%-----------------------------------LICENSE------------------------------------%
%   This file is part of Mathematics-and-Physics.                              %
%                                                                              %
%   Mathematics-and-Physics is free software: you can redistribute it and/or   %
%   modify it under the terms of the GNU General Public License as             %
%   published by the Free Software Foundation, either version 3 of the         %
%   License, or (at your option) any later version.                            %
%                                                                              %
%   Mathematics-and-Physics is distributed in the hope that it will be useful, %
%   but WITHOUT ANY WARRANTY; without even the implied warranty of             %
%   MERCHANTABILITY or FITNESS FOR A PARTICULAR PURPOSE.  See the              %
%   GNU General Public License for more details.                               %
%                                                                              %
%   You should have received a copy of the GNU General Public License along    %
%   with Mathematics-and-Physics.  If not, see <https://www.gnu.org/licenses/>.%
%------------------------------------------------------------------------------%
%   Author:     Ryan Maguire                                                   %
%   Date:       December 6, 2024                                               %
%------------------------------------------------------------------------------%
\documentclass{beamer}
\usepackage{amsmath}
\graphicspath{{../images/}}
\title{Reorganizing the MITx Mathematics Content}
\author{Ryan Maguire}
\date{January 13, 2025}
\usenavigationsymbolstemplate{}
\setbeamertemplate{footline}[frame number]
\graphicspath{{../images/}}
\begin{document}
    \maketitle
    \begin{frame}
        Some problems we'll discuss.
        \begin{itemize}
            \item
                Decentralized repositories.
                \begin{itemize}
                    \item
                        Nearly identical code split
                        across several different sources.
                    \item
                        Nearly identical build systems, but with slight
                        differences that make using the same files have
                        a few hiccups.
                \end{itemize}
            \item
                Bloated repository sizes.
                \begin{itemize}
                    \item
                        Mostly caused by \texttt{git} keeping track of
                        binary files.
                    \item
                        Copy / paste of problems and content also contribes.
                \end{itemize}
            \item
                It takes a while to check your changes when using
                XML workflow via \texttt{latex2edx}.
            \item
                Moving content around changes the associated URLs for a
                vertical.
        \end{itemize}
    \end{frame}
    \begin{frame}
        The mathematics department had their content split across several
        repositories.
        \begin{itemize}
            \item 18.01 residential / online.
            \item 18.02 residential / online.
            \item 18.03 residential / online.
            \item 18.06 residential.
            \item ASEs.
        \end{itemize}
        Each repo had a fork or two as well.
    \end{frame}
    \begin{frame}
        The individual repositories also have a large amount of
        bloat, taking up a few gigabytes.
        \par\hfill\par
        By contrast, the entire Linux kernel
        is less than half a gigabyte (once compressed), and non-enterprise
        GitHub accounts must keep their source under 1 gigabyte.
    \end{frame}
    \begin{frame}
        The bloat comes from primarily from two sources.
        \begin{itemize}
            \item
                Binary files being tracked by \texttt{git}.
                Things like PDFs, Python binaries (\texttt{.pyc} files),
                PNGs and other image formats.
            \item
                Lots of copy / paste across source files. A given problem
                may occur, near verbatim, in different courses and
                in different sections of the same course.
        \end{itemize}
    \end{frame}
    \begin{frame}
        Solution: A centralized repository with \textbf{source code} for
        binaries.
        \url{https://github.mit.edu/rmaguire/mitx_mathematics}
        \par\hfill\par
        This contains \textbf{all} of the mathematics departments MITx
        content, and is gradually replacing binary files for figures with
        source code to produces these figures.
        \par\hfill\par
        This has typically been written
        in \texttt{asymptote}, which is a \texttt{C++} style vector graphics
        language, but some SVGs are rendered using \texttt{C} libraries, and
        a few \texttt{tikZ} files exist.
    \end{frame}
    \begin{frame}
        Cut and paste of content is being reduced by reorganizing
        commonly occuring material into their own directories.
        \begin{itemize}
            \item
                \texttt{getting\_started} contains edX verticals that
                many courses use (such as how to use MITx, submit answers,
                write mathematical formulas, etc.). The language is written in
                a \textit{course-agnostic} manner so that it may be included
                into any mathematics course.
            \item
                \texttt{problems} contains, well, problems.
                A function \texttt{\\inputProblem} is provided allowing one
                to select a given file and give a user-provided name to the
                vertical, as well as details about the grading for the problem.
            \item
                \texttt{lectures} contains lectures that are written in
                a mostly course-independent way. If a given lecture cannot be
                written here, it goes into
                \texttt{course-number/content/unit-name/section-name/}.
                Something like \textit{how to use Euler's formula} could
                be written in a course-independent manner, and may be used
                across several courses.
        \end{itemize}
    \end{frame}
    \begin{frame}
        The source code for rendering images are split across several of
        my repositories. They are all open source, GPLv3 license:
        \begin{itemize}
            \item
                \url{https://github.com/ryanmaguire/asymptote_figures}
            \item
                \url{https://github.com/ryanmaguire/tikz_figures}
            \item
                \url{https://github.com/ryanmaguire/libtmpl}
        \end{itemize}
        We use \texttt{git} submodules in the main repository to pull from
        these sources and build. This saves a lot of space and duplication.
    \end{frame}
    \begin{frame}
        To avoid repeatedly uploading tarballs to MITs servers
        and waiting for webpages to render, a very primitive
        \LaTeX file, \texttt{standalone\_preamble.tex}, is provided which
        has many of the environments and macros provided by
        \texttt{latex2edx}, but in a way that standard compilers such
        as \texttt{pdflatex} can understand.
        \par\hfill\par
        This allows you to instantly review your content as a PDF.
        Note, however, that nothing fancy is done. This cannot handle
        Python or JavaScript scripts. The
        \texttt{edXscript} macro simply expands to nothing.
    \end{frame}
    \begin{frame}
        Lastly, the URL problem. The previous build systems have a very
        clever method of using tex counters to automate and label
        edX verticals and sections. But the order of the content matters,
        and moving around a problem or lecture or video will alter the
        URLs, potentially creating issues.
        \par\hfill\par
        Since all of the content is now in one repository, we fix this
        by having the \texttt{url\_name} attribute for an edX vertical,
        or problem, or section (etc.) set to the file path of the file.
        Works so far!
    \end{frame}
    \begin{frame}
        \begin{center}
            Thank you!
        \end{center}
    \end{frame}
\end{document}