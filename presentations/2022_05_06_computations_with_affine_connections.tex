%-----------------------------------LICENSE------------------------------------%
%   This file is part of Mathematics-and-Physics.                              %
%                                                                              %
%   Mathematics-and-Physics is free software: you can redistribute it and/or   %
%   modify it under the terms of the GNU General Public License as             %
%   published by the Free Software Foundation, either version 3 of the         %
%   License, or (at your option) any later version.                            %
%                                                                              %
%   Mathematics-and-Physics is distributed in the hope that it will be useful, %
%   but WITHOUT ANY WARRANTY; without even the implied warranty of             %
%   MERCHANTABILITY or FITNESS FOR A PARTICULAR PURPOSE.  See the              %
%   GNU General Public License for more details.                               %
%                                                                              %
%   You should have received a copy of the GNU General Public License along    %
%   with Mathematics-and-Physics.  If not, see <https://www.gnu.org/licenses/>.%
%------------------------------------------------------------------------------%
%   Author:     Ryan Maguire                                                   %
%   Date:       May 6, 2022                                                    %
%------------------------------------------------------------------------------%
\documentclass{beamer}
\usepackage{amsmath}

\title{Computations with Affine Connections}
\author{Ryan Maguire}
\date{May 6, 2022}
\usenavigationsymbolstemplate{}
\setbeamertemplate{footline}[frame number]
\begin{document}
    \maketitle
    \begin{frame}{Outline}
        An Affine connection on a manifold is used to describe the notion of
        \textit{parallel translation}. The definition of such an object is
        motivated by the \textit{covariant derivative} from the calculus of
        $\mathbb{R}^{n}$. We'll use this to compute the angle of deviation of
        a pendulum in Paris swinging north-to-south over a 24 hour period.
    \end{frame}
    \begin{frame}{Basic Definitions}
        A \textit{section} of a manifold to it's tangent bundle is a function
        $f:M\rightarrow{TM}$ such that for all $p\in{M}$ we have
        $\textrm{proj}\big(f(p)\big)=p$, where
        $\textrm{proj}\big((x,v)\big)=x$, $(x,v)\in{TM}$.
        A smooth vector field on a smooth manifold $(M,\,\mathcal{A})$ is a
        smooth section $X:M\rightarrow{TM}$.
        \par\hfill\par
        An equivalent formulation is an assignment at each $p\in{M}$ a
        tangent vector $X_{p}\in{T}_{p}M$ and this assignment is done
        \textit{smoothly}. That is, for any given function
        $f\in{C}^{\infty}(M,\,\mathbb{R})$ the function $Xf$, defined in a
        local chart $(\mathcal{U},\varphi)$ as:
        \begin{equation}
            Xf(p)=X_{p}f
                =\sum_{n=0}^{N-1}a_{n}\frac{\partial{f}}{\partial\varphi_{n}}(p)
        \end{equation}
        is smooth.
    \end{frame}
    \begin{frame}{Basic Definitions}
        Since a vector field $X:M\rightarrow{TM}$ can be applied to smooth
        functions $f\in{C}^{\infty}(M,\,\mathbb{R})$, the result being
        a smooth, it is possible to compose vector fields
        $X,Y:M\rightarrow{TM}$. That is, given a function
        $f\in{C}^{\infty}(M,\,\mathbb{R})$ and two smooth vector fields $X$ and
        $Y$, we obtain a new smooth function $X(Yf)$. The composition of two
        vector fields need not be a vector field. Using local
        coordinates $(\mathcal{U},\varphi)$, we may represent the tangent
        vectors $X_{p}$ and $Y_{p}$ as follows:
        \begin{align}
            X_{p}&=\sum_{n=0}^{N-1}a_{n}(p)\frac{\partial}{\partial\varphi_{n}}\\
            Y_{p}&=\sum_{n=0}^{N-1}b_{n}(p)\frac{\partial}{\partial\varphi_{n}}
        \end{align}
    \end{frame}
    \begin{frame}{Basic Definitions}
        The composition $X(Yf)$ is then:
        \begin{align}
            X\Big(
                \sum_{m=0}^{N-1}b_{n}\frac{\partial{f}}{\partial\varphi_{m}}
            \Big)
            &=\sum_{n=0}^{N-1}a_{n}\frac{\partial}{\partial\varphi_{n}}
            \Big(
                \sum_{m=0}^{N-1}b_{n}\frac{\partial{f}}{\partial\varphi_{m}}
            \Big)\\
            &=\sum_{n=0}^{N-1}\sum_{m=0}^{N-1}\Big(
                a_{n}\frac{\partial{b}_{m}}{\partial\varphi_{n}}
                \frac{\partial{f}}{\partial\varphi_{m}}+
                a_{n}b_{m}
                \frac{\partial^{2}{f}}{\partial\varphi_{n}\partial\varphi_{m}}
            \Big)
        \end{align}
        So $XYf$ involves second order derivatives, which is not Liebnizian.
        If we subtract $YXf$ and invoke the Clairaut formula, we see that
        $XY-YX$ involves only first order derivatives, and hence is a
        vector field. This is called the \textit{Lie Bracket} of $X$ with
        respect to $Y$, $[X,Y]=XY-YX$.
    \end{frame}
    \begin{frame}{Basic Definitions}
        A Riemannian metric on a smooth manifold $(M,\,\mathcal{A})$ is a
        function $g$ on $M$ such that for all $p\in{M}$,
        $g_{p}$ is a symmetric bilinear form that is positive-definite. That
        is, for all $u_{0},u_{1},v_{0},v_{1}\in{T}_{p}M$ and
        $a,b,c,d\in\mathbb{R}$ we have:
        \begin{align}
            g_{p}(u_{0},\,u_{1})
            &=g_{p}(u_{1},\,u_{0})\\
                g_{p}(au_{0}+bu_{1},\,cv_{0}+dv_{1})
            &=acg_{p}(u_{0},\,v_{0})+adg_{p}(u_{0},\,v_{1})+\nonumber\\
            &\hspace{1.3em}bcg_{p}(u_{1},\,v_{0})+bdg_{p}(u_{1},\,v_{1})\\
            g_{p}(u_{0},\,u_{0})&\geq{0}\\
            g_{p}(u_{0},\,u_{0})
            &=0\quad\textrm{iff}\quad{u}_{0}=0
        \end{align}
    \end{frame}
    \begin{frame}{Basic Definitions}
        Moreover, the function $g$ should vary \textit{smoothly} with $p$.
        That is, for any two smooth vector fields $X$ and $Y$, the function
        $f:M\rightarrow\mathbb{R}$ defined by:
        \begin{equation}
            f(p)=g_{p}(X_{p},Y_{p})
        \end{equation}
        should be smooth. Such a function $g$ is called a Riemannian metric,
        and a Riemannian manifold is an ordered triple
        $(M,\,\mathcal{A},\,g)$ where $(M,\,\mathcal{A})$ is a smooth manifold
        and $g$ is a Riemannian metric on $(M,\,\mathcal{A})$.
    \end{frame}
    \begin{frame}{Basic Definitions}
        The \textit{covariant} derivative in $\mathbb{R}^{n}$ gives us a way
        of specifying the \textit{derivative} of one vector field with respect
        to another. Given $X=\sum{a}_{n}(\mathbf{x})\partial{x}_{n}$ and
        $Y=\sum{b}_{m}(\mathbf{x})\partial{x}_{m}$, the covariant derivative
        of $Y$ with respect to $X$ is:
        \begin{equation}
            \sum_{n=0}^{N-1}\sum_{m=0}^{N-1}a_{n}
                \frac{\partial{b}_{m}}{\partial{x}_{n}}
                \frac{\partial}{\partial{x}_{m}}
        \end{equation}
        This is the part of $XY$ that does not involve second order terms.
    \end{frame}
    \begin{frame}{Affine Connections}
        An affine connection of a smooth manifold $(M,\,\mathcal{A})$
        axiomatizes the properties of the covariant derivative. This is a
        function
        $\nabla:\mathfrak{X}(M)\times\mathfrak{X}(M)\rightarrow\mathfrak{X}(M)$,
        where $\mathfrak{X}(M)$ is the set of all smooth vector fields on
        $(M,\,\mathcal{A})$, with the following properties:
        \begin{itemize}
            \item $\nabla$ is bilinear.
            \item $\nabla$ is $C^{\infty}(M,\mathbb{R})$ linear in the first
                coordinate:
                \begin{equation}
                    \nabla_{fX}Y=f\nabla_{X}Y
                \end{equation}
            \item $\nabla$ is Liebnizean in the second coordinate:
                \begin{equation}
                    \nabla_{X}fY=(Xf)Y+f\nabla_{X}Y
                \end{equation}
        \end{itemize}
        Recall that since $X$ is a smooth vector field, $Xf$ is a smooth
        function, so $(Xf)Y$ is the product of a smooth function with a smooth
        vector field, which is again a smooth vector field. The third condition
        is thus the sum of two vector fields, which is a vector field, meaning
        all of this is well defined.
    \end{frame}
    \begin{frame}{Affine Connections}
        All of this is defined for smooth manifolds and no Riemannian metric is
        yet needed. Affine connections are called \textit{torsion free} if they
        are related to the Lie bracket:
        \begin{equation}
            \nabla_{X}Y-\nabla_{Y}X=[X,Y]
        \end{equation}
        Given a Riemannian manifold $(M,\,\mathcal{A},\,g)$, the connection is
        said to be \textit{compatible} with $g$ if for all smooth vector fields
        $X,Y,Z$ we have
        \begin{equation}
            Xg(Y,Z)=g(\nabla_{X}Y,\,Z)+g(Y,\,\nabla_{X}Z)
        \end{equation}
        A \textit{Levi-Civita} connection is an affine connection that is
        torsion free and is compatible with the metric $g$.
    \end{frame}
    \begin{frame}{Affine Connections}
        \begin{theorem}[Fundamental Theorem of Riemannian Geometry]
            Every Riemannian manifold $(M,\,\mathcal{A},\,g)$ has a unique
            Levi-Civita connection $\nabla$.
        \end{theorem}
    \end{frame}
    \begin{frame}{Parallel Transport}
        Given a smooth curve $\gamma:[0,\,1]\rightarrow{M}$, if $\gamma$ is
        injective, then the image $\gamma\big[[0,1]\big]$ defines a
        submanifold with boundary in $M$. The velocity vector
        $\dot{\gamma}(t_{0})$ is defined as the derivation
        $\dot{\gamma}(t_{0}):C^{\infty}(M,\mathbb{R})\rightarrow\mathbb{R}$
        given by:
        \begin{equation}
            \dot{\gamma}(t_{0})(f)=
                \frac{\textrm{d}}{\textrm{d}t}\Big(f\circ\gamma\Big)
        \end{equation}
        Note that $f\circ\gamma$ is a function from $[0,1]$ to $\mathbb{R}$ so
        we may differentiate in the usual sense. $\dot{\gamma}$ defines a vector
        field on a closed submanifold with boundary of $M$. It is always
        possible to extend a smooth vector field on a closed submanifold to a
        smooth vector field on all of $M$. This follows from a partition of
        unity argument.
    \end{frame}
    \begin{frame}{Parallel Transport}
        \begin{theorem}
            If $(M,\,\mathcal{A})$ is a smooth manifold, if $\nabla$ is an
            affine connection on $M$, if $\gamma$ is a smooth injective curve
            in $M$, if $X$ and $Y$ are smooth extensions of $\dot{\gamma}$,
            and if $Z$ is a smooth vector field, then for all $p\in{M}$ such
            that $p=\gamma(t)$ for some $t\in[0,1]$, we have:
            \begin{equation}
                \nabla_{X_{p}}Z_{p}=\nabla_{Y_{p}}Z_{p}
            \end{equation}
        \end{theorem}
        Because of this we may abuse notation and write
        $\nabla_{\dot{\gamma}}Z$ to mean the derivative of the vector field $Z$
        along the curve $\gamma$.
    \end{frame}
    \begin{frame}{Parallel Transport}
        A vector field that is \textit{parallel} along a curve $\gamma$ is a
        smooth vector field $X$ such that $\nabla_{\dot{\gamma}}X=0$. Given a
        curve $\gamma$ and a tangent vector $v\in{T}_{\gamma(0)}M$ we can
        \textit{solve} for a vector field that is parallel along $\gamma$ in
        local coordinates by solving a system of differential equations. This
        allows one numerically solve for how a given tangent vector will be
        transported along a curve.
    \end{frame}
    \begin{frame}{The Foucault Pendulum}
        Let's use these ideas on $\mathbb{S}^{2}$.
        \par\hfill\par
        Given a Riemannian manifold $(M,\,\mathcal{A},\,g)$
        and a smooth embedding
        $f:X\rightarrow{M}$ of a smooth manifold $(X,\,\mathcal{A}_{X})$ it is
        possible to endow $(X,\,\mathcal{A}_{X})$ with a metric via
        \textit{pull-back}. Give $x\in{X}$, $u_{0},u_{1}\in{T}_{x}X$, let
        $p=f(x)$ and $v_{0}=\textrm{d}f_{x}(u_{0})$,
        $v_{1}=\textrm{d}f_{x}(u_{1})$, where $\textrm{d}f_{x}$ is the
        differential push-forward of $f$ at the point $x$. We may defined
        $\tilde{g}_{x}$ via:
        \begin{equation}
            \tilde{g}_{x}(u_{0},u_{1})=g_{p}(v_{0},v_{1})
        \end{equation}
        Since the differential push-forward of a smooth vector field is a
        smooth vector field, if $g$ is smooth, then so is $\tilde{g}$. This
        gives $(X,\,\mathcal{A}_{X},\,\tilde{g})$ the structure of a Riemannian
        manifold.
    \end{frame}
    \begin{frame}{The Foucault Pendulum}
        The round metric on $\mathbb{S}^{2}$ is obtained via pull-back.
        $\mathbb{R}^{3}$ has the standard dot product:
        \begin{equation}
            \langle\mathbf{x}|\mathbf{y}\rangle
                =\sum_{n=0}^{N-1}{x}_{n}{y}_{n}
        \end{equation}
        The inclusion mapping $\iota:\mathbb{S}^{2}\rightarrow\mathbb{R}^{3}$
        is a smooth embedding, and hence induces a metric on $\mathbb{S}^{2}$
        via pull-back of the dot product. This Riemannian metric on
        $\mathbb{S}^{2}$ is called the \textit{round metric}.
    \end{frame}
    \begin{frame}{The Foucault Pendulum}
        There is a correspondence between vector fields on $\mathbb{S}^{2}$ and
        functions $X:\mathbb{S}^{2}\rightarrow\mathbb{R}^{3}$ such that
        for all $p\in\mathbb{S}^{2}$ we have
        $\langle{p}|X_{p}\rangle=0$. A smooth vector field is in particular a
        smooth function $X:\mathbb{S}^{2}\rightarrow\mathbb{R}^{3}$ with this
        property. The function
        $\nabla:\mathfrak{X}(\mathbb{S}^{2})\times\mathfrak{X}(\mathbb{S}^{2})%
            \rightarrow\mathfrak{X}(\mathbb{S}^{2})$ defined by:
        \begin{equation}
            \nabla_{X_{p}}Y_{p}(f)=
            \textrm{d}Y_{p}(X_{p})(f)+\langle{X}_{p}|Y_{p}\rangle{f}(p)
        \end{equation}
        determines a Levi-Civita connection on $\mathbb{S}^{2}$.
    \end{frame}
    \begin{frame}{The Foucault Pendulum}
        Now imagine it is 1851, you are in Paris, and have a 67 meter long
        pendulum attached to the dome of the French Panth\'{e}on. You have it
        swinging north-to-south. You then ask what will the angle be after one
        full rotation of the Earth.
        \par\hfill\par
        It is easy to convince your self that if you were in Brazil on the
        equator and performed this experiment, the angle would not deviate at
        all. It is also easy to believe that if you performed this experiment
        next to polar bears at the north pole the pendulum would simply rotate
        with the Earth a full $2\pi$ radians.
    \end{frame}
    \begin{frame}{The Foucault Pendulum}
        The path the pendulum traverses is a circle of constant latitude.
        We can cover this in a single coordinate chart via stereographic
        projection about either the north or south pole. In these coordinates
        $x$ and $y$ correspond to east-west and south-north directions,
        respectively. The system of differential equations we need to solve is:
        \begin{align}
            \ddot{x}(t)&=-x+4\pi\dot{y}(t)\sin(\phi)\\
            \ddot{y}(t)&=-y-4\pi\dot{x}(t)\sin(\phi)
        \end{align}
        Here $\phi$ is the angle of latitude of Paris. This converts into a
        single complex differential equation:
        \begin{equation}
            \ddot{z}(t)+4\pi{i}\sin(\phi)\dot{z}(t)+z(t)=0
        \end{equation}
    \end{frame}
    \begin{frame}{The Foucault Pendulum}
        We can solve directly:
        \begin{equation}
            z(t)=\exp\Big(-2\pi{i}\sin(\phi)t\Big)\Big(A\exp(it)+B\exp(-it)\Big)
        \end{equation}
        where $A$ and $B$ are constants corresponding to the initial tangent
        vector $v$ (which is north-south). Here, $t$ is measured in days. The
        angle of deviation after one rotation of the Earth is then
        $-2\pi\sin(\phi)$. The latitude of Paris is 48 degrees and 51 minutes,
        or about 0.85 radians. The resulting deviation is
        $-4.73$, or about $-271.1$ degrees. The pendulum will be oscillating
        in the east-west direction.
    \end{frame}
\end{document}
