%-----------------------------------LICENSE------------------------------------%
%   This file is part of Mathematics-and-Physics.                              %
%                                                                              %
%   Mathematics-and-Physics is free software: you can redistribute it and/or   %
%   modify it under the terms of the GNU General Public License as             %
%   published by the Free Software Foundation, either version 3 of the         %
%   License, or (at your option) any later version.                            %
%                                                                              %
%   Mathematics-and-Physics is distributed in the hope that it will be useful, %
%   but WITHOUT ANY WARRANTY; without even the implied warranty of             %
%   MERCHANTABILITY or FITNESS FOR A PARTICULAR PURPOSE.  See the              %
%   GNU General Public License for more details.                               %
%                                                                              %
%   You should have received a copy of the GNU General Public License along    %
%   with Mathematics-and-Physics.  If not, see <https://www.gnu.org/licenses/>.%
%------------------------------------------------------------------------------%
%   Author:     Ryan Maguire                                                   %
%   Date:       June 24, 2025                                                  %
%------------------------------------------------------------------------------%
\documentclass{beamer}
\usepackage{graphicx}
\usepackage{amsmath}
\graphicspath{{../images/}}
\title{Knot Theory and Complexity}
\author{Ryan Maguire}
\date{June 26, 2025}
\usenavigationsymbolstemplate{}
\setbeamertemplate{footline}[frame number]
\begin{document}
    \maketitle
    \begin{frame}{Outline}
        \begin{itemize}
            \item Knot polynomials.
            \item Complexity theory.
            \item Computing the Jones polynomial with a computer.
        \end{itemize}
    \end{frame}
    \begin{frame}{Knot Polynomials}
        The Alexander-Conway polynomial of a link $L$ can be computed using its
        skein relation. Pick any crossing in the diagram we have:
        \begin{equation}
            \nabla_{L_{+}}(t)-\nabla_{L_{-}}(t)=t\nabla_{L_{0}}(t)
        \end{equation}
        where $L_{+}$ replaces the original crossing with a positive one,
        $L_{-}$ uses a negative crossing, and $L_{0}$ smooths the crossing,
        effectively deleting it.
        \par\hfill\par
        Question to have in mind, if you have an $N$ crossing knot, how long
        will it take to compute $\nabla_{L}(t)$ by doing this?
    \end{frame}
    \begin{frame}{Knot Polynomials}
        Let's examine another invariant you've already studied a bit, the
        Jones polynomial. The Kauffman bracket polynomial can be computed,
        using the Bar-Natan notation / convention, using the following
        diagram.
        \begin{figure}
            \centering
            \resizebox{0.75\textwidth}{!}{%
                \includegraphics{kauffman_bracket_polynomial}
            }
            \caption{The Kauffman bracket using Bar-Natan's normalization.}
        \end{figure}
        Call the diagram on the center the \textit{zero smoothing} of this
        particular crossing, and the diagram on the right the
        \textit{one smoothing}. The naming is justified later, we can use
        binary to aid our computations.
    \end{frame}
    \begin{frame}{Knot Polynomials}
        This recursive definition is completed by defining:
        \begin{align}
            \langle{L\cup{\bigcirc}}\rangle
            &=(q+q^{-1})\langle{L}\rangle\\
            \langle{\bigcirc}\rangle&=1
        \end{align}
        Note, this definition is slightly different from the one given by
        Kauffman. Given a link $L$, and denoting the zero smoothing by
        $L_{0}$ and the one smoothing via $L_{1}$, the original recursive
        definition is:
        \begin{equation}
            \langle{L}\rangle
            =A^{-1}\langle{L_{0}}\rangle+A\langle{L_{1}}\rangle
        \end{equation}
        To eliminate circles in the plane, we do:
        \begin{equation}
            \langle{L\cup\bigcirc}\rangle
            =(-A^{2}-A^{-2})\langle{L}\rangle
        \end{equation}
    \end{frame}
    \begin{frame}{Knot Polynomials}
        Both the Bar-Natan method and the Kauffman definition are made into
        true invariants once we normalize by a factor involving the writhe.
        This is the Jones polynomial.
        \par\hfill\par
        We do not have two invariants here, it is possible to go back and
        forth between the two conventions. The Bar-Natan definition has some
        nice computational benefits, we will stick with this one.
    \end{frame}
    \begin{frame}{Knot Polynomials}
        Question, if a knot or link has $N$ crossings, how long will it take
        to compute the Jones polynomial using this method?
    \end{frame}
    \begin{frame}{Knot Polynomials}
        The answer is $\textrm{poly}(N)\cdot{2}^{N}$, where
        $\textrm{poly}(N)$ is some polynomial function that depends on how
        long simple things like adding and multiplying numbers or
        Laurent polynomials takes. This is quite long!
        \par\hfill\par
        At each crossing we create two branches, one where the zero smoothing
        occurs, and one where the one smooth occurs. For $N$ crossings there
        will be $2^{N}$ \textit{leaves} in our smoothing tree.
    \end{frame}
    \begin{frame}{Knot Polynomials}
        The leaves correspond to the \textit{cube of resolutions}, which is
        all of the $2^{N}$ possible complete smoothings (meaning there are
        no crossings left) of the knot or link diagram.
        \begin{figure}
            \centering
            \includegraphics{trefoil_knot_cube_of_resolutions}
            \caption{Cube of resolutions for the Trefoil}
        \end{figure}
    \end{frame}
    \begin{frame}{Knot Polynomials}
        \begin{figure}
            \centering
            \resizebox{!}{0.8\textheight}{
                \includegraphics{figure_eight_knot_cube_of_resolutions}
            }
            \caption{Cube of resolutions for the Figure-Eight}
        \end{figure}
    \end{frame}
    \begin{frame}{Knot Polynomials}
        Is it really this hard? Let's consider a simpler problem, computing
        the $n^{\textrm{th}}$ Fibonacci number. The definition is:
        \begin{equation}
            F_{n+2}=F_{n+1}+F_{n}
        \end{equation}
        With $F_{0}=F_{1}=1$. Na\"{i}vely, this has the same branching problem.
        To compute $F_{n+2}$, you first need to compute $F_{n+1}$ and $F_{n}$,
        and creating a simple program to do this will require $2^{n}$ steps.
        But is it really that hard to compute the Fibonacci numbers?
    \end{frame}
    \begin{frame}{Knot Polynomials}
        No, this can be done in polynomial time, meaning we need at most
        $\textrm{poly}(n)$ steps. To do this, note that computing $F_{n+1}$
        is not independent from computing $F_{n}$ since
        $F_{n+1}=F_{n}+F_{n-1}$. To compute $F_{n+2}$ you can simply start with
        $F_{0}=F_{1}=1$, then compute $F_{2}=F_{1}+F_{0}=1+1=2$, and then
        $F_{3}$, and so on. If we assume that arithmetic (adding numbers)
        runs in constant time, then $F_{n}$ can be computed in linear time.
    \end{frame}
    \begin{frame}{Knot Polynomials}
        Returning to the Alexander-Conway polynomial, how long does it take?
        The skein relation is not needed, there is a simpler method involving
        matrices. If $K$ is a knot with $N$ crossings, then by Euler's formula
        $(V-E+F=2)$, the knot with cut the plane into $N+2$ faces, where the
        arcs of the knot form the edges and the crossings form the vertices.
        \par\hfill\par
        We can create a certain $(N+2)\times{N}$ matrix with affine entries
        (meaning the entries are of the form $a+bt$ with integer coefficients
        $a$ and $b$). The Alexander-Conway polynomial is computed by
        considering a particular $N\times{N}$ minor and taking the determinant.
        \par\hfill\par
        How long does it take to compute a determinant?
    \end{frame}
    \begin{frame}{Knot Polynomials}
        The elementary method we learn about in school (co-factor expansion,
        also called the Laplace expansion) requires $N!$ steps. $N!$ is
        \textbf{much} worse than $2^{N}$. But if you learn more about linear
        algebra you will learn about LU decomposition, which allows us to
        compute determinants in cubic time.
        \par\hfill\par
        \textbf{The Alexander-Conway polynomial can be computed in polynomial}
        \textbf{time}.
    \end{frame}
    \begin{frame}{Knot Polynomials}
        So, how hard is it to compute the Jones polynomial? Can we do better
        than $2^{N}$? We can indeed, there are methods of doing this in
        $2^{C\sqrt{N}}$ time, where $C$ is a fixed constant that is independent
        of the knot or link diagram.
        \par\hfill\par
        Can we compute the Jones polynomial in polynomial time?
    \end{frame}
    \begin{frame}{Complexity Theory}
        Before answering this question, let's discuss complexity theory.
        Loosely speaking, we use the symbol $\textrm{P}$ to denote all of
        the problems that can be solved in polynomial time.
        \par\hfill\par
        Sorting a collection of $N$ numbers can be done in $N\log(N)$ times,
        which is bounded by $N^{2}$, so sorting is a polynomial time problem.
        \par\hfill\par
        Search for a particular word in a dictionary with $N$ words can be
        done in $\log(N)$ time, which is bounded by $N$, so searching is also
        polynomial in time.
    \end{frame}
    \begin{frame}{Complexity Theory}
        Every positive integer greater than one has a unique factorization
        in terms of prime numbers. Can finding this factorization be done in
        polynomial time?
        \par\hfill\par
        This is not currently known, but lets consider a similar problem.
        If you are given an integer $N$, and you \textit{guess} a prime
        factorization $p_{0},\,\cdots,\,p_{n}$, how long would it take to
        verify that your factorization is correct?
    \end{frame}
    \begin{frame}{Complexity Theory}
        You could simply multiply your factorization
        $p_{0}\cdots{p}_{n}$ and see if you get $N$. If you do not, then you
        do not have the correct factorization. Multiplication is indeed
        polynomial in time, so
        \textbf{verifying a solution runs in polynomial-time}.
        \par\hfill\par
        The set of problems that can be verified in polynomial time is
        denoted $\textrm{NP}$.
        \par\hfill\par
        How does $\textrm{P}$ related to $\textrm{NP}$?
    \end{frame}
    \begin{frame}{Complexity Theory}
        It is not hard to see that $\textrm{P}\subseteq\textrm{NP}$ is true,
        but it is currently unknown (and, in fact, it is a million dollar
        problem) if $\textrm{P}=\textrm{NP}$.
    \end{frame}
    \begin{frame}{Computing the Jones Polynomial with a Computer}
        How does the Jones polynomial tie in to this?
        The Jones polynomial is $\#\textrm{P-Hard}$ and $\textrm{NP-Hard}$,
        which roughly means that it is as least as hard as any
        $\textrm{P}$ or $\textrm{NP}$ problem. Because of this it is highly
        unlikely that the Jones polynomial can be computed in polynomial time.
    \end{frame}
    \begin{frame}{Computing the Jones Polynomial with a Computer}
        Let's study the trefoil diagram again.
        \begin{figure}
            \centering
            \includegraphics{trefoil_knot_cube_of_resolutions}
            \caption{Cube of resolutions for the Trefoil}
        \end{figure}
    \end{frame}
    \begin{frame}
        We can compute the Kauffman bracket via:
        \begin{equation}
            \label{eqn:kauffman_bracket}%
            \langle{L}\rangle=\sum_{n=0}^{2^{N}-1}
                (-q)^{w(n)}(q+q^{-1})^{c(n)}
        \end{equation}
        Here, $w(n)$ is the \textit{Hamming weight} of $n$, the number of 1's
        that occur in the binary representation of $n$. $c(n)$ is the
        \textit{circle counting function}, the number of disjoint circles that
        result from the complete resolution of $L$ corresponding to $n$.
        To compute $\langle{L}\rangle$ we need only compute $c(n)$.
    \end{frame}
    \begin{frame}{An Algorithm for the Jones Polynomial}
        First, thicken the knot. All of the crossings then become
        ``four-way intersections."
        \begin{figure}
            \centering
            \includegraphics{trefoil_knot_framed_001}
            \caption{Framed Left-Handed Trefoil}
            \label{fig:trefoil_knot_framed_001}
        \end{figure}
    \end{frame}
    \begin{frame}{An Algorithm for the Jones Polynomial}
        The aforementioned four-way intersections.
        \begin{figure}
            \centering
            \includegraphics{thickened_crossings}
            \caption{Signed Crossings in a Framed Knot}
            \label{fig:thickened_crossings}
        \end{figure}
    \end{frame}
    \begin{frame}{An Algorithm for the Jones Polynomial}
        A smoothing amounts to a roadblock.
        \begin{figure}
            \centering
            \includegraphics{thickened_negative_crossing_smoothings}
            \caption{Smoothing a Negative Crossing in a Framed Knot}
            \label{fig:thickened_negative_crossing_smoothings}
        \end{figure}
        \begin{figure}
            \centering
            \includegraphics{thickened_positive_crossing_smoothings}
            \caption{Smoothing a Positive Crossing in a Framed Knot}
            \label{fig:thickened_positive_crossing_smoothings}
        \end{figure}
    \end{frame}
    \begin{frame}{An Algorithm for the Jones Polynomial}
        We can compute $c(n)$ by following along the knot, keeping track of
        the road blocks and the orientations, until we get back to where we
        started. This amounts to a single cycle. We do this for all of the
        roads corresponding to a smoothing.
        \par\hfill\par
        More advanced algorithms that allow us to compute the Jones polynomial
        in subexponential time, $2^{C\sqrt{N}}$, are possible, but are
        considerably more sophisticated. They use tools like tangles, braids,
        and other concepts that are common in knot theory.
    \end{frame}
    \begin{frame}
        Our algorithm can be implemented in a computer using
        \textit{Gauss code}.
        \begin{figure}
            \centering
            \resizebox{!}{0.8\textheight}{%
                \includegraphics{trefoil_knot_oriented_with_extended_gauss_code}
            }
            \caption{Oriented Trefoil with Gauss Code}
        \end{figure}
    \end{frame}
\end{document}
