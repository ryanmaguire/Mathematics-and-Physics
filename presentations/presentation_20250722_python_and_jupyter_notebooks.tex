%-----------------------------------LICENSE------------------------------------%
%   This file is part of Mathematics-and-Physics.                              %
%                                                                              %
%   Mathematics-and-Physics is free software: you can redistribute it and/or   %
%   modify it under the terms of the GNU General Public License as             %
%   published by the Free Software Foundation, either version 3 of the         %
%   License, or (at your option) any later version.                            %
%                                                                              %
%   Mathematics-and-Physics is distributed in the hope that it will be useful, %
%   but WITHOUT ANY WARRANTY; without even the implied warranty of             %
%   MERCHANTABILITY or FITNESS FOR A PARTICULAR PURPOSE.  See the              %
%   GNU General Public License for more details.                               %
%                                                                              %
%   You should have received a copy of the GNU General Public License along    %
%   with Mathematics-and-Physics.  If not, see <https://www.gnu.org/licenses/>.%
%------------------------------------------------------------------------------%
%   Author:     Ryan Maguire                                                   %
%   Date:       July 20, 2025                                                  %
%------------------------------------------------------------------------------%
\documentclass{beamer}
\usepackage{color}
\usepackage{graphicx}
\usepackage{amsmath}
\usepackage{tikz}
\usepackage{listings}
\usepackage{adjustbox}
\usetikzlibrary{decorations.markings, arrows.meta}
\graphicspath{{../images/}}
\title{Python and Jupyter Notebooks}
\author{Ryan Maguire}
\date{July 22, 2025}
\usenavigationsymbolstemplate{}
\setbeamertemplate{footline}[frame number]

\DeclareFixedFont{\ttb}{T1}{txtt}{bx}{n}{12}
\DeclareFixedFont{\ttm}{T1}{txtt}{m}{n}{12}

\definecolor{deepblue}{rgb}{0,0,0.5}
\definecolor{deepred}{rgb}{0.6,0,0}
\definecolor{deepgreen}{rgb}{0,0.5,0}

\lstset{
    language=Python,
    basicstyle=\ttm,
    morekeywords={self},
    keywordstyle=\ttb\color{deepblue},
    emph={MyClass,__init__},
    emphstyle=\ttb\color{deepred},
    stringstyle=\color{deepgreen},
    frame=tb,
    showstringspaces=false
}

\begin{document}
    \maketitle
    \begin{frame}{Outline}
        \begin{itemize}
            \item Programming in Python.
            \item Literate Programming.
            \item Jupyter Notebooks and Markdown.
        \end{itemize}
    \end{frame}
    \begin{frame}{Python}
        Python is an object oriented programming language that
        emphasizes readability. The core philosophical principles are
        outlines in PEP 20\footnote{%
            \textit{Python Enhancement Proposal}.
        }, the Zen of Python:
        \begin{itemize}
            \item
                Beautiful is better than ugly.
            \item
                Explicit is better than implicit.
            \item
                Simple is better than complex.
            \item
                Complex is better than complicated.
            \item
                Flat is better than nested.
            \item
                Sparse is better than dense.
            \item
                Readability counts.
        \end{itemize}
    \end{frame}
    \begin{frame}{Python}
        If you have familiarity in \texttt{C} / \texttt{C++} or
        \texttt{Java}, this means there are a few key differences that take
        some getting used to.
        \begin{itemize}
            \item No curly brackets everywhere.
            \item No semi-colons at the end of a statement.
            \item \textbf{Significant whitespace}.
            \item Statements end at the end of a line.
        \end{itemize}
        Let's see a simple example.
    \end{frame}
    \begin{frame}[fragile]{Python}
        \begin{adjustbox}{width=0.5\textwidth,keepaspectratio}
            \begin{lstlisting}
import math

def dot(point0, point1):
    """
        Computes the Euclidean dot product of two points.
    """
    return point0[0]*point1[0] + point0[1]*point1[1]

def norm(point):
    """
        Computes the distance between a point and the origin.
    """
    return math.sqrt(dot(point, point))

def angle(point0, point1):
    """
        Computes the angle made by two points in the plane.
    """
    dot_prod = dot(point0, point1)
    norm0 = norm(point0)
    norm1 = norm(point1)

    return math.acos(dot_prod / (norm0 * norm1))

if __name__ == "__main__":
    p = [1.0, 1.0]
    q = [-1.0, 1.0]
    theta = angle(p, q)
    print(theta)
            \end{lstlisting}
        \end{adjustbox}
    \end{frame}
    \begin{frame}{Python}
        Placing this in a file \texttt{angle.py} and typing
        \texttt{python3 angle.py} will print out
        \texttt{1.5707963267948966}.
        \par\hfill\par
        Some things to note.
        \begin{enumerate}
            \item
                We use \textbf{\texttt{import}} to include external libraries
                or \textit{packages} into our program for us to use.
            \item
                \textbf{\texttt{def}} defines a function.
                Splitting up long and complicated programs into smaller
                pieces helps make readability and maintenance a lot easier.
            \item
                There are \textit{four} spaces between the start of a line and
                the start of a function. Python uses significant whitespace to
                determine where a function starts and ends. It also does this
                with if-then statements, for and while loops, classes, and
                just about anything else that is nested.
        \end{enumerate}
    \end{frame}
    \begin{frame}{Python}
        You are not required to use four spaces (but this is quite common),
        but you must be consistent. Using four spaces in some spots and
        two spaces elsewhere will cause errors, and your program will not run.
    \end{frame}
    \begin{frame}{Python}
        Since we intend to work with knots and quandles, we need data
        structures that are a little more complicated than just real numbers
        and functions. This is where \textit{classes} come into use.
        \par\hfill\par
        A class is a combination of data (perhaps the DT code of a knot, or
        the Cayley table of a quandle), and \textit{methods}, which are
        functions that are able to access and modify this data, and interact
        with other classes and variables.
        \par\hfill\par
        Let's try a simple example.
    \end{frame}
    \begin{frame}[fragile]{Python}
        \begin{adjustbox}{width=0.5\textwidth,keepaspectratio}
            \begin{lstlisting}
import math

class Vec3:
    """
        Implements a 3D vector object.
    """
    def __init__(self, x_component, y_component, z_component):
        """
            Creates a 3D vector from its Cartesian coordinates.
        """
        self.x = x_component
        self.y = y_component
        self.z = z_component

    def zenith(self):
        """
            Computes the angle this vector makes with the north pole.
        """

        # This is the azimuthal, or cylindrical, part of the vector.
        rho = math.sqrt(self.x*self.x + self.y*self.y)

        # The zenith angle is the angle made by rho and z.
        return 0.5 * math.pi - math.atan2(self.z, rho)

if __name__ == "__main__":
    point = Vec3(3.0, 4.0, 5.0)
    angle = point.zenith()
    print(angle)
            \end{lstlisting}
        \end{adjustbox}
    \end{frame}
    \begin{frame}[fragile]{Literate Programming}
        Complex mathematical objects will often have lots of data and methods.
        For knot theory examples, see:
        \begin{itemize}
            \item
                \footnotesize\url{%
                    https://github.com/%
                        sagemath/sage/tree/develop/src/sage/knots%
                }
            \item
                \footnotesize\url{%
                    https://github.com/3-manifolds/SnapPy%
                }
            \item
                \footnotesize\url{%
                    https://github.com/3-manifolds/Spherogram%
                }
        \end{itemize}
        These libraries are very big and complex, but can be useful tools
        for learning how other mathematicians implement various data
        structures and algorithms.
    \end{frame}
    \begin{frame}[fragile]{Literate Programming}
        In the previous example we saw \textit{comments}. Comments in Python
        start with the \texttt{\#} symbol and end when the line ends.
        In \textit{literate programming}, we write our programs for the human
        reader, not just for the computer. Literate programming is more than
        just leaving comments that help explain the algorithm, though comments
        are important too.
    \end{frame}
    \begin{frame}[fragile]{Literate Programming}
        In a literate program we mix together detailed explanations of an
        algorithm with the source code that actually implements the algorithm.
        This idea was introduced by Donald Knuth in the 1980s, and the concept
        has been gaining popularity in recent years. There are many ways to
        write literate programs (\textit{Mathematica Notebooks} are common),
        but we will be concentrating on \textbf{Jupyter Notebooks}.
    \end{frame}
    \begin{frame}[fragile]{Literate Programming}
        Jupyter notebooks mix together code (usually in Python, but you may
        use C, C++, Java, and a handful of other languages) with
        descriptive text that is written in \textit{Markdown}. Markdown is
        similar to \LaTeX, you use a simple syntax to describe your text, and
        then the Jupyter notebook renders the text and a nice format.
        \par\hfill\par
        Let's take a look at one of my Jupyter notebooks that I used to help
        teach astronomy and optics.
    \end{frame}
    \begin{frame}{Jupyter Notebooks}
        Ayush Jain has provided his Mathematica notebooks, and the syntax can
        be quickly converted into Python (especially easy if you use the
        \texttt{sympy} package). You are highly encouraged to read these
        notebooks and try to understand the code. It provides a method of
        compute the cocycle polynomia invariant, and you can adapt it to your
        own project.
    \end{frame}
\end{document}
