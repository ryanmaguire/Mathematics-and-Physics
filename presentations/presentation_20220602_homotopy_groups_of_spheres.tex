%-----------------------------------LICENSE------------------------------------%
%   This file is part of Mathematics-and-Physics.                              %
%                                                                              %
%   Mathematics-and-Physics is free software: you can redistribute it and/or   %
%   modify it under the terms of the GNU General Public License as             %
%   published by the Free Software Foundation, either version 3 of the         %
%   License, or (at your option) any later version.                            %
%                                                                              %
%   Mathematics-and-Physics is distributed in the hope that it will be useful, %
%   but WITHOUT ANY WARRANTY; without even the implied warranty of             %
%   MERCHANTABILITY or FITNESS FOR A PARTICULAR PURPOSE.  See the              %
%   GNU General Public License for more details.                               %
%                                                                              %
%   You should have received a copy of the GNU General Public License along    %
%   with Mathematics-and-Physics.  If not, see <https://www.gnu.org/licenses/>.%
%------------------------------------------------------------------------------%
%   Author:     Ryan Maguire                                                   %
%   Date:       June 2, 2022                                                   %
%------------------------------------------------------------------------------%
\documentclass{beamer}
\usepackage{amsmath}

\title{Homotopy Groups of Spheres}
\author{Ryan Maguire}
\date{June 2, 2022}
\usenavigationsymbolstemplate{}
\setbeamertemplate{footline}[frame number]
\begin{document}
    \maketitle
    \begin{frame}
        The fundamental group is the group of equivalence classes of curves up
        to homotopy with the operation of concatenation. That is, given a
        topology space $(X,\tau)$ and a point $x_{0}\in{X}$, we define:
        \begin{equation}
            \tilde{\pi}_{1}(X,x_{0})=
                \{\,f\in{C}^{0}(\mathbb{S}^{1},X)\;|\;
                    f((1,0))=x_{0}\;\}
        \end{equation}
        Here $C^{0}(\mathbb{S}^{1},X)$ is the set of continuous functions from
        $\mathbb{S}^{1}$ to $X$. We define the equivalence relation
        $R$ on $\tilde{\pi}_{1}(X,x_{0})$ by $f\simeq{g}$ if and only if $f$ is
        homotopic to $g$. The set $\pi_{1}(X,x_{0})$ is the set of equivalence
        classes $\tilde{\pi}_{1}(X,x_{0})/R$. The group operation is that of
        concatenating curves, it is often denoted $*$.
    \end{frame}
    \begin{frame}
        The higher order homotopy groups $\pi_{n}(X,x_{0})$ are obtained by
        considering mappings of $\mathbb{S}^{n}$ into $X$. The group operation
        is obtained more easily if we think of $\mathbb{S}^{n}$ as the quotient
        space $[0,1]^{n}/\partial[0,1]^{n}$. The group operation is obtained by
        concatenation along the last coordinate of $[0,1]^{n}$.
        \par\hfill\par
        For a path connected space $(X,\tau)$, for any two points
        $x_{0},x_{1}\in{X}$, the homotopy groups
        $\pi_{n}(X,x_{0})$ and $\pi_{n}(X,x_{1})$ are isomorphic. Because of
        this, when considering spheres, we just write $\pi_{n}(\mathbb{S}^{m})$
        and omit the base point.
    \end{frame}
    \begin{frame}
        Computing the fundamental group of the circle is a well-known problem
        that takes up part of an algebraic topology course. The universal cover
        of $\mathbb{S}^{1}$ is the real line, and the group of deck
        transformations of this covering are integer shifts of the real line,
        $x\mapsto{x}+n$ for some $n\in\mathbb{Z}$. The fundamental group is
        isomorphic to the group of deck transformations, showing that
        $\pi_{1}(\mathbb{S}^{1})\simeq\mathbb{Z}$.
    \end{frame}
    \begin{frame}
        For higher order homotopy groups, the $n^{th}$ homotopy group of a path
        connected, locally path-connected, semi-locally simply path connected
        topological space $(X,\tau)$ is isomorphic to the $n^{th}$ homotopy
        group of the universal cover of $(X,\tau)$. Any continuous function
        $f:\mathbb{S}^{n}\rightarrow{X}$ lifts to a continuous function
        $\tilde{f}:\mathbb{S}^{n}\rightarrow\tilde{X}$, where
        $(\tilde{X},\tilde{\tau})$ is the universal cover of $(X,\tau)$,
        since $\mathbb{S}^{n}$ is simply connected for $n>1$. Any
        homotopy applied to $\tilde{f}$ may be projected down to a homotopy on
        $f$. Since the universal cover of $\mathbb{S}^{1}$ is
        $\mathbb{R}$, which is contractible, $\pi_{n}(\mathbb{S}^{1})$ is
        trivial for all $n>1$.
    \end{frame}
    \begin{frame}
        We so far have $\pi_{n}(\mathbb{S}^{1})$ for all $n\geq{1}$. Let us now
        compute $\pi_{n}(\mathbb{S}^{m})$ for all $n<m$. Any continuous function
        $f:\mathbb{S}^{n}\rightarrow\mathbb{S}^{m}$, with $n<m$, is homotopic
        to a continuous function
        $\tilde{f}:\mathbb{S}^{n}\rightarrow\mathbb{S}^{m}$ that is
        \textit{not} surjective. $\mathbb{S}^{m}$ minus a point is homeomorphic
        to $\mathbb{R}^{m}$ by the stereographic projection about the deleted
        point. Since $\mathbb{R}^{m}$ is contractible, this induces a homotopy
        between $\tilde{f}$ and a constant mapping, which shows $f$ is homotopic
        to a point. Hence, $\pi_{n}(\mathbb{S}^{m})$ is trivial for
        $n<m$.
    \end{frame}
    \begin{frame}
        Next on the list is $\pi_{n}(\mathbb{S}^{n})$. We have computed the
        case when $n=1$. Given a path connected space $(X,\tau)$ there is a
        homomorphism $h:\pi_{n}(X)\rightarrow{H}_{n}(X)$ called the
        \textit{Hurewicz homomorphism}. The Hurewicz theorem says the following:
        \begin{theorem}
            If $(X,\tau)$ is a path connected topological space, and if
            $h_{n}:\pi_{n}(X)\rightarrow{H}_{n}(X)$ is the Hurewicz
            homomorphism, then if $n=1$, $h$ induces an isomorphism between
            $H_{1}(X)$ and the Abelianization of $\pi_{1}(X)$, and if $n>1$ and
            $\pi_{m}(X)$ is trivial for all $m<n$, then
            $h$ is an isomorphism.
        \end{theorem}
        We have shown that $\mathbb{S}^{n}$ has trivial $m$ homotopy groups for
        $m<n$ meaning $\pi_{n}(\mathbb{S}^{n})$ is isomorphic to
        $H_{n}(\mathbb{S}^{n})$, which is $\mathbb{Z}$.
    \end{frame}
    \begin{frame}
        We now have $\pi_{n}(\mathbb{S}^{m})$ for $n\leq{m}$. Homology can no
        longer assist us since the homology groups of a manifold are zero beyond
        the dimension of the manifold. If this were true of homotopy groups we'd
        have a rather trivial problem at hand. To compute
        $\pi_{3}(\mathbb{S}^{2})$ we'll need a few concepts.
        \par\hfill\par
        A fiber bundle is an ordered quadruple
        $(E,B,p,F)$ where $p:E\rightarrow{B}$ is a continuous surjective
        function such that for all $x\in{B}$ the fiber $p^{-1}[\{x\}]$ is
        homeomorphic, with the subspace topology, to $F$. Moreover, $p$
        satisfies the \textit{local trivialization} property that for all
        $x\in{B}$ there is an open subset $\mathcal{U}\subseteq{B}$ with
        $b\in\mathcal{U}$ such that
        $\pi^{-1}[\mathcal{U}]$ is homeomorphic to $\mathcal{U}\times{F}$.
    \end{frame}
    \begin{frame}
        A Fibration is a generalization of a fiber bundle, it is an ordered
        triple $(E,B,p)$ such that for any space $X$ and any homotopy
        $H:X\times[0,1]\rightarrow{B}$ and for any continuous function
        $\tilde{f}_{0}$ lifting $H_{0}$, there is a homotopy
        $\tilde{H}:X\times[0,1]\rightarrow{E}$ that lifts $H$. That is,
        $p\circ\tilde{H}=H$. If $E$ is path connected, the fibers of any two
        points of $B$ are homotopy equivalent. This homotopy equivalence class
        is usually referred to as \textit{the} fiber $F$. Fiber bundles
        $(E,B,p,F)$ are hence a special case of fibrations where the fibers of
        all points of $B$ are not just homotopy equivalent, but are
        homeomorphic.
    \end{frame}
    \begin{frame}
        There is a long exact sequence of homotopy groups of a fibration. For
        simplicity, let us think of fiber bundles $(E,B,p,F)$ rather than the
        more abstract fibrations. So the fibers of $B$ are all homeomorphic,
        rather than just homotopy equivalent. There are homomorphisms:
        \begin{equation}
            \cdots\rightarrow
                \pi_{N+1}(F)\rightarrow\pi_{N+1}(E)\rightarrow\pi_{N+1}(B)
                \rightarrow\pi_{N}(F)\rightarrow\cdots
        \end{equation}
        that make this a long exact sequence. We can compute
        $\pi_{3}(\mathbb{S}^{2})$ using this sequence and the
        \textit{Hopf Fibration}.
    \end{frame}
    \begin{frame}
        We may identify $\mathbb{S}^{3}$ with the unit quaternions,
        $\mathbf{p}\in\mathbb{H}$ with $||\mathbf{p}||=1$. We can then define
        the following continuous function
        $f:\mathbb{H}\rightarrow\mathbb{R}^{3}$:
        \begin{equation}
            f\big((a,b,c,d)\big)
                =\big(2(ac+bd),\,2(bc-ad),\,a^{2}+b^{2}-c^{2}-d^{2}\big)
        \end{equation}
        where $(a,b,c,d)=\mathbf{p}\in\mathbb{H}$. If $||\mathbf{p}||=1$, then
        $||f(\mathbf{p})||=1$ showing that this is also a map from
        $\mathbb{S}^{3}$ to $\mathbb{S}^{2}$. With a bit of work one can see
        that the fiber of every point in $\mathbb{S}^{2}$ is a circle, showing
        that this is a fiber bundle
        $(\mathbb{S}^{3},\mathbb{S}^{2},f,\mathbb{S}^{1})$.
    \end{frame}
    \begin{frame}
        The long exact homotopy sequence then yields:
        \begin{equation}
            \pi_{3}(\mathbb{S}^{1})\rightarrow\pi_{3}(\mathbb{S}^{3})
            \rightarrow\pi_{3}(\mathbb{S}^{2})\rightarrow\pi_{2}(\mathbb{S}^{1})
        \end{equation}
        But we know $\pi_{3}(\mathbb{S}^{1})=0$ and
        $\pi_{2}(\mathbb{S}^{1})=0$ meaning
        $\pi_{3}(\mathbb{S}^{3})$ is isomorphic to
        $\pi_{3}(\mathbb{S}^{2})$. Since we've computed
        $\pi_{3}(\mathbb{S}^{3})$ using the Hurewicz theorem, we have
        $\pi_{3}(\mathbb{S}^{2})\simeq\mathbb{Z}$.
    \end{frame}
    \begin{frame}
        Equations can be boring and I'd like to visually describe
        $\mathbb{S}^{3}$ being foliated by circles. To do this we attempt to
        foliate $\mathbb{R}^{3}$ with circles first. We start with the unit
        circle in the $xy$ plane. Next we place a torus of inner radius
        $\epsilon$ around this circle. Instead of thinking of it as a torus,
        we think of it as the union of planar circles each of which making the
        same very small angle with the $xy$ plane. Next we consider a larger
        torus around our $\mathbb{S}^{1}$ in the $xy$ plane, again thinking of
        it as the union of planar circles, but now making a larger angle with
        the $xy$ plane. We continue doing this, growing the torii, thinking of
        it as circles making larger angles with the $xy$ plane.
        \par\hfill\par
        We nearly foliate all of $\mathbb{R}^{3}$ with circles until we reach
        the point where the angle between the circles and the $xy$ plane is
        $\frac{\pi}{2}$. In this instance the \textit{circles} are really the
        $z$ axis, and they all lie on top of each other. We then realize that
        $\mathbb{S}^{3}$ is $\mathbb{R}^{3}$ with a point
        \textit{at infinity} and the $z$ axis is really a circle containing
        the point at infinity. This foliates the entirety of $\mathbb{S}^{3}$
        with circles.
    \end{frame}
    \begin{frame}
        The following is courtesy of Niles Johnson's Sage code, released under
        the GNU General Public License, version 2.
        \begin{figure}
            \resizebox{0.7\textwidth}{!}{%
                \includegraphics{../images/hopf_fibration.png}
            }
            \caption{Visualization of the Hopf Fibration}
            \label{fig:hopf_fibration}
        \end{figure}
    \end{frame}
    \begin{frame}
        The Freudenthal suspension theorem tells us the \textit{diagonals} of
        the homotopy groups of spheres eventually stabilize.
        \begin{theorem}
            If $(X,\tau)$ is a path connected topological space with
            $\pi_{k}(X)$ trivial for all $k\leq{n}$ for some $n\in\mathbb{N}$,
            the $\pi_{n}(X)$ is isomorphic to
            $\pi_{n+1}(\Sigma{X})$, where $\Sigma$ is the suspension of $X$.
        \end{theorem}
        The suspension of $\mathbb{S}^{n}$ is $\mathbb{S}^{n+1}$ giving us the
        following corollary:
        \begin{theorem}
            $\pi_{n+k}(\mathbb{S}^{n})$ is isomorphic to
            $\pi_{n+k+1}(\mathbb{S}^{n+1})$ for all $n>k+1$.
        \end{theorem}
        Below these diagonals is all zero, as we've seen, but above these
        diagonals is almost random. The computations involve spectral sequences.
        One result of note is that $\pi_{n}(\mathbb{S}^{m})$ is always finite
        except for the diagonals and super diagonals described by the
        previous theorem.
    \end{frame}
\end{document}
