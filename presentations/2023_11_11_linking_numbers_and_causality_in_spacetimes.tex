%-----------------------------------LICENSE------------------------------------%
%   This file is part of Mathematics-and-Physics.                              %
%                                                                              %
%   Mathematics-and-Physics is free software: you can redistribute it and/or   %
%   modify it under the terms of the GNU General Public License as             %
%   published by the Free Software Foundation, either version 3 of the         %
%   License, or (at your option) any later version.                            %
%                                                                              %
%   Mathematics-and-Physics is distributed in the hope that it will be useful, %
%   but WITHOUT ANY WARRANTY; without even the implied warranty of             %
%   MERCHANTABILITY or FITNESS FOR A PARTICULAR PURPOSE.  See the              %
%   GNU General Public License for more details.                               %
%                                                                              %
%   You should have received a copy of the GNU General Public License along    %
%   with Mathematics-and-Physics.  If not, see <https://www.gnu.org/licenses/>.%
%------------------------------------------------------------------------------%
%   Author:     Ryan Maguire                                                   %
%   Date:       October 24, 2023                                               %
%------------------------------------------------------------------------------%
\documentclass{beamer}
\usepackage{graphicx}
\usepackage{amsmath}
\graphicspath{{../images/}}
\title{Linking Numbers and Causality in Spacetimes}
\author{Ryan Maguire}
\date{November 11, 2023}
\usenavigationsymbolstemplate{}
\setbeamertemplate{footline}[frame number]
\begin{document}
    \maketitle
    \begin{frame}{Outline}
        \begin{itemize}
            \item Spacetimes
            \item Causality and Linking
            \item Linking and Affine Linking Numbers
        \end{itemize}
    \end{frame}
    \begin{frame}{Spacetimes}
        A Riemannian manifold is a smooth manifold $M$ equipped with a
        \textit{Riemannian metric}, a means of measuring angles between
        tangent vectors that varies smoothly between tangent spaces. That is,
        for each $p\in{M}$ there is a positive-definite symmetric bilinear form
        $g_{p}:T_{p}M\times{T}_{p}M\rightarrow\mathbb{R}$ and for every pair of
        smooth vector fields $X,Y$ the function
        \begin{equation}
            p\mapsto{g}_{p}(X_{p},\,Y_{p})
        \end{equation}
        is smooth.
        \par\hfill\par
        Semi-Riemannian manifolds (also called pseudo-Riemannian manifolds)
        generalize this slightly. The positive-definite condition is relaxed to
        non-degeneracy. For each $p\in{M}$ and for all non-zero $v\in{T}_{p}M$
        there is some tangent vector $w\in{T}_{p}M$ such that
        $g_{p}(v,\,w)\ne{0}$.
    \end{frame}
    \begin{frame}{Spacetimes}
        Sylvester's theorem of inertia tells us that for every $p\in{M}$
        there is a chart $(x_{1},\,\cdots,\,x_{n})$ and three numbers
        $n_{0},\,n_{+},\,n_{-}$ that are constant (independent of the chart)
        such that $n=n_{0}+n_{+}+n_{-}$ and:
        \begin{equation}
            g_{p}=\sum_{k=1}^{n_{+}}\textrm{d}x_{k}^{2}
                -\sum_{k=1}^{n_{-}}\textrm{d}x_{k+n_{+}}^{2}
        \end{equation}
        Non-degeneracy means $n_{0}=0$. The \textit{signature} of a
        semi-Riemannian metric is the ordered pair $(n_{+},n_{-})$.%
        \footnote{%
            Riemannian manifolds have signature $(n,0)$.
        }
    \end{frame}
    \begin{frame}{Spacetimes}
        A \textit{Lorentz} manifold is a semi-Riemannian manifold $(X,\,g)$ of
        dimension $n+1$ with signature $(n,1)$. That is, for every point
        $p\in{X}$ there is a chart $(x_{1},\,\cdots,\,x_{n},\,t)$ such that:
        \begin{equation}
            g_{p}=-\textrm{d}t^{2}+\sum_{k=1}^{n}\textrm{d}x_{k}^{2}
        \end{equation}
        The \textit{null vectors} in $T_{p}M$ are those that satisfy
        $g_{p}(v,\,v)=0$. The equation above tells us that these vectors
        satisfy the equation of a cone.
    \end{frame}
    \begin{frame}{Spacetimes}
        Intuitively, null vectors represent particles traveling at the speed of
        light. We can see this by abusing notation and letting $c=1$. We have:
        \begin{align}
            -c^{2}\textrm{d}t^{2}+\sum_{k=1}^{n}\textrm{d}x_{k}^{2}&=0\\
            \Rightarrow
            \sum_{k=1}^{n}\Big(\frac{\textrm{d}x_{k}}{\textrm{d}t}\Big)^{2}
                &=c^{2}\\
            \Rightarrow
            \sqrt{\sum_{k=1}^{n}\Big(\frac{\textrm{d}x_{k}}{\textrm{d}t}\Big)^{2}}
            &=c
        \end{align}
        This last equation says the \textit{speed} the tangent represents is
        equal to $c$, which we're taking to be the speed of light.
    \end{frame}
    \begin{frame}{Spacetimes}
        The \textit{light-cone} at $p$, the set of all null vectors in
        $T_{p}M$, splits $T_{p}M$ into three parts: the interior, the exterior,
        and the cone itself. The interior consists of two connected components.
        \par\hfill\par
        A \textit{time-orientation} is a smooth choice of one of these connected
        components as one varies the point $p$.%
        \footnote{%
            Note: orientable and time-orientable are different. The (open)
            M\"{o}bius strip is not orientable, but it can be time-orientable,
            pending the metric $g$. An annulus is orientable, but it can be
            non-time-orientable, again pending $g$. All four combinations of
            orientable and time-orientable are possible. These are separate
            notions.
        }
        The chosen component is called the \textit{future direction} at the
        point $p$.
        \par\hfill\par
        A spacetime is a Lorentz manifold $(X,\,g)$ with a time orientation.
    \end{frame}
    \begin{frame}{Spacetimes}
        The future and past of a point $p$, denoted $J_{p}^{+}$ and $J_{p}^{-}$,
        respectively, are all points in $X$ that can be connected to $p$ by a
        curve $\gamma$ whose speed is always bounded by the speed of light.
        That is:
        \begin{equation}
            g_{\gamma(t)}(\dot{\gamma}(t),\,\dot{\gamma}(t))\leq{0}
        \end{equation}
        for all $t$.
        \par\hfill\par
        \textit{Nice} spacetimes have the following properties:
        \begin{itemize}
            \item
                $J_{p}^{+}\cap{J}_{q}^{-}$ is compact for all $p,q\in{X}$.
            \item
                No time travel.
        \end{itemize}
        No time travel means there is no future-pointing curve $\gamma$ (not
        exceeding the speed of light) that starts at a point $p$ and later
        returns to $p$.
        \par\hfill\par
        These spacetimes are called \textit{globally hyperbolic}.
    \end{frame}
    \begin{frame}{Spacetimes}
        \begin{theorem}[Geroch, 1970]
            Globally hyperbolic spacetimes $(X,\,g)$ are homeomorphic to
            $M\times\mathbb{R}$ for some topological manifold $M$.
        \end{theorem}
        \begin{theorem}[Bernal-S\'{a}nchez, 2003]
            Globally hyperbolic spacetimes $(X,\,g)$ are diffeomorphic to
            $M\times\mathbb{R}$ for some smooth manifold $M$, and the
            restriction of $g$ to $M\times\{t\}$ can be made Riemannian for all
            $t\in\mathbb{R}$.
        \end{theorem}
        Moreover, $M\times\{t\}$ is a \textit{Cauchy hypersurface},
        a submanifold with the property that particles traveling at
        less-than-or-equal-to the speed of light will intersect $M$ at
        precisely one time moment $t_{0}$.
    \end{frame}
    \begin{frame}{Causality and Linking}
        The \textit{sky} of a point $p$ in a spacetime $(X,\,g)$ is the space
        of all light-rays passing through $p$.
        \par\hfill\par
        If $(X,\,g)$ is globally hyperbolic, the sky of a point $p$ can be given
        a more explicit topological and geometric structure. Let $M$ be a
        Cauchy surface so that $X$ is diffeomorphic to $M\times\mathbb{R}$ and
        let $t_{0}\in\mathbb{R}$ be the unique time such that
        $p\in{M}\times\{t_{0}\}$.
        \par\hfill\par
        Since $(M,\,g|_{M})$ is Riemannian the spherical cotangent bundle
        $ST^{*}M$ can be realized as unit-length elements of $T^{*}M$.
        \par\hfill\par
        For any ray of light $\gamma$ passing through $p$ the linear functional
        $\phi:T_{p}M\rightarrow\mathbb{R}$ defined by:
        \begin{equation}
            \phi(v)=g_{p}\big(v,\,\dot{\gamma}(t_{0})\big)
        \end{equation}
        is non-zero. It hence defines a point in $ST^{*}M$.
    \end{frame}
    \begin{frame}{Causality and Linking}
        The sky of a point can thus be identified with the fibre of $p$ under
        the canonical projection $\textrm{proj}_{M}:ST^{*}M\rightarrow{M}$.
        \par\hfill\par
        The skies of two points $p,q\in{X}$ can be used to answer questions
        about \textit{causality}. That is, whether or not it is possible for
        $p$ and $q$ to communicate with each other by sending messages that
        do not travel faster than light.
        \par\hfill\par
        In some sense the skies of $p$ and $q$ may be \textit{linked} in
        $ST^{*}M$.
    \end{frame}
    \begin{frame}{Causality and Linking}
        $ST^{*}M$ naturally has a contact structure given by the Liouville
        form. One may then ask of the skies of two points are
        \textit{Legendrian linked}, instead of just topologically linked.
        This leads to the following theorem.
        \begin{theorem}[Chernov, Nemirovksi, 2009]
            The skies of causally related points in a globally hyperbolic
            spacetime are Legendrian linked.
        \end{theorem}
        For the remainder of the talk we'll talk about other linking ideas that
        can be used to study causality.
    \end{frame}
    \begin{frame}{Linking and Affine Linking Numbers}
    \end{frame}
\end{document}
