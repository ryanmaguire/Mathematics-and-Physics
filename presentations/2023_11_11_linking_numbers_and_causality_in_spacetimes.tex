%-----------------------------------LICENSE------------------------------------%
%   This file is part of Mathematics-and-Physics.                              %
%                                                                              %
%   Mathematics-and-Physics is free software: you can redistribute it and/or   %
%   modify it under the terms of the GNU General Public License as             %
%   published by the Free Software Foundation, either version 3 of the         %
%   License, or (at your option) any later version.                            %
%                                                                              %
%   Mathematics-and-Physics is distributed in the hope that it will be useful, %
%   but WITHOUT ANY WARRANTY; without even the implied warranty of             %
%   MERCHANTABILITY or FITNESS FOR A PARTICULAR PURPOSE.  See the              %
%   GNU General Public License for more details.                               %
%                                                                              %
%   You should have received a copy of the GNU General Public License along    %
%   with Mathematics-and-Physics.  If not, see <https://www.gnu.org/licenses/>.%
%------------------------------------------------------------------------------%
%   Author:     Ryan Maguire                                                   %
%   Date:       October 24, 2023                                               %
%------------------------------------------------------------------------------%
\documentclass{beamer}
\usepackage{graphicx}
\usepackage{amsmath}
\graphicspath{{../images/}}
\title{Linking Numbers and Causality in Spacetimes}
\author{Ryan Maguire}
\date{November 11, 2023}
\usenavigationsymbolstemplate{}
\setbeamertemplate{footline}[frame number]
\begin{document}
    \maketitle
    \begin{frame}{Outline}
        \begin{itemize}
            \item Contact Manifolds
            \item Spacetimes
            \item Causality and Linking
            \item Linking and Affine Linking Numbers
            \item Causality
        \end{itemize}
    \end{frame}
    \begin{frame}{Contact Manifolds}
        Contact manifolds are the odd dimensional analogue of symplectic
        manifolds. Both ideas arise from classical mechanics.
        \begin{itemize}
            \item
                The \textit{phase-space} of a particle $(p,\,q)$, $p$ being
                momentum, $q$ be position, yields an even dimensional symplectic
                manifold.
            \item
                Hypersurfaces of constant energy yield odd dimensional contact
                manifolds.
        \end{itemize}
        The exact definition attempts to characterize
        \textit{maximum non-integrability}, the opposite of the
        Frobenius theorem.
    \end{frame}
    \begin{frame}{Contact Manifolds}
        A contact manifold is a smooth odd dimensional $2n+1$ manifold $M$ with
        a distribution of co-dimension 1 hyperplanes in the tangent bundle $TM$
        with the property that no submanifold $S$ of dimension $m>n$ has
        its tangent spaces given as subspaces of the hyperplanes in the
        distribution.
        \par\hfill\par
        Locally this amounts to the existence of a one-form $\alpha$ such that
        \begin{equation}
            \alpha\land(\textrm{d}\,\alpha)^{k}\ne{0}
        \end{equation}
        The hyperplane at a point $p\in{M}$ is given by the kernel of
        $\alpha_{p}:T_{p}M\rightarrow\mathbb{R}$.
    \end{frame}
    \begin{frame}{Contact Manifolds}
        The Darboux theorem makes contact manifolds a lot more concrete. It says
        that for a contact manifold $M$ and for any $p\in{M}$ there is a chart
        $(\mathcal{U},\,\varphi)$ about $p$ such that:
        \begin{equation}
            \alpha=\textrm{d}\varphi_{2n+1}-
                \sum_{k=1}^{n}\varphi_{2k}\;\textrm{d}\varphi_{2k-1}
        \end{equation}
        For $\mathbb{R}^{3}$ we can use a global chart and this becomes:
        \begin{equation}
            \alpha=\textrm{d}z-y\textrm{d}x
        \end{equation}
        The hyperplane at $(x,\,y,\,z)$ is spanned by the vectors:
        \begin{equation}
            \mathbf{v}=\partial{y}\quad\quad\mathbf{w}=\partial{x}+y\partial{z}
        \end{equation}
    \end{frame}
    \begin{frame}{Contact Manifolds}
        \begin{figure}
            \centering
            \resizebox{\textwidth}{!}{\includegraphics{darboux_form_001.pdf}}
            \caption{The Standard Contact Structure on $\mathbb{R}^{3}$}
            \label{fig:contact_on_r3}
        \end{figure}
    \end{frame}
    \begin{frame}{Contact Manifolds}
        The spherical cotangent bundle of a Riemannian manifold $(M,\,g)$
        naturally carries a contact structure. $STM\subseteq{TM}$ can be
        defined as all pairs $(p,\,v)\in{TM}$ with $g_{p}(v,\,v)=1$, and the
        metric $g$ gives us an isomorphism between $TM$ and $T^{*}M$. $ST^{*}M$
        can be defined as the image of $STM$ under this isomorphism.
        \par\hfill\par
        A point $\tilde{v}\in{ST}^{*}M$ can be identified with a linear
        functional $v:T_{p}M\rightarrow\mathbb{R}$ modulo multiplication by a
        positive real number, where $p=\textrm{proj}_{M}(\tilde{v})$, the
        canonical projection from $ST^{*}M$ to $M$. This corresponds to the
        hyperplane $\textrm{ker}(v)\subseteq{T}_{p}M$.
    \end{frame}
    \begin{frame}{Contact Manifolds}
        Pulling this back by $\textrm{d}\,\textrm{proj}_{M}$ gives us a
        hyperplane distribution in $ST^{*}M$:
        \begin{equation}
            \chi=\{\,\textrm{d}\,
                \textrm{proj}_{M}^{-1}\big[\textrm{ker}(v)\big]\;|\;
                    \tilde{v}\in{S}T^{*}M\,\}
        \end{equation}
        This distribution of hyperplanes yields a contact structure on
        $ST^{*}M$, the one-form $\alpha$ that invokes it is called the
        \textit{Liouville form}.
    \end{frame}
    \begin{frame}{Spacetimes}
        Shifting slightly from Riemannian and contact manifolds, we now consider
        Lorentz manifolds, which are special types of semi-Riemannian manifolds.
        Semi-Riemannian manifolds are ordered pairs $(M,\,g)$ where $M$ is an
        $n$ dimensional smooth manifolds and $g$ is a
        \textit{pseudo-Riemannian} metric, also called a semi-Riemannian metric.
        \par\hfill\par
        Pseudo-Riemannian metrics retain the symmetric bilinearity conditions,
        but relax \textit{positive-definiteness} to \textit{non-degeneracy}.
        That is, for all $p\in{M}$, and for all non-zero $v\in{T}_{p}M$, there
        is a $w\in{T}_{p}M$ such that $g_{p}(v,\,w)\ne{0}$.
    \end{frame}
    \begin{frame}{Spacetimes}
        Positive-definiteness implies non-degeneracy (choose $w=v$), but not
        the other way around. The \textit{Minkowski metric} on
        $\mathbb{R}^{n+1}$ is defined by:
        \begin{equation}
            g=-\textrm{d}t^{2}+\sum_{k=1}^{n}\textrm{d}x_{k}^{2}
        \end{equation}
        $g$ is indeed a pseudo-Riemannian metric on $\mathbb{R}^{n+1}$, but not
        a Riemannian one.
        \par\hfill\par
        The set of \textit{null vectors} ($g_{p}(v,\,v)=0$) in Minkowsky space
        satisfies the equation of a cone:
        \begin{equation}
            \textrm{d}t=\sqrt{\sum_{k=1}^{n}\textrm{d}x_{k}^{2}}
        \end{equation}
    \end{frame}
    \begin{frame}{Spacetimes}
        Sylvester's theorem tells us that for any point $p\in{M}$ and for any
        chart $(\mathcal{U},\,\varphi)$ about $p$ the metric $g$ will be given
        by the same number of positive components and negative components.
        \par\hfill\par
        That is, there are fixed integers $(n_{+},\,n_{-})$ such that
        $n_{+}+n_{-}=n$, and $g$ contains $n_{+}$ positive parts and $n_{-}$
        negative parts in any representation of it. This is the
        \textit{signature} of the metric.
        \par\hfill\par
        A Lorentz manifold is a
        semi-Riemannian manifold $(X,\,g)$ of dimension $n+1$
        with signature $(n,\,1)$.%
        \footnote{%
            Riemannian manifolds have signature $(n,\,0)$.
        }
    \end{frame}
    \begin{frame}{Spacetimes}
        At each point in a Lorentz manifold $(X,\,g)$ the set of null vectors
        is given by a cone.
        \par\hfill\par
        For $n\geq{2}$ this splits $T_{p}M$ into three
        parts: A connected open subset (the exterior of the cone), a
        disconnected open subset (the interior of the cone),
        and the cone itself.
        \par\hfill\par
        A \textit{time-orientation} on a Lorentz manifold is smooth choice of
        one of the components of the interior part of the cone as one varies
        from point to point. This choice is called the \textit{future}
        direction.%
        \footnote{%
            Orientability and time-orientability are separate notions. All four
            combinations of orientable and time-orientable are possible on
            Lorentz manifolds.
        }
        \par\hfill\par
        A \textit{spacetime} is a time-oriented Lorentz manifold.
    \end{frame}
    \begin{frame}{Spacetimes}
        We interpret vectors $v\in{T}_{p}X$ with
        $g_{p}(v,\,v)\leq{0}$ as those that have speeds less than that of light.
        This interpretation becomes evident when we abuse notation a bit:
        \begin{align}
            -c^{2}\textrm{d}t^{2}+\sum_{k=1}^{n}\textrm{d}x_{k}^{2}
            &\leq{0}\\
            \Rightarrow
            \sqrt{\sum_{k=1}^{n}\textrm{d}x_{k}^{2}}
            &\leq{c}\textrm{dt}\\
            \Rightarrow
            \sqrt{\sum_{k=1}^{n}\big(\frac{\textrm{d}x_{k}}{\textrm{d}t}\big)^{2}}
            &\leq{c}\\
            \Rightarrow
            ||\mathbf{v}||\leq{c}
        \end{align}
        where $||\mathbf{v}||$ is meant to represent the speed of the particle
        and $c$ is the speed of light, taken to be 1 for convenience.
    \end{frame}
    \begin{frame}{Spacetimes}
        The future and past of a point $p$, denoted $I_{p}^{+}$ and $I_{p}^{-}$,
        respectively, are all points in $X$ that can be connected to $p$ by a
        curve whose speed is always bounded by the speed of light.
        \par\hfill\par
        \textit{Nice} spacetimes have the following properties:
        \begin{itemize}
            \item
                $I_{p}^{+}\cap{I}_{q}^{-}$ is compact for all $p,q\in{X}$.
            \item
                No time travel.
        \end{itemize}
        No time travel means there is no future-pointing curve $\gamma$ (not
        exceeding the speed of light) that starts at a point $p$ and later
        returns to $p$.
        \par\hfill\par
        These spacetimes are called \textit{globally hyperbolic}.
    \end{frame}
    \begin{frame}{Spacetimes}
        \begin{theorem}[Geroch, 1970]
            Globally hyperbolic spacetimes $(X,\,g)$ are homeomorphic to
            $M\times\mathbb{R}$ for some topological manifold $M$.
        \end{theorem}
        \begin{theorem}[Bernal-S\'{a}nchez, 2003]
            Globally hyperbolic spacetimes $(X,\,g)$ are diffeomorphic to
            $M\times\mathbb{R}$ for some smooth manifold $M$, and the
            restriction of $g$ to $M\times\{t\}$ can be made Riemannian for all
            $t\in\mathbb{R}$.
        \end{theorem}
        Morever, $M\times\{t\}$ is a \textit{Cauchy hypersurface}, a submanifold
        with the property that particles that at less-than-or-equal-to the
        speed of light will intersect $M$ at precisely one time moment $t_{0}$.
    \end{frame}
    \begin{frame}{Causality and Linking}
        The \textit{sky} of a point $p$ in a spacetime $(X,\,g)$ is the space
        of all light-rays passing through $p$.
        \par\hfill\par
        If $(X,\,g)$ is globally hyperbolic, the sky of a point $p$ can be given
        a more explicit topological and geometric structure. Let $M$ be a
        Cauchy surface so that $X$ is diffeomorphic to $M\times\mathbb{R}$ and
        let $t_{0}\in\mathbb{R}$ be the unique time such that
        $p\in{M}\times\{t_{0}\}$. Since $(M,\,g|_{M})$ is Riemannian, for any
        ray of light $\gamma$ passing through $p$ the linear functional
        $\phi:T_{p}M\rightarrow\mathbb{R}$ defined by:
        \begin{equation}
            \phi(v)=g_{p}\big(v,\,\dot{\gamma}(t)\big)
        \end{equation}
        is non-zero. It hence defines a point in $ST^{*}M$. The sky of a point
        can thus be identified with the fibre of $p$ under the canonical
        projection $\textrm{proj}_{M}:ST^{*}M\rightarrow{M}$.
    \end{frame}
    \begin{frame}{Causality and Linking}
        The skies of two points $p,q\in{X}$ can be used to answer questions
        about \textit{causality}. That is, whether or not it is possible for
        $p$ and $q$ to communicate with each other by sending messages that
        do not travel faster than light.
        \par\hfill\par
        In some sense the skies of $p$ and $q$ may be \textit{linked} in
        $ST^{*}M$.
    \end{frame}
\end{document}
