%-----------------------------------LICENSE------------------------------------%
%   This file is part of Mathematics-and-Physics.                              %
%                                                                              %
%   Mathematics-and-Physics is free software: you can redistribute it and/or   %
%   modify it under the terms of the GNU General Public License as             %
%   published by the Free Software Foundation, either version 3 of the         %
%   License, or (at your option) any later version.                            %
%                                                                              %
%   Mathematics-and-Physics is distributed in the hope that it will be useful, %
%   but WITHOUT ANY WARRANTY; without even the implied warranty of             %
%   MERCHANTABILITY or FITNESS FOR A PARTICULAR PURPOSE.  See the              %
%   GNU General Public License for more details.                               %
%                                                                              %
%   You should have received a copy of the GNU General Public License along    %
%   with Mathematics-and-Physics.  If not, see <https://www.gnu.org/licenses/>.%
%------------------------------------------------------------------------------%
%   Author:     Ryan Maguire                                                   %
%   Date:       December 6, 2024                                               %
%------------------------------------------------------------------------------%
\documentclass{beamer}
\usepackage{amsmath}
\graphicspath{{../images/}}
\title{Tait Graphs for Virtual Knots}
\author{Ryan Maguire}
\date{December 8, 2024}
\usenavigationsymbolstemplate{}
\setbeamertemplate{footline}[frame number]
\graphicspath{{../images/}}
\begin{document}
    \maketitle
    \begin{frame}{Outline}
        \begin{itemize}
            \item Tait Graphs for Classical Knots.
            \item What are Virtual Knots?
            \item Tait Graphs for Virtual Knots.
            \item The Jones Polynomial via Tait Graphs.
        \end{itemize}
        This talk is mostly pictures, I hope you enjoy.
    \end{frame}
    \begin{frame}{Tait Graphs for Classical Knots}
        We start with a link diagram (oriented or unoriented), like
        Fig.~\ref{fig:hopf_link_diagram}.
        \begin{figure}
            \centering
            \includegraphics{hopf_link_diagram}
            \label{fig:hopf_link_diagram}
            \caption{An Unoriented Hopf Link}
        \end{figure}
    \end{frame}
    \begin{frame}{Tait Graphs for Classical Knots}
        Pick a random face and place a dot in the middle. For each face that
        is connected to your initial one by going \textit{across} (no going
        left, and no going right) a crossing, place a dot in middle of it as
        well. Keep doing this recursively until there are no new faces to
        put dots in.
        \begin{figure}
            \centering
            \includegraphics{hopf_link_unoriented_tait_graph_001}
            \label{fig:hopf_link_unoriented_tait_graph_001}
            \caption{A Hopf Link with Dots in the Faces}
        \end{figure}
    \end{frame}
    \begin{frame}{Tait Graphs for Classical Knots}
        Every face that has a dot in it, shade the entire region gray, and
        connect any two adjacent dots with an edge, an associate a sign to this
        edge based on the drawings below.
        \begin{figure}
            \centering
            \includegraphics{tait_graph_sign}
            \label{fig:tait_graph_sign}
            \caption{Signing the Edges of the Tait Graph}
        \end{figure}
    \end{frame}
    \begin{frame}{Tait Graphs for Classical Knots}
        To make things more visual, color the negative edges red, and the
        positive edges blue.
        \begin{figure}
            \centering
            \includegraphics{hopf_link_unoriented_tait_graph_002}
            \label{fig:hopf_link_unoriented_tait_graph_002}
            \caption{Tait Graph for the Hopf Link}
        \end{figure}
    \end{frame}
    \begin{frame}{Tait Graphs for Classical Knots}
        Remove the link diagram, and viola, you have a Tait graph.
        \begin{figure}
            \centering
            \includegraphics{hopf_link_tait_graph_no_link_diagram_001}
            \label{fig:hopf_link_tait_graph_no_link_diagram_001}
            \caption{Tait Graph for the Hopf Link}
        \end{figure}
    \end{frame}
    \begin{frame}{Tait Graphs for Classical Knots}
        Remove the link diagram, and viola, you have a Tait graph.
        \begin{figure}
            \centering
            \includegraphics{hopf_link_tait_graph_no_link_diagram_001}
            \label{fig:hopf_link_tait_graph_no_link_diagram_001}
            \caption{Tait Graph for the Hopf Link}
        \end{figure}
    \end{frame}
    \begin{frame}{Tait Graphs for Classical Knots}
        Depending on your choice of initial face, you may get a different
        graph. There are two possibilities, and they are planar duals.
        \begin{figure}
            \centering
            \resizebox{!}{0.6\textheight}{%
                \includegraphics{%
                    trefoil_tait_graph_with_both_graphs_and_knot_diagram_001
                }
            }
            \label{fig:trefoil_tait_graph_with_both_graphs_and_knot_diagram_001}
            \caption{Tait Graphs for the Right-Handed Trefoil}
        \end{figure}
    \end{frame}
    \begin{frame}{The End}
        \begin{center}
            Thank you!
        \end{center}
    \end{frame}
\end{document}
