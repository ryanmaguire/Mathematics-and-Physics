%-----------------------------------LICENSE------------------------------------%
%   This file is part of Mathematics-and-Physics.                              %
%                                                                              %
%   Mathematics-and-Physics is free software: you can redistribute it and/or   %
%   modify it under the terms of the GNU General Public License as             %
%   published by the Free Software Foundation, either version 3 of the         %
%   License, or (at your option) any later version.                            %
%                                                                              %
%   Mathematics-and-Physics is distributed in the hope that it will be useful, %
%   but WITHOUT ANY WARRANTY; without even the implied warranty of             %
%   MERCHANTABILITY or FITNESS FOR A PARTICULAR PURPOSE.  See the              %
%   GNU General Public License for more details.                               %
%                                                                              %
%   You should have received a copy of the GNU General Public License along    %
%   with Mathematics-and-Physics.  If not, see <https://www.gnu.org/licenses/>.%
%------------------------------------------------------------------------------%
%   Author:     Ryan Maguire                                                   %
%   Date:       June 3, 2022                                                   %
%------------------------------------------------------------------------------%
\documentclass{beamer}
\usepackage{amsmath}

\title{Curvature and the Einstein Field Equations}
\author{Ryan Maguire}
\date{June 3, 2022}
\usenavigationsymbolstemplate{}
\setbeamertemplate{footline}[frame number]
\begin{document}
    \maketitle
    \begin{frame}
        Many of the notions of curvature require only a semi-Riemannian metric
        and a choice of affine connection.
        \par\hfill\par
        It is almost universal amongst physicists and mathematicians to work
        with the unique Levi-Civita connection that a given semi-Riemannian
        metric induces.
        \par\hfill\par
        We'll discuss several types of curvatures, and their uses in
        describing the Einstein field equations.
    \end{frame}
    \begin{frame}
        In physics it is common to work in a coordinate chart
        $(\mathcal{U},\varphi)$ and express all physical quantities in terms of
        this chart. The semi-Riemannian metric $g$ becomes a matrix
        $g_{\mu\nu}$ with entries
        $g_{\mu\nu}=g(\partial\varphi_{\mu},\partial\varphi_{\nu})$, which is
        called the \textit{metric tensor} in general relativity. Other tensors
        and tensor fields will be described similarly.
    \end{frame}
    \begin{frame}
        The first tensor to describe is the stress-energy tensor $T_{\mu\nu}$.
        It is the gravitational analogue of the stress tensor from Newtonian
        mechanics and describes the density and flux of energy in the
        manifold $(M,g)$, which is always chosen to be Lorentzian.
        \par\hfill\par
        The Einstein field equations relate the stress-energy tensor and the
        metric tensor to Ricci curvature and scalar curvature.
    \end{frame}
    \begin{frame}
        The Ricci curvature is described in terms of the Riemann curvature
        tensor field (It's a tensor field, not a tensor). Given the affine
        connection $\nabla$ on the semi-Riemannian manifold, the Riemann
        curvature tensor field is defined in one of two equivalent ways.
        It is a function $R:\mathfrak{X}(M)^{3}\rightarrow\mathfrak{X}(M)$
        \begin{equation}
            R(X,Y)Z=
                \nabla_{X}\nabla_{Y}Z-\nabla_{Y}\nabla_{X}Z-\nabla_{[X,Y]}Z
        \end{equation}
        Where $[X,Y]$ is the Lie bracket. We can also write this as:
        \begin{equation}
            R(X,Y)=[\nabla_{X},\nabla_{Y}]-\nabla_{[X,Y]}
        \end{equation}
        again using the Lie bracket. With this we see that the Riemann
        curvature tensor field measures the failure of the second derivative to
        commute.
    \end{frame}
    \begin{frame}
        If $\nabla$ is a Levi-Civita connection (torsion free and compatible
        with the metric), then there are several identities the Riemann
        curvature tensor field enjoys. These identities can be combined with
        the Einstein field equations to prove the local conservation of
        energy and momentum, classical laws of Newtonian mechanics which still
        hold in general relativity.
        \begin{itemize}
            \item $R$ is trilinear over $C^{\infty}(M,\mathbb{R})$.
            \item The Bianchi identity holds:
                \begin{equation}
                    R(X,Y)Z+R(Y,Z)X+R(Z,X)Y=0
                \end{equation}
        \end{itemize}
        The Bianchi identity cyclicly permutes the vector fields. It is the
        Bianchi identity that helps one prove conservation of momentum and
        energy.
    \end{frame}
    \begin{frame}
        The quadruple product relates the Riemann curvature tensor field to the
        semi-Riemannian metric. It is defined as:
        \begin{equation}
            (X,Y,Z,T)=g\big(R(X,Y)Z,T\big)
        \end{equation}
        There are several identities for this operation, which are again useful
        for the proof of various theorems in the framework of general
        relativity.
        \begin{align}
            (X,Y,Z,T)&=-(Y,X,Z,T)\\
            (X,Y,Z,T)&=-(X,Y,T,Z)\\
            (X,Y,Z,T)&=(Z,T,X,Y)
        \end{align}
        Lastly, an analogue of the Bianchi identity:
        \begin{equation}
            (X,Y,Z,T)+(Y,Z,X,T)+(Z,X,Y,T)=0
        \end{equation}
    \end{frame}
    \begin{frame}
        These identities combine to give the following theorem.
        \begin{theorem}
            If $(\mathcal{U},\varphi)$ is a chart in a spacetime $(M,g)$,
            if $\nabla$ is the unique Levi-Civita connection on $M$, and if
            $T$ is the stress-energy tensor, then:
            \begin{equation}
                \sum_{n=0}^{N-1}\nabla_{\partial\varphi_{n}}T_{n,m}=0
            \end{equation}
        \end{theorem}
        This is the analogue of the conservation of momentum and energy laws
        that occur in Newtonian mechanics. The proof is about a page and simply
        uses the identities of the Riemannian curvature tensor field, the
        quadruple product, and the Einstein field equations which will be
        stated soon.
    \end{frame}
    \begin{frame}
        The Einstein field equations relate the stress-energy tensor to the
        Ricci and scalar curvatures. The Ricci curvature is defined in terms of
        the Riemann curvature tensor field. There are two ways of doing this.
        \par\hfill\par
        In the Riemann setting ($g$ is positive-definite), fix $p\in{M}$ and
        $x=z_{n}\in{T}_{p}M$ to be unit length. Since $T_{p}M$ is an $n$
        dimensional real inner product space, we may extend $z_{n}$ via the
        Gram-Schmidt procedure to an orthonormal basis. Label these other
        elements $z_{1},\dots,z_{n-1}$. The Ricci curvature about $p$ is defined
        as:
        \begin{equation}
            \textrm{Ric}_{p}(x)=\frac{1}{n-1}\sum_{k=1}^{n}g_{p}
                \Big(R(x,z_{k})x,z_{k}\Big)
        \end{equation}
        It is a theorem that this result is independent of the choice of
        basis.
    \end{frame}
    \begin{frame}
        In the semi-Riemannian setting $T_{p}M$ is not an inner product space
        since $g$ can, in general, fail to be positive definite. Such is the
        case in spacetimes with signature $(+,+,+,-)$. Fix two vector fields
        $Y$ and $Z$. Given a vector field $X$, the mapping
        $X\mapsto{R}(X,Y)Z$ is linear at each tangent space. Because of this
        one may define the \textit{trace} of this mapping. This is the Ricci
        curvature tensor.
        \begin{equation}
            \textrm{Ric}_{p}(Y,Z)=\textrm{tr}
                \big(X_{p}\mapsto{R}_{p}(X_{p},Y_{p})Z_{p}\big)
        \end{equation}
        In local coordinates $(\mathcal{U},\varphi)$ it can be given by a
        matrix $R_{\mu\nu}$.
    \end{frame}
    \begin{frame}
        The Ricci curvature can be completely described by the sectional
        curvature, which is one of the older notions of curvature dating back
        to a time when differential geometry dealt solely with regular surfaces
        and curves. The sectional curvature of a 2-dimensional subspace
        $\delta$ of the tangent space $T_{p}M$ is given by:
        \begin{equation}
            K_{\delta}=\frac{(v,w,v,w)}{A(v,w)}
                =\frac{g_{p}\big(R(v,w)v,w\big)}{\sqrt{||v||^{2}\,||w||^{2}-g_{p}(v,w)^{2}}}
        \end{equation}
        where $v$ and $w$ are two tangent vectors that span $\delta$, and
        $A(v,w)$ is the area of the parallelogram with sides $v$ and $w$.
        $K_{\delta}$ is independent of choice of basis since a change of basis
        can be made by a combination of moves
        $(x,y)\mapsto(y,x)$, $(x,y)\mapsto(\lambda{x},y)$, $\lambda\ne{0}$, and
        $(x,y)\mapsto(x+\lambda{y},y)$. These operations are reflection,
        scaling, and shearing, respectively. All of these are invariant under
        formula above showing $K_{\delta}$ is independent of basis.
    \end{frame}
    \begin{frame}
        For constant curvature manifolds the Ricci curvature is given by a
        simple formula:
        \begin{equation}
            R_{\mu\nu}=(n-1)Kg_{\mu\nu}
        \end{equation}
        where $K$ is the constant curvature of the manifold.
        It is probably not the case that the spacetime we live in is constant
        curvature.
    \end{frame}
    \begin{frame}
        The scalar curvature is defined directly by the Ricci curvature. Given
        the Riemannian definition, $\textrm{Ric}_{p}(x)$, given a basis
        $\{z_{1},\dots,z_{n}\}$ of $T_{p}M$, the scalar curvature is defined by:
        \begin{equation}
            K(p)=\frac{1}{n}\sum_{k=1}^{n}\textrm{Ric}_{p}(z_{k})
        \end{equation}
        It is independent of choice of basis. With respect to the second
        definition, we can define:
        \begin{equation}
            K(p)=\textrm{tr}(R_{\mu\nu})
        \end{equation}
    \end{frame}
    \begin{frame}
        The Einstein tensor is defined in terms of the Ricci and scalar tensors.
        We have:
        \begin{equation}
            G_{\mu\nu}=R_{\mu\nu}-\frac{1}{2}Kg_{\mu\nu}
        \end{equation}
        Where $R_{\mu\nu}$ is the Ricci tensor, $K$ is the scalar curvature,
        and $g_{\mu\nu}$ is the metric tensor. The Einstein field equations are:
        \begin{equation}
            G_{\mu\nu}+\Lambda{g}_{\mu\nu}=\kappa{T}_{\mu\nu}
        \end{equation}
        Where $T_{\mu\nu}$ is the stress-energy tensor. $\Lambda$ is the
        cosmological constant, and $\kappa$ is the Einstein gravitational
        constant.
    \end{frame}
    \begin{frame}
        In practice, one measures the stress-energy tensor and the Einstein
        tensor and wishes to solve for the metric in the Einstein field
        equation. A common simplification is to suppose the spacetime you are
        working in is a vacuum containing no mass-energy. The Einstein field
        equations simplify to:
        \begin{equation}
            G_{\mu\nu}+\Lambda{g}_{\mu\nu}=0
        \end{equation}
        Expanding the Einstein tensor in terms of the Ricci and scalar
        curvature, we get:
        \begin{equation}
            R_{\mu\nu}-\frac{1}{2}Kg_{\mu\nu}+\Lambda{g}_{\mu\nu}=0
        \end{equation}
    \end{frame}
    \begin{frame}
        This is a purely geometrical problem. Depending on the value of
        $\Lambda$ there are several known spacetimes with metrics that satisfy
        the Einstein field equations.
        \begin{itemize}
            \item Minkowski spacetime $\mathbb{M}^{3,1}$
            \item Milne spacetime
            \item Schwarzschild vacuum spacetime
            \item Kerr vacuum
        \end{itemize}
        The value of $\Lambda$ was originally thought to be zero, and Einstein
        retracted it from the equation. In the late 1990's it was discovered
        the inflation of the universe is accelerating, indicating the constant
        may be positive. One possible value involves the Hubble constant, given
        by:
        \begin{equation}
            \Lambda=1.1056\times{10}^{-52}\,\textrm{m}^{-2}
        \end{equation}
    \end{frame}
\end{document}
