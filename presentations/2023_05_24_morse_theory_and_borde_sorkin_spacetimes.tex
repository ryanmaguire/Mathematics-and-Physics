%-----------------------------------LICENSE------------------------------------%
%   This file is part of Mathematics-and-Physics.                              %
%                                                                              %
%   Mathematics-and-Physics is free software: you can redistribute it and/or   %
%   modify it under the terms of the GNU General Public License as             %
%   published by the Free Software Foundation, either version 3 of the         %
%   License, or (at your option) any later version.                            %
%                                                                              %
%   Mathematics-and-Physics is distributed in the hope that it will be useful, %
%   but WITHOUT ANY WARRANTY; without even the implied warranty of             %
%   MERCHANTABILITY or FITNESS FOR A PARTICULAR PURPOSE.  See the              %
%   GNU General Public License for more details.                               %
%                                                                              %
%   You should have received a copy of the GNU General Public License along    %
%   with Mathematics-and-Physics.  If not, see <https://www.gnu.org/licenses/>.%
%------------------------------------------------------------------------------%
%   Author:     Ryan Maguire                                                   %
%   Date:       April 27, 2023                                                 %
%------------------------------------------------------------------------------%
\documentclass{beamer}
\usepackage{amsmath}
\title{Morse Theory and Borde-Sorkin Spacetimes}
\author{Ryan Maguire}
\date{May 24, 2023}
\usenavigationsymbolstemplate{}
\setbeamertemplate{footline}[frame number]
\newtheorem{conjecture}{Conjecture}
\begin{document}
    \maketitle
    \begin{frame}{Outline}
        \begin{itemize}
            \item Morse Theory
            \item Borde-Sorkin Spacetimes
            \item The Borde-Sorkin Conjecture
        \end{itemize}
    \end{frame}
    \begin{frame}{Motivation}
        Since the late 90s some physicists and mathematicians have been
        exploring spacetimes with \textit{broken} parts, isolated points with
        degenerate metrics, and changing spacial topologies.
        \par\hfill\par
        Many have also rejected this since they often have bad topological
        properties. Borde and Sorkin proposed a resolution to this issue by
        considering topology-changing spacetimes that are constructed via
        \textit{Morse functions}.
        \par\hfill\par
        These \textit{Borde-Sorkin} spacetimes may be necessary in the
        development of a consistent theory of quantum gravity. Borde and Sorkin
        have conjectured that these spacetimes are \textit{causally continuous},
        so long as the index of the Morse points is not 1 or $n-1$.
    \end{frame}
    \begin{frame}{Morse Theory}
        Marston Morse (C.E. 1892 - 1977) observed that much of the topology of
        a smooth manifold $M$ can be observed from a general smooth function
        $f:M\rightarrow\mathbb{R}$. For smooth surfaces $S$ embedded in
        $\mathbb{R}^{3}$ the \textit{elevation} function serves as a good
        example.
        \par\hfill\par
        Given $(x,\,y,\,z)\in{S}$ define $f(x,\,y,\,z)=z$. The fiber of a real
        number $r$ yields \textit{contour lines} on the surface. You may also
        find isolated points and \textbf{X}'s (double points). Higher order
        points are possible, like triple points where the contour lines meet
        like an asterisk \textbf{*}, but small perturbations in $f$ will
        remove this.
    \end{frame}
    \begin{frame}{Morse Theory}
        If $\mathbf{p}=(x,\,y,\,z)\in{S}$ is not a critical point, meaning the
        gradient $\nabla(f)_{\mathbf{p}}$ is non-zero, and if $r=f(\mathbf{p})$,
        then it seems intuitive, at least by experiment, that
        $f^{-1}\big[(-\infty,\,r)\big]$ and
        $f^{-1}\big[(-\infty,\,r+\varepsilon)\big]$ are homeomorphic for small
        enough $\varepsilon$. That is, the topology does not change until one
        reaches critical points. The elevation function on the standard
        embedding of $\mathbb{T}^{2}$ into $\mathbb{R}^{3}$ serves as the
        quintessential example.
    \end{frame}
    \begin{frame}{Morse Theory}
        The \textit{index} of a non-degenerate (non-singular Hessian matrix)
        critial point $\mathbf{p}$ is the maximum number of independent
        directions in which the value of $f$ \textit{decreases}
        from $\mathbf{p}$.
        \par\hfill\par
        We may define this for more general manifolds by saying the index is
        the maximum value possible for the dimension of a subspace of
        $T_{\mathbf{p}}S$ such that the Hessian is negative definite
        ($vHv^{T}\leq{0}$ for all $v\in{T}_{\mathbf{p}}S$ and
        $vHv^{T}=0$ if and only if $v=\mathbf{0}$). By Sylvester's Law of
        Inertia this value is independent of chart chosen for $\mathbf{p}$.
    \end{frame}
    \begin{frame}{Morse Theory}
        A Morse function on a smooth manifold $M$ is a smooth function
        $f:M\rightarrow\mathbb{R}$ that has no degenerate critical points.
        \begin{theorem}
            If $M$ is a smooth manifold, then there exists a Morse function
            $f:M\rightarrow\mathbb{R}$.
        \end{theorem}
        \begin{theorem}
            If $M$ is a smooth manifold, then most functions
            $f\in{C}^{\infty}(M,\,\mathbb{R})$ are Morse functions.
            That is, the set of Morse functions is open and dense in the
            $C^{2}$ topology on $C^{\infty}(M,\,\mathbb{R})$.
        \end{theorem}
    \end{frame}
    \begin{frame}{Borde-Sorkin Spacetimes}
        Spacetimes are Lorentz manifolds (signature $(n,\,1)$) with a chosen
        time orientation. At no points in the spacetime is the metric allowed
        to deviate from being non-degenerate, nor may the signature change.
        Sorkin proposed the following construction in 1989.
        \par\hfill\par
        Let $\mathcal{M}$ be a compact cobordism of dimension $n$
        (a smooth manifold with boundary such that $\partial\mathcal{M}$ is the
        disjoint union of two closed $n-1$ dimensional smooth manifolds). Let
        $h$ be a Riemannian metric on $\mathcal{M}$, $\zeta>1$ a constant, and
        $f:\mathcal{M}\rightarrow\mathbb{R}$ a Morse function
        (the notation here follows Garcia-Heveling, 2022).
    \end{frame}
    \begin{frame}{Borde-Sorkin Spacetimes}
        By the Morse lemma, at a given critical point $p$ of $f$ there is a
        chart $(\mathcal{U},\,\varphi)$ such that:
        \begin{equation}
    	    f(x)=\sum_{k=1}^{n}a_{k}\varphi_{k}(x)^{2}
        \end{equation}
        for all $x\in\mathcal{U}$ where $a_{k}\ne{0}$ for each $k$. It follows
        that critical points of a Morse function are topologically isolated.
    \end{frame}
    \begin{frame}{Borde-Sorkin Spacetimes}
        The \textit{Morse Metric} of $(\mathcal{M},\,h,\,f,\,\zeta)$ is
        defined on $\mathcal{M}\times\mathbb{R}$ via:
        \begin{equation}
            g=||\textrm{d}f||^{2}_{h}h-\zeta\textrm{d}f^{2}
        \end{equation}
        We let $M=\textrm{Int}(\mathcal{M})$, and
        abuse notation by endowing $M$ with the subspace metric
        $g|_{M}$ and labelling this $g$. At non-critical points of $f$ we have
        a spacetime. At the critical points $\textrm{d}f$ is zero, we see that
        $g=0-0=0$, a degenerate metric.
    \end{frame}
    \begin{frame}{Borde-Sorkin Spacetimes}
        Note that, for non-critical points $p\in{M}$, $(M,\,g)$ is somewhat
        \textit{locally} globally hyperbolic (locally globally is a strange
        phrase). That is, locally $(M,\,g)$ has the structure of a Riemann
        manifold crossed with $\mathbb{R}$,
        $M$ acting locally as a Cauchy surface.
        \par\hfill\par
        Borde-Sorkin spacetimes can intuitively be thought of as
        globally hyperbolic spacetimes \textit{glued} together at the
        critical points of the Morse function.
    \end{frame}
    \begin{frame}{Borde-Sorkin Conjecture}
        The spacetimes were invented to allow topology-changing
        without the poor properties such features usually exhibit. It has been
        conjectured that the pathologies arise with casually discontinuous
        spaces.
        \par\hfill\par
        Following Hawking and Sachs, 1973, casually continuous spacetimes are,
        roughly speaking, those in which the future and past of an observer
        vary in a well behaved manner under continuous perturbations to the
        metric $g$ or to the position of the observer.
        \par\hfill\par
        \begin{conjecture}[Borde-Sorkin]
            The Morse spacetime $(M,\,g)$ is casually continuous if and only
            if the index of all critical points of $f$ is different than
            $1$ and $n-1$.
        \end{conjecture}
    \end{frame}
    \begin{frame}{Borde-Sorkin Conjecture}
        \begin{theorem}[Garcia-Heveling, 2022]
            If $(M,\,g)$ is a Borde-Sorkin spacetime coming from a Morse
            geometry $(\mathcal{M},\,h,\,f,\,\zeta)$, if $f$ has one critical
            point $p_{c}$, and if there is a chart $(\mathcal{U},\,\varphi)$
            containing $p_{c}$ such that:
            \begin{align}
                f(x)&=\sum_{k=1}^{n}a_{k}\varphi_{k}(x)^{2}\\
                h&=\sum_{k=1}^{n}\textrm{d}\varphi_{k}^{2}
            \end{align}
            where $a_{k}\ne{0}$ for all $k$, and where:
            \begin{equation}
                \frac{1}{\zeta}<\Big|\frac{a_{k}}{a_{j}}\Big|<\zeta\quad\textrm{and}\quad
                \frac{5}{8}\leq\Big|\frac{a_{k}}{a_{j}}\Big|\leq\frac{8}{5}
            \end{equation}
        \end{theorem}
        For all $a_{k},a_{j}$, then $(M,g)$ is causally continuous.
    \end{frame}
\end{document}
