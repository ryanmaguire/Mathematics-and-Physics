%-----------------------------------LICENSE------------------------------------%
%   This file is part of Mathematics-and-Physics.                              %
%                                                                              %
%   Mathematics-and-Physics is free software: you can redistribute it and/or   %
%   modify it under the terms of the GNU General Public License as             %
%   published by the Free Software Foundation, either version 3 of the         %
%   License, or (at your option) any later version.                            %
%                                                                              %
%   Mathematics-and-Physics is distributed in the hope that it will be useful, %
%   but WITHOUT ANY WARRANTY; without even the implied warranty of             %
%   MERCHANTABILITY or FITNESS FOR A PARTICULAR PURPOSE.  See the              %
%   GNU General Public License for more details.                               %
%                                                                              %
%   You should have received a copy of the GNU General Public License along    %
%   with Mathematics-and-Physics.  If not, see <https://www.gnu.org/licenses/>.%
%------------------------------------------------------------------------------%
%   Author:     Ryan Maguire                                                   %
%   Date:       April 27, 2023                                                 %
%------------------------------------------------------------------------------%
\documentclass{beamer}
\usepackage{amsmath}
\title{Morse Theory and Borde-Sorkin Spacetimes}
\author{Ryan Maguire}
\date{May 24, 2023}
\usenavigationsymbolstemplate{}
\setbeamertemplate{footline}[frame number]
\begin{document}
    \maketitle
    \begin{frame}{Outline}
        \begin{itemize}
            \item Morse Theory
            \item Borde-Sorkin Spacetimes
            \item The Borde-Sorkin Conjecture
        \end{itemize}
    \end{frame}
    \begin{frame}{Motivation}
        Since the late 90s some physicists and mathematicians have been
        exploring spacetimes with \textit{broken} parts, isolated points with
        degenerate metrics, and changing spacial topologies.
        \par\hfill\par
        Many have also rejected this since they often have bad topological
        properties. Borde and Sorkin proposed a resolution to this issue by
        considering topology-changing spacetimes that are constructed via
        \textit{Morse functions}.
        \par\hfill\par
        These \textit{Borde-Sorkin} spacetimes may be necessary in the
        development of a consistent theory of quantum gravity. Borde and Sorkin
        have conjectured that these spacetimes are \textit{causally continuous},
        so long as the index of the Morse points is not 1 or $n-1$.
    \end{frame}
    \begin{frame}{Morse Theory}
        Marston Morse (C.E. 1892 - 1977) observed that much of the topology of
        a smooth manifold $M$ can be observed from a general smooth function
        $f:M\rightarrow\mathbb{R}$. For smooth surfaces $S$ embedded in
        $\mathbb{R}^{3}$ the \textit{elevation} function serves as a good
        example.
        \par\hfill\par
        Given $(x,\,y,\,z)\in{S}$ define $f(x,\,y,\,z)=z$. The fiber of a real
        number $r$ yields \textit{contour lines} on the surface. You may also
        find isolated points and \textbf{X}'s (double points). Higher order
        points are possible, like triple points where the contour lines meet
        like an asterisk \textbf{*}, but small perturbations in $f$ will
        remove this.
    \end{frame}
    \begin{frame}{Morse Theory}
        If $\mathbf{p}=(x,\,y,\,z)\in{S}$ is not a critical point, meaning the
        gradient $\nabla{f}(\mathbf{P})$ is non-zero, and if $r=f(\mathbf{p})$
        then it seems intuitive, at least by experiment, that
        $f^{-1}\big[(-\infty,\,a)\big]$ and
        $f^{-1}\big[(-\infty,\,a+\varepsilon)\big]$ are homeomorphic for small
        enough $\varepsilon$. That is, the topology does not change until one
        reaches critical points. The elevation function on the standard
        embedding of $\mathbb{T}^{2}$ into $\mathbb{R}^{3}$ serves as the
        quintessential example.
    \end{frame}
    \begin{frame}{Morse Theory}
        The \textit{index} of a non-degenerate (non-singular Hessian matrix)
        critial point $\mathbf{p}$ is the maximum number of independent
        directions in which the value of $f$ \textit{decreases}
        from $\mathbf{p}$.
        \par\hfill\par
        We may define this for more general manifolds by saying the index is
        the maximum value possible for the dimension of a subspace of
        $T_{\mathbf{p}}S$ such that the Hessian is negative definite
        ($vHv^{T}\leq{0}$ for all $v\in{T}_{\mathbf{p}}S$ and
        $vHv^{T}=0$ if and only if $v=\mathbf{0}$). By Sylvester's Law of
        Inertia this value is independent of chart chosen for $\mathbf{p}$.
    \end{frame}
    \begin{frame}{Morse Theory}
        A Morse function on a smooth manifold $M$ is a smooth function
        $f:M\rightarrow\mathbb{R}$ that has no degenerate critical points.
        \begin{theorem}
            If $M$ is a smooth manifold, then there exists a Morse function
            $f:M\rightarrow\mathbb{R}$.
        \end{theorem}
        \begin{theorem}
            If $M$ is a smooth manifold, then most functions
            $f\in{C}^{\infty}(M,\,\mathbb{R})$ are Morse functions.
            That is, the set of Morse functions is open and dense in the
            $C^{2}$ topology on $C^{\infty}(M,\,\mathbb{R})$.
        \end{theorem}
    \end{frame}
\end{document}
