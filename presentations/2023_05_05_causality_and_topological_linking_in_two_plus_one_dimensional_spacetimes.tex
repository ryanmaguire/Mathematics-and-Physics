%-----------------------------------LICENSE------------------------------------%
%   This file is part of Mathematics-and-Physics.                              %
%                                                                              %
%   Mathematics-and-Physics is free software: you can redistribute it and/or   %
%   modify it under the terms of the GNU General Public License as             %
%   published by the Free Software Foundation, either version 3 of the         %
%   License, or (at your option) any later version.                            %
%                                                                              %
%   Mathematics-and-Physics is distributed in the hope that it will be useful, %
%   but WITHOUT ANY WARRANTY; without even the implied warranty of             %
%   MERCHANTABILITY or FITNESS FOR A PARTICULAR PURPOSE.  See the              %
%   GNU General Public License for more details.                               %
%                                                                              %
%   You should have received a copy of the GNU General Public License along    %
%   with Mathematics-and-Physics.  If not, see <https://www.gnu.org/licenses/>.%
%------------------------------------------------------------------------------%
%   Author:     Ryan Maguire                                                   %
%   Date:       April 27, 2023                                                 %
%------------------------------------------------------------------------------%
\documentclass{beamer}
\usepackage{amsmath}
\title{Causality and Topological Linking in Two Plus One Dimensional Spacetimes}
\author{Ryan Maguire}
\date{May 5, 2023}
\usenavigationsymbolstemplate{}
\setbeamertemplate{footline}[frame number]
\begin{document}
    \maketitle
    \begin{frame}{Outline}
        \begin{itemize}
            \item Spacetimes and causality.
            \item Cotangent bundles and spherical cotangent bundles.
            \item Causality in globally hyperbolic $2+1$ dimensional spacetimes.
        \end{itemize}
    \end{frame}
    \begin{frame}{Spacetimes and Causality}
        An $n+1$ dimensional spacetime is a Lorentz manifold $(M,\,g)$
        (a semi-Riemannian manifold with signature $(n,\,1)$, or $(1,\,n)$ in
        some physics communities) with a chosen time orientation. That is, at
        each point a future direction is chosen and this choice is done
        continuously.
        \par\hfill\par
        A time-like curve is a differentiable curve $\gamma$ such that
        $g\big(\dot{\gamma}(t),\,\dot{\gamma}(t)\big)<0$ for all $t$. A
        light-like curve is one such that
        $g\big(\dot{\gamma}(t),\,\dot{\gamma}(t)\big)=0$ for all $t$. Lastly,
        causal curve satisfy
        $g\big(\dot{\gamma}(t),\,\dot{\gamma}(t)\big)\leq{0}$.
    \end{frame}
    \begin{frame}{Spacetimes and Causality}
        Causal curves represent the transmittance of real data since we may not
        exceed the speed of light. To see this, given a point $p\in{M}$ we may
        find a chart $(\mathcal{U},\,\varphi)$ with $p\in\mathcal{U}$ such that:
        \begin{equation}
            g=-\textrm{d}t^{2}+\sum_{k=1}^{n}\textrm{d}x_{k}^{2}
        \end{equation}
        where $\textrm{d}x_{k}=\textrm{d}\varphi_{k}$ and
        $\textrm{d}t=\textrm{d}\varphi_{n+1}$. If $\gamma$ is causal, this says:
        \begin{align}
            \sum_{k=1}^{n}\textrm{d}x_{k}^{2}(\dot{\gamma})
                &\leq\textrm{d}t^{2}(\dot{\gamma})\\
            \Rightarrow
            \sum_{k=1}^{n}\frac{\textrm{d}x_{k}^{2}}{\textrm{d}t^{2}}&\leq{1}
        \end{align}
    \end{frame}
    \begin{frame}{Spacetimes and Causality}
        The sum of the squares of the components is the square of the norm of
        the velocity vector, so the square of the speed.
        \par\hfill\par
        In natural units one takes the speed of light to be $c=1$, so this
        final inequality states that the speed of the curve never exceeds that
        of light.
        \par\hfill\par
        Causally related points in a spacetime $(M,\,g)$ are those that can be
        connected by a causal curve.
    \end{frame}
    \begin{frame}{Spacetimes and Causality}
        There are lots of spacetimes one can ponder, most of which are
        not physically relevant (but perhaps still fun to think about). Two
        reasonable restrictions are often placed on our manifolds.
        \begin{itemize}
            \item There is no time travel.
            \item Given two points $p$ and $q$, the intersection of the causal
                  future $J_{p}^{+}$ and causal past $J_{q}^{-}$ is compact.
        \end{itemize}
        The set $J_{p}^{+}$ is the set of all points in $M$ that can be reached
        by a causal future directed curve from $p$. Similarly $J_{q}^{-}$ is
        the set of points that can be reached by causal past directed curves
        from $q$. Such a spacetime is called \textit{globally hyperbolic}.
    \end{frame}
    \begin{frame}{Spacetimes and Causality}
        The structure of globally hyperbolic spacetimes is well understood.
        \begin{theorem}[Geroch's Splitting Theorem, 1979]
            If $(M,\,g)$ is a globally hyperbolic spacetime, then $M$ is
            homeomorphic to $S\times\mathbb{R}$ where $S$ is a Cauchy surface,
            a hypersurface such that every inextensible light-like geodesic
            intersects $S$ exactly once.
        \end{theorem}
        This is a \textit{topological} theorem and does not give us any smooth
        or geometrical information, but it is useful nonetheless. It seriously
        restricts the possible structure of globally hyperbolic spacetimes.
    \end{frame}
    \begin{frame}{Spacetimes and Causality}
        A strenghthening of this theorem exists.
        \begin{theorem}[Bernal, A. and Sanchez, M., 2003]
            If $(M,\,g)$ is a globally hyperbolic spacetime, then there is a
            Riemannian Cauchy surface $S$ (a Cauchy surface such that the
            restriction of $g$ to $S$ is a Riemannian metric) such that
            $M$ is diffeomorphic to $S\times\mathbb{R}$.
        \end{theorem}
    \end{frame}
    \begin{frame}{Spherical Cotangent Bundles}
        This talk aims to discuss the Topological Low Conjecture
        for $2+1$ dimensional globally hyperbolic spacetimes. Moreover,
        spacetimes where the Cauchy surface $S$ has a universal cover
        diffeomorphic to $\mathbb{R}^{2}$.
        \par\hfill\par
        To do this requires the notion of linking in the spherical cotangent
        bundle of a Cauchy surface. The cotangent bundle $T^{*}M$ is
        constructed in a similar manner to the tangent bundle $TM$ using
        local trivialization and smoothly \textit{gluing} copies of
        $\mathbb{R}^{n}$ to each point in $M^{n}$. By removing the zero section
        we may quotient by the action of multiplication by positive real numbers
        and obtain the \textit{spherical cotangent bundle}
        $ST^{*}M$.
    \end{frame}
    \begin{frame}{Spherical Cotangent Bundles}
        Given a smooth manifold (no Riemannian or semi-Riemannian metric needed)
        there is a standard method of inventing a symplectic form on
        $T^{*}M$ which restricts to a contact form on $ST^{*}M$. The
        construction is made quite explicit with the existence of a
        Riemannian metric $g$.
        \par\hfill\par
        The metric induces a map $\tilde{g}:TM\rightarrow{T}^{*}M$ as follows.
        Given a vector field $X\in\mathfrak{X}(M)$ we define:
        \begin{equation}
            \tilde{g}(X)(p,\,v)=g(X_{p},\,v)
        \end{equation}
        This is a one-form, at each point $p$ it takes in tangent vectors and
        returns real numbers, and since $g$ is a Riemannian metric this varies
        smoothly. Thus $\tilde{g}$ maps vector fields to one-forms so it is
        a function $\tilde{g}:TM\rightarrow{T}^{*}M$.
    \end{frame}
    \begin{frame}{Spherical Cotangent Bundles}
        Locally in some chart $(\mathcal{U},\,\varphi)$ we may represent
        $\tilde{g}$ as a matrix with components $\tilde{g}_{i,\,j}$. The
        Liouville form $\Omega$ is then:
        \begin{equation}
            \Omega=\sum_{i=1}^{n}\sum_{j=1}^{n}
                \tilde{g}_{i,\,j}\textrm{d}x_{i}\land\textrm{d}v_{j}+
            \sum_{k=1}^{n}\sum_{i=1}^{n}\sum_{j=1}^{n}
                \frac{\partial\tilde{g}_{i,\,j}}{\partial{x}_{k}}
                v_{i}\textrm{d}x_{j}\land\textrm{d}x_{k}
        \end{equation}
        where we represent $\varphi$ by
        $(x_{1},\,\dots,\,x_{n},\,v_{1},\,\dots,\,v_{n})$. The restriction of
        this to $ST^{*}M$ yields a contact structure. Note since we have a
        Riemannian metric $STM$ lives as a subspace of $TM$ by
        considering points $(p,\,v)$ with $g_{p}(v,\,v)=1$. $ST^{*}M$ similarly
        lives as a subspace of $T^{*}M$.
    \end{frame}
    \begin{frame}{Spherical Cotangent Bundles}
        For the sake of visualization it helps to know the topological
        structure of spherical cotangent bundles. A \textit{trivializable}
        tangent (or cotangent) bundle is one that is homeomorphic to
        $M\times\mathbb{R}^{n}$. Similarly a trivializable spherical tangent
        (or cotangent) bundle is one that may be written as
        $M\times\mathbb{S}^{n-1}$. Four results help.
        \begin{theorem}
            If the Euler characteristic of $M$ is non-zero, then $TM$ is not
            trivializable.
        \end{theorem}
        \begin{theorem}
            If $M_{1}$ and $M_{2}$ have trivializable tangent (or cotangent)
            bundles, then $M_{1}\times{M}_{2}$ does as well.
        \end{theorem}
        \begin{theorem}
            Parallelizable manifolds yield trivializable bundles.
        \end{theorem}
        \begin{theorem}
            A parallelizable manifold is orientable.
        \end{theorem}
    \end{frame}
    \begin{frame}{Spherical Cotangent Bundles}
        The only closed orientable surface with euler characteristic zero is
        the torus, which happens to be parallelizable. The plane is also
        parallelizable. For $n$ dimensional parallelizable manifolds $M$ the
        spherical cotangent bundle $ST^{*}M$ is homeomorphic to
        $M\times{S}^{n-1}$.
        \par\hfill\par
        For the torus we get $\mathbb{T}^{3}$, the
        three-torus, and for the plane we have
        $\mathbb{R}^{2}\times\mathbb{S}^{1}$, the thickened torus. Both of these
        spaces have methods of visualizing which helps us create drawings.
    \end{frame}
    \begin{frame}{Causality and Linking}
        Robert Low first conjectured that for causally related points $p,q$ in
        (certain) $2+1$ dimensional spacetimes $(X,\,g)$ the \textit{skies},
        which live in the space of all future directed inextensible null
        pre-geodesics, of these points are topologically linked.
        \par\hfill\par
        This can be made quite explicit in $2+1$ dimensional Minkowski space
        which is given by the semi-Riemannian metric on $\mathbb{R}^{3}$ $g$
        defined by:
        \begin{equation}
            g=\textrm{d}x^{2}+\textrm{d}y^{2}-\textrm{d}t^{2}
        \end{equation}
        The future direction at each point is \textit{up}, i.e. the positive
        $t$ direction (which coincides with the $z$ axis in $\mathbb{R}^{3}$).
    \end{frame}
    \begin{frame}{Causality and Linking}
        Given two points $p,q\in\mathbb{M}^{2,\,1}$ that are causally related
        either there exists a light ray between them or not. If there is the
        skies intersect, which is non-trivial. Otherwise the skies form two
        nested circles in the Cauchy surface $\mathbb{R}^{2}$ (the larger
        circle is the one for the event further in the \textit{past}).
        \par\hfill\par
        To make life simpler we can suppose $p$ and $q$ have the same spacial
        component and only differ in time. That is, $p=(x,\,y,\,t_{0})$ and
        $q=(x,\,y,\,t_{1})$ with $t_{0}<t_{1}$. In this case the two skies are
        actually concentric circles in $\mathbb{R}^{2}$ centered at $(x,\,y)$.
    \end{frame}
    \begin{frame}{Causality and Linking}
        The skies embed naturally into $ST^{*}\mathbb{R}^{2}$, which is
        diffeomorphic to the thickened torus as follows. The plane is
        homeomorphic to the open unit disk via:
        \begin{equation}
            f(x,\,y)=\frac{(x,\,y)}{1+\sqrt{x^{2}+y^{2}}}
        \end{equation}
        We may parameterize the thickened torus by elements of the unit disk
        $\mathbb{D}^{2}$ and points on the circle $\mathbb{S}^{1}$. Given a
        curve in the plane we use $f$ to map it to the disk, and then examine
        angles given by the unit normal to the curve. This gives us a point in
        the disk and an angle on the circle, yielding a unique point in the
        thickened torus.
    \end{frame}
    \begin{frame}{Causality and Linking}
        For two concentric circles we end up with a parameterization of the
        Hopf link. This is shown below.
        \begin{figure}
            \centering
            \resizebox{0.6\textwidth}{!}{%
                \includegraphics{../images/linking_detects_causality_001.pdf}
            }
            \caption{Linking Detects Causality}
            \label{fig:linking_detects_causality_001}
        \end{figure}
    \end{frame}
    \begin{frame}{Causality and Linking}
        It is possible to get other links. For two distinct points in the same
        Cauchy surface (which are hence not causally related) it is also easy
        to show that the skies form unlinked disjoint circles.
        \par\hfill\par
        If one considers the Riemannian metric induced by pullback of the
        mapping $F:\mathbb{R}^{2}\rightarrow\mathbb{R}$ by:
        \begin{equation}
            F(x,\,y)=\frac{1}{1+x^{2}+y^{2}}
        \end{equation}
        (This is a surface in $\mathbb{R}^{3}$ so we may steal the standard
        metric) and warp product this with $\mathbb{R}$ one gets a
        Minkowski-like space with a \textit{bump} around the origin.
        Causally related points can yield different topological links, such as
        the Whitehead link, in this scenario.
    \end{frame}
    \begin{frame}{Causality and Linking}
        The topological Low conjecture does hold for $2+1$ dimensional globally
        hyperbolic spaces where the Cauchy surface admits a covering by an open
        domain in $\mathbb{R}^{2}$. In $3+1$ and higher dimensions this is
        false, Low himself found a counterexample. Here one must
        replace \textit{topological} linking with \textit{Legendrian} linking,
        where we consider the Liouville form on the spherical cotangent bundle
        of the Cauchy surface.
        \par\hfill\par
        This becomes hard to visualize, even for the Minkowski space $M^{3,1}$
        since the spherical cotangent bundle is a five dimensional topological
        manifold. For $M^{3,1}$ it is $\mathbb{R}^{3}\times\mathbb{S}^{2}$.
    \end{frame}
    \begin{frame}{Causality and Linking}
        \begin{theorem}[Chernov, Nemirovski 2008]
            If $(X,\,g)$ is a globally hyperbolic spacetime, if $M$ is a
            spacelike (Riemannian) Cauchy surface of dimension $m\geq{2}$, and
            if $M$ has a smooth covering by an open subset of $\mathbb{R}^{m}$,
            then two causally related points $p,q\in{X}$ have Legendrian linked
            skies in $ST^{*}M$.
        \end{theorem}
        This hints to us that Legendrian knot and link theory may be much
        richer than its topological counterpart.
    \end{frame}
    \begin{frame}
        \centering
        Thanks!
    \end{frame}
\end{document}
