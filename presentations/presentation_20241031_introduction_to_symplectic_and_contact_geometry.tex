%-----------------------------------LICENSE------------------------------------%
%   This file is part of Mathematics-and-Physics.                              %
%                                                                              %
%   Mathematics-and-Physics is free software: you can redistribute it and/or   %
%   modify it under the terms of the GNU General Public License as             %
%   published by the Free Software Foundation, either version 3 of the         %
%   License, or (at your option) any later version.                            %
%                                                                              %
%   Mathematics-and-Physics is distributed in the hope that it will be useful, %
%   but WITHOUT ANY WARRANTY; without even the implied warranty of             %
%   MERCHANTABILITY or FITNESS FOR A PARTICULAR PURPOSE.  See the              %
%   GNU General Public License for more details.                               %
%                                                                              %
%   You should have received a copy of the GNU General Public License along    %
%   with Mathematics-and-Physics.  If not, see <https://www.gnu.org/licenses/>.%
%------------------------------------------------------------------------------%
%   Author:     Ryan Maguire                                                   %
%   Date:       March 29, 2024                                                 %
%------------------------------------------------------------------------------%
\documentclass{beamer}
\usepackage{amsmath}
\graphicspath{{../images/}}
\title{Introduction to Symplectic and Contact Geometry}
\author{Ryan Maguire}
\date{October 31, 2024}
\usenavigationsymbolstemplate{}
\setbeamertemplate{footline}[frame number]
\graphicspath{{../images/}}
\begin{document}
    \maketitle
    \begin{frame}{Outline}
        \begin{itemize}
            \item Historical background.
            \item Exterior algebra and exterior calculus.
            \item The tautological one-form and the Liouville form.
            \item Canonical contact structures.
        \end{itemize}
    \end{frame}
    \begin{frame}{Historical Background}
        The \textit{Lagrangian} of a mechanical system is define via:
        \[
            \mathcal{L}=T-V
        \]
        where $T$ is the kinetic energy and $V$ is the potential energy of the
        system. After studying calculus of variations, you can apply the
        \textit{Euler-Lagrange equation} to this expression to get a new, but
        equivalent, formulation of Newton's second law:
        \[
            \frac{\partial\mathcal{L}}{\partial{q}}
            -\frac{\textrm{d}}{\textrm{d}t}
            \frac{\partial\mathcal{L}}{\partial\dot{q}}
        \]
        where $q$ describes the position, $\dot{q}$ is the velocity,
        and $t$ is time.
    \end{frame}
    \begin{frame}
        Plotting the Lagrangian involves a curve (as a function of $t$)
        that has $q$ coordinates and $\dot{q}$ coordinates. This collection of
        ordered pairs $(q,\,\dot{q})$ is called the \textit{phase-space} of the
        system.
        \par\hfill\par
        We go from Lagrangian mechanics to Hamiltonian mechanics by considering
        the Hamiltonian. This replaces the \textit{position-velocity} coordinate
        system $(q,\,\dot{q})$ with the \textit{momentum-position} coordinate
        system $(p,\,q)$.
        \[
            \mathcal{H}
            =p\dot{q}-\mathcal{L}
        \]
    \end{frame}
    \begin{frame}{Historical Background}
        The dynamics of this system can be obtained
        from studying the differential
        of the Hamiltonian, $\textrm{d}\mathcal{H}$. To do this in a general
        setting requires some exterior algebra and understanding the exterior
        derivative.
    \end{frame}
    \begin{frame}{Exterior Algebra and Calculus}
        The \textit{wedge} product is something used to measure areas, volumes,
        and hypervolumes (higher-dimensional volumes). It can be seen as a
        generalization of the determinant of a matrix, or of the cross product
        that appears in $\mathbb{R}^{3}$.
        \par\hfill\par
        Given two vectors $\mathbf{v}$ and $\mathbf{w}$, the wedge product is
        some new object $\mathbf{v}\land\mathbf{w}$ such that the
        \textit{magnitude} of this expression is precisely the area of the
        parallelogram spanned by $\mathbf{v}$ and $\mathbf{w}$.
        It is also called
        the \textit{exterior product} because $\mathbf{v}\land\mathbf{w}$ is
        something \textit{outside of the vector space}. That is, it is not
        necessarily a vector itself.
    \end{frame}
    \begin{frame}{Exterior Algebra and Calculus}
        We can continue, requiring that
        $\mathbf{u}\land\mathbf{v}\land\mathbf{w}$
        is some object that describes the volume of the
        parallelpiped formed by the
        three vectors $\mathbf{u}$, $\mathbf{v}$, and $\mathbf{w}$.
        \par\hfill\par
        Such a construction must satisfy a few properties. Firstly,
        $\mathbf{v}\land\mathbf{v}=0$.
        This is because the \textit{parallelogram}
        a vector makes with itself is really just a line segment, and line
        segments do not have any area.
        \par\hfill\par
        Similarly, if we are dealing with $n$ dimensional vectors,
        then if we have
        $n+1$ vectors $\mathbf{v}_{1},\,\dots,\,\mathbf{v}_{n+1}$, then the
        wedge product $\mathbf{v}_{1}\land\cdots\land\mathbf{v}_{n+1}$ must be
        zero. This is because there can't be any $n+1$ volume in an
        $n$ dimensional space.
    \end{frame}
    \begin{frame}{Exterior Algebra and Calculus}
        We can create a calculus on such objects by introducing the
        \textit{exterior derivative}.
        \par\hfill\par
        We start with a $0$ form. A $0$ form is just a smooth function.
        \par\hfill\par
        A $1$ form is something that takes in a
        vector and returns a real number. There are some more restrictions in
        the precise definition (stuff related to tangent bundles and smooth
        sections), but this statement suffices for now.
        \par\hfill\par
        A $k$ form is a linear combination of $k$ wedges of $1$ forms. That is,
        a $k$ wedge (also called a $k$ \textit{blade}), is and object of the
        form $\alpha_{1}\land\cdots\land\alpha_{k}$, where each $\alpha_{n}$
        is a $1$ form. A \textit{simple} $k$ form is just a $k$ blade.
        A general $k$ form is a finite linear combination of $k$ blades.
    \end{frame}
    \begin{frame}{Exterior Algebra and Calculus}
        The exterior derivative is defined as follows.
        The exterior derivative of
        a zero form is given by differentials. This produces a 1-form:
        \[
            \textrm{d}(f)=\textrm{d}f
        \]
        The differential of a wedge of $k$ forms is given by:
        \[
            d\left(\alpha\land\beta\right)
            =\textrm{d}\alpha\land\beta
            +(-1)^{k}\alpha\land\textrm{d}\beta
        \]
        We perform the general exterior derivative by requiring that
        differentiation is linear. The final key property property is
        that differentiating twice gives you zero:
        \[
            \textrm{d}\left(\textrm{d}\alpha\right)=\textrm{d}^{2}\alpha=0
        \]
    \end{frame}
    \begin{frame}{The Tautological One-Form and the Liouville Form}
        The Hamiltonian can be rewritten using the exterior derivative. We have:
        \[
            \mathcal{H}
            =\sum_{n}p_{n}\land\textrm{d}q_{n}-\mathcal{L}
        \]
        The differential of the Hamiltonian is just the exterior derivative of
        this expression:
        \[
            \begin{aligned}
                \textrm{d}\mathcal{H}
                &=\textrm{d}\left(
                    \sum_{n}p_{n}\land\textrm{d}q_{n}-\mathcal{L}
                \right)\\
                &=\sum_{n}\textrm{d}\left(
                    p_{n}\land\textrm{d}q_{n}
                \right)-\textrm{d}\mathcal{L}\\
                &=\sum_{n}
                    \textrm{d}p_{n}\land\textrm{d}q_{n}
                    +(-1)^{k}p_{n}\land\textrm{d}^{2}q_{k}
                    -\textrm{d}\mathcal{L}\\
                    &=\sum_{n}
                    \textrm{d}p_{n}\land\textrm{d}q_{n}
                    -\textrm{d}\mathcal{L}
            \end{aligned}
        \]
        Where we used the fact that $\textrm{d}^{2}\alpha=0$ for any $\alpha$.
    \end{frame}
    \begin{frame}{The Tautological One-Form and the Liouville Form}
        Consider the case where $\mathcal{L}=0$. That is, the kinetic energy and
        the potential energy are equal, and therefore the Lagrangian is zero.
        \par\hfill\par
        In this case, our Hamiltonian becomes:
        \[
            \mathcal{H}
            =\sum_{n}p_{n}\land\textrm{d}q_{n}
        \]
        The position $\mathbf{q}=(q_{1},\,\cdots,\,q_{k})$ should vary smoothly
        as a function of time, meaning this is a smooth function (a 0 form).
        Similarly the momentum $\mathbf{p}=(p_{1},\,\cdots,\,p_{n})$ should be
        smooth. $\mathcal{H}$ is thus a sum of exterior products of $0$ forms
        and $1$ forms, meaning it is a $1$ form. This is called the
        \textit{tautological one form}.
    \end{frame}
    \begin{frame}{The Tautological One-Form and the Liouville Form}
        The exterior product of the tautological one form is then:
        \[
            \textrm{d}\mathcal{H}
            =\sum_{n}\textrm{d}p_{n}\land\textrm{d}q_{n}
        \]
        This is called the \textit{Liouville form}.
    \end{frame}
    \begin{frame}{The Tautological One-Form and the Liouville Form}
        A symplectic form on a smooth manifold $M$ is function
        $\omega$ defined on the
        tangent bundle $TM$ (which acts as our \textit{phase space})
        that axiomatizes the properties of the Liouville form. In particular:
        \begin{itemize}
            \item
                $\omega$ acts like a \textit{dot product}. That is, it is a
                \textit{bilinear pairing}.
            \item
                $\omega$ is \textit{skew-symmetric}, meaning
                $\omega(X,\,Y)=-\omega(Y,\,X)$. This comes from the alternating
                property of the wedge product.
            \item
                $\omega$ is non-degenerate. If $X$ is fixed and
                $\omega(X,\,Y)=0$ for all $Y$, then $X$ is zero.
            \item
                $\omega$ is \textit{closed}, meaning the exterior derivative
                of it is zero. To see that the Liouville form is closed,
                note that if we apply the exterior derivative to it we will
                have a sum of $\textrm{d}^{2}p_{n}\land\textrm{d}q_{n}$ on the
                left and $\textrm{d}p_{n}\land\textrm{d}^{2}q_{n}$ on the
                right. All of this will be zero.
        \end{itemize}
    \end{frame}
    \begin{frame}{The Tautological One-Form and the Liouville Form}
        A bilinear product can be described by a matrix $A$. Given vectors
        $X$ and $Y$, we write:
        \[
            \omega(X)=X^{T}AY
        \]
        where $X^{T}$ denotes the transpose of $X$. The skew-symmetric property
        of $\omega$ implies that $A^{T}=-A$, and the non-degenerate property
        means $\textrm{det}(A)\ne{0}$. Because of this we can note that
        $M$ must be \textit{even dimensional}. This is because if the dimension
        is odd, then the matrix $A$ must have determinant zero since:
        \[
            \begin{aligned}
                \textrm{det}(A)
                &=\textrm{det}(A^{T})\\
                &=\textrm{det}(-A)\\
                &=(-1)^{n}\textrm{det}(A)
            \end{aligned}
        \]
        Hence, either $n$ is even or $\textrm{det}(A)=0$. For $A$ to be
        non-degenerate we then require $n$ to be even.
    \end{frame}
    \begin{frame}{Canonical Contact Structures}
        The cotangent bundle of a smooth manifold always has a symplectic
        structure, given by the Liouville form we've seen before.
        If we restrict our attention to the \textit{spherical cotangent bundle},
        and restrict the Liouville form as well, we create a
        \textit{contact structure}.
        \par\hfill\par
        This is the canonical contact structure on the spherical cotangent
        bundle of a manifold.
        \par\hfill\par
        As is often the case in physics, should we have a metric $g$ on the
        manifold $M$, there is then a correspondence (an isomorphism) between
        the spherical cotangent bundle and the spherical tangent bundle.
        It is frequent in the literature to consider these objects the same.
    \end{frame}
    \begin{frame}{Canonical Contact Structures}
        The contact structure has a nice physical interpretation. We are
        considering hypersurfaces in $TM$ where the Hamiltonian $\mathcal{H}$
        is constant.
        \par\hfill\par
        These are surfaces of constant energy! Since energy is a conserved
        quantity, particles should move within these surfaces.
    \end{frame}
    \begin{frame}{The End}
        \begin{center}
            Thank you!
        \end{center}
    \end{frame}
\end{document}
