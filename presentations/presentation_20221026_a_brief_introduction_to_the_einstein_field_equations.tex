%-----------------------------------LICENSE------------------------------------%
%   This file is part of Mathematics-and-Physics.                              %
%                                                                              %
%   Mathematics-and-Physics is free software: you can redistribute it and/or   %
%   modify it under the terms of the GNU General Public License as             %
%   published by the Free Software Foundation, either version 3 of the         %
%   License, or (at your option) any later version.                            %
%                                                                              %
%   Mathematics-and-Physics is distributed in the hope that it will be useful, %
%   but WITHOUT ANY WARRANTY; without even the implied warranty of             %
%   MERCHANTABILITY or FITNESS FOR A PARTICULAR PURPOSE.  See the              %
%   GNU General Public License for more details.                               %
%                                                                              %
%   You should have received a copy of the GNU General Public License along    %
%   with Mathematics-and-Physics.  If not, see <https://www.gnu.org/licenses/>.%
%------------------------------------------------------------------------------%
%   Author:     Ryan Maguire                                                   %
%   Date:       October 26, 2022                                               %
%------------------------------------------------------------------------------%
\documentclass{beamer}
\usepackage{amsmath}

\title{A Brief Introduction to the Einstein Field Equations}
\author{Ryan Maguire}
\date{October 26, 2022}
\usenavigationsymbolstemplate{}
\setbeamertemplate{footline}[frame number]
\begin{document}
    \maketitle
    \begin{frame}
        Many of the notions of curvature require only a semi-Riemannian metric
        and a choice of affine connection.
        \par\hfill\par
        It is almost universal amongst physicists and mathematicians to work
        with the unique Levi-Civita connection that a given semi-Riemannian
        metric induces.
        \par\hfill\par
        We'll discuss several types of curvatures, and their uses in
        describing the Einstein field equations.
    \end{frame}
    \begin{frame}{Preliminaries}
        A manifold is a topological space $(M,\,\tau)$ that is:
        \begin{itemize}
            \item Hausdorff
            \item Second Countable
            \item Locally Euclidean
        \end{itemize}
        Some authors replace second countability with a plethora of other
        not-necessarily-equivalent notions. Paracompactness is a common one,
        others like the notion of $\sigma\textrm{-compactness}$. If the space
        is required to be \textit{connected}, all of these ideas are the same%
        \footnote{I believe there are 50+ alternatives to second countability
                  one can use if the space is connected, but I've forgotten the
                  reference. So don't quote me.}
        \par\hfill\par
        Most authors omit the topology $\tau$ altogether.
    \end{frame}
    \begin{frame}{Preliminaries}
        A chart in a manifold $(M,\,\tau)$ is an ordered pair
        $(\mathcal{U},\,\varphi)$ where $\mathcal{U}\in\tau$ is an open subset
        and $\varphi:\mathcal{U}\rightarrow\mathbb{R}^{n}$ is a continuous
        injective open mapping for some $n\in\mathbb{N}$. The \textit{dimension}
        of the chart is this value $n$.
        \begin{theorem}[Brauer's Invariance of Domain]
            If $f:\mathbb{R}^{n}\rightarrow\mathbb{R}^{n}$ is a continuous
            injective mapping, and if $\mathcal{U}\subseteq\mathbb{R}^{n}$ is
            open, then $f[\mathcal{U}]$ is open.
        \end{theorem}
        This would make $f$ a continuous injective open mapping, meaning it is
        a homeomorphism onto the image. This has two corollaries.
        \begin{theorem}
            $\mathbb{R}^{n}$ is homeomorphic to $\mathbb{R}^{m}$ if and only if
            $n=m$.
        \end{theorem}
    \end{frame}
    \begin{frame}{Preliminaries}
        \begin{theorem}
            If $(M,\,\tau)$ is locally Euclidean, if $x\in{M}$, and if
            $(\mathcal{U},\,\varphi)$ and $(\mathcal{V},\,\psi)$ are charts in
            $M$ with $x\in\mathcal{U}$ and $x\in\mathcal{V}$, then the charts
            have the same dimension.
        \end{theorem}
        The proof is performed by examining the map $\varphi\circ\psi^{-1}$ and
        noting it induces a homeomorphism from an open subset of
        $\mathbb{R}^{n}$ to an open subset of $\mathbb{R}^{m}$, meaning $n=m$.
        \par\hfill\par
        This says that dimension is \textit{locally} constant. If the manifold
        is connected, dimension is a constant. The only way to have a manifold
        with a 1 dimensional component and a 2 dimension component is via
        disjoint unions, like the disjoint union of a line and a sphere.
    \end{frame}
    \begin{frame}{Preliminaries}
        Examining this composition map allows us to define differentiability.
        With very few exceptions (topological vector spaces where the
        Fr\'{e}chet derivative is definable) functions between topological
        spaces have no notion of differentiability. Manifolds have the ability
        to define such things.
        \par\hfill\par
        Given two charts $(\mathcal{U},\,\varphi)$ and $(\mathcal{V},\,\psi)$
        where $\mathcal{U}\cap\mathcal{V}\ne\emptyset$, the function
        $\varphi\circ\psi^{-1}$ is a continuous function from an open subset of
        $\mathbb{R}^{n}$ to another open subset of $\mathbb{R}^{n}$. It is then
        perfectly valid to ask if this function has partial derivatives, or
        second partial derivatives, and so on. One can even ask if the function
        is \textit{smooth}, having all partial derivatives of all orders.
        \par\hfill\par
        If $\varphi\circ\psi^{-1}$ and $\psi\circ\varphi^{-1}$ are smooth, we
        say the charts $(\mathcal{U},\,\varphi)$ and $(\mathcal{V},\,\psi)$ are
        smoothly compatible. We'll say this is vauously true if
        $\mathcal{U}\cap\mathcal{V}=\emptyset$.
    \end{frame}
    \begin{frame}{Preliminaries}
        An atlas on a manifold $(M,\,\tau)$ is a collection of charts
        $(\mathcal{U}_{\alpha},\,\varphi_{\alpha})$ such that
        $\bigcup_{\alpha}\mathcal{U}_{\alpha}=M$.
        \par\hfill\par
        A smooth atlas is an atlas where all charts are smoothly compatible.
        A maximal smooth atlas is a smooth atlas with, intuitively, as many
        smoothly compatible charts possible. A smooth manifold is a topological
        manifold with a maximal smooth atlas.
        \par\hfill\par
        Given two smooth manifolds $M$ and $N$, a smooth function is a function
        $F:M\rightarrow{N}$ such that for every $x\in{M}$ there is a chart
        $(\mathcal{U},\,\varphi)$ in $M$ and a chart $(\mathcal{V},\,\psi)$ in
        $N$ such that $x\in\mathcal{U}$, $F[\mathcal{U}]\subseteq\mathcal{V}$,
        and the function $\psi\circ{F}\circ\varphi^{-1}$ is smooth.
    \end{frame}
    \begin{frame}{Preliminaries}
        A side-note,
        the $F[\mathcal{U}]\subseteq\mathcal{V}$ criterion is important.
        Without it the function $F:\mathbb{R}\rightarrow\mathbb{R}$
        defined by
        \begin{equation}
            F(x)=
            \begin{cases}
                0&x\leq{0}\\
                1&x>0
            \end{cases}
        \end{equation}
        would be considered smooth, but it's not even continuous!
        \par\hfill\par
        A diffeomorphism is a function $F:M\rightarrow{N}$ that is bijective,
        smooth, and such that $F^{-1}$ is smooth.
        For dimensions 0, 1, 2, and 3, every topological manifold has a
        maximal smooth atlas that is unique up to diffeomorphism. Things get
        weird in dimension 4 and higher where it is possible for the same
        manifold to have different smooth structures and for some manifolds to
        have \textbf{zero} smooth structures. Some compact manifolds need not
        be \textit{smoothable}.
    \end{frame}
    \begin{frame}{Preliminaries}
        Smooth manifolds have a notion of tangent spaces. Given a smooth
        manifold $M$ and a point $x\in{M}$ the tangent space at $x$ is
        denoted $T_{x}M$. This can be described via derivations, which are
        functions $D:C^{\infty}(M,\,\mathbb{R})\rightarrow\mathbb{R}$, that take
        in smooth functions and output real numbers, such that:
        \begin{align}
            D(af+bg)&=aD(f)+bD(g)\\
            D(fg)&=f(x)D(g)+D(f)g(x)
        \end{align}
        That is, $D$ is linear and Liebnizean. If $M$ is an $n$ dimensional
        manifold, the set of all derivations at $x\in{M}$ forms an $n$
        dimensional real vector space.
    \end{frame}
    \begin{frame}{Preliminaries}
        If $(\mathcal{U},\,\varphi)$ is a chart containing $x\in{M}$ then the
        partial derivative operators $\partial\varphi_{k}$, $k=0,\dots,n-1$,
        form a basis for $T_{x}M$:
        \begin{equation}
            \partial\varphi_{k}(f)
            =\frac{\partial}{\partial{x}_{k}}\Big(f\circ\varphi^{-1}\Big)
        \end{equation}
        Note, since $f\in{C}^{\infty}(M,\,\mathbb{R})$, it is smooth, and
        $f\circ\varphi^{-1}$ is a smooth function from an open subset of
        $\mathbb{R}^{n}$ to $\mathbb{R}$ so it is valid to take the partial
        derivative in the $k^{th}$ component.
    \end{frame}
    \begin{frame}
        The tangent bundle of a manifold is formed by taking all $T_{x}M$ for
        each $x\in{M}$ and gluing them together in a natural way. For
        $M=\mathbb{R}$ the tangent bundle is $\mathbb{R}^{2}$, and for
        $M=\mathbb{S}^{1}$ the tangent bundle is
        $\mathbb{S}^{1}\times\mathbb{R}$. These are called \textit{trivial}
        tangent bundles. Most tangent bundles are not of the form
        $\mathbb{R}^{n}\times{M}$, the easiest example is $\mathbb{S}^{2}$.
        \par\hfill\par
        The tangent bundle is denoted $TM$ and is a smooth manifold of dimension
        $2n$. A vector field on a manifold is a smooth function
        $V:M\rightarrow{T}M$ such that for all $x\in{M}$ the element
        $V(x)$ is of the form $V(x)=(x,\,v)$. That is, $V$ assigns to every
        element $x\in{M}$ a tangent vector $v\in{T}_{x}M$ that starts at the
        point $x$. The image $V(x)$ is often denoted $V_{x}$.
    \end{frame}
    \begin{frame}{Preliminaries}
        A Riemannian metric is a function $g$ that assigns to every
        $x\in{M}$ a function $g_{x}:T_{x}M\times{T}_{x}M\rightarrow\mathbb{R}$
        that mimics the dot product in Euclidean space. That is, for all
        $v_{0},v_{1},w\in{T}_{x}M$ and $a_{0},a_{1}\in\mathbb{R}$ we have:
        \begin{align}
            g_{x}(a_{0}v_{0}+a_{1}v_{1},\,w)&=a_{0}g_{x}(v_{0},\,w)
                +a_{1}g_{x}(v_{1},\,w)\\
                g_{x}(w,\,a_{0}v_{0}+a_{1}v_{1})
            &=a_{0}g_{x}(w,\,v_{0})+a_{1}g_{x}(w,\,v_{1})\\
                g_{x}(v_{0},\,v_{1})&=g_{x}(v_{1},\,v_{0})\\
                g_{x}(w,\,w)&>{0}\quad{w}\ne{0}
        \end{align}
        The first two say $g_{x}$ is bilinear, the third equation makes $g_{x}$
        symmetric, and the last condition is called
        positive-definiteness.
    \end{frame}
    \begin{frame}{Preliminaries}
	    The algebraist can summarize this by stating that
        $g_{x}$ is a symmetric positive-definite bilinear form. But if you're
        like me and forget these words half the time, then remember the
        Euclidean dot product.\footnote{That's how I made the previous slide.}
        \par\hfill\par
    	The metric $g$ should also be \textit{smooth}. That is, for all
    	$x\in{X}$ and for every smooth vector field $V,\,W$ on $M$ the function
    	$f:M\rightarrow\mathbb{R}$ defined by $f(x)=g_{x}(V_{x},\,W_{x})$
    	should be smooth.
    \end{frame}
    \begin{frame}{Preliminaries}
        Riemannian metrics give us a means of smoothly measuring angles
        between tangent vectors on a manifold at a given tangent space. Namely,
        one can define, for two tangent vectors $v,w\in{T}_{x}M$, the function:
        \begin{equation}
            \angle(v,\,w)=\cos^{-1}\Big(
                \frac{g_{x}(v,\,w)}{g_{x}(v,\,v)\,g_{x}(w,\,w)}
            \Big)
        \end{equation}
    \end{frame}
    \begin{frame}{Preliminaries}
        Pseudo-Riemannian metrics are formed by replacing the
        positive-definite requirement with non-degeneracy. That is, for all
        non-zero tangent vectors $v\in{T}_{x}M$, there is a tangent vector
        $w\in{T}_{x}M$ such that $g_{x}(v,\,w)\ne{0}$. Positive-definiteness
        implies non-degenerate, given $v\ne{0}$ simply choose $w=v$.
        Pseudo-Riemannian metrics are thus a generalization of Riemannian
        metrics.
    \end{frame}
    \begin{frame}{Preliminaries}
        Tangent vectors are derivations on $C^{\infty}(M,\,\mathbb{R})$.
        Cotangent vectors are functions that take in tangent vectors and
        return real numbers in a linear way. Much like tangent vectors can be
        represented explicitly with charts, so can cotangent vectors. Given
        $(\mathcal{U},\,\varphi)$ we can define:
        \begin{equation}
            \textrm{d}\varphi_{k}(\partial\varphi_{\ell})
            =\begin{cases}
                0&k\ne\ell\\
                1&k=\ell
            \end{cases}
        \end{equation}
    \end{frame}
    \begin{frame}{Preliminaries}
        \begin{theorem}[Sylvester's Law of Inertia]
            If $M$ is a smooth manifold of dimension $N$, if $g$ is a
            pseudo-Riemannian metric on $M$, and if $x\in{M}$, then there is a
            chart $(\mathcal{U},\,\varphi)$ and a fixed integer $0\leq{n}<N$
            such that for all $v,w\in{T}_{x}M$ we have:
            \begin{equation}
                g_{x}(v,\,v)
                =\sum_{k=0}^{n-1}\textrm{d}\varphi_{k}(v)^{2}-
                    \sum_{k=n}^{N-1}\textrm{d}\varphi_{k}(v)^{2}
            \end{equation}
        \end{theorem}
        The \textit{signature} of the metric $g$ is the tuple
        $(1,\,\dots,1,\,\,-1,\,\,\dots,\,-1)$ where there are $n$ positives and
        $N-n$ negatives.\footnote{Why is it called Sylvester's Law of Inertia?
            Excellent question.}
    \end{frame}
    \begin{frame}{Preliminaries}
        Pseudo-Riemannian metrics allow one to define affine connections, which
        are tools for transporting tangent vectors around a manifold in a
        parallel fashion, and defining things like curvature.
        Let $\mathfrak{X}(M)$ denote the set of all smooth vector fields on
        $M$. An affine connections is a function
        $\nabla:\mathfrak{X}(M)\times\mathfrak{X}(M)\rightarrow\mathfrak{X}(M)$
        such that for all $X,Y,Z\in\mathfrak{X}(M)$, $a,b\in\mathbb{R}$, and
        $f,g\in{C}^{\infty}(M,\,\mathbb{R})$ we have:
        \begin{align}
            \nabla_{aX+bY}Z&=a\nabla_{X}Z+b\nabla_{Y}Z\\
            \nabla_{Z}(aX+bY)&=a\nabla_{Z}X+b\nabla_{Z}Y\\
            \nabla_{fX}Y&=f\nabla_{X}Y\\
            \nabla_{X}(fY)&=D_{X}Y+f\nabla_{X}Y
        \end{align}
        where $D_{X}f$ is the directional derivative of $f$ in the direction of
        $X$. This is an attempt to axiomatize the contravariant derivative that
        occurs in multi-variable analysis.
    \end{frame}
    \begin{frame}{Preliminaries}
        The general affine connection is rarely discussed in mathematics and
        physics. Two more desirable properties are usually added,
        \textit{torsion-free} and \textit{compatibility}. Torsion is defined
        in terms of the Lie bracket. Given two vector fields
        $X,Y\in\mathfrak{X}(M)$ the Lie bracket is another vector field
        $[X,\,Y]$ defined by:
        \begin{equation}
            [X,\,Y]=XY-YX
        \end{equation}
        The composition $XY$ need not be a vector field because of second order
        terms that make it not linear, but $-YX$ always kills those factors.
        Hence $[X,\,Y]$ is a vector field.
    \end{frame}
    \begin{frame}{Preliminaries}
        A torsion free connection is a connection $\nabla$ such that for all
        $X,\,Y\in\mathfrak{X}(M)$ we have:
        \begin{equation}
            \nabla_{X}Y-\nabla_{Y}X-[X,\,Y]=0
        \end{equation}
    \end{frame}
    \begin{frame}{Preliminaries}
        Compatibility with the metric is defined in terms of parallel
        translation. Say you have a tangent vector $v$ at a point
        $x\in{M}$ and you want to move it along a curve
        $\gamma:[0,\,1]\rightarrow{M}$, $\gamma(0)=x$, in a manner that is
        \textit{parallel}. This equates to solving for a vector field $X$
        using differential equations in coordinates, and you seek
        the solution to:
        \begin{align}
            \nabla_{\gamma'(t)}X&=0\\
            X_{\gamma(0)}&=v
        \end{align}
    \end{frame}
    \begin{frame}{Preliminaries}
        An affine connection that is compatible with $g$ is one such that
        parallel transport is an isometry. That is, if $v,w\in{T}_{x}M$, if
        $\gamma:[0,\,1]\rightarrow{M}$ is a smooth curve, and if
        $v',w'$ are the results of parallel transport of $v$ and $w$,
        respectively, for 1 second, then:
        \begin{equation}
            g_{\gamma(0)}(v,\,w)
            =g_{\gamma(1)}(v',\,w')
        \end{equation}
        A Levi-Civita connection is an affine connection that is torsion free
        and compatible with the metric.
    \end{frame}
    \begin{frame}{Preliminaries}
        \begin{theorem}[Fundamental Theorem of Semi-Riemannian Geometry]
            If $M$ is a smooth manifold, and if $g$ is a pseudo-Riemannian
            metric on $M$, then there is a unique Levi-Civita connection
            $\nabla$ on $M$.
        \end{theorem}
        Lastly, a Lorentzian metric is a pseudo-Riemannian metric on $M$
        with signature $(N-1,\,1)$, or $(+1,\,\dots,\,+1,\,-1)$. The
        \textit{negative} dimension is taken to be time. The Lorentzian
        \textit{norm} (it's not a true norm) is given by:
        \begin{equation}
            ||\cdot||^{2}=\sum_{k=0}^{N-2}\textrm{d}x_{k}^{2}-\textrm{d}t^{2}
        \end{equation}
        And that's it, no more preliminary stuff. On to physics!
    \end{frame}
    \begin{frame}{General Relativity and Curvature}
        In physics it is common to work in a coordinate chart
        $(\mathcal{U},\varphi)$ and express all physical quantities in terms of
        this chart. The semi-Riemannian metric $g$ becomes a matrix
        $g_{\mu\nu}$ with entries
        $g_{\mu\nu}=g(\partial\varphi_{\mu},\partial\varphi_{\nu})$, which is
        called the \textit{metric tensor} in general relativity. Other tensors
        and tensor fields will be described similarly.
    \end{frame}
    \begin{frame}{General Relativity and Curvature}
        The first tensor to describe is the stress-energy tensor $T_{\mu\nu}$.
        It is the gravitational analogue of the stress tensor from Newtonian
        mechanics and describes the density and flux of energy in the
        manifold $(M,g)$, which is always chosen to be Lorentzian.
        \par\hfill\par
        The Einstein field equations relate the stress-energy tensor and the
        metric tensor to Ricci curvature and scalar curvature.
    \end{frame}
    \begin{frame}{General Relativity and Curvature}
        The Ricci curvature is described in terms of the Riemann curvature
        tensor field (It's a tensor field, not a tensor). Given the affine
        connection $\nabla$ on the semi-Riemannian manifold, the Riemann
        curvature tensor field is defined in one of two equivalent ways.
        It is a function $R:\mathfrak{X}(M)^{3}\rightarrow\mathfrak{X}(M)$
        \begin{equation}
            R(X,Y)Z=
                \nabla_{X}\nabla_{Y}Z-\nabla_{Y}\nabla_{X}Z-\nabla_{[X,Y]}Z
        \end{equation}
        Where $[X,Y]$ is the Lie bracket. We can also write this as:
        \begin{equation}
            R(X,Y)=[\nabla_{X},\nabla_{Y}]-\nabla_{[X,Y]}
        \end{equation}
        again using the Lie bracket. With this we see that the Riemann
        curvature tensor field measures the failure of the second derivative to
        commute.
    \end{frame}
    \begin{frame}{General Relativity and Curvature}
        If $\nabla$ is a Levi-Civita connection (torsion free and compatible
        with the metric), then there are several identities the Riemann
        curvature tensor field enjoys. These identities can be combined with
        the Einstein field equations to prove the local conservation of
        energy and momentum, classical laws of Newtonian mechanics which still
        hold in general relativity.
        \begin{itemize}
            \item $R$ is trilinear over $C^{\infty}(M,\mathbb{R})$.
            \item the Bianchi identity holds:
                \begin{equation}
                    R(X,Y)Z+R(Y,Z)X+R(Z,X)Y=0
                \end{equation}
        \end{itemize}
        The Bianchi identity cyclicly permutes the vector fields. It is the
        Bianchi identity that helps one prove conservation of momentum and
        energy.
    \end{frame}
    \begin{frame}{General Relativity and Curvature}
        The quadruple product relates the Riemann curvature tensor field to the
        semi-Riemannian metric. It is defined as:
        \begin{equation}
            (X,Y,Z,T)=g\big(R(X,Y)Z,T\big)
        \end{equation}
        There are several identities for this operation, which are again useful
        for the proof of various theorems in the framework of general
        relativity.
        \begin{align}
            (X,Y,Z,T)&=-(Y,X,Z,T)\\
            (X,Y,Z,T)&=-(X,Y,T,Z)\\
            (X,Y,Z,T)&=(Z,T,X,Y)
        \end{align}
        Lastly, an analogue of the Bianchi identity:
        \begin{equation}
            (X,Y,Z,T)+(Y,Z,X,T)+(Z,X,Y,T)=0
        \end{equation}
    \end{frame}
    \begin{frame}{General Relativity and Curvature}
        These identities combine to give the following theorem.
        \begin{theorem}
            If $(\mathcal{U},\varphi)$ is a chart in a spacetime $(M,g)$,
            if $\nabla$ is the unique Levi-Civita connection on $M$, and if
            $T$ is the stress-energy tensor, then:
            \begin{equation}
                \sum_{n=0}^{N-1}\nabla_{\partial\varphi_{n}}T_{n,m}=0
            \end{equation}
        \end{theorem}
        This is the analogue of the conservation of momentum and energy laws
        that occur in Newtonian mechanics. The proof is about a page and simply
        uses the identities of the Riemannian curvature tensor field, the
        quadruple product, and the Einstein field equations which will be
        stated soon.
    \end{frame}
    \begin{frame}{General Relativity and Curvature}
        The Einstein field equations relate the stress-energy tensor to the
        Ricci and scalar curvatures. The Ricci curvature is defined in terms of
        the Riemann curvature tensor field. There are two ways of doing this.
        \par\hfill\par
        In the Riemann setting ($g$ is positive-definite), fix $p\in{M}$ and
        $x=z_{n}\in{T}_{p}M$ to be unit length. Since $T_{p}M$ is an $n$
        dimensional real inner product space, we may extend $z_{n}$ via the
        Gram-Schmidt procedure to an orthonormal basis. Label these other
        elements $z_{1},\dots,z_{n-1}$. The Ricci curvature about $p$ is defined
        as:
        \begin{equation}
            \textrm{Ric}_{p}(x)=\frac{1}{n-1}\sum_{k=1}^{n}g_{p}
                \Big(R(x,z_{k})x,z_{k}\Big)
        \end{equation}
        It is a theorem that this result is independent of the choice of
        basis.
    \end{frame}
    \begin{frame}{General Relativity and Curvature}
        In the semi-Riemannian setting $T_{p}M$ is not an inner product space
        since $g$ can, in general, fail to be positive definite. Such is the
        case in spacetimes with signature $(+,+,+,-)$. Fix two vector fields
        $Y$ and $Z$. Given a vector field $X$, the mapping
        $X\mapsto{R}(X,Y)Z$ is linear at each tangent space. Because of this
        one may define the \textit{trace} of this mapping. This is the Ricci
        curvature tensor.
        \begin{equation}
            \textrm{Ric}_{p}(Y,Z)=\textrm{tr}
                \big(X_{p}\mapsto{R}_{p}(X_{p},Y_{p})Z_{p}\big)
        \end{equation}
        In local coordinates $(\mathcal{U},\varphi)$ it can be given by a
        matrix $R_{\mu\nu}$.
    \end{frame}
    \begin{frame}{General Relativity and Curvature}
        The Ricci curvature can be completely described by the sectional
        curvature, which is one of the older notions of curvature dating back
        to a time when differential geometry dealt solely with regular surfaces
        and curves. The sectional curvature of a 2-dimensional subspace
        $\delta$ of the tangent space $T_{p}M$ is given by:
        \begin{equation}
            K_{\delta}=\frac{(v,w,v,w)}{A(v,w)}
                =\frac{g_{p}\big(R(v,w)v,w\big)}
                      {\sqrt{||v||^{2}\,||w||^{2}-g_{p}(v,w)^{2}}}
        \end{equation}
        where $v$ and $w$ are two tangent vectors that span $\delta$, and
        $A(v,w)$ is the area of the parallelogram with sides $v$ and $w$.
        $K_{\delta}$ is independent of choice of basis since a change of basis
        can be made by a combination of moves
        $(x,y)\mapsto(y,x)$, $(x,y)\mapsto(\lambda{x},y)$, $\lambda\ne{0}$, and
        $(x,y)\mapsto(x+\lambda{y},y)$. These operations are reflection,
        scaling, and shearing, respectively. All of these are invariant under
        formula above showing $K_{\delta}$ is independent of basis.
    \end{frame}
    \begin{frame}{General Relativity and Curvature}
        For constant curvature manifolds the Ricci curvature is given by a
        simple formula:
        \begin{equation}
            R_{\mu\nu}=(n-1)Kg_{\mu\nu}
        \end{equation}
        where $K$ is the constant curvature of the manifold.
        It is probably not the case that the spacetime we live in is constant
        curvature.
    \end{frame}
    \begin{frame}{General Relativity and Curvature}
        The scalar curvature is defined directly by the Ricci curvature. Given
        the Riemannian definition, $\textrm{Ric}_{p}(x)$, given a basis
        $\{z_{1},\dots,z_{n}\}$ of $T_{p}M$, the scalar curvature is defined by:
        \begin{equation}
            K(p)=\frac{1}{n}\sum_{k=1}^{n}\textrm{Ric}_{p}(z_{k})
        \end{equation}
        It is independent of choice of basis. With respect to the second
        definition, we can define:
        \begin{equation}
            K(p)=\textrm{tr}(R_{\mu\nu})
        \end{equation}
    \end{frame}
    \begin{frame}{General Relativity and Curvature}
        The Einstein tensor is defined in terms of the Ricci and scalar tensors.
        We have:
        \begin{equation}
            G_{\mu\nu}=R_{\mu\nu}-\frac{1}{2}Kg_{\mu\nu}
        \end{equation}
        Where $R_{\mu\nu}$ is the Ricci tensor, $K$ is the scalar curvature,
        and $g_{\mu\nu}$ is the metric tensor. The Einstein field equations are:
        \begin{equation}
            G_{\mu\nu}+\Lambda{g}_{\mu\nu}=\kappa{T}_{\mu\nu}
        \end{equation}
        Where $T_{\mu\nu}$ is the stress-energy tensor. $\Lambda$ is the
        cosmological constant, and $\kappa$ is the Einstein gravitational
        constant.
    \end{frame}
    \begin{frame}{General Relativity and Curvature}
        In practice, one measures the stress-energy tensor and the Einstein
        tensor and wishes to solve for the metric in the Einstein field
        equation. A common simplification is to suppose the spacetime you are
        working in is a vacuum containing no mass-energy. The Einstein field
        equations simplify to:
        \begin{equation}
            G_{\mu\nu}+\Lambda{g}_{\mu\nu}=0
        \end{equation}
        Expanding the Einstein tensor in terms of the Ricci and scalar
        curvature, we get:
        \begin{equation}
            R_{\mu\nu}-\frac{1}{2}Kg_{\mu\nu}+\Lambda{g}_{\mu\nu}=0
        \end{equation}
    \end{frame}
    \begin{frame}{General Relativity and Curvature}
        This is a purely geometrical problem. Depending on the value of
        $\Lambda$ there are several known spacetimes with metrics that satisfy
        the Einstein field equations.
        \begin{itemize}
            \item Minkowski spacetime $\mathbb{M}^{3,1}$
            \item Milne spacetime
            \item Schwarzschild vacuum spacetime
            \item Kerr vacuum
        \end{itemize}
        The value of $\Lambda$ was originally thought to be zero, and Einstein
        retracted it from the equation. In the late 1990's it was discovered
        the inflation of the universe is accelerating, indicating the constant
        may be positive. One possible value involves the Hubble constant, given
        by:
        \begin{equation}
            \Lambda=1.1056\times{10}^{-52}\,\textrm{m}^{-2}
        \end{equation}
    \end{frame}
\end{document}
