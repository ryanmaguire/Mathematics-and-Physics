%-----------------------------------LICENSE------------------------------------%
%   This file is part of Mathematics-and-Physics.                              %
%                                                                              %
%   Mathematics-and-Physics is free software: you can redistribute it and/or   %
%   modify it under the terms of the GNU General Public License as             %
%   published by the Free Software Foundation, either version 3 of the         %
%   License, or (at your option) any later version.                            %
%                                                                              %
%   Mathematics-and-Physics is distributed in the hope that it will be useful, %
%   but WITHOUT ANY WARRANTY; without even the implied warranty of             %
%   MERCHANTABILITY or FITNESS FOR A PARTICULAR PURPOSE.  See the              %
%   GNU General Public License for more details.                               %
%                                                                              %
%   You should have received a copy of the GNU General Public License along    %
%   with Mathematics-and-Physics.  If not, see <https://www.gnu.org/licenses/>.%
%------------------------------------------------------------------------------%
%   Author:     Ryan Maguire                                                   %
%   Date:       June 24, 2025                                                  %
%------------------------------------------------------------------------------%
\documentclass{beamer}
\usepackage{graphicx}
\usepackage{amsmath}
\usepackage{tikz}
\usetikzlibrary{decorations.markings, arrows.meta}
\graphicspath{{../images/}}
\title{Fundamental Quandle and the Cocycle Invariant}
\author{Ryan Maguire}
\date{July 17, 2025}
\usenavigationsymbolstemplate{}
\setbeamertemplate{footline}[frame number]
\begin{document}
    \maketitle
    \begin{frame}{Outline}
        \begin{itemize}
            \item The knot group and the Wirtinger presentation.
            \item Fundamental quandles.
            \item 2-cocycles and the cocycle invariant.
        \end{itemize}
    \end{frame}
    \begin{frame}{The Knot Group and the Wirtinger Presentation}
        The fundamental group of a space $X$ is the set of
        \textit{equivalence classes} of loops
        $\gamma:\mathbb{S}^{1}\rightarrow{X}$ that start and end at a chosen
        fixed point $x_{0}\in{X}$, where two loops $\gamma_{0}$ and
        $\gamma_{1}$ are \textit{equivalent} if we can deform one into the
        other continuously without cutting or tearing.
        \par\hfill\par
        The group operation is \textit{concatenation}. Given
        $\gamma_{0},\,\gamma_{1}:\mathbb{S}^{1}\rightarrow{X}$, we form
        $\gamma_{2}:\mathbb{S}^{1}\rightarrow{X}$ by first following along
        $\gamma_{0}$, and then following along $\gamma_{1}$.
    \end{frame}
    \begin{frame}{The Knot Group and the Wirtinger Presentation}
        Note, the curve $\gamma$ can intersect itself. It does not need to
        be a \textit{knot}, or an embedding.
        \par\hfill\par
        We denote the fundamental group of $X$ with chosen point $x_{0}$
        by $\pi_{1}(X,\,x_{0})$.
        \par\hfill\par
        For most spaces (including all the ones we will care about),
        the chosen point $x_{0}$ does not matter, and hence we may simply write
        $\pi_{1}(X)$.
    \end{frame}
    \begin{frame}{The Knot Group and the Wirtinger Presentation}
        What do we think $\pi_{1}(\mathbb{R}^{3})$ is?
    \end{frame}
    \begin{frame}{The Knot Group and the Wirtinger Presentation}
        What do we think $\pi_{1}(\mathbb{S}^{1})$ is?
    \end{frame}
    \begin{frame}{The Knot Group and the Wirtinger Presentation}
        What about a torus, $\mathbb{T}=\mathbb{S}^{1}\times\mathbb{S}^{1}$?
        \begin{figure}
            \centering
            \includegraphics{torus}
            \caption{A Torus}
        \end{figure}
    \end{frame}
    \begin{frame}{The Knot Group and the Wirtinger Presentation}
        Here's a tricky one, what about the Klein bottle?
        \begin{figure}
            \centering
            \resizebox{!}{0.5\textwidth}{%
                \includegraphics{klein_bottle}
            }
            \caption{A Klein Bottle}
        \end{figure}
    \end{frame}
    \begin{frame}{The Knot Group and the Wirtinger Presentation}
        \begin{align}
            \pi_{1}(\mathbb{R}^{3})
            &=\textrm{trivial}\\
            \pi_{1}(\mathbb{S}^{1})
            &=\mathbb{Z}\\
            \pi_{1}(\mathbb{T})
            &=\langle{a,\,b\;|\;ab=ba}\rangle\\
            \pi_{1}(\textrm{Klein})
            &=\langle{a,\,b\;|\;aba=b}\rangle
        \end{align}
    \end{frame}
    \begin{frame}{The Knot Group and the Wirtinger Presentation}
        The knot group of a knot $K$ is the fundamental group of the
        complement, $\pi_{1}(\mathbb{R}^{3}\setminus{K})$.
        \par\hfil\par
        This group has a presentation, known as the
        \textit{Wirtinger presentation}, of the form:
        \begin{equation}
            \pi_{1}(\mathbb{R}^{3}\setminus{K})
            =\langle
                x_{1},\,\dots,\,x_{N}\;|\;wx_{n}w^{-1}=x_{m},\,\dots
            \rangle
        \end{equation}
        Where $w$ is a \textit{word}, a product of the generators
        $x_{1},\,\dots,\,x_{N}$, which come from the \textit{arcs} in
        a diagram.
    \end{frame}
    \begin{frame}{The Knot Group and the Wirtinger Presentation}
        Kervaire proved the following.
        \begin{theorem}
            A Wirtinger presentation for a group $G$
            corresponds to a knot $K$ if and only if:%
            \footnote{%
                This \textit{knot} $K$ might live in higher dimensions.
            }
            \begin{enumerate}
                \item
                    The Abelianization of $G$ is $\mathbb{Z}$.
                \item
                    The second homology of the group is trivial.
                \item
                    The presentation is finite.
                \item
                    $G$ is the \textit{normal closure} of a single generator.
            \end{enumerate}
        \end{theorem}
    \end{frame}
    \begin{frame}{The Knot Group and the Wirtinger Presentation}
        Conditions 1 and 2 are pretty straight forward once we study algebraic
        topology. The Abelianization of $\pi_{1}(X)$ is $H_{1}(X)$, the
        first homology group, and for any knot $K$ it is well-known that
        $H_{1}(\mathbb{R}^{3}\setminus{K})=\mathbb{Z}$.
        \par\hfill\par
        The second condition (second homology group is trivial)
        says \textit{there are no spheres in the knot complement}.
        This theorem goes back to the 1950s
        (Papakyriakopoulos, 1957).
        \par\hfill\par
        Conditions 3 and 4 are the Wirtinger condition.
    \end{frame}
    \begin{frame}{The Knot Group and the Wirtinger Presentation}
        The trefoil has knot group
        $\langle{x,\,y\;|\;(xy)^{-1}yxy=x}\rangle$.
        We can simplify this to:
        \begin{equation}
            \pi_{1}(\mathbb{R}^{3}\setminus\textrm{trefoil})
            =\langle{x,\,y\;|\;x^{2}=y^{3}}\rangle
        \end{equation}
    \end{frame}
    \begin{frame}{The Knot Group and the Wirtinger Presentation}
        The figure-eight is given by:
        \begin{equation}
            \pi_{1}(\mathbb{R}^{3}\setminus\textrm{figure-eight})
            =\langle
                x,\,y\;|\;yxy^{-1}xy=xyx^{-1}yx
            \rangle
        \end{equation}
    \end{frame}
    \begin{frame}{Fundamental Quandles}
        A quandle is a set $Q$ and a binary operation
        $*:Q\times{Q}\rightarrow{Q}$ such that:
        \begin{enumerate}
            \item
                $x*x=x$
            \item
                The function $f_{x}(y)=y*x$ is invertible for each $x\in{Q}$.
            \item
                $a*(b*c)=(a*b)*(a*c)$.
        \end{enumerate}
        Given a group $G$, the \textit{conjugate} of an element
        $a\in{G}$ by an element $g\in{G}$ is the element
        $g^{-1}ag\in{G}$. Wirtinger presentations are precisely conjugations
        of elements of the group by words.
    \end{frame}
    \begin{frame}{Fundamental Quandles}
        If $G$ is any group, then we automatically have a quandle by defining
        \begin{equation}
            x*y=y^{-1}xy
        \end{equation}
        The fact that this is a quandle follows directly from some basic
        group properties:
        \begin{enumerate}
            \item
                $a*a=a^{-1}aa=a$
            \item
                $\varphi_{x}(y)=xy$ and $\tilde{\varphi}_{x}(y)=y^{-1}x$
                are both bijections, a fundamental fact about groups.
                Because of this, $f_{x}(y)=y^{-1}xy$ is the composition of
                bijections, which is hence a bijection, and therefore this
                function has an inverse.
            \item
                Conjugation self-distributes.
        \end{enumerate}
    \end{frame}
    \begin{frame}{Fundamental Quandles}
        Observation: The Wirtinger presentation consists of conjugations,
        and conjugations define quandles. We take the Wirtinger presentation
        of the knot group, and use it as the presentation of a quandle.
        This is the \textbf{fundamental quandle} of the knot. It detects
        a knot up to mirror images and inversions (swapping the direction).
    \end{frame}
    \begin{frame}{Fundamental Quandles}
        The fundamental quandle is indeed a knot invariant, and hence any
        invariant of the quandle is also an invariant of the knot.
        \par\hfill\par
        For example, suppose we fix a knot $K$ and let $Q(K)$ denote the
        fundamental quandle. Given any quandle $R$, we can consider the
        set of all \textit{quandle homomorphisms}
        $\varphi:Q(K)\rightarrow{R}$, denoted $\textrm{Hom}\big(Q(K),\,R\big)$,
        which are functions with
        $\varphi(x*y)=\varphi(x)*\varphi(y)$. The size of this set, or its
        \textit{cardinality}, is another invariant of the knot $K$.
    \end{frame}
    \begin{frame}{Cocycle Invariant}
        Given a group $G$, there is something called the
        \textit{group ring} $\mathbb{Z}[G]$. Groups only have one operation,
        but we are more familiar with having two (addition and multiplication).
        An element of $\mathbb{Z}[G]$ is a finite combination
        $n_{1}g_{1}+\cdots+n_{N}g_{N}$ where $n_{m}\in\mathbb{Z}$ and
        $g_{m}\in{G}$. Addition and multiplication are carried out using
        the distributive law, together with the usual arithmetic properties
        of $\mathbb{Z}$, and by applying the group operation.
    \end{frame}
    \begin{frame}{Cocycle Invariant}
        As an example, suppose $G$ is a group with
        $g_{1},g_{2}\in{G}$. Then:
        \begin{equation}
            (g_{1}+2g_{2})(3g_{1}-g_{2})
            =3g_{1}^{2}+6g_{2}*g_{1}-g_{1}*g_{2}-2g_{2}^{2}
        \end{equation}
        Note that, in general, groups do not need to be \textit{Abelian},
        meaning $g_{1}*g_{2}$ and $g_{2}*g_{1}$ might be different
        (think about matrices).
    \end{frame}
    \begin{frame}{Cocycle Invariant}
        The cocycle invariant is given as follows. Fix a knot $K$ and a group
        $G$ (it does not need to be Abelian, but it is easier to deal with
        Abelian groups), and a quandle $R$. Given a diagram $D$ for $K$, we let
        $X$ denote the set of \textit{arcs}, which are the curves that
        start at a given under crossing, pass over all of the over crossings,
        and then stop at the next under crossing.
    \end{frame}
    \begin{frame}{Cocycle Invariant}
        \begin{figure}
            \centering
            \includegraphics{trefoil_knot_001}
            \caption{Trefoil Knot}
        \end{figure}
    \end{frame}
    \begin{frame}{Cocycle Invariant}
        A coloring of $D$ with respect to the quandle $R$ is a function
        $C:A\rightarrow{R}$ such that at each crossing we have:
        \begin{equation}
            C(\textrm{lower-right})*C(\textrm{lower-left})
            =C(\textrm{upper-left})
        \end{equation}
    \end{frame}
    \begin{frame}{Cocycle Invariant}
        Tricolorability can be recovered using certain 3-element quandles.
        \begin{figure}
            \centering
            \resizebox{!}{0.5\textwidth}{%
                \includegraphics{trefoil_tricolor}
            }
            \caption{Trefoil Knot}
        \end{figure}
    \end{frame}
    \begin{frame}{Cocycle Invariant}
        A 2-cocycle is a function
        $\phi:R\times{R}\rightarrow{G}$ with $\phi(x,\,x)=0$ and:
        \begin{equation}
            \phi(x,\,y)\phi(x*y,\,z)=\phi(x,\,z)\phi(x*z,\,y*z)
        \end{equation}
        Note, if $G$ is Abelian we replace multiplication with addition.
        The \textit{Boltzmann weight} for a crossing $\tau$ is
        $B(C,\,\tau)=\phi(x_{\tau},\,y_{\tau})^{\textrm{sgn}(\tau)}$,
        where $x_{\tau}$ and $y_{\tau}$ are the bottom-right and bottom-left
        arcs for the crossing $\tau$, respectively, and
        $\textrm{sgn}(\tau)$ is the sign of the crossing.
    \end{frame}
    \begin{frame}{Cocycle Invariant}
        Letting $\textrm{Col}_{R}(K)$ denote the set of all colorings of $K$
        by the quandle $R$, the cocycle invariant is an element of the
        group-ring $\mathbb{Z}[G]$, defined by:
        \begin{equation}
            \Phi_{R,\,G}(K)
            =\sum_{C\in\textrm{Col}_{R}(K)}\prod_{\tau}B(C,\,\tau)
        \end{equation}
    \end{frame}
    \begin{frame}{Cocycle Invariant}
        This definition is very involved, and next week when we learn
        how to write Python code we will see how to make this more
        computational. We conclude with some results.
    \end{frame}
    \begin{frame}{Cocycle Invariant}
        \begin{theorem}
            There are 26 quandles $R$ and groups $G$ such that
            $\Phi_{R,\,G}(K)$ is a perfect invariant (up to mirrors and
            inverses) for all knots with less than 12 crossings. There are
            60 such pairs that detect all but 13 knots with less than
            13 crossings.
        \end{theorem}
    \end{frame}
    \begin{frame}{Cocycle Invariant}
        The following has been conjectures. If $K$ and $K^{\prime}$ are
        distinct knots, then there is a group $G$ and two distinct
        quandles $R$, $R^{\prime}$ such that
        $\Phi_{R,\,G}(K)\ne\Phi_{R^{\prime},\,G}(K^{\prime})$. That is,
        the cocycle invariant might be a perfect invariant for knots.
    \end{frame}
\end{document}
