%-----------------------------------LICENSE------------------------------------%
%   This file is part of Mathematics-and-Physics.                              %
%                                                                              %
%   Mathematics-and-Physics is free software: you can redistribute it and/or   %
%   modify it under the terms of the GNU General Public License as             %
%   published by the Free Software Foundation, either version 3 of the         %
%   License, or (at your option) any later version.                            %
%                                                                              %
%   Mathematics-and-Physics is distributed in the hope that it will be useful, %
%   but WITHOUT ANY WARRANTY; without even the implied warranty of             %
%   MERCHANTABILITY or FITNESS FOR A PARTICULAR PURPOSE.  See the              %
%   GNU General Public License for more details.                               %
%                                                                              %
%   You should have received a copy of the GNU General Public License along    %
%   with Mathematics-and-Physics.  If not, see <https://www.gnu.org/licenses/>.%
%------------------------------------------------------------------------------%
%   Author:     Ryan Maguire                                                   %
%   Date:       April 27, 2023                                                 %
%------------------------------------------------------------------------------%
\documentclass{beamer}
\usepackage{amsmath}
\title{Numerical Experiments with the Khovanov Homology of Twist Knots}
\author{Ryan Maguire}
\date{April 29, 2023}
\newtheorem{question}{Question}
\usenavigationsymbolstemplate{}
\setbeamertemplate{footline}[frame number]
\begin{document}
    \maketitle
    \begin{frame}{Outline}
        \begin{itemize}
            \item Conjectures on Legendrian and tranversely simple knots.
            \item New data for knots up to 19 crossings.
            \item A peculiar find about twist knots.
        \end{itemize}
    \end{frame}
    \begin{frame}{Previous Conjectures}
        Last year I presented the following conjecture.
        \begin{flushleft}
            If $K$ is a topological knot (or link) type that is Legendrian
            (or transversally) simple, then the Khovanov homology of $K$
            distinguishes it. That is, if $\tilde{K}$ is another knot with the
            same Khovanov homology as $K$, then $\tilde{K}$ is topologically
            identical to $K$.
        \end{flushleft}
        Legendrian simple means the Legendrian representations of the knot
        are uniquely identified by their Thurston-Bennequin and rotation
        numbers (transversally simple has a similar definition).
        \par\hfill\par
        Numerical evidence was presented for all knots up to 17 crossings.
        This has been expanded.
    \end{frame}
    \begin{frame}{Previous Conjectures}
        It is known that the torus knots form a family of Legendrian simple
        knots. Etnyre, Ng, and Vertesi have also completely classified when
        twist knots are Legendrian simple. Indeed, then $m_{n}$ twist knot is
        Legendrian simple if and only if $m\geq{-3}$. Their work further
        classifies when twist knots are transversally simple. Suffice it to say
        not all twist knots are transversally simple.
        \par\hfill\par
        Lastly, Ng, Chongchitmate, An, Liang, and Manocha have written Java
        code to find Legendrian representations of knots. Their work has led
        to the conjecture that certain knot types are Legendrian simple, but
        not confirmed. For example the $6_{2}$ knot has two
        distinct Legendrian representations with $(tb,\,rot)=(-7,\,2)$ and
        hence this is not Legendrian simple, but the \textit{mirror} of $6_{2}$
        may be.
    \end{frame}
    \begin{frame}{Previous Conjectures}
        Using torus knots, twist knots, and the conjecturally Legendrian simple
        knots from the Legendrian knot atlas, all prime knots up to 19
        crossings have been compared with these families to test the conjectured
        (expanding our results beyond 17).
        \par\hfill\par
        A table of all Jones polynomials of all prime knots up to this many
        crossings has been tabulated (as well as Alexander and HOMFLY
        polynomials, but these were not directly needed for our conjecture).
        This data will soon be made available publically. We use this to search
        which knots may have the same Khovanov homology as a torus, twist, or
        conjecturally Legendrian simple knot (Khovanov homology categorifies
        the Jones polynomial).
    \end{frame}
    \begin{frame}{New Finds}
        After analyzing the data we find several knots with the
        same Jones polynomial as a torus, twist, or conjecturally Legendrian
        simple knot. At this point we compute the Khovanov polynomial
        (which contains the torsion-free data of the Khovanov homology of the
        knots). We can make the following claims:
        \begin{theorem}
            If $K$ is a prime knot with less than or equal to 19 crossings,
            and if $T$ is a torus or twist knot with the same Khovanov
            polynomial (or Khovanov homology) as $K$,
            then $T$ is equivalent to $K$.
        \end{theorem}
    \end{frame}
    \begin{frame}{New Finds}
        The degree of the Jones polynomial of a knot with $n$ crossings is
        bounded by a constant multiple of $n$.
        \par\hfill\par
        Using well-known formulas for
        the Jones polynomials of torus and twist knots we need only search
        through a finite set of knots to make the previous claim.
    \end{frame}
    \begin{frame}{New Finds}
        This motivates the following question.
        \begin{question}
            Does the Khovanov polynomial distinguish twist knot?
            Does Khovanov homology?
        \end{question}
        It is known that this is true in the three simplest cases.
        Khovanov homology \textit{does} distinguish the unknot, trefoils, and
        figure eight knot. It is also known that the Jones polynomial
        does \textit{not}, the the figure eight has the sames Jones polynomial
        as a certain 11 crossing knot.
    \end{frame}
\end{document}
