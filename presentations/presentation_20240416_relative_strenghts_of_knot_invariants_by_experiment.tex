%-----------------------------------LICENSE------------------------------------%
%   This file is part of Mathematics-and-Physics.                              %
%                                                                              %
%   Mathematics-and-Physics is free software: you can redistribute it and/or   %
%   modify it under the terms of the GNU General Public License as             %
%   published by the Free Software Foundation, either version 3 of the         %
%   License, or (at your option) any later version.                            %
%                                                                              %
%   Mathematics-and-Physics is distributed in the hope that it will be useful, %
%   but WITHOUT ANY WARRANTY; without even the implied warranty of             %
%   MERCHANTABILITY or FITNESS FOR A PARTICULAR PURPOSE.  See the              %
%   GNU General Public License for more details.                               %
%                                                                              %
%   You should have received a copy of the GNU General Public License along    %
%   with Mathematics-and-Physics.  If not, see <https://www.gnu.org/licenses/>.%
%------------------------------------------------------------------------------%
%   Author:     Ryan Maguire                                                   %
%   Date:       March 29, 2024                                                 %
%------------------------------------------------------------------------------%
\documentclass{beamer}
\usepackage{amsmath}
\graphicspath{{../images/}}
\title{Relative Strengths of Knot Invariants by Experiment}
\author{Ryan Maguire}
\date{April 16, 2024}
\usenavigationsymbolstemplate{}
\setbeamertemplate{footline}[frame number]
\graphicspath{{../images/}}
\begin{document}
    \maketitle
    \begin{frame}{Outline}
        \begin{enumerate}
            \item The Alexander Polynomial.
            \item The Jones Polynomial.
            \item Khovanov Homology.
            \item The HOMFLY-PT Polynomial.
            \item Results and Statistics.
        \end{enumerate}
    \end{frame}
    \begin{frame}{The Alexander Polynomial}
        One of the oldest invariants in knot theory, the Alexander polynomial
        is not-so-easy to describe, but very easy to compute. The original
        definition involves a certain infinite cyclic cover of the complement
        of a knot embedded into $\mathbb{R}^{3}$
        (see \cite[p.~53]{LickorishKnotTheory}), but the
        computation can be explained fairly quickly using knot diagrams.
        \par\hfill\par
        For a planar diagram of a knot there will be $N$ crossings and
        $N+2$ faces using Euler's formula. Create a $N\times(N+2)$ adjacency
        matrix and fill the entries with affine combinations of terms in the
        set $\{\,1,\,-1,\,t,\,-t\,\}$ based on the orientation of the crossings.
        \cite[p.~49]{LivingstonKnotTheory} contains the details.
        \par\hfill\par
        Cross out two columns corresponding to faces that do not share a
        border and take the determinant. The result is the Alexander polynomial
        up to a multiple of $t^{n}$.
    \end{frame}
    \begin{frame}{The Alexander Polynomial}
        Creating the Alexander matrix runs in $O(N^{2})$ time, where $N$
        is the number of crossings. The way determinants are taught in
        linear algebra yields a terrifying $O(N!)$ computation, but $LU$
        decomposition can speed this up to $O(N^3)$. The Alexander polynomial
        is hence a polynomial time invariant.
        \par\hfill\par
        We can speed the computation up even further using tangles
        (\cite{BarNatanPolynomialTimeKnotPolynomials}) and implementations
        exist in \cite{SnapPy} and \cite{MaguireLibtmpl}.
    \end{frame}
    \begin{frame}{The Alexander Polynomial}
        Having such a quick computation means we can tabulate the invariant
        for the 352+ million prime knots up to crossing number 19 in less than
        a day.
        \par\hfill\par
        This speed comes at a cost, the invariant is not very good at
        distinguishing knots. Many distinct knots have the same Alexander
        polynomial. Furthermore, we have the following classical theorem
        (see \cite{KawauchiSurveyKnotTheory1996}).
        \begin{theorem}
            If $p\in\mathbb{Z}[t,\,t^{-1}]$ is a Laurent polynomial such that
            $p(t)=p(t^{-1})$ and $p(1)=\pm{1}$, then there is a knot $K$
            such that $\Delta_{K}(t)=p(t)$ where $\Delta_{K}$ is the
            Alexander polynomial of $K$.
        \end{theorem}
        So Alexander polynomials are not uncommon.
    \end{frame}
    \begin{frame}{The Jones Polynomial}
        The Jones polynomial is a knot invariant that is often called
        \textit{stronger} than the Alexander polynomial, but the two are not
        directly comparable. There are knots with the same Alexander polynomial
        but different Jones polynomials ($6_{1}$ and $9_{46}$ in the Rolfsen
        table), but there are also knots with the
        same Jones polynomial and different Alexander polynomials
        (the Figure-Eight $4_{1}$ and the non-alternating knot
        $K11N19$).
        \par\hfill\par
        At the end of this talk we will justify why the Jones polynomial is
        said to be stronger, but keep those examples in your back pocket!
    \end{frame}
    \begin{frame}{The Jones Polynomial}
        The original definition uses the Temperley-Lieb algebra, braid groups,
        and the Markov trace \cite{JonesJonesPolyForDummies}, but it is
        computationally
        beneficial to think in terms of the Kauffman bracket. This is defined
        in terms of smoothings of crossings.
        \par\hfill\par
        \begin{figure}
            \centering
            \includegraphics{resolving_crossings.pdf}
            \caption{Smoothing a Crossing}
        \end{figure}
    \end{frame}
    \begin{frame}{The Jones Polynomial}
        The Kauffman bracket $\langle{L}\rangle$ of a link diagram $L$ is
        defined recursively as follows\footnote{%
            This follows the conventions in \cite{BarNatanKhovanovJones}
            which differ slightly from the original in
            \cite{KauffmanStateModels}.
        }
        \begin{align}
            \langle{\emptyset}\rangle&=1\\
            \langle{\mathbb{S}^{1}\sqcup{L}}\rangle
            &=(q+q^{-1})\langle{L}\rangle\\
            \langle{L}\rangle
            &=\langle{L_{n,\,0}}\rangle-q\langle{L_{n,\,1}}\rangle\\
        \end{align}
        where $\mathbb{S}^{1}\sqcup{L}$ denotes the disjoint union of a
        circle and $L$ unlinked in $\mathbb{R}^{3}$, $L_{n,\,0}$ denotes the
        zero smoothing at the $n^{\small\textrm{th}}$ crossing, and
        similarly $L_{n,\,1}$ denotes the one smoothing at the
        $n^{\small\textrm{th}}$ crossing.
    \end{frame}
    \begin{frame}{The Jones Polynomial}
        This is indeed well-defined and invariant
        under the type II and type III Reidemeister moves. Type I moves scale
        the result by $q^{-1}$ and $-q^{2}$, depending on the sign of the
        introduced crossing.
        \par\hfill\par
        There are two ways to make the resulting polynomial invariant under
        type I moves. We can normalize by the writhe, writing
        \begin{equation}
            J_{L}(q)=(-q)^{n_{+}-2n_{-}}\langle{L}\rangle
        \end{equation}
        where $n_{+}$ and $n_{-}$ are the number of positive and negative
        crossings in the diagram, respectively.
        \par\hfill\par
        Alternatively we could scale $\langle{L}\rangle$ by the monomial
        $q^{k}$ such that the zeroth term has degree zero. That is, multiply
        by $q^{-\textrm{mindeg}(\langle{L}\rangle)}$ so that the result is an
        actual polynomial, instead of a Laurent polynomial.
    \end{frame}
    \begin{frame}{The Jones Polynomial}
        There are pros and cons to both definitions. The far more common
        definition, $J_{L}$, has the nice property that
        $J_{L}(q)=J_{m(L)}(q^{-1})$ where $m(L)$ is the mirror of $L$,
        the result of composing the embedding for $L$ with the
        reflection $z\mapsto{-z}$.
        \par\hfill\par
        The alternative definition means we can skip a writhe computation,
        but this runs in $O(N)$ so it is not a big deal. However, since
        $p(q)=q^{-\textrm{mindeg}(\langle{L}\rangle)}\langle{L}\rangle$ is a
        valid polynomial, this alternative method defines a
        meromorphic map $z\mapsto{z}-p(z)/p'(z)$ and induces a Newton fractal
        in the complex plane with Newton basins and a Julia set.
        \par\hfill\par
        The topology of these
        sets are dependent only on the polynomial $p$, and are hence knot
        invariants themselves.
    \end{frame}
    \begin{frame}{The Jones Polynomial}
        \begin{figure}
            \centering
            \resizebox{0.7\textwidth}{!}{%
                \includegraphics{newton_fractal_right_trefoil_jones.png}
            }
            \caption{Newton Fractal for the Right-Handed Trefoil}
        \end{figure}
    \end{frame}
    \begin{frame}{The Jones Polynomial}
        \begin{figure}
            \centering
            \resizebox{0.7\textwidth}{!}{%
                \includegraphics{newton_fractal_right_trefoil_jones_unreduced.png}
            }
            \caption{Unreduced Newton Fractal for the Right-Handed Trefoil}
        \end{figure}
    \end{frame}
    \begin{frame}{The Jones Polynomial}
        \begin{figure}
            \centering
            \resizebox{0.7\textwidth}{!}{%
                \includegraphics{newton_fractal_left_trefoil_jones.png}
            }
            \caption{Newton Fractal for the Left-Handed Trefoil}
        \end{figure}
    \end{frame}
    \begin{frame}{The Jones Polynomial}
        \begin{figure}
            \centering
            \resizebox{0.7\textwidth}{!}{%
                \includegraphics{newton_fractal_left_trefoil_jones_unreduced.png}
            }
            \caption{Unreduced Newton Fractal for the Left-Handed Trefoil}
        \end{figure}
    \end{frame}
    \begin{frame}{The Jones Polynomial}
        Computationally the Jones polynomial is at most
        $O(N)$ away from the Kauffman bracket, so we will restrict our
        algorithmic efforts to computing $\langle{L}\rangle$.
        \par\hfill\par
        The definition gives us a recursive algorithm that is extremely
        simple to implement. This can be done in just a handful of lines using
        \texttt{C} or \texttt{Python}. The simplicity comes at a cost,
        the algorithm is not only exponential in time, but also in space.
        After about a dozen crossings one may need several gigabytes of
        memory for the computation, and 17 crossings and higher is not
        feasible on most personal computers.
    \end{frame}
    \begin{frame}
        The recursion tree can be expanded into an iterative formula using
        induction. For your $N$ crossing link diagram each number between 0 and
        $2^{N}-1$ uniquely corresponds to a smoothing of all crossings in the
        figure.
        \par\hfill\par
        Given $0\leq{n}\leq{2}^{N}-1$ write $n$ in binary. If the
        $m^{\small\textrm{th}}$ bit is a zero, perform a zero smoothing at
        the $m^{\small\textrm{th}}$ crossing. Otherwise perform a one
        smoothing. Do this for each bit in the number.
    \end{frame}
    \begin{frame}
        The Kauffman bracket is then computed via
        \begin{equation}
            \langle{L}\rangle
            =\sum_{n=0}^{2^{N}-1}(-q)^{w(n)}(q+q^{-1})^{c(n)}
        \end{equation}
        where $w(n)$ is the \textit{Hamming weight}, the number of ones in the
        binary expansion of $n$, and $c(n)$ is the
        \textit{cycle counting function}, which counts the number of cycles
        in the plane that result from the complete smoothing corresponding to
        the number $n$.
    \end{frame}
    \begin{frame}{The Jones Polynomial}
        \begin{figure}
            \centering
            \includegraphics{trefoil_knot_cube_of_resolutions.pdf}
            \caption{Cube of Resolutions for the Trefoil}
        \end{figure}
    \end{frame}
    \begin{frame}{The Jones Polynomial}
        \begin{figure}
            \centering
            \resizebox{0.55\textwidth}{!}{%
                \includegraphics{figure_eight_knot_cube_of_resolutions.pdf}
            }
            \caption{Cube of Resolutions for the Figure-Eight}
        \end{figure}
    \end{frame}
    \begin{frame}{The Jones Polynomial}
        By representing knots and links with extended Gauss codes we can
        use this summation to devise explicit algorithms for the Kauffman
        bracket. This is done in \cite{MaguireJones}.
        \par\hfill\par
        While still running in
        $O(2^{N})$ time the space requirements have been drastically reduced
        to $O(N)$. A 32 crossing knot can be dealt with on a laptop, though
        the computation will take several minutes to a few hours or days,
        depending on the CPU.
    \end{frame}
    \begin{frame}{The Jones Polynomial}
        This invariant has been tabulated for all prime knots up to 19 crossings
        in \cite{JonesData} using the algorithms in \cite{MaguireJones} and
        \cite{Burton2018HOMFLFixedParameter}.
        \par\hfill\par
        We take a moment to outline one more algorithm since it will be of
        use to us for the computation of Khovanov homology. It uses a
        \textit{symbolic calculus} by manipulating planar diagram codes,
        or PD codes. So first, what are PD codes?
    \end{frame}
    \begin{frame}{The Jones Polynomial}
        Take a knot diagram and orient it, labeling the arcs from $0$ to
        $2N-1$ in increasing order. Trace your finger along the drawing and at
        each under crossing write down \texttt{X[a,b,c,d]} where
        \texttt{a} is the arc behind you, \texttt{b} is to your right,
        \texttt{c} is the arc in front, and \texttt{d} is the one on the left.
        \par\hfill\par
        The result is a string of $N$ $4-$tuples. The algorithm starts by
        observing what the Kauffman relation does to PD code.
    \end{frame}
    \begin{frame}{The Jones Polynomial}
        \begin{figure}
            \centering
            \includegraphics{resolving_crossings_pd_label.pdf}
            \caption{Resolving a Crossing with PD Code}
        \end{figure}
    \end{frame}
    \begin{frame}{The Jones Polynomial}
        The Kauffman relation tells us to perform the symbolic replacement
        \begin{equation}
            \texttt{X[a,b,c,d]}\mapsto\texttt{P[a,b]P[c,d]}-q\texttt{P[a,d]P[b,c]}
        \end{equation}
        We do this for every crossing and obtain the product
        \begin{equation}
            \langle{L}\rangle
            =\prod_{k=n}^{N-1}
            (\texttt{P[an,bn]P[cn,dn]}-q\texttt{P[an,dn]P[bn,cn]})
        \end{equation}
        where \texttt{X[an,bn,cn,dn]} represents the
        $n^{\small\textrm{th}}$ crossing.
    \end{frame}
    \begin{frame}{The Jones Polynomial}
        After expanding this we obtain a polynomial whose coefficients are
        formal products of ordered pairs. We reduce further and obtain a
        Laurent polynomial with integer coefficients by appealing to the
        Kauffman relations.
        \par\hfill\par
        Should we see the product
        \texttt{P[a,b]P[a,b]} we may replace it with \texttt{P[a,b]} since
        both expressions tell us the arcs \texttt{a} and \texttt{b} are now
        connected. Similarly, we may replace \texttt{P[a,b]P[b,c]} with
        \texttt{P[a,c]}.
        \par\hfill\par
        Lastly, \texttt{P[a,a]} tells us we have a cycle. Should we find
        such an expression we replace it with $q+q^{-1}$.
    \end{frame}
    \begin{frame}{The Jones Polynomial}
        Expanding the product on the previous page results in $2^{N}$
        terms, meaning this algorithm is still exponential. Fortunately there
        is a trick to significantly speed up our computation.
        \par\hfill\par
        Instead of expanding the product all at once, we pick a crossing and
        perform the smoothing
        $\texttt{X[a,b,c,d]}\mapsto\texttt{P[a,b]P[c,d]}-q\texttt{P[a,d]P[b,c]}$.
        We then pick the crossing that has the most arcs in common with
        \texttt{X[a,b,c,d]} and smooth that one. That is, we look for the
        crossing \texttt{X[e,f,g,h]} such that
        \texttt{a,b,c,d} and \texttt{e,f,g,h} have as many terms in common as
        possible.
    \end{frame}
    \begin{frame}{The Jones Polynomial}
        We combine the products and simplify using the previous relations.
        By adding the crossings in such a way we grow our
        \textit{computational front} as slowly as possible, meaning we have
        less work to do in the end.
        \par\hfill\par
        This idea is implemented in \cite{KatlasJones} and
        \cite{MaguireJones} and gives a considerable speed boost for many
        knot diagrams.
    \end{frame}
    \begin{frame}{Khovanov Homology}
        The other benefit is that it can be generalized and allow us to compute
        Khovanov homology. Khovanov homology is a link invariant that
        associates a chain complex to a link diagram, the homology of which
        is a link invariant.
        \par\hfill\par
        Following \cite{BarNatanKhovanovJones} we again consider the cube of
        resolutions and associate to each vertex a vector space (or module)
        $V^{\otimes{c(n)}}$ where $V$ is a graded vector space (or free module)
        of graded dimension $q+q^{-1}$. The chain groups are obtained by
        direct sums over the vertices of the same Hamming weight. The chain
        maps are defined by summing over local maps defined on the edges of
        the cube.
    \end{frame}
    \begin{frame}{The Jones Polynomial}
        \begin{figure}
            \centering
            \includegraphics{trefoil_knot_cube_of_resolutions.pdf}
            \caption{Cube of Resolutions for the Trefoil}
        \end{figure}
    \end{frame}
    \begin{frame}{Khovanov Homology}
        The resulting homology can be decomposed into homogeneous parts
        \cite{KatlasKhoHo}, and the \textit{Khovanov polynomial} is defined
        by the resulting Poincar\'{e} polynomial
        \begin{equation}
            Kh_{L}(q,\,t)=\sum_{n,\,m}q^{n}t^{m}
                \textrm{dim}\big(Kh_{m}^{n}(L)\big)
        \end{equation}
        The Jones polynomial is recovered via
        \begin{equation}
            J_{L}(q)=Kh_{L}(q,\,-1)
        \end{equation}
        In \cite{BarNatan2006FASTKH} an explicit algorithm is outlined for
        the computation and this is implemented in \cite{JavaKhv2}.\footnote{%
            Work in progress implementations in \texttt{C} can be found in
            \cite{MaguireKnotData} and \cite{MaguireLibtmpl}.
        }
    \end{frame}
    \begin{frame}{Khovanov Homology}
        The Khovanov polynomial of all prime knots up to 17 crossings has been
        tabulated and is available in \cite{KhovanovData}.
        \par\hfill\par
        The computation took 6 weeks using fairly powerful hardware.
        19 crossings would have taken about 3 years. Optimizations for low
        crossing knots are underway, as is experimenting with parallelization.
        This may allow the computation to be done in months, instead of years.
    \end{frame}
    \begin{frame}{HOMFLY-PT Polynomial}
        The final invariant to be experimented with is the HOMFLY-PT
        polynomial. Like Khovanov homology, it generalizes the Jones polynomial,
        but also generalizes the Alexander polynomial. It is defined via
        \begin{align}
            P(\mathbb{S}^{1})&=1\\
            \alpha{P}(L_{n,\,+})-\alpha^{-1}P(L_{n,\,-})
            &=zP(L_{n,\,0})
        \end{align}
        Where $L_{n,\,+}$ denotes the $n^{\small\textrm{th}}$ crossing as a
        positive crossing, $L_{n,\,-}$ denotes swapping the strands over each
        other to a negative crossing, and $L_{n,\,0}$ is the
        orientation preserving smoothing.
    \end{frame}
    \begin{frame}{HOMFLY-PT Polynomial}
        The Jones and Alexander polynomial, $J_{K}$ and $\Delta_{K}$,
        respectively, are recovered via
        \begin{align}
            J_{K}(q)&=P(q^{-2},\,q-q^{-1})\\
            \Delta_{K}(q)&=P(1,\,q-q^{-1})
        \end{align}
        So the computation is at least as hard as the Jones polynomial, and
        indeed it is \textbf{NP-Hard} \cite{HOMFLYPTNPHard}.
    \end{frame}
    \begin{frame}{HOMFLY-PT Polynomial}
        Three algorithms exist, of which I'm still in the early stages of
        experimenting.
        \begin{enumerate}
            \item Kauffman's skein template algorithm \cite{KauffmanStateModelsLinkPolynomials}.
            \item Gouesbet, Meunier-Guttin-Cluzel, and Letellier's algorithm using Gauss codes \cite{GouesbetHOMFLYAlgorithm}.
            \item Burton's modification of Kauffman's algorithm \cite{Burton2018HOMFLFixedParameter}.
        \end{enumerate}
        Burton's algorithm gives the best performance, and he provided a
        \texttt{C++} implementation as well \cite{regina}. Slight experiments
        with this code have led to the tabulation of the HOMFLY-PT polynomial
        of all prime knots up to 19 crossings
        \cite{HOMFLYData}.
    \end{frame}
    \begin{frame}{Results}
        \begin{table}
            \centering
            \resizebox{0.8\textwidth}{!}{%
                \begin{tabular}{| l | l |}
                    \hline
                    Keyword & Description\\
                    \hline
                    Cr     & Crossing number, largest number of crossings considered.\\
                    Unique & Number of polynomials that occur for one knot.\\
                    Almost & Number of polynomials that occur for two knots.\\
                    Total  & Total number of distinct polynomials in list.\\
                    Knots  & Total number of knots in list.\\
                    FracU  & Unique / Total\\
                    FracT  & Total / Knots\\
                    FracK  & Unique / Knots\\
                    \hline
                \end{tabular}
            }
            \caption{Legend for the Statistics Table}
        \end{table}
    \end{frame}
    \begin{frame}{Results}
        \begin{table}
            \centering
            \resizebox{\textwidth}{!}{%
                \begin{tabular}{| r | r | r | r | r | r | r | r |}
                    \hline
                    Cr &  Unique  &  Almost  &   Total   &   Knots    &  FracU   &  FracT   &  FracK\\
                    \hline
                    03 &        1 &        0 &         1 &         1 & 1.000000 & 1.000000 & 1.000000\\
                    04 &        2 &        0 &         2 &         2 & 1.000000 & 1.000000 & 1.000000\\
                    05 &        4 &        0 &         4 &         4 & 1.000000 & 1.000000 & 1.000000\\
                    06 &        7 &        0 &         7 &         7 & 1.000000 & 1.000000 & 1.000000\\
                    07 &       14 &        0 &        14 &        14 & 1.000000 & 1.000000 & 1.000000\\
                    08 &       35 &        0 &        35 &        35 & 1.000000 & 1.000000 & 1.000000\\
                    09 &       84 &        0 &        84 &        84 & 1.000000 & 1.000000 & 1.000000\\
                    10 &      223 &       13 &       236 &       249 & 0.944915 & 0.947791 & 0.895582\\
                    11 &      626 &       77 &       710 &       801 & 0.881690 & 0.886392 & 0.781523\\
                    12 &     1981 &      345 &      2420 &      2977 & 0.818595 & 0.812899 & 0.665435\\
                    13 &     6855 &     1695 &      9287 &     12965 & 0.738129 & 0.716313 & 0.528731\\
                    14 &    25271 &     7439 &     37578 &     59937 & 0.672495 & 0.626958 & 0.421626\\
                    15 &   105246 &    35371 &    170363 &    313230 & 0.617775 & 0.543891 & 0.336002\\
                    16 &   487774 &   173677 &    829284 &   1701935 & 0.588187 & 0.487260 & 0.286600\\
                    17 &  2498968 &   894450 &   4342890 &   9755328 & 0.575416 & 0.445181 & 0.256164\\
                    18 & 13817237 &  4863074 &  24116048 &  58021794 & 0.572948 & 0.415638 & 0.238139\\
                    19 & 82712788 & 27409120 & 141439472 & 352152252 & 0.584793 & 0.401643 & 0.234878\\
                    \hline
                \end{tabular}
            }
            \caption{Statistics for the Jones Polynomial}
        \end{table}
    \end{frame}
    \begin{frame}{Results}
        This raises the question, if $K_{n}$ is the number of prime
        knots with at most $n$ crossings, and if $J_{n}$ is the number of
        unique Jones polynomials for prime knots up to $n$ crossings, does
        $J_{n}/K_{n}$ converge to zero?
    \end{frame}
    \begin{frame}{Results}
        \begin{table}
            \centering
            \resizebox{\textwidth}{!}{%
                \begin{tabular}{| r | r | r | r | r | r | r | r |}
                    \hline
                    Cr & Unique  & Almost  &  Total  &  Knots  &  FracU   &  FracT   &  FracK\\
                    \hline
                    03 &       1 &       0 &       1 &       1 & 1.000000 & 1.000000 & 1.000000\\
                    04 &       2 &       0 &       2 &       2 & 1.000000 & 1.000000 & 1.000000\\
                    05 &       4 &       0 &       4 &       4 & 1.000000 & 1.000000 & 1.000000\\
                    06 &       7 &       0 &       7 &       7 & 1.000000 & 1.000000 & 1.000000\\
                    07 &      14 &       0 &      14 &      14 & 1.000000 & 1.000000 & 1.000000\\
                    08 &      35 &       0 &      35 &      35 & 1.000000 & 1.000000 & 1.000000\\
                    09 &      84 &       0 &      84 &      84 & 1.000000 & 1.000000 & 1.000000\\
                    10 &     225 &      12 &     237 &     249 & 0.949367 & 0.951807 & 0.903614\\
                    11 &     641 &      71 &     718 &     801 & 0.892758 & 0.896380 & 0.800250\\
                    12 &    2051 &     326 &    2462 &    2977 & 0.833063 & 0.827007 & 0.688949\\
                    13 &    7223 &    1636 &    9539 &   12965 & 0.757207 & 0.735750 & 0.557115\\
                    14 &   27317 &    7441 &   39222 &   59937 & 0.696471 & 0.654387 & 0.455762\\
                    15 &  118534 &   36867 &  182598 &  313230 & 0.649153 & 0.582952 & 0.378425\\
                    16 &  578928 &  187639 &  919835 & 1701935 & 0.629382 & 0.540464 & 0.340159\\
                    17 & 3167028 & 1001101 & 5033403 & 9755328 & 0.629202 & 0.515965 & 0.324646\\
                    \hline
                \end{tabular}
            }
            \caption{Statistics for the Khovanov Polynomial}
        \end{table}
    \end{frame}
    \begin{frame}{Results}
        We can ask the same question for Khovanov homology. Labeling
        $Kh_{n}$ and $K_{n}$ in a similar way, does
        $Kh_{n}/K_{n}$ converge to zero?
        \par\hfill\par
        Note, as expected, $Kh_{n}>J_{n}$. This simply reiterates the fact
        that $Kh_{L}(q,\,-1)=J_{L}(q)$ for any link $L$.
        \par\hfill\par
        While $4_{1}$ and $K11n19$ have the same Jones polynomial
        (but different Alexander polynomials), their Khovanov polynomials
        differ, further cementing the claim that the knots are different.
    \end{frame}
    \begin{frame}{Results}
        \begin{table}
            \centering
            \resizebox{\textwidth}{!}{%
                \begin{tabular}{| r | r | r | r | r | r | r | r |}
                    \hline
                    Cr & Unique  & Almost  &  Total  &  Knots  &  FracU   &  FracT   &  FracK\\
                    \hline
                    03 &         1 &        0 &         1 &         1 & 1.000000 & 1.000000 & 1.000000\\
                    04 &         2 &        0 &         2 &         2 & 1.000000 & 1.000000 & 1.000000\\
                    05 &         4 &        0 &         4 &         4 & 1.000000 & 1.000000 & 1.000000\\
                    06 &         7 &        0 &         7 &         7 & 1.000000 & 1.000000 & 1.000000\\
                    07 &        14 &        0 &        14 &        14 & 1.000000 & 1.000000 & 1.000000\\
                    08 &        35 &        0 &        35 &        35 & 1.000000 & 1.000000 & 1.000000\\
                    09 &        84 &        0 &        84 &        84 & 1.000000 & 1.000000 & 1.000000\\
                    10 &       241 &        4 &       245 &       249 & 0.983673 & 0.983936 & 0.967871\\
                    11 &       730 &       34 &       765 &       801 & 0.954248 & 0.955056 & 0.911361\\
                    12 &      2494 &      210 &      2724 &      2977 & 0.915565 & 0.915015 & 0.837756\\
                    13 &      9475 &     1302 &     11044 &     12965 & 0.857932 & 0.851832 & 0.730814\\
                    14 &     39401 &     7170 &     48329 &     59937 & 0.815266 & 0.806330 & 0.657374\\
                    15 &    186799 &    38833 &    238614 &    313230 & 0.782850 & 0.761785 & 0.596364\\
                    16 &    979987 &   209669 &   1266261 &   1701935 & 0.773922 & 0.744013 & 0.575808\\
                    17 &   5559808 &  1157938 &   7175287 &   9755328 & 0.774855 & 0.735525 & 0.569925\\
                    18 &  33722920 &  6480965 &  42857755 &  58021794 & 0.786857 & 0.738649 & 0.581211\\
                    19 & 213355372 & 36387952 & 264839694 & 352152252 & 0.805602 & 0.752060 & 0.605861\\
                    \hline
                \end{tabular}
            }
            \caption{Statistics for the HOMFLY-PT Polynomial}
        \end{table}
    \end{frame}
    \begin{frame}{Results}
        Most surprising, defining $H_{n}$ to be the number of unique
        HOMFLY-PT polynomials for prime knots up to $n$ crossings, we see that
        $H_{n}/K_{n}$ is not monotonic. Also surprising is how high the ratio
        is for 19 crossings.
        \par\hfill\par
        It would be intriguing to know what the
        $\textrm{lim}\;\textrm{sup}$ and $\textrm{lim}\;\textrm{inf}$ are.
    \end{frame}
    \begin{frame}{Results}
        Future work
        \begin{enumerate}
            \item Finish FastKH reimplementation.
            \begin{itemize}
                \item Parallelize batch computations.
                \item Parallelize individual computations?
                \item Avoid arbitrary precision arithmetic.
                \item \texttt{C} vs. \texttt{Java}, will there be noticeable
                    performance differences?
            \end{itemize}
            \item Rewrite Alexander algorithms.
            \begin{itemize}
                \item The ones used are in \texttt{Sage}, \texttt{Python}, and
                    \texttt{Mathematica}. My \texttt{C} routines are in the
                    early stages. Gotten nice speed improvements so far.
            \end{itemize}
            \item Fixed parameter tractability for Khovanov homology?
            \item Speed tests!
            \begin{itemize}
                \item More than half a dozen Jones polynomial algorithms.
                \item A few Alexander polynomial algorithms.
                \item Three HOMFLY-PT algorithms.
                \item Which are the fastest for general knots?
            \end{itemize}
            \item Do all of this again for 20 crossings!
        \end{enumerate}
    \end{frame}
    \begin{frame}{The End}
        \begin{center}
            Thank You!
        \end{center}
    \end{frame}
    \begin{frame}[allowframebreaks]{References}
        \bibliographystyle{annotate}
        \bibliography{bib.bib}
    \end{frame}
\end{document}
