%-----------------------------------LICENSE------------------------------------%
%   This file is part of Mathematics-and-Physics.                              %
%                                                                              %
%   Mathematics-and-Physics is free software: you can redistribute it and/or   %
%   modify it under the terms of the GNU General Public License as             %
%   published by the Free Software Foundation, either version 3 of the         %
%   License, or (at your option) any later version.                            %
%                                                                              %
%   Mathematics-and-Physics is distributed in the hope that it will be useful, %
%   but WITHOUT ANY WARRANTY; without even the implied warranty of             %
%   MERCHANTABILITY or FITNESS FOR A PARTICULAR PURPOSE.  See the              %
%   GNU General Public License for more details.                               %
%                                                                              %
%   You should have received a copy of the GNU General Public License along    %
%   with Mathematics-and-Physics.  If not, see <https://www.gnu.org/licenses/>.%
%------------------------------------------------------------------------------%
%   Author:     Ryan Maguire                                                   %
%   Date:       February 17, 2022                                              %
%------------------------------------------------------------------------------%
\documentclass{beamer}
\usepackage{amsmath}
\graphicspath{{../images/}}
\title{Khovanov Homology and Legendrian Simple Knots}
\author{Ryan Maguire}
\date{March 5, 2024}
\usenavigationsymbolstemplate{}
\setbeamertemplate{footline}[frame number]
\begin{document}
    \maketitle
    \begin{frame}{Outline}
        \begin{enumerate}
            \item Knots and invariants.
            \item Contact topology.
            \item Results.
        \end{enumerate}
    \end{frame}
    \begin{frame}{Knots and Links}
        Intuitively we all know what a knot is, a loop of string in space that
        may be tangled in some way. They were a part
        of various cultures long before they became of interest to the
        mathematical community.
        \begin{enumerate}
            \item The Celtic book of Kells has many complicated knot drawings.
            \item The Viking legend of Hildr contains the Borromean rings.
            \item Tibetan Buddhism uses the endless knot as a religious symbol.
            \item The hammer of Thor, Mj\"{o}lnir, occasionally depicts the Whitehead link.
        \end{enumerate}
        Other knots can be found in Islamic, Jewish, and Japanese cultures,
        with some examples being well over 1000 years old.
    \end{frame}
    \begin{frame}{Knots and Links}
        Mathematically knots are relatively new. Early investigations occurred
        in the 1700s and early 1800s, but the theory picked up popularity in
        the mid and late 1800s for two reasons.
        \begin{enumerate}
            \item
                Many problems in electromagnetism involved charged loops
                of wire that may be linked together.
            \item
                Vortex theory, an early attempt at explaining the structure of
                atoms, used knots and links to describe chemical properties.
        \end{enumerate}
        Vortex theory was eventually disproved by J. J. Thompson, ironically
        as he was attempting to provide evidence for it. Knots were not
        entirely abandoned and were absorbed into topology in the
        early $20^{\small\textrm{th}}$ century.
    \end{frame}
    \begin{frame}{Knots and Links}
        Why might we care about knots and links? Two theorems I like.
        \begin{theorem}[Lickorish-Wallace, 1960]
            Every compact orientable connected three dimensional manifold
            can be obtained by taking the 3-sphere $\mathbb{S}^{3}$ and cutting
            out a link $L$ and gluing it back together with some twists.%
            \footnote{In technical terms, this is a Dehn surgery.}
        \end{theorem}
        \begin{theorem}[Steinitz-Tait, 1877, 1922]
            Every connected planar graph with signed edges can be represented
            by a knot diagram. Conversely, every knot diagram can be
            represented by two connected graphs with signed edges that are
            related to each other as planar duals.
        \end{theorem}
    \end{frame}
    \begin{frame}{Knots and Links}
        So what is a knot?
        A \textbf{knot} is a smooth or polygonal embedding
        $\gamma:\mathbb{S}^{1}\rightarrow\mathbb{R}^{3}$.
        A \textbf{knot diagram} is a projection of a knot onto the plane where
        we mark the overlapping points in some way so that we can easily
        identify which part is going over and which is under.
        \begin{figure}
            \centering
            \includegraphics{trefoil_knot_001}
            \caption{Knot Diagram}
        \end{figure}
    \end{frame}
    \begin{frame}
        We'll say two knots are \textbf{equivalent} if we can
        smoothly deform one into the other without tearing or cutting.
        \par\hfill\par
        Knot diagrams turn geometric and analytical objects
        (functions and embeddings) into combinatorial ones. In the 1920s it was
        shown that the notion of knot equivalence can be demonstrated
        using these diagrams and the three \textbf{Reidemeister moves}.
    \end{frame}
    \begin{frame}{Knots and Links}
        \begin{figure}
            \centering
            \begin{minipage}[b]{0.49\textwidth}
                \centering
                \resizebox{\textwidth}{!}{%
                    \includegraphics{reidemeister_1_move}
                }
                \caption{Type I Move}
            \end{minipage}
            \hfill
            \begin{minipage}[b]{0.49\textwidth}
                \centering
                \resizebox{\textwidth}{!}{%
                    \includegraphics{reidemeister_2_move}
                }
                \caption{Type II Move}
            \end{minipage}
        \end{figure}
    \end{frame}
    \begin{frame}{Knots and Links}
        \begin{figure}
            \centering
            \resizebox{\textwidth}{!}{%
                \includegraphics{reidemeister_3_move}
            }
            \caption{Type III Move}
        \end{figure}
    \end{frame}
    \begin{frame}{Knots and Links}
        In practice we now have a means of distinguishing two knot diagrams.
        The number of moves can be quite high, however. Even if one of the
        diagrams is a simple circle with no other crossings, if the other
        diagram has $n$ crossing it may take more than
        $10^{26}\times{n}^{11}$ Reidemeister moves to undo it.\footnote{%
            This is the best known bound at the time of this writing.
        }
        \par\hfill\par
        Determining if a knot diagram represents an unknotted circle then
        takes roughly $3^{10^{26}\times{n}^{11}}$ steps and we quickly
        approach the \textit{transcomputational problem} as $n$ increases.%
        \footnote{%
            See the \textit{Bremermann limit} for details on transcomputation.
        }
        \par\hfill\par
        Instead of doing this we invent \textit{invariants}, which are usually
        algebraic objects attached to knot diagrams that do not change with
        the three Reidemeister moves. We'll be working with two of these.
    \end{frame}
    \begin{frame}{The Jones Polynomial}
        The Jones polynomial assigns a Laurent polynomial, a polynomial in
        $q$ and $q^{-1}$ with integer coefficients, to knot diagrams and is
        an invariant. We can define it pictorially using the Kauffman bracket
        in terms of smoothings.
        \begin{figure}
            \centering
            \includegraphics{resolving_crossings}
            \caption{Smoothing Crossings}
        \end{figure}
    \end{frame}
    \begin{frame}{The Jones Polynomial}
        The Kauffman bracket is defined recursively. Given a knot (or link)
        diagram $L$ we write:
        \begin{align}
            \langle{\emptyset}\rangle
            &=1\\
            \langle{\mathbb{S}^{1}\sqcup{L}}\rangle
            &=(q+q^{-1})\langle{L}\rangle\\
            \langle{L}\rangle
            &=\langle{L_{n,\,0}}\rangle-q\langle{L_{n,\,1}}\rangle
        \end{align}
        where $\mathbb{S}^{1}\sqcup{L}$ represents the disjoint union of $L$
        and a circle. The Jones polynomial $J_{L}(q)$ is obtained from
        $\langle{L}\rangle$ by normalizing it by a factor
        $\pm{q}^{k}$ for a particular $k$ that depends on the diagram. The
        computation of this factor is linear in the number of crossings so we
        can restrict our attention to the Kauffman bracket.
    \end{frame}
    \begin{frame}{The Jones Polynomial}
        If we follow this recursive formula we end up with $2^{n}$ different
        ways of completely smoothing the diagram so that there are no crossings
        left.
        \par\hfill\par
        If we label the crossings 1 to $n$ then every possible smoothing
        uniquely corresponds to a number between 1 and $2^{n}$. Write your
        number $1\leq{k}\leq{2}^{n}$ in binary. If the $m^{\small\textrm{th}}$
        bit is 0 do a 0-smoothing at the $m^{\small\textrm{th}}$ crossing,
        otherwise do a 1-smoothing. The resulting
        $2^{n}$ pictures is called the \textbf{cube of resolutions} for the
        diagram.
    \end{frame}
    \begin{frame}{The Jones Polynomial}
        \begin{figure}
            \centering
            \includegraphics{trefoil_knot_cube_of_resolutions}
            \caption{Cube of Resolutions for the Trefoil}
        \end{figure}
    \end{frame}
    \begin{frame}{The Jones Polynomial}
        \begin{figure}
            \centering
            \resizebox{!}{0.8\textheight}{%
                \includegraphics{figure_eight_knot_cube_of_resolutions}
            }
            \caption{Cube of Resolutions for the Figure-Eight}
        \end{figure}
    \end{frame}
    \begin{frame}{The Jones Polynomial}
        The Kauffman bracket is obtained by counting the cycles in the
        smoothings and summing over them with appropriates weights. We obtain
        the formula
        \begin{equation}
            \langle{L}\rangle
            =\sum_{k=1}^{2^{n}}(-q)^{w(k)}(q+q^{-1})^{c(k)}
        \end{equation}
        where $w(k)$ is the Hamming weight, the number of 1's that occur in
        the binary expansion of $k$, and $c(k)$ is the number of cycles
        corresponding to the $k^{\small\textrm{th}}$ complete smoothing.
    \end{frame}
    \begin{frame}{The Jones Polynomial}
        Several algorithms exist for the computation of the Jones polynomial
        and Kauffman bracket, and the complexity is known to be
        \textbf{NP-Hard}. The computation has been studied in the
        quantum aspect and a polynomial-time additive approximation method
        exists.
        \par\hfill\par
        The algorithm we'll discuss is one of the simplest, but it lends itself
        to a clever speed-up trick and a means of computing Khovanov homology
        as well, which we'll soon discuss.
    \end{frame}
    \begin{frame}
        We first define $PD$ code. Take a knot diagram and pick a starting
        point. Trace your finger around the knot and label the arcs in
        increasing order. At each under crossing write
        \texttt{X[a,b,c,d]} where \texttt{a} is arc you're walking on,
        \texttt{b} is the strand to your right, \texttt{c} is the strand
        in front, and \texttt{d} is the strand to the left.
        \begin{figure}
            \centering
            \resizebox{0.4\textwidth}{!}{%
                \includegraphics{trefoil_knot_arcs_labeled}
            }
            \caption{Trefoil with Arcs Labeled}
        \end{figure}
    \end{frame}
    \begin{frame}{The Jones Polynomial}
        The $PD$ code of the diagram is the ordered sequence of quadruples
        \texttt{X[a,b,c,d]} for each under crossing. For the trefoil on the
        previous page we get \texttt{X[1,5,2,4]}, \texttt{X[3,1,4,0]},
        \texttt{X[5,3,0,2]}.
        \par\hfill\par
        The Kauffman bracket is now computed with a \textit{symbolic calculus}.
        \begin{figure}
            \centering
            \includegraphics{resolving_crossings_pd_label}
            \caption{Motive for the Symbolic Calculus}
        \end{figure}
    \end{frame}
    \begin{frame}{The Jones Polynomial}
        The Kauffman relation tells us to replace \texttt{X[a,b,c,d]} with
        $\texttt{P[a,b]P[c,d]}-q\texttt{P[a,d]P[b,c]}$. The Kauffman bracket
        becomes a product of these formal polynomials:
        \begin{equation}
            \prod_{k=1}^{n}(
                \texttt{P[ak,bk]P[ck,dk]}-q\texttt{P[ak,dk]P[bk,cd]}
            )
        \end{equation}
        After expanding we have formal products of ordered pairs
        \texttt{P[a,b]}. We simplify via
        $\texttt{P[a,b]P[b,c]}\mapsto\texttt{P[a,c]}$ and
        $\texttt{P[a,b]P[a,b]}\mapsto\texttt{P[a,b]}$. The Kauffman relation
        then tells us to remove cycles and replace them with $q+q^{-1}$.
        This amounts to $\texttt{P[a,a]}\mapsto{q}+q^{-1}$.
    \end{frame}
    \begin{frame}{The Jones Polynomial}
        This gives us a quick algorithm to implement, but we can improve it.
        Rather than smoothing all of the crossings at once and expanding we
        pick one and resolve it using
        \begin{equation}
            \nonumber
            \texttt{X[a,b,c,d]}
            \mapsto\texttt{P[a,b]P[c,d]}-q\texttt{P[a,d]P[b,c]}
        \end{equation}
        We then choose the next crossing \texttt{X[e,f,g,h]} with the most
        numbers in common with \texttt{X[a,b,c,d]}. We perform the same
        replacement and simplify. We then add the next crossing with the
        most numbers in common with the ones we've already worked with.
        \par\hfill\par
        By adding crossings in this way we grow our
        \textit{computational front} minimally, ensuring we have less work to
        do in the end. This simple trick can dramatically improve performance.
    \end{frame}
    \begin{frame}{The Jones Polynomial}
        We've implemented several algorithms for the Jones polynomial. Using
        this we were able to tabulate the invariant for all prime knots up to
        19 crossings, over 352 million knots.
        \par\hfill\par
        We also examined the strength of the invariant. Below is the
        legend for the table on the next page.
        \begin{table}
            \centering
            \resizebox{0.8\textwidth}{!}{%
                \begin{tabular}{| l | l |}
                    \hline
                    Keyword & Description\\
                    \hline
                    Cr     & Crossing number, largest number of crossings considered.\\
                    Unique & Number of polynomials that occur for one knot.\\
                    Almost & Number of polynomials that occur for two knots.\\
                    Total  & Total number of distinct polynomials in list.\\
                    Knots  & Total number of knots in list.\\
                    FracU  & Unique / Total\\
                    FracT  & Total / Knots\\
                    FracK  & Unique / Knots\\
                    \hline
                \end{tabular}
            }
            \caption{Legend for the Statistics Table}
        \end{table}
    \end{frame}
    \begin{frame}{The Jones Polynomial}
        \begin{table}
            \centering
            \resizebox{\textwidth}{!}{%
                \begin{tabular}{| r | r | r | r | r | r | r | r |}
                    \hline
                    Cr &  Unique  &  Almost  &   Total   &   Knots    &  FracU   &  FracT   &  FracK\\
                    \hline
                    03 &        1 &        0 &         1 &         1 & 1.000000 & 1.000000 & 1.000000\\
                    04 &        2 &        0 &         2 &         2 & 1.000000 & 1.000000 & 1.000000\\
                    05 &        4 &        0 &         4 &         4 & 1.000000 & 1.000000 & 1.000000\\
                    06 &        7 &        0 &         7 &         7 & 1.000000 & 1.000000 & 1.000000\\
                    07 &       14 &        0 &        14 &        14 & 1.000000 & 1.000000 & 1.000000\\
                    08 &       35 &        0 &        35 &        35 & 1.000000 & 1.000000 & 1.000000\\
                    09 &       84 &        0 &        84 &        84 & 1.000000 & 1.000000 & 1.000000\\
                    10 &      223 &       13 &       236 &       249 & 0.944915 & 0.947791 & 0.895582\\
                    11 &      626 &       77 &       710 &       801 & 0.881690 & 0.886392 & 0.781523\\
                    12 &     1981 &      345 &      2420 &      2977 & 0.818595 & 0.812899 & 0.665435\\
                    13 &     6855 &     1695 &      9287 &     12965 & 0.738129 & 0.716313 & 0.528731\\
                    14 &    25271 &     7439 &     37578 &     59937 & 0.672495 & 0.626958 & 0.421626\\
                    15 &   105246 &    35371 &    170363 &    313230 & 0.617775 & 0.543891 & 0.336002\\
                    16 &   487774 &   173677 &    829284 &   1701935 & 0.588187 & 0.487260 & 0.286600\\
                    17 &  2498968 &   894450 &   4342890 &   9755328 & 0.575416 & 0.445181 & 0.256164\\
                    18 & 13817237 &  4863074 &  24116048 &  58021794 & 0.572948 & 0.415638 & 0.238139\\
                    19 & 82712788 & 27409120 & 141439472 & 352152252 & 0.584793 & 0.401643 & 0.234878\\
                    \hline
                \end{tabular}
            }
            \caption{Statistics for the Jones Polynomial}
        \end{table}
    \end{frame}
    \begin{frame}{Khovanov Homology}
        Khovanov homology is our next invariant, which generalizes the Jones
        polynomial. We replace a polynomial with a homological object. The
        Khovanov bracket is defined via
        \begin{align}
            [[\emptyset]]&=0\rightarrow\mathbb{Z}\rightarrow{0}\\
            [[\mathbb{S}^{1}\sqcup{L}]]&=V\otimes[[L]]\\
            [[L]]&=\mathcal{F}
            \big(
                0\rightarrow[[L_{n,\,0}]]
                \rightarrow[[L_{n,\,1}]]\{1\}\rightarrow{0}
            \big)
        \end{align}
        where $V$ is a graded vector space of graded dimension
        $q+q^{-1}$\;\footnote{%
            Free modules work too.
        }
        and $\mathcal{F}$ is the flatten operation that takes a
        double complex into a chain complex by direct sums along diagonals.
    \end{frame}
    \begin{frame}{Khovanov Homology}
        The differential between $[[L_{n,\,0}]]$ and
        $[[L_{n,\,1}]]\{1\}$ is defined pictorially. We use the cube of
        resolutions of the knot or link diagram, for example.
        Recalling our previous binary notation, if two strings differ in only
        one place then there is an edge between them in the cube of resolutions.
        \par\hfill\par
        The edge describes a cobordism
        (a pair of pants) that either fuses two cycles into one or splits a cycle
        into two. Fusing amounts to a homomorphism between
        $V\otimes{V}$ and $V$, whereas splitting needs a map from
        $V$ to $V\otimes{V}$.
    \end{frame}
    \begin{frame}{Khovanov Homology}
        These are the homomorphisms.
        \begin{align}
            m(v_{-}\otimes{v}_{-})&=\mathbf{0}\\
            m(v_{-}\otimes{v}_{+})&=v_{-}\\
            m(v_{+}\otimes{v}_{-})&=v_{-}\\
            m(v_{+}\otimes{v}_{+})&=v_{+}\\
            \Delta(v_{-})&=v_{-}\otimes{v}_{-}\\
            \Delta(v_{+})&=v_{+}\otimes{v}_{-}+v_{-}\otimes{v}_{+}
        \end{align}
        The differential is defined by an alternating sum of these
        homomorphisms along strings of equal hamming weight.
    \end{frame}
    \begin{frame}{The Jones Polynomial}
        \begin{figure}
            \centering
            \includegraphics{trefoil_knot_cube_of_resolutions}
            \caption{Cube of Resolutions for the Trefoil}
        \end{figure}
    \end{frame}
    \begin{frame}{Khovanov Homology}
        The differential does indeed square to zero and we get a homology
        out of this. The $r^{\small\textrm{th}}$ homology group
        $Kh^{r}(L)$ is the direct sum of homogeneous parts
        $Kh^{r}_{s}(L)$ and the Khovanov polynomial of the diagram is the
        Poincar\'{e} polynomial of the homology
        \begin{equation}
            Kh_{L}(q,\,t)=
            \sum_{r,\,s}q^{r}t^{s}\textrm{dim}\big(Kh_{s}^{r}(L)\big)
        \end{equation}
        The Jones polynomial is recovered via
        \begin{equation}
            J_{L}(q)=Kh_{L}(q,\,-1)
        \end{equation}
    \end{frame}
    \begin{frame}{Khovanov Homology}
        The symbolic calculus can be modified for Khovanov homology and the
        Khovanov polynomial.
        \par\hfill\par
        By experimenting with the \texttt{JavaKh} library we were able to
        tabulate the Khovanov polynomial of all prime knots up to 17 crossings.
        This took about two months, 19 crossings would have taken two years.
        We're currently trying to speed up the computations to make this more
        approachable.
    \end{frame}
    \begin{frame}{Khovanov Homology}
        \begin{table}
            \centering
            \resizebox{\textwidth}{!}{%
                \begin{tabular}{| r | r | r | r | r | r | r | r |}
                    \hline
                    Cr & Unique  & Almost  &  Total  &  Knots  &  FracU   &  FracT   &  FracK\\
                    \hline
                    03 &       1 &       0 &       1 &       1 & 1.000000 & 1.000000 & 1.000000\\
                    04 &       2 &       0 &       2 &       2 & 1.000000 & 1.000000 & 1.000000\\
                    05 &       4 &       0 &       4 &       4 & 1.000000 & 1.000000 & 1.000000\\
                    06 &       7 &       0 &       7 &       7 & 1.000000 & 1.000000 & 1.000000\\
                    07 &      14 &       0 &      14 &      14 & 1.000000 & 1.000000 & 1.000000\\
                    08 &      35 &       0 &      35 &      35 & 1.000000 & 1.000000 & 1.000000\\
                    09 &      84 &       0 &      84 &      84 & 1.000000 & 1.000000 & 1.000000\\
                    10 &     225 &      12 &     237 &     249 & 0.949367 & 0.951807 & 0.903614\\
                    11 &     641 &      71 &     718 &     801 & 0.892758 & 0.896380 & 0.800250\\
                    12 &    2051 &     326 &    2462 &    2977 & 0.833063 & 0.827007 & 0.688949\\
                    13 &    7223 &    1636 &    9539 &   12965 & 0.757207 & 0.735750 & 0.557115\\
                    14 &   27317 &    7441 &   39222 &   59937 & 0.696471 & 0.654387 & 0.455762\\
                    15 &  118534 &   36867 &  182598 &  313230 & 0.649153 & 0.582952 & 0.378425\\
                    16 &  578928 &  187639 &  919835 & 1701935 & 0.629382 & 0.540464 & 0.340159\\
                    17 & 3167028 & 1001101 & 5033403 & 9755328 & 0.629202 & 0.515965 & 0.324646\\
                    \hline
                \end{tabular}
            }
            \caption{Statistics for the Khovanov Polynomial}
        \end{table}
    \end{frame}
    \begin{frame}{HOMFLY-PT}
        Slight digression, the HOMFLY-PT polynomial was also investigated.
        It too generalizes the Jones polynomial (and also the Alexander
        polynomial).
        \par\hfill\par
        It is different than the Khovanov polynomial, there are
        knots with different HOMFLY-PT polynomials but the same Khovanov
        polynomial, and vice-versa.
        \par\hfill\par
        Using the \texttt{regina} library we've tabulated the HOMFLY-PT
        polynomial of all prime knots up to 19 crossings.
    \end{frame}
    \begin{frame}{HOMFLY-PT}
        \begin{table}
            \centering
            \resizebox{\textwidth}{!}{%
                \begin{tabular}{| r | r | r | r | r | r | r | r |}
                    \hline
                    Cr & Unique  & Almost  &  Total  &  Knots  &  FracU   &  FracT   &  FracK\\
                    \hline
                    03 &         1 &        0 &         1 &         1 & 1.000000 & 1.000000 & 1.000000\\
                    04 &         2 &        0 &         2 &         2 & 1.000000 & 1.000000 & 1.000000\\
                    05 &         4 &        0 &         4 &         4 & 1.000000 & 1.000000 & 1.000000\\
                    06 &         7 &        0 &         7 &         7 & 1.000000 & 1.000000 & 1.000000\\
                    07 &        14 &        0 &        14 &        14 & 1.000000 & 1.000000 & 1.000000\\
                    08 &        35 &        0 &        35 &        35 & 1.000000 & 1.000000 & 1.000000\\
                    09 &        84 &        0 &        84 &        84 & 1.000000 & 1.000000 & 1.000000\\
                    10 &       241 &        4 &       245 &       249 & 0.983673 & 0.983936 & 0.967871\\
                    11 &       730 &       34 &       765 &       801 & 0.954248 & 0.955056 & 0.911361\\
                    12 &      2494 &      210 &      2724 &      2977 & 0.915565 & 0.915015 & 0.837756\\
                    13 &      9475 &     1302 &     11044 &     12965 & 0.857932 & 0.851832 & 0.730814\\
                    14 &     39401 &     7170 &     48329 &     59937 & 0.815266 & 0.806330 & 0.657374\\
                    15 &    186799 &    38833 &    238614 &    313230 & 0.782850 & 0.761785 & 0.596364\\
                    16 &    979987 &   209669 &   1266261 &   1701935 & 0.773922 & 0.744013 & 0.575808\\
                    17 &   5559808 &  1157938 &   7175287 &   9755328 & 0.774855 & 0.735525 & 0.569925\\
                    18 &  33722920 &  6480965 &  42857755 &  58021794 & 0.786857 & 0.738649 & 0.581211\\
                    19 & 213355372 & 36387952 & 264839694 & 352152252 & 0.805602 & 0.752060 & 0.605861\\
                    \hline
                \end{tabular}
            }
            \caption{Statistics for the HOMFLY-PT Polynomial}
        \end{table}
    \end{frame}
    \begin{frame}{Khovanov Homology}
        Back to Khovanov homology. While it has been conjectured that the
        Jones polynomial distinguishes the unknot, it is \textit{known} that
        Khovanov homology does.
        \begin{theorem}[Kronheimer-Mrowka, 2011]
            If a knot has the same Khovanov homology as the unknot, then it
            is equivalent to the unknot.
        \end{theorem}
        It is now known the Khovanov homology also detects the trefoils,
        figure eight, and the cinquefoils. These results will motivate our
        work discussed later.
    \end{frame}
    \begin{frame}{Contact Topology}
        Before diving into the results, lets discuss contact topology. The
        theory derives itself from physics, classical Hamiltonian mechanics
        in particular. Particles in $n$ dimensions can be described by
        $2n$ coordinates, their position and momentum. This is the
        \textit{phase space} coordinates.
        \par\hfill\par
        By considering hypersurfaces of constant kinetic energy we obtain
        $2n-1$ dimensional objects. The properties of these manifolds are
        axiomatize to create contact structures.
    \end{frame}
    \begin{frame}{Contact Topology}
        A \textbf{contact manifold} is a smooth $2n+1$ dimensional
        manifold $X$ together with a
        collection of smooth charts $(\mathcal{U}_{i},\,\varphi_{i})$ and
        one-forms $\alpha_{i}$ such that the charts cover the manifold and
        the $\alpha_{i}$ satisfy
        \begin{equation}
            \alpha_{i}\land(\textrm{d}\alpha_{i})^{n}\ne{0}
        \end{equation}
        and such that $\alpha_{i}$ and $\alpha_{k}$ agree whenever
        $\mathcal{U}_{i}$ and $\mathcal{U}_{k}$ overlap.
    \end{frame}
    \begin{frame}{Contact Topology}
        The kernels of the one-forms describe co-dimension one planes in the
        tangent space of each point in the manifold.
        This strange condition on the $\alpha$ is called
        \textit{maximal non-integrability}. It means there is no hypersurface
        of dimension greater than $n$ that is everywhere tangent to this
        collection of planes.
        \par\hfill\par
        The Darboux theorem tells us locally any such manifold has a chart
        $(\mathcal{U},\,\varphi)$ where the one-form $\alpha$ is given by:
        \begin{equation}
            \alpha=\sum_{k=1}^{n}
                \textrm{d}\varphi_{2k}-\varphi_{1}\textrm{d}\varphi_{2k+1}
        \end{equation}
    \end{frame}
    \begin{frame}{Contact Topology}
        For $\mathbb{R}^{3}$ this tells us we get a contact structure by using
        a single global chart and the one-form
        \begin{equation}
            \alpha=\textrm{d}z-y\textrm{d}x
        \end{equation}
        This is the standard contact structure on $\mathbb{R}^{3}$. At each
        point $(x,\,y,\,z)$ we see that
        $\partial{y}$ and $\partial{x}+y\partial{z}$ span the kernel of
        $\alpha$ meaning we can explicitly draw the hyperplane distribution.
    \end{frame}
    \begin{frame}{Contact Topology}
        \begin{figure}
            \centering
            \resizebox{\textwidth}{!}{%
                \includegraphics{darboux_form_001}
            }
            \caption{Standard Contact Structure on $\mathbb{R}^{3}$}
        \end{figure}
    \end{frame}
    \begin{frame}{Contact Topology}
        While it is impossible for a surface to be everywhere tangent, it is
        possible for curves, or \textit{knots}, to be. A \textit{Legendrian}
        knot is a smooth embedding
        $\gamma:\mathbb{S}^{1}\rightarrow\mathbb{R}^{3}$ such that
        $\alpha(\dot{\gamma}(t))=0$ for each $t\in\mathbb{S}^{1}$.
        \par\hfill\par
        This restriction takes away a degree of freedom from the knot since
        the $y$ coordinate must satisfy
        \begin{align}
            \textrm{d}z-y\textrm{d}x&=0\\
            \Rightarrow
            y&=\frac{\textrm{d}z}{\textrm{d}x}\\
            \Rightarrow{y}(t)&=
            \frac{\textrm{d}z/\textrm{d}t}{\textrm{d}x/\textrm{d}t}\\
            \Rightarrow
            y(t)&=\frac{\dot{z}(t)}{\dot{x}(t)}
        \end{align}
    \end{frame}
    \begin{frame}{Contact Topology}
        \begin{figure}
            \centering
            \resizebox{0.4\textwidth}{!}{%
                \includegraphics{legendrian_unknot_002}
            }
            \caption{Legendrian Unknot}
        \end{figure}
    \end{frame}
    \begin{frame}{Contact Topology}
        For this to be well defined when $\dot{x}(t)=0$ we also need
        $\dot{z}(t)$ to approach zero as well. The value $y$ is also finite,
        and since the circle is compact the range of $y$ is also bounded.
        Hence in a knot diagram there will be no
        \textit{vertical tangencies} and instead we obtain cusps.
        \begin{figure}
            \centering
            \resizebox{0.45\textwidth}{!}{%
                \includegraphics{legendrian_unknot_cusps_001.pdf}
            }
            \caption{Legendrian Unknot Diagram}
        \end{figure}
    \end{frame}
    \begin{frame}{Contact Topology}
        Two Legendrian knots are \textbf{equivalent} if we can smoothly
        deform one into the other, keeping the knot Legendrian at each stage
        of the deformation.
        \par\hfill\par
        It is possible for two knots to be topologically equivalent but
        different as Legendrian embeddings. To distinguish Legendrian knots
        then requires Legendrian invariants. The two simplest are the
        Thurston-Bennequin $tb$ and rotation numbers $rot$. A
        \textbf{Legendrian simple} knot is a knot where all Legendrian
        embeddings are uniquely determined by these two invariants.
        \par\hfill\par
        It is known that are torus knots are Legendrian simple.
    \end{frame}
    \begin{frame}{Contact Topology}
        The contact structure also allows us to describe transverse knots,
        those that are everywhere transverse to the distribution of hyperplanes.
        We can also define transverse invariants and transversally simple
        knots.
        \par\hfill\par
        The twist knot knots with a positive number of twist serve as our
        example of transversally simple knots.
    \end{frame}
    \begin{frame}{Results and Conjectures}
        The knots where Khovanov homology is known to uniquely distinguish are
        all Legendrian simple. We've conjectured that all such knots may be
        detectable.
        \par\hfill\par
        We computed the Jones polynomial of all prime knots of
        up to 19 crossings and compared these with the Jones polynomial of
        torus knots. At the end of this computation four matches were found.
    \end{frame}
    \begin{frame}{Results and Conjectures}
        \begin{table}
            \centering
            \resizebox{\textwidth}{!}{%
                \begin{tabular}{| l | l | l |}
                    \hline
                        Torus Knot&
                        Non-Torus Knot&
                        Jones Polynomial\\
                    \hline
                        $T(2,\,5)$&
                        \texttt{dciaFHjEbg}&
                        $-q^{14}+q^{12}-q^{10}+q^{8}+q^{4}$\\
                    \hline
                        $T(2,\,7)$&
                        \texttt{fJGkHlICEABd}&
                        $-q^{20}+q^{18}-q^{16}+q^{14}-q^{12}+q^{10}+q^{6}$\\
                    \hline
                        $T(2,\,11)$&
                        \texttt{gHlImJnKBDFAce}&
                        $-q^{32}+q^{30}-q^{28}+q^{26}-q^{24}+q^{22}-q^{20}+q^{18}-q^{16}+q^{14}+q^{10}$\\
                    \hline
                        $T(2,\,5)$&
                        \texttt{iNHlPJqCoKFmdABgE}&
                        $-q^{14}+q^{12}-q^{10}+q^{8}+q^{4}$\\
                    \hline
                \end{tabular}%
            }
            \caption{Knots whose Jones polynomial matches that of a Torus Knot}
        \end{table}
        From this the unknot conjecture cannot be generalized to torus knots
        or Legendrian simple knots. In each case the Khovanov polynomials
        are different.
        \begin{theorem}
            If a prime knot $K$ has less than or equal to 19 crossings and has
            the same Khovanov polynomial, or Khovanov homology, as a torus
            knot, than it is equivalent to it.
        \end{theorem}
    \end{frame}
    \begin{frame}{Results and Conjectures}
        A similar search was performed with the twist knots for transversally
        simple knots. A lot more matches were found but in each case the
        Khovanov polynomials differed.
        \begin{table}
            \centering
            \resizebox{\textwidth}{!}{%
                \begin{tabular}{| l | l | l |}
                    \hline
                        Twist Knot&
                        Non-Twist Knot&
                        Jones Polynomial\\
                    \hline
                        $m_{2}$&
                        \texttt{eikGbHJCaFd}&
                        $q^{4}-q^{2}+1-q^{-2}+q^{-4}$\\
                    \hline
                        $m_{3}$&
                        \texttt{dgikFHaEjbc}&
                        $-q^{12}+q^{10}-q^{8}+2q^{6}-q^{4}+q^{2}$\\
                    \hline
                        $m_{3}$&
                        \texttt{gfJKHlaIEBCD}&
                        $-q^{12}+q^{10}-q^{8}+2q^{6}-q^{4}+q^{2}$\\
                    \hline
                        $m_{3}$&
                        \texttt{hGJaMlCdEKBfI}&
                        $-q^{12}+q^{10}-q^{8}+2q^{6}-q^{4}+q^{2}$\\
                    \hline
                        $m_{5}$&
                        \texttt{bhDGijCkaef}&
                        $-q^{16}+q^{14}-q^{12}+2q^{10}-2q^{8}+2q^{6}-q^{4}+q^{2}$\\
                    \hline
                        $m_{6}$&
                        \texttt{cefIgbajkDh}&
                        $q^{12}-q^{10}+q^{8}-2q^{6}+2q^{4}-2q^{2}+2-q^{-2}+q^{-4}$\\
                    \hline
                        $m_{6}$&
                        \texttt{femIbaJKLCGHd}&
                        $q^{12}-q^{10}+q^{8}-2q^{6}+2q^{4}-2q^{2}+2-q^{-2}+q^{-4}$\\
                    \hline
                        $m_{6}$&
                        \texttt{jpIFNMrClqOhkEDabg}&
                        $q^{12}-q^{10}+q^{8}-2q^{6}+2q^{4}-2q^{2}+2-q^{-2}+q^{-4}$\\
                    \hline
                        $m_{7}$&
                        \texttt{cgjFHIaDEkb}&
                        $-q^{20}+q^{18}-q^{16}+2q^{14}-2q^{12}+2q^{10}-2q^{8}+2q^{6}-q^{4}+q^{2}$\\
                    \hline
                        $m_{8}$&
                        \texttt{knIHoBjCDQrMPaeLgF}&
                        $q^{16}-q^{14}+q^{12}-2q^{10}+2q^{8}-2q^{6}+2q^{4}-2q^{2}+2-q^{-2}+q^{-4}$\\
                    \hline
                        $m_{9}$&
                        jopIFMrDlqNhkEabcg&
                        $-q^{24}+q^{22}-q^{20}+2q^{18}-2q^{16}+2q^{14}-2q^{12}+2q^{10}-2q^{8}+2q^{6}-q^{4}+q^{2}$\\
                    \hline
                \end{tabular}%
            }
            \caption{Knots whose Jones polynomial matches that of a Twist Knot}
            \label{table:matching_twist_knots}
        \end{table}
    \end{frame}
    \begin{frame}{Results and Conjectures}
        \begin{theorem}
            If a prime knot $K$ has 19 or fewer crossings and the same
            Khovanov homology or Khovanov polynomial as a twist knot, then it
            is equivalent to it.
        \end{theorem}
        An interesting thing to note is that not all twist knots are
        transversally or Legendrian simple. This may lead one to conjecture
        that Khovanov homology is able to detect twist knots in general.
    \end{frame}
    \begin{frame}{Results and Conjectures}
        We also looked through the conjectured Legendrian simple knots in
        the Legendrian knot atlas. Once again many matches were found for the
        Jones polynomial, but the Khovanov polynomials were all different.
        \begin{table}
            \centering
            \resizebox{\textwidth}{!}{%
                \begin{tabular}{| l | l | l |}
                    \hline
                    Ng Knot&Matching Knot&Jones Polynomial\\
                    \hline
                    $m(6_{2})$&\texttt{glfoJcbKMNDaHIe}&$q^{10}-2q^{8}+2q^{6}-2q^{4}+2q^{2}-1+q^{-2}$\\
                    \hline
                    $m(6_{2})$&\texttt{hknEGmDbJLaIfc}&$q^{10}-2q^{8}+2q^{6}-2q^{4}+2q^{2}-1+q^{-2}$\\
                    \hline
                    $m(6_{2})$&\texttt{gKHlmIdJCEABf}&$q^{10}-2q^{8}+2q^{6}-2q^{4}+2q^{2}-1+q^{-2}$\\
                    \hline
                    $m(6_{2})$&\texttt{ehkjmGIaFlcbd}&$q^{10}-2q^{8}+2q^{6}-2q^{4}+2q^{2}-1+q^{-2}$\\
                    \hline
                    $m(7_{3})$&\texttt{hgelkIbaJFcd}&$-q^{18}+q^{16}-2q^{14}+3q^{12}-2q^{10}+2q^{8}-q^{6}+q^{4}$\\
                    \hline
                    $m(7_{4})$&\texttt{gfkHlbjIDaec}&$-q^{16}+q^{14}-2q^{12}+3q^{10}-2q^{8}+3q^{6}-2q^{4}+q^{2}$\\
                    \hline
                    $m(9_{48})$&\texttt{gnoqKDjIMrpEaHblfc}&$q^{2}-3+4q^{-2}-4q^{-4}+6q^{-6}-4q^{-8}+3q^{-10}-2q^{-12}$\\
                    \hline
                    $m(9_{49})$&\texttt{lFKJIOAEnDCpBhmG}&$q^{-4}-2q^{-6}+4q^{-8}-4q^{-10}+5q^{-12}-4q^{-14}+3q^{-16}-2q^{-18}$\\
                    \hline
                    $m(10_{128})$&\texttt{eHPNqGJlBFoiaDCkM}&$-q^{20}+q^{18}-2q^{16}+2q^{14}-q^{12}+2q^{10}-q^{8}+q^{6}$\\
                    \hline
                    $m(10_{128})$&\texttt{edjkaGIlFbch}&$-q^{20}+q^{18}-2q^{16}+2q^{14}-q^{12}+2q^{10}-q^{8}+q^{6}$\\
                    \hline
                    $m(10_{136})$&\texttt{igDKHJaEbFC}&$q^{6}-2q^{4}+2q^{2}-2+3q^{-2}-2q^{-4}+2q^{-6}-q^{-8}$\\
                    \hline
                    $10_{145}$&\texttt{eoHKqGJnCFmPDibaL}&$-q^{20}+q^{18}-q^{16}+q^{14}+q^{4}$\\
                    \hline
                    $10_{145}$&\texttt{kNJIpHLFECoMGABd}&$-q^{20}+q^{18}-q^{16}+q^{14}+q^{4}$\\
                    \hline
                    $10_{161}$&\texttt{hOqrljsnMeipFAgkbcd}&$-q^{22}+q^{20}-q^{18}+q^{16}-q^{14}+q^{12}+q^{6}$\\
                    \hline
                \end{tabular}%
            }
            \caption{Conjectured Legendrian Simple Knots}
        \end{table}
    \end{frame}
    \begin{frame}{Future Work}
        We were able to implement several algorithms and get computations
        for the Jones, HOMFLY-PT, and Alexander (not discussed here)
        polynomials in a reasonable amount of time. All three invariants
        have been tabulated to prime knots up to 19 crossings.
        \par\hfill\par
        The Khovanov computation was still too slow. The previous tabulation
        effort stopped at 16, and we've been able to push this to 17. By
        introducing parallel computing and making some optimizations we may
        be able to get to 19 crossings in a few months, instead of a few years.
    \end{frame}
    \begin{frame}{The End}
        \begin{center}
            Thank You!
        \end{center}
    \end{frame}
\end{document}

