%-----------------------------------LICENSE------------------------------------%
%   This file is part of Mathematics-and-Physics.                              %
%                                                                              %
%   Mathematics-and-Physics is free software: you can redistribute it and/or   %
%   modify it under the terms of the GNU General Public License as             %
%   published by the Free Software Foundation, either version 3 of the         %
%   License, or (at your option) any later version.                            %
%                                                                              %
%   Mathematics-and-Physics is distributed in the hope that it will be useful, %
%   but WITHOUT ANY WARRANTY; without even the implied warranty of             %
%   MERCHANTABILITY or FITNESS FOR A PARTICULAR PURPOSE.  See the              %
%   GNU General Public License for more details.                               %
%                                                                              %
%   You should have received a copy of the GNU General Public License along    %
%   with Mathematics-and-Physics.  If not, see <https://www.gnu.org/licenses/>.%
%------------------------------------------------------------------------------%
%   Author:     Ryan Maguire                                                   %
%   Date:       January 24, 2023                                               %
%------------------------------------------------------------------------------%
\documentclass{beamer}
\usepackage{amsmath}

\title{Pade Approximants and the Remez Exchange Algorithm}
\author{Ryan Maguire}
\date{January 24, 2023}
\usenavigationsymbolstemplate{}
\setbeamertemplate{footline}[frame number]
\begin{document}
    \maketitle
    \begin{frame}{Outline}
        \begin{itemize}
            \item Taylor / Maclaurin Series
            \item Chebyshev Polynomials
            \item The Remez Exchange Algorithm
            \item Pade Approximants
        \end{itemize}
    \end{frame}
    \begin{frame}{Polynomial Approximations}
        If you have an analytic real-valued function you could, in principle,
        compute the function to arbitrary precision using the Taylor series.
        \begin{equation}
            f(x)=\sum_{n=0}^{\infty}\frac{f^{(n)}(x_{0})}{n!}(x-x_{0})^{n}
        \end{equation}
        where $f^{(n)}$ denotes the $n^{th}$ derivative of $f$, and
        $f^{(0)}(x_{0})$ is simply $f(x_{0})$. This is likely familiar to any
        student of Calculus.
    \end{frame}
    \begin{frame}{Polynomial Approximations}
        Given a fixed $N\in\mathbb{N}$ one might ask what is the
        \textit{best} polynomial approximation of degree $N$ of a function $f$
        on some interval $[a,b]$? There are a few ways to phrase what
        \textit{best} should mean.
        \begin{itemize}
            \item Is easy to understand and implement.
            \item Minimizes the root-mean-square error.
            \item Minimizes the sup norm of $f$ on $[a,b]$.
        \end{itemize}
    \end{frame}
    \begin{frame}{Polynomial Approximations}
        The anwer to $1.)$ is probably the Taylor / Maclaurin series.
        It is rather easy
        to understand and implement, and does a good job approximating a
        function near the point $x_{0}$. There are several theorems, like the
        alternating series test, that also give bounds on the error.
        \par\hfill\par
        Least-squares methods give the answer to $2.)$, and statisticians,
        applied mathematicians, and physics make frequent use of this.
        \par\hfill\par
        We want to deal with the third problem. The issue with only caring
        about the RMS error is that there may be parts of $[a,b]$ where
        the approximation is horrible. Minimizing the sup norm means your
        approximation is good for all numbers under consideration.
    \end{frame}
    \begin{frame}{Chebyshev Polynomials}
        To solve this we use the Remez exchange algorithm. This involves
        Chebyshev polynomials, so it'd be useful to discuss these briefly.
        The defining characteristic is:
        \begin{equation}
            T_{n}\big(\cos(\theta)\big)=\cos(n\theta)
        \end{equation}
        but it is easier to work with their recurrence relation:
        \begin{equation}
            T_{n+2}(x)=2xT_{n+1}(x)-T_{n}(x)
        \end{equation}
        where $T_{0}(x)=1$ and $T_{1}(x)=x$.
    \end{frame}
    \begin{frame}{Chebyshev Polynomials}
        These polynomials come from a Sturm-Liouville system:
        \begin{equation}
            (1-x^{2})\ddot{y}-x\dot{y}+n^{2}y=0
        \end{equation}
        On the interval $[-1,1]$ the weight is
        $w(x)=(1-x^{2})^{-\frac{1}{2}}$ meaning we can approximate smooth
        functions on $[-1,1]$ via:
        \begin{equation}
            f(x)
            =\sum_{n=0}^{\infty}a_{n}T_{n}(x)
            =\sum_{n=0}^{\infty}T_{n}(x)\int_{-1}^{1}
                \frac{f(x)T_{n}(x)}{\sqrt{1-x^2}}\,\textrm{d}x
        \end{equation}
        and stopping this sum at some appropriate integer $N\in\mathbb{N}$.
    \end{frame}
    \begin{frame}{Chebyshev Polynomials}
        Chebyshev polynomials have been well-studied and their use is
        widespread. Evaluation of special functions, like Bessel functions and
        Fresnel integrals, often involves Chebyshev expansions at some point.
        \par\hfill\par
        The Remez algorithm that we'll be discussing uses the extrema of these
        polynomials. From $T_{n}(\cos(\theta))=\cos(n\theta)$ we can see that
        the $|T_{n}(x)|$ attains it's maximums at:
        \begin{equation}
            x_{k}=\cos\Big(\frac{k\pi}{n}\Big)
        \end{equation}
        for $0\leq{k}\leq{n}$.
    \end{frame}
    \begin{frame}{The Remez Exchange Algorithm}
        Now we want to find the best polynomial approximation of a smooth
        function $f$ on some interval $[a,b]$. We can linearly translate this
        back to $[-1,1]$, the domain of the Chebyshev polynomials, and then
        translate back at the end, so for ease we suppose $a=-1$ and $b=1$.
        We start with $N+2$ samples $x_{0},\dots,x_{N+1}$, where $N$ is the
        desired degree of the polynomial. These are taken to be the extrema of
        $T_{N+1}(x)$. We then set up the following $(N+2)\times(N+2)$ system of
        linear equations:
        \begin{equation}
            \Big(\sum_{n=0}^{N}b_{k}x_{k}^{n}\Big)+(-1)^{k}\varepsilon=f(x_{k})
        \end{equation}
        for each $0\leq{k}\leq{N+1}$. The variable $\varepsilon$ is the
        approximate supremum error for the polynomial $p(x)$ defined by:
        \begin{equation}
            p(x)=\sum_{k=0}^{N}b_{k}x^{k}
        \end{equation}
    \end{frame}
    \begin{frame}{The Remez Exchange Algorithm}
        Solving this system of equations gives us a guess $p_{0}(x)=p(x)$ to
        the minimax polynomial, the polynomial which minimizes the sup norm.
        We compute $p_{1}$ as follows.
        \par\hfill\par
        Compute $|f(x)-p_{0}(x)|$ and find approximations to the local
        extrema. This is done by applying Newton's method to the samples
        $x_{k}$. These new values are our new samples $y_{k}$. We replace the
        $x_{k}$ with these $y_{k}$ and repeat the process from the previous
        slide.
    \end{frame}
    \begin{frame}{The Remez Exchange Algorithm}
        As we apply more iterations, the value $\varepsilon$ starts to stabilize
        to some constant. Once this happens we are done, and we have found the
        minimax polynomial. The error in this approximation is given by
        $\varepsilon$.
        \par\hfill\par
        To be precise, we are done when the local extrema of $f(x)-p_{n}(x)$
        are all of the same magnitude and oscillate. It is a well-known theorem
        that the minimax polynomial of $f$ on $[a,\,b]$ of degree $N$ is the
        unique polynomial that achieves this property. This is the
        equioscillation theorem.
    \end{frame}
    \begin{frame}{The Remez Exchange Algorithm}
        As an example, the coefficients for $\arctan(x^2)$ on a small interval.
        \begin{align}
            1.000000000000000000000\\
            0.333333333333329318027\\
            -0.199999999998764832476\\
            0.142857142725034663711\\
            -0.111111104054623557880\\
            0.090908871334365065619\\
            -0.076918762050448299949\\
            0.066610731373875312066\\
            -0.058335701337905734864\\
            0.049768779946159323601\\
            -0.036531572744216915527\\
            0.062858201153657823623
        \end{align}
        Note these approximate the Taylor coefficients very well.
    \end{frame}
    \begin{frame}{Pade Approximants}
        Now, for some history. In the $7^{th}$ century, Indian mathematician
        Bhaskara I. writes:
        \begin{quote}
            I briefly state the rule (for finding the bhujaphala
            and the kotiphala, etc.) without making use of the
            Rsine-differences 225, etc. Subtract the degrees of a bhuja
            (or koti) from the degrees of a half circle (that is, 180 degrees).
            Then multiply the remainder by the degrees of the bhuja or koti
            and put down the result at two places. At one place subtract the
            result from 40500. By one-fourth of the remainder (thus obtained),
            divide the result at the other place as multiplied by the
            'anthyaphala (that is, the epicyclic radius). Thus is obtained
            the entire bahuphala (or, kotiphala) for the sun, moon or the
            star-planets. So also are obtained the direct and inverse Rsines.
        \end{quote}
    \end{frame}
    \begin{frame}{Pade Approximants}
        Let's be thankful for modern notation, and write:
        \begin{equation}
            \sin(x)\approx\frac{4x(180-x)}{40500-x(180-x)}
        \end{equation}
        where $x$ is in degrees. This approximation beats the Taylor polynomial
        of similar degree by a good margin. Bhaskara did not write where this
        formula came from, but it is possible to reverse engineer it.
    \end{frame}
    \begin{frame}{Pade Approximants}
        The Taylor polynomial of degree $N$ can be described as the unique
        polynomial $p$ that satisfies:
        \begin{equation}
            \frac{\textrm{d}^{n}}{\textrm{d}\,x^n}p(x_{0})=
            \frac{\textrm{d}^{n}}{\textrm{d}\,x^n}f(x_{0})
        \end{equation}
        for $0\leq{n}\leq{N}$. The $(N,M)$ Pade approximant is computed via:
        \begin{equation}
            R(x)=\frac{\sum_{n=0}^{N}a_{n}(x-x_{0})^{n}}
                {1+\sum_{m=1}^{M}b_{m}(x-x_{0})^{m}}
        \end{equation}
        and requiring that:
        \begin{equation}
            \frac{\textrm{d}^{n}}{\textrm{d}\,x^n}R(x_{0})=
            \frac{\textrm{d}^{n}}{\textrm{d}\,x^n}f(x_{0})
        \end{equation}
        for $0\leq{n}\leq{N+M}$.
    \end{frame}
    \begin{frame}{Pade Approximants}
        This requirement leads to an $(N+M+1)\times(N+M+1)$ system of equations
        (with the $N+1$ unknowns $a_{0},\,\dots,\,a_{N}$ and
        the $M$ unknowns $b_{1},\,\dots,\,b_{M}$). Solving this gives us the
        coefficients of the Pade approximant, which can be efficiently
        evaluated using Horner's method twice and one division.
        \par\hfill\par
        The Pade approximant is often more accurate than the Taylor polynomial
        of similar degree. The method is used frequent in software libraries
        when quadruple precision ($2^{-112}\approx{10}^{-34}$ relative error)
        is required.
    \end{frame}
\end{document}
