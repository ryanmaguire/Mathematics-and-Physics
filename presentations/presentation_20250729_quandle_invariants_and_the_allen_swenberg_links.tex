%-----------------------------------LICENSE------------------------------------%
%   This file is part of Mathematics-and-Physics.                              %
%                                                                              %
%   Mathematics-and-Physics is free software: you can redistribute it and/or   %
%   modify it under the terms of the GNU General Public License as             %
%   published by the Free Software Foundation, either version 3 of the         %
%   License, or (at your option) any later version.                            %
%                                                                              %
%   Mathematics-and-Physics is distributed in the hope that it will be useful, %
%   but WITHOUT ANY WARRANTY; without even the implied warranty of             %
%   MERCHANTABILITY or FITNESS FOR A PARTICULAR PURPOSE.  See the              %
%   GNU General Public License for more details.                               %
%                                                                              %
%   You should have received a copy of the GNU General Public License along    %
%   with Mathematics-and-Physics.  If not, see <https://www.gnu.org/licenses/>.%
%------------------------------------------------------------------------------%
%   Author:     Ryan Maguire                                                   %
%   Date:       July 29, 2025                                                  %
%------------------------------------------------------------------------------%
\documentclass{beamer}
\usepackage{graphicx}
\usepackage{amsmath}
\graphicspath{{../images/}}
\title{Quandle Invariants and the Allen-Swenberg Links}
\author{Ryan Maguire}
\date{July 29, 2025}
\usenavigationsymbolstemplate{}
\setbeamertemplate{footline}[frame number]
\begin{document}
    \maketitle
    \begin{frame}
        \begin{itemize}
            \item
                Symplectic forms.
            \item
                Symplectic quandles.
            \item
                The coloring polynomial for symplectic quandles.
            \item
                The Allen-Swenberg examples.
        \end{itemize}
    \end{frame}
    \begin{frame}{Symplectic Forms}
        The \textit{Lagrangian} of a mechanical system is define via:
        \[
            \mathcal{L}=T-V
        \]
        where $T$ is the kinetic energy and $V$ is the potential energy of the
        system. After studying calculus of variations, you can apply the
        \textit{Euler-Lagrange equation} to this expression to get a new, but
        equivalent, formulation of Newton's second law:
        \[
            \frac{\partial\mathcal{L}}{\partial{q}}
            -\frac{\textrm{d}}{\textrm{d}t}
            \frac{\partial\mathcal{L}}{\partial\dot{q}}
        \]
        where $q$ describes the position, $\dot{q}$ is the velocity,
        and $t$ is time.
    \end{frame}
    \begin{frame}{Symplectic Forms}
        Plotting the Lagrangian involves a curve (as a function of $t$)
        that has $q$ coordinates and $\dot{q}$ coordinates. This collection of
        ordered pairs $(q,\,\dot{q})$ is called the \textit{phase-space} of the
        system.
        \par\hfill\par
        We go from Lagrangian mechanics to Hamiltonian mechanics by considering
        the Hamiltonian. This replaces the \textit{position-velocity} coordinate
        system $(q,\,\dot{q})$ with the \textit{momentum-position} coordinate
        system $(p,\,q)$.
        \[
            \mathcal{H}
            =p\dot{q}-\mathcal{L}
        \]
    \end{frame}
    \begin{frame}{Symplectic Forms}
        Consider the case where $\mathcal{L}=0$. That is, the kinetic energy and
        the potential energy are equal, and therefore the Lagrangian is zero.
        \par\hfill\par
        In this case, our Hamiltonian becomes:
        \[
            \mathcal{H}
            =\sum_{n}p_{n}\land\textrm{d}q_{n}
        \]
        The position $\mathbf{q}=(q_{1},\,\cdots,\,q_{k})$ should vary smoothly
        as a function of time, meaning this is a smooth function (a 0 form).
        Similarly the momentum $\mathbf{p}=(p_{1},\,\cdots,\,p_{n})$ should be
        smooth. $\mathcal{H}$ is thus a sum of exterior products of $0$ forms
        and $1$ forms, meaning it is a $1$ form. This is called the
        \textit{tautological one form}.
    \end{frame}
    \begin{frame}{Symplectic Forms}
        The \textit{differential} of the tautological one form is given by:
        \[
            \textrm{d}\mathcal{H}
            =\sum_{n}\textrm{d}p_{n}\land\textrm{d}q_{n}
        \]
        Where $\land$ is a special operation called the
        \textit{exterior product}.
        This is called the \textit{Liouville form}.
    \end{frame}
    \begin{frame}{Symplectic Forms}
        A symplectic form is a type of function that generalizes the
        Liouville form from classical mechanics. In particular, given a
        vector field $V$ over a field $k$ (think of $\mathbb{R}^{n}$ over
        $\mathbb{R}$, for example), a symplectic form is an
        \textit{anti-symmetric non-degenerate bilinear form}.
        \par\hfill\par
        We'll explore what this means on the next slide.
    \end{frame}
    \begin{frame}{Symplectic Forms}
        A bilinear form is a function $\omega:V\times{V}\rightarrow{k}$ that
        is linear in both components. That is, $\omega$ takes in two vectors
        and returns a scalar, and it is linear in each slot:
        \begin{equation}
            \begin{array}{rcl}
                \displaystyle
                \omega(av_{0}+bv_{1},\,w)
                &=&
                \displaystyle
                a\omega(v_{0},\,w)+b\omega(v_{1},\,w)\\[1em]
                \displaystyle
                \omega(v,\,aw_{0}+bw_{1})
                &=&
                \displaystyle
                a\omega(v,\,w_{0})+b\omega(v,\,w_{1})
            \end{array}
        \end{equation}
        Anti-symmetry (also called \textit{skew} symmetry) means
        $\omega(v,\,w)=-\omega(w,\,v)$. The non-degeneracy requirement means
        that if $\omega(v,\,w)=0$ for all $w$, then $v$ must be the zero vector.
    \end{frame}
    \begin{frame}{Symplectic Forms}
        A bilinear product can be described by a matrix $A$. Given vectors
        $X$ and $Y$, we write:
        \[
            \omega(X,\,Y)=X^{T}AY
        \]
        where $X^{T}$ denotes the transpose of $X$. The skew-symmetric property
        of $\omega$ implies that $A^{T}=-A$, and the non-degenerate property
        means $\textrm{det}(A)\ne{0}$. Because of this we can note that
        $V$ must be \textit{even dimensional}. This is because if the dimension
        is odd, then the matrix $A$ must have determinant zero since:
        \[
            \begin{aligned}
                \textrm{det}(A)
                &=\textrm{det}(A^{T})\\
                &=\textrm{det}(-A)\\
                &=(-1)^{n}\textrm{det}(A)
            \end{aligned}
        \]
        Hence, either $n$ is even or $\textrm{det}(A)=0$. For $A$ to be
        non-degenerate we then require $n$ to be even.
    \end{frame}
    \begin{frame}{Symplectic Quandles}
        Given a vector space $V$ over a field $k$, if we have a symplectic
        form $\omega$ then we can make $V$ into a quandle using the quandle
        operation:
        \begin{equation}
            v\triangleright{w}
            =v+\omega(v,\,w)w
        \end{equation}
        The inverse quandle operation is:
        \begin{equation}
            v\triangleright^{-1}{w}
            =v-\omega(v,\,w)w
        \end{equation}
    \end{frame}
    \begin{frame}{The Coloring Polynomial for Symplectic Quandles}
        Given a knot $K$ and a symplectic quandle $T$, we may consider the
        homomorphism set $\textrm{Hom}(Q(K),\,T)$, where $Q(K)$ is the
        fundamental quandle of $K$.
        \par\hfill\par
        $Q(K)$ is determined by the relations:
        \begin{equation}
            x_{\ell}=x_{n}\triangleright{x}_{m}
        \end{equation}
        where $\ell$, $n$, and $m$ are the indices for the arcs at a given
        crossing (taken in the correct order).
    \end{frame}
    \begin{frame}{The Coloring Polynomial for Symplectic Quandles}
        To color $K$ with the quandle $T$ requires us to label the arcs
        $\ell$, $n$, and $m$ with elements of $T$ satisfying:
        \begin{equation}
            x_{\ell}
            =x_{n}+\omega(x_{n},\,x_{m})x_{m}
        \end{equation}
        where $\omega$ is the symplectic form from $T$.
    \end{frame}
    \begin{frame}{The Coloring Polynomial for Symplectic Quandles}
        If we fix a prime number $p$, then $\mathbb{Z}_{p}$ becomes a field
        (with addition and multiplication defined $\textrm{mod}\,\textrm{n}$).
        $(\mathbb{Z}_{p})^{2}$ is thus a two dimension $\mathbb{Z}_{p}$
        vector space, and given any anti-symmetric matrix $A$ with entries in
        $\mathbb{Z}_{p}$ and non-zero determinant, this would automatically
        produce a symplectic quandle.
    \end{frame}
    \begin{frame}{The Coloring Polynomial for Symplectic Quandles}
        The simplest symmetric non-degenerate matrix is the identity:
        \begin{equation}
            I=
            \begin{bmatrix}
                1&0\\
                0&1
            \end{bmatrix}
        \end{equation}
        The simples anti-symmetric non-degenerate matrix is probably:
        \begin{equation}
            I=
            \begin{bmatrix}
                0&1\\
                -1&0
            \end{bmatrix}
        \end{equation}
    \end{frame}
    \begin{frame}{The Coloring Polynomial for Symplectic Quandles}
        Once we have chosen $A$, to compute the coloring polynomial we need to
        solve the following non-linear vector equations:
        \begin{equation}
            x_{\ell}=x_{n}\triangleright{x}_{m}=x_{n}+x_{n}^{T}Ax_{m}
        \end{equation}
        Since $(\mathbb{Z}_{p})^{2}$ is two dimensional, each one of these
        vector equations is really two scalar equations. If $K$ has $N$
        crossings there are then $2N$ equations in $2N$ unknowns.
    \end{frame}
    \begin{frame}{The Allen-Swenberg Examples}
        Performing this problem with the connected sum of two Hopf links
        produces $8$ equations in $8$ unknowns. The solution set can be found
        by the aid of a computer, and after counting how many \textit{distinct}
        solutions there are, we may compute the coloring polynomial.
    \end{frame}
    \begin{frame}{The Allen-Swenberg Examples}
        The first Allen-Swenberg link has 45 crossings, hence searching for
        all possible solutions via brute force requires checking
        $p^{2\cdot{45}}=p^{90}$ different combinations!
        \par\hfill\par
        Fortunately we may split the link into smaller parts, solving the
        problem \textit{locally}, and then joining these individual solutions
        together into a solution for the entire link. This is what Jain's
        Mathematica program does.
    \end{frame}
    \begin{frame}{The Allen-Swenberg Examples}
        You may do this using \texttt{sympy} as well. You need to:
        \begin{itemize}
            \item
                Create the $2N$ system of equations in the $2N$
                symbolic variables.
            \item
                Solve these equations \textit{locally}. If you break your
                link into sections with at most $k$ crossings, the number of
                possible combinations is $p^{2k}$. If $k$ is small this search
                can easily be done with a computer.
            \item
                Join your solutions for the subdivisions of the link into a
                solution for the entire link.
                \item
                Count the number of distinct solutions.
        \end{itemize}
    \end{frame}
    \begin{frame}{The Allen-Swenberg Examples}
        \begin{center}
            Good luck!
        \end{center}
    \end{frame}
\end{document}
