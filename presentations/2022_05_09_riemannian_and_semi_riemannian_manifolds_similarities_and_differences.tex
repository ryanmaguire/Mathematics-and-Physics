%-----------------------------------LICENSE------------------------------------%
%   This file is part of Mathematics-and-Physics.                              %
%                                                                              %
%   Mathematics-and-Physics is free software: you can redistribute it and/or   %
%   modify it under the terms of the GNU General Public License as             %
%   published by the Free Software Foundation, either version 3 of the         %
%   License, or (at your option) any later version.                            %
%                                                                              %
%   Mathematics-and-Physics is distributed in the hope that it will be useful, %
%   but WITHOUT ANY WARRANTY; without even the implied warranty of             %
%   MERCHANTABILITY or FITNESS FOR A PARTICULAR PURPOSE.  See the              %
%   GNU General Public License for more details.                               %
%                                                                              %
%   You should have received a copy of the GNU General Public License along    %
%   with Mathematics-and-Physics.  If not, see <https://www.gnu.org/licenses/>.%
%------------------------------------------------------------------------------%
%   Author:     Ryan Maguire                                                   %
%   Date:       May 9, 2022                                                    %
%------------------------------------------------------------------------------%
\documentclass{beamer}
\usepackage{amsmath}

\title{Riemannian and Semi-Riemannian Manfifolds}
\subtitle{Similarities and Differences}
\author{Ryan Maguire}
\date{May 9, 2022}
\usenavigationsymbolstemplate{}
\setbeamertemplate{footline}[frame number]
\begin{document}
    \maketitle
    \begin{frame}
        A smooth manifold is a locally Euclidean Hausdorff topological space
        that is second countable such that there is an atlas $\mathcal{A}$
        consisting of smoothly compatible charts.
        \par\hfill\par
        Diffeomorphisms are
        smooth bijective functions with smooth inverses.
        \par\hfill\par
        Things like \textit{distance} and \textit{volume} are not preserved by
        diffeomorphisms, indeed these concepts aren't even well-defined for
        the general smooth manifold.
    \end{frame}
    \begin{frame}
        A smooth vector field on a manifold $M$ is a smooth section of the
        tangent bundle $X:M\rightarrow{TM}$.
        \par\hfill\par
        Let's define a
        \textit{generalized metric} to be a function $g$ on $M$ such that for
        all $p\in{M}$, $g_{p}:T_{p}M\rightarrow\mathbb{R}$ is a smooth
        symmetric bilinear form. By smooth it is meant that for all smooth
        vector fields $X,Y:M\rightarrow{TM}$ the function
        $g_{p}(X_{p},Y_{p})$, which is a function from $M$ to $\mathbb{R}$, is
        smooth.
        \par\hfill\par
        At this point there is no requirement of positive definiteness
        or non-degeneracy.
    \end{frame}
    \begin{frame}
        A \textit{Riemannian} metric on a manifold $M$ is a generalized metric
        $g$ that is positive-definite. That is, for all $p\in{M}$ and for all
        $v\in{T}_{p}M$, $g_{p}(v,v)\geq{0}$ and $g_{p}(v,v)=0$ if and only if
        $v$ is the zero tangent vector.
        \par\hfill\par
        A \textit{semi-Riemannian} metric is a generalized metric that is
        non-degenerate. That is, for all $p\in{M}$ and all non-zero
        $v\in{T}_{p}M$, there is a $w\in{T}_{p}M$ such that
        $g_{p}(v,w)\ne{0}$.
    \end{frame}
    \begin{frame}
        By Sylvester's Law of Inertia (I have absolutely no idea why it is
        called that), for each $p\in{M}$ there is a basis of $T_{p}M$ such that
        the matrix representation of the symmetric bilinear form $g_{p}$
        is diagonal and consists entirely of $1$, $0$, and $-1$ on the diagonal.
        \par\hfill\par
        By Sylvester's Conservation of Inertia (again, no idea why the name),
        the number of $1$'s, $0$'s, and $-1$'s is a constant. This allows one to
        define the \textit{signature} of a generalized metric. This is the
        ordered triple $(a,b,c)$ where $a\in\mathbb{N}$ is the number of 1's
        in such a representation, $b\in\mathbb{N}$ is the number of 0's, and
        $c\in\mathbb{N}$ is the number of $-1$'s.
    \end{frame}
    \begin{frame}
        An $N\in\mathbb{N}$ dimensional Riemannian manifold is just a smooth
        manifold with a generalized metric of signature $(N,0,0)$. A
        semi-Riemannian manifold is a smooth manifold with a generalized metric
        of signature $(n,0,m)$ with $n+m=N$, $n,m\geq{0}$.
        \par\hfill\par
        Many of the results about Riemannian manifolds hold for semi-Riemannian,
        and many do not not. In this talk we'll discuss some of these central
        ideas.
        \par\hfill\par
        Note, since a semi-Riemannian metric $g$ is non-degenerate by
        definition, the number $b$ in the signature $(a,b,c)$ of $g$ is always
        zero. Because of this many authors write the signature of $g$ to be
        $(a,c)$.
    \end{frame}
    \begin{frame}
        An affine connection on a smooth manifold is a function
        $\nabla:\mathfrak{X}(M)\times\mathfrak{X}(M)\rightarrow\mathfrak{X}(M)$
        such that:
        \begin{itemize}
            \item $\nabla$ is bilinear.
            \item $\nabla$ is $C^{\infty}(M,\mathbb{R})$ linear in the first
                component. That is, $\nabla_{fX}Y=f\nabla_{X}Y$.
            \item $\nabla$ is Liebnizean in the second component. That is,
                $\nabla_{X}fY=(Xf)Y+f\nabla_{X}Y$.
        \end{itemize}
        A torsion-free affine connection is one compatible with the
        \textit{Lie Bracket}. That is:
        \begin{equation}
            \nabla_{X}Y-\nabla_{Y}X=[X,Y]
        \end{equation}
    \end{frame}
    \begin{frame}
        Given a generalized metric $g$ on a smooth manifold $M$, a
        \textit{compatible} affine connection is an affine connection
        $\nabla$ such that for all smooth vector fields $X,Y,Z$ we have:
        \begin{equation}
            Xg_{p}(Y_{p},Z_{p})=
                g_{p}(\nabla_{X}Y,Z)+g_{p}(Y,\nabla_{X}Z)
        \end{equation}
        A Levi-Civita connection on a smooth manifold with a (generalized)
        metric $g$ is an affine connection that is torsion free and compatible
        with $g$.
    \end{frame}
    \begin{frame}
        A theorem dating back to the early 1900's states if $M$ is a smooth
        manifold with a \textit{Riemannian} metric, then there is a unique
        Levi-Civita connection on $M$. This generalizes to the semi-Riemannian
        case.
        \begin{theorem}[Fundamental Theorem of Semi-Riemannian Geometry]
            If $M$ is a smooth manifold, and if $g$ is a semi-Riemannian metric,
            then there is a unique Levi-Civita connection on $M$.
        \end{theorem}
        This does not generalize to generalized metrics, the non-degenaracy
        is needed. Indeed, take $g$ to be the \textit{zero} metric with
        signature $(0,N,0)$. Then \textit{any} affine connection is compatible
        with $g$ since the compatibility equation reduces to $0=0+0$, which is
        true.
    \end{frame}
    \begin{frame}
        For both Riemannian and semi-Riemannian the fundamental theorem
        generalizes as follows. Define any function
        $F:\mathfrak{X}(M)\times\mathfrak{X}(M)\rightarrow\mathfrak{X}(M)$.
        We have the following.
        \begin{theorem}[Generalized Fundamental Theorem of Semi-Riemannian Geometry]
            If $M$ is a smooth manifold with a semi-Riemannian metric $g$,
            then there is a unique affine connection that is compatible with $g$
            such that for all $X,Y\in\mathfrak{X}(M)$ we have:
            \begin{equation}
                \nabla_{X}Y-\nabla_{Y}X-[X,Y]=F(X,Y)
            \end{equation}
            That is, there is a unique affine connection compatible with $g$
            with the prescribed torsion.
        \end{theorem}
        The proof is the same as the proof of the previous theorem. One takes
        the \textit{Koszul formula}, shows that any such metric mush satisfy it,
        and that the equation does indeed define an affine connection
        compatible with $g$.
    \end{frame}
    \begin{frame}
        Affine connections can be defined locally on curves, and need not be
        defined on the entirety of $M$. A smooth curve $\gamma$ in $M$ is called
        a \textit{geodesic} with respect to $\nabla$ if:
        \begin{equation}
            \nabla_{\dot{\gamma}}\dot{\gamma}=0
        \end{equation}
        This makes sense for Riemannian, semi-Riemannian, or generalized
        metrics, meaning there is always a way to define geodesics.
    \end{frame}
    \begin{frame}
        What differs is the use of geodesics to define a
        \textit{distance function} (a regular metric-space metric) on $M$
        induced by $g$. For Riemannian metrics $g$ one may define:
        \begin{equation}
            d(p,q)=
            \begin{cases}
                \underset{\gamma:p\rightarrow{q}}{\inf}\int_{\gamma}
                    \sqrt{g_{\gamma(t)}(\dot{\gamma(t)},\dot{\gamma(t)})}
                    \textrm{d}t&p\textrm{ connected to }q\\
                1&\textrm{else}
            \end{cases}
        \end{equation}
        Positive-definiteness shows that this is well-defined, and with a
        bit of work one can show this transforms $(M,d)$ into a metric space.
    \end{frame}
    \begin{frame}
        It is not so easy to extend this idea to semi-Riemannian manifolds, at
        least without modification. Take a \textit{Lorentz manifold}, which is
        a semi-Riemannian manifold with metric $(N-1,0,1)$. Given $p\in{M}$
        there are points $v\in{T}_{p}M$ such that
        $g_{p}(v,v)=0$ but $v\ne{0}$. The set of all such points is called the
        \textit{light-cone} of $p$ since the set of points satisfies the
        equation:
        \begin{equation}
            \Big(\sum_{n=0}^{N-2}\textrm{d}x_{n}^{2}\Big)-\textrm{d}t^{2}=0
        \end{equation}
        Solving for $\textrm{d}t$ as a function of the other one-forms
        $\textrm{d}x_{n}$ gives the equation of a cone. Abusing notation, it is
        the \textit{light-cone} since we have:
        \begin{equation}
            \sum_{n=0}^{N-2}\Big(\frac{\textrm{d}x_{n}}{\textrm{d}t}\Big)^{2}=1
        \end{equation}
        That is, the velocity vector has norm 1 which in \textit{natural units}
        corresponds to the speed of light.
    \end{frame}
    \begin{frame}
        The formula for distance between points does not work with a Lorentzian
        metric. There are vectors $v\in{T}_{p}M$ with
        $g_{p}(v,v)<0$, the so-called \textit{time-like} vectors which
        represent movement at less than the speed of light. The square root of
        $g_{\gamma}(t),(\dot{v}(t),\dot{v}(t))$ is not a real number so this formula
        does not make sense. Moreover, when it does make sense it may not define
        a metric. Points that differ by a light-like curve (a curve with
        $g_{\gamma(t)}(\dot{\gamma}(t),\dot{\gamma}(t))=0$) will have a
        \textit{distance} between them of zero.
    \end{frame}
    \begin{frame}
        Another celebrated theorem of Riemannian geometry does not hold in
        semi-Riemannian geometry.
        \begin{theorem}[Hopf-Rinow Theorem]
            If $(M,g)$ is a connected Riemannian manifold, then the following
            are equivalent ($d$ being the metric induced by $g$):
            \begin{enumerate}
                \item Closed bounded subsets of $M$ are compact.
                \item $(M,d)$ is a complete metric space.
                \item M is geodesically complete.
            \end{enumerate}
        \end{theorem}
        Geodesically complete means geodesics may flow for all time. This
        theorem is \textit{false} for (and not well-posed) for semi-Riemannian
        manifolds. However, even if we omit the second statement, conditions
        1 and 3 are not equivalent in a semi-Riemannian manifold.
    \end{frame}
    \begin{frame}
        Let $M=\mathbb{R}^{2}\setminus\{(0,0)\}$. Define $g$ on $M$ as follows:
        \begin{equation}
            g=2\frac{\textrm{d}x\,\textrm{d}y}{x^{2}+y^{2}}
        \end{equation}
        The function $\lambda:M\rightarrow{M}$ defined by
        $\lambda((x,y))=(2x,2y)$ is an isometry. Let $\Gamma$ be the subgroup
        of the isometry group of $(M,g)$ generated by $\lambda$. There is a
        properly discontinuous group action of $\Gamma$ on $M$ and the space
        $M/\Gamma$ is, topologically, the torus $\mathbb{T}^{2}$. In particular,
        $M/\Gamma$ is compact and hence \textit{any} metric on $M/\Gamma$ that
        induces it's topology must be bounded. The induced metric gives a
        \textit{Lorentz surface}, and this ordered pair
        $(M/\Gamma,\tilde{g})$ is called the
        \textit{Clifton-Pohl} torus.
    \end{frame}
    \begin{frame}
        The Clifton-Pohl torus is \textit{not} geodesically complete, even
        though it is compact. The geodesic:
        \begin{equation}
            \gamma(t)=\Big(\frac{1}{1-t},0\Big)
        \end{equation}
        in $M$ induces a geodesic $\tilde{\gamma}$ in $M/\Gamma$, but this
        induced curve cannot flow for time $t\geq{1}$.
    \end{frame}
    \begin{frame}
        A theorem that surprised me goes as follows. First, a classic result
        about \textit{Riemannian} metrics.
        \begin{theorem}
            If $M$ is a smooth manifold, then there is a Riemannian metric $g$
            on $M$ and $(M,g)$ is a Riemannian manifold.
        \end{theorem}
        This \textit{fails} for semi-Riemannian.
        \begin{theorem}
            There is no Lorentz metric (signature $(1,0,1)$) on $\mathbb{S}^{2}$.
        \end{theorem}
    \end{frame}
    \begin{frame}
        What's truly surprising is that the existence of metrics of a given
        signature is entirely related to the algebraic topology of the
        underlying manifold.
        \par\hfill\par
        A real smooth vector bundle over a smooth manifold $M$ is an ordered
        pair $(E,\pi)$ where $E$ is a smooth manifold and
        $\pi:E\rightarrow{M}$ is a smooth surjection such that for all
        $p\in{M}$, $\pi^{-1}[\{p\}]$ has the structure of a finite dimensional
        real vector space. Moreover, for all $p\in{M}$ there is an open
        subset $\mathcal{U}\subseteq{M}$ with $p\in\mathcal{U}$ such that
        $\mathcal{U}\times\mathbb{R}^{N}$ is diffeomorphic to
        $\pi^{-1}[\mathcal{U}]$. This is an immediate generalization of the
        tangent bundle of a smooth manifold.
        \par\hfill\par
        For connected vector bundles, the dimension of the vector space of the
        fiber of $p$ is a constant for all $p\in{M}$. This is the
        \textit{rank} of the vector bundle.
    \end{frame}
    \begin{frame}
        \begin{theorem}
            A smooth manifold $M$ has a metric with signature $(p,0,q)$ if and
            only if there are smooth real vector bundles $(E,\pi_{E})$ and
            $(F,\pi_{F})$ of ranks $p$ and $q$, respectively,
            such that $TM\simeq{E}\oplus{F}$.
        \end{theorem}
        For the case of Lorentzian manifolds, $p=N-1$ and $q=1$. That is, we
        wish to write $TM$ as the product of a rank $N-1$ and a
        \textit{line bundle}.
    \end{frame}
    \begin{frame}
        \begin{theorem}
            A smooth manifold $M$ has a Lorentz metric if and only if
            $M$ is non-compact, or $M$ has Euler characteristic zero.
        \end{theorem}
        The torus has Euler characteristic zero, and we've already given a
        Lorentz metric on it. There are others, such that the Lorentz metric
        induced by the Minkowski metric on $\mathbb{R}^{2}$ by the group action
        of $\mathbb{Z}^{2}$ of integer translations. The sphere has Euler
        characteristic 2 and is compact, so there is no Lorentz metric on it.
    \end{frame}
\end{document}
