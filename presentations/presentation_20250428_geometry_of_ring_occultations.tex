%-----------------------------------LICENSE------------------------------------%
%   This file is part of Mathematics-and-Physics.                              %
%                                                                              %
%   Mathematics-and-Physics is free software: you can redistribute it and/or   %
%   modify it under the terms of the GNU General Public License as             %
%   published by the Free Software Foundation, either version 3 of the         %
%   License, or (at your option) any later version.                            %
%                                                                              %
%   Mathematics-and-Physics is distributed in the hope that it will be useful, %
%   but WITHOUT ANY WARRANTY; without even the implied warranty of             %
%   MERCHANTABILITY or FITNESS FOR A PARTICULAR PURPOSE.  See the              %
%   GNU General Public License for more details.                               %
%                                                                              %
%   You should have received a copy of the GNU General Public License along    %
%   with Mathematics-and-Physics.  If not, see <https://www.gnu.org/licenses/>.%
%------------------------------------------------------------------------------%
%   Author:     Ryan Maguire                                                   %
%   Date:       December 6, 2024                                               %
%------------------------------------------------------------------------------%
\documentclass{beamer}
\usepackage{amsmath}
\graphicspath{{../images/}}
\title{Reorganizing the MITx Mathematics Content}
\author{Ryan Maguire}
\date{January 13, 2025}
\usenavigationsymbolstemplate{}
\setbeamertemplate{footline}[frame number]
\graphicspath{{../images/}}
\begin{document}
    \maketitle
    \begin{frame}
        \begin{equation}
            \begin{array}{rcl}
                \mathbf{R}_{c}
                &=&(x,\,y,\,z)\\[1em]
                \boldsymbol{\rho}_{0}
                &=&
                \left(
                    \rho_{0}\cos(\phi_{0}),\,\rho_{0}\sin(\phi_{0}),\,0
                \right)\\[1em]
                \boldsymbol{\rho}
                &=&
                \left(
                    \rho\cos(\phi),\,\rho\sin(\phi),\,0
                \right)\\[1em]
                D
                &=&
                \|\mathbf{R}_{c}-\boldsymbol{\rho}_{0}\|
            \end{array}
        \end{equation}
    \end{frame}
    \begin{frame}
        \begin{figure}
            \centering
            \resizebox{\textwidth}{!}{%
                \includegraphics{fresnel_diffraction_geometry_saturn_001}
            }
            \caption{Occultation Observation of Saturn's Rings}
        \end{figure}
    \end{frame}
    \begin{frame}
        \begin{equation}
            \begin{array}{rcl}
                \mathbf{u}
                &=&
                \mathbf{R}_{c}-\boldsymbol{\rho}_{0}\\[1em]
                &=&
                \left(
                    x-\rho_{0}\cos(\phi_{0}),\,
                    y-\rho_{0}\sin(\phi_{0}),\,
                    z
                \right)
            \end{array}
        \end{equation}
        In an \textit{ideal} geometry, $y=\rho_{0}\sin(\phi_{0})$.
        $D$ is then:
        \begin{equation}
            D=\sqrt{\left(x-\rho_{0}\cos(\phi_{0})\right)^{2}+z^{2}}
        \end{equation}
        Letting $B$ denote the angle made by $\mathbf{u}$ and the ring plane,
        we may write $x$ and $z$ as follows:
        \begin{equation}
            \begin{array}{rcl}
                \displaystyle
                x
                &=&
                \displaystyle
                \rho_{0}\cos(\phi_{0})+D\cos(B)\\[1em]
                \displaystyle
                z
                &=&
                \displaystyle
                D\sin(B)
            \end{array}
        \end{equation}
    \end{frame}
    \begin{frame}
        In this ideal setting, $\mathbf{u}$ can be written explicitly as
        \begin{equation}
            \mathbf{u}=\left(
                D\cos(B),\,0,\,D\sin(B)
            \right)
        \end{equation}
        Define:
        \begin{equation}
            \begin{array}{rcl}
                \displaystyle
                \hat{\mathbf{u}}
                &=&
                \displaystyle
                \frac{\mathbf{u}}{D}\\[1em]
                &=&
                \displaystyle
                \left(
                    \cos(B),\,0,\,\sin(B)
                \right)
            \end{array}
        \end{equation}
        The Fresnel kernel $\psi$ is given by:
        \begin{equation}
            \psi
            =
            k\Big(
                \|\boldsymbol{\rho}-\mathbf{R}_{c}\|
                +\hat{\mathbf{u}}\cdot\left(
                    \boldsymbol{\rho}-\mathbf{R}_{c}
                \right)
            \Big)
        \end{equation}
        Replacing $\mathbf{R}_{c}$ with $\mathbf{u}+\boldsymbol{\rho}_{0}$,
        we get:
        \begin{equation}
            \psi=
            k\Big(
                \|\boldsymbol{\rho}-\boldsymbol{\rho}_{0}-\mathbf{u}\|
                +\hat{\boldsymbol{u}}\cdot\left(
                    \boldsymbol{\rho}-\boldsymbol{\rho}_{0}-\mathbf{u}
                \right)
            \Big)
        \end{equation}
    \end{frame}
    \begin{frame}
        Introduce the auxiliary parameters $\xi$ and $\eta$ defined by:
        \begin{equation}
            \begin{array}{rcl}
                \displaystyle
                \xi
                &=&
                \displaystyle
                \frac{\cos(B)}{D}\Big(
                    \rho\cos(\phi)-\rho_{0}\cos(\phi_{0})
                \Big)\\[1em]
                \displaystyle
                \eta
                &=&
                \displaystyle
                \frac{\rho^{2}+\rho_{0}^{2}-2\rho\rho_{0}\cos(\phi-\phi_{0})}
                    {D^{2}}
            \end{array}
        \end{equation}
        The Fresnel kernel is then:
        \begin{equation}
            \psi
            =kD\left(
                \sqrt{1-2\xi+\eta}
                +\xi-1
            \right)
        \end{equation}
    \end{frame}
    \begin{frame}
        The diffraction pattern $\hat{T}(\rho_{0})$ is given by:
        \begin{equation}
            \begin{array}{rcl}
                \displaystyle
                \hat{T}(\rho_{0})
                &=&
                \displaystyle
                \frac{\sin(B)}{i\lambda}
                \int_{0}^{\infty}
                    \int_{0}^{2\pi}
                        \rho{T}(\rho)
                        \frac{\exp(i\psi)}
                            {\|\boldsymbol{\rho}-\mathbf{R}_{c}\|}\,
                        \textrm{d}\phi\,\textrm{d}\rho\\[1em]
                &=&
                \displaystyle
                \frac{\sin(B)}{i\lambda}
                \int_{0}^{\infty}\rho{T}(\rho)\left(
                    \int_{0}^{2\pi}
                        \frac{\exp(i\psi)}
                            {\|\boldsymbol{\rho}-\mathbf{R}_{c}\|}\,
                        \textrm{d}\phi
                \right)\,\textrm{d}\rho
            \end{array}
        \end{equation}
    \end{frame}
\end{document}
