%-----------------------------------LICENSE------------------------------------%
%   This file is part of Mathematics-and-Physics.                              %
%                                                                              %
%   Mathematics-and-Physics is free software: you can redistribute it and/or   %
%   modify it under the terms of the GNU General Public License as             %
%   published by the Free Software Foundation, either version 3 of the         %
%   License, or (at your option) any later version.                            %
%                                                                              %
%   Mathematics-and-Physics is distributed in the hope that it will be useful, %
%   but WITHOUT ANY WARRANTY; without even the implied warranty of             %
%   MERCHANTABILITY or FITNESS FOR A PARTICULAR PURPOSE.  See the              %
%   GNU General Public License for more details.                               %
%                                                                              %
%   You should have received a copy of the GNU General Public License along    %
%   with Mathematics-and-Physics.  If not, see <https://www.gnu.org/licenses/>.%
%------------------------------------------------------------------------------%
%   Author:     Ryan Maguire                                                   %
%   Date:       April 27, 2023                                                 %
%------------------------------------------------------------------------------%
\documentclass{beamer}
\usepackage{amsmath}
\title{Numerical Experiments with the Khovanov Homology of Twist Knots}
\author{Ryan Maguire}
\date{April 30, 2023\par\hfill\par{KiW 49 $\frac{0x0F}{0x10}$}}
\newtheorem{question}{Question}
\usenavigationsymbolstemplate{}
\setbeamertemplate{footline}[frame number]
\begin{document}
    \maketitle
    \begin{frame}{Outline}
        \begin{itemize}
            \item Conjectures on Legendrian and tranversely simple knots.
            \item New data for knots up to 19 crossings.
            \item A peculiar find about twist knots.
        \end{itemize}
    \end{frame}
    \begin{frame}{Legendrian Knots}
        A knot is a smooth embedding of $\mathbb{S}^{1}$ into $\mathbb{R}^{3}$
        or $\mathbb{S}^{3}$. We can get a richer theory by considering the
        standard contact structures of these manifolds.
        \par\hfill\par
        We seek a distribution of planes in $\mathbb{R}^{3}$ with the property
        that no surface (open or closed) can be everywhere tangent to this
        distribution. This is obtained in $\mathbb{R}^{3}$ by the one-form
        $\textrm{d}z-y\textrm{d}x$, the hyperplane at $(x,\,y,\,z)$ being the
        kernel of this. This is spanned by the vectors
        $\partial{y}$ and $y\partial{z}+\partial{x}$
    \end{frame}
    \begin{frame}{Legendrian Knots}
        \begin{figure}
            \centering
            \resizebox{\textwidth}{!}{%
                \includegraphics{../images/darboux_form_001.pdf}
            }
            \caption{The Darboux Form for $\mathbb{R}^{3}$}
        \end{figure}
    \end{frame}
    \begin{frame}{Legendrian Knots}
        Contact structures can be similarly defined for odd $2k+1$ dimensional
        manifolds, locally given by a form $\alpha$ such that
        $\alpha\land(\textrm{d}\alpha)^{k}\ne{0}$ (This strange criterion is
        related to the Frobenius theorem on integrability).
        \begin{theorem}[Darboux]
            Given a point $p$ in a $2k+1$ dimensional contact manifold $M$ there
            is a coordinate chart $(\mathcal{U},\,\varphi)$ containing $p$ such
            that the contact structure is given by:
            \begin{equation}
                \alpha=\textrm{d}\varphi_{0}-
                \sum_{n=1}^{k}\varphi_{2n-1}\textrm{d}\varphi_{2n}
            \end{equation}
        \end{theorem}
        So locally all contact manifolds look like the twisting hyperplane
        distribution given before.
    \end{frame}
    \begin{frame}{Legendrian Knots}
        Given a $2k+1$ dimensional contact manifold the integer $k$ tells you
        the highest dimension a submanifold can be while being everywhere
        tangent to the contact structure. For three dimensional manifolds this
        is $k=1$, so we get knots in our space.
        \begin{definition}
            A Legendrian submanifold in a contact manifold $M$ is a submanifold
            $K$ such that $K$ is everywhere tangent to the hyperplane
            distribution. A Legendrian knot is a 1-dimensional Legendrian
            submanifold.
        \end{definition}
    \end{frame}
    \begin{frame}{Legendrian Knots}
        Contact structures arise naturally from the spherical cotangent bundle
        of any (positive dimensional) smooth manifold $M$. The contact form is
        given by the so-called Liouville form.
        \par\hfill\par
        Every knot can be made Legendrian by an appropriate perturbation. A
        \textit{Legendrian} isotopy is an isotopy $H_{t}$ of a Legendrian knot
        $K$ such that $H_{t}$ is legendrian for all $t\in[0,\,1]$. Legendrian
        isotopy and topological isotopy are different notions.
    \end{frame}
    \begin{frame}{Legendrian Knots}
        Given a (certain) spacetime $(M,\,g)$ and two points, $p,\,q\in{M}$,
        the skies $\mathbb{S}_{p}$ and $\mathbb{S}_{q}$ of these two points
        are the spaces of all future-directed inextensible light-like geodesics
        at these two points. These live naturally in the spherical cotangent
        bundle of some submanifold of the
        spacetime so one may ask if they are linked.
        \begin{theorem}[Low Conjecture]
            In (certain) 2+1 dimensional spacetimes, if two points are causally
            related (impossible to send data between the two without exceeding
            the speed of light) then their skies are topologically linked.
        \end{theorem}
        This does not, topologically, generalize to higher dimensions.
    \end{frame}
    \begin{frame}{Legendrian Knots}
        \begin{theorem}[Chernov, Nemirovski]
            The Low conjecture holds for (certain) spacetimes if one replaces
            topological linking with Legendrian linking.
        \end{theorem}
        This hints at Legendrian linking (and knotting) having more information
        than topological linking.
    \end{frame}
    \begin{frame}{Legendrian Knots}
        Legendrian equivalence implies topological equivalence since a
        Legendrian isotopy is indeed a topological isotopy. This does not
        reverse so we need invariants that distinguish Legendrian links.
        \par\hfill\par
        By examining the Darboux form, if we have a parameterization
        $\gamma:[0,\,1]\rightarrow\mathbb{R}^{3}$ of a Legendrian knot,
        $\gamma(t)=(x(t),\,y(t),\,z(t))$, we find the following constraint:
        \begin{equation}
            y(t)=\frac{z'(t)}{x'(t)}=\frac{\textrm{d}z}{\textrm{d}x}
        \end{equation}
        At the left and right extremes of a knot diagram we see $x'$ hit zero
        meaning the \textit{front diagram} of the knot will have cusps.
    \end{frame}
    \begin{frame}{Legendrian Knots}
        \begin{figure}
            \centering
            \resizebox{0.6\textwidth}{!}{%
                \includegraphics{../images/legendrian_unknot_cusps_001.pdf}
            }
            \caption{Front Diagram of a Legendrian Unknot}
        \end{figure}
    \end{frame}
    \begin{frame}{Legendrian Knots}
        \begin{figure}
            \centering
            \resizebox{0.6\textwidth}{!}{%
                \includegraphics{../images/legendrian_unknot_001.pdf}
            }
            \caption{Legendrian Unknot Embedding in $\mathbb{R}^{3}$}
        \end{figure}
    \end{frame}
    \begin{frame}{Legendrian Knots}
        \begin{figure}
            \centering
            \resizebox{0.4\textwidth}{!}{%
                \includegraphics{../images/legendrian_unknot_002.pdf}
            }
            \caption{Legendrian Unknot with Hyperplane Distribution}
        \end{figure}
    \end{frame}
    \begin{frame}{Legendrian Knots}
        The Thurston-Bennequin of a front diagram is defined as the writhe
        minus the number of right cusps.
        \begin{definition}
            A Legendrian simple knot is a topological knot $K$ such that all
            Legendrian representations of $K$ are uniquely determined by their
            Thurston-Bennequin number $tb$ and their rotation number $rot$.
        \end{definition}
        There is a similar notion of \textit{transversely simple} for transverse
        representations of a topological knot.
    \end{frame}
    \begin{frame}{Previous Conjectures}
        Last year I presented the following conjecture.
        \begin{flushleft}
            If $K$ is a topological knot (or link) type that is Legendrian
            (or transversally) simple, then the Khovanov homology of $K$
            distinguishes it. That is, if $\tilde{K}$ is another knot with the
            same Khovanov homology as $K$, then $\tilde{K}$ is topologically
            identical to $K$.
        \end{flushleft}
        This is somewhat motivated by the theorem Mrowka and Kronheimer which
        shows that Khovanov homology is an unknot detector.
        Numerical evidence was presented for all knots up to 17 crossings.
        This has been expanded.
    \end{frame}
    \begin{frame}{Previous Conjectures}
        It is known that the torus knots form a family of Legendrian simple
        knots. Etnyre, Ng, and Vertesi have also completely classified when
        twist knots are Legendrian simple. Indeed, the $m_{n}$ twist knot is
        Legendrian simple if and only if $n\geq{-3}$. Their work further
        classifies when twist knots are transversally simple. Suffice it to say
        not all twist knots are transversally simple.
        \par\hfill\par
        Lastly, Ng, Chongchitmate, An, Liang, and Manocha have written Java
        code to find Legendrian representations of knots. Their work has led
        to the conjecture that certain knot types are Legendrian simple, but
        not confirmed. For example the $6_{2}$ knot has two
        distinct Legendrian representations with $(tb,\,rot)=(-7,\,2)$ and
        hence this is not Legendrian simple, but the \textit{mirror} of $6_{2}$
        may be.
    \end{frame}
    \begin{frame}{Previous Conjectures}
        Using torus knots, twist knots, and the conjecturally Legendrian simple
        knots from the Legendrian knot atlas, all prime knots up to 19
        crossings have been compared with these families to test the conjecture
        (expanding our results beyond 17).
        \par\hfill\par
        A table of all Jones polynomials of all prime knots up to this many
        crossings has been tabulated (as well as Alexander and HOMFLY
        polynomials, but these were not directly needed for our conjecture).
        This data will soon be made available publically. We use this to search
        which knots may have the same Khovanov homology as a torus, twist, or
        conjecturally Legendrian simple knot (Khovanov homology categorifies
        the Jones polynomial).
    \end{frame}
    \begin{frame}{New Finds}
        After analyzing the data we find several knots with the
        same Jones polynomial as a torus, twist, or conjecturally Legendrian
        simple knot. At this point we compute the Khovanov polynomial
        (which contains the torsion-free data of the Khovanov homology of the
        knots). We can make the following claims:
        \begin{theorem}
            If $K$ is a prime knot with less than or equal to 19 crossings,
            and if $T$ is a torus or twist knot with the same Khovanov
            polynomial (or Khovanov homology) as $K$,
            then $T$ is equivalent to $K$.
        \end{theorem}
    \end{frame}
    \begin{frame}{New Finds}
        The degree of the Jones polynomial of a knot with $n$ crossings is
        bounded by a constant multiple of $n$.
        \par\hfill\par
        Using well-known formulas for
        the Jones polynomials of torus and twist knots we need only search
        through a finite set of knots to make the previous claim.
    \end{frame}
    \begin{frame}{New Finds}
        This motivates the following question.
        \begin{question}
            Does the Khovanov polynomial distinguish twist knot?
            Does Khovanov homology?
        \end{question}
        It is known that this is true in the three simplest cases.
        Khovanov homology \textit{does} distinguish the unknot, trefoils, and
        figure eight knot. It is also known that the Jones polynomial
        does \textit{not}, the figure eight has the sames Jones polynomial
        as a certain 11 crossing knot.
    \end{frame}
    \begin{frame}{New Finds}
        The strangeness here is that not all twist knots are Legendrian simple,
        but our search found no matches for any twist knots.
        \par\hfill\par
        The next avenue to explore is the overall strength of the Khovanov
        polynomial. Perhaps the twist knots aren't too special, perhaps many
        knot types have unique Khovanov polynomials.
        \par\hfill\par
        The Khovanov, Alexander, Jones, and HOMFLY polynomials for the first
        350 millions knots has (nearly) been tabulated so we will soon be able
        to experiment with the relative strengths of these invariants.
    \end{frame}
    \begin{frame}
        \centering
        Thank you!
    \end{frame}
\end{document}
