%-----------------------------------LICENSE------------------------------------%
%   This file is part of Mathematics-and-Physics.                              %
%                                                                              %
%   Mathematics-and-Physics is free software: you can redistribute it and/or   %
%   modify it under the terms of the GNU General Public License as             %
%   published by the Free Software Foundation, either version 3 of the         %
%   License, or (at your option) any later version.                            %
%                                                                              %
%   Mathematics-and-Physics is distributed in the hope that it will be useful, %
%   but WITHOUT ANY WARRANTY; without even the implied warranty of             %
%   MERCHANTABILITY or FITNESS FOR A PARTICULAR PURPOSE.  See the              %
%   GNU General Public License for more details.                               %
%                                                                              %
%   You should have received a copy of the GNU General Public License along    %
%   with Mathematics-and-Physics.  If not, see <https://www.gnu.org/licenses/>.%
%------------------------------------------------------------------------------%
%   Author:     Ryan Maguire                                                   %
%   Date:       July 23, 2025                                                  %
%------------------------------------------------------------------------------%
\documentclass{beamer}
\usepackage{graphicx}
\usepackage{amsmath}
\usepackage{tikz}
\usetikzlibrary{decorations.markings, arrows.meta}
\graphicspath{{../images/}}
\title{Causality and Linking}
\author{Ryan Maguire}
\date{July 25, 2025}
\usenavigationsymbolstemplate{}
\setbeamertemplate{footline}[frame number]
\begin{document}
    \maketitle
    \begin{frame}
        \begin{itemize}
            \item
                Tangent bundles and spherical tangent bundles.
            \item
                The space of light rays.
            \item
                Causality and Linking.
            \item
                The Allen-Swenberg examples.
        \end{itemize}
    \end{frame}
    \begin{frame}{Tangent Bundles and Spherical Tangent Bundles}
        Throughout a \textit{space} is going to be any two dimensional
        object like a plane ($\mathbb{R}^{2}$) or a sphere
        ($\mathbb{S}^{2}$) or a torus ($\mathbb{T}^{2}$).
        \par\hfill\par
        At each point in any of these spaces there is a
        \textit{best approximate tangent plane}. We form a new space
        by \textit{gluing} all of these spaces together in a continuous
        manner. This is called the \textit{tangent bundle}.
        \par\hfill\par
        What is the dimension of such a space?
    \end{frame}
    \begin{frame}{Tangent Bundles and Spherical Tangent Bundles}
        Since the tangent bundle of a two dimensional object happens to be
        a four dimensional object, it becomes very difficult to visualize.
        Let's take a step down and consider one dimensional objects.
        As far as topology is concerned, there are two: a line and a circle.
        \par\hfill\par
        We replace \textit{best approximate planes} with
        \textit{best approximate lines} (just like in calculus).
        What's the tangent bundle of the real line?
        What's the tangent bundle of the circle?
    \end{frame}
    \begin{frame}{Tangent Bundles and Spherical Tangent Bundles}
        For a one dimensional space $X$, the tangent bundle is just
        $X\times\mathbb{R}$. For two dimensional objects, life is not so
        easy. The tangent bundle of $\mathbb{S}^{2}$ is \textbf{not}
        the product $\mathbb{S}^{2}\times\mathbb{R}^{2}$. However, the
        tangent bundle of $X=\mathbb{T}^{2}$ or $X=\mathbb{R}^{2}$ is indeed
        $X\times\mathbb{R}^{2}$.
    \end{frame}
    \begin{frame}{Tangent Bundles and Spherical Tangent Bundles}
        Part of the reason the sphere is so difficult is the
        \textit{hairy ball theorem}. It is impossible to comb a full head of
        hair without creating a part or a cow-lick. In other words, there is
        not \textit{tangent vector field} along the sphere that is continuous
        and is never zero.
    \end{frame}
    \begin{frame}{Tangent Bundles and Spherical Tangent Bundles}
        This has a surprising consequence in computer graphics. For a
        vector in the plane, $(x,\,y)$, there is a simple formula for an
        \textit{orthogonal vector}, simply $(-y,\,x)$. The dot product is:
        \begin{equation}
            (x,\,y)\cdot(-y,\,x)
            =-xy+yx
            =0
        \end{equation}
        \textbf{There is not such formula for 3D}. This can be quite annoying,
        especially when working with projections and other geometric
        transformations, something quite common in computer graphics.
    \end{frame}
    \begin{frame}{Tangent Bundles and Spherical Tangent Bundles}
        The consequence of the hairy ball theorem is that the tangent bundle
        for the sphere is not simply $\mathbb{S}^{2}\times\mathbb{R}^{2}$.
        Tangent bundles become more difficult to imagine in higher dimensions.
    \end{frame}
    \begin{frame}{Tangent Bundles and Spherical Tangent Bundles}
        To obtain the spherical tangent bundle, we do the following.
        For each tangent plane, we remove all points that are not
        \textit{unit distance} from the point on the surface. The result is a
        bunch of circles that are centered on the points in the surface.
        \par\hfill\par
        What is the dimension of such a space?
    \end{frame}
    \begin{frame}{Tangent Bundles and Spherical Tangent Bundles}
        A circle is 1 dimensional, and the surface is 2 dimensional, so we have
        $2+1=3$ dimensions in total. The spherical tangent bundle for a surface
        is a $3$ dimensional space.
        \par\hfill\par
        What is the spherical tangent bundle for the plane $\mathbb{R}^{2}$?
    \end{frame}
    \begin{frame}{The Space of Light Rays}
        Given an observer in a spacetime, the set of all possible directions
        light can go is a topological sphere. If the observer lives in a
        $2+1$ dimensional spacetime (two dimensions for space, one for time),
        then this is a circle. The space of all light rays can be identified
        with the spherical tangent bundle of surface
        (the two dimensional part).\footnote{%
            Technically we would use the spherical \textit{cotangent} bundle.
            Since our surfaces are actually \textit{Riemannian}, the
            spherical tangent bundle and the spherical cotangent bundle can
            be identified as the same object. This slight difference in name
            will not bother us.
        }
    \end{frame}
    \begin{frame}{The Space of Light Rays}
        Since the spherical tangent bundle of the Euclidean plane is the
        thickened torus, $\mathbb{R}^{2}\times\mathbb{S}^{1}$, we can make
        our skies very visual. We can write:
        \begin{equation}
            \begin{array}{rcl}
                \displaystyle
                x
                &=&
                \displaystyle
                (R+r\cos(\theta))cos(\phi)\\
                \displaystyle
                y
                &=&
                \displaystyle
                (R+r\cos(\theta))sin(\phi)\\
                \displaystyle
                z
                &=&
                \displaystyle
                r\sin(\theta)
            \end{array}
        \end{equation}
        where $(R,\,\phi)$ is a point in the plane in polar coordinates,
        and $\theta$ represents the angle on the circle $\mathbb{S}^{1}$.
    \end{frame}
    \begin{frame}{The Space of Light Rays}
        Given two observers, we can draw their skies explicitly in the
        thickened torus and then see if we have a link. If we have
        an observer at $(x,\,y,\,0)$ and another at $(x,\,y,\,t)$ for some
        $t>0$, we get the following:
    \end{frame}
    \begin{frame}{The Space of Light Rays}
        \begin{figure}
            \centering
            \resizebox{0.6\textwidth}{!}{%
                \includegraphics{linking_detects_causality_001}
            }
            \caption{Linking Detects Causality}
        \end{figure}
    \end{frame}
    \begin{frame}{The Space of Light Rays}
        Its a Hopf link!
    \end{frame}
    \begin{frame}{Linking and Causality}
        Since we know linking detects causality for $2+1$ dimensional
        spacetimes, we might ask if certain link invariants alone are able
        to detect causality. That is, if you have a link that represents
        the skies of observers, and if you compute some link invariants
        (say, the Jones polynomial, or Alexander-Conways polynomial, or
        Khovanov homology, or knot Floer homology, or the cocycle invariant),
        can this invariant tell you whether or not the observers are causally
        related?
    \end{frame}
    \begin{frame}{Linking and Causality}
        First, a small caveat. What does it mean if this \textit{link}
        isn't a proper link at all and two circles (the skies for the
        two observers) intersect? Are the observers causally related?
    \end{frame}
    \begin{frame}{Linking and Causality}
        Intersecting skies mean the observers can see each other, so they
        are causally related. Here linking means either linked in the usual
        sense, or the skies intersect.
    \end{frame}
    \begin{frame}{Linking and Causality}
        Some of these link invariants are known to detect causality
        (like Khovanov homology), some are conjectured to
        (like the Jones polynomial), and some are conjectured not to
        (like the Alexander-Conway polynomial).
    \end{frame}
    \begin{frame}{The Allen-Swenberg Links}
        Allen and Swenberg show that the connect sum of two Hopf links is
        sky-like. That is, it is a link the does indeed come from the skies
        of observers. Moreover, in a certain spacetime these observers are
        not causally related. They then construct an infinite family of links
        with the same Alexander-Conway polynomial, but which are fairly
        more complicated.
    \end{frame}
    \begin{frame}{The Allen-Swenberg Links}
        \begin{figure}
            \centering
            \resizebox{!}{0.75\textheight}{%
                \includegraphics{allen_swenberg_link}
            }
            \caption{The Allen-Swenberg Link}
        \end{figure}
    \end{frame}
    \begin{frame}{The Allen-Swenberg Links}
        \textbf{Question:} Why does this not completely prove that
        the Alexander-Conway polynomial does not detect causality?
    \end{frame}
    \begin{frame}{The Allen-Swenberg Links}
        We do not know if these links come from the skies of observers.
        For example, in the simple Minkowski space, the trefoil knot never
        occurs as the sky of an observer. The sky of any point in Minkowski
        space is an unknotted circle. In a similar manner, these new links
        may not occur as the skies of observers in a spacetime, but there's
        reason to suspect that they are.
    \end{frame}
    \begin{frame}{The Allen-Swenberg Links}
        Allen and Swenberg provided numerical evidence that suggests that
        the Jones polynomial might detect causality. Your project is to use
        these quandle polynomials with any quandle of your choosing and
        investigate whether or not this invariant detects causality.
        \par\hfill\par
        You are highly encouraged to read the Allen-Swenberg paper,
        Jack Leventhal's paper, and Ayush Jain's paper. Nelson's work are
        excellent supplements, and I have many more readings should you be
        interested.
    \end{frame}
\end{document}
