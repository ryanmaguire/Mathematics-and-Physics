\documentclass[oneside]{book}                                                  %
%----------------------------------Preamble------------------------------------%
\makeatletter                                                                  %
    \def\input@path{{../}}                                                     %
\makeatother                                                                   %
%---------------------------Packages----------------------------%
\usepackage{geometry}
\geometry{b5paper, margin=1.0in}
\usepackage[T1]{fontenc}
\usepackage{graphicx, float}            % Graphics/Images.
\usepackage{natbib}                     % For bibliographies.
\bibliographystyle{agsm}                % Bibliography style.
\usepackage[french, english]{babel}     % Language typesetting.
\usepackage[dvipsnames]{xcolor}         % Color names.
\usepackage{listings, lstlinebgrd}      % Verbatim-Like Tools.
\usepackage{mathtools, esint, mathrsfs} % amsmath and integrals.
\usepackage{amsthm, amsfonts}           % Fonts and theorems.
\usepackage{tabularx}
\usepackage{tcolorbox}                  % Frames around theorems.
\usepackage{upgreek}                    % Non-Italic Greek.
\usepackage{paracol}                    % Two-column styling.
\usepackage{wrapfig}                    % Wrap text around figure.
\usepackage{fmtcount, etoolbox}         % For the \book{} command.
\usepackage[newparttoc]{titlesec}       % Formatting chapter, etc.
\usepackage{titletoc}                   % Allows \book in toc.
\usepackage[nottoc]{tocbibind}          % Bibliography in toc.
\usepackage[titles]{tocloft}            % ToC formatting.
\usepackage{multicol, enumitem}         % Multi-column/enumerate.
\usepackage{import}                     % Import external files.
\usepackage{pgfplots, tikz}             % Drawing/graphing tools.
\usetikzlibrary{
    calc,                   % Calculating right angles and more.
    angles,                 % Drawing angles within triangles.
    arrows.meta,            % Latex and Stealth arrows.
    quotes,                 % Adding labels to angles.
    positioning,            % Relative positioning of nodes.
    decorations.markings,   % Adding arrows in the middle of a line.
    patterns,
    arrows,
    shapes,
    shapes.geometric,
    cd,
    hobby,
    babel
}                                       % Libraries for tikz.
\pgfplotsset{compat=1.9}                % Version of pgfplots.
\usepackage[font=scriptsize,
            labelformat=simple,
            labelsep=colon]{subcaption} % Subfigure captions.
\usepackage[font={scriptsize},
            hypcap=true,
            labelsep=colon]{caption}    % Figure captions.
\usepackage{hyperref}                   % Allows for hyperlinks.
\hypersetup{
    colorlinks=true,
    linkcolor=blue,
    filecolor=magenta,
    urlcolor=Cerulean,
    citecolor=SkyBlue
}                           % Colors for hyperref.
\usepackage[toc,acronym,nogroupskip]{glossaries} % Glossaries and acronyms.
\usepackage[subpreambles=false]{standalone}      % Complileable sub files.

% Various font stuff from kiwi.
% Use this for Times text and Computer Modern math
%\usepackage{times}

% Quite nice
%\usepackage[charter, greekfamily=, greekuppercase=italicized]{mathdesign}
%\usepackage[utopia, greekuppercase=italicized]{mathdesign}    % Math is narrower

% Use this for Times text and math
%\usepackage{newtxtext}
%\usepackage[libertine,cmintegrals]{newtxmath}
%\usepackage{fix-cm}

%\usepackage{txfontsb}
% or
%\usepackage{mathptmx}

%\usepackage[scaled=0.92]{helvet}
%\renewcommand{\rmdefault}{ptm}

%\usepackage{mathpazo}    % add possibly `sc` and `osf` options
%\usepackage{eulervm}

%\usepackage{fourier}
%\renewcommand{\rmdefault}{ptm}
%\usepackage{mathptm}

%\usepackage{fontspec}
%\setmainfont{lmodern}

%\usepackage[varg]{txfonts}
%\usepackage{fouriernc}
%\usepackage{mathpazo}

%\usepackage{bookman}
%\usepackage[scaled]{uarial}
%\usepackage[scaled]{helvet}
%\renewcommand*\familydefault{\sfdefault}
%\usepackage[math]{anttor}

%\newcommand\fgeorgia{\fontfamily{jvn}\selectfont}
%\newcommand\ftimes{\fontfamily{ptm}\selectfont}
%\newcommand\fhelvetica{\fontfamily{phv}\selectfont}
%\newcommand\fcourier{\fontfamily{pcr}\selectfont}
%\newcommand\fbookman{\fontfamily{pbk}\selectfont}
%\newcommand\fnewcentury{\fontfamily{pnc}\selectfont}
%\newcommand\fpalatino{\fontfamily{ppl}\selectfont}
%\newcommand\favantgarde{\fontfamily{pag}\selectfont}
%\newcommand\fnormal{\normalfont}
%\newcommand\fsize[1]{\ifnum#1>0\fontsize{#1}{#1}\selectfont\else\normalsize\fi}
%------------------------Theorem Styles-------------------------%
% Define theorem style for default spacing and normal font.
\newtheoremstyle{normal}
    {\topsep}               % Amount of space above the theorem.
    {\topsep}               % Amount of space below the theorem.
    {}                      % Font used for body of theorem.
    {}                      % Measure of space to indent.
    {\bfseries}             % Font of the header of the theorem.
    {}                      % Punctuation between head and body.
    {.5em}                  % Space after theorem head.
    {}

% Define theorem style for default spacing with italicized font.
\newtheoremstyle{normalit}{\topsep}{\topsep}
                {\itshape}{}{\bfseries}{}{.5em}{}

% Italic header environment.
\newtheoremstyle{thmit}{\topsep}{\topsep}{}{}{\itshape}{}{0.5em}{}

% Define italicized environments.
\theoremstyle{normalit}
\newtheorem{theorem}{Theorem}[section]
\newtheorem{lemma}{Lemma}[section]
\newtheorem{corollary}{Corollary}[section]
\newtheorem{proposition}{Proposition}[section]
\newtheorem*{theorem*}{Theorem}

% Define environments with italic headers.
\theoremstyle{thmit}
\newtheorem*{solution}{Solution}
\newtheorem*{fsolution}{Solution}

% Define default environments.
\theoremstyle{normal}
\newtheorem{example}{Example}[section]
\newtheorem{definition}{Definition}[section]
\newtheorem{problem}{Problem}[section]
\newtheorem{question}{Question}[section]
\newtheorem{remark}{Remark}[section]
\newtheorem{properties}{Properties}[section]
\newtheorem{notation}{Notation}[section]
\newtheorem{axiom}{Axiom}[section]
\newtheorem*{properties*}{Properties}
\newtheorem*{remark*}{Remark}
\newtheorem*{definition*}{Definition}
\theoremstyle{plain}

% Define framed environment.
\tcbuselibrary{most}
\newtcbtheorem[use counter*=theorem]{ftheorem}{Theorem}%
    {colback=green!5,colframe=green!35!black,
     fonttitle=\bfseries\upshape}{th}

\newtcbtheorem[use counter*=example]{fdefinition}{Definition}%
    {fonttitle=\bfseries\upshape,
     colback=blue!5!white,colframe=blue!75!black}{def}

\newtcbtheorem[use counter*=example]{fexample}{Example}%
    {fonttitle=\bfseries\upshape,
     colback=red!5!white,colframe=red!75!black}{ex}

\newtcbtheorem[use counter*=notation]{fnotation}{Notation}%
    {fonttitle=\bfseries\upshape,
     colback=SeaGreen!5!white,colframe=SeaGreen!75!black}{ex}

\newtcbtheorem[use counter*=corollary]{fcorollary}{Corollary}%
    {fonttitle=\bfseries\upshape,
     colback=Orchid!5!white,colframe=Orchid!75!black}{ex}

\newenvironment{bproof}{\textit{Proof.}}{\hfill$\square$}
\tcolorboxenvironment{bproof}{blanker,breakable,left=5mm,
                             before skip=10pt,after skip=10pt,
                             borderline west={1mm}{0pt}{red}}
\tcolorboxenvironment{fsolution}
    {enhanced jigsaw,colframe=cyan,interior hidden,breakable}

%--------------------Declared Math Operators--------------------%
\DeclareMathOperator{\Refl}{Refl}           % Reflection operator.
\DeclareMathOperator{\Span}{Span}           % Span of a set of vectors.
\DeclareMathOperator{\Card}{Card}           % Cardinality of set.
\DeclareMathOperator{\Ord}{Ord}             % Ordinal of ordered set.
\DeclareMathOperator{\Tr}{Tr}               % Trace of matrix.
\DeclareMathOperator{\adjoint}{adj}         % Adjoint of matrix.
\DeclareMathOperator{\rk}{rk}               % Rank of operator.
\DeclareMathOperator{\nul}{nul}             % Null space of operator.
\DeclareMathOperator{\sgn}{sgn}             % Sign of a number.
\DeclareMathOperator{\multideg}{mutlideg}   % Multi-Degree (Graphs).
\DeclareMathOperator{\GCD}{GCD}             % Greatest common denominator.
\DeclareMathOperator{\LM}{LM}               % Leading monomial
\DeclareMathOperator{\LC}{LC}               % Leading coefficient.
\DeclareMathOperator{\LT}{LT}               % Leading term.
\DeclareMathOperator{\LCM}{LCM}             % Least common multiple.
\DeclareMathOperator{\Mon}{Mon}             % Monomial.
\DeclareMathOperator{\Spec}{Spec}           % Spectrum.
\DeclareMathOperator{\proj}{proj}           % Projection.
\DeclareMathOperator{\comp}{comp}           % Component.
\DeclareMathOperator{\sinc}{sinc}           % Sinc function.
\DeclareMathOperator{\Ima}{Im}              % Image of operator.
\DeclareMathOperator{\Prin}{Prin}           % Principal value.
\DeclareMathOperator{\Mod}{mod}             % Modulus.
%------------------------New Commands---------------------------%
\DeclarePairedDelimiter\norm{\lVert}{\rVert}
\DeclarePairedDelimiter\ceil{\lceil}{\rceil}
\DeclarePairedDelimiter\floor{\lfloor}{\rfloor}
\newcommand*\diff{\mathop{}\!\mathrm{d}}
\newcommand*\Diff[1]{\mathop{}\!\mathrm{d^#1}}
\renewcommand{\mod}{\ \Mod}
\renewcommand*{\glstextformat}[1]{\textcolor{RoyalBlue}{#1}}
\renewcommand{\glsnamefont}[1]{\textbf{#1}}
\renewcommand\labelitemii{$\circ$}
\renewcommand\thesubfigure{\arabic{chapter}.\arabic{figure}}
\renewcommand\thesubfigure{%
    \arabic{chapter}.\arabic{figure}.\arabic{subfigure}}
\addto\captionsenglish{\renewcommand{\figurename}{Fig.}}
%------------------------Book Command---------------------------%
\makeatletter
\renewcommand\@pnumwidth{1cm}
\newcounter{book}
\renewcommand\thebook{\@Roman\c@book}
\newcommand\book{%
    \if@openright
        \cleardoublepage
    \else
        \clearpage
    \fi
    \thispagestyle{plain}%
    \if@twocolumn
        \onecolumn
        \@tempswatrue
    \else
        \@tempswafalse
    \fi
    \null\vfil
    \secdef\@book\@sbook
}
\def\@book[#1]#2{%
    \ifnum \c@secnumdepth >-3\relax
        \refstepcounter{book}%
        \addcontentsline{toc}{book}{
            \bookname\ \thebook:\hspace{1em}#1
        }
    \else
        \addcontentsline{toc}{book}{#1}%
    \fi
    \markboth{}{}%
    {\centering
     \interlinepenalty \@M
     \normalfont
     \ifnum \c@secnumdepth >-2\relax
       \huge\bfseries \bookname\nobreakspace\thebook
       \par
       \vskip 20\p@
     \fi
     \Huge \bfseries #2\par}%
    \@endbook}
\def\@sbook#1{%
    {\centering
     \interlinepenalty \@M
     \normalfont
     \Huge \bfseries #1\par}%
    \@endbook}
\def\@endbook{
    \vfil\newpage
        \if@twoside
            \if@openright
                \null
                \thispagestyle{empty}%
                \newpage
            \fi
        \fi
        \if@tempswa
            \twocolumn
        \fi
}
\newcommand*\l@book[2]{%
    \ifnum \c@tocdepth >-2\relax
        \addpenalty{-\@highpenalty}%
        \addvspace{2.25em \@plus\p@}%
        \setlength\@tempdima{3em}%
        \begingroup
            \parindent \z@ \rightskip \@pnumwidth
            \parfillskip -\@pnumwidth
            {
                \leavevmode
                \Large \bfseries #1\hfil \hb@xt@\@pnumwidth{
                    \hss #2
                }
            }
            \par
            \nobreak
            \global\@nobreaktrue
            \everypar{\global\@nobreakfalse\everypar{}}%
        \endgroup
    \fi}
\newcommand\bookname{Book}
\renewcommand{\thebook}{\texorpdfstring{\Numberstring{book}}{book}}
\providecommand*{\toclevel@book}{-2}
\makeatother
\titlecontents{chapter}[0pt]
    {\bfseries}
    {\chaptername\ \thecontentslabel:\quad}
    {}
    {\hfill\contentspage}
\titleformat{\part}[display]
    {\Large\bfseries}
    {\partname\nobreakspace\thepart}
    {0mm}
    {\Huge\bfseries}
    \titlecontents{part}[0pt]
    {\large\bfseries}
    {\partname\ \thecontentslabel: \quad}
    {}
    {\hfill\contentspage}
\newcommand{\MarkRightAngle}[4][.3cm]
    {\coordinate (tempa) at ($(#3)!#1!(#2)$);
     \coordinate (tempb) at ($(#3)!#1!(#4)$);
     \coordinate (tempc) at ($(tempa)!0.5!(tempb)$);%midpoint
     \draw (tempa) -- ($(#3)!2!(tempc)$) -- (tempb);}
%--------------------------LENGTHS------------------------------%
% Spacings for the Table of Contents.
\addtolength{\cftsecnumwidth}{1ex}
\addtolength{\cftsubsecindent}{1ex}
\addtolength{\cftsubsecnumwidth}{1ex}
\addtolength{\cftfignumwidth}{1ex}
\addtolength{\cfttabnumwidth}{1ex}

% Spacing for multi-column and enumerate environments.
\setlength{\multicolsep}{6pt}
\setlist[enumerate]{itemsep=0pt,topsep=3pt}

% Indent and paragraph spacing.
\setlength{\parindent}{0em}
\setlength{\parskip}{0em}                                                           %
%----------------------------------GLOSSARY------------------------------------%
\makenoidxglossaries                                                           %
\loadglsentries{glossary}                                                      %
\loadglsentries{acronym}                                                       %
\makeindex                                                                     %
%---------------------------------Title Page-----------------------------------%
\title{Mathematics and Physics}                                                %
\author{Ryan Maguire}                                                          %
\date{\vspace{-5ex}}                                                           %
%--------------------------------Main Document---------------------------------%
%   Some things to talk about:
%       Topology that generated Lebesgue sigma algebra.
%       Not every sigma algebra is generated by a topology.
%           The sigma algebra of countable and cocountable sets is, however.
%       Notes on on the measure of projections.
%           Is this measure a continuous function of angle?
%               Not in general. Subset of unit circle with rational x-axis.
%               What about if compact? Connected? Path connected?
%
%   Develop set theory from algebras of sets. Prove associativity, DeMorgan's
%   Law's, etc. Use this to define set difference in terms of complement and
%   intersection. Go into Stone's Representation Theorem.
%
%   Set difference is neither associative nor commutative.
%
%   Cardinality of R, Q, Z, N, other examples. Algebraic numbers.
%   Prove the power set is strictly larger.
%
%   Discuss vertical line test for functions and it's abstract analog.
%
%   Complement and inverse as unary operators (functions)
%
%   Continuum hypothesis and bijection from R to power set of N.
%   Add a comment about ordered pair notation vs. interval notation.
%
%   Prove (a,b)=(c,d) iff a=c and b=d.
%
%   Talk about or, and, the symbols $\land$ and $\lor$, if then statements,
%   logical negation, truth tables, quantifies ($\exists$ and $\forall$). Talk
%   about theorems and proofs, hypotheses and conclusions. Vacuous truths.
%   Provide examples. Converse and contrapositive.
%
%   Topology notes:
%       Sierpinski space. Particular point topology. Pseudocompactness.
%       Define neighborhood, open neighborhood, interior, exterior.
%       Boundary. Closure minus set, or whole space minus interior and exterior.
%       Isolated points, perfect sets, countability & perfect sets.
%       Density, limit points, nowhere dense. Continuity, homemomorphisms.
%       

\newcommand*{\TOPPATH}{books}
\newcommand*{\PATH}{\TOPPATH/}
\newcounter{endpage}
\begin{document}
    \pagenumbering{roman}
    \maketitle
    \tableofcontents
    \listoffigures
    \listoftables
    \clearpage
    \pagenumbering{gobble}
    \book{Foundations}
        \pagenumbering{arabic}
        \renewcommand{\PATH}{\TOPPATH/Foundations}
        \part{Set-Theory}
            %------------------------------------------------------------------------------%
\begingroup
    \ifcsname\PATH\endcsname
        \newcommand{\PATH}{books/Foundations/ZFC}
        \newcommand{\OLDPATH}{\PATH}
    \else
        \newcommand{\OLDPATH}{\PATH}
        \renewcommand{\PATH}{books/Foundations/ZFC}
    \fi
    \chapter{Zermelo-Fraenkel Set Theory}
        \label{chapt:Zermelo_Fraenkel_Set_Theory}%
        We'll develop mathematics from an axiomatic view built on set theory,
        adopting as truths the few postulates of Zermelo and Fraenkel. We'll
        then add the axiom of choice and proceed from there to define many
        familiar concepts and prove some basic results that are often taken for
        granted. The existence of many types of sets will be proven, rather than
        accepting these things as trivial truths.
        %------------------------------------------------------------------------------%
\section{The Axioms of Zermelo and Fraenkel}
    The first thing to do is define what \textit{sets}\index{Set} are.
    \begin{fdefinition}{Sets}{Sets}
        A \gls{set} is a collection of objects (or elements), none of which is
        the set itself.
    \end{fdefinition}
    If we wish to stand on a truly solid foundation, it seems we're off to a bad
    start. In defining sets we used the words \textit{collection} and
    \textit{objects}, neither of which have been defined. This is the problem
    found in Chapt.~\ref{chapt:Logic} when defining connectives. To begin
    stating definitions and theorems we need the existence of a \textit{thing}.
    Sets act as our thing. We know they exist, but we don't know how to define
    them all to well. Nevertheless, we can describe how they behave and what
    they can do, as well as how to obtain new sets from pre-existing ones.
    \begin{fnotation}{Element Notation}{Element_Notation}
        If $A$ is a \gls{set} and if $x$ is an element\index{Element (Sets)} of
        $A$, then we denote this by writing $x\in{A}$. If $x$ is not an element
        of $A$, we write $x\notin{A}$.
    \end{fnotation}
    We do not yet know that sets exist. Pedagogically it seems poor to wait
    for examples, so we'll speak loosely for the moment so we may familiarize
    ourselves with the notation.
    \begin{fexample}{Using Element Notation}{Using_Element_Notation}
        Given a set $A$ that contains only a few objects, we can represent $A$
        by listing out the elements, separated by commas, and enclosing them in
        braces. Suppose $A$ is the set that contains three distinct objects
        labelled $a$, $b$, and $c$. We then write:
        \begin{equation}
            A=\big\{\,a,\,b,\,c\,\big\}
        \end{equation}
        If we are told that there is a fourth object $d$ that is different from
        $a$, $b$, and $c$, then we can use the notation defined in
        Not.~\ref{not:Element_Notation} to write the following:
        \par\hfill\par
        \begin{subequations}
            \begin{minipage}[b]{0.49\textwidth}
                \begin{equation}
                    a\in{A}
                \end{equation}
            \end{minipage}
            \hfill
            \begin{minipage}[b]{0.49\textwidth}
                \begin{equation}
                    d\notin{A}
                \end{equation}
            \end{minipage}
        \end{subequations}
        \par\vspace{2.5ex}
        The notation $a\in{A}$ should be read as \textit{a is an element of A},
        or \textit{a is contained in A}, or simply \textit{a is in A}.
        Similarly, the notation $d\notin{A}$ should be read as
        \textit{d is not an element of A}, or \textit{d is not contained in A}.
        \par\hfill\par
        $A$ is an example of a \textit{finite} set\index{Finite Set}, moreover
        it contains only three elements. For larger sets we rely on other
        methods to write them down. One such means is to indicate a pattern and
        use an ellipses to show that it goes on. Such a description is vague and
        lacks rigor, but can be helpful when the pattern is obvious. The set of
        all \textit{natural} numbers\index{Natural Numbers}, or non-negative
        integers (denoted $\mathbb{N}$) can be loosely represented by writing:
        \begin{equation}
            \label{eqn:Natural_Numbers_Ellipses}%
            \mathbb{N}=\big\{\,0,\,1,\,2,\,3,\,4,\,5,\,\dots\,\big\}
        \end{equation}
        Using our developed notation, we can write:
        \par\hfill\par
        \begin{subequations}
            \begin{minipage}[b]{0.49\textwidth}
                \begin{equation}
                    23\in\mathbb{N}
                \end{equation}
            \end{minipage}
            \hfill
            \begin{minipage}[b]{0.49\textwidth}
                \begin{equation}
                    \minus{4}\notin\mathbb{N}
                \end{equation}
            \end{minipage}
        \end{subequations}
        \par\vspace{2.5ex}
        Letting $\mathbb{Z}_{n}$ denote all non-negative integers between 0 and
        $n-1$, we have:
        \begin{equation}
            \label{eqn:Z_n_Ellipses}%
            \mathbb{Z}_{n}=\big\{0,\,1,\,2,\,\dots,\,n-1\,\big\}
        \end{equation}
        Thus $17\in\mathbb{Z}_{18}$ but $19\notin\mathbb{Z}_{18}$. Lastly, we
        present the integers\index{Integers}.
        \begin{equation}
            \label{eqn:Integers_Ellipses}%
            \mathbb{Z}=\big\{\,\dots,\,\minus{3},\,\minus{2},\,\minus{1},
                             \,0,\,1,\,2,\,3,\dots\,\big\}
        \end{equation}
    \end{fexample}
    In our definition of a set (Def.~\ref{def:Sets}) we explicitly required
    that sets cannot contain themselves. That is, if $A$ is a set, then
    $A\notin{A}$. This requirement was introduced to avoid paradoxes discovered
    by Bertrand Russell\index{Russell, Bertrand} in 1901. Allow us to neglect
    this requirement for a moment and reveal why it is essential. Recall from
    logic that a system of mathematics is inconsistent if one can prove a
    contradiction within the theory. In Naive Set Theory\index{Naive Set Theory}
    we allow the \textit{axiom of unrestricted comprehension}%
    \index{Axiom!of Unrestricted Comprehension}. This allows us to constructs
    sets as any definable collection. That is, if we have a proposition $P$,
    then we can define a set $A$ as the set of all objects that satisfy $P$.
    We can write:
    \begin{equation}
        A=\big\{\,x\;|\;P(x)\,\big\}
    \end{equation}
    Problems with such a loose definition arise instantly. Let $P$ be the
    proposition \textit{true if x is a set, false otherwise}. Then
    $A=\{\,x\;|\;P(x)\,\}$ can be read in plain English as the
    \textit{set of all sets}\index{Set!of All Sets}. A natural question would be
    whether or not $A$ then contains itself. That is, is $A\in{A}$? Russell's
    paradox arises by defining proper sets to be sets $B$ such that
    $B\notin{B}$, and improper sets to be sets $B$ such that $B\in{B}$. Using
    the \textit{Law of the Excluded Middle}\index{Law of the Excluded Middle}
    (which we will prove later), one has that every set is either proper or
    improper.
    \begin{ftheorem}{Russell's Paradox}{Russells_Paradox}
        Naive Set Theory is inconsistent.\index{Russell's Paradox}
    \end{ftheorem}
    \begin{bproof}
        For let $P$ be the proposition \textit{true if} $x\notin{x}$,
        \textit{false otherwise}. Let $A$ be the set defined by this
        proposition:
        \begin{equation}
            A=\big\{\,x\;|\;P(x)\,\big\}
        \end{equation}
        That is, $A$ is the set of all sets that do not contain themselves.
        Suppose $A\in{A}$. If $A\in{A}$ then $P(A)$ is true. That is, $A$ is a
        proper set. But proper sets do not contain themselves and $A\in{A}$, a
        contradiction. Thus $A\notin{A}$. But if $A\notin{A}$ than $P(A)$ is
        false. But if $P(A)$ is false, than $A$ is a improper set. But then
        $A\in{A}$, a contradiction as $A\notin{A}$. Thus $A\in{A}$ if and only
        if $A\notin{A}$, a contradiction. Therefore, Naive Set Theory is
        inconsistent.
    \end{bproof}
    Our development of Zermelo-Fraenkel Set Theory is to avoid this paradox and
    attempt to develop a consistent system of mathematics. The proof of
    Russell's Paradox (Thm.~\ref{thm:Russells_Paradox}) relied on the
    \textit{Law of the Excluded Middle}\index{Law of the Excluded Middle} which
    states that, given a proposition $P$, either $P$ is true or its negation is
    true. Thus we have shown that the axiom of unrestricted
    comprehension\index{Axiom!of Unrestricted Comprehension} and the law of the
    excluded middle are not compatible. This is quite unfortunate as the law of
    the excluded middle is essential in mathematics as it allows one to prove
    things via contradiction\index{Proof by Contradiction}. That is, given some
    statement we assume the opposite is true and arrive at a contradiction thus
    showing the negation of our statement is false. We then invoke the law of
    the excluded middle to show that our original statement is true. The axioms
    of Zermelo and Fraenkel, together with the axiom of choice (a system
    commonly abbreviated as ZFC) are able to prove the validity of the law of
    the excluded middle. That is, if ZFC is consistent, then so is the law of
    the excluded middle. This is one of the reasons for studying ZFC in detail.
    \par\hfill\par
    The first collection of axioms were proposed in 1908 by
    Ernst Zermelo\index{Zermelo, Ernst}. Subtle problems were pointed out by
    Abraham Fraenkel\index{Fraenkel, Abraham} in 1920, and in 1921 the system of
    Zermelo-Fraenkel Set Theory\index{Zermelo-Fraenkel Set Theory} came to be.
    The requirement that a set does not contain itself is sufficient to avoid
    Russell's paradox. This is equivalent to the
    \textit{axiom of regularity}\index{Axiom!of Regularity}. We will prove the
    equivalence of this axiom with our definition once we have obtained the law
    of the excluded middle.
    \subsection{Subsets and Equality}
        To delve more into set theory it would be convenient to know that at
        least \textit{one} set exists. The axiom of the empty
        set\index{Axiom!of the Empty Set} gives us such an existance.
        \begin{faxiom}{Axiom of the Empty Set}{Axiom_of_the_Empty_Set}
            There exists a set $\emptyset$ (the \gls{empty set}) such that for
            all $x$ it is true that $x\notin\emptyset$.\index{Empty Set}
            \begin{equation*}
                \exists_{\emptyset}:\forall_{x}\big(\neg(x\in\emptyset)\big)
            \end{equation*}
        \end{faxiom}
        The empty set is the set that contains no elements. As such, some choose
        to write $\emptyset=\{\}$. Note that the empty set is different from the
        set $\{\emptyset\}$. The empty set contains no elements whereas
        $\{\emptyset\}$ contains one elements (it contains the empty set).
        Indeed, the equality of $\emptyset$ and $\{\emptyset\}$ would violate
        our requirement that sets do not contain themselves. Any set that
        contains \textit{something} is called non-empty.
        \begin{fdefinition}{Non-Empty Set}{Non_Empty_Set}
            A \gls{non-empty set} is a \gls{set} $A$ such that there exists an
            $x$ such that $x\in{A}$.\index{Non-Empty Set}
        \end{fdefinition}
        \begin{example}
            The terminology is somewhat redundant, and essentially every set we
            deal with is non-empty. Indeed, there is only one empty set
            (see Thm.~\ref{thm:Empty_Set_is_Unique}). Thus, every other set one
            thinks of ($\mathbb{N},\mathbb{Z},\mathbb{Q},\mathbb{R},\mathbb{C}$,
            etc.) is non-empty.
        \end{example}
        \begin{example}
            It's possible to write down some formula for a set that ultimately
            leads to the empty set. For consider the \textit{set of all}
            \textit{rational numbers whose square is two}. This set turns out to
            be empty since there is no rational that satisfies this criterion.
            That is, $\sqrt{2}$ is known to be an irrational number. Thus, the
            set specified by our proposition is the empty set.
        \end{example}
        \begin{example}
            Going in the other direction, it is possible to write a formula for
            a set that appears empty, but is indeed not. The set of all
            $p\textrm{-Sylow}$ subgroups\index{Sylow Subgroups} of a non-empty
            finite group (Discussed in Book~\ref{book:Algebra}) is a non-empty
            set, but there's no reason to believe so from the start.
        \end{example}
        A set is entirely determined by its elements, and as such repetition and
        order cannot be accounted for. That is, the sets $\{a,b\}$ and
        $\{a,a,b\}$ must be considered the same since they contain precisely the
        same elements. This will be made clear once equality has been defined.
        In a similar manner, sets have no sense of order and thus $\{a,b\}$ and
        $\{b,a\}$ are equivalent. It then becomes a task to invent some new
        object that does have a notion of order. To do this requires the concept
        of a \textit{function}\index{Function}, and it is our current aim to
        develop this topic.
        \par\hfill\par
        To rigorously show that the examples in the previous paragraph are equal
        requires a definition of equality. This is the
        \textit{axiom of extensionality}\index{Axiom!of Extensionality}. First,
        we define the familiar symbol for equality\index{Equality} in terms of
        containment.
        \begin{fnotation}{Equality}{Equality}
            If $A$ and $B$ are sets, then $A=B$ if and only if for all sets
            $C$, $C\in{A}$ if and only if $C\in{B}$, and for all sets $D$,
            $A\in{D}$ if and only if $B\in{D}$.
            \begin{equation*}
                \forall_{A}\forall_{B}(A=B)
                \Longleftrightarrow\Big(
                    \forall_{C}(C\in{A}\Leftrightarrow{C}\in{B})
                    \land\forall_{D}(A\in{D}\Leftrightarrow{B}\in{D})\Big)
            \end{equation*}
        \end{fnotation}
        \begin{example}
            Consider the set of all planets in the solar system, and consider
            the set of the eight largest objects in the solar system other than
            the sun. These two sets are equal since the eight largest objects
            (other than the sun) are the eight planets (sorry Pluto), and the
            set of planets form the eight largest objects. The tricky part is
            to check that for any set one can name, it is true that if the set
            of planets lies in the set, then the set of the eight largest
            objects not equal to the sun lie in this set as well, and vice
            versa. This is almost an impossible task, and so we rely on the
            \textit{axiom of extensionality} the demonstration of equality.
        \end{example} 
        The axiom of extensionality says that to check for equality it suffices
        to show that for all $C$, $C\in{A}$ if and only if $C\in{B}$.
        That is, there is no need to check that for all $D$, $A\in{D}$ if and
        only if $B\in{D}$. For simplicity, the axiom of extensionality may be
        taken as the definition of equality.
        \begin{faxiom}{Axiom of Extensionality}{Axiom_of_Extensionality}
            If $A$ and $B$ are sets, and if for all $x$ it is true that
            $x\in{A}$ if and only if $x\in{B}$, then $A=B$. That is, $A$ and $B$
            are equal sets.\index{Axiom!of Extensionality}
            \begin{equation*}
                \forall_{x}\forall_{y}\Big(\forall_{z}(z\in{x}\Leftrightarrow
                z\in{y})\Longleftrightarrow\big(x=y\big)\Big)
            \end{equation*}
        \end{faxiom}
        \begin{example}
            Returning to our example of planets, we have seen that the set of
            all planets and the set of the eight largest objects other than the
            sun contain precisely the same elements. By the axiom of
            extensionality, we thus have equality amongst these two.
        \end{example}
        We won't adopt this axiom directly, but restate equality as a definition
        using the language of subsets\index{Set!Subset}. This will make proving
        various things easier in the proceeding sections. The notions are
        equivalent. Subsets are sets that are defined in terms of another given
        set by simply removing some (or none, or all) of the elements.
        \begin{fdefinition}{Subsets}{Subsets}
            A \gls{subset} of a \gls{set} $B$ is a set $A$ such that for all
            $x\in{A}$ it is true that $x\in{B}$. If $A$ is a subset of $B$ we
            write $A\subseteq{B}$. Otherwise, we write $A\nsubseteq{B}$.
            \index{Subset}
            \begin{equation*}
                \forall_{A}\forall_{B}\Big(\big(A\subseteq{B}\big)
                \Longleftrightarrow
                \forall_{x}\big(x\in{A}\Rightarrow{x}\in{B}\big)\Big)
            \end{equation*}
        \end{fdefinition}
        We can often visualize sets as blobs in the plane. Using such a visual,
        we can envision subsets as well (Fig.~\ref{fig:Subset_Blobs}). Given a
        blob $B$, a subset of $B$ is another blob $A$ that is entirely contained
        within $B$.
        \begin{figure}[H]
            \centering
            %--------------------------------Dependencies----------------------------------%
%   tikz                                                                       %
%-------------------------------Main Document----------------------------------%
\begin{tikzpicture}[line width=0.2mm, scale=1.2]

    % Coordinates for the bigger blob.
    \coordinate (P1) at ( 0.0, -2.0);
    \coordinate (P2) at ( 1.0, -1.0);
    \coordinate (P3) at ( 1.5,  1.0);
    \coordinate (P4) at ( 0.0,  2.0);
    \coordinate (P5) at (-3.0,  0.0);

    % Coordinates for the inner blob.
    \coordinate (Q1) at ( 0.0, -1.0);
    \coordinate (Q2) at ( 1.0,  0.0);
    \coordinate (Q3) at ( 0.5,  0.5);
    \coordinate (Q4) at (-0.5,  0.5);
    \coordinate (Q5) at (-1.0,  0.0);

    % Coordindates to label things.
    \coordinate (A) at (-0.1, -0.2);
    \coordinate (B) at (-1.5,  0.5);

    % Draw the bigger blob.
    \draw[fill=red, opacity=0.4] (P1) to [out=0,    in=-120] (P2)
                                      to [out=60,   in=-45]  (P3)
                                      to [out=135,  in=0]    (P4)
                                      to [out=-180, in=70]   (P5)
                                      to [out=-110, in=-180] cycle;

    % Draw the inner blob.
    \draw[fill=cyan, opacity=0.8] (Q1) to [out=0,    in=-120]  (Q2)
                                       to [out=60,   in=20]    (Q3)
                                       to [out=-160, in=45]    (Q4)
                                       to [out=-135, in=90]    (Q5)
                                       to [out=-90,  in=180]   cycle;

    % Labels for the two blobs.
    \node at (A) {$A$};
    \node at (B) {$B$};
\end{tikzpicture}

            \caption{Visualizing Subsets as Blobs}
            \label{fig:Subset_Blobs}
        \end{figure}
        \begin{example}
            Consider the set of natural numbers $\mathbb{N}$ and the set of
            integers $\mathbb{Z}$ (loosely defined in
            Eqn.~\ref{eqn:Natural_Numbers_Ellipses} and
            Eqn.~\ref{eqn:Integers_Ellipses}, respectively). It can be seen that
            every natural number is also an integer, and thus we have:
            \begin{equation}
                \mathbb{N}\subseteq\mathbb{Z}
            \end{equation}
            Letting $\mathbb{Q}$ denote the rational numbers $p/q$, where
            $p,q\in\mathbb{Z}$ and $q$ is non-zero, we can see that $\mathbb{Q}$
            contains $\mathbb{Z}$ as a subset. That is, setting $q=1$ and
            allowing $p$ to vary over $\mathbb{Z}$ gives us every integer. Thus:
            \begin{equation}
                \mathbb{Z}\subseteq\mathbb{Q}
            \end{equation}
            We can continue with the real numbers and the complex numbers as
            well, creating a chain of subsets:
            \begin{equation}
                \mathbb{N}\subseteq\mathbb{Z}\subseteq\mathbb{Q}
                \subseteq\mathbb{R}\subseteq\mathbb{C}
            \end{equation}
        \end{example}
        We can use subsets to define equality and to provide examples
        of the \textit{axiom schema of specification}%
        \index{Axiom!Schema of Specification}. It is important to note the
        distinction between the symbols $\in$ and $\subseteq$. The symbol $\in$
        is used to denote that some object $x$ is an \textit{element}%
        \index{Element (Sets)} of some set. That is, $x\in{A}$ indicates that
        $x$ is an element of $A$. This does not necessarily imply
        $x\subseteq{A}$, but this \textit{does} imply that $\{x\}\subseteq{A}$.
        That is, if $x\in{A}$, then the set that contains only $x$ is a subset
        of $A$. Moreover, the notions are not mutually exclusive. It is possible
        for $A$ to be a set such that $x\in{A}$ and $x\subseteq{A}$. For let
        $A=\{\emptyset\}$. For any set $A$ it is true that
        $\emptyset\subseteq{A}$ (see Thm.~\ref{thm:Emptyset_Is_Subset}). But
        from how $A$ is defined, we have that $\emptyset\in{A}$. Thus it is true
        that both $\emptyset\in{A}$ and $\emptyset\subseteq{A}$.
        \begin{fexample}{Elementary Examples of Subsets}
                        {Elementary_Examples_of_Subsets}
            Let $A$ and $B$ be the sets defined by:
            \par\hfill\par
            \begin{subequations}
                \begin{minipage}[b]{0.49\textwidth}
                    \begin{equation}
                        A=\big\{\,a,\,b,\,c\,\big\}
                    \end{equation}
                \end{minipage}
                \hfill
                \begin{minipage}[b]{0.49\textwidth}
                    \begin{equation}
                        B=\big\{\,a,\,b,\,c,\,d\,\big\}
                    \end{equation}
                \end{minipage}
            \end{subequations}
            \par\vspace{2.5ex}
            where we assume that $a$, $b$, $c$, and $d$ are distinct objects.
            From the definition of subsets (Def.~\ref{def:Subsets}):
            \par\hfill\par
            \begin{subequations}
                \begin{minipage}[b]{0.49\textwidth}
                    \begin{equation}
                        A\subseteq{B}
                    \end{equation}
                \end{minipage}
                \hfill
                \begin{minipage}[b]{0.49\textwidth}
                    \begin{equation}
                        B\nsubseteq{A}
                    \end{equation}
                \end{minipage}
            \end{subequations}
            \par\vspace{2.5ex}
            This is true since from the definition of $A$ and $B$, every element
            of $A$ is also an element of $B$. The converse of this is not true
            since there is an element of $B$ that is not an element of $A$
            (namely, the element $d$). That is, $d\in{B}$ but $d\notin{A}$ and
            therefore $B\nsubseteq{A}$.
        \end{fexample}
        The example shown in Ex.~\ref{ex:Elementary_Examples_of_Subsets} shows
        how we can define equality of sets. We see that $A\subseteq{B}$, but
        $B\nsubseteq{A}$. If we have two sets $A$ and $B$ such that
        $A\subseteq{B}$ and $B\subseteq{A}$ then it would be impossible to
        discern between the two. This gives us our new definition of equality.
        We now prove this equivalence with the axiom of extensionality%
        \index{Axiom!of Extensionality} (Ax.~\ref{ax:Axiom_of_Extensionality}).
        \begin{theorem}
            \label{thm:Equivalent_Def_of_Equality}%
            If $A$ and $B$ are sets, then $A=B$ if and only if $A\subseteq{B}$
            and $B\subseteq{A}$.
        \end{theorem}
        \begin{proof}
            For by the axiom of extensionality
            (Ax.~\ref{ax:Axiom_of_Extensionality}), $A=B$ if and only if, for
            all $x$ it is true that $x\in{A}$ if and only if $x\in{B}$. But then
            $x\in{A}$ implies that $x\in{B}$, and thus $A\subseteq{B}$
            (Def.~\ref{def:Subsets}). But also $x\in{B}$ implies $x\in{A}$, and
            therefore $B\subseteq{A}$. Moreover, if $A\subseteq{B}$ and
            $B\subseteq{A}$, then for all $x\in{A}$ it is true that $x\in{B}$
            and for all $x\in{B}$ it is true that $x\in{A}$
            (Def.~\ref{def:Subsets}), and therefore $x\in{A}$ if and only if
            $x\in{B}$. Thus, $A=B$ if and only if $A\subseteq{B}$ and
            $B\subseteq{A}$.
        \end{proof}
        With this, we can redefine the notion of
        \textit{equal sets}\index{Equal Sets}.
        \begin{fdefinition}{Equal Sets}{Equal_Sets}
            \Glspl{equal set} are \glspl{set} $A$ and $B$, denoted $A=B$, such
            that $A\subseteq{B}$ and $B\subseteq{A}$.\index{Equal Sets}
            \begin{equation*}
                \forall_{A}\forall_{B}(A=B)
                \Leftrightarrow
                (A\subseteq{B}\land{B}\subseteq{A})
            \end{equation*}
        \end{fdefinition}
        Def.~\ref{def:Equal_Sets} is justified by
        Thm.~\ref{thm:Equivalent_Def_of_Equality}, and thus there is no
        contradiction with the axiom of extensionality
        (Ax.~\ref{ax:Axiom_of_Extensionality}). If $A$ and $B$ are not equal, we
        write $A\ne{B}$. 
        \begin{lexample}{More Examples of Subsets}{More_Examples_of_Subsets}
            Using the notation from Ex.~\ref{ex:Using_Element_Notation}, for all
            $n\in\mathbb{N}$ we have:
            \begin{equation}
                \mathbb{Z}_{n}\subseteq\mathbb{N}
            \end{equation}
            Let's define $\mathbb{N}_{e}$ and $\mathbb{N}_{o}$ to be the sets of
            even\index{Even Integers} and odd\index{Odd Integers} non-negative
            integers, respectively:
            \par
            \begin{subequations}
                \begin{minipage}[b]{0.49\textwidth}
                    \begin{equation}
                        \label{eqn:Even_Pos_Ints_Ellipses}%
                        \mathbb{N}_{e}=\big\{\,0,\,2,\,4,\,6,\,8,\,\dots\,\big\}
                    \end{equation}
                \end{minipage}
                \hfill
                \begin{minipage}[b]{0.49\textwidth}
                    \begin{equation}
                        \label{eqn:Odd_Pos_Ints_Ellipses}%
                        \mathbb{N}_{o}=\big\{\,1,\,3,\,5,\,7,\,9,\,\dots\,\big\}
                    \end{equation}
                \end{minipage}
            \end{subequations}
            \par\vspace{2.5ex}
            From this we see the following two expressions are true:
            \par\hfill\par
            \begin{subequations}
                \begin{minipage}[b]{0.49\textwidth}
                    \begin{equation}
                        \mathbb{N}_{o}\subseteq\mathbb{N}
                    \end{equation}
                \end{minipage}
                \hfill
                \begin{minipage}[b]{0.49\textwidth}
                    \begin{equation}
                        \mathbb{N}_{e}=\mathbb{N}
                    \end{equation}
                \end{minipage}
            \end{subequations}
            \par\vspace{2.5ex}
            Moreover we see that $\mathbb{N}_{o}$ and $\mathbb{N}_{e}$ have no
            elements in common. That is, they are \textit{disjoint}%
            \index{Disjoint Sets}. We can represent this symbolically by
            writing:
            \par\hfill\par
            \begin{subequations}
                \begin{minipage}[b]{0.49\textwidth}
                    \begin{equation}
                        \mathbb{N}_{o}\nsubseteq\mathbb{N}_{e}
                    \end{equation}
                \end{minipage}
                \hfill
                \begin{minipage}[b]{0.49\textwidth}
                    \begin{equation}
                        \mathbb{N}_{e}\nsubseteq\mathbb{N}_{o}
                    \end{equation}
                \end{minipage}
            \end{subequations}
            \par\vspace{2.5ex}
            We can also think of trivial examples. We see that:
            \par\hfill\par
            \begin{subequations}
                \begin{minipage}[b]{0.49\textwidth}
                    \begin{equation}
                        \mathbb{Z}_{3}\subseteq\mathbb{Z}_{4}
                    \end{equation}
                \end{minipage}
                \hfill
                \begin{minipage}[b]{0.49\textwidth}
                    \begin{equation}
                        \mathbb{Z}_{4}\nsubseteq\mathbb{Z}_{3}
                    \end{equation}
                \end{minipage}
            \end{subequations}
            \par\vspace{2.5ex}
            This is because every element of $\mathbb{Z}_{3}$ is contained in
            $\mathbb{Z}_{4}$, but $3\in\mathbb{Z}_{4}$ but
            $3\notin\mathbb{Z}_{3}$. It may seem like bad notation to write
            $3\notin\mathbb{Z}_{3}$, but since we want $\mathbb{Z}_{n}$ to have
            $n$ elements, and since we started counting at zero, we have that
            $n\notin\mathbb{Z}_{n}$ for all $n\in\mathbb{N}$. Such counting
            schemes are common in computer science, but there's disagreement in
            mathematics as to whether $0\in\mathbb{N}$ or not. We will use the
            \textit{axiom of infinity}\index{Axiom!of Infinity} to prove the
            existence of $\mathbb{N}$, and in doing so it will be natural to
            define $\mathbb{N}$ as a set that contains $0$.
        \end{lexample}
        While Def.~\ref{def:Equal_Sets} is indeed equivalent to the axiom of
        extensionality, this definition creates a few problems. As discussed
        previously, sets have no notion of order and cannot account for
        repetition. For let $A$, $B$, and $C$ be the sets defined by:
        \par
        \begin{subequations}
            \begin{minipage}[b]{0.31\textwidth}
                \begin{equation}
                    A=\big\{\,a,\,b\,\big\}
                \end{equation}
            \end{minipage}
            \hfill
            \begin{minipage}[b]{0.36\textwidth}
                \begin{equation}
                    B=\big\{\,a,\,a,\,b\,\big\}
                \end{equation}
            \end{minipage}
            \hfill
            \begin{minipage}[b]{0.31\textwidth}
                \begin{equation}
                    C=\big\{\,b,\,a\,\big\}
                \end{equation}
            \end{minipage}
        \end{subequations}
        \par\vspace{2.5ex}
        All three of these sets are equal by both the definition of equality
        (Def.~\ref{def:Equal_Sets})\index{Equal Sets} and the axiom of
        extensionality\index{Axiom!of Extensionality}. It seems clear that
        $A\subseteq{B}$, but it is also true that $B\subseteq{A}$. This is
        because $B$ contains only the elements $a$ and $b$. While $a$ is
        included twice, repetition cannot be accounted for and $B$ is entirely
        determined by $a$ and $b$. But $A$ also contains $a$ and $b$, and
        therefire $B\subseteq{A}$. By the definition of equality
        (Def.~\ref{def:Equal_Sets}), we have that $A=B$. In a similar manner,
        $A=C$. From the definition of subsets, for any set $A$ we see that
        $A\subseteq{A}$ (see Thm.~\ref{thm:Reflexivity_of_Inclusion}). It would
        be nice to distinguish between subsets that aren't the entire set
        itself. These are called proper subsets\index{Set!Subset!Proper}, and
        we can define them in terms of equality.
        \begin{fdefinition}{Proper Subsets}{Proper_Subsets}
            A \gls{proper subset} of a \gls{set} $B$ is a set $A$ such that
            $A\subseteq{B}$ and $A\ne{B}$. We write $A\subsetneq{B}$
            to denote that $A$ is a proper subset of $B$.\index{Proper Subset}
            \begin{equation*}
                \forall_{A}\forall_{B}(A\subsetneq{B})
                \Leftrightarrow(A\subseteq{B}\land{A}\ne{B})
            \end{equation*}
        \end{fdefinition}
        The symbols $\subseteq$ and $\subsetneq$ are analogous to the notations
        of inequalities that one finds in calculus: $\leq$ and $<$. In many
        texts, the two symbols $\subseteq$ and $\subset$ are taken to be
        identical, which may cause confusion. In an attempt to reduce confusion,
        $\subseteq$ will denote any subset, $\subsetneq$ denotes a proper
        subset, and the symbol $\subset$ will be avoided.
        \begin{lexample}{Proper Subsets}{Proper_Subsets}
            Let $A$ and $B$ be sets defined as follows:
            \par
            \begin{subequations}
                \begin{minipage}[b]{0.49\textwidth}
                    \centering
                    \begin{equation}
                        A=\big\{\,a,\,b,\,c\,\big\}
                    \end{equation}
                \end{minipage}
                \hfill
                \begin{minipage}[b]{0.49\textwidth}
                    \centering
                    \begin{equation}
                        B=\big\{\,a,\,b,\,c,\,d\,\big\}
                    \end{equation}
                \end{minipage}
            \end{subequations}
            \par\vspace{2.5ex}
            Then $A\subseteq{B}$, since every element of $A$ is an element of
            $B$, but $B\nsubseteq{A}$ since $d\in{B}$ and $d\notin{A}$.
            Therefore $A\ne{B}$, and thus $A$ is a proper subset of $B$. We can
            denote this by writing $A\subsetneq{B}$.
        \end{lexample}
        \begin{example}
            Returning to more concrete examples, $\mathbb{N}$ is a proper subset
            of $\mathbb{Z}$. To see this, note that $\minus{1}\in\mathbb{Z}$ but
            $\minus{1}\notin\mathbb{N}$. Indeed, none of the negative integers
            are natural numbers, but they are integers. We can write this by:
            \begin{equation}
                \mathbb{N}\subsetneq\mathbb{Z}
            \end{equation}
            Similarly, $\mathbb{Q}$ contains numbers that are not integers,
            for example $1/2$. Thus, $\mathbb{Z}$ is also a proper subset of
            $\mathbb{Q}$. Lastly, since $\sqrt{2}$ is not a rational number, the
            set of rational numbers must then be a proper subset of the set of
            real numbers.
        \end{example}
        We now introduce the \textit{axiom schema of specification}.
        \begin{faxiom}{Axiom Schema of Specification}
                      {Axiom_Schema_of_Specification}
            If $A$ is a set and if $P$ is a proposition, then there exists a set
            $B$ such that $x\in{B}$ if and only if $x\in{A}$ and $P(x)$ is true.
            We can write this as:\index{Axiom!Schema of Specification}
            \begin{equation*}
                B=\big\{\,x\in{A}\;|\;P(x)\,\big\}
            \end{equation*}
            Using our formal language, we have:
            \begin{equation*}
                \forall_{A}\forall_{P}\exists_{B}:
                \forall_{x}\Big(x\in{B}\Leftrightarrow
                \big(x\in{A}\land{P}(x)\big)\Big)
            \end{equation*}
        \end{faxiom}
        Ax.~\ref{ax:Axiom_Schema_of_Specification} is different from the
        inconsistent axiom of unrestricted comprehension%
        \index{Axiom!of Unrestricted Comprehension} in that we can only speak of
        elements that are already defined and contained in some other set. That
        is, this new axiom does not allow us to talk about the
        \textit{set of all sets}\index{Set!of All Sets}, and so we have avoided
        the crux of Russell's paradox.
        \par\hfill\par
        This allows us to use the Set-Builder method of constructing sets. We
        loosely defined then natural numbers $\mathbb{N}$ and the integers
        $\mathbb{Z}$ (From the German \textit{Zahl}) by
        Eqns.~\ref{eqn:Natural_Numbers_Ellipses} and
        \ref{eqn:Integers_Ellipses}, respectively. It would be more
        difficult (but not impossible) to describe the set of rational numbers%
        \index{Rational Numbers} in such a way. Instead, we use set builder
        notation if it is known that $\mathbb{Q}$ is contained in some larger
        set $\mathbb{R}$ (the \textit{real} numbers)\index{Real Numbers}.
        \begin{equation}
            \mathbb{Q}=\Big\{\;\frac{p}{q}\in\mathbb{R}\;\big|\;
                                p,\,q\in\mathbb{Z}\textrm{ and }q\ne{0}\;\Big\}
        \end{equation}
        That is, the rational numbers are the set of all real numbers which can
        be written as the ratios of integers with non-zero denominator. The
        Axiom Schema of Specification states that this is is a valid method of
        describing sets. It is also known as the axiom of separation%
        \index{Axiom!of Separation}.
        \begin{example}
            We can describe the sets $\mathbb{Z}$, $\mathbb{N}$,
            $\mathbb{N}_{e}$, and $\mathbb{N}_{o}$ using set-builder notation if
            we assume these belong to some larger set $\mathbb{R}$. We define
            $\mathbb{Z}$ by:
            \index{Natural Numbers}\index{Even Integers}\index{Odd Integers}%
            \index{Integers}
            \begin{equation}
                \mathbb{Z}=
                \big\{\,n\in\mathbb{R}\;|\;n\textrm{ is an integer}\,\big\}
            \end{equation}
            From here we can define $\mathbb{N}$ by:
            \begin{equation}
                \mathbb{N}=\{\,n\in\mathbb{Z}\;|\;n\geq{0}\,\}
            \end{equation}
            Furthermore, $\mathbb{N}_{e}$ and $\mathbb{N}_{0}$ can be described
            as follows:
            \par
            \begin{subequations}
                \begin{minipage}[b]{0.495\textwidth}
                    \centering
                    \begin{equation}
                        \label{eqn:Even_Pos_Ints_Set_Builder}%
                        \mathbb{N}_{e}=
                        \big\{n\in\mathbb{N}\;|\;n\textrm{ is even}\big\}
                    \end{equation}
                \end{minipage}
                \hfill
                \begin{minipage}[b]{0.495\textwidth}
                    \centering
                    \begin{equation}
                        \label{eqn:Odd_Pos_Ints_Set_Builder}%
                        \mathbb{N}_{o}=
                        \big\{n\in\mathbb{N}\;|\;n\textrm{ is odd}\big\}
                    \end{equation}
                \end{minipage}
            \end{subequations}
            \par\vspace{2.5ex}
            Such notation is justified by the axiom schema of specification%
            \index{Axiom!Schema of Specification}.
        \end{example}
        We are not adopting these definitions since they lack rigor. These
        examples build intuition behind the notation and the axioms, but we will
        develop arithmetic from an axiomatic viewpoint.
    \subsection{Ordered Pairs and Unions}
        We now wish to solve the issue previously raised that sets do
        not have order. We'll develop a new object, called ordered pairs%
        \index{Ordered Pair}, that can distinguish such things. The definition
        we'll adopt is due to Kuratowski\index{Kuratowski, Kazimierz} and uses
        the following form:
        \begin{equation}
            (a,\,b)=\Big\{\,\big\{\,a\,\big\},\,\big\{\,a,\,b\,\big\}\,\Big\}
        \end{equation}
        We now prove such a set exists within the framework of ZFC.
        \begin{faxiom}{Axiom of Pairing}{Axiom_of_Pairing}
            If $A$ and $B$ are sets, then there exists a set $\mathcal{C}$
            such that $A\in\mathcal{C}$ and $B\in\mathcal{C}$.
            \index{Axiom!of Pairing}
            \begin{equation*}
                \forall_{A}\forall_{B}\exists_{\mathcal{C}}:
                (A\in\mathcal{C}\land{B}\in\mathcal{C})
            \end{equation*}
        \end{faxiom}
        The set hypothesized to exist in this axiom may be very large, we have
        no way of knowing. What we want from this is a set that contains two
        elements $A$ and $B$, and only those elements. We obtain this by
        combining pairing with specification.
        \begin{theorem}
            \label{thm:Existence_of_Set_Built_from_Two_Sets}%
            If $A$ and $B$ are sets, then there exists a set $D$ such that,
            for all $x$ it is true that $x\in{D}$ if and only if $x=A$ or
            $x=B$. That is:
            \begin{equation}
                D=\{\,A,\,B\,\}
            \end{equation}
        \end{theorem}
        \begin{proof}
            By the axiom of pairing (Ax.~\ref{ax:Axiom_of_Pairing}) there
            exists a set $\mathcal{C}$ such that $A\in\mathcal{C}$ and
            $B\in\mathcal{C}$. Let $P$ be the proposition
            \textit{true if} $x=A$ \textit{or} $x=B$, \textit{false otherwise}.
            By the axiom schema of specification
            (Ax.~\ref{ax:Axiom_Schema_of_Specification}), there is a set
            $D$ such that $x\in{D}$ if and only if $x\in\mathcal{C}$ and
            $P(x)$ is true. But then $x\in{D}$ if and only if
            $x\in\mathcal{C}$ and $x=A$ or $x\in\mathcal{C}$ and $x=B$.
            But $A\in\mathcal{C}$ and $B\in\mathcal{C}$, and thus
            $x\in{D}$ if and only if $x=A$ or $x=B$.
        \end{proof}
        By the axiom of extensionality\index{Axiom!of Extensionality}
        (Ax.~\ref{ax:Axiom_of_Extensionality}), the set hypothesized in
        Thm.~\ref{thm:Existence_of_Set_Built_from_Two_Sets} is unique, and thus
        there is no trouble in \textit{defining} the symbol $\{A,B\}$ to be the
        unique set that contains the elements $A$ and $B$ and only those
        elements. That is, we develop the new notation:
        \begin{fnotation}{Finite Set Notation}{Finite_Set_Notation}
            If $A$ and $B$ are sets, then $\{A,B\}$ is the unique set such that
            for all $x$, $x\in\{A,B\}$ if and only if $x=A$ or $x=B$.
            \begin{equation*}
                \forall_{x}\Big(\big(x\in\{A,B\}\big)
                \Leftrightarrow\big((x=A)\lor(x=B)\big)\Big)
            \end{equation*}
        \end{fnotation}
        \begin{theorem}
            \label{thm:Existence_of_Set_Containing_Set}%
            If $A$ is a set, then there is a set $B$ such that $x\in{B}$ if
            and only if $x=A$. That is, there exists a set $B$ such that:
            \begin{equation}
                B=\{\,A\,\}
            \end{equation}
        \end{theorem}
        \begin{proof}
            For since $A$ is a set, by
            Thm.~\ref{thm:Existence_of_Set_Built_from_Two_Sets} there exists
            a set $B=\{A,\,A\}$. But then $x\in{B}$ if and only if $x=A$.
        \end{proof}
        We can apply Not.~\ref{not:Finite_Set_Notation} to a single set $A$ and
        similarly define what the notation $\{A\}$ means. With this, we can now
        prove the existence of ordered pairs.
        \begin{ltheorem}{Existence of Ordered Pairs}{Existence_of_Ordered_Pairs}
            If $A$ and $B$ are sets, then there is a set $(A,\,B)$ such that,
            for all $x$ it is true that $x\in(A,\,B)$ if only if $x=\{A\}$
            or $x=\{A,B\}$.
        \end{ltheorem}
        \begin{proof}
            For by Thm.~\ref{thm:Existence_of_Set_Containing_Set}, there is
            a set $\{A\}$ such that $x\in\{A\}$ if and only if $x=A$.
            But by Thm.~\ref{thm:Existence_of_Set_Built_from_Two_Sets}, there
            is a set $\{A,\,B\}$ such that $x\in\{A,\,B\}$ and if only
            if $x=A$ or $x=B$. But again by
            Thm.~\ref{thm:Existence_of_Set_Built_from_Two_Sets}, since
            $\{A\}$ and $\{A,\,B\}$ are sets, there is a set $(A,\,B)$ such
            that $x\in(A,\,B)$ if and only if $x=\{A\}$ or $x=\{A,\,B\}$.
        \end{proof}
        Thm.~\ref{thm:Existence_of_Ordered_Pairs} asserts the existence of
        ordered pairs\index{Ordered Pair}, as defined by Kuratowski, and allows
        us to provide the following definition.
        \begin{fdefinition}{Ordered Pairs}{Ordered_Pairs}
            The \gls{ordered pair} of a \gls{set} $x$ with respect to a set
            $y$ is the set:\index{Ordered Pair}
            \begin{equation*}
                (x,\,y)=\big\{\,\{\,x\,\},\,\{\,x,\,y\,\}\,\big\}
            \end{equation*}
        \end{fdefinition}
        Kuratowski\index{Kuratowski, Kazimierz} first put forward this definition
        in 1921 and this does precisely what we want it to do and orders
        elements. That is, if $x$ and $y$ are distinct, then
        $(x,\,y)\ne(y,\,x)$. The caveat with this definition is the following
        reduction:
        \begin{equation}
            (x,\,x)
            =\big\{\,\{\,x\,\},\,\{\,x,\,x\,\}\,\big\}
            =\big\{\,\{\,x\,\},\{\,x\,\}\,\big\}
            =\big\{\,\{\,x\,\}\,\big\}
        \end{equation}
        Prior to Kuratowski there existed a definition due to Norbert Wiener%
        \index{Wiener, Norbert}, put forward in 1914. His definition grew out of
        Bertrand Russell's\index{Russell, Bertrand} Type
        Theory\index{Type Theory} which was an attempt to rid set theory of the
        paradoxes he discovered. Wiener writes:
        \begin{equation}
            (x,\,y)_{W}=\Big\{\,\big\{\,\{\,x\,\},\,\emptyset\,\big\},\,
                                \big\{\,\{\,y\,\}\,\big\}\Big\}
        \end{equation}
        Returning to Kuratowski's definition (Def.~\ref{def:Ordered_Pairs}),
        consider the ordered pair $(1,\,2)$, where we take for granted that
        $1\ne{2}$. We have:
        \begin{equation}
            (1,\,2)=\big\{\,\{\,1\,\},\,\{\,1,\,2\,\}\,\big\}
        \end{equation}
        Swapping and computing $(2,\,1)$, we obtain:
        \begin{equation}
            (2,\,1)=\big\{\,\{\,2\,\},\,\{\,2,\,1\,\}\,\big\}
        \end{equation}
        We know that sets cannot distinguish order, so
        $\{\,1,\,2\,\}=\{\,2,\,1\,\}$. Thus:
        \par
        \begin{subequations}
            \begin{minipage}[b]{0.49\textwidth}
                \begin{equation}
                    (1,\,2)=\big\{\,\{\,1\,\},\,\{\,1,\,2\,\}\,\big\}
                \end{equation}
            \end{minipage}
            \hfill
            \begin{minipage}[b]{0.49\textwidth}
                \begin{equation}
                    (2,\,1)=\big\{\,\{\,2\,\},\,\{\,1,\,2\,\}\,\big\}
                \end{equation}
            \end{minipage}
        \end{subequations}
        \par\vspace{2.5ex}
        Combining these equations, we now have that:
        \begin{equation}
            (1,\,2)\ne(2,\,1)
        \end{equation}
        To see this, note that both sets contain the element $\{1,\,2\}$, but
        $\{1\}$ is an element of $(1,\,2)$ and not an element of $(2,\,1)$,
        and thus $(1,\,2)\nsubseteq(2,\,1)$. Similarly, $\{2\}$ is an element
        of $(2,\,1)$ but not an element of $(1,\,2)$, and therefore
        $(2,\,1)\nsubseteq(1,\,2)$. From the definition of equality
        (Def.~\ref{def:Equal_Sets}), we have that these sets are not equal.
        \par\hfill\par
        There's is an unfortunate doubling of notation that occurs in
        mathematics, and $(a,b)$ has two common meanings. The first meaning is
        the ordered pair which we've just defined, and the second is the
        \textit{open interval}\index{Open Interval} defined in the context of a
        \textit{partially ordered set}\index{Partially Ordered Set}.
        The most common example is when discussing the real numbers
        $\mathbb{R}$, $(a,b)$ denotes the set of all real numbers $x$ such that
        $a<x$ and $x<b$. Hopefully it will be clear what the notation means when
        a theorem or example is being presented, but we will be explicit when
        ambiguity can arise.
        \par\hfill\par
        The natural thing from here is to construct the
        \textit{Cartesian Product}\index{Cartesian Product} of two sets. This is
        the set of all ordered pairs\index{Ordered Pair} $(a,\,b)$ where $a$
        belongs to some set $A$ and $b$ belongs to another set $B$. To prove
        such a set exists requires two more axioms.
        \begin{faxiom}{Axiom of Union}{Axiom_of_Union}
            If $\mathcal{O}$ is a set, then there exists a set $\mathcal{F}$
            such that, for all $A$ such that $A\in\mathcal{O}$ and for all
            $x$ such that $x\in{A}$, it is true that $x\in\mathcal{F}$.%
            \index{Axiom!of Union}
            \begin{equation*}
                \forall_{\mathcal{O}}\exists_{\mathcal{F}}\forall_{x}:
                \big(\exists_{A\in\mathcal{O}}(x\in{A})\big)
                \Rightarrow{x}\in\mathcal{F}
            \end{equation*}
        \end{faxiom}
        This states that, given a collection of sets $\mathcal{O}$, there exists
        a larger set which contains the elements of the constituent sets of
        $\mathcal{O}$. One question that arises is
        \textit{what happens if our collection is empty}? That is, if
        $\mathcal{O}=\emptyset$, is there any meaning behind the equation
        (the notation $\cup$ is defined in Def.~\ref{def:Union_over_a_Set}):
        \begin{equation}
            \mathcal{F}=\bigcup_{\mathcal{U}\in\mathcal{O}}\mathcal{U}
        \end{equation}
        There is, and $\mathcal{F}$ will be the empty set. That is,
        $\mathcal{F}=\emptyset$. This is true in a vacuous sense and can be
        proved via contradiction with the law of the excluded middle.
        \begin{ltheorem}{Existence of the Union of Sets}{Existence_of_Unions}
            If $\mathcal{O}$ is a set, then there exists a set $\mathcal{F}$
            such that, for all $x$ it is true that $x\in\mathcal{F}$ if and
            only if there is a set $A\in\mathcal{O}$ such that $x\in{A}$.
        \end{ltheorem}
        \begin{proof}
            For by the axiom of union (Ax.~\ref{ax:Axiom_of_Union}), there
            exists a set $\mathcal{A}$ such that, for all $A\in\mathcal{O}$
            and for all $x\in{A}$, it is true that $x\in\mathcal{A}$. Let
            $P$ be the proposition \textit{true if there exists a set}
            $A\in\mathcal{O}$ \textit{such that} $x\in{A}$,
            \textit{false otherwise}. Then, by the axiom schema of specification
            (Ax.~\ref{ax:Axiom_Schema_of_Specification}) there exists a set
            $\mathcal{F}$ such that:
            \begin{equation}
                \mathcal{F}=\big\{\,x\in\mathcal{A}\;|\;P(x)\,\big\}
            \end{equation}
            But then $x\in\mathcal{F}$ if and only if $x\in\mathcal{A}$ and
            $P(x)$ is true. But if $P(x)$ is true, then $x\in\mathcal{A}$, and
            thus $x\in\mathcal{F}$ if and only if there is a set
            $A\in\mathcal{O}$ such that $x\in{A}$.
        \end{proof}
        We define the set $\mathcal{F}$ described in
        Thm.~\ref{thm:Existence_of_Unions} as the \textit{union}%
        \index{Union (Sets)} over the set $\mathcal{O}$. This set is often
        called the index set\index{Index Set} for which we take the union over.
        \begin{fdefinition}{Union over a Set}{Union_over_a_Set}
            The \gls{union over a set} $\mathcal{O}$ is the set:
            \index{Union (Sets)}
            \begin{equation*}
                \bigcup_{\mathcal{U}\in\mathcal{O}}\mathcal{U}
                =\big\{\,x\;|\;\textrm{There exists a set }
                         \mathcal{U}\in\mathcal{O}\textrm{ such that }
                         x\in\mathcal{U}\big\}
            \end{equation*}
            Using our formal language:
            \begin{equation*}
                \forall_{x}\Big(
                    x\in\bigcup_{\mathcal{U}\in\mathcal{O}}\mathcal{U}
                    \Leftrightarrow\exists_{\mathcal{U}\in\mathcal{O}}:
                    x\in\mathcal{U}\Big)
            \end{equation*}
        \end{fdefinition}
        This is very convenient if we have a collection of sets defined, but it
        would be nice to form the union over two given sets without considering
        them as part of a larger collection. This can be done by combining the
        axiom of union\index{Axiom!of Union} with
        pairing\index{Axiom!of Pairing}.
        \begin{theorem}
            \label{thm:Union_of_Two_Sets_Exists}%
            If $A$ and $B$ are sets, then there exists a set $C$ such that
            $x\in{C}$ if and only if either $x\in{A}$ or $x\in{B}$.
        \end{theorem}
        \begin{proof}
            For by Thm.~\ref{thm:Existence_of_Set_Built_from_Two_Sets},
            there exists a set $\mathcal{O}$ such that $y\in\mathcal{O}$ if and
            only if $y=A$ or $y=B$. That is, $\mathcal{O}=\{A,\,B\}$.
            But by Thm.~\ref{thm:Existence_of_Unions} there exists a set
            $C$ such that $x\in{C}$ if and only if there exist a set
            $F\in\mathcal{O}$ such that $x\in{F}$. But then $x\in{C}$ if and
            only if either $x\in{A}$ or $x\in{B}$.
        \end{proof}
        This allows us to define our first \textit{operation} of two sets.
        \begin{fdefinition}{Union of Two Sets}{Union_of_Two_Sets}
            The \gls{union of two sets} $A$ and $B$ is the set $A\cup{B}$
            defined by:
            \begin{equation*}
                A\cup{B}=\big\{\,x\;|\;x\in{A}\textrm{ or }x\in{B}\,\big\}
            \end{equation*}
        \end{fdefinition}
        In our definition of the union over a set and the union of two sets
        we have slightly abused our set-builder notation. The axiom schema
        of specification allows us to write a set as
        $A=\{\,x\in{B}\,|\,P(x)\,\}$, given some set $B$ that is already known
        to exists, and some proposition $P$. These two definitions
        (Def.~\ref{def:Union_over_a_Set} and \ref{def:Union_of_Two_Sets})
        are justified by the theorems we have proven, and so there is no
        contradiction.
        \begin{example}
            Again using the notation found in Eqn.~\ref{eqn:Z_n_Ellipses}, if we
            let $\mathbb{Z}_{n}$ denote the integers between $0$ and $n-1$, we
            have the following: If $n$ is less than $m$, then:
            \begin{equation}
                \mathbb{Z}_{n}\cup\mathbb{Z}_{m}=\mathbb{Z}_{m}
            \end{equation}
            This is because every element of $\mathbb{Z}_{n}$ is already
            and element of $\mathbb{Z}_{m}$, and thus taking the union adds
            nothing new to $\mathbb{Z}_{m}$, so the resulting set is
            $\mathbb{Z}_{m}$ itself.
        \end{example}
        \begin{example}
            Denoting the even and odd non-negative integers by $\mathbb{N}_{e}$
            and $\mathbb{N}_{o}$, respectively, we see that:
            \begin{equation}
                \mathbb{N}_{e}\cup\mathbb{N}_{o}=\mathbb{N}
            \end{equation}
            This is because every non-negative integer $n\in\mathbb{N}$ is
            either even or odd, and thus either $n\in\mathbb{N}_{e}$ or
            $n\in\mathbb{N}_{o}$. Taking the union therefore gives the entire
            set $\mathbb{N}$. The union does not add anything more than
            $\mathbb{N}$ since $\mathbb{N}_{e}\subseteq\mathbb{N}$ and
            $\mathbb{N}_{o}\subseteq\mathbb{N}$.
        \end{example}
        \begin{fexample}{Union of Two Sets}{Union_of_Two_Sets}
            Let $A$ and $B$ be the sets defined by:
            \par\hfill\par
            \begin{subequations}
                \begin{minipage}[b]{0.49\textwidth}
                    \begin{equation}
                        A=\{\,a,\,b,\,c\,\}
                    \end{equation}
                \end{minipage}
                \hfill
                \begin{minipage}[b]{0.49\textwidth}
                    \begin{equation}
                        B=\{\,c,\,1,\,2\,\}
                    \end{equation}
                \end{minipage}
            \end{subequations}
            \par\vspace{2.5ex}
            The union of $A$ and $B$ is the set that contains all of the
            elements of $A$ and all of the elements of $B$, and only such
            elements. That is:
            \begin{equation}
                A\cup{B}=\{\,a,\,b,\,c,\,1,\,2\,\}
            \end{equation}
            Even though $c\in{A}$ and $c\in{B}$, $c$ only appears once in the
            union. This is because sets cannot account for repetition, so
            including $c$ twice would be redundant.
        \end{fexample}
        The union of two sets can again be visualized by considering blobs
        in the plane. Let $A$ and $B$ be two circles that overlap somewhere in
        the middle. The union $A\cup{B}$ can then be represented by shading in
        the region covered by either $A$ or $B$
        (see Fig.~\ref{fig:Union_of_Two_Sets}). Such a drawing is called a
        \textit{Venn diagram}.
        \begin{figure}[H]
            \centering
            \captionsetup{type=figure}
            \documentclass[crop,class=article]{standalone}
%----------------------------Preamble-------------------------------%
\usepackage{tikz}                       % Drawing/graphing tools.
%--------------------------Main Document----------------------------%
\begin{document}
    \begin{tikzpicture}
        \draw (-2.5,-2) rectangle (2.5,2);
        \fill[cyan] (-0.8cm,0) circle (1.5cm);
        \fill[cyan] (0.8cm,0) circle (1.5cm);
        \draw (-0.8cm,0) circle (1.5cm);
        \draw (0.8cm,0) circle (1.5cm);
        \node at (-1,1.1) {$A$};
        \node at (1,1.1) {$B$};
        \node at (-1,1.8) {$A\cup{B}$};
    \end{tikzpicture}
\end{document}
            \caption{The Union of Two Sets}
            \label{fig:Union_of_Two_Sets}
        \end{figure}
        Fig.~\ref{fig:Union_of_Two_Sets} can be extended to an
        arbitrary collection of sets. For the sake of simplicity, a Venn
        diagram for the union of three sets is shown in
        Fig.~\ref{fig:Union_of_Three_Sets}. We can combine the axiom schema of
        specification (Ax.~\ref{ax:Axiom_Schema_of_Specification}) with the
        existence of the union of two sets to define intersections.
        \begin{figure}[H]
            \centering
            \captionsetup{type=figure}
            \begin{tikzpicture}
    % Coordinates for the centers of the circles.
    \coordinate (C1) at (-1.3,  0.00);
    \coordinate (C2) at ( 1.3,  0.00);
    \coordinate (C3) at ( 0.0, -2.15);

    % Coordinates for the labels.
    \coordinate (A) at (-1.3, 1.2);
    \coordinate (B) at ( 1.3, 1.2);
    \coordinate (C) at ( 0.0, -3.5);
    \coordinate (U) at ( 0.0, 2.5);

    % Rectangle indicating the universe set.
    \draw (-4, -4.5) rectangle (4, 3.2);

    % Fill in the circle with cyan.
    \draw[fill=cyan, draw=none] (C1) circle (2);
    \draw[fill=cyan, draw=none] (C2) circle (2);
    \draw[fill=cyan, draw=none] (C3) circle (2);

    % Give outlines to the circles.
    \draw (C1) circle (2);
    \draw (C2) circle (2);
    \draw (C3) circle (2);

    % Labels.
    \node at (A) {$A$};
    \node at (B) {$B$};
    \node at (C) {$C$};
    \node at (U)
        {\Large{$\underset{{\mathcal{U}\in\{A,B,C\}}}{\bigcup}\mathcal{U}$}};
\end{tikzpicture}
            \caption{The Union of Three Sets}
            \label{fig:Union_of_Three_Sets}
        \end{figure}
        \begin{theorem}
            If $A$ and $B$ are sets, then there exists a set $\mathcal{C}$
            such that, for all $x$ it is true that $x\in\mathcal{C}$ if and
            only if $x\in{A}$ and $x\in{B}$.
        \end{theorem}
        \begin{proof}
            For by Thm.~\ref{thm:Union_of_Two_Sets_Exists}, there exists
            a set $A\cup{B}$ such that, for all $x$ is is true that
            $x\in{A}\cup{B}$ if and only if $x\in{A}$ or $x\in{B}$. Let
            $P$ be the proposition \textit{True if} $x\in{A}$ \textit{and}
            $x\in{B}$, \textit{false otherwise}. Then by the axiom schema
            of specification (Ax.~\ref{ax:Axiom_Schema_of_Specification})
            there is a set $\mathcal{C}$ such that:
            \begin{equation}
                \mathcal{C}=\big\{\,x\in{A}\cup{B}\;|\;P(x)\,\big\}
            \end{equation}
            But then $x\in\mathcal{C}$ if and only if $x\in{A}\cup{B}$ and
            $x\in{A}$ and $x\in{B}$. But if $x\in{A}$ and $x\in{B}$, then
            $x\in{A}$, and thus $x\in{A}\cup{B}$
            (Def.~\ref{def:Union_of_Two_Sets}). That is, $P(x)$ implies that
            $x\in{A}\cup{C}$. Therefore, $x\in\mathcal{C}$
            if and only if $x\in{A}$ and $x\in{B}$.
        \end{proof}
        \begin{fdefinition}{Intersection of Two Sets}
                           {Intersection_of_Two_Sets}
            The \gls{intersection of two sets} $A$ and $B$ is the set:
            \begin{equation*}
                A\cap{B}
                =\big\{\,x\in{A}\cup{B}\;|\;
                    a\in{A}\textrm{ and }b\in{B}\,\big\}
            \end{equation*}
        \end{fdefinition}
        We'll need one brief theorem about intersections to allow us to prove
        that certain sets are not equal.
        \begin{theorem}
            \label{thm:Lemma_for_Anti_Russells_Paradox}%
            If $A$ and $B$ are sets, if $x\in{B}$, and if $x\notin{A}\cap{B}$,
            then $x\notin{A}$.
        \end{theorem}
        \begin{proof}
            For if $x\notin{A}\cap{B}$ then either $x\notin{A}$ or $x\notin{B}$
            (Def.~\ref{def:Intersection_of_Two_Sets}). But $x\in{B}$, and
            therefore $x\notin{A}$.
        \end{proof}
        \begin{fexample}{Intersections of Two Sets}
                        {Intersections_of_Two_Sets}
            If we let $A$ and $B$ be the sets defined by:
            \par\hfill\par
            \begin{subequations}
                \begin{minipage}[b]{0.49\textwidth}
                    \begin{equation}
                        A=\{\,a,\,b,\,c\,\}
                    \end{equation}
                \end{minipage}
                \hfill
                \begin{minipage}[b]{0.49\textwidth}
                    \begin{equation}
                        B=\{\,c,\,1,\,2\,\}
                    \end{equation}
                \end{minipage}
            \end{subequations}
            \par\vspace{2.5ex}
            we have that the intersection is then:
            \begin{equation}
                A\cap{B}=\{\,1\,\}
            \end{equation}
            Letting $\mathbb{N}_{e}$ and $\mathbb{N}_{o}$ once again denote
            the even and odd non-negative integers, respectively, we have:
            \begin{equation}
                \mathbb{N}_{e}\cap\mathbb{N}_{o}=\emptyset
            \end{equation}
            This is precisely what was meant earlier when it was claimed that
            $\mathbb{N}_{e}$ and $\mathbb{N}_{o}$ are disjoint%
            \index{Disjoint Sets}. If $n,m\in\mathbb{N}$ and if $n<m$, we have:
            \begin{equation}
                \mathbb{Z}_{n}\cap\mathbb{Z}_{m}=\mathbb{Z}_{n}
            \end{equation}
            This is because every element of $\mathbb{Z}_{n}$ is an element
            of $\mathbb{Z}_{m}$. This creates a more general theorem that if
            $A\subseteq{B}$, then $A\cap{B}=A$. Intersections thus seem like
            antonyms of unions.
        \end{fexample}
        Similar to how unions can be visualized with Venn diagrams
        (Fig.~\ref{fig:Union_of_Two_Sets}), so can the intersection of
        two sets. We draw two circles that overlap slightly, and consider the
        region contained in both (see Fig.~\ref{fig:Intersection_of_Two_Sets}).
        \begin{figure}[H]
            \centering
            \documentclass[crop,class=article]{standalone}
%----------------------------Preamble-------------------------------%
\usepackage{tikz}                       % Drawing/graphing tools.
%--------------------------Main Document----------------------------%
\begin{document}
    \begin{tikzpicture}
        \draw (-2.5,-2) rectangle (2.5,2);
        \draw (-0.8cm,0) circle (1.5cm);
        \draw (0.8cm,0) circle (1.5cm);
        \draw[fill=cyan]
            (0,-1.26886) arc(-57.77:57.77:1.5)
                         arc(122.231:237.7690:1.5);
        \node at (-1,1.1) {$A$};
        \node at (1,1.1) {$B$};
        \node at (-1,1.8) {$A\cap{B}$};
    \end{tikzpicture}
\end{document}
            \caption{Venn Diagram for Intersection}
            \label{fig:Intersection_of_Two_Sets}
        \end{figure}
        We can extend this further and define the intersection over any
        collection of sets.
        \begin{theorem}
            \label{thm:Existence_of_Arbitrary_Intersetions}%
            If $\mathcal{O}$ is a set, then there exists a set
            $\mathcal{C}$ such that, for all $x$ it is true that
            $x\in\mathcal{C}$ if and only if for all
            $\mathcal{U}\in\mathcal{O}$ it is true that $x\in\mathcal{U}$.
            That is:
            \begin{equation}
                \mathcal{C}=\bigcap_{\mathcal{U}\in\mathcal{O}}\mathcal{U}
            \end{equation}
        \end{theorem}
        \begin{proof}
            For by Thm.~\ref{thm:Existence_of_Unions} there is a set
            $\bigcup\mathcal{U}$ such that for all $x$ it is true that
            $x\in\bigcup\mathcal{U}$ if and only if for all
            $\mathcal{U}\in\mathcal{O}$ it is true that $x\in\mathcal{U}$. Let
            $P$ be the proposition \textit{True if for all}
            $\mathcal{U}\in\mathcal{O}$ \textit{it is true that}
            $x\in\mathcal{U}$, \textit{false otherwise}. Then by the
            axiom schema of specification
            (Ax.~\ref{ax:Axiom_Schema_of_Specification}), there exists the set:
            \begin{equation}
                \mathcal{C}
                =\Big\{\,x\in\bigcup_{\mathcal{U}\in\mathcal{O}}\mathcal{U}
                    \;|\;P(x)\,\Big\}
            \end{equation}
            But then $x\in\mathcal{C}$ if and only if $x\in\bigcup\mathcal{U}$
            and $P(x)$ is true. But if $P(x)$ is true, then for all
            $\mathcal{U}\in\mathcal{O}$ it is true that $x\in\mathcal{U}$. But
            then there is a $\mathcal{U}\in\mathcal{O}$ such that
            $x\in\mathcal{U}$, and thus $P(x)$ implies that
            $x\in\bigcup\mathcal{U}$. Therefore $x\in\mathcal{C}$ if and only if
            for all $\mathcal{U}\in\mathcal{O}$ it is true that
            $x\in\mathcal{U}$.
        \end{proof}
        It is common to consider some \textit{universal} set, of which all
        other sets of current consideration are drawn from. Using this the
        definition of arbitrary intersection is defined as the subset of
        this universal set such that every element of this subset is
        contained in every element of the arbitrary collection. One may then
        ask what would happen if the collection is empty. Using this
        definition the intersection would be the entire universal set
        in a vacuous sense. That is, there would be no $x$ in the universe
        that fails the definition of the intersection over an empty
        collection, and thus the intersection is everything. Letting $X$
        denote our universe, we obtain:
        \begin{equation}
            \emptyset=\bigcup_{\mathcal{U}\in\emptyset}\mathcal{U}
                \subseteq\bigcap_{\mathcal{U}\in\emptyset}\mathcal{U}
            =X
        \end{equation}
        It seems like unions should always be bigger. Indeed, for any
        non-empty collection, the intersection over the collection is a
        subset of the union over the collection. Because of this we do
        not adopt this definition of the intersection over a collection,
        but rather require in our construction the use of the union over
        the collection, and then use the axiom schema of specification to
        pick the subset of all elements of the union that belong to every
        element of the collection. Thus:
        \begin{equation}
            \bigcap_{\mathcal{U}\in\emptyset}\mathcal{U}
            \subseteq\bigcup_{\mathcal{U}\in\emptyset}\mathcal{U}
            =\emptyset
        \end{equation}
        And from this we conclude the intersection is empty as well.
        \begin{fdefinition}{Intersection Over a Collection}
                           {Intersection_Over_a_Collection}
            The \gls{intersection over a set} $\mathcal{O}$
            of sets is the set:
            \begin{equation*}
                \bigcap_{E\in\mathcal{O}}E
                =\Big\{\,x\in\bigcup_{E\in\mathcal{O}}E\;\Big|\;x\in{E}
                    \textrm{ for all }E\in\mathcal{O}\,\Big\}
            \end{equation*}
        \end{fdefinition}
        We can extend our Venn diagram for larger collections as well
        (see Fig.~\ref{fig:Intersection_of_Three_Sets}).
        \begin{figure}[H]
            \centering
            \begin{tikzpicture}
    % Coordinates for the centers of the circles.
    \coordinate (C1) at (-1.3,  0.00);
    \coordinate (C2) at ( 1.3,  0.00);
    \coordinate (C3) at ( 0.0, -2.15);

    \coordinate (O)  at ( 0.0000, -0.7500);
    \coordinate (P1) at ( 0.6817, -0.2697);
    \coordinate (P2) at (-0.6817, -0.2697);
    \coordinate (P3) at ( 0.0000, -1.5198);

    % Coordinates for the labels.
    \coordinate (A) at (-1.3, 1.2);
    \coordinate (B) at ( 1.3, 1.2);
    \coordinate (C) at ( 0.0, -3.5);
    \coordinate (U) at ( 0.0, 2.5);

    % Rectangle indicating the universe set.
    \draw (-4, -4.5) rectangle (4, 3.3);

    % Fill in the intersection with cyan.
    \draw[fill=cyan, draw=none] (P1) arc(70.07:109.93:2)
                                     arc(187.8:229.50:2)
                                     arc(310.54:352.25:2);

    % Give outlines to the circles.
    \draw (C1) circle (2);
    \draw (C2) circle (2);
    \draw (C3) circle (2);

    % Labels.
    \node at (A) {$A$};
    \node at (B) {$B$};
    \node at (C) {$C$};
    \node at (U)
        {\Large{$\underset{\mathcal{U}\in\{A,B,C\}}{\bigcap}\mathcal{U}$}};
\end{tikzpicture}
            \caption{Intersection of Three Sets}
            \label{fig:Intersection_of_Three_Sets}
        \end{figure}
        Next on the list of axioms is that of \textit{regularity}.
        \begin{faxiom}{Axiom of Regularity}{Axiom_of_Regularity}
            If $A$ is a non-empty set, then there is an element $B\in{A}$
            such that $A\cap{B}=\emptyset$.\index{Axiom!of Regularity}
            \begin{equation*}
                \forall_{A}(\exists_{x\in{A}})\Rightarrow
                \exists_{y}:\Big((y\in{A})\land
                \big((y\cap{A})=\emptyset\big)\Big)
            \end{equation*}
        \end{faxiom}
        This axiom is often seen as unnecessary by many working mathematicians
        and indeed it's use seems to only lie in set theory and foundations.
        That is, unlike the axioms of choice and union which are widely
        applicable to analysis and topology, regularity seems to only be useful
        to set theoriests. Nevertheless, it is vital to support the claim that
        ZFC is a good system to base mathematics on. We will combine this with
        pairing to prove that for any set $A$ it is true that $A\notin{A}$. That
        is, Zermelo-Fraenkel set theory is free of Russell's
        paradox\index{Russell's Paradox}.
        \begin{theorem}
            \label{thm:Anti_Russells_Paradox}%
            If $A$ is a set, then $A\notin{A}$.
        \end{theorem}
        \begin{proof}
            For if $A$ is a set, then $\{A\}$ is a set
            (Thm.~\ref{thm:Existence_of_Set_Containing_Set}). But since
            $A\in\{A\}$, $\{A\}$ is a non-empty set
            (Def.~\ref{def:Non_Empty_Set}). Thus by the axiom of regularity
            (Ax.~\ref{ax:Axiom_of_Regularity}), there is a set $B\in\{A\}$ such
            that $B\cap\{A\}=\emptyset$. But $B\in\{A\}$ if and only if
            $B=A$, and therefore $A\cap\{A\}=\emptyset$. Thus, by the axiom of
            the empty set (Ax.~\ref{ax:Axiom_of_the_Empty_Set}), for all $x$ it
            is true that $x\notin{A}\cap\{A\}$ and therefore
            $A\notin{A}\cap\{A\}$. But $A\in\{A\}$ and therefore
            $A\notin{A}$ (Thm.~\ref{thm:Lemma_for_Anti_Russells_Paradox}).
        \end{proof}
        \begin{theorem}
            \label{thm:Containment_NEqual_Underlying_Set}%
            If $A$ and $B$ are sets and if $A\in{B}$, then $A\ne{B}$.
        \end{theorem}
        \begin{proof}
            For $A\notin{A}$ (Thm.~\ref{thm:Anti_Russells_Paradox}) and
            $A\in{B}$ and therefore it is not true that for all $x$, $x\in{A}$
            if and only if $x\in{B}$. Therefore, by the Axiom of
            Extensionality (Ax.~\ref{ax:Axiom_of_Extensionality}), $A\ne{B}$.
        \end{proof}
        \begin{theorem}
            If $A$ is a set, then $A\ne\{A\}$.
        \end{theorem}
        \begin{proof}
            \label{thm:Cor_of_Containment_NEqual_Underlying_Set}%
            For if $A$ is a set, then $A\notin{A}$
            (Thm.~\ref{thm:Anti_Russells_Paradox}). But if $A$ is a set, then
            $\{A\}$ is a set (Thm.~\ref{thm:Existence_of_Set_Containing_Set}).
            But $A\in\{A\}$, and thus $A\ne\{A\}$
            (Thm.~\ref{thm:Containment_NEqual_Underlying_Set}).
        \end{proof}
        This quick theorem will eventually prove the well known result that
        $0\ne{1}$. It also shows that there is no set of all
        sets\index{Set!of All Sets}.
    \subsection{The Axiom of the Power Set}
        Continuing in our goal of constructing order, we move on to the
        Cartesian product of two sets $A$ and $B$. This is the collection of
        all ordered pairs $(a,b)$ such that $a\in{A}$ and $b\in{B}$. To prove
        such a set exists requires the \textit{axiom of the power set}.
        \begin{faxiom}{Axiom of the Power Set}{Axiom_of_the_Power_Set}
            If $X$ is a set, then there exists a set $\mathscr{P}$ such that,
            for all $A\subseteq{X}$, it is true that $A\in\mathscr{P}$.
            \index{Axiom!of the Power Set}
            \begin{equation*}
                \forall_{X}\exists_{\mathscr{P}}:
                \forall_{A}(A\subseteq{X})\Rightarrow{A}\in\mathscr{P}
            \end{equation*}
        \end{faxiom}
        Again, much like the axiom of union and the axiom of pairing, this
        set may be bigger than we would like. We wish to find a set, called
        the \textit{power set}, the contains all of the subsets of a given
        set $A$ and nothing else. Combining the axiom of the power set
        with the axiom schema of specification gives us such existence.
        \begin{ltheorem}{Existence of the Power Set}
                        {Existence_of_the_Power_Set}
            If $A$ is a set, then there exists a set $\mathcal{P}(A)$
            such that, for all $x$ it is true that $x\in\mathcal{P}(A)$ if and
            only if $x\subseteq{A}$.
        \end{ltheorem}
        \begin{proof}
            For by the axiom of the power set
            (Ax.~\ref{ax:Axiom_of_the_Power_Set}), there is a set
            $\mathscr{P}(A)$ such that for all $x\subseteq{A}$ it is true
            that $x\in\mathscr{P}(A)$. Let $P$ be the proposition
            \textit{true if} $x\subseteq{A}$, \textit{false otherwise}. By the
            axiom schema of specification
            (Ax.~\ref{ax:Axiom_Schema_of_Specification}), there is a set
            $\mathcal{P}(A)$ such that:
            \begin{equation}
                \mathcal{P}(A)=\{\,x\in\mathscr{P}(A)\;|\;P(x)\,\}
            \end{equation}
            But if $P(x)$ is true, then $x$ is a subset of $A$, and therefore
            $x\in\mathscr{P}(A)$. Thus $x\in\mathcal{P}(A)$ if and only if
            $x\subseteq{A}$.
        \end{proof}
        With this we now define the \textit{power set} of a given set.
        \begin{fdefinition}{Power Set}{Power_Set}
            The \gls{power set} of a \gls{set} $X$ is the set $\mathcal{P}(X)$
            defined by:\index{Power Set}
            \begin{equation*}
                \mathcal{P}(X)=\{\,A\;|\;A\subseteq{X}\,\}
            \end{equation*}
            That is, the set of all subsets of $X$.
            \begin{equation*}
                \forall_{A}\forall_{B}\Big(B\in\mathcal{P}(A)\Longleftarrow
                    B\subseteq{A}\Big)
            \end{equation*}
        \end{fdefinition}
        Thm.~\ref{thm:Existence_of_the_Power_Set} justifies such a definition.
        The power set of a set is a crucial construction for when one discusses
        the \textit{cardinality} of sets, denoted $\textrm{Card}(A)$. This
        describes the \textit{size} of a set in a very precise manner. A theorem
        that will eventually be proved known as
        \textit{Cantor's Theorem}\index{Cantor's Power Set Theorem} shows that
        the power set of a set is always strictly \textit{larger} than the
        original set. That is:
        \begin{equation}
            \textrm{Card}(A)<\textrm{Card}\big(\mathcal{P}(A)\big)
        \end{equation}
        This will be made precise soon enough. The axiom of the power set allows
        us to build \textit{larger} sets from a given set. That is, given a set
        $A$ it is true that $A\ne\mathcal{P}(A)$ (otherwise we would violate the
        axiom of regularity). But since $A\in\mathcal{P}(A)$ we see that the
        power set is in a sense \textit{larger}.
        \begin{example}
            If $A=\{1,2\}$, then the power set is:
            \begin{equation}
                \mathcal{P}(A)=\big\{\,\emptyset,\,\{1\},\,\{2\},\,
                    \{1,2\}\,\big\}
            \end{equation}
            We must consider the empty set since $\emptyset\subseteq{A}$.
            Now suppose $A=\{1,2,3\}$:
            \begin{equation}
                \mathcal{P}(A)=\big\{\,\emptyset,\,\{1\},\,\{2\},\,\{3\},\,
                    \{1,2\},\,\{1,3\},\,\{2,3\},\,\{1,2,3\}\,\big\}
            \end{equation}
            We see that a set with 2 elements has a power set with 4 elements
            and a set with 3 elements has a power set with 8. This pattern
            continues for finite sets and if $A$ has $n$ elements, then
            $\mathcal{P}(A)$ has $2^{n}$ elements.
        \end{example}
        \begin{example}
            When we consider the case of an \textit{infinite} set $A$ we have
            that $\mathcal{P}(A)$ is a strictly larger set and this creates a
            paradoxical heirarchy of infinities. The smallest heirarchy is
            that of the \textit{countable} infinite sets, like $\mathbb{N}$.
            Everything larger is called \textit{uncountable}. It will be
            shown that the following is true:
            \begin{equation}
                \Card\big(\mathcal{P}(\mathbb{N})\big)=
                \Card(\mathbb{R})
            \end{equation}
            where again $\mathbb{N}$ denotes the non-negative integers and
            $\mathbb{R}$ denotes the set of all \textit{real} numbers. We
            can loosely show this by using the binary representation of real
            numbers. A real number may be thought of as an infinite decimal.
            For example, $\pi=3.1415926\dots$ and $1=1.000\dots$ We can
            also represent real numbers as a sequence of zeroes and ones and
            this is the \textit{binary} representation. For
            $A\subseteq\mathbb{N}$ and let $r_{A}=0.n_{1}n_{2}\hdots$ where:
            \begin{equation}
                n_{i}=
                \begin{cases}
                    0,&i\notin{A}\\
                    1,&i\in{A}
                \end{cases}
            \end{equation}
            Thus for each $A\in\mathcal{P}(\mathbb{N})$ there is a real
            number $r_{A}$ such that $0\leq{r}_{A}\leq{1}$ that is
            associated with it, and moreover to every real number between
            zero and one there is a subset of $\mathbb{N}$ associated with
            it. The tricky numbers to see are zero and one, but note that
            $r_{\emptyset}$ is associated to 0 and $r_{\mathbb{N}}$ gets
            paired with 1. To show that $\mathbb{R}$ and
            $\mathcal{P}(\mathbb{N})$ are the same size requires us to
            refine this association so that every element of
            $\mathcal{P}(\mathbb{N})$ uniquely corresponds to an element of
            $\mathbb{R}$, and vice-versa.
        \end{example}
    \subsection{Cartesian Products and Functions}
        In the previous section we introduced ordered pairs and the notion of
        the power set. We can use both of these concepts to define and prove
        the existence of \textit{Cartesian products}. Intuitively we want to
        define $A\times{B}$ to be the set of all ordered pairs $(a,b)$ where
        $a\in{A}$ and $b\in{B}$:
        \begin{equation}
            A\times{B}=\big\{\,(a,\,b)\;|\;a\in{A}\textrm{ and }b\in{B}\,\big\}
        \end{equation}
        But recalling Def.~\ref{def:Ordered_Pairs}, ordered pairs are sets
        of the form $\{\{a\},\{a,b\}\}$. Thus elements of $A\times{B}$ are
        contained in the power set of the power set of $A\cup{B}$:
        \begin{equation}
            A\times{B}\subseteq\mathcal{P}\big(\mathcal{P}(A\cup{B})\big)
        \end{equation}
        We can combine the axiom of the power set with the axiom schema of
        specification to obtain the existence of the Cartesian product of
        two sets.
        \begin{theorem}
            \label{thm:Ordered_Pair_Subset_of_Power_Set}%
            If $A$ and $B$ are sets, if $a\in{A}$ and $b\in{B}$, then
            $(a,b)\subseteq\mathcal{P}(A\cup{B})$.
        \end{theorem}
        \begin{proof}
            For if $a\in{A}$ and $b\in{B}$, then
            $(a,b)=\{\,\{\,a\,\},\,\{\,a,\,b\,\}\,\}$
            (Def.~\ref{def:Ordered_Pairs}). But if $a\in{A}$, then $a\in{A}$
            or $a\in{B}$, and thus $a\in{A}\cup{B}$
            (Def.~\ref{def:Union_of_Two_Sets}). But then
            $\{\,a\,\}\subseteq{A}\cup{B}$ (Def.~\ref{def:Subsets}). But if
            $b\in{B}$, then $b\in{A}$ or $b\in{B}$, and thus $b\in{A}\cup{B}$
            (Def.~\ref{def:Union_of_Two_Sets}). But then
            $\{\,a,\,b\,\}\subseteq{A}\cup{B}$ (Def.~\ref{def:Subsets}).
            But then $\{\,a\,\}\subseteq{A}\cup{B}$ and
            $\{\,a,\,b\,\}\subseteq{A}\cup{B}$, and thus
            $(a,b)\subseteq\mathcal{P}(A\cup{B})$ (Def.~\ref{def:Power_Set}).
        \end{proof}
        \begin{ltheorem}{Existence of the Cartesian Product}
                        {Existence_of_the_Cartesian_Product}
            If $A$ and $B$ are sets, then there exists a set $A\times{B}$
            such that, for all $z$, $z\in{A}\times{B}$ if and only if there
            is an $a\in{A}$ and $b\in{B}$ such that $z=(a,b)$.
        \end{ltheorem}
        \begin{proof}
            For if $A$ and $B$ are sets, then by the axiom of union
            (Ax.~\ref{ax:Axiom_of_Union}), $A\cup{B}$ is a set. But if
            $A\cup{B}$ is a set, by the axiom of the power set
            (Ax.~\ref{ax:Axiom_of_the_Power_Set}), $\mathcal{P}(A\cup{B})$ is
            a set, where $\mathcal{P}(X)$ denotes the power set of $X$. But if
            $\mathcal{P}(A\cup{B})$ is a set, then
            $\mathcal{P}(\mathcal{P}(A\cup{B}))$ is a set
            (Ax.~\ref{ax:Axiom_of_the_Power_Set}). But then
            $z\in\mathcal{P}(\mathcal{P}(A\cup{B}))$ if and only if
            $z\subseteq\mathcal{P}(A\cup{B})$ (Def.~\ref{def:Power_Set}).
            But if $a\in{A}$ and $b\in{B}$, then
            $(a,b)\subseteq\mathcal{P}(A\cup{B})$
            (Thm.~\ref{thm:Ordered_Pair_Subset_of_Power_Set}), and therefore
            $(a,b)\in\mathcal{P}(\mathcal{P}(A\cup{B}))$
            (Def.~\ref{def:Power_Set}). Let $P$ be the proposition
            \textit{True if there exists} $a\in{A}$ \textit{and}
            $b\in{B}$ \textit{such that} $z=(a,b)$, \textit{false otherwise}.
            Then by the axiom schema of specification
            (Ax.~\ref{ax:Axiom_Schema_of_Specification}), there exists a
            set $A\times{B}$ such that:
            \begin{equation}
                A\times{B}=
                \{\,z\in\mathcal{P}\big(\mathcal{P}(A\cup{B})\big)\;|\;
                    P(z)\,\}
            \end{equation}
            But it was proved that $P(z)$ implies that
            $z\in\mathcal{P}(\mathcal{P}(A\cup{B}))$. Thus $z\in{A}\times{B}$
            if and only if there exists $a\in{A}$ and $b\in{B}$
            such that $z=(a,b)$.
        \end{proof}
        \begin{fdefinition}{Cartesian Product of Two Sets}
                           {Cartesian_Product_of_Two_Sets}
            The \gls{Cartesian product} of two \glspl{set} $A$ and $B$ is the
            set
            \begin{equation*}
                A\times{B}
                =\{\,(a,\,b)\;|\;a\in{A}\textrm{ and }b\in{B}\,\}
            \end{equation*}
        \end{fdefinition}
        Note that since, in general, $(a,b)\ne(b,a)$, it is generally true that
        $A\times{B}\ne{B}\times{A}$. Indeed, equality occurs if and only if
        $A=B$ (or if either set is empty).
        \begin{fexample}{Basic Cartesian Products}{Basic_Cartesian_Products}
            Let $A$ and $B$ be sets defined as follows:
            \par
            \begin{subequations}
                \begin{minipage}[b]{0.49\textwidth}
                    \centering
                    \begin{equation}
                        A=\{\,1,\,2,\,3\,\}
                    \end{equation}
                \end{minipage}
                \hfill
                \begin{minipage}[b]{0.49\textwidth}
                    \centering
                    \begin{equation}
                        B=\{\,a,\,b\,\}
                    \end{equation}
                \end{minipage}
            \end{subequations}
            \par\vspace{2.5ex}
            Let's compute $A\times{B}$ and $B\times{A}$. From the definition
            (Def.~\ref{def:Cartesian_Product_of_Two_Sets}) we have:
            \begin{equation}
                A\times{B}=\{\,(a,b)\;|\;a\in{A}\textrm{ and }b\in{B}\,\}
            \end{equation}
            Using this, we can compute:
            \begin{equation}
                A\times{B}=\big\{\,(1,a),\,(2,a),\,(3,a),\,
                                   (1,b),\,(2,b),\,(3,b)\,\big\}
            \end{equation}
            Computing $B\times{A}$, we have:
            \begin{equation}
                B\times{A}=\big\{\,(a,\,1),\,(a,\,2),\,(a,\,3),\,
                                   (b,\,1),\,(b,\,2),\,(b,\,3)\,\big\}
            \end{equation}
            Now if we suppose that $a$ is not equal to 1, then we see that
            $(a,1)$ is a different element than $(1,a)$, and thus $A\times{B}$
            is not equal to $B\times{A}$. Next, compute $A\times{A}$:
            \begin{equation}
                \begin{split}
                    A\times{A}=\Big\{\,(1,1),\,(1,2),\,&(1,3),
                                       (2,1),\,(2,2),\,\\&(2,3),
                                       (3,1),\,(3,2),\,(3,3)\,\Big\}
                \end{split}
            \end{equation}
            And finally $B\times{B}$:
            \begin{equation}
                B\times{B}=\big\{\,(a,\,a),\,(a,\,b),
                                 \,(b,\,a),\,(b,\,b)\,\big\}
            \end{equation}
            Equality of $A\times{B}$ and $B\times{A}$ is achieved if and only
            if $A=B$, or if either set is the empty set.
        \end{fexample}
        Note that in Ex.~\ref{ex:Basic_Cartesian_Products}, the \textit{size}
        of the Cartesian product of two sets was simply the product of the
        number of elements of the constituent sets. That is, we see that $A$
        has three elements and $B$ has two elements, but also that
        $A\times{B}$ has six elements. Moreover, $A\times{A}$ has nine
        elements and $B\times{B}$ has four. This pattern holds for the
        Cartesian products of any two \textit{finite} sets.
        \par\hfill\par
        It is common to consider the Cartesian product of a set with itself.
        That is, given a set $A$, we are often interested in $A\times{A}$. We
        denote this by writing $A^{2}$. One such example is when we consider
        the set of real numbers $\mathbb{R}$. The Cartesian product
        $\mathbb{R}^{2}$ is called the \textit{Euclidean Plane},
        or the \textit{Cartesian Plane}, after Euclid of Alexandria%
        \index{Euclide of Alexandria} and Ren\'{e} Descartes%
        \index{Descartes, Ren\'{e}}. This is because $\mathbb{R}^{2}$ is used to
        model both planar geometry and analytical geometry, of which Euclid and
        Descartes were pioneers of, respectively. The term Cartesian products
        is in honor of Ren\'{e} Descartes, as well.
        Let $\mathbb{R}$ denote the set of real numbers, and let
        $A=\mathbb{R}$ and $B=\mathbb{R}$. Then we have:
        \begin{equation}
            A\times{B}=\mathbb{R}\times\mathbb{R}\equiv\mathbb{R}^{2}
        \end{equation}
        Where the symbol $\equiv$ again means that $\mathbb{R}^{2}$ is
        defined by this expression. Using the definition of Cartesian
        products (Def.~\ref{def:Cartesian_Product_of_Two_Sets}), we obtain:
        \begin{equation}
            \mathbb{R}^{2}=\{\;(x,y)\,:\,x\in\mathbb{R}
                               \textrm{ and }y\in\mathbb{R}\;\}
        \end{equation}
        That is, $\mathbb{R}^{2}$ is the set of all ordered pairs of real
        numbers. The first term is called the $x$ coordinate, and similarly the
        second term is called the $y$ coordinate. We envision this as a
        \textit{plane} of points, each one corresponding to an ordered pair
        $(x,y)$. This is depicted in Fig.~\ref{fig:Cartesian_Plane}.
        \begin{figure}[H]
            \centering
            %--------------------------------Dependencies----------------------------------%
%   tikz                                                                       %
%       arrows.meta                                                            %
%-------------------------------Main Document----------------------------------%
\begin{tikzpicture}[%
    >=Latex,
    line width=0.2mm,
    line cap=round,
    font=\Large
]
    % Coordinates for the points.
    \coordinate (x) at (2.2, 0.0);
    \coordinate (y) at (0.0, 2.9);
    \coordinate (z) at (2.2, 2.9);

    % Draw a grid.
    \draw[style=help lines] (-0.3, -0.3) grid (7.9, 7.9);

    % Axes.
    \begin{scope}[thick]
        \draw[->] (-0.3, 0) to (8.4, 0) node [above] {$\mathbb{R}$};
        \draw[->] (0, -0.3) to (0, 8.4) node [right] {$\mathbb{R}$};
    \end{scope}

    % Draw dashed lines to the point.
    \begin{scope}[densely dashed]
        \draw (x) to (z);
        \draw (y) to (z);
    \end{scope}

    % Draw dots marking the various points.
    \draw[fill=black] (x) circle (0.6mm);
    \draw[fill=black] (y) circle (0.6mm);
    \draw[fill=black] (z) circle (0.6mm);

    % Label the points x, y, and the dot (x,y) in the plane.
    \node at (x) [below=0.1]     {$x$};
    \node at (y) [left=0.1]      {$y$};
    \node at (z) [above right]   {$(x,\,y)$};
\end{tikzpicture}
            \caption{The Cartesian Plane $\mathbb{R}^{2}$}
            \label{fig:Cartesian_Plane}
        \end{figure}
        Consider further the set $\mathbb{N}^{2}$. That is, letting
        $\mathbb{N}$ denote the set of natural numbers
        (Eqn.~\ref{eqn:Natural_Numbers_Ellipses}), letting $A=\mathbb{N}$ and
        $B=\mathbb{N}$ we have:
        \begin{equation}
            A\times{B}=\mathbb{N}\times\mathbb{N}\equiv\mathbb{N}^{2}
        \end{equation}
        Again using the definition of Cartesian products
        (Def.~\ref{def:Cartesian_Product_of_Two_Sets}), we have:
        \begin{equation}
            \mathbb{N}^{2}=
            \{\,(n,\,m)\;|\;n\in\mathbb{N}\textrm{ and }m\in\mathbb{N}\,\}
        \end{equation}
        We can visualize this as a subset of $\mathbb{R}^{2}$ by drawing a
        lattice of points in the Cartesian plane
        (Fig.~\ref{fig:Lattice_Cart_Prod_of_N_with_N}).
        \begin{figure}[H]
            \centering
            %--------------------------------Dependencies----------------------------------%
%   tikz                                                                       %
%       arrows.meta                                                            %
%-------------------------------Main Document----------------------------------%
\begin{tikzpicture}[%
    >=Latex,
    line width=0.2mm,
    line cap=round
]

    % Axes.
    \begin{scope}[thick, font=\Large]
        \draw[->] (0, 0) to (8.4, 0) node [above] {$\mathbb{N}$};
        \draw[->] (0, 0) to (0, 8.4) node [right] {$\mathbb{N}$};
    \end{scope}

    \foreach\x in{1, 2, 3, 4, 5, 6, 7, 8}{
        \foreach\y in{1, 2, 3, 4, 5, 6, 7, 8}{
            \draw[fill=black] (\x, \y) circle (0.2mm);
        }
        \draw (\x, -0.1) to (\x, 0.1) node [below=1ex] {$\x$};
        \draw (-0.1, \x) to (0.1, \x) node [left=1ex]  {$\x$};
    }
\end{tikzpicture}
            \caption{The Lattice $\mathbb{N}^{2}$}
            \label{fig:Lattice_Cart_Prod_of_N_with_N}
        \end{figure}
        This can then be consider a subset of the Euclidean plane
        $\mathbb{R}^{2}$. That is, at every ordered pair of integers $(m,n)$,
        we place a point in the Euclidean plane whose $x$ coordinate is $m$ and
        whose $y$ coordinate is $n$. We can also be more abstract and general in
        our examples. Consider the following sets:
        \par
        \begin{subequations}
            \begin{minipage}[b]{0.49\textwidth}
                \centering
                \begin{equation}
                    A=\{\,\textrm{Point, Line 1, Line 2}\,\}
                \end{equation}
            \end{minipage}
            \hfill
            \begin{minipage}[b]{0.49\textwidth}
                \centering
                \begin{equation}
                    B=\{\,\textrm{Point, Line}\,\}
                \end{equation}
            \end{minipage}
        \end{subequations}
        \par\vspace{2.5ex}
        We can visually represent the Cartesian product $A\times{B}$ by
        drawing $A$ in green and $B$ in red, as shown in
        Fig.~\ref{fig:Cartesian_Product_Example}. The Cartesian Product
        $A\times{B}$ is the set formed by connecting all of the points
        from $A$ and $B$ in the plane. This is shown in blue.
        \begin{figure}[H]
            \centering
            %--------------------------------Dependencies----------------------------------%
%   tikz                                                                       %
%       arrows.meta                                                            %
%-------------------------------Main Document----------------------------------%
\begin{tikzpicture}[%
    >=Latex,
    line width=0.2mm,
    line cap=round
]

    % Draw green to indicate the set A.
    \begin{scope}[green]

        % Draw some points.
        \draw[fill=green] (1, 0) circle (0.3mm);
        \draw[fill=green] (2, 0) circle (0.3mm);
        \draw[fill=green] (5, 0) circle (0.3mm);
        \draw[fill=green] (6, 0) circle (0.3mm);
        \draw[fill=green] (7, 0) circle (0.3mm);

        % Draw lines.
        \draw (2, 0) to (5, 0);
        \draw (6, 0) to (7, 0);
    \end{scope}

    % Draw red to denote the set B.
    \begin{scope}[red]

        % Draw in some points.
        \draw[fill=red] (0, 1) circle (0.3mm);
        \draw[fill=red] (0, 2) circle (0.3mm);
        \draw[fill=red] (0, 5) circle (0.3mm);

        % Draw a line.
        \draw (0, 2) to (0, 5);
    \end{scope}

    % Use blue to mark AxB (Cartesian product).
    \begin{scope}[blue]

        % Fill in points.
        \draw[fill=blue] (1, 1) circle (0.3mm);
        \draw[fill=blue] (1, 2) circle (0.3mm);
        \draw[fill=blue] (1, 5) circle (0.3mm);
        \draw[fill=blue] (2, 1) circle (0.3mm);
        \draw[fill=blue] (5, 1) circle (0.3mm);
        \draw[fill=blue] (6, 1) circle (0.3mm);
        \draw[fill=blue] (7, 1) circle (0.3mm);
        \draw[fill=blue] (2, 2) circle (0.3mm);
        \draw[fill=blue] (2, 5) circle (0.3mm);
        \draw[fill=blue] (5, 2) circle (0.3mm);
        \draw[fill=blue] (5, 5) circle (0.3mm);
        \draw[fill=blue] (6, 2) circle (0.3mm);
        \draw[fill=blue] (7, 2) circle (0.3mm);
        \draw[fill=blue] (6, 5) circle (0.3mm);
        \draw[fill=blue] (7, 5) circle (0.3mm);

        % Draw lines.
        \draw (1, 2) to (1, 5);
        \draw (2, 1) to (5, 1);
        \draw (6, 1) to (7, 1);

        % Fill in rectangles.
        \draw[fill=blue, opacity=0.4] (2, 2) to (5, 2) to (5, 5)
                                             to (2, 5) to cycle;
        \draw[fill=blue, opacity=0.4] (6, 2) to (7, 2) to (7, 5)
                                             to (6, 5) to cycle;
        \draw (2, 2) to (5, 2) to (5, 5) to (2, 5) to cycle;
        \draw (6, 2) to (7, 2) to (7, 5) to (6, 5) to cycle;
    \end{scope}
\end{tikzpicture}
            \caption[Cartesian Product of Two Sets]
                {The Cartesian Product of Two Sets. $A$ is
                 in \textcolor{green}{Green},
                 $B$ is in \textcolor{red}{red}, and
                 $A\times{B}$ is in \textcolor{blue}{blue}.}
            \label{fig:Cartesian_Product_Example}
        \end{figure}
        Cartesian products are not \textit{associative}. That is, given three
        sets $A$, $B$, and $C$, there is no clear way to take the Cartesian
        product of these since:
        \begin{equation}
            A\times(B\times{C})\ne(A\times{B})\times{C}
        \end{equation}
        To see this, note that the elements of $A\times(B\times{C})$ are
        ordered pairs of the form $\big(a,\,(b,\,c)\big)$, whereas elements of
        $(A\times{B})\times{C}$ are of the form $\big((a,\,b),\,c\big)$. When
        we write $A\times{B}\times{C}$ we really want ordered \textit{triples}
        of the form $(a,\,b,\,c)$.
        Much the way ordered pairs have been
        defined, we can modify Kuratowski's approach and define ordered
        triples and ordered $n$ tuples. Rather than doing this we will use the
        language of functions to define higher order Cartesian products.
        \begin{fdefinition}{Functions}{Function}
            A \gls{function} from a \gls{set} $A$ to a set $B$ is a \gls{subset}
            $f\subseteq{A}\times{B}$, denoted $f:A\rightarrow{B}$, such that
            for all $x\in{A}$ there is a unique $y\in{B}$ such that
            $(x,y)\in{f}$. $A$ is called the domain of $f$
            and $B$ is called the codomain.
        \end{fdefinition}
        We're used to hearing that a function is a rule that assigns to an
        input value $x$ some output value $f(x)$. It may seem hard to justify,
        then, why we've defined a function as a subset of the Cartesian
        product. But note the requirement that, for each $x\in{A}$, there is a
        \textit{unique} $y\in{B}$ such that $(x,y)\in{f}$. We call this unique
        element the \textit{image} of $x$ under the function $f$ and write
        $y=f(x)$. The condition that there is a unique such value $y$ to each
        $x$ is called the \textit{vertical line test} when graphing functions
        of the form $f:\mathbb{R}\rightarrow\mathbb{R}$
        (Fig.~\ref{fig:Function_R_to_R_Subset_Cart_Prod}). Simply, given such
        a function, if one draws a vertical line in the plane, then it must
        intersect the graph of $f$ once and only once. This provides a
        quick means of discerning functions from non-functions.
        \begin{lexample}{The Square Function}{Square_Function}
            If we can come up with some rule that assigns to every element
            $a\in{A}$ a unique element of $B$, then we can use this rule to
            define a function $f:A\rightarrow{B}$. Such a rule often comes
            in the form of a \textit{formula}. We write the unique element that
            $a$ corresponds to as $f(b)$. For example, let $A=\mathbb{R}$ and
            let $B=\mathbb{R}$. We can define a function by the squaring
            formula:
            \begin{equation}
                f(x)=x\cdot{x}=x^{2}
            \end{equation}
            Once we know that $x^{2}$ gives a unique number
            (which will require some notion of arithmetic), we can define
            the function $f:\mathbb{R}\rightarrow\mathbb{R}$ by:
            \begin{equation}
                f=\{\,(x,\,x^{2})\in\mathbb{R}^{2}\;|\;x\in\mathbb{R}\,\}
            \end{equation}
            Usually we'll define functions by their formula's, rather than
            expressing them explicitly as subsets of the Cartesian product.
        \end{lexample}
        In the field of mathematical analysis we are often concerned with
        functions involving real numbers. For the sake of intuition, let
        us consider functions of the form $f:\mathbb{R}\rightarrow\mathbb{R}$.
        Any curve that we draw left-to-right, without picking up the pencil,
        will be a valid function.
        (See Fig.~\ref{fig:Function_R_to_R_Subset_Cart_Prod}).
        \begin{figure}[H]
            \centering
            %--------------------------------Dependencies----------------------------------%
%   xcolor                                                                     %
%   amssymb                                                                    %
%   tikz                                                                       %
%       arrows.meta                                                            %
%       patterns                                                               %
%-------------------------------Main Document----------------------------------%
\begin{tikzpicture}[%
    >=Latex,
    line width=0.2mm,
    line cap=round,
    scale=1.2
]
    % Coorindates for the curve.
    \coordinate (P1) at (-3.85, -2.00);
    \coordinate (P2) at (-2.00, -3.00);
    \coordinate (P3) at ( 0.00,  0.00);
    \coordinate (P4) at ( 2.00,  3.00);
    \coordinate (P5) at ( 3.85,  3.80);

    \draw[%
        pattern=north west lines,
        pattern color=Green!80!Black,
        opacity=0.5,
        draw=white
    ]   (-3.9, -3.9) rectangle (3.9, 3.9);

    \begin{scope}[thick, font=\Large]
        \draw[<->] (-4.2, 0) to (4.2, 0) node [above] {$\mathbb{R}$};
        \draw[<->] (0, -4.2) to (0, 4.2) node [right] {$\mathbb{R}$};
    \end{scope}

    \draw[draw=blue] (P1) to [out=-30, in=150]  (P2)
                          to [out=-30, in=210]  (P3)
                          to [out=30,  in=180]  (P4)
                          to [out=0,   in=200]  (P5);
    \draw[fill=white, draw=white] 
        (1.3, 2) rectangle node {$\textcolor{blue}{f}$} (1.6, 1.5);
\end{tikzpicture}
            \caption[Example of a Function $f:\mathbb{R}\rightarrow\mathbb{R}$]
                    {Example of a function $f:\mathbb{R}\rightarrow\mathbb{R}$.
                     The Cartesian product $\mathbb{R}\times\mathbb{R}$ is
                     shown in \textcolor{green!80!black}{green}, and the
                     function $f\subseteq\mathbb{R}\times\mathbb{R}$ is shown
                     in \textcolor{blue}{blue}.}
            \label{fig:Function_R_to_R_Subset_Cart_Prod}
        \end{figure}
        Let $g\subseteq\mathbb{R}\times\mathbb{R}$ be defined as follows:
        \begin{equation}
            g=\big\{\,(x,\,y)\in\mathbb{R}^{2}\;|\;y^{2}=x\,\big\}
        \end{equation}
        It is tempting to label $g$ by writing $g(x)=\sqrt{x}$, but $g$ is
        not a function for it fails two of the requirements of a function.
        Firstly, for any $x>0$, there are two values $y_{1}$ and $y_{2}$
        whose square is equal to $x$. Indeed, if $y_{1}$ is one such value,
        then setting $y_{2}=\minus{y}_{1}$ will result in a second
        distinct value. Thus $g$ does not have the uniqueness property
        required for functions. Moreover, if $x<0$, then there is no such
        value $y\in\mathbb{R}$ such that $(x,y)\in{g}$, and thus $g$ also
        lacks the existence property. In terms of the vertical line test,
        there are points $x$ such that the vertical line through
        $(x,\,0)$ intersects $g$ twice, and there are points such that the
        vertical line does not intersect at all. The graph of $g$ is shown
        in Fig.~\ref{fig:SQRT_Not_a_Function}.
        \begin{figure}[H]
            \centering
            %--------------------------------Dependencies----------------------------------%
%   xcolor                                                                     %
%   amssymb                                                                    %
%   tikz                                                                       %
%       arrows.meta                                                            %
%-------------------------------Main Document----------------------------------%
\begin{tikzpicture}[%
    >=Latex,
    line width=0.2mm,
    line cap=round,
    scale=1.2
]
    % Coorindates for the curve.
    \coordinate (P1) at (-3.85, -2.00);
    \coordinate (P2) at (-2.00, -3.00);
    \coordinate (P3) at ( 0.00,  0.00);
    \coordinate (P4) at ( 2.00,  3.00);
    \coordinate (P5) at ( 3.85,  3.80);

    % Draw a green mesh indicating the Cartesian plane.
    \foreach\x in {-40, -39, ..., 39}{
        \draw[draw=green, line width=0.1mm] (\x/10, -4) to (-4, \x/10);
        \draw[draw=green, line width=0.1mm] (4, \x/10)  to (\x/10, 4);
    }
    \draw[draw=green, line width=0.1mm] (4, 4)  to (4, 4);

    \begin{scope}[thick, font=\Large]
        \draw[<->] (-4.3,  0.0) to (4.3, 0.0) node [above] {$\mathbb{R}$};
        \draw[<->] ( 0.0, -4.3) to (0.0, 4.3) node [right] {$\mathbb{R}$};
    \end{scope}

    \draw[draw=red] (3.880000, -1.969772) to (3.840000, -1.959592)
                                          to (3.800000, -1.949359)
                                          to (3.760000, -1.939072)
                                          to (3.720000, -1.928730)
                                          to (3.680000, -1.918333)
                                          to (3.640000, -1.907878)
                                          to (3.600000, -1.897367)
                                          to (3.560000, -1.886796)
                                          to (3.520000, -1.876166)
                                          to (3.480000, -1.865476)
                                          to (3.440000, -1.854724)
                                          to (3.400000, -1.843909)
                                          to (3.360000, -1.833030)
                                          to (3.320000, -1.822087)
                                          to (3.280000, -1.811077)
                                          to (3.240000, -1.800000)
                                          to (3.200000, -1.788854)
                                          to (3.160000, -1.777639)
                                          to (3.120000, -1.766352)
                                          to (3.080000, -1.754993)
                                          to (3.040000, -1.743560)
                                          to (3.000000, -1.732051)
                                          to (2.960000, -1.720465)
                                          to (2.920000, -1.708801)
                                          to (2.880000, -1.697056)
                                          to (2.840000, -1.685230)
                                          to (2.800000, -1.673320)
                                          to (2.760000, -1.661325)
                                          to (2.720000, -1.649242)
                                          to (2.680000, -1.637071)
                                          to (2.640000, -1.624808)
                                          to (2.600000, -1.612452)
                                          to (2.560000, -1.600000)
                                          to (2.520000, -1.587451)
                                          to (2.480000, -1.574802)
                                          to (2.440000, -1.562050)
                                          to (2.400000, -1.549193)
                                          to (2.360000, -1.536229)
                                          to (2.320000, -1.523155)
                                          to (2.280000, -1.509967)
                                          to (2.240000, -1.496663)
                                          to (2.200000, -1.483240)
                                          to (2.160000, -1.469694)
                                          to (2.120000, -1.456022)
                                          to (2.080000, -1.442221)
                                          to (2.040000, -1.428286)
                                          to (2.000000, -1.414214)
                                          to (1.960000, -1.400000)
                                          to (1.920000, -1.385641)
                                          to (1.880000, -1.371131)
                                          to (1.840000, -1.356466)
                                          to (1.800000, -1.341641)
                                          to (1.760000, -1.326650)
                                          to (1.720000, -1.311488)
                                          to (1.680000, -1.296148)
                                          to (1.640000, -1.280625)
                                          to (1.600000, -1.264911)
                                          to (1.560000, -1.249000)
                                          to (1.520000, -1.232883)
                                          to (1.480000, -1.216553)
                                          to (1.440000, -1.200000)
                                          to (1.400000, -1.183216)
                                          to (1.360000, -1.166190)
                                          to (1.320000, -1.148913)
                                          to (1.280000, -1.131371)
                                          to (1.240000, -1.113553)
                                          to (1.200000, -1.095445)
                                          to (1.160000, -1.077033)
                                          to (1.120000, -1.058301)
                                          to (1.080000, -1.039230)
                                          to (1.040000, -1.019804)
                                          to (1.000000, -1.000000)
                                          to (0.960000, -0.979796)
                                          to (0.920000, -0.959166)
                                          to (0.880000, -0.938083)
                                          to (0.840000, -0.916515)
                                          to (0.800000, -0.894427)
                                          to (0.760000, -0.871780)
                                          to (0.720000, -0.848528)
                                          to (0.680000, -0.824621)
                                          to (0.640000, -0.800000)
                                          to (0.600000, -0.774597)
                                          to (0.560000, -0.748331)
                                          to (0.520000, -0.721110)
                                          to (0.480000, -0.692820)
                                          to (0.440000, -0.663325)
                                          to (0.400000, -0.632456)
                                          to (0.360000, -0.600000)
                                          to (0.320000, -0.565685)
                                          to (0.280000, -0.529150)
                                          to (0.240000, -0.489898)
                                          to (0.200000, -0.447214)
                                          to (0.160000, -0.400000)
                                          to (0.120000, -0.346410)
                                          to (0.080000, -0.282843)
                                          to (0.040000, -0.200000)
                                          to (0.000000, 0.000000) 
                                          to (0.040000, 0.200000)
                                          to (0.080000, 0.282843)
                                          to (0.120000, 0.346410)
                                          to (0.160000, 0.400000)
                                          to (0.200000, 0.447214)
                                          to (0.240000, 0.489898)
                                          to (0.280000, 0.529150)
                                          to (0.320000, 0.565685)
                                          to (0.360000, 0.600000)
                                          to (0.400000, 0.632456)
                                          to (0.440000, 0.663325)
                                          to (0.480000, 0.692820)
                                          to (0.520000, 0.721110)
                                          to (0.560000, 0.748331)
                                          to (0.600000, 0.774597)
                                          to (0.640000, 0.800000)
                                          to (0.680000, 0.824621)
                                          to (0.720000, 0.848528)
                                          to (0.760000, 0.871780)
                                          to (0.800000, 0.894427)
                                          to (0.840000, 0.916515)
                                          to (0.880000, 0.938083)
                                          to (0.920000, 0.959166)
                                          to (0.960000, 0.979796)
                                          to (1.000000, 1.000000)
                                          to (1.040000, 1.019804)
                                          to (1.080000, 1.039230)
                                          to (1.120000, 1.058301)
                                          to (1.160000, 1.077033)
                                          to (1.200000, 1.095445)
                                          to (1.240000, 1.113553)
                                          to (1.280000, 1.131371)
                                          to (1.320000, 1.148913)
                                          to (1.360000, 1.166190)
                                          to (1.400000, 1.183216)
                                          to (1.440000, 1.200000)
                                          to (1.480000, 1.216553)
                                          to (1.520000, 1.232883)
                                          to (1.560000, 1.249000)
                                          to (1.600000, 1.264911)
                                          to (1.640000, 1.280625)
                                          to (1.680000, 1.296148)
                                          to (1.720000, 1.311488)
                                          to (1.760000, 1.326650)
                                          to (1.800000, 1.341641)
                                          to (1.840000, 1.356466)
                                          to (1.880000, 1.371131)
                                          to (1.920000, 1.385641)
                                          to (1.960000, 1.400000)
                                          to (2.000000, 1.414214)
                                          to (2.040000, 1.428286)
                                          to (2.080000, 1.442221)
                                          to (2.120000, 1.456022)
                                          to (2.160000, 1.469694)
                                          to (2.200000, 1.483240)
                                          to (2.240000, 1.496663)
                                          to (2.280000, 1.509967)
                                          to (2.320000, 1.523155)
                                          to (2.360000, 1.536229)
                                          to (2.400000, 1.549193)
                                          to (2.440000, 1.562050)
                                          to (2.480000, 1.574802)
                                          to (2.520000, 1.587451)
                                          to (2.560000, 1.600000)
                                          to (2.600000, 1.612452)
                                          to (2.640000, 1.624808)
                                          to (2.680000, 1.637071)
                                          to (2.720000, 1.649242)
                                          to (2.760000, 1.661325)
                                          to (2.800000, 1.673320)
                                          to (2.840000, 1.685230)
                                          to (2.880000, 1.697056)
                                          to (2.920000, 1.708801)
                                          to (2.960000, 1.720465)
                                          to (3.000000, 1.732051)
                                          to (3.040000, 1.743560)
                                          to (3.080000, 1.754993)
                                          to (3.120000, 1.766352)
                                          to (3.160000, 1.777639)
                                          to (3.200000, 1.788854)
                                          to (3.240000, 1.800000)
                                          to (3.280000, 1.811077)
                                          to (3.320000, 1.822087)
                                          to (3.360000, 1.833030)
                                          to (3.400000, 1.843909)
                                          to (3.440000, 1.854724)
                                          to (3.480000, 1.865476)
                                          to (3.520000, 1.876166)
                                          to (3.560000, 1.886796)
                                          to (3.600000, 1.897367)
                                          to (3.640000, 1.907878)
                                          to (3.680000, 1.918333)
                                          to (3.720000, 1.928730)
                                          to (3.760000, 1.939072)
                                          to (3.800000, 1.949359)
                                          to (3.840000, 1.959592)
                                          to (3.880000, 1.969772);
    \draw[fill=white, draw=white] 
        (1.3, 2.0) rectangle node {$\textcolor{red}{g}$} (1.6, 1.5);
\end{tikzpicture}
            \caption[Example of a Non-Function]
                {$g\subseteq\mathbb{R}\times\mathbb{R}$ is not a function
                 since it fails the vertical line test.}
            \label{fig:SQRT_Not_a_Function}
        \end{figure}
        We need not only consider functions of the form
        $f:\mathbb{R}\rightarrow\mathbb{R}$, nor functions
        $f:\mathcal{U}\rightarrow\mathcal{V}$, where $\mathcal{U}$ and
        $\mathcal{V}$ are subsets of $\mathbb{R}$, and we can allow for
        arbitrary abstract functions. Let $A$ and $B$ be defined as follows:
        \par
        \begin{subequations}
            \begin{minipage}[b]{0.49\textwidth}
                \centering
                \begin{equation}
                    A=\{\,1,\,2,\,3,\,4\,\}
                \end{equation}
            \end{minipage}
            \hfill
            \begin{minipage}[b]{0.49\textwidth}
                \centering
                \begin{equation}
                    B=\{\,a,\,b,\,c\,\}
                \end{equation}
            \end{minipage}
        \end{subequations}
        \par\vspace{2.5ex}
        Similar to the vertical line test, we can devise a visual to
        discerning functions from non-functions for abstract sets.
        We represent the elements of $A$ and $B$ as points in some blob
        in the plane, and then draw arrows between the points
        $x\in{A}$ and $y\in{b}$ indicating that $(x,\,y)\in{f}$.
        This allows us to determine if a given $f\subseteq{A}\times{B}$ is a
        functions ore not. Every point in $A$ must be mapped to a unique point
        in $B$. That is, every point in $A$ must have one and only one arrow
        connecting it to some point in $B$. Examples of valid functions
        are shown in Fig.~\ref{fig:Abstract_Functions}, and non-functions
        are shown in Fig.~\ref{fig:Abstract_Non_Functions}.
        \begin{figure}[H]
            \centering
            \begin{subfigure}[b]{0.49\textwidth}
                \centering
                \resizebox{\textwidth}{!}{%
                    \input{tikz/Function_Example_002.tex}
                }
                \subcaption{A Valid Function.}
            \end{subfigure}
            \begin{subfigure}[b]{0.49\textwidth}
                \centering
                \resizebox{\textwidth}{!}{%
                    \input{tikz/Function_Example_003.tex}
                }
                \subcaption{Another Valid Function.}
            \end{subfigure}
            \caption{Visual for Abstract Functions}
            \label{fig:Abstract_Functions}
        \end{figure}
        \begin{figure}[H]
            \centering
            \begin{subfigure}[b]{0.49\textwidth}
                \centering
                \resizebox{\textwidth}{!}{%
                    %--------------------------------Dependencies----------------------------------%
%   tikz                                                                       %
%       arrows.meta                                                            %
%-------------------------------Main Document----------------------------------%
\begin{tikzpicture}[%
    >=latex,
    line width=0.2mm,
    line cap=round,
    scale=1.2
]
    % Coorindates.
    \coordinate (a) at ( 1.5,  0.75);
    \coordinate (b) at ( 1.5, -0.00);
    \coordinate (c) at ( 1.5, -0.75);
    \coordinate (1) at (-1.5,  1.20);
    \coordinate (2) at (-1.5,  0.40);
    \coordinate (3) at (-1.5, -0.40);
    \coordinate (4) at (-1.5, -1.20);
    \coordinate (A) at (-1.5,  2.50);
    \coordinate (B) at ( 1.5,  2.50);

    % Ellipses representing the sets A and B.
    \draw[thick] (-1.5, 0.0) ellipse (1 and 2);
    \draw[thick] ( 1.5, 0.0) ellipse (1 and 2);

    % Draw circles for the various points.
    \draw[fill=black] (a) circle (0.4mm);
    \draw[fill=black] (b) circle (0.4mm);
    \draw[fill=black] (c) circle (0.4mm);
    \draw[fill=black] (1) circle (0.4mm);
    \draw[fill=black] (2) circle (0.4mm);
    \draw[fill=black] (3) circle (0.4mm);
    \draw[fill=black] (4) circle (0.4mm);

    % Draw paths indicating mappings.
    \begin{scope}[->]
        \draw[shorten >=0.8mm] (1) to (a);
        \draw[shorten >=0.8mm] (2) to (b);
        \draw[shorten >=0.8mm] (3) to (c);
    \end{scope}

    % Labels.
    \node at (A)         {$A$};
    \node at (B)         {$B$};
    \node at (a) [right] {$a$};
    \node at (b) [right] {$b$};
    \node at (c) [right] {$c$};
    \node at (1) [left]  {$1$};
    \node at (2) [left]  {$2$};
    \node at (3) [left]  {$3$};
    \node at (4) [left]  {$4$};
\end{tikzpicture}
                }
                \subcaption{Fails Existence.}
            \end{subfigure}
            \begin{subfigure}[b]{0.49\textwidth}
                \centering
                \resizebox{\textwidth}{!}{%
                    \input{tikz/Non_Function_Example_003.tex}
                }
                \subcaption{Fails Uniqueness.}
            \end{subfigure}
            \caption{Non-Functions}
            \label{fig:Abstract_Non_Functions}
        \end{figure}
        It is possible to count the total number of functions from $A$ to $B$.
        Since every element of $A$ needs to be mapped to some element of $B$,
        and since there are 4 elements in $A$ and 3 elements in $B$, the total
        number of functions $f:A\rightarrow{B}$ is $4^{3}=64$. On the other
        hand, the total number of subsets of $A\times{B}$ is $2^{12}=4096$
        (we will justify this when we discuss the \textit{cardinality} of
        sets). Thus, if we were to randomly pick a subset of $A\times{B}$, the
        odds are that it is almost certainly \textit{not} a function
        (1.5625\%). Thus, functions are very special subsets.
        There is a frequent need to discuss the \textit{set of all functions}
        from a given set $A$ into another set $B$. To ensure we don't create
        a function version of Russell's paradox, we prove such a set exists.
        \begin{theorem}
            If $A$ and $B$ are sets, then there exists a set $\mathcal{F}$ such
            that, for all $f$ it is true that $f\in\mathcal{F}$ if and only if
            $f$ is a function from $A$ to $B$, $f:A\rightarrow{B}$.
        \end{theorem}
        \begin{proof}
            For if $A$ and $B$ are sets, then by
            Thm.~\ref{thm:Existence_of_the_Cartesian_Product} the set
            $A\times{B}$ exists. But by the axiom of the power set
            (Ax.~\ref{ax:Axiom_of_the_Power_Set}) the power set of $A\times{B}$,
            $\mathcal{P}(A\times{B})$, exists. Let $P$ be the proposition
            \textit{True if} $f$ \textit{is a function from} $A$ \textit{to}
            $B$, \textit{false otherwise}. Then by axiom schema of specification
            (Ax.~\ref{ax:Axiom_Schema_of_Specification}), there is a set
            $\mathcal{F}$ such that:
            \begin{equation}
                \mathcal{F}=\big\{\,f\in\mathcal{P}(A\times{B})\;|
                    \;P(f)\,\big\}
            \end{equation}
            But then for all $f$, $f\in\mathcal{F}$ if and only if
            $f\in\mathcal{F}$ and $P(f)$ is true. But if $P(f)$ is true then
            $f$ is a function from $A$ to $B$, and thus by the definition of a
            function (Def.~\ref{def:Function}) $f\subseteq{A}\times{B}$. But
            then by the definition of the power set (Def.~\ref{def:Power_Set})
            we have that $f\in\mathcal{P}(A\times{B})$. Thus $P(f)$ implies
            $f\in\mathcal{F}$. Therefore $f\in\mathcal{F}$ if and only if
            $P(f)$. That is, $f\in\mathcal{F}$ if and only if $f$ is a function
            from $A$ to $B$.
        \end{proof}
        There is non-standard notation when discussing the set of all functions
        from a given set $A$ to a set $B$:
        \begin{fnotation}{Set of All Functions}{Set_of_All_Functions}
            If $A$ and $B$ are sets, the set of all functions from $A$ to $B$,
            $f:A\rightarrow{B}$, is denoted as either $\mathcal{F}(A,B)$ or
            $B^{A}$.
        \end{fnotation}
        The notation $B^{A}$ is common in many areas such as topology and
        algebra, especially when $A=B$. The \textit{topological space} $I^{I}$,
        which is the set of all functions from the \textit{closed unit inverval}
        to itself, is often used to construct examples and counterexamples.
        In analysis the notation $\mathcal{F}(A,B)$ seems to be more common,
        in particular $\mathcal{C}(A,B)$ is often used to denote the set of all
        \textit{continuous} functions from $A$ to $B$, provided the word
        continuous has meaning. Since the notation is not universal nor standard
        across the various disciplines, an attempt will be made to specify what
        $B^{A}$ or $\mathcal{F}(A,B)$ means before using it in a theorem or
        counterexample.
        \begin{fdefinition}{Image of a Point}{Image_of_Point}
            The image of an element $x$ in a set $A$ under a function
            $f:A\rightarrow{B}$ is the unique value $y\in{B}$ such that
            $(x,y)\in{f}$. We write $y=f(x)$.
        \end{fdefinition}
        This allows us to define functions by simply specifying what the
        image of each $x\in{A}$ is. Restating our previous claim, if we can
        define some formula such that for each $x\in{A}$ there is a unique
        $f(x)\in{B}$ such that the formula takes $x$ to $f(x)$, then we can
        define $f$ as the set of all such ordered pairs $(x,f(x))$, and this
        will be a function.
        \begin{fnotation}{Image Notation}{Image_Notation}
            If $A$ and $B$ are sets, if $f:A\rightarrow{B}$ is a function,
            if $x\in{A}$ and if $y=f(x)\in{B}$, then we denote this by
            writing $x\overset{f}{\longmapsto}{y}$ or just $x\mapsto{y}$.
        \end{fnotation}
        Throughout we will almost exclusively use the notation $y=f(x)$ rather
        than $x\mapsto{y}$. The reasons are purely aesthetic and both notations
        are common in mathematics. In a similar manner, we can define the image
        of an entire subset.
        \begin{theorem}
            If $A$ and $B$ are sets, if $f:A\rightarrow{B}$ is a function,
            and if $\mathcal{U}\subseteq{A}$, then there is a set
            $\mathcal{V}\subseteq{B}$ such that, for all $y$ it is true that
            $y\in\mathcal{V}$ if and only if $y\in{B}$ and such that there is
            an $x\in\mathcal{U}$ such that $y=f(x)$.
        \end{theorem}
        \begin{proof}
            For let $P$ be the proposition \textit{True if there exists}
            $x\in\mathcal{U}$ \textit{such that} $y=f(x)$,
            \textit{false otherwise}. By the axiom schema of specification
            (Ax.~\ref{ax:Axiom_Schema_of_Specification}) there is a set
            $\mathcal{V}$ such that, for all $y$ it is true that
            $y\in\mathcal{V}$ if and only if $y\in{B}$ and $P(y)$ is true. That
            is, $y\in\mathcal{V}$ if and only if $y\in{B}$ and if there is an
            $x\in\mathcal{U}$ such that $y=f(x)$.
        \end{proof}
        \begin{fdefinition}{Image of a Subset}{Image_of_Subset}
            The image of a subset $\mathcal{U}$ of a set $A$ under a function
            $f:A\rightarrow{B}$ is the set:
            \begin{equation*}
                f\big(\mathcal{U}\big)=
                    \{\,y\in{B}\;|\;\textrm{There exists }x\in\mathcal{U}
                                    \textrm{ such that }y=f(x)\,\}
            \end{equation*}
            That is, the set of all points in $B$ that are the
            image of points in $\mathcal{U}$.
        \end{fdefinition}
        This definition of the image of a subset was given in such a manner
        so that it only relies on the axiom schema of specification to
        justify it's existence. We could also use the notation:
        \begin{equation}
            f\big(\mathcal{U})=\{\,f(x)\in{B}\;|\;x\in\mathcal{U}\,\}
        \end{equation}
        Writing the definition of the image of a subset in such a way is
        justified by the \textit{axiom schema of replacement}, but we've not yet
        included this axiom in our system. This axiom deals with
        \textit{class functions} and will be dealt with later.
        \begin{example}
            If $f:\mathbb{R}\rightarrow\mathbb{R}$ is the function $f(x)=x^{2}$,
            then $f(\mathbb{R})=[0,\infty)$, where $[0,\infty)$ is defined by:
            \begin{equation}
                [0,\infty)=\{\,x\in\mathbb{R}\;|\;x\geq{0}\}
            \end{equation}
            To see this, not that every positive real number $y$ gets mapped to
            by at least one real number (notably, the positive square root
            $\sqrt{y}$). Zero is mapped to as well since $0^{2}=0$. However,
            none of the negative numbers are the image of any element of
            $\mathbb{R}$ since the square of a real number is always
            non-negative.
        \end{example}
        We can visualize functions and images by using blobs in the plane. Given
        some sub-blob of a set $A$, the image of this will be another sub-blob
        of $B$. Note that if $f:A\rightarrow{B}$ is a function, it does
        \textbf{not} need to be true that $f(A)=B$. These are special functions
        that are called \textit{surjective}\index{Function!Surjective} and are
        discussed in Chapt.~\ref{chapt:Function_Theory}. Such a drawing of the
        general case is shown in Fig.~\ref{fig:Image_of_Point_and_Subset}.
        \begin{figure}[H]
            \centering
            \captionsetup{type=figure}
            \begin{tikzpicture}[>=Latex]
    \coordinate (U1) at (-5.0, -2.0);
    \coordinate (U2) at (-3.5, -2.0);
    \coordinate (U3) at (-0.5, -0.5);
    \coordinate (U4) at (-2.0,  2.0);
    \coordinate (U5) at (-3.3,  1.6);
    \coordinate (U6) at (-4.0,  2.0);
    \coordinate (U7) at (-5.0,  0.0);

    \coordinate (V1) at (5.0,  2.0);
    \coordinate (V2) at (4.0,  2.0);
    \coordinate (V3) at (2.0,  0.0);
    \coordinate (V4) at (2.0, -2.0);
    \coordinate (V5) at (4.0, -1.0);
    \coordinate (V6) at (5.0, -1.0);

    \coordinate (S1) at (-4.0,  0.0);
    \coordinate (S2) at (-3.0, -1.0);
    \coordinate (S3) at (-2.5,  0.0);
    \coordinate (S4) at (-3.0,  0.5);

    \coordinate (T1) at (3.0,  0.0);
    \coordinate (T2) at (3.5, -0.8);
    \coordinate (T3) at (4.5,  0.0);
    \coordinate (T4) at (3.5,  0.8);

    \coordinate (x)  at (-2.8, -0.3);
    \coordinate (fx) at (3.4,  -0.4);

    \draw[fill=blue,opacity=0.5,draw=black,thick]
        (U1)    to[out=0,  in=-150] (U2)
                to[out=30, in=-90]  (U3)
                to[out=90, in=-60]  (U4)
                to[out=120, in=-30] (U5)
                to[out=150,in=10]   (U6)
                to[out=-170,in=90]  (U7)
                to[out=-90,in=180]  cycle;

    \draw[fill=red!80!white,opacity=0.5,draw=black,thick]
        (V1)    to[out=180, in=0]       (V2)
                to[out=180, in=60]      (V3)
                to[out=-120,in=120]     (V4)
                to[out=-60, in=180]     (V5)
                to[out=0,   in=-120]    (V6)
                to[out=60,  in=0]       cycle;

    \draw[fill=blue!80!white] (S1)  to[out=-150,in=180] (S2)
                                    to[out=0,   in=-90] (S3)
                                    to[out=90,  in=-60] (S4)
                                    to[out=120, in=30]  cycle;

    \draw[fill=red!80!white] (T1)   to[out=-150,in=180] (T2)
                                    to[out=0,   in=-90] (T3)
                                    to[out=90,  in=0]   (T4)
                                    to[out=180, in=30]  cycle;

    \draw[fill=black] (x)  circle (0.3mm);
    \draw[fill=black] (fx) circle (0.3mm);
    \node at (-2.5, 1.0) {\Large{$A$}};
    \node at ( 4.6, 1.3) {\Large{$B$}};
    \node at (-3.3, 0.0) {\large{$\mathcal{U}$}};
    \node at ( 3.7, 0.2) {\large{$f(\mathcal{U})$}};
    \node at (x)  [below] {$x$};
    \node at (fx) [right] {$f(x)$};
    \draw[->,shorten >= 1.5mm,shorten <= 1.5mm]
        (x) to[out=-30,in=-150] node[below]{$f$} (fx);
\end{tikzpicture}
            \caption{Image of a Subset and of a Point under a Function}
            \label{fig:Image_of_Point_and_Subset}
        \end{figure}
        If we consider a function $f:A\rightarrow{B}$ and the image of the
        entire set $A$ we obtain the \textit{range} of $f$. That is, the range
        is the set $f(A)\subseteq{B}$. In a similar manner to the forward image
        of a function, we can define the pre-image. First, we prove such things
        exist.
        \begin{theorem}
            \label{thm:Existence_of_Pre_Image}%
            If $A$ and $B$ are sets, if $f:A\rightarrow{B}$ is a function from
            $A$ to $B$, and if $\mathcal{V}\subseteq{B}$, then there is a set
            $\mathcal{U}\subseteq{A}$ such that for all $x$ it is true that
            $x\in\mathcal{U}$ if and only if $x\in{A}$ and $f(x)\in\mathcal{V}$.
        \end{theorem}
        \begin{proof}
            For let $P$ be the proposition \textit{True if} $x\in{A}$
            \textit{and} $f(x)\in\mathcal{V}$, \textit{false otherwise}. Then
            by the axiom schema of specification
            (Ax.~\ref{ax:Axiom_Schema_of_Specification}) there is a set
            $\mathcal{U}$ such that:
            \begin{equation*}
                \mathcal{U}=\big\{\,x\in{A}\;|\;P(x)\,\big\}
            \end{equation*}
            But $P(x)$ implies $x\in{A}$ and thus $x\in\mathcal{U}$ if and only
            if $x\in{A}$ and $f(x)\in\mathcal{V}$.
        \end{proof}
        \begin{fdefinition}{Pre-Image of a Subset}{Pre_Image_of_Subset}
            The \gls{pre-image} of a \gls{subset} $\mathcal{V}\subseteq{B}$
            under a \gls{function} $f:A\rightarrow{B}$ is the set:
            \begin{equation}
                f^{\minus{1}}(\mathcal{V})
                =\big\{\,x\in{X}\;|\;f(x)\in\mathcal{V}\,\big\}
            \end{equation}
        \end{fdefinition}
        The pre-image of a set behaves a lot differently than the image, and
        this will be explored in detail when functions are discussed. The cause
        of the discrepancy is the requirement that elements of $A$ map uniquely
        to elements of $B$, but a single element in $B$ can be the image of
        many different points in $A$. This gives rise to the notion of a
        \textit{fiber} of a point in $B$.
        \begin{theorem}
            If $A$ and $B$ are sets, if $f:A\rightarrow{B}$ is a function, and
            if $b\in{A}$, then there is a set $\mathcal{U}\subseteq{A}$ such
            for all $x\in{A}$ it is true that $x\in\mathcal{U}$ if and only if
            $f(x)=b$.
        \end{theorem}
        \begin{proof}
            For by Thm.~\ref{thm:Existence_of_Set_Containing_Set}, $\{b\}$ is
            a set and $\{b\}\subseteq{B}$ (Def.~\ref{def:Subsets}). But if
            $\{b\}$ is a subset of $B$, then there is a set
            $\mathcal{U}\subseteq{A}$ such that for all $x\in{A}$ it is true
            that $x\in\mathcal{U}$ if and only if $f(x)\in\{b\}$
            (Thm.~\ref{thm:Existence_of_Pre_Image}). But $f(x)\in\{b\}$ if and
            only if $f(x)=b$ (Thm.~\ref{thm:Existence_of_Set_Containing_Set}).
            Thus, for all $x\in{A}$, $x\in\mathcal{U}$ if and only if $f(x)=b$.
        \end{proof}
        \begin{fdefinition}{Fiber of an Element}{Fiber_of_Element}
            The fiber of an element $b$ in a set $B$ under a function
            $f:A\rightarrow{B}$ from a set $A$ to a set $B$ is the pre-image
            of the set $\{b\}$. That is:
            \begin{equation*}
                f^{\minus{1}}(\{b\})
                =\big\{\,a\in{A}\:|\;f(a)=b\,\big\}
            \end{equation*}
        \end{fdefinition}
        Since the set $b\in{B}$ we have that $\{b\}\subseteq{B}$
        (Def.~\ref{def:Subsets}) and thus there is no need to prove again that
        the fiber of an element $b\in{B}$ is a valid set, since
        Thm.~\ref{thm:Existence_of_Pre_Image} applies.
    \subsection{The Axiom of Choice and Diaconescu's Theorem}
        The next two axioms to be introduced are the most controversial of those
        listed in ZFC: The \textit{axiom of infinity} and the
        \textit{axiom of choice}. While the axiom of infinity only has a
        small number of critics, the axiom of choice is far more contentious.
        Choice is equivalent to many other statements that come across in
        almost all forms of mathematics (analysis, algebra, topology, etc.).
        Many of which are theorems we would \textit{want} to be true, and so
        accepting the axiom of choice allows us to prove them. In particular,
        the axioms presented thus far can be combined with the axiom of choice
        to prove the \textit{Law of the Excluded Middle}, a result known as
        Diaconescu's theorem, and this is our current goal.
        \begin{faxiom}{Axiom of Choice}{Axiom_of_Choice}
            If $\mathcal{O}$ is a non-empty set such that for all
            $\mathcal{U}\in\mathcal{O}$ it is true that $\mathcal{U}$ is
            non-empty, and if $\mathcal{F}$ is the union over $\mathcal{O}$:
            \index{Axiom!of Choice}
            \begin{equation*}
                \mathcal{F}=\bigcup_{\mathcal{U}\in\mathcal{O}}\mathcal{U}
            \end{equation*}
            then there is a function $f:\mathcal{O}\rightarrow\mathcal{F}$ such
            that, for all $x\in\mathcal{O}$, $f(x)\in{x}$
        \end{faxiom}
        Such a function is called a choice function. The axiom can be made
        obviously true if we word it one way, and obviously false if we word it
        another. To convince one that is it true will require talking about
        products. The Cartesian product has been defined using ordered pairs
        as defined by Kuratowski and allows us to order two elements. Given two
        sets $A$ and $B$ we can define an equivalent notion of the Cartesian
        product using the set of all functions from $\mathbb{Z}_{2}=\{0,1\}$
        into $A\cup{B}$ with a particular property:
        \begin{equation}
            A\times{B}=
            \big\{\,f:\mathbb{Z}_{2}\rightarrow{A}\cup{B}\;|\;
                f(0)\in{A}\textrm{ and }f(1)\in{B}\,\big\}
        \end{equation}
        To see why this is equivalent to the actual Cartesian product, note that
        the Cartesian product $A\times{B}$ is the set of all ordered pairs whose
        first entry lies in $A$ and whose second entry lies in $B$. Given
        $a\in{A}$ and $b\in{B}$, let $f:\mathbb{Z}_{2}\rightarrow{A}\cup{B}$ be
        the function such that $f(0)=a$ and $f(1)=b$. Then we can identify the
        ordered pair $(a,b)$ with $f$. Indeed, $(a,b)=(f(0),f(1))$ making our
        identification very explicit. We can now generalize to a collection of
        $n$ different sets and define the ordered $n$ tuple over a collection of
        $n$ sets to be the set of all functions from $\mathbb{Z}_{n}$ into the
        union over this collection with a similar property:
        \begin{equation}
            \prod_{k\in\mathbb{Z}_{n}}A_{k}
            =\big\{f:\mathbb{Z}_{n}\rightarrow\bigcup_{k\in\mathbb{Z}_{n}}^{n}
                A_{k}\;|\;f(k)\in{A}_{k}
                \textrm{for all }k\in\mathbb{Z}_{n}\big\}
        \end{equation}
        Given the function $f$ that maps $k$ to $a_{k}\in{A}_{k}$, we can
        identify $f$ with the ordered $n$ tuple:
        \begin{equation}
            f=(a_{0},\,a_{1},\,\dots,\,a_{k},\,\dots,\,a_{n-1})
            =(f(0),\,f(1),\,\dots,\,f(k),\,\dots,\,f(n-1))
        \end{equation}
        And thus we have a more general way of defining products. Note that
        swapping the order of the product is equivalent to changing functions,
        and thus we have that two $n$ tuples are equal if and only if all of
        their entries are equal. What's nice about our function definition is
        that it allows one to define products over \textit{arbitrary}
        collections. This is crucial for topology and analysis as we often wish
        to speak of \textit{infinite dimensional} spaces that are constructed
        using these abstract products. Given a set $I$, often called the
        \textit{index set}, such that for all $\mathcal{U}\in{I}$ it is true
        $\mathcal{U}$ is a set, we can form the product over $I$ by defining
        this to be the collection of all functions from $I$ into the union over
        $I$.
        \begin{equation}
            \prod_{i\in{I}}A_{i}
            =\big\{\,f:I\rightarrow\bigcup_{i\in{I}}A_{i}\;|\;
                f(i)\in{A}_{i}\textrm{ for all }i\in{I}\,\big\}
        \end{equation}
        The axiom of choice is thus equivalent to the statement
        \textit{The infinite product of non-empty sets is non-empty}. These
        functions that identify $k$ with the $k^{th}$ set are precisely
        choice functions. Phrasing it like this we see that the axiom of choice
        is somewhat obvious. The infinite product of non-empty sets is most
        likely enormous! Claiming it's non-empty thus seems trivial. It's thus
        unfortunate that this claim can not be proven with the other axioms
        we've developed. As stated before, the axiom of choice is equivalent to
        many other statements such as \textit{Zorn's lemma, Tychonoff's theorem,
        the well ordering theorem, every vector space has a basis, every set has
        a group structure}, and countless more. Many of these theorems have many
        applications to algebra, analysis, and topology, and since we would like
        to use them to prove other things we are forced to accept the axiom of
        choice. Many theorems in real analysis hide the use of the axiom of
        choice by constructing sequences \textit{by induction}. An attempt will
        be made to be very clear whenever the axiom of choice is used in a
        proof.
        \par\hfill\par
        We conclude this section by presenting Diaconescu's theorem.
        \begin{ftheorem}{Diaconescu's Theorem}{Diaconescus_Theorem}
            If $P$ is a proposition on sets and if $x$ is a set, then either
            $P(x)$ is true or $P(x)$ is false. That is, $P\lor\neg{P}$ is true.
        \end{ftheorem}
        \begin{bproof}
            Let $0=\emptyset$. By
            Thm.~\ref{thm:Existence_of_Set_Containing_Set}, we have that the set
            $\{0\}$ exists. Let $1=\{0\}$. Then since $0\in{1}$, $0\ne{1}$
            (Thm.~\ref{thm:Containment_NEqual_Underlying_Set}). Since $0$ and
            $1$ are sets, by Thm.~\ref{thm:Existence_of_Set_Built_from_Two_Sets}
            we have that the set $\{0,1\}$ exists. Let $Q$ be the proposition
            \textit{true if P(x) or } $x=0$, \textit{false otherwise}. By the
            axiom schema of specification
            (Ax.~\ref{ax:Axiom_Schema_of_Specification}) there exists a set
            $\mathcal{U}$ such that:
            \begin{equation}
                \mathcal{U}=\{\,x\in\{\,0,\,1\,\}\;|\;Q(x)\,\}
            \end{equation}
            Similarly, let $Q$ be the proposition \textit{true if P(x) or}
            $x=1$, \textit{false otherwise}. By the axiom schema of
            specification we have that the following set exists:
            \begin{equation}
                \mathcal{V}=\{\,x\in\{\,0,\,1\,\}\;|\;R(x)\,\}
            \end{equation}
            By Thm.~\ref{thm:Existence_of_Set_Built_from_Two_Sets}, we have that
            the set $\{\,\mathcal{U},\,\mathcal{V}\,\}$ exists. By the axiom of
            choice (Ax.~\ref{ax:Axiom_of_Choice}), there exists a function
            $f:\{\mathcal{U},\mathcal{V}\}\rightarrow%
             \bigcup\{\mathcal{U},\mathcal{V}\}$ such that
            $f(\mathcal{U})\in\mathcal{U}$ and $f(\mathcal{V})\in\mathcal{V}$.
            But then, by the definition of $\mathcal{U}$, either
            $f(\mathcal{U})=0$ or $P(x)$ is true. Similarly, either
            $f(\mathcal{V})=1$ or $P(x)$ is true. But since $0\ne{1}$, either
            $f(\mathcal{U})\ne{f}(\mathcal{V})$ or $P(x)$ is true. Again by the
            axiom of extensionality, and by the definition of $\mathcal{U}$ and
            $\mathcal{V}$, if $P(x)$ is true then $\mathcal{U}=\mathcal{V}$. But
            then $f(\mathcal{U})=f(\mathcal{V})$. But then, by the
            contrapositive, $\neg{P}(x)$ implies that
            $f(\mathcal{U})\ne{f}(\mathcal{V})$. But by extensionality, either
            $f(\mathcal{U})=f(\mathcal{V})$ or
            $f(\mathcal{U})\ne{f}(\mathcal{V})$, and thus either $P(x)$ or
            $\neg{P}(x)$. That is, $P\lor\neg{P}$ is true.
        \end{bproof}
        We can now prove things via \textit{proof by contradiction}. While we
        have made great efforts to justify every step of a proof thus far, we
        will often omit mention of Diaconescu's theorem as a justification for
        the law of the excluded middle and simply use it freely. We may rest
        easy knowing that we've proved the equivalence within the framework of
        ZFC. There are two more axioms remaining, that of infinity and
        replacement. The axiom of infinity is best introduced when we construct
        the natural numbers, and from there build the real numbers, and thus we
        shall delay its development briefly. The axiom of replacement needs a
        notion of class and thus will also be postponed.
        \section{The Structure of Sets}
    We've developed two \textit{operations} on sets thus far, that of union and
    intersections. Several of the properties of these two operations give rise
    to a structure known as a \textit{Boolean algebra}. Many theorems about sets
    can thus be proven in an algebraic setting by using the structure of a
    Boolean algebra, and thus it is our current goal to prove the basics about
    unions and operations, and then to develop the notion of a Boolean algebra.
    \subsection{Basic Theorems}
        With the law of the excluded middle in our toolbelt, we can now rapidly
        prove many basic and familiar results.
        \begin{theorem}
            \label{thm:Emptyset_Is_Subset}%
            If $A$ is a set, then $\emptyset\subseteq{A}$.
        \end{theorem}
        \begin{proof}
            For suppose not. Then there is an $x\in\emptyset$ such that
            $x\notin{A}$, a contradiction as for all $x$, it is true that
            $x\notin\emptyset$ (Def.~\ref{ax:Axiom_of_the_Empty_Set}).
            Therefore $\emptyset\subseteq{A}$.
        \end{proof}
        \begin{theorem}
            \label{thm:Subset_is_Transitive}%
            If $A$, $B$, and $C$ are sets, if $A\subseteq{B}$, and if
            $B\subseteq{C}$, then $A\subseteq{C}$.
        \end{theorem}
        \begin{proof}
            For suppose not. Then there is an $x\in{A}$ such that $x\notin{C}$.
            But $A$ is a subset of $B$ and thus $x\in{B}$
            (Def.~\ref{def:Subsets}). But $B$ is a subset of $C$ and therefore
            $x\in{C}$ (Def.~\ref{def:Subsets}). But $x\notin{C}$, a
            contradiction. Therefore, $A\subseteq{C}$.
        \end{proof}
        \begin{theorem}
            \label{thm:Set_Is_Subset_Of_Self}%
            If $A$ is a set, then $A\subseteq{A}$.
        \end{theorem}
        \begin{proof}
            Suppose not. Then there is an $x\in{A}$ such that $x\notin{A}$, a
            contradiction. Therefore $A\subseteq{A}$.
        \end{proof}
        We can now rigorously restate our claim that the empty set is unique.
        \begin{theorem}
            If $\emptyset'$ is a set with no elements,
            then $\emptyset=\emptyset'$.
        \end{theorem}
        \begin{proof}
            For suppose not. But $\emptyset'$ is a set, and thus
            $\emptyset\subseteq\emptyset'$ (Thm.~\ref{thm:Emptyset_Is_Subset}).
            Therefore $\emptyset'\nsubseteq\emptyset$. But then there is an $x$
            such that $x\in\emptyset'$ and $x\notin\emptyset$. But $\emptyset'$
            contains no elements, a contradiction. Thus
            $\emptyset'\subseteq\emptyset$. Therefore,
            $\emptyset=\emptyset'$ (Def.~\ref{def:Equal_Sets}).
        \end{proof}
        We can use this to define \textit{the} empty set.
        \begin{fdefinition}{The Empty Set}{Empty_Set}
            The empty set is the unique set $\emptyset$ such that for all $x$
            it is true that $x\notin\emptyset$.
        \end{fdefinition}
        \begin{theorem}
            \label{thm:Subsets_of_Equal_Sets}%
            If $A$, $B$, and $C$ are sets, if $A=B$, and if $C\subseteq{A}$,
            then $C\subseteq{B}$.
        \end{theorem}
        \begin{proof}
            For if $A=B$, then $A\subseteq{B}$ (Def.~\ref{def:Equal_Sets}). But
            if $C\subseteq{A}$ and $A\subseteq{B}$, then $C\subseteq{B}$
            (Thm.~\ref{thm:Subset_is_Transitive}). Therefore, etc.
        \end{proof}
        \begin{theorem}
            \label{thm:Equality_Reflexive}%
            If $A$ is a set, then $A=A$.
        \end{theorem}
        \begin{proof}
            For if $A$ is a set then $A\subseteq{A}$
            (Thm.~\ref{thm:Set_Is_Subset_Of_Self}). Thus,
            $A=A$ (Def.~\ref{def:Equal_Sets}).
        \end{proof}
        \begin{theorem}
            \label{thm:Equality_Symmetric}%
            If $A$ and $B$ are sets and if $A=B$, then $B=A$.
        \end{theorem}
        \begin{proof}
            For suppose not. If $B\ne{A}$, then either $B\nsubseteq{A}$ or
            $A\nsubseteq{B}$. But $A=B$, and thus $A\subseteq{B}$  and
            $B\subseteq{A}$ (Def.~\ref{def:Equal_Sets}),
            a contradiction. Therefore, etc.
        \end{proof}
        \begin{theorem}
            \label{thm:Equality_Transitive}%
            If $A$, $B$, and $C$ are sets, if $A=B$, and if $B=C$, then $A=C$.
        \end{theorem}
        \begin{proof}
            For if $B=C$, then $C\subseteq{B}$ (Def.~\ref{def:Equal_Sets}). But
            if $A=B$, then $B=A$ (Thm.~\ref{thm:Equality_Symmetric}). But if
            $B=A$ and and $C\subseteq{B}$, then $C\subseteq{A}$
            (Thm.~\ref{thm:Subsets_of_Equal_Sets}). And if $A=B$, then
            $A\subseteq{B}$ (Def.~\ref{def:Equal_Sets}). But if $B=C$ and
            $A\subseteq{B}$, then $A\subseteq{C}$
            (Thm.~\ref{thm:Subsets_of_Equal_Sets}). But it was proved that
            $C\subseteq{A}$, and thus $A=C$ (Def.~\ref{def:Equal_Sets}).
        \end{proof}
        These three properties,
        Thms.~\ref{thm:Equality_Reflexive}-%
        \ref{thm:Equality_Transitive}, are the key ingredients to define
        \textit{equivalence relations}. Equivalence relations are used to model
        the notion of equality in more abstract settings and are fundamental in
        the study of algebra and topology.
        \begin{theorem}
            \label{thm:Prop_Subset_Not_Equal}%
            If $A$ and $B$ are sets, and if $A\subsetneq{B}$, then there is an
            $x\in{B}$ such that $x\notin{A}$.
        \end{theorem}
        \begin{proof}
            For suppose not. Then for all $x\in{B}$ it is true that $x\in{A}$.
            But then $B\subseteq{A}$ (Def.~\ref{def:Subsets}).
            But $A\subseteq{B}$ and thus $A=B$ (Def.~\ref{def:Equal_Sets}).
            But $A\subsetneq{B}$ and therefore $A\ne{B}$, a contradiction.
            Therefore, etc.
        \end{proof}
        Theorem \ref{thm:Prop_Subset_Not_Equal} can be used as an equivalent
        definition of a proper subset. That is, a proper subset is a subset that
        is missing at least one element.
    \subsection{Operations on Sets}
        Similar to the arithmetic of real numbers, there are standard operations
        that can be performed on sets to obtain new sets. The four most common
        operations are union, intersection, set difference, and symmetric
        difference. As stated before, we wish to build the structure of sets in
        an algebraic manner. To do this requires the notion that the operations
        of intersection and unions are \textit{commutative},
        \textit{distributive}, have \textit{identities}, and have
        \textit{complements}.
        \begin{ltheorem}{Commutative Law of Unions}{Commutative_Law_of_Unions}
            If $A$ and $B$ are sets, then $A\cup{B}=B\cup{A}$.
        \end{ltheorem}
        \begin{proof}
            For if $x\in{A}\cup{B}$, then either $x\in{A}$
            or $x\in{B}$, or both
            (Def.~\ref{def:Union_of_Two_Sets}). But then either
            $x\in{B}$ or $x\in{A}$, or both, and therefore
            $x\in{B}\cup{A}$ (Def.~\ref{def:Union_of_Two_Sets}).
            But then for all $x\in{A}\cup{B}$ it is true that
            $x\in{B}\cup{A}$, and therefore
            $A\cup{B}\subseteq{B}\cup{A}$
            (Def.~\ref{def:Subsets}). Similarly,
            $B\cup{A}\subseteq{A}\cup{B}$, and thus
            $A\cup{B}=B\cup{A}$ (Def.~\ref{def:Equal_Sets}).
            Therefore, etc.
        \end{proof}
        When taking the union of two sets, we obtain a \textit{larger} set, in
        a sense. Again relying on the analogy of arithmetic, given two
        non-negative integers $a$ and $b$, it is true that $a\leq{a}+b$.
        Equality is obtained if and only if either $a$ or $b$ is equal to zero.
        The empty set thus acts as the \textit{zero} of unions. Also, given
        three non-negative integers $a$, $b$, and $c$, if $b\leq{c}$, then
        $a+b\leq{a}+c$. A similar result will hold for sets and unions.
        \begin{theorem}
            \label{thm:Union_is_Bigger}%
            If $A$ and $B$ are sets, then $A\subseteq{A}\cup{B}$.
        \end{theorem}
        \begin{proof}
            For suppose not. Then there is an $x\in{A}$ such
            that $x\notin{A}\cup{B}$. But if $x\in{A}$, then
            $x\in{A}$ or $x\in{B}$ and thus $x\in{A}\cup{B}$
            (Def.~\ref{def:Union_of_Two_Sets}), a
            contradiction. Therefore, etc.
        \end{proof}
        \begin{theorem}
            \label{thm:Union_With_Lesser_Set}%
            If $A$, $B$, and $C$ are sets, and if
            $B\subseteq{C}$, then
            $A\cup{B}\subseteq{A}\cup{C}$.
        \end{theorem}
        \begin{proof}
            For if $x\in{A}\cup{B}$, then either $x\in{A}$,
            or $x\in{B}$, or both
            (Def.~\ref{def:Union_of_Two_Sets}). But $B$ is a
            subset of $C$, and therefore if $x\in{B}$, then
            $x\in{C}$ (Def.~\ref{def:Subsets}).
            Thus, if $x\in{A}$ or $x\in{B}$, then
            $x\in{A}$ or $x\in{C}$, and therefore
            $x\in{A}\cup{C}$ (Def.~\ref{def:Union_of_Two_Sets}).
            Thus, $A\cup{B}\subseteq{A}\cup{C}$
            (Def.~\ref{def:Subsets}). Therefore, etc.
        \end{proof}
        \begin{theorem}
            If $A$, $B$, $C$, and $D$ are sets, if
            $A\subseteq{C}$, and if $B\subseteq{D}$, then
            $A\cup{B}\subseteq{C}\cup{D}$.
        \end{theorem}
        \begin{proof}
            For if $B\subseteq{D}$, then
            $A\cup{B}\subseteq{A}\cup{D}$
            (Thm.~\ref{thm:Union_With_Lesser_Set}).
            But $A\cup{D}=D\cup{A}$
            (Thm.~\ref{thm:Commutative_Law_of_Unions}).
            But if $A\subseteq{C}$, then
            $D\cup{A}\subseteq{D}\cup{C}$
            (Thm.~\ref{thm:Union_With_Lesser_Set}). But
            $D\cup{C}=C\cup{D}$
            (Thm.~\ref{thm:Commutative_Law_of_Unions}).
            And if $A\cup{B}\subseteq{A}\cup{D}$ and
            $A\cup{D}\subseteq{C}\cup{D}$, then
            $A\cup{B}\subseteq{C}\cup{D}$
            (Thm.~\ref{thm:Subset_is_Transitive}).
            Therefore, etc.
        \end{proof}
        Taking the union of subsets is redundant, as we
        simply obtain the larger set. This starts to break
        down the analogy between sets and arithmetic, since
        there is only one \textit{zero}. That is, there is
        only one number $b$ such that $a+b=a$, and that is
        $b=0$. While any subset acts as a \textit{zero} of a
        given set, the empty set has the property that it
        acts as a zero for \textit{every} set. It is the only
        set with this property, and thus the analogy with
        arithmetic is restored.
        \begin{theorem}
            \label{thm:Union_With_Subset}%
            If $A$ and $B$ are sets, and if
            $A\subseteq{B}$, then $A\cup{B}=B$.
        \end{theorem}
        \begin{proof}
            For if $A$ and $B$ are sets, then
            $B\subseteq{A}\cup{B}$
            (Thm.~\ref{thm:Union_is_Bigger}).
            But if $A\subseteq{B}$, then for all $x\in{A}$,
            it is true that $x\in{B}$
            (Def.~\ref{def:Subsets}). Thus if $x\in{A}$ or if
            $x\in{B}$, then $x\in{B}$. But then, for all
            $x\in{A}\cup{B}$, it is true that $x\in{B}$, and
            therefore $A\cup{B}\subseteq{B}$
            (Def.~\ref{def:Subsets}). Thus,
            $A\cup{B}=B$ (Def.~\ref{def:Equal_Sets}).
            Therefore, etc.
        \end{proof}
        \begin{theorem}
            \label{thm:Union_with_Emptyset}%
            If $A$ is a set, then $A=\emptyset\cup{A}$.
        \end{theorem}
        \begin{proof}
            For $\emptyset\subseteq{A}$
            (Thm.~\ref{thm:Emptyset_Is_Subset}) and
            therefore $\emptyset\cup{A}=A$
            (Thm.~\ref{thm:Union_With_Subset}).
        \end{proof}
        \begin{theorem}
            \label{thm:Empty_Set_Is_Zero_for_Unions}%
            If $A$ is a set such that, for any set $B$, it is
            true that $A\cup{B}=B$, then $A$ is the
            empty set.
        \end{theorem}
        \begin{proof}
            For suppose not. If $A\ne\emptyset$, then there
            is an $x\in{A}$ (Def.~\ref{ax:Axiom_of_the_Empty_Set}).
            But then $B=\{A\}$ is a set
            (Def.~\ref{def:Sets}). But then $x\in{A}\cup{B}$
            (Def.~\ref{def:Union_of_Two_Sets}). But $x\notin{B}$,
            and thus $A\cup{B}\ne{B}$
            (Def.~\ref{def:Equal_Sets}), a contradiction
            since $A$ is such that for any set $B$, it is
            true that $A\cup{B}=B$. Therefore, etc.
        \end{proof}
        Thm.~\ref{thm:Empty_Set_Is_Zero_for_Unions} proves
        the assertion that the empty set is the zero of set
        union. The converse of
        Thm.~\ref{thm:Union_With_Subset} can be proved as
        well.
        \begin{theorem}
            \label{thm:Conv_Union_Is_Bigger}%
            If $A$ and $B$ are sets, and if
            $A\cup{B}\subseteq{A}$, then $A\cup{B}=A$.
        \end{theorem}
        \begin{proof}
            For $A\subseteq{A}\cup{B}$
            (Thm.~\ref{thm:Union_is_Bigger}). But by
            hypothesis, $A\cup{B}\subseteq{A}$. But then
            $A=A\cup{B}$ (Def.~\ref{def:Equal_Sets}).
            Therefore, etc.
        \end{proof}
        \begin{theorem}
            \label{thm:Union_is_Equal}%
            If $A$ and $B$ are sets, and if
            $A\cup{B}\subseteq{A}$, then $B\subseteq{A}$.
        \end{theorem}
        \begin{proof}
            For if $A\cup{B}\subseteq{A}$, then
            $A\cup{B}=A$
            (Thm.~\ref{thm:Conv_Union_Is_Bigger}). And also,
            $B\subseteq{A}\cup{B}$
            (Thm.~\ref{thm:Union_is_Bigger}). But if
            $A\cup{B}=A$ and $B\subseteq{A}\cup{B}$, then
            $B\subseteq{A}$
            (Thm.~\ref{thm:Subsets_of_Equal_Sets}).
            Therefore, etc.
        \end{proof}
        We'll wrap up unions by showing that the operation
        is associative. Once again relying on the analogy
        of arithmetic, given three real numbers $a$, $b$,
        and $c$, it is true that $a+(b+c)=(a+b)+c$. This
        is called the associative law of addition. Combining
        this law with the commutative law shows that the
        order in which three real numbers are added is
        irrelevant. Applying induction, we see that given
        any finite collection of real numbers, the order in
        which we add them is again irrelevant. The same holds
        true for the union of sets.
        \begin{theorem}
            \label{thm:Redundant_Union}%
            If $A$, $B$, and $C$ are sets and if $A\subseteq{B}$, then
            $A\cup(B\cup{C})=B\cup{C}$
        \end{theorem}
        \begin{proof}
            For $B\cup{C}\subseteq{A}\cup(B\cup{C})$
            (Thm.~\ref{thm:Union_is_Bigger}). But
            $A\cup(B\cup{C})=(A\cup{B})\cup{C}$
            (Thm.~\ref{thm:Associative_Law_of_Unions}).
            And since $A$ is a subset of $B$, $A\cup{B}=B$
            (Thm.~\ref{thm:Union_With_Subset}), and thus
            $(A\cup{B})\cup{C}=B\cup{C}$. Thus, $B\cup{C}=A\cup(B\cup{C})$
            (Thm.~\ref{thm:Equality_Transitive}). Therefore, etc.
        \end{proof}
        \begin{figure}[H]
            \centering
            \captionsetup{type=figure}
            \centering
            %--------------------------------Dependencies----------------------------------%
%   tikz                                                                       %
%-------------------------------Main Document----------------------------------%
\begin{tikzpicture}[line width=0.2mm]

    % Coordinates for the two circles.
    \coordinate (A) at ( 0.5, -0.3);
    \coordinate (B) at ( 0.0,  0.0);
    \coordinate (I) at (-2.3,  1.5);

    % Draw a rectangle for the universe set.
    \draw (-3,-2.3) rectangle (2.5,2.3);

    \draw[fill=cyan] (A) circle (0.85);
    \draw (B) circle (2);

    \node at (A) [above=0.1cm] {$A$};
    \node at (B) [above=1.0cm] {$B$};
    \node at (I) {$A\cap{B}$};
\end{tikzpicture}
            \caption{Visual for Thm.~\ref{thm:Intersection_of_Subset}.}
            \label{fig:Union_Intersection_venn_diagram}
        \end{figure}
        \begin{ltheorem}{Commutative Law of Intersections}{Commut_Law_Intersec}
            If $A$ and $B$ are sets, then $A\cap{B}=B\cap{A}$.
        \end{ltheorem}
        \begin{proof}
            For if $x\in{A}\cap{B}$, then $x\in{A}$ and
            $x\in{B}$. But then $x\in{B}$ and $x\in{A}$,
            and therefore $x\in{B}\cap{A}$
            (Def.~\ref{def:Intersection_of_Two_Sets}). But then
            for all $x\in{A}\cap{B}$ it is true that
            $x\in{B}\cap{A}$, and therefore
            $A\cup{B}\subseteq{B}\cup{A}$
            (Def.~\ref{def:Subsets}). Similarly,
            $B\cap{A}\subseteq{A}\cap{B}$, and thus
            $A\cap{B}=B\cap{A}$ (Def.~\ref{def:Equal_Sets}).
            Therefore, etc.
        \end{proof}
        \begin{theorem}
            \label{thm:Intersection_is_Smaller}%
            If $A$ snd $B$ are sets, then
            $A\cap{B}\subseteq{A}$.
        \end{theorem}
        \begin{proof}
            If $x\in{A}\cap{B}$, then $x\in{A}$ and
            $x\in{B}$, and thus $x\in{A}$. Therefore, etc.
        \end{proof}
        \begin{theorem}
            \label{thm:Intersection_with_Lesser_Set}%
            If $A$, $B$, and $C$ are sets, and if
            $B\subseteq{C}$, then
            $A\cap{B}\subseteq{A}\cap{C}$.
        \end{theorem}
        \begin{proof}
            For if $x\in{A}\cap{B}$, then $x\in{A}$ and
            $x\in{B}$ (Def.~\ref{def:Intersection_of_Two_Sets}).
            But $B$ is a subset of $C$, and thus if
            $x\in{B}$, then $x\in{C}$
            (Def.~\ref{def:Subsets}). But then $x\in{A}$ and
            $x\in{C}$, and therefore $x\in{A}\cap{C}$
            (Def.~\ref{def:Intersection_of_Two_Sets}). But
            then $A\cap{B}\subseteq{A}\cap{C}$
            (Def.~\ref{def:Subsets}). Therefore, etc.
        \end{proof}
        \begin{theorem}
            \label{thm:Intersection_is_Equal}%
            If $A$ and $B$ are sets, and if
            $A=A\cap{B}$, then $A\subseteq{B}$.
        \end{theorem}
        \begin{proof}
            For suppose not. Then there is an $x\in{A}$ such
            that $x\notin{B}$. But since $A=A\cap{B}$,
            if $x\in{A}$ then $x\in{A}\cap{B}$
            (Def.~\ref{def:Equal_Sets}). But if
            $x\in{A}\cap{B}$, then $x\in{B}$
            (Thm.~\ref{thm:Intersection_is_Smaller}),
            a contradiction. Therefore, etc.
        \end{proof}
        \begin{theorem}
            \label{thm:Intersection_of_Subset}%
            If $A$ and $B$ are sets, and if
            $A\subseteq{B}$, then $A\cap{B}=A$.
        \end{theorem}
        \begin{proof}
            For $A\cap{B}\subseteq{A}$
            (Thm.~\ref{thm:Intersection_is_Smaller}). But
            since $A$ is a subset of $B$, if $x\in{A}$, then
            $x\in{B}$ (Def.~\ref{def:Subsets}). But then
            $x\in{A}\cap{B}$
            (Def.~\ref{def:Intersection_of_Two_Sets}). Therefore,
            $A\subseteq{A}\cap{B}$ (Def~\ref{def:Subsets})
            and thus $A=A\cap{B}$ (Def~\ref{def:Equal_Sets}).
            Therefore, etc.
        \end{proof}
        \begin{theorem}
            \label{thm:Conv_Intersection_is_Smaller}%
            If $A$ and $B$ are sets, and if
            $A\subseteq{A}\cap{B}$, then $A=A\cap{B}$.
        \end{theorem}
        \begin{proof}
            For $A\cap{B}\subseteq{A}$
            (Thm.~\ref{thm:Intersection_is_Smaller}). But
            by hypothesis, $A\subseteq{A}\cap{B}$, and thus
            $A=A\cap{B}$ (Def.~\ref{def:Equal_Sets}).
            Therefore, etc.
        \end{proof}
        \begin{theorem}
            If $A$ is a set, then $\emptyset\cap{A}=\emptyset$.
        \end{theorem}
        \begin{proof}
            For $\emptyset\subseteq{A}$ (Thm.~\ref{thm:Emptyset_Is_Subset}), and
            therefore $\emptyset\cap{A}=\emptyset$
            (Thm.~\ref{thm:Intersection_of_Subset}).
        \end{proof}
        \begin{theorem}
            \label{thm:Redundant_Intersection}%
            If $A$, $B$, and $C$ are sets and if
            $B\subseteq{A}$, then
            $A\cap(B\cap{C})=B\cap{C}$.
        \end{theorem}
        \begin{proof}
            For $A\cup(B\cup{C})\subseteq{B}\cup{C}$
            (Thm.~\ref{thm:Intersection_is_Smaller}). But
            $A\cap(B\cap{C})=(A\cap{B})\cap{C}$
            (Thm.~\ref{thm:Assoc_Law_Intersec}).
            And since $B$ is a subset of $A$,
            $A\cap{B}=A$
            (Thm.~\ref{thm:Intersection_of_Subset}),
            and thus $(A\cap{B})\cap{C}=B\cap{C}$. Thus,
            $B\cap{C}=A\cap(B\cap{C})$
            (Thm.~\ref{thm:Equality_Transitive}).
            Therefore, etc.
        \end{proof}
        \begin{theorem}
            \label{thm:First_Pseudo_Dist_Law_Union}%
            If $A$, $B$, and $C$ are sets, then
            $(B\cap{C})\subseteq(A\cup{B})\cap(A\cup{C})$.
        \end{theorem}
        \begin{proof}
            For $B\subseteq{A}\cup{B}$
            (Thm.~\ref{thm:Union_is_Bigger}). But then
            $B\cap{C}\subseteq(A\cup{B})\cap{C}$
            (Thm.~\ref{thm:Intersection_with_Lesser_Set}).
            But $C\subseteq{A}\cup{C}$
            (Thm.~\ref{thm:Union_is_Bigger}), and thus
            $(A\cup{B})\cap{C}%
            \subseteq(A\cup{B})\cap{A}\cup{C}$
            (Thm.~\ref{thm:Intersection_with_Lesser_Set}).
            But it was just proved that
            $B\cap{C}\subseteq(A\cup{B})\cap{C}$, and
            therefore by transivity,
            $(B\cap{C})\subseteq(A\cup{B})\cap(A\cup{C})$
            (Thm.~\ref{thm:Subset_is_Transitive}).
            Therefore, etc.
        \end{proof}
        \begin{ltheorem}{Distributive Law of Unions}
            {Distributive_Law_Union}
            If $A$, $B$, and $C$ are sets, then
            $A\cup(B\cap{C})=(A\cup{B})\cap(A\cup{C})$.
        \end{ltheorem}
        \begin{proof}
            For $(B\cap{C})\subseteq(A\cup{B})\cap(A\cup{C})$
            (Thm.~\ref{thm:First_Pseudo_Dist_Law_Union}).
            But then:
            \begin{equation}
                A\cup(B\cap{C})\subseteq
                A\cup\Big((A\cup{B})\cap(A\cup{C})\Big)
            \end{equation}
            But $A\cup((A\cup{B})\cap(A\cup{C}))%
                =(A\cup{B})\cap(A\cup{C})$, and therefore:
            \begin{equation}
                A\cup(B\cap{C})\subseteq
                (A\cup{B})\cap(A\cup{C})
            \end{equation}
        \end{proof}
        \begin{ltheorem}{Distributive Law of Intersections}
            {Distributive_Law_Intersections}
            If $A$, $B$, and $C$ are sets, then
            $A\cap(B\cup{C})=(A\cap{B})\cup(A\cap{C})$.
        \end{ltheorem}
        \begin{proof}
            Hi
        \end{proof}
        If $A$ and $B$ are sets, and if
        $C\subseteq{A}\cup{B}$, then
        either $C\subseteq{A}$ or $C\subseteq{B}$, or both.
        It is possible that $C\subseteq{A}\cup{B}$ and yet
        $C$ and $B$ have no elements in common, as long
        as $C\subseteq{A}$. As an example,
        take $A$ and $B$ to be disjoint sets. Then
        $A\subseteq{A}\cup{B}$, yet $A$ and $B$ have no
        elements in common. If $C\subseteq{A}\cap{B}$, then
        it must be true that $C\subseteq{A}$ and
        $C\subseteq{B}$.
        Using arithmetic as an analogy, the empty set
        acts somewhat like a zero element. It is an identity
        element under set unions, and collapses everything
        down to zero under set intersections. Continuing
        with this analogy, we discuss set difference.
        \begin{ldefinition}{Set Difference}{Set_Difference}
            The set difference of a set $A$ with respect to
            a set $B$, denoted $B\setminus{A}$, is the set:
            \begin{equation}
                B\setminus{A}=\{x\in{B}:x\notin{A}\}
            \end{equation}
        \end{ldefinition}
        \begin{ldefinition}{Symmetric Difference}
            {Symmetric_Difference}
            The symmetric difference of $A$ and $B$, denoted
            $A\ominus{B}$, is the set:
            \begin{equation}
                A\ominus{B}
                =(A\cup{B})\setminus(A\cap{B})
            \end{equation}
        \end{ldefinition}
        While set difference appears similar to subtraction,
        the two have their differences. For any two real
        numbers $a$ and $b$, it is always true that
        $b=a-(a-b)$. For sets this is not true. For let $A$
        be the empty set, and let $B$ be non-empty.
        Then $A\setminus(A\setminus{B})=\emptyset$, which
        is not $B$. Set differences can not be easily
        simplified. The notion is not associative, nor is it
        commutative. If there is a larger \textit{universe}
        set, then set difference can be related to
        intersection.
        \begin{theorem}
            \label{thm:Set_Difference_As_Intersection}%
            If $A$, $B$, and $C$ are sets, and if
            $A\subseteq{C}$ and $B\subseteq{C}$, then:
            \begin{equation}
                B\setminus{A}=B\cap(C\setminus{A})
            \end{equation}
        \end{theorem}
        \begin{proof}
            For if $x\in{B}\setminus{A}$, then
            $x\in{B}$ and $x\notin{A}$. But
            $B\subseteq{C}$, and thus if $x\in{B}$, then
            $x\in{C}$. But if $x\notin{A}$, then
            $x\in{C}\setminus{A}$. Therefore
            $B\setminus{A}\subseteq{B}\cap(C\setminus{A})$.
            Similarly,
            $B\cap(C\setminus{A})\subseteq{B}\setminus{A}$,
            and therefore
            $B\setminus{A}={B}\cap(C\setminus{A})$.
        \end{proof}
        Similar to unions and intersections,
        set differences and symmetric differences can be
        visualized by Venn diagrams, as shown in
        Fig.~\ref{fig:Difference_Sym_Venn_Diagram}.
        \begin{figure}[H]
            \centering
            \captionsetup{type=figure}
            \begin{subfigure}[b]{\textwidth}
                \centering
                \documentclass[crop,class=article]{standalone}
%----------------------------Preamble-------------------------------%
\usepackage{tikz}                       % Drawing/graphing tools.
%--------------------------Main Document----------------------------%
\begin{document}
    \begin{tikzpicture}
        \draw (-2.5,-2) rectangle (2.5,2);
        \fill[cyan] (-0.8cm,0) circle (1.5cm);
        \fill[white] (0.8cm,0) circle (1.5cm);
        \draw (-0.8cm,0) circle (1.5cm);
        \draw (0.8cm,0) circle (1.5cm);
        \node at (-1,1.1) {$A$};
        \node at (1,1.1) {$B$};
        \node at (-1,1.75) {$A\setminus{B}$};
    \end{tikzpicture}
\end{document}
                \subcaption{Set Difference}
            \end{subfigure}
            \begin{subfigure}[b]{\textwidth}
                \centering
                %--------------------------------Dependencies----------------------------------%
%   tikz                                                                       %
%-------------------------------Main Document----------------------------------%
\begin{tikzpicture}[line width=0.2mm]

    % Coordinates for the centers of the circles.
    \coordinate (C1) at (-1.3, 0);
    \coordinate (C2) at ( 1.3, 0);

    % Coordinates for the labels.
    \coordinate (A) at (-1.3, 1.2);
    \coordinate (B) at ( 1.3, 1.2);
    \coordinate (S) at ( 0.0, 2.5);

    % Rectangle indicating the universe set.
    \draw (-4, -3) rectangle (4, 3);

    % Fill in the circle with cyan.
    \draw[fill=cyan, draw=none] (C1) circle (2);
    \draw[fill=cyan, draw=none] (C2) circle (2);

    % Fill in the circle with cyan.
    \draw[fill=white, draw=none] (0, -1.51987) arc(-49.46:49.46:2)
                                               arc(130.54:229.46:2);

    % Give outlines to the circles.
    \draw (C1) circle (2);
    \draw (C2) circle (2);

    % Labels.
    \node at (A) {$A$};
    \node at (B) {$B$};
    \node at (S) {$A\ominus{B}$};
\end{tikzpicture}
                \subcaption{Symmetric Difference}
            \end{subfigure}
            \caption[Venn Diagrams for Set Difference
                    and Symmetric Difference]
                    {Venn Diagrams Depicting the Set
                    Difference and Symmetric Difference
                    of the Sets $A$ and $B$.}
            \label{fig:Difference_Sym_Venn_Diagram}
        \end{figure}
        The concept of set difference can then be used to define the
        concept of complement.
        Thm.~\ref{thm:Set_Difference_As_Intersection} can then be
        translated into the notation of complements as follows:
        \begin{theorem}
            If $A$, $B$, and $\Omega$ are sets,
            $A,B\subseteq\Omega$, and if $A^{C}$ is the
            complement of $A$ with respect to $\Omega$, then:
            \begin{equation}
                B\setminus{A}=B\cap{A}^{C}
            \end{equation}
        \end{theorem}
        \begin{proof}
            By the definition of complement,
            $A^{C}=\Omega\setminus{A}$.
            As $A\subseteq\Omega$ and $B\subseteq\Omega$, by
            Thm.~\ref{thm:Set_Difference_As_Intersection},
            $B\setminus{A}=B\cap(\Omega\setminus{A})$,
            and therefore $B\setminus{A}=B\cap{A}^{C}$.
        \end{proof}
        The main result about complements are known as
        DeMorgan's Laws. The laws relate unions and
        intersections by means of complements. The general
        laws hold for arbitrary unions and arbitrary
        intersections, as will be shown later.
        \begin{ftheorem}{DeMorgan's Laws}{MEASURE_DEMORGAN}
            If $A$, $B$, and $\Omega$ are sets, if
            $A\subseteq\Omega$ and $B\subseteq\Omega$, then:
            \begin{subequations}
                \begin{align}
                    \big(A\cap{B}\big)^{C}
                    &=A^{C}\cup{B}^{C}\\
                    \big(A\cup{B}\big)^{C}
                    &=A^{C}\cap{B}^{C}
                \end{align}
            \end{subequations}
        \end{ftheorem}
        With this, we can prove some results about
        set differences.
        \begin{theorem}
            If $A$ and $B$ are sets, then:
            \begin{equation}
                A=\big(A\cap{B}\big)
                    \cup\big(A\setminus{B}\big)
            \end{equation}
        \end{theorem}
        \begin{proof}
            For let $\Omega=A\cup{B}$. Then
            $A\subseteq\Omega$ and $B\subseteq\Omega$,
            and thus:
            \begin{subequations}
                \begin{align}
                    \big(A\cap{B})\cup\big(A\setminus{B}\big)
                    &=\big(A\cap{B}\big)
                        \cup\big(A\cap{B}^{C}\big)\\
                    &=A\cap(B\cup{B}^{C})\\
                    &=A\cap\Omega
                \end{align}
            \end{subequations}
            But by Thm.~\ref{thm:Intersection_is_Smaller},
            $A\cap\Omega=A$. Therefore, etc.
        \end{proof}
        \begin{theorem}
            If $A$, $B$, and $C$ are sets, then:
            \begin{equation}
                A\cap\big(B\setminus{C}\big)
                =\big(A\cap{B}\big)\cap\big(A\setminus{C}\big)
            \end{equation}
        \end{theorem}
        \begin{proof}
            For:
            \begin{subequations}
                \begin{align}
                    A\cap\big(B\setminus{C}\big)
                    &=A\cap\big(B\cap{C}^{C}\big)\\
                    &=\big(A\cap{A}\big)
                        \cap\big(B\cap{C}^{C}\big)\\
                    &=\big(A\cap{B}\big)
                        \cap\big(A\cap{C}^{C}\big)\\
                    &=\big(A\cap{B}\big)
                        \cap\big(A\setminus{C}\big)
                \end{align}
            \end{subequations}
        \end{proof}
        Intersections do distribute over set differences.
        \begin{theorem}
            If $A$, $B$, and $C$ are sets, then:
            \begin{equation}
                A\cap(B\setminus{C})=
                (A\cap{B})\setminus(A\cap{C})
            \end{equation}
        \end{theorem}
        \begin{proof}
            For:
            \begin{subequations}
                \begin{align}
                    \big(A\cap{B}\big)\setminus
                        \big(A\cap{C}\big)
                    &=\big(A\cap{B}\big)
                        \cap\big(A\cap{C}\big)^{C}\\
                    &=\big(A\cap{B}\big)
                        \cap\big(A^{C}\cup{C}^{C}\big)\\
                    &=\big[\big(A\cap{B}\big)\cap{A}^{C}\big]
                        \cup\big[\big({A}\cap{B}\big)
                        \cap{C}^{C}\big]\\
                    &=\big[\big(A\cap{A}^{C}\big)\cap{B}\big]
                        \cup\big[\big(A\cap{B}\big)
                        \cap{C}^{C}\big]\\
                    &=\emptyset\cup\big[\big(A\cap{B}\big)
                        \cap{C}^{C}\big]\\
                    &=\big(A\cap{B}\big)\cap{C}^{C}\\
                    &=A\cap\big(B\cap{C}^{C}\big)\\
                    &=A\cap\big(B\setminus{C}\big)
                \end{align}
            \end{subequations}
            Therefore, etc.
        \end{proof}
        Unions do not, however. For let $A$ be non-empty
        and let $A=B=C$. Then $A\cup(B\setminus{C})=A$, but
        $(A\cup{B})\setminus(A\cup{C})=\emptyset$.
        \begin{theorem}
            If $A$ and $B$ are sets and $A\subset B$,
            then $B\setminus(B\setminus A)=A$.
        \end{theorem}
        \begin{proof}
            For:
            \begin{align}
                Yo
            \end{align}
            $[x\in B\setminus(B\setminus{A})]%
            \Rightarrow[x\in{B}\land{x}\notin%
            \{x\in{B}:x\notin{A}\}]%
            \Rightarrow[x\in{A}\subset{B}]$.
            $[x\in{A}]\Rightarrow[x\notin{B}\setminus{A}]%
            \Rightarrow[x\in{B}\setminus(B\setminus{A})]$.
        \end{proof}
        The previous theorem shows that $(A^C)^{C}=A$.
        % Untrue garbage.
        % If $A$ and $B$ are sets, and if $C\subseteq{A}\cup{B}$, then
        % either $C\subseteq{A}$ or $C\subseteq{B}$, or both. It is
        % possible that $C\subseteq{A}\cup{B}$ and yet $C$ and $B$ have no
        % elements in common, as long as $C\subseteq{A}$. As an example,
        % take $A$ and $B$ to be disjoint sets. Then $A\subseteq{A}\cup{B}$,
        % yet $A$ and $B$ have no elements in common. If
        % $C\subseteq{A}\cap{B}$, then it must be true that
        % $C\subseteq{A}$ and $C\subseteq{B}$.
        % As with the notions of unions and intersections, set differences and
        % symmetric differences can be visualized using Venn diagrams.
        \begin{theorem}
            \label{thm:MEASURE_THEORY_SET_DIFFERENCE_AS_INTERSECTION}
            If $A$, $B$, and $C$ are sets, and if $A\subseteq{C}$
            and $B\subseteq{C}$, then:
            \begin{equation}
                B\setminus{A}=B\cap(C\setminus{A})
            \end{equation}
        \end{theorem}
        \begin{proof}
            For if $x\in{B}\setminus{A}$, then
            $x\in{B}$ and $x\notin{A}$. But
            $B\subseteq{C}$, and thus if $x\in{B}$, then $x\in{C}$.
            But if $x\notin{A}$, then $x\in{C}\setminus{A}$. Therefore
            $B\setminus{A}\subseteq{B}\cap(C\setminus{A})$.
            Similarly, $B\cap(C\setminus{A})\subseteq{B}\setminus{A}$,
            and therefore $B\setminus{A}={B}\cap(C\setminus{A})$.
        \end{proof}
        While set difference appears similar to subtraction that one finds in
        basic arithmetic, the two have their differences. For any two real
        numbers $a$ and $b$, $b=a-(a-b)$. For sets this is not true. For let $A$
        be the empty set, and let $B$ be non-empty. Then
        $A\setminus(A\setminus{B})=\emptyset$, which is not $B$.
        Also, while it may seems convincing that
        $A\setminus(B\setminus{A})=A\setminus{B}$, this is not true. For
        let $A$ be a non-empty set and let $B=A$. Then
        $A\setminus(B\setminus{A})=A$, but $A\setminus{B}=\emptyset$.
        The concept of set difference can then be used to define the
        concept of complement.
        %------------------------------------------------------------------------------%
\section{Relations}
    \label{Section:ZFC:Elementary_Set_Theory:Relations}%
    \begin{fdefinition}{Relation on a Set}{Relation_on_a_Set}
        A \gls{relation} on a \gls{set} $A$ is a \gls{subset} $R$ of the
        \gls{Cartesian product} $A\times{A}$.
    \end{fdefinition}
    We use a special notation for relations on a set.
    \begin{fnotation}{Relation Notation}{Relation_Notation}
        If $A$ is a set, if $R$ is a relation on $A$, and if $(a,b)\in{R}$, we
        write $aRb$.
        \begin{equation*}
            \forall_{x}\forall_{y}\big(aRb\big)\Leftrightarrow
            \big((a,b)\in{R}\big)
        \end{equation*}
    \end{fnotation}
    For a relation $R$ it is not necessary true that $aRb$ implies $bRa$, nor is
    it necessarily true that $aRa$. These are called symmetric and reflexive
    relations, respectively.
    \begin{lexample}{Examples of Relations}{Examples_of_Relations}
        Let $A=\mathbb{R}$ and consider the relation of equality. That is, let
        $R_{=}\subseteq\mathbb{R}^{2}$ be defined by:
        \begin{equation}
            R_{=}=\{\,(x,y)\in\mathbb{R}^{2}\;|\;x=y\,\}
        \end{equation}
        Then $R_{=}$ is a relation on $\mathbb{R}^{2}$. Rather than writing
        $(x,y)\in{R_{=}}$ or $xR_{=}y$ we commonly write $x=y$. Note that this
        relation is defined entirely by the \textit{diagonal} of the Cartesian
        product $\mathbb{R}\times\mathbb{R}$. Another simple relation is that of
        ordering. Let $R_{<}$ be defined as follows:
        \begin{equation}
            R_{<}=\{\,(x,y)\in\mathbb{R}^{2}\;|\;x<y\,\}
        \end{equation}
        This is also a relation since it is a subset of the Cartesian product,
        but it is a slightly more complicated one. There are many
        \textit{off-diagonal} elements of this relation.
    \end{lexample}
    \begin{theorem}
        If $B$ is a set, if $A\subseteq{B}$, and if $R$ is a relation on $B$,
        then there is a relation $R_{A}$ such that $R_{A}$ is a relation on
        $A$ and $R_{A}\subseteq{R}$.
    \end{theorem}
    \begin{proof}
        For let $P$ be the proposition \textit{True if} $(x,y)\in{A}^{2}$,
        \textit{false otherwise}. By the axiom schema of specification
        (Ax.~\ref{ax:Axiom_Schema_of_Specification}) there is a set:
        \begin{equation}
            R_{A}=\big\{\,(x,y)\in{R}\;|\;P\big((x,y)\big)\,\big\}
        \end{equation}
        But then $(x,y)\in{R}_{A}$ if and only if $(x,y)\in{R}$ and
        $(x,y)\in{A}^{2}$.
    \end{proof}
    This set is called the \textit{restriction} of $R$ to the subset $A$.
    \begin{fdefinition}{Restriction of a Relation}{Restriction_of_a_Relation}
        The restriction of a relation $R$ on a set $B$ to a subset $A$ is the
        set $R_{A}$ defined by:
        \begin{equation*}
            R_{A}=\big\{\,(x,y)\in{R}\;|\;(x,y)\in{A}^{2}\,\big\}
        \end{equation*}
    \end{fdefinition}
    There are many basic properties that relations have, and we prove them now.
    \begin{theorem}
        \label{thm:Cartesian_Product_Is_Relation}%
        If $A$ is a set, then $A\times{A}$ is a relation on $A$.
    \end{theorem}
    \begin{proof}
        For if $A$ is a set, then
        $A\times{A}\subseteq{A}\times{A}$. Therefore, etc.
    \end{proof}
    \begin{theorem}
        \label{thm:Empty_Set_Is_Relation}%
        If $A$ is a set, and then $\emptyset$ is a relation
        on $A$.
    \end{theorem}
    \begin{proof}
        For if $A$ is a set, then
        $\emptyset\subseteq{A}\times{A}$. Therefore, etc.
    \end{proof}
    \begin{theorem}
        Set inclusion $\subseteq$ is a relation. Proper set inclusion
        $\subsetneq$ is a relation. These define partial orderings.
    \end{theorem}
    \begin{fdefinition}{Domain of a Relation}{Domain_of_a_Relation}
        The \gls{domain (relation)} of a \gls{relation} $R$ on a \gls{set} $A$
        is the set:
        \begin{equation*}
            \textrm{dom}(R)=\big\{a\in{A}\;|\;\exists{b}\in{A}
                \textrm{ such that }aRb\big\}
        \end{equation*}
    \end{fdefinition}
    \begin{fdefinition}{Range of a Relation}{Range_of_a_Relation}
        The \gls{range (relation)} of a \gls{relation} $R$ on a \gls{set} $A$ is
        the set:
        \begin{equation*}
            \textrm{ran}(R)=\big\{b\in{A}\;|\;\exists{a}\in{A}
                \textrm{ such that }aRb\big\}
        \end{equation*}
    \end{fdefinition}
    \begin{fdefinition}{Field of a Relation}{Field_of_a_Relation}
        The \gls{field (relation)} of a \gls{relation} $R$ on a set $A$ is the
        set:
        \begin{equation*}
            \textrm{field}(R)=\textrm{dom}(R)\cup\textrm{ran}(R)
        \end{equation*}
        Where $\textrm{dom}(R)$ is the \gls{domain (relation)} of $R$ and
        $\textrm{ran}(R)$ is the \gls{range (relation)} of $R$.
    \end{fdefinition}
    These provide the two most basic examples of relations on a
    set. The empty set is the relation that says no two elements
    are related. Indeed, even single elements are unrelated to
    themselves. The second, the entire Cartesian product
    $A\times{A}$, says that everything is related. These are the
    two extreme cases, but provide useful examples and
    counterexamples in various contexts. More useful is that the
    union and intersection of relations is also a relation. We
    prove this now.
    \begin{theorem}
        \label{thm:Intersection_of_Relations_Is_Relation}%
        If $A$ is a set and if $R_{1}$ and $R_{2}$ are relations
        on $A$, then $R_{1}\cap{R}_{2}$ is a relation on $A$.
    \end{theorem}
    \begin{proof}
        For let $R=R_{1}\cap{R}_{2}$ and suppose $R$ is not a
        relation on $A$. Then there is an $x\in{R}$ such that
        $x\notin{A}\times{A}$. But if $x\in{R}$ then
        $x\in{R}_{1}$ and $x\in{R}_{2}$. But for all
        $x\in{R}_{1}$, $x\in{A}\times{A}$, since $R_{1}$ is a
        relation on $A$, a contradiction as
        $x\notin{A}\times{A}$. Therefore, $R$ is a relation on
        $A$.
    \end{proof}
    \begin{theorem}
        \label{thm:Set_Theory_Union_of_Relations_Is_Relation}
        If $A$ is a set and if $R_{1}$ and $R_{2}$ are relations
        on $A$, then $R_{1}\cup{R}_{2}$ is a relation on $A$.
    \end{theorem}
    \begin{proof}
        For let $R=R_{1}\cup{R}_{2}$ and suppose $R$ is not a
        relation on $A$. Then there is an $x\in{R}$ such that
        $x\notin{A}\times{A}$. But if $x\in{R}$ then
        $x\in{R}_{1}$ or $x\in{R}_{2}$. But for all $x\in{R}_{1}$
        and for all $x\in{R}_{2}$,
        $x\in{A}\times{A}$, since $R_{1}$ and $R_{2}$ are
        relations on $A$, a contradiction. Therefore, etc.
    \end{proof}
    \begin{theorem}
        If $A$ is a set and $R$ is a relation on $A$, then there
        is a relation $\mathcal{U}$ on $A$ such that
        $R\cap\mathcal{U}=R$.
    \end{theorem}
    \begin{proof}
        For let $\mathcal{U}={A}\times{A}$. Then by
        Thm.~\ref{thm:Cartesian_Product_Is_Relation}, $A\times{A}$ is
        a relation on $A$. But since $R$ is a relation,
        $R\subseteq{A}\times{A}$. But then
        $R\cap\mathcal{U}=R$. Therefore, etc.
    \end{proof}
    \begin{theorem}
        If $A$ is a set and $R$ is a relation on $A$, then there
        is a relation $\mathcal{U}$ on $A$ such that
        $R\cup\mathcal{U}=R$
    \end{theorem}
    \begin{proof}
        For let $\mathcal{U}=\emptyset$. Then by
        Thm.~\ref{thm:Empty_Set_Is_Relation},
        $\mathcal{U}$ is a relation. But if $R$ is a set, then
        $R\cup\emptyset=R$. Thus, $R\cup\mathcal{U}=R$.
        Therefore, etc.
    \end{proof}
    Since a general relation is simply a subset of $A\times{A}$,
    there's not much structure on them, and thus there's not a lot
    that can be said about them. We can add more constraints to
    certain relations to get the more familiar properties
    we're used to.
    \begin{fdefinition}{Reflexive Relations}{Reflexive_Relations}
        A reflexive relation on a set $A$ is a
        relation $R$ on $A$ such that for all $a\in{A}$
        it is true that $aRa$.
    \end{fdefinition}
    A reflexive relation on $A$ is simply any subset of
    $A\times{A}$ that contains the entire \textit{diagonal}. That,
    all of the pairs $(a,a)$. A reflexive relation can contain more
    than this, however. The only strict requirement is that
    $aRa$ for all $a\in{A}$.
    \begin{theorem}
        If $A$ is a set, and if $R_{1}$ and $R_{2}$ are reflexive
        relations on $A$, then $R_{1}\cap{R}_{2}$ is a reflexive
        relation on $A$.
    \end{theorem}
    \begin{proof}
        For let $R=R_{1}\cap{R}_{2}$. Then by
        Thm.~\ref{thm:Intersection_of_Relations_Is_Relation}, $R$ is a relation.
        Suppose $R$ is not reflexive.
        Then there is an $a\in{A}$ such that $(a,a)\notin{R}$. But
        if $a\in{A}$, then $(a,a)\in{R}_{1}$, since $R_{1}$ is
        reflexive. Similarly, $(a,a)\in{R}_{2}$ since $R_{2}$ is
        reflexive. But if $(a,a)\in{R}_{1}$ and $(a,a)\in{R}_{2}$,
        then $(a,a)\in{R}$ since $R=R_{1}\cap{R}_{2}$, a
        contradiction. Therefore, $R$ is reflexive.
    \end{proof}
    \begin{theorem}
        If $A$ is a set, if $R_{1}$ is a reflexive relation on
        $A$, and if $R_{2}$ is a relation on $A$, then
        $R_{1}\cup{R}_{2}$ is a reflexive relation on $A$.
    \end{theorem}
    \begin{proof}
        For let $R=R_{1}\cup{R}_{2}$. Since $R_{1}$ and $R_{2}$ are
        relations, by
        Thm.~\ref{thm:Set_Theory_Union_of_Relations_Is_Relation},
        $R$ is a relation. Suppose it is not reflexive.
        Then there is an $a\in{A}$ such that
        $(a,a)\notin{R}$. But if $a\in{A}$ then $(a,a)\in{R}_{1}$
        since $R_{1}$ is reflexive. But if $(a,a)\in{R}_{1}$ then
        $(a,a)\in{R}_{1}\cup{R}_{2}$, a contradiction.
        Therefore, etc.
    \end{proof}
    Given an arbitrary relation $R$ on a set $A$, it may not be
    true that $R$ is reflexive. It may often be useful to add in
    only the necessary points of $A$ that will make $R$
    reflexive. This is called the reflexive closure of $R$.
    \begin{fdefinition}{Reflexive Closure of a Relation}
                       {Reflexive_Closure_of_Relation}
        The reflexive closure of a relation $R$ on a set $A$
        is the set:
        \begin{equation}
            S=R\cup\{(a,a):a\in{A}\}
        \end{equation}
    \end{fdefinition}
    \begin{theorem}
        If $A$ is a set, $R$ is a relation on $A$, and if $S$ is the
        reflexive closure of $R$, then $S$ is a reflexive relation on $A$.
    \end{theorem}
    \begin{theorem}
        \label{thm:Set_Theory_Refl_Clos_Is_Smallest_Refl_With_R}
        If $A$ is a set, if $R$ is a relation on $A$, if
        $S$ is the reflexive closure of $R$, and if $T$ is a
        reflexive relation on $A$ such that $R\subseteq{T}$, then
        $S\subseteq{T}$.
    \end{theorem}
    \begin{proof}
        For if $x\in{S}$, then either $x\in{R}$ or there is an
        $a\in{A}$ such that $x=(a,a)$. But if $x\in{R}$, then
        $x\in{T}$ since $R\subseteq{T}$. If $x\notin{R}$ then
        there is an $a\in{A}$ such that $x=(a,a)$. But $T$ is
        reflexive, and therefore $(a,a)\in{T}$. But then
        $x\in{T}$. Therefore, $S\subseteq{T}$.
    \end{proof}
    Thm.~\ref{thm:Set_Theory_Refl_Clos_Is_Smallest_Refl_With_R}
    says that the reflexive closure of a relation $R$ is, in a sense,
    the \textit{smallest} relation that is reflexive and contains
    $R$ as a subset.
    \begin{theorem}
        If $A$ is a set, $R_{1}$ and $R_{2}$ are relations on $A$,
        and if $S_{1}$ and $S_{2}$ are the reflexive closures of
        $R_{1}$ and $R_{2}$, respectively, then the reflexive closure
        of $R_{1}\cap{R}_{2}$ is:
        \begin{equation}
            S=S_{1}\cap{S}_{2}
        \end{equation}
    \end{theorem}
    \begin{proof}
        By the definition of reflexive closure, we have:
        \begin{align}
            S_{1}&=R_{1}\cup\{(a,a):a\in{A}\}
            \tag{Def.~\ref{def:Reflexive_Closure_of_Relation}}\\
            S_{1}&=R_{2}\cup\{(a,a):a\in{A}\}
            \tag{Def.~\ref{def:Reflexive_Closure_of_Relation}}\\
            \nonumber
            S_{1}\cap{S}_{2}&=
            (R_{1}\cup\{(a,a):a\in{A}\})
            \cap(R_{2}\cup\{(a,a):a\in{A}\})\\
            &=(R_{1}\cap{R}_{2})
            \cup\{(a,a):a\in{A}\}
            \tag{Distributive Law}
        \end{align}
        But by the definition of the transitive closure of
        $R_{1}\cap{R}_{2}$:
        \begin{equation}
            S=(R_{1}\cap{R}_{2})\cup\{(a,a):a\in{A}\}
            \tag{Def.~\ref{def:Reflexive_Closure_of_Relation}}
        \end{equation}
        Therefore, etc.
    \end{proof}
    \begin{fdefinition}{Symmetric Relation}{Symmetric_Relation}
        A symmetric relation on a set $A$ is a
        relation $R$ on $A$ such that for all $a,b\in{A}$
        such that $aRb$, it is true that $bRa$.
    \end{fdefinition}
    \begin{theorem}
        If $A$ is a set, if $S_{1}$ and $S_{2}$ are symmetric relations
        on $A$, then $S_{1}\cap{S}_{2}$ is a symmetric relation on $A$.
    \end{theorem}
    \begin{proof}
        For since $S_{1}$ and $S_{2}$ are relations, $S_{1}\cap{S}_{2}$ is a
        relation (Thm.~\ref{thm:Intersection_of_Relations_Is_Relation}). Suppose
        it is not symmetric. Then there is an $(x,y)\in{S}_{1}\cap{S}_{2}$ such
        that $(y,x)\notin{S}_{1}\cap{S}_{2}$. But if
        $(x,y)\in{S}_{1}\cap{S}_{2}$, then $(x,y)\in{S}_{1}$ and
        $(x,y)\in{S}_{2}$ (Def.~\ref{def:Intersection_of_Two_Sets}). But $S_{1}$
        is symmetric and if $(x,y)\in{S}_{1}$, then $(y,x)\in{S}_{1}$
        (Def.~\ref{def:Symmetric_Relation}). Similarly $(y,x)\in{S}_{2}$, and
        therefore $(y,x)\in{S}_{1}\cap{S}_{2}$
        (Def.~\ref{def:Intersection_of_Two_Sets})), a contradiction. Therefore,
        $S_{1}\cap{S}_{2}$ is symmetric.
    \end{proof}
    \begin{fdefinition}{Transitive Relation}{Transitive_Relation}
        A transitive relation on a set $A$ is a relation $R$ on $A$
        such that for all $a,b,c\in{A}$ such that $aRb$ and $bRc$,
        is it true that $aRc$.
    \end{fdefinition}
    \begin{theorem}
        \label{thm:Entire_Cartesian_is_Transitive}%
        If $A$ is a set, then $A\times{A}$ is a transitive relation on $A$.
    \end{theorem}
    \begin{proof}
        For suppose not. Then there exists $a,b,c\in{A}$ such that
        $(a,b)\in{A}\times{A}$ and $(b,c)\in{A}\times{A}$, yet
        $(a,c)\in{A}\times{A}$. But if $a\in{A}$ and $c\in{A}$, then
        $(a,c)\in{A}\times{A}$ (Def.~\ref{def:Cartesian_Product_of_Two_Sets}), a
        contradiction. Therefore $A\times{A}$ is a transitive relation on $A$.
    \end{proof}
    Using the Cartesian product definition of a relation, we can visualize the
    requirement imposed on transitive relations in the diagram below
    (Fig.~\ref{fig:Transitive_Relation_Diagram}).
    \begin{figure}[H]
        \centering
        \captionsetup{type=figure}
        \begin{tikzpicture}
    \foreach\x in {0,1,2,3,4,5,6,7,8,9}{
        \foreach\y in {0,1,2,3,4,5,6,7,8,9}{
            \draw[fill=black] (\x,\y) circle (0.1);
        }
        \node at (\x, -1) {$\x$};
        \node at (-1, \x) {$\x$};
    }

    \draw[fill=red,draw=red,opacity=0.4] ( 1.7, -0.3) rectangle (2.3, 9.3);
    \draw[fill=red,draw=red,opacity=0.4] (-0.3,  5.7) rectangle (9.3, 6.3);

    \draw[draw=blue,fill=cyan,opacity=0.5] (2, 4) circle (0.2);
    \draw[draw=blue,fill=cyan,opacity=0.5] (4, 6) circle (0.2);
    \draw[draw=blue,fill=cyan,opacity=0.5] (2, 6) circle (0.2);
\end{tikzpicture}
        \caption{Diagram for a Transitive Relation}
        \label{fig:Transitive_Relation_Diagram}
    \end{figure}
    Given a point $(a,b)$ that is in the relation and another point $(b,c)$, for
    the relation to be transitive requires $(a,c)$ to be contained in it. That
    is, if we take the first coordinate from the first element and the second
    coordinate from the second element and then combine them to form a new
    ordered pair, this element must also be in the relation.
    \begin{theorem}
        If $A$ is a set, if $T_{1}$ and $T_{2}$ are transitive relations on $A$,
        and if $R=T_{1}\cap{T}_{2}$, then $R$ is a transitive relation.
    \end{theorem}
    \begin{proof}
        For since $T_{1}$ and $T_{2}$ are relations, $T_{1}\cap{T}_{2}$ is a
        relation (Thm.~\ref{thm:Intersection_of_Relations_Is_Relation}). Suppose
        it is not transitive. Then there are $(x,y),(y,z)\in{R}$ such that
        $(x,z)\notin{R}$ (Def.~\ref{def:Transitive_Relation}). But if
        $(x,y),(y,z)\in{R}$, then by the definition of intersection,
        $(x,y),(y,z)\in{T}_{1}$ and $(x,y),(y,z)\in{T}_{1}$
        (Def.~\ref{def:Intersection_of_Two_Sets}). But $T_{1}$ is transitive, and
        thus if $xT_{1}y$ and $yT_{1}z$, then $xT_{1}z$. But similarly $T_{2}$
        is transitive, and therefore $xT_{2}z$. But then $(x,z)\in{T}_{1}$ and
        $(x,z)\in{T}_{2}$, and thus $(x,z)\in{T}_{1}\cap{T}_{2}$, a
        contradiction. Therefore, $R$ is transitive.
    \end{proof}
    \begin{example}
        The requirement that both relations $T_{1}$ and $T_{2}$ are transitive
        cannot be weakened. For consider the relations $S$ and $T$ on
        $\mathbb{Z}_{3}$ defined by:
        \par\hfill\par
        \begin{subequations}
            \begin{minipage}[b]{0.49\textwidth}
                \centering
                \begin{equation}
                    S=\big\{\,(0,1),\,(1,2)\,\}
                \end{equation}
            \end{minipage}
            \hfill
            \begin{minipage}[b]{0.49\textwidth}
                \centering
                \begin{equation}
                    T=\big\{\,(0,1),\,(1,2),\,(0,1)\,\}
                \end{equation}
            \end{minipage}
        \end{subequations}
        \par\vspace{2.5ex}
        Then $T$ is transitive and $S$ is not. Moreover $S\subseteq{T}$, and
        hence $S\cap{T}=S$ (Thm.~\ref{thm:Intersection_with_Subset}), and
        therefore the intersection is not transitive. This example is
        demonstrated in
        Fig.~\subref{fig:Trans_Intersect_Non_Trans_May_Not_Be_Trans}. The
        opposite is possible, and to construct an example we need only find a
        transitve relation $T$ and a non-transitive relation $S$ such that
        $T\subseteq{S}$. Define:
        \par\hfill\par
        \begin{subequations}
            \begin{minipage}[b]{0.49\textwidth}
                \centering
                \begin{equation}
                    T=\big\{\,(0,0)\,\}
                \end{equation}
            \end{minipage}
            \hfill
            \begin{minipage}[b]{0.49\textwidth}
                \centering
                \begin{equation}
                    S=\big\{\,(0,0),\,(0,1),\,(1,2)\,\}
                \end{equation}
            \end{minipage}
        \end{subequations}
        \par\vspace{2.5ex}
        Then $T$ is transitive, $S$ is not, and $S\cap{T}=T$
        (See \subref{fig:Trans_Int_Trans_May_Not_Be_Trans}).
    \end{example}
    \begin{figure}[H]
        \centering
        \captionsetup{type=figure}
        \begin{subfigure}[b]{0.49\textwidth}
            \centering
            \begin{tikzpicture}
    \foreach\x in {0,1,2}{
        \foreach\y in {0,1,2}{
            \draw[fill=black] (\x,\y) circle (0.4mm);
        }
    }
    \node at (-1, 2) {$a$};
    \node at (-1, 1) {$b$};
    \node at (-1, 0) {$c$};
    \node at (2, 3) {$a$};
    \node at (1, 3) {$b$};
    \node at (0, 3) {$c$};
    \draw[draw=blue,fill=none] (0,1) circle (0.3);
    \draw[draw=blue,fill=none] (1,0) circle (0.3);
    \draw[draw=blue,fill=none] (2,2) circle (0.3);
    \draw[draw=red,fill=none] (-0.4243, 0.5757) rectangle (0.4243, 1.4243);
    \draw[draw=red,fill=none] ( 0.5757,-0.4243) rectangle (1.4243, 0.4243);
\end{tikzpicture}
            \subcaption{The intersection is not transitive}
            \label{fig:Trans_Intersect_Non_Trans_May_Not_Be_Trans}
        \end{subfigure}
        \hfill
        \begin{subfigure}[b]{0.49\textwidth}
            \centering
            \begin{tikzpicture}
    \foreach\x in {0,1,2}{
        \foreach\y in {0,1,2}{
            \draw[fill=black] (\x,\y) circle (0.4mm);
        }
        \node at (\x, -1) {$\x$};
        \node at (-1, \x) {$\x$};
    }
    \draw[draw=blue,fill=none] (0,2) circle (0.3);
    \draw[draw=red,fill=none] (-0.4, 1.6) rectangle (0.4, 2.4);
    \draw[draw=red,fill=none] (-0.4, 0.6) rectangle (0.4, 1.4);
    \draw[draw=red,fill=none] ( 0.6,-0.4) rectangle (1.4, 0.4);
\end{tikzpicture}
            \subcaption{The intersection is transitive}
            \label{fig:Trans_Int_Trans_May_Not_Be_Trans}
        \end{subfigure}
        \label{fig:Intersection_of_Transitive_and_Non_Transitive_Relations}
        \caption{The Intersection of Transitive and Non-Transitive Relations}
    \end{figure}
    We can strengthened our claim that the intersection of two transitive
    relations is again transitive and show that any arbitrary intersection will
    again be transitive.
    \begin{theorem}
        \label{thm:Intersection_of_Transitive_is_Transitive}%
        If $A$ is a set, if $\mathcal{P}(A\times{A})$ denotes the power set of
        $A\times{A}$, if $\mathcal{O}\subseteq\mathcal{P}(A\times{A})$ is such
        that for all $\mathcal{U}\in\mathcal{O}$ it is true that $\mathcal{U}$
        is a transitive relation on $A$, if $\mathcal{T}$ is defined by:
        \begin{equation}
            \mathcal{T}=\bigcap_{\mathcal{U}\in\mathcal{O}}\mathcal{U}
        \end{equation}
        Then $\mathcal{T}$ is a transitive relation on $A$.
    \end{theorem}
    \begin{proof}
        For suppose not. Then there exists $a,b,c\in{A}$ such that
        $(a,b)\in\mathcal{T}$ and $(b,c)\in\mathcal{T}$, yet
        $(a,c)\notin\mathcal{T}$. But if $(a,b)\in\mathcal{T}$, then for all
        $\mathcal{U}\in\mathcal{O}$ it is true that $(a,b)\in\mathcal{U}$
        (Def.~\ref{def:Intersection_Over_a_Collection}). Similarly, for all
        $\mathcal{U}\in\mathcal{O}$ it is true that $(b,c)\in\mathcal{U}$.
        But by hypothesis, for all $\mathcal{U}\in\mathcal{O}$ it is true that
        $\mathcal{U}$ is a transitive relation and thus if $(a,b)\in\mathcal{U}$
        and $(b,c)\in\mathcal{U}$, then it is true that $(a,c)\in\mathcal{U}$
        (Def.~\ref{def:Transitive_Relation}). But then for all
        $\mathcal{U}\in\mathcal{O}$ it is true that $(a,c)\in\mathcal{U}$, and
        therefore $(a,c)\in\mathcal{T}$
        (Def.~\ref{def:Intersection_Over_a_Collection}), a contradiction.
        Therefore, $\mathcal{T}$ is a transitive relation on $A$.
    \end{proof}
    This allows us to define the transitive closure of any relation $R$ on a set
    $A$. It is, in a sense, the \textit{smallest} transitive relation that
    contains $R$.
    \begin{theorem}
        If $A$ is a set and if $R$ is a relation on $A$, then there exists a
        transitive relation $\mathcal{T}$ on $A$ such that
        $R\subseteq\mathcal{T}$ and such that for transitive relations $T$ on
        $A$ such that $R\subseteq{T}$ it is true that $\mathcal{T}\subseteq{T}$.
    \end{theorem}
    \begin{proof}
        For let $P$ be the proposition \textit{True if} $S$
        \textit{is a transitive relation on} $A$ \textit{such that}
        $R\subseteq{S}$, \textit{false otherwise}. Then by the axiom schema of
        specification (Ax.~\ref{ax:Axiom_Schema_of_Specification}) there exists
        a set:
        \begin{equation}
            \mathcal{O}=\big\{\,S\in\mathcal{P}(A\times{A})\;|\;P(S)\,\big\}
        \end{equation}
        But then for all $S\in\mathcal{O}$, $P(S)$ is true and therefore
        $R\subseteq{S}$ and $S$ is transitive. Moreover, $\mathcal{O}$ is
        non-empty since by Thm.~\ref{thm:Entire_Cartesian_is_Transitive},
        $A\times{A}$ is a transitive relation. Define $\mathcal{T}$ by:
        \begin{equation}
            \mathcal{T}=\bigcap_{S\in\mathcal{O}}S
        \end{equation}
        Then by Thm.~\ref{thm:Intersection_of_Transitive_is_Transitive},
        $\mathcal{T}$ is a transitive relation. Moreover, suppose $S$ is a
        transitive relation such that $R\subseteq{S}$. But if $S$ is a relation
        on $A$, then $S\subseteq{A}\times{A}$
        (Def.~\ref{def:Relation_on_a_Set}) and therefore
        $S\in\mathcal{P}(A\times{A})$ (Def.~\ref{def:Power_Set}). But if $S$ is
        a transitive relation and if $R\subseteq{S}$, then $P(S)$ is true, and
        therefore $S\in\mathcal{P}$. Thus, $\mathcal{T}\subseteq{S}$.
    \end{proof}
    \begin{fdefinition}{Transitive Closure}{Transitive_Closure}
        The transitive closure of a relation $R$ on a set
        $A$ is the the set $R^{t}\subseteq{A}\times{A}$ defined by:
        \begin{equation}
            R^{t}
        \end{equation}
    \end{fdefinition}
    \begin{fdefinition}{Asymmetric Relation}{Assymetric_Relation}
        An asymmetric relation on a set $A$ is a relation $R$
        on $A$ such that for all $a,b\in{A}$ such that $aRb$
        it is true that $(b,a)\notin{R}$.
    \end{fdefinition}
    \begin{fdefinition}{Total Relation}{Total_Relation}
        A total relation on a set $A$ is a relation $R$ on $A$ such
        that for all $a,b\in{A}$ it is true that either
        $aRb$ or $bRa$, or both.
    \end{fdefinition}
    The notion of equality can be defined as a relation
    with the following properties:
    \begin{enumerate}
        \item Equality is Reflexive: $a=a$ for all $a\in{A}$.
        \item Equality is Symmetric: $a=b$ if and only if $b=a$.
        \item Equality is Transitive: If $a=b$ and $b=c$, then $a=c$.
        \item The relation is uniquely defined by the set
              $\{(a,a)\in A\times A:a\in A\}$.
    \end{enumerate}
    That is, equality can be seen as the \textit{diagonal} in the
    Cartesian product $A\times{A}$.
    \begin{fdefinition}{Antisymmetric Relation}
        An antisymmetric relation on a set $A$ is a relation $R$ on $A$
        such that for all $a,b\in{A}$ such that $aRb$ and $bRa$, it
        is true that $a=b$.
    \end{fdefinition}
    \begin{fdefinition}{Equivalence Relation}{Equivalence_Relation}
        An equivalence relation on a set $A$ is a relation $R$ on $A$ such that
        $R$ is reflexive, symmetric, and transitive.
    \end{fdefinition}
    Equivalence relations attempt to model equality. They are fundamental in
    mathematics as they allow us to define \textit{equivalence classes}, which
    are used to define quotients. There are many examples such as quotient
    topologies, quotient groups, quotient rings, and quotient modules, all of
    which will be discussed later.
    \begin{fdefinition}{Equivalence Class}{Equivalence_Class}
        The equivalence class of an element $x$ in a set $A$ by an
        equivalence relation $R$ is the set:
        \begin{equation*}
            [x]=\{\,y\in{A}\;|\;xRy\,\}
        \end{equation*}
    \end{fdefinition}
    It's important to note that the term class here is different from the notion
    of a collection of sets. And equivalence class of an element $x$ in a set
    $A$ under an equivalene relation $R$ will indeed be a set in $ZFC$.
    \begin{theorem}
        \label{thm:Equivalence_Classes_Disjoint_or_Equal}%
        If $A$ is a set, if $R$ is an equivalence relation on $A$, and if
        $x,y\in{A}$, then either $[x]=[y]$ or $[x]\cap[y]=\emptyset$.
    \end{theorem}
    \begin{proof}
        For suppose not and suppose $[x]\ne[y]$ and that
        $[x]\cap[y]\ne\emptyset$. That is, suppose:
        \begin{equation*}
            \neg([x]=[y])\land\neg([x]\cap[y]=\emptyset)
        \end{equation*}
        If $[x]\cap[y]$ is non-empty then there is a
        $z\in{A}$ such that $z\in[x]$ and $z\in[y]$
        (Def.~\ref{def:Non_Empty_Set}). But if $z\in[x]$, then $xRz$
        (Def.~\ref{def:Equivalence_Class}). But also
        $z\in[y]$ and therefore $yRz$. But $R$ is an equivalence relation and
        is therefore symmetric (Def.~\ref{def:Equivalence_Relation}) and thus
        if $yRz$ then $zRy$ (Def.~\ref{def:Symmetric_Relation}). But an
        equivalence relation is also transitive, and thus if $xRz$ and $zRy$,
        then $xRy$ (Def.~\ref{def:Transitive_Relation}). But if $[x]\ne[y]$ then
        either $[x]\nsubseteq[y]$ or $[y]\nsubseteq[x]$. Suppose
        $[x]\nsubseteq[y]$ and let $a\in[x]$ be such that $a\notin[y]$. But
        if $a\in[x]$ then $xRa$ (Def.~\ref{def:Equivalence_Class}). But since
        equivalence relations are symmetric, if $xRa$, then $aRx$. But it was
        proven that $xRy$ and since equivalence relations are transitive, if
        $aRx$ and $xRy$, then $aRy$. But again if $aRy$, then $yRa$ and
        therefore $a\in[y]$, a contradiction. Therefore $[x]\subseteq[y]$.
        Similarly, $[y]\subseteq[x]$ and therefore $[x]=[y]$, a contradiction.
        By the law of the excluded middle, the negation is true:
        \begin{equation*}
            \neg\big(\neg([x]=[y])\land\neg([x]\cap[y]=\emptyset)\big)
            =([x]=[y])\lor([x]\cap[y]=\emptyset)
        \end{equation*}
        Thus, either $[x]=[y]$ or $[x]\cap[y]=\emptyset$.
    \end{proof}
    \begin{fdefinition}{Quotient Set}{Quotient_Set}
        The quotient set of a set $A$ by an equivalence relation $R$ on $A$ is
        the set:
        \begin{equation*}
            A/R=\{\,[x]\in\mathcal{P}(A)\;|\;x\in{A}\,\}
        \end{equation*}
        Where $[x]$ is the equivalence class of $x$ under $R$.
    \end{fdefinition}
    \begin{example}
        The definition of the quotient set comes naturally when one considers
        functions between sets. Suppose $A$ and $B$ are sets, and suppose
        $f:A\rightarrow{B}$ is a function. In general, it may not be true that
        $f(a_{1})=f(a_{2})$ implies that $a_{1}=a_{2}$, and so we wish to find a
        subset of $A$ with this property. The quotient set does this. Let
        $R$ be the relation:
        \begin{equation}
            R=\{\,(a,b)\in{A}^{2}\;|\;f(a)=f(b)\,\}
        \end{equation}
        If we form the quotient set $A/R$ and consider the projective mapping
        $\pi:A\rightarrow{A}/R$ that sends $a\in{A}$ to its equivalence class.
        That is, $\pi(a)=[a]$. We then seek a function
        $\tilde{f}:A/R\rightarrow{B}$ such that $\tilde{f}\circ{\pi}=f$.
        That is, we wish to make the diagram below \textit{commute}.
        \begin{figure}[H]
            \centering
            \begin{tikzpicture}[%
    >=latex,
    every path/.style={->},
    line width=0.2mm,
    line cap=round
]
    \node (A) at (0.0,  0.0) {$A$};
    \node (AR) at (0.0, -2.0) {$A/R$};
    \node (B) at  (2.0,  0.0) {$B$};
    \path (A) edge node [above]        {$f$}         (B);
    \path (AR) edge node [below right] {$\tilde{f}$} (B);
    \path (A) edge node [left]         {$\pi$}       (AR);
\end{tikzpicture}
            \label{fig:Comm_Diagram_Quotient_Set}
            \caption{Commutative Diagram for the Quotient Set}
        \end{figure}
        So we need to map $[x]$ to $f(x)$. That is, $\tilde{f}([x])=f(x)$. For
        this problem to be well posed requires that the equivalence class that
        make up the elements of $A/R$ come from equivalence relations. That is,
        that the relation $R$ is transitive, symmetric, and reflexive.
    \end{example}
    \begin{theorem}
        If $A$ is a set and if $R$ is an equivalence relation on $A$, then
        $A/R$ is a partition of $A$.
    \end{theorem}
    \begin{proof}
        For by Thm.~\ref{thm:Equivalence_Classes_Disjoint_or_Equal}, if
        $\mathcal{U},\mathcal{V}\in{A}/R$, then either
        $\mathcal{U}=\mathcal{V}$ or $\mathcal{U}\cap\mathcal{V}=\emptyset$.
        But also, for all $x\in{A}$, there is a $\mathcal{U}\in{A}/R$ such that
        $x\in\mathcal{U}$ since $x\in[x]$ and $[x]\in{R}/A$. Therefore,
        $A/R$ is a partition of $A$.
    \end{proof}
    \chapter{Function Theory}
        \label{chapt:Function_Theory}%
        Functions serve as a basic tool for studying mathematics, so much so
        that it is often taken as fundamental and no definition is given. We've
        adopted the definition that a function\index{Function} from a set $A$ to
        a set $B$, $f:A\rightarrow{B}$, is a subset of the Cartesian product
        $A\times{B}$ with a few properties (see Def.~\ref{def:Function}). We now
        take the time to examine the implications of this definition.
        \section{Definitions}
    Given a function $f:X\rightarrow{Y}$, and any non-empty subset
    $S\subseteq{X}$, the image $f(S)$ is non-empty. This is not true for the
    pre-image of a function. For let $f:\mathbb{R}\rightarrow\mathbb{R}$ be
    defined by $f(x)=1$ for all $x\in\mathbb{R}$. Then, for any subset
    $S\subset\mathbb{R}$
    such that $1\notin{S}$, we have that $f^{\minus{1}}(S)=\emptyset$.
    There are many examples of functions, but certain ones are easier
    to study than others. We give some of these special functions names.
    \begin{ldefinition}{Injective Functions}{Injective_Function}
        An \gls{injective function} is a function
        $f:X\rightarrow{Y}$ such that, for all
        $x,y\in{X}$ such that $x\ne{y}$, it is true that
        $f(x)\ne{f}(y)$.
    \end{ldefinition}
    That is, an injective function is a function
    $f:X\rightarrow{Y}$ such that $f(x_{1})=f(x_{2})$
    if and only if $x_{1}=x_{2}$. Such functions are also
    called \textit{one-to-one}.
    \begin{lexample}{}{Natural_Log_Is_Injective}
        Consider the natural logarithm
        $\ln:\mathbb{R}^{+}\rightarrow\mathbb{R}$. This is an injective
        function. For let $x,y\in\mathbb{R}^{+}$ be such that
        $x\ne{y}$. Suppose $\ln(x)=\ln(y)$. But then:
        \begin{equation}
            \ln(x)-\ln(y)=\ln\Big(\frac{x}{y}\Big)=0
        \end{equation}
        Recall the definition of the natural logarithm:
        \begin{equation}
            \ln(t)=\int_{1}^{t}\frac{1}{x}\diff{x}
        \end{equation}
        But then $\ln(t)=0$ if and only if $t=1$. Thus $x=y$, a
        contradiction. Therefore $\ln$ is an injective function. Not
        every function is injective, for define
        $f:\mathbb{R}\rightarrow\mathbb{R}$ by $f(x)=x^{2}$. Then, for
        all $x\in\mathbb{R}^{+}$, $f(\minus{x})=f(x)$, and thus $f$
        cannot be an injective function.
    \end{lexample}
    One might think that most functions are not injective,
    and indeed for the \textit{finite} case, this is true.
    For let $A$ and $B$ be finite sets with $n$ and $m$
    elements, respectively. If $m<n$, there can't be
    any injective function. Consider the case when $n=m$.
    Then we are simply counting the number of ways to
    permute the elements of $A$. This is $n!$. On the
    other hand, the total number of functions is
    $n^{n}$. Thus, the ratio of the number of injective
    functions to the number of functions is
    $n!/n^{n}$, and this decays to zero rapidly as
    $n$ get's large. Finally, if $m>n$, then the total
    number of injective functions is
    $n!\binom{m}{n}$, where $\binom{m}{n}$ is the
    binomial coefficient. The total number of functions
    is $n^{m}$. The ratio is thus:
    \begin{equation}
        \frac{n!\binom{m}{n}}{n^{m}}=\frac{n!\frac{m!}{n!(m-n)!}}{n^{m}}
                                    =\frac{m!}{(m-n)!n^{m}}
    \end{equation}
    And again, this decays rapidly to zero and $n$ and $m$
    get large. Later, when we define infinite sets
    and the notion of Cardinality, we'll show that this
    trend continues. That is, in a sense, \textit{most}
    functions from a set $A$ to a sufficiently large set
    $B$ are not injective. Next, we define
    \textit{surjective} functions.
    \begin{ldefinition}{Surjective Functions}{Surjective_Function}
        A \gls{surjective function} is a function
        $f:X\rightarrow{Y}$ such that $f(X)=Y$.
        That is, for all $y\in{Y}$, there is an
        $x\in{X}$ such that $f(x)=y$.
    \end{ldefinition}
    That is, every point $y\in{Y}$ gets mapped to by
    at least one point in $X$. It may also be true that
    many points in $X$ map to the same point in $Y$.
    The notions of surjective functions and injective
    functions are distinct, and neither implies the
    other. Surjective functions are also called
    \textit{onto}.
    \begin{ldefinition}{Bijective Functions}{Bijective_Function}
        A \gls{bijective function} is a function
        that is both injective and surjective.
    \end{ldefinition}
    \begin{theorem}
        \label{thm:Image_of_Empty_Set_Is_Empty}%
        If $A$ and $B$ are sets, and if $f:A\rightarrow{B}$
        is a function, then:
        \begin{equation}
            f(\emptyset)=\emptyset
        \end{equation}
    \end{theorem}
    \begin{theorem}
        If $A$ and $B$ are sets, and if $f:A\rightarrow{B}$
        is a function, then:
        \begin{equation}
            f^{-1}(\emptyset)=\emptyset
        \end{equation}
    \end{theorem}
    \begin{theorem}
        If $X$ and $Y$ are sets, if $A\subseteq{X}$, and if
        $f:X\rightarrow{Y}$ is a function such that
        $f(A)=\emptyset$, then $A=\emptyset$.
    \end{theorem}
    \begin{proof}
        For suppose not. If $A\ne\emptyset$, then there is an $x\in{A}$.
        But then $f(x)\in{f}(A)$, a contradiction as $f(A)=\emptyset$.
    \end{proof}
    \begin{theorem}
        If $X$ and $Y$ are sets, if $B$ is a subset of $Y$,
        and if $f:X\rightarrow{Y}$ is a function, then:
        \begin{equation}
            f\big(f^{-1}(B)\big)\subseteq{B}
        \end{equation}
    \end{theorem}
    \begin{proof}
        For if $y\in{f(f^{-1}(B))}$, then there is an
        $x\in{f^{-1}(B)}$ such that $y=f(x)$. But if
        $x\in{f^{-1}(B)}$, then $f(x)\in{B}$. Thus,
        $y\in{B}$. Therefore, etc.
    \end{proof}
    \begin{theorem}
        If $X$ and $Y$ are non-empty sets and if there exists
        $y_{1},y_{2}\in{Y}$ such that $y_{1}\ne{y}_{2}$, then
        there is a function $f:X\rightarrow{Y}$ and a
        $B\subseteq{Y}$ such that:
        \begin{equation}
            f\big(f^{-1}(B)\big)\ne{B}
        \end{equation}
    \end{theorem}
    \begin{proof}
        \begin{subequations}
            For if $X$ and $Y$ are non-empty, let $f:X\rightarrow{Y}$
            be defined by:
            \begin{equation}
                f=\{(x,y_{1}):x\in{X}\}
            \end{equation}
            Then $f$ is a function, since $f\subseteq{X}\times{Y}$
            as $y_{1}\in{Y}$. Moreover, for all $x\in{X}$ there is a
            unique $y\in{Y}$ such that $(x,y)\in{f}$. Thus, $f$ is a
            function from $X$ to $Y$. However since for all
            $x\in{X}$, $f(x)=y_{1}$, we have that:
            \begin{equation}
                f^{-1}(\{y_{2}\})=\emptyset
            \end{equation}
            For suppose $x\in{f}^{-1}(\{y_{2}\})$.
            Then $f(x)=y_{2}$, but for all $x\in{X}$, $f(x)=y_{1}$,
            and $y_{1}\ne{y}_{2}$. Thus
            $f^{-1}(\{y_{2}\})=\emptyset$. But by
            Thm.~\ref{thm:Image_of_Empty_Set_Is_Empty},
            $f(\emptyset)=\emptyset$. Therefore:
            \begin{equation}
                f\big(f^{-1}(\{y_{2}\})\big)=\emptyset
            \end{equation}
            But $\{y_{2}\}\ne\emptyset$ and
            $\{y_{2}\}\subseteq{Y}$. Therefore, etc.
        \end{subequations}
    \end{proof}
    \begin{theorem}
        If $X$ and $Y$ are sets, if $A$ is a subset of $X$,
        and if $f:X\rightarrow{Y}$ is a function, then:
        \begin{equation}
            A\subseteq{f^{-1}}\big(f(A)\big)
        \end{equation}
    \end{theorem}
    \begin{proof}
        For if $x\in{A}$, then there is a $y\in{f}(A)$ such that
        $f(x)=y$. But then $x\in{f^{-1}(f(A))}$. Therefore, etc.
    \end{proof}
    \begin{theorem}
        If $X$ and $Y$ are sets, if $A_{1}$ and $A_{2}$ are
        subsets of $X$ such that $A_{1}\subseteq{A}_{2}$,
        and if $f:X\rightarrow{Y}$ is a function, then:
        \begin{equation}
            f(A_{1})\subseteq{f}(A_{2})
        \end{equation}
    \end{theorem}
    \begin{proof}
        For if $y\in{f}(A_{1})$, then there is an $x\in{A}_{1}$
        such that $f(x)=y$. But $A_{1}\subseteq{A}_{2}$, and
        therefore $x\in{A}_{2}$. But if $x\in{A}_{2}$, then
        $f(x)\in{f}(A_{2})$. Thus, $y\in{f}(A_{2})$. Therefore, etc.
    \end{proof}
    \begin{theorem}
        If $X$ and $Y$ are sets, if $B_{1}$ and $B_{2}$ are subsets of
        $Y$ such that $B_{1}\subseteq{B}_{2}$, and if $f:X\rightarrow{Y}$
        is a function, then:
        \begin{equation}
            f^{-1}(B_{1})\subseteq{f^{-1}}(B_{2})
        \end{equation}
    \end{theorem}
    \begin{proof}
        For if $x\in{f}^{-1}(B_{1})$, then there is a
        $y\in{B}_{1}$ such that $f(x)=y$. But
        $B_{1}\subseteq{B}_{2}$, and therefore $y\in{B}_{2}$.
        Thus, $x\in{f}^{-1}(B_{2})$. Therefore, etc.
    \end{proof}
    \begin{theorem}
    If $f:A\rightarrow B$, $A_1,A_2\subset A$, then $f(A_1 \cup A_2) = f(A_1)\cup f(A_2)$.
    \end{theorem}
    \begin{proof}
    $[y\in f(A_1\cup A_2)]\Rightarrow [\exists x\in A_1 \cup A_2:y=f(x)]\Rightarrow [y \in f(A_1)\cup f(A_2)]$. $[y\in f(A_1)\cup f(A_2)]\Rightarrow \big[[\exists x\in A_1] \lor[\exists x\in A_2]: y=f(x)\big]\Rightarrow [x\in A_1\cup A_2]\Rightarrow [f(x)\in f(A_1\cup A_2)]$
    \end{proof}
    \begin{theorem}
        If $f:A\rightarrow B$, $A_{1},A_{}2\subset A$, then
        $f(A_{1}\cap{A}_{2})\subset{f}(A_{1})\cap{f}(A_{2})$.
    \end{theorem}
    \begin{proof}
        $[y\in f(A_1 \cap A_2)]\Rightarrow [\exists x\in A_1 \cap A_2:y=f(x)]\Rightarrow [x\in A_1 \land x \in A_2] \Rightarrow[y \in f(A_1)\cap f(A_2)]$.
    \end{proof}
    \begin{theorem}
        If $A$ and $B$ are sets, $f:A\rightarrow{B}$ is a function,
        and $B_{1},B_{2}\subseteq{B}$, then:
        \begin{equation}
            f^{-1}(B_{1}\cup{B}_{2})=f^{-1}(B_{1})\cup{f}^{-1}(B_{2})
        \end{equation}
    \end{theorem}
    \begin{proof}
        For if $x\in{B}_{1}\cup{B}_{2}$, then
        $f(x)\in{B}_{1}\cup{B}_{2}$. but then either
        $f(x)\in{B}_{1}$ or $f(x)\in{B}_{2}$, and therefore
        $x\in{f}^{\minus{1}}(B_1)\cup{f}^{\minus{1}}(B_2)$. But if
        $x\in{f}^{\minus{1}}(B_{1})\cup{f}^{\minus{1}}(B_2)$, then
        $f(x)\in{B}_{1}$ or $f(x)\in{B}_{2}$. Therefore
        $f(x)\in{B}_{1}\cup{B}_{2}$. Thus, $x\in{f}^{-1}(B_1\cup{B}_2)$.
    \end{proof}
    \begin{theorem}
        If $A$ and $B$ are sets, $f:A\rightarrow{B}$ is a function,
        and $B_{1},B_{2}\subseteq{B}$, then:
        \begin{equation}
            f^{-1}(B_{1}\cap{B}_{2})=f^{-1}(B_{1})\cap{f}^{-1}(B_{2})
        \end{equation}
    \end{theorem}
    \begin{proof}
        $[x\in f^{-1}(B_1\cap B_2)]\Rightarrow [f(x) \in B_1 \cap B_2]\Rightarrow [f(x)\in B_1\land f(x) \in B_2 ]\Rightarrow [x\in f^{-1}(B_1)\cap f^{-1}(B_2)]$. $[x\in f^{-1}(B_1)\cap f^{-1}(B_2)]\Rightarrow [x\in f^{-1}(B_1)\land x\in f^{-1}(B_2)]\Rightarrow [f(x) \in B_1\land f(x) \in B_2]\Rightarrow [f(x)\in B_1\cap B_2]\Rightarrow [x\in f^{-1}(B_1\cap B_2)]$.
    \end{proof}
    \begin{theorem}
    If $f:A\rightarrow B$, $B_1 \subset B$, then $f^{-1}(B\setminus B_1) = f^{-1}(B)\setminus f^{-1}(B_1)$.
    \end{theorem}
    \begin{proof}
    $[x\in f^{-1}(B\setminus B_1)]\Leftrightarrow [f(x)\notin B_1]\Leftrightarrow [x\in f^{-1}(B)\setminus f^{-1}(B_1)]$
    \end{proof}
    If $f:A\rightarrow B$, the image of $A$ under $f$
    is often called the range (A is often called the domain).
    \begin{ldefinition}{Permutations}{Permutations}
        A permutation on a set $A$ is a bijective function
        $f:A\rightarrow{A}$.
    \end{ldefinition}
    \begin{theorem}
    If $f:A\rightarrow B$ is bijective, then $f^{-1}$ is bijective.
    \end{theorem}
    \begin{proof}
    $[f^{-1}(y_1) = f^{-1}(y_2)]\Rightarrow [\exists x\in A:[f(x) = y_1]\land [f(x)=y_2]]\Rightarrow [y_1=y_2]$. By definition, $f^{-1}$ is surjective.
    \end{proof}
    \begin{definition}
    If $f:A\rightarrow B$ and $g:B\rightarrow C$, then $g\circ f:A\rightarrow C$ is defined by the image $g(f(x)), x\in A$. 
    \end{definition}
    \begin{theorem}
    If $f:A\rightarrow B$, $g:B\rightarrow C$, and $\mathcal{V}\subset C$, then $(g\circ g)^{-1}(\mathcal{V}) = f^{-1}(g^{-1}(\mathcal{V}))$.
    \end{theorem}
    \begin{proof}
    $[x\in (g\circ f)^{-1}(\mathcal{V})]\Leftrightarrow [g(f(x))\in \mathcal{V}] \Leftrightarrow [f(x)\in g^{-1}(\mathcal{V})]\Leftrightarrow [x\in f^{-1}(g^{-1}(\mathcal{V}))]$.
    \end{proof}
    \begin{theorem}
    If $f:A\rightarrow B$ is bijective, $g:B\rightarrow C$ is bijective, then $g\circ f$ is bijective.
    \end{theorem}
    \begin{proof}
    $\big[[f(A) = B]\land [g(B) = C]\big]\Rightarrow [g(f(A)) = g(B) = C]$. $[g(f(x_1))=g(f(x_2))]\Leftrightarrow [f(x_1)=f(x_2)]\Leftrightarrow [x_1=x_2]$.
    \end{proof}
    \begin{theorem}
    If $f:A\rightarrow B$ is bijective, $A_1\subset A$, and $f(A_1) = B$, then $A_1=A$.
    \end{theorem}
    \begin{proof}
    $\Big[\big[[A_1^c \ne \emptyset]\Rightarrow [f(A_1^c) \ne \emptyset]\big]\land[f(A_1)\cap f(A_1^c) = \emptyset]\Big]\Rightarrow [\exists y\in B:y\notin f(A_1)]$, a contradiction.
    \end{proof}
        \section{Binary Operations}
    Binary operations are the standard tools that one uses when they develope
    arithmetic. As such, the most familiar examples of binary operations are
    those of addition, multiplication, and subtraction with real numbers.
    On the other hand, division is \textit{not} a binary operation on the real
    numbers since division by zero is undefined. To make this explicit we need
    to give a rigorous definition to binary operations. We can do this with the
    language of functions\index{Function} and by using the Cartesian product
    \index{Cartesian Product} of a set $A$ with itself.
    \begin{fdefinition}{Binary Operation}{Binary_Operation}
        A \gls{binary operation} on a set $A$ is a function
        $*:A\times{A}\rightarrow{A}$.
    \end{fdefinition}
    \begin{example}
        Let $\mathbb{R}$ be the set of real numbers and $+$ denote the addition
        of two real numbers. Then $+$ is a binary operation on $\mathbb{R}$.
        Similarly, if $\cdot$ denotes the multiplication of two real numbers,
        than it two is a binary operation on $\mathbb{R}$. For division, $\div$,
        we are lacking the requirement that \textit{for all}
        $(a,b)\in\mathbb{R}^{2}$ there is a unique $c\in\mathbb{R}$ such that
        $a\div{b}=c$, since if $b=0$ our expression is undefined. That is, this
        is not a function from $\mathbb{R}^{2}$ to $\mathbb{R}$. If we consider
        all of the non-zero elements, then division is a binary operation. That
        is, division is a binary operation on $\mathbb{R}\setminus\{0\}$.
    \end{example}
    \begin{lexample}{Binary Operation on the Set of Functions}
                    {Binary_Operation_on_the_Set_of_Functions}
        If $A$ is a set, and if $\mathcal{F}(A,A)$ denotes the set of all
        functions $f:A\rightarrow{A}$, and if $\circ$ denotes function
        composition, then $\circ$ is a binary operation on $\mathcal{F}(A,A)$.
        That is, for any two functions $f,g\in\mathcal{F}(A,A)$, the composition
        $g\circ{f}:A\rightarrow{A}$ is again an element of $\mathcal{F}(A,A)$
    \end{lexample}
    Just like functions, there are three important conditions that a binary
    operation must satisfy. Given any ordered pair $(a,b)\in{A}^{2}$, it must
    be true that $*(a,b)$ is defined. This comes from the definition of a
    function on a set (Def.~\ref{def:Function}). Next, the image of $(a,b)$ must
    be unique. That is, if $*(a,b)=c$ and $*(a,b)=d$, then $c=d$. Note that this
    is not the same as requiring that $*(a,b)=*(b,a)$, and in general this is
    not true. Such binary operations are called
    \textit{commutative}\index{Commutative Operation}. Lastly, for any
    $(a,b)\in{A}^{2}$, $*(a,b)$ must be an element of $A$. That is,
    $*(a,b)\in{A}$. All of these requirements come from the definition of a
    function, so in a sense it is redundant to repeat these. In practice one
    defines a binary operation by a formula $\varphi$, and it then becomes
    necessary to show that this formula satisfies these properties before we can
    rightly call it a binary operation.
    \begin{example}
        Let $A=\mathbb{Z}_{2}$ and consider all of the binary operations on
        $\mathbb{Z}_{2}$. We can count these by constructing tables:
        \begin{table}[H]
            \centering
            \begin{tabular}{c|c}
                $(x,y)$&$*(x,y)$\\
                \hline
                $(0,0)$&0\\
                $(0,1)$&0\\
                $(1,0)$&1\\
                $(1,1)$&0
            \end{tabular}
            \label{tab:Binary_Operation_on_Z_2}
            \caption{Simple Binary Operation on $\mathbb{Z}_{2}$}
        \end{table}
        This is one such binary operation, there are 15 others. To see this,
        recall that the number of functions from a set $A$ to a set $B$, where
        both $A$ and $B$ are finite sets with $m$ and $n$ elements,
        respectively, is $n^{m}$. Since $\mathbb{Z}_{2}$ has 2 elements, and
        since a binary operation is a function
        $*\mathbb{Z}_{2}\times\mathbb{Z}_{2}\rightarrow\mathbb{Z}_{2}$, the
        total number of binary operations is $2^{(2^{2})}=2^{4}=16$. In general,
        if $A$ has $n$ elements, and if $B$ is the set of all binary operations
        on $A$, then:
        \begin{equation}
            \textrm{Card}(B)=n^{(n^2)}
        \end{equation}
    \end{example}
    \begin{example}
        Let's consider some formula that take in numbers and return numbers, and
        see if they can define operations on various sets. Suppose we have:
        \begin{equation}
            a*b=\{\,r\in\mathbb{R}\;|\;r^{2}=|ab|\,\}
        \end{equation}
        Where $|ab|$ denotes the absolute value of $a$ times $b$. If we take the
        positive square root we can write this as $a*b=\sqrt{|ab|}$. If we
        consider this formula on the rational numbers $\mathbb{Q}$, does it
        define a function? One might recall that $\sqrt{2}$ is not a rational
        number. That is, it is \textit{irrational}. Thus $1*2$ is not a rational
        number, and so $*$ is not a binary operation on $\mathbb{Q}$. It is a
        binary operation on $\mathbb{R}$, however. Suppose we change the formula
        to state:
        \begin{equation}
            a*b=\{\,r\in\mathbb{R}\:|\;r^{2}-ab=0\,\}
        \end{equation}
        and where we consider this formula to take inputs from $\mathbb{R}$.
        This is not a binary operation since it is poorly defined. That is,
        should $1*1=1$, or should $1*1=\minus{1}$? The formula is ambgious and
        thus $*$ is not a binary operation.
    \end{example}
    \begin{example}
        If we consider subtraction on the integers $\mathbb{Z}$, this is a
        binary operation. The operation is well defined and returns an integer
        for all integer inputs. If instead we consider subtraction on
        $\mathbb{N}$, this is \textit{not} a binary operation since it may take
        in non-negative integers and return a negative integer. For example,
        $1-2=\minus{1}$, and $\minus{1}\notin\mathbb{N}$. A simple fix for this
        is considering again the absolute value function. If we define
        $n*m=|n-m|$, then $*$ is indeed a binary operation on $\mathbb{N}$.
    \end{example}
    \begin{fnotation}{Binary Operation}{Binary_Operation}
        If $A$ is a set and if $*:A\times{A}\rightarrow{A}$ is a binary
        operation on $A$, for any ordered pair $(a,b)\in{A}^{2}$, the image
        of $*(a,b)$ is denoted $a*b$.
    \end{fnotation}
    It is occasionally useful to think of binary operations purely as functions,
    and so we will use function notation at these times. For the most part we
    will stick with notation defined in Not.~\ref{not:Binary_Operation}. There
    are several types of binary operations worth studying, and several key
    properties that these operations can have. One of the most fundamental is
    the existence of a \textit{unital} element, also known as an identity.
    \begin{fdefinition}{Left Unital Element}{Left_Unital_Element}
        A left unital element in a \gls{set} $A$ under a \gls{binary operation}
        $*$ on $A$ is an element $e_{L}\in{A}$ such that, for all $a\in{A}$ it
        is true that $e_{L}*a=a$.
    \end{fdefinition}
    \begin{example}
        From the definition of a left unital element
        (Def.~\ref{def:Left_Unital_Element}) it would seem natural to define a
        right unital element. The importance is to note that left and right
        unital elements need not be equal. Indeed, if $A$ is a set and $*$ is
        a binary operation, given a left identity $e_{L}$ and a right identity
        $e_{R}$ it will be true that $e_{R}=e_{L}$ and thus all left and right
        unital elements will be the same
        (see Thm.~\ref{thm:left_and_right_identity_implies_identity}). Thus to
        find counterexamples to the claim that the existence of a left unital
        element implies the existence of a right unital element we need to think
        of strange operations. Let $A=\mathbb{R}$ and let $*$ be defined by
        $a*b=b$ for all $a,b\in\mathbb{R}$. Then every element of $\mathbb{R}$
        is a left unital element. Moreover, none of the element of $\mathbb{R}$
        are right unital elements.
    \end{example}
    \begin{fdefinition}{Right Unital Element}{Right_Unital_Element}
        A right unital element of a \gls{set} $A$ under a \gls{binary operation}
        $*$ is an element $e_{R}$ such that for all $a\in{A}$ it is true that
        $a*e_{R}=a$.
    \end{fdefinition}
    \begin{example}
        Consider $\mathbb{R}$ with the operation $*$ defined by $a*b=a+b+1$.
        This operation has a right unital element, $\minus{1}$. For:
        \begin{equation}
            a*(\minus{1})=a+(\minus{1})+1=a+0=a
        \end{equation}
        And this is true for all $a\in\mathbb{R}$, so $\minus{1}$ is a right
        unital element. It turns out this is also a left unital element, and
        hence a unital element, and this can be proven if addition is known to
        be a \textit{commutative} operation.
    \end{example}
    \begin{theorem}
        \label{thm:left_and_right_identity_implies_identity}%
        If $A$ is a set, if $*$ is a binary operation on $A$, if $e_{L}$ is a
        left unital element of $A$, and if $e_{R}$ is a right unital element of
        $a$, then $e_{L}=e_{R}$.
    \end{theorem}
    \begin{proof}
        For:
        \begin{equation}
            e_{L}=e_{L}*e_{R}=e_{R}
        \end{equation}
        And thus $e_{L}=e_{R}$.
    \end{proof}
    \begin{example}
        Consider a non-empty set $A$ and the set of all functions from $A$ to
        itself, $\mathcal{F}(A,A)$. Let $\circ$ denote the binary operation of
        function composition. Then $\mathcal{F}(A,A)$ has a right identity under
        $\circ$, and a left identity. For
        the identity function $\textrm{id}_{A}$ acts as a right identity:
        \begin{equation}
            (f\circ\textrm{id}_{A})(x)
            =f\big(\textrm{id}_{A}(x)\big)
            =f(x)
        \end{equation}
        And thus $\textrm{id}_{A}$ is a right identity. By
        Thm.~\ref{thm:left_and_right_identity_implies_identity}, any left
        identity must also be a right identity, and so the likely candidate to
        check is $\textrm{id}_{A}$. And indeed we have:
        \begin{equation}
            (\textrm{id}_{A}\circ{f})(x)
            =\textrm{id}_{A}\big(f(x)\big)
            =f(x)
        \end{equation}
        And thus $\textrm{id}_{A}$ is a left identity as well.
    \end{example}
    \begin{fdefinition}{Unital Element}{Unital_Element}
        A \gls{unital element} of a \gls{set} $A$ under a \gls{binary operation}
        $*$ is an element $e\in{A}$ that is both a right unital element and a
        left unital element.
    \end{fdefinition}
    \begin{example}
        Let $\mathbb{R}$ be the set of real numbers and let $+$ be the usual
        notion of addition. Then 0 is a unital element of $\mathbb{R}$ with
        respect to this operation. That is, for any real number $x$ we have
        $x+0=0+x=x$. For multiplication the unital element is 1. This is because
        $1\cdot{x}=x\cdot{1}=x$. Subtraction has a right unital element, and
        again it is 0 since $x-0=x$, but no left identity. To see this, suppose
        $e-x=x$ for all $x$. Applying some algebra we have that $e=2x$, meaning
        there is no constant $e\in\mathbb{R}$ such that for all $x$, $e-x=x$.
        Since subtraction has no left unital element, it has no unital element
        either.
    \end{example}
    \begin{theorem}
        \label{thm:Unital_Elements_are_Unique}%
        If $A$ is a set, if $*$ is a binary operation on $A$, and if $e$ and
        $e'$ are unital elements of $A$, then $e=e'$
    \end{theorem}
    \begin{proof}
        For:
        \begin{equation}
            e=e*e'=e'
        \end{equation}
        And thus by transitivity, $e=e'$.
    \end{proof}
    The next thing to discuss is that of inverses. There are five types, but in
    practice only one of these is discussed.
    \begin{fdefinition}{Weak Right Inverse}{Weak_Right_Inverse}
        A weak right inverse of an element $a$ in a \gls{set} $A$ under a
        \gls{binary operation} $*$ on $A$ is an element $b\in{A}$ such that
        $a*b$ is a right unital element.
    \end{fdefinition}
    This definition will not recieve much use until we discuss
    groups\index{Group}. A group is a set with a binary operation $*$ that has
    a unital element, inverse elements, and is associatied (to be defined soon).
    As it turns out these conditions are stronger than necessary and it suffices
    to check that there are weak right inverses and a right unital element. The
    next thing to define is right inverses.
    \begin{fdefinition}{Right Inverse}{Right_Inverse}
        A right inverse of an element $a$ in a \gls{set} $A$ under a
        \gls{binary operation} is an element $b\in{A}$ such that $a*b$ is a
        \gls{unital element}.
    \end{fdefinition}
    Here, we've simply strengthened the requirement that $a*b$ not only be a
    right unital element, but also a left unital element as well. A right
    inverse is therefore necessarily a weak right inverse.
    \begin{fdefinition}{Weakly Left Invertible}{Weakly_Left_Invertible}
        A weakly left invertible element of a \gls{set} $A$ under a
        \gls{binary operation} $*$ is an element $a\in{A}$ such that there
        exists a $b\in{A}$ such that $b*a$ is a left unital element.
    \end{fdefinition}
    \begin{fdefinition}{Left Invertible Element}{Left_Inverse}
        A left invertible element of a \gls{set} $A$ under a
        \gls{binary operation} is an element $a\in{A}$ such that there exists a
        $b\in{A}$ such that $b*a$ is a \gls{unital element}.
    \end{fdefinition}
    \begin{fdefinition}{Invertible Element}{Invertible_Element}
        An invertible element of a a \gls{set} $A$ under a
        \gls{binary operation} is an element $a\in{A}$ that is both
        left invertible and right invertible.
    \end{fdefinition}
    \begin{theorem}
        \label{thm:Unital_Elements_Are_Invertible}%
        If $A$ is a set, if $*$ is a binary operation on $A$, and if $e$ is a
        unital element of $A$, then $e$ is an invertible element.
    \end{theorem}
    \begin{proof}
        For since $e$ is a unital element, it is true that $e=e*e$
        (Def.~\ref{def:Unital_Element}). Therefore $e$ is invertible element
        (Def.~\ref{def:Invertible_Element}).
    \end{proof}
    \begin{fdefinition}{Commutative Operation}{Commutative_Operation}
        A \gls{commutative operation} on a \gls{set} $A$ is a
        \gls{binary operation} $*$ such that for all $(a,b)\in{A}^{2}$ it is
        true that $a*b=b*a$.
    \end{fdefinition}
    \begin{fdefinition}{Associative Operation}{Associative_Operation}
        A \gls{associative operation} on a \gls{set} $A$ is a
        \gls{binary operation} $*$ such that, for all $a,b,c\in{A}$ it is true
        that $a*(b*c)=(a*b)*c$.
    \end{fdefinition}
    \begin{example}
        Consider a finite set $A$ and consider the set of all functions from
        $\mathbb{Z}_{n}$ to $A$. That is, $\mathcal{F}_{n}(\mathbb{Z}_{n},A)$.
        Define $A[x]$ by:
        \begin{equation}
            \mathcal{F}=\bigcup_{n\in\mathbb{N}}\mathcal{F}_{n}
        \end{equation}
        That is, the set of all finite sequences in $A$. We can form an
        associative operation on this set by defining the concatenation
        operation. Given $f,g\in\mathcal{F}$, suppose
        $f\in\mathcal{F}(\mathbb{Z}_{m},A)$ and
        $g\in\mathcal{F}(\mathbb{Z}_{n},A)$. We define
        $f*g\in\mathcal{F}(\mathbb{Z}_{m+n},A)$ as follows:
        \begin{equation}
            (f*g)(k)=
            \begin{cases}
                f(k),&k\in\mathbb{Z}_{m}\\
                g(k-m),&k\in\mathbb{Z}_{m+n}\textrm{ and }k\geq{m}
            \end{cases}
        \end{equation}
        That is, given two sequences $f_{0},f_{1},\dots,f_{m-1}$ and
        $g_{0},g_{1},\dots,g_{n-1}$, we concatenate them to form the sequence
        $f_{0},\dots,f_{m-1},g_{0},\dots,g_{n-1}$. This operation is associative
        since if $f,g,h\in\mathcal{F}$, then:
        \begin{subequations}
            \begin{align}
                f*(g*h)&=(f_{0},f_{1},\dots,f_{m-1})
                    *(g_{0},g_{1},\dots,g_{m-1},h_{0},h_{1},\dots,h_{r-1})\\
                &=f_{0},f_{1},\dots,f_{m-1},
                    g_{0},g_{1},\dots,g_{m-1},h_{0},h_{1},\dots,h_{r-1}\\
                &=(f_{0},f_{1},\dots,f_{m-1},
                    g_{0},g_{1},\dots,g_{m-1})*(h_{0},h_{1},\dots,h_{r-1})\\
                &=(f*g)*h
            \end{align}
        \end{subequations}
        If $A$ has more than one point than $*$ is not commutative. For let
        $f,g:\mathbb{Z}_{1}\rightarrow{A}$ be defined by $f(0)=a$ and $g(0)=b$,
        respectively. Then $f*g=a,b$ but $g*f=b,a$, and thus $f*g\ne{g}*f$.
        There is, however, an identity. Consider a $\mathbb{Z}_{0}$, which is
        the empty set. Any function from $\mathbb{Z}_{0}$ to $A$ is therefore
        the \textit{empty sequence}. If we concatenate $f$ with the empty
        sequence we get back $f$, and this then acts as our unital element.
    \end{example}
    \begin{theorem}
        \label{thm:Assoc_Op_Inverses_are_Unique}%
        If $A$ is a set, if $*$ is a binary operation on $A$, if $a$ is an
        invertible element, and if $b,c\in{A}$ are such that $b*a$ and $a*c$
        are unital elements, then $b=c$.
    \end{theorem}
    \begin{proof}
        For:
        \begin{align}
            b&=b*(a*c)
            \tag{$a*b$ is a unital element}\\
            &=(b*a)*c
            \tag{Associativity}\\
            &=c
            \tag{$b*a$ is a unital element}
        \end{align}
        And therefore $b=c$.
    \end{proof}


        \section{Boolean Algebras}
    \label{sec:Boolean_Algebra}%
    We now attempt to make set theory more algebraic. We wish to model
    as an object the triple $(\mathcal{P}(X),\cup,\cap)$, where
    $\mathcal{P}(X)$ is the \gls{power set} of some set, and $\cup$ and
    $\cap$ and union and intersection, respectively. These can be seen as
    binary operations on $\mathcal{P}(X)$. We take a few of the properties
    of this structure and state them as the definition for our new object:
    \textit{Boolean Algebras}.
    \begin{fdefinition}{Complement in a Boolean Algebra}
                       {Complement}
        A complement of a \gls{set} $A$ with respect to two
        \glspl{binary operation} $*$ and $\circ$ is an element
        $a^{\minus{1}}\in{A}$ such that:
        \begin{equation*}
            a*a^{\minus{1}}=a^{\minus{1}}*a=e_{\circ}
            \quad\quad
            a\circ{a}^{\minus{1}}=a^{\minus{1}}\circ{a}=e_{*}
        \end{equation*}
        Where $e_{\circ}$ and $e_{*}$ are the \glspl{unital element} of
        $\circ$ and $*$, respectively.
    \end{fdefinition}
    \begin{fdefinition}{Boolean Algebras}{Boolean_Algebra}
        A Boolean algebra is a set $A$ with two
        \glslink{commutative operation}{commutative} \glspl{binary operation}
        $\circ$ and $*$ on $A$ with \glspl{unital element} $e_{*}$ and
        $e_{\circ}$, respectively, such that:
        \begin{itemize}
            \item[1.)]  $\circ$ \glslink{distributive operation}{distributes}
                        over $*$ and $*$ distributes over $\circ$.
            \item[2.)]  For all $a\in{A}$ there is a complement of $a$.
        \end{itemize}
    \end{fdefinition}
    The second property is known as the complement property and it is very
    different from the notion of inverses. An inverse of an element $a$ with
    respective to an operation $\cdot$ is such that $a\cdot{b}$ is a unital
    element with respect to the operation $\cdot$. A complement produces a
    unital element with respect the the \textit{other} operation. That is,
    $a*a^{\minus{1}}$ is a unital element of $\circ$, and not $*$. Similarly,
    $a\circ{a}^{\minus{1}}$ is a unital element of $*$ and not $\circ$.
    \begin{theorem}
        If $(A,\circ,*)$ is a Boolean algebra and if $b\in{X}$ is a
        unital element of $\circ$, then $b=e_{\circ}$.
    \end{theorem}
    \begin{proof}
        For unital elements are unique
        (Thm.~\ref{thm:Unital_Elements_are_Unique}), and therefore
        $b=e_{\circ}$.
    \end{proof}
    \begin{theorem}
        If $(X,\circ,*)$ is a Boolean algebra and if $b\in{X}$ is a
        unital element of $*$, then $b=e_{*}$.
    \end{theorem}
    \begin{proof}
        For unital elements are unique
        (Thm.~\ref{thm:Unital_Elements_are_Unique}), and therefore $b=e_{*}$.
    \end{proof}
    \begin{theorem}
        \label{thm:Bool_Alg_Boundary_of_Circ}%
        If $(X,\circ,*)$ is a Boolean algebra, if $e_{*}$ is the unital
        element of $*$, and if $a\in{X}$, then $a\circ{e}_{*}=e_{*}$.
    \end{theorem}
    \begin{proof}
        For if $a\in{X}$ then there is an $a^{\minus{1}}\in{X}$ such that
        $a\circ{a}^{\minus{1}}=e_{*}$ (Def.~\ref{def:Boolean_Algebra}).
        But then:
        \par\vspace{-2.5ex}
        \begin{minipage}[t]{0.51\textwidth}
            \centering
            \begin{align}
                e_{*}&=a\circ{a}^{\minus{1}}
                \tag{Complement}\\
                &=a\circ(a^{\minus{1}}*e_{*})
                \tag{Identity}\\
                &=(a\circ{a}^{\minus{1}})*(a\circ{e}_{*})
                \tag{Distributivity}
            \end{align}
        \end{minipage}
        \hfill
        \begin{minipage}[t]{0.47\textwidth}
            \centering
            \begin{align}
                &=e_{*}*(a\circ{e}_{*})
                \tag{Complement}\\
                &=a\circ{e}_{*}
                \tag{Identity}
            \end{align}
        \end{minipage}
        \par\vspace{2.5ex}
        And therefore $e_{*}=a\circ{e}_{*}$.
    \end{proof}
    This theorem is equivalent to the notion that a Boolean algebra is a
    bounded lattice\index{Bounded Lattice} and the $e_{*}$ is a boundary. The
    theorem holds for $\circ$ as well.
    \begin{theorem}
        \label{thm:Bool_Alg_Boundary_of_Star}%
        If $(X,\circ,*)$ is a Boolean algebra, if $e_{*}$ is the unital
        element of $*$, and if $a\in{X}$, then $a*{e}_{\circ}=e_{\circ}$.
    \end{theorem}
    \begin{proof}
        For if $a\in{X}$ then there is an $a^{\minus{1}}\in{X}$ such that
        $a*{a}^{\minus{1}}=e_{\circ}$ (Def.~\ref{def:Boolean_Algebra}).
        But then:
        \par\vspace{-2.5ex}
        \begin{minipage}[t]{0.51\textwidth}
            \centering
            \begin{align}
                e_{\circ}&=a*{a}^{\minus{1}}
                \tag{Complement}\\
                &=a*(a^{\minus{1}}\circ{e}_{\circ})
                \tag{Identity}\\
                &=(a*a^{\minus{1}})\circ(a*e_{*})
                \tag{Distributivity}
            \end{align}
        \end{minipage}
        \hfill
        \begin{minipage}[t]{0.47\textwidth}
            \centering
            \begin{align}
                &=e_{\circ}\circ(a*e_{\circ})
                \tag{Complement}\\
                &=a*e_{\circ}
                \tag{Identity}
            \end{align}
        \end{minipage}
        \par\vspace{2.5ex}
        And therefore $e_{\circ}=a*e_{\circ}$.
    \end{proof}
    \begin{theorem}
        If $(A,\circ,*)$ is a Boolean algebra, if $e_{\circ}$ and $e_{*}$
        are the unital elements of $\circ$ and $*$, respectively, then
        $e_{\circ}$ is the complement of $e_{*}$ and $e_{*}$ is the
        complement of $e_{\circ}$.
    \end{theorem}
    \begin{proof}
        From identity:
        \par\vspace{-2.5ex}
        \begin{subequations}
            \begin{minipage}[b]{0.49\textwidth}
                \centering
                \begin{equation}
                    e_{\circ}\circ{e}_{*}=e_{*}\circ{e}_{\circ}=e_{*}
                    \tag{Identity}
                \end{equation}
            \end{minipage}
            \hfill
            \begin{minipage}[b]{0.49\textwidth}
                \centering
                \begin{equation}
                    e_{\circ}*{e}_{*}=e_{*}*e_{\circ}=e_{\circ}
                    \tag{Identity}
                \end{equation}
            \end{minipage}
        \end{subequations}
        \par\vspace{2.5ex}
        Thus, $e_{*}$ is a complement of $e_{\circ}$ and $e_{\circ}$ is a
        complement of $e_{*}$ (Def.~\ref{def:Complement}).
    \end{proof}
    Every element of a Boolean algebra is idempotent with respect to both
    operations.
    \begin{theorem}
        \label{thm:Bool_Alg_Idempotent_of_Star}%
        If $(A,\circ,*)$ is a Boolean algebra and if $a\in{A}$, then $a*a=a$.
    \end{theorem}
    \begin{proof}
        For:
        \par\vspace{-2.5ex}
        \begin{subequations}
            \begin{minipage}[t]{0.49\textwidth}
                \centering
                \begin{align}
                    a&=a*e_{*}
                    \tag{Identity}\\
                    &=a*(a\circ{a}^{\minus{1}})
                    \tag{Complement}\\
                    &=(a*a)\circ(a*a^{\minus{1}})
                    \tag{Distributivity}
                \end{align}
            \end{minipage}
            \hfill
            \begin{minipage}[t]{0.49\textwidth}
                \centering
                \begin{align}
                    &=(a*a)\circ{e}_{\circ}
                    \tag{Complement}\\
                    &=a*a\tag{Identity}
                \end{align}
            \end{minipage}
        \end{subequations}
        \par\vspace{2.5ex}
        And therefore $a=a*a$.
    \end{proof}
    \begin{theorem}
        \label{thm:Bool_Alg_Idempotent_of_Circ}%
        If $(A,\circ,*)$ is a Boolean algebra and if $a\in{A}$, then
        $a\circ{a}=a$.
    \end{theorem}
    \begin{proof}
        For:
        \par\vspace{-2.5ex}
        \begin{subequations}
            \begin{minipage}[t]{0.49\textwidth}
                \centering
                \begin{align}
                    a&=a\circ{e}_{\circ}
                    \tag{Identity}\\
                    &=a\circ(a*a^{\minus{1}})
                    \tag{Complement}\\
                    &=(a\circ{a})*(a\circ{a}^{\minus{1}})
                    \tag{Distributivity}
                \end{align}
            \end{minipage}
            \hfill
            \begin{minipage}[t]{0.49\textwidth}
                \centering
                \begin{align}
                    &=(a\circ{a})*{e}_{*}
                    \tag{Complement}\\
                    &=a\circ{a}\tag{Identity}
                \end{align}
            \end{minipage}
        \end{subequations}
        \par\vspace{2.5ex}
        And therefore $a=a\circ{a}$.
    \end{proof}
    \begin{theorem}
        \label{thm:Bool_Alg_aob_equal_a_acb_equal_a_implies_a_equal_b}%
        If $(A,\circ,*)$ is a Boolean algebra, if $a,b\in{A}$, if $a\circ{b}=a$,
        and if $a*b=a$, then $b=a$.
    \end{theorem}
    \begin{proof}
        For:
        \par
        \begin{minipage}[b]{0.49\textwidth}
            \centering
            \begin{align}
                b&=b*e_{*}\tag{Identity}\\
                 &=b*(a\circ a^{-1})\tag{Complement}\\
                 &=(b*a)\circ(b* a^{-1})\tag{Distributivity}\\
                 &=a\circ (b* a^{-1})\tag{Hypothesis}
            \end{align}
        \end{minipage}
        \hfill
        \begin{minipage}[b]{0.49\textwidth}
            \centering
            \begin{align}
                &=(a\circ b)*(a\circ a^{-1})\tag{Distributivity}\\
                &=(a\circ{b})*e_{*}\tag{Complement}\\
                &=a\circ{b}\tag{Identity}\\
                &=a\tag{Hypothesis}
            \end{align}
        \end{minipage}
        \par\vspace{2.5ex}
        And therefore $a=b$.
    \end{proof}
    \begin{theorem}
        If $(A,\circ,*)$ is a Boolean algebra, if $a\in{A}$ is such that
        $a=a^{\minus{1}}$, then $a=e_{\circ}=e_{*}$.
    \end{theorem}
    \begin{proof}
        For let $a\in{A}$ and let $a=a^{\minus{1}}$. Then by
        Thm.~\ref{thm:Bool_Alg_Idempotent_of_Star}:
        \begin{equation}
            a=a*a=a*a^{-1}=e_{\circ}
        \end{equation}
        Similarly, $a=e_{*}$.
    \end{proof}
    \begin{theorem}
        If $(A,\circ,*)$ is a Boolean algebra, if $a\in{A}$, and if $b,c\in{A}$
        are complements of $a$, then $b=c$.
    \end{theorem}
    \begin{proof}
        For:
        \par
        \begin{minipage}[t]{0.49\textwidth}
            \centering
            \begin{align}
                b&=b*e
                \tag{Identity}\\
                &=b*(a\circ{c})
                \tag{Complement}\\
                &=(b*a)\circ(b*c)
                \tag{Distributivity}\\
                &=e_{\circ}\circ(b*c)
                \tag{Complement}\\
                &=(c*a)\circ(b*c)
                \tag{Complement}
            \end{align}
        \end{minipage}
        \hfill
        \begin{minipage}[t]{0.49\textwidth}
            \centering
            \begin{align}
                &=(c*a)\circ(c*b)
                \tag{Commutativity}\\
                &=c\circ(a*b)
                \tag{Distributivity}\\
                &=c\circ{e}_{\circ}
                \tag{Complement}\\
                &=c
                \tag{Identity}
            \end{align}
        \end{minipage}
        \par\vspace{2.5ex}
        Therefore, $b=c$.
    \end{proof}
    \begin{theorem}
        If $(A,\circ,*)$ is a Boolean algebra and if $a\in{A}$,
        then $(a^{\minus{1}})^{\minus{1}}=a$.
    \end{theorem}
    \begin{proof}
        For:
        \begin{align}
            a&=a*e_{*}
            \tag{Identity}\\
            &=a*\big(a^{\minus{1}}\circ(a^{\minus{1}})^{\minus{1}})
            \tag{Complement}\\
            &=(a\circ{a}^{\minus{1}})*
                \big(a\circ(a^{\minus{1}})^{\minus{1}}\big)
            \tag{Distributivity}\\
            &=e_{*}*\big(a\circ(a^{\minus{1}})^{\minus{1}}\big)
            \tag{Complement}\\
            &=a\circ(a^{\minus{1}})^{\minus{1}}
            \tag{Identity}
        \end{align}
        And similarly $a*(a^{\minus{1}})^{\minus{1}}=a$. But if $a*b=a$ and
        $a\circ{b}=a$, then $a=b$
        (Thm.~\ref{thm:Bool_Alg_aob_equal_a_acb_equal_a_implies_a_equal_b}).
        Therefore, $a=(a^{\minus{1}})^{\minus{1}}$.
    \end{proof}
    \begin{ltheorem}{Absorption Laws}{Absorption_Law}
        If $(A,\circ,*)$ is a Boolean algebra, if $a\in{A}$ and if $b\in{A}$,
        then $a*(a\circ{b})=a$ and $a\circ(a*{b})=a$.
    \end{ltheorem}
    \begin{proof}
        For:
        \par
        \begin{minipage}[t]{0.59\textwidth}
            \centering
            \begin{align}
                a*(a\circ{b})&=(a*e_{*})\circ(a*b)
                \tag{Identity}\\
                &=a*(e_{*}\circ{b})
                \tag{Distributivity}
            \end{align}
        \end{minipage}
        \hfill
        \begin{minipage}[t]{0.39\textwidth}
            \centering
            \begin{align}
                &=a*e_{*}
                \tag{Thm.~\ref{thm:Bool_Alg_Boundary_of_Circ}}\\
                &=a
                \tag{Identity}
            \end{align}
        \end{minipage}
        \par\vspace{2.5ex}
        And therefore $a*(a\circ{b})=a$. Similarly:
        \par
        \begin{minipage}[t]{0.59\textwidth}
            \centering
            \begin{align}
                a\circ(a*b)&=(a\circ{e}_{\circ})*(a\circ{b})
                \tag{Identity}\\
                &=a\circ(e_{\circ}*b)
                \tag{Distributivity}
            \end{align}
        \end{minipage}
        \hfill
        \begin{minipage}[t]{0.39\textwidth}
            \begin{align}
                &=a\circ{e}_{\circ}
                \tag{Thm.~\ref{thm:Bool_Alg_Boundary_of_Star}}\\
                &=a
                \tag{Identity}
            \end{align}
        \end{minipage}
        \par\vspace{2.5ex}
        And therefore $a\circ(a*b)=a$.
    \end{proof}
    We can weaken the hypothesis of
    Thm.~\ref{thm:Bool_Alg_aob_equal_a_acb_equal_a_implies_a_equal_b} to obtain
    a more general result.
    \begin{theorem}
        \label{thm:Bool_Alg_aob_equal_acb_implies_a_equal_b}
        If $(A,\circ,*)$ is a Boolean algebra, if $a,b\in{A}$, and if
        $a*b=a\circ{b}$, then $a=b$.
    \end{theorem}
    \begin{proof}
        For:
        \begin{align}
            a&=a*e_{*}
            \tag{Identity}\\
            &=a*(b\circ{b}^{\minus{1}})
            \tag{Complement}\\
            &=(a*b)\circ(a*b^{\minus{1}})
            \tag{Distributivity}\\
            &=(a\circ{b})\circ(a*b^{\minus{1}})
            \tag{Hypothesis}\\
            &=((a\circ{b})\circ{a})*
                \big((a\circ{b})\circ{b}^{\minus{1}}\big)
            \tag{Distributivity}\\
            &=\big((a\circ{b})\circ{a}\big)*
                \big(a\circ(b\circ{b}^{\minus{1}})\big)
            \tag{Associativity}\\
            &=\big((a\circ{b})\circ{a}\big)*(a\circ{e}_{*})
            \tag{Complement}\\
            &=\big((a\circ{b})\circ{a}\big)*e_{*}
            \tag{Thm.~\ref{thm:Bool_Alg_Boundary_of_Circ}}\\
            &=(a\circ{b})\circ{a}
            \tag{Identity}\\
            &=(a\circ{a})\circ{b}
            \tag{Associativity and Commutativity}\\
            &=a\circ{b}
            \tag{Thm~\ref{thm:Bool_Alg_Idempotent_of_Circ}}
        \end{align}
        Thus $a=a\circ{b}$. But $a\circ{b}=a*{b}$, and so $a=a*{b}$. By
        Thm.~\ref{thm:Bool_Alg_aob_equal_a_acb_equal_a_implies_a_equal_b},
        $a=b$.
    \end{proof}
            \begin{definition} For $a\in S$, an inverse, or normal inverse, of the First Operation is an element $b\in S$ such that $a\circ b=e_{\circ}$. An inverse of the Second Operation is similarly defined. The normal inverses are denoted $a^{*}$ and $a^{\circ}$.
            \end{definition}
            \begin{theorem} If $a\in S$ has a normal inverse for either operation, than it is unique.
            \end{theorem}
            \begin{proof} For suppose not. Let $a\in S$ have a normal inverse for the First Operation. That is, there is an $a^{\circ}\in S$ such that $a\circ a^{\circ}=e_{\circ}$ and let $a'^{\circ}$ be a second normal inverse not equal to the first. But then $a^{\circ}=a^{\circ}\circ e_{\circ}=a^{\circ}\circ (a\circ a'^{\circ})$ and from associativity we have $a^{\circ}=(a^{\circ}\circ a)\circ a'^{\circ}=a'^{\circ}$. Thus, the normal inverse is unique. Similarly if there is an inverse for the Second Operation
            \end{proof}
            \begin{theorem} If $a\in S$ has a normal inverse, say $a'$, for one operation, then $a^{-1}=a'^{-1}$.
            \end{theorem}
            \begin{proof} For let $a\in S$ have a normal inverse $a'$ for the First Operation. That is, $a\circ a' = e_{\circ}$. But $a' \circ a'^{-1}=e_{*}$, and from theorem 1.3 $a\circ e_{*}=e_{*}$. So $a\circ (a' \circ a'^{-1})=e_{*}$. And from theorem 1.4, $a\circ a=a$, so we have $(a\circ a)\circ (a'\circ a'^{-1}=a\circ (a\circ a')\circ a'^{-1}=a\circ a'^{-1}=e_{*}$. But $a\circ a^{-1}=e_{\circ}$. And pseudo-inverses are unique. Thus, $a^{-1}=a'^{-1}$. 
            \end{proof}
            \begin{theorem} The identities have normal inverses for their respective operations.
            \end{theorem}
            \begin{proof} As normal inverses are unique, it suffices to find inverses for both identities. But $e_{\circ}\circ e_{\circ}=e_{\circ}$, so $e_{\circ}$ is its own inverse for the First Operation. Similarly, $e_{*}*e_{*}=e_{*}$.
            \end{proof}
            \begin{theorem} \textbf{(The Not-A-Field Theorem)} Only the identities have normal inverses.
            \end{theorem}
            \begin{proof} For suppose not. Suppose $a\in S,\ a\ne e_{\circ},\ a\ne e_{*}$ and a has an inverse for the First Operation. That is $\exists a^{\circ}\in S|\ a\circ a^{\circ}=e_{\circ}$. But by theorem 1.4, $a\circ a^{\circ}=(a\circ a)\circ a^{\circ}$. By associativity, we have $e_{\circ}=a\circ a^{\circ} = a\circ (a\circ a^{\circ})=a\circ e_{\circ}=a$. Thus, $a=e_{\circ}$. But by hypothesis, $a\ne e_{\circ}$. Thus, there is no inverse for $a$. Similarly, a has no inverse for the Second Operation.
            \end{proof}
            \begin{theorem}
            There exist pseudo-fields with only one element.
            \end{theorem}
            \begin{proof}
            For let $e_{\circ} = e_{*}$, and let no other elements be in the set. 
            \end{proof}
            \begin{theorem}
            A pseud-field has one element if and only if $e_{\circ} = e_{*}$.
            \end{theorem}
            \begin{proof}
            For suppose there is another element $a \ne e_{\circ}$. But then $a \circ e_{\circ} = a$, but also $a \circ e_{\circ} = a \circ e_{*} = e_{*}$. So $a = e_{*}$. If there is only one element, then clearly $e_{\circ} = e_{*}$ as otherwise there would be two elements.
            \end{proof}
            \begin{definition} A generating set on a pseudo-field is a subset $g_S \subset S$ such that every element of $S$ can be written as a finite combination of elements in $g_S$ using $\circ$ or $*$.
            \end{definition}
            \begin{theorem}
            The number of elements in a finite pseudo-field is a power of 2.
            \end{theorem}
            \begin{proof}
            Consider the set of all generators $g_S$ on $S$. Clearly for all such generators, $1\leq |g_S|\leq |S|$. Let $G$ be the smallest generator, such that $|G| \leq |g_S|$ for any other given generator. 
            \end{proof}
        \section{Sequences and Matrices}
    Matrices\index{Matrix} are the fundamental object studied in
    linear algebra\index{Linear Algebra}, and are used in the study of general
    algebra as well. To discuss the more interesting properties requires some
    notion of arithmetic that we do not yet posses. In particular, matrices are
    most interesting when there is an underlying \textit{ring}\index{Ring}
    structure. For now we simply introduce the set theoretic definition of a
    matrix, relate this to the familiar \textit{grid of numbers} definition, and
    provide examples.
    \begin{fdefinition}{Matrix}
        An $n\times{m}$, $n,m\in\mathbb{N}$, matrix on a set $X$ is a function
        $A:\mathbb{Z}_{n}\times\mathbb{Z}_{m}\rightarrow{X}$.
    \end{fdefinition}
    \chapter{The Real Numbers}
        Now that the notion of function has been developed, and most of ZFC laid
        out, we take the time to discuss the next axiom on the list:
        \textit{the axiom of infinity}\index{Axiom!of Infinity}. This axiom,
        while controversial, posits the existence of an infinite set and allows
        us to construct the natural numbers and the real numbers in the
        framework of set theory.
    \chapter{Arithmetic}
        \section{Elementary Number Theory}
    \begin{theorem}
        \label{thm:Equiv_Classes_Form_Partition}%
        If $A$ is a set, if $R$ is an equivalence relation on $A$, and
        if $A/R$ is the quotient set, then $A/R$ is a partition of $A$.
    \end{theorem}
    \begin{proof}
        For all $x\in{A}$ it is true that $[x]\in{A}/R$ and since $R$ is
        an equivalence relation we have $x\in[x]$, hence $A/R$ covers
        $A$. Moreover, if $\mathcal{U},\mathcal{V}\in{A}/R$ and if
        $\mathcal{U}\cap\mathcal{V}$ is non-empty, then there is an
        $x\in{R}$ such that $[x]\in\mathcal{U}$ and $[x]\in\mathcal{V}$.
        But if $[y]\in\mathcal{U}$ and $[z]\in\mathcal{V}$, then
        $yRx$ and $xRz$. But then $yRz$ since $R$ is an equivalence
        relation and therefore $[y]\in\mathcal{V}$. Similarly,
        $[z]\in\mathcal{U}$. Hence, either $\mathcal{U}=\mathcal{V}$ or
        they are disjoint. Therefore, $A/R$ is a partition of $A$.
    \end{proof}
    \begin{theorem}
        If $A$ is a set, and if $\mathcal{O}\subseteq\powset{X}$ is a
        partition of $A$, then there is an equivalence relation $R$ on
        $A$ such that $\mathcal{O}=A/R$.
    \end{theorem}
    \begin{proof}
        For let $R\subseteq{A}\times{A}$ be defined by:
        \begin{equation}
            R=\{\,(x,y)\in{A}\times{A}\;|\;
                \exists_{\mathcal{U}\in\mathcal{O}}
                (x,y\in\mathcal{U})\,\}
        \end{equation}
        then $R$ is an equivalence relation. Since $\mathcal{O}$ is a
        partition, for all $x\in{A}$ there is a
        $\mathcal{U}\in\mathcal{O}$ such that $x\in\mathcal{U}$. But
        then $x\in\mathcal{U}$ and $x\in\mathcal{U}$, and therefore
        $(x,x)\in{R}$. That is, $xRx$. Moreover, if $xRy$ then there is
        a $\mathcal{U}\in\mathcal{O}$ such that $x\in\mathcal{U}$ and
        $y\in\mathcal{U}$. But then $y\in\mathcal{U}$ and
        $x\in\mathcal{U}$ and hence $yRx$. Lastly, if $xRy$ and $yRz$,
        then there is a set $\mathcal{U}\in\mathcal{O}$ and a set
        $\mathcal{V}\in\mathcal{O}$ such that $x,y\in\mathcal{U}$ and
        $y,z\in\mathcal{V}$. But $\mathcal{O}$ is a partition, and hence
        either $\mathcal{U}\cap\mathcal{V}=\emptyset$ or
        $\mathcal{U}=\mathcal{V}$. But
        $\mathcal{U}\cap\mathcal{V}\ne\emptyset$ since $y\in\mathcal{U}$
        and $y\in\mathcal{V}$. Therefore, $\mathcal{U}=\mathcal{V}$ and
        thus, since $z\in\mathcal{V}$, it is true that
        $z\in\mathcal{U}$. That is, $xRz$. Hence, $R$ is an equivalence
        relation. Moreover, by definition, $A/R=\mathcal{O}$.
    \end{proof}
    \begin{theorem}
        \label{thm:Fibers_of_Func_Form_Equiv_Relation}%
        If $A$ and $B$ are sets, if $f:A\rightarrow{B}$ is a function,
        and if $R\subseteq{A}\times{A}$ is the relation defined by:
        \begin{equation}
            R=\{\,(x,y)\in{A}\times{A}\;|\;f(x)=f(y)\,\}
        \end{equation}
        then $R$ is an equivalence relation on $A$.
    \end{theorem}
    \begin{proof}
        For all $x\in{A}$ it is true that $f(x)=f(x)$, and hence $xRx$.
        Moreover, if $x,y\in{A}$ and $xRy$, then $f(x)=f(y)$. But
        equality is reflexive, and hence $f(y)=f(x)$. But then $yRx$.
        Lastly, by the transitivity of equality, if $xRy$ and $yRz$,
        then $f(x)=f(y)$ and $f(y)=f(z)$, hence $f(x)=f(z)$. But then
        $xRz$. Thus, $R$ is and equivalence relation.
    \end{proof}
    \begin{theorem}
        \label{thm:Weak_Euc_Division_Alg}%
        If $n,m\in\mathbb{N}^{+}$, then there exists $q,r\in\mathbb{N}$
        such that $n=q\cdot{m}+r$ with $r<m$.
    \end{theorem}
    \begin{proof}
        For let $G\subseteq\mathbb{N}$ be defined by:
        \begin{equation}
            G=\{\,k\in\mathbb{N}\;|\;n-k\cdot{m}\geq{0}\,\}
        \end{equation}
        if $n<m$, then $G=\{0\}$ and hence choosing $r=n$ and $q=0$
        works. If $n=m$, $q=1$ and $r=0$ does the trick. Otherwise, $G$
        is non-empty and bounded since it is bounded by $n$, and hence
        there is a greatest element. Let $q\in{G}$ be the greatest
        element, and let $r=n-q\cdot{m}$. Since $q\in{G}$, $r>0$.
        Moreover, $r<m$. For if not, then
        $r-m=n-q\cdot{m}-m=n-(q+1)\cdot{m}\geq{0}$, which contradicts
        the maximality of $q$. But then $n=kq+r$ and $r<m$.
    \end{proof}
    There's a strengthening of this theorem, which dates back to
    antiquity.
    \begin{ftheorem}{Euclid's Division Algorithm}
                    {Euclid_Division_Algorithm}
        If $m,n\in\mathbb{Z}$, and if $m\ne{0}$, then there exists
        unique $q\in\mathbb{Z}$ and $r\in\mathbb{N}$ such that $r<|m|$
        and:
        \begin{equation*}
            n=mq+r
        \end{equation*}
    \end{ftheorem}
    \begin{bproof}
        For if $n=0$, then let $r=0$ and $q=0$. If $n>0$, and if $m>0$,
        then by Thm.~\ref{thm:Weak_Euc_Division_Alg} there exists
        $q,r$ such that $n=q\cdot{m}+r$ with $r<m$. If $n>0$ and $m<0$,
        then $\minus{m}>0$. But then by
        Thm.~\ref{thm:Weak_Euc_Division_Alg} there exists $q,r$ such
        that $n=q\cdot(\minus{m})+r$ with $r<\minus{m}$. But then
        $n=(\minus{q})\cdot{m}+r$, and $r<|m|$. If $n<0$ and $m>0$,
        then $\minus{n}>0$ and hence by
        Thm.~\ref{thm:Weak_Euc_Division_Alg} there exists
        $q,r\in\mathbb{N}$ such that $r<m$ and $\minus{n}=q\cdot{m}+r$.
        But then:
        \begin{equation}
            n=(\minus{q})\cdot{m}-r=(\minus{q}-1)\cdot{m}+(m-r)
        \end{equation}
        but since $0\leq{r}<m$ we have $0\leq{m}-r<m$. Lastly, if $n<0$
        and $m<0$, then $\minus{n}>0$ and $\minus{m}>0$. But then by
        Thm.~\ref{thm:Weak_Euc_Division_Alg} there exists $q,r$ with
        $\minus{n}=q\cdot(\minus{m})+r$ and $r<\minus{m}$. But then
        $n=q\cdot{m}-r$, and hence $n=(q+1)\cdot{m}+(\minus{r}-m)$. But 
        $r<\minus{m}$, and therefore $0<\minus{r}-m$ and
        $\minus{r}-m<|m|$. Moreover, $q$ and $r$ are unique. For suppose
        $n=q_{0}\cdot{m}+r_{0}$ and $n=q_{1}\cdot{m}+r_{1}$. If
        $q_{0}\ne{q}_{1}$, then either $q_{0}<q_{1}$ or $q_{1}<q_{0}$.
        Suppose $q_{0}<q_{1}$ and let $k=q_{1}-q_{0}$. But then
        \begin{equation}
            n=q_{0}\cdot{m}+r_{0}=(q_{1}-k)\cdot{m}+r_{0}
                =q_{1}\cdot{m}+r_{0}-k\cdot{m}
                =q_{1}\cdot{m}+r_{1}
        \end{equation}
        and therefore $r_{1}=r_{0}-k\cdot{m}$, so
        $r_{0}=r_{1}+k\cdot{m}$, a contradiction since $r_{0}<m$. Hence,
        $q_{0}=q_{1}$. But then taking the difference we obtain
        $r_{0}=r_{1}$. Thus, they are unique.
    \end{bproof}
    \begin{fdefinition}{Divisor of an Integer}{Divisor_of_Integer}
        A divisor of an integer $n\in\mathbb{Z}$ is an integer
        $m\in\mathbb{Z}$ such that there exists and integer
        $k\in\mathbb{Z}$ where $k\cdot{m}=n$. We denote this by $m|n$.
    \end{fdefinition}
    \begin{theorem}
        \label{thm:One_is_Divisor}%
        If $n\in\mathbb{Z}$, then $1|n$.
    \end{theorem}
    \begin{proof}
        For $n=1\cdot{n}$, and hence $1$ is a divisor of $n$
        (Def.~\ref{def:Divisor_of_Integer}).
    \end{proof}
    \begin{theorem}
        \label{thm:Greater_is_not_divisor_of_lesser}%
        If $n,m\in\mathbb{N}^{+}$ and $m<n$, then $n$ does not divide
        $m$.
    \end{theorem}
    \begin{proof}
        For suppose not. If $n|m$, then there exists $k\in\mathbb{Z}$
        such that $m=k\cdot{n}$ (Def.~\ref{def:Divisor_of_Integer}). But
        $n,m\in\mathbb{N}^{+}$ and therefore $n>0$ and $m>0$. But
        $m=k\cdot{n}$, and hence $k>0$. But then
        $m=k\cdot{n}\leq{1}\cdot{n}=n$, a contradiction since $m<n$.
    \end{proof}
    Well ordering of $\mathbb{N}$. Well ordering of countable set.
    \begin{fdefinition}{Greatest Common Divisor}{GCD}
        A greatest common divisor of two integers
        $n,m\in\mathbb{Z}\setminus\{0\}$ is an integer
        $k\in\mathbb{N}^{+}$ such that $k|n$, $k|m$, and for all
        $j\in\mathbb{N}^{+}$ where $j|n$ and $j|m$ it is true that
        $j\leq{k}$.
    \end{fdefinition}
    We say \textit{a} greatest divisor since we don't know
    \textit{a priori} if there are many, or if there is one at all.
    \begin{theorem}
        \label{thm:GCD_Existence_Theorem}%
        If $n,m\in\mathbb{Z}\setminus\{0\}$, then there exists a
        greatest common divisor.
    \end{theorem}
    \begin{proof}
        For let $G\subseteq\mathbb{N}^{+}$ be defined by:
        \begin{equation}
            G=\{\,k\in\mathbb{N}^{+}\;|\;
                k\textrm{ divides }n\textrm{ and }
                k\textrm{ divides }m\,\}
        \end{equation}
        Then $G$ is non-empty since $1\in{G}$. That is, $1|n$ and $1|m$
        (Thm.~\ref{thm:One_is_Divisor}). If $k\in{G}$, then
        $k$ divides $|n|$ and $|m|$. But since
        $n,m\in\mathbb{Z}\setminus\{0\}$ it is true that
        $|n|,|m|\in\mathbb{N}^{+}$. But then if $k$ divides $|n|$ and $|m|$,
        then $k<|n|$ and $k<|m|$
        (Thm.~\ref{thm:Greater_is_not_divisor_of_lesser}). Hence, $G$ is
        bounded above. But then there is a greatest element $k\in{G}$.
        But then $k|m$ and $k|n$, and for all $j\in\mathbb{N}^{+}$ such
        that $j|m$ and $j|n$ it is true that $j\leq{k}$. Hence, $k$ is
        a greatest common divisor (Def.~\ref{def:GCD}).
    \end{proof}
    \begin{theorem}
        \label{thm:GCD_Unique}%
        If $n,m\in\mathbb{Z}\setminus\{0\}$, if $k$ is a greatest common
        divisor of $n$ and $m$, and if $j$ is a greatest common
        divisor of $n$ and $m$, then $k=j$.
    \end{theorem}
    \begin{proof}
        For if not, then either $j<k$ or $k<j$. But if $j<k$ then $j$ is
        not the greatest common divisor since there exists an element
        $k\in\mathbb{N}^{+}$ such that $k|m$, $k|n$, and such that
        $j<k$, a contradiction (Def.~\ref{def:GCD}). Similarly
        $k\not<j$, and hence $j=k$.
    \end{proof}
    Now that we know there's a unique greatest common divisor, we
    use the following notation:
    \begin{fnotation}{Greatest Common Divisor}{GCD}
        The greatest common divisor of $n,m\in\mathbb{Z}\setminus\{0\}$
        is denoted $\GCD(n,m)$.
    \end{fnotation}
    The algorithmic way to go about showing there exists a unique
    greatest common divisor is to apply Euclid's algorithm. We perform
    division with remainder repeatedly until we're left with no
    remaining term. Suppose we're given $n,m\in\mathbb{N}^{+}$ and want
    to compute $\GCD(n,m)$. We do:
    \par
    \begin{subequations}
        \begin{minipage}[b]{0.49\textwidth}
            \centering
            \begin{align}
                n&=q_{0}m+r_{0}\\
                m&=q_{1}r_{0}+r_{2}\\
                r_{0}&=q_{2}r_{1}+r_{2}
            \end{align}
        \end{minipage}
        \hfill
        \begin{minipage}[b]{0.49\textwidth}
            \centering
            \begin{align}
                r_{n-3}&=q_{n-1}r_{n-2}+r_{n-1}\\
                r_{n-2}&=q_{n}r_{n-1}+r_{n}\\
                r_{n-1}&=q_{n+1}r_{n}
            \end{align}
        \end{minipage}
    \end{subequations}
    \par\vspace{2.5ex}
    The $\GCD$ is the last non-zero remainder term.
    \begin{example}
        Let's compute the $\GCD$ of $34$ and $51$. We have:
        \twocolumneq{34=1\cdot{34}+17}{34=2\cdot{17}}
        and so $\GCD(34,51)=17$. Perhaps bigger numbers will better
        demonstrate the algorithm. Let's compute $\GCD(57970,10353)$. We
        obtain:
        \par
        \begin{subequations}
            \begin{minipage}[b]{0.49\textwidth}
                \centering
                \begin{align}
                    57970&=5\cdot{10353}+6205\\
                    10353&=1\cdot{6205}+4148\\
                    6205&=1\cdot{4148}+2057
                \end{align}
            \end{minipage}
            \hfill
            \begin{minipage}[b]{0.49\textwidth}
                \centering
                \begin{align}
                    4148&=2\cdot{2057}+34\\
                    2057&=60\cdot{34}+17\\
                    34&=2\cdot{17}
                \end{align}
            \end{minipage}
        \end{subequations}
        \par\vspace{2.5ex}
        after running the gauntlet, the last equation has zero
        remainder: $34=2\cdot{17}+0$ and hence $\GCD(57970,10353)=17$.
    \end{example}
    \begin{fdefinition}{Least Common Multiple}{LCM}
        A least common multiple of integers
        $n,m\in\mathbb{Z}\setminus\{0\}$ is an integer
        $k\in\mathbb{N}^{+}$ where $n|k$, $m|k$, and for all
        $j\in\mathbb{N}^{+}$ with $n|k$ and $m|k$ it is true that
        $k\leq{j}$.
    \end{fdefinition}
    Again, we use the phrasing \textit{a} least common multiple since
    the existence and uniqueness of such a concept has not yet been
    proved. We now perform this task.
    \begin{theorem}
        \label{thm:Integer_Divides_Multiple_of_Self}%
        If $n,m\in\mathbb{Z}$, if $k=n\cdot{m}$, then $n$ divides $k$.
    \end{theorem}
    \begin{proof}
        For by hypothesis there exists an $m\in\mathbb{Z}$ such that
        $n\cdot{m}=k$ and hence $n$ divides $k$
        (Def.~\ref{def:Divisor_of_Integer}).
    \end{proof}
    \begin{theorem}
        \label{thm:LCM_Existence_Theorem}%
        If $n,m\in\mathbb{Z}\setminus\{0\}$, then there is a least
        common multiple of $n$ and $m$.
    \end{theorem}
    \begin{proof}
        For let $G\subseteq\mathbb{N}^{+}$ be defined by:
        \begin{equation}
            G=\{\,k\in\mathbb{N}^{+}\;|\;
                n\textit{ divides }k\textrm{ and }
                m\textit{ divides }k\,\}
        \end{equation}
        Then $G$ is non-empty since $|n|\cdot|m|\in{G}$
        (Thm.~\ref{thm:Integer_Divides_Multiple_of_Self}). But if $G$ is
        a non-empty subset of $\mathbb{N}^{+}$, then there is a least
        element $k\in{G}$. But then $n$ divides $k$, $m$ divides $k$,
        and for all $j$ with $n|j$ and $m|j$ it is true that
        $k\leq{j}$. Hence, $k$ is a least common multiple of $n$ and $m$
        (Def.~\ref{def:LCM}).
    \end{proof}
    Much like the greatest common divisor, we now prove that least
    common multiples are unique and then assign a notation to the
    concept.
    \begin{theorem}
        \label{thm:LCM_Unique}%
        If $n,m\in\mathbb{Z}^{+}$, if $k$ is a least common multiple of
        $n$ and $m$, and if $j$ is a least common multiple of $n$ and
        $m$, then $k=j$.
    \end{theorem}
    \begin{proof}
        For if not, then either $j<k$ or $k<j$, violating minimality.
    \end{proof}
    \begin{fnotation}{Least Common Multiple}{LCM}
        The least common multiple of $n,m\in\mathbb{Z}\setminus\{0\}$ is
        denoted $\LCM(n,m)$.
    \end{fnotation}
    One of the most useful theorems in number theory is from the French
    mathematician \'{E}tienne B\'{e}zout. The theorem, known as
    B\'{e}zout's identity, was proved in the $18^{th}$ century in the
    setting of polynomials, but when applied to integers it allows to us
    make many theorems of great historical importance into short
    corollaries. For example, as we will show, the effort in proving
    Euclid's prime number lemma and the Chinese remainder theorem are
    greatly diminshed by applying the identity. Later, in the context of
    \textit{rings}, a generalization will be given.
    \begin{ftheorem}{B\'{e}zout's Identity}{Bezout_Identity}
        If $a,b\in\mathbb{N}\setminus\{0\}$, and if $d=\GCD(a,b)$, then
        there exist $n,m\in\mathbb{Z}$ such that:
        \begin{equation*}
            a\cdot{n}+b\cdot{m}=d
        \end{equation*}
    \end{ftheorem}
    \begin{bproof}
        For let $G\subseteq\mathbb{Z}\setminus\{0\}$ be defined by:
        \begin{equation}
            G=\{\,k\in\mathbb{N}\setminus\{0\}\;|\;
                \textrm{ There exists }n,m\in\mathbb{Z}
                \textrm{ such that }k=an+bm\big)\,\}
        \end{equation}
        then $G$ is non-empty since $a^{2}+b^{2}\in{G}$. But then there
        is a least element $d\in{G}$. By Euclid's division algorithm,
        there exists $q\in\mathbb{Z}$ and $r\in\mathbb{N}$ such that
        $r<d$ and $a=q\cdot{d}+r$
        (Thm.~\ref{thm:Euclid_Division_Algorithm}). But $d\in{G}$, and
        hence there are $n,m\in\mathbb{Z}$ such that
        $d=a\cdot{n}+b\cdot{m}$. But then:
        \begin{equation}
            r=a-qd=a-q(na+mb)=a-aqn-bqm=a(1-qn)+b\cdot(\minus{q}m)
        \end{equation}
        But if $r>0$, then $r\in{G}$ and $r<d$, a contradiction since
        $d$ is the least element of $G$. Hence, $r=0$ and $d$ divides
        $a$ (Def.~\ref{def:Divisor_of_Integer}). Similarly, $d$ divides
        $b$. If $c\in\mathbb{N}^{+}$ is such that $c|a$ and $c|b$, then
        there exists $s,t\in\mathbb{Z}$ such that $a=c\cdot{s}$ and
        $b=c\cdot{t}$. But then:
        \begin{equation}
            d=a\cdot{n}+b\cdot{m}=(c\cdot{s})n+(c\cdot{t})m=c(sn+tm)
        \end{equation}
        and therefore $d$ divides $c$. But it $d$ divides $c$, then
        $d$ is not greater than $c$
        (Thm.~\ref{thm:Greater_is_not_divisor_of_lesser}), and therefore
        $c\leq{d}$. Thus, $d$ is the greatest common divisor
        (Def.~\ref{def:GCD}).
    \end{bproof}
    \begin{theorem}
        \label{thm:Divisor_of_AB_Divides_GCD}%
        If $a,b,n\in\mathbb{Z}\setminus\{0\}$, and if $n|a$ and $n|b$,
        then $n|\GCD(a,b)$.
    \end{theorem}
    \begin{proof}
        By B\'{e}zout's identity there exists $s,t\in\mathbb{Z}$ such
        that $as+bt=\GCD(a,b)$ (Thm.~\ref{thm:Bezout_Identity}). But
        $n|a$ and $n|b$, and thus there exists $u,v\in\mathbb{Z}$ such
        that $nu=a$ and $nv=b$ (Def.~\ref{def:Divisor_of_Integer}). But
        then $n(us+vt)=\GCD(a,b)$. Therefore, $n$ divides $\GCD(a,b)$
        (Def.~\ref{def:Divisor_of_Integer}).
    \end{proof}
    \begin{theorem}
        \label{thm:n_Div_AB_then_A_Div_AS_BT}%
        If $a,b,s,t,n\in\mathbb{Z}$, if $n|a$, and if $n|b$, then $a$
        divides $as+bt$.
    \end{theorem}
    \begin{proof}
        For if $n$ divides $a$ and $b$, then there exists
        $u,v\in\mathbb{Z}$ such that $nu=a$ and $nv=b$
        (Def.~\ref{def:Divisor_of_Integer}). But then
        $as+bt=nus+nvt=n(us+bt)$, and thus $n$ divides $as+bt$
        (Def.~\ref{def:Divisor_of_Integer}).
    \end{proof}
    As claimed, many great theorems from antiquity become corollaries of
    B\'{e}zout's identity. We now demonstrate this.
    \begin{ftheorem}{Euclid's Prime Number Lemma}
                    {Euclid_Prime_Number_Lemma}
        If $a,b,p\in\mathbb{Z}\setminus\{0\}$, if $\GCD(a,p)=1$, and if
        $p$ divides $a\cdot{b}$, then $p$ divides $b$.
    \end{ftheorem}
    \begin{bproof}
        Since $\GCD(a,p)=1$, by B\'{e}zout's identity there exist
        integers $n,m\in\mathbb{Z}$ such that $an+pm=1$
        (Thm.~\ref{thm:Bezout_Identity}). But then $ban+bpm=b$. But $p$
        divides $ab$ and hence there is a $k\in\mathbb{Z}$ such that
        $kp=ab$ (Def.~\ref{def:Divisor_of_Integer}) and therefore
        $kpn+bpm=b$. But then $p(kn+bm)=b$, and hence $p$ divides $b$
        (Def.~\ref{def:Divisor_of_Integer}).
    \end{bproof}
    Euclid's phrasing of this theorem, as presented in his
    \textit{elements}, goes as follows:
    \begin{theorem}
        \label{thm:Prime_Div_AB_then_PdivA_or_PdivB}%
        If $a,b\in\mathbb{N}^{+}$, if $p\in\mathbb{N}^{+}$ is prime, and
        if $p$ divides $a\cdot{b}$, then either $p$ divides $a$ or $p$
        divides $b$.
    \end{theorem}
    \begin{proof}
        For if $p$ does not divide $a$, then $\GCD(a,p)=1$, and thus
        $p$ divides $b$ (Thm.~\ref{thm:Euclid_Prime_Number_Lemma}).
        Similary, if $p$ does not divide $b$ then it divides $a$.
    \end{proof}
    \begin{theorem}
        If $m,n\in\mathbb{N}^{+}$, if $l$ is the least common multiple of
        $n$ and $m$, and if $d$ is the greatest common divisor of $n$ and
        $m$, then $l\cdot{d}=n\cdot{m}$.
    \end{theorem}
    \begin{ftheorem}{Fundamental Theorem of Arithmetic}
                    {Fundamental_Theorem_of_Arithmetic}
        If $n\in\mathbb{N}^{+}$, then there exists a unique
        $n\in\mathbb{N}^{+}$, a unique strictly increasing sequence
        $P:\mathbb{Z}_{n}\rightarrow\mathbb{N}^{+}$ and a unique
        sequence $N:\mathbb{Z}_{n}\rightarrow\mathbb{N}^{+}$ such that
        for all $k\in\mathbb{Z}_{n}$ it is true that $P_{k}$ is prime
        and:
        \begin{equation*}
            n=\prod_{k\in\mathbb{Z}_{n}}P_{k}^{N_{k}}
        \end{equation*}
    \end{ftheorem}
    That is, every integer has a unique prime factorization. If $n$ is a
    prime, the factorization is simply
    $P:\mathbb{Z}_{1}\rightarrow\mathbb{N}^{+}$ defined by $P_{0}=n$ and
    $N:\mathbb{Z}_{1}\rightarrow\mathbb{N}^{+}$ with $N_{0}=1$.
    \begin{theorem}
        \label{thm:Composite_N_Exists_AB_N_Div_AB_and_N_NDiv_A_or_B}%
        If $n\in\mathbb{N}^{+}$ is not prime, then there exists
        integers $a,b\in\mathbb{N}^{+}$ such that $n$ divides
        $ab$, but $n$ does not divide $a$ and $n$ does not divide $b$.
    \end{theorem}
    \begin{proof}
        For of $n$ is composite, then there is a prime $p$ that divides
        $n$. Let $Q=n/p$. By the fundamental theorem of arithmetic,
        there exists a subset
        $S\subseteq\mathbb{N}^{+}\times\mathbb{N}^{+}$ such that for all
        $(q,n)\in{S}$ it is true that $q$ is prime and:
        \begin{equation}
            Q=\prod_{(p,n)\in{S}}q^{n}
        \end{equation}
        Since there are infinitely many primes, there is a prime $P$ not
        contained in the projection of $S$ into the first variable. Let
        $a=p$ and $b=Q\cdot{P}$. Then $n$ does not divide $a$ or $b$,
        but it does divide $a\cdot{b}$.
    \end{proof}
    \begin{example}
        The smallest example we have to work with is 4, and so we direct
        our attention there. 4 divides 12, and $12=4\cdot{3}$. Following
        the proof of
        Thm.~\ref{thm:Composite_N_Exists_AB_N_Div_AB_and_N_NDiv_A_or_B},
        we remove a prime from 4 and are left with 2. We then pick a
        prime that is not in the prime factorization of 4 and multiply
        by this. That is, we have $p=2$, $Q=2$, and $P=3$. We set
        $a=p=2$ and $b=Q\cdot{P}=2\cdot{3}=6$. Then 4 does not divide 2
        since $2<4$ and moreover 4 does not divide 6, but it does divide
        $2\cdot{6}=12$.
    \end{example}
    \begin{ftheorem}{Chinese Remainder Theorem}
                    {Chinese_Remainder_Theorem}
        If $n\in\mathbb{N}^{+}$ is an integer, if
        $N:\mathbb{Z}_{n}\rightarrow\mathbb{N}$ is such that for all
        $i,j\in\mathbb{Z}_{n}$ with $i\ne{j}$ it is true that
        $\GCD(N_{i},N_{j})=1$, and if
        $A:\mathbb{Z}_{n}\rightarrow\mathbb{N}$ is a sequence of
        integers such that $A_{i}<N_{i}$ for all $i\in\mathbb{Z}_{n}$,
        then there is a unique $x\in\mathbb{N}$ such that for all
        $i\in\mathbb{Z}_{n}$ it is true that $x\equiv{A}_{i}\mod{N}_{i}$
        and $x<\prod_{k\in\mathbb{Z}_{n}}N_{k}$
    \end{ftheorem}
    \begin{bproof}
        By induction. The base case, since $N_{1}$ and $N_{2}$ are
        relatively prime, by B\'{e}zout's identity there exist integers
        $m_{1},m_{2}\in\mathbb{Z}$ such that $m_{1}N_{1}+m_{2}N_{2}=1$.
        Let $x=m_{2}N_{2}A_{1}+m_{1}N_{1}A_{2}$. In the induction case,
        there is an $x$ such that $x\equiv{A}_{i}\mod{N}_{i}$ for
        $i\in\mathbb{Z}_{n-1}$ and $x\equiv{A}_{n}A_{n+1}\mod{N}_{n}N_{n+1}$
        since $N_{n}$ and $N_{n+1}$ are relatively prime.
    \end{bproof}
    We can get some use out of this by studying the Euler totient
    function.
    \begin{theorem}
        \label{thm:Rel_Prime_to_Prod_of_Rel_Primes}%
        If $n,m\in\mathbb{N}$, if $\GCD(n,m)=1$, if $k\in\mathbb{N}$,
        and if $k<mn$, then $\GCD(k,n\cdot{m})=1$ if and only if
        $\GCD(k,n)=1$ and $\GCD(k.m)=1$
    \end{theorem}
    \begin{proof}
        Somehow the Chinese remainder theorem does this. Moving on for
        now.
    \end{proof}
    $S_{1}\subseteq{S}_{2}$, we see that
    $\varphi(a)$ divides $\varphi(b)$.  \begin{fdefinition}{Euler Totient Function}{Euler_Totient_Func}
        The Euler totient function is the function
        $\varphi:\mathbb{N}^{+}\rightarrow\mathbb{N}^{+}$ defined by:
        \begin{equation*}
            \varphi(n)=\cardinality{\{\,k\in\mathbb{N}\;|\;
                k\leq{n}\textrm{ and }\GCD(k,n)=1\,\}}
        \end{equation*}
    \end{fdefinition}
    Since there are so many different uses of the symbol $\varphi$, when
    a theorem pertains to the Euler totient function we will explicitly
    say so.
    \begin{theorem}
        \label{thm:Euler_Totient_of_Prime}%
        If $p\in\mathbb{N}$ is a prime number and if $\varphi$ is the
        Euler totient function, then $\varphi(p)=p-1$.
    \end{theorem}
    \begin{proof}
        For all $k\in\mathbb{N}^{+}$ such that $k<p$ it is true that
        $\GCD(k,p)=1$ since $p$ is prime. Hence, $\varphi(p)$ is equal
        to $\cardinality{\mathbb{Z}_{p}\setminus\{0\}}$ which  is $p-1$.
    \end{proof}
    \begin{theorem}
        \label{thm:Euler_Totient_Multiplicative}%
        If $a,b\in\mathbb{N}$, if $\varphi$ is the Euler totient
        function, if $\GCD(a,b)=1$, then:
        \begin{equation}
            \varphi(a\cdot{b})=\varphi(a)\cdot\varphi(b)
        \end{equation}
    \end{theorem}
    \begin{proof}
        For since $\GCD(n,m)=1$, for all $k<mn$ it is true that
        $\GCD(k,mn)=1$ if and only if $\GCD(k,m)=1$ and $\GCD(k,n)=1$
        (Thm.~\ref{thm:Rel_Prime_to_Prod_of_Rel_Primes}). Hence, there
        are $\varphi(a)\cdot\varphi(b)$ such elements.
    \end{proof}
    \begin{theorem}
        \label{thm:Euler_Totient_Powers_of_Primes}%
        If $p\in\mathbb{N}$ is prime, if $\varphi$ is the Euler totient
        function, and if $n\in\mathbb{N}^{+}$, then
        $\varphi(p^{n})=p^{n-1}(p-1)$.
    \end{theorem}
    \begin{proof}
        For if $p$ is prime, and if $m\in\mathbb{Z}_{p}\setminus\{0\}$,
        then since the only factors of $p^{n}$ are powers of $p$,
        $\GCD(p^{n},m)=p^{k}$ for some $k\in\mathbb{Z}_{n}$. There are
        $p\cdot{p}^{n-1}$ elements that are multiples of $p$, and hence
        $p^{n}-p^{n-1}$ elements that are coprime. Hence,
        $\varphi(p^{n})=p^{n}-p^{n-1}=p^{n-1}(p-1)$.
    \end{proof}
    We can combine the fundamental theorem of arithmetic together with
    these theorems to quickly compute the Euler totient function of a
    given value.
    \begin{example}
        The prime factorization of 12 is $2^{2}\cdot{3}$. Thus, we can
        compute $\varphi$ as follows:
        \begin{equation}
            \varphi(12)=\varphi(2^{2}\cdot{3})
                =\varphi(2^{2})\cdot\varphi(3)
                =2^{2-1}(2-1)\cdot(3-1)=2\cdot{1}\cdot{2}=4
        \end{equation}
        we can also just count out the relatively prime elements of
        $\mathbb{N}$ that are less than 12, and we obtain 1, 5, 7, and
        11. Powers of primes are particularly easy:
        \begin{equation}
            \varphi(16)=\varphi(2^{4})=2^{4-1}(2-1)=2^{3}=8
        \end{equation}
        the relatively prime elements are 1, 3, 5, 7, 9, 11, 13, and 15.
        That is, all of the odd numbers less than 16. Lastly, let's try
        $\varphi(75)$. We obtain:
        \begin{equation}
            \varphi(75)=\varphi(5^{2}*3)=5^{2-1}(5-1)\cdot(3-1)
                =5\cdot{4}\cdot{2}=40
        \end{equation}
    \end{example}
    \begin{theorem}
        \label{thm:SQRT_Primes_are_Irrational}%
        If $p\in\mathbb{N}^{+}$ is a prime number, then $\sqrt{p}$ is
        irrational.
    \end{theorem}
    \begin{proof}
        For suppose not. If $\sqrt{p}$ is rational, then there exists
        $a,b\in\mathbb{Z}$ such that $b\ne{0}$, $\GCD(a,b)=1$, and
        $\sqrt{p}=a/b$. But then $a^{2}=pb^{2}$ and thus $p$ divides
        $a\cdot{a}$ (Def.~\ref{def:Divisor_of_Integer}). Since $p$ is
        prime, by Euclid's prime number lemma $p$ divides $a$
        (Thm.~\ref{thm:Euclid_Prime_Number_Lemma}). But then there is a
        $k\in\mathbb{Z}$ such that $a=p\cdot{k}$. But then
        $a^{2}=p^{2}k^{2}=pb^{2}$, and hence $pk^{2}=b^{2}$. But then
        $p$ divides $b^{2}$ (Def.~\ref{def:Divisor_of_Integer}) and thus
        since $p$ is prime, $p$ divides $b$
        (Thm.~\ref{thm:Euclid_Prime_Number_Lemma}). But then $p$ divides
        $a$ and $b$, a contradiction since $\GCD(a,b)=1$ and $1<p$.
        Hence, $\sqrt{p}$ is irrational.
    \end{proof}
    Only finitely many $n$ have $\varphi(n)=N$ for a fixed $N$, where
    $\varphi$ is the Euler totient function.
    \begin{theorem}
        \label{thm:A_DIV_B_then_EulerTotA_Div_EulerTotB}%
        If $a,b\in\mathbb{N}^{+}$, if $a|b$, and if $\varphi$ is the
        Euler totient function, then $\varphi(a)$ divides $\varphi(b)$.
    \end{theorem}
    \begin{proof}
        For by the fundamental theorem of arithmetic
        (Thm.~\ref{thm:Fundamental_Theorem_of_Arithmetic}) there exist
        integers $m,n\in\mathbb{N}^{+}$, sequences
        $P,M:\mathbb{Z}_{m}\rightarrow\mathbb{N}^{+}$ and
        $Q,N:\mathbb{Z}_{n}\rightarrow\mathbb{N}^{+}$ such that for all
        $j\in\mathbb{Z}_{m}$ and $k\in\mathbb{Z}_{n}$, $P_{j}$ and
        $Q_{k}$ are prime, and:
        \twocolumneq{a=\prod_{j\in\mathbb{Z}_{m}}P_{j}^{M_{j}}}
                    {b=\prod_{k\in\mathbb{Z}_{n}}Q_{k}^{N_{k}}}
        But since $a$ divides $b$ there exists a $k\in\mathbb{Z}$ such
        that $b=a\cdot{k}$ (Def.~\ref{def:Divisor_of_Integer}). But
        since $P$ is a strictly increasing sequence and hence the
        $P_{j}$ are distinct, and since all of the $P_{j}$ are prime,
        all of the $P_{j}^{M_{j}}$ are relatively prime. Hence:
        \begin{align*}
            \varphi(a)&=\varphi\Big(
                \prod_{j\in\mathbb{Z}_{m}}P_{j}^{M_{j}}
            \Big)\\
            &=\prod_{j\in\mathbb{Z}_{m}}\varphi(P_{j}^{M_{j}})
                \tag{Thm.~\ref{thm:Euler_Totient_Multiplicative}}\\
            &=\prod_{j\in\mathbb{Z}_{m}}P_{j}^{N_{j}-1}(P_{j}-1)
                \tag{Thm.~\ref{thm:Euler_Totient_Powers_of_Primes}}
        \end{align*}
        and
        \begin{align*}
            \varphi(b)&=\varphi\Big(
                \prod_{k\in\mathbb{Z}_{n}}Q_{k}^{M_{k}}
            \Big)\\
            &=\prod_{k\in\mathbb{Z}_{n}}\varphi(Q_{k}^{M_{k}})
                \tag{Thm.~\ref{thm:Euler_Totient_Multiplicative}}\\
            &=\prod_{k\in\mathbb{Z}_{n}}Q_{k}^{N_{k}-1}(Q_{k}-1)
                \tag{Thm.~\ref{thm:Euler_Totient_Powers_of_Primes}}
        \end{align*}
        since $a$ divides $b$, and from the uniqueness of the sequences
        $P$ and $Q$, we see that $\varphi(a)$ divides $\varphi(b)$.
    \end{proof}
    \subsection{Exam I}
        \begin{problem}
            Find an integer $n$ such that $\gcd(n,4)=2$ and
            $\gcd(n,6)=3$, or prove that no such integer exists.
        \end{problem}
        \begin{proof}[Solution 1]
            If $\gcd(n,4)=2$, then ${2}\vert{n}$, and thus
            $\exists_{k\in\mathbb{Z}}:n=2k$. But
            $\gcd(n,6)=\gcd(2k,2\cdot 3)=2\gcd(k,3)$. But
            $\gcd(n,6)=3$, and therefore $2\gcd(k,3)=3$, a
            contradiction as $3$ is odd. No such $n$ exists.
        \end{proof}
        \begin{proof}[Solution 2]
            If $\gcd(n,4)=2$, then ${2}\vert{n}$, and thus
            $\exists_{j\in\mathbb{Z}}:n=2j$. If $\gcd(n,6)=3$,
            then ${3}\vert{n}$. Therefore
            $\exists_{k\in\mathbb{Z}}:n=3k$. But then $2j=3k$.
            As $3$ is odd, $k$ must be even. Therefore,
            $\exists_{m\in\mathbb{Z}}:k=2m$. But then
            $n=3k=3(2m)=6m$. Thus, ${6}\vert{n}$. But then
            $\gcd(n,6)=6$, a contradiction as $\gcd(n,6)=3$.
        \end{proof}
        \begin{proof}[Solution 3]
            If $\gcd(n,4)=2$, then ${2}\vert{n}$, and thus
            $\exists_{k\in\mathbb{Z}}:n=2k$. But $\gcd(n,6)=3$,
            and therefore $\exists_{x,y\in\mathbb{Z}}:nx+6y=3$.
            But $nx+6y=2kx+6y=2(kx+3y)$, and $nx+6y=3$, and
            therefore $2(nx+3y)=3$, a contradiction as $3$ is
            odd. No such $n$ exists.
        \end{proof}
        \begin{problem}
            Prove or disprove the following:
            \begin{enumerate}
                \begin{multicols}{2}
                    \item ${20}\vert{300}$
                    \item If $a>0$, then ${a}\vert{1}$
                    \item $\forall_{a,b>0}$, either
                        ${a}\vert{b}$ or ${b}\vert{a}$
                    \item $\forall_{a,b,c>0}$, if ${a}\vert{b}$
                        and ${a}\vert{(b+c)}$,
                        then ${a}\vert{(c-b)}$
                    \item $\forall_{a,b,c>0}$, if ${a}\vert{b}$
                        and ${a}\vert{c}$, then 
                        ${a}\vert{(b^{2}+c^{2})}$
                    \item $\forall_{a,b,c>0}$, if ${a}\vert{b}$
                        and $a\vert{(b^{2}+c^{2})}$, then
                        ${a}\vert{c}$
                    \item $\forall_{a,b,c>0}$, if ${a}\vert{b}$
                        and ${b}\vert{c}$, then $a\leq c$
                    \item If $a,b,c>0$, then
                        $\gcd(a,bc)\geq\gcd(a,b)$
                    \item If $a,b,c>0$, then
                        $\gcd(a,c-a)=\gcd(a+c,c)$
                    \item If $p$ is prime and
                        ${p^{3}}\vert{abc}$, then ${p}\vert{a}$
                    \item If $a+b$ is prime, then $ab$ is even.
                    \item If $a$ and $b$ are composite, then
                        $a+b$ is composite.
                    \item If $p$ is prime and ${p}\vert{a^{2}}$,
                        then $p^{2}\vert{a^{2}}$
                    \item If $0<b<a$, then $a^{2}-b^{2}$ is
                        composite.
                \end{multicols}
            \end{enumerate}
        \end{problem}
        \begin{proof}[Solution]
            \
            \begin{enumerate}
                \item True, for $300=20\cdot 15$
                \item False, for $2>0$, but $2$ does not divide
                    $1$
                \item False, for $5>0$ and $7>0$ but $5$ does
                    not divide $7$ and $7$ does not
                    divide $5$ for they are prime.
                \item True. If ${a}\vert{b}$, then
                    $\exists_{n\in\mathbb{Z}}:b=na$. If
                    ${a}\vert{(b+c)}$, then
                    $\exists_{m\in\mathbb{Z}}:b+c=ma$. But we
                    have that $c=ma-b=ma-na=a(m-n)$,
                    and therefore ${a}\vert{c}$. But then
                    $b-c=a(2n-m)$, so ${a}\vert{(b-c)}$
                \item True. If ${a}\vert{b}$ then
                    $\exists_{n\in\mathbb{Z}}:b=an$.
                    If ${a}\vert{c}$, then
                    $\exists_{m\in\mathbb{Z}}:c=am$. But then
                    $b^{2}+c^{2}=a^{2}n^{2}+a^{2}m^{2}%
                     =a(an^{2}+am^{2})$, and therefore
                    ${a}\vert{(b^{2}+c^{2})}$
                \item False. Let $a=4$, $b=8$, and $c=6$.
                    Then $b=2a$, $b^{2}+c^{2}=25a$, but $4$
                    does not divide $6$.
                \item True. If $a,b,c>0$ and ${a}\vert{b}$,
                    then $\exists_{n\in\mathbb{N}}:b=na$,
                    and therefore $a\leq b$. If
                    ${b}\vert{c}$, then
                    $\exists_{m\in\mathbb{N}}:c=mb$. But then
                    $b\leq c$. But $a\leq b$, and therefore
                    $a\leq c$
                \item True. If ${n}\vert{a}$ and ${n}\vert{b}$,
                    then ${n}\vert{a}$ and ${n}\vert{bc}$, and
                    therefore $\gcd(a,b)\leq\gcd(a,bc)$
                \item True. If ${n}\vert{a}$ and
                    ${n}\vert{(c-a)}$, then ${n}\vert{c}$. But
                    then ${n}\vert{(a+c)}$. If ${n}\vert{c}$
                    and ${n}\vert{(a+c)}$, then ${n}\vert{c}$.
                    But then ${n}\vert{(c-a)}$, and therefore
                    $\gcd(a,c-a)=\gcd(a+c,c)$
                \item False. Let $a=6$ and $c=10$. Then
                    $\gcd(a,b)=\gcd(6,10)=2$, and
                    $\gcd(a+c,c-a)=\gcd(16,4)=4$.
                \item False. Let $p=5$, $a=2$, $b=5$, and $c=25$.
                    Then $p$ is prime, ${p^{3}}\vert{abc}$, but
                    $5$ does not divide $2$
                \item False. Let $a=b=1$. Then $a+b=2$, which
                    is prime, but $ab=1$, which is odd.
                \item False. Let $a=9$, and $b=8$. Then $a$ and
                    $b$ are composite, but $a+b=17$,
                    which is prime.
                \item True. If ${p}\vert{a^{2}}$, then
                    $\exists_{n\in\mathbb{Z}}:a^{2}=np$. But, as
                    $p$ is prime, $a$ does not divide $p$, and
                    therefore $a=\frac{n}{a}p$. That is,
                    ${p}\vert{a}$. Therefore, ${p}\vert{a^{2}}$
                \item False. Let $a=9$ and $b=8$. Then
                    $9^{2}-8^{2}=81-64=17$, which is prime.
            \end{enumerate}
        \end{proof}
        \begin{problem}
            Use Euclid's Algorithm to compute $\gcd(201,62)$.
        \end{problem}
        \begin{proof}[Solution]
            \begin{align*}
                201&=62\cdot 3+15\\
                62&=15\cdot 5+2\\
                15&=2\cdot 7+1\\
                2&=1\cdot 2+0
            \end{align*}
            $\gcd(201,62)=1$
        \end{proof}
        \begin{problem}
            Find all integer solutions to $201x+62y=1$
        \end{problem}
        \begin{proof}[Solution 1]
            From the previous problem, we have:
            \begin{equation*}
                3+\frac{1}{4+\frac{1}{7}}=\frac{94}{29}
            \end{equation*}
            So $201(29)+62(-94)=1$. The general solution
            is therefore $x=29+62k$ and $y=-94-201k$ for
            all $k\in\mathbb{Z}$.
        \end{proof}
        \begin{proof}[Solution 2]
            From the previous problem, we have:
            \begin{align*}
                1&=15-2\cdot7&
                &=201\cdot(1+28)+62\cdot(-3-7-84)\\
                &=(201-63\cdot3)-(62-15\cdot4)\cdot7&
                &=201\cdot29+62\cdot(-94)\\
                &=(201-62\cdot3)-(62-(201-62\cdot3)\cdot4)\cdot7
            \end{align*}
            The general solution is $x=29+62k$ and $y=-94-201k$
        \end{proof}
        \begin{problem}
            Solve the following:
            \begin{enumerate}
                \begin{multicols}{2}
                    \item ${300^{3}+400^{4}}\mod{6}$
                    \item ${300^{3}+400^{4}}\mod{5}$
                    \item ${3^{1}}\mod{10}$
                    \item Last digit of $333^{222}$
                    \item ${1212^{11}}\mod{13}$
                    \item If $m$ is odd and $66\equiv{4}\mod{m}$,
                        what is $m$?
                    \item ${(21)(34)+765}\mod{9}$
                    \item ${48^{237}}\mod{4}$
                    \item ${3+3^{3}+3^{5}+3^{7}+3^{9}}\mod{8}$
                    \item If $2x\equiv{5}\mod{21}$, what is
                        ${x}\mod{21}$?
                \end{multicols}
            \end{enumerate}
        \end{problem}
        \begin{proof}[Solution]
            \par\hfill\par
            \begin{enumerate}
                \item We have
                    ${6}\vert{300}\Rightarrow%
                     300^{3}\equiv{0}\mod{6}$.
                    Also
                    $400\equiv{4}\mod{6}\Rightarrow%
                     400^{4}\equiv{4^{4}}\mod{6}%
                     ={256}\mod{6}\equiv{4}$
                \item
                    ${5}\vert{300}\Rightarrow{300^{3}}%
                     \equiv{0}\mod{5}$,
                    ${5}\vert{400}\Rightarrow{400^{4}}%
                     \equiv{0}\mod{5}$.
                    ${300^{3}+400^{4}}\equiv{0}\mod{5}$
                \item
                    ${3}\cdot{7}={21}\equiv{1}\mod{10}%
                     \Rightarrow{3^{-1}}\equiv{7}\mod{10}$
                \item
                    ${333}\equiv{3}\mod{10}\Rightarrow%
                     {333^{222}}\equiv{3^{222}}\mod{10}$. But
                    $3^{222}=9(3^{2})^{110}$, and
                    $9^{110}={81^{55}}\equiv{1}\mod{10}$.
                    So, ${333^{222}}\equiv{9}\mod{10}$
                \item
                    ${1212}\equiv{3}\mod{13}$, and
                    $3^{11}=9\cdot((3^{3})^{3}={9}\cdot{27}^{3}$.
                    But ${27}\equiv{1}\mod{13}$. So
                    ${1212^{11}}\equiv{9}\mod{13}$
                \item ${62}\equiv{0}\mod{m}$. But
                    $62={31}\cdot{2}$. $m=31$
                \item ${21}\equiv{3}\mod{9}$,
                    ${34}\equiv{7}\mod{9}$, and
                    ${765}\equiv{0}\mod{9}$. So we have
                    ${3}\cdot{7}={21}\equiv{3}\mod{9}$
                \item ${48}\equiv{0}\mod{4}$.
                \item $3^{2}\equiv{1}\mod{8}$,
                    $3^{5}\equiv{{3}\cdot{3^{4}}}\mod{8}%
                     \equiv{3}\mod{8}$,
                    $3^{7}\equiv{{3}\cdot{3^{6}}}\mod{8}%
                     \equiv{3}\mod{8}$, and finally
                    ${3^{9}}\equiv{{3}\cdot{3^{8}}}\mod{8}%
                     \equiv{3}\mod{8}$. So we have
                    $3+3+3+3+3={15}\equiv{7}\mod{8}$
                \item If ${2x}\equiv{5}\mod{21}$, then
                    $x\equiv{{5}\cdot{2^{-1}}}\mod{21}$.
                    But ${2^{-1}}\equiv{11}\mod{21}$, so
                    ${x}\equiv{{5}\cdot{11}}\mod{21}$ and
                    ${5}\cdot{11}={55}\equiv{13}\mod{21}$.
            \end{enumerate}
        \end{proof}
        \begin{problem}
            Find all integers $n,m\geq{0}$ such that
            $5^{n}-4^{m}=1$.
        \end{problem}
        \begin{proof}[Solution]
            $n=m=1$ is a solution since
            $5-4=1$. Suppose there is another solution.
            Note that $5^{0}-4^{0}=1-1=0$,
            $5^{1}-4^{0}=5-1=4$, and $5^{0}-4^{1}=1-4=-3$.
            If $m\geq{1}$ and $n\geq{2}$, we have
            $5^{n}-4^{m}>5^{n}-1\geq25-4=21>1$. If $m\geq{2}$,
            then $4^{m}$ is divisible by 8, and thus
            $4^{m}\mod{8}=0$. If $(n,m)$ is a solution, then
            $1=5^{n}-4^{n}\equiv{5^{n}}\mod{8}$, and thus
            $5^{n}\equiv{1}\mod{8}$. If $n$ is even, then
            $5^{2k}=25^{k}\equiv{1}\mod{8}$. If $n$ is odd, then
            $5^{2k+1}\equiv{5}\mod{8}$. Thus $n$ must be even if it
            is a solution. But if $5^{n}-4^{m}=1$,
            then $5^{n}-4^{m}\equiv{1}\mod{3}$. But
            $5^{n}-4^{m}\equiv{(-1)^{n}-(1)^{m}}\mod{3}$. But $n$ is
            even, and thus $5^{n}-4^{m}\equiv{0}\mod{8}$. But then
            $1\equiv{0}\mod{3}$, a contradiction. Thus, there is
            no other solution. $n=m=1$ is the only solution.
        \end{proof}
        \section{Modulo Arithmetic}
    \begin{theorem}
        \label{thm:Modulo_n_is_Equiv_Relation}%
        If $n\in\mathbb{N}$, and if
        $R\subseteq\mathbb{Z}\times\mathbb{Z}$ is defined by:
        \begin{equation}
            R=\{\,(a,b)\in\mathbb{Z}^{2}\;|\;
                n\textrm{ divides }b-a\,\}
        \end{equation}
        then $R$ is an equivalence relation on $\mathbb{Z}$.
    \end{theorem}
    \begin{proof}
        For $aRa$ since $a-a=0$ and $n$ divides 0. If $aRb$, then $n$
        divides $b-a$ and hence there is a $k\in\mathbb{Z}$ such that
        $nk=b-a$ (Def.~\ref{def:Divisor_of_Integer}). But then
        $n(\minus{k})=a-b$, and thus $n$ divides $a-b$
        (Def.~\ref{def:Divisor_of_Integer}), hence $bRa$. Lastly, if
        $aRb$ and $bRc$, then there exist integers $j,k\in\mathbb{Z}$
        such that $nj=b-a$ and $nk=c-b$. But then:
        \begin{equation}
            n(k+j)=nk+nk=(c-b)+(b-a)=c-a
        \end{equation}
        and therefore $aRc$.
    \end{proof}
    \begin{fdefinition}{Ring of Integers Modulo $n$}{Ring_Ints_Mod_N}
        The ring of integers modulo $n\in\mathbb{N}$ is the quotient set
        $\mathbb{Z}/R$ where $R$ is the equivalence relation
        $R\subseteq\mathbb{Z}^{2}$ defined by:
        \begin{equation*}
            R=\{\,(a,b)\in\mathbb{Z}^{2}\;|\;n\textrm{ divides }b-a\,\}
        \end{equation*}
        We denote this $\mathbb{Z}/n\mathbb{Z}$.
    \end{fdefinition}
    \begin{theorem}
        \label{thm:Z_n_is_Bij_onto_Z_mod_n}%
        If $n\in\mathbb{N}^{+}$, if $\mathbb{Z}/n\mathbb{Z}$ is the ring
        of integers modulo $n$, and if the function
        $\pi:\mathbb{Z}\rightarrow\mathbb{Z}/n\mathbb{Z}$ is the
        canonical projection map: $\pi(n)=[n]$, then
        $\pi|_{\mathbb{Z}_{n}}$ is bijective.
    \end{theorem}
    \begin{proof}
        For let $x\in\mathbb{Z}/n\mathbb{Z}$. Then there exists a
        representative $k\in\mathbb{Z}$ such that $[k]=x$. But by
        Euclid's division algorithm there exists $q\in\mathbb{Z}$ and
        $r\in\mathbb{N}$ such that $r<n$ and $k=qn+r$. But then
        $qn=k-r$ and hence $n$ divides $k-r$
        (Def.~\ref{def:Divisor_of_Integer}). But then $r\in[k]$, and
        hence $\pi(r)=\pi(k)=x$. And since $r<n$, $r\in\mathbb{Z}_{n}$.
        Hence, $\pi|_{\mathbb{Z}_{n}}$ is surjective. Moreover, if
        $a,b\in\mathbb{Z}_{n}$ and if $a\ne{b}$, then either $a<b$ or
        $b<a$. Suppose $a<b$. If $\pi(a)=\pi(b)$, then $[a]=[b]$ and
        hence $n$ divides $b-a$. But since $a,b\in\mathbb{Z}_{n}$, $a<n$
        and $b<n$, and thus $b-a<n$. But since $a<b$,
        $b-a\in\mathbb{N}^{+}$. But then $n$ the greater divides $b-a$
        the lesser, a contradiction
        (Thm.~\ref{thm:Greater_is_not_divisor_of_lesser}). Hence,
        $\pi|_{\mathbb{Z}_{n}}$ is injective, and is therefore a
        bijection. 
    \end{proof}
    In the case of $n=0$ we see that $\mathbb{Z}/0\mathbb{Z}$ reduces to
    just $\mathbb{Z}$. That is, the equivalence classes are just
    $[x]=\{x\}$ and thus
    $\pi:\mathbb{Z}\rightarrow\mathbb{Z}/0\mathbb{Z}$ is a bijection. We
    now wish to endow $\mathbb{Z}/n\mathbb{Z}$ with an arithmetic. We do
    so by borrowing the arithmetic from $\mathbb{Z}$ by using Euclid's
    division algorithm. We define this as follows.
    \begin{fdefinition}{Addition Modulo $n$}{Addition_Mod_n}
        The additive operation $\tilde{+}$ on $\mathbb{Z}/n\mathbb{Z}$
        (relabelled $Z_{n}$ for brevity) for $n\in\mathbb{N}^{+}$ is the
        set:
        \begin{equation*}
            \tilde{+}=\big\{\,\big((x,y),z\big)
                \in\big(Z_{n}\times{Z}_{n}\big)\times{Z}_{n}\;\big|\;
                \exists_{a,b\in\mathbb{Z}}
                \big(x=[a],y=[b],z=[a+b]\big)\big\}
        \end{equation*}
    \end{fdefinition}
    We only used the labelling $Z_{n}$ so that the equation didn't
    run off the page, and will not use it regularly, but rather stick
    to $\mathbb{Z}/n\mathbb{Z}$. The definition seems strange, but
    recall a function $f:A\rightarrow{B}$ is a subset
    $f\subseteq{A}\times{B}$ with certain properties, and a binary
    operation on $A$ is a function $*:A\times{A}\rightarrow{A}$. Hence,
    a binary operation is a particular subset of
    $(A\times{A})\times{A}$. We are now tasked with showing the
    definition given by Def.~\ref{def:Addition_Mod_n} forms a valid well
    defined binary operation on $\mathbb{Z}/n\mathbb{Z}$. To show that
    it is a binary operation amounts to showing that for every pair of
    elements $(x,y)$ there is a unique $z$ corresponding to this.
    \begin{theorem}
        \label{thm:Mod_Addition_is_Bin_Op}%
        If $n\in\mathbb{N}^{+}$, and if $\tilde{+}$ is the additive
        operation on $\mathbb{Z}/n\mathbb{Z}$, the $\tilde{+}$ is a
        binary operation.
    \end{theorem}
    \begin{proof}
        For if $x,y\in\mathbb{Z}/n\mathbb{Z}$, then there exists
        $a,b\in\mathbb{Z}$ such that $x=[a]$ and $y=[b]$. But then
        $\big((x,y)[a+b]\big)\in\tilde{+}$
        (Def.~\ref{def:Addition_Mod_n}). If $z\in\mathbb{Z}$ is such
        that $\big((x,y),z\big)\in\tilde{+}$, then there exists
        $\alpha,\beta,\gamma\in\mathbb{Z}$ such that
        $x=[\alpha]$, $y=[\beta]$, and $z=[\alpha+\beta]$. But if
        $x=[\alpha]$ and $x=[a]$, then $n$ divides $a-\alpha$ and hence
        there is a $j\in\mathbb{Z}$ such that $jn=a-\alpha$
        (Def.~\ref{def:Divisor_of_Integer}). Similarly there is a
        $k\in\mathbb{Z}$ such that $kn=b-\beta$. But then:
        \begin{equation}
            a+b-(\alpha+\beta)=(a-\alpha)+(b-\beta)=jn+kn=(j+k)n
        \end{equation}
        and hence $n$ divides $a+b-(\alpha+\beta)$
        (Def.~\ref{def:Divisor_of_Integer}). But then
        $[a+b]=[\alpha+\beta]$, and hence for all
        $x,y\in\mathbb{Z}/n\mathbb{Z}$ there is a unique
        $z\in\mathbb{Z}/n\mathbb{Z}$ such that
        $\big((x,y),z\big)\in\tilde{+}$, and hence $\tilde{+}$ is
        function. Moreover, since the domain of $\tilde{+}$ is
        $\mathbb{Z}/n\mathbb{Z}\times\mathbb{Z}/n\mathbb{Z}$ and the
        range is $\mathbb{Z}/n\mathbb{Z}$, $\tilde{+}$ is a binary
        operation.
    \end{proof}
    Since $\mathbb{Z}/n\mathbb{Z}$ can be put into a bijection with
    $\mathbb{Z}_{n}$ for $n\in\mathbb{N}^{+}$
    (Thm.~\ref{thm:Z_n_is_Bij_onto_Z_mod_n}), and since $\tilde{+}$
    gives a binary operation on $\mathbb{Z}/n\mathbb{Z}$
    (Thm.~\ref{thm:Mod_Addition_is_Bin_Op}) it custom to pull this
    binary operation back to $\mathbb{Z}_{n}$ and relabel it simply as
    $+$. It's always poor to use the same symbol for two different
    things that are frequently used in the same context, but alas it is
    the standard. In this case it is rather justifiable since
    $[j+k]$ is simply the remainder term of $j+k$ after division by $n$.
    \begin{example}
        The most common example one comes across of modulo arithmetic is
        in $\mathbb{Z}_{12}$ since this represents a clock. If it is 11
        A.M. and you wait for 3 hours, the time will then be 2 P.M. and
        hence $11+3=2$, quite paradoxical. One might claim ``Aha! I use
        military time!'' but then we simply apply the argument to
        $\mathbb{Z}_{24}$ and ask what time is 23 hours + 3 hours? That
        answer is 2 in the morning, hence $23+3=2$. There's no mystery
        once one realizes we are simply using the arithmetic of
        $\mathbb{Z}/n\mathbb{Z}$.
    \end{example}
    With modulo addition defined, we now turn to multiplication. Similar
    to addition, we build this new operation by borrowing from our
    familiar multiplicative operation on $\mathbb{Z}$ and thus push
    down to the equivalence classes in $\mathbb{Z}/n\mathbb{Z}$.
    \begin{fdefinition}{Multiplication Modulo $n$}{Multiplication_Mod_n}
        The multiplicative operation $\tilde{\cdot}$ on
        $\mathbb{Z}/n\mathbb{Z}$
        (relabelled $Z_{n}$ for brevity) for $n\in\mathbb{N}^{+}$
        is the set:
        \begin{equation*}
            \tilde{\cdot}=\big\{\,\big((x,y),z\big)
                \in\big(Z_{n}\times{Z}_{n}\big)\times{Z}_{n}\;\big|\;
                \exists_{a,b\in\mathbb{Z}}
                \big(x=[a],y=[b],z=[a\cdot{b}]\big)\big\}
        \end{equation*}
    \end{fdefinition}
    \begin{theorem}
        \label{thm:Mult_Mod_n_is_Bin_Op}%
        If $n\in\mathbb{N}^{+}$, if $\tilde{\cdot}$ is the
        multiplicative operation on $\mathbb{Z}/n\mathbb{Z}$, then
        $\tilde{\cdot}$ is a binary operation on
        $\mathbb{Z}/n\mathbb{Z}$.
    \end{theorem}
    \begin{proof}
        For if $x,y\in\mathbb{Z}/n\mathbb{Z}$ then there exist
        $n,m\in\mathbb{Z}$ such that $x=[n]$ and $y=[m]$. But then
        $\big((x,y),[n\cdot{m}]\big)\in\tilde{\cdot}$
        (Def.~\ref{def:Multiplication_Mod_n}). If
        $\big((x,y),z\big)\in\tilde{\cdot}$, then there exists
        $\alpha,\beta\in\mathbb{Z}/n\mathbb{Z}$ such that
        $x=[\alpha]$, $y=[\beta]$, and $z=[\alpha\cdot\beta]$
        (Def.~\ref{def:Multiplication_Mod_n}). But if $x=[\alpha]$ and
        $x=[a]$, then $[a]=[\alpha]$ and hence $n$ divides $a-\alpha$.
        Similarly, $n$ divides $b-\beta$. But then there exists
        $j,k\in\mathbb{Z}$ such that $jn=a-\alpha$ and $kn=b-\beta$
        (Def.~\ref{def:Divisor_of_Integer}). But then
        $a=jn+\alpha$ and $b-kn+\beta$, hence:
        \begin{subequations}
            \begin{align}
                (a\cdot{b})-(\alpha\cdot\beta)
                    &=(\alpha+jn)(\beta+kn)-(\alpha\cdot\beta)\\
                    &=\alpha\beta+jn\beta+kn\alpha-\alpha\cdot\beta\\
                    &=jn\beta+kn\alpha\\
                    &=n(j\beta+k\alpha)
            \end{align}
        \end{subequations}
        and hence $n$ divides $a\cdot{b}-\alpha\cdot\beta$. But then
        $[a\cdot{b}]=[\alpha\cdot\beta]$ and therefore for all
        $x,y\in\mathbb{Z}/n\mathbb{Z}$ there is a unique
        $z\in\mathbb{Z}/n\mathbb{Z}$ such that
        $\big((x,y),z\big)\in\tilde{\cdot}$. Thus, $\tilde{\cdot}$ is a
        binary operation on $\mathbb{Z}/n\mathbb{Z}$.
    \end{proof}
    While we occasional use $\mathbb{Z}_{n}$, $+$, and $\cdot$ in place
    of $\mathbb{Z}/n\mathbb{Z}$, $\tilde{+}$, and $\tilde{\cdot}$, it is
    worthwhile to note that the elements of $\mathbb{Z}/n\mathbb{Z}$ are
    \textit{not} the same as the elements of $\mathbb{Z}_{n}$.
    $\mathbb{N}$ was constructed from the axiom of infinity and the
    elements look like
    $\emptyset,\{\emptyset\},\{\emptyset,\{\emptyset\}\}$, and so forth.
    Meanwhile $\mathbb{Z}/n\mathbb{Z}$ was constructed from an
    equivalence relation, and hence the elements of
    $\mathbb{Z}/n\mathbb{Z}$ are elements of the \textit{power set} of
    $\mathbb{N}$. Indeed, we know precisely what these elements are:
    \begin{equation}
        \begin{split}
            \mathbb{Z}/n\mathbb{Z}=
            \Big\{\,&\{\,0,\,\pm{n},\,\pm{2n},\,\pm{3n},\,\dots\,\},\\
                &\{\,1,\,1\pm{n},\,1\pm{2n},\,1\pm{3n},\,\dots\,\},\\
                &\{\,2,\,2\pm{n},\,2\pm{2n},\,2\pm{3n},\,\dots\,\},\\
                &\dots,\\
                &\{\,n-1,\,n-1\pm{n},\,n-1\pm{2n},\,n-1\pm{3n},\,
                    \dots\,\}\Big\}
        \end{split}
    \end{equation}
    set theoretically these are different.
    Thm.~\ref{thm:Z_n_is_Bij_onto_Z_mod_n} tells us they're the same
    size and the way we've defined modulo arithmetic mimics the notion
    division with remainder in $\mathbb{Z}_{n}$. That is,
    writing $a+b=qn+r$ with $0\leq{r}<r$, we have defined
    $[a]\tilde{+}[b]=[r]$, and similary for $a\cdot{b}=sn+t$, we have
    $[a]\tilde{\cdot}[b]=[t]$. It is in this sense that we are justified
    in using the same notation.
    \begin{example}
        A common application of modulo arithmetic that one sees in
        elementary number theory is the computation of the last few
        digits of large powers of numbers. For example, consider $3^120$
        and suppose we want to know the last digit of this number.
        That's equivalent to asking what is the smallest representative
        of the equivalence class of $3^{120}$ in the ring of integers
        modulo 10. First we reduce the problem and recognize that
        $120=2\cdot{60}$, and hence
        $3^{120}=3^{2\cdot{60}}=(3^{2})^{60}$. Well $3^{2}=9$, which
        is equivalent to $\minus{1}$ in $\mathbb{Z}/10\mathbb{Z}$, and
        so we next consider $(\minus{1})^{60}$. But this is just 1.
        But 1 is congruent to 1 in $\mathbb{Z}/10\mathbb{Z}$ and there
        we have it: The last digit of $3^{120}$ is 1. Indeed, using your
        favorite computer language, we can compute and obtain:
        \begin{equation*}
            3^{120}=
            1797010299914431210413179829509605039731475627537851106401
        \end{equation*}
        and so our calculation was correct. Let us try $2^{2000}$ and
        attain the last 2 digits (But perhaps not compute the actual
        value). We note that $2000=10*200$ and every computer scientist
        recognizes instantly that $2^{10}=1024$, which is congruent to
        24 in $\mathbb{Z}/100\mathbb{Z}$. So we are left with
        $24^{200}$. We look at the exponent again, note that it is equal
        to $2\cdot{100}$ and $24^{2}$ seems a far simpler computation.
        We get $24^{2}=576$, the last two digits of which are 76, and so
        we're down to $76^{100}$. We break this into $(76^{2})^{50}$ and
        upon computing we get $76^{2}=5776$ and so the last two digits
        are once again 76, meaning we can continue decomposing away all
        of the powers of 2, and we are left with
        $75^{25}=76^{24}\cdot{76}$. But then removing the powers of 2
        away from 24 ($24=2^{3}\cdot{3}$), we are left with
        $76^{3}\cdot{76}=76^{4}$ which then reduces to $76$. The last
        two digits of $2^{2000}$ are 76.
    \end{example}
    \begin{theorem}
        \label{thm:Equiv_Class_of_0_is_Add_Identity_Mod_n}%
        If $n\in\mathbb{N}^{+}$, if $[0]\in\mathbb{Z}/n\mathbb{Z}$ is
        the equivalence class of 0 in the ring of integers modulo $n$,
        and if $x\in\mathbb{Z}/n\mathbb{Z}$, then $x\tilde{+}[0]=x$.
    \end{theorem}
    \begin{proof}
        For if $x\in\mathbb{Z}/n\mathbb{Z}$ then there is a
        $k\in\mathbb{Z}$ such that $x=[k]$. But then:
        \begin{equation}
            x\tilde{+}[0]=[k]\tilde{+}[0]=[k+0]=[k]=x
        \end{equation}
    \end{proof}
    \begin{theorem}
        \label{thm:Mod_Add_is_Assoc}%
        If $n\in\mathbb{N}^{+}$, if $x,y,z\in\mathbb{Z}/n\mathbb{Z}$,
        then $(x\tilde{+}y)\tilde{+}z=x\tilde{+}(y\tilde{+}z)$.
    \end{theorem}
    \begin{proof}
        For if $x,y,z\in\mathbb{Z}/n\mathbb{Z}$, then there exists
        $i,j,k\in\mathbb{Z}$ such that $x=[i]$, $y=[j]$, and $z=[k]$.
        But then:
        \par
        \begin{subequations}
            \begin{minipage}[t]{0.54\textwidth}
                \centering
                \begin{align}
                    (x\tilde{+}y)\tilde{+}z
                    &=([i]\tilde{+}[j])\tilde{+}[k]
                        \tag{Hypothesis}\\
                    &=[i+j]\tilde{+}[k]
                        \tag{Def.~\ref{def:Addition_Mod_n}}\\
                    &=[(i+j)+k]
                        \tag{Def.~\ref{def:Addition_Mod_n}}\\
                    &=[i+(j+k)]
                        \tag{Associativity}
                \end{align}
            \end{minipage}
            \hfill
            \begin{minipage}[t]{0.44\textwidth}
                \begin{align}
                    &=[i]\tilde{+}[j+l]
                        \tag{Def.~\ref{def:Addition_Mod_n}}\\
                    &=[i]\tilde{+}([j]\tilde{+}[k])
                        \tag{Def.~\ref{def:Addition_Mod_n}}\\
                    &=x\tilde{+}(y\tilde{+}z)
                        \tag{Hypothesis}
                \end{align}
            \end{minipage}
        \end{subequations}
        \par\vspace{2.5ex}
        proving the claim.
    \end{proof}
    \begin{theorem}
        \label{thm:Additive_Inv_Mod_n}%
        If $n\in\mathbb{N}^{+}$, and if $x\in\mathbb{Z}/n\mathbb{Z}$,
        then there is a $y\in\mathbb{Z}/n\mathbb{Z}$ such that
        $x\tilde{+}y=[0]$.
    \end{theorem}
    \begin{proof}
        For if $x\in\mathbb{Z}/n\mathbb{Z}$, then there is a
        $k\in\mathbb{Z}$ such that $x=[k]$. Let $y=[\minus{k}]$. But
        then:
        \begin{equation}
            x\tilde{+}y=[k]\tilde{+}[\minus{k}]
            =[k+(\minus{k})]=[0]
        \end{equation}
        proving the claim.
    \end{proof}
    \begin{theorem}
        \label{thm:Equiv_Class_of_1_is_Mult_Identity_Mod_n}%
        If $n\in\mathbb{N}^{+}$, if $[1]\in\mathbb{Z}/n\mathbb{Z}$ is
        the equivalence class of 1 in the ring of integers modulo $n$,
        and if $x\in\mathbb{Z}/n\mathbb{Z}$, then $x\tilde{\cdot}[1]=x$.
    \end{theorem}
    \begin{proof}
        For if $x\in\mathbb{Z}/n\mathbb{Z}$, then there is a
        $k\in\mathbb{Z}$ such that $x=[k]$. But then:
        \begin{equation}
            x\tilde{\cdot}[1]=[k]\tilde{\cdot}[1]=[k\cdot{1}]=[k]=x
        \end{equation}
    \end{proof}
    \begin{theorem}
        \label{thm:Mod_Mult_is_Assoc}%
        If $n\in\mathbb{N}^{+}$, if $x,y,z\in\mathbb{Z}/n\mathbb{Z}$,
        then $(x\tilde{\cdot}y)\tilde{\cdot}z%
        =x\tilde{\cdot}(y\tilde{\cdot}z)$.
    \end{theorem}
    \begin{proof}
        For if $x,y,z\in\mathbb{Z}/n\mathbb{Z}$, then there exists
        $i,j,k\in\mathbb{Z}$ such that $x=[i]$, $y=[j]$, and $z=[k]$.
        But then:
        \par
        \begin{subequations}
            \begin{minipage}[t]{0.54\textwidth}
                \centering
                \begin{align}
                    (x\tilde{\cdot}y)\tilde{\cdot}z
                    &=([i]\tilde{\cdot}[j])\tilde{\cdot}[k]
                        \tag{Hypothesis}\\
                    &=[i\cdot{j}]\tilde{\cdot}[k]
                        \tag{Def.~\ref{def:Multiplication_Mod_n}}\\
                    &=[(i\cdot{j})\cdot{k}]
                        \tag{Def.~\ref{def:Multiplication_Mod_n}}\\
                    &=[i\cdot(j\cdot{k})]
                        \tag{Associativity}
                \end{align}
            \end{minipage}
            \hfill
            \begin{minipage}[t]{0.44\textwidth}
                \begin{align}
                    &=[i]\tilde{\cdot}[j\cdot{l}]
                        \tag{Def.~\ref{def:Multiplication_Mod_n}}\\
                    &=[i]\tilde{\cdot}([j]\tilde{\cdot}[k])
                        \tag{Def.~\ref{def:Multiplication_Mod_n}}\\
                    &=x\tilde{\cdot}(y\tilde{\cdot}z)
                        \tag{Hypothesis}
                \end{align}
            \end{minipage}
        \end{subequations}
        \par\vspace{2.5ex}
        proving the claim.
    \end{proof}
    \begin{theorem}
        \label{thm:Invertible_Mod_n_iff_Relatively_Prime}%
        If $n\in\mathbb{N}^{+}$, if $x\in\mathbb{Z}/n\mathbb{Z}$, then
        there exists a $y\in\mathbb{Z}/n\mathbb{Z}$ such that
        $x\tilde{\cdot}{y}=[1]$ if and only if there is a
        $k\in\mathbb{Z}$ such that $x=[k]$ and $\GCD(k,n)=1$.
    \end{theorem}
    \begin{proof}
        For if $k$ is a representative for $x$ and $\GCD(k,n)=1$, then
        by B\'{e}zout's identity there exist $a,b\in\mathbb{Z}$ such
        that $a\cdot{k}+b\cdot{n}=1$ (Thm.~\ref{thm:Bezout_Identity}).
        But then $b\cdot{n}=1-a\cdot{k}$ and hence $n$ divides
        $1-a\cdot{k}$ (Def.~\ref{def:Divisor_of_Integer}). But then by
        the definition of the ring of integers modulo $n$,
        $1\in[a\cdot{k}]$ (Def.~\ref{def:Ring_Ints_Mod_N} and
        hence $[a]\tilde{\cdot}[k]=[1]$. In the other direction, if
        $x$ has an inverse $y$, then there are representatives $k,m$
        such that $x=[k]$ and $y=[m]$. But then
        $x\tilde{\cdot}{y}=[k]\tilde{\cdot}[m]=[k\cdot{m}]$. But by
        hypothesis $x\tilde{\cdot}{y}=[1]$ and hence $[k\cdot{m}]=[1]$.
        But then $n$ divides $1-km$ and hence there is a
        $j\in\mathbb{Z}$ such that $jn=1-km$
        (Def.~\ref{def:Divisor_of_Integer}). But then $nj+km=1$ and
        therefore $\GCD(k,n)=1$.
    \end{proof}
    Hence every non-zero element of $\mathbb{Z}/p\mathbb{Z}$ for some prime
    $p$ is invertible.
    \begin{example}
        Other tricks that use modulo arithmetic appear in disguise when one
        studies divisibility tricks. If $a=\sum{a}_{n}10^{n}$ is a finite
        sum, then $a$ is congruent of $\sum{a}_{n}$. Simply use the
        additivity of modulo addition, and use the fact that
        $10^{n}\equiv{1}\mod{9}$. Another trick comes from studying if a
        number is divisible by 3. Applying the same trick, we see that if
        3 divides $n$, then it divides the sum of its digits (in base 10).
        Conversely, if 3 divides the sum of the digits of $n$ (in base 10),
        then 3 divides $n$.
    \end{example}
    \begin{example}
        Looking back at a previous example, let's compute the last digit of
        $9^{n}$ for any $n\in\mathbb{N}$. We note that
        $9\equiv\minus{1}\mod{10}$ and hence we only need to consider
        $(\minus{1})^{n}$. But $(\minus{1})^{n}$ is 1 if $n$ is even and
        $\minus{1}$ is $n$ is odd. Therefore the last digit of $9^{n}$ is
        9 if $n$ is odd and 1 if $n$ is even. And indeed, the pattern holds
        true: 1, 9, 81, 729, 6561, and so on.
    \end{example}
    Come back to excercises (DF Chapt 1).
    \chapter{Cardinality}
        \section{Cardinality}
        Functions can also be called maps or mappings. The unique point
        $b\in{B}$ such that $(a,b)\in{f}$ is often called the image of
        $a$ under $f$. We sometimes write $a\mapsto{b}$, but most often
        will write $f(a)=b$.
        The image of a subset $A\subseteq{X}$ is the set of all points
        that get mapped onto by the function $f$ by the elements in $A$.
        In a similar manner we can define the opposite of this notion,
        called the pre-image.
        \begin{axiom}
            If $X$ is a non-empty set such that for all $x\in{X}$,
            $x\ne\emptyset$, then there is a function
            $f:X\rightarrow\bigcup_{x\in{X}}x$ such that,
            for all $x\in{X}$, $f(x)\in{x}$.
        \end{axiom}
        This is called the axiom of choice. It can be made into
        a blatantly obvious statement, by choosing a more careful
        wording, however many of the results it gives are far
        from intuitive. This says that, given a collection of sets,
        each of which is non-empty, one may choose a single element from
        each set. This choosing is the function $f$, and is often called
        a \textit{choice function}. For those interested, the axiom
        of choice is consistent with modern set theory (Called
        Zermelo-Fraenkel set theory, or ZF). It may thus be rejected
        or accepted without logical contradiction.
        \begin{theorem}
            \label{theorem:Set_Theory_Image_of_Empty_Set_Is_Empty}
            If $A$ and $B$ are sets, and if $f:A\rightarrow{B}$
            is a function, then:
            \begin{equation}
                f(\emptyset)=\emptyset
            \end{equation}
        \end{theorem}
        \begin{proof}
            For suppose not. Let $y\in{f}(\emptyset)$.
            But then there is an
            $x\in\emptyset$ such that $f(x)=y$, a contradiction
            ince for all $x$, $x\notin\emptyset$.
            Thus, $f(\emptyset)=\emptyset$.
        \end{proof}
        \begin{theorem}
            If $A$ and $B$ are sets, and if $f:A\rightarrow{B}$ is a function, then:
            \begin{equation}
                f^{-1}(\emptyset)=\emptyset
            \end{equation}
        \end{theorem}
        \begin{proof}
            For suppose not. Then there is an $x\in{X}$ such that
            $f(x)\in\emptyset$, a contradiction since for all $x$,
            $f(x)\notin\emptyset$. Therefore, etc.
        \end{proof}
        \begin{theorem}
            If $X$ and $Y$ are sets, if $A\subseteq{X}$, and if
            $f:X\rightarrow{Y}$ is a function such that
            $f(A)=\emptyset$, then $A=\emptyset$.
        \end{theorem}
        \begin{proof}
            For suppose not. If $A\ne\emptyset$, then there is an
            $x\in{A}$. But then $f(x)\in{f}(A)$, a contradiction as
            $f(A)=\emptyset$. Therefore, etc.
        \end{proof}
        \begin{theorem}
            If $X$ and $Y$ are sets, if $B$ is a subset of $Y$,
            and if $f:X\rightarrow{Y}$ is a function, then:
            \begin{equation}
                f\big(f^{-1}(B)\big)\subseteq{B}
            \end{equation}
        \end{theorem}
        \begin{proof}
            For if $y\in{f(f^{-1}(B))}$, then there is an
            $x\in{f^{-1}(B)}$ such that $y=f(x)$. But if
            $x\in{f^{-1}(B)}$, then $f(x)\in{B}$. Thus,
            $y\in{B}$. Therefore, etc.
        \end{proof}
        \begin{theorem}
            If $X$ and $Y$ are non-empty sets and if there exists
            $y_{1},y_{2}\in{Y}$ such that $y_{1}\ne{y}_{2}$, then
            there is a function $f:X\rightarrow{Y}$ and a
            $B\subseteq{Y}$ such that:
            \begin{equation}
                f\big(f^{-1}(B)\big)\ne{B}
            \end{equation}
        \end{theorem}
        \begin{proof}
            \begin{subequations}
                For if $X$ and $Y$ are non-empty, let $f:X\rightarrow{Y}$
                be defined by:
                \begin{equation}
                    f=\{(x,y_{1}):x\in{X}\}
                \end{equation}
                Then $f$ is a function, since $f\subseteq{X}\times{Y}$
                as $y_{1}\in{Y}$. Moreover, for all $x\in{X}$ there is a
                unique $y\in{Y}$ such that $(x,y)\in{f}$. Thus, $f$ is a
                function from $X$ to $Y$. However since for all
                $x\in{X}$, $f(x)=y_{1}$, we have that:
                \begin{equation}
                    f^{-1}(\{y_{2}\})=\emptyset
                \end{equation}
                For suppose $x\in{f}^{-1}(\{y_{2}\})$.
                Then $f(x)=y_{2}$, but for all $x\in{X}$, $f(x)=y_{1}$,
                and $y_{1}\ne{y}_{2}$. Thus
                $f^{-1}(\{y_{2}\})=\emptyset$. But by
                Thm.~\ref{theorem:Set_Theory_Image_%
                          of_Empty_Set_Is_Empty},
                $f(\emptyset)=\emptyset$. Therefore:
                \begin{equation}
                    f\big(f^{-1}(\{y_{2}\})\big)=\emptyset
                \end{equation}
                But $\{y_{2}\}\ne\emptyset$ and
                $\{y_{2}\}\subseteq{Y}$. Therefore, etc.
            \end{subequations}
        \end{proof}
        \begin{theorem}
            If $X$ and $Y$ are sets, if $A$ is a subset of $X$,
            and if $f:X\rightarrow{Y}$ is a function, then:
            \begin{equation}
                A\subseteq{f^{-1}}\big(f(A)\big)
            \end{equation}
        \end{theorem}
        \begin{proof}
            For if $x\in{A}$, then there is a
            $y\in{f}(A)$ such that $f(x)=y$. But then
            $x\in{f^{-1}(f(A))}$. Therefore, etc.
        \end{proof}
        \begin{theorem}
            If $X$ and $Y$ are sets, if $A_{1}$ and $A_{2}$ are
            subsets of $X$ such that $A_{1}\subseteq{A}_{2}$,
            and if $f:X\rightarrow{Y}$ is a function, then:
            \begin{equation}
                f(A_{1})\subseteq{f}(A_{2})
            \end{equation}
        \end{theorem}
        \begin{proof}
            For if $y\in{f}(A_{1})$, then there is an $x\in{A}_{1}$
            such that $f(x)=y$. But $A_{1}\subseteq{A}_{2}$, and
            therefore $x\in{A}_{2}$. But if $x\in{A}_{2}$, then
            $f(x)\in{f}(A_{2})$. Thus, $y\in{f}(A_{2})$. Therefore, etc.
        \end{proof}
        \begin{theorem}
            If $X$ and $Y$ are sets, if $B_{1}$ and $B_{2}$ are subsets of
            $Y$ such that $B_{1}\subseteq{B}_{2}$, and if $f:X\rightarrow{Y}$
            is a function, then:
            \begin{equation}
                f^{-1}(B_{1})\subseteq{f^{-1}}(B_{2})
            \end{equation}
        \end{theorem}
        \begin{proof}
            For if $x\in{f}^{-1}(B_{1})$, then there is a
            $y\in{B}_{1}$ such that $f(x)=y$. But
            $B_{1}\subseteq{B}_{2}$, and therefore $y\in{B}_{2}$.
            Thus, $x\in{f}^{-1}(B_{2})$. Therefore, etc.
        \end{proof}
        \begin{theorem}
        If $f:A\rightarrow B$, $A_1,A_2\subset A$, then $f(A_1 \cup A_2) = f(A_1)\cup f(A_2)$.
        \end{theorem}
        \begin{proof}
        $[y\in f(A_1\cup A_2)]\Rightarrow [\exists x\in A_1 \cup A_2:y=f(x)]\Rightarrow [y \in f(A_1)\cup f(A_2)]$. $[y\in f(A_1)\cup f(A_2)]\Rightarrow \big[[\exists x\in A_1] \lor [\exists x\in A_2]: y=f(x)\big]\Rightarrow [x\in A_1\cup A_2]\Rightarrow [f(x)\in f(A_1\cup A_2)]$
        \end{proof}
        \begin{theorem}
        If $f:A\rightarrow B$, $A_1,A_2\subset A$, then $f(A_1\cap A_2)\subset f(A_1)\cap f(A_2)$.
        \end{theorem}
        \begin{proof}
        $[y\in f(A_1 \cap A_2)]\Rightarrow [\exists x\in A_1 \cap A_2:y=f(x)]\Rightarrow [x\in A_1 \land x \in A_2] \Rightarrow[y \in f(A_1)\cap f(A_2)]$.
        \end{proof}
        \begin{theorem}
        If $f:A\rightarrow B$, $B_1,B_2\subset B$, then $f^{-1}(B_1\cup B_2) = f^{-1}(B_1)\cup f^{-1}(B_2)$.
        \end{theorem}
        \begin{proof}
        $[x\in B_1\cup B_2]\Rightarrow [f(x)\in B_1\cup B_2]\Rightarrow [f(x)\in B_1\lor f(x)\in B_2]\Rightarrow [x\in f^{-1}(B_1)\cup f^{-1}(B_2)]$. $[x \in f^{-1}(B_1)\cup f^{-1}(B_2)]\Rightarrow [f(x)\in B_1\lor f(x) \in B_2]\Rightarrow [f(x) \in B_1\cup B_2]\Rightarrow [x\in f^{-1}(B_1\cup B_2)]$.
        \end{proof}
        \begin{theorem}
        If $f:A\rightarrow B$, $B_1,B_2\subset B$, then $f^{-1}(B_1\cap B_2) = f^{-1}(B_1)\cap f^{-1}(B_2)$.
        \end{theorem}
        \begin{proof}
        $[x\in f^{-1}(B_1\cap B_2)]\Rightarrow [f(x) \in B_1 \cap B_2]\Rightarrow [f(x)\in B_1\land f(x) \in B_2 ]\Rightarrow [x\in f^{-1}(B_1)\cap f^{-1}(B_2)]$. $[x\in f^{-1}(B_1)\cap f^{-1}(B_2)]\Rightarrow [x\in f^{-1}(B_1)\land x\in f^{-1}(B_2)]\Rightarrow [f(x) \in B_1\land f(x) \in B_2]\Rightarrow [f(x)\in B_1\cap B_2]\Rightarrow [x\in f^{-1}(B_1\cap B_2)]$.
        \end{proof}
        \begin{theorem}
        If $f:A\rightarrow B$, $B_1 \subset B$, then $f^{-1}(B\setminus B_1) = f^{-1}(B)\setminus f^{-1}(B_1)$.
        \end{theorem}
        \begin{proof}
        $[x\in f^{-1}(B\setminus B_1)]\Leftrightarrow [f(x)\notin B_1]\Leftrightarrow [x\in f^{-1}(B)\setminus f^{-1}(B_1)]$
        \end{proof}
        If $f:A\rightarrow B$, the image of $A$ under $f$
        is often called the range (A is often called the domain).
        \begin{ldefinition}{Permutations}{Permutations}
            A permutation on a set $A$ is a bijective function
            $f:A\rightarrow{A}$.
        \end{ldefinition}
        \begin{theorem}
        If $f:A\rightarrow B$ is bijective, then $f^{-1}$ is bijective.
        \end{theorem}
        \begin{proof}
        $[f^{-1}(y_1) = f^{-1}(y_2)]\Rightarrow [\exists x\in A:[f(x) = y_1]\land [f(x)=y_2]]\Rightarrow [y_1=y_2]$. By definition, $f^{-1}$ is surjective.
        \end{proof}
        \begin{definition}
        If $f:A\rightarrow B$ and $g:B\rightarrow C$, then $g\circ f:A\rightarrow C$ is defined by the image $g(f(x)), x\in A$.
        \end{definition}
        \begin{theorem}
        If $f:A\rightarrow B$, $g:B\rightarrow C$, and $\mathcal{V}\subset C$, then $(g\circ g)^{-1}(\mathcal{V}) = f^{-1}(g^{-1}(\mathcal{V}))$.
        \end{theorem}
        \begin{proof}
        $[x\in (g\circ f)^{-1}(\mathcal{V})]\Leftrightarrow [g(f(x))\in \mathcal{V}] \Leftrightarrow [f(x)\in g^{-1}(\mathcal{V})]\Leftrightarrow [x\in f^{-1}(g^{-1}(\mathcal{V}))]$.
        \end{proof}
        \begin{theorem}
        If $f:A\rightarrow B$ is bijective, $g:B\rightarrow C$ is bijective, then $g\circ f$ is bijective.
        \end{theorem}
        \begin{proof}
        $\big[[f(A) = B]\land [g(B) = C]\big]\Rightarrow [g(f(A)) = g(B) = C]$. $[g(f(x_1))=g(f(x_2))]\Leftrightarrow [f(x_1)=f(x_2)]\Leftrightarrow [x_1=x_2]$.
        \end{proof}
        \begin{theorem}
        If $f:A\rightarrow B$ is bijective, $A_1\subset A$, and $f(A_1) = B$, then $A_1=A$.
        \end{theorem}
        \begin{proof}
        $\Big[\big[[A_1^c \ne \emptyset]\Rightarrow [f(A_1^c) \ne \emptyset]\big]\land[f(A_1)\cap f(A_1^c) = \emptyset]\Big]\Rightarrow [\exists y\in B:y\notin f(A_1)]$, a contradiction.
        \end{proof}
        \begin{lexample}{}{Image_is_Nonempty}
            Given a function $f:X\rightarrow{Y}$, and any
            non-empty subset $S\subseteq{X}$, the image
            $f(S)$ is non-empty. This is not true for the
            pre-image of a function. For let
            $f:\mathbb{R}\rightarrow\mathbb{R}$ be defined by
            $f(x)=1$ for all $x\in\mathbb{R}$. Then, for any
            subset $S\subset\mathbb{R}$ such that
            $1\notin{S}$, we have that
            $f^{\minus{1}}(S)=\emptyset$.
        \end{lexample}
        There are many examples of functions, but certain
        ones are easier to study than others. We give some
        of these special functions names.
        \begin{ldefinition}{Injective Functions}
              {Funct_Analysis_Injective_Function}
            An \gls{injective function} is a function
            $f:X\rightarrow{Y}$ such that, for all
            $x,y\in{X}$ such that $x\ne{y}$, it is true that
            $f(x)\ne{f}(y)$.
        \end{ldefinition}
        That is, an injective function is a function
        $f:X\rightarrow{Y}$ such that $f(x_{1})=f(x_{2})$
        if and only if $x_{1}=x_{2}$. Such functions are also
        called \textit{one-to-one}.
        \begin{lexample}{}{Natural_Log_is_Injective}
            Consider the natural logarithm
            $\ln:\mathbb{R}^{+}\rightarrow\mathbb{R}$. This
            is an injective function. For let
            $x,y\in\mathbb{R}^{+}$ be such that $x\ne{y}$.
            Suppose $\ln(x)=\ln(y)$. But then:
            \begin{equation}
                \ln(x)-\ln(y)=\ln\Big(\frac{x}{y}\Big)=0
            \end{equation}
            Recall the definition of the natural logarithm:
            \begin{equation}
                \ln(t)=\int_{1}^{t}\frac{1}{x}\diff{x}
            \end{equation}
            But then $\ln(t)=0$ if and only if $t=1$. Thus
            $x=y$, a contradiction. Therefore $\ln$ is an
            injective function. Not every function is
            injective, for define
            $f:\mathbb{R}\rightarrow\mathbb{R}$ by
            $f(x)=x^{2}$. Then, for all $x\in\mathbb{R}^{+}$,
            $f(\minus{x})=f(x)$, and thus $f$ cannot be an
            injective function.
        \end{lexample}
        One might think that most functions are not injective,
        and indeed for the \textit{finite} case, this is true.
        For let $A$ and $B$ be finite sets with $n$ and $m$
        elements, respectively. If $m<n$, there can't be
        any injective function. Consider the case when $n=m$.
        Then we are simply counting the number of ways to
        permute the elements of $A$. This is $n!$. On the
        other hand, the total number of functions is
        $n^{n}$. Thus, the ratio of the number of injective
        functions to the number of functions is
        $n!/n^{n}$, and this decays to zero rapidly as
        $n$ get's large. Finally, if $m>n$, then the total
        number of injective functions is
        $n!\binom{m}{n}$, where $\binom{m}{n}$ is the
        binomial coefficient. The total number of functions
        is $n^{m}$. The ratio is thus:
        \begin{equation}
            \frac{n!\binom{m}{n}}{n^{m}}=
            \frac{n!\frac{m!}{n!(m-n)!}}{n^{m}}
            =\frac{m!}{(m-n)!n^{m}}
        \end{equation}
        And again, this decays rapidly to zero and $n$ and $m$
        get large. Later, when we define infinite sets
        and the notion of Cardinality, we'll show that this
        trend continues. That is, in a sense, \textit{most}
        functions from a set $A$ to a sufficiently large set
        $B$ are not injective.
        \begin{ldefinition}{Inverse of Injective Functions}
                           {Inverse_Function}
            The inverse of an injective function $f:A\rightarrow{B}$
            is the function $f^{-1}:f(A)\rightarrow{A}$ defined by
            $f^{-1}(y)=x$, where $x\in{A}$ is the unique element
            such that $f(x)=y$.
        \end{ldefinition}
        Next, we define \textit{surjective} functions.
        \begin{ldefinition}{Surjective Functions}
              {Funct_Analysis_Surjective_Function}
            A \gls{surjective function} is a function
            $f:X\rightarrow{Y}$ such that $f(X)=Y$.
            That is, for all $y\in{Y}$, there is an
            $x\in{X}$ such that $f(x)=y$.
        \end{ldefinition}
        That is, every point $y\in{Y}$ gets mapped to by
        at least one point in $X$. It may also be true that
        many points in $X$ map to the same point in $Y$.
        The notions of surjective functions and injective
        functions are distinct, and neither implies the
        other. Surjective functions are also called
        \textit{onto}.
        \begin{ldefinition}{Bijective Functions}
              {Funct_Analysis_Bijective_Function}
            A \gls{bijective function} is a function
            that is both injective and surjective.
        \end{ldefinition}
        Sets $X$ and $Y$ such that there
        exists a bijective function $f:X\rightarrow{Y}$ are
        called \textit{equivalent}. Such sets can be said
        to have the same size. We say that $X$ is strictly
        smaller than $Y$ if there is an injective function
        $f:X\rightarrow{Y}$, but no bijective function.
        Being countable means you can write
        the elements out in a list, or a
        one-to-one correspondence with all of
        the positive integers. Many sets are countable,
        including the whole numbers, integers, rational
        numbers, and \textit{algebraic} numbers. The
        union of finitely many countable sets is also
        countable, as is the union of countably many
        countable sets.
        \begin{example}
            The set of all positive even integers is
            countable. For let $\mathbb{N}_{e}$ be the
            set of all even integers and define
            $f:\mathbb{N}\rightarrow\mathbb{N}_{e}$ be
            $f(n)=2n$ for all $n\in\mathbb{N}$. This is
            a bijection, and thus $\mathbb{N}_{e}$ is
            countable. The set of all odd positive integers
            is countable, as shown by letting
            $f(n)=2n-1$. Even though the set of even
            integers may seem ``smaller,'' than the set of
            all integers, they are equivalent. The set of
            all integers $\mathbb{Z}$ is also countable.
            For let $f:\mathbb{N}\rightarrow\mathbb{Z}$
            be defined as:
            \begin{equation}
                f(n)=
                \begin{cases}
                    \frac{1}{2}(n-1),&n\textrm{ odd}\\
                    -\frac{n}{2},&n\textrm{ even}
                \end{cases}
            \end{equation}
        \end{example}
        Any set that is infinite (Not finite) contains a
        countable subset. Thus, $\mathbb{N}$ can be
        considered as the \textit{smallest} infinite set.
        \begin{theorem}
            If $A$ is an infinite set, then there exists
            $S\subseteq{A}$ such that $S$ is countabl e.
        \end{theorem}
        \begin{proof}
            For as $A$ is infinite, for all $n\in\mathbb{N}$
            there exists a set $B\subseteq{A}$ such that
            $|B|=n$. For all $n\in\mathbb{N}$,
            define the following:
            \begin{equation}
                \mathcal{S}_{n}=\{B\subseteq{A}:|B|=n\}
            \end{equation}
            Let $\mathcal{S}$ be defined as:
            \begin{equation}
                \mathcal{S}=\{\mathcal{S}_{n}:n\in\mathbb{N}\}
            \end{equation}
            Then $\mathcal{S}$ is countable, for
            $a:\mathbb{N}\rightarrow\mathcal{S}$ defined
            by $a_{n}=\mathcal{S}_{n}$ is a bijection.
            By the axiom of choice, there is a function:
            \begin{equation}
                \alpha:\mathcal{S}\rightarrow
                \bigcup_{n=1}^{\infty}\mathcal{S}_{n}
            \end{equation}
            Such that, for all $x\in\mathcal{S}$,
            $\alpha(x)\in{x}$. But then, for all
            $x\in\mathcal{S}$, $\alpha(x)$ is a subset
            of $A$. But for all $x\in\mathcal{S}$, there
            is an $n\in\mathbb{N}$ such that
            $a_{n}=x$. Thus, let $S$ be the following:
            \begin{equation}
                S=\bigcup_{n=1}^{\infty}\alpha(a_{n})
            \end{equation}
        \end{proof}
        \begin{theorem}
            \label{thm:Funct_Countable_Union_of_Countable}
            If $A$ is a countable set such that for all
            $\mathcal{U}\in{A}$, $\mathcal{U}$ is a
            countable set, and if for all $a,b\in{A}$,
            $a\cap{b}=\emptyset$, then
            $\bigcup_{\mathcal{U}\in{A}}\mathcal{U}$
            is countable set.
        \end{theorem}
        \textit{Sketch of Proof.} The proof of
        Thm.~\ref{thm:Funct_Countable_Union_of_Countable}
        follows in the same manner
        as proving that the rationals are countable. Since
        there are countably many sets, write them out in
        a list $\mathcal{U}_{1}$, $\mathcal{U}_{2}$, and
        so on. Then write out the elements in a table as
        follows:
        \begin{table}[H]
            \captionsetup{type=table}
            \centering
            \begin{tabular}{ccccc}
                $u_{11}$&$u_{12}$&$u_{13}$
                &$u_{14}$&$\hdots$\\
                $u_{21}$&$u_{22}$&$u_{23}$
                &$u_{24}$&$\hdots$\\
                $u_{31}$&$u_{32}$&$u_{33}$
                &$u_{34}$&$\hdots$\\
                $u_{41}$&$u_{42}$&$u_{43}$
                &$u_{44}$&$\hdots$\\
                $\vdots$&$\vdots$&$\vdots$
                &$\vdots$&$\ddots$
            \end{tabular}
            \caption{Construction of a Bijection on the
                     Countable Union of Countably Infinite
                     Sets.}
            \label{table:Func_Countable_Union_of_Countable}
        \end{table}
        Where $u_{nm}$ is the $m^{th}$ element of
        $\mathcal{U}_{n}$.
        Using the \textit{diagonal argument},
        we obtain:
        \begin{table}[H]
            \captionsetup{type=table}
            \centering
            \begin{tabular}{|c|c|c|c|c|c|c|c|c|c|c|}
                \hline
                $\mathbb{N}$&1&2&3&4&5&6&7&8&9&$\hdots$\\
                \hline
                $\bigcup_{\mathcal{U}\in{A}}\mathcal{U}$&
                $u_{11}$&$u_{12}$&$u_{21}$&$u_{13}$&
                $u_{22}$&$u_{31}$&$u_{14}$&$u_{23}$&
                $u_{32}$&$\hdots$\\
                \hline
            \end{tabular}
            \caption{The Bijection Between $\mathbb{N}$ and
                     $\bigcup_{\mathcal{U}\in{A}}\mathcal{U}$}
            \label{table:Func_Bijection_on_Countable_Union}
        \end{table}
        In the absence of the requirement that
        $a\cap{b}=\emptyset$ for all pairs in $\mathcal{U}$,
        we still have that the union is, at most, countable.
        The mapping we found would be a
        \textit{surjection}, rather than a bijection.
        The union is then either finite or countable. The
        Cantor-Schr\"{o}der-Bernstein Theorem can often be
        used to help identify the size of a set. This says
        that if $A$ and $B$ are sets such that there exists
        a surjective function $f:A\rightarrow{B}$ and a
        surjective function $g:B\rightarrow{A}$, then there
        is a bijective function $h:A\rightarrow{B}$. The
        requirement that $f$ and $g$ both be surjective
        can be replaced with the requirement that they both
        be injective. This is similar to saying that if
        $\mathrm{Card}(A)\leq\mathrm{Card}(B)$ and
        $\mathrm{Card}(B)\leq\mathrm{Card}(A)$,
        then $\mathrm{Card}(A)=\mathrm{Card}(B)$. Here, $\mathrm{Card}(A)$
        denotes the \textit{cardinality} of the set $A$.
        \begin{theorem}
            $\mathbb{R}$ is uncountable.
        \end{theorem}
        \textit{Sketch of Proof.} We'll show that the unit
        interval $(0,1)$ is uncountable. Suppose not.
        Let $r_{ij}$ be the $j^{th}$ decimal of the $i^{th}$
        element in the list. We construct the real number
        $d$ as follows: If $d_{j}$ denotes the $j^{th}$
        decimal in $d$, let $d_{j}=r_{jj}+1$ if
        $r_{jj}\ne{9}$, and $d_{j}=0$ otherwise. Then
        $d\in(0,1)$, but $d$ is not on the list. For it's not
        the $n^{th}$ element, for it differs in the
        $n^{th}$ decimal place. Thus there is no bijection.
        Therefore, $(0,1)$ is uncountable. By extension,
        $\mathbb{R}$ is uncountable.
        \par\hfill\par
        \vspace{-2ex}
        For a set $X$, we often write
        $\mathcal{P}(X)$ to denote the
        \textit{power set} of $X$. This is the
        set of all subsets of $X$.
        For any set $X$ you can show that $X$ is
        strictly smaller than $\mathcal{P}(X)$.
        For example, $\mathcal{P}(\mathbb{N})$
        can be shown to be equivalent to $\mathbb{R}$.
        Since $\mathbb{N}$ is stricly smaller than
        $\mathbb{R}$, one might ask if there exists
        a set $X$ such that $\mathbb{N}$ is strictly
        smaller than $X$, but $X$ is strictly smaller
        than $\mathbb{R}$. Continuing, you can ask the
        same thing about $\mathbb{R}$ and
        $\mathcal{P}(\mathbb{R})$, and so on.
        This is called the continuum hypothesis.
        It turns out to be independent of
        the standard axioms of mathematics.
        We begin by talking about cardinality. This is the
        \textit{size} of a set. For an infinite set it
        doesn't make sense to talk about the \textit{number}
        of elements, but we can specify what it means two
        sets to have the same size.
        \begin{ldefinition}{Equivalent Sets}{Equivalent_Sets}
            Equivalent sets are set $A$ and $B$ such that
            there exists a bijection $f:A\rightarrow{B}$.
        \end{ldefinition}
        The notion of equivalent sets defines an equivalence
        relation on sets. That is, the notion is reflexive,
        symmetric, and transitive.
        \begin{theorem}
            If $A$ is a set, then $A$ is equivalent to $A$.
        \end{theorem}
        \begin{proof}
            For let $\mathrm{id}_{A}:A\rightarrow{A}$
            be the identity mapping, $\mathrm{id}_{A}(x)=x$,
            then $\mathrm{id}_{A}$ is a bijection, and thus
            $A$ is equivalent to $A$.
        \end{proof}
        \begin{theorem}
            If $A$ and $B$ are sets and if $A$ is equivalent
            to $B$, then $B$ is equivalent to $A$.
        \end{theorem}
        \begin{proof}
            For if $A$ is equivalent to $B$, then there is
            a bijection $f:A\rightarrow{B}$. But if $f$ is a
            bijection, then the inverse function
            $f^{-1}:B\rightarrow{A}$ is well-defined and is
            a bijection. Thus $B$ is equivalent to $A$.
        \end{proof}
        \begin{theorem}
            If $A$, $B$, and $C$ are sets, if $A$ is
            equivalent to $B$, and if $B$ is equivalent to
            $C$, then $A$ is equivalent to $C$.
        \end{theorem}
        \begin{proof}
            For if $A$ is equivalent to $B$, then there is
            a bijection $f:A\rightarrow{B}$. But if $B$ is
            equivalent ot $C$, then there is a bijection
            $g:B\rightarrow{C}$. But then
            $g\circ{f}:A\rightarrow{C}$ is a bijection, and
            thus $A$ and $C$ are equivalent.
        \end{proof}
        A bijection is a function that is both injective and
        surjective. Thus, two equivalent sets can be put
        into a one-to-one correspondence and can be said to
        have the same size. We then say that $A$ and $B$
        have the same cardinality. The notation is written
        as $|A|=|B|$ or $\mathrm{Card}(A)=\mathrm{Card}(B)$. Cardinality
        splits sets into one of three categories.
        \begin{ldefinition}{Finite Sets}{Finite_Sets}
            A finite set is a set $A$ such that there exists
            an $n\in\mathbb{N}$ such that there is a
            bijection $f:\mathbb{Z}_{n}\rightarrow{A}$, or
            such that $A=\emptyset$.
        \end{ldefinition}
        Sets that are not finite are called infinite. There
        are two types of infinite sets. Let $\mathbb{N}$
        denote the set of positive integers, or
        \textit{natural} numbers.
        \begin{ldefinition}{Countably Infinite Sets}
              {Countably_Infinite}
            A countably infinite set is a set $A$ such that
            is a bijection $f:\mathbb{N}\rightarrow{A}$.
        \end{ldefinition}
        Combining the notions of finite sets and countably
        infinite sets, we get the notion of
        \textit{countable} sets.
        \begin{ldefinition}{Countable Sets}
              {Countable_Sets}
            A countable set is a set $A$ such that $A$ is
            either finite or countably infinite.
        \end{ldefinition}
        Countable sets are also called \textit{listable}.
        This is because if $A$ is a countably infinite set,
        and if $a:\mathbb{N}\rightarrow{A}$ is a bijection,
        we can write $A$ as:
        \begin{equation}
            A=\{\;a_{n}\,:\,n\in\mathbb{N}\;\}
            =\{\,a_{1},\,a_{2},\,\dots,\,a_{k},\,\dots\,\}
        \end{equation}
        If $A$ is finite, and if
        $a:\mathbb{Z}_{n}\rightarrow{A}$ is a
        bijection, then we can write:
        \begin{equation}
            A=\{\;a_{n}\,:\,n\in\mathbb{Z}_{n}\;\}
             =\{\,a_{1},\,\dots,\,a_{n}\,\}
        \end{equation}
        Recall that functions $a:\mathbb{N}\rightarrow{A}$
        are called \textit{sequences}, and the image of
        $n\in\mathbb{N}$ is written $a_{n}$, rather than
        $a(n)$.
        \begin{lexample}{}{Countable_Sets}
            There are many commonly discussed sets that are
            countably infinite. $\mathbb{N}$ is a trivial
            such example, but also $\mathbb{N}_{e}$ and
            $\mathbb{N}_{o}$, the sets of even and odd
            positive integers, respectively. For consider as
            bijections the following functions:
            \par
            \begin{subequations}
                \begin{minipage}[b]{0.49\textwidth}
                    \centering
                    \begin{equation}
                        f_{e}(n)=2n
                    \end{equation}
                \end{minipage}
                \hfill
                \begin{minipage}[b]{0.49\textwidth}
                    \centering
                    \begin{equation}
                        f_{0}(n)=2n-1
                    \end{equation}
                \end{minipage}
                \par\vspace{2.5ex}
                The set of all integers, $\mathbb{Z}$ is also
                countable, as shown in
                Fig.~\ref{fig:Bijection_N_and_Z}.
                One bijection is:
                \begin{equation}
                    f(n)=
                    \begin{cases}
                        \frac{n}{2},&n\mod{2}=0\\
                        \frac{1-n}{2},&n\mod{2}=1
                    \end{cases}
                \end{equation}
            \end{subequations}
            Any subset of $\mathbb{Z}$ is countable,
            and this is true of any countable set.
        \end{lexample}
        \begin{figure}[H]
            \centering
            \captionsetup{type=figure}
            %--------------------------------Dependencies----------------------------------%
%   amssymb                                                                    %
%   tikz                                                                       %
%       arrows.meta                                                            %
%-------------------------------Main Document----------------------------------%
\begin{tikzpicture}[%
    >=latex
]
    \draw[<->, thick] (-5, 0) to (5, 0) node[below] {$\mathbb{Z}$};
    \foreach\x in {-4, -3, -2, -1, 0, 1, 2, 3, 4}{%
        \draw (\x, -0.1) to (\x, 0.1);
        \node at (\x, -0.4) {\x};
    }
    \draw[->] (0, 0.2) arc(180:0:0.5 and 0.4);
    \draw[->] (1, -0.6) arc(0:-180:1 and 0.6);
    \draw[->] (-1, 0.2) arc(180:0:1.5 and 0.7);
    \draw[->] (2, -0.6) arc(0:-180:2 and 0.8);
    \draw[->] (-2, 0.2) arc(180:0:2.5 and 0.9);
    \draw[->] (3, -0.6) arc(0:-180:3 and 1);
    \draw[->] (-3, 0.2) arc(180:0:3.5 and 1.1);
    \draw[->] (4, -0.6) arc(0:-180:4 and 1.2);
\end{tikzpicture}
            \caption{Diagram of a Bijection Between
                     $\mathbb{N}$ and $\mathbb{Z}$.}
            \label{fig:Bijection_N_and_Z}
        \end{figure}
        One of the standard results about countable sets is
        that their subsets are also countable. This theorem
        relies, in a very subtle way, the use of the axiom
        of choice. There are a few stepping stones to get
        there. We will accept the various
        Cantor-Schr\"{o}eder-Bernstein theorems, which say
        the following:
        \begin{ltheorem}
              {First Cantor-Schr\"{o}eder-Bernstein Theorem}
              {First_Cantor_Schroeder_Bernstein}
            If $A$ and $B$ are sets such that there is an injective
            function $f:A\rightarrow{B}$ and an injective function
            $g:B\rightarrow{A}$, then there is a bijective function
            $h:A\rightarrow{B}$.
        \end{ltheorem}
        \begin{ltheorem}
              {Second Cantor-Schr\"{o}eder-Bernstein Theorem}
              {Second_Cantor_Schroeder_Bernstein}
            If $A$ and $B$ are sets such that there is a surjective
            function $f:A\rightarrow{B}$ and a surjective function
            $g:B\rightarrow{A}$, then there is a bijective function
            $h:A\rightarrow{B}$.
        \end{ltheorem}
        \par\hfill\par
        Using cardinalities, this says that if
        $\mathrm{Card}(A)\leq\mathrm{Card}(B)$ and $\mathrm{Card}(B)\leq\mathrm{Card}(A)$, then
        $\mathrm{Card}(A)=\mathrm{Card}(B)$. With this notation it becomes more
        intuitive. We will use this to prove that various sets are
        countable. Many sets that appear to be larger than $\mathbb{N}$
        can shown to to be the same size as $\mathbb{N}$ by finding
        a simple injective function, without finding an explicit
        bijection.
        \begin{ltheorem}
              {Third Cantor-Schr\"{o}eder-Bernstein Theorem}
              {Third_Cantor_Schroeder_Bernstein}
            If $A$, $B$, and $C$ are sets such that
            $A\subseteq{B}\subseteq{C}$, and if $A$ and $C$ are equivalent
            sets, then $B$ and $C$ are equivalent sets.
        \end{ltheorem}
        \par\hfill\par
        This says that if $\mathrm{Card}(A)\leq\mathrm{Card}(B)\leq\mathrm{Card}(C)$,
        and if $\mathrm{Card}(A)=\mathrm{Card}(C)$, then $\mathrm{Card}(B)=\mathrm{Card}(C)$.
        \begin{theorem}
            \label{thm:Measure_Theory_NxN_Is_Countable}
            $\mathbb{N}\times\mathbb{N}$ is countably infinite.
        \end{theorem}
        \begin{proof}
            There is a trivial injection
            $f:\mathbb{N}\rightarrow\mathbb{N}\times\mathbb{N}$
            defined by:
            \begin{equation}
                f(n)=(n,0)
            \end{equation}
            There is also an injection
            $g:\mathbb{N}\times\mathbb{N}\rightarrow\mathbb{N}$
            defined by:
            \begin{equation}
                g(n.m)=2^{n}3^{m}
            \end{equation}
            Since 2 and 3 are co-prime, if
            $g(n_{1},m_{1})=g(n_{2},m_{2})$, then
            $(n_{1},m_{1})=(n_{2},m_{2})$. Thus, $g$ is an injection.
            By the Cantor-Schr\"{o}eder-Bernstein Theorem, there is a
            bijection $h:\mathbb{N}\rightarrow\mathbb{N}\times\mathbb{N}$.
        \end{proof}
        One can intuitively see that the set of all positive
        rational numbers $\mathbb{Q}^{+}$ is countable by examining
        the zig-zag pattern shown in
        Fig.~\ref{fig:Bijection_N_and_Q_Plus}.
        Thm.~\ref{thm:Measure_Theory_NxN_Is_Countable} also
        shows this in a more rigorous way that. We can create
        a one-to-one correspondence with
        $\mathbb{N}\times\mathbb{N}$ by mapping
        $pq^{\minus{1}}\mapsto(p,q)$. Thus $\mathbb{Q}^{+}$
        and $\mathbb{N}\times\mathbb{N}$ are equivalent sets.
        But $\mathbb{N}\times\mathbb{N}$ and $\mathbb{N}$
        are equivalent sets, and therefore $\mathbb{Q}^{+}$
        is countable.
        Thm.~\ref{thm:Measure_Theory_NxN_Is_Countable} can also be used
        to show that the countable union of countable sets is also
        countable.
        \begin{ltheorem}{Equivalence of Countable Sets}
              {Countable_iff_exists_inj_to_N}
            A set $A$ is countable if and only if there is an injective
            function $f:A\rightarrow\mathbb{N}$.
        \end{ltheorem}
        Thm.~\ref{thm:Countable_iff_exists_inj_to_N} seems
        intuitively obvious, the injective function is
        simply the listing function. For a finite set, this
        is precisely how one constructs such an injection.
        For an infinite set $A$, this is equivalent to
        showing that any infinite subset of $\mathbb{N}$ is
        equivalent to $\mathbb{N}$. The standard proof
        using \textit{induction}, but actually has the axiom
        of choice underlying it.
        \begin{theorem}
            If $\mathcal{A}$ is a countably infinite set
            such that, for all $A\in\mathcal{A}$, $A$ is
            a non-empty countable set, then the set:
            \begin{equation}
                S=\bigcup_{A\in\mathcal{A}}A
            \end{equation}
            Is a countable set.
        \end{theorem}
        \begin{proof}
            If $\mathcal{F}$ is finite, then we are done. Suppose not.
            Let $A:\mathbb{N}\rightarrow\mathcal{A}$ be a bijection,
            and define:
            \begin{equation}
                S=\bigcup_{n\in\mathbb{N}}A_{n}
            \end{equation}
            Also, let:
            \begin{equation}
                \mathcal{F}_{n}
                =\{f:A_{n}\rightarrow\mathbb{N}:
                    f\textrm{ is injective}\}
            \end{equation}
            Since, for all $n\in\mathbb{N}$, $A_{n}$ is
            non-empty and countable, $\mathcal{F}_{n}$
            is non-empty. Let:
            \begin{equation}
                \mathcal{F}
                =\bigcup_{n\in\mathbb{N}}\mathcal{F}_{n}
            \end{equation}
            Thus, by the axiom of choice, there is a function
            $F:\mathbb{N}\rightarrow\mathcal{F}$ such that, for all
            $n\in\mathbb{N}$, $F_{n}\in\mathcal{F}_{n}$. For
            $x\in{S}$, let:
            \begin{equation}
                \varphi_{x}
                =\inf\{n\in\mathbb{N}:x\in{A}_{n}\}
            \end{equation}
            By the well-ordering of $\mathbb{N}$, for all
            $x\in{S}$, $\varphi_{x}$ is well defined. Let
            $\phi:S\rightarrow\mathbb{N}\times\mathbb{N}$
            be defined by:
            \begin{equation}
                \phi(x)
                =\big(\varphi_{x},F_{\varphi_{x}}(x)\big)
            \end{equation}
            Then $\phi$ is an injection. For if
            $\big(\varphi_{x},F_{\varphi_{x}}(x)\big)=%
             \big(\varphi_{y},F_{\varphi_{x}}(y)\big)$, then
            $\varphi_{x}=\varphi_{y}$, and thus
            $F_{\varphi(x)}(x)=F_{\varphi(x)}(y)$. But
            $F_{\varphi_{x}}$ is an injection, and
            thus $x=y$. Therefore $\phi$ is an injection.
            But $\mathbb{N}\times\mathbb{N}$ and $\mathbb{N}$
            are equivalent sets, and thus there's an
            injection $f:\mathbb{N}\times\mathbb{N}$. And
            the composition of injective functions is again
            injective, and thus
            $\phi\circ{f}:S\rightarrow\mathbb{N}$ is an
            injective function. But by
            Thm.~\ref{thm:Countable_iff_exists_inj_to_N},
            if there is an injective function
            $f:S\rightarrow\mathbb{N}$, then $S$ is
            countable. Therefore, etc.
        \end{proof}
        \begin{theorem}
            If $X$ is infinite, then there exists a
            countably infinite set $A\subseteq{X}$.
        \end{theorem}
        \begin{proof}
            If $A$ is finite, then we are done. Suppose not.
            For $n\in\mathbb{N}$, let:
            \begin{equation}
                A_{n}
                =\{g:\mathbb{Z}_{n}\rightarrow{A}:f\textrm{ is inective}\}
            \end{equation}
            Also, define:
            \begin{equation}
                \mathcal{F}=\bigcup_{n\in\mathbb{N}}A_{n}
            \end{equation}
            But by the axiom of choice, there is a function
            $f:\mathbb{N}\rightarrow\mathcal{F}$ such that
            $f_{n}\in{A}_{n}$. But then, for all
            $n\in\mathbb{N}$, the range of $f_{n}$ is finite.
            \begin{equation}
                A=\bigcup_{n\in\mathbb{N}}f_{n}
                    \Big(\mathbb{Z}_{n}\Big)
            \end{equation}
            Then $A\subseteq{X}$ is countably infinite.
        \end{proof}
        The set of rational numbers, $\mathbb{Q}$, is also
        countable. We may intuitively think of $\mathbb{N}$
        as being smaller than $\mathbb{Q}$, since there are
        simple \textit{surjections} that can be constructed
        from $\mathbb{Q}$ to $\mathbb{N}$. There is also a
        surjection from $\mathbb{N}$ onto $\mathbb{Q}^{+}$,
        as is shown in Fig.~\ref{fig:Bijection_N_and_Q_Plus}.
        To construct such a surjection, write out all of the
        positive rational numbers in a grid so that $a_{nm}$
        is the number $n/m$. Then zig-zag along the diagonals
        to construct the function. Thus there is a surjection
        $f:\mathbb{Q}^{+}\rightarrow\mathbb{N}$
        and a surjection
        $g:\mathbb{N}\rightarrow\mathbb{Q}^{+}$.
        \begin{figure}[H]
            \centering
            \captionsetup{type=figure}
            \resizebox{0.7\textwidth}{!}{%
                %--------------------------------Dependencies----------------------------------%
%   tikz                                                                       %
%       arrows.meta                                                            %
%-------------------------------Main Document----------------------------------%
\begin{tikzpicture}[%
    >=latex
]
    \foreach\y in {1, 2, 3, 4, 5, 6}{%
        \foreach\x in {1, 2, 3, 4, 5, 6}{%
            \node (\x\y) at (\x, 7-\y) {$\frac{\x}{\y}$};
        }
    }
    \foreach\x in {1, 2, 3, 4, 5, 6}{%
        \node at (7, \x) {$\cdots$};
        \node at (\x, 0) {$\vdots$};
    }
    \node at (7, 0) {$\ddots$};
    \draw[->] (11) to (12);
    \draw[->] (12) to (21);
    \draw[->] (21) to (31);
    \draw[->] (31) to (22);
    \draw[->] (22) to (13);
    \draw[->] (13) to (14);
    \draw[->] (14) to (23);
    \draw[->] (23) to (32);
    \draw[->] (32) to (41);
    \draw[->] (41) to (51);
    \draw[->] (51) to (42);
    \draw[->] (42) to (33);
    \draw[->] (33) to (24);
    \draw[->] (24) to (15);
    \draw[->] (15) to (16);
    \draw[->] (16) to (25);
    \draw[->] (25) to (34);
    \draw[->] (34) to (43);
    \draw[->] (43) to (52);
    \draw[->] (52) to (61);
    \draw[->] (61) to (62);
    \draw[->] (62) to (53);
    \draw[->] (53) to (44);
    \draw[->] (44) to (35);
    \draw[->] (35) to (26);
    \draw[->] (36) to (45);
    \draw[->] (45) to (54);
    \draw[->] (54) to (63);
    \draw[->] (64) to (55);
    \draw[->] (55) to (46);
    \draw[->] (56) to (65);
\end{tikzpicture}
            }
            \caption{Diagram of a Surjection from
                     $\mathbb{N}$ onto $\mathbb{Q}^{+}$.}
            \label{fig:Bijection_N_and_Q_Plus}
        \end{figure}
        We can modify Fig.~\ref{fig:Bijection_N_and_Q_Plus}
        slightly to create a surjection between $\mathbb{N}$
        and $\mathbb{Q}$, see
        Fig.~\ref{fig:Bijection_N_and_Q}.
        It is important to note that this bijection will not
        preserve the order of the rational numbers. The
        bijection will have to jump around back and forth.
        Any such bijection will be forced to do this, as the
        rationals are everywhere dense on $\mathbb{R}$. Any
        monotonic sequence of $\mathbb{Q}$ cannot possibly
        be a bijection.
        \begin{theorem}
            If $A$ is a countably infinite set and
            $B\subseteq{A}$, then $B$ is countable.
        \end{theorem}
        \begin{proof}
            As $A$ is countably infinite, there is a bijection
            $a:\mathbb{N}\rightarrow{A}$. Define:
            \begin{equation}
                K=\{n\in\mathbb{N}:a_{n}\in{B}\}
            \end{equation}
            As $B\subseteq{A}$,
            this set contains a subsequence of points in
            $\mathbb{N}$ that get mapped into $B$. If $K$ is finite,
            then $B$ is finite, and if not then $K$ is countably
            infinite, and thus $B$ is countably infinite.
        \end{proof}
        \begin{figure}[H]
            \centering
            \captionsetup{type=figure}
            \resizebox{\textwidth}{!}{%
                %--------------------------------Dependencies----------------------------------%
%   amsmath                                                                    %
%   tikz                                                                       %
%       arrows.meta                                                            %
%-------------------------------Main Document----------------------------------%
\begin{tikzpicture}[%
    >=latex
]
    \foreach\y in {1, 2, 3, 4}{%
        \foreach\x in {-4, -3, -2, -1, 0, 1, 2, 3, 4}{%
            \node (\x\y) at (\x, 7-\y) {$\frac{\x}{\y}$};
        }
    }
    \foreach\x in {-4, -3, -2, -1, 0, 1, 2, 3, 4}{%
        \node at (\x, 2) {$\vdots$};
    }
    \foreach\y in {3, 4, 5, 6}{%
        \node at (5, \y) {$\cdots$};
        \node at (-5, \y) {$\cdots$};
    }
    \node at (5, 2) {$\ddots$};
    \node at (-5, 2) {$\reflectbox{\ensuremath{\ddots}}$};
    \draw[->] (01) to (11);
    \draw[->] (11) to (12);
    \draw[->] (12) to (02);
    \draw[->] (02) to (-12);
    \draw[->] (-12) to (-11);
    \draw[->] (-1, 6.3) to (-1, 6.6)
                        to (2, 6.6)
                        to (2, 6.3);
    \draw[->] (21) to (22);
    \draw[->] (22) to (23);
    \draw[->] (23) to (13);
    \draw[->] (13) to (03);
    \draw[->] (03) to (-13);
    \draw[->] (-13) to (-23);
    \draw[->] (-23) to (-22);
    \draw[->] (-22) to (-21);
    \draw[->] (-2, 6.3) to (-2, 6.8)
                        to (3, 6.8)
                        to (3, 6.3);
    \draw[->] (31) to (32);
    \draw[->] (32) to (33);
    \draw[->] (33) to (34);
    \draw[->] (34) to (24);
    \draw[->] (24) to (14);
    \draw[->] (14) to (04);
    \draw[->] (04) to (-14);
    \draw[->] (-14) to (-24);
    \draw[->] (-24) to (-34);
    \draw[->] (-34) to (-33);
    \draw[->] (-33) to (-32);
    \draw[->] (-32) to (-31);
    \draw[->] (-3, 6.3) to (-3, 7)
                        to (4, 7)
                        to (4, 6.3);
    \draw[->] (41) to (42);
    \draw[->] (42) to (43);
    \draw[->] (43) to (44);
    \draw[->] (44) to (4, 2.3);
    \draw[->] (-4, 2.3) to (-44);
    \draw[->] (-44) to (-43);
    \draw[->] (-43) to (-42);
    \draw[->] (-42) to (-41);
\end{tikzpicture}
            }
            \caption{Diagram of a Surjection from
                     $\mathbb{N}$ onto $\mathbb{Q}$.}
            \label{fig:Bijection_N_and_Q}
        \end{figure}
        \begin{theorem}
            If $A$ is an infinite set, then there exists a
            countable subset $B\subseteq{A}$.
        \end{theorem}
        \begin{proof}
            If $A$ is infinite then there is an
            $a_{1}\in{A}$. But, as $A$ is infinite,
            $A\setminus\{a_{1}\}$ is infinite, and there
            is an $a_{2}\in{A}\setminus\{a_{1}\}$. Continuing
            we obtain a sequence of distinct elements in $A$.
            Let $B=\{a_{n}:n\in\mathbb{N}\}$. Then
            $B\subseteq{A}$, and $B$ is countable.
        \end{proof}
        \begin{lexample}{}{Disjoint_Union_of_Intervals_is_Countable}
            Suppose we have a collection of disjoint intervals
            of $\mathbb{R}$. This collection is either finite
            or countable. For in every interval, choose a
            rational number $q_{n}$. Let
            $A=\{q_{1},q_{2},\hdots\}$. Then
            $A\subseteq\mathbb{Q}$, and thus $A$ is either
            finite or countable. But this is also an enumeration
            of the intervals in the collection, and thus the
            collection is either finite or countable.
        \end{lexample}
        Given a countable collection of sets
        $A=\{\mathcal{A}_{1},\mathcal{A}_{2},\hdots\}$ such
        that, for all $n\in\mathbb{N}$, $\mathcal{A}_{n}$ is
        also a countable set, then the union is countable. That is:
        \begin{equation}
            B=\bigcup_{n=1}^{\infty}\mathcal{A}_{n}
        \end{equation}
        is a countable set. The proof of this is a mimicry of
        the proof of the countability of $\mathbb{Q}$. Not
        every set is either finite or countable. The real numbers,
        $\mathbb{R}$, is an example of an \textit{uncountable}
        set. First, some notes on the power set of a set.
        \begin{lexample}{}{Basic_Power_Set}
            \begin{subequations}
                If $\Omega=\{1,2\}$, then the power set is:
                \begin{equation}
                    \mathcal{P}(\Omega)=
                    \big\{\emptyset,\{1\},\{2\},\{1,2\}\big\}
                \end{equation}
                We must consider the empty set, since for any set
                $A$, $\emptyset\subseteq{A}$. As another example,
                let $\Omega=\{1,2,3\}$. Then:
                \begin{equation}
                    \mathcal{P}(\Omega)=
                    \big\{\emptyset,\{1\},\{2\},\{3\},\{1,2\},
                      \{1,3\},\{2,3\},\{1,2,3\}\big\}
                \end{equation}
                We see that, in the first example, a set with
                2 elements has a power set with 4 elements. In the
                second example we see that a set with 3 elements has
                a power set with 8 elements. This pattern continues
                for finite sets. If $A$ has $n$ elements, then
                $\mathcal{P}(A)$ has $2^{n}$ elements. If
                $A$ is an infinite set, then $\mathcal{P}(A)$ is
                uncountable. Indeed:
                \begin{equation}
                    \mathrm{Card}\big(\mathcal{P}(\mathbb{N})\big)=
                    \mathrm{Card}(\mathbb{R})
                \end{equation}
                We can show this by using the binary representation
                of real numbers. We construct a bijection as
                follows: If $A\subseteq\mathbb{N}$, then
                let $r_{A}=0.n_{1}n_{2}\hdots$ where:
                \begin{equation}
                    n_{i}=
                    \begin{cases}
                        0,&i\notin{A}\\
                        1,&i\in{A}
                    \end{cases}
                \end{equation}
                The function
                $f:\mathcal{P}(\mathbb{N})\rightarrow[0,1]$
                defined by $f(A)=r_{A}$ is thus a bijection.
                That is, every element of $[0,1]$ gets mapped to in
                a one-to-one manner. The potentially tricky numbers are
                0 and 1, but $f(\emptyset)=0$, and $f(\mathbb{N})=1$.
                Thus $\mathcal{P}(\mathbb{N})$ and $[0,1]$ are of the
                same cardinality. But $(0,1)$ and $\mathbb{R}$
                are of the same cardinality. To see this, consider
                the graph of the function
                $g:(0,1)\rightarrow\mathbb{R}$ defined as:
                \begin{equation}
                    g(x)=\frac{2x-1}{x(1-x)}
                \end{equation}
                This is a bijection between the unit interval
                $(0,1)$ and $\mathbb{R}$. One can also use the
                \textit{stereographic projection} to show this.
                But also $[0,1]$ and $(0,1)$ have the same cardinality.
                For this, consider the following function:
                \begin{equation}
                    f(x)=
                    \begin{cases}
                        \frac{1}{2},&x=0\\
                        \frac{1}{2^{n+2}},&x=\frac{1}{2^{n}}\\
                        x,&\textrm{Otherwise}
                    \end{cases}
                \end{equation}
                A graph of this is shown in
                Fig.~\ref{fig:Measure_Theory_Bijection_Closed_I_to_Open}.
                Therefore, $\mathbb{R}$ and
                $\mathcal{P}(\mathbb{N})$ have the same cardinality.
                This can then be used to show that $\mathbb{R}$ is
                uncountable.
            \end{subequations}
        \end{lexample}
        \begin{figure}[H]
            \centering
            \captionsetup{type=figure}
            \documentclass[crop,class=article]{standalone}
%----------------------------Preamble-------------------------------%
\usepackage{tikz}                       % Drawing/graphing tools.
\usetikzlibrary{arrows.meta}            % Latex and Stealth arrows.
%--------------------------Main Document----------------------------%
\begin{document}
    \begin{tikzpicture}[>=Latex, scale=2]
        \draw[->] (-0.15in, 0) to (1.1in, 0) node[above] {$x$};
        \draw[->] (0, -0.15in) to (0, 1.1in) node[right] {$y$};
        \draw (0, 0) to (1in, 1in);
        \draw[fill=black, draw=black] (0, 0.5in) circle (0.3mm);
        \foreach\x in{1in, 0.5in, 0.25in, 0.125in, 0.0625in, 0.03125in}{
            \draw[fill=white, draw=black] (\x, \x) circle (0.3mm);
            \draw[fill=black, draw=black] (\x, 0.25*\x) circle (0.3mm);
        }
    \end{tikzpicture}
\end{document}
            \caption{Bijection from $[0,1]$ to $(0,1)$.}
            \label{fig:Measure_Theory_Bijection_Closed_I_to_Open}
        \end{figure}
        The power set of any set is strictly larger than the
        original set. If $\Omega$ is finite with $n$ elements, it
        can be shown that $\mathcal{P}(\Omega)$ has $2^{n}$
        elements. For infinite sets, there is a trivial surjection
        from $\mathcal{P}(\Omega)$ onto $\Omega$: for any element
        $x$, let $f(\{x\})=x$. This shows that
        $\mathrm{Card}(\Omega)\leq\mathrm{Card}(\mathcal{P}(\Omega))$. We now show
        that the inequality is strict.
        \begin{theorem}
            If $\Omega$ is a set, then there is no bijection
            $f:\Omega\rightarrow\mathcal{P}(\Omega)$
        \end{theorem}
        \begin{proof}
            For suppose not, and let
            $f:\Omega\rightarrow\mathcal{P}(\Omega)$ be such a
            bijection. Define:
            \begin{equation}
                A=\{x\in\Omega:x\in{f}(x)\}
            \end{equation}
            Then $A\subseteq\Omega$, and thus
            $A\in\mathcal{P}(\Omega)$. But then the complement of
            $A$ is also an element of $\mathcal{P}(\Omega)$. But
            $f$ is a bijection and thus there is an $x\in\Omega$
            such that $f(x)=A^{C}$. If $x\in{f}(x)$, then
            $x\in{A}$, a contradiction as $f(x)=A^{C}$, and thus
            $x\in{A}^{C}$ as well. Therefore $x\notin{f}(x)$. But
            then $x\in{A}^{C}$. But, from the definition of $A$,
            since $x\in{A}^{C}$ and $f(x)=A^{C}$, $x\in{f}(x)$
            and thus $x\in{A}$, a contradiction. Thus there is no
            $x$ such that $f(x)=A^{C}$. Therefore, $f$ is not a
            bijection.
        \end{proof}
        From this we conclude that $\mathcal{P}(\mathbb{N})$
        is an uncountable infinite set. But since $\mathbb{R}$
        and $\mathcal{P}(\mathbb{N})$ have the same cardinality,
        $\mathbb{R}$ is also uncountable.
        If a set $A$ has the same cardinality as $\mathbb{R}$,
        we say that $A$ has the cardinality of the continuum.
        \begin{lexample}{}{Bijection_from_closed_to_open_interval}
            There is a bijection between the open unit
            square $(0,1)\times(0,1)$ and the open unit interval
            $(0,1)$. For an element $(x,y)\in(0,1)\times(0,1)$,
            let $z\in(0,1)$ be defined as
            $z=0.x_{1}y_{1}x_{2}y_{2}x_{3}y_{3}\dots$ This is
            a bijection, for all $(x,y)$ in the square there is
            a corresponding $z\in(0,1)$, and for all
            $z\in(0,1)$ there is a corresponding element of
            $(0,1)\times(0,1)$. We can say that $(x,y)$ can
            be coded into $z$, and $z$ can be decoded into
            $(x,y)$. Hence, $(0,1)\times(0,1)$ has the cardinality
            of the continuum. By stereographic projection and induction
            we obtain:
            \par\hfill\par
            \begin{subequations}
                \begin{minipage}[b]{0.49\textwidth}
                    \centering
                    \begin{equation}
                        \mathrm{Card}(\mathbb{R}^{2})=\mathrm{Card}(\mathbb{R})
                    \end{equation}
                \end{minipage}
                \hfill
                \begin{minipage}[b]{0.49\textwidth}
                    \centering
                    \begin{equation}
                        \mathrm{Card}(\mathbb{R}^{n})=\mathrm{Card}(\mathbb{R})
                    \end{equation}
                \end{minipage}
                \par
            \end{subequations}
        \end{lexample}
        \begin{lexample}{}{Space_of_Sequences}
            Consider the set of all real-valued sequences. We've seen
            that any real number can be represented as a function
            $f:\mathbb{N}\rightarrow\{0,1\}$. A real-valued sequence
            is a function $a:\mathbb{N}\rightarrow\mathbb{R}$, and
            thus the set of real-valued sequences can be seen as the
            set of functions whose domain is $\mathbb{N}$ and whose
            range is the set of all functions
            $f:\mathbb{N}\rightarrow\{0,1\}$. So given a sequence
            $a$, the image of $a_{n}$, for $n\in\mathbb{N}$, is a
            function $f_{n}:\mathbb{N}\rightarrow\{0,1\}$. Therefore
            the set of all real-valued sequences can be represented
            as the set of all functions
            $g:\mathbb{N}\times\mathbb{N}\rightarrow\{0,1\}$, where
            $g(n,m)=f_{n}(m)$. But $\mathbb{N}\times\mathbb{N}$ is
            countable, and thus the set of all functions of the form
            $g:\mathbb{N}\times\mathbb{N}\rightarrow\{0,1\}$ has the
            same cardinality as the set of all functions of the form
            $f:\mathbb{N}\rightarrow\{0,1\}$. But this has the
            cardinality of the continuum. Therefore, the set of all
            real-valued sequences has the cardinality of the continuum.
        \end{lexample}

    \renewcommand{\PATH}{\OLDPATH}
\endgroup
    %     \part{Category Theory}
    %         \input{\PATH/Category_Theory/Category_Theory.tex}
    %     \part{Model Theory}
    %         \documentclass[crop=false,class=book,oneside]{standalone}
%----------------------------Preamble-------------------------------%
%---------------------------Packages----------------------------%
\usepackage{geometry}
\geometry{b5paper, margin=1.0in}
\usepackage[T1]{fontenc}
\usepackage{graphicx, float}            % Graphics/Images.
\usepackage{natbib}                     % For bibliographies.
\bibliographystyle{agsm}                % Bibliography style.
\usepackage[french, english]{babel}     % Language typesetting.
\usepackage[dvipsnames]{xcolor}         % Color names.
\usepackage{listings, lstlinebgrd}      % Verbatim-Like Tools.
\usepackage{mathtools, esint, mathrsfs} % amsmath and integrals.
\usepackage{amsthm, amsfonts}           % Fonts and theorems.
\usepackage{tabularx}
\usepackage{tcolorbox}                  % Frames around theorems.
\usepackage{upgreek}                    % Non-Italic Greek.
\usepackage{paracol}                    % Two-column styling.
\usepackage{wrapfig}                    % Wrap text around figure.
\usepackage{fmtcount, etoolbox}         % For the \book{} command.
\usepackage[newparttoc]{titlesec}       % Formatting chapter, etc.
\usepackage{titletoc}                   % Allows \book in toc.
\usepackage[nottoc]{tocbibind}          % Bibliography in toc.
\usepackage[titles]{tocloft}            % ToC formatting.
\usepackage{multicol, enumitem}         % Multi-column/enumerate.
\usepackage{import}                     % Import external files.
\usepackage{pgfplots, tikz}             % Drawing/graphing tools.
\usetikzlibrary{
    calc,                   % Calculating right angles and more.
    angles,                 % Drawing angles within triangles.
    arrows.meta,            % Latex and Stealth arrows.
    quotes,                 % Adding labels to angles.
    positioning,            % Relative positioning of nodes.
    decorations.markings,   % Adding arrows in the middle of a line.
    patterns,
    arrows,
    shapes,
    shapes.geometric,
    cd,
    hobby,
    babel
}                                       % Libraries for tikz.
\pgfplotsset{compat=1.9}                % Version of pgfplots.
\usepackage[font=scriptsize,
            labelformat=simple,
            labelsep=colon]{subcaption} % Subfigure captions.
\usepackage[font={scriptsize},
            hypcap=true,
            labelsep=colon]{caption}    % Figure captions.
\usepackage{hyperref}                   % Allows for hyperlinks.
\hypersetup{
    colorlinks=true,
    linkcolor=blue,
    filecolor=magenta,
    urlcolor=Cerulean,
    citecolor=SkyBlue
}                           % Colors for hyperref.
\usepackage[toc,acronym,nogroupskip]{glossaries} % Glossaries and acronyms.
\usepackage[subpreambles=false]{standalone}      % Complileable sub files.

% Various font stuff from kiwi.
% Use this for Times text and Computer Modern math
%\usepackage{times}

% Quite nice
%\usepackage[charter, greekfamily=, greekuppercase=italicized]{mathdesign}
%\usepackage[utopia, greekuppercase=italicized]{mathdesign}    % Math is narrower

% Use this for Times text and math
%\usepackage{newtxtext}
%\usepackage[libertine,cmintegrals]{newtxmath}
%\usepackage{fix-cm}

%\usepackage{txfontsb}
% or
%\usepackage{mathptmx}

%\usepackage[scaled=0.92]{helvet}
%\renewcommand{\rmdefault}{ptm}

%\usepackage{mathpazo}    % add possibly `sc` and `osf` options
%\usepackage{eulervm}

%\usepackage{fourier}
%\renewcommand{\rmdefault}{ptm}
%\usepackage{mathptm}

%\usepackage{fontspec}
%\setmainfont{lmodern}

%\usepackage[varg]{txfonts}
%\usepackage{fouriernc}
%\usepackage{mathpazo}

%\usepackage{bookman}
%\usepackage[scaled]{uarial}
%\usepackage[scaled]{helvet}
%\renewcommand*\familydefault{\sfdefault}
%\usepackage[math]{anttor}

%\newcommand\fgeorgia{\fontfamily{jvn}\selectfont}
%\newcommand\ftimes{\fontfamily{ptm}\selectfont}
%\newcommand\fhelvetica{\fontfamily{phv}\selectfont}
%\newcommand\fcourier{\fontfamily{pcr}\selectfont}
%\newcommand\fbookman{\fontfamily{pbk}\selectfont}
%\newcommand\fnewcentury{\fontfamily{pnc}\selectfont}
%\newcommand\fpalatino{\fontfamily{ppl}\selectfont}
%\newcommand\favantgarde{\fontfamily{pag}\selectfont}
%\newcommand\fnormal{\normalfont}
%\newcommand\fsize[1]{\ifnum#1>0\fontsize{#1}{#1}\selectfont\else\normalsize\fi}
%------------------------Theorem Styles-------------------------%
% Define theorem style for default spacing and normal font.
\newtheoremstyle{normal}
    {\topsep}               % Amount of space above the theorem.
    {\topsep}               % Amount of space below the theorem.
    {}                      % Font used for body of theorem.
    {}                      % Measure of space to indent.
    {\bfseries}             % Font of the header of the theorem.
    {}                      % Punctuation between head and body.
    {.5em}                  % Space after theorem head.
    {}

% Define theorem style for default spacing with italicized font.
\newtheoremstyle{normalit}{\topsep}{\topsep}
                {\itshape}{}{\bfseries}{}{.5em}{}

% Italic header environment.
\newtheoremstyle{thmit}{\topsep}{\topsep}{}{}{\itshape}{}{0.5em}{}

% Define italicized environments.
\theoremstyle{normalit}
\newtheorem{theorem}{Theorem}[section]
\newtheorem{lemma}{Lemma}[section]
\newtheorem{corollary}{Corollary}[section]
\newtheorem{proposition}{Proposition}[section]
\newtheorem*{theorem*}{Theorem}

% Define environments with italic headers.
\theoremstyle{thmit}
\newtheorem*{solution}{Solution}
\newtheorem*{fsolution}{Solution}

% Define default environments.
\theoremstyle{normal}
\newtheorem{example}{Example}[section]
\newtheorem{definition}{Definition}[section]
\newtheorem{problem}{Problem}[section]
\newtheorem{question}{Question}[section]
\newtheorem{remark}{Remark}[section]
\newtheorem{properties}{Properties}[section]
\newtheorem{notation}{Notation}[section]
\newtheorem{axiom}{Axiom}[section]
\newtheorem*{properties*}{Properties}
\newtheorem*{remark*}{Remark}
\newtheorem*{definition*}{Definition}
\theoremstyle{plain}

% Define framed environment.
\tcbuselibrary{most}
\newtcbtheorem[use counter*=theorem]{ftheorem}{Theorem}%
    {colback=green!5,colframe=green!35!black,
     fonttitle=\bfseries\upshape}{th}

\newtcbtheorem[use counter*=example]{fdefinition}{Definition}%
    {fonttitle=\bfseries\upshape,
     colback=blue!5!white,colframe=blue!75!black}{def}

\newtcbtheorem[use counter*=example]{fexample}{Example}%
    {fonttitle=\bfseries\upshape,
     colback=red!5!white,colframe=red!75!black}{ex}

\newtcbtheorem[use counter*=notation]{fnotation}{Notation}%
    {fonttitle=\bfseries\upshape,
     colback=SeaGreen!5!white,colframe=SeaGreen!75!black}{ex}

\newtcbtheorem[use counter*=corollary]{fcorollary}{Corollary}%
    {fonttitle=\bfseries\upshape,
     colback=Orchid!5!white,colframe=Orchid!75!black}{ex}

\newenvironment{bproof}{\textit{Proof.}}{\hfill$\square$}
\tcolorboxenvironment{bproof}{blanker,breakable,left=5mm,
                             before skip=10pt,after skip=10pt,
                             borderline west={1mm}{0pt}{red}}
\tcolorboxenvironment{fsolution}
    {enhanced jigsaw,colframe=cyan,interior hidden,breakable}

%--------------------Declared Math Operators--------------------%
\DeclareMathOperator{\Refl}{Refl}           % Reflection operator.
\DeclareMathOperator{\Span}{Span}           % Span of a set of vectors.
\DeclareMathOperator{\Card}{Card}           % Cardinality of set.
\DeclareMathOperator{\Ord}{Ord}             % Ordinal of ordered set.
\DeclareMathOperator{\Tr}{Tr}               % Trace of matrix.
\DeclareMathOperator{\adjoint}{adj}         % Adjoint of matrix.
\DeclareMathOperator{\rk}{rk}               % Rank of operator.
\DeclareMathOperator{\nul}{nul}             % Null space of operator.
\DeclareMathOperator{\sgn}{sgn}             % Sign of a number.
\DeclareMathOperator{\multideg}{mutlideg}   % Multi-Degree (Graphs).
\DeclareMathOperator{\GCD}{GCD}             % Greatest common denominator.
\DeclareMathOperator{\LM}{LM}               % Leading monomial
\DeclareMathOperator{\LC}{LC}               % Leading coefficient.
\DeclareMathOperator{\LT}{LT}               % Leading term.
\DeclareMathOperator{\LCM}{LCM}             % Least common multiple.
\DeclareMathOperator{\Mon}{Mon}             % Monomial.
\DeclareMathOperator{\Spec}{Spec}           % Spectrum.
\DeclareMathOperator{\proj}{proj}           % Projection.
\DeclareMathOperator{\comp}{comp}           % Component.
\DeclareMathOperator{\sinc}{sinc}           % Sinc function.
\DeclareMathOperator{\Ima}{Im}              % Image of operator.
\DeclareMathOperator{\Prin}{Prin}           % Principal value.
\DeclareMathOperator{\Mod}{mod}             % Modulus.
%------------------------New Commands---------------------------%
\DeclarePairedDelimiter\norm{\lVert}{\rVert}
\DeclarePairedDelimiter\ceil{\lceil}{\rceil}
\DeclarePairedDelimiter\floor{\lfloor}{\rfloor}
\newcommand*\diff{\mathop{}\!\mathrm{d}}
\newcommand*\Diff[1]{\mathop{}\!\mathrm{d^#1}}
\renewcommand{\mod}{\ \Mod}
\renewcommand*{\glstextformat}[1]{\textcolor{RoyalBlue}{#1}}
\renewcommand{\glsnamefont}[1]{\textbf{#1}}
\renewcommand\labelitemii{$\circ$}
\renewcommand\thesubfigure{\arabic{chapter}.\arabic{figure}}
\renewcommand\thesubfigure{%
    \arabic{chapter}.\arabic{figure}.\arabic{subfigure}}
\addto\captionsenglish{\renewcommand{\figurename}{Fig.}}
%------------------------Book Command---------------------------%
\makeatletter
\renewcommand\@pnumwidth{1cm}
\newcounter{book}
\renewcommand\thebook{\@Roman\c@book}
\newcommand\book{%
    \if@openright
        \cleardoublepage
    \else
        \clearpage
    \fi
    \thispagestyle{plain}%
    \if@twocolumn
        \onecolumn
        \@tempswatrue
    \else
        \@tempswafalse
    \fi
    \null\vfil
    \secdef\@book\@sbook
}
\def\@book[#1]#2{%
    \ifnum \c@secnumdepth >-3\relax
        \refstepcounter{book}%
        \addcontentsline{toc}{book}{
            \bookname\ \thebook:\hspace{1em}#1
        }
    \else
        \addcontentsline{toc}{book}{#1}%
    \fi
    \markboth{}{}%
    {\centering
     \interlinepenalty \@M
     \normalfont
     \ifnum \c@secnumdepth >-2\relax
       \huge\bfseries \bookname\nobreakspace\thebook
       \par
       \vskip 20\p@
     \fi
     \Huge \bfseries #2\par}%
    \@endbook}
\def\@sbook#1{%
    {\centering
     \interlinepenalty \@M
     \normalfont
     \Huge \bfseries #1\par}%
    \@endbook}
\def\@endbook{
    \vfil\newpage
        \if@twoside
            \if@openright
                \null
                \thispagestyle{empty}%
                \newpage
            \fi
        \fi
        \if@tempswa
            \twocolumn
        \fi
}
\newcommand*\l@book[2]{%
    \ifnum \c@tocdepth >-2\relax
        \addpenalty{-\@highpenalty}%
        \addvspace{2.25em \@plus\p@}%
        \setlength\@tempdima{3em}%
        \begingroup
            \parindent \z@ \rightskip \@pnumwidth
            \parfillskip -\@pnumwidth
            {
                \leavevmode
                \Large \bfseries #1\hfil \hb@xt@\@pnumwidth{
                    \hss #2
                }
            }
            \par
            \nobreak
            \global\@nobreaktrue
            \everypar{\global\@nobreakfalse\everypar{}}%
        \endgroup
    \fi}
\newcommand\bookname{Book}
\renewcommand{\thebook}{\texorpdfstring{\Numberstring{book}}{book}}
\providecommand*{\toclevel@book}{-2}
\makeatother
\titlecontents{chapter}[0pt]
    {\bfseries}
    {\chaptername\ \thecontentslabel:\quad}
    {}
    {\hfill\contentspage}
\titleformat{\part}[display]
    {\Large\bfseries}
    {\partname\nobreakspace\thepart}
    {0mm}
    {\Huge\bfseries}
    \titlecontents{part}[0pt]
    {\large\bfseries}
    {\partname\ \thecontentslabel: \quad}
    {}
    {\hfill\contentspage}
\newcommand{\MarkRightAngle}[4][.3cm]
    {\coordinate (tempa) at ($(#3)!#1!(#2)$);
     \coordinate (tempb) at ($(#3)!#1!(#4)$);
     \coordinate (tempc) at ($(tempa)!0.5!(tempb)$);%midpoint
     \draw (tempa) -- ($(#3)!2!(tempc)$) -- (tempb);}
%--------------------------LENGTHS------------------------------%
% Spacings for the Table of Contents.
\addtolength{\cftsecnumwidth}{1ex}
\addtolength{\cftsubsecindent}{1ex}
\addtolength{\cftsubsecnumwidth}{1ex}
\addtolength{\cftfignumwidth}{1ex}
\addtolength{\cfttabnumwidth}{1ex}

% Spacing for multi-column and enumerate environments.
\setlength{\multicolsep}{6pt}
\setlist[enumerate]{itemsep=0pt,topsep=3pt}

% Indent and paragraph spacing.
\setlength{\parindent}{0em}
\setlength{\parskip}{0em}
%----------------------------GLOSSARY-------------------------------%
\makeglossaries
\loadglsentries{../../glossary}
\loadglsentries{../../acronym}
%--------------------------Main Document----------------------------%
\begin{document}
    \ifx\ifmain\undefined
        \pagenumbering{roman}
        \title{Model Theory}
        \author{Ryan Maguire}
        \date{\vspace{-5ex}}
        \maketitle
        \tableofcontents
        \clearpage
        \chapter*{Model Theory}
        \addcontentsline{toc}{chapter}{Model Theory}
        \markboth{}{MODEL THEORY}
        \vspace{10ex}
        \setcounter{chapter}{1}
        \pagenumbering{arabic}
    \else
        \chapter{Model Theory}
    \fi
\end{document}
    % \addtocontents{toc}{\protect\newpage}
    % \clearpage

    % \setcounter{endpage}{\thepage}
    % \pagenumbering{gobble}
    % \book{Algebra}
    %     \renewcommand{\PATH}{\TOPPATH/Algebra}
    %     \pagenumbering{arabic}
    %     \setcounter{page}{\value{endpage}}
    %     \part{Group Theory}
    %         \chapter{Group Theory}
    \section{Relations}
        \begin{ldefinition}{Relation on a Set}{Relation}
            A relation on a set $A$ is a subset $R$ of $A\times{A}$.
            That is, $R\subseteq{A}\times{A}$. For elements
            $(a,b)\in{R}$ we write $aRb$.
        \end{ldefinition}
        For a relation $R$ it is not necessary true that $aRb$
        implies $bRa$, nor is it necessarily true that $aRa$. These
        are called symmetric and reflexive relations, respectively.
        There are many basic properties that relations have, and we
        prove them now.
        \begin{theorem}
            \label{thm:Cartesian_Product_Is_Relation}%
            If $A$ is a set, then $A\times{A}$ is a relation on $A$.
        \end{theorem}
        \begin{proof}
            For if $A$ is a set, then
            $A\times{A}\subseteq{A}\times{A}$. Therefore, etc.
        \end{proof}
        \begin{theorem}
            \label{thm:Empty_Set_Is_Relation}%
            If $A$ is a set, and then $\emptyset$ is a relation
            on $A$.
        \end{theorem}
        \begin{proof}
            For if $A$ is a set, then
            $\emptyset\subseteq{A}\times{A}$. Therefore, etc.
        \end{proof}
        These provide the two most basic examples of relations on a
        set. The empty set is the relation that says no two elements
        are related. Indeed, even single elements are unrelated to
        themselves. The second, the entire Cartesian product
        $A\times{A}$, says that everything is related. These are the
        two extreme cases, but provide useful examples and
        counterexamples in various contexts. More useful is that the
        union and intersection of relations is also a relation. We
        prove this now.
        \begin{theorem}
            \label{thm:Intersection_of_Relations_Is_Relation}%
            If $A$ is a set and if $R_{1}$ and $R_{2}$ are relations
            on $A$, then $R_{1}\cap{R}_{2}$ is a relation on $A$.
        \end{theorem}
        \begin{proof}
            For let $R=R_{1}\cap{R}_{2}$ and suppose $R$ is not a
            relation on $A$. Then there is an $x\in{R}$ such that
            $x\notin{A}\times{A}$. But if $x\in{R}$ then
            $x\in{R}_{1}$ and $x\in{R}_{2}$. But for all
            $x\in{R}_{1}$, $x\in{A}\times{A}$, since $R_{1}$ is a
            relation on $A$, a contradiction as
            $x\notin{A}\times{A}$. Therefore, $R$ is a relation on
            $A$.
        \end{proof}
        \begin{theorem}
            \label{thm:Set_Theory_Union_of_Relations_Is_Relation}
            If $A$ is a set and if $R_{1}$ and $R_{2}$ are relations
            on $A$, then $R_{1}\cup{R}_{2}$ is a relation on $A$.
        \end{theorem}
        \begin{proof}
            For let $R=R_{1}\cup{R}_{2}$ and suppose $R$ is not a
            relation on $A$. Then there is an $x\in{R}$ such that
            $x\notin{A}\times{A}$. But if $x\in{R}$ then
            $x\in{R}_{1}$ or $x\in{R}_{2}$. But for all $x\in{R}_{1}$
            and for all $x\in{R}_{2}$,
            $x\in{A}\times{A}$, since $R_{1}$ and $R_{2}$ are
            relations on $A$, a contradiction. Therefore, etc.
        \end{proof}
        \begin{theorem}
            If $A$ is a set and $R$ is a relation on $A$, then there
            is a relation $\mathcal{U}$ on $A$ such that
            $R\cap\mathcal{U}=R$.
        \end{theorem}
        \begin{proof}
            For let $\mathcal{U}={A}\times{A}$. Then by
            Thm.~\ref{thm:Set_Theory_Entire_%
                      Cartesian_Product_Is_Relation}, $A\times{A}$ is
            a relation on $A$. But since $R$ is a relation,
            $R\subseteq{A}\times{A}$. But then
            $R\cap\mathcal{U}=R$. Therefore, etc.
        \end{proof}
        \begin{theorem}
            If $A$ is a set and $R$ is a relation on $A$, then there
            is a relation $\mathcal{U}$ on $A$ such that
            $R\cup\mathcal{U}=R$
        \end{theorem}
        \begin{proof}
            For let $\mathcal{U}=\emptyset$. Then by
            Thm.~\ref{thm:Set_Theory_Empty_Set_Is_Relation},
            $\mathcal{U}$ is a relation. But if $R$ is a set, then
            $R\cup\emptyset=R$. Thus, $R\cup\mathcal{U}=R$.
            Therefore, etc.
        \end{proof}
        Since a general relation is simply a subset of $A\times{A}$,
        there's not much structure on them, and thus there's not a lot
        that can be said about them. We can add more constraints to
        certain relations to get the more familiar properties
        we're used to.
        \begin{ldefinition}{Reflexive Relations}
            A reflexive relation on a set $A$ is a
            relation $R$ on $A$ such that for all $a\in{A}$
            it is true that $aRa$.
        \end{ldefinition}
        A reflexive relation on $A$ is simply any subset of
        $A\times{A}$ that contains the entire \textit{diagonal}. That,
        all of the pairs $(a,a)$. A reflexive relation can contain more
        than this, however. The only strict requirement is that
        $aRa$ for all $a\in{A}$.
        \begin{theorem}
            If $A$ is a set, and if $R_{1}$ and $R_{2}$ are reflexive
            relations on $A$, then $R_{1}\cap{R}_{2}$ is a reflexive
            relation on $A$.
        \end{theorem}
        \begin{proof}
            For let $R=R_{1}\cap{R}_{2}$. Then by
            Thm.~\ref{thm:Set_Theory_Intersection_of_%
                      Relations_Is_Relation}, $R$ is a relation.
            Suppose $R$ is not reflexive.
            Then there is an $a\in{A}$ such that $(a,a)\notin{R}$. But
            if $a\in{A}$, then $(a,a)\in{R}_{1}$, since $R_{1}$ is
            reflexive. Similarly, $(a,a)\in{R}_{2}$ since $R_{2}$ is
            reflexive. But if $(a,a)\in{R}_{1}$ and $(a,a)\in{R}_{2}$,
            then $(a,a)\in{R}$ since $R=R_{1}\cap{R}_{2}$, a
            contradiction. Therefore, $R$ is reflexive.
        \end{proof}
        \begin{theorem}
            If $A$ is a set, if $R_{1}$ is a reflexive relation on
            $A$, and if $R_{2}$ is a relation on $A$, then
            $R_{1}\cup{R}_{2}$ is a reflexive relation on $A$.
        \end{theorem}
        \begin{proof}
            For let $R=R_{1}\cup{R}_{2}$. Since $R_{1}$ and $R_{2}$ are
            relations, by
            Thm.~\ref{thm:Set_Theory_Union_of_Relations_Is_Relation},
            $R$ is a relation. Suppose it is not reflexive.
            Then there is an $a\in{A}$ such that
            $(a,a)\notin{R}$. But if $a\in{A}$ then $(a,a)\in{R}_{1}$
            since $R_{1}$ is reflexive. But if $(a,a)\in{R}_{1}$ then
            $(a,a)\in{R}_{1}\cup{R}_{2}$, a contradiction.
            Therefore, etc.
        \end{proof}
        Given an arbitrary relation $R$ on a set $A$, it may not be
        true that $R$ is reflexive. It may often be useful to add in
        only the necessary points of $A$ that will make $R$
        reflexive. This is called the reflexive closure of $R$.
        \begin{ldefinition}{Reflexive Closure of a Relation}
              {Reflexive_Closure_of_Relation}
            The reflexive closure of a relation $R$ on a set $A$
            is the set:
            \begin{equation}
                S=R\cup\{(a,a):a\in{A}\}
            \end{equation}
        \end{ldefinition}
        \begin{theorem}
            If $A$ is a set, $R$ is a relation on $A$, and if $S$ is the
            reflexive closure of $R$, then $S$ is a reflexive relation on $A$.
        \end{theorem}
        \begin{theorem}
            \label{thm:Set_Theory_Refl_Clos_Is_Smallest_Refl_With_R}
            If $A$ is a set, if $R$ is a relation on $A$, if
            $S$ is the reflexive closure of $R$, and if $T$ is a
            reflexive relation on $A$ such that $R\subseteq{T}$, then
            $S\subseteq{T}$.
        \end{theorem}
        \begin{proof}
            For if $x\in{S}$, then either $x\in{R}$ or there is an
            $a\in{A}$ such that $x=(a,a)$. But if $x\in{R}$, then
            $x\in{T}$ since $R\subseteq{T}$. If $x\notin{R}$ then
            there is an $a\in{A}$ such that $x=(a,a)$. But $T$ is
            reflexive, and therefore $(a,a)\in{T}$. But then
            $x\in{T}$. Therefore, $S\subseteq{T}$.
        \end{proof}
        Thm.~\ref{thm:Set_Theory_Refl_Clos_Is_Smallest_Refl_With_R}
        says that the reflexive closure of a relation $R$ is, in a sense,
        the \textit{smallest} relation that is reflexive and contains
        $R$ as a subset.
        \begin{theorem}
            If $A$ is a set, $R_{1}$ and $R_{2}$ are relations on $A$,
            and if $S_{1}$ and $S_{2}$ are the reflexive closures of
            $R_{1}$ and $R_{2}$, respectively, then the reflexive closure
            of $R_{1}\cap{R}_{2}$ is:
            \begin{equation}
                S=S_{1}\cap{S}_{2}
            \end{equation}
        \end{theorem}
        \begin{proof}
            By the definition of reflexive closure, we have:
            \begin{align}
                S_{1}&=R_{1}\cup\{(a,a):a\in{A}\}
                \tag{Def.~\ref{def:Reflexive_Closure_of_Relation}}\\
                S_{1}&=R_{2}\cup\{(a,a):a\in{A}\}
                \tag{Def.~\ref{def:Reflexive_Closure_of_Relation}}\\
                \nonumber
                S_{1}\cap{S}_{2}&=
                (R_{1}\cup\{(a,a):a\in{A}\})
                \cap(R_{2}\cup\{(a,a):a\in{A}\})\\
                &=(R_{1}\cap{R}_{2})
                \cup\{(a,a):a\in{A}\}
                \tag{Distributive Law}
            \end{align}
            But by the definition of the transitive closure of
            $R_{1}\cap{R}_{2}$:
            \begin{equation}
                S=(R_{1}\cap{R}_{2})\cup\{(a,a):a\in{A}\}
                \tag{Def.~\ref{def:Set_Theory_Reflexive_%
                               Closure_of_Relation}}
            \end{equation}
            Therefore, etc.
        \end{proof}
        \begin{ldefinition}{Symmetric Relations}
            A symmetric relation on a set $A$ is a
            relation $R$ on $A$ such that for all $a,b\in{A}$
            such that $aRb$, it is true that $bRa$.
        \end{ldefinition}
        \begin{ldefinition}{Transitive Relation}
            A transitive relation on a set $A$ is a relation $R$ on $A$
            such that for all $a,b,c\in{A}$ such that $aRb$ and $bRc$,
            is it true that $aRc$.
        \end{ldefinition}
        \begin{ldefinition}{Transitive Closure}
            The transitive closure of a relation $R$ on a set
            $A$ is the the set $R^{t}\subseteq{A}\times{A}$ defined by:
            \begin{equation}
                R^{t}
            \end{equation}
        \end{ldefinition}
        \begin{ldefinition}{Asymmetric Relation}
            An asymmetric relation on a set $A$ is a relation $R$
            on $A$ such that for all $a,b\in{A}$ such that $aRb$
            it is true that $(b,a)\notin{R}$.
        \end{ldefinition}
        \begin{ldefinition}{Total Relation}
            A total relation on a set $A$ is a relation $R$ on $A$ such
            that for all $a,b\in{A}$ it is true that either
            $aRb$ or $bRa$, or both.
        \end{ldefinition}
        The notion of equality can be defined as a relation
        with the following properties:
        \begin{enumerate}
            \item Equality is Reflexive: $a=a$ for all $a\in{A}$.
            \item Equality is Symmetric: $a=b$ if and only if $b=a$.
            \item Equality is Transitive: If $a=b$ and $b=c$, then $a=c$.
            \item The relation is uniquely defined by the set
                  $\{(a,a)\in A\times A:a\in A\}$.
        \end{enumerate}
        That is, equality can be seen as the \textit{diagonal} in the
        Cartesian product $A\times{A}$.
        \begin{ldefinition}{Antisymmetric Relation}
            An antisymmetric relation on a set $A$ is a relation $R$ on $A$
            such that for all $a,b\in{A}$ such that $aRb$ and $bRa$, it
            is true that $a=b$.
        \end{ldefinition}
    \section{Functions}
        \begin{ldefinition}{Functions}
            A function from a set $A$ to a set $B$, denoted
            $f:A\rightarrow B$, is a subset of $A\times{B}$
            such that for all $a\in{A}$ there is a unique
            $b\in{B}$, denoted $f(a)$, such that $(a,b)\in{f}$.
        \end{ldefinition}
        Functions can also be called maps or mappings. The unique point
        $b\in{B}$ such that $(a,b)\in{f}$ is often called the image of
        $a$ under $f$. We sometimes write $a\mapsto{b}$, but most often
        will write $f(a)=b$.
        \begin{ldefinition}{Image of a Set}
            The image of a subset $A$ of a set $X$ under a function
            $f:X\rightarrow{Y}$ is the set:
            \begin{equation}
                f(A)=\{f(x)\in{Y}:x\in{A}\}
            \end{equation}
        \end{ldefinition}
        The image of a subset $A\subseteq{X}$ is the set of all points
        that get mapped onto by the function $f$ by the elements in $A$.
        In a similar manner we can define the opposite of this notion,
        called the pre-image.
        \begin{ldefinition}{Pre-Image}
            The pre-image of a subset $B$ of a set $Y$ under a function
            $f:X\rightarrow{Y}$ is the set:
            \begin{equation}
                f^{-1}(B)=\{x\in{X}:f(x)\in{B}\}
            \end{equation}
        \end{ldefinition}
        \begin{axiom}
            If $X$ is a non-empty set such that for all $x\in{X}$,
            $x\ne\emptyset$, then there is a function
            $f:X\rightarrow\bigcup_{x\in{X}}x$ such that,
            for all $x\in{X}$, $f(x)\in{x}$.
        \end{axiom}
        This is called the axiom of choice. It can be made into
        a blatantly obvious statement, by choosing a more careful
        wording, however many of the results it gives are far
        from intuitive. This says that, given a collection of sets,
        each of which is non-empty, one may choose a single element from
        each set. This choosing is the function $f$, and is often called
        a \textit{choice function}. For those interested, the axiom
        of choice is consistent with modern set theory (Called
        Zermelo-Fraenkel set theory, or ZF). It may thus be rejected
        or accepted without logical contradiction.
        \begin{theorem}
            \label{theorem:Set_Theory_Image_of_Empty_Set_Is_Empty}
            If $A$ and $B$ are sets, and if $f:A\rightarrow{B}$
            is a function, then:
            \begin{equation}
                f(\emptyset)=\emptyset
            \end{equation}
        \end{theorem}
        \begin{proof}
            For suppose not. Let $y\in{f}(\emptyset)$.
            But then there is an
            $x\in\emptyset$ such that $f(x)=y$, a contradiction 
            ince for all $x$, $x\notin\emptyset$.
            Thus, $f(\emptyset)=\emptyset$.
        \end{proof}
        \begin{theorem}
            If $A$ and $B$ are sets, and if $f:A\rightarrow{B}$ is a function, then:
            \begin{equation}
                f^{-1}(\emptyset)=\emptyset
            \end{equation}
        \end{theorem}
        \begin{proof}
            For suppose not. Then there is an $x\in{X}$ such that
            $f(x)\in\emptyset$, a contradiction since for all $x$,
            $f(x)\notin\emptyset$. Therefore, etc.
        \end{proof}
        \begin{theorem}
            If $X$ and $Y$ are sets, if $A\subseteq{X}$, and if
            $f:X\rightarrow{Y}$ is a function such that
            $f(A)=\emptyset$, then $A=\emptyset$.
        \end{theorem}
        \begin{proof}
            For suppose not. If $A\ne\emptyset$, then there is an
            $x\in{A}$. But then $f(x)\in{f}(A)$, a contradiction as
            $f(A)=\emptyset$. Therefore, etc.
        \end{proof}
        \begin{theorem}
            If $X$ and $Y$ are sets, if $B$ is a subset of $Y$,
            and if $f:X\rightarrow{Y}$ is a function, then:
            \begin{equation}
                f\big(f^{-1}(B)\big)\subseteq{B}
            \end{equation}
        \end{theorem}
        \begin{proof}
            For if $y\in{f(f^{-1}(B))}$, then there is an
            $x\in{f^{-1}(B)}$ such that $y=f(x)$. But if
            $x\in{f^{-1}(B)}$, then $f(x)\in{B}$. Thus,
            $y\in{B}$. Therefore, etc.
        \end{proof}
        \begin{theorem}
            If $X$ and $Y$ are non-empty sets and if there exists
            $y_{1},y_{2}\in{Y}$ such that $y_{1}\ne{y}_{2}$, then
            there is a function $f:X\rightarrow{Y}$ and a
            $B\subseteq{Y}$ such that:
            \begin{equation}
                f\big(f^{-1}(B)\big)\ne{B}
            \end{equation}
        \end{theorem}
        \begin{proof}
            \begin{subequations}
                For if $X$ and $Y$ are non-empty, let $f:X\rightarrow{Y}$
                be defined by:
                \begin{equation}
                    f=\{(x,y_{1}):x\in{X}\}
                \end{equation}
                Then $f$ is a function, since $f\subseteq{X}\times{Y}$
                as $y_{1}\in{Y}$. Moreover, for all $x\in{X}$ there is a
                unique $y\in{Y}$ such that $(x,y)\in{f}$. Thus, $f$ is a
                function from $X$ to $Y$. However since for all
                $x\in{X}$, $f(x)=y_{1}$, we have that:
                \begin{equation}
                    f^{-1}(\{y_{2}\})=\emptyset
                \end{equation}
                For suppose $x\in{f}^{-1}(\{y_{2}\})$.
                Then $f(x)=y_{2}$, but for all $x\in{X}$, $f(x)=y_{1}$,
                and $y_{1}\ne{y}_{2}$. Thus
                $f^{-1}(\{y_{2}\})=\emptyset$. But by
                Thm.~\ref{theorem:Set_Theory_Image_%
                          of_Empty_Set_Is_Empty},
                $f(\emptyset)=\emptyset$. Therefore:
                \begin{equation}
                    f\big(f^{-1}(\{y_{2}\})\big)=\emptyset
                \end{equation}
                But $\{y_{2}\}\ne\emptyset$ and
                $\{y_{2}\}\subseteq{Y}$. Therefore, etc.
            \end{subequations}
        \end{proof}
        \begin{theorem}
            If $X$ and $Y$ are sets, if $A$ is a subset of $X$,
            and if $f:X\rightarrow{Y}$ is a function, then:
            \begin{equation}
                A\subseteq{f^{-1}}\big(f(A)\big)
            \end{equation}
        \end{theorem}
        \begin{proof}
            For if $x\in{A}$, then there is a
            $y\in{f}(A)$ such that $f(x)=y$. But then
            $x\in{f^{-1}(f(A))}$. Therefore, etc.
        \end{proof}
        \begin{theorem}
            If $X$ and $Y$ are sets, if $A_{1}$ and $A_{2}$ are
            subsets of $X$ such that $A_{1}\subseteq{A}_{2}$,
            and if $f:X\rightarrow{Y}$ is a function, then:
            \begin{equation}
                f(A_{1})\subseteq{f}(A_{2})
            \end{equation}
        \end{theorem}
        \begin{proof}
            For if $y\in{f}(A_{1})$, then there is an $x\in{A}_{1}$
            such that $f(x)=y$. But $A_{1}\subseteq{A}_{2}$, and
            therefore $x\in{A}_{2}$. But if $x\in{A}_{2}$, then
            $f(x)\in{f}(A_{2})$. Thus, $y\in{f}(A_{2})$. Therefore, etc.
        \end{proof}
        \begin{theorem}
            If $X$ and $Y$ are sets, if $B_{1}$ and $B_{2}$ are subsets of
            $Y$ such that $B_{1}\subseteq{B}_{2}$, and if $f:X\rightarrow{Y}$
            is a function, then:
            \begin{equation}
                f^{-1}(B_{1})\subseteq{f^{-1}}(B_{2})
            \end{equation}
        \end{theorem}
        \begin{proof}
            For if $x\in{f}^{-1}(B_{1})$, then there is a
            $y\in{B}_{1}$ such that $f(x)=y$. But
            $B_{1}\subseteq{B}_{2}$, and therefore $y\in{B}_{2}$.
            Thus, $x\in{f}^{-1}(B_{2})$. Therefore, etc.
        \end{proof}
        \begin{theorem}
        If $f:A\rightarrow B$, $A_1,A_2\subset A$, then $f(A_1 \cup A_2) = f(A_1)\cup f(A_2)$.
        \end{theorem}
        \begin{proof}
        $[y\in f(A_1\cup A_2)]\Rightarrow [\exists x\in A_1 \cup A_2:y=f(x)]\Rightarrow [y \in f(A_1)\cup f(A_2)]$. $[y\in f(A_1)\cup f(A_2)]\Rightarrow \big[[\exists x\in A_1] \lor[\exists x\in A_2]: y=f(x)\big]\Rightarrow [x\in A_1\cup A_2]\Rightarrow [f(x)\in f(A_1\cup A_2)]$
        \end{proof}
        \begin{theorem}
        If $f:A\rightarrow B$, $A_1,A_2\subset A$, then $f(A_1\cap A_2)\subset f(A_1)\cap f(A_2)$.
        \end{theorem}
        \begin{proof}
        $[y\in f(A_1 \cap A_2)]\Rightarrow [\exists x\in A_1 \cap A_2:y=f(x)]\Rightarrow [x\in A_1 \land x \in A_2] \Rightarrow[y \in f(A_1)\cap f(A_2)]$.
        \end{proof}
        \begin{theorem}
        If $f:A\rightarrow B$, $B_1,B_2\subset B$, then $f^{-1}(B_1\cup B_2) = f^{-1}(B_1)\cup f^{-1}(B_2)$.
        \end{theorem}
        \begin{proof}
        $[x\in B_1\cup B_2]\Rightarrow [f(x)\in B_1\cup B_2]\Rightarrow [f(x)\in B_1\lor f(x)\in B_2]\Rightarrow [x\in f^{-1}(B_1)\cup f^{-1}(B_2)]$. $[x \in f^{-1}(B_1)\cupf^{-1}(B_2)]\Rightarrow [f(x)\in B_1\lor f(x) \in B_2]\Rightarrow [f(x) \in B_1\cup B_2]\Rightarrow [x\in f^{-1}(B_1\cup B_2)]$.
        \end{proof}
        \begin{theorem}
        If $f:A\rightarrow B$, $B_1,B_2\subset B$, then $f^{-1}(B_1\cap B_2) = f^{-1}(B_1)\cap f^{-1}(B_2)$.
        \end{theorem}
        \begin{proof}
        $[x\in f^{-1}(B_1\cap B_2)]\Rightarrow [f(x) \in B_1 \cap B_2]\Rightarrow [f(x)\in B_1\land f(x) \in B_2 ]\Rightarrow [x\in f^{-1}(B_1)\cap f^{-1}(B_2)]$. $[x\in f^{-1}(B_1)\capf^{-1}(B_2)]\Rightarrow [x\in f^{-1}(B_1)\land x\in f^{-1}(B_2)]\Rightarrow [f(x) \in B_1\land f(x) \in B_2]\Rightarrow [f(x)\in B_1\cap B_2]\Rightarrow [x\in f^{-1}(B_1\cap B_2)]$.
        \end{proof}
        \begin{theorem}
        If $f:A\rightarrow B$, $B_1 \subset B$, then $f^{-1}(B\setminus B_1) = f^{-1}(B)\setminus f^{-1}(B_1)$.
        \end{theorem}
        \begin{proof}
        $[x\in f^{-1}(B\setminus B_1)]\Leftrightarrow [f(x)\notin B_1]\Leftrightarrow [x\in f^{-1}(B)\setminus f^{-1}(B_1)]$
        \end{proof}
        If $f:A\rightarrow B$, the image of $A$ under $f$
        is often called the range (A is often called the domain).
        \begin{ldefinition}{Injective Functions}
            An injective function from a set $A$ to a set $B$ is a function
            $f:A\rightarrow{B}$ such that for all $x,y\in{A}$ such that
            $x\ne{y}$, it is true that $f(x)\ne{f}(y)$.
        \end{ldefinition}
        \begin{ldefinition}{Inverse of Injective Functions}
            The inverse of an injective function $f:A\rightarrow{B}$
            is the function $f^{-1}:f(A)\rightarrow{A}$ defined by
            $f^{-1}(y)=x$, where $x\in{A}$ is the unique element such
            that $f(x)=y$.
        \end{ldefinition}
        \begin{ldefinition}{Surjective Functions}
            A surjective function from a set $A$ to a set $B$ is
            a function $f:A\rightarrow{B}$ such that for all
            $y\in{B}$ there is an $x\in{A}$ such that $y=f(x)$.
        \end{ldefinition}
        \begin{ldefinition}{Bijective Functions}
            A bijective function from a set $A$ to a set $B$
            is a function $f:A\rightarrow{B}$ such that $f$ is
            injective and surjective.
        \end{ldefinition}
        \begin{ldefinition}{Permutations}
            A permutation on a set $A$ is a bijective function
            $f:A\rightarrow{A}$.
        \end{ldefinition}
        \begin{theorem}
        If $f:A\rightarrow B$ is bijective, then $f^{-1}$ is bijective.
        \end{theorem}
        \begin{proof}
        $[f^{-1}(y_1) = f^{-1}(y_2)]\Rightarrow [\exists x\in A:[f(x) = y_1]\land [f(x)=y_2]]\Rightarrow [y_1=y_2]$. By definition, $f^{-1}$ is surjective.
        \end{proof}
        \begin{definition}
        If $f:A\rightarrow B$ and $g:B\rightarrow C$, then $g\circ f:A\rightarrow C$ is defined by the image $g(f(x)), x\in A$. 
        \end{definition}
        \begin{theorem}
        If $f:A\rightarrow B$, $g:B\rightarrow C$, and $\mathcal{V}\subset C$, then $(g\circ g)^{-1}(\mathcal{V}) = f^{-1}(g^{-1}(\mathcal{V}))$.
        \end{theorem}
        \begin{proof}
        $[x\in (g\circ f)^{-1}(\mathcal{V})]\Leftrightarrow [g(f(x))\in \mathcal{V}] \Leftrightarrow [f(x)\in g^{-1}(\mathcal{V})]\Leftrightarrow [x\in f^{-1}(g^{-1}(\mathcal{V}))]$.
        \end{proof}
        \begin{theorem}
        If $f:A\rightarrow B$ is bijective, $g:B\rightarrow C$ is bijective, then $g\circ f$ is bijective.
        \end{theorem}
        \begin{proof}
        $\big[[f(A) = B]\land [g(B) = C]\big]\Rightarrow [g(f(A)) = g(B) = C]$. $[g(f(x_1))=g(f(x_2))]\Leftrightarrow [f(x_1)=f(x_2)]\Leftrightarrow [x_1=x_2]$.
        \end{proof}
        \begin{theorem}
        If $f:A\rightarrow B$ is bijective, $A_1\subset A$, and $f(A_1) = B$, then $A_1=A$.
        \end{theorem}
        \begin{proof}
        $\Big[\big[[A_1^c \ne \emptyset]\Rightarrow [f(A_1^c) \ne \emptyset]\big]\land[f(A_1)\cap f(A_1^c) = \emptyset]\Big]\Rightarrow [\exists y\in B:y\notin f(A_1)]$, a contradiction.
        \end{proof}
        \begin{definition}
        If $A$ is a set, then a binary operation $*$ on the set $A$ is a function from $A\times A$ to $A$.
        \end{definition}
        \begin{definition}
        A binary operation $*$ is said to be associative if and only if $a*(b*c) = (a*b)*c$.
        \end{definition}
        \begin{definition}
        An element $e\in A$ is said to be an identity element if and only if for all $a\in A$, $e*a = a*e = a$.
        \end{definition}
        \begin{definition}
        An element $b\in A$ is said to be an inverse of $a$ if and only if $a*b=b*a = e$. We write $b=a^{-1}$
        \end{definition}
        \begin{ldefinition}{Groups}
            A group is a set $G$ with a binary operation $*$,
            denoted $(G,*)$, such that: 
            \begin{enumerate}
                \item There exists an identity element $e$.
                \item For every element $a\in{A}$, there is an inverse element.
                \item The binary operation $*$ is associative.
            \end{enumerate}
        \end{ldefinition}
        Note that it is not necessarily true that $a*b = b*a$.
        These are special groups that are called Abelian.
        \begin{theorem}
            If $(G,*)$ is a group and $e$ is the identity, then it is unique.
        \end{theorem}
        \begin{proof}
            For if $e$ and $e'$ are identities, then:
            \begin{equation}
                e'=e'*e=e
            \end{equation}
            Therefore, etc.
        \end{proof}
        \begin{theorem}
            \label{thm:Group_Theory_Inverses_Are_Unique}
            If $(G,*)$ is a group, $a\in{G}$, and if $a^{-1}$ and
            $a'^{-1}$ are inverses of $a$, then $a^{-1}=a'^{-1}$.
        \end{theorem}
        \begin{proof}
            For if $e$ is the identity, and
            $a^{-1}$ and $a'^{-1}$ are inverses of $a$, then:
            \begin{align}
                a^{-1}&=a^{-1}*e
                \tag{Identitive Property}\\
                &=a^{-1}*(a*a'^{-1})
                \tag{Inverse Property}\\
                &=(a^{-1}*a)*a'^{-1}
                \tag{Associative Property}\\
                &=e*a'^{-1}
                \tag{Inverse Property}\\
                &=a'^{-1}
                \tag{Identitive Property}
            \end{align}
            Therefore, etc.
        \end{proof}
        \begin{theorem}
            If $(G,*)$ is a group, and if $e\in{G}$ is the identity,
            then $e^{-1}=e$.
        \end{theorem}
        \begin{proof}
            For $e=e*e$, and by Thm.~\ref{thm:Group_Theory_Inverses_Are_Unique}
            inverses are unique. Therefore, etc.
        \end{proof}
        \begin{theorem}
            \label{thm:Group_Theory_Inverse_of_Product}
            If $(G,*)$ is a group and $a,b\in G$, then:
            \begin{equation}
                (a*b)^{-1} = b^{-1}*a^{-1}
            \end{equation}
        \end{theorem}
        \begin{proof}
            For:
            \begin{align}
                (a*b)*(b^{-1}*a^{-1})&=
                a*(b*b^{-1})*a^{-1}
                \tag{Associative Property}\\
                &=a*(e)*a^{-1}
                \tag{Inverse Property}\\
                &=a*a^{-1}
                \tag{Identitive Property}\\
                &=e
                \tag{Inverse Property}
            \end{align}
            Thus $b^{-1}*a^{-1}$ is a right-inverse of $a*b$.
            But since $(G,*)$ is a group, right-inverses are
            left-inverses, and therefore $b^{-1}*a^{-1}$ is
            an inverse of $a*b$. But by
            Thm.~\ref{thm:Group_Theory_Inverses_Are_Unique},
            inverses are unique. Therefore, etc.
        \end{proof}
        \begin{theorem}
            If $(G,*)$ is a group and $a\in{G}$, then:
            \begin{equation}
                (a^{-1})^{-1}=a
            \end{equation}
        \end{theorem}
        \begin{proof}
            For:
            \begin{align}
                a^{-1}*(a^{-1})^{-1}
                &=(a^{-1}* a)^{-1}
                \tag{Thm.~\ref{thm:Group_Theory_Inverse_of_Product}}\\
                &=e
                \tag{Inverse Property}
            \end{align}
            From uniqueness, $(a^{-1})^{-1}=a$.
        \end{proof}
        \begin{definition}
        If $\langle G, * \rangle$ and $\langle G',\circ \rangle$ are groups and $f:G\rightarrow G'$ is a bijective function, then $f$ is said to be an isomorphism between $\langle G, *\rangle$ and $\langle G',\circ \rangle$ if and only if for all $a,b\in G$, $f(a*b) =f(a)\circ f(b)$.
        \end{definition}
        \begin{definition}
        $\langle G, *\rangle$ and $\langle G', \circ \rangle$ are said to be isomorphic if and only if there is an isomorphism between them.
        \end{definition}
        \begin{theorem}
        If $\langle G, * \rangle$ and $\langle G', \circ \rangle$ are isomorphic with identities $e_*$ and $e_{\circ}$ are the identities, then $f(e_*) = e_{\circ}$.
        \end{theorem}
        \begin{proof}
        $\forall a\in G,\ f(a)=f(a* e_*) = f(a)\circ f(e_*)$ as $f$ is an isomorphism. As identities are unique, $f(e_*) = e_{\circ}$.
        \end{proof}
        \begin{theorem}
        If $\langle G, * \rangle$ and $\langle G', \circ \rangle$ are isomorphic, with isomorphism $f$, and if $a\in G$, then $f(a^{-1}) = f(a)^{-1}$.
        \end{theorem}
        \begin{proof}
        For $e_{\circ}=f(e_*) = f(a*a^{-1}) = f(a^{-1}*a) = f(a)\circ f(a^{-1})=f(a^{-1})\circ f(a)$. As inverses are unique, $f(a^{-1})=f(a)^{-1}$.
        \end{proof}
        \begin{definition}
        A binary operation $*$ on a set $A$ is said to be commutative if and only for all $a,b\in A$, $a*b = b*a$.
        \end{definition}
        \begin{definition}
        A field is a set $F$ with two operations $+$ and $\cdot$, denoted $\langle F, +,\cdot \rangle$, with the following properties:
        \begin{enumerate}
        \item $a+b=b+a$ \hfill [Addition is Commutative]
        \item $a+(b+c)=(a+b)+c$ \hfill [Addition is Associative]
        \item $a\cdot b = b\cdot a$ \hfill [Multiplication is Commutative]
        \item $a\cdot (b\cdot c) = (a\cdot b)\cdot c$ \hfill [Multiplication is Associative]
        \item There is a $0\in F$ such that $0+a=a$ for all $a\in F$ \hfill [Existence of Additive Identity]
        \item There is a $1\in F$ such that $1\cdot a = a$ for all $a\in F$ \hfill [Existence of Multiplicative Identity]
        \item For each $a\in F$ there is a $b\in F$ such that $a+b = 0$. $b$ is denoted $-a$ \hfill [Existence of Additive Inverses]
        \item For each $a\in F$, $a\ne 0$ there is a $b\in F$ such that $a\cdot b = 1$. $b$ is denoted $a^{-1}$. \hfill [Existence of Multiplicative Inverses]
        \item $a\cdot(b+c) = a\cdot b + a\cdot c$ \hfill [Distributive Property]
        \end{enumerate}
        \end{definition}
        \begin{definition}
        A subfield of a field $\langle F,+,\cdot \rangle$ is a set $K\subset F$, such that $\langle K, +,\cdot \rangle$ is a field.
        \end{definition}
        \begin{theorem}
        In a field, $0$ and $1$ are unique.
        \end{theorem}
        \begin{proof}
        For suppose not, and let $0'$ and $1'$ be other identities. Then $1'=1'\cdot 1 = 1$ and $0'=0'+0=0$.
        \end{proof}
        \begin{theorem}
        For any field $\langle F,+,\cdot \rangle$, for any $a\in F$, $a\cdot 0 = 0$.
        \end{theorem}
        \begin{proof}
        For $0 = a\cdot 0 + (-a\cdot 0) = a\cdot(0+0) +(-a\cdot 0) = a\cdot 0 + a\cdot 0 + (-a\cdot 0) = a\cdot 0$. Thus, $a\cdot 0 = 0$.
        \end{proof}
        \begin{remark}
        If $1=0$, then $a=a\cdot 1 = a\cdot 0 = 0$, and thus every element is zero. A very boring field.
        \end{remark}
        \begin{corollary}
        In a field $\langle F, +,\cdot \rangle$, if $0\ne 1$, then $0$ has no inverse.
        \end{corollary}
        \begin{proof}
        For let $a$ be such an inverse. Then $a\cdot 0 = 1$. But for any element of $F$, $a \cdot 0 = 0$. But $0\ne 1$, a contradiction.
        \end{proof}
        \begin{theorem}
        If $a+b = 0$, then $b= (-1)\cdot a$ where $(-1)$ is the solution to $1+(-1)=0$.
        \end{theorem}
        \begin{proof}
        $a+(-1)a = a(1+(-1)) = a\cdot 0 = 0$. From uniqueness, $b=(-1)a$. We may thus write additive inverses as $-a$
        \end{proof}
        \begin{definition}
        Given two fields $\langle F,+,\cdot \rangle$ and $\langle F', +',\times \rangle$, a bijection function $f:F\rightarrow F'$ is said to be a field isomorphism if and only if for allelements $a,b\in F$, $f(a+b)=f(a)+'f(b)$, and $f(a\cdot b) = f(a)\times f(b)$
        \end{definition}
        \begin{definition}
        $\langle F,+,\cdot \rangle$ and $\langle F', +',\times \rangle$, are said to be isomorphic if and only if they have an isomorphism.
        \end{definition}
        \begin{theorem}
        Given an ismorphism between two fields $\langle F,+,\cdot \rangle$ and $\langle F', +',\times \rangle$, $f(1) = 1'$ and $f(0) = 0'$.
        \end{theorem}
        \begin{proof}
        For let $x\in F$. Then $f(x)=f(x\cdot 1) = f(x)\times f(1)$, and $f(x)=f(x+0) = f(x)+'f(0)$. Therefore, etc.
        \end{proof}
        \begin{theorem}
        In a field $\langle F,+,\cdot \rangle$, $(a+ b)^2 = a^2 + 2ab + b^2$ ($2$ being the solution to $1+1$).
        \end{theorem}
        \begin{proof}
        For $(a+b)^2 = (a+b)(a+b) = a(a+b)+b(a+b) = a^2 + ab + ba + b^2 = a^2 +ab(1+1)+b^2 = a^2 + 2ab + b^2$.
        \end{proof}
        \begin{definition}
            A group is a set $G$ and a binary relation $*$
            on $G$, denoted $(G,*)$, such that:
            \begin{enumerate}
                \item For all ${a,b,c}\in{G}$, $(a*b)*c=a*(b*c)$
                \item There is an ${e}\in{G}$ such that for all
                    ${a}\in{G}$, $a*e=e*a=a$.
                \item For all ${a}\in{G}$ there is a ${b}\in{G}$
                    such that $a*b=e$
            \end{enumerate}
        \end{definition}
        \begin{definition}
            An Abelian group is a group $(G,*)$ such that
            $*$ is commutative.
        \end{definition}
        \begin{example}
            $G=\{1\}$ is an Abelian group
            under multiplication.
            This is the trivial group.
        \end{example}
        \begin{theorem*}
            If $(G,*)$ is a group, then the following are true:
            \begin{enumerate}
                \begin{multicols}{2}
                    \item The identity ${e}\in{G}$ is unique.
                    \item If $a*b=a*c$, then $b=c$.
                    \item If $b*a=c*a$, then $b=c$.
                    \item Inverses $a^{-1}$ are unique.
                    \item $\forall_{{a,b}\in{G}}%
                        \exists_{{x}\in{G}}:a*x=b$
                    \item $(a*b)^{-1}=b^{-1}*a^{-1}$
                \end{multicols}
            \end{enumerate}
        \end{theorem*}
        \begin{definition}
            The order of a group is number of elements in the
            group.
        \end{definition}
        \begin{definition}
            A group of finite order, or a finite group,
            is a group with finitely many elements.
        \end{definition}
        \begin{definition}
            The direct product of two groups $(G,*)$ and
            $(H,\circ)$ is the group  $({G}\times{H},\star)$
            where $\star$ is the binary operation defined by
            $(g_{1},h_{1})\star(g_{2},h_{2})%
             =(g_{1}*g_{2},{h_{1}}\circ{h_{2}})$
        \end{definition}
        \begin{definition}
            A permutation group on $n$ elements is a
            group whose elements are permutations of
            $n$ elements.
        \end{definition}
        \begin{definition}
            The symmetric group on $n$ elements,
            denoted $S_{n}$, is the group formed by
            permuting $n$ elements.
        \end{definition}
        \begin{definition}
            A homomorphism from a group $(G,*)$ to
            a group $(H,\circ)$ is a function
            $h:{G}\rightarrow{H}$ such that for all
            ${a,b}\in{G}$, $h(a*b)={h(a)}\circ{h(b)}$
        \end{definition}
        \begin{definition}
            An epimorphism from a group $(G,*)$ to
            a group $(H,\circ)$ is a homomorphism
            $h:{G}\rightarrow{H}$ such that
            $h$ is surjective.
        \end{definition}
        \begin{definition}
            A monomorphism from a group $(G,*)$ to
            a group $(H,\circ)$ is a homomorphism
            $h:{G}\rightarrow{H}$ such that
            $h$ is injective.
        \end{definition}
        \begin{definition}
            An isomorphism from a group $(G,*)$ to
            a group $(H,\circ)$ is a homomorphism
            $h:{G}\rightarrow{H}$ such that
            $h$ is bijective.
        \end{definition}
        \begin{definition}
            A ring is a set $R$ and two binary operations
            on $R$, denoted $(R,\cdot,+)$, such that:
            \begin{enumerate}
                \begin{multicols}{3}
                \item $(R,+)$ is an Abelian group.
                \item $a\cdot({b}\cdot{c})%
                       =({a}\cdot{b})\cdot{c}$
                \item ${a}\cdot(b+c)%
                       ={a}\cdot{b}+{a}\cdot{c}$
                \end{multicols}
            \end{enumerate}
        \end{definition}
        \begin{definition}
            A ring with identity is a ring $(R,\cdot,+)$
            such that there is a ${1}\in{R}$ such that for
            all ${a}\in{R}$, ${a}\cdot{1}={1}\cdot{a}=a$.
        \end{definition}
        \begin{remark}
            Left and right identities are elements such
            that ${e_{L}}\cdot{a}=a$ and ${e_{R}}\cdot{a}=a$.
            If inverses $a_{L}^{-1}$ and $a_{R}^{-1}$ exist
            for $a$, then $a_{L}^{-1}=a_{R}^{-1}$. That is,
            the inverse is the same for both right and left
            identities.
        \end{remark}
        \begin{definition}
            A commutative ring is a ring $(R,\cdot,+)$ such that
            $\cdot$ is commutative.
        \end{definition}
        \begin{definition}
            A commutative ring with identity is a
            ring with identity such that $\cdot$
            is commutative.
        \end{definition}
        \begin{definition}
            A Field is a commutative ring with identity
            $(F,\cdot,+)$
            such that for all ${a}\in{F}$ such that
            $a$ is not an identity with respect to $+$,
            there is a $b\in{F}$ such that ${a}\cdot{b}=1$.
        \end{definition}
        \begin{definition}
            Equivalent sets are sets $A$ and $B$ such that
            there exists a bijective function
            $f:{A}\rightarrow{B}$
        \end{definition}
        \begin{definition}
            A finite set is a set $A$ such that there
            is an ${n}\in{\mathbb{N}}$ such that $A$
            is equivalent to $\mathbb{Z}_{n}$.
        \end{definition}
        \begin{definition}
            A countable set (Or a denumerable set) is a
            set $A$ that is equivalent to $\mathbb{N}$.
        \end{definition}
        \begin{definition}
            An uncountable set is a set that is neither finite
            nor countable.
        \end{definition}
        \begin{theorem*}
            Set Equivalence is an equivalence relation.
        \end{theorem*}
        This equivalence allows to classify all sets by the
        number of elements they contain or, more generally,
        by their cardinality. We say that two sets $A$ and
        $B$ have the same cardinality, denoted
        $\Card(A)$, if and only if $A$ and $B$ are equivalent.
        \begin{theorem*}
            The following are true:
            \begin{enumerate}
                \begin{multicols}{2}
                    \item $\Card(A)=0$ if and only if
                          $A=\emptyset$.
                    \item If ${A}\sim{\mathbb{Z}_{n}}$, then
                          $\Card(A)=n$.
                \end{multicols}
            \end{enumerate}
        \end{theorem*}
        \begin{definition}
            A finite cardinal number is a cardinal
            number of a finite set.
        \end{definition}
        \begin{definition}
            The standard ordering on the finite cardinal
            number is $0<1<\hdots<n<n+1<\hdots$
        \end{definition}
        Thus, if $A$ and $B$ are finite sets, then we write
        $\Card(A)<\Card(B)$ if $A$ is equivalent to a
        subset of $B$ but not equivalent to $B$.
        We take this notion and generalize to
        all sets. For $A$ and $B$, we write
        $\Card(A)<\Card(B)$ if $A$ is equivalent to a subset
        of $B$ but is not equivalent to $B$. This is the
        same as saying $A$ is equivalent to a subset of $B$,
        but $B$ is not equivalent to a subset of $A$.
        We write that
        $\Card(A)\leq\Card(B)$ is $A$ is equivalent to a
        subset of $B$.
        \begin{theorem*}[Schr\"{o}der-Bernstein Theorem]
            If $A$ and $B$ are sets such that
            $\Card(A)\leq\Card(B)$ and
            $\Card(B)\leq\Card(A)$, then
            $\Card(A)=\Card(B)$.
        \end{theorem*}
        \begin{theorem*}
            The following are true:
            \begin{enumerate}
                \item If $\Card(A)\leq\Card(B)$ and
                      $\Card(B)\leq\Card(A)$, then
                      $\Card(A)\leq\Card(C)$.
                \item If $\Card(A)\leq\Card(B)$, then
                      $\Card(A)+\Card(C)\leq\Card(B)+\Card(C)$
            \end{enumerate}
        \end{theorem*}
        \begin{theorem*}
            If ${A}\subset{B}\subset{C}$, and
            $\Card(A)=\Card(C)$, then $\Card(B)=\Card(C)$
        \end{theorem*}
        \begin{theorem*}
            If $f:{X}\rightarrow{Y}$ is a function,
            then $\Card(f(X))\leq\Card(X)$.
        \end{theorem*}
        \begin{proof}
            Note that $f^{-1}(\{y\})$ creates a set of
            mutually disjoint subsets of $X$. By the
            axiom of choice there is a function
            $F:{f(X)}\rightarrow{X}$
            such that for all ${y}\in{f(X)}$,
            ${F(y)}\in{f^{-1}(\{y\})}$. But since these
            sets are disjoint, $F$ is injective.
            Thus, $f(X)$ is equivalent to a subset of $X$.
            Therefore, $\Card(f(X))\leq\Card(X)$.
        \end{proof}
        The Schr\"{o}der-Bernstein theorem can be restated
        equivalently as ``If $A$ is equivalent to a subset
        of $B$ and $B$ is equivalent to a subset of $A$,
        then $A$ is equivalent to $B$.''
        Addition and multiplication of finite cardinals
        follows directly from the standard arithmetic
        for the natural numbers. For cardinals of infinite
        sets, the arithmetic becomes a little more complicated.
        \begin{definition}
            The sum of two cardinal numbers is the
            cardinality of the union of two disjoint sets $A$
            and $B$. That is, if ${A}\cap{B}=\emptyset$, then
            $\Card(A)+\Card(B)=\Card({A}\cup{B})$.
        \end{definition}
        \begin{theorem*}
            If $a$ and $b$ are distinct cardinal numbers,
            then there exists sets $A$ and $B$ such that
            ${A}\cap{B}=\emptyset$, $\Card(A)=a$, and
            $\Card(B)=b$.
        \end{theorem*}
        \begin{theorem*}
            If $A,B,C,$ and $D$ are sets such that
            $\Card(A)=\Card(C)$, $\Card(B)=\Card(D)$,
            and if ${A}\cap{B}=\emptyset$ and
            ${C}\cap{D}=\emptyset$, then
            $\Card({A}\cup{B})=\Card({C}\cup{D})$.
        \end{theorem*}
        \begin{theorem*}
            If $x,y,$ and $z$ are cardinal numbers, then
            $x+y=y+x$ and $x+(y+z)=(x+y)+z$.
        \end{theorem*}
        \begin{notation}
            The carinality of the set of natural numbers
            is denoted $\aleph_{0}$. That is,
            $\Card(\mathbb{N})=\aleph_{0}$
        \end{notation}
        \begin{example}
            Find the cardinal sum of $2$ and $5$. Let
            $N_{2}=\{1,2\}$ and $N_{5}=\{3,4,5,6,7\}$.
            Then $N_{2}$ and $N_{5}$ are disjoint,
            $\Card(N_{2})=2$ and $\Card(N_{5})=5$.
            Therefore $2+5=\Card(N_{2}\cup{N_{5}})$.
            But ${N_{2}}\cup{N_{5}}$ is just $\mathbb{Z}_{7}$,
            and $\Card(\mathbb{Z}_{7})=7$. Thus, $2+5=7$.
        \end{example}
        \begin{theorem*}
            If $n$ and $m$ are finite cardinalities,
            then the cardinal sum of $n$ and $m$ is the
            integer $n+m$, where $+$ is the usual
            arithmetic addition.
        \end{theorem*}
        \begin{example}
            Compute the cardinal sum
            $\aleph_{0}+\aleph_{0}$. Let
            $\mathbb{N}_{e}$ be the set of even natural
            numbers, and let $\mathbb{N}_{o}$ be the set
            of odd natural numbers. Then
            $\Card(\mathbb{N}_{e})=\aleph_{0}$,
            $\Card(\mathbb{N}_{o})=\aleph_{0}$, and
            ${\mathbb{N}_{o}}\cap{\mathbb{N}_{e}}=\emptyset$.
            Thus
            $\aleph_{0}+\aleph_{0}%
             =\Card({\mathbb{N}_{o}}\cup{\mathbb{N}_{e}})$.
            But
            ${\mathbb{N}_{o}}\cup{\mathbb{N}_{e}}%
             =\mathbb{N}$ and $\Card(\mathbb{N})=\aleph_{0}$.
            Therefore, $\aleph_{0}+\aleph_{0}=\aleph_{0}$.
        \end{example}
        \begin{example}
            Find $n+\aleph_{0}$, where $n\in\mathbb{N}$.
            We have that
            $\Card(\mathbb{Z}_{n}z)=n$ and
            $\Card(\mathbb{N}\setminus\mathbb{Z}_{n})%
             =\aleph_{0}$
            But then
            $n+\aleph_{0}=%
             \Card(\mathbb{Z}_{n}\cup%
             \mathbb{N}\setminus\mathbb{Z}_{n})%
             =\Card(\mathbb{N})=\aleph_{0}$.
            Therefore, $n+\aleph_{0}=\aleph_{0}$.
        \end{example}
        \begin{definition}
            The cardinality of the continuum,
            denoted $\mathfrak{c}$, is the
            cardinality of the set of real numbers.
            That is, $\mathfrak{c}=\Card(\mathbb{R})$.
        \end{definition}
        \begin{theorem*}
            $\Card([0,1])=\mathfrak{c}$.
        \end{theorem*}
        \begin{theorem*}
            $\Card\big((0,1)\big)=\mathfrak{c}$.
        \end{theorem*}
        \begin{theorem*}
            $\mathbb{R}$ is uncountable. That is,
            $\mathfrak{c}>\aleph_{0}$.
        \end{theorem*}
        \begin{theorem*}
            $\mathfrak{c}+\aleph_{0}=\mathfrak{c}$.
        \end{theorem*}
        \begin{proof}
            We have $\Card((0,1))=\mathfrak{c}$ and
            $\Card(\mathbb{N})=\aleph_{0}$. But
            $(0,1)\cap\mathbb{N}=\emptyset$, and thus
            $\aleph_{0}+\mathfrak{c}%
             =\Card((0,1)\cup\mathbb{N})$.
            But $\mathbb{R}\sim(0,1)$ and
            $\mathbb{N}\cup(0,1)\subset\mathbb{R}$.
            By the Schr\"{o}der-Bernstein theorem,
            $\mathbb{N}\cup(0,1)\sim\mathbb{R}$.
            Therefore, etc.
        \end{proof}
        \begin{definition}
            The product of two cardinal numbers $a$ and $b$
            is the cardinality of the cartesian product
            of two set $A$ and $B$ such that
            $\Card(A)=a$ and $\Card(B)=b$. That is,
            ${a}\times{b}=\Card({A}\times{B})$.
        \end{definition}
        \begin{theorem*}
            The following are true of cardinal numbers:
            \begin{enumerate}
                \begin{multicols}{3}
                    \item $xy=yx$
                    \item $x(yz)=(xy)z$
                    \item $x(y+z)=xy+xz$
                \end{multicols}
            \end{enumerate}
        \end{theorem*}
        \begin{proof}[Proof of Part 3]
            Let $A,B,$ and $C$ be disjoint.
            Then
            ${A}\times{({B}\cup{C})}%
             =({A}\times{B})\cup({A}\times{C})$, and thus
            $\Card({A}\times{({B}\cup{C})})%
             =\Card(({A}\times{B})\cup({A}\times{C}))$.
            But ${A}\times{B}$ and ${A}\times{C}$ are disjoint.
            Thus we have
            $\Card(({A}\times{B})\cup({A}\times{C}))%
             =\Card({A}\times{B})+\Card({A}\times{C})$.
            Therefore, etc.
        \end{proof}
        \begin{theorem*}
            If $\Card(T)=x$ and
            $F:{T}\rightarrow{\mathcal{P}(T)}$
            is a set-valued mapping such that for all
            ${t}\in{T}$ we have that
            $\Card(F(t))=y$ and
            for all ${t}\ne{t}$,
            ${F(t)}\cap{F(t')}=\emptyset$, then
            $\Card(\cup_{t=1}^{N}F(t))=xy$
        \end{theorem*}
        \begin{example}
            Let $f:{\mathbb{N}^{2}}\rightarrow{\mathbb{N}}$
            be defined by $f(n,m)=2^{n}3^{m}$.
            Then $f$ is injective, since $2$ and $3$
            are coprime. Therefore,
            $\aleph_{0}\times\aleph_{0}=\aleph_{0}$.
        \end{example}
        \begin{example}
            Show that $\mathbb{R}^{2}\sim\mathbb{R}$.
            Let $f:\mathbb{R}^{2}\rightarrow\mathbb{R}$
            be the rather bizarre function defined by the image
            $f(x_{0}.x_{1}x_{2}\hdots,y_{0}.y_{1}y_{2}\hdots)%
             =x_{0}y_{0}.x_{0}y_{0}x_{1}y_{1}\hdots$ Then
            $f$ is inective. But the mapping
            $g:\mathbb{R}\rightarrow\mathbb{R}^{2}$
            defined by $g(x)=(x,0)$ is also injective.
            By Schr\"{o}der-Bernstein,
            $\mathbb{R}^{2}\sim\mathbb{R}$.
        \end{example}
        \begin{definition}
           Order isomorphic set are two sets $A$ and $B$
           with well orders $<_{A}$ and $<_{B}$ such that
           there exists a bijection $f:{B}A\rightarrow{B}$
           such that for all $a_{1},a_{2}\in{A}$ such that
           $a_{1}<_{A}a_{2}$, $f(a_{1})<_{B}f(a_{2})$.
        \end{definition}
        \begin{theorem*}
           Order-Isomorphism is an equivalence relation.
        \end{theorem*}
        To every well ordered set, an ordinal number is
        assigned, denoted $\Ord(A,<_{A})$. Conversely,
        for every ordinal number there is a set with a
        well order corresponding to it. Two ordinal numbers
        are equal if and only if the well-ordered sets
        corresponding to them are order isomorphic.
        That is,
        $\Ord(A,<_{A})=\Ord(B,<_{B})$ if and only if
        $(A,<_{A})$ and $(B,<_{B})$ are order isomorphic.
        \begin{theorem*}
           If $(A,<_{A})$ and $(B,<_{B})$ are well ordered
           sets, and if $\Card(A)=\Card(B)$, then
           $(A,<_{A})$ and $(B,<_{B})$ are order
           isomorphic.
        \end{theorem*}
        The ordinal number of the empty set is $0$. The
        ordinal number of a finite set of $n$ elements with
        a well ordering is denoted $n\in\mathbb{N}$.
        The ordinal for the natural numbers $\mathbb{N}$
        with their usual well-ordering is denoted $\omega$.
        A given well-ordered set has only one cardinal number,
        but it is possible for it to have two ordinal numbers.
        \begin{definition}
           An ordinal number $\alpha$ is less than or equal
           to an ordinal number $\beta$ if there are
           well-ordered sets $(A,<_{A})$ and $(B,<_{B})$
           such that $\alpha=\Ord((A,<_{A}))$ and
           $\beta=\Ord(B,<_{B})$, and $(A,<_{B})$ is
           order isomorphic to subset of
           $(B,<_{B})$.
        \end{definition}
        \begin{theorem*}
           The only order isomorphism from a well ordered
           set $(A,<_{A})$ to itself is the identity
           isomorphism.
        \end{theorem*}
        \begin{theorem*}
           If $\alpha$ and $\beta$ are ordinal numbers and
           ${\alpha}\leq{\beta}$ and ${\beta}\leq{\alpha}$,
           then $\alpha=\beta$.
        \end{theorem*}
        \begin{theorem*}
           If $\alpha$ and $\beta$ are ordinal numbers,
           either ${\alpha}\leq{\beta}$, or
           ${\beta}\leq{\alpha}$.
        \end{theorem*}
        \begin{theorem*}
           If $\alpha$ and $\beta$ are ordinal numbers,
           either $\alpha<\beta$, $\beta<\alpha$, or
           $\alpha=\beta$.
        \end{theorem*}
        \begin{definition}
           The total ordering relation of a
           well-ordered set $(A,<_{A})$ with respect
           to a well-ordered set $(B,<_{B})$ is the ordering
           on the set ${A}\cup{B}$ defined as: For all
           $a_{1},a_{2}\in{A}$ such that $a_{1}<_{A}a_{2}$,
           $a_{1}<_{*}a_{2}$, for all $b_{1},b_{2}\in{B}$
           such that $b_{1}<_{B}b_{2}$, $b_{1}<_{*}b_{2}$,
           and for all ${a}\in{A}$ and ${b}\in{B}$,
           ${a}<_{*}{b}$.
        \end{definition}
        \begin{theorem*}
           The total ordering relation $<_{*}$ on the set
           ${A}\cup{B}$ is a well-ordering.
        \end{theorem*}
        \begin{definition}
           The ordinal sum of two ordinal numbers
           $\Ord((A,<_{A}))$ and $\Ord((B,<_{B}))$,
           where $A$ and $B$ are disjoint,
           is the ordinal number
           $\Ord(({A}\cup{B},<_{*}))$.
        \end{definition}
        \begin{theorem*}
           The following are true of ordinal numbers:
           \begin{enumerate}
               \begin{multicols}{3}
                   \item $\alpha<\beta\Rightarrow%
                          \alpha+\gamma<\beta+\gamma$
                   \item $(\alpha+\beta)+\gamma%
                          =\alpha+(\beta+\gamma)$
                   \item $\alpha+\beta=\alpha+\gamma%
                          \Rightarrow\beta=\gamma$.
               \end{multicols}
           \end{enumerate}
        \end{theorem*}
        \begin{definition}
           The lexicographic ordering on the cartesian
           product of well ordered set $(A,<_{A})$ and
           $(B,<_{B})$ is the ordering on
           ${A}\times{B}$ defined by: If ${a}<_{A}{x}$,
           then $(a,b)<_{*}(x,y)$ for all $b,y\in{B}$, and
           if $a=x$ and $b<_{B}y$, then $(a,b)<_{*}(x,y)$.
        \end{definition}
        \begin{theorem*}
           If $(A,<_{A})$ and $(B,<_{B})$ are well ordered
           sets, then the lexicographic ordering
           on ${A}\times{B}$ is a well ordering.
        \end{theorem*}
        \begin{definition}
           The ordinal product of two ordinal numbers
           $\Ord((A,<_{A}))$ and $\Ord((B,<_{B}))$,
           is $\Ord(({A}\times{B},<_{*}))$
        \end{definition}
        \begin{theorem*}
           The following are true of ordinal numbers:
           \begin{enumerate}
               \begin{multicols}{2}
                   \item $\alpha(\beta\gamma)%
                          =(\alpha\beta)\gamma$
                   \item $\alpha(\beta+\gamma)%
                          =\alpha\beta+\alpha\gamma$
               \end{multicols}
           \end{enumerate}
        \end{theorem*}
        \begin{definition}
           Relatively prime integers are integers
           $a,b\in\mathbb{N}$ such that $\gcd(a,b)=1$.
        \end{definition}
        \begin{theorem*}
           If $p$ is prime and $a\in\mathbb{N}$ is
           such that $p$ does not divide $a$, then $a$ and $p$
           are relatively prime.
        \end{theorem*}
        \begin{theorem*}
           There are infinitely many prime numbers.
        \end{theorem*}
        \begin{theorem*}
           If $a\in\mathbb{N}$, $a>1$, then either
           $a$ is a prime number, or $a$ is the product
           of finitely many primes.
        \end{theorem*}
        \begin{theorem*}
           If $a\in\mathbb{N}$, $a>1$, and if $a$ is not
           prime, then the prime expansion of $a$ is
           unique.
        \end{theorem*}
        \begin{definition}
           A diophantine equation is an equation whose
           solutions are required to be integers.
        \end{definition}
        \begin{definition}
           A linear diophantine equation in two variables
           $x$ and $y$ is an equation
           $ax+by=c$, where $a,b,c\in\mathbb{Z}$.
        \end{definition}
        \begin{theorem*}
           If $a,b,c\in\mathbb{Z}$ $d=\gcd(a,b)$,
           and if $d$ does not divide $c$,
           then $ax+by=c$ has no integral solutions.
        \end{theorem*}
        \begin{theorem*}
           If $a,b,c\in\mathbb{Z}$ $d=\gcd(a,b)$,
           and if $d$ divides $c$,
           then $ax+by=c$ has infinitely many solutions.
        \end{theorem*}
    %     \part{Ring Theory}
    %         \begingroup
    \ifcsname\PATH\endcsname
        \newcommand{\PATH}{books/Algebra/Ring_Theory}
        \newcommand{\OLDPATH}{\PATH}
    \else
        \newcommand{\OLDPATH}{\PATH}
        \renewcommand{\PATH}{books/Algebra/Ring_Theory}
    \fi
    \chapter{Definitions}
        \section{Rngs and Rings}
    \begin{fdefinition}{Rng}{Rng}
        A rng is an \gls{Abelian group} $(R,+)$ and a \gls{semigroup}
        $(R,\cdot\,)$ such that $\cdot$ is a \gls{distributive operation}
        over $+$. A rng is denoted $(R,+,\cdot\,)$.
    \end{fdefinition}
    The unital element of the Abelian group $(R,+)$ is often denoted 0 and is
    called the zero element. Zero has the unique property that multiplication
    by zero returns zero for any element.
    \begin{theorem}
        \label{thm:Mult_By_Zero_in_Rng}%
        If $(R,+,\cdot\,)$ is a rng, if 0 is the unital element of $(R,+)$, and
        if $r\in{R}$, then $r\cdot{0}=0$.
    \end{theorem}
    \begin{proof}
        For:
        \begin{align}
            0&=r\cdot{0}-r\cdot{0}
            \tag{Inverse Property of Groups}\\
            &=r\cdot(0-0)
            \tag{Distributive Property}\\
            &=r\cdot{0}
            \tag{Identity Property}
        \end{align}
        And therefore $r\cdot{0}=0$.
    \end{proof}
    \begin{fdefinition}{Ring}{Ring}
        A \gls{ring} is a \gls{rng} $(R,+,\cdot\,)$ such that $(R,\cdot\,)$ is
        a \gls{monoid}. That is, an \gls{Abelian group} $(R,+)$ and a monoid
        $(R,\cdot\,)$ such that $\cdot$ is a \gls{distributive operation} over
        $+$.
    \end{fdefinition}
    Zero is the only element that does this. The converse of
    Thm.~\ref{thm:Mult_By_Zero_in_Rng} is true.
    \begin{theorem}
        \label{thm:Mult_by_Zero_Always_Zero_Implies_Zero}%
        If $(R,+,\cdot\,)$ is a rng, if $0$ is the unital element of $(R,+)$,
        and if $r$ is such that for all $s\in{R}$ it is true that $r\cdot{s}=0$,
        then $r=0$.
    \end{theorem}
    \begin{proof}
        For if $(R,+,\cdot\,)$ is a ring, then $(R,\cdot\,)$ is a monoid
        (Def.~\ref{def:Ring}) and thus there is a unital element, 1, of
        $(R,\cdot\,)$ (Def.~\ref{def:Monoid}). But then for all $s\in{R}$
        we have:
        \begin{align}
            r+s&=(r+s)\cdot{1}
            \tag{Identity Property of 1}\\
            &=(r\cdot{1})+(s\cdot{1})
            \tag{Distributive Property}\\
            &=0+s\cdot{1}
            \tag{Hypothesis}\\
            &=s\cdot{1}
            \tag{Identity Property of 0}\\
            &=s
            \tag{Identity Property of 1}
        \end{align}
        And thus, for all $s\in{R}$, $r+s=s$. But $(R,+)$ is an Abelian group,
        and thus if $r+s=s$, then $s+r=s$ (Def.~\ref{def:Abelian_Group}) and
        therefore $r$ is a unital element of $(R,+)$
        (Def.~\ref{def:Unital_Element}). But the unital element of a group is
        unique, and therefore $r=0$.
    \end{proof}


        \section{Homomorphism}
    \begin{fdefinition}{Ring Homomorphism}{Ring_Homomorphism}
        A \gls{ring homomorphism}\index{Ring Homomorphism} from a \gls{ring}
        $(R_{1},\,+,\,\cdot\,)$ to a ring $(R_{2},\,+',\,*\,)$ is a
        \gls{function} $f:R_{1}\rightarrow{R}_{2}$ such that, for all
        $x,y\in{R}_{1}$, the following are true:
        \begin{align}
            f(x+y)&=f(x)+'f(y)
            \tag{Preservation of Addition}\\
            f(x\cdot{y})&=f(x)*f(y)
            \tag{Preservation of Multiplication}\\
            f(1_{R_{1}})&=1_{R_{2}}
            \tag{Preservation of Identities}
        \end{align}
        Where $1_{R_{1}}$ is the unital element of $R_{1}$ and
        $1_{R_{2}}$ is the unital element of $R_{2}$.
    \end{fdefinition}
    There's a special name for a homomorphism from a ring $(R,\,+,\,\cdot\,)$
    to itself.
    \begin{fdefinition}{Ring Endomorphisms}{Ring_Endomorphisms}
        A \gls{ring endomorphism}\index{Ring Endomorphism} on a \gls{ring}
        $(R,\,+,\,\cdot\,)$ is a \gls{ring homomorphism} from
        $(R,\,+,\,\cdot\,)$ to itself. That is, a ring homomorphism
        $f:R\rightarrow{R}$.
    \end{fdefinition}
    \begin{fnotation}{Set of Ring Endomorphisms}{Ring_Endomorphisms}
        The set of ring endomorphisms on a ring $\mathcal{R}=(R,\,+,\,\cdot)$
        is denoted $\textrm{End}(\mathcal{R})$.
    \end{fnotation}
    \renewcommand{\PATH}{\OLDPATH}
\endgroup
    %     \part{Fields}
    %         %------------------------------------------------------------------------------%
\begingroup
    \ifcsname\PATH\endcsname
        \newcommand{\PATH}{books/Algebra/Fields}
        \newcommand{\OLDPATH}{\PATH}
    \else
        \newcommand{\OLDPATH}{\PATH}
        \renewcommand{\PATH}{books/Algebra/Fields}
    \fi
    \chapter{Basic Properties}
        \section{Definitions}
    Given a function $f:X\rightarrow{Y}$, and any non-empty subset
    $S\subseteq{X}$, the image $f(S)$ is non-empty. This is not true for the
    pre-image of a function. For let $f:\mathbb{R}\rightarrow\mathbb{R}$ be
    defined by $f(x)=1$ for all $x\in\mathbb{R}$. Then, for any subset
    $S\subset\mathbb{R}$
    such that $1\notin{S}$, we have that $f^{\minus{1}}(S)=\emptyset$.
    There are many examples of functions, but certain ones are easier
    to study than others. We give some of these special functions names.
    \begin{ldefinition}{Injective Functions}{Injective_Function}
        An \gls{injective function} is a function
        $f:X\rightarrow{Y}$ such that, for all
        $x,y\in{X}$ such that $x\ne{y}$, it is true that
        $f(x)\ne{f}(y)$.
    \end{ldefinition}
    That is, an injective function is a function
    $f:X\rightarrow{Y}$ such that $f(x_{1})=f(x_{2})$
    if and only if $x_{1}=x_{2}$. Such functions are also
    called \textit{one-to-one}.
    \begin{lexample}{}{Natural_Log_Is_Injective}
        Consider the natural logarithm
        $\ln:\mathbb{R}^{+}\rightarrow\mathbb{R}$. This is an injective
        function. For let $x,y\in\mathbb{R}^{+}$ be such that
        $x\ne{y}$. Suppose $\ln(x)=\ln(y)$. But then:
        \begin{equation}
            \ln(x)-\ln(y)=\ln\Big(\frac{x}{y}\Big)=0
        \end{equation}
        Recall the definition of the natural logarithm:
        \begin{equation}
            \ln(t)=\int_{1}^{t}\frac{1}{x}\diff{x}
        \end{equation}
        But then $\ln(t)=0$ if and only if $t=1$. Thus $x=y$, a
        contradiction. Therefore $\ln$ is an injective function. Not
        every function is injective, for define
        $f:\mathbb{R}\rightarrow\mathbb{R}$ by $f(x)=x^{2}$. Then, for
        all $x\in\mathbb{R}^{+}$, $f(\minus{x})=f(x)$, and thus $f$
        cannot be an injective function.
    \end{lexample}
    One might think that most functions are not injective,
    and indeed for the \textit{finite} case, this is true.
    For let $A$ and $B$ be finite sets with $n$ and $m$
    elements, respectively. If $m<n$, there can't be
    any injective function. Consider the case when $n=m$.
    Then we are simply counting the number of ways to
    permute the elements of $A$. This is $n!$. On the
    other hand, the total number of functions is
    $n^{n}$. Thus, the ratio of the number of injective
    functions to the number of functions is
    $n!/n^{n}$, and this decays to zero rapidly as
    $n$ get's large. Finally, if $m>n$, then the total
    number of injective functions is
    $n!\binom{m}{n}$, where $\binom{m}{n}$ is the
    binomial coefficient. The total number of functions
    is $n^{m}$. The ratio is thus:
    \begin{equation}
        \frac{n!\binom{m}{n}}{n^{m}}=\frac{n!\frac{m!}{n!(m-n)!}}{n^{m}}
                                    =\frac{m!}{(m-n)!n^{m}}
    \end{equation}
    And again, this decays rapidly to zero and $n$ and $m$
    get large. Later, when we define infinite sets
    and the notion of Cardinality, we'll show that this
    trend continues. That is, in a sense, \textit{most}
    functions from a set $A$ to a sufficiently large set
    $B$ are not injective. Next, we define
    \textit{surjective} functions.
    \begin{ldefinition}{Surjective Functions}{Surjective_Function}
        A \gls{surjective function} is a function
        $f:X\rightarrow{Y}$ such that $f(X)=Y$.
        That is, for all $y\in{Y}$, there is an
        $x\in{X}$ such that $f(x)=y$.
    \end{ldefinition}
    That is, every point $y\in{Y}$ gets mapped to by
    at least one point in $X$. It may also be true that
    many points in $X$ map to the same point in $Y$.
    The notions of surjective functions and injective
    functions are distinct, and neither implies the
    other. Surjective functions are also called
    \textit{onto}.
    \begin{ldefinition}{Bijective Functions}{Bijective_Function}
        A \gls{bijective function} is a function
        that is both injective and surjective.
    \end{ldefinition}
    \begin{theorem}
        \label{thm:Image_of_Empty_Set_Is_Empty}%
        If $A$ and $B$ are sets, and if $f:A\rightarrow{B}$
        is a function, then:
        \begin{equation}
            f(\emptyset)=\emptyset
        \end{equation}
    \end{theorem}
    \begin{theorem}
        If $A$ and $B$ are sets, and if $f:A\rightarrow{B}$
        is a function, then:
        \begin{equation}
            f^{-1}(\emptyset)=\emptyset
        \end{equation}
    \end{theorem}
    \begin{theorem}
        If $X$ and $Y$ are sets, if $A\subseteq{X}$, and if
        $f:X\rightarrow{Y}$ is a function such that
        $f(A)=\emptyset$, then $A=\emptyset$.
    \end{theorem}
    \begin{proof}
        For suppose not. If $A\ne\emptyset$, then there is an $x\in{A}$.
        But then $f(x)\in{f}(A)$, a contradiction as $f(A)=\emptyset$.
    \end{proof}
    \begin{theorem}
        If $X$ and $Y$ are sets, if $B$ is a subset of $Y$,
        and if $f:X\rightarrow{Y}$ is a function, then:
        \begin{equation}
            f\big(f^{-1}(B)\big)\subseteq{B}
        \end{equation}
    \end{theorem}
    \begin{proof}
        For if $y\in{f(f^{-1}(B))}$, then there is an
        $x\in{f^{-1}(B)}$ such that $y=f(x)$. But if
        $x\in{f^{-1}(B)}$, then $f(x)\in{B}$. Thus,
        $y\in{B}$. Therefore, etc.
    \end{proof}
    \begin{theorem}
        If $X$ and $Y$ are non-empty sets and if there exists
        $y_{1},y_{2}\in{Y}$ such that $y_{1}\ne{y}_{2}$, then
        there is a function $f:X\rightarrow{Y}$ and a
        $B\subseteq{Y}$ such that:
        \begin{equation}
            f\big(f^{-1}(B)\big)\ne{B}
        \end{equation}
    \end{theorem}
    \begin{proof}
        \begin{subequations}
            For if $X$ and $Y$ are non-empty, let $f:X\rightarrow{Y}$
            be defined by:
            \begin{equation}
                f=\{(x,y_{1}):x\in{X}\}
            \end{equation}
            Then $f$ is a function, since $f\subseteq{X}\times{Y}$
            as $y_{1}\in{Y}$. Moreover, for all $x\in{X}$ there is a
            unique $y\in{Y}$ such that $(x,y)\in{f}$. Thus, $f$ is a
            function from $X$ to $Y$. However since for all
            $x\in{X}$, $f(x)=y_{1}$, we have that:
            \begin{equation}
                f^{-1}(\{y_{2}\})=\emptyset
            \end{equation}
            For suppose $x\in{f}^{-1}(\{y_{2}\})$.
            Then $f(x)=y_{2}$, but for all $x\in{X}$, $f(x)=y_{1}$,
            and $y_{1}\ne{y}_{2}$. Thus
            $f^{-1}(\{y_{2}\})=\emptyset$. But by
            Thm.~\ref{thm:Image_of_Empty_Set_Is_Empty},
            $f(\emptyset)=\emptyset$. Therefore:
            \begin{equation}
                f\big(f^{-1}(\{y_{2}\})\big)=\emptyset
            \end{equation}
            But $\{y_{2}\}\ne\emptyset$ and
            $\{y_{2}\}\subseteq{Y}$. Therefore, etc.
        \end{subequations}
    \end{proof}
    \begin{theorem}
        If $X$ and $Y$ are sets, if $A$ is a subset of $X$,
        and if $f:X\rightarrow{Y}$ is a function, then:
        \begin{equation}
            A\subseteq{f^{-1}}\big(f(A)\big)
        \end{equation}
    \end{theorem}
    \begin{proof}
        For if $x\in{A}$, then there is a $y\in{f}(A)$ such that
        $f(x)=y$. But then $x\in{f^{-1}(f(A))}$. Therefore, etc.
    \end{proof}
    \begin{theorem}
        If $X$ and $Y$ are sets, if $A_{1}$ and $A_{2}$ are
        subsets of $X$ such that $A_{1}\subseteq{A}_{2}$,
        and if $f:X\rightarrow{Y}$ is a function, then:
        \begin{equation}
            f(A_{1})\subseteq{f}(A_{2})
        \end{equation}
    \end{theorem}
    \begin{proof}
        For if $y\in{f}(A_{1})$, then there is an $x\in{A}_{1}$
        such that $f(x)=y$. But $A_{1}\subseteq{A}_{2}$, and
        therefore $x\in{A}_{2}$. But if $x\in{A}_{2}$, then
        $f(x)\in{f}(A_{2})$. Thus, $y\in{f}(A_{2})$. Therefore, etc.
    \end{proof}
    \begin{theorem}
        If $X$ and $Y$ are sets, if $B_{1}$ and $B_{2}$ are subsets of
        $Y$ such that $B_{1}\subseteq{B}_{2}$, and if $f:X\rightarrow{Y}$
        is a function, then:
        \begin{equation}
            f^{-1}(B_{1})\subseteq{f^{-1}}(B_{2})
        \end{equation}
    \end{theorem}
    \begin{proof}
        For if $x\in{f}^{-1}(B_{1})$, then there is a
        $y\in{B}_{1}$ such that $f(x)=y$. But
        $B_{1}\subseteq{B}_{2}$, and therefore $y\in{B}_{2}$.
        Thus, $x\in{f}^{-1}(B_{2})$. Therefore, etc.
    \end{proof}
    \begin{theorem}
    If $f:A\rightarrow B$, $A_1,A_2\subset A$, then $f(A_1 \cup A_2) = f(A_1)\cup f(A_2)$.
    \end{theorem}
    \begin{proof}
    $[y\in f(A_1\cup A_2)]\Rightarrow [\exists x\in A_1 \cup A_2:y=f(x)]\Rightarrow [y \in f(A_1)\cup f(A_2)]$. $[y\in f(A_1)\cup f(A_2)]\Rightarrow \big[[\exists x\in A_1] \lor[\exists x\in A_2]: y=f(x)\big]\Rightarrow [x\in A_1\cup A_2]\Rightarrow [f(x)\in f(A_1\cup A_2)]$
    \end{proof}
    \begin{theorem}
        If $f:A\rightarrow B$, $A_{1},A_{}2\subset A$, then
        $f(A_{1}\cap{A}_{2})\subset{f}(A_{1})\cap{f}(A_{2})$.
    \end{theorem}
    \begin{proof}
        $[y\in f(A_1 \cap A_2)]\Rightarrow [\exists x\in A_1 \cap A_2:y=f(x)]\Rightarrow [x\in A_1 \land x \in A_2] \Rightarrow[y \in f(A_1)\cap f(A_2)]$.
    \end{proof}
    \begin{theorem}
        If $A$ and $B$ are sets, $f:A\rightarrow{B}$ is a function,
        and $B_{1},B_{2}\subseteq{B}$, then:
        \begin{equation}
            f^{-1}(B_{1}\cup{B}_{2})=f^{-1}(B_{1})\cup{f}^{-1}(B_{2})
        \end{equation}
    \end{theorem}
    \begin{proof}
        For if $x\in{B}_{1}\cup{B}_{2}$, then
        $f(x)\in{B}_{1}\cup{B}_{2}$. but then either
        $f(x)\in{B}_{1}$ or $f(x)\in{B}_{2}$, and therefore
        $x\in{f}^{\minus{1}}(B_1)\cup{f}^{\minus{1}}(B_2)$. But if
        $x\in{f}^{\minus{1}}(B_{1})\cup{f}^{\minus{1}}(B_2)$, then
        $f(x)\in{B}_{1}$ or $f(x)\in{B}_{2}$. Therefore
        $f(x)\in{B}_{1}\cup{B}_{2}$. Thus, $x\in{f}^{-1}(B_1\cup{B}_2)$.
    \end{proof}
    \begin{theorem}
        If $A$ and $B$ are sets, $f:A\rightarrow{B}$ is a function,
        and $B_{1},B_{2}\subseteq{B}$, then:
        \begin{equation}
            f^{-1}(B_{1}\cap{B}_{2})=f^{-1}(B_{1})\cap{f}^{-1}(B_{2})
        \end{equation}
    \end{theorem}
    \begin{proof}
        $[x\in f^{-1}(B_1\cap B_2)]\Rightarrow [f(x) \in B_1 \cap B_2]\Rightarrow [f(x)\in B_1\land f(x) \in B_2 ]\Rightarrow [x\in f^{-1}(B_1)\cap f^{-1}(B_2)]$. $[x\in f^{-1}(B_1)\cap f^{-1}(B_2)]\Rightarrow [x\in f^{-1}(B_1)\land x\in f^{-1}(B_2)]\Rightarrow [f(x) \in B_1\land f(x) \in B_2]\Rightarrow [f(x)\in B_1\cap B_2]\Rightarrow [x\in f^{-1}(B_1\cap B_2)]$.
    \end{proof}
    \begin{theorem}
    If $f:A\rightarrow B$, $B_1 \subset B$, then $f^{-1}(B\setminus B_1) = f^{-1}(B)\setminus f^{-1}(B_1)$.
    \end{theorem}
    \begin{proof}
    $[x\in f^{-1}(B\setminus B_1)]\Leftrightarrow [f(x)\notin B_1]\Leftrightarrow [x\in f^{-1}(B)\setminus f^{-1}(B_1)]$
    \end{proof}
    If $f:A\rightarrow B$, the image of $A$ under $f$
    is often called the range (A is often called the domain).
    \begin{ldefinition}{Permutations}{Permutations}
        A permutation on a set $A$ is a bijective function
        $f:A\rightarrow{A}$.
    \end{ldefinition}
    \begin{theorem}
    If $f:A\rightarrow B$ is bijective, then $f^{-1}$ is bijective.
    \end{theorem}
    \begin{proof}
    $[f^{-1}(y_1) = f^{-1}(y_2)]\Rightarrow [\exists x\in A:[f(x) = y_1]\land [f(x)=y_2]]\Rightarrow [y_1=y_2]$. By definition, $f^{-1}$ is surjective.
    \end{proof}
    \begin{definition}
    If $f:A\rightarrow B$ and $g:B\rightarrow C$, then $g\circ f:A\rightarrow C$ is defined by the image $g(f(x)), x\in A$. 
    \end{definition}
    \begin{theorem}
    If $f:A\rightarrow B$, $g:B\rightarrow C$, and $\mathcal{V}\subset C$, then $(g\circ g)^{-1}(\mathcal{V}) = f^{-1}(g^{-1}(\mathcal{V}))$.
    \end{theorem}
    \begin{proof}
    $[x\in (g\circ f)^{-1}(\mathcal{V})]\Leftrightarrow [g(f(x))\in \mathcal{V}] \Leftrightarrow [f(x)\in g^{-1}(\mathcal{V})]\Leftrightarrow [x\in f^{-1}(g^{-1}(\mathcal{V}))]$.
    \end{proof}
    \begin{theorem}
    If $f:A\rightarrow B$ is bijective, $g:B\rightarrow C$ is bijective, then $g\circ f$ is bijective.
    \end{theorem}
    \begin{proof}
    $\big[[f(A) = B]\land [g(B) = C]\big]\Rightarrow [g(f(A)) = g(B) = C]$. $[g(f(x_1))=g(f(x_2))]\Leftrightarrow [f(x_1)=f(x_2)]\Leftrightarrow [x_1=x_2]$.
    \end{proof}
    \begin{theorem}
    If $f:A\rightarrow B$ is bijective, $A_1\subset A$, and $f(A_1) = B$, then $A_1=A$.
    \end{theorem}
    \begin{proof}
    $\Big[\big[[A_1^c \ne \emptyset]\Rightarrow [f(A_1^c) \ne \emptyset]\big]\land[f(A_1)\cap f(A_1^c) = \emptyset]\Big]\Rightarrow [\exists y\in B:y\notin f(A_1)]$, a contradiction.
    \end{proof}

    \renewcommand{\PATH}{\OLDPATH}
\endgroup
    %     \part{Modules}
    %         %--------------------------------------------------------------------------%
\chapter{Elementary Properties}
    \ifcsname\PATH\endcsname
        \newcommand{\PATH}{books/Algebra/Modules/Elementary_Properties}
    \else
        \renewcommand{\PATH}{books/Algebra/Modules/Elementary_Properties}
    \fi

    \section{Definitions}
    Given a function $f:X\rightarrow{Y}$, and any non-empty subset
    $S\subseteq{X}$, the image $f(S)$ is non-empty. This is not true for the
    pre-image of a function. For let $f:\mathbb{R}\rightarrow\mathbb{R}$ be
    defined by $f(x)=1$ for all $x\in\mathbb{R}$. Then, for any subset
    $S\subset\mathbb{R}$
    such that $1\notin{S}$, we have that $f^{\minus{1}}(S)=\emptyset$.
    There are many examples of functions, but certain ones are easier
    to study than others. We give some of these special functions names.
    \begin{ldefinition}{Injective Functions}{Injective_Function}
        An \gls{injective function} is a function
        $f:X\rightarrow{Y}$ such that, for all
        $x,y\in{X}$ such that $x\ne{y}$, it is true that
        $f(x)\ne{f}(y)$.
    \end{ldefinition}
    That is, an injective function is a function
    $f:X\rightarrow{Y}$ such that $f(x_{1})=f(x_{2})$
    if and only if $x_{1}=x_{2}$. Such functions are also
    called \textit{one-to-one}.
    \begin{lexample}{}{Natural_Log_Is_Injective}
        Consider the natural logarithm
        $\ln:\mathbb{R}^{+}\rightarrow\mathbb{R}$. This is an injective
        function. For let $x,y\in\mathbb{R}^{+}$ be such that
        $x\ne{y}$. Suppose $\ln(x)=\ln(y)$. But then:
        \begin{equation}
            \ln(x)-\ln(y)=\ln\Big(\frac{x}{y}\Big)=0
        \end{equation}
        Recall the definition of the natural logarithm:
        \begin{equation}
            \ln(t)=\int_{1}^{t}\frac{1}{x}\diff{x}
        \end{equation}
        But then $\ln(t)=0$ if and only if $t=1$. Thus $x=y$, a
        contradiction. Therefore $\ln$ is an injective function. Not
        every function is injective, for define
        $f:\mathbb{R}\rightarrow\mathbb{R}$ by $f(x)=x^{2}$. Then, for
        all $x\in\mathbb{R}^{+}$, $f(\minus{x})=f(x)$, and thus $f$
        cannot be an injective function.
    \end{lexample}
    One might think that most functions are not injective,
    and indeed for the \textit{finite} case, this is true.
    For let $A$ and $B$ be finite sets with $n$ and $m$
    elements, respectively. If $m<n$, there can't be
    any injective function. Consider the case when $n=m$.
    Then we are simply counting the number of ways to
    permute the elements of $A$. This is $n!$. On the
    other hand, the total number of functions is
    $n^{n}$. Thus, the ratio of the number of injective
    functions to the number of functions is
    $n!/n^{n}$, and this decays to zero rapidly as
    $n$ get's large. Finally, if $m>n$, then the total
    number of injective functions is
    $n!\binom{m}{n}$, where $\binom{m}{n}$ is the
    binomial coefficient. The total number of functions
    is $n^{m}$. The ratio is thus:
    \begin{equation}
        \frac{n!\binom{m}{n}}{n^{m}}=\frac{n!\frac{m!}{n!(m-n)!}}{n^{m}}
                                    =\frac{m!}{(m-n)!n^{m}}
    \end{equation}
    And again, this decays rapidly to zero and $n$ and $m$
    get large. Later, when we define infinite sets
    and the notion of Cardinality, we'll show that this
    trend continues. That is, in a sense, \textit{most}
    functions from a set $A$ to a sufficiently large set
    $B$ are not injective. Next, we define
    \textit{surjective} functions.
    \begin{ldefinition}{Surjective Functions}{Surjective_Function}
        A \gls{surjective function} is a function
        $f:X\rightarrow{Y}$ such that $f(X)=Y$.
        That is, for all $y\in{Y}$, there is an
        $x\in{X}$ such that $f(x)=y$.
    \end{ldefinition}
    That is, every point $y\in{Y}$ gets mapped to by
    at least one point in $X$. It may also be true that
    many points in $X$ map to the same point in $Y$.
    The notions of surjective functions and injective
    functions are distinct, and neither implies the
    other. Surjective functions are also called
    \textit{onto}.
    \begin{ldefinition}{Bijective Functions}{Bijective_Function}
        A \gls{bijective function} is a function
        that is both injective and surjective.
    \end{ldefinition}
    \begin{theorem}
        \label{thm:Image_of_Empty_Set_Is_Empty}%
        If $A$ and $B$ are sets, and if $f:A\rightarrow{B}$
        is a function, then:
        \begin{equation}
            f(\emptyset)=\emptyset
        \end{equation}
    \end{theorem}
    \begin{theorem}
        If $A$ and $B$ are sets, and if $f:A\rightarrow{B}$
        is a function, then:
        \begin{equation}
            f^{-1}(\emptyset)=\emptyset
        \end{equation}
    \end{theorem}
    \begin{theorem}
        If $X$ and $Y$ are sets, if $A\subseteq{X}$, and if
        $f:X\rightarrow{Y}$ is a function such that
        $f(A)=\emptyset$, then $A=\emptyset$.
    \end{theorem}
    \begin{proof}
        For suppose not. If $A\ne\emptyset$, then there is an $x\in{A}$.
        But then $f(x)\in{f}(A)$, a contradiction as $f(A)=\emptyset$.
    \end{proof}
    \begin{theorem}
        If $X$ and $Y$ are sets, if $B$ is a subset of $Y$,
        and if $f:X\rightarrow{Y}$ is a function, then:
        \begin{equation}
            f\big(f^{-1}(B)\big)\subseteq{B}
        \end{equation}
    \end{theorem}
    \begin{proof}
        For if $y\in{f(f^{-1}(B))}$, then there is an
        $x\in{f^{-1}(B)}$ such that $y=f(x)$. But if
        $x\in{f^{-1}(B)}$, then $f(x)\in{B}$. Thus,
        $y\in{B}$. Therefore, etc.
    \end{proof}
    \begin{theorem}
        If $X$ and $Y$ are non-empty sets and if there exists
        $y_{1},y_{2}\in{Y}$ such that $y_{1}\ne{y}_{2}$, then
        there is a function $f:X\rightarrow{Y}$ and a
        $B\subseteq{Y}$ such that:
        \begin{equation}
            f\big(f^{-1}(B)\big)\ne{B}
        \end{equation}
    \end{theorem}
    \begin{proof}
        \begin{subequations}
            For if $X$ and $Y$ are non-empty, let $f:X\rightarrow{Y}$
            be defined by:
            \begin{equation}
                f=\{(x,y_{1}):x\in{X}\}
            \end{equation}
            Then $f$ is a function, since $f\subseteq{X}\times{Y}$
            as $y_{1}\in{Y}$. Moreover, for all $x\in{X}$ there is a
            unique $y\in{Y}$ such that $(x,y)\in{f}$. Thus, $f$ is a
            function from $X$ to $Y$. However since for all
            $x\in{X}$, $f(x)=y_{1}$, we have that:
            \begin{equation}
                f^{-1}(\{y_{2}\})=\emptyset
            \end{equation}
            For suppose $x\in{f}^{-1}(\{y_{2}\})$.
            Then $f(x)=y_{2}$, but for all $x\in{X}$, $f(x)=y_{1}$,
            and $y_{1}\ne{y}_{2}$. Thus
            $f^{-1}(\{y_{2}\})=\emptyset$. But by
            Thm.~\ref{thm:Image_of_Empty_Set_Is_Empty},
            $f(\emptyset)=\emptyset$. Therefore:
            \begin{equation}
                f\big(f^{-1}(\{y_{2}\})\big)=\emptyset
            \end{equation}
            But $\{y_{2}\}\ne\emptyset$ and
            $\{y_{2}\}\subseteq{Y}$. Therefore, etc.
        \end{subequations}
    \end{proof}
    \begin{theorem}
        If $X$ and $Y$ are sets, if $A$ is a subset of $X$,
        and if $f:X\rightarrow{Y}$ is a function, then:
        \begin{equation}
            A\subseteq{f^{-1}}\big(f(A)\big)
        \end{equation}
    \end{theorem}
    \begin{proof}
        For if $x\in{A}$, then there is a $y\in{f}(A)$ such that
        $f(x)=y$. But then $x\in{f^{-1}(f(A))}$. Therefore, etc.
    \end{proof}
    \begin{theorem}
        If $X$ and $Y$ are sets, if $A_{1}$ and $A_{2}$ are
        subsets of $X$ such that $A_{1}\subseteq{A}_{2}$,
        and if $f:X\rightarrow{Y}$ is a function, then:
        \begin{equation}
            f(A_{1})\subseteq{f}(A_{2})
        \end{equation}
    \end{theorem}
    \begin{proof}
        For if $y\in{f}(A_{1})$, then there is an $x\in{A}_{1}$
        such that $f(x)=y$. But $A_{1}\subseteq{A}_{2}$, and
        therefore $x\in{A}_{2}$. But if $x\in{A}_{2}$, then
        $f(x)\in{f}(A_{2})$. Thus, $y\in{f}(A_{2})$. Therefore, etc.
    \end{proof}
    \begin{theorem}
        If $X$ and $Y$ are sets, if $B_{1}$ and $B_{2}$ are subsets of
        $Y$ such that $B_{1}\subseteq{B}_{2}$, and if $f:X\rightarrow{Y}$
        is a function, then:
        \begin{equation}
            f^{-1}(B_{1})\subseteq{f^{-1}}(B_{2})
        \end{equation}
    \end{theorem}
    \begin{proof}
        For if $x\in{f}^{-1}(B_{1})$, then there is a
        $y\in{B}_{1}$ such that $f(x)=y$. But
        $B_{1}\subseteq{B}_{2}$, and therefore $y\in{B}_{2}$.
        Thus, $x\in{f}^{-1}(B_{2})$. Therefore, etc.
    \end{proof}
    \begin{theorem}
    If $f:A\rightarrow B$, $A_1,A_2\subset A$, then $f(A_1 \cup A_2) = f(A_1)\cup f(A_2)$.
    \end{theorem}
    \begin{proof}
    $[y\in f(A_1\cup A_2)]\Rightarrow [\exists x\in A_1 \cup A_2:y=f(x)]\Rightarrow [y \in f(A_1)\cup f(A_2)]$. $[y\in f(A_1)\cup f(A_2)]\Rightarrow \big[[\exists x\in A_1] \lor[\exists x\in A_2]: y=f(x)\big]\Rightarrow [x\in A_1\cup A_2]\Rightarrow [f(x)\in f(A_1\cup A_2)]$
    \end{proof}
    \begin{theorem}
        If $f:A\rightarrow B$, $A_{1},A_{}2\subset A$, then
        $f(A_{1}\cap{A}_{2})\subset{f}(A_{1})\cap{f}(A_{2})$.
    \end{theorem}
    \begin{proof}
        $[y\in f(A_1 \cap A_2)]\Rightarrow [\exists x\in A_1 \cap A_2:y=f(x)]\Rightarrow [x\in A_1 \land x \in A_2] \Rightarrow[y \in f(A_1)\cap f(A_2)]$.
    \end{proof}
    \begin{theorem}
        If $A$ and $B$ are sets, $f:A\rightarrow{B}$ is a function,
        and $B_{1},B_{2}\subseteq{B}$, then:
        \begin{equation}
            f^{-1}(B_{1}\cup{B}_{2})=f^{-1}(B_{1})\cup{f}^{-1}(B_{2})
        \end{equation}
    \end{theorem}
    \begin{proof}
        For if $x\in{B}_{1}\cup{B}_{2}$, then
        $f(x)\in{B}_{1}\cup{B}_{2}$. but then either
        $f(x)\in{B}_{1}$ or $f(x)\in{B}_{2}$, and therefore
        $x\in{f}^{\minus{1}}(B_1)\cup{f}^{\minus{1}}(B_2)$. But if
        $x\in{f}^{\minus{1}}(B_{1})\cup{f}^{\minus{1}}(B_2)$, then
        $f(x)\in{B}_{1}$ or $f(x)\in{B}_{2}$. Therefore
        $f(x)\in{B}_{1}\cup{B}_{2}$. Thus, $x\in{f}^{-1}(B_1\cup{B}_2)$.
    \end{proof}
    \begin{theorem}
        If $A$ and $B$ are sets, $f:A\rightarrow{B}$ is a function,
        and $B_{1},B_{2}\subseteq{B}$, then:
        \begin{equation}
            f^{-1}(B_{1}\cap{B}_{2})=f^{-1}(B_{1})\cap{f}^{-1}(B_{2})
        \end{equation}
    \end{theorem}
    \begin{proof}
        $[x\in f^{-1}(B_1\cap B_2)]\Rightarrow [f(x) \in B_1 \cap B_2]\Rightarrow [f(x)\in B_1\land f(x) \in B_2 ]\Rightarrow [x\in f^{-1}(B_1)\cap f^{-1}(B_2)]$. $[x\in f^{-1}(B_1)\cap f^{-1}(B_2)]\Rightarrow [x\in f^{-1}(B_1)\land x\in f^{-1}(B_2)]\Rightarrow [f(x) \in B_1\land f(x) \in B_2]\Rightarrow [f(x)\in B_1\cap B_2]\Rightarrow [x\in f^{-1}(B_1\cap B_2)]$.
    \end{proof}
    \begin{theorem}
    If $f:A\rightarrow B$, $B_1 \subset B$, then $f^{-1}(B\setminus B_1) = f^{-1}(B)\setminus f^{-1}(B_1)$.
    \end{theorem}
    \begin{proof}
    $[x\in f^{-1}(B\setminus B_1)]\Leftrightarrow [f(x)\notin B_1]\Leftrightarrow [x\in f^{-1}(B)\setminus f^{-1}(B_1)]$
    \end{proof}
    If $f:A\rightarrow B$, the image of $A$ under $f$
    is often called the range (A is often called the domain).
    \begin{ldefinition}{Permutations}{Permutations}
        A permutation on a set $A$ is a bijective function
        $f:A\rightarrow{A}$.
    \end{ldefinition}
    \begin{theorem}
    If $f:A\rightarrow B$ is bijective, then $f^{-1}$ is bijective.
    \end{theorem}
    \begin{proof}
    $[f^{-1}(y_1) = f^{-1}(y_2)]\Rightarrow [\exists x\in A:[f(x) = y_1]\land [f(x)=y_2]]\Rightarrow [y_1=y_2]$. By definition, $f^{-1}$ is surjective.
    \end{proof}
    \begin{definition}
    If $f:A\rightarrow B$ and $g:B\rightarrow C$, then $g\circ f:A\rightarrow C$ is defined by the image $g(f(x)), x\in A$. 
    \end{definition}
    \begin{theorem}
    If $f:A\rightarrow B$, $g:B\rightarrow C$, and $\mathcal{V}\subset C$, then $(g\circ g)^{-1}(\mathcal{V}) = f^{-1}(g^{-1}(\mathcal{V}))$.
    \end{theorem}
    \begin{proof}
    $[x\in (g\circ f)^{-1}(\mathcal{V})]\Leftrightarrow [g(f(x))\in \mathcal{V}] \Leftrightarrow [f(x)\in g^{-1}(\mathcal{V})]\Leftrightarrow [x\in f^{-1}(g^{-1}(\mathcal{V}))]$.
    \end{proof}
    \begin{theorem}
    If $f:A\rightarrow B$ is bijective, $g:B\rightarrow C$ is bijective, then $g\circ f$ is bijective.
    \end{theorem}
    \begin{proof}
    $\big[[f(A) = B]\land [g(B) = C]\big]\Rightarrow [g(f(A)) = g(B) = C]$. $[g(f(x_1))=g(f(x_2))]\Leftrightarrow [f(x_1)=f(x_2)]\Leftrightarrow [x_1=x_2]$.
    \end{proof}
    \begin{theorem}
    If $f:A\rightarrow B$ is bijective, $A_1\subset A$, and $f(A_1) = B$, then $A_1=A$.
    \end{theorem}
    \begin{proof}
    $\Big[\big[[A_1^c \ne \emptyset]\Rightarrow [f(A_1^c) \ne \emptyset]\big]\land[f(A_1)\cap f(A_1^c) = \emptyset]\Big]\Rightarrow [\exists y\in B:y\notin f(A_1)]$, a contradiction.
    \end{proof}
    %     \part{Unsorted Stuff}
    %         \documentclass[crop=false,class=book,oneside]{standalone}
%----------------------------Preamble-------------------------------%
%---------------------------Packages----------------------------%
\usepackage{geometry}
\geometry{b5paper, margin=1.0in}
\usepackage[T1]{fontenc}
\usepackage{graphicx, float}            % Graphics/Images.
\usepackage{natbib}                     % For bibliographies.
\bibliographystyle{agsm}                % Bibliography style.
\usepackage[french, english]{babel}     % Language typesetting.
\usepackage[dvipsnames]{xcolor}         % Color names.
\usepackage{listings, lstlinebgrd}      % Verbatim-Like Tools.
\usepackage{mathtools, esint, mathrsfs} % amsmath and integrals.
\usepackage{amsthm, amsfonts}           % Fonts and theorems.
\usepackage{tabularx}
\usepackage{tcolorbox}                  % Frames around theorems.
\usepackage{upgreek}                    % Non-Italic Greek.
\usepackage{paracol}                    % Two-column styling.
\usepackage{wrapfig}                    % Wrap text around figure.
\usepackage{fmtcount, etoolbox}         % For the \book{} command.
\usepackage[newparttoc]{titlesec}       % Formatting chapter, etc.
\usepackage{titletoc}                   % Allows \book in toc.
\usepackage[nottoc]{tocbibind}          % Bibliography in toc.
\usepackage[titles]{tocloft}            % ToC formatting.
\usepackage{multicol, enumitem}         % Multi-column/enumerate.
\usepackage{import}                     % Import external files.
\usepackage{pgfplots, tikz}             % Drawing/graphing tools.
\usetikzlibrary{
    calc,                   % Calculating right angles and more.
    angles,                 % Drawing angles within triangles.
    arrows.meta,            % Latex and Stealth arrows.
    quotes,                 % Adding labels to angles.
    positioning,            % Relative positioning of nodes.
    decorations.markings,   % Adding arrows in the middle of a line.
    patterns,
    arrows,
    shapes,
    shapes.geometric,
    cd,
    hobby,
    babel
}                                       % Libraries for tikz.
\pgfplotsset{compat=1.9}                % Version of pgfplots.
\usepackage[font=scriptsize,
            labelformat=simple,
            labelsep=colon]{subcaption} % Subfigure captions.
\usepackage[font={scriptsize},
            hypcap=true,
            labelsep=colon]{caption}    % Figure captions.
\usepackage{hyperref}                   % Allows for hyperlinks.
\hypersetup{
    colorlinks=true,
    linkcolor=blue,
    filecolor=magenta,
    urlcolor=Cerulean,
    citecolor=SkyBlue
}                           % Colors for hyperref.
\usepackage[toc,acronym,nogroupskip]{glossaries} % Glossaries and acronyms.
\usepackage[subpreambles=false]{standalone}      % Complileable sub files.

% Various font stuff from kiwi.
% Use this for Times text and Computer Modern math
%\usepackage{times}

% Quite nice
%\usepackage[charter, greekfamily=, greekuppercase=italicized]{mathdesign}
%\usepackage[utopia, greekuppercase=italicized]{mathdesign}    % Math is narrower

% Use this for Times text and math
%\usepackage{newtxtext}
%\usepackage[libertine,cmintegrals]{newtxmath}
%\usepackage{fix-cm}

%\usepackage{txfontsb}
% or
%\usepackage{mathptmx}

%\usepackage[scaled=0.92]{helvet}
%\renewcommand{\rmdefault}{ptm}

%\usepackage{mathpazo}    % add possibly `sc` and `osf` options
%\usepackage{eulervm}

%\usepackage{fourier}
%\renewcommand{\rmdefault}{ptm}
%\usepackage{mathptm}

%\usepackage{fontspec}
%\setmainfont{lmodern}

%\usepackage[varg]{txfonts}
%\usepackage{fouriernc}
%\usepackage{mathpazo}

%\usepackage{bookman}
%\usepackage[scaled]{uarial}
%\usepackage[scaled]{helvet}
%\renewcommand*\familydefault{\sfdefault}
%\usepackage[math]{anttor}

%\newcommand\fgeorgia{\fontfamily{jvn}\selectfont}
%\newcommand\ftimes{\fontfamily{ptm}\selectfont}
%\newcommand\fhelvetica{\fontfamily{phv}\selectfont}
%\newcommand\fcourier{\fontfamily{pcr}\selectfont}
%\newcommand\fbookman{\fontfamily{pbk}\selectfont}
%\newcommand\fnewcentury{\fontfamily{pnc}\selectfont}
%\newcommand\fpalatino{\fontfamily{ppl}\selectfont}
%\newcommand\favantgarde{\fontfamily{pag}\selectfont}
%\newcommand\fnormal{\normalfont}
%\newcommand\fsize[1]{\ifnum#1>0\fontsize{#1}{#1}\selectfont\else\normalsize\fi}
%------------------------Theorem Styles-------------------------%
% Define theorem style for default spacing and normal font.
\newtheoremstyle{normal}
    {\topsep}               % Amount of space above the theorem.
    {\topsep}               % Amount of space below the theorem.
    {}                      % Font used for body of theorem.
    {}                      % Measure of space to indent.
    {\bfseries}             % Font of the header of the theorem.
    {}                      % Punctuation between head and body.
    {.5em}                  % Space after theorem head.
    {}

% Define theorem style for default spacing with italicized font.
\newtheoremstyle{normalit}{\topsep}{\topsep}
                {\itshape}{}{\bfseries}{}{.5em}{}

% Italic header environment.
\newtheoremstyle{thmit}{\topsep}{\topsep}{}{}{\itshape}{}{0.5em}{}

% Define italicized environments.
\theoremstyle{normalit}
\newtheorem{theorem}{Theorem}[section]
\newtheorem{lemma}{Lemma}[section]
\newtheorem{corollary}{Corollary}[section]
\newtheorem{proposition}{Proposition}[section]
\newtheorem*{theorem*}{Theorem}

% Define environments with italic headers.
\theoremstyle{thmit}
\newtheorem*{solution}{Solution}
\newtheorem*{fsolution}{Solution}

% Define default environments.
\theoremstyle{normal}
\newtheorem{example}{Example}[section]
\newtheorem{definition}{Definition}[section]
\newtheorem{problem}{Problem}[section]
\newtheorem{question}{Question}[section]
\newtheorem{remark}{Remark}[section]
\newtheorem{properties}{Properties}[section]
\newtheorem{notation}{Notation}[section]
\newtheorem{axiom}{Axiom}[section]
\newtheorem*{properties*}{Properties}
\newtheorem*{remark*}{Remark}
\newtheorem*{definition*}{Definition}
\theoremstyle{plain}

% Define framed environment.
\tcbuselibrary{most}
\newtcbtheorem[use counter*=theorem]{ftheorem}{Theorem}%
    {colback=green!5,colframe=green!35!black,
     fonttitle=\bfseries\upshape}{th}

\newtcbtheorem[use counter*=example]{fdefinition}{Definition}%
    {fonttitle=\bfseries\upshape,
     colback=blue!5!white,colframe=blue!75!black}{def}

\newtcbtheorem[use counter*=example]{fexample}{Example}%
    {fonttitle=\bfseries\upshape,
     colback=red!5!white,colframe=red!75!black}{ex}

\newtcbtheorem[use counter*=notation]{fnotation}{Notation}%
    {fonttitle=\bfseries\upshape,
     colback=SeaGreen!5!white,colframe=SeaGreen!75!black}{ex}

\newtcbtheorem[use counter*=corollary]{fcorollary}{Corollary}%
    {fonttitle=\bfseries\upshape,
     colback=Orchid!5!white,colframe=Orchid!75!black}{ex}

\newenvironment{bproof}{\textit{Proof.}}{\hfill$\square$}
\tcolorboxenvironment{bproof}{blanker,breakable,left=5mm,
                             before skip=10pt,after skip=10pt,
                             borderline west={1mm}{0pt}{red}}
\tcolorboxenvironment{fsolution}
    {enhanced jigsaw,colframe=cyan,interior hidden,breakable}

%--------------------Declared Math Operators--------------------%
\DeclareMathOperator{\Refl}{Refl}           % Reflection operator.
\DeclareMathOperator{\Span}{Span}           % Span of a set of vectors.
\DeclareMathOperator{\Card}{Card}           % Cardinality of set.
\DeclareMathOperator{\Ord}{Ord}             % Ordinal of ordered set.
\DeclareMathOperator{\Tr}{Tr}               % Trace of matrix.
\DeclareMathOperator{\adjoint}{adj}         % Adjoint of matrix.
\DeclareMathOperator{\rk}{rk}               % Rank of operator.
\DeclareMathOperator{\nul}{nul}             % Null space of operator.
\DeclareMathOperator{\sgn}{sgn}             % Sign of a number.
\DeclareMathOperator{\multideg}{mutlideg}   % Multi-Degree (Graphs).
\DeclareMathOperator{\GCD}{GCD}             % Greatest common denominator.
\DeclareMathOperator{\LM}{LM}               % Leading monomial
\DeclareMathOperator{\LC}{LC}               % Leading coefficient.
\DeclareMathOperator{\LT}{LT}               % Leading term.
\DeclareMathOperator{\LCM}{LCM}             % Least common multiple.
\DeclareMathOperator{\Mon}{Mon}             % Monomial.
\DeclareMathOperator{\Spec}{Spec}           % Spectrum.
\DeclareMathOperator{\proj}{proj}           % Projection.
\DeclareMathOperator{\comp}{comp}           % Component.
\DeclareMathOperator{\sinc}{sinc}           % Sinc function.
\DeclareMathOperator{\Ima}{Im}              % Image of operator.
\DeclareMathOperator{\Prin}{Prin}           % Principal value.
\DeclareMathOperator{\Mod}{mod}             % Modulus.
%------------------------New Commands---------------------------%
\DeclarePairedDelimiter\norm{\lVert}{\rVert}
\DeclarePairedDelimiter\ceil{\lceil}{\rceil}
\DeclarePairedDelimiter\floor{\lfloor}{\rfloor}
\newcommand*\diff{\mathop{}\!\mathrm{d}}
\newcommand*\Diff[1]{\mathop{}\!\mathrm{d^#1}}
\renewcommand{\mod}{\ \Mod}
\renewcommand*{\glstextformat}[1]{\textcolor{RoyalBlue}{#1}}
\renewcommand{\glsnamefont}[1]{\textbf{#1}}
\renewcommand\labelitemii{$\circ$}
\renewcommand\thesubfigure{\arabic{chapter}.\arabic{figure}}
\renewcommand\thesubfigure{%
    \arabic{chapter}.\arabic{figure}.\arabic{subfigure}}
\addto\captionsenglish{\renewcommand{\figurename}{Fig.}}
%------------------------Book Command---------------------------%
\makeatletter
\renewcommand\@pnumwidth{1cm}
\newcounter{book}
\renewcommand\thebook{\@Roman\c@book}
\newcommand\book{%
    \if@openright
        \cleardoublepage
    \else
        \clearpage
    \fi
    \thispagestyle{plain}%
    \if@twocolumn
        \onecolumn
        \@tempswatrue
    \else
        \@tempswafalse
    \fi
    \null\vfil
    \secdef\@book\@sbook
}
\def\@book[#1]#2{%
    \ifnum \c@secnumdepth >-3\relax
        \refstepcounter{book}%
        \addcontentsline{toc}{book}{
            \bookname\ \thebook:\hspace{1em}#1
        }
    \else
        \addcontentsline{toc}{book}{#1}%
    \fi
    \markboth{}{}%
    {\centering
     \interlinepenalty \@M
     \normalfont
     \ifnum \c@secnumdepth >-2\relax
       \huge\bfseries \bookname\nobreakspace\thebook
       \par
       \vskip 20\p@
     \fi
     \Huge \bfseries #2\par}%
    \@endbook}
\def\@sbook#1{%
    {\centering
     \interlinepenalty \@M
     \normalfont
     \Huge \bfseries #1\par}%
    \@endbook}
\def\@endbook{
    \vfil\newpage
        \if@twoside
            \if@openright
                \null
                \thispagestyle{empty}%
                \newpage
            \fi
        \fi
        \if@tempswa
            \twocolumn
        \fi
}
\newcommand*\l@book[2]{%
    \ifnum \c@tocdepth >-2\relax
        \addpenalty{-\@highpenalty}%
        \addvspace{2.25em \@plus\p@}%
        \setlength\@tempdima{3em}%
        \begingroup
            \parindent \z@ \rightskip \@pnumwidth
            \parfillskip -\@pnumwidth
            {
                \leavevmode
                \Large \bfseries #1\hfil \hb@xt@\@pnumwidth{
                    \hss #2
                }
            }
            \par
            \nobreak
            \global\@nobreaktrue
            \everypar{\global\@nobreakfalse\everypar{}}%
        \endgroup
    \fi}
\newcommand\bookname{Book}
\renewcommand{\thebook}{\texorpdfstring{\Numberstring{book}}{book}}
\providecommand*{\toclevel@book}{-2}
\makeatother
\titlecontents{chapter}[0pt]
    {\bfseries}
    {\chaptername\ \thecontentslabel:\quad}
    {}
    {\hfill\contentspage}
\titleformat{\part}[display]
    {\Large\bfseries}
    {\partname\nobreakspace\thepart}
    {0mm}
    {\Huge\bfseries}
    \titlecontents{part}[0pt]
    {\large\bfseries}
    {\partname\ \thecontentslabel: \quad}
    {}
    {\hfill\contentspage}
\newcommand{\MarkRightAngle}[4][.3cm]
    {\coordinate (tempa) at ($(#3)!#1!(#2)$);
     \coordinate (tempb) at ($(#3)!#1!(#4)$);
     \coordinate (tempc) at ($(tempa)!0.5!(tempb)$);%midpoint
     \draw (tempa) -- ($(#3)!2!(tempc)$) -- (tempb);}
%--------------------------LENGTHS------------------------------%
% Spacings for the Table of Contents.
\addtolength{\cftsecnumwidth}{1ex}
\addtolength{\cftsubsecindent}{1ex}
\addtolength{\cftsubsecnumwidth}{1ex}
\addtolength{\cftfignumwidth}{1ex}
\addtolength{\cfttabnumwidth}{1ex}

% Spacing for multi-column and enumerate environments.
\setlength{\multicolsep}{6pt}
\setlist[enumerate]{itemsep=0pt,topsep=3pt}

% Indent and paragraph spacing.
\setlength{\parindent}{0em}
\setlength{\parskip}{0em}
%----------------------------GLOSSARY-------------------------------%
\makeglossaries
\loadglsentries{../../glossary}
\loadglsentries{../../acronym}
%--------------------------Main Document----------------------------%
\begin{document}
    \ifx\ifmain\undefined
        \pagenumbering{roman}
        \title{Combinatorics}
        \author{Ryan Maguire}
        \date{\vspace{-5ex}}
        \maketitle
        \tableofcontents
        \clearpage
        \chapter*{Combinatorics}
        \addcontentsline{toc}{chapter}{Combinatorics}
        \markboth{}{COMBINATORICS}
        \vspace{10ex}
        \setcounter{chapter}{1}
        \pagenumbering{arabic}
    \else
        \chapter{Combinatorics}
    \fi
    \section{Introduction}
        We use the following notation:
        \begin{equation}
            [n]=\mathbb{Z}_{n}=\{1,2,\hdots,n\}
        \end{equation}
        In combinatorics we study functions of the form
        $f:[n]\rightarrow[m]$. These functions can be
        one-to-one, onto, or both. A permutation of the
        elements of $[n]$ is simply a bijection
        $f:[n]\rightarrow[n]$. A partial permutation is a
        permutation of length $k\leq{n}$ of the set
        $[n]$. That is, a permutation on some subset of
        $\mathbb{Z}_{n}$. We all study sets. In particular,
        the power set of $[n]$ and subsets from $k$ chosen
        elements of $[n]$. Another topic of study is that of
        lattice paths on $\mathbb{Z}\times\mathbb{Z}$. That
        it, paths from $(i,j)$ to $(n,m)$ using a prescribed
        set of rules. For example, how many ways can you get
        from $(0,0)$ to $(21,7)$ if you're only allowed to move
        North and East. Restricted paths impose more rules,
        for example the number of moves east must be greater
        than the number of moves north. There are also things
        called Catalan paths and Motzkin paths.
        \par\hfill\par
        Another topic in combinatorics is that of words. An
        alphabet is a set $[n]$, and we wish to study the number
        of words of length $k$ in $[n]$. Binary words are words
        when the alphabet is $\{0,1\}$. There are also words
        with a prescribed number for each letter.
        There are also circular arrangements of the elemnts
        in $[n]$, and the idea of multi-sets. Multi-sets
        are sets that allow for repetition. From elementary
        set theory, we have:
        \begin{equation}
            \{a,b,c\}=\{a,a,b,c\}
        \end{equation}
        Sets are uniquely defined by the elements they contain.
        Multi-sets allow for repetition, and this distringuishes
        two different sets. We write:
        \begin{equation}
            A=\{\{1,1,1,2,3,3\}\}
        \end{equation}
        Note that $A\ne\{\{1,2,3\}\}$. The multiplicty of
        an elements $a\in{A}$ is the number of times the
        element $a$ occurs in the multi-set $A$. To know how
        many multi-sets chosen from $[n]$ with $k$ elements,
        we wish to study the following equation:
        \begin{equation}
            \sum_{i=1}^{n}m_{i}=k
        \end{equation}
        Where $m_{i}$ is the multiplicity of the $i^{th}$
        element of $[n]$. Where wish to find integer solutions
        to this equation, in particular solutions with
        non-negative integers. We can also place restrictions
        on the multi-sets, for example by requiring that
        each element occurs at least once. Thus we'd have:
        \begin{equation}
            \sum_{i=1}^{n}m_{i}=k
            \quad\quad
            m_{i}\geq{1}
        \end{equation}
        A partition of $[n]$ is a collection of sets
        $B_{1},b_{2},\dots,B_{k}$ such that:
        \begin{equation}
            \cup_{i=1}^{k}B_{i}=[n]
            \quad\quad
            B_{i}\cap{B}_{j}=\emptyset
            \quad{i}\ne{j}
        \end{equation}
        The $B_{i}$ are called \textrm{blocks}. We also
        study partitions of numbers. Given $n\geq{0}$, we
        want $\lambda=(\lambda_{1},\dots,\lambda_{\ell})$ such
        that:
        \begin{equation}
            \lambda_{i+1}\leq\lambda_{i},
            \quad\quad
            i=1,2,\dots,\ell-1
        \end{equation}
        And such that:
        \begin{equation}
            \sum_{i=1}{\ell}\lambda_{i}=n
        \end{equation}
        Another commonly studied object is a graph. Labelled
        tress, colorings of graphs, and spanning trees. Finally
        we study tableaux's. These are fillings of arrays
        of boxes with objects, such as numbers, sets, and
        multi-sets.
    \section{Counting Techniques}
        \subsection{Basic Numbers}
            \begin{equation}
                n^{k}=\big|\{f:[k]\rightarrow[n]\}\big|=
                \textrm{The number of words over the alphabet }
                [n]
            \end{equation}
            \begin{equation}
                \binom{n}{k}=
                \frac{n!}{k!(n-k)!}=
                \textrm{The number of ways to choose }k
                \textrm{ objects from }[n]
            \end{equation}
            Stirling's number of the second kind, denoted
            $S(k,n)$, is the number of ways to partition
            $[n]$ into $n$ blocks. $P(n)$ is the number of
            partitions of the integer $n$. And lastly,
            $n!$ is the number of bijections from $[n]$ into
            $[n]$. That is, $n!$ is the number of permutations
            on $[n]$.
        \subsection{Basic Counting Principles}
            Sum Rule (Divide and Conquer): If you cannot
            count the set, divide it into pieces and count
            the pieces.
            \begin{align}
                S&=\bigcup_{i=1}^{k}S_{i}
                \quad\quad
                S_{i}\cap{S}_{j}=\emptyset,
                \quad{i}\ne{j}\\
                |S|&=\sum_{i=1}^{k}|S_{i}|
            \end{align}
            Application: Classify or partition the elements
            in $S$ according to a set of mutually disjoint
            properties $p_{1},\dots,p_{k}$ and let:
            \begin{equation}
                S_{k}=\{x\subseteq{S}:p_{k}(x)\}
            \end{equation}
            We often use such a scheme to prove recurrences.
            For example Pascal's triangle:
            \begin{equation}
                \binom{n}{k}=
                \binom{n-1}{k-1}+\binom{n-1}{k}
            \end{equation}
            We can prove this by plugging in the formula and
            simplifying, but we wish to give a combinatorial
            proof. Let $S\subseteq[n]$, where
            $\Card(S)=k\leq{n}$. Define:
            \begin{align}
                S_{1}&=\{A\subseteq{S}:n\in{A}\}\\
                S_{2}=\{A\subseteq{S}:n\notin{A}\}
            \end{align}
            Then $S_{1}$ and $S_{2}$ partition $S$, and thus
            $\Card(S)=\Card(S_{1})+\Card(S_{2})$. But:
            \begin{align}
                \Card(S_{1})&=\binom{n-1}{k-1}\\
                \Card(S_{2})&=\binom{n-1}{k}
            \end{align}
            This completes the proof. Next is the difference
            rule (Count the opposite):
            \begin{align}
                S&\subseteq\mathcal{U}\\
                \overline{S}&=\mathcal{U}\setminus{S}\\
                \Card(S)&=\Card(\mathcal{U})
                    -\Card(\overline{S})
            \end{align}
            For example, how many permutations on $[n]$ are
            there so that 1 and 2 are not next to each other?
            We count 1 and 2 being together as 12 or 21.
            It's thus easier to think of this as one element.
            So we're counting the number of permutation on
            $[n-1]$ by consider $12$ as one element, and
            again by consider $21$ as one element. This
            gives $2(n-1)!$. By taking the difference, we
            have:
            \begin{equation}
                \Card(S)=n!-2(n-1)!=(n-1)!(n-2)
            \end{equation}
    \section{Posets}
        Let $n\in\mathbb{N}$, $X=2^{[n]}=\mathcal{P}(\mathbb{Z}_{n})$, and
        let consider $(X,\subseteq)$, where $\subseteq$ is the partial ordering
        of inclusion. If $\Card(A)=\Card(B)$, then either $A=B$, or
        $A$ and $B$ are not comparable, and hence $\subseteq$ is a partial
        order.
        \begin{theorem}
            For any $n\geq{1}$, $2^{[n]}$ has a SCD.
        \end{theorem}
        \begin{proof}
            Let $x\in{2}^{[n]}$. Then $x$ has a binary representation. We
            want to create 1-0 pairs as if they were parentheses. For example,
            suppse $x=\{1,5,7\}$. Then $x=0110001$. We match this up to
            $)(()))(\leftrightarrow0(())01$. To get the chain containing
            $x$, we need to describe how to go up and how to go down
            in the chain. To go up, take the right-most unpaired zero
            and change it to a one. To go down, take the left-most one
            and change it to a zero. For the example of $x=\{1,5,6\}$, we have:
            \begin{equation*}
                0110000\rightarrow0110001\rightarrow0110011\rightarrow1110011
            \end{equation*}
            The size of the smallest set in the chain the the number of
            parenthesizations. If this number is $i$, then the largest set
            has size $(n-2i)+i=n-i$. By construction, the chain is saturated.
            At every step we only add one element. Thus, the constructions
            produce symmetric chains. Notice that we never produce new
            1-0 pairings in the algorithm. Thus, all of the sets in the chain
            have the same pairings. So two sets $X$ and $Y$ produce either
            the same chain or disjoint chains. For example, consider
            $2^{[4]}$. The chain is:
            \begin{align*}
                0000\rightarrow0001\rightarrow
                0011\rightarrow0111\rightarrow1111\\
                0010\rightarrow0110\rightarrow1110\\
                0100\rightarrow0101\rightarrow1101\\
                1000\rightarrow1001\rightarrow1011
            \end{align*}
        \end{proof}
        \begin{ltheorem}{Sperner's Theorem}
            Any anti-chain of $2^{[n]}$ elements has at most
            $\binom{n}{\floor{n/2}}$ subsets.
        \end{ltheorem}
        \begin{proof}
            Notice that in the diagram of $2^{[n]}$ each level contains
            $\binom{n}{k}$ elements. An SCD partitions $2^{[n]}$. Let
            $G$ be an $SCD$. with $m$ chains. Then the maximum number of
            incomparable elements is $m$. therwise, two imcomparable elements
            are in the same chain. Thus, $m\geq\binom{n}{\floor{n/2}}$. Also,
            each chain will intersect a level in the graph of the poset at most
            once. Therefore, $m\leq\binom{n}{\floor{n/2}}$. Thus, etc.
        \end{proof}
        The existence of a symmetric chain decomposition gives an elegant
        combinatorial proof that the sequence
        $\binom{n}{k}$, $k=0,1,\dots,n$, is unimodal. To prove unimodality, it
        would suffice to show that, for $k\leq{n/2}$, there exists an
        injection from $k$ subsets to $k+1$ subsets. If we have a SCD, map
        a $k$ subset to it's successor in the chain. This gives the injection.
    \section{Binomial Coefficients and Multi-Sets}
        Recall that a multi-set is similar to a set, except that repetitions
        are allowed. For example, if we consider $[3]$, then a multi-set of
        of this could be:
        \begin{equation}
            \{\{1,1,2,2,3\}\}
        \end{equation}
        This has 5 elements, and is a mutli-set of size 5.
        \begin{theorem}
            The number of $k$ multisets of an $n$ element set is:
            \begin{equation}
                \frac{n^{\overline{k}}}{k!}
                =\frac{n(n+1)\cdots(n+k-1)}{k!}
                =\binom{n+k-1}{k}
            \end{equation}
        \end{theorem}
        \begin{proof}
            For let $S$ be the set of multi-sets of size $k$ of elements
            of an $n$ element set, and let $T$ be subsets of size $k$ in
            $[n+k-1]$. We need to produce a map $f:S\rightarrow[T]$. Let:
            \begin{equation}
                M=\{\{1\leq{a}_{1}\leq{a}_{2}\leq\cdots\leq{a}_{k}\leq{n}\}\}
            \end{equation}
            This maps to the set:
            \begin{equation}
                A=\{1\leq{a}_{1}<a_{2}+1<a_{3}+2<\cdots<a_{k}+k-1\leq{n}+k-1\}
            \end{equation}
            This mapping is reversible. Therefore, etc.
        \end{proof}
        \begin{example}
            Let $n=3$, and $k=5$. Also, define:
            \begin{equation}
                M=\{\{1,1,2,2,3\}\}
            \end{equation}
            Then:
            \begin{equation}
                A=\{1,2,4,5,7\}\subseteq[7]
            \end{equation}
        \end{example}
        Multi-sets can be seen as binary sequences (Stars and bars). For
        example, let $M=\{3,3,4,7,7\}$. We can write this as
        $||**|*|||**$. This helps count out the repetitions of various elements
        in the multi-set. $\binom{n}{k}$ can be seen as the number of
        functions that map $k$ elements to $0$ and $n-k$ elements to $1$.
        We can generalize to functions $[n]\rightarrow[m]$. Let
        $k_{1},k_{2},\dots,x_{m}$ be such that:
        \begin{equation}
            \sum_{i=1}^{m}k_{i}=n
        \end{equation}
        And such that, for all $i$, $k_{i}\geq{0}$. Then
        $\binom{n}{k_{1},\dots,k_{m}}$ is the number of ways to map
        $[n]\rightarrow[m]$ such that $k_{i}$ elements map to $i$, where:
        \begin{align}
            \binom{n}{k_{1},k_{2},\dots,k_{m}}&=
            \binom{n}{k_{1}}\binom{n-k_{1}}{k_{2}}
            \binom{n-k_{1}-k_{2}}{k_{3}}\cdots\binom{k_{m}}{k_{m}}\\
            &=\frac{n!}{k_{1}!k_{2}!\cdots{k}_{m}!}
        \end{align}
        \begin{ltheorem}{Multinomial Theorem}
            \begin{equation}
                (x_{1}+x_{2}+\cdots+x_{m})^{n}=
                \sum_{k_{1}+\cdots+k_{m}=n}
                \binom{n}{k_{1},\dots,k_{m}}x_{1}^{k_{1}}\cdots{x}_{m}^{k_{m}}
            \end{equation}
        \end{ltheorem}
        \subsection{Lattice Paths}
            Let $\mathbb{Z}^{d}$ be an integer lattice of dimension $d$, where
            $d\in\mathbb{N}$ and $d\geq{1}$.
            \begin{ldefinition}{Lattice Path}
                A lattice path in $\mathbb{Z}^{d}$ with $k$ steps in
                $S\subseteq\mathbb{Z}^{d}$ is a subset
                $L\subseteq\mathbb{Z}^{d}$ such that $L=\{v_{1},\dots,v_{k}\}$
                such that, for all $i=1,2,\dots,k-1$, $v_{i+1}-v_{i}\in{S}$.
            \end{ldefinition}
            \begin{example}
                If $d=2$, $S=\{(0,1),(1,0)\}$, then there are 6 paths
                from $(0,0)$ to $(2,2)$.
            \end{example}
            \begin{theorem}
                if $v=(a_{1},\cdots,a_{d})\in\mathbb{Z}^{d}$ and if $e_{i}$ is
                the $i^{th}$ unit vector in $\mathbb{Z}^{d}$, then the number
                of lattice paths in $\mathbb{Z}^{d}$ from the origin to
                $v$ with steps in $\{e_{i}:i\in\mathbb{d}\}$ is given by
                the multinomial coefficient
                $\binom{\norm{v}_{1}}{a_{1},\dots,a_{d}}$.
            \end{theorem}
            \begin{proof}
                For let $v_{0},\cdots,v_{k}$ be a path. Then
                $v_{1}-v_{0},v_{2}-v_{1},\dots,v_{k}-v_{k-1}$ consist of
                the $e_{i}$. Thus there are $a_{1}$ $e_{1}$'s,
                $a_{2}$ $e_{2}$'s, and so on. The total number is thus the
                multinomial coefficient.
            \end{proof}
            \begin{theorem}
                The number of lattice paths from $(0,0)$ to $(n,m)$ with
                steps in $\{(0,1),(1,0)\}$ is $\binom{n+m}{n}$.
            \end{theorem}
        \subsection{The Involution Principle}
            \begin{theorem}
                The number of lattice paths from $(i,j)$ to $(m,n)$ using
                steps $(1,0)$ and $(0,1)$ is $\binom{m-i+n-j}{m-i}$.
            \end{theorem}
            Given a set $S$ and a partition
            $S=S^{+}\cup{S}^{-}$ into a negative part $S^{-}$ and a
            positive part $S^{+}$, then $S$ is called a signed set. We
            are interested in computing $\Card(S^{+})-\Card(S^{-})$.
            \begin{ldefinition}{Sign Reversing Involution}
                A sign reversing involution is an involution
                $\psi:S\rightarrow{S}$ such that for all $x\in{S}$
                such that $\psi(x)\ne{x}$, then
                $\psi(x)\in{S}^{+}$ for all $x\in{S}^{-}$ and
                $\psi(x)\in{S}^{-}$ for all $x\in{S}^{+}$.
            \end{ldefinition}
            \begin{theorem}
                If $\psi$ is a sign reversing involution, if
                $F^{+}$ are the fixed points of $\psi$ in $S^{+}$,
                and $F^{-}$ are the fixed points are $\psi$ in
                $S^{-}$, then:
                \begin{align}
                    \Card(S^{+}\setminus{F}^{+})&=
                    \Card(S^{-}\setminus{F}^{-})\\
                    \Card(S^{+})-\Card(S^{-})&=
                    \Card(F^{+})-\Card(F^{-})
                \end{align}
            \end{theorem}
            Suppose we are given a set $X$ and we want to compute
            $\Card(X)$. Embed $X$ into a signed set $S=S^{+}\cup{S}^{-}$
            such that for all $x\in{X}$ there is a corresponding
            $s\in{S^{+}}$.
            \begin{ldefinition}{Catalan Path}
                A Catalan path is a lattice path from $(0,0)$ to
                $(n,n)$ using steps $(0,1)$ and $(1,0)$ such that the
                path never crosses the line $x=y$.
            \end{ldefinition}
            We are interested in counting the number of Catalan paths
            for a given $n\in\mathbb{N}$. The first few numbers are
            $1,2,5,14,42,\dots$ and occur frequently in mathematics.
            Let $S^{+}$ be the set of paths from $(1,0)$ to $(n+1,n)$
            and $S^{-}$ be the set of paths from $(0,1)$ to $(n+1,n)$.
            Using the previous theorem:
            \begin{align}
                \Card(S^{+})&=\binom{2n}{n}\\
                \Card(S^{-})&=\binom{2n}{n-1}
            \end{align}
            Now we need to embed the Catalan paths into $S^{+}$.
            The embedding comes from shifting the graphs 1 unit to the
            right. Note that the image never touches the line $x=y$.
            Define a sign reversing involution $\psi:S\rightarrow{S}$
            by letting $P$ be any path in $S$ that does not touch
            $x=y$, and defining $\psi(P)=P$. If $P$ touches or crosses
            $x=y$, let $p_{0}$ be the first such crossing. Let
            $\psi(P)$ be the path from $(1,0)$
            (Respectively, from $(0,1)$), such that the points from
            $(0,0)$ to $p_{0}$ are reflected, and the points from
            $p$ to $(n+1,n)$ stay the same. Now $F^{-}$ is empty, since
            given any path from $(1,0)$ to $(n+1,n)$, it must cross
            the line $y=x$. Thus there are no fixed points in $S^{-}$.
            But then:
            \begin{equation}
                \Card(F^{+})=\Card(S^{+})-\Card(S^{-})
            \end{equation}
            But the Catalan number $C_{n}$ is equal to the size of
            $F^{+}$, and thus we have:
            \begin{equation}
                C_{n}=\binom{2n}{n}-\binom{2n}{n-1}
                =\frac{1}{n+1}\binom{2n}{n}
            \end{equation}
        \subsection{Diagonal Lattice Paths}
            \begin{ldefinition}{Diagonal Lattice Paths}
                A diagonal lattice path is a lattice path with steps
                $(1,1)$ and $(1,-1)$.
            \end{ldefinition}
            \begin{lexample}
                Consider all diagonal paths from $(0,0)$ to $(4,0)$.
                Since any step increasing the $x$ coordinate by 1,
                there must be 4 steps in the lattice path. But since
                the path must end at 0, there must be an equal number
                of steps that go up as there are steps that go down.
                So, we must have two up steps and two down steps.
                The total number of diagonal lattice paths is thus
                $\binom{4}{2}=6$. In general, the total number of
                lattice paths from $(0,0)$ to $(2n,0)$ is
                $\binom{2n}{n}$.
            \end{lexample}
            These diagonal lattice paths can be seen as binary words with
            $d=(1,-1)$ and $(u=1,1)$ such that the number of occurences
            of $d$ is equal to the number of occurences of $u$. We can
            establish a correspondence between diagonal lattice paths and
            Catalan paths by considering as the bijection a reflection
            about the $x=y$ axis, and then a rotation by $45^{\circ}$.
            \begin{ldefinition}{Dyck Paths}
                A Dyck is a diagonal lattice path that never goes below
                it's starting point.
            \end{ldefinition}
    \section{q-Analogues}
        In combinatorics, a $q$ analogue of a counting function, such
        as $n!$, is typically a polynomial in $q$ which evaluates to
        the function if we set $q=1$, and if not a polynomial we take
        the limit as $q\rightarrow{1}$. We want the q-analogue to
        preserve the same reccurence properties that the counting
        function has.
        \begin{lexample}
            A q-Analogue of a real number $x\in\mathbb{R}$ could be:
            \begin{equation}
                [x]_{q}=\frac{1-q^{x}}{1-q}
            \end{equation}
            Taking the limit as $q\rightarrow{1}$, we see that this
            expression evaluates to $x$ by using L'H\^{o}pital's Rule.
            If $x=n\in\mathbb{N}$, then:
            \begin{equation}
                \frac{1-q^{n}}{1-q}=1+q+\cdots+q^{n-1}
            \end{equation}
            This allows us to construct a q-Analogue of $n!$:
            \begin{equation}
                [n]_{q}!=[1]_{q}[2]_{q}\cdots[n]_{q}
            \end{equation}
            This can be used to put statistics on sets.
        \end{lexample}
        \begin{ldefinition}{Statistic on a Finite Set}
            A statistic on a finite set $S$ is a function
            $f:S\rightarrow\mathbb{N}_{0}$
        \end{ldefinition}
        Let $S_{n}$ denote the symmetric group, which is the set of
        permutations of $1,2,\dots,n$ under the operation of composition.
        Then:
        \begin{equation}
            \Card(S_{n})=n!
        \end{equation}
        \begin{ldefinition}{Inversion of a Word}
            An inversion of a word $\sigma$ is a pair $(i,j)$, where
            $1\leq{i}<j\leq{n}$, where $\sigma_{i}>\sigma_{j}$.
        \end{ldefinition}
        \begin{lexample}
            Let $\sigma=(132)(45)(6)(7)$. Then $(1,3)$ is an inversion,
            since $\sigma_{1}=3>\sigma_{3}=2$. 
        \end{lexample}
        \begin{ldefinition}{Inversion Statistic}
            The inversion statistic on $S_{n}$ is the number of
            inversions of $\sigma\in{S}_{n}$.
        \end{ldefinition}
        \begin{theorem}
            If $S_{n}$ is the permutation group, then:
            \begin{equation}
                \sum_{\sigma\in{S}_{n}}q^{\textrm{inv}(\sigma)}
                =[n]_{q}!
            \end{equation}
        \end{theorem}
    \section{Lecture 6}
        Last week we introduced q-Analogs, and proved the following
        identities:
        \begin{equation}
            \sum_{\sigma\in{S}_{n}}q^{inv(\sigma)}=
            \sum_{\sigma\in{S}_{n}}q^{maj(\sigma)}
            =[n]_{q}!]
        \end{equation}
        Where $inv(\sigma)$ is the number of inversions, and
        $maj(\sigma)$ is the number of descents. We now want a
        q-Analog of $\binom{n}{k}$. Recall the inversion tables:
        \begin{equation}
            \mathcal{I}_{n}=
                \{(a_{1},\dots,a_{n}):0\leq{a}_{i}\leq{i}\}
        \end{equation}
        We can write this as:
        \begin{equation}
            \mathcal{I}_{n}=
                \{0\}\times\{0,1\}\times\{0,1,2\}\times\cdots
                \times\{0,1,2,\dots,n-1\}
        \end{equation}
        From this we obtain:
        \begin{equation}
            \Card(\mathcal{I}_{n})=n!
        \end{equation}
        We define the function
        $\Psi_{1}:\mathcal{I}_{n}\rightarrow{S}_{n}$ by mapping:
        \begin{equation}
            \Psi_{1}(a_{1},\dots,a_{n})=\sigma
        \end{equation}
        Where $\sigma$ is the permutation with inversions
        $a_{1},\dots,a_{n}$, and $a_{i}$ is the number inversions
        created by $i$. We also define
        $\Psi_{2}:\mathcal{I}_{n}\rightarrow{S}_{n}$ and
        re-interpreted $a_{i}$ to be the contribution of $i$ to the
        major index. Then $\Psi=\Psi_{2}\circ\Psi_{1}^{\minus{1}}$
        is a bijection from $S_{n}$ to itself. We now want to find
        a good q-Analog for $\binom{n}{k}$ that would satisfy
        similar properties as the binomial coefficient. One nice
        property is Pascal's Identity:
        \begin{equation}
            \binom{n}{k}=\binom{n-1}{k-1}+\binom{n-1}{k}
        \end{equation}
        Perhaps the obvious choice is to choose:
        \begin{equation}
            \binom{n}{k}_{q}=
            \frac{[n]_{q}!}{[k]_{q}![n-k]_{q}!}
        \end{equation}
        These are called the Gaussian polynomials, and it seems
        surprising that these are polynomials in the first place,
        since it appears to be a rational function. However, we
        can see just by plugging in that:
        \begin{equation}
            \binom{n+1}{k}_{q}\ne
            \binom{n}{k}_{q}+\binom{n}{k-1}_{q}
        \end{equation}
        And thus this is not a good q-Analog for the binomial
        coefficients. Let:
        \begin{equation}
            \mathcal{R}(1^{k}0^{n-k})=\{
            \textrm{Set of binary words of length $n$ with $k$ 1's}
            \}
        \end{equation}
        Then:
        \begin{equation}
            \Card\Big(\mathcal{R}(1^{k}0^{n-k})\Big)=
            \binom{n}{k}
        \end{equation}
        This is equivalent to saying:
        \begin{equation}
            k!(n-k)!\Card\Big(\mathcal{R}(1^{k}0^{n-k})\Big)
            =n!
        \end{equation}
        But the left-hand side of this equation s the cardinality
        of $S_{k}\times{S}_{n-k}\times\mathcal{R}(1^{k}0^{n-k})$,
        and the right-hand side is the cardinality of
        $S_{n}$. We need to define a function:
        \begin{equation}
            f:S_{k}\times{S}_{n-k}\times\mathcal{R}(1^{k}0^{n-k})
        \end{equation}
        Add $n-k$ to the numbers in the permutation $S_{k}$.
        For example consider:
        \begin{equation}
            (132,14523,10011000)\mapsto
            (687,14523,10011000)
        \end{equation}
        Send the left-most number in order from left to right
        to the 1's in the binary word (The third entry). Send the
        second entry to the 0's in the binary word.
        So, finally we have:
        \begin{equation}
            (132,14523,10011000)\mapsto
            (61487523)
        \end{equation}
        We now want to show that:
        \begin{equation}
            \sum_{r\in\mathcal{R}(1^{k}0^{n-k})}q^{inv(r)}
            =\binom{n}{k}_{q}
            =\frac{[n]_{q}!}{[k]_{q}![n-k]_{q}!}
        \end{equation}
        We can do this in a similar manner as before. We need a
        function $f$ from
        $S_{n-k}\times{S}_{k}\times\mathcal{R}(1^{k}0^{n-k})$ that
        is bijective. Let's use the one defined previously. We
        now need to show that $f$ preserves inversions.
        \begin{theorem}
            \begin{equation}
                \binom{n+1}{k}_{q}=
                q^{k}\binom{n}{k}_{q}+
                \binom{n}{k-1}_{q}
            \end{equation}
        \end{theorem}
        \begin{ltheorem}{Foata's Theorem}
            \begin{equation}
                \sum_{r\in\mathcal{R}(1^{k}0^{n-k})}q^{maj(r)}
                =\binom{n}{k}_{q}
            \end{equation}
        \end{ltheorem}
        \subsection{Lattice Paths and Gaussian Polynomials}
            \begin{ldefinition}{Partitions of Integers}
                A partition of $\mathbb{Z}_{n}$, $n\in\mathbb{N}$,
                is a weakly decreasing sequence
                $\lambda=(\lambda_{1},\dots,\lambda_{\ell})$,
                such that:
                \begin{equation}
                    |\lambda|=\sum_{k=1}^{\ell}\lambda_{k}=n
                \end{equation}
                $|\lambda|$ is called the weight of $\lambda$ and
                $\lambda_{i}$ are called the parts of $\lambda$.
                The length of $\lambda$ is the number of non-zero
                parts.
            \end{ldefinition}
            A young diagram is a graphical representation of a
            partition $\lambda=(\lambda_{1},\dots,\lambda_{\ell})$.
            The conjugate of a partition $\lambda$ is obtained by
            transposing the Young diagram of $\lambda$. For example:
            \begin{equation}
                (3,3,1)\mapsto(3,2,2)
            \end{equation}
            We denote the conjugate by $\lambda'$. Recall that
            $\binom{n+m}{n}$ is the number of lattice paths from
            $(0,0)$ to $(n,m)$ using steps $(1,0)$ and $(0,1)$.
            \begin{theorem}
                There exists a bijection between the set of lattice
                paths from $(0,0)$ to $(m,n)$ and the set of
                partitions of such that $\lambda_{1}\leq{m}$ and
                $\ell(\lambda)\leq{n}$.
            \end{theorem}
            If $p(m,n)$ is the number of partitions that fit in
            the $m\times{n}$ rectangle, that is
            $\ell(\lambda)\leq{N}$ and $\lambda_{1}\leq{m}$, then
            $p(m,n)=\binom{n+m}{n}$. A statistic on partitions is
            given by the weight of $\lambda$, $|\lambda|$.
            \begin{theorem}
                For $m,n\in\mathbb{N}$:
                \begin{equation}
                    \binom{n}{m}_{q}=
                    \sum_{\lambda\subseteq[m^{n}]}q^{|\lambda|}
                \end{equation}
            \end{theorem}
            \begin{proof}
                We can show this by proving that the sum satisfies
                the same recurrence relation and initial conditions
                as the q binomial.
            \end{proof}
    \section{Lecture 7 (I Think)}
        We're currently discussing q-analogues. We want to extend
        the q-analogue defined for the factorial function to the
        binomial coefficient. The Gaussian polynomials are one
        such attempt at this:
        \begin{equation}
            \binom{n}{k}_{q}=
            \frac{[n]_{q}!}{[k]_{q}![n-k]_{q}!}
        \end{equation}
        Another such attempt was to sum over all binary words of
        length $n$ with $k$ one's, and obtain:
        \begin{equation}
            \binom{n}{k}_{q}=
            \sum_{r\in\mathcal{R}(1^{k},0^{n-k})}
                q^{inv(r)}
        \end{equation}
        Using this definition, we obtained the following equation:
        \begin{equation}
            \binom{n+1}{k}_{q}=
            q^{k}\binom{n}{k}_{q}+
            \binom{n}{k-1}_{q}
        \end{equation}
        Then we discussed lattice paths $L(m,n)$, which are paths
        from $(0,0)$ to $(m,n)$ using steps in $(1,0)$ and $(0,1)$.
        Next we discuessed partitions of numbers.
        \begin{table}[H]
            \centering
            \captionsetup{type=table}
            \begin{tabular}{|c|c|}
                \hline
                $n$&Partitions\\
                \hline
                0&$\emptyset$\\
                \hline
                1&$(1)$\\
                \hline
                2&$(2),(1,1)$\\
                \hline
                3&$(3),(2,1),)1,1,1)$\\
                \hline
            \end{tabular}
            \caption{Caption}
            \label{tab:my_label}
        \end{table}
        Given such a partition, we assign the weight to be:
        \begin{equation}
            |\lambda|=\sum_{k=1}^{\ell}\lambda_{k}
        \end{equation}
        Then we define:
        \begin{equation}
            P(m,n)=
            \{(\lambda_{1},\dots,\lambda_{\ell}:
                \lambda_{1}\leq{m},\ell(\lambda)\leq{n}\}
        \end{equation}
        We showed that there is a bijection between
        $L(m,n)$ and $P(m,n)$. Thus, we have:
        \begin{equation}
            \Card\Big(P(n,m)\Big)=\binom{m+n}{m}
        \end{equation}
        \begin{theorem}
            If $m,n\in\mathbb{N}$, then:
            \begin{equation}
                \binom{m+n}{m}_{q}=
                \sum_{\lambda\in{P}(m,n)}q^{|\lambda|}
            \end{equation}
        \end{theorem}
        \begin{proof}
            The strategy of the proof is to show that this sum
            satisfies the same initial conditions and the same
            recurrence as the original definition. We have that
            $p(m,0)=1=q^{0}$ since there is only the empty
            partition in the rectangle $m\times{0}$, and similarly
            $p(0,n)=1=q^{0}$ since there is only the empty
            partition in the rectangle $0\times{n}$. Moreover,
            $p(m,m)=1=q^{0}$ since:
            \begin{equation}
                \binom{m}{m}_{q}=
                \frac{[m]_{q}}{[0]_{q}[m]_{q}}=
                \frac{[m]_{q}}{[m]_{q}}=1
            \end{equation}
            We now must show that the recurrence relation is
            satisfied. We want:
            \begin{equation}
                p(m,n)=q^{m}p(n-1,m)+p(n,m-1)
            \end{equation}
            We have that:
            \begin{align}
                p(m,n)=&
                \sum_{\lambda\in{P}(m,n)}q^{|\lambda|}\\
                &=\sum_{\lambda_{1}=m}q^{|\lambda|}+
                \sum_{\lambda_{1}<m}q^{|\lambda|}\\
                &=q^{m}\sum_{\lambda\in{P}(m,n-1)}q^{|\lambda|}
                +\sum_{\lambda\in{P}(m-1,n)}q^{|\lambda|}
            \end{align}
            This completes the proof.
        \end{proof}
        \begin{ldefinition}{$x$ Factorization}
            Let $w\in{X}^{*}$ be a word in the alphabet $X$.
            Let $x\in{X}$ and suppose $w=vy$, where $v$ is a word
            and $y$ is a letter in $X$. That is, $y$ is the last
            letter of $w$. Then the factorization is:
            \begin{equation}
                w=v_{1}y_{1}\dots{v}_{k}y_{k}
            \end{equation}
            Where, if $y>x$ ($X$ is totally ordered):
            \begin{equation}
                y_{i}>x,
                \quad\quad
                y_{i}\in{X}
            \end{equation}
            \begin{equation}
                v_{i}\in{L}_{x}^*
            \end{equation}
            Where:
            \begin{equation}
                L_{x}=\{a:a\leq{x}\}
            \end{equation}
            If $y\leq{x}$, then:
            \begin{equation}
                y_{i}\leq{x}\quad\quad
                y_{i}\in{X}
            \end{equation}
            and:
            \begin{equation}
                v_{i}\in{R}_{x}^{*}
            \end{equation}
            Where:
            \begin{equation}
                R_{x}=\{a:a>x\}
            \end{equation}
        \end{ldefinition}
        \begin{lexample}
            Let $x=3$ and let:
            \begin{equation}
                w=125312641237
            \end{equation}
            Then $y=7$, and thus $y>x$. Then we can write:
            \begin{equation}
                w=|12|5|312|6||4|123|7
            \end{equation}
            We allow for empty words. As another example,
            consider:
            \begin{equation}
                w=135712136412
            \end{equation}
            Then $y=2$, and thus $y<2$. We obtain:
            \begin{equation}
                w=1|3|57|2|1|3|641|2
            \end{equation}
        \end{lexample}
        \begin{ltheorem}{Foata's Theorem}
            The following is true:
            \begin{equation}
                \sum_{r\in\mathcal{R}(1^{k},0^{n-k})}q^{maj(r)}
                =\binom{n}{k}_{q}
            \end{equation}
        \end{ltheorem}
        \begin{proof}
            We want to define a bijection
            $\varphi$ from $\mathcal{R}(1^{k},0^{n-k})$ to itself
            such that, for any $r$, we have $maj(r)=inv(\varphi(r))$.
            Let $X\subseteq\mathbb{N}$. Let $X^{*}$ be the set
            of all words over $X$. For example if $X=\{0,1\}$, then
            $X^{*}$ is the set of all binary words. Define
            $\varphi:X^{*}\rightarrow{X}^{*}$ be such that
            $maj(w)=inv(\varphi(w))$ for any $w\in{X}^{*}$.
            Note that inversions and descents are defined in the
            same way as for permutations. This is why we required
            the set to be totally ordered, $\mathbb{N}$ in our
            case. Define $\gamma_{x}:X^{*}\rightarrow{X}^{*}$ by:
            \begin{equation}
                \gamma_{x}(w)=
                \begin{cases}
                    \emptyset,&w=\emptyset\\
                    y_{1}v_{1}\dots{y}_{k}v_{k},&
                    w=v_{1}y_{1}\dots{v}_{k}y_{k}
                \end{cases}
            \end{equation}
            We define $\varphi$ as follows:
            \begin{equation}
                \varphi(w)=
                \begin{cases}
                    \emptyset,&w=\emptyset\\
                    w,&w\in{X}\\
                    \gamma_{x}(\varphi(v)),&
                    w=vx,x\in{X}
                \end{cases}
            \end{equation}
            This is a recursive definition. For example, let:
            \begin{equation}
                w=121314
            \end{equation}
            Then $\varphi(1)=1$, and thus
            $\varphi(12)=\gamma_{2}(\phi(1))2=12$. The first
            interesting case is with three elements. We have:
            \begin{equation}
                \varphi(121)=
                \gamma_{1}(\varphi(12))1=
                \gamma_{1}(12)1=211
            \end{equation}
            Note that $inv(211)=2$ and $maj(121)=2$. This function
            works since, if $w\in{X}^{*}$, then let
            $r_{x}$ be the number of letters in $w$ that are
            greater than $x$, and let $\ell_{x}$ be the number
            of letters in $w$ that are less than or equal to $x$.
            Let $v\in{X}^{*}$ and $x\in{X}$. Then:
            \begin{equation}
                inv(vx)=inv(v)+r_{x}(v)
            \end{equation}
            Also, when the last letter of $v$ is less than or
            equal to $x$, we have:
            \begin{equation}
                inv(\gamma_{x}(v))=inv(v)-r_{x}(v)
            \end{equation}
            And otherwise we have:
            \begin{equation}
                inv(\gamma_{x}(v))=inv(v)+\ell_{x}(v)
            \end{equation}
            Moreover, if the last letter $v$ is less than or
            equal to $x$, then:
            \begin{equation}
                maj(vx)=maj(v)
            \end{equation}
            And otherwise:
            \begin{equation}
                maj(vx)+|v|
            \end{equation}
        \end{proof}
    \section{Lecture 9}
        As a summary, we we studying q-Analogues. We have shown:
        \begin{equation}
            \sum_{\sigma\in{S}_{n}}q^{inv(\sigma)}=
            \sum_{\sigma\in{S}_{n}}q^{maj(\sigma)}=
            [n]_{q}!
        \end{equation}
        Also:
        \begin{equation}
            \sum_{w\in\mathcal{R}(1^{k},0^{n-k})}
            q^{inv(w)}=
            \sum_{w\in\mathcal{R}(1^{k},0^{n-k})}
            q^{maj(w)}=
            \binom{n}{k}_{q}
        \end{equation}
        \begin{ltheorem}{q-Binomial Theorem}
            The following is true:
            \begin{equation}
                \prod_{i=1}^{n}
                (x+q^{i}y)=
                \sum{q}^{\binom{n-k+1}{2}}
                \binom{n}{k}_{q}x^{k}y^{n-k}
            \end{equation}
        \end{ltheorem}
        For all $n,k\in\mathbb{N}$, we have:
        \begin{equation}
            \binom{n+k}{k}_{q}=
            \sum_{\lambda\in{P}(n,k)}q^{|\lambda|}
        \end{equation}
        We can use rising factorials to define
        $\binom{x}{k}_{q}$ for all $x\in\mathbb{R}$ and
        $k\in\mathbb{N}$. That is:
        \begin{equation}
            \binom{x}{k}_{q}=
            \frac{(1-q^{x-k+1})(1-q^{x-k+1})}{(1-q)(1-q^{2})\dots}
        \end{equation}
        \subsection{q-Catalan Analogue}
            Recall that $C_{n}$ is the number of lattice paths
            from $(0,0)$ to $(n,n)$ that do not go below or
            above the line $x=y$ in the plane. We showed earlier
            that:
            \begin{equation}
                C_{n}=\frac{1}{n+1}\binom{2n}{n}
            \end{equation}
            Recall that $L(m,n)$ is the number of lattice paths
            from $(0,0)$ to $(m,n)$. We define $L^{+}(m,n)$ to be
            the set of lattice paths from $(0,0)$ to $(m,n)$ that
            do not go below the line $y=\frac{n}{m}x$. In
            particular, $L^{+}(n,n)=C_{n}$. The Catalan numbers
            satisfy the following recurrence:
            \begin{equation}
                C_{n}=\sum_{k=1}^{n}C_{k-1}C_{n-k}
            \end{equation}
            Recall that
            $\omega:L(m,n)\rightarrow\mathcal{R}(0^{n},1^{m}$,
            where $\omega$ maps $N\rightarrow{0}$ and
            $E$ to 1, north and east. Let
            $\mathcal{R}^{+}(0^{n}1^{n})$ be the elements f
            $\mathcal{R}(1^{n}0^{n})$ that correspond to
            Catalan paths. From a previous observation, the words
            corresponding to the Catalan paths are characterized
            by alwas have more 0's than 1's for any initial word.
            \begin{ltheorem}{MacMahor's Theorem}
                The following is true:
                \begin{equation}
                    \sum_{p\in{L}^{+}(n,n)}q^{maj(w(p))}=
                    \frac{1}{[n+1]_{q}}\binom{2n}{n}_{q}
                \end{equation}
            \end{ltheorem}
            \begin{proof}
                Define the following:
                \begin{align}
                    \mathcal{R}^{\minus}(0^{n}1^{n})
                        &=\mathcal{R}(0^{n}1^{n})
                        -\mathcal{R}^{+}(0^{n}1^{n})\\
                    L^{\minus}(n,n)=L(n,n)-L^{+}(n,n)
                \end{align}
                Given a path $P$ in $L^{\minus}(n,n)$, let $A$ be
                the lattice point with the smallest $x$ coordinate
                among all the lattice points $(i,j)$ with
                $i-j$ maximized, whose distance from the
                $x=y$ line in the south east direction is maximized.
                Let $B$ be the lattice point just before $A$.
                Notice that the step $B\rightarrow{A}$ must be
                an east step. Create a new path as follows. Change
                the east step to a north step, and then take the
                remaining path from $A$ to $(n,n)$ and shift it
                up one and two the left one. This path ends on
                $(n-1,n+1)$. Let $\varphi$ denote the new path.
                Then:
                \begin{equation}
                    maj(w(\varphi(p)))=maj(w(p))-1
                \end{equation}
                For suppose $B\ne(0,0)$. The the step that goes to
                $B$ must be an east step. For if not, then $A$ does
                not have the smallest $x$ coordinate with maximal
                distance to the line $y=x$. If $B=(0,0)$, then the
                first position goes away. Moreover, the algorithm
                is reversible. For let $P'$ be a lattice path from
                $(0,0)$ to $(n,m)$, and let $A'$ be the point with
                maximal $x$ coordinate such that $i-j$ is maximized.
                This point corresponds to the $B$ in the previous
                path. Therefore:
                \begin{equation}
                    \sum_{w\in\mathcal{R}(1^{n}0^{n})}q^{maj(w)}=
                    \sum_{w\in\mathcal{R}(1^{n+1}0^{n-1})}
                        q^{maj(w)+1}=
                    q\binom{2n}{n+1}
                \end{equation}
            \end{proof}
            There is another q-analogue for $C_{n}$ due to
            Carlitz and Riordan. Let $p\in{L}^{+}(n,n)$ and
            define $a_{i}(p)$ to be the number of complete
            squares between the path and the $x=y$ line in row
            $i$. The number $a_{i}(p)$ is called the length of
            the $i^{th}$ row of $p$ and the sequence
            $(a_{1}(p),\dots,a_{n}(p))$ is called the
            co-area vector of $p$. The co-area statistic on
            $p$ is defined as:
            \begin{equation}
                Coarea(p)=\sum_{i=1}^{n}a_{i}(p)
            \end{equation}
            \begin{ltheorem}{Carlitz-Riordan Theorem}
                The following is true:
                \begin{equation}
                    C_{n}(q)=\sum_{p\in{L}^{+}(n,n)}q^{Coarea(p)}
                \end{equation}
            \end{ltheorem}
        Using the Carlitz-Riordan theorem, we can show the
        following result.
        \begin{theorem}
            The following is true:
            \begin{equation}
                C_{n}(q)=\sum_{k=1}^{n}q^{k-1}
                    C_{k-1}(q)C_{n-k}(q)
            \end{equation}
        \end{theorem}
        If we set $x\mapsto{q}^{i}x$, and $y\mapsto{1}$ in the
        q-binomial theorem, then we obtain:
        \begin{equation}
            (\minus{x};q)=\prod_{k=0}^{n-1}(q^{i}x+1)=
            \sum_{k=0}^{n}q^{\binom{k}{2}}\binom{n}{k}_{q}x^{k}
        \end{equation}
        Using this q-binomial, we get the following.
        \begin{theorem}
            If $h,n,m\in\mathbb{N}$, then:
            \begin{equation}
                \sum_{k=0}^{n}q^{(n-k)(h-k)}\binom{n}{k}_{q}
                \binom{m}{n-k}_{q}=
                \binom{m+n}{h}_{q}
            \end{equation}
        \end{theorem}
    \section{Generating Functions}
        Given a q-analogue and a set $S$, we can then define a
        statistic, $\lambda$. We then have:
        \begin{equation}
            \sum_{s\in{S}}q^{\lambda(s)}=
            \sum_{i=0}^{n}a_{i}q^{i}
        \end{equation}
        Where $a_{i}$ is the number of elements in $S$ with
        statistic value $i$. Thus we can think of a
        statistic $:S\rightarrow\mathbb{N}$, called the
        value function.
        \begin{lexample}
            Let $n\in\mathbb{N}$ and consider
            $S=\mathcal{P}(\mathbb{Z}_{n})$. One easy statistic
            we can place on $S$ is the cardinality function. That is,
            we define $f:S\rightarrow\mathbb{N}$ by:
            \begin{equation}
                f(\omega)=\Card(\omega)
                \quad\quad
                \omega\in\mathcal{P}(\mathbb{Z}_{n})
            \end{equation}
            Let's compute this a different way. Given a set
            $A\subseteq\mathcal{P}(\mathbb{Z}_{n})$, either
            $1\in{A}$ or $1\notin{A}$. Similarly, either
            $2\in{A}$ or $2\notin{A}$. For all $k\in\mathbb{Z}$,
            either $k\in{A}$ or $k\notin{A}$. Thus, we have:
            \begin{equation}
                (q^{1}+q^{0})\cdots(q^{1}+q^{0})=
                \prod_{k=1}^{n}(q^{1}+q^{0})
                =[2]_{q}^{n}
                =\sum_{k=0}^{n}\binom{n}{k}q^{k}
            \end{equation}
        \end{lexample}
        Next we want to consider $S$ being infinite. To get
        a generating function we require that $a_{i}$ is equal
        to the number of elements in $S$ with value $i$ being finite.
        We obtain the following power series:
        \begin{equation}
            a_{0}q^{0}+a_{1}q^{1}+\dots
            =\sum_{i=0}^{\infty}a_{i}q^{i}
        \end{equation}
        Let $\mathbb{C}[[q]]$ denote the ring of formal power
        series. This is a ring. For let:
        \begin{subequations}
            \begin{align}
                A(q)&=\sum_{i=0}^{\infty}a_{i}q^{i}\\
                B(q)&=\sum_{i=0}^{\infty}b_{i}q^{i}
            \end{align}
        \end{subequations}
        Then the sum is well defined, and we have:
        \begin{equation}
            A(q)+B(q)=
            \sum_{i=0}^{\infty}(a_{i}+b_{i})q^{i}
        \end{equation}
        We can also define the product by using the convolution
        product, or Cauchy sums:
        \begin{equation}
            A(q)B(q)=\sum_{i=0}^{\infty}c_{i}q^{i}
        \end{equation}
        Where:
        \begin{equation}
            c_{k}=\sum_{i=0}^{k}a_{i}b_{k-i}
        \end{equation}
        Some properties of $\mathbb{C}[[q]]$ is that it is a
        commutative ring. This is because we considered the
        coefficients to be over $\mathbb{C}$. Moreover, it is an
        integral domain. That is, $\mathbb{C}[[q]]$ has no zero
        divisors. The units, or invertible elements, are formal
        power series such that $a_{o}\ne{0}$. Indeed, this is a
        necessary and sufficient condition for an element to be
        invertible. For example, consider:
        \begin{equation}
            A(q)=\sum_{i=0}^{\infty}q^{i}
        \end{equation}
        This is a geometric sum, and we can show that for
        $|q|<1$, this formal sum is a convergent sum and evaluates
        to $(1-q)^{\minus{1}}$. However, for all formal sums, this
        formal power series has an inverse, and the inverse is
        indeed $(1-q)^{\minus{1}}$. Ivan Niven has a nice article
        on $\mathbb{C}[qq]]$ in the American Mathematical Monthly,
        1969. This has applications in counting partitions of
        numbers. See Andrews Theory of Partitions. Let $p(n)$ be
        the number of partitions on $n$. Then:
        \begin{equation}
            \sum_{n=0}^{\infty}p(n)q^{n}=
            1+q+2q^{2}+3q^{3}+5q^{4}+7q^{5}+11q^{6}+15q^{7}+\dots
        \end{equation}
        The value function is thus the weight of the partion
        $|\lambda|=\lambda_{1}+\dots+\lambda_{m}$.
        \begin{ltheorem}{Euler's Thoemre}
            If $P$ is the set of partitions, then:
            \begin{equation}
                \sum_{\lambda\in{P}}q^{|\lambda|}=
                \prod_{k=1}^{\infty}\frac{1}{q-q^{i}}
            \end{equation}
        \end{ltheorem}
        Give an arbitrary partition $\lambda$, consider the parts
        of size one. $\lambda$ does not have a part of size 1 or
        $\lambda$ has a part of size one, or $\lambda$ has a part of
        size two, and so on. Now do the same for each of the parts
        of size $k$, in general.
        \begin{ltheorem}{Euler's Other Theorem}
            The number of partitions with $n$ distinct parts
            is equal to the number of partitions of $n$
            with only odd parts.
        \end{ltheorem}
        \begin{proof}
            For:
            \begin{equation}
                \sum_{n=0}^{\infty}a_{n}q^{n}=
                \prod_{i=1}^{\infty}(1+q^{i})=
                \prod_{i=1}^{\infty}\frac{(q+q^{i})(1-q^{i})}{1-q^{i}}
            \end{equation}
            We can then simplify:
        \end{proof}
        \begin{ldefinition}{Durfee Square}
            The largest square that fits into a partition
            $\lambda$ is called the Durfee square of $\lambda$.
        \end{ldefinition}
        \begin{ldefinition}{Self-Conjugate Partition}
            A self-conjugate partition is a partition
            $\lambda$ such that $\lambda=\lambda'$, where
            $\lambda'$ is the conjugate of $\lambda$.
        \end{ldefinition}
        \begin{ltheorem}{Euler's Other-Other Theorem}
            The following is true:
            \begin{equation}
                \sum_{\lambda=\lambda'}q^{|\lambda|}=
                \prod_{n=0}^{\infty}(1+q^{2n+1})
            \end{equation}
            Where $\lambda=\lambda'$ are all of the self-conjugate
            partitions.
        \end{ltheorem}
        The product is the generating function for partitions with
        distinct odd parts. Euler's theorem then says that the
        generating function for this set is the equal to the sum
        over all of the self-conjugate partitions.
    \section{Euler's Theorem}
        \begin{equation}
            \sum_{n\in\mathbb{N}}p(n)q^{n}=
            \prod_{i=1}^{\infty}\frac{1}{1-q^{i}}
        \end{equation}
        Where $p(n)$ is the number of partitations of $n$.
        We also define the Euler function, not to be confused
        with the Euler totient function, as:
        \begin{equation}
            \phi(q)=\prod_{i=1}^{\infty}(1-q^{i})
        \end{equation}
        Let's try to simplify this:
        \begin{equation}
            \phi(q)=\prod_{i=1}^{\infty}(1-q^{i})=
            \sum_{k=0}^{\infty}b_{k}q^{k}
        \end{equation}
        We want to find the $b_{k}$. Multiplying through by
        the original series from Euler's theorem, we get:
        \begin{equation}
            \Big(\sum_{i=1}^{\infty}b_{i}q^{i}\Big)
            \Big(\sum_{j=0}^{\infty}p(j)q^{j}\Big)=1
        \end{equation}
        Using the convolution product, we have for all $k\geq{1}$:
        \begin{equation}
            \sum_{j=0}^{k}b_{j}p(k-j)=0
        \end{equation}
        This gives a recursion for $p(k)$. This gives us
        Euler's Pentagonal Number Theorem.
        \begin{ltheorem}{Euler's Pentagonal Number Theorem}
            \begin{subequations}
                \begin{align}
                    \phi(q)&=\prod_{i=1}^{\infty}(1-q^{i})\\
                    &=1+\sum_{m=1}^{\infty}(\minus{1})^{m}
                    \Big(q^{\frac{m(3m-2)}{2}}+
                        q^{\frac{m(3m+1)}{2}}\Big)\\
                    &=\sum_{m=\minus\infty}^{\infty}
                        (\minus{1})^{m}q^{\frac{m(3m-1)}{2}}
                \end{align}
            \end{subequations}
        \end{ltheorem}
        Let $p_{e}(d,n)$ denote the number of partitions $n$ with
        distinct parts and even length. Similarly, define
        $p_{o}(d,x)$ for odd length.
        \begin{theorem}
            \begin{equation}
                p_{e}(d,n)-p_{o}(d,n)=
                \begin{cases}
                    (\minus{1})^{n},&n=\frac{m(3m\pm{1})}{2}\\
                    0,&\textrm{Otherwise}
                \end{cases}
            \end{equation}
        \end{theorem}
        After some reflection, it should be easy to see that
        the first case is the inverse of the second case. Moreover,
        cases 1, 2, and 3 cover all partitions with distinct
        parts. For any partition in case three only one of
        $a$ or $b$ is true. The bijection constructed proves
        Euler's Pentagonal Theorem. Now we can compute $p$:
        \begin{subequations}
            \begin{align}
                p(0)&=1\\
                p(1)&=1\\
                p(2)&=2\\
                p(3)&=3\\
                p(4)&=p(3)+p(2)=5\\
                p(5)&=p(4)+p(3)-p(0)=7\\
                p(6)&=p(5)+p(4)-p(1)=11\\
                p(7)&=p(6)+p(5)-p(2)-p(0)=15
            \end{align}
        \end{subequations}
        And in general:
        \begin{equation}
            p(n)=p(n-1)+p(n-2)-p(n-5)-p(n-7)+\dots
        \end{equation}
        Where we add and subtrack over the pentagonal numbers.
        Gauss then turned to the question of computing powers
        of $\phi(q)$.
        \begin{ltheorem}{Gauss's Pentagonal Theorem}
            \begin{equation}
                \phi(q)^{3}=\prod_{i=1}^{\infty}(1-q^{i})^{3}
                =\sum_{r=0}^{\infty}(\minus{1})^{r}(2r+1)
                    q^{\frac{r(r+1)}{2}}
            \end{equation}
        \end{ltheorem}
        This identity occurs in many different areas of mathematics,
        such as homological algebra, complex analysis, and
        hyperbolic geometry. The proof comes from jacobi's
        Triple Product Identity.
        \begin{ltheorem}{Jacobi's Triple Product Identity}
            \begin{equation}
                \sum_{n=\minus\infty}^{\infty}z^{n}q^{n^{2}}=
                    \prod_{n=0}^{\infty}(1-q^{2n+2})
                        (1+zq^{2n+1})(1+z^{\minus{1}}q^{2n+1})
            \end{equation}
        \end{ltheorem}
        The proof of Gauss' identity then uses Sylvester's
        bijection. To get this from Jacobi, do a shift of
        index starting from $n=0$ to $n=1$. Differentiate both
        sides with respect to $q$, and then put
        $z=\minus{q}$. Finally, map $q^{2}$ to $q$.
        \par\hfill\par
        Felix Klein computed $\phi(q)^{8}$. In the theory of
        modular forms there is something called the $\tau$
        function, due to Ramanujan. This has the property that:
        \begin{equation}
            \sum_{n=1}^{\infty}\tau(n)q^{n-1}=
            \phi(q)^{24}=\prod_{m=1}^{\infty}(1-q^{m})^{24}
        \end{equation}
        Freeman Dyson also had some contributions to this subject
        and came up with nice formula for $\phi(q)^{d}$ when
        $d=3,8,10,14,15,21,24,26,28,35,36,\dots$ With the
        exception of $26$, these are the dimensions of the Lie
        algebras. Ian McDonald, working on the same problem, saw
        this as well. He came up with the following:
        \begin{equation}
            \phi(q)^{n^{2}-1}=\sum\varepsilon(k_{1},\dots,k_{n})
                \prod_{i=1}^{n}\binom{k_{i}}{n-i}q^{k_{n}}
        \end{equation}
        Where this is summed over all tuples $(k_{1},\dots,k_{n})$
        of non-negative integers such that:
        \begin{equation}
            \sum_{i=1}^{n}k_{i}^{2}=\sum_{i=1}^{k}k_{i}+
            \sum_{i=1}^{n-1}k_{i}k_{i+1}
            +k_{n}k_{1}
        \end{equation}
        And where $\varepsilon(k_{1},\dots,k_{m})=\pm{1}$.
    \section{Generating Function for Multisets}
        Recall that a multiset is a colection with repetition.
        For example:
        \begin{equation}
            A=\{\{1,1,1,3,3,4\}\}
        \end{equation}
        This is different from the set
        $B=\{1,1,1,3,3,4\}$, since sets cannot account for
        repetition. That is, $B$ can be reduced down to
        $B=\{1,3,4\}$. Note that partitions are multisets and
        we have shown that $\binom{k+(n-1)}{k}$ is the
        number of multi-sets of size $k$ chosen from the
        set $[n]$. We want to compute the generating function
        for the number of multi-sets of size $k$:
        \begin{equation}
            f(M)=\sum_{M}q^{|M|}
        \end{equation}
        Where $M$ is a multi-set of elements in $[n]$, and
        $|M|$ denotes the number of elements in $M$, with
        repetitions included. If $M$ is an arbitrary multi-set,
        then either $1\ne{M}$, or $1\in{M}$, or $1,1\in{M}$,
        and so on. So, in general, we get:
        \begin{equation}
            \sum_{k=0}^{\infty}q^{k}=\frac{1}{1-q}
        \end{equation}
        In general:
        \begin{equation}
            \sum_{M}q^{|M|}=\frac{1}{(1-q)^{n}}
            =\sum_{k=0}^{\infty}\binom{n-1+k}{k}q^{k}
        \end{equation}
        From the binomial theorem, we get:
        \begin{equation}
            (1+q)^{n}=
            \sum_{k=0}^{n}\binom{n}{k}q^{k}
        \end{equation}
        And thus, we can define:
        \begin{equation}
            \binom{\minus{n}}{k}=
            \binom{n-1+k}{k}(\minus{1})^{k}
        \end{equation}
        Extending the binomial coefficient to all
        $n\in\mathbb{Z}$. Next, recall the q-binomial theorem.
        \begin{equation}
            \sum_{k=0}^{n}\binom{n}{k}_{q}q^{\binom{k}{2}}x^{k}
            =\prod_{k=1}^{n-1}(1+q^{k}x)
        \end{equation}
        And also:
        \begin{equation}
            \sum_{k=0}^{\infty}\binom{n+k}{k}_{q}x^{k}=
            \prod_{i=0}^{n}(1-q^{i}x)^{\minus{1}}
        \end{equation}
        See Stanley and MacDonald.
    \section{Symmetric Functions}
        \subsection{Symmetric Polynomials}
            Let $x_{1},\dots,x_{N}$ commute, and let
            $f\in\mathbb{C}[x_{1},\dots,x_{N}]$. Then $f$ is
            symmetric if, for all $\sigma\in{S}_{N}$, then:
            \begin{equation}
            f(x_{1},\dots,x_{N})
            =f(x_{\sigma(1)},\dots,x_{\sigma(N)})
            \end{equation}
            Where $S_{N}$ is the symmetric group, and
            $\sigma$ is any permutation. It is convenient to
            work with infinitely many variables. We impose the
            requirements that there are countable many variables,
            so that we may list them, and that the commute. In
            this case we have power series instead of polynomials.
            We thus get sums of the form:
            \begin{equation}
                f=\sum_{\alpha}C_{\alpha}x^{\alpha}
            \end{equation}
            Where $\alpha$ is a sequence of non-negative integers.
            We require that the sum over $\alpha$ be finite, and
            thus this implies that all of the monomials are of
            finite degree. $f$ is said to be homogeneous if all
            of the monomials have the same degree. We require that
            $f$ be invariant under any permutation
            $\sigma:\mathbb{N}\rightarrow\mathbb{N}$.
            \begin{ldefinition}{Monomial Symmetric Basis}
                A monomial symmetric basis is:
                \begin{equation}
                    m_{\lambda}=m_{\lambda}(x)+
                    \sum_{\alpha}x^{\alpha}
                \end{equation}
                Where $\alpha$ is a rearrangement of $\lambda$.
            \end{ldefinition}
            That is, $m_{\lambda}$ is the sum of all monomials
            in $x_{i}$ whose exponents are the parts of $\lambda$.
            \begin{example}
                \begin{subequations}
                    \begin{align}
                        m_{1,1}=\sum_{i<j}x_{i}x_{j}&=
                            x_{1}x_{2}+x_{1}x_{3}+\dots
                            +x_{2}x_{3}+\dots\\
                        m_{2,1,1}(x_{1},x_{2},x_{3})&=
                            x_{1}^{2}x_{2}x_{2}+x_{1}x_{2}^{2}x_{3}+
                                x_{1}x_{2}x_{3}^{2}\\
                        m_{2}(x)=\sum_{i=1}^{\infty}x_{i}^{2}
                    \end{align}
                \end{subequations}
            \end{example}
            \begin{ldefinition}{Elementary Symmetric Function}
                The elementary symmetric function is defined as:
                \begin{equation}
                    e_{k}=m_{1^{k}}=
                    \sum_{i_{1}<i_{2}<\dots<i_{k}}
                        x_{i_{1}}x_{i_{2}}\cdots{x}_{i_{k}}
                \end{equation}
                For $k\in\mathbb{N}$.
            \end{ldefinition}
            Note that:
            \begin{equation}
                \prod_{i=1}^{\infty}(1+zx_{i})=
                \sum_{n=0}^{\infty}e_{n}z^{n}
            \end{equation}
            \begin{ldefinition}{Power Symmetric Function}
                The power symmetric function is defined as:
                \begin{equation}
                    P_{k}=m_{k}=
                    \sum_{i=1}^{\infty}x_{i}^{k}
                \end{equation}
                For $k\in\mathbb{N}$. That is, $k\geq{1}$.
            \end{ldefinition}
            \begin{ldefinition}
                  {Complete Homogeneous Symmetric Function}
                The complete homogeneous symmetric function is
                defined as:
                \begin{equation}
                    h_{k}=\sum_{\lambda+k}m_{\lambda}
                \end{equation}
                That is, the sum of all monomials of degree $k$.
            \end{ldefinition}
            \begin{theorem}
                The generating function for the homogeneous
                symmetric function is:
                \begin{equation}
                    \prod_{i=1}^{\infty}\frac{1}{1-zx_{i}}=
                    \sum_{n=0}^{\infty}h_{n}z^{n}
                \end{equation}
                Where $h_{0}$ is defined as $h_{0}=1$.
            \end{theorem}
            An endomorphism is a function such that
            $w(fg)=w(f)w(g)$, $w(f+g)=w(f)+w(g)$, and
            $w(cf)=cw(f)$.
            \begin{ldefinition}{$\omega$ Involution}
                The $\omega$ involution is the endomorphism defined
                by:
                \begin{equation}
                    \omega(p_{k})=(\minus{1})^{k=1}P_{k}
                \end{equation}
                And:
                \begin{equation}
                    \omega(p_{\lambda})=
                    (\minus{1})^{n-\ell(\lambda)}P_{\lambda}
                \end{equation}
            \end{ldefinition}
\end{document}
    %         \input{books/Algebra/Discrete_Structures.tex}
    %         \chapter{Linear Algebra}
    \section{Linear Algebra}
        \begin{lexample}
            If $V$ and $W$ are $2-$dimensional subspaces in
            $\mathbb{R}^{4}$, what are the possible dimensions of
            $V\cap W$. If $V$ and $W$ are subspaces, then
            ${V}\cap{W}$ is subspace, and
            $\dim({V}\cap{W}\leq\min(\{\dim(V),\dim(W)\})$
            we have in our problem that $\dim\{V\cap W\}\leq 2$. We
            now must show that $V\cap W$ can have dimensions 0,1, or
            2. If $V=\{(x,y,0,0):x,y\in\mathbb{R}\}$ and
            $W=\{(0,0,z,w):z,w\in \mathbb{R}\}$, then
            ${V}\cap{W}=\{(0,0,0,0)\}$ which has dimension $0$.
            If $V=\{(x,y,0,0):x,y\in\mathbb{R}\}$ and
            $W=\{(0,y,z,0):y,z\in\mathbb{R}\}$,
            then ${V}\cap{W}=\{(0,y,0,0):y\in\mathbb{R}\}$
            which has dimension $1$. Finally, if $V=W$ then
            ${V}\cap{W}=V$, which has dimension $2$. So, the only
            possibilities are $0,1$, or $2$.
        \end{lexample}
        A system of linear equations can be written
        using an equivalent matrix notation.
        \begin{example}
            \begin{align*}
                a_{11}x_{1}+a_{12}x_{2}+a_{13}x_{3}&=b_{1}\\
                a_{21}x_{1}+a_{22}x_{2}+a_{23}x_{3}&=b_{2}\\
                a_{31}x_{1}+a_{32}x_{2}+a_{33}x_{3}&=b_{3}\\
            \end{align*}
            This is equivalent to either of the following:
            \begin{align*}
                \begin{bmatrix}
                    a_{11}&a_{12}&a_{13}\\
                    a_{21}&a_{22}&a_{31}\\
                    a_{31}&a_{32}&a_{33}
                \end{bmatrix}
                \begin{bmatrix}
                    x_{1}\\
                    x_{2}\\
                    x_{3}
                \end{bmatrix}
                &=
                \begin{bmatrix}
                    b_{1}\\
                    b_{2}\\
                    b_{3}
                \end{bmatrix}
                &
                \mathbf{A}\mathbf{x}
                &=\mathbf{b}
            \end{align*}
        \end{example}
        Matrices can also be written as $\mathbf{A}=(a_{ij})$.
        The following rules are used to
        define matrix arithmetic.
        \begin{enumerate}
            \item Addition: To add two matrices, add their
                corresponding elements. That is, if
                $\mathbf{A}=(a_{ij})$ and $\mathbf{B}=(b_{ij})$,
                then $\mathbf{A}+\mathbf{B}=(a_{ij}+b_{ij})$.
                Matrix addition is only defined on matrices of
                the same size.
            \item Scale multiplication: To multiply a
                matrix by a real or complex number $c$,
                multiply this number to every element. That is,
                if $\mathbf{A}=(a_{ij})$, then
                $c\mathbf{A}=({c}\cdot{a_{ij}})$
            \item Matrix Multiplication: The product of
                and ${M}\times{N}$ matrix with an
                ${N}\times{P}$ matrix is defined by
                $\mathbf{C}=\mathbf{A}\mathbf{B}$, where
                $(c_{ij})=(\sum_{k=1}^{N}a_{ik}b_{kj})$. Note
                that it is possible for
                $\mathbf{A}\mathbf{B}\ne\mathbf{B}\mathbf{A}$.
                Indeed, it is possible for
                $\mathbf{A}\mathbf{B}$ to be defined, whereas
                $\mathbf{B}\mathbf{A}$ is undefined.
        \end{enumerate}
        \begin{example}
            Let the following be true:
            \begin{align*}
                A&=
                \begin{bmatrix}
                    1&2\\
                    3&4
                \end{bmatrix}
                &
                B&=
                \begin{bmatrix}
                    5&6\\
                    7&8
                \end{bmatrix}
            \end{align*}
            Then, using the defined rules, we have:
            \begin{align*}
                A+B&=
                \begin{bmatrix}
                    6&8\\
                    10&12
                \end{bmatrix}
                &
                5A&=
                \begin{bmatrix}
                    5&10\\
                    15&20
                \end{bmatrix}
                \\
                AB&=
                \begin{bmatrix}
                    19&22\\
                    43&50
                \end{bmatrix}
                &
                BA&=
                \begin{bmatrix}
                    23&34\\
                    31&46
                \end{bmatrix}
            \end{align*}
        Note that even in this trivial example,
        ${AB}\ne{BA}$.
        \end{example}
        \begin{definition}
            The ${n}\times{n}$ identity matrix is the matrix
            $I_{n}=(I_{ij})$, where
            $I_{ij}=%
            \begin{cases}%
             0,&{i}\ne{j}\\%
             1,&{i}={j}%
            \end{cases}$
        \end{definition}
        \begin{definition}
            An inverse matrix of an ${n}\times{n}$ matrix
            $A$ is a matrix $A^{-1}$ such that
            $AA^{-1}=A^{-1}A=I_{n}$
        \end{definition}
        Not every matrix has an inverse matrix. If one
        does exists, there are many properties it contains.
        \begin{theorem*}
            The following are true:
            \begin{enumerate}
                \item If $\mathbf{A}$ and $\mathbf{B}$
                    are invertible ${n}\times{n}$ matrices,
                    then $\mathbf{A}\mathbf{B}$ is invertible
                    and
                    $\mathbf{A}\mathbf{B}^{-1}%
                     =\mathbf{B}^{-1}\mathbf{A}^{-1}$
                \item If $\mathbf{A}$ is an invertible matrix,
                    then $\mathbf{A}^{-1}$ is an invertible
                    matrix and
                    $(\mathbf{A}^{-1})^{-1}=\mathbf{A}$
            \end{enumerate}
        \end{theorem*}
        \begin{definition}
            The trace of an ${n}\times{n}$ matrix
            $A$ is the sum of
            it's diagonal: $\Tr(A)=\sum_{i=1}^{n}a_{ii} $
        \end{definition}
        \begin{example}
            \begin{equation*}
                \Tr\Bigg(
                \begin{bmatrix}
                    4&5&6\\
                    1&2&3\\
                    8&8&3
                \end{bmatrix}
                \Bigg)
                =4+2+3=9
            \end{equation*}
        \end{example}
        \begin{definition}
            The determinant of a ${2}\times{2}$ matrix
            $A=%
             \begin{bmatrix}%
                a&b\\%
                c&d%
             \end{bmatrix}$
            is $\det(A)=ad-bc$
        \end{definition}
        \begin{definition}
            The minor of the $i^{th}$ row and $j^{th}$
            column of an ${n}\times{n}$ matrix $\mathbf{A}$,
            denoted $M_{ij}$, is the determinant of the
            ${(n-1)}\times{(n-1)}$ matrix formed by
            removing the $i^{th}$ row and $j^{th}$ column
            from $\mathbf{A}$.
        \end{definition}
        \begin{definition}
            The cofactor of the minor $M_{ij}$ of an
            ${n}\times{n}$ matrix $\mathbf{A}$,
            denoted $C_{ij}$, is $(-1)^{i+j}M_{ij}$.
        \end{definition}
        \begin{example}
            \begin{align*}
                A&=
                \begin{bmatrix}
                    7&1&3\\
                    1&3&5\\
                    17&4&20
                \end{bmatrix}
                &
                M_{11}
                &=
                \det\Bigg(\begin{bmatrix}
                         3&5\\
                         4&20
                     \end{bmatrix}
                    \Bigg)
                =40
                &
                C_{11}
                &=(-1)^{1+1}M_{11}=40
            \end{align*}
        \end{example}
        \begin{definition}
            The determinant of an ${n}\times{n}$ matrix
            $\mathbf{A}$ is
            $\det(A)=\sum_{j=1}^{n}a_{1j}C_{1j}$
        \end{definition}
        \begin{theorem*}
            If $\mathbf{A}$ is an ${n}\times{n}$ matrix
            and ${1}\leq{i}\leq{n}$, then
            $\det(A)=\sum_{j=1}^{n}a_{ij}C_{ij}$
        \end{theorem*}
        \begin{definition}
            The transpose of an ${n}\times{m}$ matrix
            $\mathbf{A}$, denoted $\mathbf{A}^{T}$,
            is the ${m}\times{n}$ matrix formed by
            swapping the rows and columns of $\mathbf{A}$
            with each other. That is $(a_{ij})^{T}=(a_{ji})$.
        \end{definition}
        \begin{definition}
            A symmetric matrix is a matrix $\mathbf{A}$
            such that $\mathbf{A}^{T}=\mathbf{A}$
        \end{definition}
        \begin{theorem*}
            If $\mathbf{A}$ is an ${n}\times{m}$ matrix and
            $\mathbf{B}$ is an ${m}\times{p}$ matrix, then
            the following are true:
            \begin{enumerate}
                \begin{multicols}{2}
                    \item $(\mathbf{A}^{T})^{T}=\mathbf{A}$
                    \item $(\mathbf{A}+\mathbf{B})^{T}%
                           =\mathbf{A}^{T}+\mathbf{B}^{T}$
                    \item $(k\mathbf{A})^{T}=k\mathbf{A}^{T}$
                    \item $(\mathbf{A}\mathbf{B})^{T}%
                           =\mathbf{B}^{T}\mathbf{A}^{T}$
                \end{multicols}
            \end{enumerate}
        \end{theorem*}
        \begin{definition}
            The adjoint of an ${n}\times{n}$ matrix
            $\mathbf{A}$, denoted $\adjoint(\mathbf{A})$,
            is the matrix $(C_{ij})^{T}$.
        \end{definition}
        \begin{theorem*}
            If ${\det(\mathbf{A})}\ne{0}$, then $\mathbf{A}$
            is invertible and
            $\mathbf{A}^{-1}=%
             \frac{1}{\det(\mathbf{A})}\adjoint(\mathbf{A})$
        \end{theorem*}
        \begin{theorem*}
            If $\mathbf{A}$ and $\mathbf{B}$ are
            ${n}\times{n}$ matrices, then the following
            are true:
            \begin{enumerate}
                \begin{multicols}{3}
                    \item $\det(\mathbf{A})%
                           =\det(\mathbf{A}^{T})$
                    \item $\det(k\mathbf{A})%
                           =k^{n}\det(\mathbf{A})$
                    \item $\det(\mathbf{A}\mathbf{B})%
                           =\det(\mathbf{A})\det(\mathbf{B})$
                \end{multicols}
            \end{enumerate}
        \end{theorem*}
        \begin{theorem*}
            A matrix $\mathbf{A}$ is invertible if and only
            if ${\det(\mathbf{A})}\ne{0}$
        \end{theorem*}
        \begin{theorem*}
            If $\mathbf{A}$ is invertible, then
            $\det(\mathbf{A}^{-1})=\frac{1}{\det(\mathbf{A})}$
        \end{theorem*}
        The differential equation
        $\sum_{k=0}^{n}a_{k}y^{(k)}(x)$ Can be expression
        in terms of the characteristic polynomial
        $\sum_{k=0}^{n}a_{k}D^{k}$. Factoring this linear
        operator into $\Pi_{k=0}^{n}(D-r_{k})$,
        the general solution is
        $y(x)=\sum_{k=1}^{n}c_{k}e^{r_{k}x}$. If some of the
        $r_{k}$ repeat, we have $c_{k}x^{m_{k}-1}e^{r_{k}x}$,
        where $m_{k}$ is the number of repetitions.
        In general, if we have
        $\Pi_{k=0}^{n}(D-r_{k})^{m_{k}}$, the general
        solution is
        $y(x)=%
         \sum_{k=1}^{n}c_{k}e^{r_{k}x}%
         (\sum_{j=0}^{m_{k}-1}x^{j})$
        \begin{example}
            \
            \begin{enumerate}
                \item $y'''-4y''+4y'=0$ has the characteristic
                    polynomial $D(D-2)^{2}$, so
                    $y(x)=c_{1}+c_{2}e^{2x}+c_{3}xe^{2x}$
            \end{enumerate}
        \end{example}
        In linear algebra, the determinant
        $\det(\mathbf{A}-\lambda{I})$ is the characteristic
        polynomial of the square matrix $\mathbf{A}$.
        \begin{definition}
            A vector space $V$ over a Field (Set of scalars)
            $F$ is a set $V$ with two operations
            $+$ and $\cdot$
            such that the following are true:
            \begin{enumerate}
                \begin{multicols}{3}
                    \item $\forall_{{\mathbf{a},%
                                     \mathbf{b}}\in{V}}$
                          ${\mathbf{a}+\mathbf{b}}\in{V}$
                    \item $\mathbf{a}+\mathbf{b}%
                           =\mathbf{b}+\mathbf{a}$
                    \item $\mathbf{a}+(\mathbf{b}+\mathbf{c})%
                           =(\mathbf{a}+\mathbf{b})+\mathbf{c}$
                    \item $\forall_{\mathbf{a}\in{V}}%
                           \exists_{\mathbf{b}\in{V}}:%
                           \mathbf{a}+\mathbf{b}=\mathbf{0}$
                    \item $\forall_{{k}\in{F},\mathbf{a}\in{V}}$
                          $k\mathbf{a}\in{V}$
                    \item $k(\mathbf{a}+\mathbf{b})%
                           =k\mathbf{a}+k\mathbf{b}$
                    \item $(k_{1}+k_{2})\mathbf{a}%
                           =k_{1}\mathbf{a}+k_{2}\mathbf{a}$
                    \item $1\mathbf{a}=\mathbf{a}$
                    \item $k_{1}(k_{2}\mathbf{a})%
                           =(k_{1}k_{2})\mathbf{a}$
                \end{multicols}
            \end{enumerate}
        \end{definition}
        \begin{theorem*}
            If $V$ is a vector space, then there is a
            $\mathbf{0}\in{V}$ such that for all
            $\mathbf{a}\in{V}$,
            $\mathbf{a}+\mathbf{0}=\mathbf{a}$
        \end{theorem*}
        \begin{definition}
            A linearly dependent subset of a vector space
            $V$ (Over $\mathbb{R}$)
            is a subset ${S}\subset{V}$ such that
            there exists an $N\in\mathbb{N}$, a non-zero
            $a_{n}:\mathbb{Z}_{N}\rightarrow\mathbb{R}$
            and an injective function
            $\mathbf{v}_{n}:\mathbb{Z}_{N}\rightarrow{V}$
            such that
            $\sum_{k=1}^{N}a_{n}\mathbf{v}_{n}=\mathbf{0}$
        \end{definition}
        \begin{definition}
            A linearly independent subset of a vector space
            $V$ is a subset ${S}\subset{V}$ that is not
            linearly dependent.
        \end{definition}
        \begin{theorem*}
            If $V\subset\mathbb{R}^{n}$ has more than
            $n$ vectors, then $V$ is linearly dependent.
        \end{theorem*}
        \begin{definition}
            The rank of a matrix is the number
            of linearly independent columns of
            the matrix.
        \end{definition}
        \begin{example}
            Let
            $\mathbf{A}=[A_{1}\ A_{2}]$
            where $A_{1}=(1,2)^{T}$ and
            $A_{2}=(2,4)^{T}$. So
            $2A_{1}+(-1)A_{2}=(0,0)^{T}=\mathbf{0}$. Therefore
            $\{A_{1},A_{2}\}$ is a linearly independent
            subset. Thus, $\rk(\mathbf{A})=1$.
        \end{example}
        \begin{definition}
            A matrix with full rank is a square
            ${n}\times{n}$ matrix $\mathbf{A}$ such that
            $\rk(\mathbf{A})=n$.
        \end{definition}
        \begin{theorem*}
            If $\mathbf{A}$ is a square matrix and
            $\det(\mathbf{A})\ne{0}$, then $\mathbf{A}$
            has full rank.
        \end{theorem*}
        \begin{theorem*}
            If $\mathbf{A}$ is a square matrix
            with full rank, then it is invertible.
        \end{theorem*}
        \begin{definition}
            A finite basis of a vector space $V$ is a
            linearly independent subset ${S}\subset{V}$
            where
            $S=\{\mathbf{v}_{k}\}_{k=0}^{n}$
            and for all
            $\mathbf{x}\in{V}$ there is an
            $a_{n}:\mathbb{Z}_{n}\rightarrow\mathbb{R}$
            such that
            $\mathbf{x}=\sum_{k=1}^{n}a_{k}\mathbf{v}_{k}$
        \end{definition}
        \begin{theorem*}
            All bases of a vector space $V$ have the
            same number of elements.
        \end{theorem*}
        \begin{definition}
            The dimension of a vector space
            $V$ is the number of elements in any
            basis of $V$.
        \end{definition}
        \begin{definition}
            An inner product on a vector space $V$ is a function
            $\langle|\rangle:{V}\times{V}\rightarrow\mathbb{R}$
            such that:
            \begin{enumerate}
                \begin{multicols}{3}
                    \item $\langle{v_{1},v_{2}}\rangle%
                           =\langle{v_{2},v_{3}}\rangle$
                    \item $\langle{v_{1},v_{1}}\rangle\geq{0}$
                    \item $\langle{v_{1}+v_{2},v_{3}}\rangle%
                           =\langle{v_{1},v_{3}}\rangle%
                           +\langle{v_{2},v_{3}}\rangle$
                \end{multicols}
            \end{enumerate}
        \end{definition}
        \begin{definition}
            The Euclidean inner product on $\mathbb{R}^{n}$
            is defined as:
            $\langle{\mathbf{x},\mathbf{y}}\rangle%
             =\sum_{k=1}^{n}x_{k}y_{k}$
        \end{definition}
        \begin{definition}
            An eigenvector of an ${n}\times{n}$ matrix
            $\mathbf{A}$ is a vector
            $\mathbf{x}\in\mathbb{R}^{n}$
            such that there exists a $\lambda\in\mathbb{R}$
            such that
            $\mathbf{A}\mathbf{x}=\lambda\mathbf{x}$
        \end{definition}
        \begin{definition}
            An eigenvalue of an ${n}\times{n}$ matrix
            $\mathbf{A}$ is a real number
            $\lambda\in\mathbb{R}$ such that there is
            a vector $\mathbf{x}\in\mathbf{R}^{n}$ such
            that $\mathbf{A}\mathbf{x}=\lambda\mathbf{x}$
        \end{definition}
        \begin{definition}
            The characteristic equation, or
            the characteristic polynomial, of an
            ${n}\times{n}$ matrix $\mathbf{A}$
            is $\det(\lambda{I}-\mathbf{A})=0$
        \end{definition}
        \begin{definition}
            A diagonalizable matrix is an
            ${n}\times{n}$ matrix
            $\mathbf{A}$ such that there exists
            an invertible matrix $\mathbf{B}$
            such that
            $\mathbf{A}=\mathbf{B}^{-1}\mathbf{A}\mathbf{B}$
        \end{definition}
        \begin{theorem*}
            The following are true:
            \begin{enumerate}
                \item If $\mathbf{A}$ is an ${n}\times{n}$
                    diagonable matrix, then $\mathbf{A}$
                    has $n$ linearly independent
                    eigenvectors.
                \item If $\mathbf{A}$ is an ${n}\times{n}$
                    matrix with $n$ linearly independent
                    eigenvectors, then $\mathbf{A}$
                    is diagonalizable.
                \item A symmetric matrix has all real
                    eigenvalues.
            \end{enumerate}
        \end{theorem*}
    \section{Miscellaneous Lecture Notes}
        \subsection{Orthogonal Projections}
        \begin{definition}
        The span of
        $\mathcal{W}=\{X_{i}\}_{1}^{k}\subset\mathbb{R}^n$
        is the set
        $\Span(\mathcal{W})=\{\sum_{i=1}^{k}a_{i}X_{i}:a_{i}\in \mathbb{R}\}$.
        \end{definition}
        \begin{definition}
        A linearly dependent subset of $\mathbb{R}^{n}$ is a subset $S\subset\mathbb{R}^{n}$ such that there exists a finite subset $\{X_{i}\}_{i=1}^{k}\subset S$ and a subset $\{a_{i}\}_{i=1}^{k}\subset \mathbb{R}\setminus \{0\}$ such that $\sum_{i=1}^{k}a_{i}X_{i}=\mathbf{0}$
        \end{definition}
        \begin{definition}
        A linearly independent subset of $\mathbb{R}^{n}$ is a subset $S\subset \mathbb{R}^{n}$ that is not linearly dependent.
        \end{definition}
        \begin{theorem}
        If $S\subset\mathbb{R}^{n}$ is linearly independent, then $|S|\leq n$.
        \end{theorem}
        \begin{theorem}
        If $\mathcal{W}\subset\mathbb{R}^{n}$ is linearly independent and $|\mathcal{W}| = k$, then $\Span(\mathcal{W})$ is a $k$ dimensional subspace of $\mathbb{R}^n$.
        \end{theorem}
        If we have a linearly independent subset $\mathcal{W}=\{X_1,\hdots, X_k\}\subset\mathbb{R}^n$, and some other vector $Y$, we may wish to consider the orthogonal projection of $Y$ onto the $k$ dimensional subspace spanned by $\mathcal{W}$. That is, we may wish to find a vector $Y'\in\Span(X_1,\hdots, X_k)$ such that $Y-Y'$ is orthogonal to $\Span(X_1,\hdots, X_k)$.
        \begin{theorem}
        If $\{X_1,\hdots, X_k\}\subset\mathbb{R}^n$ is linearly independent, $\mathcal{W} = \Span(X_1,\hdots, X_n)$, and if $Y\in \mathbb{R}^n$ is such that $Y\perp X_i$ for $i=1,2,\hdots, k$, then $\forall_{Z\in \mathcal{W}}$, $Y\perp Z$.
        \end{theorem}
        \begin{proof}
        For let $Y\in \mathbb{R}^n$ be such that for $i=1,2,\hdots, k$, $Y\perp X_i$. Let $Z\in \mathcal{W}$. Then $Z= \sum_{i=1}^{k} a_i X_i$, where $a_i\in \mathbb{R}$. But then $\langle Y, Z\rangle = \sum_{i=1}^{k} a_i \langle Y, X_i\rangle = \sum_{i=1}^{k} a_i\cdot 0 = 0$. 
        \end{proof}
        \begin{lemma}
        If $P$ is an $n\times k$ matrix whose columns are linearly independent, then $P^TP$ is invertible.
        \end{lemma}
        \begin{proof}
        If $P^TPX = 0$, then $PX$ is orthogonal to the columns of $P$. But $PX$ is a linear combination of the columns of $P$, and thus $PX$ is orthogonal to itself. Thus, $PX = 0$. But the columns of $P$ are linearly independent, if $PX = 0$, then $X=0$. Thus $P^TPX = 0$ if and only if $X=0$. $P^TP$ is invertible.
        \end{proof}
        We need $X_{i}^T(Y-Y')=0$. Let $X_{i}=(x_{1i},x_{2i},\hdots,x_{ni})^{T},P=(x_{ij})$. Then $P^T(Y-Y') = 0$, so $P^TY = P^T Y'$. But $Y'\in \mathcal{W}$, so $Y' = \sum_{i=1}^{k} c_i X_i = P(c_1,\hdots, c_k)^T = PC$. Thus, $C = (P^TP)^{-1}P^TY$. Therefore $Y'=P(P^TP)^{-1}P^T Y$.
        \begin{definition}
        The projection matrix of $\Span(X_{1},\hdots,X_{k})$ is $P(P^TP)^{-1}P^T$, where $P=(x_{ij})$.
        \end{definition}
        \begin{theorem}
        If $Q = P(P^TP)^{-1}P^T$ is a projection matrix for a subspace $\mathcal{W}$, then $Q^T =Q$.
        \end{theorem}
        \begin{proof}
        $Q^T=(P(P^{T}P)^{-1}P^{T})^{T}=(P^{T})^{T}(P(P^{T}P)^{-1})^{T}= P((P^{T}P)^{-1}g)^{T}P^{T}=P(P^{T}P)^{-1}P^{T}=Q$
        \end{proof}
        \begin{theorem}
        If $Q = P(P^TP)^{-1}P^T$ is a projection matrix for a subspace $\mathcal{W}$, then $Q^2 = Q$.
        \end{theorem}
        \begin{proof}
        $Q^{2}=P(P^TP)^{-1}P^TP(P^TP)^{-1}P^T= P((P^{T}P)^{-1}(P^{T}P))(P^{T}P)^{-1}P^{T}=P(P^{T}P)^{-1}P^{T}=Q$
        \end{proof}
        \begin{theorem}
        If $Q$ is an $n\times n$ matrix, $Q = Q^{2}$, and $Q=Q^{T}$, then there is a subspace $\mathcal{W}\subset \mathbb{R}^{n}$ such that $Q$ is the projection matrix of $\mathcal{W}$.
        \end{theorem}
        \subsection{Reflections}
        Let $\mathcal{W}$ be a plane passing through the origin, and suppose we want to reflect a vector $v$ across this plane. Let $u$ be a unit vector along $\mathcal{W}^{\perp}$. That is, $u$ is normal to the plane. The projection of $v$ along the line through $u$ is then given by $\hat{v} = Proj_{u}(v) = u(u^Tu)^{-1}u^Tv$. But $u$ is a unit vector, and therefore $u^Tu = 1$, so $\hat{v} = uu^T v$. Let $Q_u = uu^T$. The definition of the reflection of $v$ across $\mathcal{W}$ is the vector $\Refl_{\mathcal{W}}(v)$ such that has the same magnitude as $v$ lying on the opposite side of $\mathcal{W}$. Thus $v-\Refl_{\mathcal{W}}(v) = 2Q_u v$, and so we have:
        \begin{equation*}
            \Refl_{\mathcal{W}}(v) = v-2Q_u v = (I-2Q_u)v =(I-2uu^T)v
        \end{equation*}
        \begin{definition}
        $H_{\mathcal{W}} = I-2uu^T$ is called the Reflection (Householder) Matrix for $\mathcal{W}$.
        \end{definition}
        \begin{definition}
        An orthogonal matrix is a matrix $P$ such that $P^TP = I$.
        \end{definition}
        \subsection{Lecture Notes on Orthogonal Matrices}
        \begin{definition}
        An orthoganal matrix is a $n\times n$ matrix $A$ such that $A^{T}A = I$.
        \end{definition}
        \begin{theorem}
        If $A$ is an orthogonal matrix, then $A^T = A^{-1}$.
        \end{theorem}
        \begin{proof}
        For $A^TA = I$, and inverses are unique. Thus $A^T = A^{-1}$.
        \end{proof}
        If we let $A_{i} = Ae_{i}$, then $A^TA = (A_{i}^{T}A_{j}) = I$. Therefore $A_i^TA_j = \delta_{ij}$.
        \begin{theorem}
        If $A$ is an $n\times n$ real-valued matrix and $A_i = Ae_i$, $i=1,2,\hdots, n$, then $A$ is orthogonal if and only if $\{A_1,\hdots, A_n\}$ is an orthonormal set of vectors.
        \end{theorem}
        \begin{proof}
        If $A$ is orthogonal, then $A_{i}^{T}A_{j} = \delta_{ij}$, and from this we have orthonormality. If $\{A_1,\hdots, A_n\}$ is orthonormal, then $A^TA = I$ and is therefore orthogonal.
        \end{proof}
        \begin{theorem}
        The following statements are equivalent:
        \begin{enumerate}
        \begin{multicols}{3}
            \item $A$ is orthogonal
            \item $\forall_{X\in\mathbb{R}^{n}}$, $\norm{AX} = \norm{X}$
            \item $\forall_{X,Y\in\mathbb{R}^{n}}$, $\langle AX, AY\rangle = \langle X, Y\rangle$
        \end{multicols}
        \end{enumerate}
        \end{theorem}
        \begin{proof}
        We show $1\Rightarrow 2 \Rightarrow 3 \Rightarrow 1$.
        \begin{enumerate}
            \item If $A^TA = I$, then $\norm{AX}^{2}=(AX)^{T}AX=X^{T}A^{T}AX=X^{T}X=\norm{X}^{2}$. Therefore, $\norm{AX}=\norm{X}$.
            \item If $A$ is a square matrix such that $\forall_{X\in\mathbb{R}^{n}}$, $\norm{AX} = \norm{X}$, then:
            \begin{equation*}
                \norm{X+Y}^{2}=(X+Y)^T(X+Y)=X^TX+X^TY+Y^TX+Y^TY=\norm{X}^2+2X^TY+\norm{Y}^2
            \end{equation*}
            But:
            \begin{equation*}
                \norm{A(X+Y)}^{2}=\norm{AX+AY}^{2}=\norm{AX}^{2}+2(AX)^{T}AY+\norm{AY}^2=\norm{X}^{2}+2(AX)^{T}AY+\norm{Y}^2
            \end{equation*}
            Therefore $(AX)^TAY = X^TY$. That is, $\langle AX, AY\rangle = \langle X, Y\rangle$.
            \item If $A$ is a square matrix such that $\forall_{X,Y\in \mathbb{R}^n}$, $\langle AX, AY\rangle = \langle X, Y\rangle$, then $\langle Ae_{i}, Ae_{j}\rangle=\langle e_i,e_j\rangle=\delta_{ij}$
            Therefore, $A$ is orthogonal.
        \end{enumerate}
        \end{proof}
        \begin{theorem}
        If $A$ and $B$ are $n\times n$ orthogonal matrices, then $AB$ is orthogonal.
        \end{theorem}
        \begin{proof}
        For if $A^{T}A = I$ and $B^{T}B = I$, then $(AB)^{T}AB = B^{T}A^{T}AB = B^{T}IB = B^{T}B = I$. $AB$ is orthogonal.
        \end{proof}
        \begin{theorem}
        \label{theorem:LINEAR_ALGEBRA_orthogonal_matrices_have_determinant_pm_1}
        If $A$ is an $n\times n$ orthogonal matrix, then $\det(A) = 1$ or $-1$.
        \end{theorem}
        \begin{proof}
        For $\det(I) = \det(A^TA) = \det(A^T)\det(A) = \det(A)^2$. Thus, $\det(A) = \pm 1$.
        \end{proof}
        \begin{remark}
        The converse of Thm.~\ref{theorem:LINEAR_ALGEBRA_orthogonal_matrices_have_determinant_pm_1} is false.
        \end{remark}
        Recall that if $u\in \mathbb{R}^n$ is a unit vector and $W = u^{\perp}$, then $H=2uu^T$ is the reflection matrix for $W$. Reflections preserve distance, and therefore $H$ must be orthogonal.
        \newpage
        \begin{theorem}
        If $A$ is an $n\times n$ orthogonal matrix, then there exist $k$ $n\times n$ reflection matrices $H_1,\hdots, H_k$, $0\leq k \leq n$, such that $A = \prod_{i=1}^{j}H_i$.
        \end{theorem}
        \begin{proof}
        By induction. The base case is trivial. Suppose it holds for $n-1$. Let $z = Ae_n$, and let $H$ be the reflection matrix that exchanges $z$ and $e_n$. Then $HAe_n = Hz = e_n$, so $HA$ fixes $e_n$. But $HA$ is an orthogonal matrix, and thus preserves distances and angles. Thus $HA$ maps $\mathbb{R}^{n-1}$ onto itself, and thus by induction there are $H_2,\hdots, H_k$ such that $HA = \prod_{i=2}^{k} H_i$. Letting $H_{1}=H$, we have $A = HHA = \prod_{i=1}^{k}H_i$.
        \end{proof}
        \begin{theorem}
        If $H$ is a reflection matrix, then $\det(H) = -1$.
        \end{theorem}
        \begin{theorem}
        If $A$ is an orthogonal matrix and $A=\prod_{i=1}^{k} H_i$, then $\det(A) = (-1)^k$.
        \end{theorem}
        If $A$ is an orthogonal $2\times 2$ matrix, then we know that columns must be unit vectors that are also orthogonal (Orthonormal). That is, the two columns must lie on the unit circle about the origin. So we may express the first column as $(\cos(\theta),\sin(\theta))$ for some angle $\theta$. There are then two options for the seconds column: $(-\sin(\theta),\cos(\theta))$ or $(\sin(\theta),-\cos(\theta))$. The first is the rotation matrix which rotates $\mathbb{R}^2$ counterclockwise around the origin, and the second is the reflection matrix that makes a reflection across the line that makes an angle $\frac{\theta}{2}$ with the $x-$axis. 
        \begin{theorem}
        If $A$ is a $3\times 3$ orthogonal matrix and $\det(A) = 1$, then $A$ is a rotation matrix.
        \end{theorem}
        \begin{proof}
        $A$ must be the product of $0,1,2,$ or $3$ reflection matrices. If $\det(A) = 1$, then $A$ is the product of an even number of reflections, and thus either $A=I$ or $A$ is the product of two reflections, and is thus a rotation matrix.
        \end{proof}
        \begin{theorem}
        If $A$ is a $3\times 3$ orthogonal matrix, $\det(A) = -1$, and $A=A^T$, then either $A=-I$ or $A$ is a reflection matrix.
        \end{theorem}
        \begin{proof}
        If $\det(A)=-1$, then $A$ is the product of an odd number of reflections, either $1$ or $3$. If $A$ is a single reflection, then $A=H$ for some Householder matrix $H$. Thus $A^T = A$. Conversely, if $A = A^T$ and $\det(A) = -1$, then $\det(-A) = 1$, and $-A^T = -A = -A^{-1}$. Therefore $-A$ is a rotation whose square is the identity. If $A\ne I$, then $A$ must be a rotation of $\pi/2$ around some axis, and thus $A$ is a reflection.
        \end{proof}
        \begin{theorem}
        If $A$ is a $3\times 3$ orthogonal matrix, $\det(A) = -1$, and $A\ne A^T$, then $A$ is the product of three reflections.
        \end{theorem}
        \begin{proof}
        If $\det(A) = -1$, and $A\ne A^T$, then $A$is not a rotation or a pure reflection, and is thus a product of $3$ reflection matrices.
        \end{proof}
        \begin{theorem}
        If $A$ and $B$ are $3\times 3$ rotation matrices, then $AB$ is a rotation matrix.
        \end{theorem}
        \begin{proof}
        For $A$ and $B$ must be orthogonal, and thus $AB$ is orthogonal. But $\det(AB) = \det(A)\det(B) = 1\cdot 1 = 1$, and thus $AB$ is an orthogonal matrix with determinant equal to $1$, and is therefore a rotation matrix.
        \end{proof}
        \subsection{Rotations}
        The $2\times 2$ matrix $A_{\theta}$ rotates the plane $\mathbb{R}^2$ counterclockwise by $\theta$ around the origin. The question that then arises is, ``Is there a similar way to do this for $\mathbb{R}^3$?" The simple case would be rotating by an angle $\theta$ about the $z-$axis, analogous the rotating the Earth by $\theta$ about the North Pole. This fixes the $z-$axis and acts on the $xy$ plane only. This can be represented by the matrix $S_{\theta}$.
        \begin{equation*}
         A_{\theta}=\begin{bmatrix*}[r]\cos(\theta) & -\sin(\theta) \\ \sin(\theta) & \cos(\theta)\end{bmatrix*}\quad\quad\quad\quad S_{\theta}=\begin{bmatrix*}\cos(\theta) & -\sin(\theta) & \phantom{sin}0 \\ \sin(\theta) & \phantom{-}\cos(\theta) & \phantom{sin}0\\ 0 & \phantom{-}0 & \phantom{sin}1 \end{bmatrix*}   
        \end{equation*}
        $S_{\theta}$ is an orthogonal matrix. That is, $S_{\theta} S_{\theta}^T = I$, and therefore $S_{\theta}^T = S_{\theta}^{-1}$. Suppose we want to rotate by an angle $\theta$ about a different axis. Let $\mathbf{u}$ be a unit vector pointing in the direction of the axis of rotation and let $R_{\theta,\mathbf{u}}$ be the new rotation matrix. To compute $R_{\theta,\mathbf{u}}$ choose any unit vector $\mathbf{v}$ that is orthogonal to $\mathbf{u}$. Let $\mathbf{w} = \mathbf{u}\times \mathbf{v}$. Then $\{\mathbf{u},\mathbf{v},\mathbf{w}\}$ is an orthonormal basis of $\mathbb{R}^3$ such that $\mathbf{v}\times \mathbf{w} = \mathbf{u}$. Let:
        \begin{equation*}
            P = \begin{bmatrix} v_1 & w_1 & u_1 \\ v_2 & w_2 & u_2 \\ v_3 & w_3 & u_3 \end{bmatrix}
        \end{equation*}
        The columns of $P$ form an orthonormal set, and therefore $P$ is orthogonal. In particular:
        \begin{align*}
            P^{T}\mathbf{v}&=e_{1}&P^{T}\mathbf{w}&=e_{2}&P^{T}\mathbf{u}=e_{3}
        \end{align*}
        \begin{theorem}
        If $\theta \in [0,2\pi]$ and $\mathbf{u}\in \mathbb{R}^3$ is a unit vector, then $R_{\theta, \mathbf{u}} = PS_{\theta}P^T$.
        \end{theorem}
        \begin{proof}
        For $PS_{\theta}P^T\mathbf{u}=\mathbf{u}$, $PS_{\theta}P^{T}\mathbf{v}=\cos(\theta)\mathbf{v}+\sin(\theta)\mathbf{w}$, and $PS_{\theta}P^{T}\mathbf{w}=-\sin(\theta)\mathbf{v}+\cos(\theta)\mathbf{w}$
        Thus, if $X = a\mathbf{v}+b\mathbf{w}+c\mathbf{u}$, then $PS_{\theta}P^TX=a(\cos(\theta)\mathbf{v}+\sin(\theta)\mathbf{w})+b(-\sin(\theta)\mathbf{v}+\cos(\theta)\mathbf{w})+c\mathbf{u}=R_{\theta,\mathbf{u}}X$
        \end{proof}
        From the orthogonality of $P$ and $S_{\theta}$ we have that $R_{\theta,\mathbf{u}}$ is also orthogonal.
        \begin{theorem}
        \label{theorem:LINEAR_ALGEBRA_a_rotation_matrix_is_an_orthoganal_matrix_with_determinant_1}
        A rotation matrix $R$ is an orthogonal matrix with determinant $1$.
        \end{theorem}
        \begin{proof}
        For $R^{T}R=(PS_{\theta}P^{T})^{T}PS_{\theta}P^{T}=PS_{\theta}^{T}P^{T}PS_{\theta}P^{T}=PS_{\theta}^{T}S_{\theta}P^{T}=PP^{T}=I$. But also we have $\det(R)=\det(PS_{\theta}P^T)=\det(P)\det(S_{\theta})\det(P^T)=\det(P)\det(P^{-1})=1$
        \end{proof}
        \begin{remark}
        The converse of Thm.~\ref{theorem:LINEAR_ALGEBRA_a_rotation_matrix_is_an_orthoganal_matrix_with_determinant_1} is also true.
        \end{remark}
        We now turn to the question of how to compute the rotation of $\mathbb{R}^3$ represented by a given orthogonal matrix. If $R$ is an orthogonal matrix such that $\det(R) = 1$, how do we compute the angle of rotation? First recall that the trace of a matrix is the sum of the diagonal components, $\Tr(A) = \sum_{i=1}^{n}a_{ii}$.
        \begin{theorem}
        If $A$ and $B$ are $n\times n$ matrices, then $\Tr(AB) = \Tr(BA)$
        \end{theorem}
        \begin{theorem}
        If $R$ is a rotation matrix of angle $\theta$, then $\cos(\theta) = \frac{\Tr(R) - 1}{2}$.
        \end{theorem}
        \begin{proof}
        For $\Tr(R)=\Tr(PS_{\theta}P^{-1})=\Tr(PP^{-1}S_{\theta}) = \Tr(S_{\theta})=1+2\cos(\theta)\Rightarrow \cos(\theta)=\tfrac{\Tr(R)-1}{2}$
        \end{proof}
        This doesn't tell us everything, as we still don't know $\mathbf{u}$, and $\cos(\theta) = \cos(-\theta)$, so we still don't know the sign of $\theta$. Since $R$ is an orthogonal matrix, $R^T = R^{-1}$. So if $\mathbf{u}$ lies on the axis of rotation, then $(R-R^T)\mathbf{u} = (R-R^{-1})\mathbf{u} = 0$. We can find the axis of rotation by determining the null space of $R-R^T$. 
        \begin{equation*}
            R = \begin{bmatrix*}[c] r_{11} & r_{12} & r_{13} \\ r_{21} & r_{22} & r_{23} \\ r_{31} & r_{32} & r_{33} \end{bmatrix*} \Rightarrow R-R^{T} = \begin{bmatrix*}[c] 0 & r_{12} - r_{21} & r_{13} - r_{31} \\ r_{21} - r_{12} & 0 & r_{23}-r_{32} \\ r_{31} - r_{13} & r_{32} - r_{23} & 0 \end{bmatrix*}\equiv \begin{bmatrix*}[r] 0 & \alpha & \beta \\ -\alpha & 0 & \gamma \\ -\beta & -\gamma & \phantom{-}0 \end{bmatrix*}
        \end{equation*}
        This suggests that $\mathbf{u}$ is parallel to $(-\gamma, \beta, -\alpha)^{T} = (r_{32}-r_{23}, r_{13}-r_{31}, r_{21}-r_{12})^{T}$.
        \begin{theorem}
        If $R$ is a rotation matrix such that $R\ne R^T$, then the axis of rotation of $R$ is parallel to $\mathbf{q}=(-\gamma, \beta, -\alpha)^{T} = 2\sin(\theta)\mathbf{u}$, where $\mathbf{u}$ is a unit vector about the axis of rotation.
        \end{theorem}
        \begin{proof}
        Let $R = PS_{\theta}P^T$. Then:
        \begin{equation*}
            R-R^{T}=PS_{\theta}P^{T}-PS_{\theta}^{T}P^{T}=P(S_{\theta}-S_{\theta}^{T})P^{T}=2P\begin{bmatrix}0 & -\sin(\theta) & 0 \\ \sin(\theta) & 0 & 0 \\ 0 & 0 & 0 \end{bmatrix}P^{T}= 2\sin(\theta)(\mathbf{w}\mathbf{v}^{T} - \mathbf{v}\mathbf{w}^{T})
        \end{equation*} 
        Where $\mathbf{v}$ is orthogonal to $\mathbf{u}$ and $\mathbf{w} = \mathbf{u}\times \mathbf{v}$. Thus, $\mathbf{q}=(-\gamma,\beta,-\alpha)^{T}=2\sin(\theta)\mathbf{v}\times\mathbf{w}=2\sin(\theta)\mathbf{u}$
        \end{proof}
        \begin{remark}
        What about the case when $R-R^T = 0$? When this happens either $\theta = 0$ or $\theta = \pi$. If $\theta = 0$, then this is the identity rotation and thus $R = I$, and we are done. If $R\ne I$, then $\theta = \pi$. To find out the axis of rotation, we have that:
        \begin{equation*}
            R = PS_{\pi}P^T = \begin{bmatrix*}[r]-1 & 0 & \phantom{-}0 \\ 0 & -1 & 0 \\ 0 & 0 & 1 \end{bmatrix*} = -\mathbf{v}\mathbf{v}^T-\mathbf{w}\mathbf{w}^T +\mathbf{u}\mathbf{u}^T    
        \end{equation*}
        But $\mathbf{v},\mathbf{w},$ and $\mathbf{u}$ form an orthonormal basis, and therefore $\mathbf{v}\mathbf{v}^T + \mathbf{w}\mathbf{w}^T+\mathbf{u}\mathbf{u}^T = I$. Thus, $R = -I+2\mathbf{u}\mathbf{u}^T$, so $\mathbf{u} \mathbf{u}^T = \frac{1}{2}(R+I)$. But the columns of $\mathbf{u}\mathbf{u}^T$ are parallel to $\mathbf{u}$, and therefore we can determine $\mathbf{u}$ by normalizing one of the columns of $\frac{1}{2}(R+I)$.
        \end{remark}
        \newpage
        \subsection{The Matrix Exponential}
        \begin{definition}
        If $A$ is an $n\times n$ matrix, then the exponential of $A$ is $e^{A} =\sum_{k=0}^{\infty} \frac{A^k}{k!}$.
        \end{definition}
        \begin{remark}
        Notationally, we write $A^0 = I$. For any complex-valued matrix $A$ of finite dimension, it can be shown that this sum converges.
        \end{remark}
        \begin{theorem}
        If $A$ and $P$ are complex $n\times n$ matrices and $P$ invertible, then $e^{P^{-1}AP} = P^{-1}e^{A}P$.
        \end{theorem}
        \begin{proof}
        For all $m\in \mathbb{N}$, $(P^{-1}AP)^{m} = P^{-1}A^mP$. Thus:
        \begin{equation*}
            e^{P^{-1}AP} = \sum_{k=0}^{\infty} P^{-1}\frac{A^k}{k!}P = P^{-1}\big(\sum_{k=0}^{\infty} \frac{A^k}{k!}\big)P = P^{-1}e^A P
        \end{equation*}
        \end{proof}
        \begin{theorem}
        If $0$ is the zero matrix, then $e^0 = I$.
        \end{theorem}
        \begin{theorem}
        If $A$ is an $n\times n$ matrix and $m\in \mathbb{N}$, then $A^{m} e^{A} = e^{A} A^{m}$.
        \end{theorem}
        \begin{proof}
        For $A^{m} e^{A} = A^{m} \sum_{k=0}^{\infty} \frac{A^{k}}{k!} = \sum_{k=0}^{\infty} \frac{A^{k+m}}{k!} = \big(\sum_{k=0}^{\infty} \frac{A^k}{k!}\big)A^{m}$.
        \end{proof}
        \begin{theorem}
        If $A$ is an $n\times n$ matrix, then $e^{A^{T}} = (e^{A})^{T}$.
        \end{theorem}
        \begin{proof}
        For $e^{A^{T}} = \sum_{k=0}^{\infty} \frac{(A^{T})^{k}}{k!} = \sum_{k=0}^{\infty} \frac{(A^{k})^{T}}{k!} = \big(\sum_{k=0}^{\infty} \frac{A^{k}}{k!}\big)^{T} = (e^{A})^{T}$.
        \end{proof}
        \begin{theorem}
        If $A$ and $B$ are $n\times n$ matrices and if $AB = BA$, then $Ae^{B} = e^{B} A$.
        \end{theorem}
        \begin{proof}
        For $Ae^{B} = A\sum_{k=0}^{\infty} \frac{B^{k}}{k!} = \sum_{k=0}^{\infty} A\frac{B^{k}}{k!} = \sum_{k=0}^{\infty} \frac{B^{k}}{k!}A = \big(\sum_{k=0}^{\infty} \frac{B^{k}}{k!}\big)A = e^{B}A$.
        \end{proof}
        \begin{theorem}
        If $A$ and $B$ are $n\times n$ matrices and $AB = BA$, then $e^{A}e^{B} = e^{B}e^{A}$.
        \end{theorem}
        \begin{proof}
        For:
        \begin{align*}
            e^A e^B &= e^A\sum_{k=0}^{\infty}\frac{B^k}{k!}=\sum_{k=0}^{\infty} e^A\frac{B^k}{k!}= \sum_{k=0}^{\infty} \big(\sum_{j=0}^{\infty} \frac{A^j}{j!}\big) \frac{B^k}{k!}= \sum_{k=0}^{\infty}\big(\sum_{j=0}^{\infty} \frac{A^j}{j!}\frac{B^k}{k!}\big)\\
            &=\sum_{k=0}^{\infty}\big(\sum_{j=0}^{\infty} \frac{B^k}{k!}\frac{A^j}{j!}\big)=\sum_{k=0}^{\infty}\big(\sum_{j=0}^{\infty} \frac{B^k}{k!}\big)\frac{A^j}{j!}= \sum_{k=0}^{\infty}\frac{B^k}{k!}\sum_{j=0}^{\infty}\frac{A^j}{j!}=e^{B}e^{A}
        \end{align*}
        \end{proof}
        It is NOT true that $e^{A+B}=e^{A}e^{B}$, in general. Matrix exponentiation lacks this feature.
        \begin{theorem}
        If $A$ is an $n\times n$ matrix and $s,t\in \mathbb{C}$, then $e^{A(s+t)} = e^{As}e^{At}$.
        \end{theorem}
        \begin{proof}
        For $e^{As}e^{At} = \sum_{j=0}^{\infty} \sum_{k=0}^{\infty} \frac{A^{j+k}s^jt^k}{j!k!}$. Letting $n = j+k$, so $j = n-k$, we have:
        \begin{equation*}
            \sum_{n=0}^{\infty} \sum_{k=0}^{\infty} \frac{A^n}{n!}\frac{n!}{k!(n-k)!}s^{n-k}t^k = \sum_{n=0}^{\infty}\frac{A^n}{n!}\big(\sum_{k=0}^{\infty} \frac{n!}{k!(n-k)!}s^{n-k}t^k\big)    
        \end{equation*}
        From the binomial theorem, the expression inside the parenthesis is equal to $(s+t)^n$. So we have $e^{As}e^{At}=\sum_{n=0}^{\infty} \frac{A^n(t+s)^n}{n!} = e^{A(s+t)}$.
        \end{proof}
        \begin{theorem}
        If $A$ is an $n\times n$ matrix, then $e^A$ is invertible and $(e^A)^{-1} = e^{-A}$.
        \end{theorem}
        \begin{proof}
        For $I = e^{0} = e^{A(1-1)} = e^Ae^{-A}$. Thus $(e^{A})^{-1} = e^{-A}$.
        \end{proof}
        \begin{theorem}
        If $A$ is an $n\times n$ matrix and $t\in \mathbb{R}$, then $\frac{d}{dt}\big(e^{At}\big) = Ae^{At}$.
        \end{theorem}
        \begin{proof}
        For $\underset{h\rightarrow 0}\lim \frac{e^{A(t+h)}-e^{At}}{h} = e^{At}\underset{h\rightarrow 0}\lim \frac{e^{Ah}-I}{h} = e^{At}\underset{h\rightarrow 0}\lim\big[A+\frac{A^2}{2!}h+\hdots\big] = e^{At}A = Ae^{At}$.
        \end{proof}
        \begin{theorem}
        If $A$ and $B$ are $n\times n$ matrices and $AB=BA$, then $e^{A+B} = e^{A}e^{B}$.
        \end{theorem}
        \begin{proof}
        For let $g(t) = e^{(A+B)t}e^{-Bt}e^{-At}$. Then from commutativity of $A$ and $B$, $g'(t) = 0$. But then $g(t)$ is a constant. From the definition, $g(0) = I$, and thus $g(t) = I$. So $e^{(A+B)t}e^{-Bt}e^{-At} = I$, and therefore $e^{(A+B)t} = e^{At}e^{Bt}$.
        \end{proof}
        \begin{theorem}
        If $A^{2} = 0$, then $e^{A} = I+A$.
        \end{theorem}
        \begin{proof}
        For $e^{A} = I+A+A^{2}\big(\frac{I}{2!}+\frac{A}{3!}+\hdots\big) = I+A+0 = I+A$.
        \end{proof}
        \subsection{Linear Systems of Ordinary Differential Equations}
        Consider the equation $y' = ky$, where $k$ is some constant. We can solve this via calculus using separation of variables:
        \begin{equation*}
            \frac{y'}{y} = k\Rightarrow \int \frac{y'}{y}dx = \int kdx \Rightarrow \ln(y) = kx+c \Rightarrow y = e^c e^{kx}    
        \end{equation*}
        Setting $x=0$, we have $e^c = y_0$. So $y = y_0e^{kx}$. Let us solve this a different way: Let $F(x) = e^{-kx}y$, and let $y'=kx$. Differentiating we have:
        \begin{equation*}
            F'(x)=-ke^{kx}y+e^{-kx}y'=-kye^{-kx}+e^{-kx}ky=0    
        \end{equation*}
        So $F'(x) = 0$, and therefore $F(x)$ is a constant. Setting $x=0$, we have $F(x) = y_0$. So $y = y_0e^{kx}$. This shows us that $y_0e^{kx}$ is the $only$ solution to this problem. Let:
        \begin{equation*}
            Y(t) = \begin{bmatrix} y_1(t) \\ y_2(t)\end{bmatrix}    
        \end{equation*}
        Consider $Y'(t) = AY(t)$, where $A$ is an $n\times n$ matrix. Let $F(t) = e^{-At}Y(t)$. Then $F'(t) = 0$, and $Y(t) = Y_0 e^{At}$.
        \begin{theorem}
        If $Y:\mathbb{R}\rightarrow \mathbb{R}^n$ is a differentiable function such that $Y'(t) = AY(t)$, where $A$ is a diagonalizable matrix with eigenvalues $\lambda_1,\hdots, \lambda_n$ and eigenvectors $v_1,\hdots, v_n$, then $Y(t) = \sum_{k=1}^{n} \lambda_k e^{\lambda_k t}v_k$
        \end{theorem}
    \section{Problem Sets}
        \subsection{Problem Set I}
        \begin{problem}
        Find the point on the line $y=4x$ which is closest to the point $(2,5)$.
        \end{problem}
        \begin{proof}[Solution 1]
        Given a vector $\mathbf{v}$ that is parallel to the line $y$, we know that the vector $\mathbf{w}$ from $(2,5)$ to the point $(x,y)$ that minimizes the distance from $y=4x$ to the point $(2,5)$ will satisfy $\langle \mathbf{v}, \mathbf{w}\rangle = 0$. That is:
        \begin{equation*}
            \big\langle (1,4), (2-x,5-y)\big\rangle = 0\Rightarrow 2-x+4(5-y) = 0 \Rightarrow 22 - x - 4 y = 0    
        \end{equation*}
        But $y = 4x$, and thus $22-17x = 0 \Rightarrow x= \frac{22}{17}$. The point of least distance is $\frac{22}{17}(1,4)$.
        \end{proof}
        \begin{proof}[Solution 2]
        This point is the projection of the vector $(2,5)^T$ onto $(1,4)^T$. That is:
        \begin{equation*}
            \mathbf{P} = \frac{\begin{bmatrix}2 & 5 \end{bmatrix} \begin{bmatrix} 1 \\ 4 \end{bmatrix}}{\begin{bmatrix} 1 & 4 \end{bmatrix} \begin{bmatrix} 1 \\ 4 \end{bmatrix}} \begin{bmatrix} 1 \\ 4 \end{bmatrix} = \frac{22}{17} \begin{bmatrix} 1 \\ 4\end{bmatrix}
        \end{equation*}
        \end{proof}
        \begin{problem}
        Show that $\mathbf{x}\mathbf{y}^T + \mathbf{y}\mathbf{x}^T$ is symmetric.
        \end{problem}
        \begin{proof}[Solution]
        Recall that a matrix is symmetric if it is equal to its transpose. Thus, we must show $A = A^T$. But for any $n\times n$ matrices $A$ and $B$, $(A+B)^T = A^T + B^T$, and $(AB)^T = B^T A^T$, and $(A^T)^T = A$. Thus, given our matrix $A= \mathbf{x}\mathbf{y}^T + \mathbf{y}\mathbf{x}^T$, we have that $A^T = (\mathbf{x}\mathbf{y}^T + \mathbf{y}\mathbf{x}^T)^T = (\mathbf{x}\mathbf{y}^T)^T + (\mathbf{y}\mathbf{x}^T)^T = (\mathbf{y}^T)^T\mathbf{x}^T + (\mathbf{x}^T)^T\mathbf{y}^T = \mathbf{y}\mathbf{x}^T + \mathbf{x}\mathbf{y}^T = \mathbf{x}\mathbf{y}^T + \mathbf{y}\mathbf{x}^T = A$
        \end{proof}
        \begin{problem}
        Compute the product $\begin{bmatrix*}[r] 2 & -1 \\ 3 & 1\end{bmatrix*} \begin{bmatrix*}[r] -1 & \phantom{-}2 & \phantom{-}3 & \phantom{-}1 \\ 2 & -2 & 1 & -1 \end{bmatrix*}$
        \end{problem}
        \begin{proof}[Solution]
        \begin{align*}
            \begin{bmatrix*}[r] 2 & -1 \\ 3 & 1\end{bmatrix*} \begin{bmatrix*}[r] -1 & \phantom{-}2 & \phantom{-}3 & \phantom{-}1 \\ 2 & -2 & 1 & -1 \end{bmatrix*}&=\begin{bmatrix} 2(-1)+(-1)2 & 2\cdot 2 + (-1)(-2) & 2\cdot 3 + (-1)1 & 2\cdot 1 + (-1)(-1) \\ 3(-1)+1\cdot 2 & 3\cdot 2 + 1(-2) & 3\cdot 3 + 1\cdot 1 & 3\cdot 1 + 1(-1)\end{bmatrix}\\
            &=\begin{bmatrix} -4 & 6 & 5 &3 \\ -1 & 4 & 10 & 2\end{bmatrix}
        \end{align*}
        \end{proof}
        \begin{problem}
        Find the equation of the plane that contains $P_{1}(2,2,1),P_{2}(2,3,2)$, and $P_{3}(-1,3,1)$.
        \end{problem}
        \begin{proof}[Solution]
        It suffices to find a vector normal to this plane. We have that:
        \begin{equation*}
            \overrightarrow{P_1P_2} = (0,1,1)^T \quad\quad\quad\quad \overrightarrow{P_1P_3} = (-3,1,0)^T
        \end{equation*}
        Then both vectors are parallel to the plane, and thus $\overrightarrow{P_1P_2}\times \overrightarrow{P_1P_3}=(-1,3,3)^T$ is perpendicular to the plane. Suppose $Q=(x,y,z)$ is a point in the plane. Then the relative position vector $P_1 Q = (x-2,y-2,z-1)^T$ is orthogonal to $(-1,3,3)^T$. Thus:
        \begin{align*}
            (x-2,y-2,z-1)(-1,3,3)^T &= 0\\
            \Rightarrow 2-x+3y-6+3z-3 &= 0\\
            \Rightarrow x-3y-3z +7 &= 0   
        \end{align*}
        This is the equation of the plane.
        \end{proof}
        \begin{problem}
        Let $S=\Span(\mathbf{x}_1,\mathbf{x}_{2})$, where $\mathbf{x}_{1}=(1,-1,2)^{T}$, $\mathbf{x}_{2}=(-1,2,2)^{T}$. Find a basis for $S^{\perp}$
        \end{problem}
        \begin{proof}[Solution]
        We seek a vector in $\mathbf{x}_3\in\mathbb{R}^3$ such that $\langle \mathbf{x}_3, \mathbf{x}_{i}\rangle = 0$, $i=1,2$. That is:
        \begin{equation*}
            \begin{bmatrix*}[r] 1 & -1 & \phantom{-}2 \\ 0 & 1 & 4 \end{bmatrix*}\begin{bmatrix} x_1 \\ x_2 \\ x_3 \end{bmatrix} = 0    
        \end{equation*}
        Solving gives us $x_2 = -4x_3$, $x_1=-6x_3$. $\{(-6,-4,1)\}$ is a basis.
        \end{proof}
        \begin{problem}
        For the matrix $A = \begin{bmatrix} 1 & 2 & 2 \\ -1 & -1 & 0 \end{bmatrix}$, find a basis for the following:
        \begin{enumerate}
        \begin{multicols}{4}
            \item $R(A^T)$
            \item $N(A)$
            \item $R(A)$
            \item $N(A^T)$
        \end{multicols}
        \end{enumerate}
        \end{problem}
        \begin{proof}[Solution]
        The row-echelon form of $A$ and $A^{T}$ are given below:
        \begin{align*}
            A'&=\begin{bmatrix}1&1&0\\0&1&2\end{bmatrix} & (A^{T})'&=\begin{bmatrix*}[r]1&0\\0&-1\\0&0\end{bmatrix*}
        \end{align*}
        \begin{enumerate}
            \item The rows of $A'$ give us a basis for $R(A^T)$ of $\{(1,1,0),(0,1,2)\}$
            \item $N(A) = \{x\in \mathbb{R}^3: Ax = 0\}$. Solving $A'x=0$ gives us a basis of $\{(2,-2,1)\}$
            \item The non-zero rows of $(A^{T})'$ give us a basis of $\{(1,0), (0,-1)\}$.
            \item $N(A^T)= \{x\in \mathbb{R}^2: A^T x = 0\}$. $A'x = 0$ gives us $x_1 = 0$ and $-x_2 = 0$. $N(A^T) = \{(0,0)\}$.
        \end{enumerate}
        \end{proof}
        \subsection{Problem Set II}
        \begin{problem}
        Find a point on the line $y=5x$ that is closest to the point $(1,3)$.
        \end{problem}
        \begin{proof}[Solution]
        Pick a point on the line, say $\mathbf{w} = (1,5)^T$. The point $P$ is the projection of $\mathbf{v} = (1,3)^T$ onto the line $y=5x$, and thus:
        \begin{equation*}
            P = \frac{v^T w}{w^T w} = \frac{\begin{bmatrix}1 & 5 \end{bmatrix}\begin{bmatrix}1 \\ 5\end{bmatrix}}{\begin{bmatrix}1 & 5 \end{bmatrix}\begin{bmatrix}1 \\ 5\end{bmatrix}}(1,5)^T = \frac{8}{13}(1,5)^T
        \end{equation*}
        \end{proof}
        \begin{problem}
        Is $A = xy^T - yx^T$ symmetric? ($x$ and $y$ are $n\times 1$ vectors)
        \end{problem}
        \begin{proof}[Solution]
        In general, no. For if it were, then $A-A^T = 0$. But:
        \begin{align*}
            0&=A-A^T=xy^{T}-yx^{T}-(xy^{T}-yx^{T})^{T}=xy^{T}-yx^{T}-[(xy^{T})^{T}-(yx^{T})^{T}]\\
            &=xy^T - yx^T - [yx^T - xy^T]=2xy^T-2yx^T=2A\Rightarrow xy^{T}-yx^{T}=0\Rightarrow xy^{T}=yx^{T} 
        \end{align*}
        As this is not, in general, true, $A$ is not necessarily symmetric.
        \end{proof}
        \begin{problem}
        Compute the product $\begin{bmatrix*}[r] -1 & \phantom{-}3 \\ 4 & 2 \end{bmatrix*} \begin{bmatrix*}[r] -1 & \phantom{-}1 & \phantom{-}2 & -2 \\ 2 & 3 & 1 & 1 \end{bmatrix*}$
        \end{problem}
        \begin{proof}[Solution]
        \begin{align*}
            \begin{bmatrix*}[r] -1 & \phantom{-}3 \\ 4 & 2 \end{bmatrix*} \begin{bmatrix*}[r] -1 & \phantom{-}1 & \phantom{-}2 & -2 \\ 2 & 3 & 1 & 1 \end{bmatrix*}=\begin{bmatrix*}[r] \phantom{-}1+6 & -1+9 & -2+3 & \phantom{-}2+3 \\ -4+4 & \phantom{-}4+6 & \phantom{-}8+2 & -8+2 \end{bmatrix*}=\begin{bmatrix*}[r] 7 & 8 & 1 & 5 \\ 0 & 10 & 10 & -6 \end{bmatrix*}
        \end{align*}
        \end{proof}
        \begin{problem}
        Find the equation of the plane that passes through $P_1(2,2,2), P_2(2,3,4), P_3(-1,3,3)$.
        \end{problem}
        \begin{proof}[Solution]
        $\overrightarrow{P_1P_2} = (0,1,2)^{T}$, $\overrightarrow{P_1 P_3} = (-3,1,1)^{T}$. So:
        \begin{equation*}
            \overrightarrow{N} = \begin{vmatrix*}[r] \hat{\mathbf{i}} & \hat{\mathbf{j}} & \hat{\mathbf{k}} \\ 0 & 1 & 2 \\ -3 & \phantom{-}1 & \phantom{-}1 \end{vmatrix*} = \hat{\mathbf{i}}(1-2) + \hat{\mathbf{j}}(0+6) + \hat{\mathbf{k}}(0+3)=\begin{bmatrix*}[r]-1 \\ -6 \\ 3\end{bmatrix*}   
        \end{equation*}
        For a point $P=(x,y,z)$ in the plane, $\langle \overrightarrow{P_1P}, \overrightarrow{N}\rangle = 0$. Thus, $x + 6y - 3z =0$
        \end{proof}
        \begin{problem}
        Let $S=\Span(\{(2,1,2)^T, (-2,-1,3)^T\})$. Find a basis for $S^{\perp}$.
        \end{problem}
        \begin{proof}[Solution]
        Let $A$ and it's row-echelon form be the matrices shown below. Then $S^{\perp} = N(A)$.
        \begin{align*}
            A&=\begin{bmatrix*}[r] 2 & 1 & 2 \\ -2 & -1 & \phantom{-}3\end{bmatrix*} & A'&=\begin{bmatrix} 2 & 1 & 2 \\ 0 & 0 & 5 \end{bmatrix}
        \end{align*}
        Solving for $A'x = 0$ gives us a basis of $\{(1,-2,0)\}$
        \end{proof}
        \begin{problem}
        For the matrix $A = \begin{bmatrix*}[r] 2 & 3 & 4 \\ -2 & -2 & \phantom{-}0 \end{bmatrix*}$, find a basis for the following:
        \begin{enumerate}
        \begin{multicols}{4}
            \item $R(A^T)$
            \item $N(A)$
            \item $R(A)$
            \item $N(A^T)$
        \end{multicols}
        \end{enumerate}
        \end{problem}
        \begin{proof}[Solution]
        $A$ and $A^{T}$ have the following row-echelon forms:
        \begin{align*}
            A'&=\begin{bmatrix}1&1&0\\0&1&4\end{bmatrix} & (A^{T})'=&\begin{bmatrix}1&0\\0&1\\0&0\end{bmatrix}
        \end{align*}
        \begin{enumerate}
            \item Putting $A$ into row-echelon  form and reading off the rows, we obtain the basis $\{(1,1,0),(0,1,4)\}$
            \item $N(A) = \{x\in \mathbb{R}^3:  Ax = 0\}$. This gives us a basis of $\{(4,-4,1)\}$
            \item The non-zero rows of $(A^{T})'$ give us a basis of $\{(1,0),(0,1)\}$
            \item $N(A^T) = \{x\in \mathbb{R}^2: A^Tx = 0\}$. Solving $A'^{T}x=0$ gives us $x_1 = 0$ and $x_2 = 0$. $N(A^T) = \{(0,0)\}$
        \end{enumerate}
        \end{proof}
        \subsection{Problem Set III}
        \begin{problem}
        Let $A,B,C$ be $n\times n$ matrices. Is $A = BC^T + CB^T$ symmetric?
        \end{problem}
        \begin{proof}[Solution]
        A matrix is symmetric if $A = A^T$. If $A = BC^T+CB^T$, then:
        \begin{align*}
            A^{T}=(BC^{T}+CB^{T})^{T}=(BC^{T})^{T}+(CB^{T})^{T}=(C^{T})^{T}B^{T}+(B^{T})^{T}C^{T}=CB^{T}+BC^{T}=A
        \end{align*}
        $A$ is symmetric.
        \end{proof}
        \begin{problem}
        Compute $\norm{x}_1, \norm{x}_2, \norm{x}_3$ for $x = (2,-3,1)^T$
        \end{problem}
        \begin{proof}[Solution]
        By definition, for $x\in \mathbb{R}^n$, $\norm{x}_p = (\sum_{k=1}^{n}|x_k|^p)^{1/p}$. So we have the following:
        \begin{enumerate}
            \item $\norm{x}_1 = |2|+|-3|+|1| = 2+3+1 = 6$
            \item $\norm{x}_2 = (|2|^2+|-3|^2+|1|^2)^{1/2} = (4+9+1)^{1/2} = \sqrt{14}$
            \item $\norm{x}_3 = (|2|^3+|-3|^3+|1|^3)^{1/3} = (8+27+1)^{1/3} = \sqrt[3]{36}$
        \end{enumerate}
        \end{proof}
        \newpage
        \begin{problem}
        For the matrix $A = \begin{bmatrix*}[r] \phantom{-}2 & -2 & \phantom{-}4 \\ -1 & 1 & -2 \end{bmatrix*}$, find a basis for the following:
        \begin{enumerate}
        \begin{multicols}{4}
            \item $R(A^T)$
            \item $N(A)$
            \item $R(A)$
            \item $N(A^T)$
        \end{multicols}
        \end{enumerate}
        \end{problem}
        \begin{proof}[Solution]
        $A$ and $A^{T}$ have the following row-echelon forms:
        \begin{align*}
           A'&=\begin{bmatrix*}[r]1&-1&2\\0&0&0\end{bmatrix*} & (A^{T})'&=\begin{bmatrix*}[r]-2&1\\0&0\\0&0\end{bmatrix*}
        \end{align*}
        \begin{enumerate}
            \item The non-zero rows of $A'$ give a basis of $\{(1,-1,2)\}$
            \item $N(A) = \{x\in \mathbb{R}^3: Ax = 0\}$. Solving $A'x=0$ gives a basis of $\{(1,1,0),(-2,0,1)\}$
            \item The non-zero rows of $(A^{T})'$ give a basis of $\{(-2,1)\}$
            \item $N(A^T) = \{x\in \mathbb{R}^2: A^T x = 0\}$. Solving $(A^{T})'x=0$ gives a basis of $\{(1,2)\}$
        \end{enumerate}
        \end{proof}
        \begin{problem}
        Find the least-squares solution to the following system:
        \begin{align*}
            x_{1}-x_{2}\phantom{2} &=2\\
            x_{1}+x_{2}\phantom{2} &= 0\\
            x_{1}+2x_{2} &=-1
        \end{align*}
        \end{problem}
        \begin{proof}[Solution]
        We want the solution to $A^T A x = A^T b$. We have:
        \begin{align*}
            A&= \begin{bmatrix*}[r]1&-1\\1&1\\1&2\end{bmatrix*} & b&=\begin{bmatrix*}[r]2\\0\\-1\end{bmatrix*} & A^{T}Ax&=A^{T}b\Rightarrow\begin{bmatrix}3&1\\1&9\end{bmatrix} \begin{bmatrix}x_{1}\\x_{2}\end{bmatrix}=\begin{bmatrix*}[r]1\\-6\end{bmatrix*}
        \end{align*}
        The solution is $x = \frac{1}{26}(15,-19)^T$
        \end{proof}
        \begin{problem}
        Let $\theta\in\mathbb{R}$ and let $\mathbf{x}_1 = (\cos(\theta), \sin(\theta))^{T}$, $\mathbf{x}_2 = (-\sin(\theta), \cos(\theta))^{T}$. Show that $\{\mathbf{x}_1,\mathbf{x}_2\}$ is an orthonormal basis for $\mathbb{R}^2$. Write $\mathbf{y}=(-2,3)^{T}$ as a linear combination $\mathbf{y}=c_{1} \mathbf{x}_{1}+c_{2}\mathbf{x}_{2}$
        \end{problem}
        \begin{proof}[Solution]
        They are orthonormal for $\mathbf{x}_1^T \mathbf{x}_2 = -\cos(\theta)\sin(\theta) + \cos(\theta)\sin(\theta) = 0$, and since $\norm{\mathbf{x}_1}=\norm{\mathbf{x}_2}= (\sin^2(\theta)+\cos^2(\theta))^{1/2}=1$. Let $c_{1}=\langle\mathbf{y},\mathbf{x}_{1}\rangle$ and $c_{2}=\langle\mathbf{y},\mathbf{x}_{2}\rangle$. Then $c_1 = -2\cos(\theta)+3\sin(\theta)$ and $c_2 = 2\sin(\theta)+3\cos(\theta)$. Therefore, $\mathbf{y}=(-2\cos(\theta)+3\sin(\theta))\mathbf{x}_1+(2\sin(\theta)+3\cos(\theta)\mathbf{x}_2)$
        \end{proof}
        \subsection{Problem Set IV}
        \begin{problem}
        Find the eigenvalues and associated eigenspaces of $A = \begin{bmatrix}4 & 5 \\ 2 & 1 \end{bmatrix}$
        \end{problem}
        \begin{proof}[Solution]
        We need to compute $\det(A-\lambda I)=0$. This gives us:
        \begin{equation*}
            \begin{vmatrix} 4-\lambda & 5 \\ 2 & 1-\lambda \end{vmatrix} = (4-\lambda)(1-\lambda)-10 = 0
        \end{equation*}
        The solutions to this are $\lambda_1 = 6, \lambda_2 = -1$. Solving $Ax = \lambda x$ yields the eigenspaces. We have:
        \begin{equation*}
            \begin{bmatrix} 4 & 5 \\ 2 & 1 \end{bmatrix} \begin{bmatrix} x_1 \\ x_2 \end{bmatrix}=-\begin{bmatrix} x_1 \\ x_2 \end{bmatrix}\quad\quad\quad\quad\begin{bmatrix} 4 & 5 \\ 2 & 1 \end{bmatrix} \begin{bmatrix} x_1 \\ x_2 \end{bmatrix}=6\begin{bmatrix} x_1 \\ x_2 \end{bmatrix}
        \end{equation*}
        These give solutions $x_2(-1,1)^T$ and $x_2 (\frac{5}{2},1)^T$, where $x_2$ is a free variable.
        \end{proof}
        \begin{problem}
        Show that for a $2\times 2$ matrix $A$, $\lambda^2 - \Tr(A)\lambda + \det(A) = 0$, where $\lambda$ is an eigenvalue of $A$.
        \end{problem}
        \begin{proof}[Solution]
        For we have that $\det(A-\lambda I) = 0$. But:
        \begin{equation*}
            \det(A-\lambda I)=\begin{vmatrix} a-\lambda & b \\ c & d-\lambda \end{vmatrix}=(a-\lambda)(d-\lambda)-bc=\lambda^2-(a+d)\lambda+ad-bc=\lambda^{2}-\Tr(A)\lambda+\det(A)
        \end{equation*}
        Therefore, if $\lambda$ is an eigenvalue of $A$, then $\lambda^2 - \Tr(A) \lambda + \det(A) = 0$.
        \end{proof}
        \begin{problem}
        Find the eigenvalues and associated eigenspaces for $A = \begin{bmatrix} 1 & 1 & 1 \\ 0 & 2 & 1 \\ 0 & 0 & 3\end{bmatrix}$
        \end{problem}
        \begin{proof}[Solution]
        Recall that the determinant expansion can be done along any row. Thus:
        \begin{align*}
            \det(A-\lambda I) &= \begin{vmatrix} 1-\lambda & 1 & 1 \\ 0 & 2-\lambda & 1 \\ 0 & 0 & 3-\lambda \end{vmatrix}=0\begin{vmatrix} 1 & 1 \\ 2-\lambda & 1 \end{vmatrix}-0 \begin{vmatrix} 1-\lambda & 1 \\ 0 & 1 \end{vmatrix} + (3-\lambda)\begin{vmatrix} 1-\lambda & 1 \\ 0 & 2-\lambda\end{vmatrix}\\
            &= (3-\lambda)(1-\lambda)(2-\lambda)    
        \end{align*}
        The solutions are $\lambda_1 = 1,\ \lambda_2 = 2,\ \lambda_3 = 3$. The eigenspaces correspond to the solutions of the equation $Ax = \lambda x$. Thus we get:
        \begin{equation*}
            \begin{bmatrix} 1 & 1 & 1 \\ 0 & 2 & 1 \\ 0 & 0 & 3 \end{bmatrix}\begin{bmatrix} x \\ y \\ z \end{bmatrix} = \lambda \begin{bmatrix}x \\ y \\ z\end{bmatrix}    
        \end{equation*}
        This gives 3 different equations for each value of $\lambda$.
        \begin{equation*}
            Ax=x\Rightarrow x=(1,0,0)^{T}\quad\quad\quad\quad Ax=2x\Rightarrow x=(1,1,0)^{T}\quad\quad\quad\quad Ax=3x\Rightarrow x=(1,1,1)^{T}
        \end{equation*}
        \end{proof}
        \subsection{Problem Set V}
        \begin{problem}
        Factor $\begin{bmatrix} 4 & 2 \\ 2 & 1 \end{bmatrix}$ into the form $PDP^T$, where $D$ is a diagonal and $P$ is orthogonal.
        \end{problem}
        \begin{proof}[Solution]
        The eigenvalues of $A$ are the solutions to $(4-\lambda)(1-\lambda)-4=0$: $\lambda_1 = 0$, $\lambda_2 = 5$. The eigenvectors are solutions to:
        \begin{equation*}
            \begin{bmatrix} 4 & 2 \\ 2 & 1 \end{bmatrix} \begin{bmatrix} x \\ y \end{bmatrix} = \lambda \begin{bmatrix} x \\ y \end{bmatrix}
        \end{equation*}
        Which gives us $\frac{1}{\sqrt{5}}(2,1)^T$ and $\frac{1}{\sqrt{5}}(-1,2)^T$. Thus:
        \begin{equation*}
            P = \frac{1}{\sqrt{5}}\begin{bmatrix} -1 & 2 \\ 2 & 1 \end{bmatrix}\quad\quad D = \begin{bmatrix} 0 & 0 \\ 0 & 5 \end{bmatrix}\quad\quad P^{T} = \frac{1}{\sqrt{5}}\begin{bmatrix} -1 & 2 \\ 2 & 1 \end{bmatrix}
        \end{equation*}
        \end{proof}
        \begin{problem}
        Solve the differential equation $Y'(t) = \begin{bmatrix} 4 & 2 \\ 2 & 1 \end{bmatrix} Y(t)$ with $Y(0) = \begin{bmatrix} -1 \\ 4 \end{bmatrix}$
        \end{problem}
        \begin{proof}[Solution]
        We know from the previous problem that the eigenvalues and eigenvectors are distinct, and thus $Y(t) = \alpha V_1 e^{\lambda_1 t} + \beta V_2 e^{\lambda_2 t}$ where $\lambda_{i}$ are the distinct eigenvalues, and $V_{i}$ are the distinct eigenvectors. Solving for the initial condition:
        \begin{equation*}
            \frac{1}{\sqrt{5}}\begin{bmatrix} 2 & -1 \\ 1 & 2 \end{bmatrix}\begin{bmatrix} \alpha \\ \beta \end{bmatrix}=\begin{bmatrix} -1 \\ 4 \end{bmatrix}\Rightarrow \begin{bmatrix} \alpha \\ \beta \end{bmatrix}=\frac{1}{\sqrt{5}}\begin{bmatrix} -1 & 2 \\ 2 & 1 \end{bmatrix}\begin{bmatrix} -1 \\ 4 \end{bmatrix}=\frac{1}{\sqrt{5}} \begin{bmatrix} 9 \\ 2 \end{bmatrix}
        \end{equation*}
        Thus, $Y(t) = \frac{9}{5}(-1,2)^T + \frac{2}{5} (2,1)^T e^{5t}$ 
        \end{proof}
        \begin{problem}
        Solve the following:
        \begin{enumerate}
            \item Let $A$ be an $n\times n$ complex Hermitian matrix such that $A^4=I$. What are the possible eigenvalues of $A$?
            \item If $A$ is an $n\times n$ complex matrix and $A^4 = I$, what are the possible eigenvalues?
        \end{enumerate}
        \end{problem}
        \begin{problem}
        Using least squares, find the line in $\mathbb{R}^2$ that best fits $\{(2,1),\ (3,2),\ (4,2),\ (5,3)\}$
        \end{problem}
        \begin{proof}[Solution]
        We want a line $y=mx+b$ that best fits the points. Setting up the problem, we get:
        \begin{equation*}
            \begin{bmatrix} 1 & 2 \\ 1 & 3 \\ 1 & 4 \\ 1 & 5 \end{bmatrix} \begin{bmatrix} b \\ m \end{bmatrix} = \begin{bmatrix} 1 \\ 2 \\ 2 \\ 3\end{bmatrix}   
        \end{equation*}
        This has no solution. Let $A$ be the left-most matrix. Then:
        \begin{equation*}
            A^T = \begin{bmatrix} 1 & 1 & 1 & 1 \\ 2 & 3 & 4 & 5 \end{bmatrix}\Rightarrow A^{T}A = \begin{bmatrix} 4 & 14 \\ 14 & 54 \end{bmatrix}
        \end{equation*}
        We now solve $A^{T}AX$:
        \begin{equation*}
            \begin{bmatrix} 4 & 14 \\ 14 & 54 \end{bmatrix} \begin{bmatrix} b \\ m \end{bmatrix} =  A^T \begin{bmatrix} 1 \\ 2 \\ 2 \\ 3 \end{bmatrix} = \begin{bmatrix} 8 \\ 31 \end{bmatrix}   
        \end{equation*}
        The solution gives us $y = \frac{3}{5}x-\frac{1}{10}$
        \end{proof}
        \begin{problem}
        Find the projection matrix $P$ that projects $\mathbb{R}^4$ onto the line through the origin spanned by the vector $(2,1,-1,-1)$.
        \end{problem}
        \begin{problem}
        Consider the rotation matrix $R$ shown below. Compute the axis vector $\textbf{u}$ and both the sine and cosine of the counterclockwise angle $\theta$ such that $R = R_{\theta,\textbf{u}}$
        \begin{equation*}
            R = \begin{bmatrix*}[r] -\frac{4}{9} & -\frac{7}{9} & \frac{4}{9} \\ \frac{1}{9} & \frac{4}{9} & \frac{8}{9} \\ -\frac{8}{9} & \frac{4}{9} & -\frac{1}{9} \end{bmatrix*}
        \end{equation*}
        \end{problem}
        \begin{problem}
        Find an orthonormal basis for the column space of the matrix:
        \begin{equation*}
            A = \begin{bmatrix*}[r] 1 & 1 & 1 \\ 0 & 3 & 1 \\ 2 & 2 & 2 \\ 2 & 4 & 3 \\ -1 & \phantom{-}2 & \phantom{-}0 \end{bmatrix*}
        \end{equation*}
        \end{problem}
        \begin{proof}[Solution]
        We use Gram-Schmidt to do this. Let $v_{1}=(1,0,2,2,-1)$. Normalizing gives us:
        \begin{equation*}
            e_{1} = \frac{1}{\sqrt{10}}(1,0,2,2,-1)^T    
        \end{equation*}
        We then compute:
        \begin{align*}
            (1,3,2,4,2)^T-\tfrac{(1,3,2,4,2)^T(1,0,2,2,-1)}{(1,0,2,2,-1)^T (1,0,2,2,-1)}(1,0,2,2,-1)^{T}&=(1,3,2,4,2)^{T}-\tfrac{11}{10}(1,0,2,2,-1)^{T}\\
            &=(-\tfrac{1}{10},3,-\tfrac{2}{10},\tfrac{18}{10},\tfrac{33}{10})^{T}=\tfrac{1}{10}(-1,30,-2,18,33)^{T}
        \end{align*}
        Thus:
        \begin{equation*}
            e_{2}=\tfrac{\frac{1}{10}(-1,30,-2,18,33)}{\norm{\frac{1}{10}(-1,30,-2,18,33)}}=\frac{1}{\sqrt{2318}}(-1,30,-2,18,33)
        \end{equation*}
        Finishing off, we compute:
        \begin{equation*}
            \mathbf{v}_{3}=(1,1,2,3,0)^{T}-\tfrac{(1,1,2,3,0)^T(1,0,2,2,-1)}{10}(1,0,2,2,-1)^T-\tfrac{(1,1,2,3,0)^T(1,3,2,4,2)}{34}(1,3,2,4,2,0)^{T}
        \end{equation*}
        Finally, $e_3=\frac{\textbf{v}_{3}}{\norm{\textbf{v}_{3}}}$
        \end{proof}
        \begin{problem}
        Eliminate crossterms and classify the conic section $6x^2 - 4xy+3y^2 = 1$
        \end{problem}
        \newpage
        \subsection{Problem Set VI}
        \begin{problem}
        Let $\begin{bmatrix*}[r] 1 & 0 & 3 & \vline & 1 \\ 0 & \phantom{-}1 & -2 & \vline & 3 \\ 1 & 2 & 0 & \vline & 0 \end{bmatrix*}$ be an augmented matrix.
        \begin{enumerate}
            \item Solve the system using Gaussian elimination.
            \item Express $(1,3,0)^{T}$ as a linear combination of the column vectors of the coefficient matrix.
            \item Use elementary matrices to find the LU decomposition of the coefficient matrix.
        \end{enumerate}
        \end{problem}
        \begin{proof}[Solution]
        In order,
        \begin{enumerate}
            \item 
            \begin{align*}
                \begin{bmatrix*}[r] 1 & 0 & 3 & \vline & 1 \\ 0 & \phantom{-}1 & -2 & \vline & 3 \\ 1 & 2 & 0 & \vline & 0 \end{bmatrix*} &\underset{r_{2}\leftrightarrow r_{3}\phantom{2}}{\longrightarrow} \begin{bmatrix*}[r] 1 & 0 & 3 & \vline & \phantom{-}1 \\ 1 & 2 & 0 & \vline & 0 \\ 0 & \phantom{-}1 & -2 & \vline & 3 \end{bmatrix*} \underset{r_{2}-r_{1}\phantom{3}}{\longrightarrow} \begin{bmatrix*}[r] 1 & 0 & 3 & \vline & 1 \\ 0 & \phantom{-} 2 & -3 & \vline & -1 \\ 0 & 1 & -2 & \vline & 3 \end{bmatrix*}\\
                &\underset{r_{2}\div 2\phantom{2_{2}}}{\longrightarrow} \begin{bmatrix*}[r] 1 & 0 & 3 & \vline & 1 \\ 0 & 1 & -\tfrac{3}{2} & \vline & -\tfrac{1}{2} \\ 0 & \phantom{-}1 & -2 & \vline & 3\end{bmatrix*} \underset{r_{3}-r_{2}\phantom{3}}{\longrightarrow} \begin{bmatrix*}[r] 1 & 0 & 3 & \vline & 1 \\ 0 & 1 & -\tfrac{3}{2} & \vline & -\tfrac{1}{2} \\ 0 & \phantom{-}0 & -\tfrac{1}{2} & \vline & \tfrac{7}{2} \end{bmatrix*}\\
                &\underset{r_{3}\cdot(-2)}{\longrightarrow} \begin{bmatrix*}[r] 1 & 0 & 3 & \vline & 1 \\ 0 & \phantom{-}1 & -\tfrac{3}{2} & \vline & -\tfrac{1}{2} \\ 0 & 0 & 1 & \vline & -7 \end{bmatrix*} \underset{r_{1}-3r_{3}}{\longrightarrow} \begin{bmatrix*}[r] 1 & 0 & 0 & \vline & 22 \\ 0 & 1 & -\tfrac{3}{2} & \vline & -\tfrac{1}{2} \\ 0 & \phantom{-}0 & \phantom{-}1 & \vline & -7 \end{bmatrix*}\\
                &\underset{r_{2}+\frac{3}{2}r_{3}}{\longrightarrow} \begin{bmatrix*}[r] 1 & 0 & 0 & \vline & 22 \\ 0 & 1 & 0 & \vline & -11 \\ 0 & 0 & 1 & \vline & -7 \end{bmatrix*}
            \end{align*}
            \item $(1,3,0)^{T}=22(1,0,1)^{T}-11(0,1,2)^{T}-7(3,-2,0)^{T}$
            \item
            \begin{equation*}
                A = \begin{bmatrix*}[r] 1 & 0 & 0 \\ 0 & 1 & 0 \\ 1 & 2 & 1 \end{bmatrix*} \begin{bmatrix*}[r] 1 & 0 & 3 \\ 0 & \phantom{-}1 & -2 \\ 0 & 0 & 1 \end{bmatrix*}
            \end{equation*}
        \end{enumerate}
        \end{proof} 
        \begin{problem}
        Let $A = \begin{bmatrix*}[r] 1 & 0 & 0 \\ 2 & 1 & 0 \\ 3 & 4 & 1 \end{bmatrix*}$, $B=\begin{bmatrix*}[r]1 & 0 & 0 \\ -2 & 1 & \phantom{-}0 \\ 5 & -4 & 1 \end{bmatrix*}$, and $C = \begin{bmatrix*}[r] 2 & 3 \\ -1 & 0 \\ 1 & 1 \end{bmatrix*}$. 
        \begin{enumerate}
        \begin{multicols}{3}
            \item Solve $AC+BC$
            \item Solve $AB$
            \item Does $A = B^{-1}$?
        \end{multicols}
        \end{enumerate}
        \end{problem}
        \begin{proof}[Solution]
        In order,
        \begin{enumerate}
            \item $AC+BC = (A+B)C = \begin{bmatrix*}[r] 2 & 0 & 0 \\ 0 & 2 & 0 \\ 8 & 0 & 2 \end{bmatrix*} \begin{bmatrix*}[r] 2 & 3 \\ -1 & 0 \\ 1 & 1 \end{bmatrix*} = \begin{bmatrix*}[r] 4 & 6 \\ -2 & 0 \\ 18 & 26 \end{bmatrix*}$
            \item $AB = \begin{bmatrix*}[r] 1 & 1 & 0 \\ 0 & -15 & \phantom{-}0 \\ 0 & 0 & 1 \end{bmatrix*}$
            \item No, for if $A=B^{-1}$ then $AB=I$, but this is not true.
        \end{enumerate}
        \end{proof}
        \begin{problem}
        If $A$ and $B$ are $n\times n$ invertible matrices, what is $(AB)^{-1}$?
        \end{problem}
        \begin{proof}[Solution]
        As $A^{-1}$ and $B^{-1}$ exist, and as $A$ and $B$ are of the same dimension, $B^{-1}A^{-1}$ exists. But $(B^{-1}A^{-1})(AB) = B^{-1}(A^{-1}A)B = B^{-1}IB = B^{-1}B = I$. As inverses are unique, $(AB)^{-1} = B^{-1}A^{-1}$.
        \end{proof}
        \begin{problem}
        If $A$ and $B$ are $n\times n$ matrices, what is $(A+B)^2$?
        \end{problem}
        \begin{proof}[Solution]
        $(A+B)^2 =(A+B)(A+B) = A(A+B)+B(A+B)=A^2+AB+BA+B^2$. Note: It is not true in general that $AB=BA$, and thus we cannot simplify further.
        \end{proof}
        \begin{problem}
        If $A$ and $A^T$ are $n\times n$ invertible matrices, show that $(A^T)^{-1} = (A^{-1})^T$
        \end{problem}
        \begin{proof}[Solution]
        For $A^T(A^{-1})^T = (A^{-1}A)^T = I^T = I$. As inverses are unique, $(A^T)^{-1} = (A^{-1})^T$
        \end{proof}
        \begin{problem}
        What are the solutions of:
        \begin{enumerate}
        \begin{multicols}{2}
            \item $\begin{bmatrix*}[r] 1 & 1 & 0 & 0 & \vline & -1 \\ 0 & 1 & 0 & 0 & \vline & 3 \\ 0 & 0 & 1 & 1 & \vline & 2 \\ 0 & 0 & 1 & 1 & \vline & 1 \end{bmatrix*}$
            \item $\begin{bmatrix*}[r] 1 & 1 & 0 & 0 & \vline & -1 \\ 0 & 1 & 0 & 0 & \vline & 3 \\ 0 & 0 & 1 & 1 & \vline & 1 \\ 0 & 0 & 1 & 1 & \vline & 1 \end{bmatrix*}$
        \end{multicols}
        \end{enumerate}
        \end{problem}
        \begin{proof}[Solution]
        In order,
        \begin{enumerate}
            \item No solution as the bottom two rows say $x_3 + x_4 = 2$ and $x_3 + x_4 = 1$. An impossibility.
            \item The entire space $S = \{(-4,3,x,1-x):x\in \mathbb{R}\}$.
        \end{enumerate}
        \end{proof}
        \begin{problem}
        If $A,B,$ and $C$ are $n\times n$ invertible matrices, then solve the following equations for $X$:
        \begin{enumerate}
        \begin{multicols}{3}
            \item $XA+B=C$
            \item $AX+B=X$
            \item $XA+C=X$
        \end{multicols}
        \end{enumerate}
        \end{problem}
        \begin{proof}
        In order,
        \begin{enumerate}
            \item $XA +B=C\Rightarrow XA = C-B \Rightarrow X = (C-B)A^{-1}$
            \item $AX+B = X\Rightarrow AX-X=-B \Rightarrow (A-I)X=-B \Rightarrow X = -(A-I)^{-1}B$
            \item $XA+C = X \Rightarrow XA-X = -C \Rightarrow X(A-I) = -C \Rightarrow X = -C(A-I)^{-1}$
        \end{enumerate}
        \end{proof}
        \subsection{Problem Set VII}
        \begin{problem}
        Determine the basis of the given vector space over the given field.
        \begin{enumerate}
        \begin{multicols}{3}
            \item $V=\mathbb{R}$ over $K=\mathbb{R}$
            \item $V=\mathbb{C}$ over $K=\mathbb{C}$
            \item $V=\mathbb{C}$ over $K=\mathbb{R}$
        \end{multicols}
        \end{enumerate}
        \end{problem}
        \begin{proof}[Solution]
        In order,
        \begin{enumerate}
            \item The set $\{1\}$ is a basis. Let $r \in \mathbb{R}$. Then $r=1\cdot r$.
            \item The set $\{(1,0)\}$ is a basis. Let $z\in \mathbb{Z}$. Then $z\cdot(1,0) = z$
            \item The set $\{(1,0),(0,1)\}$ is a basis. Let $z=a+bi\in \mathbb{Z}$. Then $z = a(1,0)+b(0,1)$.
        \end{enumerate}
        \end{proof}
        \begin{problem}
        What is the nullspace of an $n\times n$ matrix $A$ with real entries?
        \end{problem}
        \begin{proof}[Solution]
        The nullspace is the set $N(A) = \{X\in \mathbb{R}^n: AX = 0\}$
        \end{proof}
        \begin{problem}
        A matrix $A$ and its row reduced form $A'$ are shown below. What is the rank of $A$?
        \begin{equation*}
            A=\begin{bmatrix*}[r] 1 & 2 & 3 & 4 \\ -1 & -1 & -4 & -2 \\ 3 & 4 & 11 & 8 \end{bmatrix*} \quad\quad\quad\quad A' = \begin{bmatrix} 1 & 0 & 5 & 0 \\ 0 & 1 & -1 & 2 \\ 0 & 0 & 0 & 0 \end{bmatrix}
        \end{equation*}
        \end{problem}
        \begin{proof}[Solution]
        The rank is the dimension of the space spanned by the column vectors of the matrix. Using the row-reduced form, we see that these columns span $\mathbb{R}^2$ and thus the matrix has rank $2$.
        \end{proof}
        \begin{problem}
        What is the rank-nullity theorem?
        \end{problem}
        \begin{proof}[Solution]
        For an $n\times n$ matrix $A$, $\rk(A)+\nul(A) = n$.
        \end{proof}
        \newpage
        \subsection{Problem Set VIII}
        \begin{problem}
        Let $T:\mathbb{R}^3\rightarrow \mathbb{R}^2$ be defined by $T\begin{bmatrix} x_1 \\ x_2 \\ x_3 \end{bmatrix} = \begin{bmatrix} x_3 \\ x_1+x_2 \end{bmatrix}$.
        \begin{enumerate}
            \item Determine $\ker(T)$.
            \item Determine the dimensions of $\ker(T)$.
            \item Using the Nullity Theorem, determine the dimension of im$(T)$.
        \end{enumerate}
        \end{problem}
        \begin{proof}[Solution]
        In order,
        \begin{enumerate}
            \item If $T(x_{1},x_{2},x_{3})^{T} = 0$, then $x_3=0$ and $x_{1}+x_{2}=0$. $\ker(T)=\{(x,-x,0):x\in \mathbb{R}\}$
            \item This is a line through the origin, so the dimension is $1$ 
            \item The Nullity Theorem states that $\dim(\ker(T))+\dim(im(T)) = \dim(\mathbb{R}^3) = 3$. Thus $\dim(im(T)) = 2$.
        \end{enumerate}
        \end{proof}
        \begin{problem}
        Find the matrix representation of $T$ (Previous problem) in the standard basis of $\mathbb{R}^3$.
        \end{problem}
        \begin{proof}[Solution]
        $Te_1 = (0,1)^T$, $T e_2 = (0,1)^T$, and $Te_3 = (1,0)^T$. The matrix representation is $T=\begin{bmatrix} 0 & 0 & 1 \\ 1 & 1 & 0 \end{bmatrix}$
        \end{proof}
        \begin{problem}
        Let $P_n$ be the set of all polynomials with real coefficients of degree less than $n$. The standard basis is $\{1,x,\hdots, \ x^{n-1}\}$. Let $D:P_3 \rightarrow P_2$ be defined by $D(p) = 5\frac{dp}{dx}$. Determine the matrix representation of $D$ with respect to the standard basis.
        \end{problem}
        \begin{proof}[Solution]
        We need only check how $D$ acts on the basis vectors. $D(1) = 0+0x$, $D(x) = 1+0x$, $D(x^2) = 0+2x$. So, we have $D = \begin{bmatrix} 0 & 2 & 0 \\ 1 & 0 & 0 \end{bmatrix}$
        \end{proof}
        \begin{problem}
        Let $V$ be a vector space over $\mathbb{R}$ and let $S$ be a subspace of $V$.
        \begin{enumerate}
        \begin{multicols}{2}
            \item Define $S^{\perp}$.
            \item If $S=\Span\{ (1,2,1)^T, (1-1,2)^T\}$, what is $S^{\perp}$?
        \end{multicols}
        \end{enumerate}
        \end{problem}
        \begin{proof}[Solution]
        In order,
        \begin{enumerate}
            \item $S^{\perp} = \{x\in V: \forall y\in S, x^T y = 0\}$.
            \item Using the definition, the equations below give us $S^{\perp}=\{x_{3}(-\frac{5}{3},\frac{1}{3},1):x_{3}\in \mathbb{R}\}$
            \begin{equation*}
                \begin{bmatrix}1&2&1\\1&-1&2\end{bmatrix}\begin{bmatrix}x_1\\x_2\\x_3\end{bmatrix}=\begin{bmatrix}0\\0\end{bmatrix}\Leftrightarrow\begin{bmatrix}1&0&\frac{5}{3} \\0&1&\frac{-1}{3}\end{bmatrix}\begin{bmatrix}x_1\\x_2\\x_3\end{bmatrix}=\begin{bmatrix}0\\0\end{bmatrix}
            \end{equation*}
        \end{enumerate}
        \end{proof}
        \begin{problem}
        \
        \begin{enumerate}
            \item Let $V$ be a vector space over $\mathbb{R}$. Define an inner product.
            \item What is the difference between the standard dot product in $\mathbb{R}^n$ and an inner product? Can a vector space have more than one inner product?
            \item If $\langle x,y \rangle = xy$, what is $\norm{x}$?
        \end{enumerate}
        \end{problem}
        \begin{proof}[Solution]
        In order,
        \begin{enumerate}
            \item An inner product is a function from $\mathbb{R}\times \mathbb{R}\rightarrow \mathbb{R}$ with the following properties:
            \begin{enumerate}
                \item $\langle ax+by,z\rangle = a\langle x,z\rangle+b\langle y,z\rangle$
                \item $\langle x,y\rangle = \langle y,x \rangle$
                \item $\langle x,x\rangle \geq 0$
            \end{enumerate}
            \item An inner product is a generalization of the standard dot product. The dot product is itself an inner product, but not all inner products are dot products. There are infinitely many inner products for $\mathbb{R}$. Let $n\in \mathbb{N}$ be arbitrary, then $\langle x,y \rangle = nxy$ is an inner product.
            \item $\norm{x} = \sqrt{\langle x,x \rangle } = \sqrt{x^2}= |x|$.
        \end{enumerate}
        \end{proof}
        \begin{problem}
        Let $V = C[-1,1]$ and let $\langle f,g\rangle = \int_{-1}^{1} f(x)g(x)dx$.
        \begin{enumerate}
            \item Show that $f(x)=1$ and $g(x) = x$ are orthogonal with respect to this inner product.
            \item Determine $\norm{f}$ and $\norm{g}$.
            \item Show that $f$ and $g$ satisfy the Pythagorean Law.
        \end{enumerate}
        \end{problem}
        \begin{proof}[Solution]
        In order,
        \begin{enumerate}
        \begin{multicols}{2}
            \item $\langle 1,x\rangle=\int_{-1}^{1}xdx=0$
            \item $\norm{1} = \sqrt{ \int_{-1}^{1} dx} = \sqrt{2}$, $\norm{x} = \sqrt{\int_{-1}^{1}x^2dx} = \sqrt{\frac{2}{3}}$
        \end{multicols}
            \item $\norm{1+x}^2 = \langle 1+x,1+x\rangle = \langle 1,1\rangle + 2\langle 1,x \rangle + \langle x,x\rangle = \norm{1}^2 + \norm{x}^2$
        \end{enumerate}
        \end{proof}
        \begin{problem}
        Let $V$ be any inner product space. State and prove the Pythagorean Theorem for inner product spaces.
        \end{problem}
        \begin{proof}[Solution]
        The Pythagorean Theorem for Inner Product Spaces state that if $V$ is an inner product space with inner product $\langle, \rangle$, and if $\langle x,y\rangle = 0$, then $\norm{x}^2+\norm{y}^2 = \norm{x+y}^2$. For $\norm{x+y}^2 = \langle x+y,x+y\rangle = \langle x,x\rangle + 2\langle x,y\rangle +\langle y,y\rangle$. But as $x$ and $y$ are orthogonal, $\langle x,y \rangle = 0$. Thus $\norm{x+y}^2 = \langle x,x\rangle + \langle y,y\rangle = \norm{x}^2+\norm{y}^2$. $\norm{x+y}^2 =\norm{x}^2+\norm{y}^2$.
        \end{proof}
        \begin{problem}
        Prove that if $V$ is an inner product space and $S$ is a subspace of $V$, then $S^{\perp}$ is a subspace of $V$.
        \end{problem}
        \begin{proof}[Solution]
        We must check that $0\in S^{\perp}$ and that $S^{\perp}$ is closed under addition and scalar multiplication.
        \begin{enumerate}
            \item For all $x\in S$, $\langle 0,x \rangle = 0$, and thus $0\in S^{\perp}$.
            \item If $x,y\in S^{\perp}$ and $z\in S$, then $\langle x+y,z\rangle = \langle x,z\rangle + \langle y,z\rangle = 0+0=0$. Thus $x+y\in S^{\perp}$.
            \item If $x\in S^{\perp}$, $y\in S$, and $\alpha$ is a scalar, then $\langle \alpha x,y \rangle = \alpha \langle x,y \rangle = \alpha \cdot 0 = 0$. Thus $\alpha x \in S^{\perp}$. $S^{\perp}$ is a subspace.
        \end{enumerate}
        \end{proof}
    \section{Problems from Salem State}
    %         \documentclass[crop=false,class=book,oneside]{standalone}
%----------------------------Preamble-------------------------------%
%---------------------------Packages----------------------------%
\usepackage{geometry}
\geometry{b5paper, margin=1.0in}
\usepackage[T1]{fontenc}
\usepackage{graphicx, float}            % Graphics/Images.
\usepackage{natbib}                     % For bibliographies.
\bibliographystyle{agsm}                % Bibliography style.
\usepackage[french, english]{babel}     % Language typesetting.
\usepackage[dvipsnames]{xcolor}         % Color names.
\usepackage{listings, lstlinebgrd}      % Verbatim-Like Tools.
\usepackage{mathtools, esint, mathrsfs} % amsmath and integrals.
\usepackage{amsthm, amsfonts}           % Fonts and theorems.
\usepackage{tabularx}
\usepackage{tcolorbox}                  % Frames around theorems.
\usepackage{upgreek}                    % Non-Italic Greek.
\usepackage{paracol}                    % Two-column styling.
\usepackage{wrapfig}                    % Wrap text around figure.
\usepackage{fmtcount, etoolbox}         % For the \book{} command.
\usepackage[newparttoc]{titlesec}       % Formatting chapter, etc.
\usepackage{titletoc}                   % Allows \book in toc.
\usepackage[nottoc]{tocbibind}          % Bibliography in toc.
\usepackage[titles]{tocloft}            % ToC formatting.
\usepackage{multicol, enumitem}         % Multi-column/enumerate.
\usepackage{import}                     % Import external files.
\usepackage{pgfplots, tikz}             % Drawing/graphing tools.
\usetikzlibrary{
    calc,                   % Calculating right angles and more.
    angles,                 % Drawing angles within triangles.
    arrows.meta,            % Latex and Stealth arrows.
    quotes,                 % Adding labels to angles.
    positioning,            % Relative positioning of nodes.
    decorations.markings,   % Adding arrows in the middle of a line.
    patterns,
    arrows,
    shapes,
    shapes.geometric,
    cd,
    hobby,
    babel
}                                       % Libraries for tikz.
\pgfplotsset{compat=1.9}                % Version of pgfplots.
\usepackage[font=scriptsize,
            labelformat=simple,
            labelsep=colon]{subcaption} % Subfigure captions.
\usepackage[font={scriptsize},
            hypcap=true,
            labelsep=colon]{caption}    % Figure captions.
\usepackage{hyperref}                   % Allows for hyperlinks.
\hypersetup{
    colorlinks=true,
    linkcolor=blue,
    filecolor=magenta,
    urlcolor=Cerulean,
    citecolor=SkyBlue
}                           % Colors for hyperref.
\usepackage[toc,acronym,nogroupskip]{glossaries} % Glossaries and acronyms.
\usepackage[subpreambles=false]{standalone}      % Complileable sub files.

% Various font stuff from kiwi.
% Use this for Times text and Computer Modern math
%\usepackage{times}

% Quite nice
%\usepackage[charter, greekfamily=, greekuppercase=italicized]{mathdesign}
%\usepackage[utopia, greekuppercase=italicized]{mathdesign}    % Math is narrower

% Use this for Times text and math
%\usepackage{newtxtext}
%\usepackage[libertine,cmintegrals]{newtxmath}
%\usepackage{fix-cm}

%\usepackage{txfontsb}
% or
%\usepackage{mathptmx}

%\usepackage[scaled=0.92]{helvet}
%\renewcommand{\rmdefault}{ptm}

%\usepackage{mathpazo}    % add possibly `sc` and `osf` options
%\usepackage{eulervm}

%\usepackage{fourier}
%\renewcommand{\rmdefault}{ptm}
%\usepackage{mathptm}

%\usepackage{fontspec}
%\setmainfont{lmodern}

%\usepackage[varg]{txfonts}
%\usepackage{fouriernc}
%\usepackage{mathpazo}

%\usepackage{bookman}
%\usepackage[scaled]{uarial}
%\usepackage[scaled]{helvet}
%\renewcommand*\familydefault{\sfdefault}
%\usepackage[math]{anttor}

%\newcommand\fgeorgia{\fontfamily{jvn}\selectfont}
%\newcommand\ftimes{\fontfamily{ptm}\selectfont}
%\newcommand\fhelvetica{\fontfamily{phv}\selectfont}
%\newcommand\fcourier{\fontfamily{pcr}\selectfont}
%\newcommand\fbookman{\fontfamily{pbk}\selectfont}
%\newcommand\fnewcentury{\fontfamily{pnc}\selectfont}
%\newcommand\fpalatino{\fontfamily{ppl}\selectfont}
%\newcommand\favantgarde{\fontfamily{pag}\selectfont}
%\newcommand\fnormal{\normalfont}
%\newcommand\fsize[1]{\ifnum#1>0\fontsize{#1}{#1}\selectfont\else\normalsize\fi}
%------------------------Theorem Styles-------------------------%
% Define theorem style for default spacing and normal font.
\newtheoremstyle{normal}
    {\topsep}               % Amount of space above the theorem.
    {\topsep}               % Amount of space below the theorem.
    {}                      % Font used for body of theorem.
    {}                      % Measure of space to indent.
    {\bfseries}             % Font of the header of the theorem.
    {}                      % Punctuation between head and body.
    {.5em}                  % Space after theorem head.
    {}

% Define theorem style for default spacing with italicized font.
\newtheoremstyle{normalit}{\topsep}{\topsep}
                {\itshape}{}{\bfseries}{}{.5em}{}

% Italic header environment.
\newtheoremstyle{thmit}{\topsep}{\topsep}{}{}{\itshape}{}{0.5em}{}

% Define italicized environments.
\theoremstyle{normalit}
\newtheorem{theorem}{Theorem}[section]
\newtheorem{lemma}{Lemma}[section]
\newtheorem{corollary}{Corollary}[section]
\newtheorem{proposition}{Proposition}[section]
\newtheorem*{theorem*}{Theorem}

% Define environments with italic headers.
\theoremstyle{thmit}
\newtheorem*{solution}{Solution}
\newtheorem*{fsolution}{Solution}

% Define default environments.
\theoremstyle{normal}
\newtheorem{example}{Example}[section]
\newtheorem{definition}{Definition}[section]
\newtheorem{problem}{Problem}[section]
\newtheorem{question}{Question}[section]
\newtheorem{remark}{Remark}[section]
\newtheorem{properties}{Properties}[section]
\newtheorem{notation}{Notation}[section]
\newtheorem{axiom}{Axiom}[section]
\newtheorem*{properties*}{Properties}
\newtheorem*{remark*}{Remark}
\newtheorem*{definition*}{Definition}
\theoremstyle{plain}

% Define framed environment.
\tcbuselibrary{most}
\newtcbtheorem[use counter*=theorem]{ftheorem}{Theorem}%
    {colback=green!5,colframe=green!35!black,
     fonttitle=\bfseries\upshape}{th}

\newtcbtheorem[use counter*=example]{fdefinition}{Definition}%
    {fonttitle=\bfseries\upshape,
     colback=blue!5!white,colframe=blue!75!black}{def}

\newtcbtheorem[use counter*=example]{fexample}{Example}%
    {fonttitle=\bfseries\upshape,
     colback=red!5!white,colframe=red!75!black}{ex}

\newtcbtheorem[use counter*=notation]{fnotation}{Notation}%
    {fonttitle=\bfseries\upshape,
     colback=SeaGreen!5!white,colframe=SeaGreen!75!black}{ex}

\newtcbtheorem[use counter*=corollary]{fcorollary}{Corollary}%
    {fonttitle=\bfseries\upshape,
     colback=Orchid!5!white,colframe=Orchid!75!black}{ex}

\newenvironment{bproof}{\textit{Proof.}}{\hfill$\square$}
\tcolorboxenvironment{bproof}{blanker,breakable,left=5mm,
                             before skip=10pt,after skip=10pt,
                             borderline west={1mm}{0pt}{red}}
\tcolorboxenvironment{fsolution}
    {enhanced jigsaw,colframe=cyan,interior hidden,breakable}

%--------------------Declared Math Operators--------------------%
\DeclareMathOperator{\Refl}{Refl}           % Reflection operator.
\DeclareMathOperator{\Span}{Span}           % Span of a set of vectors.
\DeclareMathOperator{\Card}{Card}           % Cardinality of set.
\DeclareMathOperator{\Ord}{Ord}             % Ordinal of ordered set.
\DeclareMathOperator{\Tr}{Tr}               % Trace of matrix.
\DeclareMathOperator{\adjoint}{adj}         % Adjoint of matrix.
\DeclareMathOperator{\rk}{rk}               % Rank of operator.
\DeclareMathOperator{\nul}{nul}             % Null space of operator.
\DeclareMathOperator{\sgn}{sgn}             % Sign of a number.
\DeclareMathOperator{\multideg}{mutlideg}   % Multi-Degree (Graphs).
\DeclareMathOperator{\GCD}{GCD}             % Greatest common denominator.
\DeclareMathOperator{\LM}{LM}               % Leading monomial
\DeclareMathOperator{\LC}{LC}               % Leading coefficient.
\DeclareMathOperator{\LT}{LT}               % Leading term.
\DeclareMathOperator{\LCM}{LCM}             % Least common multiple.
\DeclareMathOperator{\Mon}{Mon}             % Monomial.
\DeclareMathOperator{\Spec}{Spec}           % Spectrum.
\DeclareMathOperator{\proj}{proj}           % Projection.
\DeclareMathOperator{\comp}{comp}           % Component.
\DeclareMathOperator{\sinc}{sinc}           % Sinc function.
\DeclareMathOperator{\Ima}{Im}              % Image of operator.
\DeclareMathOperator{\Prin}{Prin}           % Principal value.
\DeclareMathOperator{\Mod}{mod}             % Modulus.
%------------------------New Commands---------------------------%
\DeclarePairedDelimiter\norm{\lVert}{\rVert}
\DeclarePairedDelimiter\ceil{\lceil}{\rceil}
\DeclarePairedDelimiter\floor{\lfloor}{\rfloor}
\newcommand*\diff{\mathop{}\!\mathrm{d}}
\newcommand*\Diff[1]{\mathop{}\!\mathrm{d^#1}}
\renewcommand{\mod}{\ \Mod}
\renewcommand*{\glstextformat}[1]{\textcolor{RoyalBlue}{#1}}
\renewcommand{\glsnamefont}[1]{\textbf{#1}}
\renewcommand\labelitemii{$\circ$}
\renewcommand\thesubfigure{\arabic{chapter}.\arabic{figure}}
\renewcommand\thesubfigure{%
    \arabic{chapter}.\arabic{figure}.\arabic{subfigure}}
\addto\captionsenglish{\renewcommand{\figurename}{Fig.}}
%------------------------Book Command---------------------------%
\makeatletter
\renewcommand\@pnumwidth{1cm}
\newcounter{book}
\renewcommand\thebook{\@Roman\c@book}
\newcommand\book{%
    \if@openright
        \cleardoublepage
    \else
        \clearpage
    \fi
    \thispagestyle{plain}%
    \if@twocolumn
        \onecolumn
        \@tempswatrue
    \else
        \@tempswafalse
    \fi
    \null\vfil
    \secdef\@book\@sbook
}
\def\@book[#1]#2{%
    \ifnum \c@secnumdepth >-3\relax
        \refstepcounter{book}%
        \addcontentsline{toc}{book}{
            \bookname\ \thebook:\hspace{1em}#1
        }
    \else
        \addcontentsline{toc}{book}{#1}%
    \fi
    \markboth{}{}%
    {\centering
     \interlinepenalty \@M
     \normalfont
     \ifnum \c@secnumdepth >-2\relax
       \huge\bfseries \bookname\nobreakspace\thebook
       \par
       \vskip 20\p@
     \fi
     \Huge \bfseries #2\par}%
    \@endbook}
\def\@sbook#1{%
    {\centering
     \interlinepenalty \@M
     \normalfont
     \Huge \bfseries #1\par}%
    \@endbook}
\def\@endbook{
    \vfil\newpage
        \if@twoside
            \if@openright
                \null
                \thispagestyle{empty}%
                \newpage
            \fi
        \fi
        \if@tempswa
            \twocolumn
        \fi
}
\newcommand*\l@book[2]{%
    \ifnum \c@tocdepth >-2\relax
        \addpenalty{-\@highpenalty}%
        \addvspace{2.25em \@plus\p@}%
        \setlength\@tempdima{3em}%
        \begingroup
            \parindent \z@ \rightskip \@pnumwidth
            \parfillskip -\@pnumwidth
            {
                \leavevmode
                \Large \bfseries #1\hfil \hb@xt@\@pnumwidth{
                    \hss #2
                }
            }
            \par
            \nobreak
            \global\@nobreaktrue
            \everypar{\global\@nobreakfalse\everypar{}}%
        \endgroup
    \fi}
\newcommand\bookname{Book}
\renewcommand{\thebook}{\texorpdfstring{\Numberstring{book}}{book}}
\providecommand*{\toclevel@book}{-2}
\makeatother
\titlecontents{chapter}[0pt]
    {\bfseries}
    {\chaptername\ \thecontentslabel:\quad}
    {}
    {\hfill\contentspage}
\titleformat{\part}[display]
    {\Large\bfseries}
    {\partname\nobreakspace\thepart}
    {0mm}
    {\Huge\bfseries}
    \titlecontents{part}[0pt]
    {\large\bfseries}
    {\partname\ \thecontentslabel: \quad}
    {}
    {\hfill\contentspage}
\newcommand{\MarkRightAngle}[4][.3cm]
    {\coordinate (tempa) at ($(#3)!#1!(#2)$);
     \coordinate (tempb) at ($(#3)!#1!(#4)$);
     \coordinate (tempc) at ($(tempa)!0.5!(tempb)$);%midpoint
     \draw (tempa) -- ($(#3)!2!(tempc)$) -- (tempb);}
%--------------------------LENGTHS------------------------------%
% Spacings for the Table of Contents.
\addtolength{\cftsecnumwidth}{1ex}
\addtolength{\cftsubsecindent}{1ex}
\addtolength{\cftsubsecnumwidth}{1ex}
\addtolength{\cftfignumwidth}{1ex}
\addtolength{\cfttabnumwidth}{1ex}

% Spacing for multi-column and enumerate environments.
\setlength{\multicolsep}{6pt}
\setlist[enumerate]{itemsep=0pt,topsep=3pt}

% Indent and paragraph spacing.
\setlength{\parindent}{0em}
\setlength{\parskip}{0em}
%----------------------------GLOSSARY-------------------------------%
\makeglossaries
\loadglsentries{../../glossary}
\loadglsentries{../../acronym}
%--------------------------Main Document----------------------------%
\begin{document}
    \ifx\ifmathcourses\undefined
        \pagenumbering{roman}
        \title{Number Theory}
        \author{Ryan Maguire}
        \date{\vspace{-5ex}}
        \maketitle
        \tableofcontents
        \clearpage
        \chapter*{Number Theory}
        \addcontentsline{toc}{chapter}{Number Theory}
        \markboth{}{NUMBER THEORY}
        \vspace{10ex}
        \setcounter{chapter}{1}
        \pagenumbering{arabic}
    \else
        \chapter{Number Theory}
    \fi
    \section{Exams from UML 92.413: Spring 2017}
        \subsection{Exam I}
            \begin{problem}
                Find an integer $n$ such that $\gcd(n,4)=2$ and
                $\gcd(n,6)=3$, or prove that no such integer exists.
            \end{problem}
            \begin{proof}[Solution 1]
                If $\gcd(n,4)=2$, then ${2}\vert{n}$, and thus
                $\exists_{k\in\mathbb{Z}}:n=2k$. But
                $\gcd(n,6)=\gcd(2k,2\cdot 3)=2\gcd(k,3)$. But
                $\gcd(n,6)=3$, and therefore $2\gcd(k,3)=3$, a
                contradiction as $3$ is odd. No such $n$ exists.
            \end{proof}
            \begin{proof}[Solution 2]
                If $\gcd(n,4)=2$, then ${2}\vert{n}$, and thus
                $\exists_{j\in\mathbb{Z}}:n=2j$. If $\gcd(n,6)=3$,
                then ${3}\vert{n}$. Therefore
                $\exists_{k\in\mathbb{Z}}:n=3k$. But then $2j=3k$.
                As $3$ is odd, $k$ must be even. Therefore,
                $\exists_{m\in\mathbb{Z}}:k=2m$. But then
                $n=3k=3(2m)=6m$. Thus, ${6}\vert{n}$. But then
                $\gcd(n,6)=6$, a contradiction as $\gcd(n,6)=3$.
            \end{proof}
            \begin{proof}[Solution 3]
                If $\gcd(n,4)=2$, then ${2}\vert{n}$, and thus
                $\exists_{k\in\mathbb{Z}}:n=2k$. But $\gcd(n,6)=3$,
                and therefore $\exists_{x,y\in\mathbb{Z}}:nx+6y=3$.
                But $nx+6y=2kx+6y=2(kx+3y)$, and $nx+6y=3$, and
                therefore $2(nx+3y)=3$, a contradiction as $3$ is
                odd. No such $n$ exists.
            \end{proof}
            \begin{problem}
                Prove or disprove the following:
                \begin{enumerate}
                    \begin{multicols}{2}
                        \item ${20}\vert{300}$
                        \item If $a>0$, then ${a}\vert{1}$
                        \item $\forall_{a,b>0}$, either
                            ${a}\vert{b}$ or ${b}\vert{a}$
                        \item $\forall_{a,b,c>0}$, if ${a}\vert{b}$
                            and ${a}\vert{(b+c)}$,
                            then ${a}\vert{(c-b)}$
                        \item $\forall_{a,b,c>0}$, if ${a}\vert{b}$
                            and ${a}\vert{c}$, then 
                            ${a}\vert{(b^{2}+c^{2})}$
                        \item $\forall_{a,b,c>0}$, if ${a}\vert{b}$
                            and $a\vert{(b^{2}+c^{2})}$, then
                            ${a}\vert{c}$
                        \item $\forall_{a,b,c>0}$, if ${a}\vert{b}$
                            and ${b}\vert{c}$, then $a\leq c$
                        \item If $a,b,c>0$, then
                            $\gcd(a,bc)\geq\gcd(a,b)$
                        \item If $a,b,c>0$, then
                            $\gcd(a,c-a)=\gcd(a+c,c)$
                        \item If $p$ is prime and
                            ${p^{3}}\vert{abc}$, then ${p}\vert{a}$
                        \item If $a+b$ is prime, then $ab$ is even.
                        \item If $a$ and $b$ are composite, then
                            $a+b$ is composite.
                        \item If $p$ is prime and ${p}\vert{a^{2}}$,
                            then $p^{2}\vert{a^{2}}$
                        \item If $0<b<a$, then $a^{2}-b^{2}$ is
                            composite.
                    \end{multicols}
                \end{enumerate}
            \end{problem}
            \begin{proof}[Solution]
                \
                \begin{enumerate}
                    \item True, for $300=20\cdot 15$
                    \item False, for $2>0$, but $2$ does not divide
                        $1$
                    \item False, for $5>0$ and $7>0$ but $5$ does
                        not divide $7$ and $7$ does not
                        divide $5$ for they are prime.
                    \item True. If ${a}\vert{b}$, then
                        $\exists_{n\in\mathbb{Z}}:b=na$. If
                        ${a}\vert{(b+c)}$, then
                        $\exists_{m\in\mathbb{Z}}:b+c=ma$. But we
                        have that $c=ma-b=ma-na=a(m-n)$,
                        and therefore ${a}\vert{c}$. But then
                        $b-c=a(2n-m)$, so ${a}\vert{(b-c)}$
                    \item True. If ${a}\vert{b}$ then
                        $\exists_{n\in\mathbb{Z}}:b=an$.
                        If ${a}\vert{c}$, then
                        $\exists_{m\in\mathbb{Z}}:c=am$. But then
                        $b^{2}+c^{2}=a^{2}n^{2}+a^{2}m^{2}%
                         =a(an^{2}+am^{2})$, and therefore
                        ${a}\vert{(b^{2}+c^{2})}$
                    \item False. Let $a=4$, $b=8$, and $c=6$.
                        Then $b=2a$, $b^{2}+c^{2}=25a$, but $4$
                        does not divide $6$.
                    \item True. If $a,b,c>0$ and ${a}\vert{b}$,
                        then $\exists_{n\in\mathbb{N}}:b=na$,
                        and therefore $a\leq b$. If
                        ${b}\vert{c}$, then
                        $\exists_{m\in\mathbb{N}}:c=mb$. But then
                        $b\leq c$. But $a\leq b$, and therefore
                        $a\leq c$
                    \item True. If ${n}\vert{a}$ and ${n}\vert{b}$,
                        then ${n}\vert{a}$ and ${n}\vert{bc}$, and
                        therefore $\gcd(a,b)\leq\gcd(a,bc)$
                    \item True. If ${n}\vert{a}$ and
                        ${n}\vert{(c-a)}$, then ${n}\vert{c}$. But
                        then ${n}\vert{(a+c)}$. If ${n}\vert{c}$
                        and ${n}\vert{(a+c)}$, then ${n}\vert{c}$.
                        But then ${n}\vert{(c-a)}$, and therefore
                        $\gcd(a,c-a)=\gcd(a+c,c)$
                    \item False. Let $a=6$ and $c=10$. Then
                        $\gcd(a,b)=\gcd(6,10)=2$, and
                        $\gcd(a+c,c-a)=\gcd(16,4)=4$.
                    \item False. Let $p=5$, $a=2$, $b=5$, and $c=25$.
                        Then $p$ is prime, ${p^{3}}\vert{abc}$, but
                        $5$ does not divide $2$
                    \item False. Let $a=b=1$. Then $a+b=2$, which
                        is prime, but $ab=1$, which is odd.
                    \item False. Let $a=9$, and $b=8$. Then $a$ and
                        $b$ are composite, but $a+b=17$,
                        which is prime.
                    \item True. If ${p}\vert{a^{2}}$, then
                        $\exists_{n\in\mathbb{Z}}:a^{2}=np$. But, as
                        $p$ is prime, $a$ does not divide $p$, and
                        therefore $a=\frac{n}{a}p$. That is,
                        ${p}\vert{a}$. Therefore, ${p}\vert{a^{2}}$
                    \item False. Let $a=9$ and $b=8$. Then
                        $9^{2}-8^{2}=81-64=17$, which is prime.
                \end{enumerate}
            \end{proof}
            \begin{problem}
                Use Euclid's Algorithm to compute $\gcd(201,62)$.
            \end{problem}
            \begin{proof}[Solution]
                \begin{align*}
                    201&=62\cdot 3+15\\
                    62&=15\cdot 5+2\\
                    15&=2\cdot 7+1\\
                    2&=1\cdot 2+0
                \end{align*}
                $\gcd(201,62)=1$
            \end{proof}
            \begin{problem}
                Find all integer solutions to $201x+62y=1$
            \end{problem}
            \begin{proof}[Solution 1]
                From the previous problem, we have:
                \begin{equation*}
                    3+\frac{1}{4+\frac{1}{7}}=\frac{94}{29}
                \end{equation*}
                So $201(29)+62(-94)=1$. The general solution
                is therefore $x=29+62k$ and $y=-94-201k$ for
                all $k\in\mathbb{Z}$.
            \end{proof}
            \begin{proof}[Solution 2]
                From the previous problem, we have:
                \begin{align*}
                    1&=15-2\cdot7&
                    &=201\cdot(1+28)+62\cdot(-3-7-84)\\
                    &=(201-63\cdot3)-(62-15\cdot4)\cdot7&
                    &=201\cdot29+62\cdot(-94)\\
                    &=(201-62\cdot3)-(62-(201-62\cdot3)\cdot4)\cdot7
                \end{align*}
                The general solution is $x=29+62k$ and $y=-94-201k$
            \end{proof}
            \begin{problem}
                Solve the following:
                \begin{enumerate}
                    \begin{multicols}{2}
                        \item ${300^{3}+400^{4}}\mod{6}$
                        \item ${300^{3}+400^{4}}\mod{5}$
                        \item ${3^{1}}\mod{10}$
                        \item Last digit of $333^{222}$
                        \item ${1212^{11}}\mod{13}$
                        \item If $m$ is odd and $66\equiv{4}\mod{m}$,
                            what is $m$?
                        \item ${(21)(34)+765}\mod{9}$
                        \item ${48^{237}}\mod{4}$
                        \item ${3+3^{3}+3^{5}+3^{7}+3^{9}}\mod{8}$
                        \item If $2x\equiv{5}\mod{21}$, what is
                            ${x}\mod{21}$?
                    \end{multicols}
                \end{enumerate}
            \end{problem}
            \begin{proof}[Solution]
                \par\hfill\par
                \begin{enumerate}
                    \item We have
                        ${6}\vert{300}\Rightarrow%
                         300^{3}\equiv{0}\mod{6}$.
                        Also
                        $400\equiv{4}\mod{6}\Rightarrow%
                         400^{4}\equiv{4^{4}}\mod{6}%
                         ={256}\mod{6}\equiv{4}$
                    \item
                        ${5}\vert{300}\Rightarrow{300^{3}}%
                         \equiv{0}\mod{5}$,
                        ${5}\vert{400}\Rightarrow{400^{4}}%
                         \equiv{0}\mod{5}$.
                        ${300^{3}+400^{4}}\equiv{0}\mod{5}$
                    \item
                        ${3}\cdot{7}={21}\equiv{1}\mod{10}%
                         \Rightarrow{3^{-1}}\equiv{7}\mod{10}$
                    \item
                        ${333}\equiv{3}\mod{10}\Rightarrow%
                         {333^{222}}\equiv{3^{222}}\mod{10}$. But
                        $3^{222}=9(3^{2})^{110}$, and
                        $9^{110}={81^{55}}\equiv{1}\mod{10}$.
                        So, ${333^{222}}\equiv{9}\mod{10}$
                    \item
                        ${1212}\equiv{3}\mod{13}$, and
                        $3^{11}=9\cdot((3^{3})^{3}={9}\cdot{27}^{3}$.
                        But ${27}\equiv{1}\mod{13}$. So
                        ${1212^{11}}\equiv{9}\mod{13}$
                    \item ${62}\equiv{0}\mod{m}$. But
                        $62={31}\cdot{2}$. $m=31$
                    \item ${21}\equiv{3}\mod{9}$,
                        ${34}\equiv{7}\mod{9}$, and
                        ${765}\equiv{0}\mod{9}$. So we have
                        ${3}\cdot{7}={21}\equiv{3}\mod{9}$
                    \item ${48}\equiv{0}\mod{4}$.
                    \item $3^{2}\equiv{1}\mod{8}$,
                        $3^{5}\equiv{{3}\cdot{3^{4}}}\mod{8}%
                         \equiv{3}\mod{8}$,
                        $3^{7}\equiv{{3}\cdot{3^{6}}}\mod{8}%
                         \equiv{3}\mod{8}$, and finally
                        ${3^{9}}\equiv{{3}\cdot{3^{8}}}\mod{8}%
                         \equiv{3}\mod{8}$. So we have
                        $3+3+3+3+3={15}\equiv{7}\mod{8}$
                    \item If ${2x}\equiv{5}\mod{21}$, then
                        $x\equiv{{5}\cdot{2^{-1}}}\mod{21}$.
                        But ${2^{-1}}\equiv{11}\mod{21}$, so
                        ${x}\equiv{{5}\cdot{11}}\mod{21}$ and
                        ${5}\cdot{11}={55}\equiv{13}\mod{21}$.
                \end{enumerate}
            \end{proof}
            \begin{problem}
                Find all integers $n,m\geq{0}$ such that
                $5^{n}-4^{m}=1$.
            \end{problem}
            \begin{proof}[Solution]
                $n=m=1$ is a solution since
                $5-4=1$. Suppose there is another solution.
                Note that $5^{0}-4^{0}=1-1=0$,
                $5^{1}-4^{0}=5-1=4$, and $5^{0}-4^{1}=1-4=-3$.
                If $m\geq{1}$ and $n\geq{2}$, we have
                $5^{n}-4^{m}>5^{n}-1\geq25-4=21>1$. If $m\geq{2}$,
                then $4^{m}$ is divisible by 8, and thus
                $4^{m}\mod{8}=0$. If $(n,m)$ is a solution, then
                $1=5^{n}-4^{n}\equiv{5^{n}}\mod{8}$, and thus
                $5^{n}\equiv{1}\mod{8}$. If $n$ is even, then
                $5^{2k}=25^{k}\equiv{1}\mod{8}$. If $n$ is odd, then
                $5^{2k+1}\equiv{5}\mod{8}$. Thus $n$ must be even if it
                is a solution. But if $5^{n}-4^{m}=1$,
                then $5^{n}-4^{m}\equiv{1}\mod{3}$. But
                $5^{n}-4^{m}\equiv{(-1)^{n}-(1)^{m}}\mod{3}$. But $n$ is
                even, and thus $5^{n}-4^{m}\equiv{0}\mod{8}$. But then
                $1\equiv{0}\mod{3}$, a contradiction. Thus, there is
                no other solution. $n=m=1$ is the only solution.
            \end{proof}
\end{document}
    %         \documentclass[crop=false,class=book,oneside]{standalone}
%----------------------------Preamble-------------------------------%
%---------------------------Packages----------------------------%
\usepackage{geometry}
\geometry{b5paper, margin=1.0in}
\usepackage[T1]{fontenc}
\usepackage{graphicx, float}            % Graphics/Images.
\usepackage{natbib}                     % For bibliographies.
\bibliographystyle{agsm}                % Bibliography style.
\usepackage[french, english]{babel}     % Language typesetting.
\usepackage[dvipsnames]{xcolor}         % Color names.
\usepackage{listings, lstlinebgrd}      % Verbatim-Like Tools.
\usepackage{mathtools, esint, mathrsfs} % amsmath and integrals.
\usepackage{amsthm, amsfonts}           % Fonts and theorems.
\usepackage{tabularx}
\usepackage{tcolorbox}                  % Frames around theorems.
\usepackage{upgreek}                    % Non-Italic Greek.
\usepackage{paracol}                    % Two-column styling.
\usepackage{wrapfig}                    % Wrap text around figure.
\usepackage{fmtcount, etoolbox}         % For the \book{} command.
\usepackage[newparttoc]{titlesec}       % Formatting chapter, etc.
\usepackage{titletoc}                   % Allows \book in toc.
\usepackage[nottoc]{tocbibind}          % Bibliography in toc.
\usepackage[titles]{tocloft}            % ToC formatting.
\usepackage{multicol, enumitem}         % Multi-column/enumerate.
\usepackage{import}                     % Import external files.
\usepackage{pgfplots, tikz}             % Drawing/graphing tools.
\usetikzlibrary{
    calc,                   % Calculating right angles and more.
    angles,                 % Drawing angles within triangles.
    arrows.meta,            % Latex and Stealth arrows.
    quotes,                 % Adding labels to angles.
    positioning,            % Relative positioning of nodes.
    decorations.markings,   % Adding arrows in the middle of a line.
    patterns,
    arrows,
    shapes,
    shapes.geometric,
    cd,
    hobby,
    babel
}                                       % Libraries for tikz.
\pgfplotsset{compat=1.9}                % Version of pgfplots.
\usepackage[font=scriptsize,
            labelformat=simple,
            labelsep=colon]{subcaption} % Subfigure captions.
\usepackage[font={scriptsize},
            hypcap=true,
            labelsep=colon]{caption}    % Figure captions.
\usepackage{hyperref}                   % Allows for hyperlinks.
\hypersetup{
    colorlinks=true,
    linkcolor=blue,
    filecolor=magenta,
    urlcolor=Cerulean,
    citecolor=SkyBlue
}                           % Colors for hyperref.
\usepackage[toc,acronym,nogroupskip]{glossaries} % Glossaries and acronyms.
\usepackage[subpreambles=false]{standalone}      % Complileable sub files.

% Various font stuff from kiwi.
% Use this for Times text and Computer Modern math
%\usepackage{times}

% Quite nice
%\usepackage[charter, greekfamily=, greekuppercase=italicized]{mathdesign}
%\usepackage[utopia, greekuppercase=italicized]{mathdesign}    % Math is narrower

% Use this for Times text and math
%\usepackage{newtxtext}
%\usepackage[libertine,cmintegrals]{newtxmath}
%\usepackage{fix-cm}

%\usepackage{txfontsb}
% or
%\usepackage{mathptmx}

%\usepackage[scaled=0.92]{helvet}
%\renewcommand{\rmdefault}{ptm}

%\usepackage{mathpazo}    % add possibly `sc` and `osf` options
%\usepackage{eulervm}

%\usepackage{fourier}
%\renewcommand{\rmdefault}{ptm}
%\usepackage{mathptm}

%\usepackage{fontspec}
%\setmainfont{lmodern}

%\usepackage[varg]{txfonts}
%\usepackage{fouriernc}
%\usepackage{mathpazo}

%\usepackage{bookman}
%\usepackage[scaled]{uarial}
%\usepackage[scaled]{helvet}
%\renewcommand*\familydefault{\sfdefault}
%\usepackage[math]{anttor}

%\newcommand\fgeorgia{\fontfamily{jvn}\selectfont}
%\newcommand\ftimes{\fontfamily{ptm}\selectfont}
%\newcommand\fhelvetica{\fontfamily{phv}\selectfont}
%\newcommand\fcourier{\fontfamily{pcr}\selectfont}
%\newcommand\fbookman{\fontfamily{pbk}\selectfont}
%\newcommand\fnewcentury{\fontfamily{pnc}\selectfont}
%\newcommand\fpalatino{\fontfamily{ppl}\selectfont}
%\newcommand\favantgarde{\fontfamily{pag}\selectfont}
%\newcommand\fnormal{\normalfont}
%\newcommand\fsize[1]{\ifnum#1>0\fontsize{#1}{#1}\selectfont\else\normalsize\fi}
%------------------------Theorem Styles-------------------------%
% Define theorem style for default spacing and normal font.
\newtheoremstyle{normal}
    {\topsep}               % Amount of space above the theorem.
    {\topsep}               % Amount of space below the theorem.
    {}                      % Font used for body of theorem.
    {}                      % Measure of space to indent.
    {\bfseries}             % Font of the header of the theorem.
    {}                      % Punctuation between head and body.
    {.5em}                  % Space after theorem head.
    {}

% Define theorem style for default spacing with italicized font.
\newtheoremstyle{normalit}{\topsep}{\topsep}
                {\itshape}{}{\bfseries}{}{.5em}{}

% Italic header environment.
\newtheoremstyle{thmit}{\topsep}{\topsep}{}{}{\itshape}{}{0.5em}{}

% Define italicized environments.
\theoremstyle{normalit}
\newtheorem{theorem}{Theorem}[section]
\newtheorem{lemma}{Lemma}[section]
\newtheorem{corollary}{Corollary}[section]
\newtheorem{proposition}{Proposition}[section]
\newtheorem*{theorem*}{Theorem}

% Define environments with italic headers.
\theoremstyle{thmit}
\newtheorem*{solution}{Solution}
\newtheorem*{fsolution}{Solution}

% Define default environments.
\theoremstyle{normal}
\newtheorem{example}{Example}[section]
\newtheorem{definition}{Definition}[section]
\newtheorem{problem}{Problem}[section]
\newtheorem{question}{Question}[section]
\newtheorem{remark}{Remark}[section]
\newtheorem{properties}{Properties}[section]
\newtheorem{notation}{Notation}[section]
\newtheorem{axiom}{Axiom}[section]
\newtheorem*{properties*}{Properties}
\newtheorem*{remark*}{Remark}
\newtheorem*{definition*}{Definition}
\theoremstyle{plain}

% Define framed environment.
\tcbuselibrary{most}
\newtcbtheorem[use counter*=theorem]{ftheorem}{Theorem}%
    {colback=green!5,colframe=green!35!black,
     fonttitle=\bfseries\upshape}{th}

\newtcbtheorem[use counter*=example]{fdefinition}{Definition}%
    {fonttitle=\bfseries\upshape,
     colback=blue!5!white,colframe=blue!75!black}{def}

\newtcbtheorem[use counter*=example]{fexample}{Example}%
    {fonttitle=\bfseries\upshape,
     colback=red!5!white,colframe=red!75!black}{ex}

\newtcbtheorem[use counter*=notation]{fnotation}{Notation}%
    {fonttitle=\bfseries\upshape,
     colback=SeaGreen!5!white,colframe=SeaGreen!75!black}{ex}

\newtcbtheorem[use counter*=corollary]{fcorollary}{Corollary}%
    {fonttitle=\bfseries\upshape,
     colback=Orchid!5!white,colframe=Orchid!75!black}{ex}

\newenvironment{bproof}{\textit{Proof.}}{\hfill$\square$}
\tcolorboxenvironment{bproof}{blanker,breakable,left=5mm,
                             before skip=10pt,after skip=10pt,
                             borderline west={1mm}{0pt}{red}}
\tcolorboxenvironment{fsolution}
    {enhanced jigsaw,colframe=cyan,interior hidden,breakable}

%--------------------Declared Math Operators--------------------%
\DeclareMathOperator{\Refl}{Refl}           % Reflection operator.
\DeclareMathOperator{\Span}{Span}           % Span of a set of vectors.
\DeclareMathOperator{\Card}{Card}           % Cardinality of set.
\DeclareMathOperator{\Ord}{Ord}             % Ordinal of ordered set.
\DeclareMathOperator{\Tr}{Tr}               % Trace of matrix.
\DeclareMathOperator{\adjoint}{adj}         % Adjoint of matrix.
\DeclareMathOperator{\rk}{rk}               % Rank of operator.
\DeclareMathOperator{\nul}{nul}             % Null space of operator.
\DeclareMathOperator{\sgn}{sgn}             % Sign of a number.
\DeclareMathOperator{\multideg}{mutlideg}   % Multi-Degree (Graphs).
\DeclareMathOperator{\GCD}{GCD}             % Greatest common denominator.
\DeclareMathOperator{\LM}{LM}               % Leading monomial
\DeclareMathOperator{\LC}{LC}               % Leading coefficient.
\DeclareMathOperator{\LT}{LT}               % Leading term.
\DeclareMathOperator{\LCM}{LCM}             % Least common multiple.
\DeclareMathOperator{\Mon}{Mon}             % Monomial.
\DeclareMathOperator{\Spec}{Spec}           % Spectrum.
\DeclareMathOperator{\proj}{proj}           % Projection.
\DeclareMathOperator{\comp}{comp}           % Component.
\DeclareMathOperator{\sinc}{sinc}           % Sinc function.
\DeclareMathOperator{\Ima}{Im}              % Image of operator.
\DeclareMathOperator{\Prin}{Prin}           % Principal value.
\DeclareMathOperator{\Mod}{mod}             % Modulus.
%------------------------New Commands---------------------------%
\DeclarePairedDelimiter\norm{\lVert}{\rVert}
\DeclarePairedDelimiter\ceil{\lceil}{\rceil}
\DeclarePairedDelimiter\floor{\lfloor}{\rfloor}
\newcommand*\diff{\mathop{}\!\mathrm{d}}
\newcommand*\Diff[1]{\mathop{}\!\mathrm{d^#1}}
\renewcommand{\mod}{\ \Mod}
\renewcommand*{\glstextformat}[1]{\textcolor{RoyalBlue}{#1}}
\renewcommand{\glsnamefont}[1]{\textbf{#1}}
\renewcommand\labelitemii{$\circ$}
\renewcommand\thesubfigure{\arabic{chapter}.\arabic{figure}}
\renewcommand\thesubfigure{%
    \arabic{chapter}.\arabic{figure}.\arabic{subfigure}}
\addto\captionsenglish{\renewcommand{\figurename}{Fig.}}
%------------------------Book Command---------------------------%
\makeatletter
\renewcommand\@pnumwidth{1cm}
\newcounter{book}
\renewcommand\thebook{\@Roman\c@book}
\newcommand\book{%
    \if@openright
        \cleardoublepage
    \else
        \clearpage
    \fi
    \thispagestyle{plain}%
    \if@twocolumn
        \onecolumn
        \@tempswatrue
    \else
        \@tempswafalse
    \fi
    \null\vfil
    \secdef\@book\@sbook
}
\def\@book[#1]#2{%
    \ifnum \c@secnumdepth >-3\relax
        \refstepcounter{book}%
        \addcontentsline{toc}{book}{
            \bookname\ \thebook:\hspace{1em}#1
        }
    \else
        \addcontentsline{toc}{book}{#1}%
    \fi
    \markboth{}{}%
    {\centering
     \interlinepenalty \@M
     \normalfont
     \ifnum \c@secnumdepth >-2\relax
       \huge\bfseries \bookname\nobreakspace\thebook
       \par
       \vskip 20\p@
     \fi
     \Huge \bfseries #2\par}%
    \@endbook}
\def\@sbook#1{%
    {\centering
     \interlinepenalty \@M
     \normalfont
     \Huge \bfseries #1\par}%
    \@endbook}
\def\@endbook{
    \vfil\newpage
        \if@twoside
            \if@openright
                \null
                \thispagestyle{empty}%
                \newpage
            \fi
        \fi
        \if@tempswa
            \twocolumn
        \fi
}
\newcommand*\l@book[2]{%
    \ifnum \c@tocdepth >-2\relax
        \addpenalty{-\@highpenalty}%
        \addvspace{2.25em \@plus\p@}%
        \setlength\@tempdima{3em}%
        \begingroup
            \parindent \z@ \rightskip \@pnumwidth
            \parfillskip -\@pnumwidth
            {
                \leavevmode
                \Large \bfseries #1\hfil \hb@xt@\@pnumwidth{
                    \hss #2
                }
            }
            \par
            \nobreak
            \global\@nobreaktrue
            \everypar{\global\@nobreakfalse\everypar{}}%
        \endgroup
    \fi}
\newcommand\bookname{Book}
\renewcommand{\thebook}{\texorpdfstring{\Numberstring{book}}{book}}
\providecommand*{\toclevel@book}{-2}
\makeatother
\titlecontents{chapter}[0pt]
    {\bfseries}
    {\chaptername\ \thecontentslabel:\quad}
    {}
    {\hfill\contentspage}
\titleformat{\part}[display]
    {\Large\bfseries}
    {\partname\nobreakspace\thepart}
    {0mm}
    {\Huge\bfseries}
    \titlecontents{part}[0pt]
    {\large\bfseries}
    {\partname\ \thecontentslabel: \quad}
    {}
    {\hfill\contentspage}
\newcommand{\MarkRightAngle}[4][.3cm]
    {\coordinate (tempa) at ($(#3)!#1!(#2)$);
     \coordinate (tempb) at ($(#3)!#1!(#4)$);
     \coordinate (tempc) at ($(tempa)!0.5!(tempb)$);%midpoint
     \draw (tempa) -- ($(#3)!2!(tempc)$) -- (tempb);}
%--------------------------LENGTHS------------------------------%
% Spacings for the Table of Contents.
\addtolength{\cftsecnumwidth}{1ex}
\addtolength{\cftsubsecindent}{1ex}
\addtolength{\cftsubsecnumwidth}{1ex}
\addtolength{\cftfignumwidth}{1ex}
\addtolength{\cfttabnumwidth}{1ex}

% Spacing for multi-column and enumerate environments.
\setlength{\multicolsep}{6pt}
\setlist[enumerate]{itemsep=0pt,topsep=3pt}

% Indent and paragraph spacing.
\setlength{\parindent}{0em}
\setlength{\parskip}{0em}
%--------------------------Main Document----------------------------%
\begin{document}
    \ifx\ifmath\undefined
        \newif\ifgeoalg
        \title{Algebraic Geometry}
        \author{Ryan Maguire}
        \date{\vspace{-5ex}}
        \maketitle
        \tableofcontents
        \chapter*{Algebraic Geometry}
        \markboth{}{ALGEBRAIC GEOMETRY}
        \setcounter{chapter}{1}
    \else
        \chapter{Algebraic Geometry}
    \fi
    \section{Notes on Cox, Little, and O'Shea}
        \subimport{./Notes_On_Cox_Little_OShea/}
            {Preliminaries}
        \subimport{./Notes_On_Cox_Little_OShea/}
            {Geometry_Algebra_and_Algorithms}
        \subimport{./Notes_On_Cox_Little_OShea/}
            {Elimination_Theory}
        \subimport{./Notes_On_Cox_Little_OShea/}
            {Groebner_Bases}
        \subimport{./Notes_On_Cox_Little_OShea/}
            {The_Algebra_Geometry_Dictionary}
        \subimport{./Notes_On_Cox_Little_OShea/}
            {Polynomials_and_Functions_on_a_Variety}
    \section{Miscellaneous Notes}
        \subimport{./Miscellaneous_Notes/}
            {Groebner_Bases}
        \subimport{./Miscellaneous_Notes/}
            {Elimination_Theory}
        \subimport{./Miscellaneous_Notes/}
            {Etale_Cohomology}
        \subimport{./Miscellaneous_Notes/}
            {The_Zariski_Topology}
        \subimport{./Miscellaneous_Notes/}
            {Notes_on_Varieties}
\end{document}
    % \addtocontents{toc}{\protect\newpage}
    % \clearpage

    % \setcounter{endpage}{\thepage}
    % \pagenumbering{gobble}
    % \book{Topology}
    %     \renewcommand{\PATH}{\TOPPATH/Topology/}
    %     \pagenumbering{arabic}
    %     \setcounter{page}{\value{endpage}}
    %     \part{Point-Set Topology}
    %         \chapter{Point-Set Topology}
    \section{Topological Spaces}
        \begin{ldefinition}{Topology}{Topology}
            A topology on a set $X$ is a subset
            $\tau\subseteq\mathcal{P}(X)$ such that
            $\emptyset\in\tau$, $X\in\tau$, and for any
            subset $\mathcal{O}\subseteq\tau$, it is true that:
            \begin{equation}
                \bigcup_{\mathcal{U}\in\mathcal{O}}\mathcal{U}\in\tau
            \end{equation}
            And such that for any finite subset
            $\mathcal{C}\subseteq\tau$, it is true that:
            \begin{equation}
                \bigcap_{\mathcal{U}\in\mathcal{C}}\mathcal{U}\in\tau
            \end{equation}
            That is, $\tau$ is closed to arbitrary unions and
            finite intersections.
        \end{ldefinition}
        \begin{ldefinition}{Topological Space}{Topological_Space}
            A topological space, denote $(X,\tau)$ is a subset $X$
            and a topology $\tau$ on $X$.
        \end{ldefinition}
        \begin{ldefinition}{Open Subsets}{Open_Subsets}
            An open subset of a topological space $(X,\tau)$ is a
            set $\mathcal{U}\subseteq{X}$ such that
            $\mathcal{U}\in\tau$.
        \end{ldefinition}
        \begin{theorem}
            \label{thm:Emptyset_Is_Open}%
            If $(X,\tau)$ is a topological space, then
            $\emptyset$ is an open subset.
        \end{theorem}
        \begin{proof}
            For since $(X,\tau)$ is a topological space,
            $\tau$ is a topology on $X$
            (Def.~\ref{def:Topological_Space}). But then
            $X\in\tau$ (Def.~\ref{def:Topology}) and
            thus $X$ is an open subset of $(X,\tau)$
            (Def.~\ref{def:Open_Subsets}). Therefore, etc.
        \end{proof}
        \begin{theorem}
            \label{thm:Whole_Space_Is_Open}%
            If $(X,\tau)$ is a topological space, then
            $X$ is an open subset.
        \end{theorem}
        \begin{proof}
            For since $(X,\tau)$ is a topological space,
            $\tau$ is a topology on $X$
            (Def.~\ref{def:Topological_Space}). But then
            $\emptyset\in\tau$ (Def.~\ref{def:Topology}) and
            thus $\emptyset$ is an open subset of $(X,\tau)$
            (Def.~\ref{def:Open_Subsets}). Therefore, etc.
        \end{proof}
        \begin{ldefinition}{Closed Subsets}{Closed_Subsets}
            A closed subset of a topological space $(X,\tau)$
            is a subset $\mathcal{C}\subseteq{X}$ such that
            $X\setminus\mathcal{C}\in\tau$. That is, the complement
            of $\mathcal{C}$ is an open subset of $(X,\tau)$.
        \end{ldefinition}
        \begin{theorem}
            \label{thm:Emptyset_Is_Closed}%
            If $(X,\tau)$ is a topological space, then
            $\emptyset$ is closed.
        \end{theorem}
        \begin{proof}
            For since $(X,\tau)$ is a topological space, $X$ is an
            open subset (Thm.~\ref{thm:Whole_Space_Is_Open}).
            But $X\setminus\emptyset=X$, and thus the complement
            of $\emptyset$ is open. Thus, $\emptyset$ is
            closed (Def.~\ref{def:Closed_Subsets}).
            Therefore, etc.
        \end{proof}
        \begin{theorem}
            \label{thm:Whole_Space_Is_Closed}%
            If $(X,\tau)$ is a topological space, then $X$ is closed.
        \end{theorem}
        \begin{proof}
            For since $(X,\tau)$ is a topological space, $\emptyset$
            is an open subset (Thm.~\ref{thm:Emptyset_Is_Open}).
            But $X\setminus{X}=\emptyset$, and thus the complement
            of $X$ is open. Thus, $X$ is closed
            (Def.~\ref{def:Closed_Subsets}). Therefore, etc.
        \end{proof}
        \begin{theorem}
            \label{thm:Comp_of_Open_is_Closed}%
            If $(X,\tau)$ is a topological space, and if
            $\mathcal{U}\subseteq{X}$ is open, then
            $X\setminus\mathcal{U}$ is closed.
        \end{theorem}
        \begin{proof}
            For if $\mathcal{U}\subseteq{X}$ is a set, then
            $X\setminus(X\setminus\mathcal{U})=\mathcal{U}$. But
            $\mathcal{U}$ is open, and thus the complement of
            $X\setminus{U}$ is open. But then $X\setminus{U}$
            is closed (Def.~\ref{def:Closed_Subsets}).
        \end{proof}
        \begin{ldefinition}{Relative Topology}{Relative_Topology}
            The relative topology of a subset
            $A\subseteq{X}$ in a topological space $(X,\tau)$ is
            the set $\tau_{A}$ defined by:
            \begin{equation}
                \tau_{A}=
                \big\{\;A\cap\mathcal{U}\,:\,
                    \mathcal{U}\in\tau\;\big\}
            \end{equation}
        \end{ldefinition}
        \begin{theorem}
            If $(X,\tau)$ is a topological space, if $A\subseteq{X}$,
            and if $\tau_{A}$ is the relative topology on $A$,
            then $\tau_{A}$ is a topology on $A$.
        \end{theorem}
        \begin{proof}
            For since $\emptyset\in\tau$, and
            $\emptyset\cap{A}=\emptyset$, we have that
            $\emptyset\in\tau_{A}$. Similarly, since
            $A\subseteq{X}$, and sicne $X\in\tau$, we see that
            $A=A\cap{X}\in\tau_{A}$. If $\mathcal{O}$ is a subset
            of $\tau_{A}$, then there is a subset
            $\Delta\subseteq\tau$ such that:
            \begin{equation}
                \mathcal{O}=\big\{\;A\cap\mathcal{U}\,:\,
                    \mathcal{U}\in\Delta\;\}
            \end{equation}
            Define $\mathcal{D}$ by:
            \begin{equation}
                \mathcal{D}=\bigcup_{\mathcal{U}\in\Delta}\mathcal{U}
            \end{equation}
            But then:
            \begin{equation}
                \bigcup_{\mathcal{V}\in\mathcal{O}}\mathcal{V}
                =\bigcup_{\mathcal{U}\in\Delta}(A\cap\mathcal{U})
                =A\cap\Big(\bigcup_{\mathcal{U}\in\Delta}\mathcal{U}
                    \Big)
                =A\cap\mathcal{D}
            \end{equation}
            But $\tau$ is a topology on $X$, and thus
            $\mathcal{D}\in\tau$ (Def.~\ref{def:Topology}). But
            then $A\cap\mathcal{D}\in\tau_{A}$
            (Def.~\ref{def:Relative_Topology}). Thus, $\tau_{A}$
            is closed to arbitrary unions.
        \end{proof}
        \begin{definition}
            If $(X,\tau)$ is a topological space and $S\subset{X}$,
            then a set $\mathcal{U}\subset S$ is said to be open in
            $S$ if and only if $\mathcal{U}\in \mathscr{T}$, where
            $\mathscr{T}$ is the relative topology on $S$.
        \end{definition}
        \begin{ldefinition}{Continuous Functions}{Cont_Func_Top}
            A continuous function from a topological space
            $(X,\tau_{X})$ to a topological space $(Y,\tau_{Y})$ is
            a function $f:X\rightarrow{Y}$ such that for all
            $\mathcal{U}\in\tau_{Y}$ it is true that
            $f^{\minus{1}}(\mathcal{U})\in\tau_{X}$. That is,
            the pre-image of open sets is open.
        \end{ldefinition}
        \begin{theorem}
            If $(X,\tau_{X})$, $(Y,\tau_{Y})$, and $(Z,\tau_{Z})$
            are topological spaces, and if the functions
            $f:X\rightarrow{Y}$ and $g:Y\rightarrow{Z}$ are
            continuous, then $g\circ{f}:X\rightarrow{Z}$
            is continuous.
        \end{theorem}
        \begin{proof}
            For if $\mathcal{V}\in\tau_{Z}$ is and open set, then
            $g^{\minus{1}}(\mathcal{V})\in\tau_{Y}$, since $g$
            is continuous. But then since $f$ is continuous,
            $f^{\minus{1}}(g^{\minus{1}}(\mathcal{V}))\in\tau_{X}$
            (Def.~\ref{def:Cont_Func_Top}). Thus $g\circ{f}$
            is continuous.
        \end{proof}
        \begin{ldefinition}
              {Convergent Sequences In Topological Spaces}
              {Conv_Seq_Top}
            A convergent sequence in a topological space $(X,\tau)$
            is a sequence $a:\mathbb{N}\rightarrow{X}$ such that
            there is an $x\in{X}$ such that, for all
            $\mathcal{U}\in\tau$ such that $x\in\mathcal{U}$, there
            is an $N\in\mathbb{N}$ such that, for all $n>N$, it is
            true that $x_{n}\in\mathcal{U}$. We denote this by
            $a_{n}\rightarrow{x}$.
        \end{ldefinition}
        \begin{ldefinition}
              {Limits of Sequences in Topological Spaces}
              {Lim_Seq_Top}
            A limit of a sequence $a:\mathbb{N}\rightarrow{X}$ in
            a topological space $(X,\tau)$ is a point $x\in{X}$
            such that $a_{n}\rightarrow{x}$.
        \end{ldefinition}
        \begin{theorem}
            There exist topological spaces with convergent sequences
            that do not have unique limits.
        \end{theorem}
        \begin{proof}
            For let $X=\{1,2,3\}$, and let
            $\tau=\{\emptyset, \{1,2\},\{1,2,3\}\}$. We see that
            $\emptyset,X\in \tau$, unions and intersections are
            in $\tau,$ and thus $\tau$ is a topology. Let:
            \begin{equation}
                x_{n}=
                \begin{cases}
                    1,&n\textrm{ odd}\\
                    2,&n\textrm{ even}
                \end{cases}
            \end{equation}
            Then $x_n \rightarrow 1$ and $x_n \rightarrow 2$. To
            see this, let $\mathcal{U}$ be an open set such that
            $1\in \mathcal{U}$. Our choices are $\{1,2\}$ and
            $\{1,2,3\}$. Then for all $n\in \mathbb{N}$,
            $x_{n}\in\mathcal{U}$, and thus $x_{n}\rightarrow{1}$.
            Similarly, $x_n \rightarrow 2$. Convergence is not
            necessarily unique in topological spaces.
        \end{proof}
        \subsection{Separation Axioms}
        \begin{ldefinition}{Fr\'{e}chet Topological Space}
              {Frechet_Topological_Space}
            A Fr\'{e}chet Topological Space is a topological
            space $(X,d)$ such that for all distinct $x,y\in{X}$
            there is an open set $\mathcal{U}$ such that
            $x\in\mathcal{U}$ and $y\notin\mathcal{U}$.
        \end{ldefinition}
        \begin{theorem}
            If $(X,\tau)$ is a Fr\'{e}chet Topological space, and
            if $x\in{X}$, $\{x\}$ is closed subset.
        \end{theorem}
        \begin{proof}
            For if $x\in{X}$, then for all $y\in{X}$ such that
            $y\ne{x}$, there is a $\mathcal{U}_{y}\in\tau$ such
            that $x\notin\mathcal{U}_{y}$ and $y\in\mathcal{U}_{y}$.
            Define $\mathcal{V}$ by:
            \begin{equation}
                \mathcal{V}\;=
                \bigcup_{y\in{X}\setminus\{x\}}\mathcal{U}_{y}
            \end{equation}
            But then $\mathcal{V}\in\tau$, since $\tau$ is a
            topology (Def.~\ref{def:Topology}). And moreover, for
            all $y\in{X}$ such that $y\ne{x}$, it is true that
            $y\in\mathcal{V}$. Lastly, $x\notin\mathcal{V}$.
            Therefore $X\setminus\mathcal{V}=\{x\}$. But if
            $\mathcal{V}$ is open, then $X\setminus\mathcal{V}$ is
            closed (Thm.~\ref{thm:Comp_of_Open_is_Closed}).
            Therefore, etc.
        \end{proof}
        \begin{ldefinition}{Hausdorff Topological Space}
              {Hausdoff_Top}
            A Hausdorff Topological space is a topological space
            $(X,\tau)$ such that, for all distinct points
            $x,y\in{X}$, there are disjoint open subsets
            $\mathcal{U}_{x}$, $\mathcal{U}_{y}\in\tau$ such that
            $x\in\mathcal{U}_{x}$ and $y\in\mathcal{U}_{y}$.
        \end{ldefinition}
        \begin{theorem}
            If $(X,\tau)$ is a Hausdorff topological space, then
            it is a Fr\'{e}chet topological space.
        \end{theorem}
        \begin{proof}
            For if $x$ and $y$ are distinct points in $X$ and if
            $(X,\tau)$ is a Hausdorff topological space, then there
            is are disjoint open subsets $\mathcal{U}_{x}$ and
            $\mathcal{U}_{y}$ such that $x\in\mathcal{U}_{x}$ and
            $y\in\mathcal{U}_{y}$. But then there is a open subset
            such that $x\notin\mathcal{U}_{y}$ and
            $y\in\mathcal{U}_{y}$. Therefore, $(X,\tau)$ is a
            Fr'{e}chet topological space.
        \end{proof}
            \begin{theorem}
            Convergence in a Hausdorff Space $(X,\tau)$ is unique.
            \end{theorem}
            \begin{proof}
            $[x_n \rightarrow x\in X]\land [x_n \rightarrow y\in X]\land[x\ne y]\Rightarrow [\exists \mathcal{U},\mathcal{V}:\mathcal{U}\cap \mathcal{V}=\emptyset\land x\in \mathcal{U}\land y\in \mathcal{V}]$. $[x_n\rightarrow x]\Rightarrow [\exists N_1\in \mathbb{N}:n>N_1\Rightarrow x_n \in \mathbb{N}]$. $[x_n\rightarrow y]\Rightarrow [N_2\in \mathbb{N}:n>N\Rightarrow x_n \in \mathcal{V}]$. $[n>\max\{N_1,N_2\}]\Rightarrow [x_n \in \mathcal{U}\cap \mathcal{V}]$, a contradiction. Therefore, etc.
            \end{proof}
            \begin{definition}
            A topological space $(X,\tau)$ is said to be regular if for each closed subset $E\subset X$ and for each point $x\in E^c$, there exist disjoint open sets $\mathcal{U}$ and $\mathcal{V}$ such that $x\in \mathcal{U}$ and $E\subset \mathcal{V}$.
            \end{definition} 
            \begin{definition}
            In a topological space $(X,\tau)$, a point $p$ is said to have a neighborhood $S\subset X$ if and only if there is a set $\mathcal{U}\subset S$ such that $\mathcal{U}\in \tau$ and $p\in \mathcal{U}$.
            \end{definition}
            \begin{definition}
            A $T_3$ space is a regular $T_1$ space.
            \end{definition}
            \begin{theorem}
            A $T_3$ space $(X,\tau)$ is a $T_2$ space.
            \end{theorem}
            \begin{proof}
            Let $x,y\in X$ be distinct. As a $T_3$ space is $T_1$, $\{x\}$ is closed. Thus $\exists \mathcal{U},\mathcal{V}\in\tau: \mathcal{U}\cap\mathcal{V}=\emptyset, \{x\}\subset \mathcal{U}$, and $y\in \mathcal{V}$.
            \end{proof}
            \begin{definition}
            A topological space $(X,\tau)$ is said to be normal if and only if for all disjoint closed subsets $E,F\subset X$, there are disjoint open sets $\mathcal{U}$ and $\mathcal{V}$ such that $E\subset \mathcal{U}$ and $F\subset \mathcal{V}$.
            \end{definition}
            \begin{definition}
            A $T_4$ space is a normal $T_1$ space.
            \end{definition}
            \begin{theorem}
            A $T_4$ space $(X,\tau)$ is a $T_3$ space.
            \end{theorem}
            \begin{proof}
            A $T_4$ space is $T_1$. If $E\underset{Closed}\subset X$ and $x\in E^c$, then $\{x\}$ is closed. Thus $\exists \mathcal{U},\mathcal{V}\in\tau: \mathcal{U}\cap\mathcal{V}=\emptyset, \{x\}\subset \mathcal{U}$, and $E\subset \mathcal{V}$.
            \end{proof}
            \begin{definition}
            A homeomorphism between two topological spaces $(X,\tau)$ and $(Y,\tau)$ is a continuous bijection $f:X\rightarrow Y$ such that $f^{-1}:Y\rightarrow X$ is continuous.
            \end{definition}
            \begin{definition}
            If $(X,\tau)$ is a topological space, and $S\subset X$, then an open cover $\mathcal{O}$ of $S$ is a set of open sets $\mathcal{U}_{\alpha}$ such that $S\subset \cup_{\alpha\in A} \mathcal{U}_{\alpha}$, where $A$ is some index set.
            \end{definition}
            \begin{definition}
            A subcover of an open cover $\mathcal{O}$ is a subset of $\mathcal{O}$ that is also a cover.
            \end{definition}
            \begin{definition}
            If $(X,\tau)$ is a topological space and $S\subset X$, then $S$ is said to be compact if and only if every open cover of $S$ has a finite subcover.
            \end{definition}
            \begin{theorem}
            If $S$ is a compact subset of a Hausdorff space, then for all $x\in S^c$ there are disjoint open sets $\mathcal{U}$ and $\mathcal{V}$ such that $x\in \mathcal{U}$ and $S\subset \mathcal{V}$.
            \end{theorem}
            \begin{proof}
            For let $x\in S^c$. For all $y\in S$ there are disjoint open sets $\mathcal{U}_y$ and $\mathcal{V}_y$ such that $x\in \mathcal{U}$ and $y\in \mathcal{V}$. But then $\cup_{y\in S} \mathcal{U}_y$ is an open cover of $S$. As $S$ is compact, there is a finite subcover, that is sets $\mathcal{V}_{y_1},\hdots, \mathcal{V}_{y_n}$ that cover $S$. But then $\cap_{k=1}^{n} \mathcal{U}_{y_k}$ is open, contains $x$ and is disjoint from $\cup_{k=1}^{n} \mathcal{V}_{y_k}$. Therefore, etc.
            \end{proof}
            \begin{theorem}
            Every compact subset of a Hausdorff space $(X,\tau)$ is closed.
            \end{theorem}
            \begin{proof}
            Let $S$ be a compact subset of a X. $\forall x\in S^c, \exists \mathcal{U}_x\in \tau:\mathcal{U}_x\cap S = \emptyset:x\in \mathcal{U}_x$. But then $S^c \subset \underset{x\in S^c}\cup\mathcal{U}_x$. But also $S\cap (\cup_{x\in S^c}\mathcal{U}_x) = \emptyset$. Thus $S^c = \cup_{x\in S^c}\mathcal{U}_x$, and therefore $S^c$ is open. Thus $S$ is closed.
            \end{proof}
            \begin{theorem}
            If $S$ is a closed subset of a compact space $(X,\tau)$, $S$ is compact.
            \end{theorem}
            \begin{proof}
            For let $\mathcal{O}$ be an open cover of $S$. As $S$ is closed, $S^c$ is open, and thus $\{S^c\} \cup \mathcal{O}$ is an open cover $X$. As $X$ is compact, there is a finite subcover, call it $\mathscr{O}$. But then $\mathscr{O}\setminus \{S^c\}$ is a finite subcover $\mathcal{O}$ that covers $S$. Thus, etc.
            \end{proof}
            \begin{theorem}
            If $f:X\rightarrow Y$ is continuous and $X$ is compact, then $f(X)\subset Y$ is compact.
            \end{theorem}
            \begin{proof}
            Let $\mathcal{O}$ be an open cover of $f(X)$. As $f$ is continuous, $\mathcal{U}\in\mathcal{O}\Rightarrow f^{-1}(\mathcal{U})$ is open in $X$. Thus $\cup_{\mathcal{U}\in \mathcal{O}} f^{-1}(\mathcal{U})$ is an open cover of $X$. As $X$ is compact, there is a finite subcover, say $\mathscr{O}$. But then $\cup_{\mathcal{V}\in \mathscr{O}} \mathcal{V}$ is a finite subcover of $\mathcal{O}$. Therefore, etc.
            \end{proof}
            \begin{theorem}
            If $f:X\rightarrow Y$ is a continuous bijection, $X$ is compact and $Y$ is Hausdorff, then $f$ is a homeomorphism.
            \end{theorem}
            \begin{proof}
            If suffices to show that if $\mathcal{U}$ is open in $X$, then $f(\mathcal{U})$ is open in $f(X)$. Let $\mathcal{U}$ be open in $X$. As $X$ is compact and $\mathcal{U}$ is open, $\mathcal{U}^c$ is compact. But then $f(\mathcal{U}^c) = f(X)\setminus f(\mathcal{U})$ is compact. Thus $f(X)\setminus f(\mathcal{U})\underset{Closed}\subset f(X)\Rightarrow f(\mathcal{U})\underset{Open}\subset f(X)$.
            \end{proof}
            \begin{definition}
            A topological space $(X,\tau)$ is said to be disconnected if and only if there are two disjoint nonempty open sets $\mathcal{U}$ and $\mathcal{V}$ such that $X = \mathcal{U}\cup \mathcal{V}$.
            \end{definition}
            \begin{theorem}
            A topological space $(X,\tau)$ is disconnected if and only if there are two non-empty disjoint closed set $\mathcal{C}$ and $\mathcal{D}$ such that $X=\mathcal{C}\cup\mathcal{D}$.
            \end{theorem}
            \begin{proof}
            $\big[\exists \mathcal{U},\mathcal{V}\in \tau: [\mathcal{U}\cap \mathcal{V}=\emptyset]\land [X=\mathcal{U}\cup \mathcal{V}]\land [\mathcal{U},\mathcal{V}\ne \emptyset]\big]\Rightarrow [X = \mathcal{U}^c\cup \mathcal{V}^c]$ thus, $X$ is the union of disjoint, non-empty closed set. $[\mathcal{C}^c,\mathcal{D}^c\in \tau]\land[\mathcal{C},\mathcal{V}\ne\emptyset]\land[\mathcal{C}\cap \mathcal{D}=\emptyset]\land[\mathcal{C}\cup\mathcal{D}=X]\Rightarrow [X=\mathcal{C}^c\cup\mathcal{D}^c].$ Thus $X$ is disconnected.
            \end{proof}
            \begin{theorem}
            $(X,\tau)$ is disconnected if and only if there is a proper, nonempty set $A\subset X$ that is both open and closed.
            \end{theorem}
            \begin{proof}
            $\big[\exists \mathcal{U},\mathcal{V}\in \tau:[\mathcal{U}\cap \mathcal{V}=\emptyset]\land [X=\mathcal{U}\cup\mathcal{V}]\land[\mathcal{U},\mathcal{V}\ne \emptyset]\big]\Rightarrow [\mathcal{U}^c = \mathcal{V}]\Rightarrow [\mathcal{U}^c\in \tau]$. Thus, $\mathcal{U}$ is open and closed.
            \end{proof}
            \begin{definition}
            A topological space is called connected if and only if it is not disconnected.
            \end{definition}
            \begin{theorem}
            If $f:X\rightarrow Y$ is a continuous function and $X$ is connected, then $f(X)$ is connected.
            \end{theorem}
            \begin{proof}
            For let $f$ be continuous and $X$ be connected. Suppose $f(X)$ is disconnected. Then there are two nonempty open disjoint sets $\mathcal{U}$ and $\mathcal{V}$ such that $f(X) = \mathcal{U}\cap \mathcal{V}$. But then their preimage is open, and thus $X=f^{-1}(\mathcal{U})\cup f^{-1}(\mathcal{V})$, and thus $X$ is disconnected, a contradiction. Thus $f(X)$ is connected.
            \end{proof}
            \begin{definition}
            If $(X,\tau)$ and $(Y,\tau')$ are topological spaces, then the product topology on the set $X\times Y$ is the set $\mathscr{T} = \{\mathcal{U}\times \mathcal{V}:\mathcal{U}\in\tau,\mathcal{V}\in \tau'\}$.
            \end{definition}
            \begin{theorem}
            The product topology is a topology.
            \end{theorem}
            \begin{proof}
            \
            \begin{enumerate}
            \item As $\emptyset \in \tau$ and $\emptyset\in \tau'$, $\emptyset =\emptyset\times \emptyset \in \mathscr{T}$.
            \item If $\mathscr{U}_{\alpha}\in \mathscr{T}$, then $\cup_{\alpha} \mathscr{U}_{\alpha} = \cup_{\alpha} (\mathcal{U}_{\alpha},\mathcal{V}_{\alpha})$. As $\tau$ and $\tau'$ are topologies, $\cup_{\alpha} \mathcal{U} \in \tau$ and $\cup_{\alpha}\mathcal{V}_{\alpha} \in \tau'$. Thus, $\cup_{\alpha}\mathscr{U}_{\alpha} \in \mathscr{T}$.
            \item $\cap_{k=1}^{n} \mathscr{U}_{k} = \cap_{k=1}^{n} (\mathcal{U}_k,\mathcal{V}_k)$. As $\tau$ and $\tau'$ are topologies, $\cap_{k=1}^{n}\mathcal{U}_k \in \tau$ and $\cap_{k=1}^{n}\mathcal{V}_{k} \in \tau'$. Thus $\cap_{k=1}^{n} \mathscr{U}_k \in \mathscr{T}$
            \end{enumerate}
            \end{proof}
            \begin{definition}
            The projection map $\pi_1$ is defined as $\pi_1:X_1\times X_2\rightarrow X_1$ by $(x_1,x_2)\mapsto x_1$. Similarly for $\pi_2$.
            \end{definition}
            \begin{theorem}
            The projection map is continuous.
            \end{theorem}
            \begin{proof}
            Let $\pi_1:X_1\times X_2\rightarrow X_1$ be the projection map, $X\times Y$ having the project topology. Let $\mathcal{U}\underset{Open}\subset X_1$. Then $f^{-1}(\mathcal{U}) = \{(x_1,x_2):x_1\in \mathcal{U}, x_2\in X_2\}$. But $\mathcal{U}$ and $X_2$ are open, and thus $f^{-1}(\mathcal{U})$ is open (In the product topology).
            \end{proof}
            \begin{definition}
            An open mapping is a function $f:X\rightarrow Y$ such that $\mathcal{U}\underset{Open}\subset X\Rightarrow f(\mathcal{U}) \underset{Open}\subset Y$.
            \end{definition}
            \begin{theorem}
            The projection map is an open mapping.
            \end{theorem}
            \begin{proof}
            For let $\mathscr{U}$ be an open set in $X\times Y$ (With the product topology). That is, there are open sets $\mathcal{U}\subset X$ and $\mathcal{V}\subset Y$ such that $\mathscr{U}= \{(x,y):x\in \mathcal{U},y\in \mathcal{V}\}$, Then $\pi_1(\mathscr{U}) =\mathcal{U}$, which is open. Therefore, etc.
            \end{proof}
            \begin{theorem}
            If $X$ and $Y$ are compact, then $X\times Y$ is compact with the product topology.
            \end{theorem}
            \begin{proof}
            For let $\mathscr{O}$ be an open cover of $X\times Y$. Then $\{\pi_X(\mathscr{U}):\mathscr{U}\in \mathscr{O}\}$ is an open cover of $X$ and $\{\pi_{Y}(\mathscr{V}):\mathscr{V}\in \mathscr{O}\}$ is an open cover of $Y$. As $X$ and $Y$ are compact, there exist finite subcovers of each, say $\mathcal{O}_X$ and $\mathcal{O}_Y$. But then $\{\pi_{X}^{-1}(\mathcal{U}):\mathcal{U}\in \mathcal{O}_X\}\cup \{\pi_{Y}^{-1}(\mathcal{V}):\mathcal{V}\in \mathcal{O}_Y\}$ is a finite subcover of $\mathscr{O}$. Thus, $X\times Y$ is compact.
            \end{proof}
            \begin{theorem}
            If $X,Y\subset Z$ are compact, $X\cup Y$ is compact.
            \end{theorem}
            \begin{proof}
            Let $\mathcal{O}$ be an open cover of $X\cup Y$. Then there is a finite subcover of $X$ and a finite subcover of $Y$, and thus the combination of these subcovers is a cover of $X\cup Y$.
            \end{proof}
    \section{Old Notes}
        Elements of a topological space are called points.
        Elements of the topology are called open subset of $X$.
        A neighborhood of a point $x$ is a set $V$ that contains
        an open subste $U$ such that $x\in{U}$. An open neighborhood
        of $x$ is an open set $U$ such that $x\in{U}$.
        \begin{example}
            The chaotic topology on a set
            $X$ is the set $\tau=\{\emptyset,X\}$.
        \end{example}
        \begin{example}
            The discrete topology on a set $X$ is
            $\tau=\mathcal{P}(X)$.
        \end{example}
        \begin{example}
            The Sierpinski topology
            on $\{0,1\}$ is the set
            $\{\emptyset,\{0\},\{0,1\}\}$.
        \end{example}
        \begin{theorem}
            If $T_{\omega}$ is a set of topologies on
            a topological space $X$, then
            $\bigcap{T_{\omega}}$ is a topology on $X$.
        \end{theorem}
        However, the union of topologies may not be a
        topology. A topology $\tau_{1}$ is set to
        be finer than a topology $\tau_{0}$ if
        $\tau_{0}\subset\tau_{1}$. An accumulation point
        of a set $A$ is a point $x$ such that, for
        all open neighborhoods $U$ of $A$,
        $U\cap{A}\ne\emptyset$.
        \begin{theorem}[Bolzano-Weierstrass Theorem]
            If $X$ is a bounded, infinite subset of
            $\mathbb{R}$, then $X$ has at least
            one accumulation point.
        \end{theorem}
        \begin{definition}
            The Euclidean topology on
            $\mathbb{R}$ is the set of
            all open sets in the sense that
            $U$ is open if, for all $x\in{U}$,
            there is an $\varepsilon>0$ such
            that $(x-\varepsilon,x+\varepsilon)\subset{U}$.
        \end{definition}
        \begin{definition}
            A closed subset of a topological space
            $(X,\tau)$ is a set $A$ such that
            $A^{C}\in\tau$.
        \end{definition}
        \begin{theorem}
            The intersection of an arbitrary collection of
            closed sets is closed. The union of finitely
            many closed sets is closed.
        \end{theorem}
        \begin{proof}
            Apply DeMorgan's theorem to the properties
            of a topological space $\tau$.
        \end{proof}
        \begin{definition}
            The closure of a set $A$,
            denoted $\overline{A}$, is the
            intersection of all closed sets
            containing $A$.
        \end{definition}
        \begin{theorem}
            If $A$ is a set in a topological space
            $(X,\tau)$, then $A\subset\overline{A}$.
        \end{theorem}
        There's also something called the derived
        set of a set $A$. The interior of $A$
        is the union of all open subset of $A$.
        The boundary of $A$ is the set difference
        of the closure of $A$ and the interior of
        $A$.
        \begin{theorem}
            If $A$ is a set, then
            the interior of $A$ is equal to
            $(\overline{A^{C}})^{C}$
        \end{theorem}
        \begin{definition}
            A dense subset of a topological
            space $(X,\tau)$ is a set $A$
            such that $\overline{A}=X$.
        \end{definition}
        \begin{definition}
            The neighborhood system of a point
            $x$ in a topological space $(X,\tau)$
            is the set of all neighborhoods of
            $x$.
        \end{definition}
        \begin{definition}
            A sequence in a topological space
            $a_{n}$ converges to a point $a$ if,
            for all open neighborhoods $U$ of $a$,
            there is an $N\in\mathbb{N}$ such that,
            for all $n>N$, $a_{n}\in{U}$.
        \end{definition}
        Limits of sequences in topological spaces are NOT
        necessarily unique. This is different from convergence
        in $\mathbb{R}$, where convergence is always unique.
        \begin{definition}
            The relative topology of a
            topological space $(X,\tau)$ with
            respect to a subset $A\subset{X}$
            is $\tau_{A}=\{A\cap{U}:U\in\tau\}$
        \end{definition}
        \begin{theorem}
            If $(X,\tau)$ is a topological space and
            $A\subset{X}$, then
            $(A,\tau_{A})$ is a topological space.
        \end{theorem}
        $(A,\tau_{A})$ is called a subspace of
        $(X,\tau)$.
        \begin{definition}
            A basis of a topological space
            $(X,\tau)$ is a subset $B$ of
            $\tau$ such that every element
            of $\tau$ is the union of some of the
            elements of $B$.
        \end{definition}
        \begin{theorem}
            A subset $B\subset\tau$ is a basis
            for $\tau$ if and only if for all
            $U\in\tau$ and all $x\in{U}$, there is
            a $V\in{B}$ such that
            $x\in{V}\subset{U}$.
        \end{theorem}
        \begin{theorem}
            If $B$ is a basis of $\tau$, then
            $U$ is open if and only if for all
            $x\in{U}$ there is a $V\in{B}$ such that
            $x\in{V}\subset{U}$.
        \end{theorem}
        \begin{theorem}
            $\mathbb{R}$ has a countable basis.
        \end{theorem}
        \begin{proof}
            For the set of open intervals
            $(p,q)$, where $p$ and $q$ are rational
            numbers, forms a basis for the standard
            topology on $\mathbb{R}$. Moreover, this
            is countable.
        \end{proof}
        \begin{definition}
            If $(X,\tau)$ is a topological space
            and $S\subset\tau$, then $S$ is a subbase
            if a finite intersection of elements of $S$
            forms a base of $\tau$.
        \end{definition}
        \begin{definition}
            A local base for a point
            $x$ in a topological space $(X,\tau)$
            is a set of open neighborhods $B_{x}$ of
            $x$ such that for all open neighborhoods $G$
            of $x$, there is a $G_{x}\in{B_{x}}$ such that
            $x\in{G_{x}}\subset{G}$.
        \end{definition}
        \begin{theorem}
            If $(X,\tau)$ is a topological space, $x\in{X}$,
            and if $B$ is a base for $\tau$, then
            the set of elements $G_{x}$ in $B$ such that
            $x\in{G_{x}}$ is a local base for $x$.
        \end{theorem}
    \section{Product Topology}
    Some preliminaries. Given a set $X$, let
    $\mathcal{P}(X)$ denote the power set of $X$. That is,
    $\mathcal{P}(X)$ is the set of all subsets of $X$.
    \begin{ldefinition}{Topological Space}{Top_Space}
        A topological space is a set $X$ and a topology
        $\tau$ on $X$, denoted $(X,\tau)$.
    \end{ldefinition}
    It is common in the literature of mathematics to drop this
    ordered pair notation, and simply call $X$ a topological space.
    To prevent confusion, we will distinguish between the two:
    $X$ is a set, $(X,\tau)$ is a topological space.
    \par\hfill\par
    The notion of a topological space is a generalization of that
    of a \textit{metric space}. We discard all properties of
    metric spaces, with the exception of the fact that open sets
    are closed under arbitrary unions and finite intersections.
    As such, we call the elements of a topology $\tau$ on a set
    $X$ the \textit{open subsets} of $X$. We wish to talk about
    the \textit{product space} formed by the Cartesian product of
    two sets and their respective topologies. We'll need to define
    the \textit{generated} topology, so we prove the following:
    \begin{theorem}
        \label{thm:Intersec_of_Tops_is_Top}%
        If $X$ is a set, and if $T$ is a set of topologies
        on $X$, then:
        \begin{equation}
            \tau=\bigcap_{t\in{T}}t
        \end{equation}
        Is a topology on $X$.
    \end{theorem}
    \begin{proof}
        First note that, since for all $t\in{T}$, $t$ is a topology,
        it is true that $\emptyset\in{t}$, and thus $\emptyset$ is
        contained in the intersection. Therefore $\emptyset\in\tau$.
        Similarly, $X\in\tau$. Given a subset
        $\mathcal{O}\subseteq\tau$, it is true that
        $\mathcal{O}\subseteq{t}$ for all $t\in{T}$. But for all
        $t\in{T}$, $t$ is a topology on $X$, and therefore the union
        of the elements of $\mathcal{O}$ are contained in $t$, and
        thus this union is contained in $\tau$. Thus, $\tau$ is
        closed to arbitrary unions. Similarly for finite
        intersections. Thus, $\tau$ is a topology on $X$.
    \end{proof}
    \begin{ldefinition}{Generated Topology}{Generated_Topology}
        The topology generated by a subset $S\subseteq\mathcal{P}(X)$
        is the set:
        \begin{equation}
            \tau=\bigcap\{\tau_{S}:\tau_{S}
                \textrm{ is a topology on $X$ and }
                S\subseteq\tau_{S}\}
        \end{equation}
        That is, the smallest topology such that the elements of $S$
        are open.
    \end{ldefinition}
    By Thm.~\ref{thm:Intersec_of_Tops_is_Top} we see
    that the topology generated by some collection of subsets of
    $X$ is indeed a topology on $X$. This notion is similar to
    the one found when one studies measure theory. For
    arbitrary topologies it is often difficult, perhaps
    even impossible, to describe explicitly the elements of the
    topology. Analogously, consider the Borel
    $\sigma\textrm{-Algebra}$ on $\mathbb{R}$. We describe
    this as the $\sigma\textrm{-Algebra}$ generated by
    semi-intervals $[a,b)$. An explicit description of the elements
    of the Borel $\sigma\textrm{-Algebra}$ is almost certainly
    impossible. With this, we can move to product spaces.
    \begin{ldefinition}{Product of Two Topologies}
          {Product_of_Two_Topologies}
        The product of two topological spaces $(X,\tau_{X})$ and
        $(Y,\tau_{Y})$ is the topological space $(X\times{Y},\tau)$,
        where $X\times{Y}$ is the Cartesian product of $X$ and $Y$,
        and where $\tau$ is the topology generated by the sets:
        \begin{equation}
            \mathcal{O}=\big\{\mathcal{U}\times\mathcal{V}:
                \mathcal{U}\in\tau_{X},\mathcal{V}\in\tau_{Y}\big\}
        \end{equation}
    \end{ldefinition}
    It is important to note that we cannot simply set the topology
    $\tau$ to be the set of all sets of the form
    $\mathcal{O}=\big\{\mathcal{U}\times\mathcal{V}\big\}$, where
    $\mathcal{U}\in\tau_{X}$ and $\mathcal{V}\in\tau_{Y}$, for
    this will most likely \textbf{not} be a topology.
    The reason being that it may fail to be closed to unions.
    \begin{figure}[H]
        \centering
        \captionsetup{type=figure}
        \begin{subfigure}[b]{0.49\textwidth}
            \centering
            \begin{tikzpicture}[>=Latex]
                \draw[->, thick] (-0.4, 0) to (4, 0)
                    node [above] {$x$};
                \draw[->, thick] (0, -0.4) to (0, 4)
                    node [right] {$y$};
                \draw (1, -0.1) to (1, 0.1);
                \node at (1, -0.4) {$a$};
                \draw (3, -0.1) to (3, 0.1);
                \node at (3, -0.4) {$b$};
                \draw (-0.1, 1) to (0.1, 1);
                \node at (-0.4, 1) {$c$};
                \draw (-0.1, 3) to (0.1, 3);
                \node at (-0.4, 3) {$d$};
                \draw[fill=cyan, opacity=0.8, draw=white]
                    (1, 1) to (1, 3) to (3, 3) to (3, 1) to cycle;
                \draw[densely dashed] (0, 1) to (3, 1);
                \draw[densely dashed] (0, 3) to (3, 3);
                \draw[densely dashed] (1, 0) to (1, 3);
                \draw[densely dashed] (3, 0) to (3, 3);
            \end{tikzpicture}
            \subcaption{The Open Rectangle $(a,b)\times(c,d)$.}
        \end{subfigure}
        \begin{subfigure}[b]{0.49\textwidth}
            \centering
            \begin{tikzpicture}[>=Latex]
                \draw[->, thick] (-0.4, 0) to (4, 0)
                    node [above] {$x$};
                \draw[->, thick] (0, -0.4) to (0, 4)
                    node [right] {$y$};
                \draw[fill=cyan, opacity=0.8, densely dashed]
                    (1, 1) to (2, 1) to (2, 2) to (3, 2)
                           to (3, 3.5) to (1.5, 3.5)
                           to (1.5, 2) to (1, 2) to cycle;
            \end{tikzpicture}
            \subcaption{A Region That Cannot be Written as
                        $\mathcal{U}\times\mathcal{V}$.}
        \end{subfigure}
        \caption{Examples of Open Subsets of $\mathbb{R}^{2}$.}
        \label{fig:Point_Set_Top_Open_Subsets_R2}
    \end{figure}
    For consider $\mathbb{R}$. The standard topology
    on $\mathbb{R}^{2}$ is constructed by considering the
    collection of all open \textit{rectangles},
    $(a,b)\times(c,d)$. However, the set of open rectangles
    will not, by itself, be a topology on $\mathbb{R}^{2}$.
    For one, the union of two rectangles may not even be
    connected: Consider two disjoint non-empty open rectangles.
    This union will \textbf{not} be a rectangle.
    But even if two open rectangles are not disjoint,
    their union may not be a rectangle. See
    Fig.~\ref{fig:Point_Set_Top_Open_Subsets_R2} for examples.
    As a final example, consider the open unit disc in
    $\mathbb{R}^{2}$. This is the set:
    \begin{equation}
        D^{2}=\{(x,y)\in\mathbb{R}^{2}:x^{2}+y^{2}<1\}
    \end{equation}
    This is not of the form $\mathcal{U}\times\mathcal{V}$ for
    some pair of sets $\mathcal{U},\mathcal{V}\subseteq\mathbb{R}$.
    However, seeing as we've called it the open unit disc, we
    would certainly like it to be open. And indeed it is, for
    it lies in the topology that is \textit{generated}
    by open rectangles.
    \begin{figure}[H]
        \centering
        \captionsetup{type=figure}
        \begin{tikzpicture}[>=Latex]
            \draw[<->, thick] (-3.3, 0) to (3.3, 0) node [above] {$x$};
            \draw[<->, thick] (0, -3.3) to (0, 3.3) node [right] {$y$};
            \draw[densely dashed] (0, 0) circle (1in);

            % First Layer
            \draw[fill=cyan, opacity=0.6, densely dashed]
                (0.7071in, 0.7071in) to (-0.7071in, 0.7071in)
                                     to (-0.7071in, -0.7071in)
                                     to (0.7071in, -0.7071in)
                                     to cycle;
            
            % Second Layer
            \draw[fill=green, opacity=0.5, densely dashed]
                (0.68in, 0.3535in) to (0.935in, 0.3535in)
                                   to (0.935in, -0.3535in)
                                   to (0.68in, -0.3535in)
                                   to cycle;
            \draw[fill=green, opacity=0.5, densely dashed]
                (-0.68in, 0.3535in) to (-0.935in, 0.3535in)
                                    to (-0.935in, -0.3535in)
                                    to (-0.68in, -0.3535in)
                                    to cycle;
            \draw[fill=green, opacity=0.5, densely dashed]
                (0.3535in, 0.68in) to (0.3535in, 0.935in)
                                   to (-0.3535in, 0.935in)
                                   to (-0.3535in, 0.68in)
                                   to cycle;
            \draw[fill=green, opacity=0.5, densely dashed]
                (0.3535in, -0.68in) to (0.3535in, -0.935in)
                                    to (-0.3535in, -0.935in)
                                    to (-0.3535in, -0.68in)
                                    to cycle;

            % Third Layer.
            \draw[fill=orange, opacity=0.6, densely dashed]
                (0.68in, 0.3535in) to (0.8212in, 0.3535in)
                                   to (0.8212in, 0.5705in)
                                   to (0.68in, 0.5707in)
                                   to cycle;
            \draw[fill=orange, opacity=0.6, densely dashed]
                (0.68in, -0.3535in) to (0.8212in, -0.3535in)
                                    to (0.8212in, -0.5705in)
                                    to (0.68in, -0.5707in)
                                    to cycle;
            \draw[fill=orange, opacity=0.6, densely dashed]
                (-0.68in, -0.3535in) to (-0.8212in, -0.3535in)
                                     to (-0.8212in, -0.5705in)
                                     to (-0.68in, -0.5707in)
                                     to cycle;
            \draw[fill=orange, opacity=0.6, densely dashed]
                (-0.68in, 0.3535in) to (-0.8212in, 0.3535in)
                                    to (-0.8212in, 0.5705in)
                                    to (-0.68in, 0.5707in)
                                    to cycle;
            \draw[fill=orange, opacity=0.6, densely dashed]
                (0.3535in, 0.68in) to (0.3535in, 0.8212in)
                                   to (0.5705in, 0.8212in)
                                   to (0.5707in, 0.68in)
                                   to cycle;
            \draw[fill=orange, opacity=0.6, densely dashed]
                (0.3535in, -0.68in) to (0.3535in, -0.8212in)
                                    to (0.5705in, -0.8212in)
                                    to (0.5707in, -0.68in)
                                    to cycle;
            \draw[fill=orange, opacity=0.6, densely dashed]
                (-0.3535in, 0.68in) to (-0.3535in, 0.8212in)
                                    to (-0.5705in, 0.8212in)
                                    to (-0.5707in, 0.68in)
                                    to cycle;
            \draw[fill=orange, opacity=0.6, densely dashed]
                (-0.3535in, -0.68in) to (-0.3535in, -0.8212in)
                                     to (-0.5705in, -0.8212in)
                                     to (-0.5707in, -0.68in)
                                     to cycle;

            % Fourth Layer
            \draw[fill=red, opacity=0.5, densely dashed]
                (0.2in, 0.93in) to (0.2in, 0.9797in)
                                to (-0.2in, 0.9797in)
                                to (-0.2in, 0.93in)
                                to cycle;
            \draw[fill=red, opacity=0.5, densely dashed]
                (0.2in, -0.93in) to (0.2in, -0.9797in)
                                 to (-0.2in, -0.9797in)
                                 to (-0.2in, -0.93in)
                                 to cycle;
            \draw[fill=red, opacity=0.5, densely dashed]
                (0.93in, 0.2in) to (0.9797in, 0.2in)
                                to (0.9797in, -0.2in)
                                to (0.93in, -0.2in)
                                to cycle;
            \draw[fill=red, opacity=0.5, densely dashed]
                (-0.93in, 0.2in) to (-0.9797in, 0.2in)
                                 to (-0.9797in, -0.2in)
                                 to (-0.93in, -0.2in)
                                 to cycle;

            % Fifth Layer
            \draw[fill=blue, opacity=0.6, densely dashed]
                (0.82in, 0.3535in) to (0.8781in, 0.3535in)
                                   to (0.8781in, 0.4784in)
                                   to (0.82in, 0.4784in)
                                   to cycle;
            \draw[fill=blue, opacity=0.6, densely dashed]
                (0.82in, -0.3535in) to (0.8781in, -0.3535in)
                                    to (0.8781in, -0.4784in)
                                    to (0.82in, -0.4784in)
                                    to cycle;
            \draw[fill=blue, opacity=0.6, densely dashed]
                (-0.82in, 0.3535in) to (-0.8781in, 0.3535in)
                                    to (-0.8781in, 0.4784in)
                                    to (-0.82in, 0.4784in)
                                    to cycle;
            \draw[fill=blue, opacity=0.6, densely dashed]
                (-0.82in, -0.3535in) to (-0.8781in, -0.3535in)
                                     to (-0.8781in, -0.4784in)
                                     to (-0.82in, -0.4784in)
                                     to cycle;
            \draw[fill=blue, opacity=0.6, densely dashed]
                (0.3535in, 0.82in) to (0.3535in, 0.8781in)
                                   to (0.4784in, 0.8781in)
                                   to (0.4784in, 0.82in)
                                   to cycle;
            \draw[fill=blue, opacity=0.6, densely dashed]
                (0.3535in, -0.82in) to (0.3535in, -0.8781in)
                                    to (0.4784in, -0.8781in)
                                    to (0.4784in, -0.82in)
                                    to cycle;
            \draw[fill=blue, opacity=0.6, densely dashed]
                (-0.3535in, -0.82in) to (-0.3535in, -0.8781in)
                                     to (-0.4784in, -0.8781in)
                                     to (-0.4784in, -0.82in)
                                     to cycle;
            \draw[fill=blue, opacity=0.6, densely dashed]
                (-0.3535in, 0.82in) to (-0.3535in, 0.8781in)
                                    to (-0.4784in, 0.8781in)
                                    to (-0.4784in, 0.82in)
                                    to cycle;

            % Sixth Layer
            \draw[fill=yellow, opacity=0.6, densely dashed]
                (0.68in, 0.5705in) to (0.7641in, 0.5705in)
                                   to (0.7641in, 0.645in)
                                   to (0.68in, 0.645in)
                                   to cycle;
            \draw[fill=yellow, opacity=0.6, densely dashed]
                (0.68in, -0.5705in) to (0.7641in, -0.5705in)
                                    to (0.7641in, -0.645in)
                                    to (0.68in, -0.645in)
                                    to cycle;
            \draw[fill=yellow, opacity=0.6, densely dashed]
                (-0.68in, -0.5705in) to (-0.7641in, -0.5705in)
                                     to (-0.7641in, -0.645in)
                                     to (-0.68in, -0.645in)
                                     to cycle;
            \draw[fill=yellow, opacity=0.6, densely dashed]
                (-0.68in, 0.5705in) to (-0.7641in, 0.5705in)
                                    to (-0.7641in, 0.645in)
                                    to (-0.68in, 0.645in)
                                    to cycle;
            \draw[fill=yellow, opacity=0.6, densely dashed]
                (0.5705in, 0.68in) to (0.5705in, 0.7641in)
                                   to (0.645in, 0.7641in)
                                   to (0.645in, 0.68in)
                                   to cycle;
            \draw[fill=yellow, opacity=0.6, densely dashed]
                (0.5705in, -0.68in) to (0.5705in, -0.7641in)
                                    to (0.645in, -0.7641in)
                                    to (0.645in, -0.68in)
                                    to cycle;
            \draw[fill=yellow, opacity=0.6, densely dashed]
                (-0.5705in, -0.68in) to (-0.5705in, -0.7641in)
                                     to (-0.645in, -0.7641in)
                                     to (-0.645in, -0.68in)
                                     to cycle;
            \draw[fill=yellow, opacity=0.6, densely dashed]
                (-0.5705in, 0.68in) to (-0.5705in, 0.7641in)
                                    to (-0.645in, 0.7641in)
                                    to (-0.645in, 0.68in)
                                    to cycle;
        \end{tikzpicture}
        \caption{Tiling of the Open Unit Disc by Rectangles.}
        \label{fig:Point_Set_Top_Unit_Disc_Rect_Tiling}
    \end{figure}
    Note that, in the tiling of the unit disc shown in
    Fig.~\ref{fig:Point_Set_Top_Unit_Disc_Rect_Tiling}, many of
    the rectangles overlap. This is to avoid excluding any points
    within the circle, and to give a clear picture. Such a tiling
    is allowed in the topology generated by open rectangles, since
    \textit{arbitrary} unions are allowed. With this figure we have
    some evidence that the topology generated by open rectangles is
    most likely the same as the standard topology on $\mathbb{R}^{2}$.
    That is: The set of all sets $\mathcal{U}\subseteq\mathbb{R}^{2}$
    such that, for all $\mathbf{x}\in\mathcal{U}$, there is an $r>0$
    such that, for all $\mathbf{y}\in\mathbb{R}^{2}$ such that
    $\norm{\mathbf{x}-\mathbf{y}}_{2}<r$, it is true that
    $\mathbf{y}\in\mathcal{U}$. Here $\norm{\cdot}_{2}$ denotes the
    standard Euclidean norm, where we compute length by
    invoking the Pythagorean formula. Rather than carrying out
    complicated computations, we can simply note that open
    balls in the $\norm{\cdot}_{2}$ norm are of the form:
    \begin{equation}
        B_{r}^{(\mathbb{R}^{2},\norm{\cdot}_{2})}(\mathbf{x})
            =\big\{\mathbf{y}\in\mathbb{R}^{2}:
                \norm{\mathbf{x}-\mathbf{y}}_{2}<r\big\}
    \end{equation}
    These are circles centered at $\mathbf{x}$.
    Contrast that with open balls in the $\norm{\cdot}_{\infty}$
    metric:
    \begin{equation}
        B_{r}^{(\mathbb{R}^{2},\norm{\cdot}_{\infty})}(\mathbf{x})
            =\big\{\mathbf{y}\in\mathbb{R}^{2}:
                \max\{|x_{1}-y_{1}|,|x_{2}-y_{2}|\}<r\big\}
    \end{equation}
    These are just squares centered at $\mathbf{x}$. And we know
    that the $\norm{\cdot}_{2}$ and $\norm{\cdot}_{\infty}$ metrics
    are equivalent, so these topologies must be the same. Thus
    we've found a slightly more inconvenient way of describing
    the topology on $\mathbb{R}^{2}$. The plus side is that
    this alternative notion generalizes to $X\times{Y}$
    when $(X,\tau_{X})$ and $(Y,\tau_{Y})$ are more general
    topological spaces.
    \par\hfill\par
    In defining the product topology of two topological spaces,
    we used the familiar notion of a Cartesian product. Elements
    of the Cartesian product $X\times{Y}$ are ordered pairs
    $(x,y)$, where $x\in{X}$ and $y\in{Y}$. We can continue to
    ordered triples $(x,y,z)$ and the general $n$ tuple
    $(x_{1},\dots,x_{n})$ and similarly define the product
    topology of $n$ topological spaces
    $(X_{1},\tau_{1}),\dots(X_{n},\tau_{n})$.
    But what if we wanted to define an \textit{infinite} product
    of infinitely many spaces? If the product is countable, we
    have some intuition for we can think of \textit{infinite} tuples
    $(x_{1},\dots,x_{n},\dots)$, but this lacks clarity.
    Rather, let's go back to the product of two topological
    spaces and redefine it. Let $\mathbb{Z}_{2}=\{1,2\}$
    and define:
    \begin{equation}
        \prod_{i=1}^{2}X_{i}=\{f:\mathbb{Z}_{2}\rightarrow
            \bigcup_{k=1}^{2}X_{k}:
            f(1)\in{X}_{1},f(2)\in{X}_{2}\}
    \end{equation}
    That is, the set of all functions from $\mathbb{Z}_{2}$ into
    $X_{1}\cup{X}_{2}$ with the property that 1 maps into $X_{1}$
    and $2$ maps into $X_{2}$. There is a clear bijection between
    this new thing and $X_{1}\times{X}_{2}$, simply map
    $(x,y)$ to the function $f$, where $f(1)=x$ and $f(2)=y$.
    But now we have a definition that really didn't depend
    on how many products we were making. Let
    $\mathbb{Z}_{n}=\{1,\dots,n\}$, and let $X_{1},\dots,X_{n}$
    be sets. We can then define:
    \begin{equation}
        \prod_{i\in\mathbb{Z}_{n}}X_{i}=
            \{f:\mathbb{Z}_{n}\rightarrow
                \bigcup_{k\in\mathbb{Z}_{n}}X_{i}:
                \forall_{i\in\mathbb{Z}_{n}},f(i)\in{X}_{i}\}
    \end{equation}
    And we can go further, defining the product for any collection
    of sets. Let's first introduce some notation. An indexing set
    for a collection of sets is some set $I$ such that we can
    write all of the sets in our collection as $X_{i}$, for
    $i\in{I}$. To improve rigor, let's say that an indexing set
    for a collection of sets $\mathcal{O}$ is some set $I$ such
    that there is a surjective function $X:I\rightarrow\mathcal{O}$,
    and let's write $X(i)=X_{i}$, for all $i\in{I}$. That is,
    for all $i\in{I}$, $X_{i}$ is a set in $\mathcal{O}$.
    We can now define the general product of sets.
    \begin{ldefinition}{Product of Sets}{Product_Set}
        The product of a collection of sets indexed by a set $I$
        is the set:
        \begin{equation}
            \prod_{i\in{I}}X_{i}=
            \{f:I\rightarrow\bigcup_{i\in{I}}X_{i}:
                \forall_{i\in{I}},f(i)\in{X}_{i}\}
        \end{equation}
    \end{ldefinition}
    This notion is well-defined for arbitrary products, countable
    or not. It is important to note that the elements of the
    product space are \textit{functions}.
    \begin{lexample}
        Nothing in the definition of an indexing set requires
        $\mathcal{O}$ to contain many sets, so let
        $\mathcal{O}=\{\mathbb{R}\}$ and let $I=\mathbb{N}$.
        Then the product is simply:
        \begin{equation}
            \prod_{n\in\mathbb{N}}\mathbb{R}=
                \{a:\mathbb{N}\rightarrow\mathbb{R}\}
        \end{equation}
        That is, the set of all sequences of real numbers.
        Thus, the countable product of $\mathbb{R}$ can be
        thought of in two ways: The set of all
        \textit{infinite} tuples
        $(x_{1},\dots,x_{n},\dots)$, or the set of all
        \textit{sequences} of real numbers.
    \end{lexample}
    All of this has been purely set theoretic: There is no topology
    yet. Given a collection of topological spaces
    $(X_{i},\tau_{i})$, is there a good topology to place on the
    product? That is, can we form a nice product topological space?
    There are two well established ways to do this:
    The \textit{obvious} way, and the \textit{correct} way.
    We first let intuition lead us astray, and define the obvious
    answer: The Box Topology.
    \begin{ldefinition}{Box Topology}{Box_Topology}
        The box topology on a collection of topological spaces
        $(X_{i},\tau_{i})$ indexed by a set $I$ is the topology
        $\tau$ on the set $X$ where:
        \begin{equation}
            X=\prod_{i\in{I}}X_{i}
        \end{equation}
        And where $\tau$ is the topology generated by the sets:
        \begin{equation}
            \mathcal{U}=
                \big\{\prod_{i\in{I}}\mathcal{U}_{i}:
                    \mathcal{U}_{i}\in\tau_{i}\big\}
        \end{equation}
        That is, $\tau$ is generated by all of the open sets
        in all of the $X_{i}$.
    \end{ldefinition}
    This is precisely what we did for $\mathbb{R}^{2}$.
    We took the topology to be the one generated by all of the open
    rectangles in the plane. Unfortunately, when the product is
    infinite, the box topology is horrible. Some problems with
    the box topology:
    \begin{enumerate}
        \item The product of compact spaces need not be compact.
        \item The product of connected spaces need not be connected.
        \item The product of metric spaces need not be metrizable.
    \end{enumerate}
    Moreover, some functions that \textit{look} continuous, and that
    we would obviously want to be continuous, are not.
    For example, let $X$ be the set of sequences in $\mathbb{R}$,
    and define $f:\mathbb{R}\rightarrow{X}$ by mapping
    $x$ to the sequence $a_{n}=x$, $n\in\mathbb{N}$. That is:
    \begin{equation}
        f(x)=x,x,x,x,\dots,x,x,\dots
    \end{equation}
    This function is \textit{nowhere} continuous in the box topology.
    So now we devise a plan to make a \textit{better} topology
    with the following property:
    Suppose $g:\mathbb{R}\rightarrow{X}$,
    where $X$ is again the space of real-valued sequences, and
    suppose $g$ is of the form:
    \begin{equation}
        g(x)=g_{1}(x),g_{2}(x),\dots,g_{n}(x),\dots
    \end{equation}
    Where $g_{k}$ is continuous for all $k\in\mathbb{N}$. We
    \textbf{require} that $g$ be continuous in the product space.
    We could simply make $\tau$ be the chaotic topology,
    $\tau=\{\emptyset,X\}$, but then \textit{every} function
    $f:A\rightarrow{X}$ is continuous, for \textit{any}
    topological space $(A,\tau_{A})$, and this is rather boring.
    So we try another approach. Given a collection of topological
    spaces $(X_{i},\tau_{i})$, we require that the product space
    $(X,\tau)$ is such that all of the projection mappings
    $p_{i}:X\rightarrow{X}_{i}$ are continuous. The projection
    mappings can be defined set theoretically using the notation
    we've developed. Given a product $X$ of sets $X_{i}$,
    the projection mapping $p_{i}:X\rightarrow{X}_{i}$ is simply:
    \begin{equation}
        p_{i}(x)=x(i)
    \end{equation}
    This looks strange, but remember that
    we've defined the product space to be a set of
    \textit{functions}, and therefore $x\in{X}$ is a function.
    Thus, the $i^{th}$ projection mapping simply
    evaluates these functions in the $i^{th}$ coordinate.
    \par\hfill\par
    In the search for a topology on the product set $X$ that makes
    all of the projection mappings $p_{i}$ continuous, we could
    simply take $\tau=\mathcal{P}(X)$. Then, for \textit{any}
    topological space $(A,\tau_{A})$, and for \textit{every}
    function $f:X\rightarrow{A}$, $f$ is continuous.
    This is overkill and we see that this is larger than the
    box topology. So all of the problems with the box topology still
    exist! So, we require that $\tau$ is the \textit{smallest}
    such topology. We now define the initial topology.
    \begin{ldefinition}{Initial Topology}{Initial_Topology}
        The initial topology on a set $X$ generated by
        a set of functions $f_{i}$ from $X$ to topological
        spaces $(X_{i},\tau_{i})$ is the set:
        \begin{equation}
            \tau=\bigcap\big\{\tau_{X}:
                \tau_{X}\textrm{ is a topology on $X$ and }
                \forall_{i\in{I}},f_{i}
                \textrm{ is continuous.}\big\}
        \end{equation}
    \end{ldefinition}
    This collection is non-empty, since $\mathcal{P}(X)$ is contained
    in it, and by Thm.~\ref{thm:Intersec_of_Tops_is_Top},
    $\tau$ is a topology on $X$. We now define the product topology.
    \begin{ldefinition}{Product Topology}
        The product topology on a set $X$ defined as the
        product of topological spaces $(X_{i},\tau_{i})$
        indexed over $I$:
        \begin{equation}
            X=\prod_{i\in{I}}X_{i}
        \end{equation}
        Is the initial topology defined by the set:
        \begin{equation}
            \mathscr{F}=
            \{p_{i}:X\rightarrow{X}_{i},p_{i}(x)=x(i)\}
        \end{equation}
        That is, the set of projection mappings.
    \end{ldefinition}
    In this construction one might have noted that the
    projection mappings are continuous in the box topology.
    Thus one might very reasonably ask if the product topology
    and the box topology are the same thing. And indeed, for
    a \textit{finite} product, they are! This makes sense, for in
    $\mathbb{R}^{2}$ we can think of the topology generated by
    rectangles, or the topology generated by the projection mappings,
    and they are the same. What's crucial is that they differ for
    infinite products.
    \begin{ltheorem}{Product Topology Basis Theorem}
        If $(X,\tau)$ is the product topological space formed by
        the topological spaces $(X_{i},\tau_{i})$, indexed by
        a set $I$, then:
        \begin{equation}
            \tau=\bigcup\prod_{i\in{I}}
                \big\{\mathcal{U}_{i}\in\tau_{i}:
                    \mathcal{U}_{i}=X_{i}
                    \textrm{ for all but finitely many sets.}
                \big\}
        \end{equation}
        That is, $\tau$ is the set of all products of open sets
        $\mathcal{U}_{i}$, such that all but finitely many of
        the $\mathcal{U}_{i}$ are the entire space $X_{i}$.
    \end{ltheorem}
    \par\hfill\par
    This seems confusing, so we illustrate with some pictures. What's
    important to note is that, if the product is infinite, then
    the box topology and the product topology differ. To see this,
    in the box topology we allowed \textit{all} products of open
    sets, whereas now we only allow the product of open sets
    $\mathcal{U}_{i}$ where, for all but finitely many $i$, we
    have $\mathcal{U}_{i}=X_{i}$. It should then be clear that,
    if $\tau_{B}$ is the box topology, and $\tau_{P}$ is the
    product topology, then $\tau_{P}\subseteq\tau_{B}$, and for
    infinite products $\tau_{P}$ is a proper subset.
    \par\hfill\par
    Let's dumb down the theorem a bit, and imagine again a world
    where $X=Y=\mathbb{R}$. Let's consider the topology generated
    by sets $\mathcal{U}\times\mathcal{V}$, where $\mathcal{U}$
    is an open subset of $\mathbb{R}$, and
    $\mathcal{V}=\mathbb{R}$. That is, rather than allowing
    the product to be over finitely many arbitrary open sets,
    we allow it to be over one, and it must be in the $x$ axis.
    In doing this we can get a sense of what the product topology
    might look like.
    \begin{figure}[H]
        \centering
        \captionsetup{type=figure}
        \begin{subfigure}[b]{0.49\textwidth}
            \centering
            \begin{tikzpicture}[>=Latex]
                \draw[->, thick] (-0.2, 0) to (5, 0) node [above] {$x$};
                \draw[->, thick] (0, -1) to (0, 5) node [right] {$y$};
                \draw[fill=cyan, opacity=0.5, draw=white]
                    (2, -1) to (2, 5) to (4, 5) to (4, -1) to cycle;
                \draw (2, -0.1) to (2, 0.1);
                \node at (2, -0.4) [left] {$a$};
                \draw (4, -0.1) to (4, 0.1);
                \node at (4, -0.4) [right] {$b$};
                \draw[densely dashed] (2, -1) to (2, 5);
                \draw[densely dashed] (4, -1) to (4, 5);
            \end{tikzpicture}
            \subcaption{Sets of the Form $\mathcal{U}\times\mathbb{R}$.}
        \end{subfigure}
        \begin{subfigure}[b]{0.49\textwidth}
            \begin{tikzpicture}[>=Latex]
                \draw[->, thick] (-1, 0) to (4, 0) node [above] {$x$};
                \draw[->, thick] (0, -1) to (0, 5) node [right] {$y$};
                \draw[fill=cyan, opacity=0.5, draw=white]
                    (-1, 2) to (4, 2) to (4, 4) to (-1, 4) to cycle;
                \draw (-0.2, 2) to (0.1, 2);
                \node at (-0.4, 2) [above] {$c$};
                \draw (-0.1, 4) to (0.1, 4);
                \node at (-0.4, 4) [below] {$d$};
                \draw[densely dashed] (-1, 2) to (4, 2);
                \draw[densely dashed] (-1, 4) to (4, 4);
            \end{tikzpicture}
            \subcaption{Sets of the Form $\mathbb{R}\times\mathcal{V}$.}
        \end{subfigure}
        \caption{Strips in the Plane.}
        \label{fig:Point_Set_Topology_Strips_in_R2}
    \end{figure}
    We can do the same thing and consider sets of the form
    $\mathbb{R}\times\mathcal{V}$, where $\mathcal{V}$ is open
    in $\mathbb{R}$. Recall that open sets in $\mathbb{R}$ are
    intervals and arbitrary collections of intervals. Using this, we
    see that the topology generated by $\mathcal{U}\times\mathbb{R}$
    is the collection of all open vertical \textit{strips},
    and $\mathbb{R}\times\mathcal{V}$ form the horizontal strips.
    See Fig.~\ref{fig:Point_Set_Topology_Strips_in_R2}.
    We expand this game to $\mathbb{R}^{3}$, and think of sets
    of the form $\mathcal{U}\times\mathcal{V}\times\mathbb{R}$,
    or any permutation of the three coordinates.
    See Fig.~\ref{fig:Point_Set_Top_Blocks_in_R3}
    \begin{figure}[H]
        \centering
        \captionsetup{type=figure}
        \begin{tikzpicture}[>=Latex]
            \draw[->, thick] (0, 0, 0) to (4, 0, 0) node [above] {$x$};
            \draw[->, thick] (0, 0, 0) to (0, 4, 0) node [right] {$y$};
            \draw[->, thick] (0, 0, 0) to (0, 0, 8) node [left] {$z$};
            \node at (1.4, -0.15, 0) {$a$};
            \node at (3.6, -0.15, 0) {$b$};
            \node at (0, 1, 0.4) {$c$};
            \node at (0, 3, 0.4) {$d$};
            \draw[densely dashed] (1.5, 0, 0) to (1.5, 3, 0);
            \draw[densely dashed] (3.5, 0, 0) to (3.5, 3, 0);
            \draw[densely dashed] (0, 1, 0) to (3.5, 1, 0);
            \draw[densely dashed] (0, 3, 0) to (3.5, 3, 0);
            \draw[densely dashed] (1.5, 3, 0) to (1.5, 3, 6);
            \draw[densely dashed] (3.5, 3, 0) to (3.5, 3, 6);
            \draw[densely dashed] (1.5, 1, 0) to (1.5, 1, 6);
            \draw[densely dashed] (3.5, 1, 0) to (3.5, 1, 6);
            \draw[densely dashed] (1.5, 3, 6) to (1.5, 0, 6);
            \draw[densely dashed] (3.5, 3, 6) to (3.5, 0, 6);
            \draw[densely dashed] (0, 1, 6) to (3.5, 1, 6);
            \draw[densely dashed] (0, 3, 6) to (3.5, 3, 6);
            \draw[densely dashed] (0, 0, 6) to (4, 0, 6);
            \draw[densely dashed] (3.5, 0, 0) to (3.5, 0, 8);
            \draw[densely dashed] (0, 0, 6) to (0, 4, 6);
            \draw[fill=cyan, opacity=0.5, draw=none]
                (1.5, 1, 6) to (3.5, 1, 6)
                            to (3.5, 3, 6)
                            to (1.5, 3, 6)
                            to cycle;
            \draw[fill=cyan, opacity=0.5, draw=none]
                (3.5, 1, 6) to (3.5, 3, 6)
                            to (3.5, 3, 0)
                            to (3.5, 1, 0)
                            to cycle;
            \draw[fill=cyan, opacity=0.5, draw=none]
                (1.5, 3, 6) to (1.5, 3, 0)
                            to (3.5, 3, 0)
                            to (3.5, 3, 6)
                            to cycle;
        \end{tikzpicture}
        \caption{Blocks in Space.}
        \label{fig:Point_Set_Top_Blocks_in_R3}
    \end{figure}
    The product topology has the following wonderful features:
    \begin{enumerate}
        \item The product of compact topological spaces is compact.
        \item The product of connected spaces is connected.
        \item The product of metric spaces is metrizable.
    \end{enumerate}
    % \addtocontents{toc}{\protect\newpage}
    % \clearpage

    % \setcounter{endpage}{\thepage}
    % \pagenumbering{gobble}
    % \book{Analysis}
    %     \renewcommand{\PATH}{\TOPPATH/Analysis/}
    %     \pagenumbering{arabic}
    %     \setcounter{page}{\value{endpage}}
    %     \part{Measure Theory}
    %         \documentclass[crop=false,class=book,oneside]{standalone}
%----------------------------Preamble-------------------------------%
%---------------------------Packages----------------------------%
\usepackage{geometry}
\geometry{b5paper, margin=1.0in}
\usepackage[T1]{fontenc}
\usepackage{graphicx, float}            % Graphics/Images.
\usepackage{natbib}                     % For bibliographies.
\bibliographystyle{agsm}                % Bibliography style.
\usepackage[french, english]{babel}     % Language typesetting.
\usepackage[dvipsnames]{xcolor}         % Color names.
\usepackage{listings, lstlinebgrd}      % Verbatim-Like Tools.
\usepackage{mathtools, esint, mathrsfs} % amsmath and integrals.
\usepackage{amsthm, amsfonts}           % Fonts and theorems.
\usepackage{tabularx}
\usepackage{tcolorbox}                  % Frames around theorems.
\usepackage{upgreek}                    % Non-Italic Greek.
\usepackage{paracol}                    % Two-column styling.
\usepackage{wrapfig}                    % Wrap text around figure.
\usepackage{fmtcount, etoolbox}         % For the \book{} command.
\usepackage[newparttoc]{titlesec}       % Formatting chapter, etc.
\usepackage{titletoc}                   % Allows \book in toc.
\usepackage[nottoc]{tocbibind}          % Bibliography in toc.
\usepackage[titles]{tocloft}            % ToC formatting.
\usepackage{multicol, enumitem}         % Multi-column/enumerate.
\usepackage{import}                     % Import external files.
\usepackage{pgfplots, tikz}             % Drawing/graphing tools.
\usetikzlibrary{
    calc,                   % Calculating right angles and more.
    angles,                 % Drawing angles within triangles.
    arrows.meta,            % Latex and Stealth arrows.
    quotes,                 % Adding labels to angles.
    positioning,            % Relative positioning of nodes.
    decorations.markings,   % Adding arrows in the middle of a line.
    patterns,
    arrows,
    shapes,
    shapes.geometric,
    cd,
    hobby,
    babel
}                                       % Libraries for tikz.
\pgfplotsset{compat=1.9}                % Version of pgfplots.
\usepackage[font=scriptsize,
            labelformat=simple,
            labelsep=colon]{subcaption} % Subfigure captions.
\usepackage[font={scriptsize},
            hypcap=true,
            labelsep=colon]{caption}    % Figure captions.
\usepackage{hyperref}                   % Allows for hyperlinks.
\hypersetup{
    colorlinks=true,
    linkcolor=blue,
    filecolor=magenta,
    urlcolor=Cerulean,
    citecolor=SkyBlue
}                           % Colors for hyperref.
\usepackage[toc,acronym,nogroupskip]{glossaries} % Glossaries and acronyms.
\usepackage[subpreambles=false]{standalone}      % Complileable sub files.

% Various font stuff from kiwi.
% Use this for Times text and Computer Modern math
%\usepackage{times}

% Quite nice
%\usepackage[charter, greekfamily=, greekuppercase=italicized]{mathdesign}
%\usepackage[utopia, greekuppercase=italicized]{mathdesign}    % Math is narrower

% Use this for Times text and math
%\usepackage{newtxtext}
%\usepackage[libertine,cmintegrals]{newtxmath}
%\usepackage{fix-cm}

%\usepackage{txfontsb}
% or
%\usepackage{mathptmx}

%\usepackage[scaled=0.92]{helvet}
%\renewcommand{\rmdefault}{ptm}

%\usepackage{mathpazo}    % add possibly `sc` and `osf` options
%\usepackage{eulervm}

%\usepackage{fourier}
%\renewcommand{\rmdefault}{ptm}
%\usepackage{mathptm}

%\usepackage{fontspec}
%\setmainfont{lmodern}

%\usepackage[varg]{txfonts}
%\usepackage{fouriernc}
%\usepackage{mathpazo}

%\usepackage{bookman}
%\usepackage[scaled]{uarial}
%\usepackage[scaled]{helvet}
%\renewcommand*\familydefault{\sfdefault}
%\usepackage[math]{anttor}

%\newcommand\fgeorgia{\fontfamily{jvn}\selectfont}
%\newcommand\ftimes{\fontfamily{ptm}\selectfont}
%\newcommand\fhelvetica{\fontfamily{phv}\selectfont}
%\newcommand\fcourier{\fontfamily{pcr}\selectfont}
%\newcommand\fbookman{\fontfamily{pbk}\selectfont}
%\newcommand\fnewcentury{\fontfamily{pnc}\selectfont}
%\newcommand\fpalatino{\fontfamily{ppl}\selectfont}
%\newcommand\favantgarde{\fontfamily{pag}\selectfont}
%\newcommand\fnormal{\normalfont}
%\newcommand\fsize[1]{\ifnum#1>0\fontsize{#1}{#1}\selectfont\else\normalsize\fi}
%------------------------Theorem Styles-------------------------%
% Define theorem style for default spacing and normal font.
\newtheoremstyle{normal}
    {\topsep}               % Amount of space above the theorem.
    {\topsep}               % Amount of space below the theorem.
    {}                      % Font used for body of theorem.
    {}                      % Measure of space to indent.
    {\bfseries}             % Font of the header of the theorem.
    {}                      % Punctuation between head and body.
    {.5em}                  % Space after theorem head.
    {}

% Define theorem style for default spacing with italicized font.
\newtheoremstyle{normalit}{\topsep}{\topsep}
                {\itshape}{}{\bfseries}{}{.5em}{}

% Italic header environment.
\newtheoremstyle{thmit}{\topsep}{\topsep}{}{}{\itshape}{}{0.5em}{}

% Define italicized environments.
\theoremstyle{normalit}
\newtheorem{theorem}{Theorem}[section]
\newtheorem{lemma}{Lemma}[section]
\newtheorem{corollary}{Corollary}[section]
\newtheorem{proposition}{Proposition}[section]
\newtheorem*{theorem*}{Theorem}

% Define environments with italic headers.
\theoremstyle{thmit}
\newtheorem*{solution}{Solution}
\newtheorem*{fsolution}{Solution}

% Define default environments.
\theoremstyle{normal}
\newtheorem{example}{Example}[section]
\newtheorem{definition}{Definition}[section]
\newtheorem{problem}{Problem}[section]
\newtheorem{question}{Question}[section]
\newtheorem{remark}{Remark}[section]
\newtheorem{properties}{Properties}[section]
\newtheorem{notation}{Notation}[section]
\newtheorem{axiom}{Axiom}[section]
\newtheorem*{properties*}{Properties}
\newtheorem*{remark*}{Remark}
\newtheorem*{definition*}{Definition}
\theoremstyle{plain}

% Define framed environment.
\tcbuselibrary{most}
\newtcbtheorem[use counter*=theorem]{ftheorem}{Theorem}%
    {colback=green!5,colframe=green!35!black,
     fonttitle=\bfseries\upshape}{th}

\newtcbtheorem[use counter*=example]{fdefinition}{Definition}%
    {fonttitle=\bfseries\upshape,
     colback=blue!5!white,colframe=blue!75!black}{def}

\newtcbtheorem[use counter*=example]{fexample}{Example}%
    {fonttitle=\bfseries\upshape,
     colback=red!5!white,colframe=red!75!black}{ex}

\newtcbtheorem[use counter*=notation]{fnotation}{Notation}%
    {fonttitle=\bfseries\upshape,
     colback=SeaGreen!5!white,colframe=SeaGreen!75!black}{ex}

\newtcbtheorem[use counter*=corollary]{fcorollary}{Corollary}%
    {fonttitle=\bfseries\upshape,
     colback=Orchid!5!white,colframe=Orchid!75!black}{ex}

\newenvironment{bproof}{\textit{Proof.}}{\hfill$\square$}
\tcolorboxenvironment{bproof}{blanker,breakable,left=5mm,
                             before skip=10pt,after skip=10pt,
                             borderline west={1mm}{0pt}{red}}
\tcolorboxenvironment{fsolution}
    {enhanced jigsaw,colframe=cyan,interior hidden,breakable}

%--------------------Declared Math Operators--------------------%
\DeclareMathOperator{\Refl}{Refl}           % Reflection operator.
\DeclareMathOperator{\Span}{Span}           % Span of a set of vectors.
\DeclareMathOperator{\Card}{Card}           % Cardinality of set.
\DeclareMathOperator{\Ord}{Ord}             % Ordinal of ordered set.
\DeclareMathOperator{\Tr}{Tr}               % Trace of matrix.
\DeclareMathOperator{\adjoint}{adj}         % Adjoint of matrix.
\DeclareMathOperator{\rk}{rk}               % Rank of operator.
\DeclareMathOperator{\nul}{nul}             % Null space of operator.
\DeclareMathOperator{\sgn}{sgn}             % Sign of a number.
\DeclareMathOperator{\multideg}{mutlideg}   % Multi-Degree (Graphs).
\DeclareMathOperator{\GCD}{GCD}             % Greatest common denominator.
\DeclareMathOperator{\LM}{LM}               % Leading monomial
\DeclareMathOperator{\LC}{LC}               % Leading coefficient.
\DeclareMathOperator{\LT}{LT}               % Leading term.
\DeclareMathOperator{\LCM}{LCM}             % Least common multiple.
\DeclareMathOperator{\Mon}{Mon}             % Monomial.
\DeclareMathOperator{\Spec}{Spec}           % Spectrum.
\DeclareMathOperator{\proj}{proj}           % Projection.
\DeclareMathOperator{\comp}{comp}           % Component.
\DeclareMathOperator{\sinc}{sinc}           % Sinc function.
\DeclareMathOperator{\Ima}{Im}              % Image of operator.
\DeclareMathOperator{\Prin}{Prin}           % Principal value.
\DeclareMathOperator{\Mod}{mod}             % Modulus.
%------------------------New Commands---------------------------%
\DeclarePairedDelimiter\norm{\lVert}{\rVert}
\DeclarePairedDelimiter\ceil{\lceil}{\rceil}
\DeclarePairedDelimiter\floor{\lfloor}{\rfloor}
\newcommand*\diff{\mathop{}\!\mathrm{d}}
\newcommand*\Diff[1]{\mathop{}\!\mathrm{d^#1}}
\renewcommand{\mod}{\ \Mod}
\renewcommand*{\glstextformat}[1]{\textcolor{RoyalBlue}{#1}}
\renewcommand{\glsnamefont}[1]{\textbf{#1}}
\renewcommand\labelitemii{$\circ$}
\renewcommand\thesubfigure{\arabic{chapter}.\arabic{figure}}
\renewcommand\thesubfigure{%
    \arabic{chapter}.\arabic{figure}.\arabic{subfigure}}
\addto\captionsenglish{\renewcommand{\figurename}{Fig.}}
%------------------------Book Command---------------------------%
\makeatletter
\renewcommand\@pnumwidth{1cm}
\newcounter{book}
\renewcommand\thebook{\@Roman\c@book}
\newcommand\book{%
    \if@openright
        \cleardoublepage
    \else
        \clearpage
    \fi
    \thispagestyle{plain}%
    \if@twocolumn
        \onecolumn
        \@tempswatrue
    \else
        \@tempswafalse
    \fi
    \null\vfil
    \secdef\@book\@sbook
}
\def\@book[#1]#2{%
    \ifnum \c@secnumdepth >-3\relax
        \refstepcounter{book}%
        \addcontentsline{toc}{book}{
            \bookname\ \thebook:\hspace{1em}#1
        }
    \else
        \addcontentsline{toc}{book}{#1}%
    \fi
    \markboth{}{}%
    {\centering
     \interlinepenalty \@M
     \normalfont
     \ifnum \c@secnumdepth >-2\relax
       \huge\bfseries \bookname\nobreakspace\thebook
       \par
       \vskip 20\p@
     \fi
     \Huge \bfseries #2\par}%
    \@endbook}
\def\@sbook#1{%
    {\centering
     \interlinepenalty \@M
     \normalfont
     \Huge \bfseries #1\par}%
    \@endbook}
\def\@endbook{
    \vfil\newpage
        \if@twoside
            \if@openright
                \null
                \thispagestyle{empty}%
                \newpage
            \fi
        \fi
        \if@tempswa
            \twocolumn
        \fi
}
\newcommand*\l@book[2]{%
    \ifnum \c@tocdepth >-2\relax
        \addpenalty{-\@highpenalty}%
        \addvspace{2.25em \@plus\p@}%
        \setlength\@tempdima{3em}%
        \begingroup
            \parindent \z@ \rightskip \@pnumwidth
            \parfillskip -\@pnumwidth
            {
                \leavevmode
                \Large \bfseries #1\hfil \hb@xt@\@pnumwidth{
                    \hss #2
                }
            }
            \par
            \nobreak
            \global\@nobreaktrue
            \everypar{\global\@nobreakfalse\everypar{}}%
        \endgroup
    \fi}
\newcommand\bookname{Book}
\renewcommand{\thebook}{\texorpdfstring{\Numberstring{book}}{book}}
\providecommand*{\toclevel@book}{-2}
\makeatother
\titlecontents{chapter}[0pt]
    {\bfseries}
    {\chaptername\ \thecontentslabel:\quad}
    {}
    {\hfill\contentspage}
\titleformat{\part}[display]
    {\Large\bfseries}
    {\partname\nobreakspace\thepart}
    {0mm}
    {\Huge\bfseries}
    \titlecontents{part}[0pt]
    {\large\bfseries}
    {\partname\ \thecontentslabel: \quad}
    {}
    {\hfill\contentspage}
\newcommand{\MarkRightAngle}[4][.3cm]
    {\coordinate (tempa) at ($(#3)!#1!(#2)$);
     \coordinate (tempb) at ($(#3)!#1!(#4)$);
     \coordinate (tempc) at ($(tempa)!0.5!(tempb)$);%midpoint
     \draw (tempa) -- ($(#3)!2!(tempc)$) -- (tempb);}
%--------------------------LENGTHS------------------------------%
% Spacings for the Table of Contents.
\addtolength{\cftsecnumwidth}{1ex}
\addtolength{\cftsubsecindent}{1ex}
\addtolength{\cftsubsecnumwidth}{1ex}
\addtolength{\cftfignumwidth}{1ex}
\addtolength{\cfttabnumwidth}{1ex}

% Spacing for multi-column and enumerate environments.
\setlength{\multicolsep}{6pt}
\setlist[enumerate]{itemsep=0pt,topsep=3pt}

% Indent and paragraph spacing.
\setlength{\parindent}{0em}
\setlength{\parskip}{0em}
%----------------------------GLOSSARY-------------------------------%
\makeglossaries
\loadglsentries{../../glossary}
\loadglsentries{../../acronym}
%--------------------------Main Document----------------------------%
\begin{document}
    \ifx\ifmathcourses\undefined
        \pagenumbering{roman}
        \title{Real Analysis}
        \author{Ryan Maguire}
        \date{\vspace{-5ex}}
        \maketitle
        \tableofcontents
        \clearpage
        \chapter*{Real Analysis}
        \addcontentsline{toc}{chapter}{Real Analysis}
        \markboth{}{REAL ANALYSIS}
        \vspace{10ex}
        \setcounter{chapter}{1}
        \pagenumbering{arabic}
    \else
        \chapter{Real Analysis}
    \fi
    \section{Notes from Rosenlicht}
        \subsection{Sets}
            Give a function $f:X\rightarrow{Y}$, the dinstinction
            between the image of a subset $S\subseteq{X}$ and a
            point $x\in{X}$ is:
            \begin{equation}
                f(\{x\})=\{f(x)\}
            \end{equation}
            Similarly for the pre-image:
            \begin{equation}
                \{f^{\minus{1}}(y)\}=f^{\minus{1}}(\{y\})
            \end{equation}
            One definition of an infinite set is that it contains
            a bijection between itself and a proper subset. Such
            sets are called Dedekind infinite, and countable choice is
            needed here. The following are true:
            \begin{align}
                (A^{C})^{C}&=A\\
                A\cup{A}&=A\cap{A}=A\cup\emptyset=A\\
                A\cap\emptyset&=\emptyset\\
                A\times\emptyset&=\emptyset
            \end{align}
            In addition, there are De Morgan's laws and the distributive
            laws. Some more identities:
            \begin{align}
                (A\setminus{B})\cap{C}&=(A\cap{C})\setminus{B}\\
                (A\cup{B})\setminus(A\cap{B})
                    &=(A\setminus{B})\cup(B\setminus{A})\\
                (A\setminus(B\setminus{C}))
                    &=(A\setminus{B})\cup(A\cap{B}\cap{C})\\
                (A\setminus{B})\times{C}
                    &=(A\times{C})\setminus(B\times{C})
            \end{align}
            Given any collection of sets $X_{i}$, $i\in{I}$, and a
            set $B$, we have:
            \begin{align}
                B\cap\Big(\bigcup_{i\in{I}}A_{i}\Big)
                    &=\bigcup_{i\in{I}}\Big(B\cap{A_{i}}\Big)
            \end{align}
            Composition is a commutative operation. That is, given
            $f:X\rightarrow{Y}$, $g:Y\rightarrow{Z}$, and
            $h:Z\rightarrow{W}$, we have:
            \begin{equation}
                h\circ(g\circ{f})=(h\circ{h})\circ{f}
            \end{equation}
            The following is also true of functions:
            \begin{subequations}
                \begin{align}
                    f(A\cup{B})&=f(A)\cup{f}(B)\\
                    f(A\cap{B})&\subseteq{f}(A)\cap{f}(B)\\
                    f^{\minus{1}}(A\cup{B})
                        &=f^{\minus{1}}(A)\cup{f}^{\minus{1}}(B)\\
                    f^{\minus{1}}(A\cap{B})
                        &=f^{\minus{1}}(A)\cap{f}^{\minus{1}}(B)\\
                    A\subseteq{f}^{\minus{1}}(f(A))\\
                    f(f^{\minus{1}}(A)\subseteq{A}
                \end{align}
            \end{subequations}
            \begin{theorem}
                If $f:X\rightarrow{Y}$ is injective, then:
                \begin{subequations}
                    \begin{align}
                        f^{\minus{1}}(f(A))&=A\\
                        f(A\cap{B})&=f(A)\cap{f}(B)
                    \end{align}
                \end{subequations}
            \end{theorem}
            \begin{theorem}
                If $f:X\rightarrow{Y}$ is surjective, then:
                \begin{equation}
                    f(f^{\minus{1}}(A))=A
                \end{equation}
            \end{theorem}
        \subsection{The Real Number System}
            The real numbers are a set $\mathbb{R}$ with several
            properties. These properties make $\mathbb{R}$ a
            complete ordered field, and indeed the only complete
            ordered field. That is, the real numbers are unique
            up to \textit{isomorphism}. There are two functions
            $+,\cdot:\mathbb{R}^{2}\rightarrow\mathbb{R}$, called
            addition and multiplication, respectively, that satisfy
            the following \textit{field axioms}:
            \begin{align}
                a+b&=b+a&
                a\cdot{b}&=b\cdot{a}
                \tag{Commutativity}\\
                a+(b+c)&=(a+b)+c&
                a\cdot(b\cdot{c})&=(a\cdot{b})\cdot{c}
                \tag{Associativity}\\
                a\cdot(b+c)&=a\cdot{b}+a\cdot{c}
                \tag{Distributive Law}\\
                \exists_{0\in\mathbb{R}}:0+a&=a&
                \exists_{1\in\mathbb{R}}:a\cdot{1}&=a
                \tag{Neutral Elements}\\
                \forall_{a\in\mathbb{R}}\exists_{b\in\mathbb{R}}:
                a+b&=0&
                \forall_{a\in\mathbb{R},a\ne{0}}
                \exists_{a^{\minus{1}}}:
                a\cdot{a}^{\minus{1}}&=1
                \tag{Inverse Elements}
            \end{align}
            By inductively using the associative laws and the
            commutative laws, we see that adding $n$ elements
            does not depend on the order in which they are
            added. Similarly for multiplication. For a general
            field, we write $(F,+,\cdot)$.
            \begin{theorem}
                If $(F,+,\cdot)$ is a field, and if $a\in{F}$, then
                the additive inverse of $a$ is unique.
            \end{theorem}
            \begin{proof}
                For suppose $b$ and $b'$ are additive inverses. Then:
                \begin{equation}
                    b=b+0=b+(a+b')=(b+a)+b'=0+b'=b'
                \end{equation}
                And therefore $b$ is unique.
            \end{proof}
            We denote the additive inverse of an element $a$ by
            writing $\minus{a}$.
            \begin{theorem}
                If $(F,+,\cdot)$ is a field, if $a,b\in{F}$, then
                there is a unique $x\in{F}$ such that
                $x+a=b$.
            \end{theorem}
            \begin{proof}
                For let $x=a-b$. Then:
                \begin{equation}
                    x+a=(b-a)+a
                    =b+(-a+a)
                    =b+0
                    =b
                \end{equation}
                Moreover, of $x'$ is a solution, then:
                \begin{equation}
                    x'=x'+0=x'+(a+(\minus{a}))=
                    (x'+a)+(\minus{a})=b+(\minus{a})=x
                \end{equation}
                Thus, $x'=x$.
            \end{proof}
            Instead of writing $b+(\minus{a})$, we
            denote this by $b-a$. This new operation is called
            subtraction. Note that it is not commutative, nor
            is it associative. Indeed, for any $a,b\in\mathbb{R}$,
            suppose $a-b=b-a$, and let $y=a-b$. Then we have that
            $y=\minus{y}$, and thus $y+y=2y=0$. This is only possible
            in $\mathbb{R}$ if $y=0$, and thus we'd require that
            $a=b$. So subtraction is not commutative in $\mathbb{R}$.
            There are fields such that $y+y=0$ and such that
            $y\ne{0}$, but such fields can't have a notion of
            \textit{order} on them. We'll discuss these later.
            Note that the notion is not associative either. Again,
            let $a=2$ and $b=c=1$. Then $a-(b-c)=2$, but
            $(a-b)-c=0$. Again we come to the conclusion that either
            $2=0$, or subtraction is not associative. In an ordered
            field, which is what $\mathbb{R}$ is, we cannot have
            $2=0$. In finite fields, this is possible.
            \begin{theorem}
                If $(F,+,\cdot)$ is a field and if $a\in{F}$
                is non-zero, then the multiplicative inverse
                of $a$ is unique.
            \end{theorem}
            \begin{proof}
                For suppose $b$ and $b'$ are multiplicative inverses
                of $a$. Then:
                \begin{equation}
                    b=b\cdot{1}=b\cdot(a\cdot{b}')=
                    (b\cdot{a})\cdot{b}'=1\cdot{b}'=b'
                \end{equation}
                And therefore $b$ is unique.
            \end{proof}
            We write the multiplicative inverse of a non-zero element
            by $a^{\minus{1}}$.
            \begin{theorem}
                If $(F,+,\cdot)$ is a field, if $a,b\in{F}$, and if
                $a\ne{0}$, then there is a unique $x\in{F}$ such that
                $x\cdot{a}=b$.
            \end{theorem}
            \begin{proof}
                For let $x=b\cdot{a}^{\minus{1}}$. Then:
                \begin{equation}
                    x\cdot{a}=(b\cdot{a}^{\minus{1}})=
                    b\cdot(a^{\minus{1}}\cdot{a})=
                    b\cdot{1}=b
                \end{equation}
                Moreoever, if $x'$ is a solution, then:
                \begin{equation}
                    x'=x'\cdot{1}=x'\cdot(a\cdot{a^{\minus{1}}})
                    =(x'\cdot{a})\cdot{a^{\minus{1}}}=
                    b\cdot{a}^{\minus{1}}=x
                \end{equation}
                Thus, $x'=x$.
            \end{proof}
            We define division by non-zero numbers by writing
            $\frac{a}{b}=a\cdot{b}^{\minus{1}}$. Other symbols are
            used for this, like $a\div{b}$, or simply $a/b$. Similar
            to subtraction, division is neither commutative nor
            associative.
            \begin{theorem}
                If $(F,+,\cdot)$ is a field, if $a,b,c\in{F}$, and if
                $a+c=b+c$, then $a=b$.
            \end{theorem}
            \begin{proof}
                For:
                \begin{equation}
                    a=a+0=a+(c-c)=(a+c)-c=(b+c)-c=b+(c-c)=0
                \end{equation}
                Therefore, etc.
            \end{proof}
            \begin{theorem}
                If $(F,+,\cdot)$ is a field, $a,b,c\in{F}$, if
                $c\ne{0}$, and if $a\cdot{c}=b\cdot{c}$, then
                $a=b$.
            \end{theorem}
            \begin{proof}
                For:
                \begin{equation}
                    a=a\cdot{1}=a\cdot(c\cdot{c}^{\minus{1}})=
                    (a\cdot{c})\cdot{c}^{\minus{1}}=
                    (b\cdot{c})\cdot{c}^{\minus{1}}=
                    b\cdot(c\cdot{c}^{\minus{1}})=
                    b\cdot{1}=b
                \end{equation}
                Therefore, etc.
            \end{proof}
            \begin{theorem}
                If $(F,\cdot,+)$ is a field, and if $a\in{F}$, then
                $a\cdot{0}=0$.
            \end{theorem}
            \begin{proof}
                For:
                \begin{equation}
                    a\cdot{0}+a\cdot{0}=a\cdot(0+0)=
                    a\cdot{0}=a\cdot{0}+0
                \end{equation}
                And therefore from the cancellation laws,
                $a\cdot{0}=0$.
            \end{proof}
            \begin{theorem}
                If $(F,+,\cdot)$ is a field, and $a\in{F}$, then
                $\minus{a}=(\minus{1})\cdot{a}$
            \end{theorem}
            \begin{proof}
                For:
                \begin{equation}
                    (\minus{1})\cdot{a}+a=
                    (\minus{1}+1)\cdot{a}=
                    0\cdot{a}=0
                \end{equation}
                From the uniqueness of inverses,
                $\minus{a}=(\minus{1})\cdot{a}$.
            \end{proof}
            \begin{theorem}
                If $(F,+,\cdot)$ is a field and $a\in{F}$, then
                $\minus(\minus{a})=a$.
            \end{theorem}
            \begin{proof}
                For:
                \begin{equation}
                    \minus(\minus{a})+(\minus{a})=
                    (\minus{1})\cdot(\minus{a})+(\minus{a})
                    =(\minus{1}+1)\cdot(\minus{a})
                    =0\cdot(\minus{a})=0
                \end{equation}
                From the uniqueness of inverses, etc.
            \end{proof}
            \begin{theorem}
                If $(F,+,\cdot)$ is a field, and if $a,b\in{F}$ are
                non-zero, then $(a\cdot{b})^{\minus{1}}=%
                                b^{\minus{1}}\cdot{a}^{\minus{1}}$.
            \end{theorem}
            \begin{proof}
                For:
                \begin{equation}
                    (a\cdot{b})
                    \cdot(b^{\minus{1}}\cdot{a}^{\minus{1}})
                    =a\cdot
                    (b\cdot{b}^{\minus{1}})\cdot{a}^{\minus{1}}
                    =a\cdot{1}\cdot{a}^{\minus{1}}=
                    a\cdot{a}^{\minus{1}}=1
                \end{equation}
                From the uniqueness of inverses, etc.
            \end{proof}
            \begin{theorem}
                If $(F,+,\cdot)$ is a field, $a\in{F}$ is non-zero,
                then $(a^{\minus{1}})^{\minus{1}}=a$.
            \end{theorem}
            \begin{proof}
                For:
                \begin{equation}
                    (a^{\minus{1}})^{\minus{1}}\cdot{a}^{\minus{1}}=
                    (a\cdot{a}^{\minus{1}})^{\minus{1}}=
                    1^{\minus{1}}=1
                \end{equation}
                From uniquness, etc.
            \end{proof}
            \begin{theorem}
                If $(F,+,\cdot)$ is a field, and $a,b,c,d\in{F}$, and
                if $b,d\ne{0}$, then:
                \begin{equation}
                    (a\cdot{b}^{\minus{1}})+(c\cdot{d}^{\minus{1}})=
                    (a\cdot{d}+b\cdot{c})\cdot(b\cdot{d})^{\minus{1}}
                    =\frac{ad+bc}{bd}
                \end{equation}
            \end{theorem}
            As stated before, the axioms of a field are not enough
            to uniquely define the real numbers. Indeed, the rational
            numbers $\mathbb{Q}$ define a field, as do the complex
            numbers $\mathbb{C}$. To see a finite field, consider the
            set $\mathbb{F}_{2}=\{0,1\}$, and consider the following
            arithmetic:
            \par
            \begin{minipage}[b]{0.49\textwidth}
                \centering
                \begin{table}[H]
                    \centering
                    \captionsetup{type=table}
                    \begin{tabular}{c|cc}
                        $+$&0&1\\
                        \hline
                        0&0&1\\
                        1&1&0
                    \end{tabular}
                    \caption{Addition in $\mathbb{F}_{2}$}
                    \label{tab:Real_Analysis_Add_in_F_2_Field}
                \end{table}
            \end{minipage}
            \hfill
            \begin{minipage}[b]{0.49\textwidth}
                \begin{table}[H]
                    \centering
                    \captionsetup{type=table}
                    \begin{tabular}{c|cc}
                        $\cdot$&0&1\\
                        \hline
                        0&0&0\\
                        1&0&1
                    \end{tabular}
                    \caption{Multiplication in $\mathbb{F}_{2}$}
                    \label{tab:Real_Analysis_Mult_in_F_2_Field}
                \end{table}
            \end{minipage}
            Then $(F,+,\cdot)$ is a field. It's a very strange field,
            since we have $1+1=0$, but alas it satisfies all of the
            properties of a field, and all of the theorem's we have
            proved still apply. Interesting, it is the only field
            with two elements. We have no choice in deciding what
            $a\cdot{b}$ means in the field, since multiplication by
            zero must give zero, and multiplication by one must give
            back the original number. Similarly for addition.
            Adding zero must not change anything, and so all
            we are left with is deciding what $1+1$ equals. But
            to be a field, there must be an additive inverse
            element. Thus we are forced to set $1+1=0$. There is
            also a field with three elements. For let
            $\mathbb{F}_{3}=\{0,1,2\}$ and define:
            \par\hfill\par
            \begin{minipage}[b]{0.49\textwidth}
                \centering
                \begin{table}[H]
                    \centering
                    \captionsetup{type=table}
                    \begin{tabular}{c|ccc}
                        $+$&0&1&2\\
                        \hline
                        0&0&1&2\\
                        1&1&2&0\\
                        2&2&0&1
                    \end{tabular}
                    \caption{Addition in $\mathbb{F}_{3}$}
                    \label{tab:Real_Analysis_Add_in_F_3_Field}
                \end{table}
            \end{minipage}
            \hfill
            \begin{minipage}[b]{0.49\textwidth}
                \begin{table}[H]
                    \centering
                    \captionsetup{type=table}
                    \begin{tabular}{c|ccc}
                        $\cdot$&0&1&2\\
                        \hline
                        0&0&0&0\\
                        1&0&1&2\\
                        2&0&2&1
                    \end{tabular}
                    \caption{Multiplication in $\mathbb{F}_{3}$}
                    \label{tab:Real_Analysis_Mult_in_F_3_Field}
                \end{table}
            \end{minipage}
            Intuition tells us that $1+1>1>0$, and thus $1+1$ cannot
            be equal to zero. Thus, to exclude finite fields we
            need to introduce the notion of order.
            \begin{enumerate}
                \item There is a subset $\mathbb{R}^{+}$
                      of $\mathbb{R}$ such that, for all
                      $a,b\in\mathbb{R}^{+}$, we
                      have $a\cdot{b}\in\mathbb{R}^{+}$ and
                      $a+b\in\mathbb{R}^{+}$.
                \item For all $a\in\mathbb{R}$, one and only one of
                      the following statements is true:
                      \begin{itemize}
                          \item $a\in\mathbb{R}^{+}$
                          \item $a=0$
                          \item $\minus{a}\in\mathbb{R}^{+}$
                      \end{itemize}
            \end{enumerate}
            $\mathbb{R}^{+}$ is called the set of positive numbers,
            and the elements such that $\minus{a}\in\mathbb{R}^{+}$
            are called negative. We define less than by writing
            $a<b$ if $b-a\in\mathbb{R}^{+}$. Similarly, we define
            greater than by writing $a>b$ is $a-b\in\mathbb{R}^{+}$.
            The less than or equal to and greater than or equal to
            symbols, denoted $\leq$ and $\geq$, respectively,
            are such that $a\leq{b}$ if $a<b$ or $a=b$, and
            similarly $a\geq{b}$ if $a>b$ or $a=b$. This defines
            $\mathbb{R}$ to be on ordered field.
            \begin{theorem}
                If $a,b\in\mathbb{R}$, then either $a=b$, $a<b$, or
                $a>b$.
            \end{theorem}
            \begin{proof}
                For either $a-b\in\mathbb{R}^{+}$, $a-b=0$, or
                $\minus(a-b)\in\mathbb{R}^{+}$. If
                $a-b\in\mathbb{R}^{+}$, then $a>b$. If $a-b=0$, then
                $a=b$. Finally, if $\minus(a-b)\in\mathbb{R}^{+}$,
                then $b-a\in\mathbb{R}^{+}$, and thus $b>a$.
            \end{proof}
            \begin{theorem}
                If $a,b,c\in\mathbb{R}$, if $a<b$, and if $b<c$, then
                $a<c$.
            \end{theorem}
            \begin{proof}
                For if $a<b$, then $b-a\in\mathbb{R}^{+}$. But if
                $b<c$, then $c-b\in\mathbb{R}^{+}$. But then:
                \begin{equation}
                    c-a=(c-b)+(b-a)\in\mathbb{R}^{+}
                \end{equation}
                Therefore, etc.
            \end{proof}
            \begin{theorem}
                If $a,b,c,d\in\mathbb{R}$, if $a<b$, and if
                $c\leq{d}$, then $a+c<b+d$.
            \end{theorem}
            \begin{proof}
                If $a<b$, then $b-a\in\mathbb{R}^{+}$. If $c\leq{d}$,
                then either $d-c\in\mathbb{R}^{+}$, or $d-c=0$. Thus:
                \begin{equation}
                    (b+d)-(a+c)=(b-a)+(d-c)\in\mathbb{R}^{+}
                \end{equation}
                Therefore, etc.
            \end{proof}
            \begin{theorem}
                If $a,b,c,d\in\mathbb{R}^{+}$, if $a<b$, and if
                $c\leq{d}$, then $a\cdot{c}<b\cdot{d}$.
            \end{theorem}
            \begin{proof}
                For if $a<b$, then $b-a\in\mathbb{R}^{+}$. But if
                $c\leq{d}$, then $d-c\in\mathbb{R}^{+}$, or
                $d-c=0$. But then
                $b\cdot{c}-a\cdot{c}=c\cdot(b-a)\in\mathbb{R}^{+}$.
                Similarly,
                $a\cdot{d}-a\cdot{c}=a\cdot(d-c)$, and thus this is
                either positive of zero. Therefore:
                \begin{equation}
                    bd-ac=(bd-ad)+(ad-ac)=
                    d(b-d)+a(d-c)\in\mathbb{R}^{+}
                \end{equation}
            \end{proof}
            \begin{theorem}
                If $a,b\in\mathbb{R}$ are negative, then $a+b$ is
                negative.
            \end{theorem}
            \begin{proof}
                For if $a$ and $b$ are negative, then
                $\minus{a}$ and $\minus{b}$ are positive. But then
                $(\minus{a})+(\minus{b})\in\mathbb{R}^{+}$. But:
                \begin{equation}
                    (\minus{a})+(\minus{b})=
                    (\minus{1})\cdot{a}+(\minus{1})\cdot{b}
                    =(\minus{1})\cdot(a+b)
                    =\minus(a+b)\in\mathbb{R}^{+}
                \end{equation}
                Thus, $a+b$ is negative.
            \end{proof}
            \begin{theorem}
                If $a,b\in\mathbb{R}$, if $a$ is positive, and if
                $b$ is negative, than $a\cdot{b}$ is negative.
            \end{theorem}
            \begin{proof}
                For if $b$ is negative, then $\minus{b}$ is
                positive, and thus:
                \begin{equation}
                    \minus(a\cdot{b})
                    =a\cdot(\minus{b})\in\mathbb{R}^{+}
                \end{equation}
                Thus, $a\cdot{b}$ is negative.
            \end{proof}
            \begin{theorem}
                If $a,b\in\mathbb{R}$ are negative, then $a\cdot{b}$
                is positive.
            \end{theorem}
            \begin{proof}
                For if $a$ and $b$ are negative, then
                $\minus{a},\minus{b}\in\mathbb{R}^{+}$. But then:
                \begin{equation}
                    a\cdot{b}=1\cdot(a\cdot{b})
                    =\big((\minus{1})\cdot(\minus{1})\big)
                    \cdot(a\cdot{b})
                    =(\minus{a})\cdot(\minus{b})\in\mathbb{R}^{+}
                \end{equation}
                And thus $a\cdot{b}$ is positive.
            \end{proof}
            \begin{theorem}
                If $a\in\mathbb{R}$, then $a^{2}\geq{0}$.
            \end{theorem}
            \begin{proof}
                For if $a$ is positive, then $a\cdot{a}$ is positive.
                If $a$ is zero, then $a\cdot{a}=0$. Finally, from the
                previous theorem, the product of two negative numbers
                is positive, and therefore if $a$ is negative, then
                $a\cdot{a}$ is positive.
            \end{proof}
            From this we have that $1=1^{1}>0$. This generalized to
            the sum of any number of squares.
            \begin{theorem}
                If $a>0$, then $a^{\minus{1}}>0$.
            \end{theorem}
            \begin{proof}
                Suppose not. Then either $a^{\minus{1}}$ is negative
                or it is zero. But it is not zero, for zero has no
                multiplicative inverse, and $a$ is an inverse of
                $a^{\minus{1}}$. Thus $a^{\minus{1}}$ is negative.
                But $a\cdot{a}^{\minus{1}}=1>0$, a contradiction.
                Therefore, $a^{\minus{1}}$ is positive.
            \end{proof}
            \begin{theorem}
                If $0<a<b$, then $0<b^{\minus{1}}<a^{\minus{1}}$.
            \end{theorem}
            \begin{proof}
                For:
                \begin{equation}
                    0<a<b\Longrightarrow
                    0<a\cdot(a^{\minus{1}}b^{\minus{1}})<
                    b\cdot(a^{\minus{1}}b^{\minus{1}})\Longrightarrow
                    0<b^{\minus{1}}<a^{\minus{1}}
                \end{equation}
            \end{proof}
            We thus have a way to distinguish $\mathbb{R}$
            from finite fields. We define the natural numbers to by
            $2=1+1$, $3=2+1$, $4=3+1$, and so on. Order also excludes
            the complex numbers, $\mathbb{C}$, since the complex
            numbers are not ordered. However, the rational numbers,
            $\mathbb{Q}$, still satisfy all of these properties and
            are too an ordered field. We need another property to
            distinguish $\mathbb{Q}$ from $\mathbb{R}$. First, a
            discussion of exponentiation and the absolute value
            function. Given a positive integer $n$, we define the
            exponentiation of a real number $r$ by
            $r^{n}=r\cdots{r}$, where multiplication is
            carried out $n$ times. From this, we get:
            \begin{subequations}
                \begin{align}
                    a^{n}\cdot{a}^{m}&=a^{n+m}\\
                    (a^{m})^{n}&=a^{mn}\\
                    (ab)^{n}&=a^{n}b^{n}
                \end{align}
            \end{subequations}
            The abolute value of a real number is defined as:
            \begin{equation}
                |a|=
                \begin{cases}
                    a,&a\geq{0}\\
                    \minus{a},&a<-
                \end{cases}
            \end{equation}
            \begin{theorem}
                If $a\in\mathbb{R}$, then $|a|\geq{0}$.
            \end{theorem}
            \begin{theorem}
                If $a,b\in\mathbb{R}$, then
                $|a\cdot{b}|=|a|\cdot|b|$.
            \end{theorem}
            \begin{theorem}
                If $a\in\mathbb{R}$, then $a^{2}=|a|^{2}$.
            \end{theorem}
            \begin{ltheorem}{Triangle Inequality}
                If $a,b\in\mathbb{R}$, then
                $|a+b|\leq|a|+|b|$.
            \end{ltheorem}
            \begin{ltheorem}{Reverse Triangle Inequality}
                If $a,b\in\mathbb{R}$, then
                $|a-b|\geq\big||a|-|b|\big|$
            \end{ltheorem}
            Note that $|x-a|<\varepsilon$ implies that
            $\varepsilon-a<x<\varepsilon+a$. Thus, the solution set
            to this inequality is all of the points that lie in the
            interval $(a-\varepsilon,a+\varepsilon)$. Now, to
            separate $\mathbb{R}$ from $\mathbb{Q}$ we need
            to introduce the idea of \textit{completeness}.
            We will do this in the form of the Least Upper
            Bound axiom.
            \begin{definition}
                An upper bound for a subset $S\subseteq\mathbb{R}$
                is a real number $r$ such that, for all $x\in{S}$,
                we have $x\leq{r}$.
            \end{definition}
            A bounded above subset is a subset with an upper bound.
            \begin{definition}
                A least upper bound for a subset
                $S\subseteq\mathbb{R}$ is a real number $r$ such
                that $r$ is an upper bound
                for $S$, and for all upper bounds $s$, we have
                $r\leq{s}$.
            \end{definition}
            From this definition we have that least upper bounds are
            unique for a given bounded above set.
            \begin{theorem}
                If $S$ is a subset of $\mathbb{R}$, if $s$ is a
                least upper bound of $S$, and if $x\in\mathbb{R}$
                is such that $x<s$, then there is a $y\in{S}$
                such that $x<y$.
            \end{theorem}
            \begin{proof}
                For suppose not. Then $x$ is an upper bound of $S$,
                a contradiction as $s$ is the least upper bound.
            \end{proof}
            Any non-empty finite subset will have a least
            upper bound. Infinite subsets need not have a least
            upper bound, and indeed $\mathbb{R}$ does not have
            one. If the least upper bound of $S$ exists, it may
            not belong to $S$. For example, the set of all
            negative numbers has zero as its least upper bound,
            but zero is not a negative number. The real
            numbers satisfy the following property:
            \begin{enumerate}
                \item For any non-empty set of real numbers that
                      is bounded from above, there is a least
                      upper bound.
            \end{enumerate}
            This axiom distringuishes the rational numbers from the
            real numbers. That is, there are bounded above subsets
            of $\mathbb{Q}$ with no least upper bound.
    \section{Old Notes}
        The real line, or real number system, is a complete ordered
        field. That is, it is complete in the sense that all
        Cauchy sequences converge, has a total order structure
        on it, and has a field structure (That of addition,
        multiplication, subtraction, and division).
        An open subset of the real line is a set $S$ such that
        for all $x\in{S}$ there is an $\varepsilon>0$ such that
        $(x-\varepsilon,x+\varepsilon)\subset{S}$. The entire
        space $\mathbb{R}$ is open, as is the empty set
        $\emptyset$. The union of
        an arbitrary collection of open sets is open, and the
        intersection of finitely many open sets is open. The
        intersection of infinitely many open sets may not be
        open, however. A set is closed if its complement is
        open. The Euclidean plane is the set of all ordered
        pairs $(a,b)$. That is,
        $\mathbb{R}^{2}=\mathbb{R}\times\mathbb{R}$. Euclidean
        space, or 3-space, is
        $\mathbb{R}^{3}=\mathbb{R}\times\mathbb{R}\times\mathbb{R}$.
        This is the set of all ordered triplets $(x,y,z)$. Similarly,
        $n$ dimensional Euclidean space is the set of all
        $n$ tuples. This is denoted $\mathbb{R}^{n}$. The distance
        between two points $\mathbf{x}$ and $\mathbf{y}$ is defined
        by the generalized Pythagorean Theorem:
        \begin{equation*}
            d(\mathbf{x},\mathbf{y})=
            \sqrt{\sum_{k=1}^{n}(x_{k}-y_{k})^{2}}
        \end{equation*}
        \begin{definition}
            A metric on a set $X$ is a function
            $d:X\times{X}\rightarrow\mathbb{R}$ such that:
            \begin{enumerate}
                \item $d(x,y)\geq{0}$ for all $x,y\in{X}$.
                \item $d(x,y)=0$ if and only if $x=y$.
                \item $d(x,y)=d(y,x)$ for all $x,y\in{X}$.
                \item $d(x,z)\leq{d(x,y)+d(y,z)}$
                      for all $x,y,z\in{X}$.
            \end{enumerate}
        \end{definition}
        There are two types of integrals defined for functions
        of a real variable: Riemann Integration and Lebesgue Integration.
        Lebesgue integration requires the notion of \textit{measure}.
    \subsection{Definitions}
        \begin{definition*}
                The tangent line of a differentiable function
                $y:\mathbb{R}\rightarrow\mathbb{R}$ at a point
                $x_{0}\in\mathbb{R}$ is the function
                $y_{T}:\mathbb{R}\rightarrow\mathbb{R}$ defined by
                $y_{T}(x)=y'(x_0)(x-x_0)+y(x_0)$ 
            \end{definition*}
        \begin{definition*}
            If $\Gamma(t)=\big(x(t),y(t)\big)$, for $a\leq t\leq b$,
            and $\Gamma'(t)=\big(x'(t),y'(t)\big)$ exists for
            $a<t<b$, then the length of $\Gamma$ from $a$ to $b$ is:
            \begin{equation}
                L=\int_{a}^{b}\sqrt{
                    \bigg(\frac{dx}{dt}\bigg)^{2}+
                    \bigg(\frac{dy}{dt}\bigg)^{2}
                }dt
            \end{equation}
        \end{definition*}
        \begin{definition*}
            The dimension of a vector space is the cardinality of
            any basis of the space. 
        \end{definition*}
        \begin{remark*}
            By the Dimension Theorem, all bases of a vector space
            have the same cardinality.
        \end{remark*}
        \begin{definition*}
            The absolute value of $x$ is
            $|x|=\begin{cases}%
                x,&x\geq 0\\ 
                -x,&x<0
            \end{cases}$
        \end{definition*}
    \subsection{Theorems}
    \begin{theorem*}[Mean Value Theorem]
        If $f:(a,b)\rightarrow\mathbb{R}$ is continuous and
        bounded, and if $x\in(a,b)$, then there is a $c\in(a,x)$
        such that $\int_{a}^{x}f=(x-a)f(c)$.
    \end{theorem*}
    \begin{theorem*}
        [Generalized Fundamental Theorem of Calculus]
        If $\mathcal{U}$ is an open non-empty subset of
        $\mathbb{R}$, $a\in\mathcal{U}$, and if
        $f:\mathcal{U}\rightarrow\mathbb{R}$
        is bounded and continuous, then
        $F:\mathcal{U}\rightarrow\mathbb{R}$
        defined by $F(x)=\int_{\mathcal{U}\cap (a,x)}f$ is
        differentiable and $F'(x)=f(x)$
    \end{theorem*}
    \begin{proof}
        For let $x\in\mathcal{U}$. Let
        $\{x_n\}_{n=1}^{\infty}\subset\mathcal{U}$
        be a sequence such that $x_{n}\rightarrow x$,
        $x\notin\{x_{n}\}_{n=1}^{\infty}$.
        As $\mathcal{U}$ is open and $x\in\mathcal{U}$,
        there is an $\varepsilon>0$ such that
        $B_{\varepsilon}(x)\subset\mathcal{U}$. But, as
        $x_{n}\rightarrow x$, there is an $N\in \mathbb{N}$ such
        that for all $n>N$, $x_{n}\in B_{\varepsilon}(x)$.
        But then for all $n>N$:
        \begin{equation*}
            \int_{\mathcal{U}\cap(a,x)}f-%
            \int_{\mathcal{U}\cap(a,x_{n})}f=%
            \int_{x_{n}}^{x}f
        \end{equation*}
        But, as $f$ is continuous, by the mean value theorem for
        all $n>N$ there is a $c_{n}\in(x_n,x)$ such that
        $\int_{x_{n}}^{x}f=(x-x_{n})f(c_{n})$. But then 
        \begin{equation*}
            \Big|\frac{\int_{x_{n}}^{x}f}{x-x_{n}}-f(x)\Big|
            =|f(c_{n})-f(x)|
        \end{equation*}
        But $c_{n}\in(x_{n},x)$, and $x_{n}\rightarrow x$, and
        therefore $c_{n} \rightarrow x$. But $f$ is continuous,
        and therefore $f(c_{n})\rightarrow f(x)$. Therefore, by
        the definition of the derivative of $F$ at $x$,
        $F'(x)=f(x)$. 
    \end{proof}
    \begin{theorem*}
        If $V$ is a vector space and $A,B\subset V$ are
        subspaces, then $A\cap B$ is a subspace and
        $\dim(A\cap B)\leq\min\{\dim(A),\dim(B)\}$
    \end{theorem*}
    \begin{theorem*}
        If $f:\mathbb{R}\rightarrow \mathbb{R}$
        is differentiable
        and $f'(x)>0$ for all $x$,
        then $f$ is strictly increasing.
    \end{theorem*}
    \begin{theorem*}
        If $f:(a,b)\rightarrow\mathbb{R}$ is continuous and
        $f(a)<0<f(b)$, then there is a $c\in (a,b)$ such that
        $f(c)=0$.
    \end{theorem*}
    \begin{theorem*}
        If $f$ is integrable on $(a,b)$, and if $c\in(a,b)$, then
        $\int_{a}^{b}f=\int_{a}^{c}f+\int_{c}^{b}f$
    \end{theorem*}
\end{document}
    %        \chapter{Measurable Spaces}
    \ifcsname\PATH\endcsname
        \newcommand{\PATH}{books/Analysis/Measure_Theory}
    \else
        \renewcommand{\PATH}{books/Analysis/Measure_Theory}
    \fi

    %------------------------------------------------------------------------------%
\section{Set Rings}
    Given a set $\Omega$, $\mathcal{P}(\Omega)$ is the set of all subsets of
    $\Omega$. Often this is too much, and too difficult to handle. Indeed,
    even $\mathcal{P}(\mathbb{R})$ is quite large and hard to get a grasp on.
    We wish to speak of collections of sets that have some structure on them.
    The first thing we will talk about is a set ring.
    \begin{fdefinition}{Set Ring}
        A set ring of a set $\Omega$ is a nonempty subset
        $\mathcal{R}\subseteq\mathcal{P}(\Omega)$ such that
        for all $A,B\in\mathcal{R}$, $A\cup{B}\in\mathcal{R}$, and
        $A\setminus{B}\in\mathcal{R}$.
    \end{fdefinition}
    \begin{fexample}{Example of Set Rings}{Example_of_Set_Rings}
        If $\Omega$ is a set, then $\mathcal{P}(\Omega)$ is a set ring of
        $\Omega$. So is the set $R=\{\emptyset$. For any $A\subset\Omega$,
        the set $R=\{A\}$ is also a set ring. If $\Omega=\{1,2,3\}$, then
        $R=\{\emptyset,\{1\},\{2,3\},\{1,2,3\}\}$ is
        a set ring on $\Omega$.
    \end{fexample}
    \begin{lexample}
        If $\Omega$ is an infinite set, and if
        $\mathcal{E}=\big\{\{x\}:x\in\Omega\big\}$, then the smallest set
        ring that contains $\mathcal{E}$ is the set of all finite subsets of
        $\Omega$. For the union of two finite sets is finite, as is the set
        difference of two finite sets, and thus this satisfies a set ring.
        Moreover, if $\mathcal{R}$ is a set ring that contains $\mathcal{E}$
        then it contains the union of any finite collection of elements in
        $\mathcal{E}$. But $\mathcal{E}$ is the set of all of the singletons
        of $\Omega$, and any finite subset of $\Omega$ can be written as the
        union of finitely many singletons. Thus, $\mathcal{R}$ is the
        smallest set ring that contains $\mathcal{E}$.
    \end{lexample}
    \begin{theorem}
        If $\Omega$ is a set, if $R$ is a set ring on $\Omega$, and if $A$ is
        a finite subset of $R$, then $\cup_{\alpha\in{A}}\alpha$ is an
        element of $R$.
    \end{theorem}
    \begin{proof}
        Apply induction to the closure of unions.
    \end{proof}
    \begin{theorem}
        If $X$ is a set, if $R$ is a set ring on $X$, and if
        $A,B\in{R}$, then $A\cup{B}\in{R}$.
    \end{theorem}
    \begin{proof}
        For $A\cap{B}=A\setminus(A\setminus{B})$, and from the closure of set
        difference, $A\cap{B}\in{R}$.
    \end{proof}
    \begin{theorem}
        If $X$ is a set, if $R$ is a set ring on $X$, and if $A$ is a finite
        subset of $R$, then $\cap_{\alpha\in{A}}\alpha$ is an element of $R$.
    \end{theorem}
    \begin{proof}
        Apply induction to the closure of intersections.
    \end{proof}
    \begin{theorem}
        If $\Omega$ is a set, if $R$ is a set ring on
        $\Omega$, if $A,B\subset\Omega$, and if
        $A\setminus{B}$, $B\setminus{A}$, and
        $A\cap{B}$ are elements of $R$, then
        $A,B\in{R}$.
    \end{theorem}
    Thus, the set ring generated by the set $\{A,B\}$ and
    the set ring generated by
    $\{A\setminus{B},B\setminus{A},A\cap{B}\}$ are the
    same.
    \begin{theorem}
        If $\Omega$ is a set and $R$ is a set ring
        of $\Omega$, then $\emptyset\in{R}$.
    \end{theorem}
    \begin{proof}
        For as $R$ is non-empty, there is an element
        $A\in{R}$. If $A=\emptyset$, then we are done.
        If not, as $R$ is closed under set difference,
        $A\setminus{A}\in{R}$. But
        $A\setminus{A}=\emptyset$.
    \end{proof}
    From this, if we have a collection $R$ of subsets of
    $\Omega$ and we wish to check if $R$ is a set ring
    of $\Omega$, there are several redundant operations
    we don't need to check. Since, for any set $A$,
    we have:
    \begin{align}
        A\setminus\emptyset&=A\\
        A\setminus{A}&=\emptyset\\
        \emptyset\setminus{A}&=\emptyset\\
        A\cup{A}&=A\\
        A\cup\emptyset&=A\\
        \emptyset\cup\emptyset&=\emptyset
    \end{align}
    Using our previous example $\Omega=\{1,2,3\}$,
    we can check laboriously that
    $R=\{\emptyset,\{1\},\{2,3\},\{1,2,3\}\}$ is a
    set ring on $\Omega$. The set
    $\{\emptyset,\{1\},\{2\},\{1,2,3\}\}$ is not
    a set ring, for $\{1,2\}=\{1\}\cup\{2\}$ is not
    an element.
    \begin{theorem}
        If $\Omega$ is a set, and if $A$ and $B$ are
        disjoint subsets of $\Omega$, then
        $R=\{\emptyset,A,B,A\cup{B}\}$ is a set ring
        on $\Omega$.
    \end{theorem}
    \begin{theorem}
        If $\Omega$ is a set, if $A$ and $B$ are
        disjoint subsets of $\Omega$, and if
        $R$ is a set ring such that $A,B\in{R}$,
        then $\{emptyset,A,B,A\cup{B}\}\subset{R}$.
    \end{theorem}
    As such, the set ring $\{\emptyset,A,B,A\cup{B}\}$
    is called the set ring generated by $A$ and $B$. We
    can continue and consider the case of three mutually
    disjoint subsets.
    \begin{theorem}
        If $\Omega$ is a set, and $A_{1},A_{2},A_{3}$ are
        mutually disjoint subsets of $\Omega$, then:
        \begin{equation}
            R=\{\emptyset,A_{1},A_{2},A_{3},
                A_{1}\cup{A}_{2},A_{1}\cup{A}_{3},
                A_{2}\cup{A}_{3},
                A_{1}\cup{A}_{2}\cup{A}_{3}\}
        \end{equation}
        is a set ring on $\Omega$.
    \end{theorem}
    Indeed, we may generalize further.
    \begin{theorem}
        If $\Omega$ is a set and if
        $A$ is a subset of $\mathcal{P}(\Omega)$ of
        $n$ elements such that, for all
        $a,b\in{A}$, $a\cap{B}=\emptyset$, then:
        \begin{equation}
            R=\{\cup_{i\in{I}}A_{i}:
            I\in\mathcal{P}(\mathbb{Z}_{n})\}
        \end{equation}
        Is a set ring on $\Omega$.
    \end{theorem}
    \begin{theorem}
        If $\Omega$ is a set, then the set of all
        finite subsets of $\Omega$ is a set ring on
        $\Omega$.
    \end{theorem}
    A left semi-interval of $\mathbb{R}$ is an interval
    of the form $[a,b)$ where $a\leq{b}$. If $a=b$, this
    is the empty set. The set of all left semi-intervals
    is not a set ring on $\mathbb{R}$ since the union
    of two semi-intervals need not be a semi-interval.
    We need to add more sets to allow this to be a
    set ring. The collection of all finite unions of
    semi-intervals of $\mathbb{R}$ is a set ring.
    First, note the following:
    \begin{equation}
        \Big(\bigcup_{n=1}^{N}[a_{n},b_{n})\Big)
        \setminus[c,d)=\bigcup_{n=1}^{N}
        \Big([a_{n},b_{n})\setminus[c,d)]
    \end{equation}
    This is again the finite union of intervals. By
    induction we see that this collection is a ring on
    $\mathbb{R}$. We have seen that a set ring is
    closed to unions and set differences, and this
    implies that rings are closed under intersections and
    closed under symmetric differences. As it turns out,
    this is an equivalent definition of a set ring.
    \begin{theorem}
        If $\Omega$ is a set and
        $R\subset\mathcal{P}(\Omega)$, then $R$ is
        a set ring of $\Omega$ if and only if $R$ is
        closed under symmetric differences and
        intersections.
    \end{theorem}
    If $R$ is a set ring on $\Omega$, and if
    $A\in{R}$, let $\chi_{A}:\Omega\rightarrow[0,1]$ be
    the indicator function defined as follows:
    \begin{equation}
        \chi_{A}(\omega)=
        \begin{cases}
            0,&\omega\notin{A}\\
            1,&\omega\in{A}
        \end{cases}
    \end{equation}
    Then we have:
    \begin{align}
        \chi_{A\cap{B}}(\omega)
        &=\chi_{A}(\omega)\chi_{B}(\omega)\\
        \chi_{A\ominus{B}}&=
        \big(\chi_{A}(\omega)+\chi_{B}(\omega)\big)
        \mod{2}
    \end{align}
    These two operations satisfy the axioms of a ring.
    That is, a ring in the algebraic sense of the word:
    A set with two operations that behave certain
    properties. It is because of this that set rings
    have earned their name.
    %        \documentclass[crop=false,class=book,oneside]{standalone}
%----------------------------Preamble-------------------------------%
%---------------------------Packages----------------------------%
\usepackage{geometry}
\geometry{b5paper, margin=1.0in}
\usepackage[T1]{fontenc}
\usepackage{graphicx, float}            % Graphics/Images.
\usepackage{natbib}                     % For bibliographies.
\bibliographystyle{agsm}                % Bibliography style.
\usepackage[french, english]{babel}     % Language typesetting.
\usepackage[dvipsnames]{xcolor}         % Color names.
\usepackage{listings, lstlinebgrd}      % Verbatim-Like Tools.
\usepackage{mathtools, esint, mathrsfs} % amsmath and integrals.
\usepackage{amsthm, amsfonts}           % Fonts and theorems.
\usepackage{tabularx}
\usepackage{tcolorbox}                  % Frames around theorems.
\usepackage{upgreek}                    % Non-Italic Greek.
\usepackage{paracol}                    % Two-column styling.
\usepackage{wrapfig}                    % Wrap text around figure.
\usepackage{fmtcount, etoolbox}         % For the \book{} command.
\usepackage[newparttoc]{titlesec}       % Formatting chapter, etc.
\usepackage{titletoc}                   % Allows \book in toc.
\usepackage[nottoc]{tocbibind}          % Bibliography in toc.
\usepackage[titles]{tocloft}            % ToC formatting.
\usepackage{multicol, enumitem}         % Multi-column/enumerate.
\usepackage{import}                     % Import external files.
\usepackage{pgfplots, tikz}             % Drawing/graphing tools.
\usetikzlibrary{
    calc,                   % Calculating right angles and more.
    angles,                 % Drawing angles within triangles.
    arrows.meta,            % Latex and Stealth arrows.
    quotes,                 % Adding labels to angles.
    positioning,            % Relative positioning of nodes.
    decorations.markings,   % Adding arrows in the middle of a line.
    patterns,
    arrows,
    shapes,
    shapes.geometric,
    cd,
    hobby,
    babel
}                                       % Libraries for tikz.
\pgfplotsset{compat=1.9}                % Version of pgfplots.
\usepackage[font=scriptsize,
            labelformat=simple,
            labelsep=colon]{subcaption} % Subfigure captions.
\usepackage[font={scriptsize},
            hypcap=true,
            labelsep=colon]{caption}    % Figure captions.
\usepackage{hyperref}                   % Allows for hyperlinks.
\hypersetup{
    colorlinks=true,
    linkcolor=blue,
    filecolor=magenta,
    urlcolor=Cerulean,
    citecolor=SkyBlue
}                           % Colors for hyperref.
\usepackage[toc,acronym,nogroupskip]{glossaries} % Glossaries and acronyms.
\usepackage[subpreambles=false]{standalone}      % Complileable sub files.

% Various font stuff from kiwi.
% Use this for Times text and Computer Modern math
%\usepackage{times}

% Quite nice
%\usepackage[charter, greekfamily=, greekuppercase=italicized]{mathdesign}
%\usepackage[utopia, greekuppercase=italicized]{mathdesign}    % Math is narrower

% Use this for Times text and math
%\usepackage{newtxtext}
%\usepackage[libertine,cmintegrals]{newtxmath}
%\usepackage{fix-cm}

%\usepackage{txfontsb}
% or
%\usepackage{mathptmx}

%\usepackage[scaled=0.92]{helvet}
%\renewcommand{\rmdefault}{ptm}

%\usepackage{mathpazo}    % add possibly `sc` and `osf` options
%\usepackage{eulervm}

%\usepackage{fourier}
%\renewcommand{\rmdefault}{ptm}
%\usepackage{mathptm}

%\usepackage{fontspec}
%\setmainfont{lmodern}

%\usepackage[varg]{txfonts}
%\usepackage{fouriernc}
%\usepackage{mathpazo}

%\usepackage{bookman}
%\usepackage[scaled]{uarial}
%\usepackage[scaled]{helvet}
%\renewcommand*\familydefault{\sfdefault}
%\usepackage[math]{anttor}

%\newcommand\fgeorgia{\fontfamily{jvn}\selectfont}
%\newcommand\ftimes{\fontfamily{ptm}\selectfont}
%\newcommand\fhelvetica{\fontfamily{phv}\selectfont}
%\newcommand\fcourier{\fontfamily{pcr}\selectfont}
%\newcommand\fbookman{\fontfamily{pbk}\selectfont}
%\newcommand\fnewcentury{\fontfamily{pnc}\selectfont}
%\newcommand\fpalatino{\fontfamily{ppl}\selectfont}
%\newcommand\favantgarde{\fontfamily{pag}\selectfont}
%\newcommand\fnormal{\normalfont}
%\newcommand\fsize[1]{\ifnum#1>0\fontsize{#1}{#1}\selectfont\else\normalsize\fi}
%------------------------Theorem Styles-------------------------%
% Define theorem style for default spacing and normal font.
\newtheoremstyle{normal}
    {\topsep}               % Amount of space above the theorem.
    {\topsep}               % Amount of space below the theorem.
    {}                      % Font used for body of theorem.
    {}                      % Measure of space to indent.
    {\bfseries}             % Font of the header of the theorem.
    {}                      % Punctuation between head and body.
    {.5em}                  % Space after theorem head.
    {}

% Define theorem style for default spacing with italicized font.
\newtheoremstyle{normalit}{\topsep}{\topsep}
                {\itshape}{}{\bfseries}{}{.5em}{}

% Italic header environment.
\newtheoremstyle{thmit}{\topsep}{\topsep}{}{}{\itshape}{}{0.5em}{}

% Define italicized environments.
\theoremstyle{normalit}
\newtheorem{theorem}{Theorem}[section]
\newtheorem{lemma}{Lemma}[section]
\newtheorem{corollary}{Corollary}[section]
\newtheorem{proposition}{Proposition}[section]
\newtheorem*{theorem*}{Theorem}

% Define environments with italic headers.
\theoremstyle{thmit}
\newtheorem*{solution}{Solution}
\newtheorem*{fsolution}{Solution}

% Define default environments.
\theoremstyle{normal}
\newtheorem{example}{Example}[section]
\newtheorem{definition}{Definition}[section]
\newtheorem{problem}{Problem}[section]
\newtheorem{question}{Question}[section]
\newtheorem{remark}{Remark}[section]
\newtheorem{properties}{Properties}[section]
\newtheorem{notation}{Notation}[section]
\newtheorem{axiom}{Axiom}[section]
\newtheorem*{properties*}{Properties}
\newtheorem*{remark*}{Remark}
\newtheorem*{definition*}{Definition}
\theoremstyle{plain}

% Define framed environment.
\tcbuselibrary{most}
\newtcbtheorem[use counter*=theorem]{ftheorem}{Theorem}%
    {colback=green!5,colframe=green!35!black,
     fonttitle=\bfseries\upshape}{th}

\newtcbtheorem[use counter*=example]{fdefinition}{Definition}%
    {fonttitle=\bfseries\upshape,
     colback=blue!5!white,colframe=blue!75!black}{def}

\newtcbtheorem[use counter*=example]{fexample}{Example}%
    {fonttitle=\bfseries\upshape,
     colback=red!5!white,colframe=red!75!black}{ex}

\newtcbtheorem[use counter*=notation]{fnotation}{Notation}%
    {fonttitle=\bfseries\upshape,
     colback=SeaGreen!5!white,colframe=SeaGreen!75!black}{ex}

\newtcbtheorem[use counter*=corollary]{fcorollary}{Corollary}%
    {fonttitle=\bfseries\upshape,
     colback=Orchid!5!white,colframe=Orchid!75!black}{ex}

\newenvironment{bproof}{\textit{Proof.}}{\hfill$\square$}
\tcolorboxenvironment{bproof}{blanker,breakable,left=5mm,
                             before skip=10pt,after skip=10pt,
                             borderline west={1mm}{0pt}{red}}
\tcolorboxenvironment{fsolution}
    {enhanced jigsaw,colframe=cyan,interior hidden,breakable}

%--------------------Declared Math Operators--------------------%
\DeclareMathOperator{\Refl}{Refl}           % Reflection operator.
\DeclareMathOperator{\Span}{Span}           % Span of a set of vectors.
\DeclareMathOperator{\Card}{Card}           % Cardinality of set.
\DeclareMathOperator{\Ord}{Ord}             % Ordinal of ordered set.
\DeclareMathOperator{\Tr}{Tr}               % Trace of matrix.
\DeclareMathOperator{\adjoint}{adj}         % Adjoint of matrix.
\DeclareMathOperator{\rk}{rk}               % Rank of operator.
\DeclareMathOperator{\nul}{nul}             % Null space of operator.
\DeclareMathOperator{\sgn}{sgn}             % Sign of a number.
\DeclareMathOperator{\multideg}{mutlideg}   % Multi-Degree (Graphs).
\DeclareMathOperator{\GCD}{GCD}             % Greatest common denominator.
\DeclareMathOperator{\LM}{LM}               % Leading monomial
\DeclareMathOperator{\LC}{LC}               % Leading coefficient.
\DeclareMathOperator{\LT}{LT}               % Leading term.
\DeclareMathOperator{\LCM}{LCM}             % Least common multiple.
\DeclareMathOperator{\Mon}{Mon}             % Monomial.
\DeclareMathOperator{\Spec}{Spec}           % Spectrum.
\DeclareMathOperator{\proj}{proj}           % Projection.
\DeclareMathOperator{\comp}{comp}           % Component.
\DeclareMathOperator{\sinc}{sinc}           % Sinc function.
\DeclareMathOperator{\Ima}{Im}              % Image of operator.
\DeclareMathOperator{\Prin}{Prin}           % Principal value.
\DeclareMathOperator{\Mod}{mod}             % Modulus.
%------------------------New Commands---------------------------%
\DeclarePairedDelimiter\norm{\lVert}{\rVert}
\DeclarePairedDelimiter\ceil{\lceil}{\rceil}
\DeclarePairedDelimiter\floor{\lfloor}{\rfloor}
\newcommand*\diff{\mathop{}\!\mathrm{d}}
\newcommand*\Diff[1]{\mathop{}\!\mathrm{d^#1}}
\renewcommand{\mod}{\ \Mod}
\renewcommand*{\glstextformat}[1]{\textcolor{RoyalBlue}{#1}}
\renewcommand{\glsnamefont}[1]{\textbf{#1}}
\renewcommand\labelitemii{$\circ$}
\renewcommand\thesubfigure{\arabic{chapter}.\arabic{figure}}
\renewcommand\thesubfigure{%
    \arabic{chapter}.\arabic{figure}.\arabic{subfigure}}
\addto\captionsenglish{\renewcommand{\figurename}{Fig.}}
%------------------------Book Command---------------------------%
\makeatletter
\renewcommand\@pnumwidth{1cm}
\newcounter{book}
\renewcommand\thebook{\@Roman\c@book}
\newcommand\book{%
    \if@openright
        \cleardoublepage
    \else
        \clearpage
    \fi
    \thispagestyle{plain}%
    \if@twocolumn
        \onecolumn
        \@tempswatrue
    \else
        \@tempswafalse
    \fi
    \null\vfil
    \secdef\@book\@sbook
}
\def\@book[#1]#2{%
    \ifnum \c@secnumdepth >-3\relax
        \refstepcounter{book}%
        \addcontentsline{toc}{book}{
            \bookname\ \thebook:\hspace{1em}#1
        }
    \else
        \addcontentsline{toc}{book}{#1}%
    \fi
    \markboth{}{}%
    {\centering
     \interlinepenalty \@M
     \normalfont
     \ifnum \c@secnumdepth >-2\relax
       \huge\bfseries \bookname\nobreakspace\thebook
       \par
       \vskip 20\p@
     \fi
     \Huge \bfseries #2\par}%
    \@endbook}
\def\@sbook#1{%
    {\centering
     \interlinepenalty \@M
     \normalfont
     \Huge \bfseries #1\par}%
    \@endbook}
\def\@endbook{
    \vfil\newpage
        \if@twoside
            \if@openright
                \null
                \thispagestyle{empty}%
                \newpage
            \fi
        \fi
        \if@tempswa
            \twocolumn
        \fi
}
\newcommand*\l@book[2]{%
    \ifnum \c@tocdepth >-2\relax
        \addpenalty{-\@highpenalty}%
        \addvspace{2.25em \@plus\p@}%
        \setlength\@tempdima{3em}%
        \begingroup
            \parindent \z@ \rightskip \@pnumwidth
            \parfillskip -\@pnumwidth
            {
                \leavevmode
                \Large \bfseries #1\hfil \hb@xt@\@pnumwidth{
                    \hss #2
                }
            }
            \par
            \nobreak
            \global\@nobreaktrue
            \everypar{\global\@nobreakfalse\everypar{}}%
        \endgroup
    \fi}
\newcommand\bookname{Book}
\renewcommand{\thebook}{\texorpdfstring{\Numberstring{book}}{book}}
\providecommand*{\toclevel@book}{-2}
\makeatother
\titlecontents{chapter}[0pt]
    {\bfseries}
    {\chaptername\ \thecontentslabel:\quad}
    {}
    {\hfill\contentspage}
\titleformat{\part}[display]
    {\Large\bfseries}
    {\partname\nobreakspace\thepart}
    {0mm}
    {\Huge\bfseries}
    \titlecontents{part}[0pt]
    {\large\bfseries}
    {\partname\ \thecontentslabel: \quad}
    {}
    {\hfill\contentspage}
\newcommand{\MarkRightAngle}[4][.3cm]
    {\coordinate (tempa) at ($(#3)!#1!(#2)$);
     \coordinate (tempb) at ($(#3)!#1!(#4)$);
     \coordinate (tempc) at ($(tempa)!0.5!(tempb)$);%midpoint
     \draw (tempa) -- ($(#3)!2!(tempc)$) -- (tempb);}
%--------------------------LENGTHS------------------------------%
% Spacings for the Table of Contents.
\addtolength{\cftsecnumwidth}{1ex}
\addtolength{\cftsubsecindent}{1ex}
\addtolength{\cftsubsecnumwidth}{1ex}
\addtolength{\cftfignumwidth}{1ex}
\addtolength{\cfttabnumwidth}{1ex}

% Spacing for multi-column and enumerate environments.
\setlength{\multicolsep}{6pt}
\setlist[enumerate]{itemsep=0pt,topsep=3pt}

% Indent and paragraph spacing.
\setlength{\parindent}{0em}
\setlength{\parskip}{0em}
%----------------------------GLOSSARY-------------------------------%
\makeglossaries
\loadglsentries{../../glossary}
\loadglsentries{../../acronym}
%--------------------------Main Document----------------------------%
\begin{document}
    \ifx\ifmathcourses\undefined
        \pagenumbering{roman}
        \title{Complex Analysis}
        \author{Ryan Maguire}
        \date{\vspace{-5ex}}
        \maketitle
        \tableofcontents
        \clearpage
        \chapter*{Complex Analysis}
        \addcontentsline{toc}{chapter}{Complex Analysis}
        \markboth{}{COMPLEX ANALYSIS}
        \vspace{10ex}
        \setcounter{chapter}{1}
        \pagenumbering{arabic}
    \else
        \chapter{Complex Analysis}
    \fi
    \section{Complex Variables}
        A complex function is a function whose argument is a complex
        variable $z=x+iy$, where $i$ is the imaginary unit. Complex
        functions can have the problem of being multi-valued, which
        is a cause for caution when dealing with them. For example,
        in the complex realm every non-zero complex number $z$
        has two square roots $\sqrt{z}$. So the square root
        function is multi-valued. Any complex function $f(z)$ can
        be written as $f(z)=u(x,y)+iv(x,y)$, where $u$ and $v$ are
        purely real functions. The function $w=f(z)$ can be seen
        as a mapping, or transformation, of the $z$ plane to
        the $w$ plane. That is, $f$ is a transformation of
        its domain onto its range, or image. A compound complex
        function is one of the form $F(z)=g(f(z))$. Since complex
        functions are functions of two variables, in a sense, one
        must be careful when considering limits of complex functions.
        \begin{example}
            What is the limit of $z/\overline{z}$ as $z\rightarrow{0}$?
            This is undefined. For:
            \begin{equation*}
                \frac{z}{\overline{z}}=\frac{x+iy}{x-iy}
            \end{equation*}
            Letting $x=0$ and taking the limit on $y$,
            we get:
            \begin{equation*}
                \frac{0+iy}{0-iy}=-1
            \end{equation*}
            Letting $y=0$ and taking the limit on $x$,
            we get:
            \begin{equation*}
                \frac{x+0i}{x-0i}=1
            \end{equation*}
            So the limit does not exist.
        \end{example}
        Continuity and the various properties of limits
        are defined similarly on $\mathbb{C}$ as for
        $\mathbb{R}$, with distance between points being
        defined by
        $d(z_{1},z_{2})=\sqrt{(x_{2}-x_{1})^{2}+(y_{2}-y_{1})^{2}}$.
        Differentiation is defined as:
        \begin{equation*}
            f'(z_{0})=\lim_{z\rightarrow{z_{0}}}\frac{f(z)-f(z_{0})}{z-z_{0}}
        \end{equation*}
        \begin{theorem}
            A complex function $f(z)=u(x,y)+iv(x,y)$ is
            differentiable if and only if it satisfies
            the Cauchy-Riemann equations:
            \begin{align*}
                \frac{\partial{u}}{\partial{x}}
                &=\frac{\partial{v}}{\partial{y}}
                &
                \frac{\partial{u}}{\partial{y}}
                &=-\frac{\partial{v}}{\partial{x}}
            \end{align*}
        \end{theorem}
        \begin{theorem}
            If $f(z)=u(x,y)+iv(x,y)$ is differentiable,
            then:
            \begin{equation*}
                f'(z)=u_{x}(x,y)+iv_{y}(x,y)
            \end{equation*}
        \end{theorem}
        \begin{definition}
            A complex function $f(z)$ is analytic,
            or holomorphic, at a point $z_{0}$ if
            it is differentiable in some neighborhood of
            $z_{0}$.
        \end{definition}
        \begin{definition}
            An entire function is a complex function
            $f(z)$ such that $f$ is analytic at every
            point $z\in\mathbb{C}$.
        \end{definition}
        \begin{definition}
            A harmonic function is a function
            $A(x,y)$ such that all of its second
            partial derivatives exists, and it
            satisfies the Laplace Equation:
            \begin{equation*}
                \nabla^{2}A
                =A_{xx}(x,y)+A_{yy}(x,y)
                =0
            \end{equation*}
        \end{definition}
        \begin{theorem}
            If $f(z)=u(x,y)+iv(x,y)$ is differentiable
            on a domain $D$, then $u$ and $v$ are
            harmonic on the domain.
        \end{theorem}
        \begin{theorem}
            A function $f(z)$ is analytic if and only if
            its real and complex parts are harmonic
            conjugates of each other.
        \end{theorem}
        \begin{definition}
            A level curve of a function $f(x,y)$ is
            a curve in $\mathbb{R}^{2}$ such that
            $f$ is constant on that curve.
        \end{definition}
        One of the most basic and fundamental results from
        complex variables is Euler's Formula:
        \begin{equation*}
            \exp(i\theta)=\cos(\theta)+i\sin(\theta)
        \end{equation*}
\end{document}
    %        \documentclass[crop=false,class=book,oneside]{standalone}                      %
%----------------------------------Preamble------------------------------------%
%---------------------------Packages----------------------------%
\usepackage{geometry}
\geometry{b5paper, margin=1.0in}
\usepackage[T1]{fontenc}
\usepackage{graphicx, float}            % Graphics/Images.
\usepackage{natbib}                     % For bibliographies.
\bibliographystyle{agsm}                % Bibliography style.
\usepackage[french, english]{babel}     % Language typesetting.
\usepackage[dvipsnames]{xcolor}         % Color names.
\usepackage{listings, lstlinebgrd}      % Verbatim-Like Tools.
\usepackage{mathtools, esint, mathrsfs} % amsmath and integrals.
\usepackage{amsthm, amsfonts}           % Fonts and theorems.
\usepackage{tabularx}
\usepackage{tcolorbox}                  % Frames around theorems.
\usepackage{upgreek}                    % Non-Italic Greek.
\usepackage{paracol}                    % Two-column styling.
\usepackage{wrapfig}                    % Wrap text around figure.
\usepackage{fmtcount, etoolbox}         % For the \book{} command.
\usepackage[newparttoc]{titlesec}       % Formatting chapter, etc.
\usepackage{titletoc}                   % Allows \book in toc.
\usepackage[nottoc]{tocbibind}          % Bibliography in toc.
\usepackage[titles]{tocloft}            % ToC formatting.
\usepackage{multicol, enumitem}         % Multi-column/enumerate.
\usepackage{import}                     % Import external files.
\usepackage{pgfplots, tikz}             % Drawing/graphing tools.
\usetikzlibrary{
    calc,                   % Calculating right angles and more.
    angles,                 % Drawing angles within triangles.
    arrows.meta,            % Latex and Stealth arrows.
    quotes,                 % Adding labels to angles.
    positioning,            % Relative positioning of nodes.
    decorations.markings,   % Adding arrows in the middle of a line.
    patterns,
    arrows,
    shapes,
    shapes.geometric,
    cd,
    hobby,
    babel
}                                       % Libraries for tikz.
\pgfplotsset{compat=1.9}                % Version of pgfplots.
\usepackage[font=scriptsize,
            labelformat=simple,
            labelsep=colon]{subcaption} % Subfigure captions.
\usepackage[font={scriptsize},
            hypcap=true,
            labelsep=colon]{caption}    % Figure captions.
\usepackage{hyperref}                   % Allows for hyperlinks.
\hypersetup{
    colorlinks=true,
    linkcolor=blue,
    filecolor=magenta,
    urlcolor=Cerulean,
    citecolor=SkyBlue
}                           % Colors for hyperref.
\usepackage[toc,acronym,nogroupskip]{glossaries} % Glossaries and acronyms.
\usepackage[subpreambles=false]{standalone}      % Complileable sub files.

% Various font stuff from kiwi.
% Use this for Times text and Computer Modern math
%\usepackage{times}

% Quite nice
%\usepackage[charter, greekfamily=, greekuppercase=italicized]{mathdesign}
%\usepackage[utopia, greekuppercase=italicized]{mathdesign}    % Math is narrower

% Use this for Times text and math
%\usepackage{newtxtext}
%\usepackage[libertine,cmintegrals]{newtxmath}
%\usepackage{fix-cm}

%\usepackage{txfontsb}
% or
%\usepackage{mathptmx}

%\usepackage[scaled=0.92]{helvet}
%\renewcommand{\rmdefault}{ptm}

%\usepackage{mathpazo}    % add possibly `sc` and `osf` options
%\usepackage{eulervm}

%\usepackage{fourier}
%\renewcommand{\rmdefault}{ptm}
%\usepackage{mathptm}

%\usepackage{fontspec}
%\setmainfont{lmodern}

%\usepackage[varg]{txfonts}
%\usepackage{fouriernc}
%\usepackage{mathpazo}

%\usepackage{bookman}
%\usepackage[scaled]{uarial}
%\usepackage[scaled]{helvet}
%\renewcommand*\familydefault{\sfdefault}
%\usepackage[math]{anttor}

%\newcommand\fgeorgia{\fontfamily{jvn}\selectfont}
%\newcommand\ftimes{\fontfamily{ptm}\selectfont}
%\newcommand\fhelvetica{\fontfamily{phv}\selectfont}
%\newcommand\fcourier{\fontfamily{pcr}\selectfont}
%\newcommand\fbookman{\fontfamily{pbk}\selectfont}
%\newcommand\fnewcentury{\fontfamily{pnc}\selectfont}
%\newcommand\fpalatino{\fontfamily{ppl}\selectfont}
%\newcommand\favantgarde{\fontfamily{pag}\selectfont}
%\newcommand\fnormal{\normalfont}
%\newcommand\fsize[1]{\ifnum#1>0\fontsize{#1}{#1}\selectfont\else\normalsize\fi}
%------------------------Theorem Styles-------------------------%
% Define theorem style for default spacing and normal font.
\newtheoremstyle{normal}
    {\topsep}               % Amount of space above the theorem.
    {\topsep}               % Amount of space below the theorem.
    {}                      % Font used for body of theorem.
    {}                      % Measure of space to indent.
    {\bfseries}             % Font of the header of the theorem.
    {}                      % Punctuation between head and body.
    {.5em}                  % Space after theorem head.
    {}

% Define theorem style for default spacing with italicized font.
\newtheoremstyle{normalit}{\topsep}{\topsep}
                {\itshape}{}{\bfseries}{}{.5em}{}

% Italic header environment.
\newtheoremstyle{thmit}{\topsep}{\topsep}{}{}{\itshape}{}{0.5em}{}

% Define italicized environments.
\theoremstyle{normalit}
\newtheorem{theorem}{Theorem}[section]
\newtheorem{lemma}{Lemma}[section]
\newtheorem{corollary}{Corollary}[section]
\newtheorem{proposition}{Proposition}[section]
\newtheorem*{theorem*}{Theorem}

% Define environments with italic headers.
\theoremstyle{thmit}
\newtheorem*{solution}{Solution}
\newtheorem*{fsolution}{Solution}

% Define default environments.
\theoremstyle{normal}
\newtheorem{example}{Example}[section]
\newtheorem{definition}{Definition}[section]
\newtheorem{problem}{Problem}[section]
\newtheorem{question}{Question}[section]
\newtheorem{remark}{Remark}[section]
\newtheorem{properties}{Properties}[section]
\newtheorem{notation}{Notation}[section]
\newtheorem{axiom}{Axiom}[section]
\newtheorem*{properties*}{Properties}
\newtheorem*{remark*}{Remark}
\newtheorem*{definition*}{Definition}
\theoremstyle{plain}

% Define framed environment.
\tcbuselibrary{most}
\newtcbtheorem[use counter*=theorem]{ftheorem}{Theorem}%
    {colback=green!5,colframe=green!35!black,
     fonttitle=\bfseries\upshape}{th}

\newtcbtheorem[use counter*=example]{fdefinition}{Definition}%
    {fonttitle=\bfseries\upshape,
     colback=blue!5!white,colframe=blue!75!black}{def}

\newtcbtheorem[use counter*=example]{fexample}{Example}%
    {fonttitle=\bfseries\upshape,
     colback=red!5!white,colframe=red!75!black}{ex}

\newtcbtheorem[use counter*=notation]{fnotation}{Notation}%
    {fonttitle=\bfseries\upshape,
     colback=SeaGreen!5!white,colframe=SeaGreen!75!black}{ex}

\newtcbtheorem[use counter*=corollary]{fcorollary}{Corollary}%
    {fonttitle=\bfseries\upshape,
     colback=Orchid!5!white,colframe=Orchid!75!black}{ex}

\newenvironment{bproof}{\textit{Proof.}}{\hfill$\square$}
\tcolorboxenvironment{bproof}{blanker,breakable,left=5mm,
                             before skip=10pt,after skip=10pt,
                             borderline west={1mm}{0pt}{red}}
\tcolorboxenvironment{fsolution}
    {enhanced jigsaw,colframe=cyan,interior hidden,breakable}

%--------------------Declared Math Operators--------------------%
\DeclareMathOperator{\Refl}{Refl}           % Reflection operator.
\DeclareMathOperator{\Span}{Span}           % Span of a set of vectors.
\DeclareMathOperator{\Card}{Card}           % Cardinality of set.
\DeclareMathOperator{\Ord}{Ord}             % Ordinal of ordered set.
\DeclareMathOperator{\Tr}{Tr}               % Trace of matrix.
\DeclareMathOperator{\adjoint}{adj}         % Adjoint of matrix.
\DeclareMathOperator{\rk}{rk}               % Rank of operator.
\DeclareMathOperator{\nul}{nul}             % Null space of operator.
\DeclareMathOperator{\sgn}{sgn}             % Sign of a number.
\DeclareMathOperator{\multideg}{mutlideg}   % Multi-Degree (Graphs).
\DeclareMathOperator{\GCD}{GCD}             % Greatest common denominator.
\DeclareMathOperator{\LM}{LM}               % Leading monomial
\DeclareMathOperator{\LC}{LC}               % Leading coefficient.
\DeclareMathOperator{\LT}{LT}               % Leading term.
\DeclareMathOperator{\LCM}{LCM}             % Least common multiple.
\DeclareMathOperator{\Mon}{Mon}             % Monomial.
\DeclareMathOperator{\Spec}{Spec}           % Spectrum.
\DeclareMathOperator{\proj}{proj}           % Projection.
\DeclareMathOperator{\comp}{comp}           % Component.
\DeclareMathOperator{\sinc}{sinc}           % Sinc function.
\DeclareMathOperator{\Ima}{Im}              % Image of operator.
\DeclareMathOperator{\Prin}{Prin}           % Principal value.
\DeclareMathOperator{\Mod}{mod}             % Modulus.
%------------------------New Commands---------------------------%
\DeclarePairedDelimiter\norm{\lVert}{\rVert}
\DeclarePairedDelimiter\ceil{\lceil}{\rceil}
\DeclarePairedDelimiter\floor{\lfloor}{\rfloor}
\newcommand*\diff{\mathop{}\!\mathrm{d}}
\newcommand*\Diff[1]{\mathop{}\!\mathrm{d^#1}}
\renewcommand{\mod}{\ \Mod}
\renewcommand*{\glstextformat}[1]{\textcolor{RoyalBlue}{#1}}
\renewcommand{\glsnamefont}[1]{\textbf{#1}}
\renewcommand\labelitemii{$\circ$}
\renewcommand\thesubfigure{\arabic{chapter}.\arabic{figure}}
\renewcommand\thesubfigure{%
    \arabic{chapter}.\arabic{figure}.\arabic{subfigure}}
\addto\captionsenglish{\renewcommand{\figurename}{Fig.}}
%------------------------Book Command---------------------------%
\makeatletter
\renewcommand\@pnumwidth{1cm}
\newcounter{book}
\renewcommand\thebook{\@Roman\c@book}
\newcommand\book{%
    \if@openright
        \cleardoublepage
    \else
        \clearpage
    \fi
    \thispagestyle{plain}%
    \if@twocolumn
        \onecolumn
        \@tempswatrue
    \else
        \@tempswafalse
    \fi
    \null\vfil
    \secdef\@book\@sbook
}
\def\@book[#1]#2{%
    \ifnum \c@secnumdepth >-3\relax
        \refstepcounter{book}%
        \addcontentsline{toc}{book}{
            \bookname\ \thebook:\hspace{1em}#1
        }
    \else
        \addcontentsline{toc}{book}{#1}%
    \fi
    \markboth{}{}%
    {\centering
     \interlinepenalty \@M
     \normalfont
     \ifnum \c@secnumdepth >-2\relax
       \huge\bfseries \bookname\nobreakspace\thebook
       \par
       \vskip 20\p@
     \fi
     \Huge \bfseries #2\par}%
    \@endbook}
\def\@sbook#1{%
    {\centering
     \interlinepenalty \@M
     \normalfont
     \Huge \bfseries #1\par}%
    \@endbook}
\def\@endbook{
    \vfil\newpage
        \if@twoside
            \if@openright
                \null
                \thispagestyle{empty}%
                \newpage
            \fi
        \fi
        \if@tempswa
            \twocolumn
        \fi
}
\newcommand*\l@book[2]{%
    \ifnum \c@tocdepth >-2\relax
        \addpenalty{-\@highpenalty}%
        \addvspace{2.25em \@plus\p@}%
        \setlength\@tempdima{3em}%
        \begingroup
            \parindent \z@ \rightskip \@pnumwidth
            \parfillskip -\@pnumwidth
            {
                \leavevmode
                \Large \bfseries #1\hfil \hb@xt@\@pnumwidth{
                    \hss #2
                }
            }
            \par
            \nobreak
            \global\@nobreaktrue
            \everypar{\global\@nobreakfalse\everypar{}}%
        \endgroup
    \fi}
\newcommand\bookname{Book}
\renewcommand{\thebook}{\texorpdfstring{\Numberstring{book}}{book}}
\providecommand*{\toclevel@book}{-2}
\makeatother
\titlecontents{chapter}[0pt]
    {\bfseries}
    {\chaptername\ \thecontentslabel:\quad}
    {}
    {\hfill\contentspage}
\titleformat{\part}[display]
    {\Large\bfseries}
    {\partname\nobreakspace\thepart}
    {0mm}
    {\Huge\bfseries}
    \titlecontents{part}[0pt]
    {\large\bfseries}
    {\partname\ \thecontentslabel: \quad}
    {}
    {\hfill\contentspage}
\newcommand{\MarkRightAngle}[4][.3cm]
    {\coordinate (tempa) at ($(#3)!#1!(#2)$);
     \coordinate (tempb) at ($(#3)!#1!(#4)$);
     \coordinate (tempc) at ($(tempa)!0.5!(tempb)$);%midpoint
     \draw (tempa) -- ($(#3)!2!(tempc)$) -- (tempb);}
%--------------------------LENGTHS------------------------------%
% Spacings for the Table of Contents.
\addtolength{\cftsecnumwidth}{1ex}
\addtolength{\cftsubsecindent}{1ex}
\addtolength{\cftsubsecnumwidth}{1ex}
\addtolength{\cftfignumwidth}{1ex}
\addtolength{\cfttabnumwidth}{1ex}

% Spacing for multi-column and enumerate environments.
\setlength{\multicolsep}{6pt}
\setlist[enumerate]{itemsep=0pt,topsep=3pt}

% Indent and paragraph spacing.
\setlength{\parindent}{0em}
\setlength{\parskip}{0em}                                                           %
%---------------------------------tikz Path------------------------------------%
\makeatletter                                                                  %
    \def\input@path{{../../../tikz/}}                                          %
\makeatother                                                                   %
%----------------------------------GLOSSARY------------------------------------%
\makeglossaries                                                                %
\loadglsentries{glossary}                                                      %
\loadglsentries{acronym}                                                       %
%--------------------------------Main Document---------------------------------%
\begin{document}
    \ifx\ifmain\undefined
        \pagenumbering{roman}
        \title{Functional Analysis}
        \author{Ryan Maguire}
        \date{\vspace{-5ex}}
        \maketitle
        \tableofcontents
        \clearpage
        \chapter*{Functional Analysis}
        \addcontentsline{toc}{chapter}{Functional Analysis}
        \markboth{}{FUNCTIONAL ANALYSIS}
        \vspace{10ex}
        \setcounter{chapter}{1}
        \pagenumbering{arabic}
    \else
        \chapter{Functional Analysis}
    \fi
    \section{Metric Spaces}
        \subsection{Basic Definitions}
            Functional analysis is concerned with normed spaces.
            This is a vector space $V$ with a function, called
            a norm, from $V$ to $[0,\infty)$. This is usually
            written $\norm{\mathbf{x}}$ for an element
            $\mathbf{x}\in{V}$. This norm must satisfy the
            folllowing for all $\mathbf{x}$, $\mathbf{y}\in{V}$:
            \begin{enumerate}
                \item $\norm{\mathbf{x}}=0$ if and only
                      if $\mathbf{x}=\mathbf{0}$
                      \hfill[Definiteness]
                \item $\norm{c\mathbf{x}}=|c|\norm{\mathbf{x}}$
                      for all $c\in\mathbb{R}$.
                      \hfill[Positiveness]
                \item $\norm{\mathbf{x}+\mathbf{y}}%
                       \leq\norm{\mathbf{x}}+\norm{\mathbf{y}}$
                      \hfill[Triangle Inequality]
            \end{enumerate}
            \begin{example}
                $\mathbb{R}$ with $\norm{x}=|x|$ and
                $\mathbb{R}^{n}$ with $\norm{\mathbf{x}}_{2}$
                defined by:
                \begin{equation*}
                    \norm{\mathbf{x}}_{2}
                    =\sqrt{\sum_{k=1}^{n}x_{k}^{2}}
                \end{equation*}
                are some of the most commonly used normed spaces.
                $\mathbb{R}^{n}$ can be thought of a the set
                of vectors in $n$ dimensions and the norm
                $\norm{\mathbf{x}}_{2}$ can be thought of as
                the length of $\mathbf{x}$ using the
                Pythagorean Theorem. There are other norms one
                can define on $\mathbb{R}^{n}$. A common
                one is the $p$ norm, defined by:
                \begin{equation*}
                    \norm{\mathbf{x}}_{p}
                    =\Big(\sum_{k=1}^{n}x_{k}^{p}\Big)^{1/p}
                \end{equation*}
                Together, $(\mathbb{R}^{n},\norm{}_{p})$ defines
                a normed space for all $p\geq{1}$. Another type
                of norm on $\mathbb{R}^{n}$ is the
                $p\textrm{-adic}$ norm.
            \end{example}
            A common family of sets,
            which we will deal with frequently, are the
            $\ell^{p}$ spaces.
            \begin{definition}
                $\ell^{p}$ is the set of all sequences
                $x:\mathbb{R}\rightarrow\mathbb{R}$ such that
                the sequence of partial sums
                $S:\mathbb{N}\rightarrow\mathbb{R}$ defined by
                $S_{N}=\sum_{n=1}^{N}|x_{n}|^{p}$ is bounded.
            \end{definition}
            \begin{definition}
                The $p$ norm on $\ell^{q}$ is the function
                $\norm{}_{p}:\ell^{q}\rightarrow\mathbb{R}$
                defined by:
                \begin{equation*}
                    \norm{x}_{p}
                    =\Big(\sum_{k=1}^{\infty}x_{k}^{p}\Big)^{1/p}
                \end{equation*}
            \end{definition}
            \begin{theorem}
                If $p\geq{1}$, then $(\ell^{p},\norm{}_{p})$
                is a normed space.
            \end{theorem}
            \begin{definition}
                The supremum norm on $\mathbb{R}^{n}$,
                $\norm{}_{\infty}:%
                 \mathbb{R}^{n}\rightarrow\mathbb{R}$,
                is the function:
                \begin{equation*}
                    \norm{\mathbf{x}}_{\infty}
                    =\max\{|x_{1}|,\hdots,|x_{n}|\}
                \end{equation*}
            \end{definition}
            \begin{theorem}
                If $n\in\mathbb{N}$, then
                $(\mathbb{R}^{n},\norm{}_{\infty})$ is a
                normed space.
            \end{theorem}
            \begin{definition}
                $\ell^{\infty}$ is the set of sequences
                $x:\mathbb{N}\rightarrow\mathbb{R}$ such that
                $x$ is bounded.
            \end{definition}
            \begin{definition}
                The supremum norm on $\ell^{\infty}$
                is the function
                $\norm{}_{\infty}:%
                 \ell^{\infty}\rightarrow\mathbb{R}$
                defined by:
                \begin{equation*}
                    \norm{x}_{\infty}
                    =\sup\{|x_{n}|:n\in\mathbb{N}\}
                \end{equation*}
            \end{definition}
            \begin{theorem}
                If $\norm{}_{\infty}$ is the
                supremum norm on $\ell^{\infty}$, then 
                $(\ell^{\infty},\norm{}_{\infty})$
                is a normed space.
            \end{theorem}
            From the fact that $\ell^{\infty}$ is a normed space
            we have that the set of convergent sequences,
            again with the $\norm{}_{\infty}$ norm, is also
            a normed space. The set of null sequences, which
            is the set of sequences that converge to zero,
            is also a normed space.
            A stranger normed space
            is the set of all bounded continuous functions
            $f:S\rightarrow\infty$ with norm
            $\norm{f}=\sup\{|f(x)|\}$. Furthermore, the
            set of all integrable functions with
            bounded integrals, with norm
            $(\int_{S}|f|^{p})^{1/p}$. If you allow integral
            to mean Lebesgue Integrable, then this becomes
            a special space denoted $L^{p}(S)$.
            \begin{definition}
                The Sobolev Space, denoted $W^{n,p}([a,b])$
                is the set of functions
                $f:[a,b]\rightarrow\mathbb{R}$ such that:
                \begin{equation*}
                    \int_{a}^{b}\sum_{k=0}^{n}
                    |f^{(k)}(x)|^{p}\diff{x}<\infty
                \end{equation*}
            \end{definition}
            \begin{definition}
                The $p$ norm on the Sobolev space
                $W^{n,q}([a,b])$ is the function
                $\norm{}_{p}:W^{n,q}([a,b])\rightarrow\mathbb{R}$
                defined by:
                \begin{equation*}
                    \Big(\int_{a}^{b}\sum_{k=0}^{n}
                    |f^{(k)}(x)|^{p}\diff{x}\Big)^{1/p}
                \end{equation*}
            \end{definition}
            \begin{theorem}
                If $p\geq{1}$, then
                $(W^{n,p}([a,b]),\norm{}_{p})$ is a
                normed space.
            \end{theorem}
            A lot of the things we wish to
            prove don't rely on the fact that all of these
            spaces are vector spaces. Really, we only care about
            the properties that the norm on the space has.
            What matters is that there's a set and a notion
            of distance on the set. This abstraction is the
            fundamental concept of a metric space.
            \begin{definition}
                A metric space is a set $S$ and a function
                $d:{X}\times{X}\rightarrow[0,\infty)$ such that:
                \begin{enumerate}
                    \item For all $x$, $y\in{X}$, $d(x,y)=0$
                          if and only if $x=y$.
                          \hfill[Definitenes]
                    \item For all $x$, $y\in{X}$,
                          $d(x,y)=d(y,x)$
                          \hfill[Symmetry]
                    \item For all $x$, $y\in{X}$,
                          $d(x,z)\leq{d(x,y)+d(y,z)}$
                          \hfill[Triangile Inequality]
                \end{enumerate}
            \end{definition}
            It turns out
            that we can actually write the following:
            \begin{definition}
                A metric space is a set $X$ and a function
                $d:{X}\times{X}\rightarrow\mathbb{R}$
                such that:
                \begin{enumerate}
                    \item $d(x,y)=0$ if and only if
                          $x=y$.
                    \item $d(x,z)\leq{d(x,y)+d(z,y)}$
                \end{enumerate}
            \end{definition}
            By writing the triangle inequality in this
            way, symmetry comes for free
            (The fact that $d(x,y)=d(y,x)$), as well
            as positivity (The fact that $d(x,y)\geq{0}$).
            Since it's easier to prove two things are
            true, rather than four things, it's nice to
            take this as the definition of a metric space,
            and then prove that the two definitions are
            equivalent.
            In a metric space $(X,d)$, $d$ is often called the
            \textit{distance function} or
            \textit{metric function}. It is meant to be an
            abstract mimicry of the absolute value function
            that is used with real numbers. Definiteness
            says the only point that is zero meters from a
            point $x$ is $x$ itself. Symmetry says the distance
            walking from $x$ to $y$ is the same as the distance
            walking from $y$ to $x$. The last rule stems from
            Euclidean geometry. It says walking from $x$ to $z$
            is shorter than (or equal to) walking from
            $x$ to $y$ and then $y$ to $z$. In Euclidean
            geometry equality is achieved only when
            $y$ lies between $x$ and $z$. In
            abstract metric spaces there may be no such
            thing as a \textit{line} between two points,
            so we need to be careful.
            \begin{example}
                $\mathbb{R}^{n}$ (for $1\leq{p}<\infty$):
                \begin{equation*}
                    d_{p}(\mathbf{x},\mathbf{y})=
                    \big(
                        \sum_{k=1}^{n}|x_{k}-y_{k}|
                    \big)^{1/p}
                    =\norm{\mathbf{x}-\mathbf{y}}_{p}
                \end{equation*}
            \end{example}
            \begin{example}
                In $\ell^{p}$, which are sequences for
                which
                $\sum_{k=1}^{\infty}|x_{k}|^{p}<\infty$,
                $d_{p}(x,y)$ forms a metric, as well
                as
                $d_{\infty}(x,y)=\sup\{|x_{k}-y_{k}|\}$,
                which is called the supremum norm.
            \end{example}
            \begin{example}
                $C(S,\mathbb{R})$, which is the
                set of continuous functions from
                $S$ to $\mathbb{R}$, letting
                $L^{p}(S)$ be the set of of functions
                such that:
                \begin{equation*}
                    \int_{S}|x(t)|^{p}<\infty
                \end{equation*}
                Then the following is a metric:
                \begin{equation*}
                    d_{p}(x,y)=
                    \bigg(
                        \int_{S}|x(t)-y(t)|^{p}dt
                    \bigg)^{1/p}
                \end{equation*}
                Also,
                $d_{\infty}(x,y)=\sup\{|x(t)-y(t)|\}$,
                which is called the supremum norm.
            \end{example}
            \begin{example}
                Let $C$ be the set of sequences such that
                $x_{n}\rightarrow{0}$. Then, with
                $d_{p}$, this forms a metric space.
                If $C_{0}$ is set of sequences with
                only finitely many non-zero terms,
                then
                $C_{0}\subset{C}\subset{\ell^{\infty}}$.
                Is there a sequence $x\in{C}$ such
                that, for all $1\leq{p}<\infty$,
                $x\notin{\ell^{p}}$.
            \end{example}
            Since the image of the metric function
            lies in $\mathbb{R}$,
            we may speak of \textit{convergence}
            in metric spaces.
            \begin{definition}
                A convergent sequence in a metric space
                $(X,d)$ is a sequence
                $x:\mathbb{N}\rightarrow{X}$ such that there
                is an $a\in{X}$ such that
                $d(a,x_{n})\rightarrow{0}$.
            \end{definition}
            \begin{definition}
                A limit of a sequence
                $x$ in a metric space $(X,d)$ is an
                $a\in{X}$ such that
                $d(x_{n},a)\rightarrow{0}$.
            \end{definition}
            Much like convergence in real numbers, limits
            in metric spaces are unique.
            \begin{theorem}
                \label{thm:Funct:Limit_of_Metric_Sequence_Unique}
                If $(X,d)$ is a metric space,
                $x:\mathbb{N}\rightarrow{X}$
                is a convergence sequence in $X$,
                and if $a$ and $b$ are limits of $x$,
                then $a=b$.
            \end{theorem}
            \begin{proof}
                For suppose not. As
                $(X,d)$ is a metric space, $d(a,b)>0$.
                Let $\varepsilon=\frac{d(a,b)}{4}$.
                Then, as $d(a,x_{n})\rightarrow{0}$
                and $\varepsilon>0$, there is an
                $N_{1}\in\mathbb{N}$ such that, for all
                $n>N_{1}$, $d(a,x_{n})<\varepsilon$. But,
                as $d(b,x_{n})\rightarrow{0}$ and
                $\varepsilon>0$, there is an $N_{2}$ such
                that, for all $n>N_{2}$,
                $d(b,x_{n})<\varepsilon$.
                Let $n=\max\{N_{1},N_{2}\}+1$.
                But then:
                \begin{equation*}
                    d(a,b)\leq{d(a,x_{n})+d(b,x_{n})}
                    <2\varepsilon=\frac{d(a,b)}{2}
                \end{equation*}
                A contradiction. Therefore, $a$ is unique.
            \end{proof}
            \begin{theorem}
                If $(X,d)$ be a metric space and if
                $x$, $y$, $z\in{X}$, then
                $|d(x,z)-d(y,z)|\leq{d(x,y)}$
            \end{theorem}
            \begin{proof}
                Suppose $d(x,z)\geq{d(y,z)}$.
                If $d(x,z)<d(y,z)$, the proof is
                symmetric. Thus we have:
                \begin{equation*}
                    |d(x,z)-d(y,z)|
                    =d(x,z)-d(y,z)
                    \leq{(d(x,y)+d(y,z))-d(y,z)}
                    =d(x,y)
                \end{equation*}
                Therefore,
                $|d(x,z)-d(y,z)|\leq{d(x,y)}$.
            \end{proof}
            \begin{theorem}
                If $(X,d)$ is a metric space
                and $x_{n}\rightarrow{a}$, then
                for all $b\in{X}$,
                $d(x_{n},b)\rightarrow{d(a,b)}$.
            \end{theorem}
            \begin{proof}
                For
                $|d(x_{n},b)-d(a,b)|\leq{d(x_{n},a)}%
                 \rightarrow{0}$.
            \end{proof}
            \begin{theorem}
                If $(V,\norm{})$ is a normed space
                and if $d:{V}\times{V}\rightarrow[0,\infty)$
                is defined by
                $d(\mathbf{x},\mathbf{y})%
                 =\norm{\mathbf{x}-\mathbf{y}}$,
                then $(V,d)$ is a metric space.
            \end{theorem}
            \begin{proof}
                In order:
                \begin{enumerate}
                    \item If $\norm{\mathbf{x}-\mathbf{y}}=0$,
                          then $\mathbf{x}=\mathbf{y}$.
                          Similarly,
                          $\norm{\mathbf{x}-\mathbf{x}}%
                           =\norm{\mathbf{0}}=0$.
                    \item $d(\mathbf{x},\mathbf{y})%
                           =\norm{\mathbf{x}-\mathbf{y}}%
                           =\norm{(-1)(\mathbf{y}-\mathbf{x})}%
                           =|-1|\norm{\mathbf{y}-\mathbf{x}}%
                           =\norm{\mathbf{y}-\mathbf{y}}%
                           =d(y,x)$
                    \item The triangle inequality follows
                          from the triangle inequality that
                          norms have.
                \end{enumerate}
            \end{proof}
            There are metric spaces that have nothing to do
            with vector spaces or norms. Metric spaces are
            a more abstract object. Every normed space
            has an associated metric space since there
            is the ``induced'' metric.
            \begin{example}
                Let $X$ be a set and let
                $d(x,y)=\begin{cases}%
                            0,&x=y\\%
                            1,&{x}\ne{y}%
                        \end{cases}$
                This is the discrete metric on $X$.
            \end{example}
            \begin{example}
                Let $X=\{a,b,c\}$, and
                $d(a,b)=1$, $d(b,c)=2$. What value
                must $d(a,c)$ have if $d$ is a metric on $X$?
                Consider the following table:
                \begin{table}[H]
                    \captionsetup{type=table}
                    \centering
                    \begin{tabular}{|c|c|c|c|}
                        \hline
                        $X$&a&b&c\\
                        \hline
                        a&0&1&?\\
                        \hline
                        b&1&0&2\\
                        \hline
                        c&?&2&0\\
                        \hline
                    \end{tabular}
                \end{table}
                This obeys everything except the triangle
                inequality. We must pick $d(a,c)$
                such that this is upheld.
                So we need the following:
                \begin{align*}
                    d(a,b)&\leq{d(a,c)+d(c,b)}&
                    d(a,c)&\leq{d(a,b)+d(b,c)}&
                    d(b,c)&\leq{d(b,a)+d(a,c)}\\
                    \Rightarrow{1}&\leq{2+d(a,c)}&
                    \Rightarrow{d(a,c)}&\leq{3}&
                    \Rightarrow{2}&\leq{1+d(a,c)}
                \end{align*}
                So we need $1\leq{d(a,c)}\leq{3}$.
                Pick $d(a,c)=2$.
                This makes $(X,d)$ a metric space.
            \end{example}
            \begin{example}
                Let $X=\mathbb{R}$ and $d(x,y)=|x-y|$.
                Then $(X,d)$ is a metric space.
            \end{example}
            \begin{example}
                $\mathbb{R}$ with
                $d(x,y)=|f(x)-f(y)|$, where
                $f:\mathbb{R}\rightarrow\mathbb{R}$
                is injective, is a metric space.
                Let $f$ be a real-valued function. Then
                from the triangle inequality
                \begin{equation*}
                    |f(x)-f(y)|\leq|f(x)-f(z)|+|f(z)-f(y)|
                \end{equation*}
                Therefore $d$ obeys the triangle inequality.
                It also obeys symmetry, for:
                \begin{equation*}
                    |f(x)-f(y)|=|(-1)(f(y)-f(x))|=|f(y)-f(x)|
                \end{equation*}
                The absolute value function is doing
                most of the work.
                But finally we require that
                $|f(x)-f(y)|=0$ if and only if
                $x=y$. But $|f(x)-f(y)|=0$ if and only
                if $f(x)=f(y)$. So we require that $f$
                is injective. If $f$ is not injective,
                then there exists $x_{1}$, $x_{2}$
                such that
                $x_{1}\ne{x_{2}}$ and yet
                $f(x_{1})=f(x_{2})$. But then
                $|f(x_{1})-f(x_{2})|=0$, contradicting the
                fact that this is a metric. If $f$ is
                injective, then this is a metric. Note
                injective functions need not be
                continuous, and can be very crazy.
            \end{example}
            \begin{example}
                $\mathbb{R}$ with
                $d(x,y)=|\tan^{-1}(x)-\tan^{-1}(y)|$ is a
                metric. Moreover, $d(x,y)<\pi$ for all
                $x,y\in\mathbb{R}$. Thus, we have found
                a metric that makes $\mathbb{R}$ a bounded
                set. As a fun fact, $x_{n}=n$ is a Cauchy
                sequence in this metric space, but
                this sequence does not converge to anything.
                Thus we've found a metric on
                $\mathbb{R}$ such that
                $(\mathbb{R},d)$ is not complete.
            \end{example}
            \begin{example}
            Can $d(x,y)=f(x-y)$ be a metric on $\mathbb{R}$
            if $f$ is differentiable? Not everywhere.
            $f$ can not be differentiable at the origin for
            $d(x,y)=f(x-y)$ to be a metric function, however
            $f$ can be differentiable everywhere else. Use
            $f(x)=|x|$ as an example.
            If $f(x-y)$ is a metric, $f$
            must be an even function. But
            then $f'(0)=0$. But $f(x-y)$ also must obey
            the triangle inequality. Therefore:
            \begin{equation*}
                f(2x)\leq{f(x)+f(x)}=2f(x)    
            \end{equation*}
            Define $h(x)$ by:
            \begin{equation*}
                h(x)=
                \left\{
                    \begin{array}{cr}
                    \frac{f(x)}{x},&x\ne{0}\\
                    0,&x=0
                    \end{array}\right.
            \end{equation*}
            Then, from
            the previous statement, $h(2x)\leq{h(x)}$.
            But then:
            \begin{equation*}
                h\Big(\frac{1}{2^{n}}\Big)\leq
                h\Big(\frac{1}{2^{n+1}}\Big)
            \end{equation*}
            But from L'H\^{o}pital's Rule,
            $h(x)\rightarrow{f'(0)}$ as $x\rightarrow{0}$.
            Therefore $h(1)\leq{f'(0)}$. But $h(1)>0$ since
            $f(x-y)$ is a metric, a contradiction.
            Therefore, $f$ can not be differentiable at
            the origin.
            \end{example}
        \subsection{Topology}
            \begin{definition}
                The open ball of radius $r>0$
                about a point $x$ in a metric space
                $(X,d)$ is the set
                $B_{r}(x)=\{y\in{X}:d(x,y)<r\}$
            \end{definition}
            The picture for this is a ``circle'' around the
            point $x$ or radius $r$. However, this circle
            can look very strange for weird metrics.
            \begin{example}
                If $X$ is a set and $d$ is the discrete metric,
                then $B_{r}(x)$ is either the point $x$
                (If $r\leq{1}$), or it is the entire set $X$.
            \end{example}
            \begin{example}
                With $X=\mathbb{R}$ and $d$ the standard metric
                $d(x,y)=|x-y|$, we have $B_{r}(x)$ is simply
                the open interval $(x-r,x+r)$.
            \end{example}
            \begin{example}
                \label{EXAMPLE:FUNCTIONAL:UNIT_BALLS_EXAMPLE}
                Let $X=\mathbb{R}^{2}$ and define
                $d_{p}(x,y)%
                 =(|x_{1}-y_{1}|^{p}+|x_{2}-y_{2}|^{p})^{1/p}$.
                For $p=2$, an open ball is a circle around
                the point $(x,y)$ of radius $r$. For $p=1$,
                we have ``diamonds'' around the point $x$.
                And for $p=\infty$ we have a square
                around $x$.
                Let $X=\mathbb{R}^{2}$ and let $d$ be the metric
                such that you can only travel parallel to the
                $y$ axis, or along the $x$ axis.
                Consider the unit balls in $(X,d)$
                about the following points:
                \begin{enumerate}
                    \begin{multicols}{4}
                        \item[a.] $(0,0)$
                        \item[b.] $(0,1)$
                        \item[c.] $(0,\frac{1}{2})$
                        \item[d.] $(\frac{1}{2},\frac{1}{2})$
                    \end{multicols}
                \end{enumerate}
                If $\mathbf{x}_{1}=(x_{1},y_{1})$ and
                $\mathbf{x}_{2}=(x_{2},y_{2})$, then we have:
                \begin{equation*}
                    d(\mathbf{x}_{1},\mathbf{x}_{2})=
                    \begin{cases}
                        |y_{2}-y_{1}|,&x_{1}=x_{2}\\
                        |x_{2}-x_{1}|+|y_{1}|+|y_{2}|,
                        &x_{1}\ne{x_{2}}
                    \end{cases}
                \end{equation*}
            About the point $(0,0)$, the unit ball
            is simply points
            $(x,y)$ such that $|x|+|y|<1$. This is a ``diamond.''
            About $(0,1)$, first note that to get to any point
            whose $x$ coordinate is not $0$, you first must travel
            the entirety of the $y$ axis. Since this length is
            already $1$, you can't go left or right
            on the $x$ axis.
            The unit ball is the line segment on the $y$ axis
            between $(0,0)$ and $(0,2)$. For the third one, if
            the $x$ coordinate changes, we have
            $0.5+|y|+|x|<1$, which implies
            $|y|+|x|<0.5$. This is again a diamond, but a
            smaller one. If the $x$ coordinate does not
            change, we have $|y-0.5|<1$. This is another
            line segment. Repeat the same arguments for the
            fourth coordinate. The diagrams are show in
            Fig.~\ref{FUNCTIONAL:HOMEWORK:2:PROBLEM:4:FIGURES}.
        \begin{figure}[H]
            \centering
            \captionsetup{type=figure}
            \subimport{../../../tikz/}{Open_Ball_in_Metric_Space_I}
            \caption{Figures for Example
                     \ref{EXAMPLE:FUNCTIONAL:UNIT_BALLS_EXAMPLE}.}
            \label{FUNCTIONAL:HOMEWORK:2:PROBLEM:4:FIGURES}
        \end{figure}
            \end{example}
            If you have a vector space and a norm on it,
            then the open balls about a point will have the
            property of convexity. Convexity is a vector space
            property, given two points the ``line'' between the
            two remains in the set. Metric spaces have no such
            notion. Since the balls of $\norm{}_{p}$ are not
            convex with $p<1$, we have that $\norm{}_{p}$ is
            a metric on $\mathbb{R}^{n}$
            if and only if $p\geq{1}$.
            \begin{definition}
                An open subset of a metric space
                $(X,d)$ is a set $S\subset{X}$ such that,
                for all $x\in{S}$, there is an
                $r>0$ such that
                $B_{r}(x)\subset{S}$.
            \end{definition}
            \begin{example}
                If $(X,d)$ is a metric space, then
                $X$ is open and $\emptyset$ is open
                (Vacuously true).
            \end{example}
            \begin{theorem}
                If $(X,d)$ is a metric space, $x\in{X}$,
                and $r>0$, then $B_{r}(x)$ is an open
                subset of $X$.
            \end{theorem}
            \begin{proof}
                If $z\in{B_{r}(x)}$, let $t=d(x,z)$.
                Then $0\leq{t}<r$. Let $r'=r-t$.
                But if $y\in{B_{r'}(z)}$, then
                $d(x,y)\leq{d(x,z)+d(y,z)}<t+r'=t+r-t=r$.
                Therefore $B_{r'}(z)\subset{B_{r}(x)}$.
            \end{proof}
            \begin{theorem}
                A finite intersection of open sets is open.
            \end{theorem}
            \begin{proof}
                If $\mathcal{U}_{1},\hdots,\mathcal{U}_{n}$
                are open and if
                $x\in\cap_{k=1}^{n}\mathcal{U}_{k}$, then there
                exists $r_{1},\hdots,r_{n}$ such that
                $B_{r_{i}}(x)\subset\mathcal{U}_{i}$. Let
                $r=\min\{r_{1},\hdots,r_{n}\}$. Then
                $B_{r}(x)\subset\cap_{k=1}^{n}\mathcal{U}_{i}$
            \end{proof}
            \begin{theorem}
                Arbitrary unions of open sets are open.
            \end{theorem}
            Infinite intersections need not be open.
            The proof above would fail since the
            $r_{i}$ can form a sequence tending to zero.
            But indeed, let $X=\mathbb{R}$ and let
            $d(x,y)=|x-y|$, and take
            $\mathcal{U}_{n}=(-\frac{1}{n},\frac{1}{n})$.
            Then all of the $\mathcal{U}_{n}$ are open,
            yet the intersection, which is the set $\{0\}$,
            is not open. All of this mumbo-jumbo creates
            the more general notion of a topological space.
            \begin{definition}
                A topological space is a set $X$ and a
                subset $\tau\subset\mathcal{P}(X)$ such that:
                \begin{enumerate}
                    \item $\emptyset,X\in\tau$
                    \item Finite intersections of sets in $\tau$
                          are also sets in $\tau$.
                    \item Arbitrary unions of sets in $\tau$
                          are also sets in $\tau$.
                \end{enumerate}
            \end{definition}
            Here, $\mathcal{P}(X)$ denotes the \textit{power set}
            of $X$. This is the set of all subsets of $X$.
            The notion of a topological space generalizes the
            notion of a metric space. There is no notion of
            distance in such spaces, and things can be weird.
            There are topological spaces that have no metric
            associated with them.
            \begin{definition}
                An open subset of a topological space
                $(X,\tau)$ is a set $\mathcal{U}\in\tau$.
            \end{definition}
            \begin{definition}
                An interior point of a subset $S$ of a topological
                space $(X,\tau)$ is a point $x\in{S}$ such that
                there is an open subset $\mathcal{U}\subseteq{S}$
                such that $x\in\mathcal{U}$.
            \end{definition}
            \begin{definition}
                The interior of a subset $S$ of a topological
                space $(X,\tau)$, denoted $\Int(S)$, is the set
                of all interior points of $S$.
            \end{definition}
            \begin{theorem}
                If $S$ is an open subset of
                $(X,\tau)$, then $\Int(S)=S$.
            \end{theorem}
            \begin{definition}
                A function from a metric space
                $(X,d_{X})$ to a metric space $(Y,d_{Y})$
                continuous at $x\in{X}$ is a function
                $f:X\rightarrow{Y}$ such that
                for all $\varepsilon>0$ there is
                a $\delta>0$ such that for all
                $x_{0}\in{X}$ such that
                $d_{X}(x,x_{0})<\delta$, we have
                $d_{Y}(f(x),f(x_{0})<\varepsilon$
            \end{definition}
            \begin{theorem}
                If $(X,\tau)$ is a topological space and
                $S\subseteq{X}$, and if $\mathcal{O}$ is the set
                of all open sets $\mathcal{U}$ such that
                $\mathcal{U}\subseteq{S}$, then
                $\Int(S)=%
                 \bigcup_{\mathcal{U}\in\mathcal{O}}\mathcal{U}$.
            \end{theorem}
            \begin{definition}
                A nowhere dense subset of a topological space
                $(X,\tau)$ is a subset $S\subseteq{X}$ such that
                $\Int(S)=\emptyset$.
            \end{definition}
            \begin{theorem}
                If $(X,d)$ is a metric space,
                $y\in{X}$, then
                $f:X\rightarrow\mathbb{R}$ defined by
                $f(x)=d(x,y)$ is uniformly continuous.
            \end{theorem}
            A surprising theorem, and the entire
            basis of the study of topology, goes as
            follows:
            \begin{theorem}
                If $(X,d_{x})$ and $(Y,d_{Y})$
                are metric spaces, then
                $f:X\rightarrow{Y}$ is continuous
                at $x\in{X}$ if and only if
                for all open subsets
                of $S\subset{Y}$ such that
                $f(x)\in{S}$, $f^{-1}(S)$ is an
                open subset of $X$.
            \end{theorem}
            This allows us to talk about continuous
            functions without a notion of metric.
            Thus, for topological spaces, this is
            the \textit{definition} of continuity.
            When the space we're discussing is a
            metric space, this theorem shows that the
            definition from topology and the defintition
            from real analysis are in fact equivalent.
            \begin{theorem}
                A function $f:X\rightarrow{Y}$ between
                metric spaces is continuous at a point
                $x\in{X}$ if and only if for all
                sequences $x_{n}$ such that
                $d_{X}(x,x_{n})\rightarrow{0}$, we have
                $d_{Y}(f(x),f(x_{n})\rightarrow{0}$.
            \end{theorem}
            We now have three different ways to talk
            about continuity. Topological spaces can be
            nastier, however. We saw in
            Thm.~\ref{thm:Funct:Limit_of_Metric_Sequence_Unique}
            that the limit of a convergent sequence in a
            metric space is unique.
            This is not true in a topological space and there
            are topological spaces with sequences
            which converge to every point in the
            space simultaneously. Indeed, it may be impossible
            to distinguish two points in a topological
            space. The ability to
            ``Separate,'' points is special.
            Hausdorff spaces can, but
            we'll save that for topology.
        \subsubsection{Closed Sets}
            \begin{definition}
                A limit point of a subset
                $S\subset{X}$ of a metric space
                $(X,d)$ is a point $a\in{X}$ such
                that there is a sequence
                $x:\mathbb{N}\rightarrow{S}$ such that
                $d(a,x_{n})\rightarrow{0}$.
            \end{definition}
            \begin{definition}
                A closed subset of a metric space $(X,d)$
                is a set $S$ such that for all $x\in{X}$ such
                that $x$ is a limit point of $S$, $x\in{S}$.
            \end{definition}
            This says that if $S$ is closed, and
            $x$ is a sequence in $S$ such
            that $x_{n}\rightarrow{a}$, then
            $a\in{S}$.
            \begin{example}
                In $\mathbb{R}$, with the standard
                metric, $(a,b)$ is open,
                $\mathbb{R}$ is open (and closed),
                $[a,b]$ is closed,
                $[a,\infty)$ is closed,
                $[a,b)$ is neither closed nor open.
            \end{example}
            \begin{example}
                If $X=(0,1)$, and
                $d(x,y)=|x-y|$, then
                $(0,1)$ is closed. This is because
                there is no sequence that converges
                to a point in the space whose limit
                is not in the space. There are no sequences
                in $X$ which converge to zero or one since,
                as far as $X$ is concerned,
                neither or these points exist.
            \end{example}
            \begin{theorem}
                If $(X,d)$ is a metric space,
                then a subset $S\subset{X}$ is open
                if and only if $X\setminus{S}$ is closed.
            \end{theorem}
            \begin{proof}
                Suppose $S$ is open, and let
                $x_{n}$ be a sequence in $S^{c}$.
                Suppose $x_{n}\rightarrow{x}$ and
                $x\in{S}$. But $S$ is open, and thus
                there is an $\varepsilon>0$ such that
                $B_{\varepsilon}(x)\subset{S}$.
                But $x_{n}\rightarrow{x}$, and thus
                this is an $N\in\mathbb{N}$ such that
                for all $n>N$, $d(x,x_{n})<\varepsilon$.
                But then for all $n>N$,
                $x_{n}\in{B_{\varepsilon}(x)}$. But
                $x_{n}\in{S^{c}}$, a contradiction.
                Therefore, $S^{c}$ is closed. On the
                other hand, if $S^{c}$ is closed
                and there is an $x\in{S}$ such that
                for all $r>0$,
                $B_{r}(x)\cap{S}\ne\emptyset$, then
                for all $n\in\mathbb{N}$ there is
                an $x_{n}\in{S^{c}}$ such that
                $d(x,x_{n})<\frac{1}{n}$. But then
                $x_{n}\rightarrow{x}$, and therefore
                $x\in{S^{c}}$. But $x\in{S}$,
                a contradiction. Thus, $S$ is open.
            \end{proof}
            In topology we take the definition of
            closed sets to be the compliment of open
            sets. This theorem shows that the
            topological definition is equivalent when we
            consider metric spaces.
            \begin{definition}
                The closure of a subset
                $S$ of a metric space
                $(X,d)$, denoted $\overline{S}$,
                is the set of all
                limit points of $S$.
            \end{definition}
            \begin{theorem}
                If $(X,d)$ is a metric space, if
                $S\subset{X}$, and if
                $\Delta$ is the set of all closed subsets
                $\mathcal{C}\subset{X}$ such that
                $S\subset\mathcal{C}$, then:
                $\overline{S}=
                 \bigcap_{\mathcal{C}\in\Delta}
                 \mathcal{C}$
            \end{theorem}
            Thus we may loosely say that
            the closure of a set $S$ is the
            ``Smallest,'' closed set that contains $S$.
            \begin{definition}
                The closed ball of radius $r>0$ about
                a point $x$ in a metric space
                $(X,d)$ is the set:
                \begin{equation*}
                    \overline{B}_{r}(x)=
                    \{y\in{X}:d(x,y)\leq{r}\}
                \end{equation*}
            \end{definition}
            There exists metric spaces $(X,d)$
            such that
            $\overline{B}_{r}(x)\ne\overline{B_{r}(x)}$.
            For take the discrete metric, $r=1$.
            Then the closure of $B_{1}(x)$ is simply
            the point $x$. However, the closed ball
            $\overline{B}_{1}(x)$ is the entire space.
            Metric spaces can be very weird like this.
            They have a property, that given a nested
            sequence of closed balls whose radius
            tends to zero, there is precisely one
            point that lies in the intersection. However,
            if the radius does not tend to zero it is
            possible that the intersection is empty.
            This is very counter-intuitive.
            \begin{definition}
                A dense subset of a metric space $(X,d)$
                is a set $S\subset{X}$ such that
                $\overline{S}=X$.
            \end{definition}
            A subset $S$ is dense in $X$ if every point
            in $X$ can be approximated arbitrarily well
            by points in $S$. For any point $a\in{X}$
            there is a sequence $x\in{S}$
            such that $x_{n}\rightarrow{a}$. The
            classic example is $\mathbb{Q}$ and
            $\mathbb{R}$. Every real number can be
            approximated arbitrary well by a rational
            number. To see this, just take the continued
            fraction of a real number and stop once
            the approximation is less than
            $\varepsilon$. When we say $\mathbb{Q}$ is
            dense in $\mathbb{R}$, we of course mean with
            respect to the standard metric on $\mathbb{R}$.
            $\mathbb{Q}$ is \textbf{not} dense in
            $\mathbb{R}$ with respect to the discrete metric.
            Indeed, if $d$ is the discrete metric on $X$,
            then $S\subset{X}$ is dense in $X$ if and only if
            $S=X$.
            \begin{example}
                $\mathbb{Q}$ is dense in $\mathbb{R}$
                with respect to $d_{p}$ for all
                $p\geq{1}$. This includes
                $d(x,y)=|x-y|$.
            \end{example}
            \begin{example}
                The set of polynomials on the interval
                $[a,b]$ are dense in the set of
                continuous functions on $[a,b]$ with
                respect to the $d_{\infty}$ metric.
                This comes from Weierstrass's Theorem.
            \end{example}
            \begin{example}
                The set of polynomials on $[a,b]$
                is dense in the set of continuous
                functions on $[a,b]$ with respect to
                the $d_{p}$ metric, for $p\geq{1}$. This
                is because:
                \begin{align*}
                    d_{p}(P,x)&=
                    \Big(
                        \int_{a}^{b}|P(t)-x(t)|^{p}\diff{t}
                    \Big)^{1/p}
                    &
                    &=\Big(
                        d_{\infty}(P,x)^{p}\int_{a}^{b}\diff{t}
                    \Big)^{1/p}\\
                    &\leq\Big(\int_{a}^{b}
                        |\max\{P(t)-x(t)\}|^{p}\diff{t}
                    \Big)^{1/p}
                    &
                    &=(b-a)^{1/p}d_{\infty}(P,x)
                \end{align*}
            \end{example}
            \begin{example}
                The continuous functions are not dense
                in the set of integrable functions,
                with respect to the supremum metric
                $d_{\infty}$. This is more or less
                because integrable functions can
                be discontinuous, or have jumps. This
                means, with respect to $d_{\infty}$,
                that no continuous functions could
                approximate such a discontinuous function
                arbitrary well.
            \end{example}
            \begin{definition}
                A separable metric space
                is a metric space $(X,d)$ with
                a countable dense subset $S$.
            \end{definition}
            \begin{example}
                $\mathbb{R}$ is separable, with
                the standard metric, since
                $\mathbb{Q}$ is countable and also
                dense in $\mathbb{R}$.
            \end{example}
            \begin{example}
                The set of continuous functions on
                $[a,b]$ is separable. For
                take the set of polynomials with
                rational coefficients. This can
                be seen as a countable union of
                countably many elements. For let
                $P_{N}$ be the set of polynomials
                of degree $N$ with rational
                coefficients. This is countable,
                and the set of all polynomials with
                rational coefficients is simply the
                union of $P_{N}$ over all $N$. This
                is dense in the set of polynomials,
                and the set of polynomials is dense
                in $C[a,b]$, and thus
                the set of polynomials with rational
                coefficients is dense in $C[a,b]$. Thus
                $C[a,b]$ is separable.
            \end{example}
            \begin{example}
                $\ell^{p}$ is separable with the
                $d_{p}$ metric, simply use elements
                with rational entries. That is,
                sequences of rational numbers.
            \end{example}
            \begin{example}
                $\ell^{p}$ with the $d_{\infty}$ metric
                is NOT separable. Consider the real
                numbers in $(0,1)$.
            \end{example}
        \subsection{Completeness}
            \begin{definition}
                A complete metric space is a metric
                space $(X,d)$ such that every
                Cauchy sequence $x_{n}$
                in $X$ converges to a point in $X$
                with respect to $d$.
            \end{definition}
            Recall that a sequence $x_{n}$ is Cauchy if
            $\forall_{\varepsilon>0}\exists_{N\in\mathbb{N}}:%
             \forall_{n,m>N},d(x_{n},x_{m})<\varepsilon$.
            Convergence with respect to $d$ means that
            $d(x,x_{n})\rightarrow{0}$.
            \begin{example}
                $\mathbb{R}$ with the standard metric
                $d(x,y)=|x-y|$ is complete.
            \end{example}
            \begin{example}
                $(\mathbb{R}^{n},d_{p})$ is also complete
                for all $n\in\mathbb{N}$.
            \end{example}
            Completeness is both a property of the set
            and the metric itself. It is not a topological
            property.
            \begin{example}
                $(\mathbb{R},d)$, where
                $d(x,y)=|\tan^{-1}(x)-\tan^{-1}(y)|$
                is \textit{not} complete. For let
                $x_{n}=n$. This is a Cauchy sequence,
                as one can see from the graph
                of $\tan^{-1}(x)$. That is, because
                $\tan^{-1}(x)\rightarrow{\pi/2}$,
                $x_{n}=n$ is a Cauchy sequence in this
                metric. Being even more rigorous, let
                $\varepsilon>0$ and
                $N=\ceil{\tan(\pi/2-\varepsilon)}$.
                Then, for all $n,m>N$,
                $d(x_{n},x_{m})%
                 =|\tan^{-1}(n)-\tan^{-1}(m)|%
                 <|\pi/2-\tan^{-1}(\min\{n,m\})|%
                 <|\pi/2-(\pi/2-\varepsilon)|%
                 =\varepsilon$. But $x_{n}$ does not
                converge. For suppose not.,
                Suppose $x_{n}=n\rightarrow{x}$.
                Then for $n>x+1$,
                $d(x_{n},x)=|\tan^{-1}(n)-\tan^{-1}(x)|%
                 <|\tan^{-1}(x+1)-\tan^{-1}(x)|$,
                so $d(x_{n},x)\not\rightarrow{0}$.
                The sequence does not converge.
            \end{example}
            Let $X=\mathbb{R}\cup\{-\infty,\infty\}$.
            Let $d:X\times{X}\rightarrow\mathbb{R}$
            be defined by
            \begin{align*}
                d(x,y)
                &=|\tan^{-1}(x)-\tan^{-1}(y)|
                &
                d(x,\infty)
                &=\frac{\pi}{2}-\tan^{-1}(x)\\
                d(-\infty,x)
                &=\frac{\pi}{2}+\tan^{-1}(x)
                &
                d(-\infty,\infty)&=\pi
            \end{align*}
            Then $d$ is a metric on $X$, and moreover
            $(X,d)$ is complete. The counterexample
            we found for $(\mathbb{R},d)$ has been
            ``filled,'' in a sense. The hole is
            no longer there. The sequence $x_{n}=n$
            now converges to $\infty$. Somewhat
            unsurpringly, $\mathbb{R}$ is
            dense in $X$, with respect to
            $d$. Every element in $X$ is the limit of
            a sequence of elements in $\mathbb{R}$.
            \begin{definition}
                A completion of a metric space
                $(X,d)$ is a complete metric space
                $(\tilde{X},\tilde{d})$ such that
                $X\subset{\tilde{X}}$ and
                the restriction of
                $\tilde{d}$ onto $X$ is equal to $d$.
            \end{definition}
            \begin{theorem}
                Every metric space has a completion.
            \end{theorem}
            \begin{definition}
                An isometry between
                metric spaces
                $(X,d_{X})$ and
                $(Y,d_{Y})$ is a function
                $f:X\rightarrow{Y}$ such that
                $d_{X}(x,y)=d_{Y}(f(x),f(y))$
                for all $x,y\in{X}$.
            \end{definition}
            \begin{definition}
                Isometric metric spaces are metric spaces
                with an isometry between them.
            \end{definition}
            \begin{theorem}
                If $(X,d)$ is a metric space
                and $(\tilde{X}_{1},\tilde{d}_{1})$
                and $(\tilde{X}_{2},\tilde{d}_{2})$
                are completions of $(X,d)$, then
                $(\tilde{X}_{1},\tilde{d}_{1})$
                and $(\tilde{X}_{2},\tilde{d}_{2})$
                are isometric.
            \end{theorem}
            This says the completion of a metric space is
            unique up to isometry.
            The Lebesgue space $L^{p}(S)$
            can be defined to be the completion of
            $C(S)$ with respect to the $d_{p}$ metric.
            \begin{theorem}
                $(C(S),d_{\infty})$ is complete.
            \end{theorem}
            \begin{proof}
                Suppose $x_{n}$ is a Cauchy sequence
                and let $\varepsilon>0$. As $x_{n}$ is
                Cauchy, there exists $N\in\mathbb{N}$
                such that for all $n,m>N$,
                $\sup|x_{m}(t)-x_{n}(t)|<\frac{\varepsilon}{3}$.
                But then for all $t\in{S}$,
                $|x_{m}(t)-x_{n}(t)|<\frac{\varepsilon}{3}$,
                for all
                $n,m>N$. That is, if $x_{n}$ is
                a Cauchy sequence in $(C(S),d_{\infty})$,
                then it is a Cauchy sequence in
                $(\mathbb{R},d_{1})$. But
                $(\mathbb{R},d_{1})$ is complete, and
                therefore, for all $t\in{S}$, there is
                an $x(t)$ such that
                $x_{n}(t)\rightarrow{x(t)}$ with respect
                to the $d_{1}$ metric on $\mathbb{R}$. We
                now need to show that $x(t)$ is a continuous
                function. That is, that
                $x(t)\in{C(S)}$. Finally we need to show that
                $x_{n}\rightarrow{d}$ with respect to
                $d_{\infty}$. We need to show that
                for all $\varepsilon>0$ and all $t\in{S}$
                there is a $\delta>0$
                such that for all $|t-t_{0}|<\delta$,
                $|x(t)-x(t_{0})|<\varepsilon$. But for
                all $n,m>N$,
                $\sup\{x_{n}(t)-x_{m}(t)\}<\frac{\varepsilon}{3}$.
                Taking the limit on $m$, we have
                $|x(t)-x_{n}(t)|<\frac{\varepsilon}{2}$.
                But $x_{n}(t)$ is continuous, and thus
                there exists $\delta>0$ such that
                for all $|t-t_{0}|<\delta$,
                $|x_{n}(t)-x_{n}(t_{0})|<\frac{\varepsilon}{3}$.
                But
                $|x(t)-x(t_{0})|\leq%
                  |x(t)-x_{n}(t)|%
                 +|x_{n}(t)-x_{m}(t)|%
                 +|x(t_{0})-x_{n}(t_{0})$
                But
                $|x(t_{0})-x_{n}(t_{0})|<%
                 \sup\{|x(t)-x_{n}(t)|\}<\frac{\varepsilon}{3}$,
                and therefore
                $|x(t)-x(t_{0})|<\varepsilon$.
                So $x(t)$ is continuous.
            \end{proof}
            The Weierstrass Approximation Theorem says that,
            for closed finite intervals $S$,
            $(C(S),d_{\infty})$ is the completion
            of the set of polynomials with respect to
            the $d_{\infty}$ metric. On the other hand,
            $(C[0,1],d_{p})$ is not complete when
            $1\leq{p}<\infty$. For define the following:
            \begin{equation*}
                H(x)=
                \begin{cases}
                    0,&0\leq{x}\leq{\frac{1}{2}}\\
                    1,&\frac{1}{2}<x\leq{1}
                \end{cases}
            \end{equation*}
            This is discontinuous and cannot be
            approximated arbitrarily well
            by any continuous function. However, the
            \textit{area} underneath $H$ can be approximated
            arbitrarily well be continuous functions. For define:
            \begin{equation*}
                x_{n}(t)=
                \begin{cases}
                    0,&0\leq{x}\leq{\frac{1}{2}-\frac{1}{n}}\\
                    n(x-\frac{1}{2}+\frac{1}{n}),
                    &\frac{1}{2}-\frac{1}{n}\leq{x}
                     \leq{\frac{1}{2}}\\
                    1,&\frac{1}{2}<{x}\leq{1}
                \end{cases}
            \end{equation*}
            Then the area under $x_{n}(t)$
            is $\frac{1}{2}+\frac{1}{2n}$, and thus
            $d_{1}(x_{n}(t),x_{m}(t))%
             =|\frac{1}{2m}-\frac{1}{2n}|$,
            and therefore $x_{n}(t)$ is a Cauchy sequence.
            But $x_{n}(t)$ does not converge in
            $(C[0,1],d_{1})$. For suppose not, suppose
            $x_{n}(t)\rightarrow{x(t)}$, and
            $x(t)\in{C[0,1]}$.
            If $x(1/2)\geq{1/2}$, then, as $x(t)$ is
            continuous, there is a $\delta>0$ such that
            for all $|t-1/2|<\delta$,
            $x(t)>1/4$. But then
            $d(x_{n},x)=\int_{0}^{1}|x(t)-x_{n}(t)|dt%
            \geq\int_{1/2-\delta/2}^{1/2}|x(t)-x_{n}(t)|dt$.
            But $|x|=|(x-y)+y|\leq{|x-y|+|y|}$,
            and thus
            $|x|-|y|\leq{|x-y|}$. From this we have
            $d(x_{n}(t),x(t))\geq%
             \int_{1/2-\delta/2}^{1/2}(x(t)-x_{n}(t))dt%
             >\int_{1/2-\delta/2}^{1/2}\frac{1}{4}dt%
             -\int_{0}^{1/2}x_{n}(t)dt%
             =\frac{1}{4}\delta-\frac{1}{2n}%
             \rightarrow{\frac{1}{4}}\delta$.
            But then $d(x_{n}(t),x(t))\not\rightarrow{0}$.
            Therefore $x_{n}(t)$ does not converge.
            \begin{theorem}
                If $1\leq{p}<\infty$, then
                $(\ell^{p},d_{p})$ is complete.
            \end{theorem}
            \begin{proof}
                Let $x_{n}$ be a Cauchy sequence
                in $(e\ell^{p},d_{p})$,
                $x_{n}=x_{n}(1),x_{n}(2),\hdots,x_{n}(k),\hdots$
                Then, for $n,m\in\mathbb{N}$,
                $d_{p}(x_{n},x_{m})%
                 =(%
                    \sum_{k=0}^{\infty}|x_{n}(k)-x_{m}(k)|^{p}%
                  )^{1/p}$
                As $x_{n}$ is Cauchy, for all 
                $\varepsilon>0$ there is an $N\in\mathbb{N}$
                such that for all $n,m>N$,
                $d_{p}(x_{n},x_{m})<\varepsilon$.
                But then, for all $n,m>N$ and all
                $k\in\mathbb{N}$,
                $|x_{n}(k)-x_{m}(k)|^{p}<d_{p}(x_{n},x_{m})^{P}%
                 <\varepsilon^{p}$.
                But then
                $|x_{n}(k)-x_{m}(k)|<\varepsilon$. Therefore
                $x_{n}(k)$ is a Cauchy sequence in
                $(\mathbb{R},d)$, and this metric space is
                complete. Therefore, for all $k\in\mathbb{N}$,
                there is a $z_{k}$ such that
                $x_{n}(k)\rightarrow{z_{k}}$. We now need to
                show that $z_{k}$ is an element of
                $\ell^{p}$ and that
                $x_{n}\rightarrow{z_{k}}$ with respect to
                the $d_{p}$ metric. For let $N\in\mathbb{N}$.
                Then
                $\sum_{k=0}^{N}|x_{n}(k)-x_{m}(k)|^{p}%
                 \leq{\sum_{k=0}^{\infty}|x_{n}(k)-x_{m}(k)|^{p}}%
                 <\varepsilon^{p}$. Taking the limit on $m$,
                we have
                $\sum_{k=0}^{N}|z_{k}-x_{n}(k)|<\varepsilon^{p}$.
                The reason we have written a finite sum is to
                avoid getting into trouble with limits. An
                infinite sum is itself a limit, and taking
                limits of limits can get very messy very easily.
                For example,
                $f(n,m)=\frac{m}{n+m}$. Taking the limit on
                $m$ first results in $1$, whereas taking the
                limit on $n$ first gives you $0$.
                That is,
                $\lim_{n}\lim_{m}f(n,m)%
                 \ne\lim_{m}\lim_{n}f(n,m)$.
                You have to
                be careful when considering limits of limits.
                With this we have shown that
                $z_{k}-x_{n}(k)\in\ell^{p}$ for all
                $n\in\mathbb{N}$. But $x_{n}\in\ell^{p}$,
                and $\ell^{p}$ is closed under addition.
                Therefore $z_{k}\in\ell^{p}$. But also,
                for $n>N$, we have
                $d_{p}(x_{n},z)<\varepsilon$. Thus,
                $x_{n}$ converges.
            \end{proof}
            \begin{theorem}
                If $(X,d)$ is complete and $S$ is a closed
                subset of $X$, then $(S,d_{S})$ is complete,
                where $d_{S}$ is the restriction of
                $d$ onto $S$.
            \end{theorem}
            \begin{proof}
                Let $x_{n}$ be a Cauchy sequence in $S$. Then
                $x_{n}\rightarrow{x}$, $x\in{X}$,
                since $x_{n}$ is Cauchy in $X$
                and $X$ is complete. Since $S$ is closed,
                $x\in{S}$. Therefore, etc.
            \end{proof}
            \begin{theorem}
                If $(X,d)$ is complete and
                $S\subset{X}$ is not closed,
                then $(S,d_{S})$ is not complete.
            \end{theorem}
            \begin{proof}
                If $S$ is not closed then there
                is a convergent sequence $x_{n}\in{S}$
                whose limit it not in $S$. But
                then $x_{n}$ is a Cauchy sequence in
                $X$, and therefore is also a
                Cauchy sequence in $S$, but
                $x_{n}$ does not converge in $S$.
                Therefore $(S,d_{S})$ is not complete.
            \end{proof}
            Recall that $c_{0}$ is the set of sequences which
            tend to zero. That is, it is the set of
            null sequences.
            \begin{theorem}
                $c_{0}$ is a closed subset of
                $(\ell^{\infty},d_{\infty})$
            \end{theorem}
            \begin{proof}[proof 1]
                Let $x_{n}$ be a sequence in $c_{0}$
                that converges to $z\in\ell^{\infty}$
                with respect to $d_{\infty}$.
                Then
                $\sup\{|x_{n}(k)-z_{k}|\}\rightarrow{0}$.
                We need to show that $z\in{c_{0}}$.
                Let $\varepsilon>0$. Let $N_{1}\in\mathbb{N}$
                be such that
                $n>N$ implies
                $\sup\{|x_{n}(k)-z_{k}\}<\frac{\varepsilon}{2}$.
                But $x_{n}\in{c_{0}}$ for all $n$, and thus
                $x_{n}(k)\rightarrow{0}$ as $k\rightarrow\infty$.
                Thus, there is an $N_{2}\in\mathbb{N}$
                such that $n>N_{2}$ implies
                $|x_{n}(k)<\varepsilon$.
                But then for $n>\max\{N_{1},N_{2}\}$,
                $|z_{k}|\leq|z_{k}-x_{n}(k)|+|x_{n}(k)|%
                 <\varepsilon$.
            \end{proof}
            \begin{proof}[Proof 2]
                We can also show that
                $c_{0}^{C}$ is open.
                Let $x\in{c_{0}^{C}}$. Then there is
                an $r>0$ and a subsequence
                $x_{k_{n}}$ of $x$ such that
                $x_{k_{n}}>r$ for all $n$.
                But then $B_{r/2}(x)$ is
                an open ball contained in $c_{0}^{C}$.
                For if $y\in{B_{r/2}(x)}$, then
                $d_{\infty}(x,y)%
                 =\sup\{|x_{n}-y_{n}|\}<r<2$,
                and thus
                $|y_{k_{n}}-x_{k_{n}}|<r/2$,
                and there for $|y_{k_{n}}|>r/2$.
                Thus, $y$ is not a null sequence and
                $c_{0}^{C}$ is open. So
                $c_{0}$ is closed.
            \end{proof}
            Let $X$ be the set of sequences with only
            finitely many nonzero terms.
            Then $(X,d_{\infty}$ is not complete.
            Let $x_{1}=(1,0,0,\hdots)$,
            $x_{2}=(1,1/2,0,0,\hdots)$,
            $x_{n}=(1,1/2,\hdots,1/n,0,0,\hdots)$.
            Then
            $d_{\infty}(x_{n},x_{m})=1/\max\{n,m\}\rightarrow{0}$.
            But clearly
            $x_{n}\rightarrow(1,1/2,\hdots,1/n,\hdots)$, which
            is an element of $c_{0}$, but not an element
            of $X$. Thus $X$ is not closed, and therefore is
            not complete. Returning to $C[0,1]$, when we had
            that sequence of continuous functions that clearly
            converged to a discontinuous functions, we still
            needed to show that there is no continuous function
            that the $x_{n}(t)$ converged to. Here we've embedded
            $X$ into a bigger space, shown that the
            sequence converges to something outside of $X$,
            in our case an element of
            $c_{0}\setminus{X}$, and then used the uniqueness
            of limits to show that the limit does
            not converge in $X$.
        \subsection{Banach's Fixed Point Theorem}
            If $(X,d)$ is a complete metric space,
            and if $T:X\rightarrow{X}$ satisfies
            the property that, for all $x$ and $y$
            in $X$, $d(T(x),T(y))<kd(x,y)$ for
            some $k<1$, then $T$ has a unique
            point $x$, called a fixed point,
            such that $T(x)=x$.
            \begin{definition}
                A contraction of a metric
                space $(X,d)$ is a function
                $T:{X}\rightarrow{X}$ such that there
                exists a $k\in(0,1)$ such that
                for all $x,y\in{X}$,
                $d(T(x),T(y))<kd(x,y)$.
            \end{definition}
            \begin{definition}
                A fixed point of a function
                $f:X\rightarrow{X}$ is a point
                $x\in{X}$ such that
                $f(x)=x$.
            \end{definition}
            \begin{theorem}[%
                Banach's Fixed Point Theorem%
            ]
                If $(X,d)$ is a complete
                metric space and $T:X\rightarrow{X}$
                is a contraction, then there is
                a unique fixed point $x\in{X}$
                with respect to $T$.
            \end{theorem}
            \begin{definition}
                A Lipschitz continuous function is a
                function $f:[a,b]\rightarrow\mathbb{R}$
                such that there is an $L\in\mathbb{R}$
                such that
                $|f(x)-f(y)|<L|x-y|$ for all
                $x,y\in[a,b]$.
            \end{definition}
            This says that the slopes of the
            secant lines of the
            function are bounded. The square root
            function $y=\sqrt{x}$ is an example
            of a function that is not Lipschitz. The
            slopes of secant lines go to infinity
            as the points tend towards the origin.
            \begin{theorem}[Picard's Theorem]
                If $f:[a,b]\times\mathbb{R}%
                    \rightarrow\mathbb{R}$
                is Lipschitz continuous,
                Then there is a unique function
                $x:[a,b]\rightarrow\mathbb{R}$
                such that
                $\frac{dx}{dt}=f(t,x(t))$ and $x(a)=a$.
            \end{theorem}
            \begin{proof}
                We prove Picard by using the
                Banach Fixed Point Theorem. First
                we write the problem as an integral
                equation.
                If $\dot{x}=f(t,x(t))$, then:
                \begin{equation*}
                    x(t)
                    =\int_{a}^{t}\frac{dx}{dt}dt
                    =x_{0}+\int_{a}^{t}f(t,x(t))dt
                \end{equation*}
                Let $(X,d)$ be $C[a,b]$ with the
                supremum norm $d_{\infty}$. Then
                $(x,d)$ is a complete metric space.
                Let $T:{X}\rightarrow{X}$ be defined
                by:
                \begin{equation*}
                    Tx=x_{0}+\int_{a}^{t}f(t,x(t))dt
                \end{equation*}
                All we need to do is show that $T$ is
                a contraction. Applying the
                Banach Fixed Point theorem then
                shows that there is a unique
                fixed point of $T$, thus showing
                that there is a unique solution
                to our original initial value problem.
                If $x,y\in{X}$, then:
                \begin{align*}
                    d(Tx,Ty)
                    &=\sup\{|Tx(t)-Ty(t)|\}\\
                    &=\sup\{
                        (x_{0}+
                         \int_{a}^{t}f(t,x(t))dt)
                       -(x_{0}+
                         \int_{a}^{t}f(t,y(t))dt)
                    \}\\
                    &=\sup\{
                        \int_{a}^{t}f(t,x(t))dt)-
                        \int_{a}^{t}f(t,y(t))dt)
                    \}\\
                    &\leq\int_{a}^{t}|
                        f(t,x(t))-f(t,y(t))|dt
                \end{align*}
                But from the Lipschitz continuity
                of $f$, we have:
                \begin{align*}
                    d(Tx,Ty)&\leq
                    L\int_{a}^{t}|x(t)-y(t)|dt\\
                    &\leq{L}(t-a)d(x,y)\\
                    &\leq{L}(b-a)d(x,y)
                \end{align*}
                So $T$ is a contraction for
                $L(b-a)<1$. Usually we can
                extend this solution by taking
                $b$ as the initial condition and
                stepping forward one interval
                at a time. We'll take a different
                approach. We have that
                $d(Tx,Ty)\leq{L}(b-a)d(x,y)$. From
                this, we obtain:
                \begin{align*}
                    d(T^{2}x,T^{2}y)
                    &\leq{L}\int_{a}^{b}d(Tx,Ty)dt\\
                    &\leq{L}\int_{a}^{t}
                        L(t-a)d(x,t)dt\\
                    &=\frac{L^{2}}{2}(t-a)^{2}d(x,y)\\
                    &\leq
                    \frac{L^{2}}{2}(b-a)^{2}d(x,y)
                \end{align*}
                Applying induction, we have:
                \begin{equation*}
                    d(T^{n}x,T^{n}y)
                    \leq\frac{L^{n}}{n!}(b-a)^{n}
                \end{equation*}
                But this tends to zero, and thus
                there is an $N$ such that,
                for all $n>N$, $T^{n}$ is a
                contraction. But then, by the
                Banach Fixed Point Theorem, there
                is a unique point $x$ such that
                $T^{n}x=x$. But then
                $Tx=T^{n}(Tx)$, and thus
                $Tx$ is a fixed point of
                $T^{n}$. But the fixed point of
                $T^{n}$ is unique, and $x$ is a
                fixed point. Therefore
                $Tx=x$. Therefore, etc.
            \end{proof}
            Without Lipschitz continuous you may
            lose uniqueness, but you still have
            existence. This is Peano's theorem.
            An example is $\dot{x}=\sqrt{x}$
            with $x(0)=0$.
            This has solutions $x(t)=0$ and
            $(t)=t^{2}/4$. Now back to compactness.
            \subsubsection{Compactness}
                \begin{definition}
                    A metric space $(X,d)$ is
                    sequentially compact if every
                    sequence in $X$ has a convergent
                    subsequence.
                \end{definition}
                In topology there is a difference
                between sequential compactness
                and regular compactness, but in
                metric spaces they turn out
                to be the same.
                A subset of $S$ of $X$ is
                compact if every sequence in
                $S$ has a subsequence which converges.
                That is, $(S,d)$ is compact.
                \begin{theorem}
                    A subset $S$ of a compact
                    metric space $(X,d)$ is compact
                    if and only if $S$ is closed.
                \end{theorem}
                \begin{proof}
                    For let $x_{n}$ be a sequence
                    in $S$. Then $x_{n}$ is a
                    sequence in $X$ and thus there
                    is a convergent subsequence
                    $x_{k_{n}}$ with a limit $x$.
                    But $x_{k_{n}}$ is in $S$ and
                    $S$ is closed, and therefore
                    $x$ is in $S$. Thus, $S$
                    is compact. Conversely, if
                    $S$ is compact, suppose it is
                    not closed. Then there is a point
                    $y\in{X}$ such that $y$ is a
                    limit point of $S$ but not
                    contained in $S$. Let
                    $x_{n}$ be a sequence that
                    converges to $y$. Then, as
                    $S$ is compact, there is
                    a convergent subsequence. But
                    the limit of this subsequence
                    is $y$, a contradiction as
                    $y\notin{S}$. Therefore $S$
                    is closed.
                \end{proof}
                \begin{theorem}
                    If $(X,d)$ is a compact metric
                    space, then
                    $(X,d)$ is complete.
                \end{theorem}
                \begin{proof}
                    If $x_{n}$ is Cauchy in $X$,
                    then there is a convergent
                    subsequence $x_{k_{n}}$
                    in $X$. But if $x_{k_{n}}$
                    converges to $x$, then
                    $x_{n}$ converges to $x$ as
                    well, as $x_{n}$ is Cauchy.
                    Therefore, $(X,d)$ is complete.
                \end{proof}
                \begin{theorem}[Heine-Borel Theorem]
                    A subset of
                    $\mathbb{R}^{n}$ is
                    compact if and only if
                    it is closed and bounded.
                \end{theorem}
                \begin{example}
                    The closed unit ball
                    of $\ell^{p}$ is not compact,
                    if $1\leq{p}\leq{\infty}$.
                    Let $x_{n}(m)$ be the sequence
                    (of sequences) such that
                    $x_{n}(m)=1$ if $n=m$, and
                    zero otherwise. Then
                    $d_{p}(x_{n},x_{m})=2^{1/p}$,
                    so $x_{n}$ has no subsequence
                    which is Cauchy. But then there
                    is no convergent subsequence
                    either, and therefore
                    $\ell^{p}$ is not compact.
                \end{example}
                \begin{example}
                    The closed unit ball in
                    $(C[0,1],d_{\infty})$ is
                    not compact. For let
                    $x_{n}(t)=t^{2^{n}}$. Then
                    (Do some calculus) the maximum of
                    $d(x_{n},x_{n+1})$ is always
                    $1/4$. So this has no subsequence
                    which is Cauchy, and thus no
                    convergent subsequence exists.
                \end{example}
                \begin{definition}
                    A metric space $X$ is totally
                    bounded if for all
                    $\varepsilon>0$ there is a finite
                    number of points $x_{n}$ such
                    that $B_{\varepsilon}(x_{n})$
                    covers the entirety of $X$.
                \end{definition}
                \begin{theorem}
                    A compact metric space is
                    totally bounded.
                \end{theorem}
                \begin{proof}
                    Suppose not. Then there is an
                    $\varepsilon>0$ such that
                    no finite collection
                    $B_{\varepsilon}(x_{n})$
                    is a covering of $X$. Let
                    $x_{1}\in{X}$. Then
                    $B_{\varepsilon}(x_{1})$ is not
                    $X$. Thus there is an $x_{2}$
                    such that
                    $x_{2}\notin%
                     B_{\varepsilon}(x_{1})$.
                    But also
                    $B_{\varepsilon}(x_{1})\cup%
                     B_{\varepsilon}(x_{2})$ is
                    not the entirety of $X$.
                    Continuing we have that there
                    is a sequence $x_{n}$ such that,
                    for all $n\ne{m}$,
                    $d(x_{n},x_{m})\geq{\varepsilon}$.
                    So there is no convergent
                    subsequence. But $X$ is
                    compact, a contradiction.
                    Therefore, etc.
                \end{proof}
                There are metric spaces that are
                bounded but not totally bounded.
                For let
                $X=\mathbb{R}$ and $d$ be the
                discrete metric. Then, for
                $\varepsilon=1/2$, the is no
                finite covering. Every point needs
                it's own ball, so the covering is
                uncountable.
                \begin{theorem}
                    If $(X,d)$ is complete and
                    totally bounded, then it
                    is compact.
                \end{theorem}
                \begin{proof}
                    Let $x_{n}$ be a sequence
                    in $X$. Let $\varepsilon=1$. Then
                    there are finitely many points
                    $y_{k}$ such that
                    $B_{\varepsilon}(y_{k})$ covers
                    $X$. Then one of these
                    balls has infinitely many of
                    the $x_{n}$. Similarly, for
                    $\varepsilon=\frac{1}{n}$, there
                    is a finite number of points
                    $y_{k}$ such that
                    $B_{\frac{1}{n}}(y_{k})$ covers
                    $X$. Thus there is a point with
                    infinitely many of the $x_{n}$
                    in it. So, we can find a
                    subsequence such that, for
                    $n,m>N$,
                    $d(x_{k_{n}},x_{k_{m}})<%
                     \frac{1}{N}$. But $(X,d)$ is
                    complete, and therefore
                    $x_{k_{n}}$ converges. Therefore
                    $x_{n}$ has a convergent
                    subsequence. Thus, $(X,d)$ is
                    compact.
                \end{proof}
                \begin{theorem}
                    Compact spaces are separable.
                \end{theorem}
                \begin{proof}
                    If $X$ is compact, then
                    it is totally bounded. But
                    then, for $\varepsilon=1/n$
                    there is a finite covering of
                    $X$ with balls of radius
                    $\varepsilon$. Then,
                    taking all of the
                    centers of all of the points
                    for all $n$ (Countable union
                    of finite points is countable),
                    we obtain a countable dense
                    subset.
                \end{proof}
                \begin{example}
                    There are ``infinite dimension''
                    sets that are also compact. Two
                    in particular worth mentioning.
                    The first is the hilbert Cube.
                    It's a subset of $\ell^{2}$
                    whose elements are such that
                    $|x_{n}|<1/n$. That is, elements
                    are sequences whose $n^{th}$
                    elements are less than
                    $1/n$. This is compact.
                    Arzela-Ascoli. Peano.
                \end{example}
    \section{Normed and Inner Product Spaces}
        \subsection{Basic Definitions}
            We're finally going to put some structure on these
            sets, and talk about vector spaces. In a metric
            space, the only thing you can really talk about
            is the distance between points. In a vector space
            we have a lot more structure. We will start off
            with vector spaces over the reals $\mathbb{R}$.
            The main properties are that there is a
            $\mathbf{0}$ element, addition is well defined
            and is both associative and commutative,
            there is a notion of scalar multiplication that
            is associative, and the distributive law holds.
            \begin{example}
                $\mathbb{R}^{n}$, with it's usual notion
                of addition, and with scalar multiplication
                defined over $\mathbb{R}$, is a vector space.
            \end{example}
            \begin{definition}
                A norm on a vector space $X$ over $\mathbb{R}$
                is a function $\norm{}:X\rightarrow\mathbb{R}$
                such that:
                \begin{enumerate}
                    \item For all $\mathbf{x}\in{X}$,
                          $\norm{\mathbf{x}}\geq{0}$ and
                          $\norm{\mathbf{x}}=0$ if and only
                          if $\mathbf{x}=\mathbf{0}$.
                          \hfill[Positive Definiteness]
                    \item For all $\mathbf{x}\in{X}$ and
                          $c\in\mathbb{R}$,
                          $\norm{c\mathbf{x}}%
                           =|c|\norm{\mathbf{x}}$
                          \hfill[Homogeneity]
                    \item For all $\mathbf{x},\mathbf{y}\in{X}$,
                          $\norm{\mathbf{x}+\mathbf{y}}%
                           \leq\norm{\mathbf{x}}%
                           +\norm{\mathbf{y}}$
                          \hfill[Triangle Inequality]
                \end{enumerate}
            \end{definition}
            We have seen before that
            $d(\mathbf{x},\mathbf{y})%
             =\norm{\mathbf{x}-\mathbf{y}}$
            defines a metric, and thus $(X,d)$ is a metric space.
            Thus, for every vector space there is an associated
            metric space, the metric $d$ called the
            \textit{induced} metric.
            \begin{definition}
                A normed vector space is a vector space
                $X$ over $\mathbb{R}$ with a norm
                $\norm{}$ on $X$.
            \end{definition}
            \begin{example}
                $\mathbb{R}^{n}$ with
                $\norm{\mathbf{x}}_{p}$, for $p\geq{1}$,
                is a normed vector space.
            \end{example}
            \begin{example}
                $\ell^{p}$ with $\norm{x}_{p}$ is
                also a normed vector space.
            \end{example}
            \begin{example}
                $C[a,b]$ equipped with the supremum norm,
                $\norm{x(t)}_{\infty}$,
                is a normed vector space.
            \end{example}
        \subsubsection{Inner Product Spaces}
            \begin{definition}
                An inner product on a vector space
                $X$ over $\mathbb{R}$ is a function
                $\langle\rangle:X\rightarrow\mathbb{R}$
                such that:
                \begin{enumerate}
                    \item For all $x\in{X}$,
                          $\langle{\mathbf{x},\mathbf{x}}%
                           \rangle\geq{0}$
                          and
                          $\langle\mathbf{x},\mathbf{x}\rangle=0$
                          if and only
                          if $\mathbf{x}=\mathbf{0}$.
                          \hfill[Positive Definiteness]
                    \item For all $\mathbf{x},\mathbf{y}\in{X}$,
                          $\langle\mathbf{x},\mathbf{y}\rangle%
                           =\langle\mathbf{y},\mathbf{x}\rangle$
                          \hfill[Symmetry]
                    \item For all
                          $\mathbf{x},\mathbf{y},\mathbf{z}%
                           \in{X}$
                          and all $\alpha,\beta\in\mathbb{R}$,
                          $\langle\alpha\mathbf{x}%
                           +\beta\mathbf{y},\mathbf{z}\rangle%
                           =\alpha\langle\mathbf{x},\mathbf{z}%
                           \rangle+\beta\langle\mathbf{y},%
                           \mathbf{z}\rangle$
                          \hfill[Linearity]
                \end{enumerate}
            \end{definition}
            \begin{example}
                $\mathbb{R}^{2}$ with
                $\langle(x_{1},x_{2}),(y_{1},y_{2})\rangle%
                 =x_{1}y_{1}+x_{2}y_{2}$ is an inner product.
                Replacing this with $\mathbb{R}^{n}$ and doing
                $\sum_{k=1}^{n}x_{k}y_{k}$ is also an inner
                product. This is the usual dot product that one
                sees in a vector calculus course. In $\ell^{2}$,
                $\sum_{k=1}^{\infty}x_{k}y_{k}$ is an inner
                product as well. Note also that
                $\sum|x_{i}y_{i}|$ converges since
                $|x_{i}y_{i}|\leq\frac{1}{2}|x_{i}^{2}|%
                 +\frac{1}{2}|y_{i}|^{2}$.
            \end{example}
            \begin{example}
                In $C[a,b]$, let
                $\langle{x(t),y(t)}\rangle%
                 =\int_{a}^{b}x(t)y(t)dt$. This defines an
                inner product.
            \end{example}
            \begin{definition}
                An inner product space is a vector space
                $X$ over $\mathbb{R}$ with an inner product
                $\langle\rangle$.
            \end{definition}
            \begin{theorem}[Cauchy-Schwarz Inequality]
                If $X$ is an inner product space
                and $x,y\in{X}$, then
                $|\langle{x,y}\rangle<\norm{x}\norm{y}$
            \end{theorem}
            \begin{proof}
                For all $y\in\mathbb{R}$,
                $\langle{x+ty,x+ty}\rangle%
                 =\langle{x,x}\rangle%
                 +2t\langle{x,y}\rangle%
                 +t^{2}\langle{y,y}\rangle%
                 =\norm{x}^{2}+2t\langle{x,y}\rangle%
                 +t^{2}\norm{y}^{2}$. Thus we have a
                quadratic in $t$. But this is always positive,
                and thus the discriminant must be non-positive. Therefore
                $(2\langle{x,y})^{2}-4\norm{x}^{2}\norm{y}^{2}%
                 \leq{0}$
                and thus
                $|\langle{x,y})|\leq\norm{x}\norm{y}$.
            \end{proof}
            \begin{theorem}
                If $X$ is a vector space over $\mathbb{R}$
                and $\langle\rangle$ is an inner product,
                then
                $\norm{\mathbf{x}}%
                 =\sqrt{\langle\mathbf{x},\mathbf{y}\rangle}$
                is a norm on $X$.
            \end{theorem}
            \begin{proof}
                Positivity, homogeneity, and definiteness are
                pretty easy. The only tricky thing to check is
                the triangle inequality. We have that
                $\norm{x+y}=\langle{x+y,x+y}\rangle$,
                and this simplify to
                $\norm{x}^{2}+2\langle{x,y}\rangle+\norm{y}^{2}$.
                But from the Cauchy-Schwartz inequality, we
                have $\langle{x,y}\rangle\leq\norm{x}\norm{y}$.
                Thus
                $\norm{x+y}^{2}\leq\norm{x}^{2}%
                 +2\norm{x}\norm{y}+\norm{y}^{2}%
                 =(\norm{x}+\norm{y})^{2}$. Taking square roots
                 completes the theorem.
            \end{proof}
            In $\mathbb{R}^{n}$, the Cauchy-Schwartz inequality
            says that the dot product of two vectors is less
            than or equal to the product of the magnitude
            of the two vectors.
            This is obvious from the fact that the dot product
            of two vector is the product of the magnitudes and
            the \textit{cosine} of the angle between them.
            Since the cosine of a number is less than or equal
            to one, this would complete the theorem.
            In $\ell^{p}$ and $L^{p}$ spaces, this is the
            special case of the H\"{o}lder inequality for
            when $p=q=2$.
        \subsubsection{Convergence in Normed Spaces}
            In a metric space, convergence meant that
            $d(x_{n},x)\rightarrow{0}$. In a normed space
            we have the induced metric, and thus we may define
            convergence as $\norm{x_{n}-x}\rightarrow{0}$.
            \begin{definition}
                A convergent sequence in a normed space $X$
                is a sequence $x_{n}$ such that there is an
                $x\in{X}$ such that
                $\norm{x_{n}-x}\rightarrow{0}$.
            \end{definition}
            Since
            $\norm{y}=\norm{(y-x)+x}\leq\norm{y-x}+\norm{x}$,
            it follows that
            $|\norm{x}-\norm{y}|\leq\norm{x-y}$.
            But then if $x_{n}\rightarrow{x}$, then
            $|\norm{x_{n}}-\norm{x}|\leq\norm{x_{n}-x}$,
            and $\norm{x_{n}-x}\rightarrow{0}$. Therefore
            $\norm{x_{n}}\rightarrow\norm{x}$. That is,
            the norm function is a continuous function.
            Similarly, if $x_{n}\rightarrow{x}$, then
            $\langle{x_{n},y}\rangle\rightarrow%
             \langle{x,y}\rangle$.
            In fact, if $x_{n}\rightarrow{x}$ and
            $y_{n}\rightarrow{y}$, then
            $\langle{x_{n},y_{n}}\rangle%
             \rightarrow\langle{x,y}\rangle$. To see this, we
            have
            $\langle{x_{n},y_{n}}\rangle-\langle{x,y}\rangle%
             =\langle{x_{n}-x,y}\rangle+\langle{x,y-y_{n}}\rangle$
            and therefore
            $|\langle{x_{n},y_{n}}\rangle-\langle{x,y}\rangle%
             \leq\norm{x_{n}-x}\norm{y_{n}}%
             +\norm{x}\norm{y-y_{n}}$. But $\norm{x-x_{n}}\rightarrow{0}$
            and $\norm{y-y_{n}}\rightarrow{0}$. But also
            $\norm{y_{n}}=\norm{(y_{n}-y)+y}\leq\norm{y_{n}-y}+\norm{y}$,
            which is bounded. Therefore
            $\langle{x_{n},y_{n}}-\langle{x,y}\rangle\rightarrow{0}$.
            So inner product spaces and normed spaces are metric spaces
            and we can define everything we did for metric spaces and all
            of the previous results remain true. That is, the notions and
            theorems pertaining to convergence, completeness, compactness,
            the notion of open and closed. All of these still make sense in
            these new spaces.
        \subsubsection{Banach Spaces and Hilbert Spaces}
            \begin{definition}
                A Banach Space is a normed vector space $X$ that is
                complete with respect to the induced metric.
            \end{definition}
            \begin{definition}
                A Hilbert Space is an inner product space $X$ that is
                complete with respect to the induced metric.
            \end{definition}
        \subsubsection{Linear Operators}
            Let $X$ and $Y$ be normed spaces. A mapping
            $T:X\rightarrow{Y}$ is called a linear operator if, for
            all $x,y\in{X}$, and for all $\alpha,\beta\in\mathbb{R}$,
            $T(\alpha{x}+\beta{y})=\alpha{T(x)}+\beta{T(y)}$. Usually, with
            operators, we simply write $Tx$ and $Ty$. Similar to how
            we write matric multiplication over vectors. In $\mathbb{R}^{n}$,
            every $n\times{n}$ matrix defines a linear operator.
            \begin{definition}
                A linear operator from a normed vector space $X$ to
                a normed vector space $Y$ is a function
                $T:X\rightarrow{Y}$ such that, for all $x,y\in{X}$
                and for all $\alpha,\beta\in\mathbb{R}$,
                $T(\alpha{x}+\beta{y})=\alpha{Tx}+\beta{Ty}$.
            \end{definition}
            \begin{definition}
                A bounded linear operator from a normed vector space
                $X$ to a normed vector space $Y$ is a linear operator
                $T:X\rightarrow{Y}$ such that there is a $K\in\mathbb{R}$
                such that for all $x\in{X}$, $\norm{Tx}\leq{K}\norm{x}$
            \end{definition}
            In a just world, ``bounded'' would mean
            $\norm{Tx}\leq{K}$. However, the only linear mapping that does
            this is the zero mapping. For if $\norm{Tx}=1$,
            then $\norm{T(2x)}=2$, and so on, and thus no linear mapping
            is bounded (With the exception of the zero mapping).
            Boundedness of a norm $T:X\rightarrow{Y}$ depends on
            the norms of the space.
            \begin{theorem}
                Bounded linear operators are continuous.
            \end{theorem}
            \begin{proof}
                If $x_{n}\rightarrow{x}$, then
                $\norm{Tx_{n}-Tx}=\norm{T(x_{n}-x)}$. But
                $T$ is bounded, and thus there is a $K$ such that
                $\norm{T(x_{n}-x)}\leq{K}\norm{x_{n}-x}$. But
                $\norm{x_{n}-x}\rightarrow{0}$. Therefore, etc.
            \end{proof}
            The converse is also true.
            \begin{theorem}
                If $T$ is a continuous linear operator,
                than there exists a $\delta>0$ such that for
                all $x\in{B}_{\delta}(0)$,
                $\norm{Tx-T0}<1$. But from linearity,
                $T0=0$, and thus $\norm{Tx}<1$. Then for any
                $z\in{Z}$, we have
                $\norm{\frac{\delta}{2}\frac{z}{\norm{z}}}=\frac{\delta}{2}$,
                and thus $\norm{T(\frac{\delta}{2}\frac{z}{\norm{z}})}<1$.
                Letting $K=\delta$, we have
                $\norm{Tx}<K\norm{x}$. Thus, $T$ is bounded.
            \end{theorem}
            Continuity at 0 implies uniform continuity since
            if $x_{n}-y_{n}\rightarrow{0}$, then
            $\norm{Tx_{n}-Ty_{n}}=\norm{T(x_{n}-y_{n})}%
             \leq{K}\norm{x_{n}-y_{n}}\rightarrow{0}$.
            The set of bounded linear operators form a vector space,
            where addition is $(S+t)(x)=(Sx)+(Tx)$, and scalar multiplication
            is defined by $(\alpha{T})(x)=\alpha(Tx)$. We must show that
            when you add two bounded linear operators, the result is a
            bounded linear operator.
            \begin{theorem}
                If $T_{1}:X\rightarrow{Y}$ and $T_{2}:X\rightarrow{Y}$
                are bounded linear operators, then $T_{1}+T_{2}$ is a
                bounded linear operator.
            \end{theorem}
            \begin{proof}
                For let $T_{1}$ and $T_{2}$ be bounded. Then there are
                $K_{1},K_{2}$ such that, for all $x\in{X}$,
                $\norm{T_{1}x}\leq{K_{1}}\norm{x}$ and
                $\norm{T_{2}x}\leq{K_{2}}\norm{x}$. But then
                $\norm{(T_{1}+T_{2})x}=\norm{T_{1}x+T_{2}x}%
                 \leq\norm{T_{1}x}+\norm{T_{2}x}%
                 \leq{K_{1}}\norm{x}+K_{2}\norm{x}$. Let $K=K_{1}+K_{2}$.
            \end{proof}
            \begin{theorem}
                If $T:X\rightarrow{Y}$ is a bounded linear operator, and
                $\alpha\in\mathbb{R}$, then $\alpha{T}$ is a bounded
                linear operator.
            \end{theorem}
            \begin{proof}
                For
                $\norm{\alpha{Tx}}=|\alpha|\norm{Tx}%
                 \leq|\alpha|K\norm{x}=K\norm{\alpha{x}}$.
            \end{proof}
            We write $B(X,Y)$ to denote the set of bounded linear
            operators from $X$ to $Y$. That is, linear operators
            $T:X\rightarrow{Y}$.
            We can define a norm on $B(X,Y)$ as follows:
            $\norm{T}_{B}%
             =\sup_{x\in{X},x\ne{0}}\{\frac{\norm{Tx}}{\norm{x}}\}$.
            This is the ``Smallest $K$,'' used as a bounded for the linear
            operator $T$. This shows that
            $\norm{Tx}_{Y}\leq\norm{T}_{B}\norm{x}_{X}$.
        \subsection{Lecture 7: October 22, 2018}
            \subsubsection{Bounded Linear Operators}
                A bounded linear operator is a function
                $T:X\rightarrow{Y}$ between normed spaces
                $X$ and $Y$ such that $T$ is linear, and
                there exists a $K\in\mathbb{R}$ such that,
                for all $x\in{X}$,
                $\norm{Tx}_{Y}\leq{K}\norm{x}_{X}$. The
                norm of $T$, $\norm{T}$, is then defined
                as the smallest such $K$. Equivalently:
                \begin{equation*}
                    \norm{T}=
                    \sup\Big\{\frac{\norm{Tx}_{Y}}{\norm{x}_{X}}:
                              x\in{X},x\ne{0}\Big\}
                    =\sup\{\norm{Tx}_{Y}:\norm{x}_{X}=1\}
                \end{equation*}
                The set of all bounded linear operators
                from a normed space $X$ to a normed space
                $Y$ is denoted $B(X,Y)$. This is a vector
                space with addition defined as
                $(T+S)x=(Tx)+(Sx)$ and $(aT)x=a(Tx)$.
                \begin{theorem}
                    $\norm{T}$ defines a norm on
                    $B(X,Y)$.
                \end{theorem}
                \begin{proof}
                    For $\norm{T}\geq{0}$ and
                    $\norm{Tx}=0$ if and only if
                    $Tx=0$ for all $x\in{X}$, and thus
                    $T$ is the zero operator. If
                    $\alpha\in\mathbb{R}$, then:
                    \begin{align*}
                        \norm{\alpha{T}}
                        &=\sup\Big\{
                            \frac{\norm{\alpha{T}x}_{Y}}{\norm{x}_{X}}:
                            x\in{X},x\ne{0}\Big\}\\
                        &=|\alpha|\sup\Big\{
                            \frac{\norm{Tx}_{Y}}{\norm{x}_{X}}:
                            x\in{X},x\ne{0}\Big\}\\
                        &=|\alpha|\norm{T}
                    \end{align*}
                    Finally, if $S,T\in{B}(X,Y)$, then:
                    \begin{align*}
                        \norm{S+T}&=\sup\Big\{
                            \frac{\norm{(S+t)x}_{Y}}{\norm{x}_{X}}:
                            x\in{X},x\ne{0}\}\\
                        &=\sup\Big\{
                            \frac{\norm{Sx+Tx}_{Y}}{\norm{x}_{X}}:
                            x\in{X},x\ne{0}\Big\}\\
                        &\leq\sup\Big\{
                            \frac{\norm{Sx}_{y}+\norm{Tx}_{Y}}
                                 {\norm{x}_{X}}:
                            x\in{X},x\ne{0}\Big\}\\
                        &\leq\norm{T}+\norm{S}
                    \end{align*}
                \end{proof}
                \begin{theorem}
                    If $Y$ is a Banach space, and 
                    if $X$ is a normed space, then
                    $B(X,Y)$ is a Banach space.
                \end{theorem}
                \begin{proof}
                    For let $T_{n}$ be a Cauchy sequence
                    in $B(X,Y)$ and let $\varepsilon>0$.
                    Then there exists $N_{0}\in\mathbb{N}$
                    such that for all $n,m>N_{0}$,
                    $\norm{T_{n}-T_{m}}<\varepsilon$. That is,
                    for all $n,m>N_{0}$:
                    \begin{align*}
                        \sup\Big\{
                            \frac{\norm{T_{n}x-T_{m}x}_{Y}}
                                 {\norm{x}_{X}}:
                            x\in{X},x\ne{0}\Big\}
                        &\leq\varepsilon\\
                        \Rightarrow
                        \frac{\norm{T_{n}x-T_{m}y}_{Y}}
                             {\norm{x}_{X}}
                        &\leq\varepsilon
                    \end{align*}
                    That is, $T_{n}x$ is a Cauchy sequence
                    in $Y$ for any fixed value $x\in{X}$.
                    But $Y$ is a Banach space, and is therefore
                    complete. But then if $T_{n}x$ is a Cauchy
                    sequence in $Y$ it has a limit $y\in{Y}$.
                    Let $Tx=\lim_{n\rightarrow\infty}T_{n}x$
                    for all $x\in{X}$.
                    Then $T\in{B(X,Y)}$. For:
                    \begin{equation*}
                        T(x+y)
                        =\lim_{n\rightarrow\infty}T_{n}(x+y)
                        =\lim_{n\rightarrow\infty}(T_{n}x+T_{n}y)
                        =Tx+Ty
                    \end{equation*}
                    And similarly $(\alpha{T})x=\alpha{T}x$.
                    Lastly, $T$ is bounded. For all $n,m>N$ we have
                    $\norm{T_{n}x-T_{m}x}_{Y}/\norm{x}_{X}<\varepsilon$.
                    Taking the limit on $m$, we have
                    $\norm{Tx-T_{n}x}_{Y}/\norm{x}_{X}\leq\varepsilon$
                    for all $n>N_{0}$. Thus,
                    $\norm{T_{n}x-Tx}_{X}\leq\varepsilon\norm{x}_{X}$.
                    But
                    $\norm{Tx-T_{n}x}_{Y}=\norm{T_{n}x-(T_{n}-Tx)}_{Y}$,
                    and therefore
                    $\norm{Tx}\leq\varepsilon\norm{x}_{X}+\norm{T_{n}x}$,
                    and $\norm{T_{n}x}\leq\norm{T_{n}}$, and therefore
                    $\norm{Tx}\leq\varepsilon\norm{X}_{X}+%
                     \norm{T}\norm{x}_{X}$. But then
                    $\norm{Tx}_{Y}\leq%
                     (\varepsilon+\norm{T_{n}})\norm{x}_{X}$.
                    But $T_{n}$ is bounded, and therefore
                    $T$ is bounded. Finally, we must show that
                    $T_{n}\rightarrow{T}$ in $B(X,Y)$ with respect
                    to the norm $\norm{T_{n}-T}$. That is, we must
                    show that $\norm{T-T_{n}}\rightarrow{0}$. This
                    follows since
                    $\norm{Tx-T_{n}x}_{Y}/\norm{x}_{X}<\varepsilon$
                    for $n>N_{0}$, and therefore
                    $\norm{T-T_{n}}<\varepsilon$. Therefore, etc.
                \end{proof}
            \subsubsection{Dual Spaces}
                So if $Y$ is a Banach space, and $X$ is any normed
                space, then $B(X,Y)$ is a Banach space. One of the
                most important cases is $Y=(\mathbb{R},||)$, where
                $||$ is the normal absolute value ``norm.''
                $B(X,\mathbb{R})$ is a Banach space, and it is
                called the continuous dual space of $X$, written
                $X'$. Elements of $X'$ are called bounded linear
                functionals. These are bounded linear operators
                whose range of the operator is the real numbers.
                The characterization, or the representation, or
                realization, of these dual spaces is a major
                topic in functional analysis. A lot of these
                theorems are do to a mathematician by the name
                of Riesz.
                \begin{example}
                    A functional takes an element of a normed
                    space $X$ and spits out a real number. For
                    example, if $X$ is the space of continuous
                    functions, then the following are
                    functionals:
                    \begin{align*}
                        f_{1}(x)&=\int_{0}^{1}x(t)t^{2}\diff{t}
                        &
                        f_{2}(x)&=x(0.5)
                        &
                        f_{3}(x)&=0
                    \end{align*}
                \end{example}
                Let $X=(\mathbb{R}^{2},\ell^{1}$. What does
                $X'$ look like? That is, what is the dual
                space of $X$? let $f:X\rightarrow\mathbb{R}$
                be defined by
                $f(x_{1},x_{2})=2x_{1}-5x_{2}$. Then
                $f\in{X'}$ and $\norm{f}=5$. More generally,
                every element of $\mathbb{R}^{2}$ defines
                and element of $X'$. Given
                $(a,b)\in\mathbb{R}^{2}$, we define
                $f(x_{1},x_{2})=ax_{1}+bx_{2}$. $f$ is then linear,
                and:
                \begin{align*}
                    |f(x_{1},x_{2})|
                    &=|ax_{1}+bx_{2}|\\
                    &\leq|a||x_{1}|+|b||x_{2}|\\
                    &\leq\max\{|a|,|b|\}(|x_{1}|+|x_{2}|)\\
                    &=\norm{(a,b)}_{\infty}
                    \norm{(x_{1},x_{2})}_{\ell^{1}}
                \end{align*}
                And therefore $f$ is bounded, as $\norm{(a,b)}_{\infty}$
                is a bound. That is, $\norm{f}\leq\norm{(a,b)}_{\infty}$.
                By choosing $x=(x_{1},x_{2})$, where $x_{1}=1$ and
                $x_{2}=0$ if $|b|\leq|a|$, and $x_{1}=0$ and
                $x_{2}=1$ otherwise, we ge
                $|f|=\max\{(a,b)\}=\norm{(a,b)}_{\infty}$.
                Therefore $\norm{f}=\norm{(a,b)}_{\infty}$. On
                the other hand, if $f\in{X'}$, let
                $a=f(1,0)$ and $b=f(0,1)$. Then, for all
                $(x_{1},x_{2})\in\mathbb{R}^{2}$:
                \begin{equation*}
                    f(x_{1},x_{2})
                    =f(x_{1}(1,0)+x_{2}(0,1))
                    =x_{1}f(1,0)+x_{2}f(0,1)
                    =ax_{1}+bx_{2}
                \end{equation*}
                So the dual of $(\mathbb{R}^{2},\ell^{1})$
                looks very much like $(\mathbb{R}^{2},\ell^{\infty})$.
                In fact, $(\mathbb{R}^{2},\ell^{1})'$ and
                $(\mathbb{R}^{2},\ell^{\infty})$ are isometric
                and isomorphic. That is, we really can't tell them
                apart and we can consider them as the same thing.
                More generally,
                $(\mathbb{R}^{n},\ell^{n})'=(\mathbb{R}^{n},\ell^{\infty})$.
                Even more general, if $p$ and $q$ are exponential
                conjugates of each other (That is,
                $\frac{1}{q}+\frac{1}{p}=1$), then
                $(\mathbb{R}^{n},\ell^{p})'=(\mathbb{R}^{n},\ell^{p})$
                for all $1\leq{p}\leq\infty$. Saying $p=\infty$ is
                equivalent to saying $q=1$. Setting $p=q=2$, we have
                $(\mathbb{R}^{n},\ell^{2})'=(\mathbb{R}^{n},\ell^{2})$.
                This is true of any Hilbert space: The dual of any
                Hilbert Space $\mathcal{H}$ is itself. That is,
                $\mathcal{H}'=\mathcal{H}$. This is one of the
                Riesz Representation Theorems. In infinite dimensions,
                $(\ell^{p})'=\ell^{q}$, where $p$ and $q$ are such that
                $\frac{1}{p}+\frac{1}{q}=1$, and $1\leq{p}<\infty$.
                Now, we cannot allow $p=\infty$. For
                $(\ell^{\infty})'$ is not equal to $\ell^{1}$.
                \begin{theorem}
                    If $1\leq{p}<\infty$ and
                    $\frac{1}{p}+\frac{1}{q}=1$, then
                    $(\ell^{p})'=\ell^{q}$.
                \end{theorem}
                \begin{proof}
                    If $(f_{1},f_{2},\hdots)\in\ell^{q}$, then let
                    $f:\ell^{p}\rightarrow\mathbb{R}$ be defined by
                    $f(x_{1},x_{2},\hdots)=\sum_{k=1}^{\infty}x_{k}f_{k}$.
                    This converges from H\"{o}lder's inequality:
                    \begin{equation*}
                        \sum_{k=1}^{\infty}|x_{k}f_{k}|
                        \leq
                        \Big(\sum_{k=1}^{\infty}f_{k}^{q}\Big)^{1/q}
                        \Big(\sum_{k=1}^{\infty}x_{i}^{p}\Big)^{1/p}
                    \end{equation*}
                    And therefore
                    $|fx|=\norm{(f_{1},f_{2},\hdots)}_{q}%
                     \norm{(x_{1},x_{2},\hdots)}_{p}$. That is,
                    Moreover $f$ is linear. Therefore
                    $f\in(\ell^{p})'$ and
                    $\norm{f}\leq\norm{(f_{1},f_{2},\hdots)}_{q}$.
                    On the other hand, let
                    $x_{i}=|f_{i}|^{q/p}\sgn(f_{i}$. Then
                    \begin{equation*}
                        fx=\sum_{k=1}^{\infty}f_{k}x_{k}
                        =\sum_{k=1}^{\infty}|f_{k}|^{q/p+1}
                    \end{equation*}
                    But $\frac{1}{p}+\frac{1}{q}=1$, and thus
                    $\frac{q}{p}+1=q$. Thus:
                    \begin{align*}
                        |fx|
                        &=\sum_{k=1}^{\infty}|f_{k}|^{q}\\
                        &=\norm{(f_{1},f_{2},\hdots)}_{q}^{q}\\
                        &=\norm{(f_{1},f_{2},\hdots)}_{q}
                            \norm{(f_{1},f_{2},\hdots)}_{q}^{q-1}\\
                        &=\norm{(f_{1},f_{2},\hdots)}_{q}
                            \Big(\sum_{k=1}^{\infty}|f_{k}^{q}
                            \Big)^{\frac{q-1}{q}}
                        &=\norm{(f_{1},f_{2},\hdots)}_{q}
                            \Big(\sum_{k=1}^{\infty}|x_{k}|^{p}
                            \Big)^{1/p}\\
                        &=\norm{(f_{1},f_{2},\hdots)}_{q}\norm{x}_{p}
                    \end{align*}
                    Therefore, $\norm{f}=\norm{(f_{1},f_{2},\hdots)}_{q}$.
                    Thus, for all $y\in\ell^{q}$ there is a bounded
                    linear operator $f\in\ell^{p}$ such that
                    $\norm{y}_{\ell^{q}}=\norm{f}_{(\ell^{p})'}$. That
                    is, every $(f_{i})\in\ell^{q}$ defines an element
                    of $(\ell^{p})'$ by
                    $fx=\sum_{k=1}^{\infty}f_{k}x_{k}$, for any
                    $(x_{i})\in\ell^{p}$. So $\ell^{q}$ can
                    be \textit{embedded} into to $(\ell^{p})'$.
                    Now we need to show that this embedding is
                    the entirety of $(\ell^{p})'$. If
                    $f\in(\ell^{p})'$, let
                    $f_{i}=f(e_{i})$, where $e_{i}$ is the
                    sequence $(0,0,\hdots,1,0,0,\hdots)$, where
                    the 1 occurs in the $i^{th}$ spot. We need
                    to show that $(f_{i})\in\ell^{q}$ and
                    $fx=\sum_{k=1}^{\infty}f_{k}x_{k}$
                    for all $x\in\ell^{p}$. If $x\in\ell^{p}$, then:
                    \begin{align*}
                        x=\sum_{k=1}^{\infty}x_{k}e_{k}
                        \Rightarrow
                        fx=f\Big(\sum_{k=1}^{\infty}x_{k}e_{k}\Big)
                        =\sum_{k=1}^{\infty}f(x_{k}e_{k})
                        =\sum_{k=1}^{\infty}x_{k}f(x_{k})
                        =\sum_{k=1}^{\infty}x_{k}f_{k}
                    \end{align*}
                    Choosing $x_{k}=|f_{k}|^{q/p}\sgn(f_{k})$ and apply
                    H\"{o}lder.
                \end{proof}
        \subsection{Lecture 8: October 29, 2018}
            \subsubsection{Review}
                If $X$ is a vector space, an inner product on
                $X$ is a mapping
                $\langle,\rangle:X\times{X}\rightarrow\mathbb{R}$
                such that:
                \begin{enumerate}
                    \item $\langle{x,x}\rangle\geq{0}$ and
                          $\langle{x,x}\rangle=0$ if and only if
                          $x=0$
                    \item $\langle{x,y}\rangle=\langle{y,x}\rangle$
                    \item $\langle{ax+by,z}\rangle%
                           =a\langle{x,z}\rangle+b\langle{y,z}\rangle$
                \end{enumerate}
                Think of the dot product for vectors. This is a
                generalization of this concept. Every inner product
                on a vector space $V$ induce a norm on $V$:
                \begin{equation*}
                    \norm{x}=\sqrt{\norm{x,x}}
                \end{equation*}
                An inner product that is complete with respect to
                the induced norm is called a Hilbert Space. A mapping
                $f:X\rightarrow\mathbb{R}$ is bounded if there is a
                $K\in\mathbb{R}$ such that, for all $x\in{X}$,
                $|f(x)|\leq{L\norm{x}}$. $f$ is linear if
                $f(ax+by)=af(x)+bf(y)$, for all $x,y\in{X}$ and all
                $a,b\in\mathbb{R}$. The smallest such $K$ that works
                is called the norm of $f$, denoted $\norm{f}$. For
                all $x\in{X}$, $|f(x)|\leq\norm{f}\norm{x}$. The
                vector space of all bounded linear functionals on
                $X$ is the dual space $X'$. This is also a Banach
                space with the functional norm $\norm{f}$. One
                question that arises is, how do we know that there
                are bounded linear functionals on a space $X$? In
                the case that $X$ is a Hilbert space, this is rather
                easy, but for a more general Banach space this is not
                that trivial. For any normed space $X$ we can at least
                one bounded linear functional because the zero mapping
                $f(x)=0$ is such a functional. The question is then
                does every normed space have a bounded linear functional
                on it? The answer is yes, and this is related to
                the Hahn-Banach Theorem. As said before, in the
                Hilbert case this is rather easy.
                \begin{theorem}
                    If $X$ is a Hilbert space, then there is a
                    non-trivial bounded linear functional
                    $f:X\rightarrow\mathbb{R}$.
                \end{theorem}
                \begin{proof}
                    If $X$ is an inner product and
                    $z\in{X}$, let $f(x)=\langle{x,z}\rangle$ for
                    all $x\in{X}$. Then $f$ is linear since the
                    inner product is linear. But moreover, from
                    Cauchy-Schwarz we have:
                    \begin{equation*}
                        |f(x)|=|\langle{x,z}\rangle|
                        \leq\norm{x}\norm{z}
                    \end{equation*}
                    And thus $\norm{f}\leq\norm{z}$
                    But $|f(z)|=\norm{z}$, so
                    $\norm{f}=\norm{z}$. $f$ is a bounded
                    linear functional.
                \end{proof}
                Riesz's Representation theorem says that this is it.
                All bounded linear functionals look like this. Thus,
                if $H$ is a Hilbert space, it's dual $H'$ is the space
                of all functions that look like
                $f(x)=\langle{x,y}\rangle$ for some $y\in{X}$. More
                precisely, if $H$ is a Hilbert space and $f\in{H'}$,
                then there is a $y\in{H}$ such that, for all
                $x\in{X}$, $f(x)=\langle{x,y\rangle}$.
            \subsubsection{Riesz's Representation Theorem}
                \begin{theorem}
                    If $H$ is a Hilbert space, and
                    $f:H\rightarrow\mathbb{R}$ is a bounded
                    linear functional, then there is a unique
                    $y\in{H}$ such that, for all $x\in{X}$,
                    $f(x)=\langle{x,y}\rangle$. Moreover,
                    $\norm{f}=\norm{y}$.
                \end{theorem}
                \begin{proof}
                    Let $f\in{H'}$ and let $N=\nul(f)$. That is,
                    $N$ is the null space of the functional $f$
                    which is the set of all points
                    $x\in{X}$ such that $f(x)=0$. The Null space
                    actually defines a closed vector space, which
                    is a subspace of $H$. If $N=H$, then
                    $f(x)=0$, and thus let $y=0$. Otherwise, let $z$
                    be a non-zero elements such that,
                    for all $x\in{N}$, $\langle{x,y}\rangle=0$.
                    For all $x,z$, $f(x)z-f(z)x\in{N}$, for:
                    \begin{equation*}
                        f(f(x)z-f(z)x)
                        =f(f(x)z)-f(f(z)z)
                        =f(x)f(z)-f(z)f(z)=0
                    \end{equation*}
                    Therefore:
                    \begin{align*}
                        \langle{f(x)z-f(z)x,z}\rangle&=0\\
                        \Rightarrow
                        |f(x)|\norm{z}^{2}-|fz|\langle{x,y}\rangle&=0\\
                        \Rightarrow
                        f(x)
                        &=\langle{x,\frac{f(z)z}{\norm{z}^{2}}}\rangle
                    \end{align*}
                    Therefore, let $y=\frac{f(z)}{\norm{z}^{2}}z$.
                    This is unique since if for all $x\in{H}$,
                    $\langle{x,y_{1}}\rangle=\langle{x,y_{2}}\rangle$,
                    then $y_{1}=y_{2}$. Finally:
                    \begin{equation*}
                        \norm{y}
                        =\frac{|f(z)|}{\norm{z}^{2}}\norm{z}
                        =\frac{|f(z)|}{\norm{z}}
                        \leq\norm{f}
                    \end{equation*}
                    But also:
                    \begin{equation*}
                        |f(x)|=|\langle{x,z}\rangle|
                        \leq\norm{x}\norm{z}
                    \end{equation*}
                    Thus, $\norm{f}\leq\norm{z}$. But
                    $\norm{z}\leq\norm{f}$. Therefore,
                    $\norm{f}=\norm{z}$.
                \end{proof}
                Much like in $\mathbb{R}^{n}$, there is a notion of
                orthogonality in a general inner product space.
                \begin{definition}
                    Orthognal elements in an inner product space $X$
                    are elements $x,y\in{X}$ such that
                    $\langle{x,y}\rangle=0$.
                \end{definition}
                There's also a notion of convexity for a general
                vector space.
                \begin{definition}
                    A convex subset of a vector space $V$ space is a
                    subset $S\subset{V}$ such that, for all
                    $x,y\in{V}$ and for all $\lambda\in\mathbb{R}$,
                    $\lambda{x}+(1-\lambda)y\in{S}$.
                \end{definition}
                \begin{theorem}
                    If $S$ is a subset of $V$, then $S$ is convex.
                \end{theorem}
                Recall that for a general metric space $X$,
                if $S\subset{X}$, we defined
                $dist(x,S)=\inf\{d(x,s):s\in{S}\}$. We proved that,
                if $S$ is compact, then there is an $s\in{S}$ such
                that $dist(x,S)=d(x,s)$. We showed that, without
                compactness, this may not be true. Indeed, even
                complete spaces may lack this property. If
                $X$ is a Hilbert space, however, this property is
                guaranteed.
                \begin{theorem}
                    If $H$ is a Hilbert space and if $S\subset{H}$
                    is a closed convex subset of $H$, then there is
                    a unique $s\in{S}$ such that
                    $dist(x,S)=\norm{x-s}$.
                \end{theorem}
                \begin{proof}
                    As $dist(x,S)=\inf\{d(x,s):s\in{S}\}$, there is
                    a sequence $x_{n}\in{S}$ such that
                    $\norm{x-x_{n}}\rightarrow{dist(x,S)}$. Then, by
                    Appolonius:
                    \begin{align*}
                        \norm{x-x_{n}}^{2}+\norm{x-x_{m}}^{2}
                        &=\frac{1}{2}\norm{x_{n}-x_{m}}^{2}
                        +\frac{1}{2}
                        \norm{\frac{1}{2}(x_{n}+x_{m})-x}^{2}\\
                        &\geq\frac{1}{2}\norm{x_{n}+x_{m}}^{2}
                        +2dist(x,S)
                    \end{align*}
                    But $\norm{x-x_{n}}\rightarrow{dist(x,S)}$ and
                    $\norm{x-x_{m}}\rightarrow{dist(x,S)}$, so:
                    \begin{equation*}
                        \frac{1}{2}\norm{x_{n}-x_{m}}^{2}
                        \leq\norm{x-x-{m}}^{2}
                        +\norm{x-x_{m}}^{2}-2dist(x,S)
                    \end{equation*}
                    Which can be made arbitrarily small. Therefore,
                    $x_{n}$ is Cauchy. But $H$ is a Hilbert space, and
                    is therefore complete, and thus $x_{n}$ converges.
                    Let $s$ be the limit. As $S$ is closed, $s\in{S}$.
                    Moreover, from construction,
                    $\norm{x-s}=dist(x,S)$. If there is another point
                    $v$, then $\norm{x-s}=\norm{x-v}$. From
                    Appolonius:
                    \begin{equation*}
                        \norm{x-s}^{2}+\norm{x-v}^{2}
                        \geq\frac{1}{2}\norm{s-v}+2dist(x,S)^{2}
                    \end{equation*}
                    But $\norm{x-s}=\norm{x-v}=dist(x,S)$, and
                    thus $\norm{s-v}=0$. Therefore, $s=v$.
                \end{proof}
                Is $S$ is a closed subspace of $H$, then it's
                automatically convex. In this case, $x-s\perp{S}$,
                where $z\perp{S}$ means that, for all
                $s\in{S}$, $\langle{s,z}\rangle=0$. For if
                $z\in{S}$, then $s+tz\in{S}$ for all $t$. Thus:
                \begin{equation*}
                    \norm{s+tz-x}\geq{dist(x,S)}=\norm{s-x}^{2}
                \end{equation*}
                And therefore:
                \begin{equation*}
                    \rangle{s-x,s-x}\rangle+2t\langle{s-x,z}
                    +t^{2}\langle{z,z}\geq{s-x}^{2}
                    \Rightarrow{t^{2}\norm{z}^{2}+2t\langle{s-x,z}}
                    \geq{0}
                \end{equation*}
                Looking at the discriminant of this polynomial, we
                have:
                \begin{equation*}
                    \langle{s-x,z}\rangle=0
                \end{equation*}
                Therefore, $s-x\perp{S}$. You obtain $s\in{S}$
                by ``dropping the perpendicular of $x$,'' onto
                $S$. That is, $s$ is the orthogonal projection
                of $x$ onto $S$. $s=P_{S}x$ where
                $P_{S}:H\rightarrow{H}$ is the orthogonal
                projection. This has a few nice properties:
                \begin{enumerate}
                    \item It is idempotent: $P_{S}^{2}=P_{S}$.
                    \item Self adjoint:
                          $\langle{P_{S}x,y}\rangle%
                           =\langle{x,P_{S}y}\rangle=$
                    \item Linear.
                    \item Bounded and $\norm{P_{S}}=1$.
                \end{enumerate}
                If $S$ is a subset of an inner product space $X$,
                we write $S^{\perp}=\{x\in{X}:\langle{x,s}\rangle=0\}$.
                This is often read aloud as ``$S$ perp'' or
                ``$S$ perpendicular.''
                \begin{theorem}
                    If $S\subset{X}$, then $S^{\perp}$ is a
                    closed subspace.
                \end{theorem}
                \begin{theorem}
                    $S\subset(S^{\perp})^{\perp}$
                \end{theorem}
                The direct sum of two subsets of a Hilbert space
                $H$ is
                $S_{1}\oplus{S_{2}}=\{ax+by:x\in{S_{1}},y\in{S_{2}}\}$.
                \begin{theorem}
                    If $H$ is a Hilbert space and $S$ is a closed
                    subspace of $H$, then $H=S\oplus{S^{\perp}}$
                \end{theorem}
                \begin{proof}
                    For $x=P_{S}(x)+(x-P_{S}(x))$, and thus there is
                    an element in $S$ and an element in $S^{\perp}$
                    such that $x$ is the sum of those two elements.
                    This is the only representation. For if
                    $x=s_{1}+s_{1}^{\perp}$ and
                    $x=s_{2}+s_{2}^{\perp}$, then stuff.
                \end{proof}
                If $X$ and $Y$ are normed spaces, and if
                $f\in{B(X,Y)}$, then
                $\{x\in{X}:f(x)=0\in{Y}\}$ is called the null
                space of $f$.
                \begin{theorem}
                    If $X$ and $Y$ are normed spaces, and if
                    $f\in{B(X,Y)}$, then $\nul(f)$ is a closed
                    linear subspace of $X$.
                \end{theorem}
                \begin{proof}
                    Obvious since $f$ is linear and continuous.
                \end{proof}
                In a Hilbert space $H$, then
                $H=\nul(f)\oplus\nul(f)^{\perp}$. Thus, if
                $\nul(f)\ne{H}$, then $\nul(f)^{\perp}\ne\{0\}$.
                That is, there exists a $z\in\nul(f)^{\perp}$ that
                is non-zero. This is the $z$ we used to prove the
                Riesz representation theorem. Riesz's
                Theorem thus says that every Hilbert space is its own dual.
        \subsection{Lecture 9: November 5, 2018}
            \subsubsection{Adjoint}
                If $H$ is a Hilbert space and
                $f\in{H'}$, then there is a $z\in{H}$
                such that $f(x)=\langle{x,z}\rangle$ for
                all $x\in{H}$. Moreover, $\norm{f}=\norm{z}$.
                The adjoint of $T\in{B(H,H)}$ is an
                operator $T^{*}:H\rightarrow{H}$ such that
                $\langle{Tx,y}\rangle=\langle{x,T^{*}}\rangle$.
                There is always such an operator for any
                $T\in{B(H,H)}$. $T^{*}$ is also bounded and
                linear. By Riesz there is a $z=T^{*}y$ such
                that $f(x)=\langle{x,T^{*}y}\rangle$. Then
                $\norm{T^{*}y}=\norm{z}=\norm{f}\leq\norm{T}\norm{y}$.
                Thus, $\norm{T^{*}}\leq\norm{T}$. Therefore
                $T^{*}\in{B(H,H)}$. $T^{*}$ is called the
                ajdoint of $T$.
                \begin{example}
                    Consider $\mathbb{R}^{n}$ with the usual
                    inner product. Let $T$ be the matrix
                    $(T_{ij})$. Then:
                    \begin{equation*}
                        (Tx)_{i}=\sum_{j=1}^{n}T_{ij}x_{j}    
                    \end{equation*}
                    and:
                    \begin{equation*}
                        \langle{Tx,y}\rangle
                        =\sum_{i=1}^{n}(Tx)_{i}y_{i}
                        =\sum_{i=1}^{n}\sum_{j=1}^{n}
                        T_{ij}x_{j}y_{i}
                        =\sum_{j=1}^{n}\sum_{i=1}^{n}
                        T_{ij}y_{i}x_{j}
                    \end{equation*}
                    If $T^{*}$ is the adjoint, then:
                    \begin{equation*}
                        \langle{x,T^{*}y}\rangle
                        =\sum_{j=1}^{n}
                        \Big(\sum_{i=1}^{n}
                             T^{*}_{ji}y_{i}\Big)x_{j}
                    \end{equation*}
                    And thus $T^{*}_{ji}=T_{ij}$.
                    That is, the adjoint
                    of $T$ is the transpose of $T$. If we were in
                    $\mathbb{C}^{n}$ we would use the complex
                    conjugate of the transpose of $T$.
                    In general, if $T=T^{*}$
                    we say that $T$ is \textit{self-adjoint}.
                    This is also called symmetric or Hermitian.
                \end{example}
                \begin{example}
                    As another example, consider
                    $H=\ell^{2}$ and let
                    $T(x_{1},x_{2},\hdots)=(x_{2},x_{3},\hdots)$.
                    This is linear, and:
                    \begin{equation*}
                        \norm{T(x_{1},x_{2},\hdots)}
                        =\norm{(x_{2},x_{3},\hdots)}
                        =\sqrt{\sum_{n=2}^{\infty}x_{n}^{2}}
                        \leq
                        \sqrt{\sum_{n=1}^{\infty}x_{n}^{2}}
                        =\norm{(x_{1},x_{2},\hdots)}
                    \end{equation*}
                    Therefore $T$ is bounded and
                    $\norm{T}\leq{1}$. But
                    $T(0,1,0,0,\hdots)=(1,0,0,\hdots)$
                    showing that $\norm{T}\geq{1}$.
                    Thus, $\norm{T}=1$. Then, from
                    the definition of $T$:
                    \begin{align*}
                        \langle{Tx,y}\rangle
                        &=\langle{(x_{2},x_{3},\hdots),
                                  (y_{1},y_{2},\hdots)}\rangle\\
                        &=x_{2}y_{1}+x_{3}y_{2}+\hdots\\
                        &={x_{1}}\cdot{0}+x_{2}y_{1}+
                        x_{3}y_{2}+\hdots\\
                        &=\langle{(x_{1},x_{2},\hdots),
                                  (0,y_{1},y_{2},\hdots)}\rangle
                    \end{align*}
                    And therefore
                    $T^{*}(y_{1},y_{2},\hdots)=(0,y_{1},y_{2},\hdots)$.
                    Also $\norm{T^{*}}=1$. In general,
                    if $T\in{B(H,H)}$
                    then $\norm{T^{*}}=\norm{T}$.
                \end{example}
                \begin{theorem}
                    If $T\in{B(H,H)}$, then
                    $T^{**}=T$.
                \end{theorem}
                \begin{theorem}
                    $\norm{T}=\norm{T^{*}}$
                \end{theorem}
                \begin{proof}
                    For $\norm{T}\leq\norm{T^{*}}$ and
                    $\norm{T^{*}}\leq\norm{T^{**}}$, but
                    $T=T^{**}$, and therefore
                    $\norm{T}=\norm{T^{*}}$.
                \end{proof}
                \begin{example}
                    Let $x=C[0,1]$ and let
                    $\langle{x,y}\rangle=\int_{0}^{1}x(t)y(t)\diff{t}$.
                    Let $K:X\times{X}\rightarrow\mathbb{R}$
                    be continuous and define $T$ by:
                    \begin{equation*}
                        Tx(t)=\int_{0}^{1}K(t,s)x(s)\diff{s}
                    \end{equation*}
                    Then for all $x\in{X}$, $Tx\in{X}$ as well,
                    since $K$ is continuous.
                    Moreover, from Cauchy-Schwarz:
                    \begin{equation*}
                        \norm{Tx}^{2}
                        =\int_{0}^{1}
                        \Big[\int_{0}^{1}
                             K(t,s)x(d)\diff{s}\Big]^{2}\diff{t}
                        \leq\int_{0}^{1}
                        \Big[\int_{0}^{1}K(t,s)^{2}\diff{s}
                        \int_{0}^{1}x(s)^{2}\diff{s}\Big]\diff{t}
                    \end{equation*}
                    But
                    $\int_{0}^{1}x(s)^{2}\diff{s}=\norm{x}^{2}$.
                    So:
                    \begin{equation*}
                        \norm{Tx}^{2}\leq
                        \norm{x}^{2}\int_{0}^{1}\int_{0}^{1}
                        K(t,s)\diff{s}\diff{t}
                    \end{equation*}
                    Therefore $T$ is bounded and:
                    \begin{equation*}
                        \norm{T}\leq
                        \sqrt{\int_{0}^{1}\int_{0}^{1}
                              K(t,s)\diff{s}\diff{t}}
                    \end{equation*}
                    Computing the adjoint:
                    \begin{align*}
                        \langle{Tx,y}\rangle
                        &=\int_{0}^{1}Tx(t)y(t)\diff{t}\\
                        &=\int_{0}^{1}\Big(
                        \int_{0}^{1}K(t,s)x(s)\diff{s}\Big)
                        y(t)\diff{t}\\
                        &=\int_{0}^{1}\Big(
                        \int_{0}^{1}K(t,s)y(t)\diff{t}\Big)
                        x(s)\diff{s}\\
                        &=\int_{0}^{1}\Big(
                        \int_{0}^{1}K(s,t)y(s)\diff{s}\Big)
                        x(t)\diff{t}\\
                        &=\int_{0}^{1}Ty(s)x(s)\diff{s}\\
                        &=\langle{Ty,x}\rangle
                    \end{align*}
                    We may swap the order of integration since
                    $K$ is continuous on a compact set.
                \end{example}
                \begin{theorem}
                    $\norm{T^{*}T}=\norm{T}^{2}$
                \end{theorem}
                \begin{proof}
                    For:
                    \begin{equation*}
                        \norm{T^{*}Tx}\leq\norm{T^{*}}
                        \norm{Tx}\leq
                        \norm{T^{*}}\norm{T}\norm{x}
                    \end{equation*}
                    And therefore $\norm{T^{*}T}\leq\norm{T}^{2}$.
                    On the other hand:
                \end{proof}
                \begin{theorem}
                    If $T$ is self-adjoint, then
                    $\norm{T}=\sup\{|\langle{Tx,x}\rangle|:\norm{x}=1\}$.
                \end{theorem}
                \begin{proof}
                    \label{thm:Funct:Norm_of_Self_%
                           Adjoint_Operator}
                    Let
                    $\alpha=\sup\{sup\{|\langle{Tx,x}\rangle|:\norm{x}=1\}$.
                    Then:
                    \begin{equation*}
                        |\langle{Tx,x}\rangle|
                        \leq\norm{Tx}\norm{x}\leq
                        \norm{T}\norm{x}^{2}
                    \end{equation*}
                    Taking the supremum over $\norm{x}=1$,
                    we have $\alpha\leq\norm{T}$.
                    But if $\norm{x}=\norm{y}=1$, then:
                    \begin{align*}
                        |\langle{Tx,y}\rangle|
                        &=|\frac{1}{4}\langle{T(x+y),x+y}\rangle
                        -\frac{1}{4}\langle{T(x-y),x-y}\rangle\\
                        &=|\frac{1}{4}\norm{x+y}^{2}
                        \langle{T}\frac{x+y}{\norm{x+y}},
                               \frac{x+y}{\norm{x+y}}\rangle
                        -\frac{1}{4}\norm{x-y}
                        \langle{T}\frac{x-y}{\norm{x-y}},
                               \frac{x-y}{\norm{x-y}}\rangle
                    \end{align*}
                    Since
                    $\langle{Tx,y}\rangle=\langle{x,Ty}\rangle$,
                    as $T$ is self-adjoint.
                    And from the definition of $\alpha$:
                    \begin{equation*}
                        |\langle{Tx,y}\rangle|
                        \leq\frac{\alpha}{4}
                        \big(\norm{x+y}^{2}+\norm{x-y}^{2}\big)
                        \leq\frac{\alpha}{4}
                        \big(2\norm{x}^{2}+2\norm{y}^{2}\big)
                        =\alpha
                    \end{equation*}
                    Let $y=Tx/\norm{Tx}$, we get:
                    \begin{equation*}
                        \langle{Tx,\frac{Tx}{\norm{Tx}}}\rangle
                        \leq\alpha
                    \end{equation*}
                    And therefore $\norm{T}\leq\alpha$. But also
                    $\alpha\leq\norm{T}$. Thus, $\norm{T}=\alpha$.
                \end{proof}
                Thm.~\ref{thm:Funct:Norm_of_Self_Adjoint_Operator}
                can fail if $T$ is not self-adjoint.
                In $\mathbb{R}^{2}$, let
                $T(x_{1},x_{2})=(0,x_{1})$. Then:
                \begin{equation*}
                    \norm{Tx}^{2}=x_{1}^{2}\leq
                    x_{1}^{2}+x_{2}^{2}
                    =\norm{x}^{2}
                \end{equation*}
                And therefore $\norm{T}\leq{1}$.
                But $T(1,0)=(0,1)$, and
                thus $\norm{T}=1$. But if $(x_{1},x_{2})$
                lies on the unit circle, then
                $|x_{1}x_{2}|\leq0.5$. Thus:
                \begin{equation*}
                    |\langle{Tx,x}\rangle|
                    =|\langle(x_{1},x_{2}),(0,x_{1})\rangle
                    =|x_{1}x_{2}|\leq\frac{1}{2}
                \end{equation*}
                Therefore
                $|\langle{Tx,x}\rangle|\leq{0.5}<\norm{T}$ for
                all $x\in\mathbb{R}^{2}$ such that $\norm{x}=1$.
            \subsubsection{Compact Operators}
                Compact operators can be defined in a more
                general spaces
                than that of Hilbert or Banach spaces.
                They can be defined
                on Topological spaces, but we won't go that far.
                For now we will simply define them on a
                general metric space.
                \begin{definition}
                    A compact mapping from a metric space $X$
                    to a metric
                    space $Y$ is a function $T:X\rightarrow{Y}$
                    such that
                    for all bounded subsets $S$ of $X$, the image
                    $T(S)$ is pre-compact in $Y$. That is,
                    $\overline{T(S)}$ is compact
                    (The closure of $T(S)$ is compact).
                \end{definition}
                \begin{theorem}
                    If $T:X\rightarrow{Y}$ is a linear compact
                    operator between normed spaces $X$ and $Y$,
                    then $T$ is continuous.
                \end{theorem}
                \begin{proof}
                    For let $S=\overline{B_{1}(\mathbf{0})}$.
                    This is bounded, so
                    $\overline{T(S)}$ is compact, and therefore bounded.
                    Let $M$ be such a bound.
                    Thus, for all $s\in\overline{S}$
                    such that $\norm{s}=1$,
                    $\norm{Ts}\leq{M}$, and therefore $\norm{T}\leq{M}$.
                    Thus $T$ is bounded and linear, and is therefore
                    continuous.
                \end{proof}
                \begin{example}
                    Every linear mapping
                    $T:\mathbb{R}^{n}\rightarrow\mathbb{R}^{m}$
                    is compact.
                    As a another example, let
                    $X=C[0,1]$ and equip this with the supremum norm.
                    Define $T$ as:
                    \begin{equation*}
                        Tx(t)=\int_{0}^{1}K(t,s)x(s)\diff{s}
                    \end{equation*}
                    Where $K:[0,1]\times[0,1]\rightarrow\mathbb{R}$
                    is continuous. This is a compact operator. For if
                    $S$ is a bounded subset then there exists
                    an $M$ such
                    that for all $x\in{S}$, $\norm{x}\leq{M}$. Thus:
                    \begin{align*}
                        \norm{Tx}=\sup|Tx(t)|
                        &=\sup\Big|\int_{0}^{1}
                        K(t,s)x(s)\diff{s}\Big|\\
                        &\leq\sup\int_{0}^{1}
                        |K(t,s)||x(s)|\diff{s}\\
                        &\leq\kappa\int_{0}^{1}|x(s)|\diff{s}\\
                        &\leq\kappa\norm{x}
                    \end{align*}
                    Where $\kappa=\sup|K(t,s)|$. $\kappa$ exists
                    since $K(t,s)$ is continuous on a compact
                    set and is therefore
                    bounded. So $T(S)$ is uniformly bounded. To apply
                    Arzela-Ascoli we need to show that
                    $T(S)$ is equicontinuous. That is, for all
                    $\varepsilon>0$ there is a $\delta>0$ such that,
                    for all $x\in{S}$, if $|t_{2}-t_{1}|<\delta$
                    then $|Tx(t_{2})-Tx(t_{1})|<\varepsilon$. If we
                    can show that $T$ satisfies this, then
                    $\overline{T(S)}$ is compact,
                    and thus $T$ is compact.
                    Let's show this. If $x\in{S}$, then:
                    \begin{align*}
                        |Tx(t_{2})-Tx(t_{1})|
                        &=\Big|\int_{0}^{1}K(t_{1},s)x(s)\diff{s}
                        -\int_{0}^{1}K(t_{2},s)x(s)\diff{s}\Big|\\
                        &=\Big|\int_{0}^{1}(K(t_{2},s)-
                        K(t_{1},s))x(s)\diff{s}\Big|\\
                        &\leq
                        \int_{0}^{1}|K(t_{2},s)-K(t_{1},s)||x(s)|
                        \diff{s}\\
                        &\leq{M}\int_{0}^{1}|K(t_{2},s)-K(t_{1},s)|
                        \diff{s}
                    \end{align*}
                    But as $K$ is uniformly continuous,
                    there is a $\delta>0$
                    such that, for all $s\in[0,1]$,
                    $|t_{2}-t_{1}|<\delta$ implies
                    $|K(t_{2},s)-K(t_{1},s)|<\varepsilon/M$.
                    Thus, $T(S)$ is equicontinuous. We can replace the
                    supremum norm with $L^{2}$ and $T$ is still compact.
                    Indeed, it is true for $L^{p}$ if we replace the
                    use of Cauchy-Schwarz with the more general
                    H\"{o}lder's Inequality. From this we have that
                    $T$ is a compact self-adjoint operator.
                \end{example}
        \subsection{Lecture 10: November 19, 2018}
            \subsubsection{Compact Linear Operators}
                A linear operator $T:X\rightarrow{Y}$
                is compact if $\overline{T(S)}$ is compact
                for all bounded $S\subset{X}$. Example,
                $Tx(t)=\int_{0}^{1}K(t,s)x(s)\diff{s}$,
                where $K$ is continuous on
                $[0,1]^{2}$, is a compact operator
                $T:C[0,1]\rightarrow{C[0,1]}$. If
                $K(x,t)=K(t,x)$ for all
                $(x,t)\in[0,1]^{2}$, then
                $T$ is a self-adjoint operator on
                $L^{2}[0,1]$. $L^{2}[0,1]$ can be seen
                as the completion of $C[0,1]$ with respect
                to the $L^{2}$ norm.
                \begin{theorem}
                    A linear operator
                    $T:X\rightarrow{Y}$ is compact if and only
                    if for all bounded sequences
                    $x:\mathbb{N}\rightarrow{X}$,
                    $Tx$ has a congergent subsequence in $Y$ 
                \end{theorem}
                We're interested mainly in the case of $Y=X$,
                and when $T:X\rightarrow{X}$ is bounded in
                linear. This is the set of all operators
                $B(X,X)$. We rewrite this as $B(X)$. That is,
                $B(X)$ is the set of all bounded linear operators
                from $X$ to itself. Recall that if
                $Y$ is a Banach space, and if $X$ is a normed
                space, then $B(X,Y)$ is a Banach space. Thus,
                if $X$ is a Banach space, then $B(X)$ is a
                Banach space. But we can also multiply elements
                in $B(X)$ by using function composition.
                If $S,T\in{B(X)}$, then $ST$ is defined by
                $(ST)(x)=S(Tx)$. But then:
                \begin{equation*}
                    \norm{(ST)x}=\norm{S(Tx)}
                    \leq\norm{S}\norm{Tx}
                    \rightarrow
                    \norm{ST}\leq\norm{S}\norm{T}
                \end{equation*}
                A Banach space with such a multiplication
                property is called a Banach Algebra. The set of
                compact linear operators on $X$ is often denoted
                $C(X)$. Thus, $C(X)\subset{B(X)}$. It's
                a two-sided closed ideal in $B(X)$. That is,
                if $S,T\in{C(X)}$, and if $a$ and $b$ are scalars,
                then $aS+bT\in{C(X)}$, $ST\in{C(X)}$,
                and $TS\in{C(X)}$. Finally, if
                $F:\mathbb{N}\rightarrow{C(X)}$ is a sequence of
                compact operators, and if $F_{n}\rightarrow{T}$,
                then $T\in{C(X)}$.
                \begin{definition}
                    Orthogonal Elements in an inner product
                    space $(X,\langle\rangle)$ are elements
                    $x,y\in{X}$ such that
                    $\langle{x,y}\rangle=0$.
                \end{definition}
                \begin{definition}
                    An orthonormal subset of an inner
                    product space $(X,\langle\rangle)$
                    is a subset $S\subset{X}$ such that for
                    all $x,y\in{S}$ such that $x\ne{y}$,
                    $\langle{x,y}\rangle=0$ and for all
                    $x\in{S}$, $\norm{x}=1$.
                \end{definition}
                \begin{theorem}
                    If $x\in{X}$ and
                    $\varphi:\mathbb{N}\rightarrow{X}$
                    is a sequence such that
                    $A=\{\varphi_{n}:n\in\mathbb{N}\}$ is an
                    orthonormal subset of $X$, then for all
                    $x\in{X}$:
                    \begin{equation*}
                        \norm{x}=
                        \sum_{n=1}^{N}\langle{x,\varphi_{n}}
                        \rangle^{2}
                        +\norm{x-\sum_{n=1}^{N}
                               \langle{x,\varphi_{n}}\rangle
                               \varphi_{n}}^{2}
                    \end{equation*}
                \end{theorem}
                \begin{proof}
                    If $m\in\mathbb{Z}_{N}$, then:
                    \begin{equation*}
                        \langle{x-\sum_{n=1}^{N}
                                \langle{x,\varphi_{n}}\rangle
                                \varphi_{n},\varphi_{m}}\rangle
                        =\langle{x,\varphi_{m}}\rangle
                        -\sum_{n=1}^{N}\langle{x,\varphi_{n}}\rangle
                        \langle{\varphi_{n},\varphi_{m}}\rangle
                    \end{equation*}
                    But $A$ is an orthonormal subset of $X$,
                    and thus if $n\ne{m}$ then
                    $\langle{\varphi_{n},\varphi_{m}}\rangle=0$
                    ad if $n=m$ then
                    $\langle{\varphi_{n},\varphi_{m}}\rangle%
                     =\norm{\varphi_{n}}=1$. In terms of the
                    Kronecker-Delta function,
                    $\langle{\varphi_{n},\varphi_{m}}\rangle%
                     =\delta_{nm}$. So we have:
                    \begin{equation*}
                        \langle{x-\sum_{n=1}^{N}
                                \langle{x,\varphi_{n}}\rangle
                                \varphi_{n},\varphi_{m}}\rangle
                        =0
                    \end{equation*}
                    But for all $N\in\mathbb{N}$,
                    $x=(x-\sum_{n=1}^{N}%
                     \langle{x,\varphi_{n}}\rangle\varphi_{n})+%
                     \sum_{n=1}^{N}\langle{x,\varphi_{n}}\rangle%
                     \varphi_{n}$,
                    and these two are orthogonal.
                    Therefore, from Pythagoras:
                    \begin{align*}
                        \norm{x}^{2}&=
                        \norm{x-\sum_{n=1}^{N}
                              \langle{x,\varphi_{n}}\varphi_{n}
                              \rangle}^{2}+
                        \norm{\sum_{n=1}^{N}
                              \langle{x,\varphi_{n}}\varphi_{n}
                              \rangle}^{2}\\
                        &=\sum_{n=1}^{N}\langle{x,\varphi_{n}}
                        \norm{\varphi_{n}}^{2}\rangle
                        +\norm{\sum_{n=1}^{N}
                        \langle{x,\varphi_{n}}\varphi_{n}
                        \rangle}^{2}
                    \end{align*}
                    But $\norm{\varphi_{n}}^{2}=1$ for all $n$.
                    Therefore, etc.
                \end{proof}
                \begin{theorem}[Bessel's Inequality]
                    $\sum_{n=1}^{N}\langle{x,\varphi_{n}}\rangle%
                     \leq\norm{x}^{2}$
                \end{theorem}
                \begin{example}
                    $A=\{e_{n}:n\in\mathbb{N}\}$ is an orthonormal
                    subset of $\ell^{2}$.
                    $A=\{\sin(nt)/\sqrt{\pi}:n\in\mathbb{N}\}$
                    is an orthonormal subset of
                    $C[0,1]$ with the $L^{2}$ inner product.
                \end{example}
                \begin{definition}
                    A basis of an innert product space $X$
                    is an orthonormal subset $A\subset{X}$
                    such that there is no orthonormal subset
                    $B\subset{X}$ such that $A\subset{B}$.
                \end{definition}
                \begin{theorem}
                    If $(X,\langle\rangle)$ is an inner product
                    space, then there is an $A\subset{X}$
                    such that $A$ is an orthonormal subset of $X$.
                \end{theorem}
                \begin{theorem}
                    If $X$ is an inner product space, and if
                    $\varphi:\mathbb{N}\rightarrow{X}$ is a sequence
                    such that $S=\{\varphi_{n}:n\in\mathbb{N}\}$
                    is a basis of $X$, then for all $x\in{X}$
                    there is a sequence
                    $a:\mathbb{N}\rightarrow\mathbb{R}$
                    such that
                    $x=\sum_{n=1}^{\infty}a_{n}\varphi_{n}$
                \end{theorem}
            \subsubsection{Summability}
                What does $\sum_{\alpha\in{A}}z_{\alpha}$
                if $z_{\alpha}\in\mathbb{R}$ for all
                $\alpha\in{A}$?
                \begin{definition}
                    We say
                    $\sum_{\alpha\in{A}}z_{\alpha}$ is summable
                    to $z\in\mathbb{R}$ if for all
                    $\varepsilon>0$ there is a subset
                    $B\subset{A}$ such that, for all
                    $C\subset{A}$ such that
                    $B\subset{C}$,
                    $|\sum_{\alpha\in{C}}z_{\alpha}-z|<\varepsilon$.
                \end{definition}
                It turns out that, if
                $\sum_{\alpha\in{A}}z_{\alpha}$ is summable, then
                only countably many $z_{\alpha}$ are non-zero.
                For all $n$, the set
                $\{z_{\alpha}:z_{\alpha}>1/n\}$ must be finite.
                The set of all non-zero elements is the
                union over all of these $n$, which is the
                countable union of finite sets, which is
                thus countable.
                \begin{theorem}
                    If $(X,\langle\rangle)$ is an inner
                    product space, if
                    $\varphi:\mathbb{N}\rightarrow{X}$ is a
                    sequence such that
                    $S=\{\varphi_{n}:n\in\mathbb{N}\}$ is a
                    basis of $X$, then:
                    \begin{equation*}
                        x=\sum_{n=1}^{\infty}
                        \langle{x,\varphi_{n}}\rangle
                        \varphi_{n}
                    \end{equation*}
                \end{theorem}
                \begin{example}
                    If $L^{2}[0,\pi]$, let
                    $\varphi_{n}(t)=\sin(nt)\sqrt{2/\pi}$. Then
                    $A=\{\varphi_{n}(t):n\in\mathbb{N}\}$.
                \end{example}
        \subsection{Lecture 11: November 26, 2018}
            Let $T$ be a linear operator on a vector space
            $T$. We say $\lambda\in\mathbb{C}$ is an
            eigenvalue of $T$ if there exists $x\ne{0}$ in
            $X$ such that $Tx=\lambda{x}$. The corresponding
            $x$ is called the eigenvector or eigenfunction.
            We're interested in the case of compact
            self-adjoint operators $T$ on a Hilbert space
            $\mathscr{H}$.
            \begin{theorem}
                There is a sequence of real eigenvalues
                $\lambda_{n}$ of $T$, finite or infinite,
                such that $0$ is the only possible
                accumulation point of $\lambda_{n}$, and
                corresponding basis of
                orthogonormal eigenvectors $x_{n}$.
            \end{theorem}
            \begin{proof}
                We'll prove this in steps. First, either
                $\norm{T}$ or $-\norm{T}$ is an eigenvalue. This
                is because:
                \begin{equation*}
                    \norm{T}=
                    \underset{\norm{x}=1}{\sup}
                    \{|\langle{Tx,x}\rangle|\}
                \end{equation*}
                This comes from the fact that
                $T=T^{*}$ for self-adjoint operators. Thus,
                either
                $\norm{T}=\langle{Tx,x}\rangle$ or
                $-\norm{T}=\langle{Tx,x}\rangle$. Choose a
                sequence $x_{n}$ such that
                $\norm{x_{n}}=1$ and
                $\langle{Tx_{n},x_{n}}\rangle%
                 \rightarrow\pm\norm{T}$. By choosing a
                subsequence, we may assume that
                $Tx_{n}$ converges. We can not assume that
                $x_{n}$ converges, however. Thus,
                $Tx_{n}\rightarrow{y}$. Then:
                \begin{align*}
                    \norm{Tx_{n}-\lambda{x}_{n}}^{2}
                    &=\langle{Tx_{n}-\lambda{x}_{n},
                              Tx_{n}-\lambda{x}_{n}}\rangle\\
                    &=\norm{Tx_{n}}^{2}
                    -2\lambda\langle{Tx_{n},x}\rangle
                    +\lambda^{2}\norm{x}^{2}\\
                    &\leq
                    \norm{T}^{2}\norm{x}^{2}
                    -2\lambda\langle{Tx_{n},x}\rangle
                    +\lambda^{2}\norm{x}^{2}
                \end{align*}
                And this converges to zero as $n$ tends to
                infinity. Thus,
                $Tx_{n}-\lambda{x}_{n}\rightarrow{0}$. It
                follows that
                $\lambda{x}_{n}=Tx_{n}-(Tx_{n}-\lambda{x}_{n})$,
                and this converges to $y$. Therefore
                $x_{n}\rightarrow{y}/\lambda$. Note that
                $\lambda$ is only equal to zero if
                $T$ is the zero operator. In this case the
                problem is trivial. Thus we may assume
                $\lambda\ne{0}$. Therefore
                $Tx_{n}\rightarrow{Ty}/\lambda$, but
                $Tx_{n}\rightarrow{y}$ as well. Thus
                $y=Ty/\lambda$. Let $\lambda_{1}=\lambda$
                and $\varphi_{1}=y/\norm{y}$. Then
                $\norm{\phi_{1}}=1$ and
                $T\varphi_{1}=\lambda_{1}\phi_{1}$. Moreover,
                let $T_{1}=T$ and let $H_{1}=\mathscr{H}$.
                Let $H_{2}=\{x\in{H}_{1}:x\perp\varphi_{1}\}$.
                Define $T_{2}:H_{2}\rightarrow{H}_{1}$ by
                $T_{2}x=T_{1}x=Tx$. This is the restriction
                of $T$ to $H_{2}$. Then
                $T_{2}H_{2}\subseteq{T}_{2}H_{2}$. For if
                $x\in{H}_{2}$, then
                $\langle{T}_{2}x,\varphi_{1}\rangle%
                 \langle{T}_{1}x,\varphi_{1}\rangle$. But
                $T$ is self-adjoint, and
                $T_{1}=T$, and therefore
                $T_{1}$ is self-adjoint. But then:
                \begin{equation*}
                    \langle{T}_{2}x,\varphi_{1}\rangle
                    =\langle{T}_{1}x,\varphi_{1}\rangle
                    =\langle{x},T_{1}\varphi_{1}\rangle
                    =\lambda_{1}\langle{x}_{1},
                        \varphi_{1}\rangle
                \end{equation*}
                Therefore $T_{2}x\in{H}_{2}$, and therefore
                $T_{2}:H_{2}\rightarrow{H}_{2}$. $T_{2}$
                is self-adjoint, for:
                \begin{align*}
                    \langle{T}_{2}x,y\rangle
                    =\langle{T}_{1}x,y\rangle
                    =\langle{x},T_{1}y\rangle
                    =\langle{x},T_{2}y\rangle
                \end{align*}
                $T_{2}$ is compact since if $x_{n}$ is
                bounded in $H_{2}$, then it's bounded in
                $H_{1}$, and thus
                $T_{2}x_{n}=T_{1}x_{n}$, which has a convergent
                subsequence,
                and therefore $T_{2}$ is compact. $H_{2}$ is
                a subspace of $\mathscr{H}$ and is closed since
                $x_{n}\in{H}_{2}$ and
                $x_{n}\rightarrow{x}$ in $\mathscr{H}$ then:
                \begin{equation*}
                    \langle{x},\varphi_{1}\rangle=
                    \langle\lim{x}_{n},\varphi_{1}\rangle=
                    \lim\langle{x}_{n},\varphi_{1}\rangle=
                    0
                \end{equation*}
                Thus $H_{2}$ is closed and is therefore complete,
                and thus $H_{2}$ is a Hilbert space. As before
                there is a $\varphi_{2}$ and a $\lambda_{2}$
                such that $\varphi_{2}\in{H}_{2}$,
                $\norm{\varphi_{2}}=1$, and:
                \begin{equation*}
                    T_{2}\varphi_{2}=\lambda_{2}\varphi_{2}
                \end{equation*}
                But then $T\varphi_{2}=\lambda_{2}\varphi_{2}$,
                where $\lambda_{2}=\pm\norm{T_{2}}$. Moreover,
                $|\lambda_{2}|<|\lambda_{1}|$. Continuing in
                this manner, let
                $H_{n}=\{x\in\mathscr{H}:%
                 x\perp\varphi_{1},\hdots,\varphi_{n-1}\}$
                and $T_{n}:H_{n}\rightarrow{H}_{n}$ be the
                restriction of $T$ onto $H_{n}$. We obtain a
                $\varphi_{n}$ such that
                $\norm{\varphi_{n}}$ and
                $T\varphi_{n}=\lambda\varphi_{n}$. Moreover
                $|\lambda_{n}|\leq|\lambda_{n-1}|$. Thus,
                $|\lambda_{n}|$ forms a monotonically
                decreasing sequence and either there is an
                $N\in\mathbb{N}$ such that
                $\lambda_{N}=0$, in which case for all $n>N$,
                $\lambda_{n}=0$ as well, or for all
                $n\in\mathbb{N}$, $|\lambda_{n}|>0$. In the first
                case it is clear that $\lambda_{n}\rightarrow{0}$.
                In the second case we have
                $\lambda_{n}\varphi_{n}=T\varphi_{n}$ has
                a convergent subsequence for $T$ is compact.
                But $|\lambda_{n}|$ is a monotonically
                decreasing sequence bounder below by zero,
                and therefore converges. Let $c$ be the limit.
                Then $\lambda_{n}x_{n}\rightarrow{cx}$.
                Moreover $\norm{\varphi_{n}-\varphi_{m}}^{2}=2$
                since $\varphi_{n}$ and $\varphi_{m}$ are
                orthogonal when $n\ne{m}$. Therefore
                $\varphi_{n}$ is not a Cauchy sequence. Thus,
                for $\lambda_{n}\varphi_{n}$ to converge,
                $c=0$.
            \end{proof}
            \begin{theorem}[Hilbert-Schmidt Theorem]
                If $x\in\mathscr{H}$, then:
                \begin{equation*}
                    Tx=\sum_{n=1}^{\infty}
                        \lambda_{n}\langle{x},\varphi_{n}\rangle
                        \varphi_{n}
                \end{equation*}
            \end{theorem}
            \begin{proof}
                For define $y_{m}$ as:
                \begin{equation*}
                    y_{m}=x-\sum_{n=1}^{m-1}
                    \langle{x},\varphi_{n}\rangle
                    \varphi_{n}
                \end{equation*}
                Then:
                \begin{equation*}
                    \langle{y}_{m},\varphi_{k}\rangle
                    =\langle
                    x-\sum_{n=1}^{m-1}\rangle{x},\varphi_{n}
                    \rangle\varphi_{n},\varphi_{k}\rangle
                    =\langle{x},\varphi_{k}\rangle-
                    \sum_{n=1}^{m-1}\langle{x},\varphi_{n}
                    \rangle\langle\varphi_{n},\varphi_{k}\rangle
                    =0
                \end{equation*}
                If $\lambda_{N}=0$, then $Ty=T_{n}y=0$ by
                setting $m=N$. Thus:
                \begin{equation*}
                    0=Ty
                    =Tx-\sum_{n=1}^{N-1}
                    \langle{x},\varphi_{n}\rangle{T}\varphi_{n}
                    =Tx-\sum_{n=1}^{N-1}
                    \lambda_{n}\langle{x},\varphi_{n}\rangle
                    \varphi_{n}
                \end{equation*}
                If $\lambda_{n}\ne{0}$ for all $n\in\mathbb{N}$,
                then $y_{m}\in{H}_{m}$ and therefore:
                \begin{equation*}
                    x=(x-y_{m})+y_{m}
                \end{equation*}
                But $x-y_{m}$ is orthogonal to $y_{m}$, and
                therefore by Pythagoras:
                \begin{equation*}
                    \norm{x}^{2}=
                    \norm{x-y_{m}}^{2}+\norm{y_{m}}^{2}
                \end{equation*}
                Therefore $\norm{y_{m}}\leq\norm{x}$. Also:
                \begin{equation*}
                    \norm{Ty_{m}}=\norm{T_{m}y_{m}}
                    \leq\norm{T_{m}}\norm{y_{m}}
                    =|\lambda_{m}|\norm{y_{m}}
                \end{equation*}
                And therefore:
                \begin{equation*}
                    \norm{Tx-\sum_{n=1}^{m-1}
                    \lambda_{n}\rangle{x},\varphi_{n}\rangle
                    \varphi_{n}}\leq|\lambda_{m}|\norm{x}
                    \rightarrow{0}
                \end{equation*}
            \end{proof}
            \begin{theorem}
                If $T$ is a compact self-adjoint operator
                on a Hilbert Space $\mathscr{H}$, then there is
                an orthogonal basis for $\mathscr{H}$
                consisting of eigenvector of $T$.
            \end{theorem}
            \begin{proof}
                For any $x\in\mathscr{H}$:
                \begin{equation*}
                    Tx=\sum_{n=1}^{\infty}\lambda_{n}
                    \rangle{x},\varphi_{n}\rangle\varphi_{n}
                \end{equation*}
                This sum may be infinite.
                Let $\{\psi_{\alpha}\}_{\alpha\in{A}}$ be
                and orthogonal basis of $\nul(T)$. Then
                $T\psi_{\alpha}=0$ for all $\alpha\in{A}$,
                and thus $0$ is an eigenvalue for all
                $\psi_{\alpha}$. But also:
                \begin{equation*}
                    \lambda_{n}\langle\varphi_{n},
                    \psi_{\alpha}\rangle
                    =\langle\lambda_{n}\varphi_{n},
                    \psi_{\alpha}\rangle
                    =\langle{T}\varphi_{n},\psi_{\alpha}\rangle
                    =\langle\varphi_{n},T\psi_{\alpha}\rangle
                    =0
                \end{equation*}
                Thus, for all $\alpha\in{A}$ and all
                $n\in\mathbb{N}$,
                $\varphi_{n}\perp\psi_{\alpha}$. Then by
                Hilbert-Schmidt, for every $x\in\mathscr{H}$:
                \begin{equation*}
                    T\big(x-\sum_{n=1}^{\infty}
                    \langle{x},\varphi_{n}\rangle\varphi_{n}\big)
                    =Tx-\sum_{n=1}^{\infty}\lambda_{n}
                    \langle{x},\varphi_{n}\rangle\varphi_{n}=0
                \end{equation*}
                Thus:
                \begin{equation*}
                    x=\sum_{n=1}^{\infty}\lambda_{n}
                    \langle{x},\varphi_{n}\rangle\varphi_{n}+
                    \sum_{\alpha\in{A}}
                    \langle{x},\psi_{\alpha}\rangle\psi_{\alpha}
                \end{equation*}
            \end{proof}
    \section{More Stuffs}
        \subsection{Lecture 12: December 3, 2018}
            Cauchy-Schwarz says that:
            \begin{equation}
                \Big(\int_{a}^{b}x(s)y(s)\diff{s}\Big)^{2}
                \leq\Big(\int_{a}^{b}x^{2}(s)\diff{s}\Big)
                \Big(\int_{a}^{b}y^{2}(s)\diff{s}\Big)
            \end{equation}
            So:
            \begin{align}
                |Tx(t)|^{2}&=
                \Big(\int_{0}^{t}x(s)\diff{s}\Big)^{2}\\
                &\leq\Big(\int_{0}^{t}1^{2}\diff{s}\Big)
                \Big(\int_{0}^{t}x(s)^{2}\diff{s}\Big)\\
                &=t\int_{0}^{t}x(s)^{2}\diff{s}\\
                &\leq{t}\int_{0}^{1}x(s)^{2}\diff{s}\\
                &=t\norm{x}_{2}^{2}
            \end{align}
            So then:
            \begin{equation}
                \int_{0}^{1}Tx(t)^{2}\diff{t}
                \leq\int_{0}^{1}t\norm{x}_{2}^{2}\diff{t}
            \end{equation}
            This then implies:
            \begin{align}
                \norm{Tx}^{2}&\leq
                \frac{1}{2}\norm{x}_{2}^{2}\\
                \Rightarrow
                \norm{T}&\leq\frac{1}{\sqrt{2}}
            \end{align}
            Letting $x(t)=1$, we have $Tx=t$. Thus:
            \begin{equation}
                \norm{Tx}^{2}=\int_{0}^{1}t^{2}\diff{t}
                =\frac{1}{3}
            \end{equation}
            And thus $\norm{Tx}=1/\sqrt{3}$. So
            $1/\sqrt{3}\leq\norm{T}\leq{1}/\sqrt{2}$.
            If $x(t)=1-t$, then $Tx(t)=t-t^{2}/2$. So
            $\norm{Tx}=\sqrt{2/15}$. We're getting closer to
            the answer. Letting $x(t)=\cos(\pi{t}/2)$ gives
            us the norm. We now have to show this. Write:
            \begin{equation*}
                x(t)=\sum_{n=1}^{\infty}b_{n}\sin(n\pi{t})
            \end{equation*}
            Then:
            \begin{equation*}
                Tx(t)=\sum_{n=1}^{\infty}
                \frac{b_{n}}{n\pi}\big[1-\cos(n\pi{t})\big]
                =\Big(\sum_{n=1}^{\infty}\frac{b_{n}}{n\pi}\Big)
                -\Big(\sum_{n=1}^{\infty}\frac{b_{n}}{n\pi}
                \cos(n\pi{t})\Big)
            \end{equation*}
            But:
            \begin{align*}
                \norm{x}^{2}&=\sum_{n=1}^{\infty}\frac{b_{n}^{2}}{2}
                &
                \norm{Tx}^{2}&=
                \Big(\sum_{n=1}^{\infty}\frac{b_{n}}{n\pi}\Big)^{2}
                +\sum_{n=1}^{\infty}\frac{b_{n}^{2}}{2n^{2}\pi^{2}}
            \end{align*}
            Let's maximize $\norm{Tx}^{2}$ subject to
            $\norm{x}^{2}$. Using Lagrange multipliers we get:
            \begin{equation*}
                \frac{2}{n\pi}\Big(\sum_{k=1}^{\infty}
                \frac{b_{k}}{k\pi}\Big)
                +\frac{b_{n}}{n^{2}\pi^{2}}
                =\lambda^{2}b_{n}
            \end{equation*}
            Letting $A=\sum_{k=1}^{\infty}b_{k}/k\pi$, we obtain:
            \begin{equation*}
                b_{n}=\frac{2n\pi}{\lambda^{2}n^{2}\pi^{2}-1}A
            \end{equation*}
            So then:
            \begin{equation*}
                A=\sum_{n=1}^{\infty}
                \frac{2}{\lambda^{2}n^{2}\pi^{2}-1}A
            \end{equation*}
            And thus:
            \begin{equation*}
                \sum_{n=1}^{\infty}
                \frac{2}{\lambda^{2}n^{2}\pi^{2}-1}=1
            \end{equation*}
            And this is the expansion of cotangent:
            \begin{equation*}
                1-\frac{1}{\lambda}\cot\big(\frac{1}{\lambda}\big)
                =1
            \end{equation*}
            And therefore:
            \begin{equation*}
                \lambda=\frac{\pi}{2},\frac{3\pi}{2},\frac{5\pi}{2},
                \hdots
            \end{equation*}
            Finally:
            \begin{equation*}
                x(t)=\sum_{n=1}^{\infty}
                b_{n}\sin(n\pi{t})=
                \sum_{n=1}^{\infty}
                \frac{2n\pi}{4n^{2}-1}A\sin(n\pi{t})
                =\sqrt{2}\cos\big(\frac{\pi}{2}t\big)
            \end{equation*}
            But before we can do all of this we need to show
            that there is such a maximum. That is, the step that
            involves Lagrange multipliers is valid. We want
            an $x$ such that $\norm{x}=1$ and
            $\norm{Tx}=\norm{T}$. Certainly there is a sequence
            such that $\norm{x_{n}}=1$ and
            $\norm{Tx_{n}}\rightarrow\norm{T}$. Since $T$ is
            compact we may assume, taking subsequences as
            necessary, that
            $Tx_{n}\rightarrow{y}$. If $T$ is self adjoint
            and $x_{n}\rightarrow{x}$, then
            $Tx=y$. It would be nice if Bolzano-Weiestrass worked
            and we could say
            $\norm{x_{n}}$ bounded implies that
            $x_{n}\rightarrow{x}$, but this is not always
            true in infinite dimensions. We'll need to
            weaken our notion of convergence for this. We
            could simply say that everything converges to
            everything, which is the chaotic topology, but
            this is not very useful for it loses uniqueness.
            \begin{definition}
                A weakly convergent sequence in an inner product
                space $X$ is a sequence
                $x:\mathbb{N}\rightarrow{X}$ such that there
                is an $a\in{X}$ such that
                for all $z\in{X}$,
                $\langle{x_{n},z}\rangle\rightarrow%
                 \langle{a,z}\rangle$. We write
                 $x_{n}\overset{w}{\rightarrow}{a}$
            \end{definition}
            For example, if $X$ is a Hilbert space and
            $e_{n}$ is an orthonormal basis, then
            $e_{n}\rightarrow{0}$ since, for all $z$:
            \begin{equation*}
                \norm{z}^{2}=\sum_{n=1}^{\infty}
                \langle{e_{n},z}\rangle^{2}
            \end{equation*}
            And thus $\langle{e_{n},z}\rangle\rightarrow{0}$.
            But $\langle{0,z}\rangle=0$, so
            $e_{n}\overset{w}{\rightarrow}{0}$. Normal convergence
            is also called strong convergence. Strong convergence
            implies weak convergence. For if
            $x_{n}\rightarrow{a}$, then
            $\langle{a-x_{n},z}\rangle\rightarrow{0}$ for all
            $z$, and thus $x_{n}\overset{w}{\rightarrow}{a}$.
            Moreover, weak limits are unique. If
            $T$ is a bounded linear operator on a Hilbert space
            $H$, and if $x_{n}$ converges weakly to $a$,
            then $Tx_{n}\overset{w}{\rightarrow}{Ta}$.
            If $x$ converges weakly to $a$ and if $T$ is a
            compact linear operator on $H$, then
            $Tx_{n}$ converges strongly to $Tx$.
            \begin{theorem}
                If $H$ is a Hilbert space,
                $T$ is a compact operator on $H$,
                and if $x:\mathbb{N}\rightarrow{H}$ is a
                weakly convergent sequence such that
                $x_{n}\overset{w}{\rightarrow}{a}$,
                then the sequence
                $y:\mathbb{N}\rightarrow{H}$ defined by
                $y_{n}=Tx_{n}$ is such that 
                $y_{n}\rightarrow{Tx}$. That is,
                $Tx_{n}$ converges strongly to $Ta$.
            \end{theorem}
            \begin{proof}
                For suppose not. As $T$ is compact and
                $x_{n}$ is bounded, there is
                a convergent subsequence. Let $a_{1}$ be the
                limit. If the limit is not unique then there
                is another convergent subsequence with
                a different limit $a_{2}$. But
                $Tx_{n}$ converges weakly to
                $a$, and strong convergence implies weak
                convergence. Therefore $a_{1}=a_{2}=a$, a
                contradiction. Therefore, $Tx_{n}$ converges
                strongly to $Ta$.
            \end{proof}
            \begin{theorem}
                If $x_{n}\overset{w}{\rightarrow}{a}$
                and $\norm{x_{n}}\rightarrow{C}$,
                then $\norm{a}\leq{C}$.
            \end{theorem}
            \begin{proof}
                For:
                \begin{align*}
                    \norm{a}^{2}&=|\langle{a,a}\rangle|\\
                    &=\underset{n\rightarrow\infty}{\lim}
                    |\langle{x_{n},a}\rangle|\\
                    &\leq\underset{n\rightarrow\infty}{\lim}
                    \norm{x_{n}}\norm{a}\\
                    &=C\norm{a}
                \end{align*}
                Dividing by $\norm{a}$ gives the result.
            \end{proof}
            We now prove that there exists $x$ such that
            $\norm{x}=1$ and $\norm{Tx}=\norm{T}$. This works
            for any compact linear operator on a Hilbert space.
            We can always find a sequence, by definition, such
            that $\norm{x_{n}}=1$ and
            $\norm{Tx_{n}}\rightarrow\norm{T}$. But since
            $x_{n}$ is bounded by 1 there is a weakly
            convergent subsequence (Still to be proved).
            This is ``Bolzano-Weierstrass,'' of infinite
            dimensions. Then $x_{n}$ converges weakly to
            $x$, and thus $Tx_{n}$ converges strongly to
            $Tx$. Then $\norm{Tx}=\norm{T}$. Finally,
            $\norm{x}\leq\lim\norm{x_{n}}=1$, and
            $\norm{T}=\norm{Tx}\leq\norm{T}\norm{x}$,
            so $\norm{x}\geq{1}$, and therefore
            $\norm{x}=1$. Let's show $T:L^{2}\rightarrow{L}^{2}$
            defined by $Tx(t)=\int_{0}^{t}x(s)\diff{s}$ is
            compact. Suppose $x_{n}$ is bounded in $L^{2}$ with
            bound $M$. That is, $\norm{x_{n}}\leq{M}$.
            Then:
            \begin{align*}
                |Tx_{n}(t)|^{2}&=
                \Big|\int_{0}^{t}x_{n}(s)^{2}\diff{s}\Big|^{2}\\
                &\leq
                \Big(\int_{0}^{1}|x_{n}(s)|\diff{s}\Big)^{2}\\
                &\leq\Big(\int_{0}^{1}\diff{s}\Big)
                \Big(\int_{0}^{1}|x_{n}(s)|^{2}\diff{s}\Big)\\
                &=\norm{x_{n}}^{2}\\
                &\leq{M}^{2}
            \end{align*}
            Taking the supremum over $t\in[0,1]$ gives:
            \begin{equation*}
                \norm{Tx_{n}}_{\infty}\leq{M}
            \end{equation*}
            So $Tx_{n}$ is bounded in $C[0,1]$. Also, if
            $0\leq{t}_{1}$ and $t_{2}\leq{1}$, then:
            \begin{align*}
                |Tx_{n}(t_{2})-Tx_{n}(t_{1})|^{2}
                &\leq\Big(\int_{t_{1}}^{t_{2}}|x_{n}(s)|
                \diff{s}\Big)^{2}\\
                &\leq\Big(\int_{t_{1}}^{t_{2}}\diff{s}\Big)
                \Big(\int_{t_{1}}^{t_{2}}|x_{n}(s)|^{2}
                \diff{s}\Big)\\
                &=(t_{2}-t_{1})\int_{t_{1}}^{t_{2}}
                |x_{n}(s)|^{2}\diff{s}\\
                &\leq(t_{2}-t_{1})\norm{x_{n}}^{2}\\
                \leq{M}^{2}(t_{2}-t_{1})
            \end{align*}
            So $|Tx_{n}(t_{2})-Tx_{n}(t_{1})|$ can be made
            arbitrarily small for $t_{2}$ and $t_{1}$ close enough,
            independent on the $n$. That is, a
            $\delta$ may be chosen independent of $n$. This is
            the criterion for equicontinuity.
            The compactness of $[0,1]$ then gives uniform
            equicontinuity. Arzela-Ascoli then says there is a
            subsequence $Tx_{n}\rightarrow{y}$,
            with $y\in{c}[0,1]$. That is,
            $\norm{Tx-y}_{\infty}\rightarrow{0}$. But then:
            \begin{align*}
                \norm{Tx_{n}-y}^{2}
                &=\int_{0}^{1}|Tx_{n}(t)-y(t)|^{2}\diff{t}\\
                &\leq\int_{0}^{1}
                \norm{Tx_{n}-y}_{\infty}^{2}\diff{t}\\
                =\norm{Tx_{n}-y}_{\infty}^{2}
            \end{align*}
            And this converges to zero. Thus,
            $Tx_{n}\rightarrow{y}$.
            \begin{theorem}[Baire Category Theorem]
                If $(X,d)$ is a complete metric and
                $C_{n}$ is a sequence of closed sets such
                that $X=\cup_{n=1}^{\infty}C_{n}$, then
                there is an $N\in\mathbb{N}$ such that
                $C_{N}$ contains an open subset.
            \end{theorem}
            \begin{proof}
                Let $r_{1}\in(0,1)$, $x_{1}\in{X}$.
                If $B_{r_{1}}(X_{1})\subset{C_{1}}$ then we're
                done. Otherwise $B_{r_{1}}(x_{1})\setminus{C_{1}}$
                is a non-empty set, so there exists
                $x_{n}\in{B}_{r_{2}}(x_{1})$, where
                $r_{2}\in(0,1/2)$. By induction, choose
                $r_{n}\in(0,1/n)$ and $x_{n}$ such that
                $x_{n}\in{B}_{r_{n-1}}(x_{n-1})$ and
                $\overline{B_{r_{n}}(x_{n})}\subset%
                 B_{r_{n-1}}(x_{n-1})$. For $n<m$,
                $x_{m}\in\overline{B_{r_{n}}(x_{n})}$,
                so $d(x_{n},x_{m})<1/n$. Then $x_{n}$ is
                Cauchy, but $X$ is complete so there is a limit
                $x$. Since $\overline{B_{r_{n}}(x_{n})}$ is
                closed, $x\in\overline{B_{r_{n}}(x_{n})}$. But
                $X=\cup_{n=1}^{\infty}C_{n}$ and thus there is
                an $N$ such that $x\in{N}$. But then
                $B_{r_{n}}(x)\subset{C}_{N})$, so $C_{N}$ contains
                an open subset.
            \end{proof}
            The Baire Category Theorem is used to prove the
            Uniform Boundedness Theorem, which is also called
            the Banach-Steinhau theorem.
            \begin{theorem}[Uniform Boundedness Theorem]
                If $H$ is a Hilbert space, and if
                $x_{n}\overset{w}{\rightarrow}{a}$, then
                $\norm{x_{n}}$ is bounded.
            \end{theorem}
            \begin{proof}
                Let
                $C_{k}=\{y\in{H}:|\langle{x_{n},y}\rangle\leq{k}\}$.
                Then $C_{k}$ is closed since $y_{j}\in{C_{k}}$
                and $y_{j}\rightarrow{y}$ implies that:
                \begin{align*}
                    |\langle{x_{n},y_{j}}\rangle|&\leq{k}\\
                    \Rightarrow
                    \underset{j\rightarrow\infty}{\lim}
                    |\langle{x_{n},y_{j}}\rangle|&\leq{k}\\
                    \Rightarrow
                    |\langle{x_{n},y}\rangle|&\leq{k}
                \end{align*}
                Moreover $H=\cup_{k=1}^{\infty}C_{k}$. By
                the Baire Category Theorem there is a
                $k\in\mathbb{N}$ such that
                $C_{k}$ contains an open subset. Let $z_{0}\in{H}$
                and $r\in\mathbb{R}$ be such that
                $B_{r}(z_{0})\subset{C_{k}}$. Let
                $y\in{H}$, $y\ne{0}$, and set $z=z_{0}+\alpha{y}$.
                where:
                \begin{equation*}
                    \alpha=\frac{r}{2\norm{y}}
                \end{equation*}
                Then $\norm{z-z_{0}}<r$, so
                $z\in{C}_{k}$. That is,
                $|\langle{x_{n},z}\rangle|\leq{k}$ for all
                $n$. Thus we have:
                \begin{align*}
                    |\langle{x_{n},y}\rangle|&=
                    |\langle{x_{n},\frac{z-z_{0}}{\alpha}}\rangle|\\
                    &\leq
                    \frac{1}{\alpha}\Big(
                    |\langle{x_{n},z\rangle}|+
                    |\langle{x_{n},z_{0}}\rangle|\\
                    &\leq
                    \frac{1}{\alpha}(k+k)\\
                    &=\frac{4k}{r}\norm{y}
                \end{align*}
                This is true of any $y\in{H}$. Choosing $y=x_{n}$,
                we get:
                \begin{equation*}
                    \norm{x_{n}}^{2}\leq\frac{4k}{r}\norm{x_{n}}
                \end{equation*}
                Dividing by $\norm{x_{n}}$ shows boundedness.
            \end{proof}
        \subsection{Lecture 13: December 10, 2018}
            \begin{equation}
                Tx(t)=\int_{0}^{1-t}x(s)\diff{s}
            \end{equation}
            If $\norm{x}_{2}\leq{M}$, then:
            \begin{equation}
                |Tx(t)|^{2}=
                \Big|\int_{0}^{1-t}x(s)\diff{s}\Big|^{2}
                \leq\Big(\int_{0}^{1-t}\diff{s}\Big)
                \Big(\int_{0}^{1-t}x(s)^{2}\diff{s}\Big)
                =(1-t)\norm{x}_{2}^{2}
            \end{equation}
            So we have:
            \begin{equation}
                \norm{Tx}^{2}=\int_{0}^{1}Tx(t)^{2}\diff{t}
                \leq\int_{0}^{1}(1-t)\norm{x}_{2}^{2}\diff{t}
                =\frac{1}{2}\norm{x}_{2}^{2}
                \leq\frac{1}{2}M^{2}
            \end{equation}
            So:
            \begin{align}
                |Tx(t_{1})-Tx(t_{2})|^{2}
                &=\Big|\int_{0}^{1-t_{1}}x(s)\diff{s}-
                \int_{0}^{1-t_{2}}x(s)\diff{s}\Big|^{2}\\
                &=\Big|\int_{1-t_{2}}^{1-t_{1}}x(s)
                \diff{s}\Big|^{2}\\
                &\leq\int_{1-t_{2}}^{1-t_{1}}\diff{s}
                \int_{1-t_{2}}^{1-t_{1}}x(s)^{2}\diff{s}\\
                &=|t_{1}-t_{2}|\norm{x}_{2}^{2}\\
                &\leq{M}^{2}|t_{1}-t_{2}|
            \end{align}
            This shows equicontinuity, and thus Arzela-Ascoli
            shows that $T$ is compact.
            \begin{theorem}[Banach-Alaoglu-Hilbert Theorem]
                If $H$ is a Hilbert space and
                $x:\mathbb{N}\rightarrow{H}$ is a bounded sequence,
                then there is a weakly convergent subsequence.
            \end{theorem}
            The next question would be ``What about Banach Space?''
            If $X$ is a normed space, we say $x_{n}$ converges
            weakly to $x$, denoted
            $x_{n}\overset{w}{\rightarrow}{x}$ if for all
            bounded linear functional $f\in{X'}$,
            $f(x_{n})\rightarrow{f(x)}$.
            For example let $X=\ell^{1}$. We have proven that the
            dual of $\ell^{1}$ is $\ell^{\infty}$ and elements
            of the dual take the form:
            \begin{equation}
                f(x)=\sum_{n=1}^{\infty}z_{i}x_{i}
            \end{equation}
            Where $z_{i}\in\ell^{\infty}$. Thus, $z_{i}$ is
            bounded and $x_{i}$ is absolutely convergent,
            since $x_{i}\in\ell^{1}$, and thus the product
            is absolutely convergent. That is,
            $x_{i}z_{i}\in\ell^{1}$. The problem with this is
            that we don't know that $X'\ne\{0\}$ for a given
            Banach space. There are plenty these, enough to
            separate points, thanks to the Hahn-Banach theorem.
            This says that if $f$ is a bounded linear functional
            on a subspace $M$ of $X$, then there exists
            $F\in{X'}$ such that
            $\norm{F}_{X'}=\norm{f}_{M'}$ and
            $F(x)=f(x)$ for all $x\in{M}$. So given $m\in{M}$,
            then $\alpha{m}\in{M}$ for al $\alpha\in\mathbb{R}$.
            Define $f(m)=k$, for some $k\in\mathbb{R}$. Then
            $f(\alpha{m})=\alpha{k}$ and
            $|f(\alpha{m})|=|\alpha||f(m)|%
             =|k|\norm{\alpha{m}}/\norm{m}$. And
             $|f(\alpha{m})|/\norm{\alpha{m}}=|k|/\norm{m}$.
            Thus $f\in{M'}$. Thus, Hahn-Banach can be exteneded
            to all of $X$. So we can for all $x\in{X}$ and for
            all $r\in\mathbb{R}$, there is a bounded linear
            function $f\in{X'}$ such that
            $f(x)=r$. If $m_{1}$ and $m_{2}$ are independent
            (That is, $m_{1}\ne\alpha{m}_{2}$ for any real
            number $\alpha$), then let
            $M=\{am_{1}+bm_{2}:a,b\in\mathbb{R}\}$. Define
            $f(am_{1}+bm_{2})=a\norm{m_{1}}$.
            Then $f\in{M'}$ so this
            can be extended to all of $X$ by the Hahn-Banach
            theorem. But $f(m_{1})=1$ and
            $f(m_{2})=0$, so $f$ separates points. Thus,
            if $x_{n}$ converges weakly to $x$ in a
            Banach space, and if $x_{n}$ also converges weakly
            to $y$, then $x=y$. The uniform boundedness theorem
            also holds if $X$ is complete. The
            Banach-Alaoglu-Hilbert theorem fails in a general
            Banach space. For example, $\ell^{1}$. We now talk
            about weak* convergence. In $X'$, we say
            $f_{n}$ converges weak* to $f$ if
            $f_{n}(x)\rightarrow{f(x)}$ for all
            $x\in{X}$. Banach-Alaoglu holds if weak is replaced
            with weak* and if $X'$ is separable. The double
            dual, $X''$, is the dual of $X'$. $X$ is
            embedded in $X''$. That is, $X$ embeds naturally
            in $X''$ as follows: Define
            $C:X\rightarrow{X''}$ as follows. If $x\in{X}$,
            $Cx(f)=f(x)$ for all $f\in{X'}$. It's easy to
            show that $\norm{cx}=\norm{x}$. Indeed,
            $C$ is an isometry on $X$ to $X''$. If $C$ is
            onto, we say that $X$ is reflexive. There are
            Banach spaces that are not reflexive that can
            be isometrically embedded into their second dual,
            but the canonical map is not such an embedding.
            \begin{theorem}
                If $H$ is a Hilbert space, then $H$ is reflexive.
            \end{theorem}
            \begin{theorem}
                If $X$ is reflexive, then
                $X=X''$.
            \end{theorem}
            \begin{theorem}
                If $X$ is reflexive, weak* convergence
                implies weak convergence.
            \end{theorem}
            From topology, a subbasis for a topological space
            is a collection of sets such that every open set can
            be written as arbitrary unions and finite intersections
            of the sets. Choose as the subbasis:
            \begin{equation}
                \{f^{-1}(-\infty,a):a\in\mathbb{R},f\in{X'}\}
            \end{equation}
            If $X'$ is separable, then this space is metrizable.
            For let $A$ be a countable dense subset, and define
            the metric $d$ as:
            \begin{equation}
                d(f,g)=\sum_{x\in{A}}
                \frac{|f(x)-g(x)|}{1+|f(x)-g(x)|}2^{-n}
            \end{equation}
            \begin{theorem}[Banach-Alaoglu]
                If $X$ is a normed vector space, and if
                $\tau$ is the weak* topology, then
                $\overline{B_{1}(0)}$ is a compact subset of
                $(X',\tau)$.
            \end{theorem}
            Adams Sobolev Spaces.
    \section{Old Notes}
        \subsection{Summary of Lectures}
            The boundary of a circle
            in $\mathbb{R}^{2}$ is nowhere dense,
            with respect to the metric on $\mathbb{R}^{2}$.
            Any open ball about any point on the circle contains
            points not on the circle, and thus it has empty
            interior. Something about $\varepsilon$ nets.
            \subsubsection{Normed Spaces and Banach Spaces}
                There are notions of subspace,
                linear combination, independence, spanning,
                dimension, basis, Hamel basis, and
                \textit{convexity}.
                Open and closed balls are convex. 
                A subspace of a Banach space is complete
                iff closed. Schauder basis.
                A Schauder basis implies separable.
                If $\{x_{1},\hdots,x_{n}\}$ is independent,
                then there exists a $c>0$ such that, for all
                $\boldsymbol{\alpha}$,
                $|\boldsymbol{\alpha}\cdot\mathbf{x}|%
                 \geq{c}\norm{\boldsymbol{\alpha}}$.
                Finite dimensional subspaces are complete,
                as are closed subspaces. In finite dimensional
                normed spaces, a space is compact if and only
                if it is closed and bounded.
                Riesz's Lemma says that if $Z$ is a subspace
                of a normed space $X$, and if $Y$ is a proper
                closed subspace of $Z$, then there is a $z\in{Z}$
                such that $\norm{z}=1$ and $D(z,Y)\geq{1/2}$.
                A corollary of this is that $B_{1}(0)$ is compact
                if and only if $X$ is finite dimensional.
            \subsubsection{Linear Operators}
                Identity, zero, differentiation,
                and integration. Domain/Range
                of a linear operator, the null space.
                Inverse of a linear operator is linear.
                $(ST)^{-1}=T^{-1}S^{-1}$.
                In finite dimension all linear operators are
                continuous. An operator is bounded
                if and only if it is continuous. If a linear
                operator is continuous at some point, then it
                is continuous everywhere. An operator is bounded
                if and only if its null space is closed.
                There is something called the extension of a
                bounded linear operator. $B(X,Y)$ is the set of
                bounder linear operators
                from $X$ to $Y$. This is complete if and only if
                $Y$ is complete. A functional is a mapping from a
                vector space $X$ into the real numbers $\mathbb{R}$.
                For continuous linear functional, continuity at
                $0$ implies continuity everywhere.
                There is something called the dual space
                $X'$, which is itself a Banach space.
            \subsubsection{Inner Product and Hilbert Spaces}
                If $x_{n}\rightarrow{x}$ and $y_{n}\rightarrow{y}$,
                then
                $\langle{x_{n},y_{n}}\rangle\rightarrow\langle{x,y}\rangle$.
                There's a notion of orthogonal sets,
                and orthonormality. If $(e_{n})$ is orthonormal basis,
                then $x=\sum\langle{x,e_{k}}\rangle{e_{k}}$
                for all $x$. Gram-Schmidt procedure.
                $\sum\alpha_{k}e_{k}$ converges if and only if
                $\sum|\alpha_{k}|^{2}$ converges.
                A set $M$ is total in a Hilbert space $H$ is the
                span of the closure of $M$ is equal
                to $H$. If $M$ is complete, then it is
                total if and only if $M^{\perp}=0$.
                Parseval's theorem. Legendre, Hermite, and Laguerre
                polynomials are things.
                Self adjoint, unitary, and normal operators.
                $T^{*}=T$, $T^{*}=T^{-1}$, and $T^{*}T=TT^{*}$. If
                $X$ is a vector space over the complex numbers,
                and if $T$ is self adjoint, then
                $\langle{Tx,x}\rangle$ is a real number for all $x$.
            \subsubsection{Compact Linear Operators}
                If $T$ is compact and linear, then it is bounded and
                continuous. An operator is compact and linear
                if and only if
                for all bounded sequences $x_{n}$,
                $Tx_{n}$ has a convergent subsequence. Compact linear
                operators form a vector space. The rank of an
                operator is
                the dimension of its image. If $T$ is linear,
                bounded, and of finite rank, then it is compact.
                If $T_{n}$ is a sequence of compact linear
                operators, if $Y$ is
                complete, and if $\norm{T_{n}-T}\rightarrow{0}$, then
                $T$ is compact. A sequence $x_{n}$ converges weakly to
                $x$ if, for all $y$,
                $\langle{x_{n},y}\rangle\rightarrow\langle{x,y}\rangle$.
                If $x_{n}$ converges weakly to $x$, then
                and if $T$ is a compact linear operator, then
                $Tx_{n}\rightarrow{Tx}$. If $H$ is a Hilbert space,
                $T$ is a compact self-adjoint operator, and if
                $x_{n}$ converges weakly to $x$, then
                $\langle{Tx_{n},x_{n}}\rangle\rightarrow\langle{Tx,x}\rangle$.
                If $T:H\rightarrow{H}$ is compact and linear, then so
                is its adjoint. The Hilbert-Schmidt theorem says that
                compact self-adjoint operators on a Hilbert space $H$
                have an orthonormal basis of eigenvectors. All of this
                has applications to integral operators and
                Sturm-Liouville Theory.
            \subsubsection{Fundamental Theorems}
                Zorn's Lemma. Hahn-Banach Theorem. Sublinear functionals.
                If $X$ is a normed space, and $Z$ is a subspace, and if
                $f\in{Z'}$, then $f$ be extended to $X$ such that
                $\norm{f}_{X}=\norm{f}_{Z}$.
                This extends Hilbert spaces by Riesz.
                If $X$ is a normed space
                and $x\ne{0}$, then there is an $f\in{X'}$ such that
                $\norm{f}=1$ and $f(x_{0})=\norm{x_{0}}$.
                For all $x$,
                $\norm{x}=\sup\{\norm{f(x)}/\norm{f}:f\in{X'},f\ne{0}\}$.
                There's a thing called bounded variation.
                If $x\in{X}$ and
                $g_{x}(f)=f(x)$ for $f\in{X'}$, then
                $g_{x}\in{X''}$ and $\norm{g_{x}}=\norm{x}$.
                Reflexive implies complete.
                Finite and Hilbert implies reflexive.
                $X'$ separable implies $X$ is separable.
                $X$ separable and reflexive implies
                $X'$ is separable.
                Strong convergence implies weak convergence.
                The converse is not true. If $X$ is finite
                dimensional, then weak convergence
                implies strong convergence. Weak convergence implies
                $\norm{x_{n}}$ is bounded. If
                $x_{n}\rightarrow{x}$ weakly, and if
                $\norm{x_{n}}\rightarrow\norm{x}$, then
                $x_{n}\rightarrow{x}$ strongly.
                Open mapping theorem.
                Closed graph theorem.
                Differentiation is a closed operator on
                $C^{1}[a,b]\rightarrow{C[a,b]}$.
    \section{Notes from Dartmouth}
    \section{Metric Spaces}
        \subsection{Basic Definitions}
            \begin{ldefinition}{Pseudo-Metric}
                  {Funct_Analysis_Pseudo_Metric}
                A pseudo-metric on a set $X$ is a function
                $\rho:X\times{X}\rightarrow[0,\infty)$ such that,
                for all $x,y,z\in{X}$, it is true that:
                \begin{align}
                    \rho(x,y)&=\rho(y,x)
                    \tag{Symmetry}\\
                    \rho(x,z)&\leq\rho(x,y)+\rho(y,z)
                    \tag{Triangle Inequality}
                \end{align}
            \end{ldefinition}
            \begin{ldefinition}{Pseudo-Metric Space}
                  {Funct_Analysis_Pseudo_Metric_Space}
                A pseudo-metric space, $(X,\rho)$, is a set
                $X$ and a pseudo-metric $\rho$ on $X$.
            \end{ldefinition}
            \begin{theorem}
                There exist pseudo-metric spaces $(X,\rho)$
                such that for all $x\in{X}$, $\rho(x,x)>0$.
            \end{theorem}
            \begin{proof}
                For let $X=\mathbb{R}$ and define
                $\rho:\mathbb{R}^{2}\rightarrow[0,\infty)$ by:
                \begin{equation}
                    \rho(x,y)=1+|x|+|y|
                \end{equation}
                Then $\rho$ is a pseudo-metric. For it is symmetric,
                since:
                \begin{equation}
                    \rho(x,y)=1+|x|+|y|=1+|y|+|x|=\rho(y,x)
                \end{equation}
                Moreover, it obeys the triangle inequality:
                \begin{subequations}
                    \begin{align}
                        \rho(x,z)&=1+|x|+|z|\\
                        &\leq{1}+|x|+|z|+2|y|+1\\
                        &=(1+|x|+|y|)+(1+|y|+|z|)\\
                        &=\rho(x,y)+\rho(y,z)
                    \end{align}
                    Thus, $\rho$ is a pseudo-metric. However, for
                    all $x\in\mathbb{R}$:
                    \begin{equation}
                        \rho(x,x)=1+|x|+|x|=1+2|x|\geq{1}>0
                    \end{equation}
                    Thus, there are no $x\in{X}$ such that
                    $\rho(x,x)=0$. Therefore, etc
                \end{subequations}
            \end{proof}
            If we require $\rho(x,x)=0$ for all $x\in{X}$, we
            can still have the case where elements cannot be
            distinguished from. That is, there may be
            $x,y\in{X}$ such that $x\ne{y}$, but
            $\rho(x,y)=0$.
            \begin{theorem}
                There exist pseudo-metric spaces $(X,\rho)$
                such that for all $x\in{X}$, $\rho(x,x)=0$,
                and there are distinct elements $x,y\in{X}$
                such that $\rho(x,y)=0$.
            \end{theorem}
            \begin{proof}
                For let $X$ have at least two distinct elements,
                and let $\rho:X^{2}\rightarrow[0,\infty)$
                be defined by:
                \begin{equation}
                    \rho(x,y)=0
                \end{equation}
                Then $\rho$ is a pseudo-metric. Symmetry and
                the triangle inequality are both trivial.
                However, since there are at least two distinct
                elements in $X$, we have unique points such that
                $\rho(x,y)=0$. Therefore, etc.
            \end{proof}
            \begin{ldefinition}{Metric Space}
                  {Funct_Analysis_Metric_Space}
                A metric space is a pseudo-metric space $(X,d)$
                such that:
                \begin{equation}
                    d(x,y)=0\Longleftrightarrow{x}=y
                    \tag{Definiteness}
                \end{equation}
            \end{ldefinition}
            \begin{ldefinition}{Semi-Norm}
                  {Funct_Analysis_Semi_Norm}
                A semi-norm on a vector space $V$ over a field
                $\mathbb{F}\subseteq\mathbb{C}$ is a function
                $\norm{\cdot}:V\rightarrow[0,\infty)$ such that, for
                all $v\in{V}$ and $\alpha\in\mathbb{F}$,
                it is true that:
                \begin{align}
                    \norm{\alpha{v}}
                    &=|\alpha|\norm{v}
                    \tag{Homogeneity}\\
                    \norm{v+w}
                    &\leq\norm{v}+\norm{w}
                    \tag{Triangle Inequality}
                \end{align}
            \end{ldefinition}
            \begin{theorem}
                If $V$ is a vector space over a field
                $\mathbb{F}\subseteq\mathbb{C}$, and if
                $\norm{\cdot}$ is semi-norm on $V$, then:
                \begin{equation}
                    \norm{\mathbf{0}}=0
                \end{equation}
            \end{theorem}
            \begin{proof}
                For:
                \begin{equation}
                    \norm{\mathbf{0}}
                    =\norm{0\mathbf{0}}
                    =|0|\norm{\mathbf{0}}=0
                \end{equation}
                Therefore, etc.
            \end{proof}
            \begin{ldefinition}{Norm}
                  {Funct_Analysis_Norm}
                A norm on a vector space $V$ over a
                field $\mathbb{F}\subseteq\mathbb{C}$
                is a semi-norm $\norm{\cdot}$ such that:
                \begin{equation}
                    \norm{\mathbf{x}}\Longrightarrow\mathbf{x}
                    =\mathbf{0}
                \end{equation}
            \end{ldefinition}
            \begin{lexample}
                Suppose that $\norm{\cdot}_{0}$ is a semi-norm
                on a vector space $V$. Define the following:
                \begin{equation}
                    N=\{v\in{V}:\norm{v}_{0}=0\}
                \end{equation}
                If follows from the definition of a semi-norm that
                $N$ is a subspace of $V$. Thus we can define a
                function on the quotient space
                $\norm{\cdot}:V/N\rightarrow[0,\infty)$ by:
                \begin{equation}
                    \norm{v+N}=\norm{v}_{0}
                \end{equation}
                We can then verify that this is well defined and
                that $\norm{\cdot}$ is a norm on $V/N$.
            \end{lexample}
            If $\norm{\cdot}$ is a norm on $V$, then we get an
            associated metric $\rho$ via:
            \begin{equation}
                \rho(v,u)=\norm{v-u}
            \end{equation}
            \begin{lexample}
                Let $(X,\mathcal{M},\mu)$ be a measure space.
                That is, $X$ is a set, $\mathcal{M}$ is
                $\sigma\textrm{-Algebra}$, and $\mu$ is a measure
                on $X$. Let $1\leq{p}<\infty$. Then:
                \begin{equation}
                    \mathcal{L}^{p}(X)=
                    \{f:X\rightarrow\mathbb{C}:
                    f\textrm{ is measurable and}
                    \int_{X}|f|^{p}\diff{\mu}<\infty.\}
                \end{equation}
                The set $\mathcal{L}^{p}(X)$ is a vector space.
                We define the semi-norm on $\mathcal{L}^{p}(X)$
                to be:
                \begin{equation}
                    \norm{f}_{p}=
                    \Big(\int_{X}|f|^{p}\diff\mu\Big)^{1/p}
                \end{equation}
                This is not a norm, since there are many
                functions such that $\norm{f}_{p}=0$, yet $f\ne{0}$.
                However, if $\norm{f}_{p}=0$, then $f=0$ $\mu$
                almost-everywhere. So we create equivalence
                classes by s comparing functions that are $\mu$
                almost-everywhere. The final thing to check is
                the triangle-inequality. It is not obvious and is a
                consequence of Minkowski's Inequality. We get
                a normed vector space by considering $N$ to be
                the set of functions $f$ such that $\norm{f}_{p}=0$,
                and we define:
                \begin{equation}
                    L^{p}(X)=\mathcal{L}^{p}(X)/N
                \end{equation}
                The analyst Halmos said that the only important
                values of $p$ are 1, 2, and $\infty$. If
                $f:X\rightarrow\mathcal{C}$ is
                measurable, then we define:
                \begin{equation}
                    \norm{f}_{\infty}=
                    \inf\{c\geq{0}:\mu\Big(\{x:|f(x)|>c\}\big)=0\}
                \end{equation}
                With the convention that $\inf\{\emptyset\}=\infty$.
                This defines a semi-norm on:
                \begin{equation}
                    \mathcal{L}^{\infty}(X)
                    =\{f:\norm{f}_{\infty}<\infty\}
                \end{equation}
                Homogeneity pops out rather quickly, but the
                triangle-inequality is still tricky. We call
                $\norm{\cdot}_{\infty}$ the essential supremum
                of $f$.
            \end{lexample}
            \begin{theorem}
                If $f:X\rightarrow\mathbb{C}$ is measurable, and if:
                \begin{equation}
                    E=\{p:\norm{f}_{p}<\infty,p\in[1,\infty)\}
                \end{equation}
                Then $E$ is connected.
            \end{theorem}
            \begin{lexample}
                Let $X$ be a finite set, let
                $\mathcal{M}=\mathcal{P}(X)$, and let $\mu$ be the
                counting measure on $X$. A function on
                $X$ is an n-tuple $x=(x_{1},\dots,x_{n})$. Then:
                \begin{equation}
                    \norm{x}_{p}=
                    \begin{cases}
                        \Big(\sum_{k=1}^{n}|x_{k}|^{p}\Big)^{1/p},
                        &1\leq{p}<\infty\\
                        \max\{|x|,x\in{X}\}
                    \end{cases}
                \end{equation}
                $\norm{\cdot}_{p}$ is a norm on $X$.
            \end{lexample}
            \begin{lexample}
                Let $X=\mathbb{N}$, the set of natural numbers. Let
                $\mathcal{M}=\mathcal{P}(X)$, and let $\mu$ be
                the counting measure. Then functions are sequences
                $a:\mathbb{N}\rightarrow\mathbb{R}$, or
                $a:\mathbb{N}\rightarrow\mathbb{C}$. Then:
                \begin{equation}
                    \norm{x}_{p}=
                    \begin{cases}
                        \Big(\sum_{n=1}^{\infty}
                            |x_{n}|^{p}\Big)^{1/p},
                        &1\leq{p}<\infty\\
                        \max\{|x|,x\in{X}\}
                    \end{cases}
                \end{equation}
                This defines a norm. Recall that a series is
                absolutely convergent if $\sum|a_{n}|<\infty$.
                Given an absolutely convergent series, the original
                series is also convergent. For this space we use
                the following notation:
                \begin{equation}
                    \ell^{p}=
                    \{a:\mathbb{R}\rightarrow\mathbb{R}:
                        \sum_{n=1}^{\infty}|a_{n}|^{p}<\infty\}
                \end{equation}
            \end{lexample}
            In general, if $X$ is a set then we can equip $X$
            with the counting measure and then if $f$ is any
            bounded function on $f$, then:
            \begin{equation}
                \norm{f}_{\infty}=\sup\{|f(x):x\in{X}\}
            \end{equation}
            For the space of sequences, we write $\ell^{\infty}(X)$.
            \begin{lexample}
                If $X$ is any set, then we define the following
                metric:
                \begin{equation}
                    \rho(x,y)=
                    \begin{cases}
                        1,&x\ne{y}\\
                        0,&x=1
                    \end{cases}
                \end{equation}
                This is often called the discrete metric. It is
                indeed a metric, and $(X,\rho)$ is a metric space.
            \end{lexample}
            \begin{lexample}
                Let $(X,\rho)$ be a metric space. If $Y\subseteq{X}$
                is a non-empty subset of $X$, we can define a new
                metric on $Y$
                be restricting $\rho$ to $Y\times{Y}$. We call
                this the metric subspace.
            \end{lexample}
            \begin{ldefinition}{Strongly Equivalent Metrics}
                  {Funct_Analysis_Strongly_Equivalent_Metrics}
                Strongly equilalent metrics on a set $X$ are metrics
                $\rho_{1}$ and $\rho_{2}$ such that there
                exists $c,d\in\mathbb{R}^{+}$ such that, for
                all $x,y\in{X}$:
                \begin{equation}
                    c\rho_{1}(x,y)\leq
                        \rho_{2}(x,y)\leq{d}\rho_{1}(x,y)
                \end{equation}
            \end{ldefinition}
            The definition of strongly equivalent metrics is indeed
            symmetric. For since $c,d\in\mathbb{R}^{+}$,
            $c^{-1}$ and $d^{-1}$ are well defined and positive,
            and thus:
            \begin{equation}
                \frac{1}{d}\rho_{2}(x,y)\leq\rho_{1}(x,y)
                \leq\frac{1}{c}\rho_{1}(x,y)
            \end{equation}
            \begin{theorem}
                If $p,q\in[1,\infty)$, then $\norm{\cdot}_{p}$ and
                $\norm{\cdot}_{q}$ are strongly equivalent.
            \end{theorem}
            \begin{proof}
                It suffices to show that for all
                $p\in[1,\infty)$ there exist
                $c,d\in\mathbb{R}^{+}$ such that:
                \begin{equation}
                    c\norm{x}_{p}\leq\norm{x}_{2}\leq{d}\norm{x}_{p}
                \end{equation}
                Also note that:
                \begin{equation}
                    \partial{\overline{B}_{1}(\mathbf{0})}=
                    \{\mathbf{x}\in\mathbb{R}^{n}:\norm{x}_{2}=1\}
                \end{equation}
                Is a closed and bounded subset of $\mathbb{R}^{n}$.
            \end{proof}
            \begin{lexample}
                Let $(X,\rho)$ be a metric space on $X$ and define:
                \begin{equation}
                    d(x,y)=\frac{\rho(x,y)}{1+\rho(x,y)}
                \end{equation}
                Then $(X,d)$ is a metric space. Definiteness
                and symmetry come rather immediately from the
                definition and the fact that $\rho$ is a metric.
                The only thing to check is the
                triangle inequality.
            \end{lexample}
            \begin{ldefinition}{Open Ball in a Metric Space}
                  {Funct_Analysis_Open_Ball}
                The open ball about a point $x$ in a metric space
                $(X,\rho)$ of radius $r\in\mathbb{R}$ is the set:
                \begin{equation}
                    B_{r}(x)=\{y\in{X}:\rho(x,y)<r\}
                \end{equation}
            \end{ldefinition}
            \begin{ldefinition}{Open Subsets of a Metric Space}
                  {Funct_Analysis_Open_in_Metric_Space}
                An open subset of a metric space $(X,\rho)$ is a set
                $\mathcal{U}\subseteq{X}$ such that for all
                $x\in\mathcal{U}$ there is an $\varepsilon>0$
                such that:
                \begin{equation}
                    B_{\varepsilon}(x)\subseteq\mathcal{U}
                \end{equation}
            \end{ldefinition}
            \begin{theorem}
                Open balls are open.
            \end{theorem}
            \begin{ldefinition}{Neighborhoods}
                  {Funct_Analysis_Neighborhoods}
                A neighborhood of a point $x$ in a metric space
                $(X,\rho)$ is a subset $D\subseteq{X}$ such
                that there is an open subset
                $\mathcal{U}\subseteq{D}$ such that
                $x\in\mathcal{U}$.
            \end{ldefinition}
            \begin{theorem}
                IF $(X,\rho)$ is a metric space, then $X$ is
                an open subset.
            \end{theorem}
            \begin{theorem}
                If $(X,\rho)$ is a metric space, then
                $\emptyset$ is an open subset.
            \end{theorem}
            \begin{theorem}
                If $\mathcal{U}_{I}$ is a collection of open subsets
                of a metric space $(X,\rho)$, and if
                $\mathcal{U}$ is defined by:
                \begin{equation}
                    \mathcal{U}=\bigcup_{i\in{I}}\mathcal{U}_{i}
                \end{equation}
                Then $\mathcal{U}$ is an open subset.
            \end{theorem}
            \begin{theorem}
                If $(X,\rho)$ is a metric space, and if
                $\mathcal{U}$ and $\mathcal{V}$ are open
                subsets, then the set $\mathcal{D}$ defined by:
                \begin{equation}
                    \mathcal{D}=\mathcal{U}\cap\mathcal{V}
                \end{equation}
                Is an open subset of $X$.
            \end{theorem}
            \begin{theorem}
                If $(X,\rho)$ is a metric space and
                $(\mathcal{E},\rho_{\mathcal{E}})$ is a subspace
                of $(X,\rho)$, then a set
                $\mathcal{U}\subseteq\mathcal{E}$
                is open in $\mathcal{E}$ if and only if there is
                an open set $\mathcal{U}$ in $X$ such that:
                \begin{equation}
                    \mathcal{V}=\mathcal{U}\cap\mathcal{E}
                \end{equation}
            \end{theorem}
            \begin{ldefinition}{Topology}
                  {Funct_Analysis_Topology}
                A topology on a set $X$ is a subset
                $\tau\subseteq\mathcal{P}(X)$ such that:
                $\emptyset\in\tau$ and $X\in\tau$, for any finite
                subset $\mathcal{C}\subseteq\tau$, it is true that:
                \begin{subequations}
                    \begin{equation}
                        \bigcap_{C\in\mathcal{C}}C\in\tau
                    \end{equation}
                    And for any subset $\mathcal{O}\subseteq\tau$
                    it is true that:
                    \begin{equation}
                        \bigcup_{\mathcal{U}\in\mathcal{O}}
                        \mathcal{U}\in\tau
                    \end{equation}
                \end{subequations}
                That is, $\tau$ is closed to finite intersections
                and arbitrary unions.
            \end{ldefinition}
            \begin{lexample}
                If $(X,\rho)$ is a metric space, then the set:
                \begin{equation}
                    \tau=\{\mathcal{U}\subseteq{X}:
                        \mathcal{U}\textrm{ is open}\}
                \end{equation}
                is a topology on $X$. This is the \textit{metric}
                topology. Not every topological space can be
                formed from a metric space. If
                $(\mathcal{E},\rho_{\mathcal{E}})$ is a subspace
                of $(X,\rho)$, then the \textit{subspace topology},
                or the relative topology, is the set:
                \begin{equation}
                    \tau_{\mathcal{E}}
                    =\{\mathcal{E}\cap\mathcal{U}:
                        \mathcal{U}\in\tau\}
                \end{equation}
                This is a topology on $\mathcal{E}$.
            \end{lexample}
            \begin{ldefinition}{Closed Subsets}
                  {Funct_Analysis_Closed_in_Metric}
                A closed subset of a metric space $(X,\rho)$
                is a set $\mathcal{C}\subseteq{X}$ such that
                $X\setminus\mathcal{C}$ is an open subset of $X$.
            \end{ldefinition}
            \begin{theorem}
                If $(X,\rho)$ is a metric space, then $X$ is closed.
            \end{theorem}
            \begin{theorem}
                If $(X,\rho)$ is a metric space, then
                $\emptyset$ is closed.
            \end{theorem}
            \begin{theorem}
                If $(X,\rho)$ is a metric space, and if
                $\mathcal{C}_{i}$ is a finite collection of
                closed subsets, then the set
                $\mathcal{C}$ defined by:
                \begin{equation}
                    \mathcal{C}=\cup_{i=1}^{n}\mathcal{C}_{i}
                \end{equation}
                is a closed subset of $X$.
            \end{theorem}
            \begin{ldefinition}{Closure of a Set}
                  {Funct_Analysis_Closure_in_Metric}
                If $(X,\rho)$ is a metric space and
                $\mathcal{E}\subseteq{X}$, then the closure
                of $\mathcal{E}$ is the set:
                \begin{equation}
                    \overline{\mathcal{E}}=
                    \bigcap\{\mathcal{F}\subseteq{X}:
                        \mathcal{E}\subseteq\mathcal{F}
                        \land\mathcal{F}\textrm{ is closed.}\}
                \end{equation}
            \end{ldefinition}
            The closure of a set $\mathcal{E}$ is the
            smallest closed set that contains $\mathcal{E}$.
            \begin{theorem}
                If $(X,\rho)$ is a metric space, and if
                $\mathcal{E}$ is a non-empty subset of $X$, then
                $x\in\overline{\mathcal{E}}$ if and only if for
                all $\varepsilon>0$:
                \begin{equation}
                    B_{\varepsilon}(x)\cap\mathcal{E}\ne\emptyset
                \end{equation}
            \end{theorem}
            \begin{ldefinition}
                  {Convergent Sequences in a Metric Space}
                  {Funct_Analysis_Convergent_Seq_in_Metric}
                A convergent sequence in a metric space
                $(X,\rho)$ is a sequence
                $a:\mathbb{N}\rightarrow{X}$ such that there is
                an $x\in{X}$ such that for all $\varepsilon>0$
                there exists an $N\in\mathbb{N}$ such that, for
                all $n\in\mathbb{N}$ and $n>N$, it is true that
                $d(x,a_{n})<\varepsilon$. We write
                $a_{n}\rightarrow{x}$.
            \end{ldefinition}
            \begin{ldefinition}{Limits of Convergent Sequences}
                  {Funct_Analysis_Limit_of_Conv_Seq_in_Metric}
                A limit of a convergent sequence
                $a:\mathbb{N}\rightarrow{X}$ in a metric space
                $(X,\rho)$ is a point $x\in{X}$ such that
                $a_{n}\rightarrow{x}$.
            \end{ldefinition}
            \begin{theorem}
                If $(X,\rho)$ is a metric space, if
                $a:\mathbb{N}\rightarrow{X}$ is a convergent
                sequence, and if $x$ and $y$ are limits of $a$,
                then $x=y$.
            \end{theorem}
            \begin{ldefinition}{Equivalent Metrics}
                  {Funct_Analysis_Equivalent_Metric}
                Equivalent metrics on a set $X$ and metrics
                $\rho$ and $d$ such that they generate the
                same topology.
            \end{ldefinition}
            \begin{theorem}
                If $X$ is a set and if $\rho$ and $d$ are
                equivalent metrics on $X$, then for all $x\in{X}$
                and for all $r\in\mathbb{R}^{+}$ there exists
                $r_{1},r_{2}\in\mathbb{R}^{+}$ such that:
                \begin{align}
                    B_{r_{1}}^{\rho}(x)&\subseteq{B}_{r}^{d}(x)\\
                    B_{r_{2}}^{d}(x)&\subseteq{B}_{r}^{\rho}(x)
                \end{align}
            \end{theorem}
            \begin{lexample}
                If $(X,\rho)$ is a metric space and if $d$
                is defined by:
                \begin{equation}
                    d(x,y)=\frac{\rho(x,y)}{1+\rho(x,y)}
                \end{equation}
                The $d$ is a metric on $X$. Moreover, $d$
                is equivalent to $\rho$. This shows that the
                notion of boundedness is not a topological one,
                but a metric property. For, given any metric
                $\rho$, $d$ is bounded. For all $x,y\in{X}$,
                $0\leq{d}(x,y)<1$. Letting $\rho$ be the
                standard metric on $\mathbb{R}$,
                $\rho(x,y)=|x-y|$, we see that the topology
                generated by this unbounded metric is equivalent
                to the topology generated by the metric:
                \begin{equation}
                    d(x,y)=\frac{|x-y|}{1+|x-y|}
                \end{equation}
            \end{lexample}
            \begin{theorem}
                If $(X,\rho)$ is a metric space, and if
                $d:X^{2}\rightarrow[0,\infty)$ is defined by:
                \begin{equation}
                    d(x,y)=\frac{\rho(x,y)}{1+\rho(x,y)}
                \end{equation}
                Then $\rho$ and $d$ are equivalent.
            \end{theorem}
            \begin{proof}
                For let $r>0$. For all $x,y\in{X}$,
                $d(x,y)\leq\rho(x,y)$, and therefore:
                \begin{equation}
                    B_{r}^{\rho}(x)\subseteq{B}_{r}^{d}(x)
                \end{equation}
                If $\rho(x,y)\leq{1}$, then
                $\rho(x,y)\leq{2}d(x,y)$. Let
                $r_{1}=\min\{r/2,1\}$, then:
                \begin{equation}
                    B_{r/2}^{d}(x)\subseteq{B}_{r}^{\rho}(x)
                \end{equation}
                Therefore, etc.
            \end{proof}
            \begin{ldefinition}
                  {Continuous Functions Between Metric Spaces}
                  {Funct_Analysis_Cont_Func_Metric}
                A continuous function from a metric space
                $(X,\rho)$ to a metric space $(Y,d)$ is a function
                $f:X\rightarrow{Y}$ such that, for all $x\in{X}$
                and for all $\varepsilon>0$, there is a
                $\delta>0$ such that:
                \begin{equation}
                    f\big(B_{\delta}^{\rho}(x)\big)\subset
                    B_{\varepsilon}^{d}\big(f(x)\big)
                \end{equation}
            \end{ldefinition}
            \begin{theorem}
                If $(X,\rho)$ and $(Y,d)$ are metric spaces, and
                if $f:X\rightarrow{Y}$ is a function, then
                the following are equivalent:
                \begin{enumerate}
                    \item $f$ is continuous at $x_{0}\in{X}$.
                    \item If $x_{n}\rightarrow{x_{0}}$ then
                          $f(x_{n})\rightarrow{f}(x_{0})$
                    \item If $\mathcal{V}$ is a neighborhood of $f(x_{0})$,
                          then $f^{-1}(\mathcal{V})$ is a neighborhodd of
                          $x_{0}$.
                \end{enumerate}
            \end{theorem}
            \begin{theorem}
                If $(X,\rho)$ and $(Y,d)$ are metric spaces, and if
                $f:X\rightarrow{Y}$ is a function that is
                continuous at $x_{0}\in{X}$, then for
                all $\mathcal{V}\subseteq{Y}$ such that
                $\mathcal{V}$ is open and $f(x_{0})\in\mathcal{V}$,
                then $f^{-1}(\mathcal{V})$ is an open subset
                of $x_{0}$.
            \end{theorem}
            \begin{ldefinition}{Uniformly Continuous Functions}
                  {Funct_Analysis_Uni_Cont_Func_Metric}
                A uniformly continuous function from a metric space
                $(X,\rho)$ to a metric space $(Y,d)$ if a function
                $f:X\rightarrow{Y}$ such that for all
                $\varepsilon>0$ there exists a $\delta>0$ such
                that, for all $x,y\in{X}$ such that
                $\rho(x,y)<\delta$, it is true that
                $d(f(x),f(y))<\varepsilon$.
            \end{ldefinition}
            \begin{theorem}
                If $f:X\rightarrow{Y}$ is uniformly continuous,
                then $f$ is continuous.
            \end{theorem}
            The converse is false. For define $f(x)=x^{2}$.
        \subsection{Completeness}
            \begin{ldefinition}{Cauchy Sequences}
                  {Funct_Analysis_Cauchy_Seq_Metric}
                A Cauchy sequence in a metric space $(X,d)$ is a
                sequence $a:\mathbb{N}\rightarrow{X}$ such that,
                for all $\varepsilon>0$ there is an
                $N\in\mathbb{N}$ such that, for all
                $n,m\in\mathbb{N}$ such that $n,m>N$,
                it is true that $d(x_{n},x_{m})<\varepsilon$.
            \end{ldefinition}
            \begin{lexample}
                Let $X=(0,2)$ with the usual metric, and let
                $a:\mathbb{N}\rightarrow{X}$ be defined by:
                \begin{equation}
                    a_{n}=\frac{1}{n}
                \end{equation}
                Then $a$ is a Cauchy sequence since:
                \begin{equation}
                    |a_{n}-a_{m}|=\frac{|n-m|}{nm}
                    <\frac{2}{\min(n,m)}
                \end{equation}
                And this converges to zero. However the sequence
                doesn't converge, since we took zero away.
            \end{lexample}
            \begin{lexample}
                Let $X=C([0,3])$ and let:
                \begin{equation}
                    \norm{f}_{1}=\int_{0}^{3}|f(x)|\diff{x}
                \end{equation}
                Then $\norm{\cdot}_{1}$ is a norm on the set of
                continuous functions, and thus induces a metric.
                Let $f_{n}$ be defined by:
                \begin{equation}
                    f_{n}(x)=
                    \begin{cases}
                        1,&x\leq{x}<2-\frac{1}{n}\\
                        Bob,\\
                        0,&x\geq{2}
                    \end{cases}
                \end{equation}
                Then $f_{n}$ is Cauchy, but does not converge.
            \end{lexample}
            \begin{ldefinition}{Complete Metric Spaces}
                  {Funct_Analysis_Complete_Metric_Space}
                A complete metric space is a metric space
                $(X,d)$ such that, for all Cauchy sequences
                $a:\mathbb{N}\rightarrow{X}$, $a$ is a convergent
                sequence.
            \end{ldefinition}
            \begin{theorem}
                If $(X,d)$ is a metric space, if
                $a:\mathbb{N}\rightarrow{X}$ is a Cauchy
                sequence, and if there is a convergent subseqence
                of $a$, then $a$ is a convergent sequence.
            \end{theorem}
            \begin{theorem}
                A normed vector space $(V,\norm{\cdot})$ is
                complete if and only if every absolutely
                convergent series converges.
            \end{theorem}
            \begin{proof}
                Suppose $V$ is complete and let $u_{n}$
                be absolutely convergent. That is, the sequence
                of partial sums:
                \begin{equation}
                    S_{N}=\sum_{n=1}^{N}\norm{u_{n}}
                \end{equation}
                Converges in $\mathbb{R}$. But then:
                \begin{equation}
                    \underset{N\rightarrow\infty}{\lim}
                    \sum_{k=N}^{\infty}\norm{u_{k}}=0
                \end{equation}
                Define:
                \begin{equation}
                    s_{n}=\sum_{k=1}^{n}u_{k}
                \end{equation}
                But if $m\geq{n}$, then:
                \begin{equation}
                    \norm{s_{n}-s_{m}}
                    \leq\sum_{k=n+1}^{m}\norm{u_{k}}
                    \leq\sum_{k=n+1}^{\infty}\norm{u_{k}}
                \end{equation}
                Thus, if $\varepsilon>0$ there is an $N\in\mathbb{N}$
                such that, for all $n\geq{N}$, it is true that:
                \begin{equation}
                    \sum_{k=n+1}^{\infty}\norm{u_{k}}<\varepsilon
                \end{equation}
                Therefore $s_{n}$ is a Cauchy sequence, and therefore
                there is an $s\in{V}$ such that $s_{n}\rightarrow{s}$.
                Proving the converse, suppose $u_{n}$ is a Cauchy
                sequence. Then there is an $N_{1}\in\mathbb{N}$ such
                that, for all $n,m\geq{N}_{1}$, we have
                $\norm{u_{n}-u_{m}}<1/2$. But then there is also an
                $N_{2}\in\mathbb{N}$ such that $N_{2}>N_{1}$, and for
                all $n,m>N_{2}$, $\norm{u_{n}-u_{m}}$. Continuing
                inductively, we find a sequence $u_{n_{k}}$ such that:
                \begin{equation}
                    \norm{u_{n_{k+1}}-u_{n_{k}}}<\frac{1}{2^{k}}
                \end{equation}
                Let $v_{k}=u_{n_{k+1}}-u_{n_{k}}$, and note that:
                \begin{equation}
                    \sum_{n=1}^{\infty}\norm{v_{n}}<\infty
                \end{equation}
                But then there is a $v\in{V}$ such that:
                \begin{equation}
                    \sum_{n=1}^{\infty}v_{n}=v
                \end{equation}
                But:
                \begin{align}
                    v&=\underset{N\rightarrow\infty}{\lim}
                        \sum_{k=1}^{N}v_{k}\\
                    &=\underset{N\rightarrow\infty}{\lim}
                        v_{n_{N+1}}-u_{n_{1}}
                \end{align}
                Therefore $u_{n_{k}}\rightarrow{v}+u_{n_{1}}$.
                But $u_{n}$ is Cauchy and thus if there is a
                convergent subsequence, then it is a convergent
                sequence. Therefore, $(X,\norm{\cdot})$ is complete.
            \end{proof}
            \begin{theorem}
                If $(X,\mathcal{M},\mu)$ is a measure space, and
                if $1\leq{p}\leq\infty$.
            \end{theorem}
            \begin{proof}
                Suppose that $f_{n}$ is a sequence of functions
                in $L^{P}(X,\mathcal{M},\mu)$ such that:
                \begin{equation}
                    \sum_{k=1}^{\infty}\norm{f_{n}}_{p}=B<\infty
                \end{equation}
                Define $G,G_{n}:X\rightarrow[0,\infty]$ be
                defined by:
                \begin{align}
                    G(x)&=\sum_{k=1}^{\infty}|f_{k}(x)|\\
                    G_{n}(x)&=\sum_{k=1}^{n}|f_{k}(x)|
                \end{align}
                Then, from the triangle inequality, we have that:
                \begin{equation}
                    \norm{G_{n}}_{p}\leq
                    \sum_{k=1}^{n}\norm{f_{k}}_{p}
                    \leq{B}
                \end{equation}
                Thus, by the monotone convergence theorem, we have:
                \begin{equation}
                    \int_{X}G(x)^{p}\diff{\mu}
                    =\underset{n\rightarrow\infty}{\lim}
                    \int_{X}G_{n}(x)^{p}\diff{\mu}\leq{B}^{p}
                \end{equation}
                Therefore $G\in\mathcal{L}^{p}(X)$, and thus
                $G(x)<\infty$ $\mu$ almost-everywhere. But then the
                original series converges $\mu$ almost everywhere.
                Define $F$ be:
                \begin{equation}
                    F(x)=
                    \begin{cases}
                        \sum_{n=1}^{\infty}f_{n}(x),
                        &|\sum_{n=1}^{\infty}f_{n}(x)|<\infty\\
                        0,&\textrm{Otherwise}
                    \end{cases}
                \end{equation}
                Then $|F(x)|\leq{G}(x)$, and thus
                $F\in\mathcal{L}^{p}(X)$. Moreover:
                \begin{equation}
                    \big|F(x)-\sum_{k=1}^{n}f_{k}(x)\big|^{p}
                    \leq{2}^{p}G(x)^{p}
                \end{equation}
                Therefore, by the Lebesgue Dominated Convergence
                Theorem, we have that:
                \begin{equation}
                    \norm{F-\sum_{k=1}^{n}f_{k}(x)}^{p}_{p}
                    \rightarrow{0}
                \end{equation}
                Therefore, $F\in{L}^{p}(X,\mathcal{M},\mu)$.
            \end{proof}
            \begin{ldefinition}
                  {Supremum Norm of Bounded Continuous Function}
                  {Funct_Analysis_Sup_Norm_of_Bound_Cont_Funcs}
                The supremum norm on set $C_{b}(X)$ of bounded
                continuous functions on a metric space $(X,d)$ is:
                \begin{equation}
                    \norm{f}_{\infty}=\sup_{x\in{X}}|f(x)|
                \end{equation}
            \end{ldefinition}
            From this, we can see thatn $f_{n}\rightarrow{f}$ if
            and only if $f_{n}\rightarrow{f}$ uniformly on $X$.
            \begin{theorem}
                $C_{b}(X)$ is complete in the supremum norm.
            \end{theorem}
            \begin{proof}
                Suppose that $f_{n}$ is a Cauchy sequence in
                $C_{b}(X)$. For for all $f_{n}$ and $x\in{X}$,
                $f_{n}(x)$ is a Cauchy sequence in $\mathbb{C}$.
                But $\mathbb{C}$ is complete, and thus there is a
                $c_{x}\in\mathbb{C}$ such that
                $f_{n}(x)\rightarrow{c}_{x}$. Let
                $f(x)=c_{x}$ for all $x\in{X}$. Then
                $f_{n}\rightarrow{f}$. For, let $\varepsilon>0$.
                Then there exists $N\in\mathbb{N}$ sch that, for
                all $n,m>{N}$ implies that:
                \begin{equation}
                    |f_{n}(x)-f_{m}(x)|<\varepsilon/2
                \end{equation}
                But then:
                \begin{equation}
                    |f_{n}(x)-f(x)|=
                    \underset{m\rightarrow\infty}{\lim}
                    |f_{n}(x)-f_{m}(x)|
                    \leq\frac{\varepsilon}{2}<\varepsilon
                \end{equation}
                But the uniform limit of continuous functions is
                continuous. Therefore, etc.
            \end{proof}
            \begin{theorem}
                If $(X,d)$ is a complete metric space, if
                $(E,d')$ is a subspace of $(X,d)$, and if
                $E$ is closed, then $(E,d')$ is complete.
            \end{theorem}
            \begin{proof}
                Suppose $E$ is closed and suppose $(x_{n})$ is a
                Cauchy sequence in $E$. Then $x_{n}$ is a Cauchy
                sequence in $X$, but $X$ is complete. Therefore
                there is an $x\in{X}$ such that
                $x_{n}\rightarrow{x}$. But $E$ is closed,
                and therefore $x\in{E}$. Now suppose $E$ is
                complete. Suppose $x_{n}$ is a sequence in $E$
                and that $x_{n}\rightarrow{y}$ in $X$. But
                convergent sequences are Cauchy sequences, and
                thus $x_{n}$ is a Cauchy sequence. But $E$ is
                complete and therefore $y\in{E}$.
                Therefore, $E$ is closed.
            \end{proof}
            \begin{ldefinition}{Bounded Metric Spaces}
                  {Funct_Analysis_Bounded_Metric_Space}
                A bounded metric space is a metric space
                $(X,d)$ such that there exists an $x\in{X}$
                and an $r>0$ such that:
                \begin{equation}
                    X\subseteq{B}_{r}^{(X,d)}(x)
                \end{equation}
            \end{ldefinition}
            \begin{ldefinition}{Diameter of a Metric Space}
                  {Funct_Analysis_Diam_of_Metric_Space}
                The diameter of a bounded metric space $(X,d)$ is:
                \begin{equation}
                    \diam(X)=\sup_{x\in{X}}\{d(x,y):x,y\in{X}\}
                \end{equation}
            \end{ldefinition}
            Every bounded metric space is contained in some
            open ball.
            \begin{theorem}
                If $(X,d)$ is a metric space, and then it
                is complete if and only if for any sequence
                of non-empty closed sets
                $F:\mathbb{N}\rightarrow\mathcal{P}(X)$ such that
                $F_{n+1}\subseteq{F}_{n}$ and
                $\diam(F_{n})\rightarrow{0}$,
                there is an $x\in{X}$ such that:
                \begin{equation}
                    \{x\}=\cap_{n=1}^{\infty}F_{n}
                \end{equation}
            \end{theorem}
            \begin{proof}
                For suppose $(X,d)$ is complete, and let
                $F:\mathbb{N}\rightarrow\mathcal{P}(X)$ be a
                sequence of non-empty subsets of $X$. Then, for
                all $n\in\mathbb{N}$, $F_{n}$ is non-empty, and
                thus there is a sequence
                $a:\mathbb{N}\rightarrow{X}$ such that, for all
                $n\in\mathbb{N}$, $x_{n}\in{F_{n}}$. But then:
                \begin{equation}
                    d(a_{n},a_{m})\leq\diam(F_{\max\{n,m\}})
                \end{equation}
                But $\diam(F_{n})\rightarrow{0}$, and therefore
                $a$ is a Cauchy sequence. But $(X,d)$ is complete,
                and therefore there is an $x\in{X}$ such that
                $a_{n}\rightarrow{x}$. Moreover, there is an
                $N\in\mathbb{N}$ such that $x\in\overline{F}_{N}$.
                But $F_{N}$ is closed, and thus $x\in{F}_{N}$.
                But for all $n>N$, $F_{n}\subseteq{F}_{N}$.
                Therefore:
                \begin{equation}
                    x\in\cap_{n=1}^{\infty}F_{n}
                \end{equation}
                If $y\in\cap_{n=1}^{\infty}F_{n}$, then
                $d(x,y)\leq\diam(F_{n})$ for all $n\in\mathbb{N}$.
                But $\diam(F_{n})\rightarrow{0}$, and thus
                $d(x,y)=0$. Therefore, $x=y$. Going the other
                way, suppose $X$ has the nested set property and
                let $a:\mathbb{N}\rightarrow{X}$ be a Cauchy
                sequence in $X$. Let
                $F:\mathbb{N}\rightarrow\mathcal{P}(X)$
                be defined by:
                \begin{equation}
                    F_{n}=\overline{\{a_{k}:k\geq{n}\}}
                \end{equation}
                Then, for all $n\in\mathbb{N}$, $F_{n}$
                is non-empty, and $F_{n+1}\subseteq{F}_{n}$.
                Moreover, $\diam(F_{n})\rightarrow{0}$. Thus, by
                the nested sequence property, there is an
                $x\in{X}$ such that $x\in\cap_{n=1}^{\infty}F_{n}$.
                But then:
                \begin{equation}
                    d(a_{n},x)\leq\diam(F_{n})\rightarrow{0}
                \end{equation}
                and therefore $a_{n}\rightarrow{x}$. Thus, $a$ is
                a Cauchy sequence and $(X,d)$ is complete.
            \end{proof}
        \subsection{Compactness}
            \begin{ldefinition}{Covers}
                  {Funct_Analysis_Covers}
                A cover of a subset $\mathcal{E}\subseteq{X}$ of
                a set $X$ is a subset
                $\mathcal{O}\subseteq\mathcal{P}(X)$ such that:
                \begin{equation}
                    \mathcal{E}\subseteq
                    \bigcup_{\mathcal{U}\in\mathcal{O}}
                        \mathcal{U}
                \end{equation}
            \end{ldefinition}
            \begin{ldefinition}{Sub-Cover}
                  {Funct_Analysis_Subcover}
                A sub-cover of a cover $\mathcal{O}$ of a subset
                $E\subseteq{X}$ of a set $X$ is a subset
                $\Delta\subseteq\mathcal{O}$ such that:
              \begin{equation}
                    \mathcal{E}\subseteq
                    \bigcup_{\mathcal{U}\in\Delta}
                        \mathcal{U}
                \end{equation}
            \end{ldefinition}
            \begin{ldefinition}{Open Covers}
                  {Funct_Analysis_Open_Cover}
                An open cover of a metric space $(X,d)$ is a cover
                $\mathcal{O}\subseteq\mathcal{P}(X)$ of $X$ such
                that, for all $\mathcal{U}\in\mathcal{O}$,
                $\mathcal{U}$ is open.
            \end{ldefinition}
            \begin{ldefinition}{Compact Sets}
                  {Funct_Analysis_Compact_Set}
                A compact metric space is a metric space $(X,d)$
                such that for any open cover $\mathcal{O}$ of
                $X$, there is a finite
                sub-cover $\Delta\subseteq\mathcal{O}$.
            \end{ldefinition}
            \begin{lexample}
                Let $X=[0,1)$ with the usual topology, and let:
                \begin{equation}
                    \mathcal{U}_{x}=[0,x)
                    \quad\quad
                    x\in(0,1)
                \end{equation}
                Then $\mathcal{O}=\{\mathcal{U}_{x}:x\in(0,1)\}$
                is an open cover of $X$, but there is no finite
                sub-cover. For given any finite sub-cover,
                there is a greatest $x$ such that
                $\mathcal{U}_{x}$ is contained in the sub-cover.
                But then for all $y\in(x,1)$, $y$ is not in
                the sub-cover. As a trivial example, any
                finite metric space is compact.
            \end{lexample}
            \begin{theorem}
                If $K$ is a subspace of $X$, then $K$ is compact
                if and only if every open cover of $K$ has a
                finite sub-cover.
            \end{theorem}
            \begin{proof}
                For suppose $(K,d_{K})$ is compact, and let
                $\mathcal{O}$ be an open cover of $K$. Then:
                \begin{equation}
                    \mathcal{O}_{K}=\{K\cup\mathcal{U}:
                        \mathcal{U}\in\mathcal{O}\}
                \end{equation}
                Is an open cover of $K$. But $K$ is compact,
                and thus there is a finite sub-cover
                $\Delta_{K}$. But then:
                \begin{equation}
                    \Delta=\{\mathcal{U}\in\mathcal{U}:
                             \mathcal{U}\cap{K}\in\Delta_{K}\}
                \end{equation}
                And this is a finite sub-cover.
            \end{proof}
            \begin{ldefinition}{Finite Intersection Property}
                  {Funct_Analysis_Finite_Intersect_Prop}
                A set with the finite intersection property
                in a metric space $(X,d)$ is a collection of sets
                $\mathscr{F}\subseteq\mathcal{P}(X)$ such that,
                for any sequence
                $F:\mathbb{Z}_{n}\rightarrow\mathscr{F}$, 
                it is true that $\cap_{k=1}^{n}F_{k}\ne\emptyset$.
            \end{ldefinition}
            \begin{theorem}
                A metric space $(X,d)$ is compact if and only if
                every collection $\mathscr{F}$ of closed sets in
                $X$ with the
                finite intersection property is such that:
                \begin{equation}
                    \bigcap_{\mathcal{C}\in\mathcal{F}}
                        \mathcal{C}\ne\emptyset
                \end{equation}
            \end{theorem}
            \begin{lexample}
                Let $F_{n}=[n,\infty)$, and let
                $\mathscr{F}=\{F_{n}:n\in\mathbb{N}\}$.
                Then $\mathscr{F}$ has the finite intersection
                property. However, the intersection over the
                entire set is empty, and hence $\mathbb{R}$
                (With the standard metric) is not compact.
            \end{lexample}
            \begin{ldefinition}{Totally Bounded Metric Space}
                  {Funct_Analysis_Totally_Bounded_Met_Space}
                A totally bounded metric space is a metric
                space $(X,d)$ such that, for all $\varepsilon>0$,
                there exists an $n\in\mathbb{N}$ and a sequence
                $a:\mathbb{Z}_{n}\rightarrow{X}$ such that:
                \begin{equation}
                    X=\cup_{k=1}^{n}B_{\varepsilon}^{(X,d)}(a_{k})
                \end{equation}
            \end{ldefinition}
            \begin{ldefinition}{$\varepsilon\textrm{-Nets}$}
                  {Funct_Analysis_epsilon_Net}
                An $\varepsilon\textrm{-Net}$ of a subspace
                $(\mathcal{E},d_{\mathcal{E}})$ of a
                metric space $(X,d)$ is a finite collection:
                \begin{equation}
                    E=\{B_{\varepsilon}^{(X,d)}(x_{k}):
                        k\in\mathbb{Z}_{n}\}
                \end{equation}
                Such that $E$ is an open cover of $\mathcal{E}$.
            \end{ldefinition}
            \begin{lexample}
                Let $X=\ell^{2}$ and let:
                \begin{equation}
                    e_{n}(x)=
                    \begin{cases}
                        1,&k=n\\
                        0,&k\ne{n}
                    \end{cases}
                \end{equation}
                Then $e_{n}\in\ell^{2}$ and $\norm{e}_{2}=1$,
                but for all $n\ne{m}$,
                $\norm{e_{n}-e_{m}}_{2}=\sqrt{2}$. Let:
                \begin{equation}
                    B_{1}=\{f\in\ell^{2}:\norm{f}_{2}\leq{1}\}
                \end{equation}
                Then $B_{1}$ is bounded, but if
                $\varepsilon=\sqrt{2}/2$ then no finite collection
                of $\varepsilon$ balls can cover $B_{1}$ since
                each ball can contain at most one of the
                $e_{n}$. Thus any cover is infinite.
            \end{lexample}
            \begin{theorem}
                A subset of $(\mathbb{R}^{n},\norm{\cdot}_{2})$
                is totally bounded if and only if it's bounded.
            \end{theorem}
            \begin{proof}
                Totally bounded implies bounded, so it suffices
                to show that if $\mathbb{R}^{n}$ is bounded then
                it is totally bounded. Let
                $\mathcal{E}\subseteq\mathbb{R}^{n}$ be bounded.
                Then there is an $r>0$ such that:
                \begin{equation}
                    \mathcal{E}\subseteq[-r,r]^{n}
                \end{equation}
                Then, compactness, stuff like that.
            \end{proof}
            This works for any norm on $\mathbb{R}^{n}$, since all
            norm's on $\mathbb{R}^{n}$ are strongly equivalent.
            \begin{ldefinition}{Sequential Compactness}
                  {Funct_Analysis_Seq_Compact}
                A sequentially compact metric space is a metric
                space $(X,d)$ such that, for all
                $a:\mathbb{N}\rightarrow{X}$, there is a convergent
                subsequence of $a$.
            \end{ldefinition}
            \begin{lexample}
                Let $X\subseteq\mathbb{R}$ be defined by:
                \begin{equation}
                    X=\{\frac{1}{n}:n\in\mathbb{N}\}\cup\{0\}
                \end{equation}
                Then $X$ is sequentially compact, with respect
                to the subspace metric.
            \end{lexample}
            \begin{theorem}
                IF $(X,d)$ is a metric space, then the following are
                equivalent:
                \begin{enumerate}
                    \item $X$ is compact.
                    \item $X$ is complete and totally bounded.
                    \item $X$ is sequentially compact.
                \end{enumerate}
            \end{theorem}
            \begin{proof}
                Suppose $(X,d)$ is not compact, and let
                $\mathcal{U}_{i}$ be an open cover with no finite
                subcover. If $X$ is totally bounded, then there is a
                finite covering of $1/2$ balls. But then at least
                one of these isn't covered by finitely many of the
                $\mathcal{U}_{i}$. Let $F_{1}$ be the closure of
                this. Then $F_{1}$ is totally bounded, and has
                a finite covering of $1/4$ balls. One of these
                must not be covered by finitely many of the
                $\mathcal{U}_{i}$. Let $F_{2}'$ be the closure
                of such a ball, and let $F_{2}=F_{1}\cap{F}_{2}'$.
                Then $F_{2}$ is closed, non-empty, and
                $\diam(F_{2})\leq{1/2}$. Continuing, we
                obtain a sequence of non-empty closed sets
                $F_{n}$ such that for all $n\in\mathbb{N}$,
                $F_{n+1}\subseteq{F}_{n}$ and
                $\diam(F_{n})<1/2^{n}$. Thus, if $X$ is complete,
                there is a unique $x$ that lies in the intersection
                of all of the $F_{n}$. But then there is a
                $\mathcal{U}_{i}$ such that $x\in\mathcal{U}_{i}$,
                and thus eventually $F_{n}\subset\mathcal{U}_{i}$,
                a contradiction. Thus, $X$ is compact. Now, suppose
                $X$ is compact and let $a:\mathbb{N}\rightarrow{X}$
                be a sequence in $X$. Let:
                \begin{equation}
                    F_{n}=\overline{\{x_{k}:l\geq{n}\}}
                    \quad\quad
                    \mathscr{F}=\{F_{n}:n\in\mathbb{N}\}
                \end{equation}
                Then $\mathscr{F}$ has the finite intersection
                property. Since $X$ is compact, the intersection of
                the $F_{n}$ is non-empty. Let $x$ be contained in
                the intersection. Then:
                \begin{equation}
                    B_{1}(x)\cap\{x_{k}:k\geq{1}\}\ne\emptyset
                \end{equation}
                Pick $n_{1}$ such that $x_{n_{1}}\in{B}_{1}(x)$.
                Then there is an $n_{2}>n_{1}$ such that
                $x_{n_{2}}\in{B}_{1/2}(x)$. Continuing, we obtain
                a subsequence $n_{k}$ such that
                $x_{k}\in{B}_{1/k}(x)$, and thus
                $x_{k}\rightarrow{x}$. Finally, we show that
                sequential compactness implies that $X$ is
                complete and totally bounded. For suppose $X$ is
                not totally bounded. Then there exists
                $\varepsilon>0$ such that $X$ has no finite
                covering of $\varepsilon$ balls. We can thus
                obtain a sequence $a:\mathbb{N}\rightarrow{X}$
                such that, for all $n\ne{m}$,
                $d(a_{n},a_{m})\geq\varepsilon$. But this
                has no convergence subsequence, for any convergent
                subsequence would be a Cauchy sequence. Moreover,
                $X$ is complete. For suppose not, and let
                $a:\mathbb{N}\rightarrow{X}$ be a Cauchy sequence
                and suppose it does not converge. But then there
                is no convergent subsequence, since Cauchy
                sequences with convergent subsequences converge.
                Thus, $X$ is complete.
            \end{proof}
            \begin{ltheorem}{Heine-Borel Theorem}
                  {Funct_Analysis_Heine_Borel}
                A subset $\mathcal{E}\subseteq\mathbb{R}^{n}$ is
                compact with respect to the standard topology if and
                only if $\mathcal{E}$ is closed and bounded.
            \end{ltheorem}
            This theorem does not generalize to other spaces. For
            consider $\ell^{2}$ and the closed unit ball about the
            origin. This is closed and bounded, but it is not
            compact. This is simply because it is not totally
            bounded, nor is it sequentially compact.
            \begin{ltheorem}{Extreme Value Theorem}
                  {Funct_Analysis_Extreme_Value_Theorem}
                If $(X,d)$ is a compact metric space and if
                $f:X\rightarrow\mathbb{R}$ is continuous, then
                $f$ attains it's maximum and minimum. In particular,
                if $f:X\rightarrow\mathbb{C}$ is continuous, then
                $f$ is bounded.
            \end{ltheorem}
            \begin{proof}
                Note that if $f:X\rightarrow\mathbb{C}$ is
                continuous, then $|f|:X\rightarrow\mathbb{R}$ is
                continuous, so we only need to prove the first
                statement. For if $X$ is compact, then $f(X)$ is
                compact, for $f$ is continuous. But then $f(X)$
                is closed and bounded. Let:
                \begin{equation}
                    M=\underset{x\in{X}}\sup\{f(x)\}
                \end{equation}
                Then, since $f(X)$ is bounded, $M\in\mathbb{R}$.
                But then there is a sequence
                $a:\mathbb{N}\rightarrow{X}$ such that
                $f(a_{n})\rightarrow{M}$. But if $X$ is
                compact, then it is sequentially compact, and
                thus there is an $x\in{X}$ an a subsequence
                $a_{k}$ such that $a_{k_{n}}\rightarrow{x}$.
                But then $f(x)=M$. Similarly for the minimum value.
            \end{proof}
        \subsection{Lebesgue Spaces}
            \begin{ldefinition}{Lebesgue Number}
                  {Funct_Analysis_Lebesgue_Number}
                A Lebesgue Number of an open cover
                $\mathcal{O}$ of a metric space $(X,d)$ is
                a non-zero number $d>0$ such that, for all
                $x\in{X}$, there exists
                a $\mathcal{U}\in\mathcal{O}$ such that:
                \begin{equation}
                    B_{d}^{(X,d)}(x)\subseteq\mathcal{U}
                \end{equation}
            \end{ldefinition}
            \begin{lexample}
                Let $X=\mathbb{R}$, and let $d$ be the
                standard metric. Let
                $\mathcal{O}=\{\mathcal{U}_{i}:i=1,2,3\}$ where:
                \begin{equation}
                    \mathcal{U}_{1}=(-\infty,1)
                    \quad\quad
                    \mathcal{U}_{2}=(0,2)
                    \quad\quad
                    \mathcal{U}_{3}=(1,\infty)
                \end{equation}
                Then $d=1/2$ is a Lebesgue number of this cover.
                Letting $X=(0,1)$ with the standard metric, for all
                $x\in{X}$ the is a $\delta_{x}>0$ such that:
                \begin{equation}
                    B_{\delta_{x}}^{(X,d)}(x)
                    \subseteq{X}
                \end{equation}
                And thus these open balls are a covering of the unit
                interval, but this covering has no Lebesgue number.
            \end{lexample}
            \begin{ltheorem}{Lebesgue Covering Lemma}
                  {Funct_Analysis_Lebesgue_Covering_Lemma}
                If $(X,d)$ is a compact metric space, and if
                $\mathcal{O}$ is an open covering of $X$, then
                $\mathcal{O}$ has a Lebesgue number.
            \end{ltheorem}
            \begin{proof}
                Suppose not. Suppose $(X,d)$ is compact, and suppose
                that $\mathcal{O}$ is a covering of $X$ with no
                Lebesgue number. But then, for all $n\in\mathbb{N}$,
                there is an $a_{n}$ such that, for all
                $\mathcal{U}\in\mathcal{O}$:
                \begin{equation}
                    B_{1/n}^{(X,d)}(a_{n})\not\subset\mathcal{U}
                \end{equation}
                But $X$ is compact, and thus there is a convergent
                subsequence such that $a_{k_{n}}\rightarrow{X}$. 
                But then there is a $\mathcal{U}\in\mathcal{O}$ such
                that $x\in\mathcal{U}$. But $\mathcal{U}$ is open,
                and thus there is an $r>0$ such that:
                \begin{equation}
                    B_{r}^{(X,d)}(x)\subseteq\mathcal{U}
                \end{equation}
                Let $N\in\mathbb{N}$ be such that, for all
                $k_{n}>N$, $d(x_{k_{n}},x)<r/2$. Let
                $n>N$ be such that $1/k_{n}<r/2$. But then:
                \begin{equation}
                    B_{1/k_{n}}(a_{k_{n}})\subseteq\mathcal{U}
                \end{equation}
                A contradiction.
            \end{proof}
            \begin{theorem}
                If $(X.d)$ is a compact metric space, if $(Y,\rho)$
                is a metric space, and if $f:X\rightarrow{Y}$ is a
                continuous function, then $f$ is
                uniformly continuous.
            \end{theorem}
            \begin{proof}
                For let $\varepsilon>0$. since $f$ is
                continuous, for all $x\in{X}$ there is a
                $\delta_{x}$ such that, for all $y\in{X}$ such
                that $d(x,y)<\delta_{x}$, it is true that
                $\rho(f(x),f(y))<\varepsilon/2$. But then:
                \begin{equation}
                    X\subseteq
                        \bigcup_{x\in{X}}B_{\delta_{x}}^{(X,d)}(x)
                \end{equation}
                But $X$ is compact, and thus this covering has a
                Lebesgue number. Let $\delta$ be such a
                Lebesgue number. But then if $d(x,y)<\delta$,
                then there is a $z\in{X}$ such that
                $x,y\in{B}_{\delta_{z}}(z)$. But then:
                \begin{equation}
                    \rho(f(x),f(x))\leq
                    \rho(f(x),f(z))+\rho(f(z),f(y))
                    <\varepsilon
                \end{equation}
            \end{proof}
        \subsection{Equicontinuity}
            \begin{ltheorem}{Arzela-Ascoli Theorem}
                  {Funct_Analysis_Arzela_Ascoli}
                If $X$ is a compact metric space, if
                $F_{n}\in{C}(X)$ is a sequence of equicontinuous
                point-wise bounded functions, then $F_{n}$ has a
                uniformly convergent subsequence.
            \end{ltheorem}
            \begin{theorem}
                If $X$ is a compact metric space and
                $\mathscr{F}\subseteq{C}(X)$ is a closed subset with
                respect to the uniform norm, and if
                $\mathscr{F}$ is equicontinuous on $X$ and point-wise
                bounded, then $\mathscr{F}$ is compact.
            \end{theorem}
            \begin{proof}
                It suffices to show that $\mathscr{F}$ is sequentially
                compact. Let $F_{n}$ be s sequence in $\mathscr{F}$.
                The by the Arzela-Ascoli theorem, there is a uniformly
                convergent subsequence $F_{k_{n}}$. But $\mathscr{F}$
                is closed, and thus the limit function is contained
                in $\mathscr{F}$. Thus, $\mathscr{F}$ is sequentially
                compact. But sequentially compact metric spaces are
                compact. Therefore, etc.
            \end{proof}
            \begin{theorem}
                If $X$ is a compact metric space and
                $\mathscr{F}\subseteq{C}(X)$ is a closed subset with
                respect to the uniform norm, and if
                $\mathscr{F}$ is equicontinuous on $X$ and point-wise
                bounded, then $\mathscr{F}$ is uniformly bounded.
            \end{theorem}
            \begin{proof}
                For $\mathscr{F}$ is compact by the previous theorem.
                But then $\mathscr{F}$ is bounded with respect to
                $\norm{\cdot}_{\infty}$. Therefore, $\mathscr{F}$ is
                uniformly bounded.
            \end{proof}
            \begin{theorem}
                If $X$ is a compact metric space and if
                $\mathscr{F}\subseteq{C}(X)$ is closed, equicontinuous,
                and uniformly bounded on $X$, then $\mathscr{F}$ is
                compact.
            \end{theorem}
            \begin{proof}
                For suppose $\mathscr{F}$ is compact. Then $\mathscr{F}$
                is closed and uniformly bounded. Thus it suffices to
                show that $\mathscr{F}$ is equicontinuous. Suppose not.
                Then there is a point $x\in{X}$ such that
                $\mathscr{F}$ is not equicontinuous at $x$. Thus,
                there exists an $\varepsilon>0$ such that, for all
                $\delta>0$, there are points $x,y$ such that
                $d(x,y)<\delta$, but $|f(x)-f(y)|\geq\varepsilon$
                for some $f\in\mathscr{F}$. Thus, for all
                $n\in\mathbb{N}$, there is an $x_{n}\in{X}$ such that
                $d(x,x_{n})<1/n$, and
                $|f_{n}(x)-f_{n}(x_{n})|\geq\varepsilon_{0}$. But if
                $\mathscr{F}$ is compact, then $f_{n}$ has a convergent
                subsequence $f_{k_{n}}$. Let $f$ be the limit.
                Since $\mathscr{F}$ is compact, $f\in\mathscr{F}$.
                But then $f_{k_{n}}(x_{k_{n}})\rightarrow{f}(x)$. But
                then there is an $N\in\mathbb{N}$ such that,
                for $k_{n}>N$,
                $\norm{f_{k_{n}}-f}_{\infty}<\varepsilon_{0}/3$.
                But then:
                \begin{align}
                    |f(x_{k_{n}})-f(x)|&=
                    |f(x_{k_{n}})-f_{k_{n}}(x_{k_{n}})
                    +f_{k_{n}}(x_{k_{n}})-f_{k_{n}}(x)
                    +f_{k_{n}}(x)-f(x)|\\
                    &\geq|f_{k_{n}}(x_{k_{n}})-f_{k_{n}}(x)|+
                    |f(x_{k_{n}})-f_{k_{n}}(x_{k_{n}})
                    +f_{k_{n}}(x)-f(x)|\\
                    &>\varepsilon
                \end{align}
                A contradiction.
            \end{proof}
        \subsection{Baire Spaces}
            \begin{ldefinition}{Baire Space}
                  {Funct_Analysis_Baire_Space}
            A Baire space is a metric space $(X,d)$ such that, for
            countable collection of open and dense sets, the
            intersection is also dense.
        \end{ldefinition}
            This is a topological property, and so Baire spaces can
            be defined for a more general topological space. The
            interior of a set in a topological space is:
            \begin{equation}
            \Int(A)=
            \bigcup\{\mathcal{U}\in\tau:\mathcal{U}\subseteq{A}\}
        \end{equation}
            \begin{theorem}
            A metric space $(X,d)$ is a Baire space if and only
            if given a countable collection $F_{n}$ of closed
            sets such that the union over all of $F_{n}$ has
            non-empty interior, then at least one of the $F_{n}$
            has non-empty interior.
            \end{theorem}
            \begin{theorem}
                There exist countable Baire spaces.
            \end{theorem}
            Suppose $\mathcal{U}\subseteq{X}$ is open at
            $x_{0}\in\mathcal{U}$. There there is a $\delta>0$ such
            $B_{\delta}(x)\subseteq\mathcal{U}$. Then:
            \begin{equation}
                \overline{B_{\delta/2}(x)}\subseteq{B}_{\delta}(x)
            \end{equation}
            Thus, $\overline{B_{\delta/2}(x)}\subseteq\mathcal{U}$
            and the diameter is less than $2\delta$.
            \begin{ltheorem}{Baire Category Theorem}
                  {Funct_Analysis_Baire_Category}
                Every complete metric space is a Baire space.
            \end{ltheorem}
            \begin{proof}
            Suppose $\mathcal{O}_{n}\subseteq{X}$ is open and
            dense for all $n\in\mathbb{N}$. Let $x_{0}\in{X}$ and
            $r_{0}>0$. It will suffice to show that:
            \begin{equation}
                B_{r_{0}}(x_{0})\cap\bigcap_{n\in\mathbb{N}}
                    \mathcal{O}_{n}\ne\emptyset
            \end{equation}
            Inductively, we create a sequence of points $x_{k}$
            and real numbers $r_{k}>0$ such that $r_{k}$ is strictly
            monotonically decreasing, and thus that:
            \begin{equation}
                \overline{B_{r_{k+1}}(x_{k+1})}
                \subseteq{B}_{r_{k}}(x_{k})\cap\mathcal{O}_{k+1}
            \end{equation}
        \end{proof}
            Consider the set of all lines through the origin with
            rational slope. The complete of any given line is the
            union of two open half planes, which are open and dense
            subsets of $\mathbb{R}^{2}$. Since we have only a countable
            collection of such lines, the intersection of the complement
            is dense in $\mathbb{R}^{2}$. Baire's Category Theorem
            holds even if $(X,d)$ is not complete, but is equivalent
            to a complete metric. For example, let $X=(0,1)$ and let
            $d(x,y)=|x-y|$ be the standard metric. This is not a
            complete space, but is homeomorphic to $\mathbb{R}$, 
            which is a complete metric space. Using this homeomorphism,
            we can find a metric $\tilde{d}$ on $(0,1)$ that is complete
            and which is equivalent to the original metric. Thus,
            $(0,1)$ is a Baire space.
            \begin{theorem}
            If $V$ is a non-empty open subset of a complete metric
            space $(X,d)$, then there is a metric $\tilde{d}$ such
            that $(V,\tilde{d})$ is complete.
        \end{theorem}
            Hence, $V$ is a Baire space. Then, given a set
            $F_{n}$ of closed subsets of $X$ such that:
            \begin{equation}
            V=\bigcup_{n=1}^{\infty}(V\cap{F}_{n})
        \end{equation}
            Then some $F_{n}\cap{V}$ has non-empty interior in $V$,
            and hence in $X$.
            \begin{theorem}
            If $X$ is a Baire space and if $f_{n}$ is a sequence
            of continuous function in $C(X)$ which converges
            point-wise to $f:X\rightarrow\mathbb{C}$, then
            the set:
            \begin{equation}
                \{x\in{X}:f\textrm{ is continuous at }x\}
            \end{equation}
            Is dense in $X$.
        \end{theorem}
            \begin{proof}
            Let $\varepsilon>0$ and define:
            \begin{align}
                A_{N}(\varepsilon)&=
                \{x:|f_{n}(x)-f_{m}(x)|\leq\varepsilon,
                    n,m\in\mathbb{N}\}\\
                &=\bigcap_{n,m\geq{N}}
                    \{x:|f_{n}(x)-f_{m}(x)|\leq\varepsilon\}
            \end{align}
            Then $A_{N}(\varepsilon)$ is closed. But also:
            \begin{equation}
                X=\bigcup_{N=1}^{\infty}A_{N}(\varepsilon)
            \end{equation}
            Thus, by the Baire category theorem, we have:
            \begin{equation}
                \mathcal{U}(\varepsilon)=
                \bigcup_{N=1}^{\infty}\Int(A_{N}(\varepsilon))
            \end{equation}
            Is non-empty and open. Moreover,
            $\mathcal{U}(\varepsilon)$ is dense. But then:
            \begin{equation}
                \mathcal{C}=\bigcap_{n=1}^{\infty}
                \mathcal{U}(\frac{1}{n})
            \end{equation}
            Is dense in $X$, and $f$ is continuous at all
            $x\in\mathcal{C}$.
        \end{proof}
            The Baire Category Theorem says that every complete metric
            space is a Baire space. The notion of Baire space is a
            topological property, and not a metric property. Thus, even
            if $(X,d)$ is not complete but is equivalent to a complete
            metric space $(X,\tilde{d})$, then $(X,d)$ is a Baire space.
            A topological space is called completely metrizable if there
            is a metric on the space that is complete and generates the
            topology. Given a complete metric space $(X,d)$, every
            non-empty open set $\mathcal{V}$ has a metric
            $d_{\mathcal{V}}$ such that $(\mathcal{V},d_{\mathcal{V}})$
            is complete, and is therefore a Baire space. Thus, if:
            \begin{equation}
                \mathcal{V}=\bigcup_{n\in\mathbb{N}}\Big(
                    \mathcal{V}\cap{F}_{n}\Big)
            \end{equation}
            Where $F_{n}$ is closed for all $n\in\mathbb{N}$, then
            for some $N\in\mathbb{N}$, $\mathcal{V}\cap{F}_{N}$ has
            interior.
            \begin{theorem}
            If $X$ is a Baire space, and if
            $F_{n}$ is a sequence of continuous functions that
            converges point-wise to $f:X\rightarrow\mathbb{C}$, then
            the set $\mathcal{D}$ defined by:
            \begin{equation}
                \mathcal{D}=
                    \{x\in{X}:\textrm{$f$ is continuous as $x$}\}
            \end{equation}
            Then $\mathcal{D}$ is dense in $X$.
        \end{theorem}
            \begin{proof}
            For let $\varepsilon>0$, and let:
            \begin{equation}
                A_{N}(\varepsilon)=
                \{x\in{X}:|f_{n}(x)-f_{m}(x)|\leq\varepsilon,n,m>N\}
            \end{equation}
            Then, for all $N\in\mathbb{N}$, $A_{N}(\varepsilon)$ is
            closed. Let $\mathcal{U}(\varepsilon)$ be defined by:
            \begin{equation}
                \mathcal{U}=\bigcup_{n\in\mathbb{N}}
                    \Int\Big(A_{N}(\varepsilon)\Big)
            \end{equation}
            Then $\mathcal{U}(\varepsilon)$ is open and dense. It
            is open for it is the union of open sets. For let
            $\mathcal{V}$ be a non-empty subset. Then:
            \begin{equation}
                \mathcal{V}=\bigcup_{n\in\mathbb{N}}
                    \Big(A_{n}(\varepsilon)\cap\mathcal{V}\Big)
            \end{equation}
            Hence there exists an $N\in\mathbb{N}$ such that:
            \begin{equation}
                A_{N}(\varepsilon)\cap\mathcal{V}\ne\emptyset
            \end{equation}
            And this has interior, and therefore:
            \begin{equation}
                \Int(A_{N}(\varepsilon))\cap\mathcal{V}\ne\emptyset
            \end{equation}
            Therefore,
            $\mathcal{V}\cap\mathcal{U}(\varepsilon)\ne\emptyset$.
            Now, define:
            \begin{equation}
                \mathcal{V}=\bigcap_{n\in\mathbb{N}}
                    \mathcal{U}\big(\frac{1}{n}\big)
            \end{equation}
            And therefore $\mathcal{C}$ is dense in $X$. We
            now want to show that $f$ is continuous for all
            $x\in\mathcal{C}$. For let $x_{0}\in\mathcal{C}$ and
            let $\varepsilon>0$. Let $k\in\mathbb{N}$ be such that
            $k^{\minus{1}}<\varepsilon$. Then
            $x_{0}\in\mathcal{U}(k^{\minus{1}})$ and thus there is
            an $N\in\mathbb{N}$ such that:
            \begin{equation}
                x_{0}\in\Int\big(A_{N}(k^{\minus{1}})\big)
            \end{equation}
            But $f_{N}$ is continuous, and thus there is a
            neighborhood $\omega$ of $x_{0}$ such that, for all
            $y\in\omega$:
            \begin{equation}
                |f_{N}(x_{0})-f_{N}(y)|<\varepsilon/3
            \end{equation}
            Shrink $\omega$ so that it resides inside of
            $\Int(A_{N}(k^{\minus{1}})$. Then:
            \begin{equation}
                |f_{n}(y)-f_{N}(y)|<k^{\minus{1}}
                \quad\quad
                n\geq{N}
            \end{equation}
            But then, use the Cauchy trick and you're down.
        \end{proof}
        \newpage
    \section{Normed Vector Spaces}
        \subsection{Basic Definitions}
            \begin{ldefinition}{Normed Vector Spaces}
                  {Funct_Analysis_Normed_Vector_Space}
                A normed vector space over a field
                $\mathbb{F}\subseteq\mathbb{C}$, denoted
                $(V,\norm{\cdot})$ is a vector space $V$ over
                $\mathbb{F}$ and a norm $\norm{\cdot}$ on $V$.
            \end{ldefinition}
        \subsection{Banach Spaces}
        \begin{ldefinition}{Banach Space}
              {Funct_Analysis_Banach_Space}
            A Banach space is a normed vector space
            $(V,\norm{\cdot})$ such that the metric $d$ induced by
            the norm $\norm{\cdot}$ is complete on $V$.
        \end{ldefinition}
        Normed spaces are special. Give $\mathbf{v}\in{V}$, and
        for $r>0$, we have:
        \begin{equation}
            B_{r}^{(V,\norm{\cdot})}(\mathbf{x})=
            B_{r}^{(V,\norm{\cdot})}(\mathbf{0})+\mathbf{x}
        \end{equation}
        That is, open balls about arbitrary points are merely
        translations of an open ball about the origin.
        \begin{equation}
            |\norm{\mathbf{v}}-\norm{\mathbf{u}}|
            \leq\norm{\mathbf{v}-\mathbf{u}}
        \end{equation}
        And thus the map $\mathbf{v}\mapsto\norm{\mathbf{v}}$ is
        continuous. The closure of an open ball is the closed ball.
        \begin{equation}
            \overline{B_{r}^{(V,\norm{\cdot})}(\mathbf{x})}
            =\{\mathbf{y}\in{V}:
                \norm{\mathbf{x}-\mathbf{y}}\leq{r}\}
        \end{equation}
        We can also multiply open balls by constants, to get
        the following:
        \begin{equation}
            \varepsilon{B}_{r}^{(V,\norm{\cdot})}(\mathbf{x})=
            B_{\varepsilon{r}}^{(V,\norm{\cdot})}(\mathbf{x})
        \end{equation}
        \begin{theorem}
            Suppose $X$ and $Y$ are normed vector spaces over
            $\mathbb{F}\subseteq\mathbb{C}$. Let $T:X\rightarrow{Y}$
            be a linear transformation. Then the following
            are equivalent:
            \begin{enumerate}
                \item $T$ is continuous.
                \item $T$ is continuous at some $x_{0}\in{X}$.
                \item There is an $\alpha>0$ such that
                      $\norm{Tx}\leq\alpha\norm{x}$.
            \end{enumerate}
        \end{theorem}
        \begin{proof}
            Suppose $T$ is continuous at $x_{0}$. Then there is a
            $\delta>0$ such that:
            \begin{equation}
                T\Big(\overline{B_{\delta}(x_{0})}\big)
                \subseteq{B}_{1}(T(x_{0}))
            \end{equation}
            But:
            \begin{align}
                T\Big(\overline{B_{\delta}(x_{0})}\big)
                &=T\Big(\overline{B_{\delta}(0)}\big)+T(x_{0})\\
                B_{1}(T(x_{0}))=
                B_{1}(0)+T(x_{0})
            \end{align}
            Now suppose $z\ne{0}$. Then:
            \begin{equation}
                \norm{T(z)}=
                \norm{\frac{\norm{z}}{\delta}
                      T\Big(\frac{\delta{z}}{\norm{z}}\Big)}
                \leq\frac{1}{\delta}\norm{z}
            \end{equation}
            Let $\alpha=\delta^{\minus{1}}$.
            Proving the next one:
            \begin{equation}
                \norm{T(x)-T(y)}=\norm{T(x-y)}
                \leq\alpha\norm{x-y}
            \end{equation}
            And so we have continuity.
        \end{proof}
        There are linear maps that are not bounded. Let
        $\ell_{1}^{0}=\Span\{e_{k}:x\in\mathbb{N}\}$. Map
        $e_{k}\rightarrow{k}e_{k}$. Let
        $\norm{\cdot}_{a}$ and $\norm{\cdot}_{b}$ be norms on
        $X$ that induce the same topology on $X$. Consider the map
        $id:(X,\norm{\cdot}_{a})\rightarrow(X,\norm{\cdot}_{b})$.
        Since the topologies are the same, $id$ is continuous.
        Then there is a $c\geq{0}$ such that:
        \begin{equation}
            \norm{x}_{b}\leq{c}\norm{x}_{a}
        \end{equation}
        We can go the other way as well, and thus we see that
        equivalence implies strongly equivalent. This is not true
        in a general metric space.
        \begin{ldefinition}{Operator Norm}
              {Funct_Analysis_Operator_Norm}
            Let $\mathscr{L}(X,Y)$ be the set of bounded linear
            transformation $T:X\rightarrow{Y}$. The operator
            norm on $T$ is:
            \begin{equation}
                \norm{T}=\sup\{\norm{T}(x):\norm{x}\leq{1}\}
            \end{equation}
        \end{ldefinition}
        \begin{theorem}
            The operator norm on $\mathscr{L}(X,Y)$ is a norm.
        \end{theorem}
        \begin{proof}
            For we have:
            \begin{equation}
                \norm{S\circ{T}}\leq\norm{S}\norm{T}
            \end{equation}
        \end{proof}
        \begin{ldefinition}{Algebra Over a Field}
              {Funct_Analysis_Algebra_Over_Field}
            An algebra over a field $\mathbb{F}$ is a vector
            space $A$ over $\mathbb{F}$ such that $A$ has a ring
            structure $(A,\times,+)$ such that:
            \begin{equation}
                \lambda(xy)=(\lambda{x})y
                =x(\lambda(y))
                \quad\quad
                x,y\in{A}
                \quad\lambda\in\mathbb{F}
            \end{equation}
        \end{ldefinition}
        \begin{lexample}
            $\mathbb{R}[x]$, $\mathbb{C}[x]$, $M_{n}(\mathbb{F})$,
            and $C_{b}(X)$.
        \end{lexample}
        \begin{ldefinition}{Normed Algebra}
              {Funct_Analysis_Normed_Algebra}
            A normed algebra is a normed space $(A,\norm{\cdot})$
            such that $A$ is an algebra and such that,
            for all $x,y\in{A}$:
            \begin{equation}
                \norm{xy}\leq\norm{x}\norm{y}
            \end{equation}
        \end{ldefinition}
        \begin{ldefinition}{Banach Algebra}
              {Funct_Analysis_Banach_Algebra}
            A Banach Algebra is a normed algebra $(A,\norm{\cdot})$
            such that $(A,\norm{\cdot})$ is a Banach space.
        \end{ldefinition}
        \begin{theorem}
            If $Y$ is a Banach space, then $\mathscr{L}(X,Y)$
            is a Banach space.
        \end{theorem}
        \begin{theorem}
            If $X$ is a Banach Algebra, then
            $\mathscr{L}(A)$ is a Banach algebra.
        \end{theorem}
        \begin{proof}
            For suppose $T_{n}$ is Cauchy in
            $\mathscr{L}(X,Y)$. Then for all $x\in{X}$,
            $T_{n}(x)$ is Cauchy in $Y$, and thus
            $T_{n}(x)$ converges to some $y\in{Y}$. Let
            $T:X\rightarrow{Y}$ be this limit function. Then
            $T:X\rightarrow{Y}$ is a linear map. But since
            $T_{n}$ is Cauchy, it is uniformly bounded. But then
            there is an $M\in\mathbb{R}^{+}$ such that:
            \begin{equation}
                \norm{T_{n}}\leq{M}
            \end{equation}
            And thus:
            \begin{equation}
                \norm{T{x}}=
                \underset{n\rightarrow\infty}{\lim}
                \norm{T_{n}x}\leq
                \lim\underset{n}{\sup}\norm{T_{n}}\norm{x}
                \leq{M}\norm{x}
            \end{equation}
            Therefore, etc.
        \end{proof}
    \section{Lecture 9, I Think}
        Let $X_{\lambda}$ be a Banach space for all
        $\lambda\in\Lambda$. Then the product is:
        \begin{equation}
            \prod_{\lambda\in\Lambda}X_{\lambda}
            =\{f:\Lambda\rightarrow\bigcup_{\lambda\in\Lambda}
            X_{\lambda}:f(\lambda)\in{X}_{\lambda}\}
        \end{equation}
        Generally, we think of the indexing set to be finite,
        $\Lambda=\{1,\dots,n\}$. The product space is then:
        \begin{equation}
            \prod_{\lambda=1}^{n}X_{\lambda}=
            X_{1}\times\dots{X}_{n}
        \end{equation}
        Functions are therefore $n$ tuples. There's no reason to
        expect that this will be a Banach space in any reasonable
        way. Thus we define the Banach Space Direct-Product.
        \begin{ldefinition}{Banach Space Direct Product}
              {Funct_Analysis_Banach_Space_Direct_Product}
            The Banach Space Direct Product of a set of Banach
            spaces $X_{\lambda}$ indexed over $\Lambda$ is:
            \begin{equation}
                \prod_{\lambda\in\Lambda}^{*}X_{\lambda}
                =\{f\in\prod_{\lambda\in\Lambda}X_{\lambda}:
                \underset{\lambda\in\Lambda}{\sup}
                f(\lambda)<\infty\}
            \end{equation}
        \end{ldefinition}
        Then $\norm{x}=\sup_{\lambda}\norm{x_{\lambda}}$ is a norm
        on the product space.
        \begin{ltheorem}{Open Mapping Theorem}
              {Funct_Analysis_Open_Mapping_Theorem}
            If $X$ and $Y$ are Banach spaces and if
            $T\in\mathcal{L}(X,Y)$ is surjective, then $T$ is an
            open map.
        \end{ltheorem}
        \begin{proof}
            It will suffice to find $r>0$ such that:
            \begin{equation}
                B_{r}^{Y}\subseteq
                T\Big(B_{1}^{X}(0)\Big)
            \end{equation}
            By homogeneity, $T(B_{\delta}^{X}(0))$ is a
            neighborhood of $0$ for all $\delta>0$. By linearity,
            $T(B_{\delta}(x))$ is a neighborhood of
            $T(x)$ for all $x\in{X}$ and for all $\delta>0$. But
            if $V\subseteq{X}$ is open, and $x\in{V}$, then there
            is a $\delta>0$ such that $B_{\delta}(x)\subseteq{V}$.
            Thus, $T(B_{\delta}(x))$ is a neighborhood of $T(x)$
            in $T(V)$. There is al an $r>0$ such that:
            \begin{equation}
                B_{r}^{Y}(0)\subseteq
                \overline{T(B_{1}^{X}(0)))}
            \end{equation}
            For let $\alpha\in(0,1)$. Note that:
            \begin{equation}
                T(B_{\alpha}(x))=\alpha{T}(B_{1}(0))
            \end{equation}
            And also:
            \begin{equation}
                B_{\alpha{r}}(0)\subseteq
                \overline{\alpha{T}(B_{1}(0))}
            \end{equation}
            Let $y\in{B}_{r}(0)$. Then there is a
            $y_{1}\in{T(B_{1}(0))}$ such that:
            \begin{equation}
                \norm{y-y_{1}}<\frac{r}{2}
            \end{equation}
            But then $y-y_{1}\in{B}_{r/2}(0)$. But then there is a
            $y_{2}\in{T}(B_{1/2}(0))$ such that:
            \begin{equation}
                \norm{y_{2}}<\frac{r}{4}
            \end{equation}
            That is:
            \begin{equation}
                \norm{y-y_{1}-y_{2}}<\frac{r}{2^{2}}
            \end{equation}
            Continuing we obtain a sequence $y_{n}$ such that:
            \begin{equation}
                y_{n}\in{T}(B_{1/2^{n}}(0))
            \end{equation}
            And such that:
            \begin{equation}
                \norm{y-\sum_{k=1}^{n}y_{k}}<
                \frac{r}{2^{n}}
            \end{equation}
            Note that there exists $x_{n}\in{X}$ such that
            $T(x_{n})=y_{n}$ and:
            \begin{equation}
                \norm{x_{n}}<\frac{1}{2^{n-1}}
            \end{equation}
            But then there is an $x\in{X}$ such that:
            \begin{equation}
                x=\sum_{n=1}^{\infty}x_{n}
            \end{equation}
            Since $X$ is complete and since this series converges.
            But $T$ is continuous, and therefore $T(x)=y$. But:
            \begin{equation}
                \norm{x}\leq\sum_{n=1}^{\infty}\norm{x_{n}}
                <\sum_{n=0}^{\infty}\frac{1}{2^{n}}=2
            \end{equation}
            That is,:
            \begin{equation}
                B_{r}(0)\subseteq{T}(B_{2}(0))
            \end{equation}
            And therefore:
            \begin{equation}
                B_{r/2}(0)\subseteq
                T_{1}(B_{1}(0))
            \end{equation}
            If $T$ is surjective, we can write:
            \begin{equation}
                Y=\bigcup_{n\in\mathbb{N}}
                \overline{T(B_{n}(0))}
            \end{equation}
            But $Y$ is a Banach space, and thus by the Baire
            category theorem, there is an $n\in\mathbb{N}$ such
            that $\overline{T(B_{n}(0))}$ has interior. Therefore,
            etc.
        \end{proof}
        \begin{lexample}
            Recall that:
            \begin{equation}
                \ell_{0}^{p}=
                \{x\in\ell^{p}:\exists{N\in\mathbb{N}},
                    \forall_{n>N},x_{n}=0\}
            \end{equation}
            Let $\ell_{0}^{p}$ be defined by:
            \begin{equation}
                \ell_{0}^{p}=\Span\{e_{n}:n\in\mathbb{N}\}
            \end{equation}
            Define $T:\ell_{0}^{2}\rightarrow\ell_{0}^{p}$ by:
            \begin{equation}
                T(e_{n})=\frac{1}{n}e_{n}
            \end{equation}
            Then $T$ is bounded and $\norm{T}\leq{1}$. Note that
            $T$ is bijective and has an inverse:
            \begin{equation}
                T^{\minus{1}}(e_{n})=ne_{n}
            \end{equation}
            And this is not bounded, so
            $T^{\minus{1}}\notin\mathcal{L}(\ell_{0}^{p})$.
            This can happen since $\ell_{0}^{P}$ is not complete.
        \end{lexample}
        \begin{ltheorem}{Inverse Mapping Theorem}
              {Funct_Analysis_Inverse_Mapping_Theorem}
            If $X$ and $Y$ are Banach spaces and
            $T\in\mathcal{L}(X,Y)$ is bijective, then
            $T^{\minus{1}}\in\mathcal{L}(Y,X)$.
        \end{ltheorem}
        \begin{proof}
            Note that $T^{\minus{1}}$ is linear. Since $T$ has to be
            open by the open mapping theorem, $T^{\minus{1}}$ is
            continuous. But continuous linear functions are bounded.
            Therefore, $T^{\minus{1}}$ is bounded.
        \end{proof}
        \begin{ltheorem}{Closed Graph Theorem}
              {Funct_Analysis_Closed_Graph_Theorem}
            If $X$ and $Y$ are Banach spaces and if
            $T:X\rightarrow{Y}$ is linear, then
            $T\in\mathcal{L}(X,Y)$ if and only if the graph of
            $T$ is closed in $X\times{Y}$.
        \end{ltheorem}
        \begin{proof}
            Note that $X\times{Y}$ is a Banach space and
            $(x_{n},y_{n})\rightarrow(x,y)$ if and only if
            $x_{n}\rightarrow{x}$ and $y_{n}\rightarrow{y}$.
            If $T$ is bounded, then the graph of $T$ is closed.
            Now, suppose that the graph of $T$ is closed. But
            then the graph is a Banach space. The projection map
            $P:T\rightarrow{X}$ given by $P((x,T(x))=x$ is
            a bounded bijection. Hence, $P^{\minus{1}}$ is bounded.
            Let $P_{2}$ be defined by $P_{2}(x,T(x))=T(x)$. Then
            $P_{2}$ is bounded. But:
            \begin{equation}
                T(x)=P_{2}\circ{P}_{1}^{\minus{1}}(x)
            \end{equation}
            And therefore $T$ is bounded.
        \end{proof}
        \begin{lexample}
            Suppose $T_{n}\in\mathcal{L}(X,Y)$ and suppose, for
            all $x\in{X}$:
            \begin{equation}
                T(x)=\underset{n\rightarrow\infty}{\lim}T_{n}(x)
            \end{equation}
            Then we have that $T$ is linear. Is
            $T\in\mathcal{L}(X,Y)$? We have:
            \begin{equation}
                \norm{T}=
                \norm{\underset{n\rightarrow\infty}{\lim}T_{n}}
                \leq\underset{n\rightarrow\infty}{\lim}\sup_{n}
                \norm{T_{n}}
            \end{equation}
        \end{lexample}
        \begin{ltheorem}{Principle of Uniform Boundedness}
              {Funct_Analysis_Prin_Uni_Bounded}
            If $X$ and $Y$ are Banach spaces, if
            $T_{\lambda}\in\mathcal{L}(X,Y)$ for all
            $\lambda\in\Lambda$, and if for all $x\in{X}$ we have:
            \begin{equation}
                \norm{\{\norm{T_{\lambda}(x)}:\lambda\in\Lambda\}}
                <\infty
            \end{equation}
            Then $\{\norm{T_{\lambda}}:\lambda\in\Lambda\}$ is
            bounded.
        \end{ltheorem}
        Recall that if $X_{\lambda}$ are Banach spaces for all
        $\lambda\in\Lambda$, where $\Lambda$ is some indexing
        set, then we can form the Banach Space direct product:
        \begin{equation}
            \prod_{\lambda\in\Lambda}^{*}X_{\lambda}
            =\Big\{x\in\prod_{\lambda\in\Lambda}X_{\lambda}:
                \norm{x}_{\lambda}<\infty\Big\}
        \end{equation}
        If $\lambda_{0}\in\Lambda$, we can define:
        \begin{equation}
            P_{\lambda}:\prod_{\lambda\in\Lambda}^{*}X_{\lambda}
            \rightarrow{X}_{\lambda_{0}}
        \end{equation}
        By defining $P_{\lambda_{0}}(x)=x(\lambda_{0})$.
        Now consider the case when $X_{\lambda}=Y$ for all
        $\lambda\in\Lambda$. We'll write:
        \begin{equation}
            Y_{\Lambda}=\prod_{\lambda\in\Lambda}Y
        \end{equation}
        The principle of uniform boundedness says that if
        $X$ and $Y$ are Banach spaces and if
        $T_{\lambda}\in\mathcal{L}(X,Y)$ for all
        $\lambda\in\Lambda$, and that for all $x\in{X}$, the set
        $\{\norm{T_{\lambda}(x)}:\lambda\in\Lambda\}$ is
        bounded, then $\{\norm{T_{\lambda}}:\lambda\in\Lambda\}$
        is bounded.
        \begin{ltheorem}{Principle of Uniform Boundedness}{}
            If $X$ and $Y$ are Banach spaces, if
            $T_{\lambda}\in\mathcal{L}(X,Y)$ for all
            $\lambda\in\Lambda$, and if for all $x\in{X}$ we 
            have that $\{\norm{T_{\lambda}(x)}:\lambda\in\Lambda\}$
            is bounded, then
            $\{\norm{T_{\lambda}}:\lambda\in\Lambda\}$ is bounded.
        \end{ltheorem}
        \begin{proof}
            For let:
            \begin{equation}
                Y_{\Lambda}=\prod_{\lambda\in\Lambda}Y
            \end{equation}
            And define $T:X\rightarrow{Y}_{\Lambda}$ by:
            \begin{equation}
                T(x)=T_{\lambda}(x)
            \end{equation}
            Then $T$ is well defined and linear. To see
            that $T$ is bounded, use the closed graph
            theorem. That is, suppose $x_{n}\rightarrow{x}$.
            We need to show that $T(x_{n})\rightarrow{T}(x)$.
            Let $\lambda\in\Lambda$. Then
            $P_{\lambda}(T(x_{n}))\rightarrow{P}_{\lambda}(y)$.
            Then $T_{n}(x_{n})(\lambda)\rightarrow{y}(\lambda)$.
            But $T(x_{n})(\lambda)=T_{\lambda}(x_{n})$, and this
            converges to $T_{\lambda}(x)$. But then
            $y(\lambda)=T_{\lambda}$ for all $\lambda$, and
            thus $y=T(x)$. Therefore
            $T\in\mathcal{L}(X,Y_{\Lambda})$. But if
            $\norm{x}\leq{1}$, then:
            \begin{equation}
                \norm{T_{\lambda}(x)}\leq
                \sup_{\lambda\in\Lambda}\norm{T_{\lambda}(x)}
                =\sup_{\lambda\in\Lambda}\norm{T(x)(\lambda)}
                =\norm{T(x)}\leq\norm{T}
            \end{equation}
            Thus, $\norm{T}\geq\norm{T_{\lambda}}$ for all
            $\lambda\in\Lambda$. Therefore, etc.
        \end{proof}
        \begin{ltheorem}{Banach-Stienhaus Theorem}
              {Funct_Analysis_Banach_Stienhaus}
            If $X$ and $Y$ are Banach spaces, and if
            $T_{n}\in\mathcal{L}(X,Y)$ converges point-wise to
            $T$, then $T\in\mathcal{L}(X,Y)$.
        \end{ltheorem}
        \begin{proof}
            By the principle of uniform boundedness,
            $T_{n}$ is uniformly bounded. Therefore $T$ is bounded.
        \end{proof}
    \section{Zorn's Lemma}
        A relation on a set $X$ is a subset
        $R\subseteq{X}\times{X}$. Given an element $(x,y)\in{R}$,
        we often write $xRy$ to denote this. Here we'll write
        $x\leq{y}$.
        \begin{ldefinition}{Ordered Sets}
              {Funct_Analysis_Ordered_Set}
            An ordered set is a set $X$ and a relation
            $\leq$ on $X$, denoted $(X,\leq)$ such that
            the following are true:
            \begin{enumerate}
                \item For all $x\in{X}$, $x\leq{x}$.
                \item For all $x,y\in{X}$ such that $x\leq{y}$ and
                      $y\leq{x}$, it is true that $x=y$.
                \item For all $x,y,z\in{X}$ such that $x\leq{y}$
                      and $y\leq{z}$, it is true that $x\leq{z}$.
            \end{enumerate}
        \end{ldefinition}
        \begin{ldefinition}{Majorants in Ordered Sets}
              {Funct_Analysis_Majorant_in_Ord_Set}
            A majorant of a subset $Y\subseteq{X}$ of
            and ordered set $(X,\leq)$ is an element $x\in{X}$
            such that, for all $y\in{Y}$, it is true that
            $y\leq{x}$.
        \end{ldefinition}
        \begin{lexample}
            Let $(X,d)$ be a metric space, and let
            $x_{0}\in{X}$. Define the following:
            \begin{equation}
                \mathscr{N}(x_{0})=
                \big\{\mathcal{V}\subseteq{X}:\mathcal{V}
                    \textrm{ is a neighborhood of $x_{0}$}\big\}
            \end{equation}
            We can order $\mathscr{N}$ by reverse containment.
            That is, We have the following relation:
            \begin{equation}
                \leq=\big\{(\mathcal{U},\mathcal{V})\in
                    \mathscr{N}(x_{0})\times\mathscr{N}(x_{0})
                    :\mathcal{V}\subseteq\mathcal{U}\big\}
            \end{equation}
            That is, we write $\mathcal{U}\leq\mathcal{V}$ if
            $\mathcal{V}$ is a subset of $\mathcal{U}$. Note
            that, for all $x_{0}$,
            $\mathscr{x_{0}}$ has a least element, or a minorant,
            but $X$ is such an element. But, if $\{x_{0}\}$ is
            not open, then there is no majorant.
        \end{lexample}
        \begin{ldefinition}{Totally Ordered Sets}
              {Funct_Analysis_Tot_Ord_Set}
            A totally ordered set is an ordered set
            $(X,\leq)$ such that, for all $x,y\in{X}$, either
            $x\leq{y}$ or $y\leq{x}$.
        \end{ldefinition}
        \begin{ldefinition}{Maximal Element}
              {Funct_Analysis_Maximal_Element}
            A maximal element of a subset $Y\subseteq{X}$ of
            a totally ordered set $(X,\leq)$ is an element $y$
            such that:
            \begin{equation}
                \{y'\in{Y}:y\leq{y}'\}=\{y\}
            \end{equation}
            Note that $y$ is not necessary a majorant for $Y$
            nor is $y$ necessarily unique.
        \end{ldefinition}
        \begin{ldefinition}{Inductively Ordered Sets}
              {Funct_Analysis_Induct_Ordered_Set}
            An inductively ordered set is an ordered set
            $)X,\leq)$ such that, for all totally ordered
            subsets $S\subseteq{X}$, there is a majorant
            $x\in{X}$ of $S$.
        \end{ldefinition}
        That is, there exists $x\in{X}$ such that, for all
        $y\in{S}$, $y\leq{x}$.
        \begin{lexample}
            Let $X=\mathbb{R}$ and consider the set
            $\mathscr{N}(0)$. Then $\mathscr{N}(0)$ is
            not inductively ordered.
        \end{lexample}
        \begin{ltheorem}{Zorn's Lemma}
              {Funct_Analysis_Zorns_Lemma}
            If $(X,\leq)$ is an inductively ordered set,
            then there is a maximal element $x\in{X}$.
        \end{ltheorem}
        For more review, see Folland's Real Analysis and
        Pedersen's Analysis Now. Zorn's lemma is used to
        prove that every vector space has a basis. We'll
        use this to discuss the notion of duals on
        normed vector spaces.
        \begin{ldefinition}{Dual of a Normed Vector Space}
              {Funct_Analysis_Dual_of_Normed_Vec_Space}
            The dual of a normed vector space $(X,\norm{\cdot})$
            is the set $X^{*}=\mathcal{L}(X,\mathbb{F})$
            with the operator norm.
        \end{ldefinition}
        \begin{theorem}
            If $(X,\norm{\cdot})_{X}$ is a normed vector space,
            then $(X^{*},\norm{\cdot})$ is a Banach space.
        \end{theorem}
        \begin{lexample}
            Let $X=\mathbb{F}^{n}$ and let $\{e_{1},\dots,e_{n}\}$
            be the standard basis. The we can define
            $e_{k}^{*}\in(\mathbb{F}^{n})^{*}$ by:
            \begin{equation}
                e_{k}^{*}(\alpha_{1}e_{1}+\cdots+\alpha_{n}e_{n})
                =\alpha_{k}
            \end{equation}
            Then $\{e_{1}^{*},\dots,e_{n}^{*}\}$ is a basis
            for $(\mathbb{F})^{*}$, and thus
            $(\mathbb{F}^{n})^{*}\simeq\mathbb{F}^{n}$.
        \end{lexample}
        \begin{lexample}
            Let $\{e_{\lambda}\}_{\lambda\in\Lambda}$ be
            a Hamel basis. Then we can define a linear basis:
            \begin{equation}
                e_{\lambda}^{*}:X\rightarrow\mathbb{F}
            \end{equation}
            But also, the set
            $\{\lambda\in\Lambda:e_{\lambda}^{*}\in{X}^{*}\}$
            is at most finite.
        \end{lexample}
        One question that arises is, given any normed vector
        space $X$, what can we say about the dual space $X^{*}$?
        We know that the zero operator is in there, but is there
        anything else?
        \begin{ldefinition}{Minkowski Functional}
              {Funct_Analysis_Minkowski_Functional}
            A Minkowski function on a normed vector space
            $(X,\norm{\cdot})$ over a field
            $\mathbb{F}\subseteq\mathbb{C}$ is a function
            $m:X\rightarrow\mathbb{R}$ such that the following are
            true:
            \begin{subequations}
                \begin{align}
                    m(x+y)&\leq{m}(x)+m(y)\\
                    m(tx)&=tm(x)\quad\quad{t}\geq{0}
                \end{align}
            \end{subequations}
        \end{ldefinition}
        \begin{lexample}
            If $\norm{\cdot}$ is a semi-norm on $X$ over
            $\mathbb{R}$ or $\mathbb{C}$, then $m(x)=\norm{x}$
            is a Minkowski functional on $X$. If we let
            $X=\ell_{\mathbb{R}}^{\infty}$, then:
            \begin{equation}
                m(x)=\underset{n\rightarrow\infty}{\lim}
                    \underset{k\leq{n}}{\sup}\{x_{n}\}
            \end{equation}
            Is also a Minkowski functional.
        \end{lexample}
        \begin{ltheorem}{Basic Extension Lemma}
              {Funct_Analysis_Basic_Extension_Lemma}
            If $m:X\rightarrow\mathbb{R}$ is a Minkowski functional
            on a vector space $X$ over $\mathbb{R}$ if
            $Y\subseteq{X}$ is a subspace, and if
            $\varphi:Y\rightarrow\mathbb{R}$ is a linear functional
            such that, for all $y\in{Y}$,
            $\varphi(y)\leq{m}(y)$, then there is a linear
            functional $\tilde{\varphi}:X\rightarrow\mathbb{R}$
            such that, for all $x\in{X}$,
            $\tilde{\varphi}(x)\leq{m}(x)$, and for all $y\in{Y}$,
            $\tilde{\varphi}(y)=\varphi(y)$.
        \end{ltheorem}
        If $X$ is a normed vector space, then we can identify
        $X$ with it's image $i(X)$ in $X^{**}$, where
        $i:X\rightarrow{X}^{**}$ is defined by:
        \begin{equation}
            i(x)(\phi)=\phi(x)
        \end{equation}
        If $X$ is a Banach space, then $i(X)$ is closed. The
        Otherwise, we call $\tilde{X}=\overline{i(X)}$ is the
        completion of $X$.
        \begin{ldefinition}{Reflexive Banach Space}
              {Funct_Analysis_Reflexive_Banach_Space}
            A reflexive banach space is a Banach space $X$ such
            that the natural map $i:X\rightarrow{X}^{**}$ is
            surjective.
        \end{ldefinition}
        If $X$ is reflexive, then $X$ is isometrically isomorphic
        to $X^{**}$. One might suspect that the converse is true,
        but that's not what the definition says. Indeed, the
        converse is not true. There are Banach spaces $X$ that
        are isometrically isomorphic to $X^{**}$ that are not
        reflexive.
        \begin{lexample}
            Let $X=\ell^{p}$, with $1<p<\infty$, and let
            $q$ be the conjugate exponent of $p$. If $y\in\ell^{q}$,
            there is a linear functional
            $\varphi_{q}^{y}\in(\ell^{p})^{*}$ defined by:
            \begin{equation}
                \varphi_{y}^{p}(x)=\sum_{n=1}^{\infty}x_{n}y_{n}
            \end{equation}
            From H\"{o}lder's inequality, this sum does indeed
            converge, and:
            \begin{equation}
                \norm{\varphi_{y}^{p}}\leq\norm{y}_{q}
            \end{equation}
            Moreover, equality is obtained:
            \begin{equation}
                \norm{\varphi_{y}^{p}}=\norm{y}_{q}
            \end{equation}
            and the map $y\mapsto\varphi_{y}^{p}$ is an
            isometric isomorphism of $\ell^{q}$ with
            $(\ell^{p})^{*}$. Consider
            $i(x)\in(\ell^{p})^{**}=(\ell^{q})^{*}$ by
            the identity above. Note that:
            \begin{equation}
                i(x)(\varphi_{y}^{p})=\varphi_{y}^{p}(x)
                =\varphi_{x}^{q}(y)
            \end{equation}
            And thus $i:X\rightarrow{X}^{**}$ is surjective, so
            $X$ is reflexive.
        \end{lexample}
        \begin{ldefinition}{Transpose of a Bounded Linear Operator}
              {Funct_Analysis_Tranpose_of_BLO}
            The transpose over a bounded linear operator
            $T:X\rightarrow{Y}$ between normed vector spaces $X$
            and $Y$ is the function $T^{*}:Y^{*}\rightarrow{X}^{*}$
            defined by:
            \begin{equation}
                T^{*}(\varphi)(x)=\varphi(T(x))
            \end{equation}
            For all $\varphi\in{Y}^{*}$.
        \end{ldefinition}
        \begin{theorem}
            If $X$ and $Y$ are normed vector spaced and if
            $T\in\mathcal{L}(X,Y)$, then
            $T^{*}\in\mathcal{L}(Y^{*},X^{*})$ and
            $\norm{T^{*}}=\norm{T}$.
        \end{theorem}
        \begin{proof}
            For:
            \begin{subequations}
                \begin{align}
                    \norm{T^{*}(\varphi)}&=
                    \underset{\norm{x}=1}{\sup}|T^{*}(\varphi)(x)|\\
                    &=\underset{\norm{x}=1}{\sup}|\varphi(T(x))|\leq
                    \underset{\norm{x}=1}{\sup}
                        \norm{\varphi}\norm{T}\norm{x}\\
                    &=\underset{\norm{x}=1}{\sup}
                        \norm{\varphi}{\norm{T}}\\
                    &\leq\norm{T}\norm{\varphi}
                \end{align}
            \end{subequations}
            Therefore, $\norm{T^{*}}\leq\norm{T}$. Let
            $\varepsilon>0$. Then there is an $x$ such that
            $\norm{x}=1$ and $\norm{T}<\norm{T(x)}+\varepsilon$.
            But there exists $\varphi\in{Y}^{*}$ such that
            $\norm{\varphi}=1$ and $\varphi(T(x))=\norm{T(x)}$.
            But then:
            \begin{subequations}
                \begin{align}
                    \norm{T^{*}}\geq\norm{T^{*}(\varphi)}
                    &\geq|T^{*}(\varphi(x))|\\
                    &=\norm{T(x)}\\
                    &>\norm{T}-\varepsilon
                \end{align}
            \end{subequations}
            By letting $\varepsilon$ tend to zero, we see that
            $\norm{T^{*}}\geq\norm{T}$. Thus,
            $\norm{T^{*}}=\norm{T}$.
        \end{proof}
        \begin{theorem}
            If $X$ and $Y$ are Banach Spaces, if $T:X\rightarrow{Y}$
            and $S:Y^{*}\rightarrow{X}^{*}$ are functions such
            that, for all $\varphi\in{Y}^{*}$ and for all $x\in{X}$,
            we have $S(\varphi)(x)=\varphi(T(x))$, then $S$ and $T$
            are bounded linear operators and $S=T^{*}$.
        \end{theorem}
        \begin{proof}
            Not if $\varphi\in{Y}^{*}$, then $S(\varphi)\in{X}^{*}$,
            and thus:
            \begin{subequations}
                \begin{align}
                    \varphi\big(T(x+\lambda{y})\big)&=
                    S(\varphi)(x+\lambda{y})\\
                    &=S(\varphi)(x)+\lambda{S}(\varphi)(y)\\
                    &=\varphi(T(x))+\lambda\varphi(T(y))\\
                    &=\varphi(T(x)+\lambda{T}(y))
                \end{align}
            \end{subequations}
            Since $\varphi\in{Y}^{*}$, we have:
            \begin{equation}
                T(x+\lambda{y})=T(x)+\lambda{T}(y)
            \end{equation}
            To see that $T$ is bounded, we use the closed graph
            theorem. Suppose $x_{n}\rightarrow{x}$ and
            $T(x_{n})\rightarrow{y}$. Then, for all
            $\varphi\in{Y}^{*}$, we have:
            \begin{subequations}
                \begin{align}
                    \varphi(y)&=\underset{n\rightarrow\infty}{\lim}
                        \varphi(T(x_{n}))\\
                        &=\underset{n\rightarrow\infty}{\lim}
                        S(\varphi)(x_{n})\\
                        &=S(\varphi)(x)\\
                        &=\varphi(T(x))
                \end{align}
            \end{subequations}
            Thus, by the closed graph theorem, $y=T(x)$.
        \end{proof}
        \subsection{Brushing Up on Topology}
            Let $(X,\tau)$ be a topological space. Recall that
            elements of $\tau$ are called open sets.
            \begin{ldefinition}{Basis of a Topology}
                  {Funct_Analysis_Basis_of_Topology}
                A basis of a topological space $(X,\tau)$ is a
                subset $\beta\subseteq\tau$ such that, for all
                $\mathcal{U}\in\tau$ and $x\in\mathcal{U}$, there
                is a $V\in\beta$ such that
                $x\in{V}\subset\mathcal{U}$.
            \end{ldefinition}
            We say that a subset $V\subseteq{X}$ is a neighborhood
            of $x$ if there is an open set
            $\mathcal{U}\in\tau$ such that
            $x\in\mathcal{U}\subseteq{V}$. We write
            $\mathscr{N}(x)$ for the collection of neighborhoods
            of $x$.
            \begin{ldefinition}{Neighborhood Basis}
                  {Funct_Analysis_Neighborhood_Basis}
                A neighborhood basis of a point $x$ in a topological
                space $(X,\tau)$ is a subset
                $\alpha\subseteq\mathscr{N}(x)$ such that
                for all $\mathcal{U}\in\mathscr{N}(x)$ there is
                a $V\in\alpha$ such that
                $x\in{V}\subseteq\mathcal{U}$.
            \end{ldefinition}
            \begin{lexample}
                In a metric space, the open balls form
                a basis for the metric topology. In $\mathbb{R}^{n}$
                every point has a neighborhood basis consisting of
                compact sets. Thus, we say that $\mathbb{R}^{n}$
                is locally compact. Thus it is useful to allow
                neighborhoods of point be be more than just open
                sets containing the point.
            \end{lexample}
            \begin{theorem}
                If $\alpha(x)$ is a neighborhood basis of
                $x$ consisting of open sets, and:
                \begin{equation}
                    \beta=\bigcup_{x\in{X}}\alpha(x)
                \end{equation}
                Then $\beta$ is a basis for $(X,\tau)$.
            \end{theorem}
            \begin{theorem}
                If $(X,\tau)$ a topological space, then
                $\beta\subseteq\tau$ is a basis if and only if,
                for all $\mathcal{U}\in\tau$, we have:
                \begin{equation}
                    \mathcal{U}=
                    \bigcup_{\underset{V\subseteq\mathcal{U}}
                        {V\in\beta}}V
                \end{equation}
            \end{theorem}
            \begin{ldefinition}{Separable Topological Space}
                  {Funct_Analysis_Sep_Top_Space}
                A separable topological space is a topological
                space $(X,\tau)$ such that there is a countable
                dense subset $\mathcal{D}\subseteq{X}$.
            \end{ldefinition}
            \begin{ldefinition}{First Countable Topological Space}
                  {Funct_Analysis_First_Count_Top_Space}
                A first countable topological space is a
                topological space $(X,\tau)$ such that, for
                all $x\in{X}$, there is a countable neighborhood
                basis $\alpha(x)\subseteq\tau$.
            \end{ldefinition}
            \begin{ldefinition}{Second Countable Topological Space}
                  {Funct_Analysis_Second_Count_Top_Space}
                A second countable topological space is a
                topological space $(X,\tau)$ such that there is
                a countable basis $\beta\subseteq\tau$.
            \end{ldefinition}
            \begin{lexample}
                Let $X$ be any set, and let $\tau=\mathcal{P}(X)$.
                This is called the discrete topology on $X$
                and is metrizable, for it is generated by the
                discrete metric on $X$. The topological space
                $(X,\tau)$ is not second countable if $X$ is an
                uncountable set.
            \end{lexample}
            \begin{theorem}
                If $(X,\tau)$ is a metrizable metric space, then
                $(X,\tau)$ is second countable if and only if
                $(X,\tau)$ is separable.
            \end{theorem}
            \begin{theorem}
                If $(X,\tau)$ is metrizable, then it is first
                countable.
            \end{theorem}
            \begin{theorem}
                If $X$ is a set and $S\subseteq\mathcal{P}(X)$,
                then there is a smallest topology, $\tau(S)$, such
                that $S\subseteq\tau(S)$.
            \end{theorem}
            \begin{proof}
                Since $\mathcal{P}(X)$ is a topology on $X$ such
                that $S\subseteq\mathcal{P}(X)$, the set of
                topologies on $X$ such that $S$ is contained in
                the topology is non-empty. Let $A$ denote this set,
                and define:
                \begin{equation}
                    \tau=\bigcap_{\tau_{A}\in{A}}\tau_{A}
                \end{equation}
                Then $\tau$ is a topology, and $S\subseteq\tau$.
                Moreover, for any topology $\tau'$ such
                that $S\subseteq\tau'$, we have $\tau\subseteq\tau$.
                Thus, $\tau$ is the smallest topology.
            \end{proof}
            \begin{theorem}
                If $\beta\subseteq\mathcal{P}(X)$ has a cover of
                $X$, then $\beta$ is a basis for
                $\tau(\beta)$ if and only if, given $\mathcal{U}$,
                $\mathcal{V}$ in $\tau(\beta)$, and
                $x\in\mathcal{U}\cap\mathcal{V}$, then there is an
                $\omega\in\beta$ such that
                $x\in\omega\subseteq\mathcal{U}\cap\mathcal{V}$.
            \end{theorem}
            \begin{theorem}
                If $X$ is a topological space, if
                $\beta\subseteq\mathcal{P}(X)$ is a cover of $X$
                such that $\beta$ is closed under intersection,
                then $\beta$ is a basis for the topological space
                $(X,\tau(\beta))$.
            \end{theorem}
            \begin{theorem}
                If $\rho\subseteq\mathcal{P}(X)$ is a cover of
                $X$, and if:
                \begin{equation}
                    \beta=\Big\{\bigcap_{i=1}^{n}V_{i}:
                        n\in\mathbb{N},V_{i}\in\rho\Big\}
                \end{equation}
                Then $\beta$ is a basis for $\tau(\rho)$. We call
                $\rho$ a sub-basis for $\tau(\rho)$.
            \end{theorem}
            \begin{ldefinition}{Topology Induced by Functions}
                  {Funct_Analysis_Top_Induced_by_Funcs}
                The topology induced on a set $X$ by a
                set $\mathcal{F}$ of functions $f$ from $X$ to
                a topological space $(Z_{F},\tau_{F})$ is the
                smallest topology on $X$ such that, for all
                $f\in\mathcal{F}$, $f$ is continuous.
            \end{ldefinition}
            \begin{theorem}
                If $X$ is a set, $\mathcal{F}$ is a set of functions
                from $X$ to a topological space $(Z_{F},\tau_{F})$,
                if $\tau$ is the topology induced by $\mathcal{F}$,
                then:
                \begin{equation}
                    \beta=\big\{f^{\minus{1}}(V):
                        f\in\mathcal{F},V\in\tau_{F}\big\}
                \end{equation}
                Is a sub-basis for $\tau$.
            \end{theorem}
            \begin{ldefinition}{Weak Topology}
                  {Funct_Analysis_Weak_Topology}
                The weak topology on a normed vector space
                $X$ is the topology on $X$ generated by 
                $X^{*}$.
            \end{ldefinition}
            That is, the weak topology is the smallest topology on
            $X$ such that every linear functional is continuous.
    \section{Stuff}
        Last time, we described the initial topology on a space
        $X$ generated by a family of functions $\mathscr{F}$
        $f:X\rightarrow(Z_{F},\tau_{F})$.
        \begin{ldefinition}{Initial Topology}
              {Funct_Analysis_Initial_Topology}
            The initial topology on a normed vector space
            $(X,\norm{\cdot})$ generated by a family of functions
            $\mathscr{F}=\{f:X\rightarrow(Z_{F},\tau_{F})\}$ such
            that $f$ is continuous for all $f\in\mathscr{F}$.
        \end{ldefinition}
        \begin{theorem}
            If $\mathcal{U}$ is defined by:
            \begin{equation}
                \mathcal{U}(\varphi,x_{0},\varepsilon)=
                \{x\in{X}:|\varphi(x)-\varphi(x_{0})|<\varepsilon\}
            \end{equation}
            For $\varphi\in{X}^{*}$, $x_{0}\in{X}$, $\varepsilon>0$,
            then $\mathcal{U}$ forms a sub-basis for the weak
            topology on $X$. Moreover, the sets:
            \begin{equation}
                \mathcal{U}(\{\varphi_{1},\dots,\varphi_{n}\},
                    x_{0},\varepsilon)\}
                    =\{x\in{X}:|\varphi_{k}(x)-\varphi_{k}(x_{0})|
                        <\varepsilon,k\in\mathbb{Z}_{n}\}
            \end{equation}
            Form an pen neighborhood basis at $x_{0}$ in the weak
            topology.
        \end{theorem}
        \begin{proof}
            It suffices to prove the second assertion. Since open
            balls in $\mathbb{F}$ form a basis for the topology,
            the collection:
            \begin{equation}
                \varphi^{\minus{1}}(B_{r}(c))=
                    \{x:|\varphi(x)-c|<r\}
            \end{equation}
            Form a sub-basis for the weak topology. But if
            $x_{0}\in\{x:|\varphi(x)-c|<r\}$, and if
            $\varepsilon=r-|\varphi(x_{0})-c|$, then
            $|\varphi(x)-\varphi(x_{0})|<\varepsilon$. Thus:
            \begin{equation}
                x_{0}\in\{x:|\varphi(x)-\varphi(x_{0})|<\varepsilon\}
                    \subseteq\{x:|\varphi(x)-c|<r\}
            \end{equation}
            Therefore, etc.
        \end{proof}
        \begin{theorem}
            If $X$ is a normed vector space, and $\tau$ is the
            normed topology and $\tau_{w}$ is the weak
            topology, then $\tau_{w}\subseteq\tau$.
        \end{theorem}
        \begin{theorem}
            If $X$ is a normed vector space, if $\tau$ is the
            normed topology, if $\tau_{w}$ is the weak topology,
            and if $X$ is finite dimensional, then
            $\tau=\tau_{w}$.
        \end{theorem}
        \begin{theorem}
            If $X$ is a normed vector space, $\tau$ is the normed
            topology, $\tau_{w}$ is th weak topology, and if
            $X$ is infinite dimensional, then
            $\tau_{w}\ne\tau$.
        \end{theorem}
        In any topology space $X$, we say that a sequence $x_{n}$
        converges to $x$ if the sequence is eventually contained in
        any neighborhood of $x$.
        \begin{theorem}
            If $x_{n}$ is a sequence in a normed vector space,
            then $x_{n}\rightarrow{x}$ weakly if and only if
            $\varphi(x_{n})\rightarrow\varphi(x)$ for all
            $\varphi\in{X}^{*}$.
        \end{theorem}
        \begin{proof}
            For suppose $x_{n}\rightarrow{x}_{0}$ weakly. Then,
            for all $\varepsilon>0$, let:
            \begin{equation}
                \mathcal{U}(\varphi,x_{0},\varepsilon)=
                    \{x:|\varphi(x)-\varphi(x_{0})|<\varepsilon\}
            \end{equation}
            This is a weak neighborhood of $x_{0}$. Hence,
            $x_{n}$ is eventually in $\mathcal{U}$. Thus,
            $\varphi(x_{n})$ is eventually in
            $B_{\varepsilon}(\varphi(x_{0}))$, and thus
            $\varphi(x_{n})\rightarrow\varphi(x_{0})$.
            Now suppose $\varphi(x_{n})\rightarrow\varphi(x_{0})$
            for all $\varphi\in{X}^{*}$. Let $V$ be a weak
            neighborhood of $x_{0}$. Then for some
            $\varphi_{1},\dots,\varphi_{n}$, and $\varepsilon>0$:
            \begin{equation}
                x_{0}\in\mathcal{U}(
                \{\varphi_{1},\dots,\varphi_{n}\},x_{0},
                    \varepsilon\})
                =\{x:|\varphi_{k}(x)-\varphi_{k}(x_{0})|
                    <\varepsilon,k\in\mathbb{Z}_{n}\}
                \subseteq{V}
            \end{equation}
            Thus, $x_{n}$ is eventually in $V$.
        \end{proof}
        \begin{theorem}
            Every weakly convergent sequence is bounded with
            respect to the normed topology.
        \end{theorem}
        \begin{proof}
            Suppose $x_{n}\rightarrow{x}$ weakly and let
            $i:X\rightarrow{X}^{**}$ be the canonical map. Then,
            for all $\varphi\in{X}^{*}$:
            \begin{equation}
                i(x_{n}(\varphi)=\varphi(x_{n})
            \end{equation}
            And this is bounded, and thus
            $i(x_{n})$ is uniformly bounded by the Uniform
            Boundedness Principle. But $i$ is isometric, and
            thus $\norm{x_{n}}$ is bounded.
        \end{proof}
        \begin{lexample}
            Let $X=\ell^{2}$. Recall that if $x\in\ell^{2}$, then:
            \begin{equation}
                \norm{x}_{2}=
                    \sqrt{\sum_{n=1}^{\infty}|x_{n}|^{2}}
            \end{equation}
            Let $e_{n}$ be the usual basis:
            \begin{equation}
                e_{n}(k)=\delta_{nk}=
                    \begin{cases}
                        1,&n=k\\
                        0,&n\ne{k}
                    \end{cases}
            \end{equation}
            We have seen that $(\ell^{2})^{*}$ is isometric to
            $\ell^{2}$. That is, if $y\in\ell^{2}$, then
            $y$ corresponds to $\varphi_{y}$ where:
            \begin{equation}
                \varphi_{y}(x)=\sum_{n=1}^{\infty}
                    x_{n}y_{n}
            \end{equation}
            And every $\varphi\in(\ell^{2})^{*}$ can be written
            this way. There is a neighborhood basis for
            $0\in\ell^{2}$ in the weak topology given by the
            sets of the form:
            \begin{equation}
                \mathcal{U}=\{x\in\ell^{2}:
                    \sum_{i=1}^{n}|\varphi_{y_{i}}(x)|^{2}
                    <\varepsilon\}
            \end{equation}
            Now define $T$ as follows:
            \begin{equation}
                T=\{\sqrt{n}e_{n}:n\in\mathbb{N}\}
            \end{equation}
            We want to find the \textit{weak} closure of $T$, which
            is the closure of $T$ with respect to the weak topology.
            That is:
            \begin{equation}
                W=\overline{T}^{w}=
                \bigcap\{F\subseteq\ell^{2}:
                    T\subseteq{F},F^{C}\in\tau_{w}\}
            \end{equation}
            Where $F^{C}$ denotes the complement with respect
            to the weak topology. Let's show that $0$ is an
            element of $W$. Suppose not. If $0\ne{W}$, then
            there is a $\mathcal{U}$ such that
            $\mathcal{U}\cap{T}=\emptyset$. But:
            \begin{equation}
                |\varphi_{y_{i}}(\sqrt{k}e_{k})|=
                \sqrt{k}|y_{i}(k)|
            \end{equation}
            Thus, for all $k\in\mathbb{N}$:
            \begin{equation}
                \sum_{i=1}^{n}|y_{i}(k)|^{2}\geq
                \frac{\varepsilon}{k}
            \end{equation}
            But then:
            \begin{equation}
                \sum_{i=1}^{n}\norm{y_{i}}_{2}^{2}
                =\sum_{i=1}^{n}\sum_{k=1}^{\infty}
                    |y_{i}(k)|^{2}
                =\sum_{k=1}^{\infty}\sum_{i=1}^{n}
                    |y_{i}(k)|^{2}\geq\sum_{k=1}^{\infty}
                        \frac{\varepsilon}{k}
            \end{equation}
            But this sum diverges, a contradiction as the functions
            are in $\ell^{2}$. Thus, $0\in{W}$.
        \end{lexample}
        Note that any sequence $x_{n}\subseteq{T}$ that converges
        to zero must be bounded. But $\{\sqrt{n}e_{n}:n\leq{n}\}$
        is weakly closed. But no sequence in $T$ can converge weakly
        to zero, and thus the weak topology is not metrizable.
        As it turns out, the weak topology is not even
        first countable. We like sequences and want to continue
        using this notion, but cannot use this for the weak
        topology on infinite dimensional normed spaces. We
        generalize by describing nets.
        \begin{ldefinition}{Directed Ordered Sets}
              {Funct_Analysis_Directed_Ord_Set}
            A directed ordered set is an ordered set
            $(\Lambda,\leq)$ such that, for all $x,y\in\Lambda$,
            there is a $z\in\Lambda$ such that
            $x\leq{z}$ and $y\leq{z}$.
        \end{ldefinition}
        \begin{lexample}
            There are two common examples. Any totally ordered
            set is automatically a directed ordered set,
            since we can just choose the max of any two elements.
            Moreover, if $(X,\tau)$ is a topologicaly space, then
            $(\tau,\subseteq)$ is a directed ordered set, since
            for $\mathcal{U}$ and $\mathcal{V}$, then
            $\mathcal{U}\cup\mathcal{V}$ is larger than both.
            Similarly with reverse containment, taking
            $\mathcal{U}\cap\mathcal{V}$ as the
            \textit{larger} set.
        \end{lexample}
        \begin{ldefinition}{Nets}
            A net on a set $X$ is a function from a directed
            set $\Lambda$ into $X$, $x:\Lambda\rightarrow{X}$.
        \end{ldefinition}
        As with sequences, we denote the image of $\lambda\in\Lambda$
        under a net $x$ by writing $x(\lambda)=x_{\lambda}$.
        We say that a net converges in a topological space if
        for any neighborhood $\mathcal{U}$ of the limit point
        $x_{0}$, there is a $\lambda_{0}\in\Lambda$ such that,
        for all $\lambda\geq\lambda_{0}$, $x_{\lambda}$, we have
        $x_{\lambda}\in\mathcal{U}$. An accumulation point of
        $x_{\lambda}$. There is also a generalized notion of
        accumulation point for nets.
        \begin{lexample}
            Every sequence is a net. Take $\Lambda=\mathbb{N}$.
        \end{lexample}
        \begin{theorem}
            If $(X,\tau)$ is a topological space, and
            $\mathcal{E}\subseteq{X}$, then
            $x\in\overline{\mathcal{E}}$ if and only if there is
            a net $a:\Lambda\rightarrow\mathcal{E}$ such that
            $a_{\lambda}\rightarrow{x}$.
        \end{theorem}
        \begin{proof}
            For suppose $x_{\lambda}\rightarrow{x}$ with
            $x_{\lambda}\in\mathcal{E}$. If
            $x\notin\overline{\mathcal{E}}$, there there is a
            $\mathcal{U}\subseteq\mathcal{O}(x)$ such that
            $\mathcal{U}\cap\mathcal{U}=\emptyset$. But
            $x_{\lambda}$ is eventually in $\mathcal{U}$,
            a contradiction. Thus, etc. This direction of
            the proof is identical for sequences, and indeed holds
            for sequences. Going the other way requires the use
            of the notion of nets. Suppose
            $x\in\overline{\mathcal{U}}$. Let
            $\Lambda$ be the set of neighborhoods of $x$. Then
            this is a directed set. Pick points to get a net.
        \end{proof}
    \section{More Normed Vector Space Stuff}
        Given a normed vector space $X$, the dual $X^{*}$ is
        a Banach space. It has a normed topology and a weak
        topology given by $X^{**}$. We can also give it the
        weak star topology: $\sigma(X^{*},X)$. This is the
        smallest (Or initial) topology determined by the
        functionals in $X$. That is, elements of $X^{*}$
        are continuous. In other words, for all
        $x\in{X}$, the mapping $\varphi\mapsto\varphi(x)$
        is continuous. Recall that a net $(\varphi_{\lambda})$
        in $X^{*}$ converges to $\varphi$ in the weak star
        topology if and only if it converges point-wise.
        That is, $\varphi_{\lambda}(x)\rightarrow\varphi(x)$
        for all $x\in{X}$.
        \begin{ltheorem}{Alaoglu's Theorem}
            If $X$ is a normed vector space, if
            $B{*}$ is the closed unit disc in the dual space:
            \begin{equation}
                B^{*}=\{\varphi\in{X}^{*}:
                    \norm{\varphi}\leq{1}\}
            \end{equation}
            Then $B^{*}$ is compact in the weak star
            topology.
        \end{ltheorem}
        \begin{proof}
            Define $D_{r}$ by:
            \begin{equation}
                D_{r}=\{z\in\mathbb{F}:|z|\leq{r}\}
            \end{equation}
            For all $r>0$. Define:
            \begin{equation}
                Z=\prod_{x\in{X}}D_{\norm{x}}
            \end{equation}
            Then, by Tychonoff's theorem, $Z$ is compact in
            the product topology. But this topology is the
            initial topology determined by the projection
            maps. And a net $(z_{\lambda})\rightarrow{z}$ in
            $Z$ if and only if $z_{\lambda}(x)\rightarrow{z}(x)$
            for all $x\in{X}$. Define
            $j:B^{*}\rightarrow{Z}$ by:
            \begin{equation}
                j(\varphi)(x)=\varphi(x)
            \end{equation}
            Then $j$ is an injective map. Moreover, $j$ has
            closed range in $Z$. For suppose $\varphi_{\lambda}$
            is a net and $j(\varphi_{\lambda})\rightarrow{z}$,
            for some $z\in{Z}$. But then, for all $x\in{X}$,
            $j(\varphi_{\lambda})(x)\rightarrow{z}(x)$, and thus
            $\varphi_{\lambda}(x)\rightarrow{z}(x)$. Thus $z$
            is linear, $z(x+y)=z(x)+z(y)$ and
            $z(\alpha{x})=\alpha{z}(x)$. Thus, $\varphi(x)=z(x)$
            and $|\varphi(x)|\leq\norm{x}$, and therefore
            $\varphi\in{B}^{*}$. Therefore $j(B^{*})$ is compact.
            Moreover, $j$ is homeomorphism between
            $B^{*}$ and $j(B^{*})$, and thus $B^{*}$ is compact.
        \end{proof}
    \section{Hilbert Spaces}
        \begin{ldefinition}{Sesqui-Linear Form}
            A Sesqui-Linear form on a vector space $V$ over a
            field $\mathbb{F}$ is a function
            $\langle{\cdot|\cdot}\rangle$ such that:
            \begin{equation}
                \langle{\alpha{x}+y|z}\rangle
                =\alpha\langle{x,z}\rangle+\langle{y,z}\rangle
                \quad\quad
                x,y,z\in{V}
                \quad
                \alpha\in\mathbb{F}
            \end{equation}
            And such that:
            \begin{equation}
                \langle{x,\alpha{y}+z}\rangle
                =\overline{\alpha}\langle{x|y}\rangle
                +\langle{x|z}\rangle
               \quad\quad
                x,y,z\in{V}
                \quad
                \alpha\in\mathbb{F}
            \end{equation}
        \end{ldefinition}
        \begin{ldefinition}{Self-Adjoint Sesqui-Linear Form}
            A self-adjoint sesqui-linear form on a vector space
            $V$ over a field $\mathbb{F}$ is a sesqui-linear
            form $\langle{\cdot|\cdot}\rangle$ such that:
            \begin{equation}
                \langle{x|y}\rangle=\overline{\langle{y|x}\rangle}
                \quad\quad
                x,y\in{V}
            \end{equation}
        \end{ldefinition}
        \begin{ldefinition}{Postive Sesqui-Linear Form}
            Positive if $\langle{x|x}\rangle\geq{0}$.
        \end{ldefinition}
        \begin{theorem}
            If $\mathbb{F}=\mathbb{C}$ and
            $\langle{\cdot|\cdot}\rangle$ is a sesqui-linear
            form, then:
            \begin{equation}
                \langle{x|y}\rangle
                =\frac{1}{4}\sum_{n=0}^{3}
                    i^{n}\langle{x+i^{n}y|x+i^{n}y}\rangle
            \end{equation}
        \end{theorem}
        \begin{theorem}
            If $\mathbb{F}=\mathbb{C}$, and
            $\langle{\cdot|\cdot}\rangle$ is a sesqui-linear
            form, then it is self-adjoint if and only if
            it is positive.
        \end{theorem}
        Thus, on a complex vector space, a positive sesqui-linear
        form is always self-adjoint.
        \begin{ldefinition}{Pre-Inner Product}
            A pre-inner product is a positive self-adjoint
            sesqui-linear form.
        \end{ldefinition}
        \begin{ldefinition}{Inner Product}
            An inner product is a pre-inner product such that
            $\langle{x|x}\rangle=0$ implies that $x=0$.
        \end{ldefinition}
        \begin{ldefinition}{Induced Semi-Norm}
            Given a pre-inner product, the induced norm is:
            \begin{equation}
                \norm{v}=\sqrt{\langle{x|y}\rangle}
            \end{equation}
        \end{ldefinition}
        If $\mathbb{F}=\mathbb{R}$, then:
        \begin{equation}
            \langle{x|y}\rangle=\norm{x+y}^{2}-\norm{x-y}^{2}
        \end{equation}
        \begin{ltheorem}{Cauchy-Schwarz Inequality}
            If $\langle{\cdot|\cdot}\rangle$ is a pre-inner product
            and $\norm{\cdot}$ is the induced semi-inner product,
            then:
            \begin{equation}
                |\langle{x|y}\rangle|\leq\norm{x}\norm{y}
            \end{equation}
            A similar result holds for $\mathbb{C}$.
        \end{ltheorem}
        \begin{proof}
            For:
            \begin{equation}
                0\leq\norm{\alpha{x}+y}^{2}
                =\langle{\alpha{x}+y|\alpha{x}+y}\rangle
                =|\alpha|^{2}\norm{x}^{2}
                +\alpha\langle{x|y}\rangle
                +\overline{\alpha\langle{x|y}\rangle}+\norm{y}^{2}
            \end{equation}
            We can simplify this further to obtain:
            \begin{equation}
                0\leq|\alpha|^{2}+2\Re(\alpha\langle{x|y}\rangle)
                    +\norm{y}^{2}
            \end{equation}
            Let $\tau\in\mathbb{F}$ be such that
            $\tau\langle{x|y}\rangle=|\langle{x|y}\rangle|$.
            Let $\alpha=t\tau$, with $t\in\mathbb{R}$. Then:
            \begin{equation}
                0\leq{t}^{2}\norm{x}^{2}+2t|\langle{x|y}\rangle
                    +\norm{y}^{2}
            \end{equation}
            But this is a quadratic with a positive discriminant,
            and thus by the quadratic formula:
            \begin{equation}
                4\langle{x|y}\rangle^{2}-4\norm{x}^{2}\norm{y}^{2}
                \leq{0}
            \end{equation}
            Therefore, etc. Smiley face.
        \end{proof}
        \begin{theorem}
            If $\langle{\cdot|\cdot}\rangle$ is a pre-inner
            product, then the induced semi-norm is a semi-norm.
        \end{theorem}
        \begin{proof}
            Apply Cauchy-Schwarz to obtain the triangle inequality.
        \end{proof}
        \begin{ldefinition}{Induced Norm}
            An induced norm is a semi-induced norm induced
            by an inner product.
        \end{ldefinition}
        \begin{theorem}
            Induced norms are norms.
        \end{theorem}
        \begin{ldefinition}{Inner Product Space}
            An inner product space Is s vector space
            $V$ over a field $\mathbb{F}$ with an
            inner product $\langle{\cdot|\cdot}\rangle$,
            denoted $(V,\langle{\cdot|\cdot}\rangle)$.
        \end{ldefinition}
        \begin{ldefinition}{Hilbert Space}
            A Hilbert space is an inner product space
            such that the induced norm is complete.
        \end{ldefinition}
        \begin{lexample}
            Let $\mathcal{H}=\mathbb{F}^{n}$ and define:
            \begin{equation}
                \langle{x|y}\rangle
                =\sum_{k=1}^{n}x_{k}\overline{y}_{k}
            \end{equation}
            Then $\langle{\cdot|\cdot}\rangle$ is an inner product.
            The induced norm is $\ell^{2}$, which is complete.
            Thus this is a Hilbert space. Extending this to
            sequences:
            \begin{equation}
                \langle{x|y}\rangle
                =\sum_{k=0}^{\infty}x_{k}\overline{y}_{k}
            \end{equation}
            Similarly, we can define:
            \begin{equation}
                \langle{f|g}\rangle
                =\int_{X}f(x)\overline{g}(x)\diff{\mu}
            \end{equation}
            For $f,g\in{L}^{2}(X,\mathcal{M},\mu)$. Remembering
            to identity functions that differ on only a set
            of measure zero, this is a Hilbert space.
        \end{lexample}
        \begin{theorem}
            If $H$ is an inner product space, and if
            $x,y\in{H}$, then:
            \begin{equation}
                \norm{x+y}^{2}+\norm{x-y}^{2}=
                2\norm{x}^{2}+2\norm{y}^{2}
            \end{equation}
        \end{theorem}
        This theorem characterizes inner product spaces.
        \begin{ltheorem}{Jordan von-Neumann Theorem}
            If $X$ is a complex Banach space such that:
            \begin{equation}
                \norm{x+y}^{2}+\norm{x-y}^{2}
                =2\norm{x}^{2}+2\norm{y}^{2}
            \end{equation}
            Then there is an inner product that induces
            the norm.
        \end{ltheorem}
        \begin{theorem}
            If $H$ is an inner product space and
            $x_{n}\rightarrow{x}$ and
            $y_{n}\rightarrow{y}$, then:
            \begin{equation}
                \langle{x_{n}|y_{n}}\rangle
                \rightarrow\langle{x|y}\rangle
            \end{equation}
        \end{theorem}
        \begin{proof}
            Recall that the norm is continuous, so
            $\norm{x_{n}}\rightarrow\norm{x}$ and
            $\norm{y}_{n}\rightarrow\norm{y}$. By Cauchy-Schwarz:
            \begin{equation}
                |\langle{x_{n}|y_{n}}\rangle-\langle{x|y}\rangle|
                \leq\langle{x_{n}-x|y_{n}}\rangle|+
                    |\langle{x|y-y_{n}}\rangle|
                \leq\norm{x-x_{n}}\norm{y_{n}}+
                    \norm{x}\norm{y-y_{n}}
                \rightarrow{0}
            \end{equation}
            Therefore, etc.
        \end{proof}
        \begin{ldefinition}{Orthogonal Elements}
              {Funct_Analysis_Orthogonal_Elements}
            Orthogonal elements of an inner product space
            $(H,\langle{\cdot|\cdot}\rangle)$ are points
            $x,y\in{H}$ such that
            $\langle{x|y}\rangle=0$.
        \end{ldefinition}
        We can define a similar notion for subsets of $H$.
        \begin{ltheorem}{Pythagoras' Theorem}
            If $x_{1},\dots,x_{n}$ are pairwise orthogonal
            elements of an inner product space $H$, then:
            \begin{equation}
                \sum_{k=1}^{n}\norm{x_{k}}^{2}=
                \norm{\sum_{k=1}^{n}x_{k}}^{2}
            \end{equation}
        \end{ltheorem}
        \begin{proof}
            For:
            \begin{align}
                \norm{\sum_{k=1}^{n}x_{k}}^{2}
                &=\langle{\sum_{k=1}^{n}x_{k}|\sum_{k=1}^{n}x_{k}}
                    \rangle\\
                &=\sum_{k=1}^{n}\sum_{j=1}^{n}
                    \langle{x_{k}|x_{j}}\rangle
                &=\sum_{k=1}^{n}\langle{x_{k}|x_{k}}\rangle\\
                &=\sum_{k=1}^{n}\norm{x_{k}}^{2}
            \end{align}
            Therefore, etc.
        \end{proof}
        \begin{theorem}
            I $C$ is a closed non-empty convex subset of a Hilbert
            space $H$, then for all $y$ there is a unique
            $x\in{C}$ such that $\dist(y,C)=\norm{x-y}$.
        \end{theorem}
        Next time, on Functional analysis:
        Direct sum, proof of the previous theorem, the fact
        that there may not be a further element. Stuff.
    \section{Even More Stuff}
        \begin{theorem}
            If $C$ is a closed non-empty convex subset of a
            Hilbert space $\mathcal{H}$, and if
            $h\in\mathcal{H}$, then there is a unique
            $c\in{C}$ such that:
            \begin{equation}
                \dist(h,C)=\norm{y-h}
            \end{equation}
            That is, there is a unique $x$ in $C$ that is
            closest to $h$.
        \end{theorem}
        \begin{proof}
            Translate $C$ by $C-h$, so we can assume $h=0$.
            Define the following:
            \begin{equation}
                \alpha=\inf\{\norm{x}:x\in{C}\}
            \end{equation}
            That is, $\alpha=\dist(0,C)$. Let $x_{n}$
            be a sequence in $C$ such that
            $\norm{x_{n}}\rightarrow\alpha$. Then, by the
            parallelogram law:
            \begin{equation}
                2(\norm{x_{n}}^{2}+\norm{x_{m}}^{2})=
                \norm{x_{n}+x_{m}}^{2}+
                \norm{x_{n}-x_{m}}^{2}
            \end{equation}
            But then, for all $n,m\in\mathbb{N}$:
            \begin{equation}
                \frac{x_{n}+x_{m}}{2}\in{C}
            \end{equation}
            Since $C$ is convex. Therefore:
            \begin{equation}
                2(\norm{x_{n}}^{2}+\norm{x_{m}}^{2})
                \geq4\alpha^{2}+\norm{x_{n}-x_{m}}^{2}
            \end{equation}
            And thus $x_{n}$ is Cauchy. Then
            $x_{n}\rightarrow{x}$ and $\norm{x}=\alpha$.
            Moreover, from convexity, $x$ is unique.
        \end{proof}
        \begin{ldefinition}{Orthogonal Complement}
            The orthogonal complement of a subset
            $S\subseteq{H}$ of an inner product space
            $H$ is the set:
            \begin{equation}
                S^{\perp}=\{y\in{H}:\forall_{x\in{A}},
                    \langle{x|y}\rangle=0\}
            \end{equation}
        \end{ldefinition}
        \begin{theorem}
            If $H$ is an inner product space and
            $S\subseteq{H}$, then $S^{\perp}$ is a closed
            subspace.
        \end{theorem}
        From linear algebra, if $W_{1}$ and $W_{2}$ are
        subsapces of a vector space $V$ such that
        $W_{1}+W_{2}=V$ and $W_{1}\cap{W}_{2}=\{0\}$, then
        we say $V=W_{1}\oplus{W}_{2}$. The map
        $P_{1}:V\rightarrow{V}$ defined by taking $v\in{V}$ to
        the unique $w_{1}\in{W}_{1}$ such that
        $v=w_{1}+w_{2}$ is called the projection of $H$
        onto $W_{1}$ along $W_{2}$.
        \begin{theorem}
            If $W$ is a closed subspace of a Hilbert space
            $\mathcal{H}$, then
            $\mathcal{H}=W\oplus{W}^{\perp}$. If
            $P_{W}:\mathcal{H}\rightarrow\mathcal{H}$ is the
            projection mapping of $H$ into $W$, then
            $P_{W}(h)$ is the closest element in $W$ to $h$.
        \end{theorem}
        \begin{proof}
            For let $h\in\mathcal{H}$ and let $x$ be the
            closed element in $W$ to $h$. Let
            $x^{\perp}=h-x$. Let $w\in{W}$ and
            $\varepsilon>0$. But then:
            \begin{subequations}
                \begin{align}
                    \norm{x^{\perp}}^{2}
                    &=\norm{h-x}^{2}\leq
                    \norm{h-(x+\varepsilon{w}}^{2}\\
                    &=\norm{h-x-\varepsilon{w}}^{2}\\
                    &=\norm{x^{\perp}-\varepsilon{w}}^{2}\\
                    &=\norm{x^{\perp}}^{2}-
                    2\varepsilon\Re(\langle{x^{\perp}|w}\rangle)
                    +\varepsilon^{2}\norm{w}^{2}
                \end{align}
            \end{subequations}
            Therefore:
            \begin{equation}
                2\varepsilon\Re(\langle{x^{\perp}|w}\rangle)
                =\varepsilon^{2}\norm{w}^{2}
            \end{equation}
            Thus, $x^{\perp}\in{W}^{\perp}$. But
            $\mathcal{H}=W+W^{\perp}$ and
            $W\cap{W}^{\perp}=\{0\}$.
            Therefore, $W\oplus{W}^{+}=\mathcal{H}$
        \end{proof}
        \begin{theorem}
            If $\mathcal{H}$ is a Hilbert space and if
            $S\subseteq\mathcal{H}$, then:
            \begin{equation}
                (S^{\perp})^{\perp}
                =\mathrm{Cl}_{\norm{\cdot}}\big(\Span(S)\big)
            \end{equation}
        \end{theorem}
        Note that given a point $h$ in an inner product space,
        $\varphi_{h}(x)=\langle{x|v}\rangle$ defins a linear
        function. By the Cauchy-Schwarz theorem:
        \begin{equation}
            |\varphi_{h}(h)|\leq\norm{h}\norm{v}
        \end{equation}
        And thus $\varphi_{v}\in{H}^{*}$ and
        $\norm{\varphi_{v}}\leq\norm{v}$. But:
        \begin{equation}
            |\varphi_{v}(v)|=\norm{v}^{2}
        \end{equation}
        And thus $\norm{\varphi_{v}}=\norm{v}$.
        \begin{ltheorem}{Riesz's Representation Theorem}
              {thm:Funct_Analysis_Riesz_Rep_Theorem}
            If $\mathcal{H}$ is a Hilbert space, then the
            map $\Phi:\mathcal{H}\rightarrow\mathcal{H}^{*}$
            given by $\Phi(v)=\varphi_{v}$ is a conjugate
            linear isometric map of $\mathcal{H}$ onto
            $\mathcal{H}^{*}$.
        \end{ltheorem}
        \begin{proof}
            By the previous remark, it suffices to show that
            $\Phi$ is surjective. Let $\varphi\in\mathcal{H}^{*}$.
            Let $W=\ker(\varphi)$. But $\varphi$ is continuous,
            and thus $W$ is a closed subspace of $\mathcal{H}$.
            If $\varphi$ is the zero function, let $v=0$. If not,
            then there exists a non-zero element
            $v\in(\ker(\varphi))^{\perp}$. Let
            $y=v/\norm{v}$.
        \end{proof}
        \begin{theorem}
            If $\mathcal{H}$ is a Hilbert space, then the
            bijection $\Phi:\mathcal{H}\rightarrow\mathcal{H}^{*}$
            mapping $v\mapsto\varphi_{v}$ is a homeomorphism
            of $\mathcal{H}$ with the weak topology onto
            $\mathcal{H}^{*}$ with the weak star topology.
        \end{theorem}
        \begin{proof}
            Recall every $\varphi\in\mathcal{H}^{*}$ is of the
            form $\varphi_{v}$ for some $v\in\mathcal{H}$.
            Thus $h_{\lambda}\mapsto{h}$ in the weak topology
            if and only if for all $v\in\mathcal{H}$:
            \begin{subequations}
                \begin{align}
                    \langle{h_{\lambda}|h}\rangle
                    &\rightarrow\langle{h|v}\rangle\\
                    \Longrightarrow
                    \langle{v|h_{\lambda}}\rangle
                    &\rightarrow\langle{v|h}\rangle\\
                    \Longrightarrow\varphi_{h_{\lambda}}(v)
                    &\rightarrow\varphi_{h}(v)
                \end{align}
            \end{subequations}
            But then $\varphi_{h_{\lambda}}\rightarrow\varphi_{h}$
            in the weak star topology.
        \end{proof}
        \begin{theorem}
            If $\mathcal{H}$ is a Hilbert space and if
            $B$ is the closed unit ball, then $B$ is
            weakly compact.
        \end{theorem}
        \begin{ldefinition}{Orthonormal Basis}
            An orthonormal basis of an inner product
            space $H$ is a subset $E\subseteq{H}$ such that, for
            all $e\in{E}$, $\norm{e}=1$, and for all distinct
            $e_{\alpha},e_{\beta}\in{E}$,
            $\langle{e_{\alpha}|e_{\beta}}\rangle=0$.
        \end{ldefinition}
        \begin{ltheorem}{Bessel's Inequality}
              {Funct_Analysis_Bessels_Inequality}
            If $H$ is an inner product, and if
            $e_{n}$ is an orthonormal sequence in $H$< then
            for all $x\in{H}$:
            \begin{equation}
                \sum_{n=1}^{\infty}|\langle{x|e_{n}}\rangle|^{2}
                    \leq\norm{x}^{2}
            \end{equation}
        \end{ltheorem}
        \begin{proof}
            For define the following:
            \begin{equation}
                x_{n}=x-\sum_{k=1}^{n}
                    \langle{x|e_{k}}\rangle{e}_{k}
            \end{equation}
            Note that $x_{n}\perp{e}_{k}$ for $k=1,\dots,n$.
            But, by the Pythagorean theorem:
            \begin{equation}
                \norm{x}^{2}=\norm{x_{n}}^{2}+
                    \sum_{k=1}^{n}|\langle{x|e_{k}}\rangle|^{2}
            \end{equation}
            Thus:
            \begin{equation}
                \norm{x}^{2}\geq\underset{n\in\mathbb{N}}{\sup}
                    \sum_{k=1}^{n}|\langle{x|e_{k}}\rangle|^{2}
                    =\sum_{k=1}^{\infty}|\langle{x|e_{k}}\rangle|^{2}
            \end{equation}
            Therefore, etc.
        \end{proof}
        \begin{ldefinition}{Orthogonal Projection}
              {def:Funct_Analysis_Orthogonal_Projection}
            The orthogonal projection of a closed subspace $W$
            of a Hilbert space $\mathcal{H}$ is the projection
            $P_{W}:\mathcal{H}\rightarrow\mathcal{H}$ of
            $W$ along $W^{\perp}$.
        \end{ldefinition}
        \begin{theorem}
            If $E$ is an orthonormal set in a Hilbert space
            $\mathcal{H}$, and if
            $\mathcal{E}=\mathrm{Cl}(\Span(E))$, then for all
            $h\in{H}$, the sum over
            $\langle{h|e_{n}}\rangle{e}_{n}$ converges and:
            \begin{equation}
                P_{\mathcal{E}}(h)=\sum_{n=1}^{\infty}
                    \langle{h|e_{n}}\rangle{e}_{n}
            \end{equation}
        \end{theorem}
    \section{Even MORE Stuff!}
        \begin{theorem}
            If $S$ and $T$ are self adjoint bounded operators
            on $\mathcal{H}$, and if $S\leq{T}$, then
            $ASA^{*}\leq{A}TA^{*}$ for all
            $A\in\mathscr{L}(\mathcal{H})$.
        \end{theorem}
        \begin{theorem}
            If $S,T$ are self-adjoint operators on $\mathcal{H}$,
            if $0\leq{S}\leq{T}$, then $\norm{S}\leq\norm{T}$.
        \end{theorem}
        \begin{proof}
            For suppose $0\leq{S}\leq{T}$, and let
            $[x,y]=\langle{Sx|y}\rangle$. Then $[\cdot,\cdot]$
            is a pre-inner product on $\mathcal{H}$, and thus
            if $\norm{x}=\norm{y}=1$, then by Cauchy-Schwarz:
            \begin{equation}
                |\langle{Sx|y}\rangle|^{2}=|[x,y]|^{2}
                \leq[x,x][y,y]=
                \langle{Sx|x}\rangle\langle{Sy|y}\rangle
            \end{equation}
            But $S\leq{T}$, and therefore:
            \begin{equation}
                \langle{Sx|x}\rangle\langle{Sy|y}\rangle
                \leq\langle{Tx|x}\rangle\langle{Ty|y}\rangle
                \leq\norm{T}^{2}
            \end{equation}
            Therefore $\norm{S}^{2}\leq\norm{T}^{2}$. Taking
            square roots completes the proof.
        \end{proof}
        \begin{theorem}
            If $S$ is a self-adjoint bounded operators on
            $\mathcal{H}$, and if $S\geq{0}$, then
            $S\leq{I}$ if and only if $\norm{S}\leq{1}$.
        \end{theorem}
        \begin{proof}
            For if $0\leq{S}\leq{U}$, then
            $\norm{S}\leq\norm{I}=1$. But if $0\leq{S}$ and
            $\norm{S}\leq{1}$, then:
            \begin{equation}
                \langle{Sx|x}\rangle\leq\norm{x}^{2}=
                \langle{x|x}\rangle
            \end{equation}
            Thus, $S\leq{I}$.
        \end{proof}
        \begin{theorem}
            If $T$ is a self-adjoint bounded operator on
            $\mathcal{H}$, if $\minus{I}\leq{T}\leq{I}$, then
            $\norm{T}\leq{1}$.
        \end{theorem}
        \begin{proof}
            Suppose $T=T^{*}$ and $\minus{I}\leq{T}\leq{I}$.
            Then:
            \begin{equation}
                \langle{T(x+y)|x+y}\rangle
                \leq\norm{x+y}^{2}
            \end{equation}
            And similarly:
            \begin{equation}
                \minus\langle{T(x-y)|x-y}\rangle
                \leq\norm{x-y}^{2}
            \end{equation}
            Summing, we obtain:
            \begin{equation}
                4\Re\big(\langle{Tx|y}\rangle\big)
                \leq\norm{x+y}^{2}+\norm{x-y}^{2}
            \end{equation}
            By the parallelogram law:
            \begin{equation}
                4\Re\big(\langle{Tx|y}\rangle\big)
                \leq{2}\norm{x}^{2}+2\norm{y}^{2}
            \end{equation}
            Therefore:
            \begin{equation}
                4|\langle{Tx|y}\rangle|
                \leq{2}\norm{x}^{2}+2\norm{y}^{2}
            \end{equation}
            But, for $\norm{x}=\norm{y}=1$,
            $\sup|\langle{Tx|y}\rangle|\leq{1}$, and therefore
            $\norm{T}\leq{1}$. On the other hand, if
            $T=T^{*}$, then:
            \begin{equation}
                |\langle{Tx|x}\rangle|\leq\norm{x}^{2}
            \end{equation}
            But $T=T^{*}$, and thus
            $\langle{Tx|x}\rangle$ is a real number. Thus:
            \begin{equation}
                \minus\langle{x|x}\rangle
                \leq\langle{Tx|x}\rangle
                \leq\langle{x|x}\rangle
            \end{equation}
            Thus, $\minus{I}\leq{T}\leq{I}$.
        \end{proof}
        Let $A\in{M}_{n}(\mathbf{F})^{\dagger}$. That is,
        $A=A^{*}$ and $A=\mathcal{U}\mathcal{D}\mathcal{U}^{*}$
        for some unitary $\mathcal{U}$ and a diagonal
        $\mathcal{D}$:
        \begin{equation}
            \mathcal{D}=
                \begin{pmatrix}
                    \lambda_{1}&0&\dots&0\\
                    0&\lambda_{2}&\dots&0\\
                    \vdots&\vdots&\ddots&0\\
                    0&0&\dots&\lambda_{n}
                \end{pmatrix}
        \end{equation}
        Where $\lambda_{k}\geq{0}$.
        \begin{theorem}
            There is a sequence o polynomials with positive
            coefficients such that:
            \begin{equation}
                \sum_{n=1}^{\infty}p_{n}(t)=1-\sqrt{1-t}
            \end{equation}
            Uniformly on $[0,1]$.
        \end{theorem}
        \begin{proof}
            Let $q_{0}=0$ and define:
            \begin{equation}
                q_{n+1}(t)=\frac{1}{2}\big(t+q_{n}(t)\big)^{2}
            \end{equation}
            For all $n\in\mathbb{N}$. By induction we see that
            $q_{n}$ is a sequence of polynomials with positive
            coefficients and such that:
            \begin{equation}
                0\leq{q}_{n}(t)\leq{1}
            \end{equation}
            For all $t\in[0,1]$, and $n\in\mathbb{N}$. Define:
            \begin{equation}
                p_{n}(t)=q_{n}(t)-q_{n-1}(t)
            \end{equation}
            For all $n\in\mathbb{N}$. But then:
            \begin{equation}
                2p_{n+1}(t)
                =q_{n}(t)^{2}-q_{n-1}(t)^{2}
                =\big(q_{n}(t)-q_{n-1}(t)\big)
                    \big(q_{n}(t)+q_{n-1}(t)\big)
                =p_{n}(t)\big(q_{n}(t)+q_{n-1}(t)\big)
            \end{equation}
            Thus, each $p_{n}$ has positive coefficients, and
            therefore:
            \begin{equation}
                q_{n}(t)\leq{q}_{n-1}(t)
            \end{equation}
            But $q_{n}$ is bounded by 1 and monotonic, and thus
            by completeness, there is a limit function. Let:
            \begin{equation}
                Q(t)=\underset{n\rightarrow\infty}{\lim}q_{n}(t)
            \end{equation}
            But then:
            \begin{equation}
                q(t)=\frac{1}{2}\big(t+q(t)^{2}\big)
            \end{equation}
            And therefore:
            \begin{equation}
                q(t)=1-\sqrt{1-t}
            \end{equation}
            Moreover, by Dini's theorem, the convergence is
            uniform. And the $p_{n}$ form a telescoping series,
            and therefore:
            \begin{equation}
                \sum_{n=1}^{\infty}p_{n}(t)=1-\sqrt{1-t}
            \end{equation}
            Therefore, etc.
        \end{proof}
        \begin{lexample}
            Define the following $2\times{2}$ matrices:
            \begin{equation}
                A=\begin{pmatrix}
                    2&1\\
                    1&1
                \end{pmatrix}
                \quad\quad
                B=\begin{pmatrix}
                    4&\minus{1}\\
                    \minus{1}&1
                \end{pmatrix}
            \end{equation}
            Then $A$ and $B$ are positive, but:
            \begin{equation}
                AB=\begin{pmatrix}
                    7&\minus{1}\\
                    2&1
                \end{pmatrix}
            \end{equation}
            And this is not symmetric, and thus not positive.
            The product of positive operators need not be
            positive.
        \end{lexample}
        \begin{theorem}
            If $S\geq{0}$, and for all $n\in\mathbb{N}$,
            $S^{n}\geq{0}$. In particular, if $p$ is a
            polynomial with positive coefficients, then
            $p(S)\geq{0}$.
        \end{theorem}
        \begin{proof}
            We have $(S^{n})^{*}=S^{n}$. Thus, yeah.
        \end{proof}
        \begin{theorem}
            If $T$ is a bounded operator, $T\geq{0}$, then
            there is a unique $A\in\mathscr{L}(\mathcal{H})$ such
            that $A\geq{0}$ and $A^{2}=T$. If $B$ commutes with
            $T$< then $B$ commutes with $A$.
        \end{theorem}
        \begin{proof}
            Let $\alpha>0$. If $A^{2}=\alpha{T}$, then
            $(\alpha^{\minus{1}/2}A)^{2}=T$. Thus we can replace
            $T$ by $\alpha{T}$ such that $\norm{T}\leq{1}$.
            Thus, $0\leq{T}\leq{I}$. Then, if $S=I-T$, then
            $0\leq{S}\leq{I}$. Let $p_{n}$ and $q_{n}$ be
            defined as before. Let:
            \begin{equation}
                S_{n}=p_{n}(S)
            \end{equation}
            Then $S_{n}\geq{0}$. Thus:
            \begin{equation}
                0\leq\sum_{k=m}^{n}p_{k}(t)=
                \sum\alpha_{r}t^{r}\leq\varepsilon
            \end{equation}
            For all $t\in[0,1]$. Note that $\alpha_{k}\geq{0}$.
            Hence, we have:
            \begin{equation}
                \norm{\sum_{k=m}^{n}S_{k}}
                \leq\sum\alpha_{r}\norm{S}^{r}\leq
                \sum\alpha{r}<\varepsilon
            \end{equation}
            Thus, $S_{k}$ forms a Cauchy sequence and therefore
            converges. Moreover:
            \begin{equation}
                R=\sum_{k=1}^{\infty}S_{k}\geq{0}
            \end{equation}
            Moreover, $0\leq{R}\leq{I}$. But:
            \begin{align}
                (I-R)^{2}
                &=\Big(I-\sum_{k=1}^{\infty}S_{k}\Big)^{2}\\
                &=\underset{n\rightarrow\infty}{\lim}
                    \big(I-q_{n}(S)\big)^{2}\\
                &=\underset{n\rightarrow\infty}{\lim}
                    \big(I-2q_{n}(S)+q_{n}(S)^{2}\big)\\
                &=I-S
            \end{align}
            And this is equal to $T$. Thus, $(I-R)^{2}=T$.
            Now, if $BT=TB$, then:
            \begin{equation}
                AB=(I-R)B
                =\underset{n\rightarrow\infty}{\lim}
                    \big(I-q_{n}(I-T)\big)B
                =B\underset{n\rightarrow\infty}{\lim}
                    \big(I-q_{n}(I-T)\big)
                =BA
            \end{equation}
            Lastly, $A$ is unique. For if $B\geq{0}$ and
            $B^{2}=T$, then $B$ commutes with $T$ and hence
            $B$ commutes with $A$. Therefore:
            \begin{equation}
                (A-B)^{2}x=(A-B)(A+B)x=(T-T)x=0
            \end{equation}
            Hence, if $y$ is in the range of $A+B$, then
            $(A-B)y=0$.Let $\mathcal{E}$ be the range of $A+B$.
            If we can show that $(A-B)y=0$ for all
            $y\in\mathcal{E}^{\perp}$, then we are done. But since
            $A$ and $B$ are self-adjoint, $\mathcal{E}^{\perp}$
            is the kernel of $A+B$. Thus:
            \begin{equation}
                \langle{Ay|y}\rangle\leq
                \langle{(A+B)z|z}\rangle=0
            \end{equation}
            Thus, for all $y\in\mathcal{E}^{\perp}$,
            $\langle{Ay|y}\rangle=0$. But $A\geq{0}$, hence there
            is a $C$ such that $A=C^{2}$ and $C\geq{0}$.
            Then:
            \begin{equation}
                \langle{Az|Z}\rangle=\norm{Cz}^{2}=0
            \end{equation}
            Thus $Cz=0$ and hence $Az=C^{2}z=0$. Similarly,
            $Bz=0$. Thus $(A-B)z=0$.
        \end{proof}
        \begin{theorem}
            If $T\geq{0}$ and $S\geq{0}$, and if
            $TS=ST$, then $TS\geq{0}$.
        \end{theorem}
        \begin{proof}
            For:
            \begin{equation}
                TS=T(\sqrt{S})^{2}=\sqrt{S}T\sqrt{S}\geq{0}
            \end{equation}
            Therefore, etc.
        \end{proof}
        Here, $A$ is called the positive square root of the
        operator $T$.
    \clearpage
    \printglossary[style=longpara]
    \chapter{Homeworks}
    \section{Homework I}
        \begin{problem}
            Show that, for $1\leq{p}\leq{q}\leq\infty$, that
            $\norm{\cdot}_{p}$ and $\norm{\cdot}_{q}$ are strongly
            equivalent.
        \end{problem}
        \begin{solution}[1]
            First, if $X$ is a set, $d_{1},d_{2},d_{3}$ are
            metrics on $X$, if $d_{1}$ is strongly equivalent to
            $d_{2}$, and $d_{2}$ is strongly equivalent to $d_{3}$,
            then $d_{1}$ is strongly equivalent to $d_{3}$.
            For if $d_{1}$ is strongly equivalent to $d_{2}$, then
            there are $\alpha,\beta>0$ such that, for all
            $x,y\in{X}$:
            \begin{equation}
                \alpha{d}_{1}(x,y)\leq{d}_{2}(x,y)
                \leq\beta{d}_{1}(x,y)
            \end{equation}
            But we have seen that strongly equivalent is a symmetric
            relation, and therefore if $d_{2}$ and $d_{3}$ are
            strongly equivalent then there are $a,b>0$ such that:
            \begin{equation}
                ad_{3}(x,y)\leq{d}_{2}(x,y)\leq{b}d_{3}(x,y)
            \end{equation}
            Therefore:
            \begin{equation}
                \frac{\alpha}{b}d_{1}(x,y)\leq{d}_{3}(x,y)
                \leq\frac{\beta}{a}d_{1}(x,y)
            \end{equation}
            Thus, strongly equivalent is a transitive relation.
            Let $n\in\mathbb{N}$, $p\in[1,\infty)$, and let
            $\mathbf{x}\in\mathbb{R}^{n}$. Then, since
            $\norm{\mathbf{x}}_{\infty}$ is the supremum norm,
            it is greater than or equal to all of the compononents
            of $\mathbf{x}$. Thus:
            \begin{equation}
                n\norm{\mathbf{x}}_{\infty}^{p}=
                \sum_{k=1}^{n}\norm{\mathbf{x}}_{\infty}^{p}
                \geq\sum_{k=1}^{n}|x_{k}|^{p}
                =\norm{\mathbf{x}}_{p}^{p}
            \end{equation}
            Taking $p^{th}$ roots, we have:
            \begin{equation}
                n^{\frac{1}{p}}\norm{\mathbf{x}}_{\infty}
                \geq\norm{\mathbf{x}}_{p}
            \end{equation}
            Going the other way:
            \begin{equation}
                \norm{\mathbf{x}}_{\infty}^{p}\leq
                \sum_{k=1}^{n}|x_{k}|^{p}
            \end{equation}
            Taking $p^{th}$ roots again, we obtain the following:
            \begin{equation}
                \norm{\mathbf{x}}_{\infty}\leq\norm{\mathbf{x}}_{p}
                \leq{n}^{\frac{1}{p}}\norm{\mathbf{x}}_{\infty}
            \end{equation}
            Thus, for all $p\in[1,\infty)$, $\norm{\cdot}_{p}$
            is strongly equivalent to $\norm{\cdot}_{\infty}$.
            But strongly equivalent is a transitive relation,
            thus for all $p,q\in[1,\infty]$, $\norm{\cdot}_{p}$
            is strongly equivalent to $\norm{\cdot}_{q}$.
        \end{solution}
        \begin{solution}[2]
            Let $p\in[1,\infty]$, and let Let
            $f:S^{n}\rightarrow\mathbb{R}$ be defined by:
            \begin{equation}
                f(\mathbf{x})=\norm{\mathbf{x}}_{p}
            \end{equation}
            But $S^{n}$ is a closed and bounded subset of
            $\mathbb{R}^{n}$, and is therefore, by the Heine-Borel
            theorem, it is compact. But continuous functions
            attain their maximum and minimum on compact sets, by
            the extreme value theorem. Thus, there are
            $\mathbf{x}_{\min}$ and $\mathbf{x}_{\max}$ such
            that, for all $\mathbf{x}\in{S}^{n}$:
            \begin{equation}
                \norm{\mathbf{x}_{\min}}_{p}\leq
                \norm{\mathbf{x}}_{p}\leq
                \norm{\mathbf{x}_{\max}}_{p}
            \end{equation}
            But also $\norm{\mathbf{x}_{\min}}_{p}>0$, for
            $\mathbf{x}_{\min}\in{S}^{n}$, and therefore
            $\mathbf{x}_{\min}\ne\mathbf{0}$. Moreover:
            \begin{equation}
                \norm{e_{i}}_{p}=1
            \end{equation}
            For all $i\in\mathbb{Z}_{n}$. But also, for all
            $\mathbf{x}\in{S}^{n}$, we have:
            \begin{equation}
                \norm{\mathbf{x}}_{2}=1
            \end{equation}
            Therefore, for all $\mathbf{x}\in{S}^{n}$:
            \begin{equation}
                \frac{\norm{\mathbf{x}_{\min}}_{p}}
                     {\norm{\mathbf{x}_{\max}}_{p}}
                \norm{\mathbf{x}}_{p}\leq
                \norm{\mathbf{x}}_{2}\leq
                \frac{\norm{\mathbf{x}_{\max}}_{p}}
                     {\norm{\mathbf{x}_{\min}}_{p}}
                \norm{\mathbf{x}}_{p}
            \end{equation}
            For the general
            $\mathbf{x}\in\mathbb{R}^{n}\setminus\{\mathbf{0}\}$,
            we can scale back to the unit $n$ sphere. Thus,
            $\norm{\cdot}_{p}$ and $\norm{\cdot}_{2}$ are
            strongly equivalent for all $p\in[1,\infty]$. Since
            strongly equivalent is a transitive relation,
            $\norm{\cdot}_{p}$ and $\norm{\cdot}_{q}$ are
            strongly equivalent for all $p,q\in[1,\infty]$.
        \end{solution}
        \begin{problem}
            Give an example of a metric on $\mathbb{R}^{n}$
            that is not strongly equivalent to $\norm{\cdot}_{p}$
            for any $p\in[1,\infty]$.
        \end{problem}
        \begin{solution}
            The discrete metric is not strongly equivalent to
            any of the metrics induced by $\norm{\cdot}_{p}$.
            For suppose not. Then:
            \begin{equation}
                \alpha\norm{\mathbf{x}}_{p}\leq
                d(\mathbf{x},\mathbf{0})\leq{1}
            \end{equation}
            Where $d$ is the discrete metric.
            But $\norm{\mathbf{x}}_{p}$ is unbounded, and thus if
            $\alpha\norm{\mathbf{x}}_{p}\leq{1}$, then $\alpha=0$.
            Thus there is no $\alpha>0$ that satisfies
            the inequality.
        \end{solution}
        \begin{problem}
            Show that, if $a,b\geq{0}$ and $0<\lambda<1$, then:
            \begin{equation}
                a^{\lambda}b^{1-\lambda}
                \leq\lambda{a}+(1-\lambda)b
            \end{equation}
        \end{problem}
        \begin{solution}
            If $a=0$ or $b=0$, then we are done. Suppose not.
            Define $t=ab^{\minus{1}}$.
            Then, if $\lambda\in(0,1)$, and $t\geq{1}$, we have:
            \begin{subequations}
                \begin{align}
                    \lambda(t^{\lambda-1}-1)&\geq0\\
                    \Rightarrow\int_{1}^{t}\lambda
                        \Big(\tau^{\lambda-1}-1\Big)\diff{\tau}
                        &\geq{0}\\
                    \Rightarrow
                    (t^{\lambda}-\lambda{t})-(1-\lambda)&\geq{0}
                \end{align}
            \end{subequations}
            For $t\in(0,1)$, we have:
            \begin{subequations}
                \begin{align}
                    \lambda(t^{\lambda-1}-1)&\leq{0}\\
                    \Rightarrow\int_{t}^{1}\lambda
                        \Big(\tau^{\lambda-1}-1\Big)\diff{\tau}
                        &\leq{0}\\
                    \Rightarrow
                    (1-\lambda)-(t^{\lambda}-\lambda{t})&\leq{0}
                \end{align}
            \end{subequations}
            Combining these two, we have for $t\in(0,\infty)$:
            \begin{equation}
                t^{\lambda}-\lambda{t}\leq{1}-\lambda
            \end{equation}
            But $t=ab^{\minus{1}}$, and thus multiplying through
            by $b$:
            \begin{equation}
                a^{\lambda}b^{\lambda-1}\leq
                \lambda{a}+(1-\lambda)b
            \end{equation}
        \end{solution}
        \begin{problem}
            Prove H\"{o}lder's Inequality: If $p\in(1,\infty)$,
            $p^{\minus{1}}+q^{\minus{1}}=1$, then:
            \begin{equation}
                \norm{fg}_{1}\leq\norm{f}_{p}\norm{g}_{q}
            \end{equation}
        \end{problem}
        \begin{solution}
            For let:
            \par
            \begin{subequations}
                \begin{minipage}[b]{0.49\textwidth}
                    \centering
                    \begin{equation}
                        \tilde{f}=\frac{f}{\norm{f}_{q}}
                    \end{equation}
                \end{minipage}
                \hfill
                \begin{minipage}[b]{0.49\textwidth}
                    \centering
                    \begin{equation}
                        \tilde{g}=\frac{g}{\norm{g}_{q}}
                    \end{equation}
                \end{minipage}
            \end{subequations}
            \par\hfill\par
            Then $\norm{\tilde{f}}_{p}=1$ and
            $\norm{\tilde{g}}_{q}=1$. But if $p\in(1,\infty)$,
            then $p^{\minus{1}}<1$, and thus
            by the previous problem, and since
            $1-p^{\minus{1}}=q^{\minus{1}}$, we have:
            \begin{equation}
                |\tilde{f}(x)\tilde{g}(x)|\leq
                    p^{\minus{1}}|\tilde{f}(x)|^{p}+
                    q^{\minus{1}}|\tilde{g}(x)|^{q}
            \end{equation}
            Integrating, we get:
            \begin{subequations}
                \begin{align}
                    \norm{\tilde{f}\tilde{g}}_{1}
                    &=\int_{\mathbb{R}}
                        |\tilde{f}(x)\tilde{g}(x)|\diff{x}\\
                    &\leq\int_{\mathbb{R}}\Big(
                        p^{\minus{1}}|\tilde{f}(x)|^{p}+
                        q^{\minus{1}}|\tilde{g}(x)|^{q}\Big)
                        \diff{x}\\
                    &=p^{\minus{1}}\norm{\tilde{f}}_{p}^{p}+
                    q^{\minus{1}}\norm{\tilde{g}}_{q}^{q}\\
                    &=p^{\minus{1}}+q^{\minus{1}}
                \end{align}
            \end{subequations}
            But $p^{\minus{1}}+q^{\minus{1}}=1$, and thus
            $\norm{\tilde{f}\tilde{g}}_{1}=1$. But from the
            definition of $\tilde{f}$ and $\tilde{g}$, we can
            multiply through by $\norm{f}_{p}\norm{g}_{q}$ to obtain:
            \begin{equation}
                \norm{fg}_{1}\leq\norm{f}_{p}\norm{g}_{q}
            \end{equation}
        \end{solution}
        \begin{problem}
            Prove Minkowski's Inequality
        \end{problem}
        \begin{problem}
            A limit point of a subspace $(E,d_{E})$ of a metric
            space $(X,d)$ is a point $x\in{X}$ such that there
            exists a sequence $a:\mathbb{N}\rightarrow{E}$ such
            that $a_{n}\rightarrow{x}$. Prove that $E$ is closed
            if and only if it has all of it's limit points.
        \end{problem}
        \begin{solution}
            Suppose $E$ is closed and let $x$ be a limit point
            of $E$. Suppose $x\in{E}^{C}$. But if $E$ is closed,
            then $E^{C}$ is open, and thus there is an $r>0$
            such that:
            \begin{equation}
                B_{r}^{(X,d)}(x)\subseteq{E}^{C}
            \end{equation}
            But if $x$ is a limit point of $E$ then there is a
            sequence $a:\mathbb{N}\rightarrow{E}$ such that
            $a_{n}\rightarrow{x}$. But if $a_{n}\rightarrow{x}$ then
            there is an $N\in\mathbb{N}$ such that, for all
            $n\in\mathbb{N}$ such that $n>N$, it is true that
            $d(x,a_{n})<r/2$. But then, for all $n>N$, we have that:
            \begin{equation}
                a_{n}\in{B}_{r}^{(X,d)}(x)
            \end{equation}
            And thus $a_{n}\in{E}^{C}$. But $a_{n}\in{E}$, a
            contradiction. Thus, $x\in{E}$. Now, suppose $x$ has
            all of it's limit points and suppose $E$
            is not closed. Then $E^{C}$ is not open. But then
            there is an $x\in{E}^{C}$ such that, for all
            $\varepsilon>0$:
            \begin{equation}
                B_{\varepsilon}^{(X,d)}(x)\cap{E}\ne\emptyset
            \end{equation}
            Define the following:
            \begin{equation}
                A_{n}=\Big\{y\in{E}:d(x,y)<\frac{1}{n}\Big\}
            \end{equation}
            Then for all $n\in\mathbb{N}$, $A_{n}$ is non-empty.
            By choice there is a sequence
            $a:\mathbb{N}\rightarrow{E}$ such that, for all
            $n\in\mathbb{N}$, $a_{n}\in{A}_{n}$. But then, for all
            $n\in\mathbb{N}$, $d(a_{n},x)<n^{\minus{1}}$. Thus
            $a_{n}\rightarrow{x}$ and therefore $x$ is a limit point
            of $E$. But $x\in{E}^{C}$, a contradiction as $E$
            contains all of its limit points. Therefore, $E$ is
            closed.
        \end{solution}
        \begin{problem}
            State and prove a result characterizing open sets in
            a metric space in terms of sequences, similar to
            the previous problem.
        \end{problem}
        \begin{solution}
            A subset $\mathcal{U}\subseteq{X}$ of a metric space
            $(X,d)$ is open if and only if for any convergence
            sequence $a:\mathbb{N}\rightarrow{X}$ such that the
            limit of $a$ is in $\mathcal{U}$, there is an
            $N\in\mathbb{N}$ such that, for all $n\in\mathbb{N}$ and
            $n>N$, it is true that $a_{n}\in\mathcal{U}$. For
            suppose $\mathcal{U}$ is open, and let
            $a:\mathbb{N}\rightarrow{X}$ be a convergent sequence
            such that there is an $x\in\mathcal{U}$ such that
            $a_{n}\rightarrow{x}$. But if $\mathcal{U}$ is open
            then there is an $\varepsilon>0$ such that:
            \begin{equation}
                B_{\varepsilon}^{(X,d)}(x)\subseteq\mathcal{U}
            \end{equation}
            But if $a_{n}\rightarrow{x}$ then there is an
            $N\in\mathbb{N}$ such that, for all $n\in\mathbb{N}$
            such that $n>N$, it is true that
            $d(x,a_{n})<\varepsilon$. But then for all $n>N$,
            $n\in\mathbb{N}$, we have
            $a_{n}\in{B}_{\varepsilon}^{(X,d)}(x)$, and thus
            $a_{n}\in\mathcal{U}$. Now suppose for any sequence
            that converges to a point in $\mathcal{U}$, the sequence
            is eventually contained within $\mathcal{U}$. Suppose
            $\mathcal{U}$ is not open. Then there is an
            $x\in\mathcal{U}$ such that, for all $\varepsilon>0$
            there is a $y\in\mathcal{U}^{C}$ such that
            $d(x,y)<\varepsilon$. Define the following:
            \begin{equation}
                A_{n}=
                \Big\{y\in\mathcal{U}^{C}:d(x,y)<\frac{1}{n}\Big\}
            \end{equation}
            Then for all $n\in\mathbb{N}$, $A_{n}$ is non-empty.
            By choice there is a sequence
            $a:\mathbb{N}\rightarrow\mathcal{U}^{C}$ such
            that $a_{n}\in{A}_{n}$. But then $a_{n}\rightarrow{x}$.
            But if $a_{n}\rightarrow{x}$, then there is an
            $N\in\mathbb{N}$ such that for all $n\in\mathbb{N}$ such
            that $n>N$ it is true that $a_{n}\in\mathcal{U}$, a
            contradiction. Therefore, $\mathcal{U}$ is open.
        \end{solution}
        \begin{problem}
            Let $\rho$ and $\sigma$ be metrics on $X$ and show that
            $\rho$ and $\sigma$ are equivalent if and only if
            they have the same convergent sequences.
        \end{problem}
        \begin{solution}
            For a sequence $a:\mathbb{N}\rightarrow{X}$ converges to
            $x\in{X}$ if and only if for all open subsets
            $\mathcal{U}\subseteq{X}$ such that $x\in\mathcal{U}$,
            there is an $N\in\mathbb{N}$ such that, for all
            $n\in\mathbb{N}$ and $n>N$, it is true that
            $a_{n}\in\mathcal{U}$. Going one way, if
            $a_{n}\rightarrow{x}$ then for all $\varepsilon>0$
            there is an $N\in\mathbb{N}$ such that for all
            $n\in\mathbb{N}$ and $n>N$, it is true that
            $d(x,a_{n})<\varepsilon$. Let $\mathcal{U}$ be an open
            subset such that $x\in\mathcal{U}$. But then there is
            an $\varepsilon>0$ such that:
            \begin{equation}
                B_{\varepsilon}^{(X,d)}(x)\subseteq\mathcal{U}
            \end{equation}
            But there is an $N\in\mathbb{N}$ such that, for all
            $n>N$, $n\in\mathbb{N}$, we have
            $a_{n}\in{B}_{\varepsilon}^{(X,d)}(x)$. Therefore
            $a_{n}\in\mathcal{U}$. Going the other way, let
            $a:\mathbb{N}\rightarrow{X}$ be a sequence such that,
            for every open set $\mathcal{U}\subseteq{X}$ such
            that $x\in\mathcal{U}$, there is an $N\in\mathbb{N}$
            such that, for all $n\in\mathbb{N}$ and $n>N$, it is
            true that $a_{n}\in\mathcal{U}$. Let:
            \begin{equation}
                A_{k}=B_{k^{\minus{1}}}^{(X,d)}(x)
            \end{equation}
            Given $\varepsilon>0$ there is a $k\in\mathbb{N}$
            such that $k^{\minus{1}}<\varepsilon$. But $A_{k}$ is
            open and $x\in{A}_{k}$, and thus there is an
            $N\in\mathbb{N}$ such that for all $n>N$ and
            $n\in\mathbb{N}$, we have that $a_{n}\in{A}_{k}$. But
            then $d(x,a_{n})<\varepsilon$, so $a_{n}\rightarrow{x}$.
            This converts the notion of convergence from a metric
            space property to a topological property. If $(X,\rho)$
            and $(X,\sigma)$ are equivalent then
            they have the same open sets, and thus convergence of
            sequences is preserved. For suppose
            $a:\mathbb{N}\rightarrow{X}$ converges to $x$ with
            respect to $\rho$. Then, for all open subsets
            $\mathcal{U}$ of $(X,d)$ such that $x\in\mathcal{U}$,
            there is an $N\in\mathbb{N}$ such that, for all
            $n\in\mathbb{N}$ and $n>N$, it is true that
            $a_{n}\in\mathcal{U}$. But if $\mathcal{U}$ is open in
            $(X,\rho)$, then it is open in $(X,\sigma)$, for the
            two metrics are equivalent. Thus $a_{n}\rightarrow{x}$
            with respect to $\sigma$.
        \end{solution}
        \begin{problem}
            Let $(X,\mathcal{M},\mu)$ be a measure space and define
            $\mathcal{U}\subseteq{L}^{1}(X)$ by:
            \begin{equation}
                \mathcal{U}=
                    \Big\{f\in{L}^{1}(X):
                        \int_{X}\Re(f)\diff{\mu}<1\Big\}
            \end{equation}
            Prove that $\mathcal{U}$ is open with respect to the
            metric induced by $\norm{\cdot}_{1}$.
        \end{problem}
        \begin{solution}
            For let $f\in{L}^{1}(X)$ and let $M=\norm{f}_{1}$. As
            $f\in{L}^{1}(X)$, $M<\infty$. Define:
            \begin{equation}
                \alpha=\int_{X}\Re(f)\diff{\mu}
            \end{equation}
            And let $\varepsilon=\min\{1/2M,1-\alpha\}$.
            Then if $\norm{f-g}_{1}<\varepsilon$ we have:
            \begin{equation}
                \int_{X}\Re(g)\diff{\mu}=
                \int_{X}\Re(g-f+f)\diff{\mu}=
                \int_{X}\Re(g-f)\diff{\mu}+\int_{X}\Re(f)\diff{\mu}
            \end{equation}
            We can simplify this further to get:
            \begin{equation}
                \int_{X}\Re(g)\diff{\mu}
                =\int_{X}\Re(g-f)\diff{\mu}+\alpha
                <\varepsilon+\alpha\leq{1}
            \end{equation}
            Therefore:
            \begin{equation}
                B_{\varepsilon}^{(L^{1}(X),\norm{\cdot}_{1})}(f)
                \subseteq\mathcal{U}
            \end{equation}
            And thus $\mathcal{U}$ is open.
        \end{solution}
        \begin{problem}
            For a metric space $(X,d)$, define
            $\dist:X\times\mathcal{P}(X)\setminus\{\emptyset\}%
             \rightarrow[0,\infty)$ by:
            \begin{equation}
                \dist(x,A)=\inf\{d(x,y):y\in{A}\}
            \end{equation}
            Prove that $\dist(x,A)=0$ if and only if
            $x\in\overline{A}$. Show that, for a fixed non-empty
            $A\subseteq{X}$, $\dist(x,A)$ is continuous. Prove that
            if $A,B\subseteq{X}$ are closed disjoint non-empty
            subsets, then there is a continuous function
            $f:X\rightarrow[0,1]$ such that $f(x)=0$
            if and only if $x\in{A}$ and $f(y)=1$ if
            and only if $x\in{B}$.
        \end{problem}
        \begin{solution}
            If $x\in\overline{A}$, then for all $\varepsilon>0$:
            \begin{equation}
                B_{\varepsilon}^{(X,d)}(x)\cap{A}\ne\emptyset
            \end{equation}
            Thus $\dist(x,A)<\varepsilon$ for all positive
            $\varepsilon$, and therefore $\dist(x,A)=0$. If
            $\dist(x,A)=0$ then for all $\varepsilon>0$ there is a
            $y\in{A}$ such that $d(x,y)<\varepsilon$. Therefore
            $x$ is a limit point of $A$, and thus $x\in\overline{A}$.
            \par\hfill\par
            Let $\varepsilon>0$ and let $x\in{X}$, and let
            $\delta=\varepsilon$. Then:
            \begin{subequations}
                \begin{align}
                    \dist(y,A)&=\inf\{d(y,z):z\in{a}\}\\
                    &\leq\inf\{d(x,y)+d(x,z):z\in{A}\}\\
                    &=d(x,y)+\inf\{d(x,z):z\in{A}\}\\
                    &=d(x,y)+\dist(x,A)
                \end{align}
            \end{subequations}
            Similarly:
            \begin{equation}
                \dist(x,A)\leq{d}(x,y)+\dist(y,A)
            \end{equation}
            And therefore, if $d(x,y)<\varepsilon$:
            \begin{equation}
                \big|\dist(x,A)-\dist(y,A)\big|\leq{d}(x,y)
                <\varepsilon
            \end{equation}
            Finally, let:
            \begin{equation}
                f(x)=\frac{\dist(x,B)}{\dist(x,A)+\dist(x,B)}
            \end{equation}
            As $A$ and $B$ are disjoint and closed, $f(x)$ is well
            defined for all $x\in{X}$. Moreover,
            $0\leq{f(x)}\leq{1}$. If $f(x)=0$, then $\dist(x,B)=0$,
            and thus $x\in\overline{B}$. But $B$ is closed, and
            therefore $x\in{B}$. If $f(x)=1$, then
            $\dist(x,A)=0$, and thus $x\in\overline{A}$. Again, as
            $A$ is closed, $x\in{A}$. But $\dist(x,B)$ is
            continuous, and $\dist(x,A)+\dist(x,B)$ is
            continuous, and the quotient of continuous functions
            is continuous. Thus, $f$ is continuous.
        \end{solution}
        \begin{problem}
            Show that a Cauchy sequence with a convergent
            subsequence is convergent.
        \end{problem}
        \begin{solution}
            For let $a:\mathbb{N}\rightarrow{X}$ be a Cauchy
            sequence and let $k:\mathbb{N}\rightarrow\mathbb{N}$ be
            such that $a\circ{k}$ is a convergent subsequence, and
            let $x$ be the limit. That is, $k$ is a stricly
            monotonically increasing sequence of natural numbers.
            Let $\varepsilon>0$. Then there is an
            $N_{1}\in\mathbb{N}$ such that, for all $k_{n}>N_{1}$,
            $n\in\mathbb{N}$, it is true that
            $d(x,a_{k_{n}})<\varepsilon/2$. But $a$ is
            Cauchy, and thus there is an $N_{2}\in\mathbb{N}$
            such that, for all $n,m\in\mathbb{N}$ such that
            $n,m>N_{2}$, it is true that
            $d(a_{n},a_{m})<\varepsilon/2$. Let
            $N=\max\{N_{1},N_{2}\}$. Then for all $n>N$,
            $n\in\mathbb{N}$, $k_{n}>N$ since $k$ is increasing,
            and thus:
            \begin{equation}
                d(a_{n},x)\leq
                d(a_{n},a_{k_{n}})+d(a_{k_{n}},x)<\varepsilon
            \end{equation}
            Therefore, $a_{n}\rightarrow{x}$.
        \end{solution}
        \begin{problem}
            Let $F:\mathbb{N}\times{X}\rightarrow\mathbb{C}$ be a
            sequence of continuous functions and let
            $f:X\rightarrow\mathbb{C}$ be such that
            $F_{n}(x)\rightarrow{f}$ uniformly. Show that $f$ is
            continuous.
        \end{problem}
        \begin{solution}
            For let $\varepsilon>0$. As $F_{n}\rightarrow{f}$
            uniformly, there is an $N\in\mathbb{N}$ such that for
            all $x\in{X}$, it is true that:
            \begin{equation}
                |F_{N}(x)-f(x)|<\frac{\varepsilon}{3}
            \end{equation}
            But $F_{N}(x)$ is continuous, and thus for $x\in{X}$
            there is a $\delta>0$ such that $d(x,x_{0})<\delta$
            implies that:
            \begin{equation}
                |F_{N}(x)-F_{N}(x_{0})|<\frac{\varepsilon}{3}
            \end{equation}
            But then:
            \begin{subequations}
                \begin{align}
                    |f(x)-f(x_{0})|&\leq
                    |f(x)-F_{N}(x)|+|F_{N}(x)-F_{N}(x_{0})|+
                    |F_{N}(x_{0})-f(x_{0})|\\
                    &<\varepsilon
                \end{align}
            \end{subequations}
            Thus, $f$ is continuous.
        \end{solution}
        \begin{problem}
            Let $X$ be a metric space. Recall that we say
            $f:X\rightarrow\mathbb{C}$ is bounded if
            $\norm{f}_{\infty}<\infty$. A sequence of functions
            $f_{n}:X\rightarrow{D}$ is uniformly bounded if there
            is an $M$ such that $\norm{f_{n}}_{\infty}<M$ for
            all $n$. Also, $f_{n}$ is uniformly Cauchy if for
            all $\varepsilon>0$ there is an $N\in\mathbb{N}$
            such that $n,m>N$ implies
            $|f_{n}(x)-f_{m}(x)|<\varepsilon$ for all $x\in{X}$.
            Show that a uniformly Cauchy sequence $f_{n}$ of
            bounded functions is uniformly bounded.
        \end{problem}
        \begin{solution}
            For let $F:\mathbb{N}\times{X}\rightarrow\mathbb{C}$
            be a sequence of functions such that, for all
            $n\in\mathbb{N}$, there is an $M_{n}\in\mathbb{R}^{+}$
            such that:
            \begin{equation}
                \norm{F_{n}}_{\infty}<M_{n}
            \end{equation}
            And such that $F$ is uniformly Cauchy. Let
            $\varepsilon=1$. Then, as $F$ is uniformly Cauchy,
            there is an $N\in\mathbb{N}$ such that, for all
            $n,m\in\mathbb{N}$ such that $n,m>N$, and for all
            $x\in{X}$, it is true that:
            \begin{equation}
                |F_{n}(x)-F_{m}(x)|<\varepsilon
            \end{equation}
            Then, for all $n>N$ and for all $x\in{X}$:
            \begin{subequations}
                \begin{align}
                    |F_{n}(x)|&=
                    |F_{n}(x)+F_{N+1}(x)-F_{N+1}(x)|\\
                    &\leq|F_{n}(x)-F_{N+1}(x)|+|F_{N+1}(x)|\\
                    &<\varepsilon+M_{N+1}\\
                    &=M_{N+1}+1
                \end{align}
            \end{subequations}
            Let:
            \begin{equation}
                M=\max\Big(
                    \{M_{n}:n\in\mathbb{Z}_{N}\}\cup\{M_{N+1}+1\}
                \Big)
            \end{equation}
            Then for all $n\in\mathbb{N}$, and for all $x\in{X}$:
            \begin{equation}
                |F_{n}(x)|\leq{M}
            \end{equation}
            Therefore, $F$ is uniformly bounded.
        \end{solution}
        \begin{problem}
            We say that $D$ is dense in $X$ if $\overline{D}=X$.
            Show that $D$ is dense if and only if $D$ meets every
            non-empty open subset.
        \end{problem}
        \begin{solution}
            Suppose $D$ is a dense subset of $X$ and let
            $\mathcal{U}\subseteq{X}$ be an open subset. Suppose
            $\mathcal{U}\cap{D}=\emptyset$. Let $x\in\mathcal{U}$. As
            $\mathcal{U}$ is open, there is an $r>0$ such that:
            \begin{equation}
                B_{r}^{(X,d)}(x)\subseteq\mathcal{U}
            \end{equation}
            But if $D$ is dense in $X$, then $x$ is a limit point
            of $D$. Thus there is a sequence
            $a:\mathbb{N}\rightarrow{D}$ such
            that $a_{n}\rightarrow{x}$. But if $a_{n}\rightarrow{x}$,
            then there is an $N\in\mathbb{N}$ such that, for all
            $n\in\mathbb{N}$ and $n>N$, we have:
            \begin{equation}
                a_{n}\in{B}_{r}^{(X,d)}(x)
            \end{equation}
            But then, $a_{n}\in\mathcal{U}$, a contradiction as
            as $a_{n}\in{D}$ and $\mathcal{U}$ and $\mathcal{D}$
            are disjoint. Therefore, etc. Now suppose
            $D$ meets every open set. Suppose
            $x\notin\overline{D}$. Define the following:
            \begin{equation}
                A_{n}=\Big\{y\in{B}_{1/n}^{(X,d)}(x):y\in{D}\Big\}
            \end{equation}
            Then $A_{n}$ is non-empty for all $n\in\mathbb{N}$,
            since $D$ meets every open set. By choice there is
            a sequence $a:\mathbb{N}\rightarrow{D}$ such
            that $a_{n}\in{A}_{n}$ for all $n\in\mathbb{N}$. But
            then $x$ is a limit point of $D$, a contradiction.
            Thus, $\overline{D}=X$.
        \end{solution}
        \begin{problem}
            Let $(x_{n})$ be a sequence in a complete metric
            space $(X,\rho)$. Suppose that
            $\rho(x_{n},x_{n+1})<2^{\minus{n}}$ for all
            $n\in\mathbb{N}$. Conclude that $(x_{n})$ is
            convergent. What if instead we have that
            $\rho(x_{n},x_{n+1})<1/n$?
        \end{problem}
        \begin{solution}
            For let $\varepsilon>0$. Let $N\in\mathbb{N}$ such that
            $2^{1-N}<\varepsilon$. But then for $n,m>N$:
            \begin{equation}
                \rho(x_{n},x_{m})\leq
                \sum_{k=\min(n,m)}^{\max(n,m)}d(x_{k},x_{k+1})
                \leq\sum_{k=N}^{\infty}d(x_{k},x_{k_{n+1}})
                \leq\sum_{k=N}^{\infty}\frac{1}{2^{k}}
            \end{equation}
            But by applying the geometric series, we have:
            \begin{equation}
                \sum_{k=N}^{\infty}\frac{1}{2^{k}}
                =2^{1-N}<\varepsilon
            \end{equation}
            Thus $x_{n}$ is Cauchy, and Cauchy sequences
            converge in a complete metric space. Therefore, etc.
            If we replace $2^{\minus{n}}$ with $n^{\minus{1}}$,
            the result may not hold. For let $X=\mathbb{R}$, which
            is complete with the standard metric, and let
            $a:\mathbb{N}\rightarrow\mathbb{R}$ be defined
            by $a_{n}=\ln(n)$. Applying some calculus, we have:
            \begin{equation}
                d(a_{n+1},a_{n})=\ln(n+1)-\ln(n)
                =\int_{n}^{n+1}\frac{1}{x}\diff{x}<\frac{1}{n}
            \end{equation}
            But $\ln(n)$ is not a convergent sequence.
        \end{solution}
        \begin{problem}
            A metric space is separable if it has a countable dense
            subset. Show that a metric space $X$ is separable if
            and only if there is a countable family $\mathcal{D}$
            of open sets such that every open set in $X$ is the union
            of open sets in $\mathcal{D}$.
        \end{problem}
        \begin{solution}
            For let $(X,d)$ be a separable metric space, and let
            $A$ be a countable dense subset. Let:
            \begin{equation}
                \mathcal{D}=\bigcup_{n\in\mathbb{N}}
                \bigcup_{a\in{A}}B_{n^{\minus{1}}}^{(X,d)}(a)
            \end{equation}
            Then $\mathcal{D}$ is countable, and
            for all $\mathcal{U}\in\mathcal{D}$, $\mathcal{U}$
            is open. Let $\mathcal{O}$ be an open subset of $X$.
            Define:
            \begin{equation}
                \mathcal{V}=\{\mathcal{U}\in\mathcal{D}:
                    \mathcal{U}\subseteq\mathcal{O}\}
            \end{equation}
            Then:
            \begin{equation}
                \bigcup_{\mathcal{U}\in\mathcal{V}}\mathcal{U}
                \subseteq\mathcal{O}
            \end{equation}
            Suppose the converse is false, and let
            $x\in\mathcal{O}$ be such that it is not contained
            in this union. But $\mathcal{O}$ is open, and thus
            there is an $r>0$ such that:
            \begin{equation}
                B_{r}^{(X,d)}(x)\subseteq\mathcal{O}
            \end{equation}
            But by the Archimedean principle, there is an
            $n\in\mathbb{N}$ such that $n^{\minus{1}}<r/2$. But $A$
            is dense in $\mathcal{O}$ and thus there is a $y\in{A}$
            such that $d(x,y)<n^{-1}$. But then:
            \begin{equation}
                x\in{B}_{n^{\minus{1}}}^{(X,d)}(y)
                \subseteq{B}_{r}^{(X,d)}(x)\subseteq\mathcal{O}
            \end{equation}
            And thus $x$ is contained in this union, a contradiction.
            Therefore, etc. Going the other way, suppose $(X,d)$
            is a metric space such that there exists a countable set
            $\mathcal{D}$ of open subsets of $X$ such that, for all
            open sets $\mathcal{O}$, $\mathcal{O}$ is the union
            of elements of $\mathcal{D}$. That is, there is a sequence
            $A:\mathbb{N}\rightarrow\mathcal{P}(X)$ such that:
            \begin{equation}
                \mathcal{D}=\{A_{n}:n\in\mathbb{N}\}
            \end{equation}
            But then by choice there is a sequence:
            \begin{equation}
                a:\mathbb{N}\rightarrow\bigcup_{n\in\mathbb{N}}A_{n}
            \end{equation}
            Such that, for all $n\in\mathbb{N}$, $a_{n}\in{A}_{n}$.
            Let $y\in{X}$ and let $\varepsilon>0$, define:
            \begin{equation}
                \mathcal{V}=B_{\varepsilon}^{(X,d)}(y)
            \end{equation}
            But then $\mathcal{V}$ is an open subset of
            $(X,d)$ and is therefore the union of elements of
            $\mathcal{D}$. That is, there is a sequence
            $k:\mathbb{N}\rightarrow\mathbb{N}$ such that:
            \begin{equation}
                \mathcal{V}=
                \bigcup_{n\in\mathbb{N}}A_{k_{n}}
            \end{equation}
            Where we write $k_{n}$ to denote $k(n)$. But then:
            \begin{equation}
                d(y,a_{k_{n}})<\varepsilon
            \end{equation}
            For all $n\in\mathbb{N}$. Thus the set:
            \begin{equation}
                \mathcal{A}=\{a_{n}:n\in\mathbb{N}\}
            \end{equation}
            Is a a countable dense subset, and $(X,d)$ is separable.
        \end{solution}
    \section{Homework II}
        \begin{problem}
            Show that $X$ is compact if and only if every collection
            of closed sets $\mathcal{F}$ with the finite intersection
            property is such that:
            \begin{equation}
                \bigcap_{F\in\mathcal{F}}F\ne\emptyset
            \end{equation}
        \end{problem}
        \begin{solution}
            For suppose $X$ is compact and suppose there is a
            collection $\mathcal{F}$ of closed sets with the finite
            intersection property such that:
            \begin{equation}
                \bigcap_{F\in\mathcal{F}}F=\emptyset
            \end{equation}
            But, for all $F\in\mathcal{F}$, $F$ is closed, and
            therefore $F^{C}$ is open. But then:
            \begin{equation}
                X=\emptyset^{C}
                =\Big(\bigcap_{F\in\mathcal{F}}F\Big)^{C}
                =\bigcup_{F\in\mathcal{F}}F^{C}
            \end{equation}
            And thus:
            \begin{equation}
                \mathcal{O}=
                    \{F^{C}:F\in\mathcal{F}\}
            \end{equation}
            Is an open cover of $X$. But $X$ is compact, and
            therefore there is a finite subcover
            $\Delta\subseteq\mathcal{O}$. But then:
            \begin{equation}
                \emptyset=
                X^{C}=\Big(\bigcup_{\mathcal{U}\in\Delta}
                    \mathcal{U}\Big)^{C}=
                    \bigcap_{\mathcal{U}\in\Delta}
                    \mathcal{U}^{C}
            \end{equation}
            But $\mathcal{U}^{C}\in\mathcal{F}$ for all
            $\mathcal{U}\in\Delta$. And $\Delta$ is finite. Thus
            there is a finite subset of $\mathcal{F}$ such that
            the intersection is empty, a contradiction as
            $\mathcal{F}$ has the finite intersection property.
            Therefore, etc. Now suppose $X$ is such that every
            collection of closed sets $\mathcal{F}$ with the
            finite intersection property is such that the
            intersection over all elements is non-empty. Suppose
            $X$ is not compact. Then there is an open cover
            $\mathcal{O}$ such that there is no finite subcover.
            Let:
            \begin{equation}
                \mathcal{F}=\{\mathcal{U}^{C}:
                    \mathcal{U}\in\mathcal{O}\}
            \end{equation}
            But then for any finite subset, the intersection is
            non-empty. For if not then $\mathcal{O}$ has a finite
            subcover, which it does not. But then $\mathcal{F}$ is
            a collection of closed sets with
            the finite intersection property, and therefore:
            \begin{equation}
                \bigcap_{F\in\mathcal{F}}F\ne\emptyset
            \end{equation}
            But:
            \begin{equation}
                \emptyset=X^{C}=\Big(
                    \bigcup_{\mathcal{U}\in\mathcal{O}}\mathcal{U}
                \Big)^{C}
                =\bigcap_{F\in\mathcal{F}}F
            \end{equation}
            A contradiction. Therefore, $X$ is compact.
        \end{solution}
        \begin{problem}
            Show that $E\subseteq{X}$ is totally bounded if and
            only if there is a finite $\varepsilon\textrm{-net}$
            for all $\varepsilon>0$.
        \end{problem}
        \begin{solution}
            If $E\subseteq{X}$ is totally bounded, then for all
            $\varepsilon>0$ there are finitely many points
            $x_{k}$, $k\in\mathbb{Z}_{n}$ such that:
            \begin{equation}
                E\subseteq\bigcup_{k=1}^{n}
                    B_{\varepsilon}^{(X,d)}(x_{k})
            \end{equation}
            But then:
            \begin{equation}
                \mathcal{O}=\{B_{\varepsilon}^{(X,d)}(x_{k}):
                    k\in\mathbb{Z}_{n}\}
            \end{equation}
            Is a finite $\varepsilon\textrm{-net}$ of $E$. If,
            for all $\varepsilon>0$, there is a finite
            $\varepsilon\textrm{-net}$ of $E$, then there are
            finitely many points $x_{k}$, $k\in\mathbb{Z}_{n}$
            such that:
            \begin{equation}
                \mathcal{O}=\{B_{\varepsilon}^{(X,d)}(x_{k}):
                    k\in\mathbb{Z}_{n}\}
            \end{equation}
            Is an open cover of $E$. But then for all
            $\varepsilon>0$ there are finitely many open balls that
            cover $E$, and therefore $E$ is totally bounded.
        \end{solution}
        \begin{problem}
            Suppose $(X,d_{X})$ is compact and that
            $f:(X,d_{X})\rightarrow(Y,d_{Y})$ is continuous.
            Show that $f(X)$ is compact.
        \end{problem}
        \begin{solution}
            For let $\mathcal{O}$ be an open cover of $f(X)$.
            But $f$ is continuous, and thus for all
            $\mathcal{U}\in\mathcal{O}$,
            $f^{\minus{1}}(\mathcal{U})$ is an open subset of
            $X$. But then:
            \begin{equation}
                \Delta=\{f^{\minus{1}}(\mathcal{U}):
                    \mathcal{U}\in\mathcal{O}\}
            \end{equation}
            Is an open cover of $X$. But $X$ is compact, and thus
            there is a finite sub-cover $\Lambda$. But then:
            \begin{equation}
                \mathscr{O}=
                \{\mathcal{U}:f^{\minus{1}}(\mathcal{U})\in\Lambda\}
            \end{equation}
            Is is a finite subcover of $f(X)$, and therefore
            $f(X)$ is compact.
        \end{solution}
        \begin{problem}
            Let $X=(0,1)$ and let $\delta_{x}>0$ be such that:
            \begin{equation}
                y\in{B}^{(X,||)}_{\delta_{x}}(x)
                \Longrightarrow
                \Big|\frac{1}{x}-\frac{1}{y}\Big|<1
            \end{equation}
            Show that:
            \begin{equation}
                \mathcal{O}=
                \{B^{(X,||)}_{\delta_{x}}(x):x\in{X}\}
            \end{equation}
            Has no Lebesgue number.
        \end{problem}
        \begin{solution}
            Suppose not, and let $d>0$ be a Lebesgue number. Then
            for all $x\in(0,1)$, there is a
            $\mathcal{U}\in\mathcal{O}$ such that:
            \begin{equation}
                B_{d}^{(X,||)}(x)\subseteq\mathcal{U}
            \end{equation}
            Let $n\in\mathbb{N}$ be such that $n^{\minus{1}}<d$.
            Let $x=n^{\minus{1}}/2$. Then $x\in(0,1)$.
            But $d>n^{\minus{1}}$, and thus:
            \begin{equation}
                B_{d}^{(X,||)}(x)=(0,x+d)
            \end{equation}
            But since $x\in(0,1)$,
            there is a $y\in(0,1)$ such that:
            \begin{equation}
                B_{d}^{(X,||)}(x)\subseteq
                B_{\delta_{y}}^{(X,||)}(y)
            \end{equation}
            But then for all $z\in(0,x+d)$, we have:
            \begin{equation}
                \Big|\frac{1}{z}-\frac{1}{y}\Big|<1
            \end{equation}
            Let $N\in\mathbb{N}$ be such that
            $N>x^{\minus{1}}+y^{\minus{1}}+2$.
            But then $N^{\minus{1}}\in(0,x+d)$, and thus:
            \begin{equation}
                \Big|\frac{1}{N^{\minus{1}}}-\frac{1}{y}\Big|<1
            \end{equation}
            But:
            \begin{equation}
                \Big|\frac{1}{N^{\minus{1}}}-\frac{1}{y}\Big|
                =|N-y^{\minus{1}}|>2
            \end{equation}
            A contradiction. Thus, $d$ is not a Lebesgue number.
        \end{solution}
        \begin{problem}
            Show that a compact metric space has a countable
            dense subset.
        \end{problem}
        \begin{solution}
            For if $(X,d)$ is compact, then it is complete and
            totally bounded. But if it is totally bounded, for all
            $n\in\mathbb{N}$ there exists an $N\in\mathbb{N}$ and
            a sequence $a:\mathbb{Z}_{N}\rightarrow{X}$ such that:
            \begin{equation}
                X=\bigcup_{k=1}^{N}B_{n^{\minus{1}}}^{(X,d)}(a_{k})
            \end{equation}
            Define the following:
            \begin{equation}
                A_{n}=\bigcup_{N\in\mathbb{N}}
                    \Big\{a:\mathbb{Z}_{N}\rightarrow{X}:
                    X=\bigcup_{k=1}^{N}
                    B_{n^{\minus{1}}}^{(X,d)}(a_{k})\Big\}
            \end{equation}
            Then, for all $n\in\mathbb{N}$, $A_{n}$ is non-empty.
            Then by choice there is a sequence:
            \begin{equation}
                f:\mathbb{N}\rightarrow
                \bigcup_{n\in\mathbb{N}}A_{n}
            \end{equation}
            Such that, for all $n$, $f_{n}\in{A}_{n}$. Let:
            \begin{equation}
                \mathcal{D}=\bigcup_{n\in\mathbb{N}}
                    \textrm{Im}(f_{n})
            \end{equation}
            Where $\textrm{Im}$ denotes the image of $f_{n}$. From
            construction, for all $n\in\mathbb{N}$,
            $\textrm{Im}(f_{n})$ is finite, and thus $\mathcal{D}$
            is the countable union of countable sets, and is
            therefore countable. Moreover,
            $\overline{\mathcal{D}}=X$. For let $x\in{X}$ and let
            $\varepsilon>0$. By the
            Archimedean property, there and an $n\in\mathbb{N}$
            such that $n^{\minus{1}}<\varepsilon$. But:
            \begin{equation}
                X=\bigcup_{y\in{f_{n}}}
                    B_{n^{\minus{1}}}^{(X,d)}(y)
            \end{equation}
            And thus there is a $y\in{f}_{n}$ such that
            $d(x,y)<n^{\minus{1}}$. But if $y\in{f}_{n}$, the
            $y\in\mathcal{D}$. Thus, $x\in\overline{\mathcal{D}}$.
            Therefore $\mathcal{D}$ is a countable dense subset.
        \end{solution}
        \begin{problem}
            Show that the family of functions $\mathcal{F}$ defined
            on $[0,1]$ by $f_{n}(x)=x^{n}$,
            is equicontinuous at each $x\in[0,1)$. 
        \end{problem}
        \begin{solution}[1]
            If $F:\mathbb{N}\times{X}\rightarrow{Y}$ is a
            sequence of continuous functions such that
            $F_{n}\rightarrow{f}$ uniformly, then
            $F$ is point-wise equicontinuous. For let $\varepsilon>0$
            and let $x\in{X}$. But $F_{n}\rightarrow{f}$ unifomly,
            and $F_{n}$ is continuous for all $n\in\mathbb{N}$,
            and therefore $f$ is continuous. But then there is
            a $\delta_{1}>0$ such that, for all $x_{0}\in{X}$
            such that $d_{X}(x,x_{0})<\delta_{1}$, we have that:
            \begin{equation}
                d_{Y}\big(f(x),f(x_{0})\big)
                <\frac{\varepsilon}{3}
            \end{equation}
            But $F_{n}\rightarrow{f}$ uniformly, and thus there is
            an $N\in\mathbb{N}$ such that, for all
            $n>N$ and $n\in\mathbb{N}$, it is true that:
            \begin{equation}
                d_{Y}\big(F_{n}(x),F_{n}(x_{0})\big)
                <\frac{\varepsilon}{3}
            \end{equation}
            But then, for all $n>N$, and for all $x_{0}\in{X}$
            such that $d_{X}(x,x_{0})<\delta_{1}$, we have that:
            \begin{equation}
                \begin{split}
                    d_{Y}\big(F_{n}(x),F_{n}(x_{0})\big)
                    \leq{d}_{Y}&\big(F_{n}(x),f(x)\big)+\\
                    &d_{Y}\big(f(x),f(x_{0})\big)+
                    d_{Y}\big(f(x_{0}),F_{n}(x_{0})\big)<\varepsilon
                \end{split}
            \end{equation}
            But $F$ is continuous, and thus for all
            $n\in\mathbb{Z}_{N}$ there is a $\delta_{n}$ such that
            $d_{X}(x,x_{0})<\delta_{n}$ implies that:
            \begin{equation}
                d_{Y}\big(F_{n}(x),F_{n}(x_{0})\big)<\varepsilon
            \end{equation}
            Let:
            \begin{equation}
                \delta=\min\Big(\{\delta_{0}\}\cup
                    \{\delta_{n}:n\in\mathbb{Z}_{N}\}\Big)
            \end{equation}
            Now, for all $x_{0}<1$, $f_{n}(x)=x^{n}$ tends
            to zero uniformly on $[0,x_{0}]$. Therefore, etc.
        \end{solution}
        \begin{solution}[2]
            For let $\varepsilon>0$, and let $x\in[0,1)$. If
            $x=0$, Let
            $\delta=\varepsilon\min\{\varepsilon,\tfrac{1}{2}\}$.
            Then, for $0\leq{x}_{0}<\delta$, and for all
            $n\in\mathbb{N}$:
            \begin{equation}
                \big|x_{0}^{n}\big|<
                \delta^{n}\leq\varepsilon\Big(\frac{1}{2}\Big)^{n}
                <\varepsilon
            \end{equation}
            Otherwise, let $\delta_{1}=\tfrac{1}{2}\min\{x,1-x\}$
            and let $y=\delta_{1}+x$. Then $0<y<1$. By the mean
            value theorem, for all $x_{0}$ there is a
            $c_{x_{0}}$ such that $|x-c_{x_{0}}|<|x-x_{0}|$ and
            such that:
            \begin{equation}
                \big|x^{n}-x_{0}^{n}\big|
                =nc_{x_{0}}^{n-1}|x-x_{0}|
            \end{equation}
            But then:
            \begin{equation}
                \big|x^{n}-x_{0}^{n}\big|<
                ny^{n}\delta
            \end{equation}
            But, since $0<y<1$, $ny^{n}$ is bounded. For let
            $f:[0,\infty)\rightarrow\mathbb{R}$ be defined by:
            \begin{equation}
                f(x)=\frac{x}{y^{1-x}}
            \end{equation}
            Then by L'H\"{o}pital, we have:
            \begin{equation}
                \underset{x\rightarrow\infty}{\lim}f(x)
                =\underset{x\rightarrow\infty}{\lim}
                    \frac{x}{y^{1-x}}
                =\underset{x\rightarrow\infty}{\lim}
                    \frac{\minus{1}}{y^{1-x}\ln(y)}
                =\underset{x\rightarrow\infty}{\lim}
                    \frac{\minus{y}^{x-1}}{\ln(y)}
            \end{equation}
            But $0<y<1$, and therefore $y^{x-1}\rightarrow{0}$
            as $x\rightarrow{0}$. Therefore:
            \begin{equation}
                \underset{x\rightarrow\infty}{\lim}f(x)=0
            \end{equation}
            But then for any sequence
            $a:\mathbb{N}\rightarrow\mathbb{R}$ such that
            $a_{n}\rightarrow\infty$, we have
            $f(a_{n})\rightarrow{0}$. Therefore,
            $ny^{n-1}$ converges to zero. But convergent sequences
            are bounded sequences, and therefore there is an
            $M\in\mathbb{R}^{+}$ such that, for all
            $n\in\mathbb{N}$:
            \begin{equation}
                \big|ny^{n-1}\big|\leq{M}
            \end{equation}
            Let $\delta=\min\{\tfrac{\varepsilon}{M},\delta_{1}\}$.
            Then for all $x_{0}\in(0,1)$ such that
            $|x-x_{0}|<\delta$, we have:
            \begin{equation}
                \big|x^{n}-x_{0}^{n}\big|=
                nc_{x_{0}}^{n-1}|x-x_{0}|<
                ny^{n-1}\delta<\varepsilon
            \end{equation}
        \end{solution}
        \begin{problem}
            Show that an equicontinuous family of functions on a
            compact metric space is uniformly equicontinuous.
        \end{problem}
        \begin{solution}
            For let $(X,d_{X})$ be a compact metric space and let
            $\mathcal{F}$ be a family of equicontinuous functions
            to a metric space $(Y,d_{Y})$.
            Let $\varepsilon>0$. Then, as $\mathcal{F}$ is
            equicontinuous, for all $x\in{X}$ there exists a
            $\delta_{x}>0$ such that, for all $f\in\mathcal{F}$,
            we have:
            \begin{equation}
                x_{0}\in{B}_{\delta_{x}}^{(X,d_{X})}(x)
                \Longrightarrow
                f(x_{0})\in{B}_{\varepsilon/2}^{(Y,d_{Y})}
                \big(f(x)\big)
            \end{equation}
            But then:
            \begin{equation}
                \mathcal{O}=\Big\{
                    B_{\delta_{x}}^{(X,d_{X})}(x):x\in{X}\Big\}
            \end{equation}
            Is an open cover of $X$. But $(X,d)$ is compact, and
            therefore this cover has a Lebesgue number $\delta>0$.
            If $x\in{X}$, then there is a $y\in{X}$ such that:
            \begin{equation}
                B_{\delta}^{(X,d_{X})}(x)\subseteq
                B_{\delta_{y}}^{(X,d_{X})}(y)
            \end{equation}
            But then, if $d_{X}(x,x_{0})<\delta$, then:
            \begin{equation}
                x_{0}\in{B}_{\delta}^{(X,d_{X})}(x)
                \Rightarrow
                x_{0}\in{B}_{\delta_{y}}^{(X,d_{X})}(y)
                \Rightarrow
                f(x_{0})\in
                B_{\varepsilon/2}^{(Y,d_{Y})}\big(f(y)\big)
            \end{equation}
            And therefore:
            \begin{equation}
                d_{Y}\big(f(x),f(x_{0})\big)\leq
                d_{Y}\big(f(x),f(y)\big)+
                d_{Y}\big(f(y),f(x_{0})\big)<\varepsilon
            \end{equation}
            Thus, $\mathcal{F}$ is uniformly equicontinuous.
        \end{solution}
        \begin{problem}
            Show that a subset of a compact metric space is compact
            if and only if it is closed.
        \end{problem}
        \begin{solution}
            For let $(X,d)$ be a compact metric space and let
            $(E,d_{E})$ be a compact subspace. Suppose $E$ is not
            closed. Then $E^{C}$ is not open, and therefore
            there is an $x\in{E}^{C}$ such that, for all
            $\varepsilon>0$:
            \begin{equation}
                B_{\varepsilon}^{(X,d)}(x)\cap
                E\ne\emptyset
            \end{equation}
            Let:
            \begin{equation}
                \mathcal{O}=\Big\{\textrm{Cl}\Big(
                    B_{\varepsilon}^{(X,d)}(x)\Big)^{C}:
                    \varepsilon\in\mathbb{R}^{+}\Big\}
            \end{equation}
            Where $\textrm{Cl}$ denotes the closure of a set.
            Then $\mathcal{O}$ is an open cover of $E$. But
            $(E,d_{E})$ is compact, and thus there is a finite
            subcover $\Delta$. But then there is a least
            $r\in\mathbb{R}^{+}$ such that:
            \begin{equation}
                \textrm{Cl}
                \Big(B_{r}^{(X,d)}(x)\Big)^{C}\in\Delta
            \end{equation}
            But then, for all $0<\varepsilon<r$, we have:
            \begin{equation}
                B_{\varepsilon}^{(X,d)}(x)\cap
                E=\emptyset
            \end{equation}
            A contradiction. Therefore, $E$ is closed. Suppose
            $(X,d)$ is compact and $E\subseteq{X}$ is closed.
            Suppose $(E,d_{E})$ is not compact. Then there is an
            open cover $\mathcal{O}_{E}$ of $E$ with no finite
            subcover. But $E$ is closed, and thus $E^{C}$ is open.
            But then:
            \begin{equation}
                \mathcal{O}_{X}=\mathcal{O}_{E}\cup
                \big\{E^{C}\big\}
            \end{equation}
            Is an open cover of $X$. But $(X,d)$ is compact,
            and therefore there is a finite subcover $\Delta_{X}$.
            But then:
            \begin{equation}
                \Delta_{E}=\Delta_{X}\setminus\big\{E^{C}\big\}
            \end{equation}
            Is a finite subcover of $\mathcal{O}_{E}$, a
            contradiction. Therefore, $(E,d_{E})$ is compact.
        \end{solution}
\end{document}
    %        \documentclass[crop=false,class=book,oneside]{standalone}
%----------------------------Preamble-------------------------------%
%---------------------------Packages----------------------------%
\usepackage{geometry}
\geometry{b5paper, margin=1.0in}
\usepackage[T1]{fontenc}
\usepackage{graphicx, float}            % Graphics/Images.
\usepackage{natbib}                     % For bibliographies.
\bibliographystyle{agsm}                % Bibliography style.
\usepackage[french, english]{babel}     % Language typesetting.
\usepackage[dvipsnames]{xcolor}         % Color names.
\usepackage{listings, lstlinebgrd}      % Verbatim-Like Tools.
\usepackage{mathtools, esint, mathrsfs} % amsmath and integrals.
\usepackage{amsthm, amsfonts}           % Fonts and theorems.
\usepackage{tabularx}
\usepackage{tcolorbox}                  % Frames around theorems.
\usepackage{upgreek}                    % Non-Italic Greek.
\usepackage{paracol}                    % Two-column styling.
\usepackage{wrapfig}                    % Wrap text around figure.
\usepackage{fmtcount, etoolbox}         % For the \book{} command.
\usepackage[newparttoc]{titlesec}       % Formatting chapter, etc.
\usepackage{titletoc}                   % Allows \book in toc.
\usepackage[nottoc]{tocbibind}          % Bibliography in toc.
\usepackage[titles]{tocloft}            % ToC formatting.
\usepackage{multicol, enumitem}         % Multi-column/enumerate.
\usepackage{import}                     % Import external files.
\usepackage{pgfplots, tikz}             % Drawing/graphing tools.
\usetikzlibrary{
    calc,                   % Calculating right angles and more.
    angles,                 % Drawing angles within triangles.
    arrows.meta,            % Latex and Stealth arrows.
    quotes,                 % Adding labels to angles.
    positioning,            % Relative positioning of nodes.
    decorations.markings,   % Adding arrows in the middle of a line.
    patterns,
    arrows,
    shapes,
    shapes.geometric,
    cd,
    hobby,
    babel
}                                       % Libraries for tikz.
\pgfplotsset{compat=1.9}                % Version of pgfplots.
\usepackage[font=scriptsize,
            labelformat=simple,
            labelsep=colon]{subcaption} % Subfigure captions.
\usepackage[font={scriptsize},
            hypcap=true,
            labelsep=colon]{caption}    % Figure captions.
\usepackage{hyperref}                   % Allows for hyperlinks.
\hypersetup{
    colorlinks=true,
    linkcolor=blue,
    filecolor=magenta,
    urlcolor=Cerulean,
    citecolor=SkyBlue
}                           % Colors for hyperref.
\usepackage[toc,acronym,nogroupskip]{glossaries} % Glossaries and acronyms.
\usepackage[subpreambles=false]{standalone}      % Complileable sub files.

% Various font stuff from kiwi.
% Use this for Times text and Computer Modern math
%\usepackage{times}

% Quite nice
%\usepackage[charter, greekfamily=, greekuppercase=italicized]{mathdesign}
%\usepackage[utopia, greekuppercase=italicized]{mathdesign}    % Math is narrower

% Use this for Times text and math
%\usepackage{newtxtext}
%\usepackage[libertine,cmintegrals]{newtxmath}
%\usepackage{fix-cm}

%\usepackage{txfontsb}
% or
%\usepackage{mathptmx}

%\usepackage[scaled=0.92]{helvet}
%\renewcommand{\rmdefault}{ptm}

%\usepackage{mathpazo}    % add possibly `sc` and `osf` options
%\usepackage{eulervm}

%\usepackage{fourier}
%\renewcommand{\rmdefault}{ptm}
%\usepackage{mathptm}

%\usepackage{fontspec}
%\setmainfont{lmodern}

%\usepackage[varg]{txfonts}
%\usepackage{fouriernc}
%\usepackage{mathpazo}

%\usepackage{bookman}
%\usepackage[scaled]{uarial}
%\usepackage[scaled]{helvet}
%\renewcommand*\familydefault{\sfdefault}
%\usepackage[math]{anttor}

%\newcommand\fgeorgia{\fontfamily{jvn}\selectfont}
%\newcommand\ftimes{\fontfamily{ptm}\selectfont}
%\newcommand\fhelvetica{\fontfamily{phv}\selectfont}
%\newcommand\fcourier{\fontfamily{pcr}\selectfont}
%\newcommand\fbookman{\fontfamily{pbk}\selectfont}
%\newcommand\fnewcentury{\fontfamily{pnc}\selectfont}
%\newcommand\fpalatino{\fontfamily{ppl}\selectfont}
%\newcommand\favantgarde{\fontfamily{pag}\selectfont}
%\newcommand\fnormal{\normalfont}
%\newcommand\fsize[1]{\ifnum#1>0\fontsize{#1}{#1}\selectfont\else\normalsize\fi}
%------------------------Theorem Styles-------------------------%
% Define theorem style for default spacing and normal font.
\newtheoremstyle{normal}
    {\topsep}               % Amount of space above the theorem.
    {\topsep}               % Amount of space below the theorem.
    {}                      % Font used for body of theorem.
    {}                      % Measure of space to indent.
    {\bfseries}             % Font of the header of the theorem.
    {}                      % Punctuation between head and body.
    {.5em}                  % Space after theorem head.
    {}

% Define theorem style for default spacing with italicized font.
\newtheoremstyle{normalit}{\topsep}{\topsep}
                {\itshape}{}{\bfseries}{}{.5em}{}

% Italic header environment.
\newtheoremstyle{thmit}{\topsep}{\topsep}{}{}{\itshape}{}{0.5em}{}

% Define italicized environments.
\theoremstyle{normalit}
\newtheorem{theorem}{Theorem}[section]
\newtheorem{lemma}{Lemma}[section]
\newtheorem{corollary}{Corollary}[section]
\newtheorem{proposition}{Proposition}[section]
\newtheorem*{theorem*}{Theorem}

% Define environments with italic headers.
\theoremstyle{thmit}
\newtheorem*{solution}{Solution}
\newtheorem*{fsolution}{Solution}

% Define default environments.
\theoremstyle{normal}
\newtheorem{example}{Example}[section]
\newtheorem{definition}{Definition}[section]
\newtheorem{problem}{Problem}[section]
\newtheorem{question}{Question}[section]
\newtheorem{remark}{Remark}[section]
\newtheorem{properties}{Properties}[section]
\newtheorem{notation}{Notation}[section]
\newtheorem{axiom}{Axiom}[section]
\newtheorem*{properties*}{Properties}
\newtheorem*{remark*}{Remark}
\newtheorem*{definition*}{Definition}
\theoremstyle{plain}

% Define framed environment.
\tcbuselibrary{most}
\newtcbtheorem[use counter*=theorem]{ftheorem}{Theorem}%
    {colback=green!5,colframe=green!35!black,
     fonttitle=\bfseries\upshape}{th}

\newtcbtheorem[use counter*=example]{fdefinition}{Definition}%
    {fonttitle=\bfseries\upshape,
     colback=blue!5!white,colframe=blue!75!black}{def}

\newtcbtheorem[use counter*=example]{fexample}{Example}%
    {fonttitle=\bfseries\upshape,
     colback=red!5!white,colframe=red!75!black}{ex}

\newtcbtheorem[use counter*=notation]{fnotation}{Notation}%
    {fonttitle=\bfseries\upshape,
     colback=SeaGreen!5!white,colframe=SeaGreen!75!black}{ex}

\newtcbtheorem[use counter*=corollary]{fcorollary}{Corollary}%
    {fonttitle=\bfseries\upshape,
     colback=Orchid!5!white,colframe=Orchid!75!black}{ex}

\newenvironment{bproof}{\textit{Proof.}}{\hfill$\square$}
\tcolorboxenvironment{bproof}{blanker,breakable,left=5mm,
                             before skip=10pt,after skip=10pt,
                             borderline west={1mm}{0pt}{red}}
\tcolorboxenvironment{fsolution}
    {enhanced jigsaw,colframe=cyan,interior hidden,breakable}

%--------------------Declared Math Operators--------------------%
\DeclareMathOperator{\Refl}{Refl}           % Reflection operator.
\DeclareMathOperator{\Span}{Span}           % Span of a set of vectors.
\DeclareMathOperator{\Card}{Card}           % Cardinality of set.
\DeclareMathOperator{\Ord}{Ord}             % Ordinal of ordered set.
\DeclareMathOperator{\Tr}{Tr}               % Trace of matrix.
\DeclareMathOperator{\adjoint}{adj}         % Adjoint of matrix.
\DeclareMathOperator{\rk}{rk}               % Rank of operator.
\DeclareMathOperator{\nul}{nul}             % Null space of operator.
\DeclareMathOperator{\sgn}{sgn}             % Sign of a number.
\DeclareMathOperator{\multideg}{mutlideg}   % Multi-Degree (Graphs).
\DeclareMathOperator{\GCD}{GCD}             % Greatest common denominator.
\DeclareMathOperator{\LM}{LM}               % Leading monomial
\DeclareMathOperator{\LC}{LC}               % Leading coefficient.
\DeclareMathOperator{\LT}{LT}               % Leading term.
\DeclareMathOperator{\LCM}{LCM}             % Least common multiple.
\DeclareMathOperator{\Mon}{Mon}             % Monomial.
\DeclareMathOperator{\Spec}{Spec}           % Spectrum.
\DeclareMathOperator{\proj}{proj}           % Projection.
\DeclareMathOperator{\comp}{comp}           % Component.
\DeclareMathOperator{\sinc}{sinc}           % Sinc function.
\DeclareMathOperator{\Ima}{Im}              % Image of operator.
\DeclareMathOperator{\Prin}{Prin}           % Principal value.
\DeclareMathOperator{\Mod}{mod}             % Modulus.
%------------------------New Commands---------------------------%
\DeclarePairedDelimiter\norm{\lVert}{\rVert}
\DeclarePairedDelimiter\ceil{\lceil}{\rceil}
\DeclarePairedDelimiter\floor{\lfloor}{\rfloor}
\newcommand*\diff{\mathop{}\!\mathrm{d}}
\newcommand*\Diff[1]{\mathop{}\!\mathrm{d^#1}}
\renewcommand{\mod}{\ \Mod}
\renewcommand*{\glstextformat}[1]{\textcolor{RoyalBlue}{#1}}
\renewcommand{\glsnamefont}[1]{\textbf{#1}}
\renewcommand\labelitemii{$\circ$}
\renewcommand\thesubfigure{\arabic{chapter}.\arabic{figure}}
\renewcommand\thesubfigure{%
    \arabic{chapter}.\arabic{figure}.\arabic{subfigure}}
\addto\captionsenglish{\renewcommand{\figurename}{Fig.}}
%------------------------Book Command---------------------------%
\makeatletter
\renewcommand\@pnumwidth{1cm}
\newcounter{book}
\renewcommand\thebook{\@Roman\c@book}
\newcommand\book{%
    \if@openright
        \cleardoublepage
    \else
        \clearpage
    \fi
    \thispagestyle{plain}%
    \if@twocolumn
        \onecolumn
        \@tempswatrue
    \else
        \@tempswafalse
    \fi
    \null\vfil
    \secdef\@book\@sbook
}
\def\@book[#1]#2{%
    \ifnum \c@secnumdepth >-3\relax
        \refstepcounter{book}%
        \addcontentsline{toc}{book}{
            \bookname\ \thebook:\hspace{1em}#1
        }
    \else
        \addcontentsline{toc}{book}{#1}%
    \fi
    \markboth{}{}%
    {\centering
     \interlinepenalty \@M
     \normalfont
     \ifnum \c@secnumdepth >-2\relax
       \huge\bfseries \bookname\nobreakspace\thebook
       \par
       \vskip 20\p@
     \fi
     \Huge \bfseries #2\par}%
    \@endbook}
\def\@sbook#1{%
    {\centering
     \interlinepenalty \@M
     \normalfont
     \Huge \bfseries #1\par}%
    \@endbook}
\def\@endbook{
    \vfil\newpage
        \if@twoside
            \if@openright
                \null
                \thispagestyle{empty}%
                \newpage
            \fi
        \fi
        \if@tempswa
            \twocolumn
        \fi
}
\newcommand*\l@book[2]{%
    \ifnum \c@tocdepth >-2\relax
        \addpenalty{-\@highpenalty}%
        \addvspace{2.25em \@plus\p@}%
        \setlength\@tempdima{3em}%
        \begingroup
            \parindent \z@ \rightskip \@pnumwidth
            \parfillskip -\@pnumwidth
            {
                \leavevmode
                \Large \bfseries #1\hfil \hb@xt@\@pnumwidth{
                    \hss #2
                }
            }
            \par
            \nobreak
            \global\@nobreaktrue
            \everypar{\global\@nobreakfalse\everypar{}}%
        \endgroup
    \fi}
\newcommand\bookname{Book}
\renewcommand{\thebook}{\texorpdfstring{\Numberstring{book}}{book}}
\providecommand*{\toclevel@book}{-2}
\makeatother
\titlecontents{chapter}[0pt]
    {\bfseries}
    {\chaptername\ \thecontentslabel:\quad}
    {}
    {\hfill\contentspage}
\titleformat{\part}[display]
    {\Large\bfseries}
    {\partname\nobreakspace\thepart}
    {0mm}
    {\Huge\bfseries}
    \titlecontents{part}[0pt]
    {\large\bfseries}
    {\partname\ \thecontentslabel: \quad}
    {}
    {\hfill\contentspage}
\newcommand{\MarkRightAngle}[4][.3cm]
    {\coordinate (tempa) at ($(#3)!#1!(#2)$);
     \coordinate (tempb) at ($(#3)!#1!(#4)$);
     \coordinate (tempc) at ($(tempa)!0.5!(tempb)$);%midpoint
     \draw (tempa) -- ($(#3)!2!(tempc)$) -- (tempb);}
%--------------------------LENGTHS------------------------------%
% Spacings for the Table of Contents.
\addtolength{\cftsecnumwidth}{1ex}
\addtolength{\cftsubsecindent}{1ex}
\addtolength{\cftsubsecnumwidth}{1ex}
\addtolength{\cftfignumwidth}{1ex}
\addtolength{\cfttabnumwidth}{1ex}

% Spacing for multi-column and enumerate environments.
\setlength{\multicolsep}{6pt}
\setlist[enumerate]{itemsep=0pt,topsep=3pt}

% Indent and paragraph spacing.
\setlength{\parindent}{0em}
\setlength{\parskip}{0em}
%----------------------------GLOSSARY-------------------------------%
\makeglossaries
\loadglsentries{../../glossary}
\loadglsentries{../../acronym}
%--------------------------Main Document----------------------------%
\begin{document}
    \ifx\ifmathcourses\undefined
        \pagenumbering{roman}
        \title{Fourier Analysis}
        \author{Ryan Maguire}
        \date{\vspace{-5ex}}
        \maketitle
        \tableofcontents
        \clearpage
        \chapter*{Fourier Analysis}
        \addcontentsline{toc}{chapter}{Fourier Analysis}
        \markboth{}{FOURIER ANALYSIS}
        \vspace{10ex}
        \setcounter{chapter}{1}
        \pagenumbering{arabic}
    \else
        \chapter{Fourier Analysis}
    \fi
    \section{Stuff}
\end{document}
    %        \chapter{Calculus on Normed Spaces}
    \section{Gateaux Derivative}
    \section{Frechet Derivative}
    \section{Malliavin Calculus}
    %        \documentclass[crop=false,class=book,oneside]{standalone}
%----------------------------Preamble-------------------------------%
%---------------------------Packages----------------------------%
\usepackage{geometry}
\geometry{b5paper, margin=1.0in}
\usepackage[T1]{fontenc}
\usepackage{graphicx, float}            % Graphics/Images.
\usepackage{natbib}                     % For bibliographies.
\bibliographystyle{agsm}                % Bibliography style.
\usepackage[french, english]{babel}     % Language typesetting.
\usepackage[dvipsnames]{xcolor}         % Color names.
\usepackage{listings, lstlinebgrd}      % Verbatim-Like Tools.
\usepackage{mathtools, esint, mathrsfs} % amsmath and integrals.
\usepackage{amsthm, amsfonts}           % Fonts and theorems.
\usepackage{tabularx}
\usepackage{tcolorbox}                  % Frames around theorems.
\usepackage{upgreek}                    % Non-Italic Greek.
\usepackage{paracol}                    % Two-column styling.
\usepackage{wrapfig}                    % Wrap text around figure.
\usepackage{fmtcount, etoolbox}         % For the \book{} command.
\usepackage[newparttoc]{titlesec}       % Formatting chapter, etc.
\usepackage{titletoc}                   % Allows \book in toc.
\usepackage[nottoc]{tocbibind}          % Bibliography in toc.
\usepackage[titles]{tocloft}            % ToC formatting.
\usepackage{multicol, enumitem}         % Multi-column/enumerate.
\usepackage{import}                     % Import external files.
\usepackage{pgfplots, tikz}             % Drawing/graphing tools.
\usetikzlibrary{
    calc,                   % Calculating right angles and more.
    angles,                 % Drawing angles within triangles.
    arrows.meta,            % Latex and Stealth arrows.
    quotes,                 % Adding labels to angles.
    positioning,            % Relative positioning of nodes.
    decorations.markings,   % Adding arrows in the middle of a line.
    patterns,
    arrows,
    shapes,
    shapes.geometric,
    cd,
    hobby,
    babel
}                                       % Libraries for tikz.
\pgfplotsset{compat=1.9}                % Version of pgfplots.
\usepackage[font=scriptsize,
            labelformat=simple,
            labelsep=colon]{subcaption} % Subfigure captions.
\usepackage[font={scriptsize},
            hypcap=true,
            labelsep=colon]{caption}    % Figure captions.
\usepackage{hyperref}                   % Allows for hyperlinks.
\hypersetup{
    colorlinks=true,
    linkcolor=blue,
    filecolor=magenta,
    urlcolor=Cerulean,
    citecolor=SkyBlue
}                           % Colors for hyperref.
\usepackage[toc,acronym,nogroupskip]{glossaries} % Glossaries and acronyms.
\usepackage[subpreambles=false]{standalone}      % Complileable sub files.

% Various font stuff from kiwi.
% Use this for Times text and Computer Modern math
%\usepackage{times}

% Quite nice
%\usepackage[charter, greekfamily=, greekuppercase=italicized]{mathdesign}
%\usepackage[utopia, greekuppercase=italicized]{mathdesign}    % Math is narrower

% Use this for Times text and math
%\usepackage{newtxtext}
%\usepackage[libertine,cmintegrals]{newtxmath}
%\usepackage{fix-cm}

%\usepackage{txfontsb}
% or
%\usepackage{mathptmx}

%\usepackage[scaled=0.92]{helvet}
%\renewcommand{\rmdefault}{ptm}

%\usepackage{mathpazo}    % add possibly `sc` and `osf` options
%\usepackage{eulervm}

%\usepackage{fourier}
%\renewcommand{\rmdefault}{ptm}
%\usepackage{mathptm}

%\usepackage{fontspec}
%\setmainfont{lmodern}

%\usepackage[varg]{txfonts}
%\usepackage{fouriernc}
%\usepackage{mathpazo}

%\usepackage{bookman}
%\usepackage[scaled]{uarial}
%\usepackage[scaled]{helvet}
%\renewcommand*\familydefault{\sfdefault}
%\usepackage[math]{anttor}

%\newcommand\fgeorgia{\fontfamily{jvn}\selectfont}
%\newcommand\ftimes{\fontfamily{ptm}\selectfont}
%\newcommand\fhelvetica{\fontfamily{phv}\selectfont}
%\newcommand\fcourier{\fontfamily{pcr}\selectfont}
%\newcommand\fbookman{\fontfamily{pbk}\selectfont}
%\newcommand\fnewcentury{\fontfamily{pnc}\selectfont}
%\newcommand\fpalatino{\fontfamily{ppl}\selectfont}
%\newcommand\favantgarde{\fontfamily{pag}\selectfont}
%\newcommand\fnormal{\normalfont}
%\newcommand\fsize[1]{\ifnum#1>0\fontsize{#1}{#1}\selectfont\else\normalsize\fi}
%------------------------Theorem Styles-------------------------%
% Define theorem style for default spacing and normal font.
\newtheoremstyle{normal}
    {\topsep}               % Amount of space above the theorem.
    {\topsep}               % Amount of space below the theorem.
    {}                      % Font used for body of theorem.
    {}                      % Measure of space to indent.
    {\bfseries}             % Font of the header of the theorem.
    {}                      % Punctuation between head and body.
    {.5em}                  % Space after theorem head.
    {}

% Define theorem style for default spacing with italicized font.
\newtheoremstyle{normalit}{\topsep}{\topsep}
                {\itshape}{}{\bfseries}{}{.5em}{}

% Italic header environment.
\newtheoremstyle{thmit}{\topsep}{\topsep}{}{}{\itshape}{}{0.5em}{}

% Define italicized environments.
\theoremstyle{normalit}
\newtheorem{theorem}{Theorem}[section]
\newtheorem{lemma}{Lemma}[section]
\newtheorem{corollary}{Corollary}[section]
\newtheorem{proposition}{Proposition}[section]
\newtheorem*{theorem*}{Theorem}

% Define environments with italic headers.
\theoremstyle{thmit}
\newtheorem*{solution}{Solution}
\newtheorem*{fsolution}{Solution}

% Define default environments.
\theoremstyle{normal}
\newtheorem{example}{Example}[section]
\newtheorem{definition}{Definition}[section]
\newtheorem{problem}{Problem}[section]
\newtheorem{question}{Question}[section]
\newtheorem{remark}{Remark}[section]
\newtheorem{properties}{Properties}[section]
\newtheorem{notation}{Notation}[section]
\newtheorem{axiom}{Axiom}[section]
\newtheorem*{properties*}{Properties}
\newtheorem*{remark*}{Remark}
\newtheorem*{definition*}{Definition}
\theoremstyle{plain}

% Define framed environment.
\tcbuselibrary{most}
\newtcbtheorem[use counter*=theorem]{ftheorem}{Theorem}%
    {colback=green!5,colframe=green!35!black,
     fonttitle=\bfseries\upshape}{th}

\newtcbtheorem[use counter*=example]{fdefinition}{Definition}%
    {fonttitle=\bfseries\upshape,
     colback=blue!5!white,colframe=blue!75!black}{def}

\newtcbtheorem[use counter*=example]{fexample}{Example}%
    {fonttitle=\bfseries\upshape,
     colback=red!5!white,colframe=red!75!black}{ex}

\newtcbtheorem[use counter*=notation]{fnotation}{Notation}%
    {fonttitle=\bfseries\upshape,
     colback=SeaGreen!5!white,colframe=SeaGreen!75!black}{ex}

\newtcbtheorem[use counter*=corollary]{fcorollary}{Corollary}%
    {fonttitle=\bfseries\upshape,
     colback=Orchid!5!white,colframe=Orchid!75!black}{ex}

\newenvironment{bproof}{\textit{Proof.}}{\hfill$\square$}
\tcolorboxenvironment{bproof}{blanker,breakable,left=5mm,
                             before skip=10pt,after skip=10pt,
                             borderline west={1mm}{0pt}{red}}
\tcolorboxenvironment{fsolution}
    {enhanced jigsaw,colframe=cyan,interior hidden,breakable}

%--------------------Declared Math Operators--------------------%
\DeclareMathOperator{\Refl}{Refl}           % Reflection operator.
\DeclareMathOperator{\Span}{Span}           % Span of a set of vectors.
\DeclareMathOperator{\Card}{Card}           % Cardinality of set.
\DeclareMathOperator{\Ord}{Ord}             % Ordinal of ordered set.
\DeclareMathOperator{\Tr}{Tr}               % Trace of matrix.
\DeclareMathOperator{\adjoint}{adj}         % Adjoint of matrix.
\DeclareMathOperator{\rk}{rk}               % Rank of operator.
\DeclareMathOperator{\nul}{nul}             % Null space of operator.
\DeclareMathOperator{\sgn}{sgn}             % Sign of a number.
\DeclareMathOperator{\multideg}{mutlideg}   % Multi-Degree (Graphs).
\DeclareMathOperator{\GCD}{GCD}             % Greatest common denominator.
\DeclareMathOperator{\LM}{LM}               % Leading monomial
\DeclareMathOperator{\LC}{LC}               % Leading coefficient.
\DeclareMathOperator{\LT}{LT}               % Leading term.
\DeclareMathOperator{\LCM}{LCM}             % Least common multiple.
\DeclareMathOperator{\Mon}{Mon}             % Monomial.
\DeclareMathOperator{\Spec}{Spec}           % Spectrum.
\DeclareMathOperator{\proj}{proj}           % Projection.
\DeclareMathOperator{\comp}{comp}           % Component.
\DeclareMathOperator{\sinc}{sinc}           % Sinc function.
\DeclareMathOperator{\Ima}{Im}              % Image of operator.
\DeclareMathOperator{\Prin}{Prin}           % Principal value.
\DeclareMathOperator{\Mod}{mod}             % Modulus.
%------------------------New Commands---------------------------%
\DeclarePairedDelimiter\norm{\lVert}{\rVert}
\DeclarePairedDelimiter\ceil{\lceil}{\rceil}
\DeclarePairedDelimiter\floor{\lfloor}{\rfloor}
\newcommand*\diff{\mathop{}\!\mathrm{d}}
\newcommand*\Diff[1]{\mathop{}\!\mathrm{d^#1}}
\renewcommand{\mod}{\ \Mod}
\renewcommand*{\glstextformat}[1]{\textcolor{RoyalBlue}{#1}}
\renewcommand{\glsnamefont}[1]{\textbf{#1}}
\renewcommand\labelitemii{$\circ$}
\renewcommand\thesubfigure{\arabic{chapter}.\arabic{figure}}
\renewcommand\thesubfigure{%
    \arabic{chapter}.\arabic{figure}.\arabic{subfigure}}
\addto\captionsenglish{\renewcommand{\figurename}{Fig.}}
%------------------------Book Command---------------------------%
\makeatletter
\renewcommand\@pnumwidth{1cm}
\newcounter{book}
\renewcommand\thebook{\@Roman\c@book}
\newcommand\book{%
    \if@openright
        \cleardoublepage
    \else
        \clearpage
    \fi
    \thispagestyle{plain}%
    \if@twocolumn
        \onecolumn
        \@tempswatrue
    \else
        \@tempswafalse
    \fi
    \null\vfil
    \secdef\@book\@sbook
}
\def\@book[#1]#2{%
    \ifnum \c@secnumdepth >-3\relax
        \refstepcounter{book}%
        \addcontentsline{toc}{book}{
            \bookname\ \thebook:\hspace{1em}#1
        }
    \else
        \addcontentsline{toc}{book}{#1}%
    \fi
    \markboth{}{}%
    {\centering
     \interlinepenalty \@M
     \normalfont
     \ifnum \c@secnumdepth >-2\relax
       \huge\bfseries \bookname\nobreakspace\thebook
       \par
       \vskip 20\p@
     \fi
     \Huge \bfseries #2\par}%
    \@endbook}
\def\@sbook#1{%
    {\centering
     \interlinepenalty \@M
     \normalfont
     \Huge \bfseries #1\par}%
    \@endbook}
\def\@endbook{
    \vfil\newpage
        \if@twoside
            \if@openright
                \null
                \thispagestyle{empty}%
                \newpage
            \fi
        \fi
        \if@tempswa
            \twocolumn
        \fi
}
\newcommand*\l@book[2]{%
    \ifnum \c@tocdepth >-2\relax
        \addpenalty{-\@highpenalty}%
        \addvspace{2.25em \@plus\p@}%
        \setlength\@tempdima{3em}%
        \begingroup
            \parindent \z@ \rightskip \@pnumwidth
            \parfillskip -\@pnumwidth
            {
                \leavevmode
                \Large \bfseries #1\hfil \hb@xt@\@pnumwidth{
                    \hss #2
                }
            }
            \par
            \nobreak
            \global\@nobreaktrue
            \everypar{\global\@nobreakfalse\everypar{}}%
        \endgroup
    \fi}
\newcommand\bookname{Book}
\renewcommand{\thebook}{\texorpdfstring{\Numberstring{book}}{book}}
\providecommand*{\toclevel@book}{-2}
\makeatother
\titlecontents{chapter}[0pt]
    {\bfseries}
    {\chaptername\ \thecontentslabel:\quad}
    {}
    {\hfill\contentspage}
\titleformat{\part}[display]
    {\Large\bfseries}
    {\partname\nobreakspace\thepart}
    {0mm}
    {\Huge\bfseries}
    \titlecontents{part}[0pt]
    {\large\bfseries}
    {\partname\ \thecontentslabel: \quad}
    {}
    {\hfill\contentspage}
\newcommand{\MarkRightAngle}[4][.3cm]
    {\coordinate (tempa) at ($(#3)!#1!(#2)$);
     \coordinate (tempb) at ($(#3)!#1!(#4)$);
     \coordinate (tempc) at ($(tempa)!0.5!(tempb)$);%midpoint
     \draw (tempa) -- ($(#3)!2!(tempc)$) -- (tempb);}
%--------------------------LENGTHS------------------------------%
% Spacings for the Table of Contents.
\addtolength{\cftsecnumwidth}{1ex}
\addtolength{\cftsubsecindent}{1ex}
\addtolength{\cftsubsecnumwidth}{1ex}
\addtolength{\cftfignumwidth}{1ex}
\addtolength{\cfttabnumwidth}{1ex}

% Spacing for multi-column and enumerate environments.
\setlength{\multicolsep}{6pt}
\setlist[enumerate]{itemsep=0pt,topsep=3pt}

% Indent and paragraph spacing.
\setlength{\parindent}{0em}
\setlength{\parskip}{0em}
%----------------------------GLOSSARY-------------------------------%
\makeglossaries
\loadglsentries{../../glossary}
\loadglsentries{../../acronym}
%--------------------------Main Document----------------------------%
\begin{document}
    \ifx\ifmathcourses\undefined
        \pagenumbering{roman}
        \title{Advanced Mathematical Analysis}
        \author{Ryan Maguire}
        \date{\vspace{-5ex}}
        \maketitle
        \tableofcontents
        \clearpage
        \chapter*{Advanced Mathematical Analysis}
        \addcontentsline{toc}{chapter}{Advanced Mathematical Analysis}
        \markboth{}{ADVANCED MATHEMATICAL ANALYSIS}
        \vspace{10ex}
        \setcounter{chapter}{1}
        \pagenumbering{arabic}
    \else
        \chapter{Advanced Mathematical Analysis}
    \fi
    \section{Stuff}
\end{document}
    %        \documentclass[crop=false,class=book,oneside]{standalone}
%----------------------------Preamble-------------------------------%
%---------------------------Packages----------------------------%
\usepackage{geometry}
\geometry{b5paper, margin=1.0in}
\usepackage[T1]{fontenc}
\usepackage{graphicx, float}            % Graphics/Images.
\usepackage{natbib}                     % For bibliographies.
\bibliographystyle{agsm}                % Bibliography style.
\usepackage[french, english]{babel}     % Language typesetting.
\usepackage[dvipsnames]{xcolor}         % Color names.
\usepackage{listings, lstlinebgrd}      % Verbatim-Like Tools.
\usepackage{mathtools, esint, mathrsfs} % amsmath and integrals.
\usepackage{amsthm, amsfonts}           % Fonts and theorems.
\usepackage{tabularx}
\usepackage{tcolorbox}                  % Frames around theorems.
\usepackage{upgreek}                    % Non-Italic Greek.
\usepackage{paracol}                    % Two-column styling.
\usepackage{wrapfig}                    % Wrap text around figure.
\usepackage{fmtcount, etoolbox}         % For the \book{} command.
\usepackage[newparttoc]{titlesec}       % Formatting chapter, etc.
\usepackage{titletoc}                   % Allows \book in toc.
\usepackage[nottoc]{tocbibind}          % Bibliography in toc.
\usepackage[titles]{tocloft}            % ToC formatting.
\usepackage{multicol, enumitem}         % Multi-column/enumerate.
\usepackage{import}                     % Import external files.
\usepackage{pgfplots, tikz}             % Drawing/graphing tools.
\usetikzlibrary{
    calc,                   % Calculating right angles and more.
    angles,                 % Drawing angles within triangles.
    arrows.meta,            % Latex and Stealth arrows.
    quotes,                 % Adding labels to angles.
    positioning,            % Relative positioning of nodes.
    decorations.markings,   % Adding arrows in the middle of a line.
    patterns,
    arrows,
    shapes,
    shapes.geometric,
    cd,
    hobby,
    babel
}                                       % Libraries for tikz.
\pgfplotsset{compat=1.9}                % Version of pgfplots.
\usepackage[font=scriptsize,
            labelformat=simple,
            labelsep=colon]{subcaption} % Subfigure captions.
\usepackage[font={scriptsize},
            hypcap=true,
            labelsep=colon]{caption}    % Figure captions.
\usepackage{hyperref}                   % Allows for hyperlinks.
\hypersetup{
    colorlinks=true,
    linkcolor=blue,
    filecolor=magenta,
    urlcolor=Cerulean,
    citecolor=SkyBlue
}                           % Colors for hyperref.
\usepackage[toc,acronym,nogroupskip]{glossaries} % Glossaries and acronyms.
\usepackage[subpreambles=false]{standalone}      % Complileable sub files.

% Various font stuff from kiwi.
% Use this for Times text and Computer Modern math
%\usepackage{times}

% Quite nice
%\usepackage[charter, greekfamily=, greekuppercase=italicized]{mathdesign}
%\usepackage[utopia, greekuppercase=italicized]{mathdesign}    % Math is narrower

% Use this for Times text and math
%\usepackage{newtxtext}
%\usepackage[libertine,cmintegrals]{newtxmath}
%\usepackage{fix-cm}

%\usepackage{txfontsb}
% or
%\usepackage{mathptmx}

%\usepackage[scaled=0.92]{helvet}
%\renewcommand{\rmdefault}{ptm}

%\usepackage{mathpazo}    % add possibly `sc` and `osf` options
%\usepackage{eulervm}

%\usepackage{fourier}
%\renewcommand{\rmdefault}{ptm}
%\usepackage{mathptm}

%\usepackage{fontspec}
%\setmainfont{lmodern}

%\usepackage[varg]{txfonts}
%\usepackage{fouriernc}
%\usepackage{mathpazo}

%\usepackage{bookman}
%\usepackage[scaled]{uarial}
%\usepackage[scaled]{helvet}
%\renewcommand*\familydefault{\sfdefault}
%\usepackage[math]{anttor}

%\newcommand\fgeorgia{\fontfamily{jvn}\selectfont}
%\newcommand\ftimes{\fontfamily{ptm}\selectfont}
%\newcommand\fhelvetica{\fontfamily{phv}\selectfont}
%\newcommand\fcourier{\fontfamily{pcr}\selectfont}
%\newcommand\fbookman{\fontfamily{pbk}\selectfont}
%\newcommand\fnewcentury{\fontfamily{pnc}\selectfont}
%\newcommand\fpalatino{\fontfamily{ppl}\selectfont}
%\newcommand\favantgarde{\fontfamily{pag}\selectfont}
%\newcommand\fnormal{\normalfont}
%\newcommand\fsize[1]{\ifnum#1>0\fontsize{#1}{#1}\selectfont\else\normalsize\fi}
%------------------------Theorem Styles-------------------------%
% Define theorem style for default spacing and normal font.
\newtheoremstyle{normal}
    {\topsep}               % Amount of space above the theorem.
    {\topsep}               % Amount of space below the theorem.
    {}                      % Font used for body of theorem.
    {}                      % Measure of space to indent.
    {\bfseries}             % Font of the header of the theorem.
    {}                      % Punctuation between head and body.
    {.5em}                  % Space after theorem head.
    {}

% Define theorem style for default spacing with italicized font.
\newtheoremstyle{normalit}{\topsep}{\topsep}
                {\itshape}{}{\bfseries}{}{.5em}{}

% Italic header environment.
\newtheoremstyle{thmit}{\topsep}{\topsep}{}{}{\itshape}{}{0.5em}{}

% Define italicized environments.
\theoremstyle{normalit}
\newtheorem{theorem}{Theorem}[section]
\newtheorem{lemma}{Lemma}[section]
\newtheorem{corollary}{Corollary}[section]
\newtheorem{proposition}{Proposition}[section]
\newtheorem*{theorem*}{Theorem}

% Define environments with italic headers.
\theoremstyle{thmit}
\newtheorem*{solution}{Solution}
\newtheorem*{fsolution}{Solution}

% Define default environments.
\theoremstyle{normal}
\newtheorem{example}{Example}[section]
\newtheorem{definition}{Definition}[section]
\newtheorem{problem}{Problem}[section]
\newtheorem{question}{Question}[section]
\newtheorem{remark}{Remark}[section]
\newtheorem{properties}{Properties}[section]
\newtheorem{notation}{Notation}[section]
\newtheorem{axiom}{Axiom}[section]
\newtheorem*{properties*}{Properties}
\newtheorem*{remark*}{Remark}
\newtheorem*{definition*}{Definition}
\theoremstyle{plain}

% Define framed environment.
\tcbuselibrary{most}
\newtcbtheorem[use counter*=theorem]{ftheorem}{Theorem}%
    {colback=green!5,colframe=green!35!black,
     fonttitle=\bfseries\upshape}{th}

\newtcbtheorem[use counter*=example]{fdefinition}{Definition}%
    {fonttitle=\bfseries\upshape,
     colback=blue!5!white,colframe=blue!75!black}{def}

\newtcbtheorem[use counter*=example]{fexample}{Example}%
    {fonttitle=\bfseries\upshape,
     colback=red!5!white,colframe=red!75!black}{ex}

\newtcbtheorem[use counter*=notation]{fnotation}{Notation}%
    {fonttitle=\bfseries\upshape,
     colback=SeaGreen!5!white,colframe=SeaGreen!75!black}{ex}

\newtcbtheorem[use counter*=corollary]{fcorollary}{Corollary}%
    {fonttitle=\bfseries\upshape,
     colback=Orchid!5!white,colframe=Orchid!75!black}{ex}

\newenvironment{bproof}{\textit{Proof.}}{\hfill$\square$}
\tcolorboxenvironment{bproof}{blanker,breakable,left=5mm,
                             before skip=10pt,after skip=10pt,
                             borderline west={1mm}{0pt}{red}}
\tcolorboxenvironment{fsolution}
    {enhanced jigsaw,colframe=cyan,interior hidden,breakable}

%--------------------Declared Math Operators--------------------%
\DeclareMathOperator{\Refl}{Refl}           % Reflection operator.
\DeclareMathOperator{\Span}{Span}           % Span of a set of vectors.
\DeclareMathOperator{\Card}{Card}           % Cardinality of set.
\DeclareMathOperator{\Ord}{Ord}             % Ordinal of ordered set.
\DeclareMathOperator{\Tr}{Tr}               % Trace of matrix.
\DeclareMathOperator{\adjoint}{adj}         % Adjoint of matrix.
\DeclareMathOperator{\rk}{rk}               % Rank of operator.
\DeclareMathOperator{\nul}{nul}             % Null space of operator.
\DeclareMathOperator{\sgn}{sgn}             % Sign of a number.
\DeclareMathOperator{\multideg}{mutlideg}   % Multi-Degree (Graphs).
\DeclareMathOperator{\GCD}{GCD}             % Greatest common denominator.
\DeclareMathOperator{\LM}{LM}               % Leading monomial
\DeclareMathOperator{\LC}{LC}               % Leading coefficient.
\DeclareMathOperator{\LT}{LT}               % Leading term.
\DeclareMathOperator{\LCM}{LCM}             % Least common multiple.
\DeclareMathOperator{\Mon}{Mon}             % Monomial.
\DeclareMathOperator{\Spec}{Spec}           % Spectrum.
\DeclareMathOperator{\proj}{proj}           % Projection.
\DeclareMathOperator{\comp}{comp}           % Component.
\DeclareMathOperator{\sinc}{sinc}           % Sinc function.
\DeclareMathOperator{\Ima}{Im}              % Image of operator.
\DeclareMathOperator{\Prin}{Prin}           % Principal value.
\DeclareMathOperator{\Mod}{mod}             % Modulus.
%------------------------New Commands---------------------------%
\DeclarePairedDelimiter\norm{\lVert}{\rVert}
\DeclarePairedDelimiter\ceil{\lceil}{\rceil}
\DeclarePairedDelimiter\floor{\lfloor}{\rfloor}
\newcommand*\diff{\mathop{}\!\mathrm{d}}
\newcommand*\Diff[1]{\mathop{}\!\mathrm{d^#1}}
\renewcommand{\mod}{\ \Mod}
\renewcommand*{\glstextformat}[1]{\textcolor{RoyalBlue}{#1}}
\renewcommand{\glsnamefont}[1]{\textbf{#1}}
\renewcommand\labelitemii{$\circ$}
\renewcommand\thesubfigure{\arabic{chapter}.\arabic{figure}}
\renewcommand\thesubfigure{%
    \arabic{chapter}.\arabic{figure}.\arabic{subfigure}}
\addto\captionsenglish{\renewcommand{\figurename}{Fig.}}
%------------------------Book Command---------------------------%
\makeatletter
\renewcommand\@pnumwidth{1cm}
\newcounter{book}
\renewcommand\thebook{\@Roman\c@book}
\newcommand\book{%
    \if@openright
        \cleardoublepage
    \else
        \clearpage
    \fi
    \thispagestyle{plain}%
    \if@twocolumn
        \onecolumn
        \@tempswatrue
    \else
        \@tempswafalse
    \fi
    \null\vfil
    \secdef\@book\@sbook
}
\def\@book[#1]#2{%
    \ifnum \c@secnumdepth >-3\relax
        \refstepcounter{book}%
        \addcontentsline{toc}{book}{
            \bookname\ \thebook:\hspace{1em}#1
        }
    \else
        \addcontentsline{toc}{book}{#1}%
    \fi
    \markboth{}{}%
    {\centering
     \interlinepenalty \@M
     \normalfont
     \ifnum \c@secnumdepth >-2\relax
       \huge\bfseries \bookname\nobreakspace\thebook
       \par
       \vskip 20\p@
     \fi
     \Huge \bfseries #2\par}%
    \@endbook}
\def\@sbook#1{%
    {\centering
     \interlinepenalty \@M
     \normalfont
     \Huge \bfseries #1\par}%
    \@endbook}
\def\@endbook{
    \vfil\newpage
        \if@twoside
            \if@openright
                \null
                \thispagestyle{empty}%
                \newpage
            \fi
        \fi
        \if@tempswa
            \twocolumn
        \fi
}
\newcommand*\l@book[2]{%
    \ifnum \c@tocdepth >-2\relax
        \addpenalty{-\@highpenalty}%
        \addvspace{2.25em \@plus\p@}%
        \setlength\@tempdima{3em}%
        \begingroup
            \parindent \z@ \rightskip \@pnumwidth
            \parfillskip -\@pnumwidth
            {
                \leavevmode
                \Large \bfseries #1\hfil \hb@xt@\@pnumwidth{
                    \hss #2
                }
            }
            \par
            \nobreak
            \global\@nobreaktrue
            \everypar{\global\@nobreakfalse\everypar{}}%
        \endgroup
    \fi}
\newcommand\bookname{Book}
\renewcommand{\thebook}{\texorpdfstring{\Numberstring{book}}{book}}
\providecommand*{\toclevel@book}{-2}
\makeatother
\titlecontents{chapter}[0pt]
    {\bfseries}
    {\chaptername\ \thecontentslabel:\quad}
    {}
    {\hfill\contentspage}
\titleformat{\part}[display]
    {\Large\bfseries}
    {\partname\nobreakspace\thepart}
    {0mm}
    {\Huge\bfseries}
    \titlecontents{part}[0pt]
    {\large\bfseries}
    {\partname\ \thecontentslabel: \quad}
    {}
    {\hfill\contentspage}
\newcommand{\MarkRightAngle}[4][.3cm]
    {\coordinate (tempa) at ($(#3)!#1!(#2)$);
     \coordinate (tempb) at ($(#3)!#1!(#4)$);
     \coordinate (tempc) at ($(tempa)!0.5!(tempb)$);%midpoint
     \draw (tempa) -- ($(#3)!2!(tempc)$) -- (tempb);}
%--------------------------LENGTHS------------------------------%
% Spacings for the Table of Contents.
\addtolength{\cftsecnumwidth}{1ex}
\addtolength{\cftsubsecindent}{1ex}
\addtolength{\cftsubsecnumwidth}{1ex}
\addtolength{\cftfignumwidth}{1ex}
\addtolength{\cfttabnumwidth}{1ex}

% Spacing for multi-column and enumerate environments.
\setlength{\multicolsep}{6pt}
\setlist[enumerate]{itemsep=0pt,topsep=3pt}

% Indent and paragraph spacing.
\setlength{\parindent}{0em}
\setlength{\parskip}{0em}
\graphicspath{{../../../images/}}   % Path to Image Folder.
%--------------------------Main Document----------------------------%
\begin{document}
    \ifx\ifmathcourses\undefined
        \pagenumbering{roman}
        \title{Chaos Theory}
        \author{Ryan Maguire}
        \date{\vspace{-5ex}}
        \maketitle
        \tableofcontents
        \chapter*{Chaos Theory}
        \markboth{}{CHAOS THEORY}
        \setcounter{chapter}{1}
        \pagenumbering{arabic}
    \else
        \chapter{Chaos Theory}
    \fi 
    \section{A Review of Differential Equations}
        \subsection{First Order Equations}
            The first differential equation that is often studied
            is $\dot{x}(t)=ax(t)$, where $\dot{x}$ denotes the
            derivative of $x$ with respect to $t$. In this equation
            $a$ is some fixed constant parameter, and each real
            value $a$ defines a different differential equation.
            We can solve this by integrating and invoking the
            fundamental theorem of calculus. In general, a
            differential equation is an equation that relates
            a differentiable functions to its derivatives.
            The general solution to a differential equation is
            the set of all functions that satisfy the
            differential equation.
            \begin{ftheorem}{}{thm:CHAOS:SIMPLE_DIFF_EQ}
                If $x(t)$ is a differentiable function such
                that $x(0)=x_{0}$ and there is a $k\in\mathbb{R}$
                such that for all $t\in\mathbb{R}$, $\dot{x}(t)=kx(t)$,
                then $x(t)=x_{0}\exp(kt)$.
            \end{ftheorem}
            \begin{bproof}
                Let $y(t)=\exp(kt)$. Then $\dot{y}(t)=ky(t)$,
                and $y(t)\ne{0}$ for all $t\in\mathbb{R}$.
                Let $F(t)=x(t)/y(t)$. Then:
                \begin{equation*}
                    \dot{F}(t)=
                    \frac{y(t)\dot{x}(t)-\dot{y}(t)x(t)}{y^{2}(t)}
                    =\frac{y(t)\big(\dot{x}(t)-kx(t)\big)}{y^{2}(t)}
                    =\frac{1}{y(t)}\big(\dot{x}(t)-kx(t)\big)
                \end{equation*}
                But $\dot{x}(t)-kx(t)=0$, and thus $\dot{F}(t)=0$ for
                all $t\in\mathbb{R}$. But then $F(t)$ is a constant
                function. Thus there is a $C\in\mathbb{R}$ such
                that $x(t)=Cy(t)$. But $x(0)=x_{0}$, and therefore
                $C=x_{0}$. Thus, $x(t)=x_{0}\exp(kt)$.
            \end{bproof}
            The proof shows that the solution to $\dot{x}(t)=kx(t)$
            is uniquely determined if $x(0)$ is known. This can
            be replaced by knowledge of $x(t_{0})$ for any point
            $t_{0}\in\mathbb{R}$.
            Such restraints are called initial conditions to a
            differential equation. Initial conditions are
            requirements that a solution to the differential
            equation must satisfy. Such requirements can be
            the value of $x(t_{0})$ at a certain point
            $t_{0}$, or a requirement on $\dot{x}(t)$.
            \begin{fexample}{}{}
                Consider the following initial value problem:
                \begin{align}
                    \dot{x}(t)&=ax(t)
                    &
                    x(t_{0})&=x_{0}
                \end{align}
                Using the previous theorem, we can translate the
                problem by $\exp(-kt_{0})$ to create the initial
                value problem $\dot{y}(t)=ky(t)$, $y(0)=x_{0}$.
                We know the solution to this is
                $y(t)=x_{0}\exp(kt)$. Thus, the solution to the
                original initial value problem is:
                \begin{equation}
                    x(t)=x_{0}\exp(k(t-t_{0}))
                \end{equation}
            \end{fexample}
            An important class of solutions to differential
            equations are \textit{equilibrium solutions}.
            \begin{fdefinition}{Equilibrium Solution}{}
                An equilibrium solution to a differential
                equation is a solution $x(t)$ such that
                $\dot{x}(t)=0$ for all $t\in\mathbb{R}$.
                That is, a constant solution $x(t)=c$.
            \end{fdefinition}
            Studying our first problem $\dot{x}(t)=kx(t)$,
            we see that the only equilibrium solution
            occurs when $x_{0}=0$. Another important concept
            is the limiting behavior of the solution to a
            differential equation. The limiting behavior
            is often dependent on the various parameters that
            may be present in a given differential equation.
            We can see this by further studying $\dot{x}(t)=kx(t)$.
            If $k=0$ we see that $x(t)$ is a constant function. That
            is, $x(t)=x_{0}$ for all $t\in\mathbb{R}$. For $k>0$,
            solutions diverges monotonically to $\infty$
            ($x_{0}>0$) or $-\infty$ ($x_{0}<0$).
            Finally, if $k<0$ then $x(t)$ converges to $0$
            for all values of $x_{0}$. This notion can be represented
            by a \textit{phase line}. A phase line is a one
            dimensional plot of the independent variable $t$ where
            equilibrium solutions are marked and arrows indicating
            convergence or divergence to the equilibrium solutions
            are drawn. Consider again the example
            we've been studying:
            $\dot{x}(t)=kx(t)$. Suppose $k<0$. We know that, for any
            $x_{0}$, the solution $x(t)$ tends to zero as $t$ tends
            to infinity. We can represent this with a phase line:
            \begin{figure}
                \captionsetup{type=figure}
                \centering
                \begin{tikzpicture}
                    \draw[%
                        postaction={decorate},
                        decoration={%
                            markings,
                            mark=at position .0 with
                                \arrowreversed{latex},
                            mark=at position .25 with
                                \arrow{stealth},
                            mark=at position .3 with \arrow{stealth},
                            mark=at position .5 with
                                {\draw[thin] (0,-1mm) -- (0,1mm)
                                 node[below=2mm] {0};},
                            mark=at position .75 with
                                \arrowreversed{stealth},
                            mark=at position .7 with
                                \arrowreversed{stealth},
                            mark=at position 1. with
                                \arrow{latex},
                        },
                    ]   (-2,0) to (2,0) node[below] {$x_{0}$};
                \end{tikzpicture}
                \caption{Phase line for $\dot{x}(t)=kx(t)$
                         when $k<0$.}
                \label{fig:CHAOS:Phase_Line_Example}
            \end{figure}
            The initial condition $x_{0}=0$ thus gives rise to a
            \textit{stable} solution to this differential equation.
            \begin{fdefinition}{Stable Equilibrium Solution}   
                {diffeq:Stable_Equilibrium_Solution}
                A stable equilibrium solution to a differential
                equation $\dot{x}(t)=f(t,x(t))$, is
                an equilibrium solution $x(t)=x_{0}$
                such that there exists a
                $\delta>0$ such that for all
                $x_{1}\in(x_{0}-\delta,x_{0}+\delta)$, the solution
                to the initial value problem
                $\dot{x}(t)=f(t,x(t))$, $x(t_{0})=x_{1}$ converges to
                $x_{0}$ as $t\rightarrow\infty$
            \end{fdefinition}
            Being purely stable can be replaced with stable from above
            or stable from below. An equilibrium solution is called
            semi-stable if it is either stable from above or stable
            from below, but not totally stable. An unstable equilibrium
            solution is an equilibrium solution that is neither stable
            nor semi-stable. The stability of solutions often depends
            on the parameters involved in the differential equation.
            In the case of $\dot{x}(t)=kx(t)$ we saw that as
            $k$ cross zero the solutions drastically change. The
            value $k=0$ is said to be a \textit{bifurcation} value
            of the differential equation. Bifurcations are values that
            a parameter can have that alter the limiting behavior of
            solutions. $k=0$ is a bifurcation since, for all $k>0$
            we have that $x(t)$ diverges monotonically, for all
            $k<0$ we see that $x(t)$ converges to zero monotonically,
            and for $k=0$ all solutions are constants. Because of this,
            we say that there is a bifurcation at $k=0$.
            \subsubsection{The Logistics Population Model}
                First order differential equations have many
                applications in both the physical and the life
                sciences. Population modeling is a common such
                application. The growth rate of a population can
                be modelled based on two simple assumptions. The
                first is that, if the population is small, the
                growth rate is roughly proportional to the population.
                The second is that, if the population is too large,
                the growth rate is negative. For example, if the
                population is too large then there is not enough food
                and thus the population will start to decrease.
                To bridge the gap between these two assumptions
                we could make the rather reasonable assumption
                that the ratio of the population growth to the
                population is linear. Thus:
                \begin{equation}
                    \label{diffeq:Simple_Logistic_Population_Model}
                    \frac{\dot{x}(t)}{x(t)}=ax(t)+b
                \end{equation}
                Since the growth rate is negative for large
                $x_{0}$ and positive for small $x_{0}$, there
                must be a value $N$, called the
                \textit{ideal population}, such that
                $\dot{x}(t)=0$ for all $t$. If we let $\alpha$ be
                the \textit{growth factor} of the population,
                Eqn.~\ref{diffeq:Simple_Logistic_Population_Model}
                becomes:
                \begin{equation}
                    \label{diffeq:Logistics_Population_Model_DE}
                    \dot{x}(t)=\alpha{x}(t)\big(1-\frac{x(t)}{N}\big)
                \end{equation}
                By hypothesis, $\alpha$ must be positive. For
                if $x(t)>N$ and if $\alpha<0$, then
                $\dot{x}(t)>0$. This would be quite strange as
                large populations would increase rapidly to infinity.
                From physical considerations we also require that
                $N$ be positive. Otherwise we'd be speaking of
                \textit{negative population}, which is nonsense.
                For the sake of easing the mathematics, we may
                normalize the problem so that the ideal population
                is $N=1$. Let $k$ be the normalized growth factor.
                This gives us the following normalized
                logistics population model:
                \begin{equation}
                    \label{diffeq:Normalized_Log_Pop_Model_DE}
                    \dot{x}(t)=kx(t)\big(1-x(t)\big)
                \end{equation}
                The logistics population model is another example
                of a first order differential equation. In general,
                a first order differential equation is of the form
                $x(t)=f(t,x(t))$. An $n^{th}$ order differential
                equation is one of the form:
                \begin{equation*}
                    x^{(n)}(t)
                    =f(t,x(t),\dot{x}(t),\ddot{x}(t),
                       x^{(3)}(t),\hdots,x^{(n-1)}(t))
                \end{equation*}
                Here we have used the notation
                $x^{(n)}(t)$ to represent the $n^{th}$ deritative
                of $x$ with respect to $t$. A special subset of the
                general $n^{th}$ order differential equation is that
                of the autonomous $n^{th}$ order differential
                equations. These are differential equations where
                $f$ is solely a function of $x$ and its derivatives.
                \begin{fdefinition}{Autonomous Differential Equation}
                    {diffeq:Autonomous_DE}
                    An autonomous $n^{th}$ order differential
                    equation is a differential equation of the
                    form:
                    \begin{equation}
                        x^{(n)}(t)=f(x(t),\hdots,x^{(n-1)}(t))
                    \end{equation}
                \end{fdefinition}
                \begin{fexample}{First Order Autonomous 
                                 Differential Equations}{}
                    Consider the case of first order autonomous
                    differential equations. Here we'd have:
                    \begin{equation}
                        \dot{x}(t)=f(x(t))
                    \end{equation}
                    If $f$ is continuous then we can integrate this
                    and obtain a solution for $x$:
                    \begin{equation}
                        F(x)\equiv\int\frac{1}{f(x)}\diff{x}
                        =\int\diff{t}
                    \end{equation}
                    If the integral on the left hand side produces
                    an invertible function, then we can solve the
                    differential equation and obtain
                    $x(t)=F^{-1}(t+C)$, where $C$ is a constant
                    of integration. An equation like this is
                    also called \textit{separable}. That is, we
                    can separate the $x$ and $y$ terms to create
                    an expression on the left hand side which is
                    written purely in $x$, and an expression on
                    the right which is written purely in $t$.
                \end{fexample}
                The logistics population model is therefore a
                nonlinear first order autonomous differential
                equation. Nonlinear differential equations are
                usually very difficult to solve. When we have
                autonomy, however, the problem can become much
                easier. In the case of the logistics model,
                we solve for $x(t)$ using standard techniques
                from Calculus.
                \begin{equation}
                    \int\frac{1}{x(1-x)}\diff{x}=k\int\diff{t}
                \end{equation}
                Recalling from elementary algebra
                the method of partial fraction decomposition,
                we have:
                \begin{equation}
                    \frac{1}{x(1-x)}=\frac{1}{x}+\frac{1}{1-x}
                \end{equation}
                This simplifies the integral, and so we obtain:
                \begin{equation}
                    \int\Big(\frac{1}{x}+\frac{1}{1-x}\Big)\diff{x}
                    =\ln(x)-\ln(1-x)
                    =\ln\big(\frac{x}{1-x}\big)
                \end{equation}
                Evaluating the right-hand side and exponentiating,
                we get:
                \begin{equation}
                    \frac{x}{1-x}=A\exp(kt)
                \end{equation}
                Where $A$ is the exponential of the constant of
                integration. The left-hand side is and invertible
                function and its inverse is $x/(1+x)$. From this we
                obtain the solution to the logistics population model:
                \begin{equation}
                    x(t)=\frac{1}{1+C\exp(-kt)}
                \end{equation}
                By studying Eqn.~\ref{diffeq:Normalized_Log_Pop_Model_DE}
                we can find two equilibrium solution: $x(t)=0$ and
                $x(t)=1$. This makes physical sense, for if $x(t)=0$
                then the population is extinct and nothing more can be
                done, and if $x(t)=1$ then the population has reached its
                ideal value. From intuition it would seem that $x(t)=0$
                is either unstable or semistable (But since we ignore
                negative populations, this would simply be unstable) and
                that $x(t)=1$ would be stable. And indeed, if
                $0<x(t)<1$, the from
                Eqn.~\ref{diffeq:Normalized_Log_Pop_Model_DE} we have
                $\dot{x}(t)>0$, so the population is increasing
                (To the ideal value). If $x(t)>1$, then $\dot{x}(t)<0$
                and thus the population is decreasing
                (Again, towards its ideal value). We can summarize this
                more generally for first-order autonomous differential
                equations. First we prove a helpful intermediate step.
                \begin{theorem}
                    If $x(t)$ is a differentiable function such that
                    $x(t)\rightarrow{a}$ as $t\rightarrow\infty$, then
                    there is a strictly increasing monotonic sequence
                    $c_{n}$ such that $\dot{x}(c_{n})\rightarrow{0}$.
                \end{theorem}
                \begin{proof}
                    For let $t_{n}=n$ for all $n\in\mathbb{N}$. Then
                    $x(t_{n})\rightarrow{a}$ as $t_{n}\rightarrow\infty$.
                    But convegent sequences are Cauchy sequences, and
                    therefore $x(t_{n+1})-x(t_{n})\rightarrow{0}$.
                    And by the mean value theorem, for all
                    $n\in\mathbb{N}$ there is a
                    $c_{n}\in(t_{n},t_{n+1})$ such that:
                    \begin{equation*}
                        \dot{x}(c_{n})
                        =\frac{x(t_{n+1})-x(t_{n})}{t_{n+1}-t_{n}}
                    \end{equation*}
                    But $t_{n+1}-t_{n}=1$, and thus
                    $\dot{x}(c_{n})=x(t_{n+1})-x(t_{n})$. But
                    $x(t_{n+1})-x(t_{n})\rightarrow{0}$ and therefore
                    $\dot{x}(c_{n})\rightarrow{0}$.
                \end{proof}
                \begin{theorem}
                    If $f:\mathbb{R}\rightarrow\mathbb{R}$ is differentiable,
                    and given the differential equation
                    $\dot{x}(t)=f(x(t))$, if $x_{0}$ is an equilibrium
                    solution and $f'(x_{0})>0$, then $x_{0}$ is
                    an unstable equilibrium solution.
                \end{theorem}
                \begin{proof}
                    If $f(x_{0})=0$ and $f'(x_{0})>0$, then there is a
                    $\delta>0$ such that, for all
                    $x\in(x_{0}-\delta,x_{0}+\delta)$, we have:
                    \begin{equation*}
                        \frac{f(x)-f(x_{0})}{x-x_{0}}>0
                    \end{equation*}
                    But $f(x_{0})=0$. Therefore, if $x_{1}>x_{0}$, we have:
                    \begin{equation*}
                        \frac{f(x_{1})-f(x_{0})}{x_{1}-x_{0}}>0
                        \Rightarrow
                        f(x_{1})-f(x_{0})>0
                        \Rightarrow
                        f(x_{1})>0
                    \end{equation*}
                    Thus, for all $x_{1}$ such that
                    $0<x_{1}-x_{0}<\delta$, the limit of
                    the solution $x(t)$ to the initial value problem
                    $\dot{x}(t)=f(x(t))$, $x(0)=x_{1}$ can't converge to
                    $x_{0}$. For if it did there would be a point
                    $t_{0}$ such that $\dot{x}(t_{0})<0$ and
                    $0<x(t_{0})-x_{0}<\delta$, a contradiction. Similarly
                    for if $x_{1}<x_{0}$. Therefore, $x_{0}$ is unstable.
                \end{proof}
                \begin{theorem}
                    If $f:\mathbb{R}\rightarrow\mathbb{R}$ is differentiable,
                    and given the differential equation
                    $\dot{x}(t)=f(x(t))$, if $x_{0}$ is an equilibrium
                    solution and $f'(x_{0})<0$, then $x_{0}$ is
                    a stable equilibrium solution.
                \end{theorem}
                \begin{proof}
                    If $f'(x_{0})<0$ then there is a $\delta>0$ such
                    that for all $x_{1}\in(x_{0}-\delta,x_{0}+\delta)$,
                    $f'(x_{1})<0$. Then for all $x_{1}$ such that
                    $0<x_{1}-x_{0}<\delta$, we have:
                    \begin{equation*}
                        \frac{f(x_{1})-f(x_{0})}{x_{1}-x_{0}}<0
                        \Rightarrow
                        f(x_{1})-f(x_{0})<0
                        \Rightarrow
                        f(x_{1})<0
                    \end{equation*}
                    
                \end{proof}
        \subsubsection{Notes on the Jordan Normal Form}
            Every square matrix $A$ is similar to an upper
            triangular matrix $J$ in
            \textit{Jordan normal form} whose diagonal entries
            are the eigenvalues of $A$. That is, there exists
            an invertible matrix $P$ such that $P^{-1}AP=J$.
            The trace of $A$ is equal to the trace of $J$:
            \begin{equation*}
                \Tr(J)
                =\Tr(P^{-1}AP)
                =\Tr(P^{-1}PA)
                =\Tr(IA)=\Tr(A)
            \end{equation*}
        \subsubsection{Notes on Conjugacy}
            \begin{minipage}[t]{0.49\textwidth}
                We have:
                \begin{align*}
                    X'(t)&=AX(t),\quad
                    X(0)=X_{0}\\
                    \Rightarrow
                    X(t)&=e^{tA}X_{0}\\
                    \Rightarrow
                    \phi^{A}(t,X_{0})
                    &=e^{tA}X_{0}
                \end{align*}
            \end{minipage}
            \vline
            \hfill
            \begin{minipage}[t]{0.49\textwidth}
                If $B=T^{-1}AT$, for some matrix $T$, then:
                \begin{align*}
                    Y'(t)&=BY(t),\quad
                    Y(0)=Y_{0}\\
                    &=T^{-1}ATY(t)\\
                    \Rightarrow
                    Y(t)
                    &=e^{tT^{-1}AT}Y_{0}\\
                    &=T^{-1}e^{tA}TY_{0}\\
                    \Rightarrow
                    \phi^{B}(t,Y_{0})
                    &=T^{-1}e^{tA}TY_{0}
                \end{align*}
            \end{minipage}
            Thus, the homeomorphism is $h(X)=T^{-1}X$,
            and we have:
            \begin{equation*}
                \phi^{B}(t,h(X_{0}))
                =\phi^{B}(t,T^{-1}X_{0})
                =T^{-1}e^{tA}T(T^{-1}X_{0})
                =T^{-1}e^{tA}X_{0}
                =h(\phi^{A}(t,X_{0}))
            \end{equation*}
\end{document}
    %        \addtocontents{toc}{\protect\newpage}

    % \book{Geometry}
    %     \part{Riemannian Geometry}
    %         \renewcommand{\PATH}{\TOPPATH/Geometry}
    %         \chapter{Semi-Riemannian Geometry}
    Here we talk about Semi-Riemannian Geometry.
    \section{Manifolds}
    The inner product
    $\langle{\cdot,\,\cdot\rangle}:\mathbb{R}^{n}\rightarrow\mathbb{R}$
    defined by:
    \begin{equation}
        \langle{\mathbf{x},\,\mathbf{y}}\rangle
        =\sum_{k=1}^{n}x_{i}y_{i}
    \end{equation}
    induces the standard norm and metric on $\mathbb{R}^{n}$:
    \par\hfill\par
    \begin{subequations}
        \begin{minipage}[b]{0.49\textwidth}
            \begin{equation}
                \norm{\mathbf{x}}
                =\sqrt{\langle{\mathbf{x},\,\mathbf{x}}\rangle}
            \end{equation}
        \end{minipage}
        \hfill
        \begin{minipage}[b]{0.49\textwidth}
            \begin{equation}
                d(\mathbf{x},\,\mathbf{y})
                =\norm{\mathbf{x}-\mathbf{y}}
            \end{equation}
        \end{minipage}
    \end{subequations}
    \par\vspace{2.5ex}
    which further generates the standard topology on $\mathbb{R}^{n}$.
    \begin{fdefinition}{Smooth Real-Valued Functions On $\mathbb{R}$}
                       {Smooth_Real_Valued_Functions_on_R}
        A smooth real-valued function on an open subset
        $\mathcal{U}\subseteq\mathbb{R}^{n}$ is a function
        $f:\mathcal{U}\rightarrow\mathbb{R}$ such that all mixed partial
        derivatives of all orders exist and are continuous for all
        $\mathbf{x}\in\mathcal{U}$.
    \end{fdefinition}
    $\mathbb{R}^{n}$ can be defined as the set of all functions
    $\mathbf{x}:\mathbb{Z}_{n}\rightarrow\mathbb{R}$. Given an element
    $\mathbf{x}\in\mathbb{R}^{n}$ and $k\in\mathbb{Z}_{n}$ we denote image
    of $k$ as $x_{k}=\mathbf{x}(k)$. This is called the $k^{th}$ coordinate
    of $\mathbf{x}$. The projection mapping
    $\pi_{k}:\mathbb{R}^{n}\rightarrow\mathbb{R}$ is the functions defined
    by $\pi(\mathbf{x})=x_{k}$.
    \begin{ldefinition}{Smooth Euclidean Functions}{Smooth_Euclidean_Functions}
        A smooth function on a subset $\mathcal{U}\subseteq\mathbb{R}^{n}$
        to $\mathbb{R}^{m}$ is a function
        $f:\mathcal{U}\rightarrow\mathbb{R}^{m}$ such that, for all
        $k\in\mathbb{Z}_{m}$, the function $\pi_{k}\circ{f}$ is a smooth
        real-valued function.
    \end{ldefinition}
    \begin{ldefinition}{Charts}{Charts}
        A chart of dimension $n\in\mathbb{N}$ in a topological space
        $(X,\tau)$, denoted $(\mathcal{U},\phi)$, is an open set
        $\mathcal{U}\in\tau$ and a function
        $\phi:\mathcal{U}\rightarrow\mathbb{R}^{n}$ such that $\phi$ is
        a homeomorphism between $\mathcal{U}$ and its image
        $\phi(\mathcal{U})$.
    \end{ldefinition}
    The coordinate functions of a chart $(\mathcal{U},\phi)$ are the
    compositions $\pi_{k}\circ\phi:\mathcal{U}\rightarrow\mathbb{R}$.
    \begin{fdefinition}{Smoothly Overlapping Charts}
                       {Smoothly_Overlapping_Charts}
        Smoothly overlapping charts of dimension $n\in\mathbb{N}$ are
        charts $(\mathcal{U}_{1},\phi_{1})$ and
        $(\mathcal{U}_{2},\phi_{2})$ of dimension $n$ on a topological
        space $(X,\tau)$ such that
        $\phi_{1}\circ\phi_{2}^{\minus{1}}:%
         \phi_{2}(\mathcal{U}_{1}\cap\mathcal{U}_{2})%
         \rightarrow\mathbb{R}^{n}$ and
        $\phi_{2}\circ\phi_{1}^{\minus{1}}:%
         \phi_{1}(\mathcal{U}_{1}\cap\mathcal{U}_{2})%
         \rightarrow\mathbb{R}^{n}$ are smooth functions.
    \end{fdefinition}
    Given charts $(\mathcal{U}_{1},\phi_{1})$ and
    $(\mathcal{U}_{2},\phi_{2})$ in a topological space $(X,\tau)$ such
    that $\mathcal{U}_{1}$ and $\mathcal{U}_{2}$ are disjoint, we see that
    the two charts are automatically smoothly overlapping in a
    vacuous sense.
    \begin{figure}[H]
        \centering
        \captionsetup{type=figure}
        %--------------------------------Dependencies----------------------------------%
%   tikz                                                                       %
%       arrows.meta                                                            %
%-------------------------------Main Document----------------------------------%
\begin{tikzpicture}[>=Latex, line width=0.2mm]
    % Coordinates for the manifold X.
    \coordinate (X0) at (-5.0,  0.0);
    \coordinate (X1) at (-3.5, -2.5);
    \coordinate (X2) at ( 1.0, -2.0);
    \coordinate (X3) at ( 5.0,  0.0);
    \coordinate (X4) at ( 0.0,  1.0);

    % Coordinates for the subset U.
    \coordinate (U0) at (-4.0, -0.5);
    \coordinate (U1) at (-3.0, -2.0);
    \coordinate (U2) at ( 1.5, -0.5);
    \coordinate (U3) at (-0.6,  0.2);

    % Coordinates for the subset V.
    \coordinate (V0) at ( 4.0,  0.0);
    \coordinate (V1) at ( 3.0, -1.5);
    \coordinate (V2) at (-1.5, -0.5);
    \coordinate (V3) at ( 0.6,  0.2);

    % Draw the manifold X.
    \draw   (X0) to[out=-90, in=120]  (X1)
                 to[out=-60, in=-170] (X2)
                 to[out=10, in=-90]   (X3)
                 to[out=90, in=0]     (X4)
                 to[out=-180, in=90]  cycle;

    % Fill in U and V first and then outline with dashes.
    % This prevents the fill option from drawing over the outline.
    % Setting opacity makes the overlapping part mix colors as well.

    % Fill in the background of U blue.
    \draw[fill=blue, opacity=0.5, draw=none]
        (U0) to[out=-90, in=-180] (U1)
             to[out=0, in=-100]   (U2)
             to[out=80, in=0]     (U3)
             to[out=-180, in=90]  cycle;

    % Fill in the background of V red.
    \draw[fill=red, opacity=0.5, draw=none]
        (V0) to[out=-90, in=0]   (V1)
             to[out=180, in=-80] (V2)
             to[out=100, in=180] (V3)
             to[out=0, in=90]    cycle;

    % Draw dashed lines around U.
    \draw[densely dashed]
        (U0) to[out=-90, in=-180] (U1)
             to[out=0, in=-100]   (U2)
             to[out=80, in=0]     (U3)
             to[out=-180, in=90]  cycle;

    \draw[densely dashed]
        (V0) to[out=-90, in=0]   (V1)
             to[out=180, in=-80] (V2)
             to[out=100, in=180] (V3)
             to[out=0, in=90]    cycle;

    \begin{scope}[xshift=-5cm, yshift=3cm]

        % Coordinates for phi of U.
        \coordinate (P0) at (0.5, 0.5);
        \coordinate (P1) at (1.5, 0.2);
        \coordinate (P2) at (3.3, 0.8);
        \coordinate (P3) at (2.8, 2.1);
        \coordinate (P4) at (2.2, 3.6);
        \coordinate (P5) at (1.2, 2.8);

        % Coordinate for some midpoint inside U.
        \coordinate (PM) at (2.0, 1.5);

        \draw[->] (-0.5,  0.0) to ( 4.0,  0.0);
        \draw[->] ( 0.0, -0.5) to ( 0.0,  4.0);

        \draw[draw=none, fill=blue!20!white]
            (P0)    to[out=-30,  in=180]    (P1)
                    to[out=0,    in=-90]    (P2)
                    to[out=90,   in=-120]   (P3)
                    to[out=60,   in=30]     (P4)
                    to[out=-150, in=60]     (P5)
                    to[out=-120, in=150]    cycle;

        \draw[densely dashed]
            (P0)    to[out=-30,  in=180]    (P1)
                    to[out=0,    in=-90]    (P2)
                    to[out=90,   in=-120]   (P3)
                    to[out=60,   in=30]     (P4)
                    to[out=-150, in=60]     (P5)
                    to[out=-120, in=150]    cycle;

        \draw[densely dashed, fill=cyan]
            (P3)    to[out=180,  in=70]   (PM)
                    to[out=-110, in=180]  (P1)
                    to[out=0,    in=-90]  (P2)
                    to[out=90,   in=-120] cycle;

        \node at (2.00, 3.0) {$\phi(\mathcal{U})$};
        \node at (2.45, 0.8) {$\phi(\mathcal{U}\cap\mathcal{V})$};
        \node at (3.50, 3.5) {\large{$\mathbb{R}^{n}$}};
    \end{scope}

    \begin{scope}[xshift=2cm, yshift=3cm]

        % Coordinates for phi of U.
        \coordinate (Q0) at (3.5, 0.5);
        \coordinate (Q1) at (2.5, 0.2);
        \coordinate (Q2) at (0.5, 0.8);
        \coordinate (Q3) at (1.2, 2.1);
        \coordinate (Q4) at (1.8, 3.6);
        \coordinate (Q5) at (2.8, 2.8);

        % Coordinate for some midpoint inside U.
        \coordinate (QM) at (2.0, 1.5);

        \draw[->] (-0.5,  0.0) to ( 4.0,  0.0);
        \draw[->] ( 0.0, -0.5) to ( 0.0,  4.0);

        \draw[draw=none, fill=red!20!white]
            (Q0)    to[out=-150,    in=0]       (Q1)
                    to[out=-180,    in=-90]     (Q2)
                    to[out=90,      in=-120]    (Q3)
                    to[out=60,      in=150]     (Q4)
                    to[out=-30,     in=60]      (Q5)
                    to[out=-120,    in=30]      cycle;

        \draw[densely dashed]
            (Q0)    to[out=-150,    in=0]       (Q1)
                    to[out=-180,    in=-90]     (Q2)
                    to[out=90,      in=-120]    (Q3)
                    to[out=60,      in=150]     (Q4)
                    to[out=-30,     in=60]      (Q5)
                    to[out=-120,    in=30]      cycle;

        \draw[densely dashed, fill=red!50!white]
            (Q3)    to[out=-60,     in=70]      (QM)
                    to[out=-110,    in=0]       (Q1)
                    to[out=-180,    in=-90]     (Q2)
                    to[out=90,      in=-120]    cycle;

        \node at (2.00, 3.0) {$\xi(\mathcal{V})$};
        \node at (1.25, 0.8) {$\xi(\mathcal{U}\cap\mathcal{V})$};
        \node at (3.50, 3.5) {\large{$\mathbb{R}^{n}$}};
    \end{scope}

    \begin{scope}[line width=0.4mm, ->, font=\large]
        \draw (-2.0, 0.5) to[out=130, in=-100] node[left]  {$\phi$} (-3.0, 3);
        \draw ( 2.5, 0.7) to[out=50,  in=-80]  node[right] {$\xi$}  ( 3.5, 3);
        \draw (-1.5, 4.5) to[out=30, in=150]
            node[above] {$\xi\circ\phi^{\minus{1}}$} ( 1.5, 4.5);
        \draw ( 1.5, 3.5) to[out=-150, in=-30]
            node[below] {$\phi\circ\xi^{\minus{1}}$} (-1.5, 3.5);
    \end{scope}

    \node at (-4.0,  0.5) {$X$};
    \node at (-3.0, -1.5) {$\mathcal{U}$};
    \node at ( 3.0, -1.3) {$\mathcal{V}$};
    \node at ( 0.0, -0.5) {$\mathcal{U}\cap\mathcal{V}$};
\end{tikzpicture}
        \caption{Smoothly Overlapping Charts}
        \label{fig:Smoothly_Overlapping_Charts}
    \end{figure}
    \begin{fdefinition}{Atlas}{Atlas}
        An atlas on a topological space $(X,\tau)$ is a set of charts
        $\mathcal{A}$ on $(X,\tau)$ such that, for all $x\in{X}$, there
        is a $(\mathcal{U},\phi)\in\mathcal{O}$ such that $x\in\mathcal{U}$.
    \end{fdefinition}
    \begin{ldefinition}{$n$ Dimensional Atlas}{n_Dimensional_Atlas}
        An atlas of dimension $n\in\mathbb{N}$ on a topological space
        $(X,\tau)$ is an atlas $\mathcal{A}$ on $(X,\tau)$ such that, for
        all $(\mathcal{U},\phi)\in\mathcal{A}$, $(\mathcal{U},\phi)$ is
        a chart of dimension $n$.
    \end{ldefinition}
    \begin{fdefinition}{Transition Function}{Transition_Function}
        The transition function of a chart $(\mathcal{U},\phi)$ with
        respect a chart $(\mathcal{V},\xi)$ on a topological space
        $(X,\tau)$ is the function
        $f:\phi(\mathcal{U}\cap\mathcal{V})%
         \rightarrow\xi(\mathcal{V}\cap\mathcal{V})$ defined by:
        \begin{equation}
            f(x)=(\xi\circ\phi^{\minus{1}})(x)
        \end{equation}
    \end{fdefinition}
    \begin{fdefinition}{Smooth Atlas}{Smooth Atlas}
        A smooth atlas on a topological space $(X,\tau)$ is an
        atlas $\mathcal{A}$ such that for all charts
        $(\mathcal{U},\phi),(\mathcal{V},\xi)\in\mathcal{A}$, the
        transition function of $(\mathcal{U},\phi)$ with respect to
        $(\mathcal{V},\xi)$ is smooth.
    \end{fdefinition}
    \begin{fdefinition}{Maximal Smooth Atlas}{Maximal_Smooth_Atlas}
        A complete atlas on a topological space $(X,\tau)$ is a smooth
        atlas $\mathcal{A}$ on $(X,\tau)$ such that, for all charts
        $(\mathcal{U},\phi)$ of $(X,\tau)$ such that $(\mathcal{U},\phi)$
        overlaps smoothly with all $(\mathcal{V},\xi)\in\mathcal{A}$, it
        is true that $(\mathcal{U},\phi)\in\mathcal{A}$.
    \end{fdefinition}
    \begin{theorem}
        If $(X,\tau)$ is a topological space and if $\mathcal{A}$ is a
        smooth atlas of dimension $n\in\mathbb{N}$ on a topological space
        $(X,\tau)$, then there is a unique maximal unique atlas
        $\mathcal{C}$ on $(X,\tau)$ such that
        $\mathcal{A}\subseteq\mathcal{C}$.
    \end{theorem}
    \begin{proof}
        For let $\mathcal{C}$ be the set of all charts on $(X,\tau)$ that
        overlap smoothly with the charts in $\mathcal{A}$. Then since
        $\mathcal{A}$ is an atlas, for all
        $(\mathcal{U},\phi)\in\mathcal{A}$ and for all
        $(\mathcal{V},\xi)\in\mathcal{A}$, we have that $(\mathcal{U},\phi)$
        and $(\mathcal{V},\xi)$ overlap smoothly, and thus
        $(\mathcal{U},\phi)\in\mathcal{C}$. Therefore
        $\mathcal{A}\subseteq\mathcal{C}$. But $\mathcal{A}$ is an
        atlas and thus for all $x\in{X}$ there is a chart
        $(\mathcal{U},\phi)\in\mathcal{A}$ such that $x\in\mathcal{U}$.
        But $\mathcal{A}\subseteq\mathcal{C}$ and thus
        $(\mathcal{U},\phi)\in\mathcal{C}$. Thus, for all $x\in{X}$ there
        is a chart $(\mathcal{U},\phi)\in\mathcal{C}$ such that
        $x\in\mathcal{U}$. Suppose
        $(\mathcal{U}_{1},\phi_{1}),%
         (\mathcal{U}_{2},\phi_{2})\in\mathcal{C}$.
        If $\mathcal{U}_{1}$ and $\mathcal{U}_{2}$ are disjoint, then
        these two charts overlap smoothly. Suppose it is non-empty and let
        $f$ be the transition function of $(\mathcal{U}_{1},\phi_{1})$ with
        respect to $(\mathcal{U}_{2},\phi_{2})$. Let
        $p\in\phi_{1}(\mathcal{U}_{1}\cap\mathcal{U}_{2})$. But then there
        is a chart $\xi\in\mathcal{A}$ such that $\phi_{2}^{\minus{1}}(p)$
        is contained in the domain of $\xi$. From the associativity of
        composition, we have:
        \begin{equation}
            \phi_{1}\circ\phi_{2}^{\minus{1}}
            =(\phi_{1}\circ\xi^{\minus{1}})\circ
             (\xi\circ\phi_{2}^{\minus{1}})
        \end{equation}
        But by the definition of $\mathcal{C}$, $\phi_{1}$ and $\phi_{2}$
        overlap smoothly with $\xi$, and thus this is the composition of
        smooth functions, and is therefore smooth. Therefore
        $\phi_{1}\circ\phi_{2}^{\minus{1}}$ is smooth and thus
        $(\mathcal{U}_{1},\phi_{1})$ and $(\mathcal{U}_{2},\phi_{2})$
        overlap smoothly. Thus, $\mathcal{C}$ is a smooth atlas. Moreover,
        it is complete from the construction. Given any other complete
        atlas $\mathcal{C}'$ that contains $\mathcal{A}$ we would have
        $\mathcal{C}\subseteq\mathcal{C}'$ and
        $\mathcal{C}'\subseteq\mathcal{C}$, and therefore
        $\mathcal{C}=\mathcal{C}'$. Thus, this completion is unique.
    \end{proof}
    \begin{fdefinition}{Smooth Manifold}{Smooth_Manifold}
        A smooth manifold of dimension $n\in\mathbb{N}$, denoted
        $(X,\tau,\mathcal{A})$ is a Hausdorff topological space
        $(X,\tau)$ with a maximal smooth atlas $\mathcal{A}$ of
        dimension $n$.
    \end{fdefinition}
    Any smooth atlas $\mathcal{A}$ on a topological space $(X,\tau)$
    defines a a smooth manifold if we let $\mathcal{C}$ be the maximal
    smooth atlas generated by $\mathcal{A}$.
    \begin{example}
        Let $(\mathbb{R}^{n},\tau_{\mathbb{R}^{n}})$ be the standard
        $n$ dimensional Euclidean space. We can define a trivial smooth
        atlas on this space by let
        $\mathcal{A}=\{(\mathbb{R}^{n},\textrm{id})\}$, where $\textrm{id}$
        is the identity function. This defines a smooth atlas. By
        considering the unique maximal smooth atlas generated by this
        we obtain the standard smooth structure on $\mathbb{R}^{n}$.
    \end{example}
    \begin{theorem}
        If $(X,\tau,\mathcal{A})$ is a smooth manifold of dimension
        $n\in\mathbb{N}$, $(\mathcal{U},\phi)\in\mathcal{A}$,
        $\mathcal{V}\in\tau$, and if $\phi_{\mathcal{V}}$ denotes
        the restriction mapping:
        $\phi_{\mathcal{V}}:\mathcal{V}\rightarrow\mathbb{R}^{n}$,
        then $(\mathcal{V},\phi_{\mathcal{V}})\in\mathcal{A}$.
    \end{theorem}
    \begin{proof}
        For since $\phi$ is a homeomorphism from $\mathcal{U}$ to
        $\phi(\mathcal{U})$, and since $\mathcal{V}\in\tau$, we have
        that $\phi_{\mathcal{V}}$ is a homeomorphism between
        $\mathcal{V}$ and $\phi_{\mathcal{V}}(\mathcal{V})$, and
        therefore $(\mathcal{V},\phi_{\mathcal{V}})$ is a chart. But
        this chart meets $(\mathcal{U},\phi)$ smoothly, and $\mathcal{A}$
        is complete. Thus,
        $(\mathcal{V},\phi_{\mathcal{A}})\in\mathcal{A}$.
    \end{proof}
    \begin{theorem}
        If $n\in\mathbb{N}$, then there is a complete atlas
        $\mathcal{A}$ on $(S^{n},\tau)$, where $\tau$ is the inherited
        topology from $\mathbb{R}^{n+1}$.
    \end{theorem}
    \begin{proof}
        For all $k\in\mathbb{Z}_{n+1}$, let $\mathcal{U}_{k}^{+}$ and
        $\mathcal{U}_{k}^{\minus}$ be defined as:
        \par\hfill\par
        \begin{minipage}[b]{0.49\textwidth}
            \begin{equation}
                \mathcal{U}_{k}^{+}
                =\{\,\mathbf{x}\in{S}^{n}\,:\,x_{k}>0\,\}
            \end{equation}
        \end{minipage}
        \hfill
        \begin{minipage}[b]{0.49\textwidth}
            \begin{equation}
                \mathcal{U}_{k}^{\minus}
                =\{\,\mathbf{x}\in{S}^{n}\,:\,x_{k}<0\,\}
            \end{equation}
        \end{minipage}
        \par\vspace{2.5ex}
        Define $\phi_{\mathcal{U}_{k}^{+}}:\mathcal{U}_{k}^{+}%
                \rightarrow\mathbb{R}^{n}$ by:
        \begin{equation}
            \phi_{\mathcal{U}_{k}^{+}}(\mathbf{x})
            =(x_{1},\,\dots,\,x_{k-1},\,x_{k+1},\,\dots,\,x_{n+1})
        \end{equation}
        That is, the mapping that projects the point onto the plane
        defined by $x_{k}=0$. Define $\phi_{\mathcal{U}_{k}^{\minus}}$
        similarly. Then all such $\phi$ are homeomorphisms from their
        domain to their image. Let $\mathcal{A}$ be defined as follows:
        \begin{equation}
            \mathcal{A}
            =\big\{\,(\mathcal{U}_{k}^{+},\,\phi_{\mathcal{U}_{k}^{+}})
                   \,:\,k\in\mathbb{Z}_{n}\big\}\bigcup
            \big\{\,(\mathcal{U}_{k}^{-},\,\phi_{\mathcal{U}_{k}^{-}})
                   \,:\,k\in\mathbb{Z}_{n}\big\}
        \end{equation}
        Then $\mathcal{A}$ is an atlas of $(S^{n},\tau)$. For if
        $\mathbf{x}\in{S}^{n}$, then $\norm{\mathbf{x}}_{2}=1$. But then
        there is a coordinate $x_{k}$ of $\mathbf{x}$ such that
        $x_{k}\ne{0}$. But then either $x_{k}>0$ or $x_{k}<0$, and thus
        either $\mathbf{x}\in\mathcal{U}_{k}^{+}$ or
        $\mathbf{x}\in\mathcal{U}_{k}^{\minus}$. If
        $(\mathcal{V}_{2},\phi_{2})$ and $(\mathcal{V}_{2},\phi_{2})$
        are charts, then either $\phi_{1}(\mathcal{V}_{1})$
        and $\phi_{2}(\mathcal{V}_{2})$ are disjoint or they are not.
        If they are disjoint, then $\phi_{1}$ and $\phi_{2}$ overlap
        smoothly. If they are not disjoint, let $\mathbf{x}$ be contained
        in the intersection. But then, for all $k\in\mathbb{Z}_{n}$,
        $\pi_{k}\circ(\phi_{1}\circ\phi_{2}^{\minus{1}})$ is smooth,
        and thus $\phi_{1}$ and $\phi_{2}$ overlap smoothly.
    \end{proof}
    \begin{fdefinition}{Open Submanifold}{Open_Submanifold}
        An open submanifold on a manifold $(X,\tau,\mathcal{A})$ is a
        an open subset $\mathcal{U}\subseteq{X}$ and the collection
        $\mathcal{A}_{\mathcal{U}}$ defined by:
        \begin{equation}
            \mathcal{A}_{\mathcal{U}}
            =\{\,(\mathcal{V},\phi)\in\mathcal{A}\,:\,
                 \mathcal{V}\subseteq\mathcal{U}\,\}
        \end{equation}
        Together with the inherited topology $\tau_{\mathcal{U}}$.
    \end{fdefinition}
    \begin{theorem}
        If $(X,\tau,\mathcal{A})$ is a smooth manifold and if
        $(\mathcal{U},\tau_{\mathcal{U}},\mathcal{A}_{\mathcal{U}})$
        is an open submanifold, then it is a smooth manifold.
    \end{theorem}
    \begin{proof}
        For by the previous theorem, $\mathcal{A}_{\mathcal{U}}$ is a
        complete atlas. Moreover, a subspace of a Hausdorff topological
        space is also a Hausdorff topological space, and hence
        $(\mathcal{U},\tau_\mathcal{U})$ is a Hausdorff space. Thus,
        $(\mathcal{U},\tau_\mathcal{U},\mathcal{A}_{\mathcal{U}})$ is
        a smooth manifold.
    \end{proof}
    \begin{fdefinition}{Product Chart}{Product_Chart}
        The product chart of an $n$ dimensional chart $(\mathcal{U},\phi)$
        on a topological space $(X,\tau_{X})$ with an $m$ dimensional chart
        $(\mathcal{V},\xi)$ on a topological space $(Y,\tau_{Y})$ is
        the ordered pair $(\mathcal{U}\times\mathcal{V},f)$ where
        $f:\mathcal{U}\times\mathcal{V}\rightarrow\mathbb{R}^{n+m}$
        defined by:
        \begin{equation}
            f(p,q)_{k}=
            \begin{cases}
                \phi(p)_{k},&k<n\\
                \xi(q)_{k},&n\leq{k}<n+m
            \end{cases}
        \end{equation}
        Where $\phi(p)_{k}$ is the $k^{th}$ coordinate of
        $\phi(p)\in\mathbb{R}^{n}$ and $\xi(q)_{k}$ is the $k^{th}$
        coordinate of $\xi(q)\in\mathbb{R}^{m}$. We denote this by
        $(\mathcal{U},\phi)\times(\mathcal{V},\xi)$.
    \end{fdefinition}
    Thinking of the elements of $\mathbb{R}^{n+m}$ as tuples of length
    $n+m$, we can write:
    \begin{equation}
        f(p,q)=\big(x_{1}(p),\dots,x_{n}(p),y_{1}(q),\dots,y_{m}(q)\big)
    \end{equation}
    \begin{theorem}
        If $(X,\tau_{X},\mathcal{A}_{X})$ and $(Y,\tau_{Y},\mathcal{A}_{Y})$
        are smooth manifolds, and if $\mathcal{A}$ is the set of all
        product charts on $X\times{Y}$, then $\mathcal{A}$ is a smooth
        atlas on $(X\times{Y},\tau_{X\times{Y}})$, where
        $\tau_{X\times{Y}}$ is the product topology.
    \end{theorem}
    \begin{proof}
        For if $p\in{X}\times{Y}$ then there is an $x\in{X}$ and a
        $y\in{Y}$ such that $p=(x,y)$. But $\mathcal{A}_{X}$ is a smooth
        atlas on $(X,\tau_{X})$, and thus if $x\in{X}$ then there is a
        $(\mathcal{U},\phi)\in\mathcal{A}_{X}$ such that $x\in\mathcal{U}$.
        Similarly, there is a $(\mathcal{V},\xi)\in\mathcal{A}_{Y}$ such
        that $y\in\mathcal{V}$. But then $p\in\mathcal{U}\times\mathcal{V}$,
        and $\mathcal{U}\times\mathcal{V}\in\tau_{X\times{Y}}$. But if
        $\phi:\mathcal{U}\rightarrow\mathbb{R}^{n}$ is a homeomorphism
        between $\mathcal{U}$ and $\phi(\mathcal{U})$ and
        $\xi:\mathcal{V}\rightarrow\mathbb{R}^{m}$ is a homemorphism
        between $\mathcal{V}$ and $\xi(\mathcal{V})$, then
        $f:\mathcal{U}\times\mathcal{V}\rightarrow\mathbb{R}^{n+m}$ is
        a homeomorphism between $\mathcal{U}\times\mathcal{V}$ and
        $f(\mathcal{U}\times\mathcal{V})$, and thus the product chart
        is a chart in $(X\times{Y},\tau_{X\times{Y}})$. Moreover, all of
        the elements of $\mathcal{A}$ are smoothly overlapping. Thus,
        $\mathcal{A}$ is an atlas on $(X\times{Y},\tau_{X\times{Y}})$.
    \end{proof}
    Using the maximal smooth atlas generated by the product atlas
    $\mathcal{A}$ creates the product manifold.
    \subsection{Smooth Mappings}
        \begin{fdefinition}{Smooth Real-Valued Functions}
                           {Smooth_Real_Valued_Functions}
            A smooth real-valued function on a manifold
            $(X,\tau,\mathcal{A})$ of dimension $n\in\mathbb{N}$ is a
            function $f:X\rightarrow\mathbb{R}$ such that, for every chart
            $(\mathcal{U},\phi)\in\mathcal{A}$, the function
            $f\circ\phi^{\minus{1}}:\phi(\mathcal{U})\rightarrow\mathbb{R}$
            is a smooth Euclidean function.
        \end{fdefinition}
        \begin{theorem}
            If $(X,\tau,\mathcal{A})$ is a manifold and if
            $f,g:X\rightarrow\mathbb{R}$ are smooth real-valued functions,
            then $(f+g):X\rightarrow\mathbb{R}$ defined by:
            \begin{equation}
                (f+g)(x)=f(x)+g(x)
                \quad\quad
                x\in{X}
            \end{equation}
            Is a smooth real-valued function.
        \end{theorem}
        \begin{theorem}
            If $(X,\tau,\mathcal{A})$ is a manifold and if
            $f,g:X\rightarrow\mathbb{R}$ are smooth real-valued functions,
            then $(f\cdot{g}):X\rightarrow\mathbb{R}$ defined by:
            \begin{equation}
                (f\cdot{g})(x)=f(x)\cdot{g}(x)
                \quad\quad
                x\in{X}
            \end{equation}
            Is a smooth real-valued function.
        \end{theorem}
        \begin{fdefinition}{Smooth Functions Between Manifolds}
                           {Smooth Functions Between Manifolds}
            A smooth function from a manifold $(X,\tau_{X},\mathcal{A}_{X})$
            of dimension $n\in\mathbb{N}$ to a manifold
            $(Y,\tau_{Y},\mathcal{A}_{Y})$ of dimension $m\in\mathbb{N}$ is
            a function $f:X\rightarrow{Y}$ such that, for every chart
            $(\mathcal{U},\phi)\in\mathcal{A}_{X}$ and for every chart
            $(\mathcal{V},\xi)\in\mathcal{A}_{Y}$, the function
            $\xi\circ{f}\circ\phi^{\minus{1}}:\xi(\mathcal{V})%
             \rightarrow\mathbb{R}^{m}$ is a smooth Euclidean function.
        \end{fdefinition}
        \begin{theorem}
            If $(X,\tau_{X},\mathcal{A}_{X})$ and
            $(Y,\tau_{Y},\mathcal{A}_{Y})$ are manifolds, if
            $A_{X}\subseteq\mathcal{A}_{X}$ is an atlas on $(X,\tau_{X})$,
            if $A_{Y}\subseteq\mathcal{A}_{Y}$ is an atlas on
            $(Y,\tau_{Y})$, and if $f:X\rightarrow{Y}$ is a function such
            that, for all $(\mathcal{U},\phi)\in{A}_{X}$ and for all
            $(\mathcal{Y},\xi)\in{A}_{y}$ it is true that
            $\xi\circ{f}\circ\phi^{\minus{1}}:\xi(\mathcal{V})%
             \rightarrow\mathbb{R}^{m}$ is a smooth Euclidean function,
            then $f$ is smooth.
        \end{theorem}
        \begin{proof}
            Since charts in $\mathcal{A}_{X}$ and $\mathcal{A}_{Y}$
            overlap smoothly with charts in $A_{X}$ and $A_{Y}$, and since
            the atlases $A_{X}$ and $A_{Y}$ cover $X$ and $Y$, respectively,
            we are done.
        \end{proof}
        \begin{theorem}
            If $(X,\tau_{X},\mathcal{A}_{X})$ is a manifold, then
            $\textrm{id}:X\rightarrow{X}$ is a smooth function.
        \end{theorem}
        \begin{theorem}
            If $(X,\tau_{X},\mathcal{A}_{X})$,
            $(Y,\tau_{Y},\mathcal{A}_{Y})$, and
            $(Z,\tau_{Z},\mathcal{A}_{Z})$ are manifolds, if
            $f:X\rightarrow{Y}$ and $g:Y\rightarrow{Z}$ are smooth, then
            $g\circ{f}:X\rightarrow{Z}$ is smooth.
        \end{theorem}
        Smoothness is a local property. A function $\phi:M\rightarrow{N}$
        is smooth at $p\in{M}$ if there is a neighborhood
        $\mathcal{U}$ of $p$ such that the restriction of $\phi$ to
        $\mathcal{U}$ is smooth. A smooth function is thus a function
        that is smooth at every point.
        \begin{theorem}
            If $(X,\mathcal{A}_{X},\tau_{X})$ and
            $(Y,\mathcal{A}_{Y},\tau_{Y})$ are manifolds and if
            $f:X\rightarrow{Y}$ is smooth, then $f$ is continuous.
        \end{theorem}
        \begin{fdefinition}{Diffeomorphism}{Diffeomorphism}
            A diffeomorphism from a manifold $(X,\tau_{X},\mathcal{A}_{X})$
            to a manifold $(Y,\tau_{Y},\mathcal{A}_{Y})$ is a bijective
            function $f:X\rightarrow{Y}$ such that $f$ and $f^{\minus{1}}$
            are smooth.
        \end{fdefinition}
        \begin{lexample}
            For any $a,b\in\mathbb{R}$ with $a<b$, the interval
            $(a,b)$ is diffeomorphic to the unit interval $(0,1)$. For
            let $\phi:(0,1)\rightarrow(a,b)$ be defined by:
            \begin{equation}
                \phi(t)=(a-b)t+b
            \end{equation}
            Then $\phi$ is a smooth bijection and it's inverse is smooth.
            Moreover, the unit interval is diffeomorphic to $\mathbb{R}$.
            For let $\xi:(0,1)\rightarrow\mathbb{R}$ be defined by:
            \begin{equation}
                \xi(t)=\frac{2t}{t(1-t)}
            \end{equation}
        \end{lexample}
        \begin{theorem}
            If $(X,\tau_{X},\mathcal{A}_{X})$ and
            $(Y,\tau_{Y},\mathcal{A}_{Y})$ are manifolds, and if
            $f:X\rightarrow{Y}$ is a diffeomorphism, then $f$ is a
            homeomorphism from $(X,\tau_{X})$ to $(Y,\tau_{Y})$.
        \end{theorem}
        \begin{proof}
            For if $f$ is a diffeomorphism, then it is a smooth bijection
            such that it's inverse is smooth. But if $f$ is smooth, then
            it is continuous and therefore it is a continuous bijection.
            But if $f^{\minus{1}}$ is smooth, then it is continuous, and
            thus $f$ is a bicontinuous bijective function, and is therefore
            a homeomorphism.
        \end{proof}
        A smooth homeomorphism need not be a diffeomorphism. The inverse
        function may not be smooth. For let
        $f:\mathbb{R}\rightarrow\mathbb{R}$ be defined by $f(x)=x^{3}$.
        Then $f$ is a homeomorphism and it's forward direction is smooth,
        but $f^{\minus{1}}$ is not smooth at the origin.
        \begin{ftheorem}{}{}
            If $A$ is a set, if $(X,\tau,\mathcal{A})$ is a manifold, and
            if $f:A\rightarrow{X}$ is an bijective function, then there
            exists a topology $\tau_{A}$ and an atlas $\mathcal{A}_{A}$
            on $X$ such that $f$ is a diffeomorphism.
        \end{ftheorem}
    %         \addtocontents{toc}{\protect\newpage}


    %\book{Physics}
    %    \import{books/Physics/}{Physics_Main.tex}
    %    \addtocontents{toc}{\protect\newpage}

    % Print glossaries and acronyms page.
    \printnoidxglossary[type=\acronymtype]
    \clearpage
    \printnoidxglossary[style=longpara]

    % Print bibliographies from all texts.
    \clearpage
    \bibliography{biblio}

    % Print the index.
    \clearpage
    \printindex
\end{document}