\documentclass[oneside]{book}                                                  %
%----------------------------------Preamble------------------------------------%
\makeatletter                                                                  %
    \def\input@path{{../}}                                                     %
\makeatother                                                                   %
%---------------------------Packages----------------------------%
\usepackage{geometry}
\geometry{b5paper, margin=1.0in}
\usepackage[T1]{fontenc}
\usepackage{graphicx, float}            % Graphics/Images.
\usepackage{natbib}                     % For bibliographies.
\bibliographystyle{agsm}                % Bibliography style.
\usepackage[french, english]{babel}     % Language typesetting.
\usepackage[dvipsnames]{xcolor}         % Color names.
\usepackage{listings}                   % Verbatim-Like Tools.
\usepackage{mathtools, esint, mathrsfs} % amsmath and integrals.
\usepackage{amsthm, amsfonts, amssymb}  % Fonts and theorems.
\usepackage{tcolorbox}                  % Frames around theorems.
\usepackage{upgreek}                    % Non-Italic Greek.
\usepackage{fmtcount, etoolbox}         % For the \book{} command.
\usepackage[newparttoc]{titlesec}       % Formatting chapter, etc.
\usepackage{titletoc}                   % Allows \book in toc.
\usepackage[nottoc]{tocbibind}          % Bibliography in toc.
\usepackage[titles]{tocloft}            % ToC formatting.
\usepackage{pgfplots, tikz}             % Drawing/graphing tools.
\usepackage{imakeidx}                   % Used for index.
\usetikzlibrary{
    calc,                   % Calculating right angles and more.
    angles,                 % Drawing angles within triangles.
    arrows.meta,            % Latex and Stealth arrows.
    quotes,                 % Adding labels to angles.
    positioning,            % Relative positioning of nodes.
    decorations.markings,   % Adding arrows in the middle of a line.
    patterns,
    arrows
}                                       % Libraries for tikz.
\pgfplotsset{compat=1.9}                % Version of pgfplots.
\usepackage[font=scriptsize,
            labelformat=simple,
            labelsep=colon]{subcaption} % Subfigure captions.
\usepackage[font={scriptsize},
            hypcap=true,
            labelsep=colon]{caption}    % Figure captions.
\usepackage[pdftex,
            pdfauthor={Ryan Maguire},
            pdftitle={Mathematics and Physics},
            pdfsubject={Mathematics, Physics, Science},
            pdfkeywords={Mathematics, Physics, Computer Science, Biology},
            pdfproducer={LaTeX},
            pdfcreator={pdflatex}]{hyperref}
\hypersetup{
    colorlinks=true,
    linkcolor=blue,
    filecolor=magenta,
    urlcolor=Cerulean,
    citecolor=SkyBlue
}                           % Colors for hyperref.
\usepackage[toc,acronym,nogroupskip,nopostdot]{glossaries}
\usepackage{glossary-mcols}
%------------------------Theorem Styles-------------------------%
\theoremstyle{plain}
\newtheorem{theorem}{Theorem}[section]

% Define theorem style for default spacing and normal font.
\newtheoremstyle{normal}
    {\topsep}               % Amount of space above the theorem.
    {\topsep}               % Amount of space below the theorem.
    {}                      % Font used for body of theorem.
    {}                      % Measure of space to indent.
    {\bfseries}             % Font of the header of the theorem.
    {}                      % Punctuation between head and body.
    {.5em}                  % Space after theorem head.
    {}

% Italic header environment.
\newtheoremstyle{thmit}{\topsep}{\topsep}{}{}{\itshape}{}{0.5em}{}

% Define environments with italic headers.
\theoremstyle{thmit}
\newtheorem*{solution}{Solution}

% Define default environments.
\theoremstyle{normal}
\newtheorem{example}{Example}[section]
\newtheorem{definition}{Definition}[section]
\newtheorem{problem}{Problem}[section]

% Define framed environment.
\tcbuselibrary{most}
\newtcbtheorem[use counter*=theorem]{ftheorem}{Theorem}{%
    before=\par\vspace{2ex},
    boxsep=0.5\topsep,
    after=\par\vspace{2ex},
    colback=green!5,
    colframe=green!35!black,
    fonttitle=\bfseries\upshape%
}{thm}

\newtcbtheorem[auto counter, number within=section]{faxiom}{Axiom}{%
    before=\par\vspace{2ex},
    boxsep=0.5\topsep,
    after=\par\vspace{2ex},
    colback=Apricot!5,
    colframe=Apricot!35!black,
    fonttitle=\bfseries\upshape%
}{ax}

\newtcbtheorem[use counter*=definition]{fdefinition}{Definition}{%
    before=\par\vspace{2ex},
    boxsep=0.5\topsep,
    after=\par\vspace{2ex},
    colback=blue!5!white,
    colframe=blue!75!black,
    fonttitle=\bfseries\upshape%
}{def}

\newtcbtheorem[use counter*=example]{fexample}{Example}{%
    before=\par\vspace{2ex},
    boxsep=0.5\topsep,
    after=\par\vspace{2ex},
    colback=red!5!white,
    colframe=red!75!black,
    fonttitle=\bfseries\upshape%
}{ex}

\newtcbtheorem[auto counter, number within=section]{fnotation}{Notation}{%
    before=\par\vspace{2ex},
    boxsep=0.5\topsep,
    after=\par\vspace{2ex},
    colback=SeaGreen!5!white,
    colframe=SeaGreen!75!black,
    fonttitle=\bfseries\upshape%
}{not}

\newtcbtheorem[use counter*=remark]{fremark}{Remark}{%
    fonttitle=\bfseries\upshape,
    colback=Goldenrod!5!white,
    colframe=Goldenrod!75!black}{ex}

\newenvironment{bproof}{\textit{Proof.}}{\hfill$\square$}
\tcolorboxenvironment{bproof}{%
    blanker,
    breakable,
    left=3mm,
    before skip=5pt,
    after skip=10pt,
    borderline west={0.6mm}{0pt}{green!80!black}
}

\AtEndEnvironment{lexample}{$\hfill\textcolor{red}{\blacksquare}$}
\newtcbtheorem[use counter*=example]{lexample}{Example}{%
    empty,
    title={Example~\theexample},
    boxed title style={%
        empty,
        size=minimal,
        toprule=2pt,
        top=0.5\topsep,
    },
    coltitle=red,
    fonttitle=\bfseries,
    parbox=false,
    boxsep=0pt,
    before=\par\vspace{2ex},
    left=0pt,
    right=0pt,
    top=3ex,
    bottom=1ex,
    before=\par\vspace{2ex},
    after=\par\vspace{2ex},
    breakable,
    pad at break*=0mm,
    vfill before first,
    overlay unbroken={%
        \draw[red, line width=2pt]
            ([yshift=-1.2ex]title.south-|frame.west) to
            ([yshift=-1.2ex]title.south-|frame.east);
        },
    overlay first={%
        \draw[red, line width=2pt]
            ([yshift=-1.2ex]title.south-|frame.west) to
            ([yshift=-1.2ex]title.south-|frame.east);
    },
}{ex}

\AtEndEnvironment{ldefinition}{$\hfill\textcolor{Blue}{\blacksquare}$}
\newtcbtheorem[use counter*=definition]{ldefinition}{Definition}{%
    empty,
    title={Definition~\thedefinition:~{#1}},
    boxed title style={%
        empty,
        size=minimal,
        toprule=2pt,
        top=0.5\topsep,
    },
    coltitle=Blue,
    fonttitle=\bfseries,
    parbox=false,
    boxsep=0pt,
    before=\par\vspace{2ex},
    left=0pt,
    right=0pt,
    top=3ex,
    bottom=0pt,
    before=\par\vspace{2ex},
    after=\par\vspace{1ex},
    breakable,
    pad at break*=0mm,
    vfill before first,
    overlay unbroken={%
        \draw[Blue, line width=2pt]
            ([yshift=-1.2ex]title.south-|frame.west) to
            ([yshift=-1.2ex]title.south-|frame.east);
        },
    overlay first={%
        \draw[Blue, line width=2pt]
            ([yshift=-1.2ex]title.south-|frame.west) to
            ([yshift=-1.2ex]title.south-|frame.east);
    },
}{def}

\AtEndEnvironment{ltheorem}{$\hfill\textcolor{Green}{\blacksquare}$}
\newtcbtheorem[use counter*=theorem]{ltheorem}{Theorem}{%
    empty,
    title={Theorem~\thetheorem:~{#1}},
    boxed title style={%
        empty,
        size=minimal,
        toprule=2pt,
        top=0.5\topsep,
    },
    coltitle=Green,
    fonttitle=\bfseries,
    parbox=false,
    boxsep=0pt,
    before=\par\vspace{2ex},
    left=0pt,
    right=0pt,
    top=3ex,
    bottom=-1.5ex,
    breakable,
    pad at break*=0mm,
    vfill before first,
    overlay unbroken={%
        \draw[Green, line width=2pt]
            ([yshift=-1.2ex]title.south-|frame.west) to
            ([yshift=-1.2ex]title.south-|frame.east);},
    overlay first={%
        \draw[Green, line width=2pt]
            ([yshift=-1.2ex]title.south-|frame.west) to
            ([yshift=-1.2ex]title.south-|frame.east);
    }
}{thm}

%--------------------Declared Math Operators--------------------%
\DeclareMathOperator{\adjoint}{adj}         % Adjoint.
\DeclareMathOperator{\Card}{Card}           % Cardinality.
\DeclareMathOperator{\curl}{curl}           % Curl.
\DeclareMathOperator{\diam}{diam}           % Diameter.
\DeclareMathOperator{\dist}{dist}           % Distance.
\DeclareMathOperator{\Div}{div}             % Divergence.
\DeclareMathOperator{\Erf}{Erf}             % Error Function.
\DeclareMathOperator{\Erfc}{Erfc}           % Complementary Error Function.
\DeclareMathOperator{\Ext}{Ext}             % Exterior.
\DeclareMathOperator{\GCD}{GCD}             % Greatest common denominator.
\DeclareMathOperator{\grad}{grad}           % Gradient
\DeclareMathOperator{\Ima}{Im}              % Image.
\DeclareMathOperator{\Int}{Int}             % Interior.
\DeclareMathOperator{\LC}{LC}               % Leading coefficient.
\DeclareMathOperator{\LCM}{LCM}             % Least common multiple.
\DeclareMathOperator{\LM}{LM}               % Leading monomial.
\DeclareMathOperator{\LT}{LT}               % Leading term.
\DeclareMathOperator{\Mod}{mod}             % Modulus.
\DeclareMathOperator{\Mon}{Mon}             % Monomial.
\DeclareMathOperator{\multideg}{mutlideg}   % Multi-Degree (Graphs).
\DeclareMathOperator{\nul}{nul}             % Null space of operator.
\DeclareMathOperator{\Ord}{Ord}             % Ordinal of ordered set.
\DeclareMathOperator{\Prin}{Prin}           % Principal value.
\DeclareMathOperator{\proj}{proj}           % Projection.
\DeclareMathOperator{\Refl}{Refl}           % Reflection operator.
\DeclareMathOperator{\rk}{rk}               % Rank of operator.
\DeclareMathOperator{\sgn}{sgn}             % Sign of a number.
\DeclareMathOperator{\sinc}{sinc}           % Sinc function.
\DeclareMathOperator{\Span}{Span}           % Span of a set.
\DeclareMathOperator{\Spec}{Spec}           % Spectrum.
\DeclareMathOperator{\supp}{supp}           % Support
\DeclareMathOperator{\Tr}{Tr}               % Trace of matrix.
%--------------------Declared Math Symbols--------------------%
\DeclareMathSymbol{\minus}{\mathbin}{AMSa}{"39} % Unary minus sign.
%------------------------New Commands---------------------------%
\DeclarePairedDelimiter\norm{\lVert}{\rVert}
\DeclarePairedDelimiter\ceil{\lceil}{\rceil}
\DeclarePairedDelimiter\floor{\lfloor}{\rfloor}
\newcommand*\diff{\mathop{}\!\mathrm{d}}
\newcommand*\Diff[1]{\mathop{}\!\mathrm{d^#1}}
\renewcommand*{\glstextformat}[1]{\textcolor{RoyalBlue}{#1}}
\renewcommand{\glsnamefont}[1]{\textbf{#1}}
\renewcommand\labelitemii{$\circ$}
\renewcommand\thesubfigure{%
    \arabic{chapter}.\arabic{figure}.\arabic{subfigure}}
\addto\captionsenglish{\renewcommand{\figurename}{Fig.}}
\numberwithin{equation}{section}

\renewcommand{\vector}[1]{\boldsymbol{\mathrm{#1}}}

\newcommand{\uvector}[1]{\boldsymbol{\hat{\mathrm{#1}}}}
\newcommand{\topspace}[2][]{(#2,\tau_{#1})}
\newcommand{\measurespace}[2][]{(#2,\varSigma_{#1},\mu_{#1})}
\newcommand{\measurablespace}[2][]{(#2,\varSigma_{#1})}
\newcommand{\manifold}[2][]{(#2,\tau_{#1},\mathcal{A}_{#1})}
\newcommand{\tanspace}[2]{T_{#1}{#2}}
\newcommand{\cotanspace}[2]{T_{#1}^{*}{#2}}
\newcommand{\Ckspace}[3][\mathbb{R}]{C^{#2}(#3,#1)}
\newcommand{\funcspace}[2][\mathbb{R}]{\mathcal{F}(#2,#1)}
\newcommand{\smoothvecf}[1]{\mathfrak{X}(#1)}
\newcommand{\smoothonef}[1]{\mathfrak{X}^{*}(#1)}
\newcommand{\bracket}[2]{[#1,#2]}

%------------------------Book Command---------------------------%
\makeatletter
\renewcommand\@pnumwidth{1cm}
\newcounter{book}
\renewcommand\thebook{\@Roman\c@book}
\newcommand\book{%
    \if@openright
        \cleardoublepage
    \else
        \clearpage
    \fi
    \thispagestyle{plain}%
    \if@twocolumn
        \onecolumn
        \@tempswatrue
    \else
        \@tempswafalse
    \fi
    \null\vfil
    \secdef\@book\@sbook
}
\def\@book[#1]#2{%
    \refstepcounter{book}
    \addcontentsline{toc}{book}{\bookname\ \thebook:\hspace{1em}#1}
    \markboth{}{}
    {\centering
     \interlinepenalty\@M
     \normalfont
     \huge\bfseries\bookname\nobreakspace\thebook
     \par
     \vskip 20\p@
     \Huge\bfseries#2\par}%
    \@endbook}
\def\@sbook#1{%
    {\centering
     \interlinepenalty \@M
     \normalfont
     \Huge\bfseries#1\par}%
    \@endbook}
\def\@endbook{
    \vfil\newpage
        \if@twoside
            \if@openright
                \null
                \thispagestyle{empty}%
                \newpage
            \fi
        \fi
        \if@tempswa
            \twocolumn
        \fi
}
\newcommand*\l@book[2]{%
    \ifnum\c@tocdepth >-3\relax
        \addpenalty{-\@highpenalty}%
        \addvspace{2.25em\@plus\p@}%
        \setlength\@tempdima{3em}%
        \begingroup
            \parindent\z@\rightskip\@pnumwidth
            \parfillskip -\@pnumwidth
            {
                \leavevmode
                \Large\bfseries#1\hfill\hb@xt@\@pnumwidth{\hss#2}
            }
            \par
            \nobreak
            \global\@nobreaktrue
            \everypar{\global\@nobreakfalse\everypar{}}%
        \endgroup
    \fi}
\newcommand\bookname{Book}
\renewcommand{\thebook}{\texorpdfstring{\Numberstring{book}}{book}}
\providecommand*{\toclevel@book}{-2}
\makeatother
\titleformat{\part}[display]
    {\Large\bfseries}
    {\partname\nobreakspace\thepart}
    {0mm}
    {\Huge\bfseries}
\titlecontents{part}[0pt]
    {\large\bfseries}
    {\partname\ \thecontentslabel: \quad}
    {}
    {\hfill\contentspage}
\titlecontents{chapter}[0pt]
    {\bfseries}
    {\chaptername\ \thecontentslabel:\quad}
    {}
    {\hfill\contentspage}
\newglossarystyle{longpara}{%
    \setglossarystyle{long}%
    \renewenvironment{theglossary}{%
        \begin{longtable}[l]{{p{0.25\hsize}p{0.65\hsize}}}
    }{\end{longtable}}%
    \renewcommand{\glossentry}[2]{%
        \glstarget{##1}{\glossentryname{##1}}%
        &\glossentrydesc{##1}{~##2.}
        \tabularnewline%
        \tabularnewline
    }%
}
\newglossary[not-glg]{notation}{not-gls}{not-glo}{Notation}
\newcommand*{\newnotation}[4][]{%
    \newglossaryentry{#2}{type=notation, name={\textbf{#3}, },
                          text={#4}, description={#4},#1}%
}
%--------------------------LENGTHS------------------------------%
% Spacings for the Table of Contents.
\addtolength{\cftsecnumwidth}{1ex}
\addtolength{\cftsubsecindent}{1ex}
\addtolength{\cftsubsecnumwidth}{1ex}
\addtolength{\cftfignumwidth}{1ex}
\addtolength{\cfttabnumwidth}{1ex}

% Indent and paragraph spacing.
\setlength{\parindent}{0em}
\setlength{\parskip}{0em}                                                           %
%----------------------------------GLOSSARY------------------------------------%
\makenoidxglossaries                                                           %
\loadglsentries{glossary}                                                      %
\loadglsentries{acronym}                                                       %
\makeindex                                                                     %
%---------------------------------Title Page-----------------------------------%
\title{Mathematics and Physics}                                                %
\author{Ryan Maguire}                                                          %
\date{\vspace{-5ex}}                                                           %
%--------------------------------Main Document---------------------------------%
%   Some things to talk about:
%       Topology that generated Lebesgue sigma algebra.
%       Not every sigma algebra is generated by a topology.
%           The sigma algebra of countable and cocountable sets is, however.
%       Notes on on the measure of projections.
%           Is this measure a continuous function of angle?
%               Not in general. Subset of unit circle with rational x-axis.
%               What about if compact? Connected? Path connected?
%
%   Develop set theory from algebras of sets. Prove associativity, DeMorgan's
%   Law's, etc. Use this to define set difference in terms of complement and
%   intersection. Go into Stone's Representation Theorem.
%   Generalized distributive law, generalized DeMorgan's Law.
%
%   Definition of partition of a set.
%   
%   Add diagram for the equivalence relation generated by arelation. Draw an
%   example where the set has 2 points, 3 points, 4 points.
%
%   Set difference is neither associative nor commutative.
%
%   Cardinality of R, Q, Z, N, other examples. Algebraic numbers.
%   Prove the power set is strictly larger.
%
%   Discuss vertical line test for functions and it's abstract analog.
%
%   Complement and inverse as unary operators (functions)
%
%   Continuum hypothesis and bijection from R to power set of N.
%   Add a comment about ordered pair notation vs. interval notation.
%
%   Prove (a,b)=(c,d) iff a=c and b=d.
%
%   Talk about or, and, the symbols $\land$ and $\lor$, if then statements,
%   logical negation, truth tables, quantifies ($\exists$ and $\forall$). Talk
%   about theorems and proofs, hypotheses and conclusions. Vacuous truths.
%   Provide examples. Converse and contrapositive.
%
%   Topology notes:
%       Sierpinski space. Particular point topology. Pseudocompactness.
%       Define neighborhood, open neighborhood, interior, exterior.
%       Boundary. Closure minus set, or whole space minus interior and exterior.
%       Isolated points, perfect sets, countability & perfect sets.
%       Density, limit points, nowhere dense. Continuity, homemomorphisms.
%       

\newcommand*{\TOPPATH}{books}
\newcommand*{\PATH}{\TOPPATH/}
\newcounter{endpage}
\begin{document}
    \pagenumbering{roman}
    \maketitle
    \tableofcontents
    \listoffigures
    \listoftables
    \clearpage
    \pagenumbering{gobble}
    \book{Foundations}
        \pagenumbering{arabic}
        \renewcommand{\PATH}{\TOPPATH/Foundations}
        \part{Set-Theory}
            %------------------------------------------------------------------------------%
\begingroup
    \ifcsname\PATH\endcsname
        \newcommand{\PATH}{books/Foundations/ZFC}
        \newcommand{\OLDPATH}{\PATH}
    \else
        \newcommand{\OLDPATH}{\PATH}
        \renewcommand{\PATH}{books/Foundations/ZFC}
    \fi
    \chapter{Logic}
        This chapter serves as providing preliminary material for the rest of
        the text and discussing how the English language will be used to
        formulate mathematics.
        \section{Formulas and Propositions}
    In the set theory we will be working with, known as
    Zermelo-Fraenkel Set Theory (abbreviated as ZFC where C stands for
    \textit{choice}) there are a few words that are left undefined as well as a
    few symbols. This is certainly unavoidable for defining all words would
    ultimately be circular, and thus we try to leave as few words and as few
    symbols possible undefined, and build up everything else from this
    foundation. When we develop set theory our undefined symbol will be
    the symbol $\in$, and our undefined word will be \textit{set}. The symbol
    $\in$ denotes containment. That is, $x\in{y}$ reads as $x$ is \textit{in}
    $y$, or $x$ is \textit{contained} in $y$. It is a
    \textit{\gls{predicate}}\index{Predicate}.
    \begin{fdefinition}{Predicate}{Predicate}
        A \gls{predicate} is a statement that takes in some parameter or
        variable $x$ and returns either \textit{True} or \textit{True}.
    \end{fdefinition}
    We have relied on the word \textit{statement} being already defined, and
    similarly for the words \textit{parameter} or \textit{variable}. For most
    this is not an issue, but for some it may indicate that mathematics may rest
    on some very shakey foundations.
    \par\hfill\par
    From our undefined symbol $\in$, we build new symbols by expressing them in
    terms of a \textit{formula}\index{Formula}, which is simply a finite
    sequence of symbols. Here the word \textit{sequence} is meant to imply that
    the \textit{order} in which we combine these symbols is important, and that
    rearranging said order may create a different inequivalent formula. We build
    formulas by defining a few symbols that stand as placeholders for standard
    words in English. There are five symbols, called
    \textit{\glspl{connective}}\index{Connective (Logic)}, that we use.
    \begin{align*}
        a\land{b}\quad
        &\textrm{True if and only if }a
        \textrm{ is true and }b\textrm{ is true}
        \tag{Conjunction}\\
        a\lor{b}\quad
        &\textrm{True if and only if }a
        \textrm{ is true or }b\textrm{ is true, or both}
        \tag{Disjunction}\\
        \neg{a}\quad
        &\textrm{True if and only if }a\textrm{ is false}
        \tag{Negation}\\
        a\Leftrightarrow{b}\quad
        &a\textrm{ is true if and only if }b\textrm{ is true}
        \tag{Equivalence}\\
        a\Rightarrow{b}\quad
        &a\textrm{ is true implies that }b\textrm{ is true}
        \tag{Implication}
    \end{align*}
    There are other symbols we adopt, such as \textit{does not imply}:
    \begin{equation*}
        a\not\Rightarrow{b}
    \end{equation*}
    But from how we shall define these notions, this new symbol is equivalent to
    a combination of the previous ones:
    \begin{equation*}
        a\not\Rightarrow{b}\Longleftrightarrow
        \neg(a\Rightarrow{b})
        \Longleftrightarrow
        a\land\neg{b}
    \end{equation*}
    There are two more symbols called
    \textit{\glspl{quantifier}}\index{Quantifier}.
    \begin{equation*}
        \forall_{x}\quad\textrm{For all }x
        \quad\quad\quad\quad
        \exists_{x}\quad\textrm{There exists }x
    \end{equation*}
    Quantifiers, together with connectives, the word \textit{set}, and the
    $\in$ symbol are combined to define new terms and new symbols. The rest of
    mathematics rest on trusting ones intuition behind these notions.
    \chapter{Zermelo-Fraenkel Set Theory}
        We'll start from the basics and develop mathematics from an axiomatic
        standpoint, adopting as truths the few postulates of Zermelo and
        Fraenkel. We'll then add the axiom of choice to our toolbelt and
        proceed from there to define and prove many of the basic and familiar
        results and concepts of mathematics. It is important not to take
        anything for granted even if something seems obvious. The existence of
        many sets will be proved, rather than accepting these things as trivial
        truths.
        %------------------------------------------------------------------------------%
\section{The Axioms of Zermelo and Fraenkel}
    The first thing to do is define what \textit{sets}\index{Set} are.
    \begin{fdefinition}{Sets}{Sets}
        A \gls{set} is a collection of objects (or elements), none of which is
        the set itself.\index{Set}
    \end{fdefinition}
    If we wish to stand on a truly solid foundation, it seems we're off to a bad
    start. In defining sets we've used the words \textit{collection} and
    \textit{objects}, neither of which have been defined. This is the problem
    found in Chapt.~\ref{chapt:Logic} when defining connectives. To begin
    stating definitions and theorems we need the existence of a \textit{thing}.
    Sets act as our thing. We know they exist, but we don't know how to define
    them all to well. Nevertheless, we can describe how they behave and what
    they can do, as well as how to obtain new sets from pre-existing ones.
    \begin{fnotation}{Element Notation}{Element_Notation}
        If $A$ is a \gls{set} and if $x$ is an element\index{Set!Element of} of
        $A$, then we denote this by writing
        \glslink{containmentsymb}{$x\in{A}$}. If $x$ is not an element
        of $A$, we write $x\notin{A}$.\index{Containment $\in$}
    \end{fnotation}
    We do not yet know that sets exist. Pedagogically it seems poor to wait
    for examples, so we'll speak loosely for the moment so we may familiarize
    ourselves with the notation.
    \newpage
    \begin{fexample}{Using Element Notation}{Using_Element_Notation}
        Given a set $A$ that contains only a few objects, we can represent $A$
        by listing out the elements, separated by commas, and enclosing them in
        braces. Suppose $A$ is the set that contains three distinct objects
        labelled $a$, $b$, and $c$. We then write:
        \begin{equation}
            A=\big\{\,a,\,b,\,c\,\big\}
        \end{equation}
        If we are told that there is a fourth object $d$ that is different from
        $a$, $b$, and $c$, then we can use the notation defined in
        Not.~\ref{not:Element_Notation} to write the following:
        \par
        \begin{subequations}
            \begin{minipage}[b]{0.49\textwidth}
                \centering
                \begin{equation}
                    a\in{A}
                \end{equation}
            \end{minipage}
            \hfill
            \begin{minipage}[b]{0.49\textwidth}
                \centering
                \begin{equation}
                    d\notin{A}
                \end{equation}
            \end{minipage}
        \end{subequations}
        \par\vspace{2.5ex}
        The notation $a\in{A}$ should be read as \textit{a is an element of A},
        or \textit{a is contained in A}, or simply \textit{a is in A}.
        Similarly, the notation $d\notin{A}$ should be read as
        \textit{d is not an element of A}, or \textit{d is not contained in A}.
        \par\hfill\par
        $A$ is an example of a \textit{finite} set\index{Set!Finite}, moreover
        it contains only three elements. For larger sets we rely on other
        methods to write them down. One such means is to indicate a pattern and
        use an ellipses to show that it goes on. Such a description is vague and
        lacks rigor, but can be helpful when the pattern is obvious. The set of
        all \textit{natural} numbers\index{Natural Numbers}, or non-negative
        integers (denoted \gls{mathbbN}) can be loosely represented by writing:
        \begin{equation}
            \label{eqn:Natural_Numbers_Ellipses}%
            \mathbb{N}=\big\{\,0,\,1,\,2,\,3,\,4,\,5,\,\dots\,\big\}
        \end{equation}
        Using our developed notation, we can write:
        \par
        \begin{subequations}
            \begin{minipage}[b]{0.49\textwidth}
                \centering
                \begin{equation}
                    23\in\mathbb{N}
                \end{equation}
            \end{minipage}
            \hfill
            \begin{minipage}[b]{0.49\textwidth}
                \centering
                \begin{equation}
                    \minus{4}\notin\mathbb{N}
                \end{equation}
            \end{minipage}
        \end{subequations}
        \par\vspace{2.5ex}
        Letting \gls{mathbbZn} denote all non-negative integers between 0 and
        $n-1$, we have:
        \begin{equation}
            \label{eqn:Z_n_Ellipses}%
            \mathbb{Z}_{n}=\big\{0,\,1,\,2,\,\dots,\,n-1\,\big\}
        \end{equation}
        Thus $17\in\mathbb{Z}_{18}$ but $19\notin\mathbb{Z}_{18}$. Lastly, we
        present the integers\index{Integers} (\gls{mathbbZ}).
        \begin{equation}
            \label{eqn:Integers_Ellipses}%
            \mathbb{Z}=\big\{\,\dots,\,\minus{3},\,\minus{2},\,\minus{1},
                             \,0,\,1,\,2,\,3,\dots\,\big\}
        \end{equation}
    \end{fexample}
    In our definition of a set (Def.~\ref{def:Sets}) we explicitly required
    that sets cannot contain themselves. That is, if $A$ is a set, then
    $A\notin{A}$. This requirement was introduced to avoid paradoxes discovered
    by Bertrand Russell\index{Russell, Bertrand} in 1901. Allow us to neglect
    this requirement for a moment and reveal why it is essential. Recall from
    logic that a system of mathematics is inconsistent if one can prove a
    contradiction within the theory. In Naive Set Theory\index{Naive Set Theory}
    we allow the \textit{axiom of unrestricted comprehension}%
    \index{Axiom!of Unrestricted Comprehension}. This allows us to construct
    sets as any definable collection. That is, if we have a
    proposition\index{Proposition} $P$, then we can define a set $A$ as the set
    of all objects that satisfy $P$. We can write:
    \begin{equation}
        A=\big\{\,x\;|\;P(x)\,\big\}
    \end{equation}
    Problems with such a loose definition arise instantly. Let $P$ be the
    proposition \textit{true if x is a set, false otherwise}. Then
    $A=\{\,x\;|\;P(x)\,\}$ can be read in plain English as the
    \textit{set of all sets}\index{Set!of All Sets}. A natural question would be
    whether or not $A$ then contains itself. That is, is $A\in{A}$? Russell's
    paradox arises by defining proper sets to be sets $B$ such that
    $B\notin{B}$, and improper sets to be sets $B$ such that $B\in{B}$. Using
    the \textit{Law of the Excluded Middle}\index{Law of the Excluded Middle}
    (which we will prove later), one has that every set is either proper or
    improper.
    \begin{ftheorem}{Russell's Paradox}{Russells_Paradox}
        Naive Set Theory is inconsistent.\index{Russell's Paradox}
        \index{Theorem!Russell's Paradox}
    \end{ftheorem}
    \begin{bproof}
        For let $P$ be the proposition \textit{true if} $x\notin{x}$,
        \textit{false otherwise}. Let $A$ be the set defined by this
        proposition:
        \begin{equation}
            A=\big\{\,x\;|\;P(x)\,\big\}
        \end{equation}
        That is, $A$ is the set of all sets that do not contain themselves.
        Suppose $A\in{A}$. If $A\in{A}$ then $P(A)$ is true. That is, $A$ is a
        proper set. But proper sets do not contain themselves and $A\in{A}$, a
        contradiction. Thus $A\notin{A}$. But if $A\notin{A}$ than $P(A)$ is
        false. But if $P(A)$ is false, than $A$ is an improper set. But then
        $A\in{A}$, a contradiction as $A\notin{A}$. Thus $A\in{A}$ if and only
        if $A\notin{A}$, a contradiction. Therefore, Naive Set Theory is
        inconsistent.
    \end{bproof}
    Our development of Zermelo-Fraenkel Set Theory%
    \index{Zermelo-Fraenkel Set Theory} is to avoid this paradox and attempt to
    develop a consistent system of mathematics. The proof of Russell's Paradox
    (Thm.~\ref{thm:Russells_Paradox}) relied on the
    \textit{Law of the Excluded Middle}\index{Law of the Excluded Middle} which
    states that, given a proposition $P$, either $P$ is true or its negation is
    true. Thus we have shown that the axiom of unrestricted
    comprehension\index{Axiom!of Unrestricted Comprehension} and the law of the
    excluded middle are not compatible. This is quite unfortunate as the law of
    the excluded middle is essential in mathematics as it allows one to prove
    things via contradiction\index{Proof!by Contradiction}. That is, given some
    statement we assume the opposite is true and arrive at a contradiction thus
    showing the negation of our statement is false. We then invoke the law of
    the excluded middle to show that our original statement is true. The axioms
    of Zermelo and Fraenkel, together with the axiom of choice (a system
    commonly abbreviated as \gls{ZFC}) are able to prove the validity of the law
    of the excluded middle. That is, if ZFC is consistent, then so is the law of
    the excluded middle. This is one of the reasons for studying ZFC in detail.
    \par\hfill\par
    The first collection of axioms were proposed in 1908 by
    Ernst Zermelo\index{Zermelo, Ernst}. Subtle problems were pointed out by
    Abraham Fraenkel\index{Fraenkel, Abraham} in 1920, and in 1921 the system of
    Zermelo-Fraenkel Set Theory\index{Zermelo-Fraenkel Set Theory} came to be.
    The requirement that a set does not contain itself, which is equivalent to
    the \textit{axiom of regularity}\index{Axiom!of Regularity}, is sufficient
    to avoid Russell's paradox. We will prove the equivalence of this axiom with
    our definition once we have obtained the law of the excluded middle.
    \subsection{Subsets and Equality}
        To delve more into set theory it would be convenient to know that at
        least \textit{one} set exists. The axiom of the empty
        set\index{Axiom!of the Empty Set} gives us such an existence.
        \begin{faxiom}{Axiom of the Empty Set}{Axiom_of_the_Empty_Set}
            There exists a set \gls{emptysetsymb} (the \gls{empty set}) such
            that for all $x$ it is true that $x\notin\emptyset$.
            \index{Empty Set}\index{Set!Empty}
            \begin{equation*}
                \exists_{\emptyset}:\forall_{x}\big(\neg(x\in\emptyset)\big)
            \end{equation*}
        \end{faxiom}
        The empty set is the set that contains no elements. As such some choose
        to write $\emptyset=\{\}$. Note that this is different from the
        set $\{\emptyset\}$ since the empty set contains no elements whereas
        $\{\emptyset\}$ contains one elements (it contains the empty set).
        Indeed, the equality of $\emptyset$ and $\{\emptyset\}$ would violate
        our requirement that sets do not contain themselves. Any set that
        contains \textit{something} is called non-empty.
        \begin{fdefinition}{Non-Empty Set}{Non_Empty_Set}
            A \gls{non-empty set} is a \gls{set} $A$ such that there exists an
            $x$ such that $x\in{A}$.\index{Set!Non-Empty}
        \end{fdefinition}
        The terminology is somewhat redundant, and essentially every set we deal
        with is non-empty. Indeed, there is only one empty set
        (see Thm.~\ref{thm:Empty_Set_is_Unique}). Thus, every other set one
        thinks of is non-empty.
        \begin{example}
            Using the notation from Ex.~\ref{ex:Using_Element_Notation}, the set
            of all natural numbers (\gls{mathbbN}) and the set of all integers
            (\gls{mathbbZ}) are non-empty since $0\in\mathbb{N}$ and
            $0\in\mathbb{Z}$. If $n$ is a positive integer, then the set
            of integers between $0$ and $n-1$ (\gls{mathbbZn}) is non-empty as
            well since $0\in\mathbb{Z}_{n}$. Note that there is some ambiguity
            behind the meaning of $\mathbb{Z}_{0}$. This stems from our
            \textit{dot dot dot} definition and it is unclear what this should
            mean for $n=0$. When we rigorously define this notation we will see
            that $\mathbb{Z}_{0}$ is empty. That is, we will say
            $k\in\mathbb{Z}_{n}$ if $k\in\mathbb{N}$ and $k<n$, a description
            that will be justified by the axiom schema of
            specification\index{Axiom!Schema of Specification}. Thus, for
            $\mathbb{Z}_{0}$ we seek an integer $k\in\mathbb{N}$ such that
            $k<0$. But there are no such integers, and thus the set is empty.
        \end{example}
        \begin{example}
            It's possible to write down some formula for a set that ultimately
            leads to the empty set. For consider the \textit{set of all}
            \textit{rational numbers whose square is two}. This set turns out to
            be empty since there is no rational that satisfies this criterion.
            That is, $\sqrt{2}$ is known to be an irrational number. Thus, the
            set specified by our proposition is the empty set.
        \end{example}
        \begin{example}
            Going in the other direction, it is possible to write a formula for
            a set that appears empty, but is indeed not. The set of all
            $p\textrm{-Sylow}$ subgroups\index{Group!Subgroup!Sylow} of a
            non-empty finite group (Discussed in Book~\ref{book:Algebra}) is a
            non-empty set, but there's no reason to believe so from the start.
        \end{example}
        A set is entirely determined by its elements, and as such repetition and
        order cannot be accounted for. Thus the sets $\{a,b\}$ and $\{a,a,b\}$
        must be considered the same since they contain precisely the same
        elements. This will be made clear once equality has been defined. In a
        similar manner, sets have no sense of order and thus $\{a,b\}$ and
        $\{b,a\}$ are equivalent. It then becomes a task to invent some new
        object that does have a notion of order. To do this requires the concept
        of a \textit{function}\index{Function}, and it is our current aim to
        develop this topic.
        \par\hfill\par
        To rigorously show that the examples in the previous paragraph are equal
        requires a definition of equality. This is the
        \textit{axiom of extensionality}\index{Axiom!of Extensionality}. First,
        we define the familiar symbol for equality (\gls{equalsymb}) in
        terms of containment (\gls{containmentsymb}).
        \begin{fnotation}{Equality}{Equality}
            If $A$ and $B$ are sets, then \glslink{equalsymb}{$A=B$} if and
            only if for all sets $C$, $C\in{A}$ if and only if $C\in{B}$, and
            for all sets $D$, $A\in{D}$ if and only if $B\in{D}$.
            \begin{equation*}
                \forall_{A}\forall_{B}\Big((A=B)\Longleftrightarrow\big(
                    \forall_{C}(C\in{A}\Leftrightarrow{C}\in{B})
                    \land\forall_{D}(A\in{D}\Leftrightarrow{B}\in{D})\big)\Big)
            \end{equation*}
        \end{fnotation}
        \begin{example}
            Consider the set of all planets in the solar system, and consider
            the set of the eight largest objects in the solar system other than
            the sun. These two sets are equal since the eight largest objects
            (other than the sun) are the eight planets (sorry Pluto), and the
            set of planets form the eight largest objects. The tricky part is
            to check that for any set one can name, it is true that if the set
            of planets lies in the set, then the set of the eight largest
            objects not equal to the sun lie in this set as well, and vice
            versa. This is almost impossible, and seemingly redundant, and so we
            rely on the \textit{axiom of extensionality} to ease the
            demonstration of equality.
        \end{example}
        The axiom of extensionality says that to check for equality it suffices
        to show that for all $C$, $C\in{A}$ if and only if $C\in{B}$.
        That is, there is no need to check that for all $D$, $A\in{D}$ if and
        only if $B\in{D}$. For simplicity, the axiom of extensionality may be
        taken as the definition of
        equality\index{Equality}\index{Set!Equal Sets}.
        \begin{faxiom}{Axiom of Extensionality}{Axiom_of_Extensionality}
            If $A$ and $B$ are sets, and if for all $x$ it is true that
            $x\in{A}$ if and only if $x\in{B}$, then $A=B$. That is, $A$ and $B$
            are equal sets.\index{Axiom!of Extensionality}
            \begin{equation*}
                \forall_{A}\forall_{B}\Big(\forall_{x}(x\in{A}\Leftrightarrow
                x\in{B})\Longleftrightarrow\big(A=B\big)\Big)
            \end{equation*}
        \end{faxiom}
        \begin{example}
            Returning to our example of planets, we have seen that the set of
            all planets and the set of the eight largest objects other than the
            sun contain precisely the same elements. By the axiom of
            extensionality, we thus have equality amongst these two.
        \end{example}
        \begin{example}
            Let \gls{mathbbR} denote the real numbers, let \gls{greaterthan} and
            \gls{leq} denote the usual notions of \textit{greater than} and
            \textit{less than or equal to}, respectively. Define $A$ and $B$ by:
            \par
            \begin{subequations}
                \begin{minipage}[b]{0.49\textwidth}
                    \centering
                    \begin{equation}
                        A=\{\,x\in\mathbb{R}\;|\;x>0\,\}
                    \end{equation}
                \end{minipage}
                \hfill
                \begin{minipage}[b]{0.49\textwidth}
                    \centering
                    \begin{equation}
                        B=\{\,y\in\mathbb{R}\;|\;y\not\leq{0}\,\}
                    \end{equation}
                \end{minipage}
            \end{subequations}
            \par\vspace{2.5ex}
            This notation will be justified by the specification axiom
            (see Ax.~\ref{ax:Axiom_Schema_of_Specification}). Then for any real
            number $x\in\mathbb{R}$, we have that $x\in{A}$ if and only if
            $x>0$. That is, $A$ is the set of all positive real numbers. But if
            $x>0$, then $x\ne{0}$ and $x$ is non-negative, so $x\not\leq{0}$.
            Thus $x$ satisfies the criterion for membership of $B$, and
            therefore $x\in{B}$. Similarly, $y\in{B}$ if and only if
            $y\not\leq{0}$ and this is just another way of stating that $y>0$,
            and hence $y\in{A}$. By the axiom of extensionality, $A=B$.
        \end{example}
        We'll restate the definition of equality\index{Equality} using the
        language of subsets\index{Set!Subset}, lessening the effort required in
        proving various things are equal. The notions are equivalent. Subsets
        are sets that are defined in terms of another given set by simply
        removing some (or none, or all) of the elements.
        \begin{fdefinition}{Subsets}{Subsets}
            A \gls{subset} of a \gls{set} $B$ is a set $A$ such that for all
            $x\in{A}$ it is true that $x\in{B}$. If $A$ is a subset of $B$ we
            write \glslink{subseteq}{$A\subseteq{B}$}. Otherwise, we write
            $A\nsubseteq{B}$.\index{Set!Subset}
            \begin{equation*}
                \forall_{A}\forall_{B}\Big(\big(A\subseteq{B}\big)
                \Longleftrightarrow
                \forall_{x}\big(x\in{A}\Rightarrow{x}\in{B}\big)\Big)
            \end{equation*}
        \end{fdefinition}
        We can often visualize sets as blobs in the plane. Using such a visual,
        we can envision subsets as well (Fig.~\ref{fig:Subset_Blobs}). Given a
        blob $B$, a subset of $B$ is another blob $A$ that is entirely contained
        within $B$.
        \begin{figure}[H]
            \centering
            %--------------------------------Dependencies----------------------------------%
%   tikz                                                                       %
%-------------------------------Main Document----------------------------------%
\begin{tikzpicture}[line width=0.2mm, scale=1.2]

    % Coordinates for the bigger blob.
    \coordinate (P1) at ( 0.0, -2.0);
    \coordinate (P2) at ( 1.0, -1.0);
    \coordinate (P3) at ( 1.5,  1.0);
    \coordinate (P4) at ( 0.0,  2.0);
    \coordinate (P5) at (-3.0,  0.0);

    % Coordinates for the inner blob.
    \coordinate (Q1) at ( 0.0, -1.0);
    \coordinate (Q2) at ( 1.0,  0.0);
    \coordinate (Q3) at ( 0.5,  0.5);
    \coordinate (Q4) at (-0.5,  0.5);
    \coordinate (Q5) at (-1.0,  0.0);

    % Coordindates to label things.
    \coordinate (A) at (-0.1, -0.2);
    \coordinate (B) at (-1.5,  0.5);

    % Draw the bigger blob.
    \draw[fill=red, opacity=0.4] (P1) to [out=0,    in=-120] (P2)
                                      to [out=60,   in=-45]  (P3)
                                      to [out=135,  in=0]    (P4)
                                      to [out=-180, in=70]   (P5)
                                      to [out=-110, in=-180] cycle;

    % Draw the inner blob.
    \draw[fill=cyan, opacity=0.8] (Q1) to [out=0,    in=-120]  (Q2)
                                       to [out=60,   in=20]    (Q3)
                                       to [out=-160, in=45]    (Q4)
                                       to [out=-135, in=90]    (Q5)
                                       to [out=-90,  in=180]   cycle;

    % Labels for the two blobs.
    \node at (A) {$A$};
    \node at (B) {$B$};
\end{tikzpicture}

            \caption{Visualizing Subsets as Blobs}
            \label{fig:Subset_Blobs}
        \end{figure}
        \begin{example}
            Consider the set of natural numbers $\mathbb{N}$ and the set of
            integers $\mathbb{Z}$ (loosely defined in
            Eqn.~\ref{eqn:Natural_Numbers_Ellipses} and
            Eqn.~\ref{eqn:Integers_Ellipses}, respectively). It can be seen that
            every natural number is also an integer, and thus we have:
            \begin{equation}
                \mathbb{N}\subseteq\mathbb{Z}
            \end{equation}
            Letting \gls{mathbbQ} denote the rational
            numbers\index{Rational Numbers} $p/q$, where $p,q\in\mathbb{Z}$ and
            $q$ is non-zero, we can see that $\mathbb{Q}$ contains $\mathbb{Z}$
            as a subset. That is, setting $q=1$ and allowing $p$ to vary over
            $\mathbb{Z}$ gives us every integer. Thus:
            \begin{equation}
                \mathbb{Z}\subseteq\mathbb{Q}
            \end{equation}
            We can continue with the real numbers (\gls{mathbbR}) and the
            complex numbers (\gls{mathbbC}) as well, creating a chain of
            subsets:
            \begin{equation}
                \mathbb{N}\subseteq\mathbb{Z}\subseteq\mathbb{Q}
                \subseteq\mathbb{R}\subseteq\mathbb{C}
            \end{equation}
        \end{example}
        \begin{example}
            Let $\mathbb{Z}_{n}$ be the set of all integers $k\in\mathbb{N}$
            such that $k<n$. If $m,n\in\mathbb{N}$ and $m<n$ we see that:
            \begin{equation}
                \mathbb{Z}_{m}\subseteq\mathbb{Z}_{n}
            \end{equation}
            This is because $\mathbb{Z}_{m}$ is the set of all $k\in\mathbb{N}$
            such that $k<m$. But since $\mathbb{Z}_{n}$ is the set of all
            $k\in\mathbb{N}$ such that $k<n$, and since $m<n$, if
            $k\in\mathbb{Z}_{m}$ then $k<m$, and thus $k<n$, which implies that
            $k\in\mathbb{Z}_{n}$.
        \end{example}
        It is important to note the distinction between the symbols for
        containment (\gls{containmentsymb}) for for subset\index{Set!Subset}
        (\gls{subseteq}). The symbol $\in$ is used to denote that some object
        $x$ is an \textit{element}\index{Set!Element of} of some set. That is,
        $x\in{A}$ indicates that $x$ is an element of $A$. This does not
        necessarily imply $x\subseteq{A}$, but this \textit{does} imply that
        $\{x\}\subseteq{A}$. That is, if $x\in{A}$, then the set that contains
        only $x$ is a subset of $A$. Moreover, the notions are not mutually
        exclusive. It is possible for $A$ to be a set such that $x\in{A}$ and
        $x\subseteq{A}$. For let $A=\{\emptyset\}$. For any set $A$ it is true
        that $\emptyset\subseteq{A}$ (see Thm.~\ref{thm:Emptyset_Is_Subset}).
        But from how $A$ is defined, we have that $\emptyset\in{A}$. Thus it is
        true that both $\emptyset\in{A}$ and $\emptyset\subseteq{A}$.
        \begin{fexample}{Elementary Examples of Subsets}
                        {Elementary_Examples_of_Subsets}
            Let $A$ and $B$ be the sets defined by:
            \par
            \begin{subequations}
                \begin{minipage}[b]{0.49\textwidth}
                    \centering
                    \begin{equation}
                        A=\big\{\,a,\,b,\,c\,\big\}
                    \end{equation}
                \end{minipage}
                \hfill
                \begin{minipage}[b]{0.49\textwidth}
                    \centering
                    \begin{equation}
                        B=\big\{\,a,\,b,\,c,\,d\,\big\}
                    \end{equation}
                \end{minipage}
            \end{subequations}
            \par\vspace{2.5ex}
            where we assume that $a$, $b$, $c$, and $d$ are distinct objects.
            From the definition of subsets (Def.~\ref{def:Subsets}):
            \par
            \begin{subequations}
                \begin{minipage}[b]{0.49\textwidth}
                    \centering
                    \begin{equation}
                        A\subseteq{B}
                    \end{equation}
                \end{minipage}
                \hfill
                \begin{minipage}[b]{0.49\textwidth}
                    \centering
                    \begin{equation}
                        B\nsubseteq{A}
                    \end{equation}
                \end{minipage}
            \end{subequations}
            \par\vspace{2.5ex}
            This is true since from the definition of $A$ and $B$, every element
            of $A$ is also an element of $B$. The converse of this is not true
            since there is an element of $B$ that is not an element of $A$
            (namely, the element $d$). That is, $d\in{B}$ but $d\notin{A}$ and
            therefore $B\nsubseteq{A}$.
        \end{fexample}
        The example shown in Ex.~\ref{ex:Elementary_Examples_of_Subsets} hints
        at how we can redefine equality of sets. We see that $A\subseteq{B}$,
        but $B\nsubseteq{A}$. If we have two sets $A$ and $B$ such that
        $A\subseteq{B}$ and $B\subseteq{A}$, then it would be impossible to
        discern between the two. This gives us our new definition of equality.
        We now prove this equivalence with the axiom of extensionality%
        \index{Axiom!of Extensionality} (Ax.~\ref{ax:Axiom_of_Extensionality}).
        \begin{theorem}
            \label{thm:Equivalent_Def_of_Equality}%
            If $A$ and $B$ are sets, then $A=B$ if and only if $A\subseteq{B}$
            and $B\subseteq{A}$.
        \end{theorem}
        \begin{proof}
            By the axiom of extensionality
            (Ax.~\ref{ax:Axiom_of_Extensionality}), $A=B$ if and only if for
            all $x$ it is true that $x\in{A}$ if and only if $x\in{B}$. But then
            $x\in{A}$ implies that $x\in{B}$, and thus $A\subseteq{B}$
            (Def.~\ref{def:Subsets}). But also $x\in{B}$ implies $x\in{A}$, and
            therefore $B\subseteq{A}$. Therefore if $A=B$, then $A\subseteq{B}$
            and $B\subseteq{A}$. Now if $A\subseteq{B}$ and $B\subseteq{A}$,
            then for all $x\in{A}$ it is true that $x\in{B}$ and for all
            $x\in{B}$ it is true that $x\in{A}$ (Def.~\ref{def:Subsets}), and
            therefore $x\in{A}$ if and only if $x\in{B}$. Thus if
            $A\subseteq{B}$ and $B\subseteq{A}$, then $A=B$. But it was just
            proved that if $A=B$, then $A\subseteq{B}$ and $B\subseteq{A}$.
            Therefore $A=B$ if and only if $A\subseteq{B}$ and $B\subseteq{A}$.
        \end{proof}
        With this, we can redefine the notion of
        \textit{equal sets}\index{Set!Equal Sets}.
        \begin{fdefinition}{Equal Sets}{Equal_Sets}
            \Glspl{equal set} are \glspl{set} $A$ and $B$, denoted
            \glslink{equalsymb}{$A=B$}, such that $A\subseteq{B}$ and
            $B\subseteq{A}$.\index{Set!Equality}
            \begin{equation*}
                \forall_{A}\forall_{B}\Big((A=B)\Longleftrightarrow
                \big((A\subseteq{B})\land(B\subseteq{A})\big)\Big)
            \end{equation*}
        \end{fdefinition}
        Def.~\ref{def:Equal_Sets} is justified by
        Thm.~\ref{thm:Equivalent_Def_of_Equality}, and thus there is no
        contradiction with the axiom of extensionality
        (Ax.~\ref{ax:Axiom_of_Extensionality}). If $A$ and $B$ are not equal, we
        write $A\ne{B}$. 
        \begin{lexample}{More Examples of Subsets}{More_Examples_of_Subsets}
            Using the notation from Ex.~\ref{ex:Using_Element_Notation}, for all
            $n\in\mathbb{N}$ we have:
            \begin{equation}
                \mathbb{Z}_{n}\subseteq\mathbb{N}
            \end{equation}
            Let's define \gls{mathbbNe} and \gls{mathbbNo} to be the sets of
            even\index{Natural Numbers!Even} and odd\index{Natural Numbers!Odd}
            non-negative integers, respectively:
            \par
            \begin{subequations}
                \begin{minipage}[b]{0.49\textwidth}
                    \centering
                    \begin{equation}
                        \label{eqn:Even_Pos_Ints_Ellipses}%
                        \mathbb{N}_{e}=\big\{\,0,\,2,\,4,\,6,\,8,\,\dots\,\big\}
                    \end{equation}
                \end{minipage}
                \hfill
                \begin{minipage}[b]{0.49\textwidth}
                    \centering
                    \begin{equation}
                        \label{eqn:Odd_Pos_Ints_Ellipses}%
                        \mathbb{N}_{o}=\big\{\,1,\,3,\,5,\,7,\,9,\,\dots\,\big\}
                    \end{equation}
                \end{minipage}
            \end{subequations}
            \par\vspace{2.5ex}
            From this we see the following two expressions are true:
            \par
            \begin{subequations}
                \begin{minipage}[b]{0.49\textwidth}
                    \centering
                    \begin{equation}
                        \mathbb{N}_{o}\subseteq\mathbb{N}
                    \end{equation}
                \end{minipage}
                \hfill
                \begin{minipage}[b]{0.49\textwidth}
                    \centering
                    \begin{equation}
                        \mathbb{N}_{e}=\mathbb{N}
                    \end{equation}
                \end{minipage}
            \end{subequations}
            \par\vspace{2.5ex}
            Moreover we see that $\mathbb{N}_{o}$ and $\mathbb{N}_{e}$ have no
            elements in common. That is, they are \textit{disjoint}%
            \index{Set!Disjoint Sets}. From this, we can write:
            \par
            \begin{subequations}
                \begin{minipage}[b]{0.49\textwidth}
                    \centering
                    \begin{equation}
                        \mathbb{N}_{o}\nsubseteq\mathbb{N}_{e}
                    \end{equation}
                \end{minipage}
                \hfill
                \begin{minipage}[b]{0.49\textwidth}
                    \centering
                    \begin{equation}
                        \mathbb{N}_{e}\nsubseteq\mathbb{N}_{o}
                    \end{equation}
                \end{minipage}
            \end{subequations}
            \par\vspace{2.5ex}
            We can also think of trivial examples:
            \par
            \begin{subequations}
                \begin{minipage}[b]{0.49\textwidth}
                    \centering
                    \begin{equation}
                        \mathbb{Z}_{3}\subseteq\mathbb{Z}_{4}
                    \end{equation}
                \end{minipage}
                \hfill
                \begin{minipage}[b]{0.49\textwidth}
                    \centering
                    \begin{equation}
                        \mathbb{Z}_{4}\nsubseteq\mathbb{Z}_{3}
                    \end{equation}
                \end{minipage}
            \end{subequations}
            \par\vspace{2.5ex}
            This is because every element of $\mathbb{Z}_{3}$ is contained in
            $\mathbb{Z}_{4}$, but $3\in\mathbb{Z}_{4}$ and
            $3\notin\mathbb{Z}_{3}$.
        \end{lexample}
        It may seem like bad notation to write $3\notin\mathbb{Z}_{3}$, but
        since we want $\mathbb{Z}_{n}$ to have $n$ elements, and since we
        started counting at zero, we have that $n\notin\mathbb{Z}_{n}$ for all
        $n\in\mathbb{N}$. Such counting schemes are common in computer science,
        but there's disagreement in mathematics as to whether $0\in\mathbb{N}$
        or not. We will use the
        \textit{axiom of infinity}\index{Axiom!of Infinity} to prove the
        existence of $\mathbb{N}$, and in doing so it will be natural to define
        $\mathbb{N}$ as a set that contains $0$.
        \par\hfill\par
        While Def.~\ref{def:Equal_Sets} is indeed equivalent to the axiom of
        extensionality, this definition creates a few problems. As discussed
        previously, sets have no notion of order and cannot account for
        repetition. For let $A$, $B$, and $C$ be the sets defined by:
        \par
        \begin{subequations}
            \begin{minipage}[b]{0.31\textwidth}
                \centering
                \begin{equation}
                    A=\big\{\,a,\,b\,\big\}
                \end{equation}
            \end{minipage}
            \hfill
            \begin{minipage}[b]{0.36\textwidth}
                \centering
                \begin{equation}
                    B=\big\{\,a,\,a,\,b\,\big\}
                \end{equation}
            \end{minipage}
            \hfill
            \begin{minipage}[b]{0.31\textwidth}
                \centering
                \begin{equation}
                    C=\big\{\,b,\,a\,\big\}
                \end{equation}
            \end{minipage}
        \end{subequations}
        \par\vspace{2.5ex}
        All three of these sets are equal by both the definition of equality
        (Def.~\ref{def:Equal_Sets})\index{Set!Equal Sets} and the axiom of
        extensionality\index{Axiom!of Extensionality}. It seems clear that
        $A\subseteq{B}$, but it is also true that $B\subseteq{A}$. This is
        because $B$ contains only the elements $a$ and $b$. While $a$ is
        included twice, repetition cannot be accounted for and $B$ is entirely
        determined by $a$ and $b$. But $A$ also contains $a$ and $b$, and
        therefore $B\subseteq{A}$. By the definition of equality
        (Def.~\ref{def:Equal_Sets}), we have that $A=B$. In a similar manner,
        $A=C$. From the definition of subsets, for any set $A$ we see that
        $A\subseteq{A}$ (see Thm.~\ref{thm:Reflexivity_of_Inclusion}). It would
        be nice to distinguish between subsets that aren't the entire set
        itself. These are called proper subsets\index{Set!Subset!Proper}, and
        we can define them in terms of equality.
        \begin{fdefinition}{Proper Subsets}{Proper_Subsets}
            A \gls{proper subset} of a \gls{set} $B$ is a set $A$ such that
            $A\subseteq{B}$ and $A\ne{B}$. We write
            \glslink{subsetneq}{$A\subsetneq{B}$} to denote that $A$ is a proper
            subset of $B$.\index{Set!Subset!Proper}
            \begin{equation*}
                \forall_{A}\forall_{B}(A\subsetneq{B})
                \Longleftrightarrow\Big((A\subseteq{B})\land(A\ne{B})\Big)
            \end{equation*}
        \end{fdefinition}
        The symbols \gls{subseteq} and \gls{subsetneq} are analogous to the
        notations of inequalities that one finds in calculus: \gls{leq} and
        \gls{lessthan}. In many texts, the two symbols $\subseteq$ and $\subset$
        are taken to be identical, which may cause confusion. In an attempt to
        reduce confusion, $\subseteq$ will denote any subset, $\subsetneq$
        denotes a proper subset, and the symbol $\subset$ will be avoided.
        \begin{lexample}{Proper Subsets}{Proper_Subsets}
            Let $A$ and $B$ be sets defined as follows:
            \par
            \begin{subequations}
                \begin{minipage}[b]{0.49\textwidth}
                    \centering
                    \begin{equation}
                        A=\big\{\,a,\,b,\,c\,\big\}
                    \end{equation}
                \end{minipage}
                \hfill
                \begin{minipage}[b]{0.49\textwidth}
                    \centering
                    \begin{equation}
                        B=\big\{\,a,\,b,\,c,\,d\,\big\}
                    \end{equation}
                \end{minipage}
            \end{subequations}
            \par\vspace{2.5ex}
            Then $A\subseteq{B}$, since every element of $A$ is an element of
            $B$, but $B\nsubseteq{A}$ since $d\in{B}$ and $d\notin{A}$.
            Therefore $A\ne{B}$, and thus $A$ is a proper subset of $B$. We can
            denote this by writing $A\subsetneq{B}$.
        \end{lexample}
        Proper subsets are subsets that are missing at least one element
        (see Thm.~\ref{thm:Prop_Subset_Not_Equal}). Returning to our claim that
        $\emptyset\subseteq{A}$ for any set $A$, for any non-empty set $A$ we
        have that $\emptyset\subsetneq{A}$. This is because for non-empty sets
        there is at least one $x$ such that $x\in{A}$
        (Def.~\ref{def:Non_Empty_Set}), whereas for all $x$ it is true that
        $x\notin\emptyset$. Thus equality cannot occur, and the empty set must
        be a proper subset. It is also true that the empty set contains no
        proper subsets. The only subset of $\emptyset$ is itself.
        \begin{example}
            Returning to more concrete examples, $\mathbb{N}$ is a proper subset
            of $\mathbb{Z}$. To see this, note that $\minus{1}\in\mathbb{Z}$ but
            $\minus{1}\notin\mathbb{N}$. Indeed, none of the negative integers
            are natural numbers, but they are integers. We can write this by:
            \begin{equation}
                \mathbb{N}\subsetneq\mathbb{Z}
            \end{equation}
            Similarly, $\mathbb{Q}$ contains numbers that are not integers,
            for example $1/2$. Thus, $\mathbb{Z}$ is also a proper subset of
            $\mathbb{Q}$. Lastly, since $\sqrt{2}$ is not a rational number, the
            set of rational numbers must then be a proper subset of the set of
            real numbers.
        \end{example}
        We now introduce the \textit{axiom schema of specification}.
        \begin{faxiom}{Axiom Schema of Specification}
                      {Axiom_Schema_of_Specification}
            If $A$ is a set and if $P$ is a proposition, then there exists a set
            $B$ such that $x\in{B}$ if and only if $x\in{A}$ and $P(x)$ is true.
            We can write this as:\index{Axiom!Schema of Specification}
            \index{Proposition}
            \begin{equation*}
                B=\big\{\,x\in{A}\;|\;P(x)\,\big\}
            \end{equation*}
            Using our formal language, we have:
            \begin{equation*}
                \forall_{A}\forall_{P}\exists_{B}:
                \forall_{x}\Big((x\in{B})\Leftrightarrow
                \big((x\in{A})\land{P}(x)\big)\Big)
            \end{equation*}
        \end{faxiom}
        Ax.~\ref{ax:Axiom_Schema_of_Specification} is different from the
        inconsistent axiom of unrestricted comprehension%
        \index{Axiom!of Unrestricted Comprehension} in that we can only speak of
        elements that are already defined and contained in some other set. That
        is, this new axiom does not allow us to talk about the
        \textit{set of all sets}\index{Set!of All Sets}, and so we have avoided
        the crux of Russell's paradox.\index{Russell's Paradox}
        \par\hfill\par
        This allows us to use the Set-Builder\index{Set-Builder Notation} method
        of constructing sets. We described the natural numbers \gls{mathbbN} and
        integers \gls{mathbbZ} (From the German \textit{Zahl}) using
        Eqns.~\ref{eqn:Natural_Numbers_Ellipses} and
        \ref{eqn:Integers_Ellipses}, respectively. It would be more difficult
        (but not impossible) to describe the set of rational numbers%
        \index{Rational Numbers} in such a way. Instead, we use set-builder
        notation if it is known that \gls{mathbbQ} is contained in some larger
        set \gls{mathbbR} (the \textit{real} numbers)\index{Real Numbers}.
        \begin{equation}
            \mathbb{Q}=\Big\{\;\frac{p}{q}\in\mathbb{R}\;\big|\;
                                p,\,q\in\mathbb{Z}\textrm{ and }q\ne{0}\;\Big\}
        \end{equation}
        That is, the rational numbers are the set of all real numbers which can
        be written as the ratios of integers with non-zero denominator. The
        Axiom Schema of Specification states that this is is a valid method of
        describing sets. It is also known as the axiom of separation%
        \index{Axiom!of Separation}.
        \begin{example}
            We can describe the sets $\mathbb{Z}$, $\mathbb{N}$,
            $\mathbb{N}_{e}$, and $\mathbb{N}_{o}$ using set-builder notation if
            we assume these belong to some larger set $\mathbb{R}$. We define
            $\mathbb{Z}$ by:
            \index{Natural Numbers}\index{Natural Numbers!Even}%
            \index{Natural Numbers!Odd}\index{Integers}
            \begin{equation}
                \mathbb{Z}=
                \big\{\,n\in\mathbb{R}\;|\;n\textrm{ is an integer}\,\big\}
            \end{equation}
            From here we can define $\mathbb{N}$ by:
            \begin{equation}
                \mathbb{N}=\{\,n\in\mathbb{Z}\;|\;n\geq{0}\,\}
            \end{equation}
            Furthermore, $\mathbb{N}_{e}$ and $\mathbb{N}_{0}$ can be described
            as follows:
            \par
            \begin{subequations}
                \begin{minipage}[b]{0.495\textwidth}
                    \centering
                    \begin{equation}
                        \label{eqn:Even_Pos_Ints_Set_Builder}%
                        \mathbb{N}_{e}=
                        \big\{n\in\mathbb{N}\;|\;n\textrm{ is even}\big\}
                    \end{equation}
                \end{minipage}
                \hfill
                \begin{minipage}[b]{0.495\textwidth}
                    \centering
                    \begin{equation}
                        \label{eqn:Odd_Pos_Ints_Set_Builder}%
                        \mathbb{N}_{o}=
                        \big\{n\in\mathbb{N}\;|\;n\textrm{ is odd}\big\}
                    \end{equation}
                \end{minipage}
            \end{subequations}
            \par\vspace{2.5ex}
            Such notation is justified by the axiom schema of specification%
            \index{Axiom!Schema of Specification}.
        \end{example}
        We are not adopting these definitions since they lack rigor. These
        examples build intuition behind the notation and the axioms, but we will
        develop arithmetic axiomatically using the \textit{axiom of infinity}%
        \index{Axiom!of Infinity}.
    \subsection{Ordered Pairs and Unions}
        We now wish to solve the issue previously raised that sets do
        not have order. We'll develop a new object, called ordered pairs%
        \index{Ordered Pair}, that can distinguish such things. The definition
        we'll adopt is due to Kuratowski\index{Kuratowski, Kazimierz} and
        defines \gls{orderedpairsymb} as follows:
        \begin{equation}
            (a,\,b)=\big\{\,\{\,a\,\},\,\{\,a,\,b\,\}\,\big\}
        \end{equation}
        We now prove such a set exists within the framework of \gls{ZFC}.
        \begin{faxiom}{Axiom of Pairing}{Axiom_of_Pairing}
            If $A$ and $B$ are sets, then there exists a set $\mathcal{C}$
            such that $A\in\mathcal{C}$ and $B\in\mathcal{C}$.
            \index{Axiom!of Pairing}
            \begin{equation*}
                \forall_{A}\forall_{B}\exists_{\mathcal{C}}:
                \Big((A\in\mathcal{C})\land(B\in\mathcal{C})\Big)
            \end{equation*}
        \end{faxiom}
        The set hypothesized to exist in this axiom may be very large, we have
        no way of knowing. What we want from this is a set that contains two
        elements $A$ and $B$, and only those elements. We obtain this by
        combining pairing with specification.
        \begin{theorem}
            \label{thm:Existence_of_Set_Built_from_Two_Sets}%
            If $A$ and $B$ are sets, then there exists a set $D$ such that
            for all $x$ it is true that $x\in{D}$ if and only if $x=A$ or
            $x=B$. That is:
            \begin{equation}
                D=\{\,A,\,B\,\}
            \end{equation}
        \end{theorem}
        \begin{proof}
            By the axiom of pairing (Ax.~\ref{ax:Axiom_of_Pairing}) there
            exists a set $\mathcal{C}$ such that $A\in\mathcal{C}$ and
            $B\in\mathcal{C}$. Let $P$ be the proposition
            \textit{true if} $x=A$ \textit{or} $x=B$, \textit{false otherwise}.
            By the axiom schema of specification
            (Ax.~\ref{ax:Axiom_Schema_of_Specification}), there is a set
            $D$ such that:
            \begin{equation}
                D=\{\,x\in\mathcal{C}\;|\;P(x)\,\}
            \end{equation}
            That is, $x\in{D}$ if and only if $x\in\mathcal{C}$ and $P(x)$ is
            true. But then $x\in{D}$ if and only if $x\in\mathcal{C}$ and $x=A$
            or $x\in\mathcal{C}$ and $x=B$. But $A\in\mathcal{C}$ and
            $B\in\mathcal{C}$, and thus $P(x)$ implies $x\in\mathcal{C}$. Thus,
            $x\in{D}$ if and only if $P(x)$ is true. That is, $x\in{D}$ if and
            only if $x=A$ or $x=B$.
        \end{proof}
        By the axiom of extensionality\index{Axiom!of Extensionality}
        (Ax.~\ref{ax:Axiom_of_Extensionality}), the set hypothesized in
        Thm.~\ref{thm:Existence_of_Set_Built_from_Two_Sets} is unique, and thus
        there is no trouble in \textit{defining} the symbol $\{A,B\}$ to be the
        unique set that contains the elements $A$ and $B$ and only those
        elements. That is, we develop the new notation:
        \begin{fnotation}{Finite Set Notation}{Finite_Set_Notation}
            If $A$ and $B$ are sets, then $\{A,B\}$ is the unique set such that
            for all $x$, $x\in\{A,B\}$ if and only if $x=A$ or $x=B$.
            \index{Set!Finite}
            \begin{equation*}
                \forall_{x}\Big(\big(x\in\{A,B\}\big)
                \Longleftrightarrow\big((x=A)\lor(x=B)\big)\Big)
            \end{equation*}
        \end{fnotation}
        \begin{theorem}
            \label{thm:Existence_of_Set_Containing_Set}%
            If $A$ is a set, then there is a set $B$ such that $x\in{B}$ if
            and only if $x=A$. That is, there exists a set $B$ such that:
            \begin{equation}
                B=\{\,A\,\}
            \end{equation}
        \end{theorem}
        \begin{proof}
            For since $A$ is a set, by
            Thm.~\ref{thm:Existence_of_Set_Built_from_Two_Sets} there exists
            a set $B=\{A,\,A\}$. But then $x\in{B}$ if and only if $x=A$.
        \end{proof}
        We can apply Not.~\ref{not:Finite_Set_Notation} to a single set $A$ and
        similarly define what the notation $\{A\}$ means. With this, we can now
        prove the existence of ordered pairs.
        \begin{ltheorem}{Existence of Ordered Pairs}{Existence_of_Ordered_Pairs}
            If $A$ and $B$ are sets, then there is a set $(A,\,B)$ such that
            for all $x$ it is true that $x\in(A,\,B)$ if only if $x=\{A\}$
            or $x=\{A,B\}$.\index{Ordered Pair}
        \end{ltheorem}
        \begin{proof}
            For by Thm.~\ref{thm:Existence_of_Set_Containing_Set}, there is
            a set $\{A\}$ such that $x\in\{A\}$ if and only if $x=A$.
            But by Thm.~\ref{thm:Existence_of_Set_Built_from_Two_Sets}, there
            is a set $\{A,\,B\}$ such that $x\in\{A,\,B\}$ and if only
            if $x=A$ or $x=B$. But again by
            Thm.~\ref{thm:Existence_of_Set_Built_from_Two_Sets}, since
            $\{A\}$ and $\{A,\,B\}$ are sets, there is a set $(A,\,B)$ such
            that $x\in(A,\,B)$ if and only if $x=\{A\}$ or $x=\{A,\,B\}$.
        \end{proof}
        \begin{fdefinition}{Ordered Pairs}{Ordered_Pairs}
            The \gls{ordered pair} of a \gls{set} $x$ with respect to a set
            $y$ is the set:\index{Ordered Pair}
            \begin{equation*}
                (x,\,y)=\big\{\,\{\,x\,\},\,\{\,x,\,y\,\}\,\big\}
            \end{equation*}
            Using our formal language:
            \begin{equation*}
                \forall_{x}\forall_{y}\forall_{z}\Big(
                    \big(z\in(x,y)\big)\Longleftrightarrow
                    \big((z=\{x\})\lor(z=\{x,y\})\big)
                \Big)
            \end{equation*}
        \end{fdefinition}
        Thm.~\ref{thm:Existence_of_Ordered_Pairs} asserts the existence of
        ordered pairs\index{Ordered Pair}, as defined by Kuratowski%.
        \index{Kuratowski, Kazimierz}, and allows us to present
        Def.~\ref{def:Ordered_Pairs} in a way that is consistent with ZFC.
        Kuratowski first put forward this definition in 1921 and it does
        precisely what we want it to do and orders elements. That is, if $x$ and
        $y$ are distinct, then $(x,\,y)\ne(y,\,x)$. The caveat with this
        definition is the following reduction:
        \begin{equation}
            (x,\,x)
            =\big\{\,\{\,x\,\},\,\{\,x,\,x\,\}\,\big\}
            =\big\{\,\{\,x\,\},\{\,x\,\}\,\big\}
            =\big\{\,\{\,x\,\}\,\big\}
        \end{equation}
        Prior to Kuratowski there existed a definition due to Norbert Wiener%
        \index{Wiener, Norbert}, put forward in 1914. His definition grew out of
        Bertrand Russell's\index{Russell, Bertrand} Type
        Theory\index{Type Theory} which was an attempt to rid set theory of the
        paradoxes he discovered. Wiener writes:
        \begin{equation}
            (x,\,y)_{W}=\Big\{\,\big\{\,\{\,x\,\},\,\emptyset\,\big\},\,
                                \big\{\,\{\,y\,\}\,\big\}\Big\}
        \end{equation}
        Returning to Kuratowski's definition (Def.~\ref{def:Ordered_Pairs}),
        consider the ordered pair $(1,\,2)$, where we take for granted that
        $1\ne{2}$. We have:
        \begin{equation}
            (1,\,2)=\big\{\,\{\,1\,\},\,\{\,1,\,2\,\}\,\big\}
        \end{equation}
        Swapping and computing $(2,\,1)$, we obtain:
        \begin{equation}
            (2,\,1)=\big\{\,\{\,2\,\},\,\{\,2,\,1\,\}\,\big\}
        \end{equation}
        We know that sets cannot distinguish order, so
        $\{\,1,\,2\,\}=\{\,2,\,1\,\}$. Thus:
        \par
        \begin{subequations}
            \begin{minipage}[b]{0.49\textwidth}
                \centering
                \begin{equation}
                    (1,\,2)=\big\{\,\{\,1\,\},\,\{\,1,\,2\,\}\,\big\}
                \end{equation}
            \end{minipage}
            \hfill
            \begin{minipage}[b]{0.49\textwidth}
                \centering
                \begin{equation}
                    (2,\,1)=\big\{\,\{\,2\,\},\,\{\,1,\,2\,\}\,\big\}
                \end{equation}
            \end{minipage}
        \end{subequations}
        \par\vspace{2.5ex}
        Combining these equations, we now have that:
        \begin{equation}
            (1,\,2)\ne(2,\,1)
        \end{equation}
        To see this, note that both sets contain the element $\{1,\,2\}$, but
        $\{1\}$ is an element of $(1,\,2)$ and not an element of $(2,\,1)$,
        and thus $(1,\,2)\nsubseteq(2,\,1)$. Similarly, $\{2\}$ is an element
        of $(2,\,1)$ but not an element of $(1,\,2)$, and therefore
        $(2,\,1)\nsubseteq(1,\,2)$. From the definition of equality
        (Def.~\ref{def:Equal_Sets}), we have that these sets are not equal.
        \par\hfill\par
        There's is an unfortunate doubling of notation that occurs in
        mathematics, and $(a,b)$ has two common meanings. The first meaning is
        the ordered pair which we've just defined, and the second is the
        \textit{open interval}\index{Interval!Open} defined in the context of a
        \textit{partially ordered set}\index{Partially Ordered Set}.
        The most common example is when discussing the real numbers
        $\mathbb{R}$, $(a,b)$ denotes the set of all real numbers $x$ such that
        $a<x$ and $x<b$. Hopefully it will be clear what the notation means when
        a theorem or example is being presented, but we will be explicit when
        ambiguity can arise.
        \par\hfill\par
        The natural thing from here is to construct the
        \textit{Cartesian Product}\index{Cartesian Product} of two sets. This is
        the set of all ordered pairs\index{Ordered Pair} $(a,\,b)$ where $a$
        belongs to some set $A$ and $b$ belongs to another set $B$. To prove
        such a set exists requires two more axioms.
        \newpage
        \begin{faxiom}{Axiom of Union}{Axiom_of_Union}
            If $\mathcal{O}$ is a set, then there exists a set $\mathcal{F}$
            such that, for all $A$ such that $A\in\mathcal{O}$ and for all
            $x$ such that $x\in{A}$, it is true that $x\in\mathcal{F}$.%
            \index{Axiom!of Union}
            \begin{equation*}
                \forall_{\mathcal{O}}\exists_{\mathcal{F}}:\forall_{x}\Big(
                \big(\exists_{A\in\mathcal{O}}:x\in{A}\big)
                \Rightarrow{x}\in\mathcal{F}\Big)
            \end{equation*}
        \end{faxiom}
        This states that, given a collection of sets $\mathcal{O}$, there exists
        a larger set which contains the elements of the constituent sets of
        $\mathcal{O}$. Similar to the axiom of pairing, $\mathcal{F}$ may be
        much larger than desired and we must invoke the axiom schema of
        specification to arrive at the \textit{union} over a collection.
        \begin{ltheorem}{Existence of the Union of Sets}{Existence_of_Unions}
            If $\mathcal{O}$ is a set, then there exists a set \gls{unionO} such
            that for all $x$ it is true that $x\in\bigcup\mathcal{O}$ if and
            only if there is a set $A\in\mathcal{O}$ such that $x\in{A}$.
        \end{ltheorem}
        \begin{proof}
            For by the axiom of union (Ax.~\ref{ax:Axiom_of_Union}), there
            exists a set $\mathcal{F}$ such that for all $A\in\mathcal{O}$
            and for all $x\in{A}$ it is true that $x\in\mathcal{F}$. Let
            $P$ be the proposition \textit{true if there exists a set}
            $A\in\mathcal{O}$ \textit{such that} $x\in{A}$,
            \textit{false otherwise}. Then, by the axiom schema of specification
            (Ax.~\ref{ax:Axiom_Schema_of_Specification}) there exists a set
            $\bigcup\mathcal{O}$ such that:
            \begin{equation}
                \bigcup\mathcal{O}=\big\{\,x\in\mathcal{F}\;|\;P(x)\,\big\}
            \end{equation}
            But then $x\in\bigcup\mathcal{O}$ if and only if $x\in\mathcal{F}$
            and $P(x)$ is true. But if $P(x)$ is true, then $x\in\mathcal{F}$,
            and thus $x\in\bigcup\mathcal{O}$ if and only if there is a set
            $A\in\mathcal{O}$ such that $x\in{A}$.
        \end{proof}
        One question that arises is
        \textit{what happens if our collection is empty}? That is, if
        $\mathcal{O}=\emptyset$, is there any meaning behind the equation:
        \begin{equation}
            \mathcal{F}=\bigcup\emptyset
        \end{equation}
        There is, and $\mathcal{F}$ will be the empty set. That is,
        $\mathcal{F}=\emptyset$. This is true in a vacuous sense and can be
        proved via contradiction with the law of the excluded middle.
        We define the set $\mathcal{F}$ described in
        Thm.~\ref{thm:Existence_of_Unions} as the \textit{union}%
        \index{Union (Sets)} over the set $\mathcal{O}$. The set $\mathcal{O}$
        is often called the index set\index{Index Set} for which we take the
        union over.
        \begin{fdefinition}{Union over a Set}{Union_over_a_Set}
            The \gls{union over a set} $\mathcal{O}$ is the set:
            \index{Set!Union}\index{Union!over a Collection}
            \begin{equation*}
                \bigcup_{\mathcal{U}\in\mathcal{O}}\mathcal{U}
                =\big\{\,x\;|\;\textrm{There exists a set }
                         \mathcal{U}\in\mathcal{O}\textrm{ such that }
                         x\in\mathcal{U}\big\}
            \end{equation*}
            Using our formal language:
            \begin{equation*}
                \forall_{\mathcal{O}}\forall_{x}\bigg(
                    \Big(x\in\bigcup_{\mathcal{U}\in\mathcal{O}}\mathcal{U}\Big)
                    \Longleftrightarrow\big(\exists_{\mathcal{U}\in\mathcal{O}}:
                    x\in\mathcal{U}\big)\bigg)
            \end{equation*}
        \end{fdefinition}
        There are two ways to write unions for arbitrary collections, and we
        will make use of both depending on scenario. The first manner we have
        already seen in Thm.~\ref{thm:Existence_of_Unions} where, given a
        collection $\mathcal{O}$, we wrote \gls{unionO} to denote the union over
        $\mathcal{O}$. The second is depicted in
        Def.~\ref{def:Union_over_a_Set}. That is, given $\mathcal{O}$ we write:
        \begin{equation}
            \bigcup\mathcal{O}=\bigcup_{\mathcal{U}\in\mathcal{O}}\mathcal{U}
        \end{equation}
        This alternative notation can be useful when we are using various
        indexing tricks to solve problems, or when combining unions with the
        various other set operations such as differences and intersections.
        \par\hfill\par
        The notion of union is very convenient if we already have a collection
        of sets defined, but it would be nice to form the union over two given
        sets without considering them as part of a larger collection. This can
        be done by combining the axiom of union\index{Axiom!of Union} with
        pairing\index{Axiom!of Pairing}.
        \begin{theorem}
            \label{thm:Union_of_Two_Sets_Exists}%
            If $A$ and $B$ are sets, then there exists a set \gls{AcupB} such
            that $x\in{A}\cup{B}$ if and only if either $x\in{A}$ or $x\in{B}$.
        \end{theorem}
        \begin{proof}
            For by Thm.~\ref{thm:Existence_of_Set_Built_from_Two_Sets} there
            exists a set $\mathcal{O}$ such that $y\in\mathcal{O}$ if and only
            if $y=A$ or $y=B$. That is, $\mathcal{O}=\{A,\,B\}$. But by
            Thm.~\ref{thm:Existence_of_Unions} there exists a set $A\cup{B}$
            such that $x\in{A}\cup{B}$ if and only if there exists a set
            $\mathcal{U}\in\mathcal{O}$ such that $x\in\mathcal{U}$. But then
            $x\in{A}\cup{B}$ if and only if either $x\in{A}$ or $x\in{B}$.
        \end{proof}
        This allows us to define our first \textit{operation} of two sets.
        \begin{fdefinition}{Union of Two Sets}{Union_of_Two_Sets}
            The \gls{union of two sets} $A$ and $B$ is the set \gls{AcupB}
            defined by:\index{Union!of Two Sets}
            \begin{equation*}
                A\cup{B}=\big\{\,x\;|\;x\in{A}\textrm{ or }x\in{B}\,\big\}
            \end{equation*}
            That is:
            \begin{equation*}
                \forall_{A}\forall_{B}\forall_{x}\Big(
                    \big(x\in{A}\cup{B}\big)
                    \Longleftrightarrow
                    \big((x\in{A})\lor(x\in{B})\big)
                \Big)
            \end{equation*}
        \end{fdefinition}
        In our definition of the union over a collection and the union of two
        sets we have slightly abused our set-builder
        notation\index{Set-Builder Notation}. The axiom schema of
        specification\index{Axiom!Schema of Specification} allows us to write a
        set as $A=\{\,x\in{B}\,|\,P(x)\,\}$ given some set $B$ that is already
        known to exists, and some proposition\index{Proposition} $P$. These two
        definitions (Defs.~\ref{def:Union_over_a_Set} and
        \ref{def:Union_of_Two_Sets}) are justified by the theorems we have
        proven, and so there is no contradiction.
        \begin{example}
            \label{ex:Union_of_Zm_and_Zn}%
            Again using the notation found in Eqn.~\ref{eqn:Z_n_Ellipses}, if we
            let $\mathbb{Z}_{n}$ denote the integers between $0$ and $n-1$, we
            have the following: If $m$ is less than $n$, then:
            \begin{equation}
                \mathbb{Z}_{m}\cup\mathbb{Z}_{n}=\mathbb{Z}_{n}
            \end{equation}
            This is because every element of $\mathbb{Z}_{m}$ is already
            and element of $\mathbb{Z}_{n}$, and thus taking the union adds
            nothing new to $\mathbb{Z}_{n}$, so the resulting set is
            $\mathbb{Z}_{m}$.
        \end{example}
        \begin{example}
            Denoting the even and odd non-negative integers by \gls{mathbbNe}
            and \gls{mathbbNo}, respectively, we see that:
            \index{Natural Numbers!Even}\index{Natural Numbers!Odd}
            \begin{equation}
                \mathbb{N}_{e}\cup\mathbb{N}_{o}=\mathbb{N}
            \end{equation}
            This is because every non-negative integer $n\in\mathbb{N}$ is
            either even or odd, and thus either $n\in\mathbb{N}_{e}$ or
            $n\in\mathbb{N}_{o}$. Taking the union therefore gives the entire
            set $\mathbb{N}$. The union does not add anything more than
            $\mathbb{N}$ since $\mathbb{N}_{e}\subseteq\mathbb{N}$ and
            $\mathbb{N}_{o}\subseteq\mathbb{N}$.
        \end{example}
        \begin{fexample}{Union of Two Sets}{Union_of_Two_Sets}
            Let $A$ and $B$ be the sets defined by:
            \par
            \begin{subequations}
                \begin{minipage}[b]{0.49\textwidth}
                    \centering
                    \begin{equation}
                        A=\{\,a,\,b,\,c\,\}
                    \end{equation}
                \end{minipage}
                \hfill
                \begin{minipage}[b]{0.49\textwidth}
                    \centering
                    \begin{equation}
                        B=\{\,c,\,1,\,2\,\}
                    \end{equation}
                \end{minipage}
            \end{subequations}
            \par\vspace{2.5ex}
            The union of $A$ and $B$ is the set that contains all of the
            elements of $A$ and all of the elements of $B$, and only such
            elements. That is:
            \begin{equation}
                A\cup{B}=\{\,a,\,b,\,c,\,1,\,2\,\}
            \end{equation}
            Even though $c\in{A}$ and $c\in{B}$, $c$ only appears once in the
            union. This is because sets cannot account for repetition, so
            including $c$ twice would be redundant.
        \end{fexample}
        The union of two sets can again be visualized by considering blobs
        in the plane. Let $A$ and $B$ be two circles that overlap somewhere in
        the middle. The union $A\cup{B}$ can then be represented by shading in
        the region covered by either $A$ or $B$
        (see Fig.~\ref{fig:Union_of_Two_Sets}). Such a drawing is called a
        \textit{Venn diagram}\index{Venn Diagram}.
        \begin{figure}[H]
            \centering
            \captionsetup{type=figure}
            %--------------------------------Dependencies----------------------------------%
%   tikz                                                                       %
%-------------------------------Main Document----------------------------------%
\begin{tikzpicture}[line width=0.2mm]

    % Coordinates for the centers of the circles.
    \coordinate (C1) at (-1.3, 0);
    \coordinate (C2) at ( 1.3, 0);

    % Coordinates for the labels.
    \coordinate (A) at (-1.3, 1.2);
    \coordinate (B) at ( 1.3, 1.2);
    \coordinate (U) at ( 0.0, 2.5);

    % Rectangle indicating the universe set.
    \draw (-4, -3) rectangle (4, 3);

    % Fill in the circle with cyan.
    \draw[fill=cyan, draw=none] (C1) circle (2);
    \draw[fill=cyan, draw=none] (C2) circle (2);

    % Give outlines to the circles.
    \draw (C1) circle (2);
    \draw (C2) circle (2);

    % Labels.
    \node at (A) {$A$};
    \node at (B) {$B$};
    \node at (U) {$A\cup{B}$};
\end{tikzpicture}
            \caption{Venn Diagram for Union}
            \label{fig:Union_of_Two_Sets}
        \end{figure}
        Fig.~\ref{fig:Union_of_Two_Sets} can be extended to an
        arbitrary collection of sets. For the sake of simplicity, a Venn
        diagram for the union of three sets is shown in
        Fig.~\ref{fig:Union_of_Three_Sets}.
        \begin{figure}[H]
            \centering
            \captionsetup{type=figure}
            \begin{tikzpicture}
    % Coordinates for the centers of the circles.
    \coordinate (C1) at (-1.3,  0.00);
    \coordinate (C2) at ( 1.3,  0.00);
    \coordinate (C3) at ( 0.0, -2.15);

    % Coordinates for the labels.
    \coordinate (A) at (-1.3, 1.2);
    \coordinate (B) at ( 1.3, 1.2);
    \coordinate (C) at ( 0.0, -3.5);
    \coordinate (U) at ( 0.0, 2.5);

    % Rectangle indicating the universe set.
    \draw (-4, -4.5) rectangle (4, 3);

    % Fill in the circle with cyan.
    \draw[fill=cyan, draw=none] (C1) circle (2);
    \draw[fill=cyan, draw=none] (C2) circle (2);
    \draw[fill=cyan, draw=none] (C3) circle (2);

    % Give outlines to the circles.
    \draw (C1) circle (2);
    \draw (C2) circle (2);
    \draw (C3) circle (2);

    % Labels.
    \node at (A) {$A$};
    \node at (B) {$B$};
    \node at (C) {$C$};
    \node at (U)
        {$\Large{\underset{{\mathcal{U}\in\{A,B,C\}}}{\bigcup}\mathcal{U}}$};
\end{tikzpicture}
            \caption{The Union of Three Sets}
            \label{fig:Union_of_Three_Sets}
        \end{figure}
        We can combine the axiom schema of specification
        (Ax.~\ref{ax:Axiom_Schema_of_Specification}) with the existence of the
        union of two sets to define intersections. The intersection of two sets,
        denoted \gls{AcapB}, is the set consisting of all elements that lie in
        both $A$ and $B$ simultaneously.
        \begin{theorem}
            If $A$ and $B$ are sets, then there exists a set \gls{AcapB}
            such that for all $x$ it is true that $x\in{A}\cap{B}$ if and
            only if $x\in{A}$ and $x\in{B}$.
        \end{theorem}
        \begin{proof}
            For by Thm.~\ref{thm:Union_of_Two_Sets_Exists}, there exists
            a set $A\cup{B}$ such that for all $x$ is is true that
            $x\in{A}\cup{B}$ if and only if $x\in{A}$ or $x\in{B}$. Let
            $P$ be the proposition \textit{True if} $x\in{A}$ \textit{and}
            $x\in{B}$, \textit{false otherwise}. Then by the axiom schema
            of specification (Ax.~\ref{ax:Axiom_Schema_of_Specification})
            there is a set $A\cap{B}$ such that:
            \begin{equation}
                A\cap{B}=\big\{\,x\in{A}\cup{B}\;|\;P(x)\,\big\}
            \end{equation}
            But then $x\in{A}\cap{B}$ if and only if $x\in{A}\cup{B}$ and
            $x\in{A}$ and $x\in{B}$. But if $x\in{A}$ and $x\in{B}$, then
            $x\in{A}$, and thus $x\in{A}\cup{B}$
            (Def.~\ref{def:Union_of_Two_Sets}). That is, $P(x)$ implies that
            $x\in{A}\cap{B}$. Therefore, $x\in{A}\cap{B}$ if and only if $P(x)$
            is true. That is, $x\in{A}\cap{B}$ if and only if $x\in{A}$ and
            $x\in{B}$.
        \end{proof}
        \begin{fdefinition}{Intersection of Two Sets}
                           {Intersection_of_Two_Sets}
            The \gls{intersection of two sets} $A$ and $B$, denoted \gls{AcapB},
            is the set:\index{Intersection!of Two Sets}
            \begin{equation*}
                A\cap{B}
                =\big\{\,x\in{A}\cup{B}\;|\;
                    a\in{A}\textrm{ and }b\in{B}\,\big\}
            \end{equation*}
            That is:
            \begin{equation*}
                \forall_{A}\forall_{B}\forall_{x}\Big(
                    \big(x\in{A}\cap{B}\big)
                    \Leftrightarrow
                    \big((x\in{A})\land(x\in{B})\big)
                \Big)
            \end{equation*}
        \end{fdefinition}
        \begin{example}
            Using Eqn.~\ref{eqn:Z_n_Ellipses} to represent $\mathbb{Z}_{n}$,
            we can see that if $m<n$ then:
            \begin{equation}
                \mathbb{Z}_{m}\cap\mathbb{Z}_{n}=\mathbb{Z}_{m}
            \end{equation}
            Since $m<n$, every element of $\mathbb{Z}_{m}$ is contained in
            $\mathbb{Z}_{n}$. But every element of $\mathbb{Z}_{n}$ that is
            not contained in $\mathbb{Z}_{m}$ will not be in the intersection.
            This is the opposite of the pattern we saw in
            Ex.~\ref{ex:Union_of_Zm_and_Zn} when we considered the union of
            $\mathbb{Z}_{m}$ and $\mathbb{Z}_{n}$. This spells out a general
            theorem: If $A\subseteq{B}$, then $A\cap{B}=A$ and $A\cup{B}=B$
            (see Thm.~\ref{thm:Intersection_with_Subset} and
            \ref{thm:Union_With_Subset}, respectively).
        \end{example}
        \begin{fexample}{Intersections of Two Sets}
                        {Intersections_of_Two_Sets}
            If we let $A$ and $B$ be the sets defined by:
            \par
            \begin{subequations}
                \begin{minipage}[b]{0.49\textwidth}
                    \centering
                    \begin{equation}
                        A=\{\,a,\,b,\,c\,\}
                    \end{equation}
                \end{minipage}
                \hfill
                \begin{minipage}[b]{0.49\textwidth}
                    \centering
                    \begin{equation}
                        B=\{\,c,\,1,\,2\,\}
                    \end{equation}
                \end{minipage}
            \end{subequations}
            \par\vspace{2.5ex}
            we have that the intersection is then:
            \begin{equation}
                A\cap{B}=\{\,1\,\}
            \end{equation}
            This is because $1$ is the only element that appears in both $A$ and
            $B$, and is hence the only member of $A\cap{B}$.
        \end{fexample}
        \begin{example}
            Recalling our comment in Ex.~\ref{ex:More_Examples_of_Subsets},
            we claimed that the set of even integers (\gls{mathbbNe}) and the
            set of odd integers (\gls{mathbbNo}) are
            \textit{disjoint}\index{Set!Disjoint Sets}. We can now be precise
            about what this means. Since even numbers are of the form $2n$ and
            odd numbers are of the form $2n+1$, there are no natural numbers
            $k\in\mathbb{N}$ that are both even and odd. Thus:
            \begin{equation}
                \mathbb{N}_{e}\cap\mathbb{N}_{o}=\emptyset
            \end{equation}
            This is our definition of disjoint sets: Those with empty
            intersection.
        \end{example}
        \begin{fdefinition}{Disjoint Sets}{Disjoint_Sets}
            \Gls{disjoint sets} are \glspl{set} $A$ and $B$ such that
            $A\cap{B}=\emptyset$.\index{Set!Disjoint Sets}
        \end{fdefinition}
        We'll need one brief theorem about intersections to allow us to prove
        that certain sets are not equal.
        \begin{theorem}
            \label{thm:Lemma_for_Anti_Russells_Paradox}%
            If $A$ and $B$ are sets, if $x\in{B}$, and if $x\notin{A}\cap{B}$,
            then $x\notin{A}$.
        \end{theorem}
        \begin{proof}
            For if $x\notin{A}\cap{B}$ then either $x\notin{A}$ or $x\notin{B}$
            (Def.~\ref{def:Intersection_of_Two_Sets}). But $x\in{B}$, and
            therefore $x\notin{A}$.
        \end{proof}
        Similar to how unions can be visualized with Venn diagrams
        (Fig.~\ref{fig:Union_of_Two_Sets}), so can the intersection of
        two sets. We draw two circles that overlap slightly, and consider the
        region contained in both (see Fig.~\ref{fig:Intersection_of_Two_Sets}).
        \index{Venn Diagram}
        \begin{figure}[H]
            \centering
            \captionsetup{type=figure}
            %--------------------------------Dependencies----------------------------------%
%   tikz                                                                       %
%-------------------------------Main Document----------------------------------%
\begin{tikzpicture}[line width=0.2mm]
    % Coordinates for the centers of the circles.
    \coordinate (C1) at (-1.3, 0);
    \coordinate (C2) at ( 1.3, 0);

    % Coordinates for the labels.
    \coordinate (A) at (-1.3, 1.2);
    \coordinate (B) at ( 1.3, 1.2);
    \coordinate (I) at ( 0.0, 2.5);

    % Rectangle indicating the universe set.
    \draw (-4, -3) rectangle (4, 3);

    % Fill in the circle with cyan.
    \draw[fill=cyan] (0, -1.51987) arc(-49.46:49.46:2) arc(130.54:229.46:2);

    % Give outlines to the circles.
    \draw (C1) circle (2);
    \draw (C2) circle (2);

    % Labels.
    \node at (A) {$A$};
    \node at (B) {$B$};
    \node at (I) {$A\cap{B}$};
\end{tikzpicture}
            \caption{Venn Diagram for Intersection}
            \label{fig:Intersection_of_Two_Sets}
        \end{figure}
        We can extend this further and define the intersection over any
        collection of sets.
        \begin{ltheorem}{Existence of the Intersection of Sets}
                        {thm:Existence_of_Arbitrary_Intersetions}
            If $\mathcal{O}$ is a set, then there exists a set \gls{intersectO}
            such that for all $x$ it is true that $x\in\bigcap\mathcal{O}$ if
            and only if $x\in\bigcup\mathcal{O}$ and for all
            $\mathcal{U}\in\mathcal{O}$ it is true that $x\in\mathcal{U}$.
        \end{ltheorem}
        \begin{proof}
            For by Thm.~\ref{thm:Existence_of_Unions} there is a set
            $\bigcup\mathcal{O}$ such that for all $x$ it is true that
            $x\in\bigcup\mathcal{O}$ if and only if there exists a
            $\mathcal{U}\in\mathcal{O}$ such that $x\in\mathcal{U}$. Let
            $P$ be the proposition \textit{True if for all}
            $\mathcal{U}\in\mathcal{O}$ \textit{it is true that}
            $x\in\mathcal{U}$, \textit{false otherwise}. Then by the
            axiom schema of specification
            (Ax.~\ref{ax:Axiom_Schema_of_Specification}), there exists the set:
            \begin{equation}
                \bigcap\mathcal{O}
                =\Big\{\,x\in\bigcup\mathcal{O}
                    \;|\;P(x)\,\Big\}
            \end{equation}
            But then $x\in\bigcap\mathcal{O}$ if and only if
            $x\in\bigcup\mathcal{O}$ and $P(x)$ is true. That is,
            $x\in\bigcap\mathcal{O}$ if and only if $x\in\bigcup\mathcal{O}$
            and for all $\mathcal{U}\in\mathcal{O}$ it is true that
            $x\in\mathcal{U}$.
        \end{proof}
        It is common to consider some \textit{universal} set, of which all
        other sets of current consideration are drawn from. Using this the
        definition of arbitrary intersection is defined as the subset of
        this universal set such that every element of this subset is
        contained in every element of the arbitrary collection. One may then
        ask what would happen if the collection is empty. Using this
        definition the intersection would be the entire universal set
        in a vacuous sense. That is, there would be no $x$ in the universe
        that fails the definition of the intersection over an empty
        collection, and thus the intersection is everything. Letting $X$
        denote our universe, we obtain:
        \begin{equation}
            \emptyset=\bigcup_{\mathcal{U}\in\emptyset}\mathcal{U}
                \subseteq\bigcap_{\mathcal{U}\in\emptyset}\mathcal{U}
            =X
        \end{equation}
        It seems like unions should always be bigger. Indeed, for any
        non-empty collection, the intersection over the collection is a
        subset of the union over the collection. Because of this we do
        not adopt this definition of the intersection over a collection,
        but rather require in our construction the use of the union over
        the collection, and then use the axiom schema of specification to
        pick the subset of all elements of the union that belong to every
        element of the collection. Thus:
        \begin{equation}
            \bigcap_{\mathcal{U}\in\emptyset}\mathcal{U}
            \subseteq\bigcup_{\mathcal{U}\in\emptyset}\mathcal{U}
            =\emptyset
        \end{equation}
        And from this we conclude the intersection is empty as well.
        \begin{fdefinition}{Intersection Over a Collection}
                           {Intersection_Over_a_Collection}
            The \gls{intersection over a set} $\mathcal{O}$
            of sets is the set \gls{intersectO} defined by:
            \begin{equation*}
                \bigcap_{\mathcal{U}\in\mathcal{O}}\mathcal{U}
                =\Big\{\,x\in\bigcup_{\mathcal{U}\in\mathcal{O}}\mathcal{U}
                    \;\Big|\;x\in\mathcal{U}\textrm{ for all }
                    \mathcal{U}\in\mathcal{O}\,\Big\}
            \end{equation*}
            That is:
            \begin{equation*}
                \forall_{\mathcal{O}}\forall_{x}\bigg(
                    \Big(x\in\bigcap_{\mathcal{U}\in\mathcal{O}}\mathcal{U}\Big)
                    \Longleftrightarrow\Big(
                        \big(
                            x\in\bigcup_{\mathcal{U}\in\mathcal{O}}\mathcal{U}
                        \big)
                        \land\big(
                            \forall_{\mathcal{U}\in\mathcal{O}}
                            (x\in\mathcal{U})
                        \big)
                    \Big)
                \bigg)
            \end{equation*}
        \end{fdefinition}
        We can extend our Venn diagram for larger collections as well
        (see Fig.~\ref{fig:Intersection_of_Three_Sets}).
        \begin{figure}[H]
            \centering
            \begin{tikzpicture}
    % Coordinates for the centers of the circles.
    \coordinate (C1) at (-1.3,  0.00);
    \coordinate (C2) at ( 1.3,  0.00);
    \coordinate (C3) at ( 0.0, -2.15);

    \coordinate (O)  at ( 0.0000, -0.7500);
    \coordinate (P1) at ( 0.6817, -0.2697);
    \coordinate (P2) at (-0.6817, -0.2697);
    \coordinate (P3) at ( 0.0000, -1.5198);

    % Coordinates for the labels.
    \coordinate (A) at (-1.3, 1.2);
    \coordinate (B) at ( 1.3, 1.2);
    \coordinate (C) at ( 0.0, -3.5);
    \coordinate (U) at ( 0.0, 2.5);

    % Rectangle indicating the universe set.
    \draw (-4, -4.5) rectangle (4, 3.3);

    % Fill in the intersection with cyan.
    \draw[fill=cyan, draw=none] (P1) arc(70.07:109.93:2)
                                     arc(187.8:229.50:2)
                                     arc(310.54:352.25:2);

    % Give outlines to the circles.
    \draw (C1) circle (2);
    \draw (C2) circle (2);
    \draw (C3) circle (2);

    % Labels.
    \node at (A) {$A$};
    \node at (B) {$B$};
    \node at (C) {$C$};
    \node at (U)
        {\Large{$\underset{\mathcal{U}\in\{A,B,C\}}{\bigcap}\mathcal{U}$}};
\end{tikzpicture}
            \caption{The Intersection of Three Sets}
            \label{fig:Intersection_of_Three_Sets}
        \end{figure}
        Much the way we've defined what it means for two sets to be disjoint, we
        can extend this to arbitrary collections. We'll say that a collections
        of sets is mutually disjoint if all distinct elements of the collection
        have empty intersection. That is, all distinct pairs are disjoint.
        \begin{fdefinition}{Mutually Disjoint Collection of Sets}
                           {Mutually_Disjoint_Collection_of_Sets}
            A mutually disjoint collection of sets is a set $\mathcal{O}$ such
            that for all $A,B\in\mathcal{O}$ such that $A\ne{B}$ it is true that
            $A\cap{B}=\emptyset$. That is, for all distinct elements
            $A,B\in\mathcal{O}$ it is true that $A$ and $B$ are
            \glslink{disjoint sets}{disjoint}.
            \index{Mutually Disjoint Collection}
        \end{fdefinition}
        The term pairwise disjoint\index{Pairwise Disjoint} is also frequently
        used in measure theory and probability. As such, one might guess that
        the notion has a fair amount of use in these subjects. Next on the list
        of axioms is that of \textit{regularity}.
        \begin{faxiom}{Axiom of Regularity}{Axiom_of_Regularity}
            If $A$ is a non-empty set, then there is an element $B\in{A}$
            such that $A\cap{B}=\emptyset$.\index{Axiom!of Regularity}
            \begin{equation*}
                \forall_{A}(\exists_{x\in{A}})\Rightarrow
                \exists_{y}:\Big((y\in{A})\land
                \big((y\cap{A})=\emptyset\big)\Big)
            \end{equation*}
        \end{faxiom}
        This axiom is often seen as unnecessary by many working mathematicians
        and indeed it's use seems to only lie in set theory and foundations.
        That is, unlike the axioms of choice and union which are widely
        applicable to analysis and topology, regularity seems to only be useful
        to set theorists. This is not entirely true if one really pays attention
        to the details. Often we are presented some object $X$, perhaps a
        topological space, or an algebraic structure like a group, and we need
        to extend this object to a larger collection. One concrete example comes
        from topology when we have a \textit{non-compact} space $(X,\tau)$ and
        we want to find a \textit{compact} space $(\tilde{X},\tilde{\tau})$ that
        contains our original space. In the construction we add a single point
        to $(X,\tau)$ called \textit{infinity}, and label it $\infty$. But what
        if the symbol $\infty$ already belongs to $X$? How do we find a new
        object that is guarenteed not to lie in $X$? The axiom of regularity
        allows us to show that $\{X\}$ does not lie in $X$, for any set $X$, and
        thus we can take this to be our new point. This, and many similar
        constructions, rely on the axiom of regularity to guarentee that our
        reasoning is not ultimately circular.
        \par\hfill\par
        Regardless of the axioms application, its existence is vital to support
        the claim that ZFC is a good system to base mathematics on. We will
        combine this with pairing to prove that for any set $A$ it is true that
        $A\notin{A}$. That is, Zermelo-Fraenkel set theory is free of Russell's
        paradox\index{Russell's Paradox}.
        \begin{theorem}
            \label{thm:Anti_Russells_Paradox}%
            If $A$ is a set, then $A\notin{A}$.
        \end{theorem}
        \begin{proof}
            For if $A$ is a set, then $\{A\}$ is a set
            (Thm.~\ref{thm:Existence_of_Set_Containing_Set}). But since
            $A\in\{A\}$, $\{A\}$ is a non-empty set
            (Def.~\ref{def:Non_Empty_Set}). Thus by the axiom of regularity
            (Ax.~\ref{ax:Axiom_of_Regularity}) there is a set $B\in\{A\}$ such
            that $B\cap\{A\}=\emptyset$. But $B\in\{A\}$ if and only if
            $B=A$, and therefore $A\cap\{A\}=\emptyset$. Thus, by the axiom of
            the empty set (Ax.~\ref{ax:Axiom_of_the_Empty_Set}), for all $x$ it
            is true that $x\notin{A}\cap\{A\}$ and therefore
            $A\notin{A}\cap\{A\}$. But $A\in\{A\}$ and therefore
            $A\notin{A}$ (Thm.~\ref{thm:Lemma_for_Anti_Russells_Paradox}).
        \end{proof}
        \begin{theorem}
            \label{thm:Containment_NEqual_Underlying_Set}%
            If $A$ and $B$ are sets and if $A\in{B}$, then $A\ne{B}$.
        \end{theorem}
        \begin{proof}
            For $A\notin{A}$ (Thm.~\ref{thm:Anti_Russells_Paradox}) and
            $A\in{B}$ and therefore it is not true that for all $x$, $x\in{A}$
            if and only if $x\in{B}$. Therefore, by the Axiom of
            Extensionality (Ax.~\ref{ax:Axiom_of_Extensionality}), $A\ne{B}$.
        \end{proof}
        \begin{theorem}
            \label{thm:Cor_of_Containment_NEqual_Underlying_Set}%
            If $A$ is a set, then $A\ne\{A\}$.
        \end{theorem}
        \begin{proof}
            For if $A$ is a set, then $A\notin{A}$
            (Thm.~\ref{thm:Anti_Russells_Paradox}). But if $A$ is a set, then
            $\{A\}$ is a set (Thm.~\ref{thm:Existence_of_Set_Containing_Set}).
            But $A\in\{A\}$, and thus $A\ne\{A\}$
            (Thm.~\ref{thm:Containment_NEqual_Underlying_Set}).
        \end{proof}
        These quick theorems will eventually prove the well known result that
        $0\ne{1}$. It also shows us that there is no set of all
        sets\index{Set!of All Sets}.
    \subsection{The Axiom of the Power Set}
        Continuing in our goal of constructing order, we move on to the
        Cartesian product\index{Cartesian Product} of two sets $A$ and $B$. This
        is the collection of all ordered pairs $(a,b)$ such that $a\in{A}$ and
        $b\in{B}$. To prove such a set exists requires the
        \textit{axiom of the power set}.
        \begin{faxiom}{Axiom of the Power Set}{Axiom_of_the_Power_Set}
            If $A$ is a set, then there exists a set $\mathscr{P}$ such that
            for all $x\subseteq{A}$ it is true that $x\in\mathscr{P}$.
            \index{Axiom!of the Power Set}
            \begin{equation*}
                \forall_{A}\exists_{\mathscr{P}}:
                \forall_{x}\Big((x\subseteq{X})\Rightarrow(x\in\mathscr{P})\Big)
            \end{equation*}
        \end{faxiom}
        Again, much like the axiom of union and the axiom of pairing, this
        set may be bigger than we would like. We wish to find a set, called
        the \textit{power set}, that contains all of the subsets of a given
        set $A$ and nothing else. Combining the axiom of the power set
        with the axiom schema of specification gives us such existence.
        \begin{ltheorem}{Existence of the Power Set}
                        {Existence_of_the_Power_Set}
            If $A$ is a set, then there exists a set \gls{powersetsymb}
            such that for all $x$ it is true that $x\in\mathcal{P}(A)$ if and
            only if $x\subseteq{A}$.\index{Set!Power Set}
        \end{ltheorem}
        \begin{proof}
            For by the axiom of the power set
            (Ax.~\ref{ax:Axiom_of_the_Power_Set}) there is a set $\mathscr{P}$
            such that for all $x\subseteq{A}$ it is true that $x\in\mathscr{P}$.
            Let $P$ be the proposition \textit{true if} $x\subseteq{A}$,
            \textit{false otherwise}. By the axiom schema of specification
            (Ax.~\ref{ax:Axiom_Schema_of_Specification}), there is a set
            $\mathcal{P}(A)$ such that:
            \begin{equation}
                \mathcal{P}(A)=\{\,x\in\mathscr{P}(A)\;|\;P(x)\,\}
            \end{equation}
            But if $P(x)$ is true, then $x$ is a subset of $A$, and therefore
            $x\in\mathscr{P}(A)$. Thus $x\in\mathcal{P}(A)$ if and only if
            $x\subseteq{A}$.
        \end{proof}
        With this we now define the \textit{power set} of a given set.
        \newpage
        \begin{fdefinition}{Power Set}{Power_Set}
            The \gls{power set} of a \gls{set} $A$ is the set $\mathcal{P}(A)$
            defined by:\index{Power Set}\index{Set!Power Set}
            \begin{equation*}
                \mathcal{P}(A)=\{\,x\;|\;x\subseteq{X}\,\}
            \end{equation*}
            That is, the set of all subsets of $X$.
            \begin{equation*}
                \forall_{A}\forall_{B}\Big(B\in\mathcal{P}(A)\Longleftrightarrow
                    B\subseteq{A}\Big)
            \end{equation*}
        \end{fdefinition}
        Again, there is some abuse of our set-builder notation, but
        Thm.~\ref{thm:Existence_of_the_Power_Set} justifies such a definition.
        The power set of a set is a crucial construction for when one discusses
        the \textit{cardinality} of sets, denoted $\textrm{Card}(A)$. This
        describes the \textit{size} of a set in a very precise manner. A theorem
        that will eventually be proved known as
        \textit{Cantor's Theorem}\index{Theorem!Cantor's Power Set Theorem}
        shows that the power set of a set is always strictly \textit{larger}
        than the original set. That is:
        \begin{equation}
            \textrm{Card}(A)<\textrm{Card}\big(\mathcal{P}(A)\big)
        \end{equation}
        This will be made precise soon enough. The axiom of the power set allows
        us to build \textit{larger} sets from a given set. This creates a
        paradoxical heirarchy of infinities. Starting with the \textit{smallest}
        infinity, the natural numbers $\mathbb{N}$, we can create a
        significantly larger set by considering $\mathcal{P}(\mathbb{N})$. We
        can continue and consider $\mathcal{P}(\mathcal{P}(\mathbb{N}))$, and
        there's no reason to stop there. At each step we create a new, massively
        larger set. This is both unintuitive and paradoxical and as such some
        may choose to reject it. This axiom is vital in the discussion of
        topology and measure theory, and so we choose to accept it as true.
        \begin{example}
            If $A=\{1,2\}$, then the power set is:
            \begin{equation}
                \mathcal{P}(A)=\big\{\,\emptyset,\,\{1\},\,\{2\},\,
                    \{1,2\}\,\big\}
            \end{equation}
            We must consider the empty set since $\emptyset\subseteq{A}$.
            Now suppose $A=\{1,2,3\}$:
            \begin{equation}
                \mathcal{P}(A)=\big\{\,\emptyset,\,\{1\},\,\{2\},\,\{3\},\,
                    \{1,2\},\,\{1,3\},\,\{2,3\},\,\{1,2,3\}\,\big\}
            \end{equation}
            We see that a set with 2 elements has a power set with 4 elements
            and a set with 3 elements has a power set with 8. This pattern
            continues for finite sets and if $A$ has $n$ elements, then
            $\mathcal{P}(A)$ has $2^{n}$ elements.
        \end{example}
        \begin{example}
            Let $A=\{a_{1},\dots,a_{n}\}$, where all of the elements $a_{k}$ are
            distinct. To count the total number of subsets we first note that
            there is one set that contains zero elements, the empty set. Next,
            there are $n$ subsets that contain one element, these are the sets
            $\{a_{k}\}$. There are $n(n-1)/2$ sets that contain two elements,
            $\{a_{i},a_{j}\}$, such that $i\ne{j}$. Continuing, we see that
            there are $\binom{n}{k}$ subsets that contain $k$ elements, where
            $\binom{n}{k}$ is the \textit{binomial coefficient}%
            \index{Binomial Coefficient}. This is defined in terms of the
            factorial function:
            \begin{subequations}
                \begin{align}
                    \binom{n}{k}
                    &=\frac{n!}{k!(n-k)!}\\
                    &=\frac{n\cdot(n-1)\cdots{2}\cdot{1}}
                        {k\cdot(k-1)\cdots{2}\cdot{1}\cdot(n-k)
                        \cdot(n-k-1)\cdots{2}\cdot{1}}\\
                    &=\frac{n\cdot(n-1)\cdots\cdot(n-k+1)}
                        {k\cdot(k-1)\cdots{2}\cdot{1}}
                \end{align}
            \end{subequations}
            To avoid undefined ratios, we define $0!=1$. Note then that
            $\binom{n}{n}=1$. This says that the number of ways to choose $n$
            element subsets from $A$ is 1. This makes sense since the only $n$
            element subset of $A$ is the entirety of $A$! To compute the size of
            $\mathcal{P}(A)$ it now suffices to sum over all of these binomial
            coefficients. Such a task can be achieved by invoking the
            \textit{binomial theorem}\index{Theorem!Binomial Theorem}. Given
            a positive integer $n$ and two real numbers $x$ and $y$, the
            binomial theorem states that:
            \begin{subequations}
                \begin{align}
                    (x+y)^{n}
                    &=\sum_{k=0}^{n}\binom{n}{k}x^{n-k}y^{k}\\
                    &=\binom{n}{0}x^{n}+\binom{n}{1}x^{n-1}y+\cdots+
                        \binom{n}{n-1}xy^{n-1}+\binom{n}{n}y^{n}
                \end{align}
            \end{subequations}
            Here, the notation $\Sigma$ is simply shorthand for denoting a long
            sum. For example:
            \par
            \begin{subequations}
                \begin{minipage}[b]{0.49\textwidth}
                    \centering
                    \begin{equation}
                        \sum_{n=1}^{3}n=1+2+3=6
                    \end{equation}
                \end{minipage}
                \hfill
                \begin{minipage}[b]{0.49\textwidth}
                    \centering
                    \begin{equation}
                        \sum_{n=1}^{3}n^{2}=1+2^{2}+3^{2}=14
                    \end{equation}
                \end{minipage}
            \end{subequations}
            \par\vspace{2.5ex}
            Setting $x=y=1$, we obtain:
            \begin{equation}
                2^{n}=\sum_{k=0}^{n}\binom{n}{k}
            \end{equation}
            and this is precisely the number of elements of $\mathcal{P}(A)$.
        \end{example}
        \begin{example}
            When we consider the case of an \textit{infinite} set $A$ we have
            that $\mathcal{P}(A)$ is a strictly larger set and this creates a
            paradoxical heirarchy of infinities. The smallest heirarchy is
            that of the \textit{countable} infinite sets, like $\mathbb{N}$.
            Everything larger is called \textit{uncountable}. It will be
            shown that the following is true:
            \begin{equation}
                \Card\big(\mathcal{P}(\mathbb{N})\big)=
                \Card(\mathbb{R})
            \end{equation}
            where again $\mathbb{N}$ denotes the non-negative integers and
            $\mathbb{R}$ denotes the set of all \textit{real} numbers. We
            can loosely show this by using the binary representation of real
            numbers. A real number may be thought of as an infinite decimal.
            For example, $\pi=3.1415926\dots$ and $1=1.000\dots$ We can
            also represent real numbers as a sequence of zeroes and ones and
            this is the \textit{binary} representation. For
            $A\subseteq\mathbb{N}$ and let $r_{A}=0.n_{1}n_{2}\hdots$ where:
            \begin{equation}
                n_{i}=
                \begin{cases}
                    0,&i\notin{A}\\
                    1,&i\in{A}
                \end{cases}
            \end{equation}
            Thus for each $A\in\mathcal{P}(\mathbb{N})$ there is a real
            number $r_{A}$ such that $0\leq{r}_{A}\leq{1}$ that is
            associated with it, and moreover to every real number between
            zero and one there is a subset of $\mathbb{N}$ associated with
            it. The tricky numbers to see are zero and one, but note that
            $r_{\emptyset}$ is associated to 0 and $r_{\mathbb{N}}$ gets
            paired with 1. To show that $\mathbb{R}$ and
            $\mathcal{P}(\mathbb{N})$ are the same size requires us to
            refine this association so that every element of
            $\mathcal{P}(\mathbb{N})$ uniquely corresponds to an element of
            $\mathbb{R}$, and vice-versa.
        \end{example}
    \subsection{Cartesian Products and Functions}
        Previously we've introduced ordered pairs and the notion of the power
        set. We can use both of these concepts to define and prove the existence
        of \textit{Cartesian products}. Intuitively we want to define
        $A\times{B}$ to be the set of all ordered pairs $(a,b)$ where $a\in{A}$
        and $b\in{B}$:
        \begin{equation}
            A\times{B}=\big\{\,(a,\,b)\;|\;a\in{A}\textrm{ and }b\in{B}\,\big\}
        \end{equation}
        But recalling Def.~\ref{def:Ordered_Pairs}, ordered pairs are sets
        of the form $\{\{a\},\{a,b\}\}$. Thus elements of $A\times{B}$ are
        contained in the power set of the power set of $A\cup{B}$:
        \begin{equation}
            A\times{B}\subseteq\mathcal{P}\big(\mathcal{P}(A\cup{B})\big)
        \end{equation}
        We can combine the axiom of the power set with the axiom schema of
        specification to obtain the existence of the Cartesian product of
        two sets.\index{Axiom!of the Power Set}
        \index{Axiom!Schema of Specification}
        \begin{theorem}
            \label{thm:Ordered_Pair_Subset_of_Power_Set}%
            If $A$ and $B$ are sets, if $a\in{A}$ and $b\in{B}$, then
            $(a,b)\subseteq\mathcal{P}(A\cup{B})$.
        \end{theorem}
        \begin{proof}
            For if $a\in{A}$ and $b\in{B}$, then
            $(a,b)=\{\,\{\,a\,\},\,\{\,a,\,b\,\}\,\}$
            (Def.~\ref{def:Ordered_Pairs}). But if $a\in{A}$, then $a\in{A}$
            or $a\in{B}$, and thus $a\in{A}\cup{B}$
            (Def.~\ref{def:Union_of_Two_Sets}). But then
            $\{\,a\,\}\subseteq{A}\cup{B}$ (Def.~\ref{def:Subsets}). But if
            $b\in{B}$, then $b\in{A}$ or $b\in{B}$, and thus $b\in{A}\cup{B}$
            (Def.~\ref{def:Union_of_Two_Sets}). But then
            $\{\,a,\,b\,\}\subseteq{A}\cup{B}$ (Def.~\ref{def:Subsets}).
            But then $\{\,a\,\}\subseteq{A}\cup{B}$ and
            $\{\,a,\,b\,\}\subseteq{A}\cup{B}$, and thus
            $(a,b)\subseteq\mathcal{P}(A\cup{B})$ (Def.~\ref{def:Power_Set}).
        \end{proof}
        \begin{ltheorem}{Existence of the Cartesian Product}
                        {Existence_of_the_Cartesian_Product}
            If $A$ and $B$ are sets, then there exists a set $A\times{B}$
            such that, for all $z$, $z\in{A}\times{B}$ if and only if there
            is an $a\in{A}$ and $b\in{B}$ such that $z=(a,b)$.
            \index{Cartesian Product}
        \end{ltheorem}
        \begin{proof}
            For if $A$ and $B$ are sets, then $A\cup{B}$ is a set
            (Thm.~\ref{thm:Union_of_Two_Sets_Exists}). But if $A\cup{B}$ is a
            set, then $\mathcal{P}(A\cup{B})$ is a set
            (Thm.~\ref{thm:Existence_of_the_Power_Set}), where $\mathcal{P}(X)$
            denotes the power set of $X$. But if $\mathcal{P}(A\cup{B})$ is a
            set, then $\mathcal{P}(\mathcal{P}(A\cup{B}))$ is a set
            (Thm.~\ref{thm:Existence_of_the_Power_Set}). But then
            $z\in\mathcal{P}(\mathcal{P}(A\cup{B}))$ if and only if
            $z\subseteq\mathcal{P}(A\cup{B})$ (Def.~\ref{def:Power_Set}).
            But if $a\in{A}$ and $b\in{B}$, then
            $(a,b)\subseteq\mathcal{P}(A\cup{B})$
            (Thm.~\ref{thm:Ordered_Pair_Subset_of_Power_Set}), and therefore
            $(a,b)\in\mathcal{P}(\mathcal{P}(A\cup{B}))$. Let $P$ be the
            proposition \textit{True if there exists} $a\in{A}$ \textit{and}
            $b\in{B}$ \textit{such that} $z=(a,b)$, \textit{false otherwise}.
            Then by the axiom schema of specification
            (Ax.~\ref{ax:Axiom_Schema_of_Specification}), there exists a
            set $A\times{B}$ such that:
            \begin{equation}
                A\times{B}=
                \{\,z\in\mathcal{P}\big(\mathcal{P}(A\cup{B})\big)\;|\;
                    P(z)\,\}
            \end{equation}
            But it was proved that $P(z)$ implies that
            $z\in\mathcal{P}(\mathcal{P}(A\cup{B}))$. Thus $z\in{A}\times{B}$
            if and only if there exists $a\in{A}$ and $b\in{B}$
            such that $z=(a,b)$.
        \end{proof}
        \begin{fdefinition}{Cartesian Product of Two Sets}
                           {Cartesian_Product_of_Two_Sets}
            The \gls{Cartesian product} of two \glspl{set} $A$ and $B$ is the
            set:\index{Cartesian Product}
            \begin{equation*}
                A\times{B}
                =\{\,(a,\,b)\;|\;a\in{A}\textrm{ and }b\in{B}\,\}
            \end{equation*}
            Formally:
            \begin{equation*}
                \forall_{A}\forall_{B}\forall_{z}\Big(
                    (z\in{A}\times{B})\Longleftrightarrow
                    \big(\exists_{x\in{A}}\land\exists_{y\in{B}}:z=(x,y)\big)
                \Big)
            \end{equation*}
        \end{fdefinition}
        Note that since, in general, $(a,b)\ne(b,a)$, it is generally true that
        $A\times{B}\ne{B}\times{A}$. Indeed, equality occurs if and only if
        $A=B$ (or if either set is empty).
        \begin{fexample}{Basic Cartesian Products}{Basic_Cartesian_Products}
            Let $A$ and $B$ be sets defined as follows:
            \par
            \begin{subequations}
                \begin{minipage}[b]{0.49\textwidth}
                    \centering
                    \begin{equation}
                        A=\{\,1,\,2,\,3\,\}
                    \end{equation}
                \end{minipage}
                \hfill
                \begin{minipage}[b]{0.49\textwidth}
                    \centering
                    \begin{equation}
                        B=\{\,a,\,b\,\}
                    \end{equation}
                \end{minipage}
            \end{subequations}
            \par\vspace{2.5ex}
            Let's compute $A\times{B}$ and $B\times{A}$. From the definition
            (Def.~\ref{def:Cartesian_Product_of_Two_Sets}) we have:
            \begin{equation}
                A\times{B}=\{\,(a,b)\;|\;a\in{A}\textrm{ and }b\in{B}\,\}
            \end{equation}
            Using this, we can compute:
            \begin{equation}
                A\times{B}=\big\{\,(1,a),\,(2,a),\,(3,a),\,
                                   (1,b),\,(2,b),\,(3,b)\,\big\}
            \end{equation}
            Computing $B\times{A}$, we have:
            \begin{equation}
                B\times{A}=\big\{\,(a,\,1),\,(a,\,2),\,(a,\,3),\,
                                   (b,\,1),\,(b,\,2),\,(b,\,3)\,\big\}
            \end{equation}
            Now if we suppose that $a$ is not equal to 1, then we see that
            $(a,1)$ is a different element than $(1,a)$, and thus $A\times{B}$
            is not equal to $B\times{A}$. Next, compute $A\times{A}$:
            \begin{equation}
                \begin{split}
                    A\times{A}=\Big\{\,(1,1),\,(1,2),\,&(1,3),
                                       (2,1),\,(2,2),\,\\&(2,3),
                                       (3,1),\,(3,2),\,(3,3)\,\Big\}
                \end{split}
            \end{equation}
            And finally $B\times{B}$:
            \begin{equation}
                B\times{B}=\big\{\,(a,\,a),\,(a,\,b),
                                 \,(b,\,a),\,(b,\,b)\,\big\}
            \end{equation}
            Equality of $A\times{B}$ and $B\times{A}$ is achieved if and only
            if $A=B$, or if either set is the empty set.
        \end{fexample}
        Note that in Ex.~\ref{ex:Basic_Cartesian_Products}, the \textit{size}
        of the Cartesian product of two sets was simply the product of the
        number of elements of the constituent sets. That is, we see that $A$
        has three elements and $B$ has two elements, but also that
        $A\times{B}$ has six elements. Moreover, $A\times{A}$ has nine
        elements and $B\times{B}$ has four. This pattern holds for the
        Cartesian products of any two \textit{finite} sets.
        \index{Set!Finite}
        \par\hfill\par
        It is common to consider the Cartesian product of a set with itself.
        That is, given a set $A$, we are often interested in $A\times{A}$. We
        denote this by writing $A^{2}$. One such example is when we consider
        the set of real numbers $\mathbb{R}$. The Cartesian product
        $\mathbb{R}^{2}$ is called the \textit{Euclidean Plane}, or the
        \textit{Cartesian Plane}\index{Euclidean Plane}\index{Cartesian Plane},
        after Euclid of Alexandria\index{Euclid of Alexandria} and Ren\'{e}
        Descartes\index{Descartes, Ren\'{e}}. This is because $\mathbb{R}^{2}$
        is used to model both planar geometry and analytical geometry, of which
        Euclid and Descartes were pioneers of, respectively. The term Cartesian
        products is in honor of Ren\'{e} Descartes, as well. Let $\mathbb{R}$
        denote the set of real numbers, and let $A=\mathbb{R}$ and
        $B=\mathbb{R}$. Then we have:
        \begin{equation}
            A\times{B}=\mathbb{R}\times\mathbb{R}\equiv\mathbb{R}^{2}
        \end{equation}
        Where the symbol \gls{equiv} means that $\mathbb{R}^{2}$ is defined by
        this expression. Using the definition of Cartesian products
        (Def.~\ref{def:Cartesian_Product_of_Two_Sets}), we obtain:
        \begin{equation}
            \mathbb{R}^{2}=\{\;(x,y)\,:\,x\in\mathbb{R}
                               \textrm{ and }y\in\mathbb{R}\;\}
        \end{equation}
        That is, $\mathbb{R}^{2}$ is the set of all ordered pairs of real
        numbers. The first term is called the $x$ coordinate, and similarly the
        second term is called the $y$ coordinate. We envision this as a
        \textit{plane} of points, each one corresponding to an ordered pair
        $(x,y)$. This is depicted in Fig.~\ref{fig:Cartesian_Plane}.
        \begin{figure}[H]
            \centering
            %--------------------------------Dependencies----------------------------------%
%   amssymb                                                                    %
%   tikz                                                                       %
%       arrows.meta                                                            %
%-------------------------------Main Document----------------------------------%
\begin{tikzpicture}[%
    >=Latex,
    line width=0.2mm,
    line cap=round,
    font=\Large
]
    % Coordinates for the points.
    \coordinate (x) at (2.2, 0.0);
    \coordinate (y) at (0.0, 2.9);
    \coordinate (z) at (2.2, 2.9);

    % Draw a grid.
    \draw[style=help lines] (-0.3, -0.3) grid (7.9, 7.9);

    % Axes.
    \begin{scope}[thick]
        \draw[->] (-0.3, 0) to (8.4, 0) node [above] {$\mathbb{R}$};
        \draw[->] (0, -0.3) to (0, 8.4) node [right] {$\mathbb{R}$};
    \end{scope}

    % Draw dashed lines to the point.
    \begin{scope}[densely dashed]
        \draw (x) to (z);
        \draw (y) to (z);
    \end{scope}

    % Draw dots marking the various points.
    \draw[fill=black] (x) circle (0.6mm);
    \draw[fill=black] (y) circle (0.6mm);
    \draw[fill=black] (z) circle (0.6mm);

    \node at (x) [below=0.1]     {$x$};
    \node at (y) [left=0.1]      {$y$};
    \node at (z) [above right]   {$(x,\,y)$};
\end{tikzpicture}
            \caption{The Cartesian Plane $\mathbb{R}^{2}$}
            \label{fig:Cartesian_Plane}
        \end{figure}
        Consider further the set $\mathbb{N}^{2}$. That is, letting
        $\mathbb{N}$ denote the set of natural numbers
        (Eqn.~\ref{eqn:Natural_Numbers_Ellipses}), letting $A=\mathbb{N}$ and
        $B=\mathbb{N}$ we have:\index{Integer Lattice}
        \begin{equation}
            A\times{B}=\mathbb{N}\times\mathbb{N}\equiv\mathbb{N}^{2}
        \end{equation}
        Again using the definition of Cartesian products
        (Def.~\ref{def:Cartesian_Product_of_Two_Sets}), we have:
        \begin{equation}
            \mathbb{N}^{2}=
            \{\,(n,\,m)\;|\;n\in\mathbb{N}\textrm{ and }m\in\mathbb{N}\,\}
        \end{equation}
        We can visualize this as a subset of $\mathbb{R}^{2}$ by drawing a
        lattice of points in the Cartesian plane
        (Fig.~\ref{fig:Lattice_Cart_Prod_of_N_with_N}).
        \begin{figure}[H]
            \centering
            %--------------------------------Dependencies----------------------------------%
%   amssymb                                                                    %
%   tikz                                                                       %
%       arrows.meta                                                            %
%-------------------------------Main Document----------------------------------%
\begin{tikzpicture}[%
    >=Latex,
    line width=0.2mm,
    line cap=round
]

    % Axes.
    \begin{scope}[thick, font=\Large]
        \draw[->] (0, 0) to (8.4, 0) node [above] {$\mathbb{N}$};
        \draw[->] (0, 0) to (0, 8.4) node [right] {$\mathbb{N}$};
    \end{scope}

    \foreach\x in{1, 2, 3, 4, 5, 6, 7, 8}{
        \foreach\y in{1, 2, 3, 4, 5, 6, 7, 8}{
            \draw[fill=black] (\x, \y) circle (0.2mm);
        }
        \draw (\x, -0.1) to (\x, 0.1) node [below=1ex] {$\x$};
        \draw (-0.1, \x) to (0.1, \x) node [left=1ex]  {$\x$};
    }
\end{tikzpicture}
            \caption{The Lattice $\mathbb{N}^{2}$}
            \label{fig:Lattice_Cart_Prod_of_N_with_N}
        \end{figure}
        This can then be consider a subset of the Euclidean plane
        $\mathbb{R}^{2}$. That is, at every ordered pair of integers $(m,n)$,
        we place a point in the Euclidean plane whose $x$ coordinate is $m$ and
        whose $y$ coordinate is $n$. We can also be more abstract and general in
        our examples. Consider the following sets:
        \par
        \begin{subequations}
            \begin{minipage}[b]{0.49\textwidth}
                \centering
                \begin{equation}
                    A=\{\,\textrm{Point, Line 1, Line 2}\,\}
                \end{equation}
            \end{minipage}
            \hfill
            \begin{minipage}[b]{0.49\textwidth}
                \centering
                \begin{equation}
                    B=\{\,\textrm{Point, Line}\,\}
                \end{equation}
            \end{minipage}
        \end{subequations}
        \par\vspace{2.5ex}
        We can visually represent the Cartesian product $A\times{B}$ by
        drawing $A$ in green and $B$ in red, as shown in
        Fig.~\ref{fig:Cartesian_Product_Example}. The Cartesian Product
        $A\times{B}$ is the set formed by connecting all of the points
        from $A$ and $B$ in the plane. This is shown in blue.
        \begin{figure}[H]
            \centering
            %--------------------------------Dependencies----------------------------------%
%   tikz                                                                       %
%       arrows.meta                                                            %
%-------------------------------Main Document----------------------------------%
\begin{tikzpicture}[%
    >=Latex,
    line width=0.2mm,
    line cap=round
]

    % Draw green to indicate the set A.
    \begin{scope}[green]

        % Draw some points.
        \draw[fill=green] (1, 0) circle (0.3mm);
        \draw[fill=green] (2, 0) circle (0.3mm);
        \draw[fill=green] (5, 0) circle (0.3mm);
        \draw[fill=green] (6, 0) circle (0.3mm);
        \draw[fill=green] (7, 0) circle (0.3mm);

        % Draw lines.
        \draw (2, 0) to (5, 0);
        \draw (6, 0) to (7, 0);
    \end{scope}

    % Draw red to denote the set B.
    \begin{scope}[red]

        % Draw in some points.
        \draw[fill=red] (0, 1) circle (0.3mm);
        \draw[fill=red] (0, 2) circle (0.3mm);
        \draw[fill=red] (0, 5) circle (0.3mm);

        % Draw a line.
        \draw (0, 2) to (0, 5);
    \end{scope}

    % Use blue to mark AxB (Cartesian product).
    \begin{scope}[blue]

        % Fill in points.
        \draw[fill=blue] (1, 1) circle (0.3mm);
        \draw[fill=blue] (1, 2) circle (0.3mm);
        \draw[fill=blue] (1, 5) circle (0.3mm);
        \draw[fill=blue] (2, 1) circle (0.3mm);
        \draw[fill=blue] (5, 1) circle (0.3mm);
        \draw[fill=blue] (6, 1) circle (0.3mm);
        \draw[fill=blue] (7, 1) circle (0.3mm);
        \draw[fill=blue] (2, 2) circle (0.3mm);
        \draw[fill=blue] (2, 5) circle (0.3mm);
        \draw[fill=blue] (5, 2) circle (0.3mm);
        \draw[fill=blue] (5, 5) circle (0.3mm);
        \draw[fill=blue] (6, 2) circle (0.3mm);
        \draw[fill=blue] (7, 2) circle (0.3mm);
        \draw[fill=blue] (6, 5) circle (0.3mm);
        \draw[fill=blue] (7, 5) circle (0.3mm);

        % Draw lines.
        \draw (1, 2) to (1, 5);
        \draw (2, 1) to (5, 1);
        \draw (6, 1) to (7, 1);

        % Fill in rectangles.
        \draw[fill=blue, opacity=0.4] (2, 2) to (5, 2) to (5, 5)
                                             to (2, 5) to cycle;
        \draw[fill=blue, opacity=0.4] (6, 2) to (7, 2) to (7, 5)
                                             to (6, 5) to cycle;
        \draw (2, 2) to (5, 2) to (5, 5) to (2, 5) to cycle;
        \draw (6, 2) to (7, 2) to (7, 5) to (6, 5) to cycle;
    \end{scope}
\end{tikzpicture}
            \caption[Cartesian Product of Two Sets]
                {The Cartesian Product of Two Sets. $A$ is
                 in \textcolor{green}{Green},
                 $B$ is in \textcolor{red}{red}, and
                 $A\times{B}$ is in \textcolor{blue}{blue}.}
            \label{fig:Cartesian_Product_Example}
        \end{figure}
        Cartesian products are not \textit{associative}%
        \index{Binary Operation!Associative}. That is, given three sets $A$,
        $B$, and $C$, there is no clear way to take the Cartesian
        product\index{Cartesian Product} of these since:
        \begin{equation}
            A\times(B\times{C})\ne(A\times{B})\times{C}
        \end{equation}
        To see this, note that the elements of $A\times(B\times{C})$ are
        ordered pairs of the form $\big(a,\,(b,\,c)\big)$, whereas elements of
        $(A\times{B})\times{C}$ are of the form $\big((a,\,b),\,c\big)$. When
        we write $A\times{B}\times{C}$ we really want ordered \textit{triples}
        of the form $(a,\,b,\,c)$.
        Much the way ordered pairs have been defined, we can modify Kuratowski's
        approach and define ordered triples\index{Ordered Triple} and ordered
        $n$ tuples\index{Ordered $n$ Tuple}. Rather than doing this we will use
        the language of functions to define higher order Cartesian products.
        \begin{fdefinition}{Functions}{Function}
            A \gls{function} from a \gls{set} $A$ to a set $B$ is a \gls{subset}
            $f\subseteq{A}\times{B}$, denoted $f:A\rightarrow{B}$, such that
            for all $x\in{A}$ there is a unique $y\in{B}$ such that
            $(x,y)\in{f}$. $A$ is called the domain of $f$
            and $B$ is called the codomain.\index{Function}
        \end{fdefinition}
        We're used to hearing that a function is a rule that assigns to an
        input value $x$ some output value $f(x)$. It may seem hard to justify,
        then, why we've defined a function as a subset of the Cartesian
        product. But note the requirement that for each $x\in{A}$ there is a
        \textit{unique} $y\in{B}$ such that $(x,y)\in{f}$. We call this unique
        element the \textit{image} of $x$ under the function $f$ and write
        $y=f(x)$. The condition that there is a unique such value $y$ to each
        $x$ is called the \textit{vertical line test}\index{Vertical Line Test}
        when graphing functions of the form $f:\mathbb{R}\rightarrow\mathbb{R}$
        (Fig.~\ref{fig:Function_R_to_R_Subset_Cart_Prod}). Simply, given such
        a function, if one draws a vertical line in the plane, then it must
        intersect the graph of $f$ once and only once. This provides a
        quick means of discerning functions from non-functions.
        \begin{lexample}{The Square Function}{Square_Function}
            If we can come up with some rule that assigns to every element
            $a\in{A}$ a unique element of $B$, then we can use this rule to
            define a function $f:A\rightarrow{B}$. Such a rule often comes
            in the form of a \textit{formula}\index{Formula}. We write the
            unique element that $a$ corresponds to as $f(b)$. For example, let
            $A=\mathbb{R}$ and let $B=\mathbb{R}$. We can define a function by
            the squaring formula:
            \begin{equation}
                f(x)=x\cdot{x}=x^{2}
            \end{equation}
            Once we know that $x^{2}$ gives a unique number
            (which will require some notion of arithmetic), we can define
            the function $f:\mathbb{R}\rightarrow\mathbb{R}$ by:
            \begin{equation}
                f=\{\,(x,\,x^{2})\in\mathbb{R}^{2}\;|\;x\in\mathbb{R}\,\}
            \end{equation}
            Usually we'll define functions by their formula's, rather than
            expressing them explicitly as subsets of the Cartesian product.
        \end{lexample}
        In the field of mathematical analysis we are often concerned with
        functions involving real numbers. For the sake of intuition, let
        us consider functions of the form $f:\mathbb{R}\rightarrow\mathbb{R}$.
        Any curve that we draw left-to-right, without picking up the pencil,
        will be a valid function.
        (See Fig.~\ref{fig:Function_R_to_R_Subset_Cart_Prod}).
        \begin{figure}[H]
            \centering
            %--------------------------------Dependencies----------------------------------%
%   xcolor                                                                     %
%   amssymb                                                                    %
%   tikz                                                                       %
%       arrows.meta                                                            %
%-------------------------------Main Document----------------------------------%
\begin{tikzpicture}[%
    >=Latex,
    line width=0.2mm,
    line cap=round,
    scale=1.2
]
    % Coorindates for the curve.
    \coordinate (P1) at (-4.00, -2.00);
    \coordinate (P2) at (-2.00, -3.00);
    \coordinate (P3) at ( 0.00,  0.00);
    \coordinate (P4) at ( 2.00,  3.00);
    \coordinate (P5) at ( 4.00,  3.90);

    % Draw a green mesh indicating the Cartesian plane.
    \foreach\x in {-40, -39, ..., 39}{
        \draw[draw=green, line width=0.1mm] (\x/10, -4) to (-4, \x/10);
        \draw[draw=green, line width=0.1mm] (4, \x/10)  to (\x/10, 4);
    }
    \draw[draw=green, line width=0.1mm] (4, 4)  to (4, 4);

    \begin{scope}[thick, font=\Large]
        \draw[<->] (-4.3,  0.0) to (4.3, 0.0) node [above] {$\mathbb{R}$};
        \draw[<->] ( 0.0, -4.3) to (0.0, 4.3) node [right] {$\mathbb{R}$};
    \end{scope}

    \draw[draw=blue] (P1) to [out=-30, in=150]  (P2)
                          to [out=-30, in=210]  (P3)
                          to [out=30,  in=180]  (P4)
                          to [out=0,   in=200]  (P5);
    \draw[fill=white, draw=white] 
        (1.3, 2.0) rectangle node {$\textcolor{blue}{f}$} (1.6, 1.4);
\end{tikzpicture}
            \caption[Example of a Function $f:\mathbb{R}\rightarrow\mathbb{R}$]
                    {Example of a function $f:\mathbb{R}\rightarrow\mathbb{R}$.
                     The Cartesian product $\mathbb{R}\times\mathbb{R}$ is
                     shown in \textcolor{green!80!black}{green}, and the
                     function $f\subseteq\mathbb{R}\times\mathbb{R}$ is shown
                     in \textcolor{blue}{blue}.}
            \label{fig:Function_R_to_R_Subset_Cart_Prod}
        \end{figure}
        Let $g\subseteq\mathbb{R}\times\mathbb{R}$ be defined as follows:
        \begin{equation}
            g=\big\{\,(x,\,y)\in\mathbb{R}^{2}\;|\;y^{2}=x\,\big\}
        \end{equation}
        It is tempting to label $g$ by writing $g(x)=\sqrt{x}$, but $g$ is
        not a function for it fails two of the requirements of a function.
        Firstly, for any $x>0$, there are two values $y_{1}$ and $y_{2}$
        whose square is equal to $x$. Indeed, if $y_{1}$ is one such value,
        then setting $y_{2}=\minus{y}_{1}$ will result in a second
        distinct value. Thus $g$ does not have the uniqueness property
        required for functions. Moreover, if $x<0$, then there is no such
        value $y\in\mathbb{R}$ such that $(x,y)\in{g}$, and thus $g$ also
        lacks the existence property. In terms of the vertical line test,
        there are points $x$ such that the vertical line through
        $(x,\,0)$ intersects $g$ twice, and there are points such that the
        vertical line does not intersect at all. The graph of $g$ is shown
        in Fig.~\ref{fig:SQRT_Not_a_Function}.
        \begin{figure}[H]
            \centering
            %--------------------------------Dependencies----------------------------------%
%   xcolor                                                                     %
%   amssymb                                                                    %
%   tikz                                                                       %
%       arrows.meta                                                            %
%       patterns                                                               %
%-------------------------------Main Document----------------------------------%
\begin{tikzpicture}[%
    >=Latex,
    line width=0.2mm,
    line cap=round,
    scale=1.2
]
    % Coorindates for the curve.
    \coordinate (P1) at (-3.85, -2.00);
    \coordinate (P2) at (-2.00, -3.00);
    \coordinate (P3) at ( 0.00,  0.00);
    \coordinate (P4) at ( 2.00,  3.00);
    \coordinate (P5) at ( 3.85,  3.80);

    \draw[%
        pattern=north west lines,
        pattern color=Green!80!Black,
        opacity=0.5,
        draw=white
    ]   (-3.9, -3.9) rectangle (3.9, 3.9);

    \begin{scope}[thick, font=\Large]
        \draw[<->] (-4.2, 0) to (4.2, 0) node [above] {$\mathbb{R}$};
        \draw[<->] (0, -4.2) to (0, 4.2) node [right] {$\mathbb{R}$};
    \end{scope}

    \draw[draw=red] (3.880000, -1.969772) to (3.840000, -1.959592)
                                          to (3.800000, -1.949359)
                                          to (3.760000, -1.939072)
                                          to (3.720000, -1.928730)
                                          to (3.680000, -1.918333)
                                          to (3.640000, -1.907878)
                                          to (3.600000, -1.897367)
                                          to (3.560000, -1.886796)
                                          to (3.520000, -1.876166)
                                          to (3.480000, -1.865476)
                                          to (3.440000, -1.854724)
                                          to (3.400000, -1.843909)
                                          to (3.360000, -1.833030)
                                          to (3.320000, -1.822087)
                                          to (3.280000, -1.811077)
                                          to (3.240000, -1.800000)
                                          to (3.200000, -1.788854)
                                          to (3.160000, -1.777639)
                                          to (3.120000, -1.766352)
                                          to (3.080000, -1.754993)
                                          to (3.040000, -1.743560)
                                          to (3.000000, -1.732051)
                                          to (2.960000, -1.720465)
                                          to (2.920000, -1.708801)
                                          to (2.880000, -1.697056)
                                          to (2.840000, -1.685230)
                                          to (2.800000, -1.673320)
                                          to (2.760000, -1.661325)
                                          to (2.720000, -1.649242)
                                          to (2.680000, -1.637071)
                                          to (2.640000, -1.624808)
                                          to (2.600000, -1.612452)
                                          to (2.560000, -1.600000)
                                          to (2.520000, -1.587451)
                                          to (2.480000, -1.574802)
                                          to (2.440000, -1.562050)
                                          to (2.400000, -1.549193)
                                          to (2.360000, -1.536229)
                                          to (2.320000, -1.523155)
                                          to (2.280000, -1.509967)
                                          to (2.240000, -1.496663)
                                          to (2.200000, -1.483240)
                                          to (2.160000, -1.469694)
                                          to (2.120000, -1.456022)
                                          to (2.080000, -1.442221)
                                          to (2.040000, -1.428286)
                                          to (2.000000, -1.414214)
                                          to (1.960000, -1.400000)
                                          to (1.920000, -1.385641)
                                          to (1.880000, -1.371131)
                                          to (1.840000, -1.356466)
                                          to (1.800000, -1.341641)
                                          to (1.760000, -1.326650)
                                          to (1.720000, -1.311488)
                                          to (1.680000, -1.296148)
                                          to (1.640000, -1.280625)
                                          to (1.600000, -1.264911)
                                          to (1.560000, -1.249000)
                                          to (1.520000, -1.232883)
                                          to (1.480000, -1.216553)
                                          to (1.440000, -1.200000)
                                          to (1.400000, -1.183216)
                                          to (1.360000, -1.166190)
                                          to (1.320000, -1.148913)
                                          to (1.280000, -1.131371)
                                          to (1.240000, -1.113553)
                                          to (1.200000, -1.095445)
                                          to (1.160000, -1.077033)
                                          to (1.120000, -1.058301)
                                          to (1.080000, -1.039230)
                                          to (1.040000, -1.019804)
                                          to (1.000000, -1.000000)
                                          to (0.960000, -0.979796)
                                          to (0.920000, -0.959166)
                                          to (0.880000, -0.938083)
                                          to (0.840000, -0.916515)
                                          to (0.800000, -0.894427)
                                          to (0.760000, -0.871780)
                                          to (0.720000, -0.848528)
                                          to (0.680000, -0.824621)
                                          to (0.640000, -0.800000)
                                          to (0.600000, -0.774597)
                                          to (0.560000, -0.748331)
                                          to (0.520000, -0.721110)
                                          to (0.480000, -0.692820)
                                          to (0.440000, -0.663325)
                                          to (0.400000, -0.632456)
                                          to (0.360000, -0.600000)
                                          to (0.320000, -0.565685)
                                          to (0.280000, -0.529150)
                                          to (0.240000, -0.489898)
                                          to (0.200000, -0.447214)
                                          to (0.160000, -0.400000)
                                          to (0.120000, -0.346410)
                                          to (0.080000, -0.282843)
                                          to (0.040000, -0.200000)
                                          to (0.000000, 0.000000) 
                                          to (0.040000, 0.200000)
                                          to (0.080000, 0.282843)
                                          to (0.120000, 0.346410)
                                          to (0.160000, 0.400000)
                                          to (0.200000, 0.447214)
                                          to (0.240000, 0.489898)
                                          to (0.280000, 0.529150)
                                          to (0.320000, 0.565685)
                                          to (0.360000, 0.600000)
                                          to (0.400000, 0.632456)
                                          to (0.440000, 0.663325)
                                          to (0.480000, 0.692820)
                                          to (0.520000, 0.721110)
                                          to (0.560000, 0.748331)
                                          to (0.600000, 0.774597)
                                          to (0.640000, 0.800000)
                                          to (0.680000, 0.824621)
                                          to (0.720000, 0.848528)
                                          to (0.760000, 0.871780)
                                          to (0.800000, 0.894427)
                                          to (0.840000, 0.916515)
                                          to (0.880000, 0.938083)
                                          to (0.920000, 0.959166)
                                          to (0.960000, 0.979796)
                                          to (1.000000, 1.000000)
                                          to (1.040000, 1.019804)
                                          to (1.080000, 1.039230)
                                          to (1.120000, 1.058301)
                                          to (1.160000, 1.077033)
                                          to (1.200000, 1.095445)
                                          to (1.240000, 1.113553)
                                          to (1.280000, 1.131371)
                                          to (1.320000, 1.148913)
                                          to (1.360000, 1.166190)
                                          to (1.400000, 1.183216)
                                          to (1.440000, 1.200000)
                                          to (1.480000, 1.216553)
                                          to (1.520000, 1.232883)
                                          to (1.560000, 1.249000)
                                          to (1.600000, 1.264911)
                                          to (1.640000, 1.280625)
                                          to (1.680000, 1.296148)
                                          to (1.720000, 1.311488)
                                          to (1.760000, 1.326650)
                                          to (1.800000, 1.341641)
                                          to (1.840000, 1.356466)
                                          to (1.880000, 1.371131)
                                          to (1.920000, 1.385641)
                                          to (1.960000, 1.400000)
                                          to (2.000000, 1.414214)
                                          to (2.040000, 1.428286)
                                          to (2.080000, 1.442221)
                                          to (2.120000, 1.456022)
                                          to (2.160000, 1.469694)
                                          to (2.200000, 1.483240)
                                          to (2.240000, 1.496663)
                                          to (2.280000, 1.509967)
                                          to (2.320000, 1.523155)
                                          to (2.360000, 1.536229)
                                          to (2.400000, 1.549193)
                                          to (2.440000, 1.562050)
                                          to (2.480000, 1.574802)
                                          to (2.520000, 1.587451)
                                          to (2.560000, 1.600000)
                                          to (2.600000, 1.612452)
                                          to (2.640000, 1.624808)
                                          to (2.680000, 1.637071)
                                          to (2.720000, 1.649242)
                                          to (2.760000, 1.661325)
                                          to (2.800000, 1.673320)
                                          to (2.840000, 1.685230)
                                          to (2.880000, 1.697056)
                                          to (2.920000, 1.708801)
                                          to (2.960000, 1.720465)
                                          to (3.000000, 1.732051)
                                          to (3.040000, 1.743560)
                                          to (3.080000, 1.754993)
                                          to (3.120000, 1.766352)
                                          to (3.160000, 1.777639)
                                          to (3.200000, 1.788854)
                                          to (3.240000, 1.800000)
                                          to (3.280000, 1.811077)
                                          to (3.320000, 1.822087)
                                          to (3.360000, 1.833030)
                                          to (3.400000, 1.843909)
                                          to (3.440000, 1.854724)
                                          to (3.480000, 1.865476)
                                          to (3.520000, 1.876166)
                                          to (3.560000, 1.886796)
                                          to (3.600000, 1.897367)
                                          to (3.640000, 1.907878)
                                          to (3.680000, 1.918333)
                                          to (3.720000, 1.928730)
                                          to (3.760000, 1.939072)
                                          to (3.800000, 1.949359)
                                          to (3.840000, 1.959592)
                                          to (3.880000, 1.969772);
    \draw[fill=white, draw=white] 
        (1.3, 2.0) rectangle node {$\textcolor{red}{g}$} (1.6, 1.5);
\end{tikzpicture}
            \caption[Example of a Non-Function]
                {$g\subseteq\mathbb{R}\times\mathbb{R}$ is not a function
                 since it fails the vertical line test.}
            \label{fig:SQRT_Not_a_Function}
        \end{figure}
        We need not only consider functions of the form
        $f:\mathbb{R}\rightarrow\mathbb{R}$, nor are we restricted to function
        liked $f:\mathcal{U}\rightarrow\mathcal{V}$ where $\mathcal{U}$ and
        $\mathcal{V}$ are subsets of $\mathbb{R}$, and we can allow for
        arbitrary abstract functions. Let $A$ and $B$ be defined as follows:
        \index{Function}
        \par
        \begin{subequations}
            \begin{minipage}[b]{0.49\textwidth}
                \centering
                \begin{equation}
                    A=\{\,1,\,2,\,3,\,4\,\}
                \end{equation}
            \end{minipage}
            \hfill
            \begin{minipage}[b]{0.49\textwidth}
                \centering
                \begin{equation}
                    B=\{\,a,\,b,\,c\,\}
                \end{equation}
            \end{minipage}
        \end{subequations}
        \par\vspace{2.5ex}
        Similar to the vertical line test\index{Vertical Line Test}, we can
        devise a visual to discerning functions from non-functions for abstract
        sets. We represent the elements of $A$ and $B$ as points in some blob
        in the plane, and then draw arrows between the points
        $x\in{A}$ and $y\in{b}$ indicating that $(x,y)\in{f}$. This allows us to
        determine if a given $f\subseteq{A}\times{B}$ is a functions ore not.
        Every point in $A$ must be mapped to a unique point in $B$. That is,
        every point in $A$ must have one and only one arrow connecting it to
        some point in $B$. Examples of valid functions are shown in
        Fig.~\ref{fig:Abstract_Functions}, and non-functions are shown in
        Fig.~\ref{fig:Abstract_Non_Functions}.
        \begin{figure}[H]
            \centering
            \begin{subfigure}[b]{0.49\textwidth}
                \centering
                \resizebox{\textwidth}{!}{%
                    %--------------------------------Dependencies----------------------------------%
%   tikz                                                                       %
%       arrows.meta                                                            %
%-------------------------------Main Document----------------------------------%
\begin{tikzpicture}[%
    >=latex,
    line width=0.2mm,
    line cap=round,
    scale=1.2
]
    % Coorindates.
    \coordinate (a) at ( 1.5,  0.75);
    \coordinate (b) at ( 1.5, -0.00);
    \coordinate (c) at ( 1.5, -0.75);
    \coordinate (1) at (-1.5,  1.20);
    \coordinate (2) at (-1.5,  0.40);
    \coordinate (3) at (-1.5, -0.40);
    \coordinate (4) at (-1.5, -1.20);
    \coordinate (A) at (-1.5,  2.50);
    \coordinate (B) at ( 1.5,  2.50);

    % Ellipses representing the sets A and B.
    \draw[thick] (-1.5, 0.0) ellipse (1 and 2);
    \draw[thick] ( 1.5, 0.0) ellipse (1 and 2);

    % Draw circles for the various points.
    \draw[fill=black] (a) circle (0.4mm);
    \draw[fill=black] (b) circle (0.4mm);
    \draw[fill=black] (c) circle (0.4mm);
    \draw[fill=black] (1) circle (0.4mm);
    \draw[fill=black] (2) circle (0.4mm);
    \draw[fill=black] (3) circle (0.4mm);
    \draw[fill=black] (4) circle (0.4mm);

    % Draw paths indicating mappings.
    \begin{scope}[->]
        \draw[shorten >=0.8mm] (1) to (c);
        \draw[shorten >=0.8mm] (2) to (a);
        \draw[shorten >=0.8mm] (3) to (b);
        \draw[shorten >=0.8mm] (4) to (a);
    \end{scope}

    % Labels.
    \node at (A)         {$A$};
    \node at (B)         {$B$};
    \node at (a) [right] {$a$};
    \node at (b) [right] {$b$};
    \node at (c) [right] {$c$};
    \node at (1) [left]  {$1$};
    \node at (2) [left]  {$2$};
    \node at (3) [left]  {$3$};
    \node at (4) [left]  {$4$};
\end{tikzpicture}
                }
                \subcaption{A Valid Function.}
            \end{subfigure}
            \begin{subfigure}[b]{0.49\textwidth}
                \centering
                \resizebox{\textwidth}{!}{%
                    %--------------------------------Dependencies----------------------------------%
%   tikz                                                                       %
%       arrows.meta                                                            %
%-------------------------------Main Document----------------------------------%
\begin{tikzpicture}[%
    >=latex,
    line width=0.2mm,
    line cap=round,
    scale=1.2
]
    % Coorindates.
    \coordinate (a) at ( 1.5,  0.75);
    \coordinate (b) at ( 1.5, -0.00);
    \coordinate (c) at ( 1.5, -0.75);
    \coordinate (1) at (-1.5,  1.20);
    \coordinate (2) at (-1.5,  0.40);
    \coordinate (3) at (-1.5, -0.40);
    \coordinate (4) at (-1.5, -1.20);
    \coordinate (A) at (-1.5,  2.50);
    \coordinate (B) at ( 1.5,  2.50);

    % Ellipses representing the sets A and B.
    \draw[thick] (-1.5, 0.0) ellipse (1 and 2);
    \draw[thick] ( 1.5, 0.0) ellipse (1 and 2);

    % Draw circles for the various points.
    \draw[fill=black] (a) circle (0.4mm);
    \draw[fill=black] (b) circle (0.4mm);
    \draw[fill=black] (c) circle (0.4mm);
    \draw[fill=black] (1) circle (0.4mm);
    \draw[fill=black] (2) circle (0.4mm);
    \draw[fill=black] (3) circle (0.4mm);
    \draw[fill=black] (4) circle (0.4mm);

    % Draw paths indicating mappings.
    \begin{scope}[->]
        \draw[shorten >=0.8mm] (1) to (a);
        \draw[shorten >=0.8mm] (2) to (a);
        \draw[shorten >=0.8mm] (3) to (a);
        \draw[shorten >=0.8mm] (4) to (a);
    \end{scope}

    % Labels.
    \node at (A)         {$A$};
    \node at (B)         {$B$};
    \node at (a) [right] {$a$};
    \node at (b) [right] {$b$};
    \node at (c) [right] {$c$};
    \node at (1) [left]  {$1$};
    \node at (2) [left]  {$2$};
    \node at (3) [left]  {$3$};
    \node at (4) [left]  {$4$};
\end{tikzpicture}
                }
                \subcaption{Another Valid Function.}
            \end{subfigure}
            \caption{Visual for Abstract Functions}
            \label{fig:Abstract_Functions}
        \end{figure}
        \begin{figure}[H]
            \centering
            \begin{subfigure}[b]{0.49\textwidth}
                \centering
                \resizebox{\textwidth}{!}{%
                    %--------------------------------Dependencies----------------------------------%
%   tikz                                                                       %
%       arrows.meta                                                            %
%-------------------------------Main Document----------------------------------%
\begin{tikzpicture}[%
    >=latex,
    line width=0.2mm,
    line cap=round,
    scale=1.2
]
    % Coorindates.
    \coordinate (a) at ( 1.5,  0.75);
    \coordinate (b) at ( 1.5, -0.00);
    \coordinate (c) at ( 1.5, -0.75);
    \coordinate (1) at (-1.5,  1.20);
    \coordinate (2) at (-1.5,  0.40);
    \coordinate (3) at (-1.5, -0.40);
    \coordinate (4) at (-1.5, -1.20);
    \coordinate (A) at (-1.5,  2.50);
    \coordinate (B) at ( 1.5,  2.50);

    % Ellipses representing the sets A and B.
    \draw[thick] (-1.5, 0.0) ellipse (1 and 2);
    \draw[thick] ( 1.5, 0.0) ellipse (1 and 2);

    % Draw circles for the various points.
    \draw[fill=black] (a) circle (0.4mm);
    \draw[fill=black] (b) circle (0.4mm);
    \draw[fill=black] (c) circle (0.4mm);
    \draw[fill=black] (1) circle (0.4mm);
    \draw[fill=black] (2) circle (0.4mm);
    \draw[fill=black] (3) circle (0.4mm);
    \draw[fill=black] (4) circle (0.4mm);

    % Draw paths indicating mappings.
    \begin{scope}[->]
        \draw[shorten >=0.8mm] (1) to (a);
        \draw[shorten >=0.8mm] (2) to (b);
        \draw[shorten >=0.8mm] (3) to (c);
    \end{scope}

    % Labels.
    \node at (A)         {$A$};
    \node at (B)         {$B$};
    \node at (a) [right] {$a$};
    \node at (b) [right] {$b$};
    \node at (c) [right] {$c$};
    \node at (1) [left]  {$1$};
    \node at (2) [left]  {$2$};
    \node at (3) [left]  {$3$};
    \node at (4) [left]  {$4$};
\end{tikzpicture}
                }
                \subcaption{Fails Existence.}
            \end{subfigure}
            \begin{subfigure}[b]{0.49\textwidth}
                \centering
                \resizebox{\textwidth}{!}{%
                    %--------------------------------Dependencies----------------------------------%
%   tikz                                                                       %
%       arrows.meta                                                            %
%-------------------------------Main Document----------------------------------%
\begin{tikzpicture}[%
    >=latex,
    line width=0.2mm,
    line cap=round,
    scale=1.2
]
    % Coorindates.
    \coordinate (a) at ( 1.5,  0.75);
    \coordinate (b) at ( 1.5, -0.00);
    \coordinate (c) at ( 1.5, -0.75);
    \coordinate (1) at (-1.5,  1.20);
    \coordinate (2) at (-1.5,  0.40);
    \coordinate (3) at (-1.5, -0.40);
    \coordinate (4) at (-1.5, -1.20);
    \coordinate (A) at (-1.5,  2.50);
    \coordinate (B) at ( 1.5,  2.50);

    % Ellipses representing the sets A and B.
    \draw[thick] (-1.5, 0.0) ellipse (1 and 2);
    \draw[thick] ( 1.5, 0.0) ellipse (1 and 2);

    % Draw circles for the various points.
    \draw[fill=black] (a) circle (0.4mm);
    \draw[fill=black] (b) circle (0.4mm);
    \draw[fill=black] (c) circle (0.4mm);
    \draw[fill=black] (1) circle (0.4mm);
    \draw[fill=black] (2) circle (0.4mm);
    \draw[fill=black] (3) circle (0.4mm);
    \draw[fill=black] (4) circle (0.4mm);

    % Draw paths indicating mappings.
    \begin{scope}[->]
        \draw[shorten >=0.8mm] (1) to (a);
        \draw[shorten >=0.8mm] (2) to (a);
        \draw[shorten >=0.8mm] (2) to (b);
        \draw[shorten >=0.8mm] (3) to (c);
        \draw[shorten >=0.8mm] (4) to (c);
    \end{scope}

    % Labels.
    \node at (A)         {$A$};
    \node at (B)         {$B$};
    \node at (a) [right] {$a$};
    \node at (b) [right] {$b$};
    \node at (c) [right] {$c$};
    \node at (1) [left]  {$1$};
    \node at (2) [left]  {$2$};
    \node at (3) [left]  {$3$};
    \node at (4) [left]  {$4$};
\end{tikzpicture}
                }
                \subcaption{Fails Uniqueness.}
            \end{subfigure}
            \caption{Non-Functions}
            \label{fig:Abstract_Non_Functions}
        \end{figure}
        It is possible to count the total number of functions from $A$ to $B$.
        Since every element of $A$ needs to be mapped to some element of $B$,
        and since there are 4 elements in $A$ and 3 elements in $B$, the total
        number of functions $f:A\rightarrow{B}$ is $4^{3}=64$. On the other
        hand, the total number of subsets of $A\times{B}$ is $2^{12}=4096$
        (we will justify this when we discuss the \textit{cardinality} of
        sets). Thus, if we were to randomly pick a subset of $A\times{B}$, the
        odds are that it is almost certainly \textit{not} a function
        (1.5625\%). Thus, functions are very special subsets.
        There is a frequent need to discuss the \textit{set of all functions}
        from a given set $A$ into another set $B$. To ensure we don't create
        a function version of Russell's paradox, we prove such a set exists.
        \begin{theorem}
            If $A$ and $B$ are sets, then there exists a set $\mathcal{F}$ such
            that, for all $f$ it is true that $f\in\mathcal{F}$ if and only if
            $f$ is a function from $A$ to $B$, $f:A\rightarrow{B}$.
            \index{Function!Set of All}
        \end{theorem}
        \begin{proof}
            For if $A$ and $B$ are sets, then by
            Thm.~\ref{thm:Existence_of_the_Cartesian_Product} the set
            $A\times{B}$ exists. But by
            Thm.~\ref{thm:Existence_of_the_Power_Set}, the power set of
            $A\times{B}$, $\mathcal{P}(A\times{B})$, exists. Let $P$ be the
            proposition \textit{True if} $f$ \textit{is a function from} $A$
            \textit{to} $B$, \textit{false otherwise}. Then by axiom schema of
            specification (Ax.~\ref{ax:Axiom_Schema_of_Specification}), there is
            a set $\mathcal{F}$ such that:
            \begin{equation}
                \mathcal{F}=\big\{\,f\in\mathcal{P}(A\times{B})\;|
                    \;P(f)\,\big\}
            \end{equation}
            But then for all $f$, $f\in\mathcal{F}$ if and only if
            $f\in\mathcal{F}$ and $P(f)$ is true. But if $P(f)$ is true then
            $f$ is a function from $A$ to $B$, and thus by the definition of a
            function (Def.~\ref{def:Function}) $f\subseteq{A}\times{B}$. But
            then by the definition of the power set (Def.~\ref{def:Power_Set})
            we have that $f\in\mathcal{P}(A\times{B})$. Thus $P(f)$ implies
            $f\in\mathcal{F}$. Therefore $f\in\mathcal{F}$ if and only if
            $P(f)$. That is, $f\in\mathcal{F}$ if and only if $f$ is a function
            from $A$ to $B$.
        \end{proof}
        There is non-standard notation when discussing the set of all functions
        from a given set $A$ to a set $B$:
        \begin{fnotation}{Set of All Functions}{Set_of_All_Functions}
            If $A$ and $B$ are sets, the set of all functions from $A$ to $B$,
            $f:A\rightarrow{B}$, is denoted as either $\mathcal{F}(A,B)$ or
            $B^{A}$.
        \end{fnotation}
        The notation $B^{A}$ is common in many areas such as topology and
        algebra, especially when $A=B$. The \textit{topological space} $I^{I}$,
        which is the set of all functions from the \textit{closed unit inverval}
        \index{Interval!Closed} to itself, is often used to construct examples
        and counterexamples. In analysis the notation $\mathcal{F}(A,B)$ seems
        to be more common, in particular $\mathcal{C}(A,B)$ is often used to
        denote the set of all \textit{continuous} functions from $A$ to $B$,
        provided the word continuous has meaning. Since the notation is not
        universal nor standard across the various disciplines, an attempt will
        be made to specify what $B^{A}$ or $\mathcal{F}(A,B)$ means before using
        it in a theorem or counterexample.
        \begin{fdefinition}{Image of a Point}{Image_of_Point}
            The \gls{image of a point} of an element $x$ in a set $A$ under a
            \gls{function} $f:A\rightarrow{B}$ is the unique value $y\in{B}$
            such that $(x,y)\in{f}$. We write $y=f(x)$.
            \index{Image!of a Point}
        \end{fdefinition}
        This allows us to define functions by simply specifying what the
        image of each $x\in{A}$ is. Restating our previous claim, if we can
        define some formula such that for each $x\in{A}$ there is a unique
        $f(x)\in{B}$ such that the formula takes $x$ to $f(x)$, then we can
        define $f$ as the set of all such ordered pairs $(x,f(x))$, and this
        will be a function.
        \begin{fnotation}{Image Notation}{Image_Notation}
            If $A$ and $B$ are sets, if $f:A\rightarrow{B}$ is a function,
            if $x\in{A}$ and if $y=f(x)\in{B}$, then we denote this by
            writing $x\overset{f}{\longmapsto}{y}$ or just $x\mapsto{y}$.
        \end{fnotation}
        Throughout we will almost exclusively use the notation $y=f(x)$ rather
        than $x\mapsto{y}$. The reasons are purely aesthetic and both notations
        are common in mathematics. In a similar manner, we can define the image
        of an entire subset.
        \begin{theorem}
            If $A$ and $B$ are sets, if $f:A\rightarrow{B}$ is a function,
            and if $\mathcal{U}\subseteq{A}$, then there is a set
            $\mathcal{V}\subseteq{B}$ such that for all $y$ it is true that
            $y\in\mathcal{V}$ if and only if $y\in{B}$ and such that there is
            an $x\in\mathcal{U}$ such that $y=f(x)$.
        \end{theorem}
        \begin{proof}
            For let $P$ be the proposition \textit{True if there exists}
            $x\in\mathcal{U}$ \textit{such that} $y=f(x)$,
            \textit{false otherwise}. By the axiom schema of specification
            (Ax.~\ref{ax:Axiom_Schema_of_Specification}) there is a set
            $\mathcal{V}$ such that, for all $y$ it is true that
            $y\in\mathcal{V}$ if and only if $y\in{B}$ and $P(y)$ is true. That
            is, $y\in\mathcal{V}$ if and only if $y\in{B}$ and if there is an
            $x\in\mathcal{U}$ such that $y=f(x)$.
        \end{proof}
        \begin{fdefinition}{Image of a Subset}{Image_of_Subset}
            The \gls{set image} of a \gls{subset} $\mathcal{U}$ of a \gls{set}
            $A$ under a \gls{function} $f:A\rightarrow{B}$ is the
            set:\index{Image!of a Set}
            \begin{equation*}
                f\big(\mathcal{U}\big)=
                    \{\,y\in{B}\;|\;\textrm{There exists }x\in\mathcal{U}
                                    \textrm{ such that }y=f(x)\,\}
            \end{equation*}
            That is, the set of all points in $B$ that are the image of points
            in $\mathcal{U}$. Formally:
            \begin{equation*}
                \forall_{A}\forall_{B}\forall_{f:A\rightarrow{B}}
                \forall_{\mathcal{U}\subseteq{A}}\forall_{y}\Big(
                    \big(y\in{f}(\mathcal{U})\big)\Longleftrightarrow
                    \big(\exists_{x\in\mathcal{U}}:y=f(x)\big)\Big)
            \end{equation*}
        \end{fdefinition}
        This definition of the image of a subset was given in such a manner
        so that it only relies on the axiom schema of specification to
        justify it's existence. We could also use the notation:
        \begin{equation}
            f\big(\mathcal{U})=\{\,f(x)\in{B}\;|\;x\in\mathcal{U}\,\}
        \end{equation}
        Writing the definition of the image of a subset in such a way is
        justified by the \textit{axiom schema of replacement}%
        \index{Axiom!Schema of Replacement}, but we've not yet included this
        axiom in our system.
        \begin{example}
            If $f:\mathbb{R}\rightarrow\mathbb{R}$ is the function $f(x)=x^{2}$,
            then:
            \begin{equation}
                f(\mathbb{R})=[0,\infty)\equiv\{\,x\in\mathbb{R}\;|\;x\geq{0}\}
            \end{equation}
            This is because every non-negative real number $y$ gets mapped to
            by at least one real number (the positive square root $\sqrt{y}$).
            None of the negative numbers are the image of any element of
            $\mathbb{R}$ since the square of a real number is always
            non-negative.
        \end{example}
        We can visualize functions and images by using blobs in the plane. Given
        some sub-blob of a set $A$, the image of this will be another sub-blob
        of $B$. Note that if $f:A\rightarrow{B}$ is a function, it does
        \textbf{not} need to be true that $f(A)=B$. These are special functions
        that are called \textit{surjective}\index{Function!Surjective} and are
        discussed in Chapt.~\ref{chapt:Function_Theory}. Such a drawing of the
        general case is shown in Fig.~\ref{fig:Image_of_Point_and_Subset}.
        \begin{figure}[H]
            \centering
            \captionsetup{type=figure}
            \begin{tikzpicture}[>=Latex]
    \coordinate (U1) at (-5.0, -2.0);
    \coordinate (U2) at (-3.5, -2.0);
    \coordinate (U3) at (-0.5, -0.5);
    \coordinate (U4) at (-2.0,  2.0);
    \coordinate (U5) at (-3.3,  1.6);
    \coordinate (U6) at (-4.0,  2.0);
    \coordinate (U7) at (-5.0,  0.0);

    \coordinate (V1) at (5.0,  2.0);
    \coordinate (V2) at (4.0,  2.0);
    \coordinate (V3) at (2.0,  0.0);
    \coordinate (V4) at (2.0, -2.0);
    \coordinate (V5) at (4.0, -1.0);
    \coordinate (V6) at (5.0, -1.0);

    \coordinate (S1) at (-4.0,  0.0);
    \coordinate (S2) at (-3.0, -1.0);
    \coordinate (S3) at (-2.5,  0.0);
    \coordinate (S4) at (-3.0,  0.5);

    \coordinate (T1) at (3.0,  0.0);
    \coordinate (T2) at (3.5, -0.8);
    \coordinate (T3) at (4.5,  0.0);
    \coordinate (T4) at (3.5,  0.8);

    \coordinate (x)  at (-2.8, -0.3);
    \coordinate (fx) at (3.4,  -0.4);

    \draw[fill=blue,opacity=0.5,draw=black,thick]
        (U1)    to[out=0,  in=-150] (U2)
                to[out=30, in=-90]  (U3)
                to[out=90, in=-60]  (U4)
                to[out=120, in=-30] (U5)
                to[out=150,in=10]   (U6)
                to[out=-170,in=90]  (U7)
                to[out=-90,in=180]  cycle;

    \draw[fill=red!80!white,opacity=0.5,draw=black,thick]
        (V1)    to[out=180, in=0]       (V2)
                to[out=180, in=60]      (V3)
                to[out=-120,in=120]     (V4)
                to[out=-60, in=180]     (V5)
                to[out=0,   in=-120]    (V6)
                to[out=60,  in=0]       cycle;

    \draw[fill=blue!80!white] (S1)  to[out=-150,in=180] (S2)
                                    to[out=0,   in=-90] (S3)
                                    to[out=90,  in=-60] (S4)
                                    to[out=120, in=30]  cycle;

    \draw[fill=red!80!white] (T1)   to[out=-150,in=180] (T2)
                                    to[out=0,   in=-90] (T3)
                                    to[out=90,  in=0]   (T4)
                                    to[out=180, in=30]  cycle;

    \draw[fill=black] (x)  circle (0.3mm);
    \draw[fill=black] (fx) circle (0.3mm);
    \node at (-2.5, 1.0) {\Large{$A$}};
    \node at ( 4.6, 1.3) {\Large{$B$}};
    \node at (-3.3, 0.0) {\large{$\mathcal{U}$}};
    \node at ( 3.7, 0.2) {\large{$f(\mathcal{U})$}};
    \node at (x)  [below] {$x$};
    \node at (fx) [right] {$f(x)$};
    \draw[->,shorten >= 1.5mm,shorten <= 1.5mm]
        (x) to[out=-30,in=-150] node[below]{$f$} (fx);
\end{tikzpicture}
            \caption{Image of a Subset and of a Point under a Function}
            \label{fig:Image_of_Point_and_Subset}
        \end{figure}
        If we consider a function $f:A\rightarrow{B}$ and the image of the
        entire set $A$ we obtain the \textit{range} of $f$. That is, the range
        is the set $f(A)\subseteq{B}$. In a similar manner to the forward image
        of a function, we can define the pre-image.
        \begin{theorem}
            \label{thm:Existence_of_Pre_Image}%
            If $A$ and $B$ are sets, if $f:A\rightarrow{B}$ is a function from
            $A$ to $B$, and if $\mathcal{V}\subseteq{B}$, then there is a set
            $\mathcal{U}\subseteq{A}$ such that for all $x$ it is true that
            $x\in\mathcal{U}$ if and only if $x\in{A}$ and $f(x)\in\mathcal{V}$.
        \end{theorem}
        \begin{proof}
            For let $P$ be the proposition \textit{True if} $x\in{A}$
            \textit{and} $f(x)\in\mathcal{V}$, \textit{false otherwise}. Then
            by the axiom schema of specification
            (Ax.~\ref{ax:Axiom_Schema_of_Specification}) there is a set
            $\mathcal{U}$ such that:
            \begin{equation*}
                \mathcal{U}=\big\{\,x\in{A}\;|\;P(x)\,\big\}
            \end{equation*}
            But $P(x)$ implies $x\in{A}$ and thus $x\in\mathcal{U}$ if and only
            if $x\in{A}$ and $f(x)\in\mathcal{V}$.
        \end{proof}
        \begin{fdefinition}{Pre-Image of a Subset}{Pre_Image_of_Subset}
            The \gls{pre-image} of a \gls{subset} $\mathcal{V}\subseteq{B}$
            under a \gls{function} $f:A\rightarrow{B}$ is the set:
            \index{Pre-Image}
            \begin{equation}
                f^{\minus{1}}(\mathcal{V})
                =\big\{\,x\in{A}\;|\;f(x)\in\mathcal{V}\,\big\}
            \end{equation}
            Using our formal language:
            \begin{equation*}
                \forall_{A}\forall_{B}\forall_{f:A\rightarrow{B}}
                \forall_{\mathcal{V}\subseteq{B}}\forall_{x}\Big(
                    \big(x\in{f}^{\minus{1}}(\mathcal{V})\big)
                    \Longleftrightarrow
                    \big((x\in{A})\land(f(x)\in\mathcal{V}))\big)\Big)
            \end{equation*}
        \end{fdefinition}
        The pre-image of a set behaves a lot differently than the image, and
        this will be explored in detail when functions are discussed. The cause
        of the discrepancy is the requirement that elements of $A$ map uniquely
        to elements of $B$, but a single element in $B$ can be the image of
        many different points in $A$. This gives rise to the notion of the
        \textit{fiber} of a point in $B$.
        \begin{theorem}
            If $A$ and $B$ are sets, if $f:A\rightarrow{B}$ is a function, and
            if $b\in{A}$, then there is a set $\mathcal{U}\subseteq{A}$ such
            for all $x\in{A}$ it is true that $x\in\mathcal{U}$ if and only if
            $f(x)=b$.
        \end{theorem}
        \begin{proof}
            For by Thm.~\ref{thm:Existence_of_Set_Containing_Set}, $\{b\}$ is
            a set and $\{b\}\subseteq{B}$ (Def.~\ref{def:Subsets}). But if
            $\{b\}$ is a subset of $B$, then there is a set
            $\mathcal{U}\subseteq{A}$ such that for all $x\in{A}$ it is true
            that $x\in\mathcal{U}$ if and only if $f(x)\in\{b\}$
            (Thm.~\ref{thm:Existence_of_Pre_Image}). But $f(x)\in\{b\}$ if and
            only if $f(x)=b$ (Thm.~\ref{thm:Existence_of_Set_Containing_Set}).
            Thus, for all $x\in{A}$, $x\in\mathcal{U}$ if and only if $f(x)=b$.
        \end{proof}
        \begin{fdefinition}{Fiber of an Element}{Fiber_of_Element}
            The \gls{fiber} of an element $b$ in a \gls{set} $B$ under a
            \gls{function} $f:A\rightarrow{B}$ from a set $A$ to a set $B$ is
            the pre-image of the set $\{b\}$. That is:\index{Fiber}
            \begin{equation*}
                f^{\minus{1}}(\{b\})
                =\big\{\,a\in{A}\:|\;f(a)=b\,\big\}
            \end{equation*}
        \end{fdefinition}
        Before we end our introduction to the concept of a function, we should
        first note the other synonomous terminology that is used and give some
        historical background. In many texts, including this one, authors use
        the word \textit{map} or \textit{mapping} to denote a function, rather
        than simply use the word function. This language has its roots in
        geometry, and particularly in the study of projective geometry. One can
        consider \textit{perspectives}, which are functions that map one line in
        the Euclidean plane to another. Consider two non-parallel lines
        $\overline{AB}$ and $\overline{CD}$ and a point $P$ in the plane that
        lies on neither of these two. Given a point $x$ which lies on
        $\overline{AB}$, we map this to the line $\overline{CD}$ by drawing a
        straight line from $P$ to $x$ and marking where this intersects
        $\overline{CD}$
        (see Fig.~\ref{fig:Example_of_Map_from_Projective_Geometry}).
        \begin{figure}[H]
            \centering
            \captionsetup{type=figure}
            \begin{tikzpicture}[>=Latex]
    \coordinate (A) at (-4.0000, -1.0000);
    \coordinate (B) at ( 4.0000,  1.0000);
    \coordinate (C) at (-4.0000,  0.0000);
    \coordinate (D) at ( 4.0000, -1.0000);
    \coordinate (P) at ( 0.0000,  1.5000);
    \coordinate (x) at ( 1.0000,  0.2500);
    \coordinate (y) at ( 1.7778, -0.7222);

    \draw[<->, shorten >=-1cm, shorten <=-1cm] (A) to (B);
    \draw[<->, shorten >=-1cm, shorten <=-1cm] (C) to (D);
    \draw (P) to (y);

    \draw[fill=black] (P) circle (0.5mm);
    \draw[fill=black] (A) circle (0.5mm);
    \draw[fill=black] (B) circle (0.5mm);
    \draw[fill=black] (C) circle (0.5mm);
    \draw[fill=black] (D) circle (0.5mm);
    \draw[fill=black] (x) circle (0.5mm);
    \draw[fill=black] (y) circle (0.5mm);

    \node at (A) [below] {$A$};
    \node at (B) [above] {$B$};
    \node at (C) [above] {$C$};
    \node at (D) [below] {$D$};
    \node at (P) [right] {$P$};
    \node at (x) [above] {$x$};
    \node at (y) [below] {$y$};
\end{tikzpicture}
            \caption{Example of a Mapping from Projective Geometry}
            \label{fig:Example_of_Map_from_Projective_Geometry}
        \end{figure}
        Such constructions play a central role in geometry, and particularly
        projective geometry, and culminate in a beautiful theorem known as
        \textit{Desargues's Theorem}\index{Theorem!Desargues's}, named after
        Girard Desargues. From this historical motivation, the word function,
        map, and mapping are synonomous in the realm of mathematics. In select
        fields such as analysis and geometry the words \textit{operator} and
        \textit{transformation} are used, and occasionally the word
        \textit{graph} is used. We will try to be clear in our useage of this
        vocabularly to rid of any potential ambiguity.
    \subsection{The Axiom of Choice and Diaconescu's Theorem}
        The next two axioms to be introduced are the most controversial of those
        listed in ZFC: The \textit{axiom of infinity} and the
        \textit{axiom of choice}. While the axiom of infinity only has a
        small number of critics, the axiom of choice is far more contentious.
        Choice is equivalent to many other statements that come across in
        almost all forms of mathematics (analysis, algebra, topology, etc.).
        Many of which are theorems we would \textit{want} to be true, and so
        accepting the axiom of choice allows us to prove them. In particular,
        the axioms presented thus far can be combined with the axiom of choice
        to prove the \textit{Law of the Excluded Middle}, a result known as
        Diaconescu's theorem, and this is our current goal.
        \begin{faxiom}{Axiom of Choice}{Axiom_of_Choice}
            If $\mathcal{O}$ is a non-empty set such that for all
            $\mathcal{U}\in\mathcal{O}$ it is true that $\mathcal{U}$ is
            non-empty, and if $\bigcup\mathcal{O}$ is the union over
            $\mathcal{O}$, then there is a function
            $f:\mathcal{O}\rightarrow\mathcal{F}$ such that for all
            $x\in\mathcal{O}$ it is true that $f(x)\in{x}$.
            Formally:\index{Axiom!of Choice}
            \begin{equation*}
                \forall_{\mathcal{O}}\big((\mathcal{O}\ne\emptyset)\land
                    (\emptyset\notin\mathcal{O})\big)
                \exists_{f:\mathcal{O}\rightarrow\bigcup\mathcal{O}}\Big(
                \forall_{x\in\mathcal{O}}\big(f(x)\in{x}\big)\Big)
            \end{equation*}
        \end{faxiom}
        Such a function is called a \textit{choice function}%
        \index{Function!Choice Function}. The axiom can be made obviously true
        or obviously false depending on how we word it. To convince one of its
        validity requires talking about products\index{Product!of Sets}. The
        Cartesian product\index{Cartesian Product} has been defined using
        ordered pairs\index{Ordered Pair} as defined by
        Kuratowski\index{Kazimierz, Kuratowski} and allows us to order two
        elements. Given two sets $A$ and $B$ we can define an equivalent notion
        using the set of all functions from $\mathbb{Z}_{2}=\{0,1\}$ into
        $A\cup{B}$ with a particular property:
        \begin{equation}
            A\times{B}=
            \big\{\,f:\mathbb{Z}_{2}\rightarrow{A}\cup{B}\;|\;
                f(0)\in{A}\textrm{ and }f(1)\in{B}\,\big\}
        \end{equation}
        To see why this is equivalent note that $A\times{B}$ is the set of all
        ordered pairs whose first entry lies in $A$ and whose second entry lies
        in $B$. Given $a\in{A}$ and $b\in{B}$, let
        $f:\mathbb{Z}_{2}\rightarrow{A}\cup{B}$ be the function such that
        $f(0)=a$ and $f(1)=b$. Then we can identify the ordered pair $(a,b)$
        with $f$. Indeed, $(a,b)=(f(0),f(1))$ making our identification very
        explicit. We can now generalize to a collection of $n$ different sets
        and define the ordered $n$ tuple over a collection of $n$ sets to be the
        set of all functions from $\mathbb{Z}_{n}$ into the union over this
        collection with a similar property:
        \begin{equation}
            \prod_{k\in\mathbb{Z}_{n}}A_{k}
            =\Big\{\,f:\mathbb{Z}_{n}\rightarrow\bigcup_{k\in\mathbb{Z}_{n}}^{n}
                A_{k}\;\big|\;f(k)\in{A}_{k}\textrm{ for all }k\in\mathbb{Z}_{n}
            \Big\}
        \end{equation}
        Given the function $f$ that maps $k$ to $a_{k}\in{A}_{k}$, we identify
        this by:
        \begin{equation}
            f=(a_{0},\,a_{1},\,\dots,\,a_{k},\,\dots,\,a_{n-1})
            =(f(0),\,f(1),\,\dots,\,f(k),\,\dots,\,f(n-1))
        \end{equation}
        And thus we have a more general way of defining products. Note that
        swapping the order of the product is equivalent to changing functions,
        so two $n$ tuples are equal if and only if all of their entries are
        equal. What's nice about our function definition is that it allows one
        to define products over \textit{arbitrary} collections. This is crucial
        for topology and analysis as we often wish to speak of
        \textit{infinite dimensional} spaces that are constructed using these
        abstract products. Given a set $I$, often called the \textit{index set},
        such that for all $\mathcal{U}\in{I}$ it is true $\mathcal{U}$ is a set,
        we can form the product over $I$ by defining this to be the collection
        of all functions from $I$ into the union over $I$.
        \begin{equation}
            \prod_{i\in{I}}A_{i}
            =\big\{\,f:I\rightarrow\bigcup_{i\in{I}}A_{i}\;|\;
                f(i)\in{A}_{i}\textrm{ for all }i\in{I}\,\big\}
        \end{equation}
        The axiom of choice\index{Axiom!of Choice} is equivalent to the
        statement \textit{The infinite product of non-empty sets is non-empty}.
        These functions that identify $k$ with the $k^{th}$ set are precisely
        choice functions. Phrasing it like this we see that the axiom of choice
        is somewhat obvious. The infinite product of non-empty sets is most
        likely enormous! Claiming it's non-empty seems trivial. It is then
        unfortunate that this claim can not be proven with the other axioms
        we've developed. As stated before, the axiom of choice is equivalent to
        many other statements such as \textit{Zorn's lemma, Tychonoff's theorem,
        the well ordering theorem, every vector space has a basis, every set has
        a group structure}, and countless more. Many of these theorems have many
        applications to algebra, analysis, and topology, and since we would like
        to use them to prove other things we are forced to accept the axiom of
        choice. Many theorems in real analysis hide the use of the axiom of
        choice by constructing sequences \textit{by induction}. An attempt will
        be made to be clear whenever the axiom of choice is used in a proof.
        \par\hfill\par
        We conclude this section by presenting Diaconescu's theorem.
        \begin{ftheorem}{Diaconescu's Theorem}{Diaconescus_Theorem}
            If $P$ is a proposition on sets and if $x$ is a set, then either
            $P(x)$ is true or the negation of $P(x)$ is true. That is,
            $P\lor\neg{P}$ is true.
            \index{Theorem!Diaconescu's}
        \end{ftheorem}
        \begin{bproof}
            Let $0=\emptyset$. By
            Thm.~\ref{thm:Existence_of_Set_Containing_Set}, we have that the set
            $\{0\}$ exists. Let $1=\{0\}$. Then since $0\in{1}$, $0\ne{1}$
            (Thm.~\ref{thm:Containment_NEqual_Underlying_Set}). Since $0$ and
            $1$ are sets, by Thm.~\ref{thm:Existence_of_Set_Built_from_Two_Sets}
            we have that the set $\{0,1\}$ exists. Let $Q$ be the proposition
            \textit{true if P(x) or } $x=0$, \textit{false otherwise}. By the
            axiom schema of specification
            (Ax.~\ref{ax:Axiom_Schema_of_Specification}) there exists a set
            $\mathcal{U}$ such that:
            \begin{equation}
                \mathcal{U}=\big\{\,x\in\{\,0,\,1\,\}\;|\;Q(x)\,\big\}
            \end{equation}
            Similarly, let $R$ be the proposition \textit{true if P(x) or}
            $x=1$, \textit{false otherwise}. By the axiom schema of
            specification we have that the following set exists:
            \begin{equation}
                \mathcal{V}=\big\{\,x\in\{\,0,\,1\,\}\;|\;R(x)\,\big\}
            \end{equation}
            By Thm.~\ref{thm:Existence_of_Set_Built_from_Two_Sets}, we have that
            the set $\{\,\mathcal{U},\,\mathcal{V}\,\}$ exists. By the axiom of
            choice (Ax.~\ref{ax:Axiom_of_Choice}), there exists a function
            $f:\{\mathcal{U},\mathcal{V}\}\rightarrow%
             \bigcup\{\mathcal{U},\mathcal{V}\}$ such that
            $f(\mathcal{U})\in\mathcal{U}$ and $f(\mathcal{V})\in\mathcal{V}$.
            But then, by the definition of $\mathcal{U}$, either
            $f(\mathcal{U})=0$ or $P(x)$ is true. Similarly, either
            $f(\mathcal{V})=1$ or $P(x)$ is true. But since $0\ne{1}$, either
            $f(\mathcal{U})\ne{f}(\mathcal{V})$ or $P(x)$ is true. Again by the
            axiom of extensionality (Ax.~\ref{ax:Axiom_of_Extensionality}), and
            by the definition of $\mathcal{U}$ and $\mathcal{V}$, if $P(x)$ is
            true then $\mathcal{U}=\mathcal{V}$. But then
            $f(\mathcal{U})=f(\mathcal{V})$. But then, by the contrapositive,
            $\neg{P}(x)$ implies that $f(\mathcal{U})\ne{f}(\mathcal{V})$. But
            by extensionality, either $f(\mathcal{U})=f(\mathcal{V})$ or
            $f(\mathcal{U})\ne{f}(\mathcal{V})$, and thus either $P(x)$ or
            $\neg{P}(x)$. That is, $P\lor\neg{P}$ is true.
        \end{bproof}
        We can now prove things via \textit{proof by contradiction}. While we
        have made great efforts to justify every step of a proof thus far, we
        will often omit mention of Diaconescu's theorem as a justification for
        the law of the excluded middle and simply use it freely. We may rest
        easy knowing that we've proved it's validity within the framework of
        ZFC. There are two more axioms remaining, that of infinity and
        replacement. The axiom of infinity is best introduced when we construct
        the natural numbers, and from there build the real numbers, and thus we
        shall delay its development briefly. The axiom of replacement needs a
        notion of class and thus will also be postponed.
        \section{The Structure of Sets}
    We've developed two \textit{operations}\index{Binary Operation} on sets thus
    far, that of union and intersection. Several of the properties of these two
    give rise to a structure known as a
    \textit{Boolean algebra}\index{Boolean Algebra}. Many theorems about sets
    can thus be proven in an algebraic setting by using the structure of a
    Boolean algebra, and thus it is our current goal to prove the basics about
    unions and operations so that we may transition to algebraic proofs.
    \subsection{Basic Theorems}
        With the law of the excluded middle in our toolbelt, we can now rapidly
        prove many basic and familiar results.
        \begin{theorem}
            \label{thm:Emptyset_Is_Subset}%
            If $A$ is a set, then $\emptyset\subseteq{A}$.
        \end{theorem}
        \begin{proof}
            For if not, then there is an $x\in\emptyset$ such that
            $x\notin{A}$ (Def.~\ref{def:Subsets}). But for all $x$ it is true
            that $x\notin\emptyset$ (Def.~\ref{def:Empty_Set}), a contradiction.
            Therefore $\emptyset\subseteq{A}$.
        \end{proof}
        This short theorem allows us to prove that the empty set is unique.
        \begin{theorem}
            \label{thm:Empty_Set_is_Unique}%
            If $\emptyset'$ is a set with no elements, then
            $\emptyset=\emptyset'$.
        \end{theorem}
        \begin{proof}
            For suppose not. But $\emptyset'$ is a set, and thus
            $\emptyset\subseteq\emptyset'$ (Thm.~\ref{thm:Emptyset_Is_Subset}).
            By the definition of equality if $\emptyset\ne\emptyset'$,
            then $\emptyset'\nsubseteq\emptyset$ (Def.~\ref{def:Equal_Sets}).
            But then there is an $x$ such that $x\in\emptyset'$ and
            $x\notin\emptyset$ (Def.~\ref{def:Subsets}). But by hypothesis
            $\emptyset'$ contains no elements, a contradiction. Thus
            $\emptyset'\subseteq\emptyset$. Therefore, $\emptyset=\emptyset'$
            (Def.~\ref{def:Equal_Sets}).
        \end{proof}
        We can use this to define \textit{the} empty set.
        \begin{fdefinition}{The Empty Set}{Empty_Set}
            The \gls{empty set} is the unique \gls{set} $\emptyset$ such that
            for all $x$ it is true that $x\notin\emptyset$.\index{Empty Set}
        \end{fdefinition}
        \begin{ltheorem}{Transitivity of Inclusion}{Transitivity_of_Inclusion}
            If $A$, $B$, and $C$ are sets, if $A\subseteq{B}$, and if
            $B\subseteq{C}$, then $A\subseteq{C}$.\index{Relation!Transitive}
            \index{Transitivity!of Inclusion}
        \end{ltheorem}
        \begin{proof}
            For suppose not. Then by the definition of subset there is an
            $x\in{A}$ such that $x\notin{C}$ (Def.~\ref{def:Subsets}). But $A$
            is a subset of $B$ and thus $x\in{B}$ (Def.~\ref{def:Subsets}).
            Similarly, $B$ is a subset of $C$ and therefore $x\in{C}$
            (Def.~\ref{def:Subsets}), a contradiction.
        \end{proof}
        Containment (\gls{containmentsymb}) is not transitive. That is, if
        $A\in{B}$ and $B\in{C}$, it may not be true that $A\in{C}$, nor is it
        necessarily true that $\{A\}\in{C}$, or even $\{A\}\subseteq{C}$. For a
        simple example, let $A=\emptyset$, $B=\{A,\{A\}\}$, and
        $C=\{B\}$. Then by definition $A\in{B}$ and $B\in{C}$, but $A\notin{C}$.
        This is because for all $x$ it is true that $x\in{C}$ if and only if
        $x=B$, and since $A\in{B}$ it is necessarily true that $A\ne{B}$
        (see Thm.~\ref{thm:Containment_NEqual_Underlying_Set}), and therefore
        $A\notin{C}$. Moreover, since $\{A\}\notin\{A\}$
        (Thm.~\ref{thm:Anti_Russells_Paradox}), we have that $\{A\}\ne{B}$.
        For $\{A\}\in{B}$ by definition and $\{A\}\notin\{A\}$, and thus by
        the axiom of extensionality (Ax.~\ref{ax:Axiom_of_Extensionality}) we
        may conclude that $\{A\}\ne{B}$. From this we may then conclude that
        $\{A\}\notin{C}$ and $\{A\}\nsubseteq{C}$.
        \par\hfill\par
        It is possible to find sets such that $A\in{B}$ and $B\in{C}$, and
        furthermore such that $A\in{C}$, we need only consider a nested chain
        of sets. Let $A=\emptyset$, $B=\{A,\{A\}\}$, and
        $C=\{A,B,\{B\}\}$. Then by construction, $A\in{B}$, $B\in{C}$, and
        $A\in{C}$. Such chains are used in the von Neumann construction of the
        integers.\index{von Neumann, John}
        \par\hfill\par
        With the transitivity of inclusion we can present a few brief
        corrolaries.
        \begin{theorem}
            \label{thm:Subsets_of_Equal_Sets}%
            If $A$, $B$, and $C$ are sets, if $A=B$, and if $C\subseteq{A}$,
            then $C\subseteq{B}$.
        \end{theorem}
        \begin{proof}
            For if $A=B$, then $A\subseteq{B}$ (Def.~\ref{def:Equal_Sets}). But
            by the transitivity of inclusion, if $C\subseteq{A}$ and
            $A\subseteq{B}$, then $C\subseteq{B}$
            (Thm.~\ref{thm:Transitivity_of_Inclusion}).
        \end{proof}
        \begin{theorem}
            \label{thm:Superset_of_Equal_Sets}%
            If $A,B$, and $C$ are sets, if $A=B$, and if $A\subseteq{C}$, then
            $B\subseteq{C}$.
        \end{theorem}
        \begin{proof}
            For if $A=B$ then $B\subseteq{A}$ (Def.~\ref{def:Equal_Sets}).
            But if $B\subseteq{A}$ and $A\subseteq{C}$, then by the transitivity
            of inclusion, $B\subseteq{C}$
            (Thm.~\ref{thm:Transitivity_of_Inclusion}).
        \end{proof}
        More than just being transitive, inclusion is also reflexive.
        \begin{ltheorem}{Reflexivity of Inclusion}{Reflexivity_of_Inclusion}
            If $A$ is a set, then $A\subseteq{A}$.
            \index{Relation!Reflexive}\index{Reflexivity!of Inclusion}
        \end{ltheorem}
        \begin{proof}
            For if not, then by the definition of subset there is an $x\in{A}$
            such that $x\notin{A}$ (Def.~\ref{def:Subsets}), a contradiction.
            Therefore $A\subseteq{A}$.
        \end{proof}
        The notion of inclusion $(A\subseteq{B})$ is also
        antisymmetric\index{Relation!Antisymmetric}. That is to say, if
        $A\subseteq{B}$ and $B\subseteq{A}$, then $A=B$. This is simply the
        definition of equality (Def.~\ref{def:Equal_Sets}). A relation that is
        transitive, reflexive, and antisymetric is known as a partial
        ordering\index{Partial Order} and will be very important in the study of
        order theory and topology. The partial ordering of inclusion on the
        power set of some given set is the quintessential example of a partial
        ordering. On the other hand, the notion of containment is not reflexive.
        That is, for any set $A$ it is true that $A\notin{A}$
        (Thm.~\ref{thm:Anti_Russells_Paradox}). Indeed, this was a desired
        characteristic of containment to allow us to avoid Russell's
        Paradox\index{Russell's Paradox}.
        \begin{ltheorem}{Reflexivity of Equality}{Equality_Reflexive}
            If $A$ is a set, then $A=A$.
            \index{Relation!Reflexive}\index{Reflexivity!of Equality}
        \end{ltheorem}
        \begin{proof}
            For if $A$ is a set then $A\subseteq{A}$
            (Thm.~\ref{thm:Reflexivity_of_Inclusion}). Thus, $A=A$
            (Def.~\ref{def:Equal_Sets}).
        \end{proof}
        \begin{ltheorem}{Symmetry of Equality}{Symmetry_of_Equality}
            If $A$ and $B$ are sets and if $A=B$, then $B=A$.
            \index{Relation!Symmetric}\index{Symmetry!of Equality}
        \end{ltheorem}
        \begin{proof}
            For suppose not. If $B\ne{A}$, then either $B\nsubseteq{A}$ or
            $A\nsubseteq{B}$. But $A=B$, and thus $A\subseteq{B}$  and
            $B\subseteq{A}$ (Def.~\ref{def:Equal_Sets}),
            a contradiction. Thus, $B=A$.
        \end{proof}
        An important but non-obvious statement is that containment
        (\gls{containmentsymb}) is \textit{not} symmetric, and indeed is the
        exact opposite, it is antisymetric. That is, for any two sets $A$ and
        $B$ it impossible for both $A\in{B}$ and $B\in{A}$, for this would
        violate the axiom of regularity\index{Axiom!of Regularity}
        (Ax.~\ref{ax:Axiom_of_Regularity}). We now prove this claim rigorously.
        \begin{ltheorem}{Antisymmetry of Containment}
                        {Antisymmetry_of_Containment}
            If $A$ and $B$ are sets, and if $A\in{B}$, then $B\notin{A}$.
            \index{Relation!Antisymmetric}\index{Antisymmetry!of Containment}
        \end{ltheorem}
        \begin{proof}
            For suppose not, and suppose $B\in{A}$. But by
            Thm.~\ref{thm:Existence_of_Set_Built_from_Two_Sets}, if $A$ and $B$
            are sets, then there is a set $\{A,B\}$ such that for all $x$ it is
            true that $x\in\{A,B\}$ if and only if $x=A$ or $x=B$. But then
            $\{A,B\}$ is a non-empty set, and thus by the axiom of regularity
            (Ax.~\ref{ax:Axiom_of_Regularity}) there is an $x\in\{A,B\}$ such
            that $x\cap\{A,B\}=\emptyset$. But by hypothesis, $A\in{B}$, and
            thus $A\in{B}\cap\{A,B\}$ (Def.~\ref{def:Intersection_of_Two_Sets}),
            and therefore $x\ne{B}$. But similarly, if $B\in{A}$, then
            $B\in{A}\cap\{A,B\}$ and thus $x\ne{B}$. But $x\in\{A,B\}$ if and
            only if $x=A$ or $x=B$, a contradiction. Therefore, $B\notin{A}$.
        \end{proof}
        An instant corrolary of this is that $\{A\}\notin{A}$.
        \begin{theorem}
            \label{thm:Set_Containing_A_is_not_Element_of_A}%
            If $A$ is a set, then $\{A\}\notin{A}$.
        \end{theorem}
        \begin{proof}
            For $A\in\{A\}$. But if $A\in\{A\}$, then $\{A\}\notin{A}$
            (Thm.~\ref{thm:Antisymmetry_of_Containment}).
        \end{proof}
        We now continue developing the structure of equality.
        \begin{ltheorem}{Transitivity of Equality}{Transitivity_of_Equality}
            If $A$, $B$, and $C$ are sets, if $A=B$, and if $B=C$, then $A=C$.
        \end{ltheorem}
        \begin{proof}
            For if $B=C$, then $C\subseteq{B}$ (Def.~\ref{def:Equal_Sets}). But
            if $A=B$, then $B=A$ (Thm.~\ref{thm:Reflexivity_of_Inclusion}). But
            if $B=A$ and $C\subseteq{B}$, then $C\subseteq{A}$
            (Thm.~\ref{thm:Subsets_of_Equal_Sets}). And if $A=B$, then
            $A\subseteq{B}$ (Def.~\ref{def:Equal_Sets}). But if $B=C$ and
            $A\subseteq{B}$, then $A\subseteq{C}$
            (Thm.~\ref{thm:Subsets_of_Equal_Sets}). But it was proved that
            $C\subseteq{A}$, and thus $A=C$ (Def.~\ref{def:Equal_Sets}).
        \end{proof}
        The three properties we've proved thus far, that of reflexivity
        (Thm.~\ref{thm:Equality_Reflexive}), symmetry
        (Thm.~\ref{thm:Symmetry_of_Equality}), and transitivity
        (Thm.~\ref{thm:Transitivity_of_Equality}) are the key ingredients to
        defining \textit{equivalence relations}\index{Equivalence Relation}.
        Equivalence relations are used to model the notion of equality in more
        abstract settings and are fundamental in the study of algebra and
        topology. We'll discuss these more in
        \S~\ref{Section:ZFC:Elementary_Set_Theory:Relations}.
        \begin{theorem}
            \label{thm:Prop_Subset_Not_Equal}%
            If $A$ and $B$ are sets, and if $A\subsetneq{B}$, then there is an
            $x\in{B}$ such that $x\notin{A}$.
        \end{theorem}
        \begin{proof}
            For suppose not. Then for all $x\in{B}$ it is true that $x\in{A}$.
            But then $B\subseteq{A}$ (Def.~\ref{def:Subsets}).
            But $A\subseteq{B}$ and thus $A=B$ (Def.~\ref{def:Equal_Sets}).
            But $A\subsetneq{B}$ and therefore $A\ne{B}$, a contradiction.
        \end{proof}
        Theorem \ref{thm:Prop_Subset_Not_Equal} can be used as an equivalent
        definition of a proper subset. That is, a proper subset is a subset that
        is missing at least one element.
    \subsection{Operations on Sets}
        Similar to the arithmetic of real numbers, there are standard operations
        that can be performed on sets to obtain new sets. The four most common
        operations are union, intersection, set difference, and symmetric
        difference. As stated before, we wish to build the structure of sets in
        an algebraic manner. To do this requires the notion that the operations
        of intersection and unions are \textit{commutative},
        \textit{distributive}, have \textit{identities}, and have
        \textit{complements}.
        \begin{ltheorem}{Commutative Law of Unions}{Commutative_Law_of_Unions}
            If $A$ and $B$ are sets, then $A\cup{B}=B\cup{A}$.
        \end{ltheorem}
        \begin{proof}
            For if $x\in{A}\cup{B}$, then either $x\in{A}$ or $x\in{B}$, or both
            (Def.~\ref{def:Union_of_Two_Sets}). But then either $x\in{B}$ or
            $x\in{A}$, or both, and therefore $x\in{B}\cup{A}$
            (Def.~\ref{def:Union_of_Two_Sets}). But then for all
            $x\in{A}\cup{B}$ it is true that $x\in{B}\cup{A}$, and therefore
            $A\cup{B}\subseteq{B}\cup{A}$ (Def.~\ref{def:Subsets}). Similarly,
            $B\cup{A}\subseteq{A}\cup{B}$, and thus
            $A\cup{B}=B\cup{A}$ (Def.~\ref{def:Equal_Sets}).
        \end{proof}
        When taking the union of two sets, we obtain a \textit{larger} set, in
        a sense. Again relying on the analogy of arithmetic, given two
        non-negative integers $a$ and $b$, it is true that $a\leq{a}+b$.
        Equality is obtained if and only if either $b$ is equal to zero. The
        empty set thus acts as the \textit{zero} of unions. Also, given three
        non-negative integers $a$, $b$, and $c$, if $b\leq{c}$, then
        $a+b\leq{a}+c$. A similar result will hold for unions.
        \begin{theorem}
            \label{thm:Union_is_Bigger}%
            If $A$ and $B$ are sets, then $A\subseteq{A}\cup{B}$.
        \end{theorem}
        \begin{proof}
            For suppose not. Then there is an $x\in{A}$ such that
            $x\notin{A}\cup{B}$. But if $x\in{A}$, then $x\in{A}$ or $x\in{B}$
            and thus $x\in{A}\cup{B}$ (Def.~\ref{def:Union_of_Two_Sets}), a
            contradiction.
        \end{proof}
        \begin{theorem}
            \label{thm:Union_With_Lesser_Set_on_Right}%
            If $A$, $B$, and $C$ are sets, and if $B\subseteq{C}$, then
            $A\cup{B}\subseteq{A}\cup{C}$.
        \end{theorem}
        \begin{proof}
            For if $x\in{A}\cup{B}$, then either $x\in{A}$, or $x\in{B}$, or
            both (Def.~\ref{def:Union_of_Two_Sets}). But $B$ is a subset of $C$,
            and therefore if $x\in{B}$, then $x\in{C}$ (Def.~\ref{def:Subsets}).
            Thus, if $x\in{A}$ or $x\in{B}$, then $x\in{A}$ or $x\in{C}$, and
            therefore $x\in{A}\cup{C}$ (Def.~\ref{def:Union_of_Two_Sets}).
            Thus, $A\cup{B}\subseteq{A}\cup{C}$ (Def.~\ref{def:Subsets}).
        \end{proof}
        \begin{theorem}
            \label{thm:Union_With_Lesser_Set_on_Left}%
            If $A$, $B$, and $C$ are sets, and if $B\subseteq{C}$, then
            $B\cup{A}\subseteq{C}\cup{A}$.
        \end{theorem}
        \begin{proof}
            For $B\cup{A}=A\cup{B}$ (Thm.~\ref{thm:Commutative_Law_of_Unions}).
            But if $B\subseteq{C}$, then $A\cup{B}\subseteq{A}\cup{C}$
            (Thm.~\ref{thm:Union_With_Lesser_Set_on_Right}). And if
            $B\cup{A}=A\cup{B}$ and $A\cup{B}\subseteq{A}\cup{C}$, then
            $B\cup{A}\subseteq{A}\cup{C}$
            (Thm.~\ref{thm:Superset_of_Equal_Sets}). But $A\cup{C}=C\cup{A}$
            (Thm.\ref{thm:Commutative_Law_of_Unions}) and if $A\cup{C}=C\cup{A}$
            and $B\cup{A}\subseteq{A}\cup{C}$, then
            $B\cup{A}\subseteq{C}\cup{A}$
            (Thm.~\ref{thm:Subsets_of_Equal_Sets}).
        \end{proof}
        \begin{theorem}
            If $A$, $B$, $C$, and $D$ are sets, if $A\subseteq{C}$, and if
            $B\subseteq{D}$, then $A\cup{B}\subseteq{C}\cup{D}$.
        \end{theorem}
        \begin{proof}
            For if $B\subseteq{D}$, then $A\cup{B}\subseteq{A}\cup{D}$
            (Thm.~\ref{thm:Union_With_Lesser_Set_on_Right}). But if
            $A\subseteq{C}$, then $A\cup{D}\subseteq{C}\cup{D}$
            (Thm.~\ref{thm:Union_With_Lesser_Set_on_Left}), obtaining the
            result.
        \end{proof}
        Taking the union of subsets is redundant, as we simply obtain the larger
        set. This starts to break down the analogy between sets and arithmetic,
        since there is only one \textit{zero}. That is, there is only one number
        $b$ such that $a+b=a$, and that is $b=0$. While any subset acts as a
        \textit{zero} of a given set, the empty set has the property that it
        acts as a zero for \textit{every} set. It is the only set with this
        property, and thus the analogy with arithmetic is slightly restored.
        \begin{theorem}
            \label{thm:Union_With_Subset}%
            If $A$ and $B$ are sets, and if $A\subseteq{B}$, then $A\cup{B}=B$.
        \end{theorem}
        \begin{proof}
            For if $A$ and $B$ are sets, then $B\subseteq{A}\cup{B}$
            (Thm.~\ref{thm:Union_is_Bigger}). But if $A\subseteq{B}$, then for
            all $x\in{A}$ it is true that $x\in{B}$ (Def.~\ref{def:Subsets}).
            Thus if $x\in{A}$ or if $x\in{B}$, then $x\in{B}$. But then, for all
            $x\in{A}\cup{B}$, it is true that $x\in{B}$, and therefore
            $A\cup{B}\subseteq{B}$ (Def.~\ref{def:Subsets}). Thus, $A\cup{B}=B$
            (Def.~\ref{def:Equal_Sets}).
        \end{proof}
        \begin{theorem}
            \label{thm:Union_with_Emptyset}%
            If $A$ is a set, then $A=\emptyset\cup{A}$.
        \end{theorem}
        \begin{proof}
            For $\emptyset\subseteq{A}$ (Thm.~\ref{thm:Emptyset_Is_Subset}) and
            therefore $\emptyset\cup{A}=A$ (Thm.~\ref{thm:Union_With_Subset}).
        \end{proof}
        \begin{theorem}
            \label{thm:Empty_Set_Is_Zero_for_Unions}%
            If $A$ is a set such that, for any set $B$ it is true that
            $A\cup{B}=B$, then $A$ is the empty set.
        \end{theorem}
        \begin{proof}
            For suppose not. If $A\ne\emptyset$, then there is an $x\in{A}$
            (Def.~\ref{def:Empty_Set}). But if $A$ is a set, then $B=\{A\}$ is a
            set (Thm.~\ref{thm:Existence_of_Set_Containing_Set}). But then
            $x\in{A}\cup{B}$ (Def.~\ref{def:Union_of_Two_Sets}). But
            $x\notin{B}$, and thus $A\cup{B}\ne{B}$
            (Def.~\ref{def:Equal_Sets}), a contradiction
            since $A$ is such that for any set $B$, it is
            true that $A\cup{B}=B$. Thus, $A$ is the empty set.
        \end{proof}
        Thm.~\ref{thm:Empty_Set_Is_Zero_for_Unions} proves the assertion that
        the empty set is the zero of set union. The converse of
        Thm.~\ref{thm:Union_With_Subset} can be proved as well.
        \begin{theorem}
            \label{thm:Conv_Union_Is_Bigger}%
            If $A$ and $B$ are sets, and if $A\cup{B}\subseteq{A}$, then
            $A\cup{B}=A$.
        \end{theorem}
        \begin{proof}
            For $A\subseteq{A}\cup{B}$ (Thm.~\ref{thm:Union_is_Bigger}). But by
            hypothesis, $A\cup{B}\subseteq{A}$. But then $A=A\cup{B}$
            (Def.~\ref{def:Equal_Sets}).
        \end{proof}
        \begin{theorem}
            \label{thm:Union_is_Equal}%
            If $A$ and $B$ are sets, and if $A\cup{B}\subseteq{A}$, then
            $B\subseteq{A}$.
        \end{theorem}
        \begin{proof}
            For if $A\cup{B}\subseteq{A}$, then $A\cup{B}=A$
            (Thm.~\ref{thm:Conv_Union_Is_Bigger}). And also,
            $B\subseteq{A}\cup{B}$ (Thm.~\ref{thm:Union_is_Bigger}). But if
            $A\cup{B}=A$ and $B\subseteq{A}\cup{B}$, then $B\subseteq{A}$
            (Thm.~\ref{thm:Subsets_of_Equal_Sets}).
        \end{proof}
        Logically, and pedagogically, it would seem appropriate to demonstrate
        the fact that union is an \textit{associative} operation on sets. We
        will instead hold off on this and prove it in a purely algebraic setting
        once we've developed Boolean algebras. Boolean algebras consist of two
        operations that are commutative, distributive, contain identities and
        complements. For the case of sets we will use union and intersection as
        our operations, and the power set of some given set as the set which
        these operations act on. First, we'll state associativity of unions:
        \begin{equation}
            A\cup(B\cup{C})=(A\cup{B})\cup{C}
        \end{equation}
        This becomes intuitively clear when one examines the Venn diagram
        shown in Fig.~\ref{fig:Union_of_Three_Sets}. Using the analogy of
        arithmetic, given three real numbers $a$, $b$, and $c$, it is true that
        $a+(b+c)=(a+b)+c$. This is called the associative law of addition.
        Combining this law with the commutative law shows that the order in
        which three real numbers are added is irrelevant. Applying induction, we
        see that given any finite collection of real numbers, the order in
        which we add them is again irrelevant. The same holds true for the union
        of sets. We now move on to intersections, and return to the associative
        law in Chapter~\ref{chapt:Function_Theory},
        (Section~\ref{sec:Boolean_Algebra}).
        \begin{ltheorem}{Commutative Law of Intersections}{Commut_Law_Intersec}
            If $A$ and $B$ are sets, then $A\cap{B}=B\cap{A}$.
        \end{ltheorem}
        \begin{proof}
            For if $x\in{A}\cap{B}$, then $x\in{A}$ and
            $x\in{B}$. But then $x\in{B}$ and $x\in{A}$,
            and therefore $x\in{B}\cap{A}$
            (Def.~\ref{def:Intersection_of_Two_Sets}). But then
            for all $x\in{A}\cap{B}$ it is true that
            $x\in{B}\cap{A}$, and therefore
            $A\cup{B}\subseteq{B}\cup{A}$
            (Def.~\ref{def:Subsets}). Similarly,
            $B\cap{A}\subseteq{A}\cap{B}$, and thus
            $A\cap{B}=B\cap{A}$ (Def.~\ref{def:Equal_Sets}).
            Therefore, etc.
        \end{proof}
        \begin{theorem}
            \label{thm:Intersection_is_Smaller}%
            If $A$ snd $B$ are sets, then
            $A\cap{B}\subseteq{A}$.
        \end{theorem}
        \begin{proof}
            If $x\in{A}\cap{B}$, then $x\in{A}$ and
            $x\in{B}$, and thus $x\in{A}$. Therefore, etc.
        \end{proof}
        \begin{theorem}
            \label{thm:Intersection_with_Lesser_Set}%
            If $A$, $B$, and $C$ are sets, and if
            $B\subseteq{C}$, then
            $A\cap{B}\subseteq{A}\cap{C}$.
        \end{theorem}
        \begin{proof}
            For if $x\in{A}\cap{B}$, then $x\in{A}$ and
            $x\in{B}$ (Def.~\ref{def:Intersection_of_Two_Sets}).
            But $B$ is a subset of $C$, and thus if
            $x\in{B}$, then $x\in{C}$
            (Def.~\ref{def:Subsets}). But then $x\in{A}$ and
            $x\in{C}$, and therefore $x\in{A}\cap{C}$
            (Def.~\ref{def:Intersection_of_Two_Sets}). But
            then $A\cap{B}\subseteq{A}\cap{C}$
            (Def.~\ref{def:Subsets}). Therefore, etc.
        \end{proof}
        \begin{figure}[H]
            \centering
            \captionsetup{type=figure}
            \centering
            \documentclass[crop,class=article]{standalone}
%----------------------------Preamble-------------------------------%
\usepackage{tikz}                       % Drawing/graphing tools.
%--------------------------Main Document----------------------------%
\begin{document}
    \begin{tikzpicture}
        \draw (-3,-2.3) rectangle (2.5,2.3);
        \draw (0,0) circle (2);
        \draw[fill=cyan] (0.5,-0.3) circle (0.85);
        \node at (0.5,-0.1) {$A$};
        \node at (0,1.3) {$B$};
        \node at (-2.3,1.5) {$A\cap{B}$};
    \end{tikzpicture}
\end{document}
            \caption{Visual for Thm.~\ref{thm:Intersection_of_Subset}.}
            \label{fig:Union_Intersection_venn_diagram}
        \end{figure}
        \begin{theorem}
            \label{thm:Intersection_is_Equal}%
            If $A$ and $B$ are sets, and if
            $A=A\cap{B}$, then $A\subseteq{B}$.
        \end{theorem}
        \begin{proof}
            For suppose not. Then there is an $x\in{A}$ such
            that $x\notin{B}$. But since $A=A\cap{B}$,
            if $x\in{A}$ then $x\in{A}\cap{B}$
            (Def.~\ref{def:Equal_Sets}). But if
            $x\in{A}\cap{B}$, then $x\in{B}$
            (Thm.~\ref{thm:Intersection_is_Smaller}),
            a contradiction. Therefore, etc.
        \end{proof}
        \begin{theorem}
            \label{thm:Intersection_of_Subset}%
            If $A$ and $B$ are sets, and if
            $A\subseteq{B}$, then $A\cap{B}=A$.
        \end{theorem}
        \begin{proof}
            For $A\cap{B}\subseteq{A}$
            (Thm.~\ref{thm:Intersection_is_Smaller}). But
            since $A$ is a subset of $B$, if $x\in{A}$, then
            $x\in{B}$ (Def.~\ref{def:Subsets}). But then
            $x\in{A}\cap{B}$
            (Def.~\ref{def:Intersection_of_Two_Sets}). Therefore,
            $A\subseteq{A}\cap{B}$ (Def~\ref{def:Subsets})
            and thus $A=A\cap{B}$ (Def~\ref{def:Equal_Sets}).
            Therefore, etc.
        \end{proof}
        \begin{theorem}
            \label{thm:Conv_Intersection_is_Smaller}%
            If $A$ and $B$ are sets, and if
            $A\subseteq{A}\cap{B}$, then $A=A\cap{B}$.
        \end{theorem}
        \begin{proof}
            For $A\cap{B}\subseteq{A}$
            (Thm.~\ref{thm:Intersection_is_Smaller}). But
            by hypothesis, $A\subseteq{A}\cap{B}$, and thus
            $A=A\cap{B}$ (Def.~\ref{def:Equal_Sets}).
            Therefore, etc.
        \end{proof}
        \begin{theorem}
            If $A$ is a set, then $\emptyset\cap{A}=\emptyset$.
        \end{theorem}
        \begin{proof}
            For $\emptyset\subseteq{A}$ (Thm.~\ref{thm:Emptyset_Is_Subset}), and
            therefore $\emptyset\cap{A}=\emptyset$
            (Thm.~\ref{thm:Intersection_of_Subset}).
        \end{proof}
        % \begin{theorem}
        %     \label{thm:Redundant_Intersection}%
        %     If $A$, $B$, and $C$ are sets and if
        %     $B\subseteq{A}$, then
        %     $A\cap(B\cap{C})=B\cap{C}$.
        % \end{theorem}
        % \begin{proof}
        %     For $A\cup(B\cup{C})\subseteq{B}\cup{C}$
        %     (Thm.~\ref{thm:Intersection_is_Smaller}). But
        %     $A\cap(B\cap{C})=(A\cap{B})\cap{C}$
        %     (Thm.~\ref{thm:Assoc_Law_Intersec}).
        %     And since $B$ is a subset of $A$,
        %     $A\cap{B}=A$
        %     (Thm.~\ref{thm:Intersection_of_Subset}),
        %     and thus $(A\cap{B})\cap{C}=B\cap{C}$. Thus,
        %     $B\cap{C}=A\cap(B\cap{C})$
        %     (Thm.~\ref{thm:Transitivity_of_Equality}).
        %     Therefore, etc.
        % \end{proof}
        \begin{theorem}
            \label{thm:First_Pseudo_Dist_Law_Union}%
            If $A$, $B$, and $C$ are sets, then
            $(B\cap{C})\subseteq(A\cup{B})\cap(A\cup{C})$.
        \end{theorem}
        \begin{proof}
            For $B\subseteq{A}\cup{B}$
            (Thm.~\ref{thm:Union_is_Bigger}). But then
            $B\cap{C}\subseteq(A\cup{B})\cap{C}$
            (Thm.~\ref{thm:Intersection_with_Lesser_Set}).
            But $C\subseteq{A}\cup{C}$
            (Thm.~\ref{thm:Union_is_Bigger}), and thus
            $(A\cup{B})\cap{C}%
            \subseteq(A\cup{B})\cap{A}\cup{C}$
            (Thm.~\ref{thm:Intersection_with_Lesser_Set}).
            But it was just proved that
            $B\cap{C}\subseteq(A\cup{B})\cap{C}$, and
            therefore by transivity,
            $(B\cap{C})\subseteq(A\cup{B})\cap(A\cup{C})$
            (Thm.~\ref{thm:Transitivity_of_Inclusion}).
            Therefore, etc.
        \end{proof}
        \begin{ftheorem}{Distributive Law of Unions}
            {Distributive_Law_Union}
            If $A$, $B$, and $C$ are sets, then:
            \begin{equation*}
                A\cup(B\cap{C})=(A\cup{B})\cap(A\cup{C})
            \end{equation*}
        \end{ftheorem}
        \begin{bproof}
            For $(B\cap{C})\subseteq(A\cup{B})\cap(A\cup{C})$
            (Thm.~\ref{thm:First_Pseudo_Dist_Law_Union}).
            But then:
            \begin{equation}
                A\cup(B\cap{C})\subseteq
                A\cup\Big((A\cup{B})\cap(A\cup{C})\Big)
            \end{equation}
            But $A\cup((A\cup{B})\cap(A\cup{C}))%
                =(A\cup{B})\cap(A\cup{C})$, and therefore:
            \begin{equation}
                A\cup(B\cap{C})\subseteq
                (A\cup{B})\cap(A\cup{C})
            \end{equation}
        \end{bproof}
        We can represent this pictorially using Venn diagrams.
        \begin{figure}[H]
            \centering
            \captionsetup{type=figure}
            \begin{tikzpicture}
    % Coordinates for the points.
    \coordinate (A) at (-2.6, 0.0);
    \coordinate (B) at ( 0.0, 0.0);
    \coordinate (C) at ( 2.6, 0.0);

    % Fill in the circle A.
    \draw[fill=cyan, draw=none] (A) circle (2);

    % Fill in the overlap between B and C.
    \draw[fill=cyan] (1.3, -1.51987) arc(-49.46:49.46:2) arc(130.54:229.46:2);

    % Draw circles outlining the regions A, B, and C.
    \draw[draw=black] (A) circle (2);
    \draw[draw=black] (B) circle (2);
    \draw[draw=black] (C) circle (2);

    % Add some labels.
    \node at (A) [above=1cm] {$A$};
    \node at (B) [above=1cm] {$B$};
    \node at (C) [above=1cm] {$C$};
\end{tikzpicture}
            \caption{Venn Diagram for the Distributive Law of Unions}
            \label{fig:Venn_Diagram_Distributive_Law_of_Union}
        \end{figure}
        \begin{ftheorem}{Distributive Law of Intersections}
            {Distributive_Law_Intersections}
            If $A$, $B$, and $C$ are sets, then:
            \begin{equation*}
                A\cap(B\cup{C})=(A\cap{B})\cup(A\cap{C})
            \end{equation*}
        \end{ftheorem}
        \begin{bproof}
            For by the distributive law of unions
            (Thm.~\ref{thm:Distributive_Law_Union}), we have:
            \begin{equation}
                    (A\cap{B})\cup(A\cap{C})
                    =\Big((A\cap{B})\cup{A}\Big)\cap\Big((A\cap{B})\cup{C}\Big)
            \end{equation}
            But $A\cap{B}\subseteq{A}$ (Thm.~\ref{thm:Intersection_is_Smaller})
            and thus $(A\cap{B})\cup{A}=A$ (Thm.~\ref{thm:Union_With_Subset}).
            Thus:
            \begin{equation}
                \Big((A\cap{B})\cup{A}\Big)\cap\Big((A\cap{B})\cup{C}\Big)
                =A\cap\Big((A\cap{B})\cup{C}\Big)
            \end{equation}
            But by the commutative law of unions
            (Thm.~\ref{thm:Commutative_Law_of_Unions}),
            $(A\cap{B})\cup{C}=C\cup(A\cap{B})$. Applying the distributive law
            again, we obtain:
            \begin{equation}
                A\cap\Big(C\cup(A\cap{B})\Big)
                =A\cap\Big((C\cup{A})\cap(C\cup{B})\Big)
            \end{equation}
            But by the associative law of intersections
            (Thm.~\ref{thm:Associative_Law_of_Intersections}) we have:
            \begin{equation}
                A\cap\Big((C\cup{A})\cap(C\cup{B})\Big)
                =\Big(A\cap(C\cup{A})\Big)\cap(C\cup{B})
            \end{equation}
            But $A\subseteq{C}\cup{A}$ (Thm.~\ref{thm:Union_is_Bigger}), and
            thus $A\cap(C\cup{A})=A$
            (Thm.~\ref{thm:Intersection_with_Lesser_Set}). But then:
            \begin{equation}
                \Big(A\cap(C\cup{A})\Big)\cap(C\cup{B})
                =A\cap(C\cup{B})
            \end{equation}
            By the transitivity of equality
            (Thm.~\ref{thm:Transitivity_of_Equality}) we conclude the result.
        \end{bproof}
        \begin{figure}[H]
            \centering
            \captionsetup{type=figure}
            \begin{tikzpicture}
                % Coordinates for the points.
                \coordinate (B) at (-2.6, 0.0);
                \coordinate (A) at ( 0.0, 0.0);
                \coordinate (C) at ( 2.6, 0.0);
            
                % Fill in the overlap between B and A.
                \draw[fill=cyan] (-1.3, -1.51987) arc(-49.46:49.46:2) arc(130.54:229.46:2);
            
                % Fill in the overlap between A and C.
                \draw[fill=cyan] (1.3, -1.51987) arc(-49.46:49.46:2) arc(130.54:229.46:2);
            
                % Draw circles outlining the regions A, B, and C.
                \draw[draw=black] (A) circle (2);
                \draw[draw=black] (B) circle (2);
                \draw[draw=black] (C) circle (2);
            
                % Add some labels.
                \node at (A) [above=1cm] {$A$};
                \node at (B) [above=1cm] {$B$};
                \node at (C) [above=1cm] {$C$};
            \end{tikzpicture}
            \caption{Venn Diagram for the Distributive Law of Intersections}
            \label{fig:Venn_Diagram_Distributive_Law_of_Intersections}
        \end{figure}
        Continuing with this analogy, we discuss set difference.
        \begin{theorem}
            \label{thm:Existence_of_Set_Difference}%
            If $A$ and $B$ are sets, then there is a set $C$ such that for all
            $x$ it is true that $x\in{C}$ if and only if $x\in{A}$ and
            $x\notin{B}$.
        \end{theorem}
        \begin{proof}
            For let $P$ be the proposition \textit{true if} $x\notin{B}$,
            \textit{false otherwise}. Then by the axiom schema of specification
            (Ax.~\ref{ax:Axiom_Schema_of_Specification}), there is a set $C$
            such that:
            \begin{equation}
                C=\{\,x\in{A}\;|\;P(x)\,\}
            \end{equation}
            But then $x\in{C}$ if and only if $x\in{A}$ and $P(x)$ is true.
            But $P(x)$ is true if and only if $x\notin{B}$. Thus $x\in{C}$ if
            and only if $x\in{A}$ and $x\notin{B}$.
        \end{proof}
        The set described in Thm.~\ref{thm:Existence_of_Set_Difference} is known
        as the set difference of $B$ with respect to $A$. We take a moment to
        discuss some of it's properties.
        \begin{fdefinition}{Set Difference}{Set_Difference}
            The set difference of a set $B$ with respect to a set $A$ is the
            set:
            \begin{equation*}
                A\setminus{B}=\{\,x\in{A}\;|\;x\notin{B}\,\}
            \end{equation*}
        \end{fdefinition}
        \begin{example}
            Let $\mathbb{Z}$ denote the integers, and $\mathbb{N}$ denote the
            set of natural numbers. Then $\mathbb{Z}\setminus\mathbb{N}$ is the
            set of all negative integers. Flipping this, we see that
            $\mathbb{N}\setminus\mathbb{Z}$ is the empty set since there are no
            non-negative integers that are not also integers.
        \end{example}
        \begin{example}
            Letting $\mathbb{R}$ denote the real numbers and $\mathbb{Q}$ denote
            the rationals, $\mathbb{R}\setminus\mathbb{Q}$ is the set of all
            \textit{irrational} numbers. Famous examples include $\sqrt{2}$,
            $\pi$, and $e$ (sometimes known as Euler's constant).
        \end{example}
        \begin{example}
            If we let $A$ and $B$ be defined as:
            \par
            \begin{subequations}
                \begin{minipage}[b]{0.49\textwidth}
                    \centering
                    \begin{equation}
                        A=\{\,a,\,b,\,c\,\}
                    \end{equation}
                \end{minipage}
                \hfill
                \begin{minipage}[b]{0.49\textwidth}
                    \centering
                    \begin{equation}
                        B=\{\,b,\,c,\,d\,\}
                    \end{equation}
                \end{minipage}
            \end{subequations}
            \par\vspace{2.5ex}
            Then we can compute directly the set difference between the two:
            \par
            \begin{subequations}
                \begin{minipage}[b]{0.49\textwidth}
                    \centering
                    \begin{equation}
                        A\setminus{B}=\{\,a\,\}
                    \end{equation}
                \end{minipage}
                \hfill
                \begin{minipage}[b]{0.49\textwidth}
                    \centering
                    \begin{equation}
                        B\setminus{A}=\{\,d\,\}
                    \end{equation}
                \end{minipage}
            \end{subequations}
            \par\vspace{2.5ex}
            Thus the symmetric difference between sets is not a
            \textit{commutative} set operation.
        \end{example}
        \begin{figure}[H]
            \centering
            \captionsetup{type=figure}
            %--------------------------------Dependencies----------------------------------%
%   tikz                                                                       %
%-------------------------------Main Document----------------------------------%
\begin{tikzpicture}[line width=0.2mm]

    % Coordinates for the centers of the circles.
    \coordinate (C1) at (-1.3, 0);
    \coordinate (C2) at ( 1.3, 0);

    % Coordinates for the labels.
    \coordinate (A) at (-1.3, 1.2);
    \coordinate (B) at ( 1.3, 1.2);
    \coordinate (U) at ( 0.0, 2.5);

    % Rectangle indicating the universe set.
    \draw (-4, -3) rectangle (4, 3);

    % Draw the two circles.
    \draw[fill=cyan]              (C1) circle (2);
    \draw[fill=white, draw=black] (C2) circle (2);

    % Add an outline to the left circle.
    \draw (C1) circle (2);

    % Labels.
    \node at (A) {$A$};
    \node at (B) {$B$};
    \node at (U) {$A\setminus{B}$};
\end{tikzpicture}
            \caption{Venn Diagram for Set Difference}
            \label{fig:Set_Diff_Venn_Diagram}
        \end{figure} 
        \begin{theorem}
            \label{thm:Set_Difference_of_Set_with_Self}%
            If $A$ is a set, then $A\setminus{A}=\emptyset$.
        \end{theorem}
        \begin{proof}
            For suppose not. But since $A\setminus{A}$ is a set,
            $\emptyset\subseteq{A}\setminus{A}
            $(Thm.~\ref{thm:Emptyset_Is_Subset}),and thus if
            $A\setminus{A}\ne\emptyset$ then $A\setminus{A}\nsubseteq\emptyset$
            (Def.~\ref{def:Equal_Sets}). But then there is an
            $x\in{A}\setminus{A}$ such that $x\notin\emptyset$. But if
            $x\in{A}\setminus{A}$, then $x\in{A}$ and $x\notin{A}$
            (Def.~\ref{def:Set_Difference}), a contradiction. Thus,
            $A\setminus{A}=\emptyset$.
        \end{proof}
        \begin{theorem}
            \label{thm:Set_Difference_is_Subset}%
            If $A$ and $B$ are sets, then $A\setminus{B}\subseteq{A}$.
        \end{theorem}
        \begin{proof}
            For suppose not. Then there is an $x\in{A}\setminus{B}$ such that
            $x\notin{A}$. But $x\in{A}\setminus{B}$ if and only if $x\in{A}$ and
            $x\notin{B}$ (Def.~\ref{def:Set_Difference}), and thus $x\in{A}$, a
            contradiction.
        \end{proof}
        \begin{theorem}
            \label{thm:Set_Difference_of_Set_and_Empty}%
            If $A$ is a set, then $A\setminus\emptyset=A$.
        \end{theorem}
        \begin{proof}
            For since $A$ and $\emptyset$ are sets,
            $A\setminus\emptyset\subseteq{A}$
            (Thm.~\ref{thm:Set_Difference_is_Subset}). Suppose they are not
            equal. Then there is an $x\in{A}$ such that
            $x\notin{A}\setminus\emptyset$ (Def.~\ref{def:Equal_Sets}). But if
            $x\in{A}$ and $x\notin{A}\setminus\emptyset$, then $x\in\emptyset$
            (Def.~\ref{def:Set_Difference}) a contradiction since for all $x$ it
            is true that $x\notin\emptyset$ (Def.~\ref{def:Empty_Set}). Thus,
            they are equal.
        \end{proof}
        \begin{example}
            More than being a non-commutative operation, set difference is not
            \textit{associative} either. For let $A$ be any non-empty set and
            let $A=B=C$. Then:
            \begin{equation}
                A\setminus(A\setminus{A})=A\setminus\emptyset=A
            \end{equation}
            Flipping this around, we have:
            \begin{equation}
                (A\setminus{A})\setminus{A}=\emptyset\setminus{A}=\emptyset
            \end{equation}
            But since $A$ is a non-empty set, $A\ne\emptyset$, thus showing
            that set difference is not associative.
        \end{example}
        \begin{ldefinition}{Symmetric Difference}{Symmetric_Difference}
            The symmetric difference of $A$ and $B$, denoted $A\ominus{B}$, is
            the set:
            \begin{equation}
                A\ominus{B}
                =(A\cup{B})\setminus(A\cap{B})
            \end{equation}
        \end{ldefinition}
        While set difference appears similar to subtraction, the two have their
        differences. For any two real numbers $a$ and $b$, it is always true
        that $b=a-(a-b)$. For sets this is not true. For let $A$ be the empty
        set, and let $B$ be non-empty. Then
        $A\setminus(A\setminus{B})=\emptyset$, which is not $B$. Set differences
        can not be easily simplified. The notion is not associative, nor is it
        commutative. If there is a larger \textit{universe} set, then set
        difference can be related to intersection.
        \begin{theorem}
            \label{thm:Set_Difference_As_Intersection}%
            If $A$, $B$, and $C$ are sets, and if $A\subseteq{C}$ and
            $B\subseteq{C}$, then:
            \begin{equation}
                B\setminus{A}=B\cap(C\setminus{A})
            \end{equation}
        \end{theorem}
        \begin{proof}
            For if $x\in{B}\setminus{A}$, then
            $x\in{B}$ and $x\notin{A}$. But
            $B\subseteq{C}$, and thus if $x\in{B}$, then
            $x\in{C}$. But if $x\notin{A}$, then
            $x\in{C}\setminus{A}$. Therefore
            $B\setminus{A}\subseteq{B}\cap(C\setminus{A})$.
            Similarly,
            $B\cap(C\setminus{A})\subseteq{B}\setminus{A}$,
            and therefore
            $B\setminus{A}={B}\cap(C\setminus{A})$.
        \end{proof}
        We can draw a Vann diagram for the symmetric difference, see
        Fig.~\ref{fig:Sym_Diff_Venn_Diagram}.
        \begin{figure}[H]
            \centering
            \captionsetup{type=figure}
            %--------------------------------Dependencies----------------------------------%
%   tikz                                                                       %
%-------------------------------Main Document----------------------------------%
\begin{tikzpicture}[line width=0.2mm]

    % Coordinates for the centers of the circles.
    \coordinate (C1) at (-1.3, 0);
    \coordinate (C2) at ( 1.3, 0);

    % Coordinates for the labels.
    \coordinate (A) at (-1.3, 1.2);
    \coordinate (B) at ( 1.3, 1.2);
    \coordinate (S) at ( 0.0, 2.5);

    % Rectangle indicating the universe set.
    \draw (-4, -3) rectangle (4, 3);

    % Fill in the circle with cyan.
    \draw[fill=cyan, draw=none] (C1) circle (2);
    \draw[fill=cyan, draw=none] (C2) circle (2);

    % Fill in the circle with cyan.
    \draw[fill=white, draw=none] (0, -1.51987) arc(-49.46:49.46:2)
                                               arc(130.54:229.46:2);

    % Give outlines to the circles.
    \draw (C1) circle (2);
    \draw (C2) circle (2);

    % Labels.
    \node at (A) {$A$};
    \node at (B) {$B$};
    \node at (S) {$A\ominus{B}$};
\end{tikzpicture}
            \caption{Venn Diagram for Symmetric Difference}
            \label{fig:Sym_Diff_Venn_Diagram}
        \end{figure}
        The concept of set difference can then be used to define the
        concept of complement.
        Thm.~\ref{thm:Set_Difference_As_Intersection} can then be
        translated into the notation of complements as follows:
        \begin{theorem}
            If $A$, $B$, and $\Omega$ are sets,
            $A,B\subseteq\Omega$, and if $A^{C}$ is the
            complement of $A$ with respect to $\Omega$, then:
            \begin{equation}
                B\setminus{A}=B\cap{A}^{C}
            \end{equation}
        \end{theorem}
        \begin{proof}
            By the definition of complement,
            $A^{C}=\Omega\setminus{A}$.
            As $A\subseteq\Omega$ and $B\subseteq\Omega$, by
            Thm.~\ref{thm:Set_Difference_As_Intersection},
            $B\setminus{A}=B\cap(\Omega\setminus{A})$,
            and therefore $B\setminus{A}=B\cap{A}^{C}$.
        \end{proof}
        The main result about complements are known as
        DeMorgan's Laws. The laws relate unions and
        intersections by means of complements. The general
        laws hold for arbitrary unions and arbitrary
        intersections, as will be shown later.
        \begin{ftheorem}{DeMorgan's Laws}{MEASURE_DEMORGAN}
            If $A$, $B$, and $\Omega$ are sets, if
            $A\subseteq\Omega$ and $B\subseteq\Omega$, then:
            \begin{subequations}
                \begin{align}
                    \big(A\cap{B}\big)^{C}
                    &=A^{C}\cup{B}^{C}\\
                    \big(A\cup{B}\big)^{C}
                    &=A^{C}\cap{B}^{C}
                \end{align}
            \end{subequations}
        \end{ftheorem}
        With this, we can prove some results about
        set differences.
        \begin{theorem}
            If $A$ and $B$ are sets, then:
            \begin{equation}
                A=\big(A\cap{B}\big)
                    \cup\big(A\setminus{B}\big)
            \end{equation}
        \end{theorem}
        \begin{proof}
            For let $\Omega=A\cup{B}$. Then
            $A\subseteq\Omega$ and $B\subseteq\Omega$,
            and thus:
            \begin{subequations}
                \begin{align}
                    \big(A\cap{B})\cup\big(A\setminus{B}\big)
                    &=\big(A\cap{B}\big)
                        \cup\big(A\cap{B}^{C}\big)\\
                    &=A\cap(B\cup{B}^{C})\\
                    &=A\cap\Omega
                \end{align}
            \end{subequations}
            But by Thm.~\ref{thm:Intersection_is_Smaller},
            $A\cap\Omega=A$. Therefore, etc.
        \end{proof}
        \begin{theorem}
            If $A$, $B$, and $C$ are sets, then:
            \begin{equation}
                A\cap\big(B\setminus{C}\big)
                =\big(A\cap{B}\big)\cap\big(A\setminus{C}\big)
            \end{equation}
        \end{theorem}
        \begin{proof}
            For:
            \begin{subequations}
                \begin{align}
                    A\cap\big(B\setminus{C}\big)
                    &=A\cap\big(B\cap{C}^{C}\big)\\
                    &=\big(A\cap{A}\big)
                        \cap\big(B\cap{C}^{C}\big)\\
                    &=\big(A\cap{B}\big)
                        \cap\big(A\cap{C}^{C}\big)\\
                    &=\big(A\cap{B}\big)
                        \cap\big(A\setminus{C}\big)
                \end{align}
            \end{subequations}
        \end{proof}
        Intersections do distribute over set differences.
        \begin{theorem}
            If $A$, $B$, and $C$ are sets, then:
            \begin{equation}
                A\cap(B\setminus{C})=
                (A\cap{B})\setminus(A\cap{C})
            \end{equation}
        \end{theorem}
        \begin{proof}
            For:
            \begin{subequations}
                \begin{align}
                    \big(A\cap{B}\big)\setminus
                        \big(A\cap{C}\big)
                    &=\big(A\cap{B}\big)
                        \cap\big(A\cap{C}\big)^{C}\\
                    &=\big(A\cap{B}\big)
                        \cap\big(A^{C}\cup{C}^{C}\big)\\
                    &=\big[\big(A\cap{B}\big)\cap{A}^{C}\big]
                        \cup\big[\big({A}\cap{B}\big)
                        \cap{C}^{C}\big]\\
                    &=\big[\big(A\cap{A}^{C}\big)\cap{B}\big]
                        \cup\big[\big(A\cap{B}\big)
                        \cap{C}^{C}\big]\\
                    &=\emptyset\cup\big[\big(A\cap{B}\big)
                        \cap{C}^{C}\big]\\
                    &=\big(A\cap{B}\big)\cap{C}^{C}\\
                    &=A\cap\big(B\cap{C}^{C}\big)\\
                    &=A\cap\big(B\setminus{C}\big)
                \end{align}
            \end{subequations}
            Therefore, etc.
        \end{proof}
        Unions do not, however. For let $A$ be non-empty
        and let $A=B=C$. Then $A\cup(B\setminus{C})=A$, but
        $(A\cup{B})\setminus(A\cup{C})=\emptyset$.
        \begin{theorem}
            If $A$ and $B$ are sets and $A\subset B$,
            then $B\setminus(B\setminus A)=A$.
        \end{theorem}
        \begin{proof}
            For:
            \begin{align}
                Yo
            \end{align}
            $[x\in B\setminus(B\setminus{A})]%
            \Rightarrow[x\in{B}\land{x}\notin%
            \{x\in{B}:x\notin{A}\}]%
            \Rightarrow[x\in{A}\subset{B}]$.
            $[x\in{A}]\Rightarrow[x\notin{B}\setminus{A}]%
            \Rightarrow[x\in{B}\setminus(B\setminus{A})]$.
        \end{proof}
        The previous theorem shows that $(A^C)^{C}=A$.
        % Untrue garbage.
        % If $A$ and $B$ are sets, and if $C\subseteq{A}\cup{B}$, then
        % either $C\subseteq{A}$ or $C\subseteq{B}$, or both. It is
        % possible that $C\subseteq{A}\cup{B}$ and yet $C$ and $B$ have no
        % elements in common, as long as $C\subseteq{A}$. As an example,
        % take $A$ and $B$ to be disjoint sets. Then $A\subseteq{A}\cup{B}$,
        % yet $A$ and $B$ have no elements in common. If
        % $C\subseteq{A}\cap{B}$, then it must be true that
        % $C\subseteq{A}$ and $C\subseteq{B}$.
        % As with the notions of unions and intersections, set differences and
        % symmetric differences can be visualized using Venn diagrams.
        \begin{theorem}
            \label{thm:MEASURE_THEORY_SET_DIFFERENCE_AS_INTERSECTION}
            If $A$, $B$, and $C$ are sets, and if $A\subseteq{C}$
            and $B\subseteq{C}$, then:
            \begin{equation}
                B\setminus{A}=B\cap(C\setminus{A})
            \end{equation}
        \end{theorem}
        \begin{proof}
            For if $x\in{B}\setminus{A}$, then
            $x\in{B}$ and $x\notin{A}$. But
            $B\subseteq{C}$, and thus if $x\in{B}$, then $x\in{C}$.
            But if $x\notin{A}$, then $x\in{C}\setminus{A}$. Therefore
            $B\setminus{A}\subseteq{B}\cap(C\setminus{A})$.
            Similarly, $B\cap(C\setminus{A})\subseteq{B}\setminus{A}$,
            and therefore $B\setminus{A}={B}\cap(C\setminus{A})$.
        \end{proof}
        While set difference appears similar to subtraction that one finds in
        basic arithmetic, the two have their differences. For any two real
        numbers $a$ and $b$, $b=a-(a-b)$. For sets this is not true. For let $A$
        be the empty set, and let $B$ be non-empty. Then
        $A\setminus(A\setminus{B})=\emptyset$, which is not $B$.
        Also, while it may seems convincing that
        $A\setminus(B\setminus{A})=A\setminus{B}$, this is not true. For
        let $A$ be a non-empty set and let $B=A$. Then
        $A\setminus(B\setminus{A})=A$, but $A\setminus{B}=\emptyset$.
        The concept of set difference can then be used to define the
        concept of complement.
        %------------------------------------------------------------------------------%
\section{Relations}
    \label{Section:ZFC:Elementary_Set_Theory:Relations}%
    \begin{fdefinition}{Relation on a Set}{Relation_on_a_Set}
        A \gls{relation} on a \gls{set} $A$ is a \gls{subset} $R$ of the
        \gls{Cartesian product} $A\times{A}$.
    \end{fdefinition}
    We use a special notation for relations on a set.
    \begin{fnotation}{Relation Notation}{Relation_Notation}
        If $A$ is a set, if $R$ is a relation on $A$, and if $(a,b)\in{R}$, we
        write $aRb$.
        \begin{equation*}
            \forall_{x}\forall_{y}\big(aRb\big)\Leftrightarrow
            \big((a,b)\in{R}\big)
        \end{equation*}
    \end{fnotation}
    For a relation $R$ it is not necessary true that $aRb$ implies $bRa$, nor is
    it necessarily true that $aRa$. These are called symmetric and reflexive
    relations, respectively.
    \begin{lexample}{Examples of Relations}{Examples_of_Relations}
        Let $A=\mathbb{R}$ and consider the relation of equality. That is, let
        $R_{=}\subseteq\mathbb{R}^{2}$ be defined by:
        \begin{equation}
            R_{=}=\{\,(x,y)\in\mathbb{R}^{2}\;|\;x=y\,\}
        \end{equation}
        Then $R_{=}$ is a relation on $\mathbb{R}^{2}$. Rather than writing
        $(x,y)\in{R_{=}}$ or $xR_{=}y$ we commonly write $x=y$. Note that this
        relation is defined entirely by the \textit{diagonal} of the Cartesian
        product $\mathbb{R}\times\mathbb{R}$. Another simple relation is that of
        ordering. Let $R_{<}$ be defined as follows:
        \begin{equation}
            R_{<}=\{\,(x,y)\in\mathbb{R}^{2}\;|\;x<y\,\}
        \end{equation}
        This is also a relation since it is a subset of the Cartesian product,
        but it is a slightly more complicated one. There are many
        \textit{off-diagonal} elements of this relation.
    \end{lexample}
    \begin{theorem}
        If $B$ is a set, if $A\subseteq{B}$, and if $R$ is a relation on $B$,
        then there is a relation $R_{A}$ such that $R_{A}$ is a relation on
        $A$ and $R_{A}\subseteq{R}$.
    \end{theorem}
    \begin{proof}
        For let $P$ be the proposition \textit{True if} $(x,y)\in{A}^{2}$,
        \textit{false otherwise}. By the axiom schema of specification
        (Ax.~\ref{ax:Axiom_Schema_of_Specification}) there is a set:
        \begin{equation}
            R_{A}=\big\{\,(x,y)\in{R}\;|\;P\big((x,y)\big)\,\big\}
        \end{equation}
        But then $(x,y)\in{R}_{A}$ if and only if $(x,y)\in{R}$ and
        $(x,y)\in{A}^{2}$.
    \end{proof}
    This set is called the \textit{restriction} of $R$ to the subset $A$.
    \begin{fdefinition}{Restriction of a Relation}{Restriction_of_a_Relation}
        The restriction of a relation $R$ on a set $B$ to a subset $A$ is the
        set $R_{A}$ defined by:
        \begin{equation*}
            R_{A}=\big\{\,(x,y)\in{R}\;|\;(x,y)\in{A}^{2}\,\big\}
        \end{equation*}
    \end{fdefinition}
    There are many basic properties that relations have, and we prove them now.
    \begin{theorem}
        \label{thm:Cartesian_Product_Is_Relation}%
        If $A$ is a set, then $A\times{A}$ is a relation on $A$.
    \end{theorem}
    \begin{proof}
        For if $A$ is a set, then
        $A\times{A}\subseteq{A}\times{A}$. Therefore, etc.
    \end{proof}
    \begin{theorem}
        \label{thm:Empty_Set_Is_Relation}%
        If $A$ is a set, and then $\emptyset$ is a relation
        on $A$.
    \end{theorem}
    \begin{proof}
        For if $A$ is a set, then
        $\emptyset\subseteq{A}\times{A}$. Therefore, etc.
    \end{proof}
    \begin{theorem}
        Set inclusion $\subseteq$ is a relation. Proper set inclusion
        $\subsetneq$ is a relation. These define partial orderings.
    \end{theorem}
    \begin{fdefinition}{Domain of a Relation}{Domain_of_a_Relation}
        The \gls{domain (relation)} of a \gls{relation} $R$ on a \gls{set} $A$
        is the set:
        \begin{equation*}
            \textrm{dom}(R)=\big\{a\in{A}\;|\;\exists{b}\in{A}
                \textrm{ such that }aRb\big\}
        \end{equation*}
    \end{fdefinition}
    \begin{fdefinition}{Range of a Relation}{Range_of_a_Relation}
        The \gls{range (relation)} of a \gls{relation} $R$ on a \gls{set} $A$ is
        the set:
        \begin{equation*}
            \textrm{ran}(R)=\big\{b\in{A}\;|\;\exists{a}\in{A}
                \textrm{ such that }aRb\big\}
        \end{equation*}
    \end{fdefinition}
    \begin{fdefinition}{Field of a Relation}{Field_of_a_Relation}
        The \gls{field (relation)} of a \gls{relation} $R$ on a set $A$ is the
        set:
        \begin{equation*}
            \textrm{field}(R)=\textrm{dom}(R)\cup\textrm{ran}(R)
        \end{equation*}
        Where $\textrm{dom}(R)$ is the \gls{domain (relation)} of $R$ and
        $\textrm{ran}(R)$ is the \gls{range (relation)} of $R$.
    \end{fdefinition}
    These provide the two most basic examples of relations on a
    set. The empty set is the relation that says no two elements
    are related. Indeed, even single elements are unrelated to
    themselves. The second, the entire Cartesian product
    $A\times{A}$, says that everything is related. These are the
    two extreme cases, but provide useful examples and
    counterexamples in various contexts. More useful is that the
    union and intersection of relations is also a relation. We
    prove this now.
    \begin{theorem}
        \label{thm:Intersection_of_Relations_Is_Relation}%
        If $A$ is a set and if $R_{1}$ and $R_{2}$ are relations
        on $A$, then $R_{1}\cap{R}_{2}$ is a relation on $A$.
    \end{theorem}
    \begin{proof}
        For let $R=R_{1}\cap{R}_{2}$ and suppose $R$ is not a
        relation on $A$. Then there is an $x\in{R}$ such that
        $x\notin{A}\times{A}$. But if $x\in{R}$ then
        $x\in{R}_{1}$ and $x\in{R}_{2}$. But for all
        $x\in{R}_{1}$, $x\in{A}\times{A}$, since $R_{1}$ is a
        relation on $A$, a contradiction as
        $x\notin{A}\times{A}$. Therefore, $R$ is a relation on
        $A$.
    \end{proof}
    \begin{theorem}
        \label{thm:Set_Theory_Union_of_Relations_Is_Relation}
        If $A$ is a set and if $R_{1}$ and $R_{2}$ are relations
        on $A$, then $R_{1}\cup{R}_{2}$ is a relation on $A$.
    \end{theorem}
    \begin{proof}
        For let $R=R_{1}\cup{R}_{2}$ and suppose $R$ is not a
        relation on $A$. Then there is an $x\in{R}$ such that
        $x\notin{A}\times{A}$. But if $x\in{R}$ then
        $x\in{R}_{1}$ or $x\in{R}_{2}$. But for all $x\in{R}_{1}$
        and for all $x\in{R}_{2}$,
        $x\in{A}\times{A}$, since $R_{1}$ and $R_{2}$ are
        relations on $A$, a contradiction. Therefore, etc.
    \end{proof}
    \begin{theorem}
        If $A$ is a set and $R$ is a relation on $A$, then there
        is a relation $\mathcal{U}$ on $A$ such that
        $R\cap\mathcal{U}=R$.
    \end{theorem}
    \begin{proof}
        For let $\mathcal{U}={A}\times{A}$. Then by
        Thm.~\ref{thm:Cartesian_Product_Is_Relation}, $A\times{A}$ is
        a relation on $A$. But since $R$ is a relation,
        $R\subseteq{A}\times{A}$. But then
        $R\cap\mathcal{U}=R$. Therefore, etc.
    \end{proof}
    \begin{theorem}
        If $A$ is a set and $R$ is a relation on $A$, then there
        is a relation $\mathcal{U}$ on $A$ such that
        $R\cup\mathcal{U}=R$
    \end{theorem}
    \begin{proof}
        For let $\mathcal{U}=\emptyset$. Then by
        Thm.~\ref{thm:Empty_Set_Is_Relation},
        $\mathcal{U}$ is a relation. But if $R$ is a set, then
        $R\cup\emptyset=R$. Thus, $R\cup\mathcal{U}=R$.
        Therefore, etc.
    \end{proof}
    Since a general relation is simply a subset of $A\times{A}$,
    there's not much structure on them, and thus there's not a lot
    that can be said about them. We can add more constraints to
    certain relations to get the more familiar properties
    we're used to.
    \begin{fdefinition}{Reflexive Relations}{Reflexive_Relations}
        A reflexive relation on a set $A$ is a
        relation $R$ on $A$ such that for all $a\in{A}$
        it is true that $aRa$.
    \end{fdefinition}
    A reflexive relation on $A$ is simply any subset of
    $A\times{A}$ that contains the entire \textit{diagonal}. That,
    all of the pairs $(a,a)$. A reflexive relation can contain more
    than this, however. The only strict requirement is that
    $aRa$ for all $a\in{A}$.
    \begin{theorem}
        If $A$ is a set, and if $R_{1}$ and $R_{2}$ are reflexive
        relations on $A$, then $R_{1}\cap{R}_{2}$ is a reflexive
        relation on $A$.
    \end{theorem}
    \begin{proof}
        For let $R=R_{1}\cap{R}_{2}$. Then by
        Thm.~\ref{thm:Intersection_of_Relations_Is_Relation}, $R$ is a relation.
        Suppose $R$ is not reflexive.
        Then there is an $a\in{A}$ such that $(a,a)\notin{R}$. But
        if $a\in{A}$, then $(a,a)\in{R}_{1}$, since $R_{1}$ is
        reflexive. Similarly, $(a,a)\in{R}_{2}$ since $R_{2}$ is
        reflexive. But if $(a,a)\in{R}_{1}$ and $(a,a)\in{R}_{2}$,
        then $(a,a)\in{R}$ since $R=R_{1}\cap{R}_{2}$, a
        contradiction. Therefore, $R$ is reflexive.
    \end{proof}
    \begin{theorem}
        If $A$ is a set, if $R_{1}$ is a reflexive relation on
        $A$, and if $R_{2}$ is a relation on $A$, then
        $R_{1}\cup{R}_{2}$ is a reflexive relation on $A$.
    \end{theorem}
    \begin{proof}
        For let $R=R_{1}\cup{R}_{2}$. Since $R_{1}$ and $R_{2}$ are
        relations, by
        Thm.~\ref{thm:Set_Theory_Union_of_Relations_Is_Relation},
        $R$ is a relation. Suppose it is not reflexive.
        Then there is an $a\in{A}$ such that
        $(a,a)\notin{R}$. But if $a\in{A}$ then $(a,a)\in{R}_{1}$
        since $R_{1}$ is reflexive. But if $(a,a)\in{R}_{1}$ then
        $(a,a)\in{R}_{1}\cup{R}_{2}$, a contradiction.
        Therefore, etc.
    \end{proof}
    Given an arbitrary relation $R$ on a set $A$, it may not be
    true that $R$ is reflexive. It may often be useful to add in
    only the necessary points of $A$ that will make $R$
    reflexive. This is called the reflexive closure of $R$.
    \begin{fdefinition}{Reflexive Closure of a Relation}
                       {Reflexive_Closure_of_Relation}
        The reflexive closure of a relation $R$ on a set $A$
        is the set:
        \begin{equation}
            S=R\cup\{(a,a):a\in{A}\}
        \end{equation}
    \end{fdefinition}
    \begin{theorem}
        If $A$ is a set, $R$ is a relation on $A$, and if $S$ is the
        reflexive closure of $R$, then $S$ is a reflexive relation on $A$.
    \end{theorem}
    \begin{theorem}
        \label{thm:Set_Theory_Refl_Clos_Is_Smallest_Refl_With_R}
        If $A$ is a set, if $R$ is a relation on $A$, if
        $S$ is the reflexive closure of $R$, and if $T$ is a
        reflexive relation on $A$ such that $R\subseteq{T}$, then
        $S\subseteq{T}$.
    \end{theorem}
    \begin{proof}
        For if $x\in{S}$, then either $x\in{R}$ or there is an
        $a\in{A}$ such that $x=(a,a)$. But if $x\in{R}$, then
        $x\in{T}$ since $R\subseteq{T}$. If $x\notin{R}$ then
        there is an $a\in{A}$ such that $x=(a,a)$. But $T$ is
        reflexive, and therefore $(a,a)\in{T}$. But then
        $x\in{T}$. Therefore, $S\subseteq{T}$.
    \end{proof}
    Thm.~\ref{thm:Set_Theory_Refl_Clos_Is_Smallest_Refl_With_R}
    says that the reflexive closure of a relation $R$ is, in a sense,
    the \textit{smallest} relation that is reflexive and contains
    $R$ as a subset.
    \begin{theorem}
        If $A$ is a set, $R_{1}$ and $R_{2}$ are relations on $A$,
        and if $S_{1}$ and $S_{2}$ are the reflexive closures of
        $R_{1}$ and $R_{2}$, respectively, then the reflexive closure
        of $R_{1}\cap{R}_{2}$ is:
        \begin{equation}
            S=S_{1}\cap{S}_{2}
        \end{equation}
    \end{theorem}
    \begin{proof}
        By the definition of reflexive closure, we have:
        \begin{align}
            S_{1}&=R_{1}\cup\{(a,a):a\in{A}\}
            \tag{Def.~\ref{def:Reflexive_Closure_of_Relation}}\\
            S_{1}&=R_{2}\cup\{(a,a):a\in{A}\}
            \tag{Def.~\ref{def:Reflexive_Closure_of_Relation}}\\
            \nonumber
            S_{1}\cap{S}_{2}&=
            (R_{1}\cup\{(a,a):a\in{A}\})
            \cap(R_{2}\cup\{(a,a):a\in{A}\})\\
            &=(R_{1}\cap{R}_{2})
            \cup\{(a,a):a\in{A}\}
            \tag{Distributive Law}
        \end{align}
        But by the definition of the transitive closure of
        $R_{1}\cap{R}_{2}$:
        \begin{equation}
            S=(R_{1}\cap{R}_{2})\cup\{(a,a):a\in{A}\}
            \tag{Def.~\ref{def:Reflexive_Closure_of_Relation}}
        \end{equation}
        Therefore, etc.
    \end{proof}
    \begin{fdefinition}{Symmetric Relation}{Symmetric_Relation}
        A symmetric relation on a set $A$ is a
        relation $R$ on $A$ such that for all $a,b\in{A}$
        such that $aRb$, it is true that $bRa$.
    \end{fdefinition}
    \begin{theorem}
        If $A$ is a set, if $S_{1}$ and $S_{2}$ are symmetric relations
        on $A$, then $S_{1}\cap{S}_{2}$ is a symmetric relation on $A$.
    \end{theorem}
    \begin{proof}
        For since $S_{1}$ and $S_{2}$ are relations, $S_{1}\cap{S}_{2}$ is a
        relation (Thm.~\ref{thm:Intersection_of_Relations_Is_Relation}). Suppose
        it is not symmetric. Then there is an $(x,y)\in{S}_{1}\cap{S}_{2}$ such
        that $(y,x)\notin{S}_{1}\cap{S}_{2}$. But if
        $(x,y)\in{S}_{1}\cap{S}_{2}$, then $(x,y)\in{S}_{1}$ and
        $(x,y)\in{S}_{2}$ (Def.~\ref{def:Intersection_of_Two_Sets}). But $S_{1}$
        is symmetric and if $(x,y)\in{S}_{1}$, then $(y,x)\in{S}_{1}$
        (Def.~\ref{def:Symmetric_Relation}). Similarly $(y,x)\in{S}_{2}$, and
        therefore $(y,x)\in{S}_{1}\cap{S}_{2}$
        (Def.~\ref{def:Intersection_of_Two_Sets})), a contradiction. Therefore,
        $S_{1}\cap{S}_{2}$ is symmetric.
    \end{proof}
    \begin{fdefinition}{Transitive Relation}{Transitive_Relation}
        A transitive relation on a set $A$ is a relation $R$ on $A$
        such that for all $a,b,c\in{A}$ such that $aRb$ and $bRc$,
        is it true that $aRc$.
    \end{fdefinition}
    \begin{theorem}
        \label{thm:Entire_Cartesian_is_Transitive}%
        If $A$ is a set, then $A\times{A}$ is a transitive relation on $A$.
    \end{theorem}
    \begin{proof}
        For suppose not. Then there exists $a,b,c\in{A}$ such that
        $(a,b)\in{A}\times{A}$ and $(b,c)\in{A}\times{A}$, yet
        $(a,c)\in{A}\times{A}$. But if $a\in{A}$ and $c\in{A}$, then
        $(a,c)\in{A}\times{A}$ (Def.~\ref{def:Cartesian_Product_of_Two_Sets}), a
        contradiction. Therefore $A\times{A}$ is a transitive relation on $A$.
    \end{proof}
    Using the Cartesian product definition of a relation, we can visualize the
    requirement imposed on transitive relations in the diagram below
    (Fig.~\ref{fig:Transitive_Relation_Diagram}).
    \begin{figure}[H]
        \centering
        \captionsetup{type=figure}
        \begin{tikzpicture}
    \foreach\x in {0,1,2,3,4,5,6,7,8,9}{
        \foreach\y in {0,1,2,3,4,5,6,7,8,9}{
            \draw[fill=black] (\x,\y) circle (0.1);
        }
        \node at (\x, -1) {$\x$};
        \node at (-1, \x) {$\x$};
    }

    \draw[fill=red,draw=red,opacity=0.4] ( 1.7, -0.3) rectangle (2.3, 9.3);
    \draw[fill=red,draw=red,opacity=0.4] (-0.3,  5.7) rectangle (9.3, 6.3);

    \draw[draw=blue,fill=cyan,opacity=0.5] (2, 4) circle (0.2);
    \draw[draw=blue,fill=cyan,opacity=0.5] (4, 6) circle (0.2);
    \draw[draw=blue,fill=cyan,opacity=0.5] (2, 6) circle (0.2);
\end{tikzpicture}
        \caption{Diagram for a Transitive Relation}
        \label{fig:Transitive_Relation_Diagram}
    \end{figure}
    Given a point $(a,b)$ that is in the relation and another point $(b,c)$, for
    the relation to be transitive requires $(a,c)$ to be contained in it. That
    is, if we take the first coordinate from the first element and the second
    coordinate from the second element and then combine them to form a new
    ordered pair, this element must also be in the relation.
    \begin{theorem}
        If $A$ is a set, if $T_{1}$ and $T_{2}$ are transitive relations on $A$,
        and if $R=T_{1}\cap{T}_{2}$, then $R$ is a transitive relation.
    \end{theorem}
    \begin{proof}
        For since $T_{1}$ and $T_{2}$ are relations, $T_{1}\cap{T}_{2}$ is a
        relation (Thm.~\ref{thm:Intersection_of_Relations_Is_Relation}). Suppose
        it is not transitive. Then there are $(x,y),(y,z)\in{R}$ such that
        $(x,z)\notin{R}$ (Def.~\ref{def:Transitive_Relation}). But if
        $(x,y),(y,z)\in{R}$, then by the definition of intersection,
        $(x,y),(y,z)\in{T}_{1}$ and $(x,y),(y,z)\in{T}_{1}$
        (Def.~\ref{def:Intersection_of_Two_Sets}). But $T_{1}$ is transitive, and
        thus if $xT_{1}y$ and $yT_{1}z$, then $xT_{1}z$. But similarly $T_{2}$
        is transitive, and therefore $xT_{2}z$. But then $(x,z)\in{T}_{1}$ and
        $(x,z)\in{T}_{2}$, and thus $(x,z)\in{T}_{1}\cap{T}_{2}$, a
        contradiction. Therefore, $R$ is transitive.
    \end{proof}
    \begin{example}
        The requirement that both relations $T_{1}$ and $T_{2}$ are transitive
        cannot be weakened. For consider the relations $S$ and $T$ on
        $\mathbb{Z}_{3}$ defined by:
        \par\hfill\par
        \begin{subequations}
            \begin{minipage}[b]{0.49\textwidth}
                \centering
                \begin{equation}
                    S=\big\{\,(0,1),\,(1,2)\,\}
                \end{equation}
            \end{minipage}
            \hfill
            \begin{minipage}[b]{0.49\textwidth}
                \centering
                \begin{equation}
                    T=\big\{\,(0,1),\,(1,2),\,(0,1)\,\}
                \end{equation}
            \end{minipage}
        \end{subequations}
        \par\vspace{2.5ex}
        Then $T$ is transitive and $S$ is not. Moreover $S\subseteq{T}$, and
        hence $S\cap{T}=S$ (Thm.~\ref{thm:Intersection_of_Subset}), and
        therefore the intersection is not transitive. This example is
        demonstrated in
        Fig.~\subref{fig:Trans_Intersect_Non_Trans_May_Not_Be_Trans}. The
        opposite is possible, and to construct an example we need only find a
        transitve relation $T$ and a non-transitive relation $S$ such that
        $T\subseteq{S}$. Define:
        \par\hfill\par
        \begin{subequations}
            \begin{minipage}[b]{0.49\textwidth}
                \centering
                \begin{equation}
                    T=\big\{\,(0,0)\,\}
                \end{equation}
            \end{minipage}
            \hfill
            \begin{minipage}[b]{0.49\textwidth}
                \centering
                \begin{equation}
                    S=\big\{\,(0,0),\,(0,1),\,(1,2)\,\}
                \end{equation}
            \end{minipage}
        \end{subequations}
        \par\vspace{2.5ex}
        Then $T$ is transitive, $S$ is not, and $S\cap{T}=T$
        (See \subref{fig:Trans_Int_Trans_May_Not_Be_Trans}).
    \end{example}
    \begin{figure}[H]
        \centering
        \captionsetup{type=figure}
        \begin{subfigure}[b]{0.49\textwidth}
            \centering
            \begin{tikzpicture}
    \foreach\x in {0,1,2}{
        \foreach\y in {0,1,2}{
            \draw[fill=black] (\x,\y) circle (0.4mm);
        }
    }
    \node at (-1, 2) {$a$};
    \node at (-1, 1) {$b$};
    \node at (-1, 0) {$c$};
    \node at (2, 3) {$a$};
    \node at (1, 3) {$b$};
    \node at (0, 3) {$c$};
    \draw[draw=blue,fill=none] (0,1) circle (0.3);
    \draw[draw=blue,fill=none] (1,0) circle (0.3);
    \draw[draw=blue,fill=none] (2,2) circle (0.3);
    \draw[draw=red,fill=none] (-0.4243, 0.5757) rectangle (0.4243, 1.4243);
    \draw[draw=red,fill=none] ( 0.5757,-0.4243) rectangle (1.4243, 0.4243);
\end{tikzpicture}
            \subcaption{The intersection is not transitive}
            \label{fig:Trans_Intersect_Non_Trans_May_Not_Be_Trans}
        \end{subfigure}
        \hfill
        \begin{subfigure}[b]{0.49\textwidth}
            \centering
            \begin{tikzpicture}
    \foreach\x in {0,1,2}{
        \foreach\y in {0,1,2}{
            \draw[fill=black] (\x,\y) circle (0.4mm);
        }
    }
    \node at (-1, 2) {$a$};
    \node at (-1, 1) {$b$};
    \node at (-1, 0) {$c$};
    \node at (2, 3) {$a$};
    \node at (1, 3) {$b$};
    \node at (0, 3) {$c$};
    \draw[draw=blue,fill=none] (0,2) circle (0.3);
    \draw[draw=red,fill=none] (-0.4243, 1.5557) rectangle (0.4243, 2.4243);
    \draw[draw=red,fill=none] (-0.4243, 0.5757) rectangle (0.4243, 1.4243);
    \draw[draw=red,fill=none] ( 0.5757,-0.4243) rectangle (1.4243, 0.4243);
\end{tikzpicture}
            \subcaption{The intersection is transitive}
            \label{fig:Trans_Int_Trans_May_Not_Be_Trans}
        \end{subfigure}
        \label{fig:Intersection_of_Transitive_and_Non_Transitive_Relations}
        \caption{The Intersection of Transitive and Non-Transitive Relations}
    \end{figure}
    We can strengthened our claim that the intersection of two transitive
    relations is again transitive and show that any arbitrary intersection will
    again be transitive.
    \begin{theorem}
        \label{thm:Intersection_of_Transitive_is_Transitive}%
        If $A$ is a set, if $\mathcal{P}(A\times{A})$ denotes the power set of
        $A\times{A}$, if $\mathcal{O}\subseteq\mathcal{P}(A\times{A})$ is such
        that for all $\mathcal{U}\in\mathcal{O}$ it is true that $\mathcal{U}$
        is a transitive relation on $A$, if $\mathcal{T}$ is defined by:
        \begin{equation}
            \mathcal{T}=\bigcap_{\mathcal{U}\in\mathcal{O}}\mathcal{U}
        \end{equation}
        Then $\mathcal{T}$ is a transitive relation on $A$.
    \end{theorem}
    \begin{proof}
        For suppose not. Then there exists $a,b,c\in{A}$ such that
        $(a,b)\in\mathcal{T}$ and $(b,c)\in\mathcal{T}$, yet
        $(a,c)\notin\mathcal{T}$. But if $(a,b)\in\mathcal{T}$, then for all
        $\mathcal{U}\in\mathcal{O}$ it is true that $(a,b)\in\mathcal{U}$
        (Def.~\ref{def:Intersection_Over_a_Collection}). Similarly, for all
        $\mathcal{U}\in\mathcal{O}$ it is true that $(b,c)\in\mathcal{U}$.
        But by hypothesis, for all $\mathcal{U}\in\mathcal{O}$ it is true that
        $\mathcal{U}$ is a transitive relation and thus if $(a,b)\in\mathcal{U}$
        and $(b,c)\in\mathcal{U}$, then it is true that $(a,c)\in\mathcal{U}$
        (Def.~\ref{def:Transitive_Relation}). But then for all
        $\mathcal{U}\in\mathcal{O}$ it is true that $(a,c)\in\mathcal{U}$, and
        therefore $(a,c)\in\mathcal{T}$
        (Def.~\ref{def:Intersection_Over_a_Collection}), a contradiction.
        Therefore, $\mathcal{T}$ is a transitive relation on $A$.
    \end{proof}
    This allows us to define the transitive closure of any relation $R$ on a set
    $A$. It is, in a sense, the \textit{smallest} transitive relation that
    contains $R$.
    \begin{theorem}
        If $A$ is a set and if $R$ is a relation on $A$, then there exists a
        transitive relation $\mathcal{T}$ on $A$ such that
        $R\subseteq\mathcal{T}$ and such that for transitive relations $T$ on
        $A$ such that $R\subseteq{T}$ it is true that $\mathcal{T}\subseteq{T}$.
    \end{theorem}
    \begin{proof}
        For let $P$ be the proposition \textit{True if} $S$
        \textit{is a transitive relation on} $A$ \textit{such that}
        $R\subseteq{S}$, \textit{false otherwise}. Then by the axiom schema of
        specification (Ax.~\ref{ax:Axiom_Schema_of_Specification}) there exists
        a set:
        \begin{equation}
            \mathcal{O}=\big\{\,S\in\mathcal{P}(A\times{A})\;|\;P(S)\,\big\}
        \end{equation}
        But then for all $S\in\mathcal{O}$, $P(S)$ is true and therefore
        $R\subseteq{S}$ and $S$ is transitive. Moreover, $\mathcal{O}$ is
        non-empty since by Thm.~\ref{thm:Entire_Cartesian_is_Transitive},
        $A\times{A}$ is a transitive relation. Define $\mathcal{T}$ by:
        \begin{equation}
            \mathcal{T}=\bigcap_{S\in\mathcal{O}}S
        \end{equation}
        Then by Thm.~\ref{thm:Intersection_of_Transitive_is_Transitive},
        $\mathcal{T}$ is a transitive relation. Moreover, suppose $S$ is a
        transitive relation such that $R\subseteq{S}$. But if $S$ is a relation
        on $A$, then $S\subseteq{A}\times{A}$
        (Def.~\ref{def:Relation_on_a_Set}) and therefore
        $S\in\mathcal{P}(A\times{A})$ (Def.~\ref{def:Power_Set}). But if $S$ is
        a transitive relation and if $R\subseteq{S}$, then $P(S)$ is true, and
        therefore $S\in\mathcal{P}$. Thus, $\mathcal{T}\subseteq{S}$.
    \end{proof}
    \begin{fdefinition}{Transitive Closure}{Transitive_Closure}
        The transitive closure of a relation $R$ on a set
        $A$ is the the set $R^{t}\subseteq{A}\times{A}$ defined by:
        \begin{equation}
            R^{t}
        \end{equation}
    \end{fdefinition}
    \begin{fdefinition}{Asymmetric Relation}{Assymetric_Relation}
        An asymmetric relation on a set $A$ is a relation $R$
        on $A$ such that for all $a,b\in{A}$ such that $aRb$
        it is true that $(b,a)\notin{R}$.
    \end{fdefinition}
    \begin{fdefinition}{Total Relation}{Total_Relation}
        A total relation on a set $A$ is a relation $R$ on $A$ such
        that for all $a,b\in{A}$ it is true that either
        $aRb$ or $bRa$, or both.
    \end{fdefinition}
    The notion of equality can be defined as a relation
    with the following properties:
    \begin{enumerate}
        \item Equality is Reflexive: $a=a$ for all $a\in{A}$.
        \item Equality is Symmetric: $a=b$ if and only if $b=a$.
        \item Equality is Transitive: If $a=b$ and $b=c$, then $a=c$.
        \item The relation is uniquely defined by the set
              $\{(a,a)\in A\times A:a\in A\}$.
    \end{enumerate}
    That is, equality can be seen as the \textit{diagonal} in the
    Cartesian product $A\times{A}$.
    \begin{fdefinition}{Antisymmetric Relation}
        An antisymmetric relation on a set $A$ is a relation $R$ on $A$
        such that for all $a,b\in{A}$ such that $aRb$ and $bRa$, it
        is true that $a=b$.
    \end{fdefinition}
    \begin{fdefinition}{Equivalence Relation}{Equivalence_Relation}
        An equivalence relation on a set $A$ is a relation $R$ on $A$ such that
        $R$ is reflexive, symmetric, and transitive.
    \end{fdefinition}
    Equivalence relations attempt to model equality. They are fundamental in
    mathematics as they allow us to define \textit{equivalence classes}, which
    are used to define quotients. There are many examples such as quotient
    topologies, quotient groups, quotient rings, and quotient modules, all of
    which will be discussed later.
    \begin{fdefinition}{Equivalence Class}{Equivalence_Class}
        The equivalence class of an element $x$ in a set $A$ by an
        equivalence relation $R$ is the set:
        \begin{equation*}
            [x]=\{\,y\in{A}\;|\;xRy\,\}
        \end{equation*}
    \end{fdefinition}
    It's important to note that the term class here is different from the notion
    of a collection of sets. And equivalence class of an element $x$ in a set
    $A$ under an equivalene relation $R$ will indeed be a set in $ZFC$.
    \begin{theorem}
        \label{thm:Equivalence_Classes_Disjoint_or_Equal}%
        If $A$ is a set, if $R$ is an equivalence relation on $A$, and if
        $x,y\in{A}$, then either $[x]=[y]$ or $[x]\cap[y]=\emptyset$.
    \end{theorem}
    \begin{proof}
        For suppose not and suppose $[x]\ne[y]$ and that
        $[x]\cap[y]\ne\emptyset$. That is, suppose:
        \begin{equation*}
            \neg([x]=[y])\land\neg([x]\cap[y]=\emptyset)
        \end{equation*}
        If $[x]\cap[y]$ is non-empty then there is a
        $z\in{A}$ such that $z\in[x]$ and $z\in[y]$
        (Def.~\ref{def:Non_Empty_Set}). But if $z\in[x]$, then $xRz$
        (Def.~\ref{def:Equivalence_Class}). But also
        $z\in[y]$ and therefore $yRz$. But $R$ is an equivalence relation and
        is therefore symmetric (Def.~\ref{def:Equivalence_Relation}) and thus
        if $yRz$ then $zRy$ (Def.~\ref{def:Symmetric_Relation}). But an
        equivalence relation is also transitive, and thus if $xRz$ and $zRy$,
        then $xRy$ (Def.~\ref{def:Transitive_Relation}). But if $[x]\ne[y]$ then
        either $[x]\nsubseteq[y]$ or $[y]\nsubseteq[x]$. Suppose
        $[x]\nsubseteq[y]$ and let $a\in[x]$ be such that $a\notin[y]$. But
        if $a\in[x]$ then $xRa$ (Def.~\ref{def:Equivalence_Class}). But since
        equivalence relations are symmetric, if $xRa$, then $aRx$. But it was
        proven that $xRy$ and since equivalence relations are transitive, if
        $aRx$ and $xRy$, then $aRy$. But again if $aRy$, then $yRa$ and
        therefore $a\in[y]$, a contradiction. Therefore $[x]\subseteq[y]$.
        Similarly, $[y]\subseteq[x]$ and therefore $[x]=[y]$, a contradiction.
        By the law of the excluded middle, the negation is true:
        \begin{equation*}
            \neg\big(\neg([x]=[y])\land\neg([x]\cap[y]=\emptyset)\big)
            =([x]=[y])\lor([x]\cap[y]=\emptyset)
        \end{equation*}
        Thus, either $[x]=[y]$ or $[x]\cap[y]=\emptyset$.
    \end{proof}
    \begin{fdefinition}{Quotient Set}{Quotient_Set}
        The quotient set of a set $A$ by an equivalence relation $R$ on $A$ is
        the set:
        \begin{equation*}
            A/R=\{\,[x]\in\mathcal{P}(A)\;|\;x\in{A}\,\}
        \end{equation*}
        Where $[x]$ is the equivalence class of $x$ under $R$.
    \end{fdefinition}
    \begin{example}
        The definition of the quotient set comes naturally when one considers
        functions between sets. Suppose $A$ and $B$ are sets, and suppose
        $f:A\rightarrow{B}$ is a function. In general, it may not be true that
        $f(a_{1})=f(a_{2})$ implies that $a_{1}=a_{2}$, and so we wish to find a
        subset of $A$ with this property. The quotient set does this. Let
        $R$ be the relation:
        \begin{equation}
            R=\{\,(a,b)\in{A}^{2}\;|\;f(a)=f(b)\,\}
        \end{equation}
        If we form the quotient set $A/R$ and consider the projective mapping
        $\pi:A\rightarrow{A}/R$ that sends $a\in{A}$ to its equivalence class.
        That is, $\pi(a)=[a]$. We then seek a function
        $\tilde{f}:A/R\rightarrow{B}$ such that $\tilde{f}\circ{\pi}=f$.
        That is, we wish to make the diagram below \textit{commute}.
        \begin{figure}[H]
            \centering
            \begin{tikzpicture}[%
    >=latex,
    every path/.style={->},
    line width=0.2mm,
    line cap=round
]
    \node (A) at (0.0,  0.0) {$A$};
    \node (AR) at (0.0, -2.0) {$A/R$};
    \node (B) at  (2.0,  0.0) {$B$};
    \path (A) edge node [above]        {$f$}         (B);
    \path (AR) edge node [below right] {$\tilde{f}$} (B);
    \path (A) edge node [left]         {$\pi$}       (AR);
\end{tikzpicture}
            \label{fig:Comm_Diagram_Quotient_Set}
            \caption{Commutative Diagram for the Quotient Set}
        \end{figure}
        So we need to map $[x]$ to $f(x)$. That is, $\tilde{f}([x])=f(x)$. For
        this problem to be well posed requires that the equivalence class that
        make up the elements of $A/R$ come from equivalence relations. That is,
        that the relation $R$ is transitive, symmetric, and reflexive.
    \end{example}
    \begin{theorem}
        If $A$ is a set and if $R$ is an equivalence relation on $A$, then
        $A/R$ is a partition of $A$.
    \end{theorem}
    \begin{proof}
        For by Thm.~\ref{thm:Equivalence_Classes_Disjoint_or_Equal}, if
        $\mathcal{U},\mathcal{V}\in{A}/R$, then either
        $\mathcal{U}=\mathcal{V}$ or $\mathcal{U}\cap\mathcal{V}=\emptyset$.
        But also, for all $x\in{A}$, there is a $\mathcal{U}\in{A}/R$ such that
        $x\in\mathcal{U}$ since $x\in[x]$ and $[x]\in{R}/A$. Therefore,
        $A/R$ is a partition of $A$.
    \end{proof}
    \chapter{Function Theory}
        \section{Definitions}
    \begin{fdefinition}{Fields}{Fields}
        A field is a commutative ring $(\mathbb{F},+,\cdot\,)$ such that, for
        all $a\in\mathbb{F}$ such that $a$ is not the unital element of
        $(\mathbb{F},+)$, it is true that $a$ is an invertible element of
        $(\mathbb{F},\cdot\,)$.
    \end{fdefinition}
    \begin{fdefinition}{Subfield}{Subfield}
        A subfield of a field $(F,+,\cdot)$ is a set $K\subset F$, such that
        $(K,+,\cdot)$ is a field.
    \end{fdefinition}
    Given an element $a\in\mathbb{F}$, if $b$ is such that
    $a+b=0$ then we write $b=\minus{a}$. Subtraction of two elements
    $a$ and $c$, denoted $a-c$, is defined as $a+(\minus{c})$. The
    structure $(\mathbb{F},+)$ forms an Abelian group. From this we have
    that the identity is unique, as are additive inverses.
    It is common in the definition of a field to require that
    $0\ne{1}$. This is because if $0=1$ then we have $\mathbb{F}=\{0\}$.
    This comes from the following.
    \begin{ltheorem}{Multiplication by Zero}{Multiplication_by_Zero}
        If $(\mathbb{F},\,+,\,\cdot\,)$ is a field, and if $a\in\mathbb{F}$,
        then $a\cdot{0}=0$.
    \end{ltheorem}
    \begin{proof}
        For we have:
        \begin{equation}
            0=a\cdot(0)-a\cdot(0)=a\cdot(0-0)=a\cdot{0}
        \end{equation}
        This simply combines the distributive law with the additive
        property of zero, completing the proof.
    \end{proof}
    \begin{theorem}
        If $(\mathbb{F},\,+,\,\cdot\,)$ is a field, and if $0=1$, then
        $\mathbb{F}=\{0\}$.
    \end{theorem}
    \begin{proof}
        For suppose not, and let $a\in\mathbb{F}$ be such that $a\ne{0}$.
        But then, by Thm.~\ref{thm:Multiplication_by_Zero}:
        \begin{equation}
            a=a\cdot{1}=a\cdot{0}=0
        \end{equation}
        And thus $a=0$, a contradiction. Therefore,
        $\mathbb{F}$ is trivial.
    \end{proof}
    It is thus common to either call such a field a trivial field, or
    to require that $0\ne{1}$.
    \begin{lexample}{Examples of Fields}{Examples_of_Fields}
        There are several fields that should be familiar to the reader.
        If we let $\mathbb{R}$ denote the real numbers and $+$ and $\cdot$
        be the usual notations of addition and multiplication, then
        $(\mathbb{R},\,+,\,\cdot\,)$ is a field. Similarly, letting
        $\mathbb{Q}$ denote the rational numbers and $\mathbb{C}$ denote
        the complex numbers, $(\mathbb{Q},\,+,\,\cdot\,)$ is a field, as
        is $(\mathbb{C},\,+,\,\cdot\,)$. There are finite fields as well.
        Let $\mathbb{F}_{2}=\{0,\,1\}$ and define multiplication and
        addition as follows:
        \par\hfill\par
        \begin{table}[H]
            \centering
            \captionsetup{type=table}
            \parbox{.45\linewidth}{%
                \centering
                \begin{tabular}{c|cc}
                    $+$&0&1\\
                    \hline
                    0&0&1\\
                    1&1&0
                \end{tabular}
            }
            \parbox{.45\linewidth}{%
                \centering
                \begin{tabular}{c|cc}
                    $\cdot$&0&1\\
                    \hline
                    0&0&0\\
                    1&0&1
                \end{tabular}
            }
            \caption{The Arithmetic of $\mathbb{F}_{2}$}
        \end{table}
        $(\mathbb{F}_{2},\,+,\,\cdot)$ forms a field. Finally, if
        $p\in\mathbb{N}$ is prime, and if $+$ and $\cdot$ are addition
        and multiplication mod $p$, respectively, then
        $(\mathbb{Z}_{p},\,+,\,\cdot\,)$ is a field.
    \end{lexample}
    \begin{fdefinition}{Vector Space}{Vector_Space}
        A vector space over a field $(\mathbb{F},\,+,\,\cdot\,)$ is a
        set $V$ and a function
        $\boldsymbol{\cdot}:\mathbb{F}\times{V}\rightarrow{V}$ and
        a binary operation $\boldsymbol{+}$ on $V$, usuall called
        scalar multiplication and vector addition, respectively, 
        such that for all $\mathbf{x},\mathbf{y},\mathbf{z}\in{V}$,
        and all $a,b\in\mathbf{F}$, the following is true:
        \begin{enumerate}
            \item $\mathbf{x}\boldsymbol{+}%
                   (\mathbf{y}\boldsymbol{+}\mathbf{z})=%
                   (\mathbf{x}\boldsymbol{+}\mathbf{y})%
                   \boldsymbol{+}\mathbf{z}$
                  \hfill[Associative of Vector Addition]
            \item $\mathbf{x}\boldsymbol{+}\mathbf{y}=%
                   \mathbf{y}\boldsymbol{+}\mathbf{x}$
                  \hfill[Commutativity of Vector Addition]
            \item There is a $\mathbf{0}\in{V}$ such that
                  $\mathbf{0}\boldsymbol{+}\mathbf{x}=\mathbf{x}$
                  \hfill[Existence of Zero Vector]
            \item For all $\mathbf{x}$ there is a $\mathbf{y}$ such that
                  $\mathbf{x}\boldsymbol{+}\mathbf{y}=\mathbf{0}$
                  \hfill[Additive Inverses]
            \item $(a\cdot{b})\boldsymbol{\cdot}\mathbf{x}=%
                    a\boldsymbol{\cdot}(b\boldsymbol{\cdot}\mathbf{x})$
                  \hfill[Compatibility of Multiplication]
            \item $(a+b)\boldsymbol{\cdot}\mathbf{x}=%
                   (a\boldsymbol{\cdot}\mathbf{x})\boldsymbol{+}%
                   (b\boldsymbol{\cdot}\mathbf{x})$
                  \hfill[Distributive Law for Field Addition]
            \item $a\boldsymbol{\cdot}(\mathbf{x}\boldsymbol{+}\mathbf{y})=%
                   (a\boldsymbol{\cdot}\mathbf{x})\boldsymbol{+}%
                   (a\boldsymbol{\cdot}\mathbf{y})$
                  \hfill[Distributive Law for Vector Addition]
        \end{enumerate}
    \end{fdefinition}
    It is quite common not to distinguish between scalar multiplication
    $\boldsymbol{\cdot}$ and field multiplication $\cdot$, which may cause
    confusion. It is also common to drop the use of a symbol altogether and
    simply representation multiplication by concatenation of the the
    two variables, for example $a\mathbf{x}$ or $ab$, which represents
    scalar multiplication and field multiplication, respectively.
    \begin{example}
        If we let $\mathbb{F}=\mathbb{R}$ and let
        $V=\mathbb{R}^{n}$, where addition, multiplication, scalar
        multiplication, and vector addition are defined in their usual
        manner, then this forms a vector space. Similarly, the space
        $C([a,b])$ of continuous functions forms a vector space over
        $\mathbb{R}$, as does $L^{2}(\mathbb{R})$, the space of
        square integrable functions.
    \end{example}
    \begin{fdefinition}{Bilinear Operations}{Bilinear_Operations}
        A bilinear operation on a vector space
        $(V,\,\boldsymbol{+},\,\boldsymbol{\cdot}\,)$ over a field
        $(\mathbf{F},\,+,\,\cdot\,)$ is a function
        $[\,]:V\times{V}\rightarrow{V}$ such that, for all
        $\mathbf{x},\mathbf{y},\mathbf{z}\in{V}$, and for all
        $a,b\in\mathbf{F}$, the following is true:
        \begin{enumerate}
            \item $[\mathbf{x}\boldsymbol{+}\mathbf{y}, \mathbf{z}]=%
                   [\mathbf{x},\mathbf{z}]\boldsymbol{+}%
                   [\mathbf{y},\mathbf{z}]$
                  \hfill[Right Distributive Law]
            \item $[\mathbf{x},\mathbf{y}\boldsymbol{+}\mathbf{z}]=%
                   [\mathbf{x},\mathbf{y}]\boldsymbol{+}%
                   [\mathbf{x},\mathbf{z}]$
                  \hfill[Left Distributive Law]
            \item $[a\boldsymbol{\cdot}\mathbf{x},%
                    b\boldsymbol{\cdot}\mathbf{y}]=%
                   (a\cdot{b})\boldsymbol{\cdot}[\mathbf{x},\mathbf{y}]$
                  \hfill[Compatibility with Scalars]
        \end{enumerate}
    \end{fdefinition}
    \begin{lexample}{Examples of Bilinear Operations}
                    {Examples_of_Bilinear_Operation}
        The quintessential example of a bilinear operation is the
        cross product that one encounters in a multivariable calculus
        course. That is, for any three vectors
        $\mathbf{x},\mathbf{y},\mathbf{z}$, we have:
        \begin{equation}
            \mathbf{x}\times(\mathbf{y}+\mathbf{z})=
            \mathbf{x}\times\mathbf{y}+\mathbf{x}\times\mathbf{z}
        \end{equation}
        Similarly for right sided multiplication. The compatibility of
        the cross product with scalar multiplication is also true:
        \begin{equation}
            (a\mathbf{x})\times(b\mathbf{y})=ab(\mathbf{x}\times\mathbf{y})
        \end{equation}
        This serves somewhat as a motivating example for bilinear
        operations. If we think of the field of invertible matrices,
        then multiplication forms a bilinear operation as well, with
        scalar multiplication being the usual entry wise operation that
        is done on matrices. Lastly, if $\langle\,\rangle$ is an inner
        product on $\mathbb{R}$ or $\mathbb{C}$, then this is a bilinear
        operation, the vector space being the underlying field itself.
    \end{lexample}
    \begin{fdefinition}{Algebra over a Field}{Algebra_over_a_Field}
        An algebra of a field $(\mathbf{F},\,+,\,\cdot\,)$ is a
        vector space $(\mathbf{V},\,\boldsymbol{+},\,\boldsymbol{\cdot}\,)$
        and a bilinear operation $[\,]:V\times{V}\rightarrow{V}$.
    \end{fdefinition}
    \begin{fdefinition}{Associative Algebra over a Field}
                       {Associative_Algebra_over_a_Field}
        An associative algebra over a field $(\mathbb{F},\,+,\,\cdot\,)$
        is an algebra $(V,[\,])$ over $\mathbb{F}$ such that, for all
        $r\in\mathbb{F}$ and for all $\mathbf{x},\mathbf{y}\in{V}$,
        the following is true:
        \begin{equation}
            r[\mathbf{x},\,\mathbf{y}]=[r\mathbf{x},\,\mathbf{y}]
                                      =[\mathbf{x},\,r\mathbf{y}]
        \end{equation}
    \end{fdefinition}
    \begin{fdefinition}{Derivation on an Algebra}{Derivation_on_an_Algebra}
        A derivation on an algebra $(V,\,[\,])$ is a function
        $D:V\rightarrow{V}$ such that for all $\mathbf{x},\mathbf{y}\in{V}$,
        the following (Liebniz's Rule) is true:
        \begin{equation}
            D([\mathbf{x},\mathbf{y}])
            =[\mathbf{x},D(\mathbf{y})]+[D(\mathbf{x}),\mathbf{y}]
        \end{equation}
    \end{fdefinition}
    \begin{theorem}
        In a field, $0$ and $1$ are unique.
    \end{theorem}
    \begin{proof}
        For suppose not, and let $0'$ and $1'$ be other identities.
        Then $1'=1'\cdot 1 = 1$ and $0'=0'+0=0$.
    \end{proof}
    \begin{theorem}
        For any field $\langle{F},+,\cdot\rangle$ and $a\in{F}$, $a\cdot{0}=0$.
    \end{theorem}
    \begin{proof}
        For:
        \begin{equation}
            0=a\cdot{0}+(\minus{a}\cdot{0})
             =a\cdot(0+0)+(\minus{a}\cdot{0})
             =a\cdot{0}+a\cdot{0}+(\minus{a}\cdot{0})
             =a\cdot 0
        \end{equation}
        Thus, $a\cdot{0}=0$.
    \end{proof}
    If $1=0$, then $a=a\cdot{1}=a\cdot{0}=0$, and thus every element is
    zero. A very boring field.
    \begin{theorem}
        In a field $\langle F, +,\cdot \rangle$, if $0\ne 1$, then $0$ has no
        inverse.
    \end{theorem}
    \begin{proof}
        For let $a$ be such an inverse. Then $a\cdot{0}=1$. But for any element
        of $F$, $a\cdot{0}=0$. But $0\ne{1}$, a contradiction.
    \end{proof}
    \begin{theorem}
        If $a+b=0$, then $b=(\minus{1})\cdot{a}$ where $(\minus{1})$ is the
        solution to $1+(\minus{1})=0$.
    \end{theorem}
    \begin{proof}
        $a+(\minus{1})a=a(1+(\minus{1}))=a\cdot{0}=0$. From uniqueness,
        $b=(\minus{1})a$. We may thus write additive inverses as $\minus{a}$.
    \end{proof}
    \begin{definition}
        Given two fields $(F,+,\cdot)$ and $(F',+',\times)$, a bijection
        function $f:F\rightarrow{F}'$ is said to be a field isomorphism if and
        only if for allelements $a,b\in{F}$, $f(a+b)=f(a)+'f(b)$, and
        $f(a\cdot{b})=f(a)\times{f}(b)$
    \end{definition}
    \begin{definition}
        $(F,+,\cdot)$ and $(F',+',\times)$, are said to be isomorphic if and
        only if they have an isomorphism.
    \end{definition}
    \begin{theorem}
        Given an ismorphism between two fields $(F,+,\cdot)$ and
        $(F', +',\times)$, $f(1)=1'$ and $f(0)=0'$.
    \end{theorem}
    \begin{proof}
        For let $x\in{F}$. Then $f(x)=f(x\cdot 1)=f(x)\times{f}(1)$, and
        $f(x)=f(x+0)=f(x)+'f(0)$. Therefore, etc.
    \end{proof}
    \begin{theorem}
        In a field $(F,+,\cdot)$, $(a+b)^{2}=a^{2}+2ab+b^{2}$
        ($2$ being the solution to $1+1$).
    \end{theorem}
    \begin{proof}
        For:
        \begin{align}
            (a+b)^{2}&=(a+b)(a+b)\\
                     &=a(a+b)+b(a+b)\\
                     &=a^{2}+ab+ba+b^{2}\\
                     &=a^{2}+ab(1+1)+b^{2}\\
                     &=a^{2}+2ab+b^{2}
        \end{align}
    \end{proof}
        \section{Binary Operations}
    Binary operations are the standard tools that one uses when they develope
    arithmetic. As such, the most familiar examples of binary operations are
    those of addition, multiplication, and subtraction with real numbers.
    On the other hand, division is \textit{not} a binary operation on the real
    numbers since division by zero is undefined. To make this explicit we need
    to give a rigorous definition to binary operations. We can do this with the
    language of functions\index{Function} and by using the Cartesian product
    \index{Cartesian Product} of a set $A$ with itself.
    \begin{fdefinition}{Binary Operation}{Binary_Operation}
        A \gls{binary operation} on a set $A$ is a function
        $*:A\times{A}\rightarrow{A}$.
    \end{fdefinition}
    \begin{example}
        Let $\mathbb{R}$ be the set of real numbers and $+$ denote the addition
        of two real numbers. Then $+$ is a binary operation on $\mathbb{R}$.
        Similarly, if $\cdot$ denotes the multiplication of two real numbers,
        than it two is a binary operation on $\mathbb{R}$. For division, $\div$,
        we are lacking the requirement that \textit{for all}
        $(a,b)\in\mathbb{R}^{2}$ there is a unique $c\in\mathbb{R}$ such that
        $a\div{b}=c$, since if $b=0$ our expression is undefined. That is, this
        is not a function from $\mathbb{R}^{2}$ to $\mathbb{R}$. If we consider
        all of the non-zero elements, then division is a binary operation. That
        is, division is a binary operation on $\mathbb{R}\setminus\{0\}$.
    \end{example}
    \begin{lexample}{Binary Operation on the Set of Functions}
                    {Binary_Operation_on_the_Set_of_Functions}
        If $A$ is a set, and if $\mathcal{F}(A,A)$ denotes the set of all
        functions $f:A\rightarrow{A}$, and if $\circ$ denotes function
        composition, then $\circ$ is a binary operation on $\mathcal{F}(A,A)$.
        That is, for any two functions $f,g\in\mathcal{F}(A,A)$, the composition
        $g\circ{f}:A\rightarrow{A}$ is again an element of $\mathcal{F}(A,A)$
    \end{lexample}
    Just like functions, there are three important conditions that a binary
    operation must satisfy. Given any ordered pair $(a,b)\in{A}^{2}$, it must
    be true that $*(a,b)$ is defined. This comes from the definition of a
    function on a set (Def.~\ref{def:Function}). Next, the image of $(a,b)$ must
    be unique. That is, if $*(a,b)=c$ and $*(a,b)=d$, then $c=d$. Note that this
    is not the same as requiring that $*(a,b)=*(b,a)$, and in general this is
    not true. Such binary operations are called
    \textit{commutative}\index{Commutative Operation}. Lastly, for any
    $(a,b)\in{A}^{2}$, $*(a,b)$ must be an element of $A$. That is,
    $*(a,b)\in{A}$. All of these requirements come from the definition of a
    function, so in a sense it is redundant to repeat these. In practice one
    defines a binary operation by a formula $\varphi$, and it then becomes
    necessary to show that this formula satisfies these properties before we can
    rightly call it a binary operation.
    \begin{example}
        Let $A=\mathbb{Z}_{2}$ and consider all of the binary operations on
        $\mathbb{Z}_{2}$. We can count these by constructing tables:
        \begin{table}[H]
            \centering
            \begin{tabular}{c|c}
                $(x,y)$&$*(x,y)$\\
                \hline
                $(0,0)$&0\\
                $(0,1)$&0\\
                $(1,0)$&1\\
                $(1,1)$&0
            \end{tabular}
            \label{tab:Binary_Operation_on_Z_2}
            \caption{Simple Binary Operation on $\mathbb{Z}_{2}$}
        \end{table}
        This is one such binary operation, there are 15 others. To see this,
        recall that the number of functions from a set $A$ to a set $B$, where
        both $A$ and $B$ are finite sets with $m$ and $n$ elements,
        respectively, is $n^{m}$. Since $\mathbb{Z}_{2}$ has 2 elements, and
        since a binary operation is a function
        $*\mathbb{Z}_{2}\times\mathbb{Z}_{2}\rightarrow\mathbb{Z}_{2}$, the
        total number of binary operations is $2^{(2^{2})}=2^{4}=16$. In general,
        if $A$ has $n$ elements, and if $B$ is the set of all binary operations
        on $A$, then:
        \begin{equation}
            \textrm{Card}(B)=n^{(n^2)}
        \end{equation}
    \end{example}
    \begin{example}
        Let's consider some formula that take in numbers and return numbers, and
        see if they can define operations on various sets. Suppose we have:
        \begin{equation}
            a*b=\{\,r\in\mathbb{R}\;|\;r^{2}=|ab|\,\}
        \end{equation}
        Where $|ab|$ denotes the absolute value of $a$ times $b$. If we take the
        positive square root we can write this as $a*b=\sqrt{|ab|}$. If we
        consider this formula on the rational numbers $\mathbb{Q}$, does it
        define a function? One might recall that $\sqrt{2}$ is not a rational
        number. That is, it is \textit{irrational}. Thus $1*2$ is not a rational
        number, and so $*$ is not a binary operation on $\mathbb{Q}$. It is a
        binary operation on $\mathbb{R}$, however. Suppose we change the formula
        to state:
        \begin{equation}
            a*b=\{\,r\in\mathbb{R}\:|\;r^{2}-ab=0\,\}
        \end{equation}
        and where we consider this formula to take inputs from $\mathbb{R}$.
        This is not a binary operation since it is poorly defined. That is,
        should $1*1=1$, or should $1*1=\minus{1}$? The formula is ambgious and
        thus $*$ is not a binary operation.
    \end{example}
    \begin{example}
        If we consider subtraction on the integers $\mathbb{Z}$, this is a
        binary operation. The operation is well defined and returns an integer
        for all integer inputs. If instead we consider subtraction on
        $\mathbb{N}$, this is \textit{not} a binary operation since it may take
        in non-negative integers and return a negative integer. For example,
        $1-2=\minus{1}$, and $\minus{1}\notin\mathbb{N}$. A simple fix for this
        is considering again the absolute value function. If we define
        $n*m=|n-m|$, then $*$ is indeed a binary operation on $\mathbb{N}$.
    \end{example}
    \begin{fnotation}{Binary Operation}{Binary_Operation}
        If $A$ is a set and if $*:A\times{A}\rightarrow{A}$ is a binary
        operation on $A$, for any ordered pair $(a,b)\in{A}^{2}$, the image
        of $*(a,b)$ is denoted $a*b$.
    \end{fnotation}
    It is occasionally useful to think of binary operations purely as functions,
    and so we will use function notation at these times. For the most part we
    will stick with notation defined in Not.~\ref{not:Binary_Operation}. There
    are several types of binary operations worth studying, and several key
    properties that these operations can have. One of the most fundamental is
    the existence of a \textit{unital} element, also known as an identity.
    \begin{fdefinition}{Left Unital Element}{Left_Unital_Element}
        A left unital element in a \gls{set} $A$ under a \gls{binary operation}
        $*$ on $A$ is an element $e_{L}\in{A}$ such that, for all $a\in{A}$ it
        is true that $e_{L}*a=a$.
    \end{fdefinition}
    \begin{example}
        From the definition of a left unital element
        (Def.~\ref{def:Left_Unital_Element}) it would seem natural to define a
        right unital element. The importance is to note that left and right
        unital elements need not be equal. Indeed, if $A$ is a set and $*$ is
        a binary operation, given a left identity $e_{L}$ and a right identity
        $e_{R}$ it will be true that $e_{R}=e_{L}$ and thus all left and right
        unital elements will be the same
        (see Thm.~\ref{thm:left_and_right_identity_implies_identity}). Thus to
        find counterexamples to the claim that the existence of a left unital
        element implies the existence of a right unital element we need to think
        of strange operations. Let $A=\mathbb{R}$ and let $*$ be defined by
        $a*b=b$ for all $a,b\in\mathbb{R}$. Then every element of $\mathbb{R}$
        is a left unital element. Moreover, none of the element of $\mathbb{R}$
        are right unital elements.
    \end{example}
    \begin{fdefinition}{Right Unital Element}{Right_Unital_Element}
        A right unital element of a \gls{set} $A$ under a \gls{binary operation}
        $*$ is an element $e_{R}$ such that for all $a\in{A}$ it is true that
        $a*e_{R}=a$.
    \end{fdefinition}
    \begin{example}
        Consider $\mathbb{R}$ with the operation $*$ defined by $a*b=a+b+1$.
        This operation has a right unital element, $\minus{1}$. For:
        \begin{equation}
            a*(\minus{1})=a+(\minus{1})+1=a+0=a
        \end{equation}
        And this is true for all $a\in\mathbb{R}$, so $\minus{1}$ is a right
        unital element. It turns out this is also a left unital element, and
        hence a unital element, and this can be proven if addition is known to
        be a \textit{commutative} operation.
    \end{example}
    \begin{theorem}
        \label{thm:left_and_right_identity_implies_identity}%
        If $A$ is a set, if $*$ is a binary operation on $A$, if $e_{L}$ is a
        left unital element of $A$, and if $e_{R}$ is a right unital element of
        $a$, then $e_{L}=e_{R}$.
    \end{theorem}
    \begin{proof}
        For:
        \begin{equation}
            e_{L}=e_{L}*e_{R}=e_{R}
        \end{equation}
        And thus $e_{L}=e_{R}$.
    \end{proof}
    \begin{example}
        Consider a non-empty set $A$ and the set of all functions from $A$ to
        itself, $\mathcal{F}(A,A)$. Let $\circ$ denote the binary operation of
        function composition. Then $\mathcal{F}(A,A)$ has a right identity under
        $\circ$, and a left identity. For
        the identity function $\textrm{id}_{A}$ acts as a right identity:
        \begin{equation}
            (f\circ\textrm{id}_{A})(x)
            =f\big(\textrm{id}_{A}(x)\big)
            =f(x)
        \end{equation}
        And thus $\textrm{id}_{A}$ is a right identity. By
        Thm.~\ref{thm:left_and_right_identity_implies_identity}, any left
        identity must also be a right identity, and so the likely candidate to
        check is $\textrm{id}_{A}$. And indeed we have:
        \begin{equation}
            (\textrm{id}_{A}\circ{f})(x)
            =\textrm{id}_{A}\big(f(x)\big)
            =f(x)
        \end{equation}
        And thus $\textrm{id}_{A}$ is a left identity as well.
    \end{example}
    \begin{fdefinition}{Unital Element}{Unital_Element}
        A \gls{unital element} of a \gls{set} $A$ under a \gls{binary operation}
        $*$ is an element $e\in{A}$ that is both a right unital element and a
        left unital element.
    \end{fdefinition}
    \begin{example}
        Let $\mathbb{R}$ be the set of real numbers and let $+$ be the usual
        notion of addition. Then 0 is a unital element of $\mathbb{R}$ with
        respect to this operation. That is, for any real number $x$ we have
        $x+0=0+x=x$. For multiplication the unital element is 1. This is because
        $1\cdot{x}=x\cdot{1}=x$. Subtraction has a right unital element, and
        again it is 0 since $x-0=x$, but no left identity. To see this, suppose
        $e-x=x$ for all $x$. Applying some algebra we have that $e=2x$, meaning
        there is no constant $e\in\mathbb{R}$ such that for all $x$, $e-x=x$.
        Since subtraction has no left unital element, it has no unital element
        either.
    \end{example}
    \begin{theorem}
        \label{thm:Unital_Elements_are_Unique}%
        If $A$ is a set, if $*$ is a binary operation on $A$, and if $e$ and
        $e'$ are unital elements of $A$, then $e=e'$
    \end{theorem}
    \begin{proof}
        For:
        \begin{equation}
            e=e*e'=e'
        \end{equation}
        And thus by transitivity, $e=e'$.
    \end{proof}
    The next thing to discuss is that of inverses. There are five types, but in
    practice only one of these is discussed.
    \begin{fdefinition}{Weak Right Inverse}{Weak_Right_Inverse}
        A weak right inverse of an element $a$ in a \gls{set} $A$ under a
        \gls{binary operation} $*$ on $A$ is an element $b\in{A}$ such that
        $a*b$ is a right unital element.
    \end{fdefinition}
    This definition will not recieve much use until we discuss
    groups\index{Group}. A group is a set with a binary operation $*$ that has
    a unital element, inverse elements, and is associatied (to be defined soon).
    As it turns out these conditions are stronger than necessary and it suffices
    to check that there are weak right inverses and a right unital element. The
    next thing to define is right inverses.
    \begin{fdefinition}{Right Inverse}{Right_Inverse}
        A right inverse of an element $a$ in a \gls{set} $A$ under a
        \gls{binary operation} is an element $b\in{A}$ such that $a*b$ is a
        \gls{unital element}.
    \end{fdefinition}
    Here, we've simply strengthened the requirement that $a*b$ not only be a
    right unital element, but also a left unital element as well. A right
    inverse is therefore necessarily a weak right inverse.
    \begin{fdefinition}{Weakly Left Invertible}{Weakly_Left_Invertible}
        A weakly left invertible element of a \gls{set} $A$ under a
        \gls{binary operation} $*$ is an element $a\in{A}$ such that there
        exists a $b\in{A}$ such that $b*a$ is a left unital element.
    \end{fdefinition}
    \begin{fdefinition}{Left Invertible Element}{Left_Inverse}
        A left invertible element of a \gls{set} $A$ under a
        \gls{binary operation} is an element $a\in{A}$ such that there exists a
        $b\in{A}$ such that $b*a$ is a \gls{unital element}.
    \end{fdefinition}
    \begin{fdefinition}{Invertible Element}{Invertible_Element}
        An invertible element of a a \gls{set} $A$ under a
        \gls{binary operation} is an element $a\in{A}$ that is both
        left invertible and right invertible.
    \end{fdefinition}
    \begin{fdefinition}{Commutative Operation}{Commutative_Operation}
        A \gls{commutative operation} on a \gls{set} $A$ is a
        \gls{binary operation} $*$ such that for all $(a,b)\in{A}^{2}$ it is
        true that $a*b=b*a$.
    \end{fdefinition}
    \begin{fdefinition}{Associative Operation}{Associative_Operation}
        A \gls{associative operation} on a \gls{set} $A$ is a
        \gls{binary operation} $*$ such that, for all $a,b,c\in{A}$ it is true
        that $a*(b*c)=(a*b)*c$.
    \end{fdefinition}
    \begin{example}
        Consider a finite set $A$ and consider the set of all functions from
        $\mathbb{Z}_{n}$ to $A$. That is, $\mathcal{F}_{n}(\mathbb{Z}_{n},A)$.
        Define $A[x]$ by:
        \begin{equation}
            \mathcal{F}=\bigcup_{n\in\mathbb{N}}\mathcal{F}_{n}
        \end{equation}
        That is, the set of all finite sequences in $A$. We can form an
        associative operation on this set by defining the concatenation
        operation. Given $f,g\in\mathcal{F}$, suppose
        $f\in\mathcal{F}(\mathbb{Z}_{m},A)$ and
        $g\in\mathcal{F}(\mathbb{Z}_{n},A)$. We define
        $f*g\in\mathcal{F}(\mathbb{Z}_{m+n},A)$ as follows:
        \begin{equation}
            (f*g)(k)=
            \begin{cases}
                f(k),&k\in\mathbb{Z}_{m}\\
                g(k-m),&k\in\mathbb{Z}_{m+n}\textrm{ and }k\geq{m}
            \end{cases}
        \end{equation}
        That is, given two sequences $f_{0},f_{1},\dots,f_{m-1}$ and
        $g_{0},g_{1},\dots,g_{n-1}$, we concatenate them to form the sequence
        $f_{0},\dots,f_{m-1},g_{0},\dots,g_{n-1}$. This operation is associative
        since if $f,g,h\in\mathcal{F}$, then:
        \begin{subequations}
            \begin{align}
                f*(g*h)&=(f_{0},f_{1},\dots,f_{m-1})
                    *(g_{0},g_{1},\dots,g_{m-1},h_{0},h_{1},\dots,h_{r-1})\\
                &=f_{0},f_{1},\dots,f_{m-1},
                    g_{0},g_{1},\dots,g_{m-1},h_{0},h_{1},\dots,h_{r-1}\\
                &=(f_{0},f_{1},\dots,f_{m-1},
                    g_{0},g_{1},\dots,g_{m-1})*(h_{0},h_{1},\dots,h_{r-1})\\
                &=(f*g)*h
            \end{align}
        \end{subequations}
        If $A$ has more than one point than $*$ is not commutative. For let
        $f,g:\mathbb{Z}_{1}\rightarrow{A}$ be defined by $f(0)=a$ and $g(0)=b$,
        respectively. Then $f*g=a,b$ but $g*f=b,a$, and thus $f*g\ne{g}*f$.
        There is, however, an identity. Consider a $\mathbb{Z}_{0}$, which is
        the empty set. Any function from $\mathbb{Z}_{0}$ to $A$ is therefore
        the \textit{empty sequence}. If we concatenate $f$ with the empty
        sequence we get back $f$, and this then acts as our unital element.
    \end{example}


        \section{Boolean Algebras}
    We now attempt to make set theory more algebraic. We wish to model
    as an object the triple $(\mathcal{P}(X),\cup,\cap)$, where
    $\mathcal{P}(X)$ is the \gls{power set} of some set, and $\cup$ and
    $\cap$ and union and intersection, respectively. These can be seen as
    binary operations on $\mathcal{P}(X)$. We take a few of the properties
    of this structure and state them as the definition for our new object:
    \textit{Boolean Algebras}.
    \begin{fdefinition}{Complement in a Boolean Algebra}
                       {Complement}
        A complement of a \gls{set} $A$ with respect to two
        \glspl{binary operation} $*$ and $\circ$ is an element
        $a^{\minus{1}}\in{A}$ such that:
        \begin{equation*}
            a*a^{\minus{1}}=a^{\minus{1}}*a=e_{\circ}
            \quad\quad
            a\circ{a}^{\minus{1}}=a^{\minus{1}}\circ{a}=e_{*}
        \end{equation*}
        Where $e_{\circ}$ and $e_{*}$ are the \glspl{unital element} of
        $\circ$ and $*$, respectively.
    \end{fdefinition}
    \begin{fdefinition}{Boolean Algebras}{Boolean_Algebra}
        A Boolean algebra is a set $A$ with two
        \glslink{commutative operation}{commutative} \glspl{binary operation}
        $\circ$ and $*$ on $A$ with \glspl{unital element} $e_{*}$ and
        $e_{\circ}$, respectively, such that:
        \begin{itemize}
            \item[1.)]  $\circ$ \glslink{distributive operation}{distributes}
                        over $*$ and $*$ distributes over $\circ$.
            \item[2.)]  For all $a\in{A}$ there is a complement of $a$.
        \end{itemize}
    \end{fdefinition}
    The second property is known as the complement property and it is very
    different from the notion of inverses. An inverse of an element $a$ with
    respective to an operation $\cdot$ is such that $a\cdot{b}$ is a unital
    element with respect to the operation $\cdot$. A complement produces a
    unital element with respect the the \textit{other} operation. That is,
    $a*a^{\minus{1}}$ is a unital element of $\circ$, and not $*$. Similarly,
    $a\circ{a}^{\minus{1}}$ is a unital element of $*$ and not $\circ$.
    \begin{theorem}
        If $(A,\circ,*)$ is a Boolean algebra and if $b\in{X}$ is a
        unital element of $\circ$, then $b=e_{\circ}$.
    \end{theorem}
    \begin{proof}
        For unital elements are unique
        (Thm.~\ref{thm:Unital_Elements_are_Unique}), and therefore
        $b=e_{\circ}$.
    \end{proof}
    \begin{theorem}
        If $(X,\circ,*)$ is a Boolean algebra and if $b\in{X}$ is a
        unital element of $*$, then $b=e_{*}$.
    \end{theorem}
    \begin{proof}
        For unital elements are unique
        (Thm.~\ref{thm:Unital_Elements_are_Unique}), and therefore $b=e_{*}$.
    \end{proof}
    \begin{theorem}
        \label{thm:Bool_Alg_Boundary_of_Circ}%
        If $(X,\circ,*)$ is a Boolean algebra, if $e_{*}$ is the unital
        element of $*$, and if $a\in{X}$, then $a\circ{e}_{*}=e_{*}$.
    \end{theorem}
    \begin{proof}
        For if $a\in{X}$ then there is an $a^{\minus{1}}\in{X}$ such that
        $a\circ{a}^{\minus{1}}=e_{*}$ (Def.~\ref{def:Boolean_Algebra}).
        But then:
        \par\vspace{-2.5ex}
        \begin{minipage}[t]{0.51\textwidth}
            \centering
            \begin{align}
                e_{*}&=a\circ{a}^{\minus{1}}
                \tag{Complement}\\
                &=a\circ(a^{\minus{1}}*e_{*})
                \tag{Identity}\\
                &=(a\circ{a}^{\minus{1}})*(a\circ{e}_{*})
                \tag{Distributivity}
            \end{align}
        \end{minipage}
        \hfill
        \begin{minipage}[t]{0.47\textwidth}
            \centering
            \begin{align}
                &=e_{*}*(a\circ{e}_{*})
                \tag{Complement}\\
                &=a\circ{e}_{*}
                \tag{Identity}
            \end{align}
        \end{minipage}
        \par\vspace{2.5ex}
        And therefore $e_{*}=a\circ{e}_{*}$.
    \end{proof}
    This theorem is equivalent to the notion that a Boolean algebra is a
    bounded lattice\index{Bounded Lattice} and the $e_{*}$ is a boundary. The
    theorem holds for $\circ$ as well.
    \begin{theorem}
        \label{thm:Bool_Alg_Boundary_of_Star}%
        If $(X,\circ,*)$ is a Boolean algebra, if $e_{*}$ is the unital
        element of $*$, and if $a\in{X}$, then $a*{e}_{\circ}=e_{\circ}$.
    \end{theorem}
    \begin{proof}
        For if $a\in{X}$ then there is an $a^{\minus{1}}\in{X}$ such that
        $a*{a}^{\minus{1}}=e_{\circ}$ (Def.~\ref{def:Boolean_Algebra}).
        But then:
        \par\vspace{-2.5ex}
        \begin{minipage}[t]{0.51\textwidth}
            \centering
            \begin{align}
                e_{\circ}&=a*{a}^{\minus{1}}
                \tag{Complement}\\
                &=a*(a^{\minus{1}}\circ{e}_{\circ})
                \tag{Identity}\\
                &=(a*a^{\minus{1}})\circ(a*e_{*})
                \tag{Distributivity}
            \end{align}
        \end{minipage}
        \hfill
        \begin{minipage}[t]{0.47\textwidth}
            \centering
            \begin{align}
                &=e_{\circ}\circ(a*e_{\circ})
                \tag{Complement}\\
                &=a*e_{\circ}
                \tag{Identity}
            \end{align}
        \end{minipage}
        \par\vspace{2.5ex}
        And therefore $e_{\circ}=a*e_{\circ}$.
    \end{proof}
    \begin{theorem}
        If $(A,\circ,*)$ is a Boolean algebra, if $e_{\circ}$ and $e_{*}$
        are the unital elements of $\circ$ and $*$, respectively, then
        $e_{\circ}$ is the complement of $e_{*}$ and $e_{*}$ is the
        complement of $e_{\circ}$.
    \end{theorem}
    \begin{proof}
        From identity:
        \par\vspace{-2.5ex}
        \begin{subequations}
            \begin{minipage}[b]{0.49\textwidth}
                \centering
                \begin{equation}
                    e_{\circ}\circ{e}_{*}=e_{*}\circ{e}_{\circ}=e_{*}
                    \tag{Identity}
                \end{equation}
            \end{minipage}
            \hfill
            \begin{minipage}[b]{0.49\textwidth}
                \centering
                \begin{equation}
                    e_{\circ}*{e}_{*}=e_{*}*e_{\circ}=e_{\circ}
                    \tag{Identity}
                \end{equation}
            \end{minipage}
        \end{subequations}
        \par\vspace{2.5ex}
        Thus, $e_{*}$ is a complement of $e_{\circ}$ and $e_{\circ}$ is a
        complement of $e_{*}$ (Def.~\ref{def:Complement}).
    \end{proof}
    Every element of a Boolean algebra is idempotent with respect to both
    operations.
    \begin{theorem}
        \label{thm:Bool_Alg_Idempotent_of_Star}%
        If $(A,\circ,*)$ is a Boolean algebra and if $a\in{A}$, then $a*a=a$.
    \end{theorem}
    \begin{proof}
        For:
        \par\vspace{-2.5ex}
        \begin{subequations}
            \begin{minipage}[t]{0.49\textwidth}
                \centering
                \begin{align}
                    a&=a*e_{*}
                    \tag{Identity}\\
                    &=a*(a\circ{a}^{\minus{1}})
                    \tag{Complement}\\
                    &=(a*a)\circ(a*a^{\minus{1}})
                    \tag{Distributivity}
                \end{align}
            \end{minipage}
            \hfill
            \begin{minipage}[t]{0.49\textwidth}
                \centering
                \begin{align}
                    &=(a*a)\circ{e}_{\circ}
                    \tag{Complement}\\
                    &=a*a\tag{Identity}
                \end{align}
            \end{minipage}
        \end{subequations}
        \par\vspace{2.5ex}
        And therefore $a=a*a$.
    \end{proof}
    \begin{theorem}
        \label{thm:Bool_Alg_Idempotent_of_Circ}%
        If $(A,\circ,*)$ is a Boolean algebra and if $a\in{A}$, then
        $a\circ{a}=a$.
    \end{theorem}
    \begin{proof}
        For:
        \par\vspace{-2.5ex}
        \begin{subequations}
            \begin{minipage}[t]{0.49\textwidth}
                \centering
                \begin{align}
                    a&=a\circ{e}_{\circ}
                    \tag{Identity}\\
                    &=a\circ(a*a^{\minus{1}})
                    \tag{Complement}\\
                    &=(a\circ{a})*(a\circ{a}^{\minus{1}})
                    \tag{Distributivity}
                \end{align}
            \end{minipage}
            \hfill
            \begin{minipage}[t]{0.49\textwidth}
                \centering
                \begin{align}
                    &=(a\circ{a})*{e}_{*}
                    \tag{Complement}\\
                    &=a\circ{a}\tag{Identity}
                \end{align}
            \end{minipage}
        \end{subequations}
        \par\vspace{2.5ex}
        And therefore $a=a\circ{a}$.
    \end{proof}
    \begin{theorem}
        \label{thm:Bool_Alg_aob_equal_a_acb_equal_a_implies_a_equal_b}%
        If $(A,\circ,*)$ is a Boolean algebra, if $a,b\in{A}$, if $a\circ{b}=a$,
        and if $a*b=a$, then $b=a$.
    \end{theorem}
    \begin{proof}
        For:
        \par
        \begin{minipage}[b]{0.49\textwidth}
            \centering
            \begin{align}
                b&=b*e_{*}\tag{Identity}\\
                 &=b*(a\circ a^{-1})\tag{Complement}\\
                 &=(b*a)\circ(b* a^{-1})\tag{Distributivity}\\
                 &=a\circ (b* a^{-1})\tag{Hypothesis}
            \end{align}
        \end{minipage}
        \hfill
        \begin{minipage}[b]{0.49\textwidth}
            \centering
            \begin{align}
                &=(a\circ b)*(a\circ a^{-1})\tag{Distributivity}\\
                &=(a\circ{b})*e_{*}\tag{Complement}\\
                &=a\circ{b}\tag{Identity}\\
                &=a\tag{Hypothesis}
            \end{align}
        \end{minipage}
        \par\vspace{2.5ex}
        And therefore $a=b$.
    \end{proof}
    \begin{theorem}
        If $(A,\circ,*)$ is a Boolean algebra, if $a\in{A}$ is such that
        $a=a^{\minus{1}}$, then $a=e_{\circ}=e_{*}$.
    \end{theorem}
    \begin{proof}
        For let $a\in{A}$ and let $a=a^{\minus{1}}$. Then by
        Thm.~\ref{thm:Bool_Alg_Idempotent_of_Star}:
        \begin{equation}
            a=a*a=a*a^{-1}=e_{\circ}
        \end{equation}
        Similarly, $a=e_{*}$.
    \end{proof}
    \begin{theorem}
        If $(A,\circ,*)$ is a Boolean algebra, if $a\in{A}$, and if $b,c\in{A}$
        are complements of $a$, then $b=c$.
    \end{theorem}
    \begin{proof}
        For:
        \par
        \begin{minipage}[t]{0.49\textwidth}
            \centering
            \begin{align}
                b&=b*e
                \tag{Identity}\\
                &=b*(a\circ{c})
                \tag{Complement}\\
                &=(b*a)\circ(b*c)
                \tag{Distributivity}\\
                &=e_{\circ}\circ(b*c)
                \tag{Complement}\\
                &=(c*a)\circ(b*c)
                \tag{Complement}
            \end{align}
        \end{minipage}
        \hfill
        \begin{minipage}[t]{0.49\textwidth}
            \centering
            \begin{align}
                &=(c*a)\circ(c*b)
                \tag{Commutativity}\\
                &=c\circ(a*b)
                \tag{Distributivity}\\
                &=c\circ{e}_{\circ}
                \tag{Complement}\\
                &=c
                \tag{Identity}
            \end{align}
        \end{minipage}
        \par\vspace{2.5ex}
        Therefore, $b=c$.
    \end{proof}
    \begin{theorem}
        If $(A,\circ,*)$ is a Boolean algebra and if $a\in{A}$,
        then $(a^{\minus{1}})^{\minus{1}}=a$.
    \end{theorem}
    \begin{proof}
        For:
        \begin{align}
            a&=a*e_{*}
            \tag{Identity}\\
            &=a*\big(a^{\minus{1}}\circ(a^{\minus{1}})^{\minus{1}})
            \tag{Complement}\\
            &=(a\circ{a}^{\minus{1}})*
                \big(a\circ(a^{\minus{1}})^{\minus{1}}\big)
            \tag{Distributivity}\\
            &=e_{*}*\big(a\circ(a^{\minus{1}})^{\minus{1}}\big)
            \tag{Complement}\\
            &=a\circ(a^{\minus{1}})^{\minus{1}}
            \tag{Identity}
        \end{align}
        And similarly $a*(a^{\minus{1}})^{\minus{1}}=a$. But if $a*b=a$ and
        $a\circ{b}=a$, then $a=b$
        (Thm.~\ref{thm:Bool_Alg_aob_equal_a_acb_equal_a_implies_a_equal_b}).
        Therefore, $a=(a^{\minus{1}})^{\minus{1}}$.
    \end{proof}
    \begin{ltheorem}{Absorption Laws}{Absorption_Law}
        If $(A,\circ,*)$ is a Boolean algebra, if $a\in{A}$ and if $b\in{A}$,
        then $a*(a\circ{b})=a$ and $a\circ(a*{b})=a$.
    \end{ltheorem}
    \begin{proof}
        For:
        \par
        \begin{minipage}[t]{0.59\textwidth}
            \centering
            \begin{align}
                a*(a\circ{b})&=(a*e_{*})\circ(a*b)
                \tag{Identity}\\
                &=a*(e_{*}\circ{b})
                \tag{Distributivity}
            \end{align}
        \end{minipage}
        \hfill
        \begin{minipage}[t]{0.39\textwidth}
            \centering
            \begin{align}
                &=a*e_{*}
                \tag{Thm.~\ref{thm:Bool_Alg_Boundary_of_Circ}}\\
                &=a
                \tag{Identity}
            \end{align}
        \end{minipage}
        \par\vspace{2.5ex}
        And therefore $a*(a\circ{b})=a$. Similarly:
        \par
        \begin{minipage}[t]{0.59\textwidth}
            \centering
            \begin{align}
                a\circ(a*b)&=(a\circ{e}_{\circ})*(a\circ{b})
                \tag{Identity}\\
                &=a\circ(e_{\circ}*b)
                \tag{Distributivity}
            \end{align}
        \end{minipage}
        \hfill
        \begin{minipage}[t]{0.39\textwidth}
            \begin{align}
                &=a\circ{e}_{\circ}
                \tag{Thm.~\ref{thm:Bool_Alg_Boundary_of_Star}}\\
                &=a
                \tag{Identity}
            \end{align}
        \end{minipage}
        \par\vspace{2.5ex}
        And therefore $a\circ(a*b)=a$.
    \end{proof}
    We can weaken the hypothesis of
    Thm.~\ref{thm:Bool_Alg_aob_equal_a_acb_equal_a_implies_a_equal_b} to obtain
    a more general result.
    \begin{theorem}
        \label{thm:Bool_Alg_aob_equal_acb_implies_a_equal_b}
        If $(A,\circ,*)$ is a Boolean algebra, if $a,b\in{A}$, and if
        $a*b=a\circ{b}$, then $a=b$.
    \end{theorem}
    \begin{proof}
        For:
        \begin{align}
            a&=a*e_{*}
            \tag{Identity}\\
            &=a*(b\circ{b}^{\minus{1}})
            \tag{Complement}\\
            &=(a*b)\circ(a*b^{\minus{1}})
            \tag{Distributivity}\\
            &=(a\circ{b})\circ(a*b^{\minus{1}})
            \tag{Hypothesis}\\
            &=((a\circ{b})\circ{a})*
                \big((a\circ{b})\circ{b}^{\minus{1}}\big)
            \tag{Distributivity}\\
            &=\big((a\circ{b})\circ{a}\big)*
                \big(a\circ(b\circ{b}^{\minus{1}})\big)
            \tag{Associativity}\\
            &=\big((a\circ{b})\circ{a}\big)*(a\circ{e}_{*})
            \tag{Complement}\\
            &=\big((a\circ{b})\circ{a}\big)*e_{*}
            \tag{Thm.~\ref{thm:Bool_Alg_Boundary_of_Circ}}\\
            &=(a\circ{b})\circ{a}
            \tag{Identity}\\
            &=(a\circ{a})\circ{b}
            \tag{Associativity and Commutativity}\\
            &=a\circ{b}
            \tag{Thm~\ref{thm:Bool_Alg_Idempotent_of_Circ}}
        \end{align}
        Thus $a=a\circ{b}$. But $a\circ{b}=a*{b}$, and so $a=a*{b}$. By
        Thm.~\ref{thm:Bool_Alg_aob_equal_a_acb_equal_a_implies_a_equal_b},
        $a=b$.
    \end{proof}
            \begin{definition} For $a\in S$, an inverse, or normal inverse, of the First Operation is an element $b\in S$ such that $a\circ b=e_{\circ}$. An inverse of the Second Operation is similarly defined. The normal inverses are denoted $a^{*}$ and $a^{\circ}$.
            \end{definition}
            \begin{theorem} If $a\in S$ has a normal inverse for either operation, than it is unique.
            \end{theorem}
            \begin{proof} For suppose not. Let $a\in S$ have a normal inverse for the First Operation. That is, there is an $a^{\circ}\in S$ such that $a\circ a^{\circ}=e_{\circ}$ and let $a'^{\circ}$ be a second normal inverse not equal to the first. But then $a^{\circ}=a^{\circ}\circ e_{\circ}=a^{\circ}\circ (a\circ a'^{\circ})$ and from associativity we have $a^{\circ}=(a^{\circ}\circ a)\circ a'^{\circ}=a'^{\circ}$. Thus, the normal inverse is unique. Similarly if there is an inverse for the Second Operation
            \end{proof}
            \begin{theorem} If $a\in S$ has a normal inverse, say $a'$, for one operation, then $a^{-1}=a'^{-1}$.
            \end{theorem}
            \begin{proof} For let $a\in S$ have a normal inverse $a'$ for the First Operation. That is, $a\circ a' = e_{\circ}$. But $a' \circ a'^{-1}=e_{*}$, and from theorem 1.3 $a\circ e_{*}=e_{*}$. So $a\circ (a' \circ a'^{-1})=e_{*}$. And from theorem 1.4, $a\circ a=a$, so we have $(a\circ a)\circ (a'\circ a'^{-1}=a\circ (a\circ a')\circ a'^{-1}=a\circ a'^{-1}=e_{*}$. But $a\circ a^{-1}=e_{\circ}$. And pseudo-inverses are unique. Thus, $a^{-1}=a'^{-1}$. 
            \end{proof}
            \begin{theorem} The identities have normal inverses for their respective operations.
            \end{theorem}
            \begin{proof} As normal inverses are unique, it suffices to find inverses for both identities. But $e_{\circ}\circ e_{\circ}=e_{\circ}$, so $e_{\circ}$ is its own inverse for the First Operation. Similarly, $e_{*}*e_{*}=e_{*}$.
            \end{proof}
            \begin{theorem} \textbf{(The Not-A-Field Theorem)} Only the identities have normal inverses.
            \end{theorem}
            \begin{proof} For suppose not. Suppose $a\in S,\ a\ne e_{\circ},\ a\ne e_{*}$ and a has an inverse for the First Operation. That is $\exists a^{\circ}\in S|\ a\circ a^{\circ}=e_{\circ}$. But by theorem 1.4, $a\circ a^{\circ}=(a\circ a)\circ a^{\circ}$. By associativity, we have $e_{\circ}=a\circ a^{\circ} = a\circ (a\circ a^{\circ})=a\circ e_{\circ}=a$. Thus, $a=e_{\circ}$. But by hypothesis, $a\ne e_{\circ}$. Thus, there is no inverse for $a$. Similarly, a has no inverse for the Second Operation.
            \end{proof}
            \begin{theorem}
            There exist pseudo-fields with only one element.
            \end{theorem}
            \begin{proof}
            For let $e_{\circ} = e_{*}$, and let no other elements be in the set. 
            \end{proof}
            \begin{theorem}
            A pseud-field has one element if and only if $e_{\circ} = e_{*}$.
            \end{theorem}
            \begin{proof}
            For suppose there is another element $a \ne e_{\circ}$. But then $a \circ e_{\circ} = a$, but also $a \circ e_{\circ} = a \circ e_{*} = e_{*}$. So $a = e_{*}$. If there is only one element, then clearly $e_{\circ} = e_{*}$ as otherwise there would be two elements.
            \end{proof}
            \begin{definition} A generating set on a pseudo-field is a subset $g_S \subset S$ such that every element of $S$ can be written as a finite combination of elements in $g_S$ using $\circ$ or $*$.
            \end{definition}
            \begin{theorem}
            The number of elements in a finite pseudo-field is a power of 2.
            \end{theorem}
            \begin{proof}
            Consider the set of all generators $g_S$ on $S$. Clearly for all such generators, $1\leq |g_S|\leq |S|$. Let $G$ be the smallest generator, such that $|G| \leq |g_S|$ for any other given generator. 
            \end{proof}
        \section{Sequences and Matrices}
    Matrices\index{Matrix} are the fundamental object studied in
    linear algebra\index{Linear Algebra}, and are used in the study of general
    algebra as well. To discuss the more interesting properties requires some
    notion of arithmetic that we do not yet posses. In particular, matrices are
    most interesting when there is an underlying \textit{ring}\index{Ring}
    structure. For now we simply introduce the set theoretic definition of a
    matrix, relate this to the familiar \textit{grid of numbers} definition, and
    provide examples.
    \begin{fdefinition}{Matrix}
        An $n\times{m}$, $n,m\in\mathbb{N}$, matrix on a set $X$ is a function
        $A:\mathbb{Z}_{n}\times\mathbb{Z}_{m}\rightarrow{X}$.
    \end{fdefinition}
        \section{Cardinality}
    As mentioned before, there is a way to discuss the size of sets in a manner
    that allows one to be precise when stating \textit{the set A is larger than}
    \textit{the set B}. If two sets are small enough, we can simply count out
    the number of elements contained in each and compare sets this way. For an
    infinite set it doesn't make sense to talk about the \textit{number} of
    elements, but we can still specify what it means two sets to have the same
    size. Sets $A$ and $B$ are equivalent if there exists a bijection
    $f:A\rightarrow{B}$. We then say that $A$ and $B$ have the same cardinality,
    and we denote this by $\textrm{Card}(A)=\textrm{Card}(B)$. A finite set is a
    set $A$ such that there is a bijection between $A$ and $\mathbb{Z}_{n}$, for
    some $n\in\mathbb{N}$. We can then view the elements of $A$ as
    $A=\{a_{1},\hdots,a_{n}\}$. A countable set is a set $A$ such that there is
    a bijection between $A$ and $\mathbb{N}$. These notions were first developed
    by Georg Cantor\index{Cantor, Georg}, and one natural question would be to
    ask if $\textrm{Card}(\mathbb{N})=\textrm{Card}(\mathbb{N})$? What about
    $\textrm{Card}(\mathbb{N})$ and $\textrm{Card}(\mathbb{R})$? This chapter
    aims to answer these questions, and develop the cardinal numbers along the
    way.
    \subsection{Equivalent Sets}
        \begin{fdefinition}{Equivalent Sets}{Equivalent_Sets}
            Equivalent sets are \glspl{set} $A$ and $B$ such that there exists a
            \gls{bijective function} $f:A\rightarrow{B}$.
        \end{fdefinition}
        The notion of equivalent sets defines an equivalence relation on sets.
        That is, the notion is reflexive, symmetric, and transitive.
        \begin{theorem}
            If $A$ is a set, then $A$ is equivalent to $A$.
        \end{theorem}
        \begin{proof}
            For the identity mapping $\textrm{id}_{A}:A\rightarrow{A}$ is a
            bijective function.
        \end{proof}
        \begin{theorem}
            If $A$ and $B$ are sets and if $A$ is equivalent to $B$, then $B$ is
            equivalent to $A$.
        \end{theorem}
        \begin{proof}
            For if $A$ is equivalent to $B$, then there is a bijection
            $f:A\rightarrow{B}$. But if $f$ is a bijection, then the inverse
            function $f^{-1}:B\rightarrow{A}$ is well-defined and is a
            bijection. Thus $B$ is equivalent to $A$.
        \end{proof}
        \begin{theorem}
            If $A$, $B$, and $C$ are sets, if $A$ is equivalent to $B$, and if
            $B$ is equivalent to $C$, then $A$ is equivalent to $C$.
        \end{theorem}
        \begin{proof}
            For if $A$ is equivalent to $B$, then there is a bijection
            $f:A\rightarrow{B}$. But if $B$ is equivalent ot $C$, then there is a
            bijection $g:B\rightarrow{C}$. But then $g\circ{f}:A\rightarrow{C}$ is a
            bijection, and thus $A$ and $C$ are equivalent.
        \end{proof}
        A bijection is a function that is both injective and surjective. Thus, two
        equivalent sets can be put into a one-to-one correspondence and can be said
        to have the same size. We then say that $A$ and $B$ have the same
        cardinality. The notation is written as $|A|=|B|$ or $\Card(A)=\Card(B)$.
        Cardinality splits sets into one of three categories.
        \begin{ldefinition}{Finite Sets}{Finite_Sets}
            A finite set is a set $A$ such that there exists
            an $n\in\mathbb{N}$ such that there is a
            bijection $f:\mathbb{Z}_{n}\rightarrow{A}$, or
            such that $A=\emptyset$.
        \end{ldefinition}
        Sets that are not finite are called infinite. There
        are two types of infinite sets. Let $\mathbb{N}$
        denote the set of positive integers, or
        \textit{natural} numbers.
        \begin{ldefinition}{Countably Infinite Sets}
              {Countably_Infinite}
            A countably infinite set is a set $A$ such that
            is a bijection $f:\mathbb{N}\rightarrow{A}$.
        \end{ldefinition}
        Combining the notions of finite sets and countably
        infinite sets, we get the notion of
        \textit{countable} sets.
        \begin{ldefinition}{Countable Sets}
              {Countable_Sets}
            A countable set is a set $A$ such that $A$ is
            either finite or countably infinite.
        \end{ldefinition}
        Countable sets are also called \textit{listable}.
        This is because if $A$ is a countably infinite set,
        and if $a:\mathbb{N}\rightarrow{A}$ is a bijection,
        we can write $A$ as:
        \begin{equation}
            A=\{\;a_{n}\,:\,n\in\mathbb{N}\;\}
            =\{\,a_{1},\,a_{2},\,\dots,\,a_{k},\,\dots\,\}
        \end{equation}
        If $A$ is finite, and if
        $a:\mathbb{Z}_{n}\rightarrow{A}$ is a
        bijection, then we can write:
        \begin{equation}
            A=\{\;a_{n}\,:\,n\in\mathbb{Z}_{n}\;\}
             =\{\,a_{1},\,\dots,\,a_{n}\,\}
        \end{equation}
        Recall that functions $a:\mathbb{N}\rightarrow{A}$
        are called \textit{sequences}, and the image of
        $n\in\mathbb{N}$ is written $a_{n}$, rather than
        $a(n)$.
        \begin{example}
            The set of all positive even integers is
            countable. For let $\mathbb{N}_{e}$ be the
            set of all even integers and define
            $f:\mathbb{N}\rightarrow\mathbb{N}_{e}$ be
            $f(n)=2n$ for all $n\in\mathbb{N}$. This is
            a bijection, and thus $\mathbb{N}_{e}$ is
            countable. The set of all odd positive integers
            is countable, as shown by letting
            $f(n)=2n-1$. Even though the set of even
            integers may seem ``smaller,'' than the set of
            all integers, they are equivalent. The set of
            all integers $\mathbb{Z}$ is also countable.
            For let $f:\mathbb{N}\rightarrow\mathbb{Z}$
            be defined as:
            \begin{equation}
                f(n)=
                \begin{cases}
                    \frac{1}{2}(n-1),&n\textrm{ odd}\\
                    -\frac{n}{2},&n\textrm{ even}
                \end{cases}
            \end{equation}
        \end{example}
        Any set that is infinite (Not finite) contains a
        countable subset. Thus, $\mathbb{N}$ can be
        considered as the \textit{smallest} infinite set.
        \begin{theorem}
            If $A$ is an infinite set, then there exists
            $S\subseteq{A}$ such that $S$ is countable.
        \end{theorem}
        \begin{proof}
            For as $A$ is infinite, for all $n\in\mathbb{N}$
            there exists a set $B\subseteq{A}$ such that
            $|B|=n$. For all $n\in\mathbb{N}$,
            define the following:
            \begin{equation}
                \mathcal{S}_{n}=\{B\subseteq{A}:|B|=n\}
            \end{equation}
            Let $\mathcal{S}$ be defined as:
            \begin{equation}
                \mathcal{S}=\{\mathcal{S}_{n}:n\in\mathbb{N}\}
            \end{equation}
            Then $\mathcal{S}$ is countable, for
            $a:\mathbb{N}\rightarrow\mathcal{S}$ defined
            by $a_{n}=\mathcal{S}_{n}$ is a bijection.
            By the axiom of choice, there is a function:
            \begin{equation}
                \alpha:\mathcal{S}\rightarrow
                \bigcup_{n=1}^{\infty}\mathcal{S}_{n}
            \end{equation}
            Such that, for all $x\in\mathcal{S}$,
            $\alpha(x)\in{x}$. But then, for all
            $x\in\mathcal{S}$, $\alpha(x)$ is a subset
            of $A$. But for all $x\in\mathcal{S}$, there
            is an $n\in\mathbb{N}$ such that
            $a_{n}=x$. Thus, let $S$ be the following:
            \begin{equation}
                S=\bigcup_{n=1}^{\infty}\alpha(a_{n})
            \end{equation}
        \end{proof}
        In the absence of the requirement that $a\cap{b}=\emptyset$ for all
        pairs in $\mathcal{U}$, we still have that the union is, at most,
        countable. The mapping we found would be a \textit{surjection}, rather
        than a bijection. The union is then either finite or countable. The
        Cantor-Schr\"{o}der-Bernstein Theorem can often be used to help identify
        the size of a set. This says that if $A$ and $B$ are sets such that
        there exists a surjective function $f:A\rightarrow{B}$ and a surjective
        function $g:B\rightarrow{A}$, then there is a bijective function
        $h:A\rightarrow{B}$. The requirement that $f$ and $g$ both be surjective
        can be replaced with the requirement that they both be injective. This
        is similar to saying that if $\Card(A)\leq\Card(B)$ and
        $\Card(B)\leq\Card(A)$, then $\Card(A)=\Card(B)$. Here, $\Card(A)$
        denotes the \textit{cardinality} of the set $A$.
        \begin{lexample}{Countable Sets}{Countable_Sets}
            There are many commonly discussed sets that are
            countably infinite. $\mathbb{N}$ is a trivial
            such example, but also $\mathbb{N}_{e}$ and
            $\mathbb{N}_{o}$, the sets of even and odd
            positive integers, respectively. For consider as
            bijections the following functions:
            \par
            \begin{subequations}
                \begin{minipage}[b]{0.49\textwidth}
                    \centering
                    \begin{equation}
                        f_{e}(n)=2n
                    \end{equation}
                \end{minipage}
                \hfill
                \begin{minipage}[b]{0.49\textwidth}
                    \centering
                    \begin{equation}
                        f_{0}(n)=2n-1
                    \end{equation}
                \end{minipage}
                \par\vspace{2.5ex}
                The set of all integers, $\mathbb{Z}$ is also
                countable, as shown in
                Fig.~\ref{fig:Bijection_N_and_Z}.
                One bijection is:
                \begin{equation}
                    f(n)=
                    \begin{cases}
                        \frac{n}{2},&n\mod{2}=0\\
                        \frac{1-n}{2},&n\mod{2}=1
                    \end{cases}
                \end{equation}
            \end{subequations}
            Any subset of $\mathbb{Z}$ is countable,
            and this is true of any countable set.
        \end{lexample}
        \begin{figure}[H]
            \centering
            \captionsetup{type=figure}
            \documentclass[crop,class=article]{standalone}
%----------------------------Preamble-------------------------------%
\usepackage{tikz}
\usepackage{amssymb}
\usetikzlibrary{arrows.meta}
%--------------------------Main Document----------------------------%
\begin{document}
    \begin{tikzpicture}[%
        >=latex
    ]
        \draw[<->, thick] (-5, 0) to (5, 0) node[below] {$\mathbb{Z}$};
        \foreach\x in {-4, -3, -2, -1, 0, 1, 2, 3, 4}{%
            \draw (\x, -0.1) to (\x, 0.1);
            \node at (\x, -0.4) {\x};
        }
        \draw[->] (0, 0.2) arc(180:0:0.5 and 0.4);
        \draw[->] (1, -0.6) arc(0:-180:1 and 0.6);
        \draw[->] (-1, 0.2) arc(180:0:1.5 and 0.7);
        \draw[->] (2, -0.6) arc(0:-180:2 and 0.8);
        \draw[->] (-2, 0.2) arc(180:0:2.5 and 0.9);
        \draw[->] (3, -0.6) arc(0:-180:3 and 1);
        \draw[->] (-3, 0.2) arc(180:0:3.5 and 1.1);
        \draw[->] (4, -0.6) arc(0:-180:4 and 1.2);
    \end{tikzpicture}
\end{document}
            \caption{Diagram of a Bijection Between
                     $\mathbb{N}$ and $\mathbb{Z}$.}
            \label{fig:Bijection_N_and_Z}
        \end{figure}
        One of the standard results about countable sets is
        that their subsets are also countable. This theorem
        relies, in a very subtle way, the use of the axiom
        of choice. There are a few stepping stones to get
        there. We will accept the various
        Cantor-Schr\"{o}eder-Bernstein theorems, which say
        the following:
        \begin{ltheorem}
              {First Cantor-Schr\"{o}eder-Bernstein Theorem}
              {First_Cantor_Schroeder_Bernstein}
            If $A$ and $B$ are sets such that there is an injective
            function $f:A\rightarrow{B}$ and an injective function
            $g:B\rightarrow{A}$, then there is a bijective function
            $h:A\rightarrow{B}$.
        \end{ltheorem}
        \begin{ltheorem}
              {Second Cantor-Schr\"{o}eder-Bernstein Theorem}
              {Second_Cantor_Schroeder_Bernstein}
            If $A$ and $B$ are sets such that there is a surjective
            function $f:A\rightarrow{B}$ and a surjective function
            $g:B\rightarrow{A}$, then there is a bijective function
            $h:A\rightarrow{B}$.
        \end{ltheorem}
        \par\hfill\par
        Using cardinalities, this says that if
        $\Card(A)\leq\Card(B)$ and $\Card(B)\leq\Card(A)$, then
        $\Card(A)=\Card(B)$. With this notation it becomes more
        intuitive. We will use this to prove that various sets are
        countable. Many sets that appear to be larger than $\mathbb{N}$
        can shown to to be the same size as $\mathbb{N}$ by finding
        a simple injective function, without finding an explicit
        bijection.
        \begin{ltheorem}
              {Third Cantor-Schr\"{o}eder-Bernstein Theorem}
              {Third_Cantor_Schroeder_Bernstein}
            If $A$, $B$, and $C$ are sets such that
            $A\subseteq{B}\subseteq{C}$, and if $A$ and $C$ are equivalent
            sets, then $B$ and $C$ are equivalent sets.
        \end{ltheorem}
        \par\hfill\par
        This says that if $\Card(A)\leq\Card(B)\leq\Card(C)$,
        and if $\Card(A)=\Card(C)$, then $\Card(B)=\Card(C)$.
        \begin{theorem}
            \label{thm:Measure_Theory_NxN_Is_Countable}
            $\mathbb{N}\times\mathbb{N}$ is countably infinite.
        \end{theorem}
        \begin{proof}
            There is a trivial injection
            $f:\mathbb{N}\rightarrow\mathbb{N}\times\mathbb{N}$
            defined by:
            \begin{equation}
                f(n)=(n,0)
            \end{equation}
            There is also an injection
            $g:\mathbb{N}\times\mathbb{N}\rightarrow\mathbb{N}$
            defined by:
            \begin{equation}
                g(n.m)=2^{n}3^{m}
            \end{equation}
            Since 2 and 3 are co-prime, if
            $g(n_{1},m_{1})=g(n_{2},m_{2})$, then
            $(n_{1},m_{1})=(n_{2},m_{2})$. Thus, $g$ is an injection.
            By the Cantor-Schr\"{o}eder-Bernstein Theorem, there is a
            bijection $h:\mathbb{N}\rightarrow\mathbb{N}\times\mathbb{N}$.
        \end{proof}
        One can intuitively see that the set of all positive
        rational numbers $\mathbb{Q}^{+}$ is countable by examining
        the zig-zag pattern shown in
        Fig.~\ref{fig:Bijection_N_and_Q_Plus}.
        Thm.~\ref{thm:Measure_Theory_NxN_Is_Countable} also
        shows this in a more rigorous way that. We can create
        a one-to-one correspondence with
        $\mathbb{N}\times\mathbb{N}$ by mapping
        $pq^{\minus{1}}\mapsto(p,q)$. Thus $\mathbb{Q}^{+}$
        and $\mathbb{N}\times\mathbb{N}$ are equivalent sets.
        But $\mathbb{N}\times\mathbb{N}$ and $\mathbb{N}$
        are equivalent sets, and therefore $\mathbb{Q}^{+}$
        is countable.
        Thm.~\ref{thm:Measure_Theory_NxN_Is_Countable} can also be used
        to show that the countable union of countable sets is also
        countable.
        \begin{ltheorem}{Equivalence of Countable Sets}
              {Countable_iff_exists_inj_to_N}
            A set $A$ is countable if and only if there is an injective
            function $f:A\rightarrow\mathbb{N}$.
        \end{ltheorem}
        Thm.~\ref{thm:Countable_iff_exists_inj_to_N} seems
        intuitively obvious, the injective function is
        simply the listing function. For a finite set, this
        is precisely how one constructs such an injection.
        For an infinite set $A$, this is equivalent to
        showing that any infinite subset of $\mathbb{N}$ is
        equivalent to $\mathbb{N}$. The standard proof
        using \textit{induction}, but actually has the axiom
        of choice underlying it.
        \begin{theorem}
            If $\mathcal{A}$ is a countably infinite set
            such that, for all $A\in\mathcal{A}$, $A$ is
            a non-empty countable set, then the set:
            \begin{equation}
                S=\bigcup_{A\in\mathcal{A}}A
            \end{equation}
            Is a countable set.
        \end{theorem}
        \begin{proof}
            If $\mathcal{F}$ is finite, then we are done. Suppose not.
            Let $A:\mathbb{N}\rightarrow\mathcal{A}$ be a bijection,
            and define:
            \begin{equation}
                S=\bigcup_{n\in\mathbb{N}}A_{n}
            \end{equation}
            Also, let:
            \begin{equation}
                \mathcal{F}_{n}
                =\{f:A_{n}\rightarrow\mathbb{N}:
                    f\textrm{ is injective}\}
            \end{equation}
            Since, for all $n\in\mathbb{N}$, $A_{n}$ is
            non-empty and countable, $\mathcal{F}_{n}$
            is non-empty. Let:
            \begin{equation}
                \mathcal{F}
                =\bigcup_{n\in\mathbb{N}}\mathcal{F}_{n}
            \end{equation}
            Thus, by the axiom of choice, there is a function
            $F:\mathbb{N}\rightarrow\mathcal{F}$ such that, for all
            $n\in\mathbb{N}$, $F_{n}\in\mathcal{F}_{n}$. For
            $x\in{S}$, let:
            \begin{equation}
                \varphi_{x}
                =\inf\{n\in\mathbb{N}:x\in{A}_{n}\}
            \end{equation}
            By the well-ordering of $\mathbb{N}$, for all
            $x\in{S}$, $\varphi_{x}$ is well defined. Let
            $\phi:S\rightarrow\mathbb{N}\times\mathbb{N}$
            be defined by:
            \begin{equation}
                \phi(x)
                =\big(\varphi_{x},F_{\varphi_{x}}(x)\big)
            \end{equation}
            Then $\phi$ is an injection. For if
            $\big(\varphi_{x},F_{\varphi_{x}}(x)\big)=%
             \big(\varphi_{y},F_{\varphi_{x}}(y)\big)$, then
            $\varphi_{x}=\varphi_{y}$, and thus
            $F_{\varphi(x)}(x)=F_{\varphi(x)}(y)$. But
            $F_{\varphi_{x}}$ is an injection, and
            thus $x=y$. Therefore $\phi$ is an injection.
            But $\mathbb{N}\times\mathbb{N}$ and $\mathbb{N}$
            are equivalent sets, and thus there's an
            injection $f:\mathbb{N}\times\mathbb{N}$. And
            the composition of injective functions is again
            injective, and thus
            $\phi\circ{f}:S\rightarrow\mathbb{N}$ is an
            injective function. But by
            Thm.~\ref{thm:Countable_iff_exists_inj_to_N},
            if there is an injective function
            $f:S\rightarrow\mathbb{N}$, then $S$ is
            countable. Therefore, etc.
        \end{proof}
        \begin{theorem}
            If $X$ is infinite, then there exists a
            countably infinite set $A\subseteq{X}$.
        \end{theorem}
        \begin{proof}
            If $A$ is finite, then we are done. Suppose not.
            For $n\in\mathbb{N}$, let:
            \begin{equation}
                A_{n}
                =\{g:\mathbb{Z}_{n}\rightarrow{A}:f\textrm{ is inective}\}
            \end{equation}
            Also, define:
            \begin{equation}
                \mathcal{F}=\bigcup_{n\in\mathbb{N}}A_{n}
            \end{equation}
            But by the axiom of choice, there is a function
            $f:\mathbb{N}\rightarrow\mathcal{F}$ such that
            $f_{n}\in{A}_{n}$. But then, for all
            $n\in\mathbb{N}$, the range of $f_{n}$ is finite.
            \begin{equation}
                A=\bigcup_{n\in\mathbb{N}}f_{n}
                    \Big(\mathbb{Z}_{n}\Big)
            \end{equation}
            Then $A\subseteq{X}$ is countably infinite.
        \end{proof}
        The set of rational numbers, $\mathbb{Q}$, is also
        countable. We may intuitively think of $\mathbb{N}$
        as being smaller than $\mathbb{Q}$, since there are
        simple \textit{surjections} that can be constructed
        from $\mathbb{Q}$ to $\mathbb{N}$. There is also a
        surjection from $\mathbb{N}$ onto $\mathbb{Q}^{+}$,
        as is shown in Fig.~\ref{fig:Bijection_N_and_Q_Plus}.
        To construct such a surjection, write out all of the
        positive rational numbers in a grid so that $a_{nm}$
        is the number $n/m$. Then zig-zag along the diagonals
        to construct the function. Thus there is a surjection
        $f:\mathbb{Q}^{+}\rightarrow\mathbb{N}$ and a surjection
        $g:\mathbb{N}\rightarrow\mathbb{Q}^{+}$. The
        Cantor-Schr\"{o}eder-Bernstein theorem says that if there is
        surjection from $A$ to $B$ and a surjection from $B$ to $A$, then
        there is a bijection between $A$ and $B$. Therefore there is a
        bijection between $\mathbb{N}$ and $\mathbb{Q}^{+}$, and
        $\mathbb{Q}^{+}$ is countable.
        \begin{figure}[H]
            \centering
            \captionsetup{type=figure}
            \resizebox{0.7\textwidth}{!}{%
                \documentclass[crop,class=article]{standalone}
%----------------------------Preamble-------------------------------%
\usepackage{tikz}
\usetikzlibrary{arrows.meta}
%--------------------------Main Document----------------------------%
\begin{document}
    \begin{tikzpicture}[%
        >=latex
    ]
        \foreach\y in {1, 2, 3, 4, 5, 6}{%
            \foreach\x in {1, 2, 3, 4, 5, 6}{%
                \node (\x\y) at (\x, 7-\y) {$\frac{\x}{\y}$};
            }
        }
        \foreach\x in {1, 2, 3, 4, 5, 6}{%
            \node at (7, \x) {$\cdots$};
            \node at (\x, 0) {$\vdots$};
        }
        \node at (7, 0) {$\ddots$};
        \draw[->] (11) to (12);
        \draw[->] (12) to (21);
        \draw[->] (21) to (31);
        \draw[->] (31) to (22);
        \draw[->] (22) to (13);
        \draw[->] (13) to (14);
        \draw[->] (14) to (23);
        \draw[->] (23) to (32);
        \draw[->] (32) to (41);
        \draw[->] (41) to (51);
        \draw[->] (51) to (42);
        \draw[->] (42) to (33);
        \draw[->] (33) to (24);
        \draw[->] (24) to (15);
        \draw[->] (15) to (16);
        \draw[->] (16) to (25);
        \draw[->] (25) to (34);
        \draw[->] (34) to (43);
        \draw[->] (43) to (52);
        \draw[->] (52) to (61);
        \draw[->] (61) to (62);
        \draw[->] (62) to (53);
        \draw[->] (53) to (44);
        \draw[->] (44) to (35);
        \draw[->] (35) to (26);
        \draw[->] (36) to (45);
        \draw[->] (45) to (54);
        \draw[->] (54) to (63);
        \draw[->] (64) to (55);
        \draw[->] (55) to (46);
        \draw[->] (56) to (65);
    \end{tikzpicture}
\end{document}
            }
            \caption{Diagram of a Surjection from
                     $\mathbb{N}$ onto $\mathbb{Q}^{+}$.}
            \label{fig:Bijection_N_and_Q_Plus}
        \end{figure}
        We can modify Fig.~\ref{fig:Bijection_N_and_Q_Plus}
        slightly to create a surjection between $\mathbb{N}$
        and $\mathbb{Q}$, see
        Fig.~\ref{fig:Bijection_N_and_Q}.
        It is important to note that this bijection will not
        preserve the order of the rational numbers. The
        bijection will have to jump around back and forth.
        Any such bijection will be forced to do this, as the
        rationals are everywhere dense on $\mathbb{R}$. Any
        monotonic sequence of $\mathbb{Q}$ cannot possibly
        be a bijection.
        \begin{theorem}
            If $A$ is a countably infinite set and
            $B\subseteq{A}$, then $B$ is countable.
        \end{theorem}
        \begin{proof}
            As $A$ is countably infinite, there is a bijection
            $a:\mathbb{N}\rightarrow{A}$. Define:
            \begin{equation}
                K=\{n\in\mathbb{N}:a_{n}\in{B}\}
            \end{equation}
            As $B\subseteq{A}$,
            this set contains a subsequence of points in
            $\mathbb{N}$ that get mapped into $B$. If $K$ is finite,
            then $B$ is finite, and if not then $K$ is countably
            infinite, and thus $B$ is countably infinite.
        \end{proof}
        \begin{figure}[H]
            \centering
            \captionsetup{type=figure}
            \resizebox{\textwidth}{!}{%
                \documentclass[crop,class=article]{standalone}
%----------------------------Preamble-------------------------------%
\usepackage{tikz}
\usepackage{amsmath}
\usetikzlibrary{arrows.meta}
%--------------------------Main Document----------------------------%
\begin{document}
    \begin{tikzpicture}[%
        >=latex
    ]
        \foreach\y in {1, 2, 3, 4}{%
            \foreach\x in {-4, -3, -2, -1, 0, 1, 2, 3, 4}{%
                \node (\x\y) at (\x, 7-\y) {$\frac{\x}{\y}$};
            }
        }
        \foreach\x in {-4, -3, -2, -1, 0, 1, 2, 3, 4}{%
            \node at (\x, 2) {$\vdots$};
        }
        \foreach\y in {3, 4, 5, 6}{%
            \node at (5, \y) {$\cdots$};
            \node at (-5, \y) {$\cdots$};
        }
        \node at (5, 2) {$\ddots$};
        \node at (-5, 2) {$\reflectbox{\ensuremath{\ddots}}$};
        \draw[->] (01) to (11);
        \draw[->] (11) to (12);
        \draw[->] (12) to (02);
        \draw[->] (02) to (-12);
        \draw[->] (-12) to (-11);
        \draw[->] (-1, 6.3) to (-1, 6.6)
                            to (2, 6.6)
                            to (2, 6.3);
        \draw[->] (21) to (22);
        \draw[->] (22) to (23);
        \draw[->] (23) to (13);
        \draw[->] (13) to (03);
        \draw[->] (03) to (-13);
        \draw[->] (-13) to (-23);
        \draw[->] (-23) to (-22);
        \draw[->] (-22) to (-21);
        \draw[->] (-2, 6.3) to (-2, 6.8)
                            to (3, 6.8)
                            to (3, 6.3);
        \draw[->] (31) to (32);
        \draw[->] (32) to (33);
        \draw[->] (33) to (34);
        \draw[->] (34) to (24);
        \draw[->] (24) to (14);
        \draw[->] (14) to (04);
        \draw[->] (04) to (-14);
        \draw[->] (-14) to (-24);
        \draw[->] (-24) to (-34);
        \draw[->] (-34) to (-33);
        \draw[->] (-33) to (-32);
        \draw[->] (-32) to (-31);
        \draw[->] (-3, 6.3) to (-3, 7)
                            to (4, 7)
                            to (4, 6.3);
        \draw[->] (41) to (42);
        \draw[->] (42) to (43);
        \draw[->] (43) to (44);
        \draw[->] (44) to (4, 2.3);
        \draw[->] (-4, 2.3) to (-44);
        \draw[->] (-44) to (-43);
        \draw[->] (-43) to (-42);
        \draw[->] (-42) to (-41);
    \end{tikzpicture}
\end{document}
            }
            \caption{Diagram of a Surjection from
                     $\mathbb{N}$ onto $\mathbb{Q}$.}
            \label{fig:Bijection_N_and_Q}
        \end{figure}
        \begin{theorem}
            If $A$ is an infinite set, then there exists a
            countable subset $B\subseteq{A}$.
        \end{theorem}
        \begin{proof}
            If $A$ is infinite then there is an
            $a_{1}\in{A}$. But, as $A$ is infinite,
            $A\setminus\{a_{1}\}$ is infinite, and there
            is an $a_{2}\in{A}\setminus\{a_{1}\}$. Continuing
            we obtain a sequence of distinct elements in $A$.
            Let $B=\{a_{n}:n\in\mathbb{N}\}$. Then
            $B\subseteq{A}$, and $B$ is countable.
        \end{proof}
        \begin{lexample}
            Suppose we have a collection of disjoint intervals
            of $\mathbb{R}$. This collection is either finite
            or countable. For in every interval, choose a
            rational number $q_{n}$. Let
            $A=\{q_{1},q_{2},\hdots\}$. Then
            $A\subseteq\mathbb{Q}$, and thus $A$ is either
            finite or countable. But this is also an enumeration
            of the intervals in the collection, and thus the
            collection is either finite or countable.
        \end{lexample}
        Given a countable collection of sets
        $A=\{\mathcal{A}_{1},\mathcal{A}_{2},\hdots\}$ such
        that, for all $n\in\mathbb{N}$, $\mathcal{A}_{n}$ is
        also a countable set, then the union is countable. That is:
        \begin{equation}
            B=\bigcup_{n=1}^{\infty}\mathcal{A}_{n}
        \end{equation}
        is a countable set. The proof of this is a mimicry of
        the proof of the countability of $\mathbb{Q}$. Not
        every set is either finite or countable. The real numbers,
        $\mathbb{R}$, is an example of an \textit{uncountable}
        set. First, some notes on the power set of a set.
        This is a bijection between the open unit interval $(0,1)$ and
        the closed unit interval $[0,1]$.
        \begin{equation}
            f(x)=
            \begin{cases}
                \frac{1}{2},&x=0\\
                \frac{1}{2^{n+2}},&x=\frac{1}{2^{n}}\\
                x,&\textrm{Otherwise}
            \end{cases}
        \end{equation}
        A graph of this is shown in
        Fig.~\ref{fig:Measure_Theory_Bijection_Closed_I_to_Open}.
        \begin{figure}[H]
            \centering
            \captionsetup{type=figure}
            \documentclass[crop,class=article]{standalone}
%----------------------------Preamble-------------------------------%
\usepackage{tikz}                       % Drawing/graphing tools.
\usetikzlibrary{arrows.meta}            % Latex and Stealth arrows.
%--------------------------Main Document----------------------------%
\begin{document}
    \begin{tikzpicture}[>=Latex, scale=2]
        \draw[->] (-0.15in, 0) to (1.1in, 0) node[above] {$x$};
        \draw[->] (0, -0.15in) to (0, 1.1in) node[right] {$y$};
        \draw (0, 0) to (1in, 1in);
        \draw[fill=black, draw=black] (0, 0.5in) circle (0.3mm);
        \foreach\x in{1in, 0.5in, 0.25in, 0.125in, 0.0625in, 0.03125in}{
            \draw[fill=white, draw=black] (\x, \x) circle (0.3mm);
            \draw[fill=black, draw=black] (\x, 0.25*\x) circle (0.3mm);
        }
    \end{tikzpicture}
\end{document}
            \caption{Bijection from $[0,1]$ to $(0,1)$.}
            \label{fig:Measure_Theory_Bijection_Closed_I_to_Open}
        \end{figure}
        The power set of any set is strictly larger than the
        original set. If $\Omega$ is finite with $n$ elements, it
        can be shown that $\mathcal{P}(\Omega)$ has $2^{n}$
        eleents. For infinite sets, there is a trivial surjection
        from $\mathcal{P}(\Omega)$ onto $\Omega$: for any element
        $x$, let $f(\{x\})=x$. This shows that
        $\Card(\Omega)\leq\Card(\mathcal{P}(\Omega))$. We now show
        that the inequality is strict.
        \begin{theorem}
            If $\Omega$ is a set, then there is no bijection
            $f:\Omega\rightarrow\mathcal{P}(\Omega)$
        \end{theorem}
        \begin{proof}
            For suppose not, and let
            $f:\Omega\rightarrow\mathcal{P}(\Omega)$ be such a
            bijection. Define:
            \begin{equation}
                A=\{x\in\Omega:x\in{f}(x)\}
            \end{equation}
            Then $A\subseteq\Omega$, and thus
            $A\in\mathcal{P}(\Omega)$. But then the complement of
            $A$ is also an element of $\mathcal{P}(\Omega)$. But
            $f$ is a bijection and thus there is an $x\in\Omega$
            such that $f(x)=A^{C}$. If $x\in{f}(x)$, then
            $x\in{A}$, a contradiction as $f(x)=A^{C}$, and thus
            $x\in{A}^{C}$ as well. Therefore $x\notin{f}(x)$. But
            then $x\in{A}^{C}$. But, from the definition of $A$,
            since $x\in{A}^{C}$ and $f(x)=A^{C}$, $x\in{f}(x)$
            and thus $x\in{A}$, a contradiction. Thus there is no
            $x$ such that $f(x)=A^{C}$. Therefore, $f$ is not a
            bijection.
        \end{proof}
        From this we conclude that $\mathcal{P}(\mathbb{N})$
        is an uncountable infinite set. But since $\mathbb{R}$
        and $\mathcal{P}(\mathbb{N})$ have the same cardinality,
        $\mathbb{R}$ is also uncountable.
        If a set $A$ has the same cardinality as $\mathbb{R}$,
        we say that $A$ has the cardinality of the continuum.
        \begin{lexample}
            There is a bijection between the open unit
            square $(0,1)\times(0,1)$ and the open unit interval
            $(0,1)$. For an element $(x,y)\in(0,1)\times(0,1)$,
            let $z\in(0,1)$ be defined as
            $z=0.x_{1}y_{1}x_{2}y_{2}x_{3}y_{3}\dots$ This is
            a bijection, for all $(x,y)$ in the square there is
            a corresponding $z\in(0,1)$, and for all
            $z\in(0,1)$ there is a corresponding element of
            $(0,1)\times(0,1)$. We can say that $(x,y)$ can
            be coded into $z$, and $z$ can be decoded into
            $(x,y)$. Hence, $(0,1)\times(0,1)$ has the cardinality
            of the continuum. By stereographic projection and induction
            we obtain:
            \par\hfill\par
            \begin{subequations}
                \begin{minipage}[b]{0.49\textwidth}
                    \begin{equation}
                        \Card(\mathbb{R}^{2})=\Card(\mathbb{R})
                    \end{equation}
                \end{minipage}
                \hfill
                \begin{minipage}[b]{0.49\textwidth}
                    \begin{equation}
                        \Card(\mathbb{R}^{n})=\Card(\mathbb{R})
                    \end{equation}
                \end{minipage}
                \par
            \end{subequations}
        \end{lexample}
        \begin{lexample}
            Consider the set of all real-valued sequences. We've seen
            that any real number can be represented as a function
            $f:\mathbb{N}\rightarrow\{0,1\}$. A real-valued sequence
            is a function $a:\mathbb{N}\rightarrow\mathbb{R}$, and
            thus the set of real-valued sequences can be seen as the
            set of functions whose domain is $\mathbb{N}$ and whose
            range is the set of all functions
            $f:\mathbb{N}\rightarrow\{0,1\}$. So given a sequence
            $a$, the image of $a_{n}$, for $n\in\mathbb{N}$, is a
            function $f_{n}:\mathbb{N}\rightarrow\{0,1\}$. Therefore
            the set of all real-valued sequences can be represented
            as the set of all functions
            $g:\mathbb{N}\times\mathbb{N}\rightarrow\{0,1\}$, where
            $g(n,m)=f_{n}(m)$. But $\mathbb{N}\times\mathbb{N}$ is
            countable, and thus the set of all functions of the form
            $g:\mathbb{N}\times\mathbb{N}\rightarrow\{0,1\}$ has the
            same cardinality as the set of all functions of the form
            $f:\mathbb{N}\rightarrow\{0,1\}$. But this has the
            cardinality of the continuum. Therefore, the set of all
            real-valued sequences has the cardinality of the continuum.
        \end{lexample}
        \begin{theorem}
            If $A$ is an infinite set, then there exists $S\subseteq{A}$ such that
            $S$ is countable.
        \end{theorem}
        \begin{proof}
            For as $A$ is infinite, for all $n\in\mathbb{N}$
            there exists a set $B\subseteq{A}$ such that
            $|B|=n$. For all $n\in\mathbb{N}$,
            define the following:
            \begin{equation}
                \mathcal{S}_{n}=\{B\subseteq{A}:|B|=n\}
            \end{equation}
            Let $\mathcal{S}$ be defined as:
            \begin{equation}
                \mathcal{S}=\{\mathcal{S}_{n}:n\in\mathbb{N}\}
            \end{equation}
            Then $\mathcal{S}$ is countable, for
            $a:\mathbb{N}\rightarrow\mathcal{S}$ defined
            by $a_{n}=\mathcal{S}_{n}$ is a bijection.
            By the axiom of choice, there is a function:
            \begin{equation}
                \alpha:\mathcal{S}\rightarrow
                \bigcup_{n=1}^{\infty}\mathcal{S}_{n}
            \end{equation}
            Such that, for all $x\in\mathcal{S}$,
            $\alpha(x)\in{x}$. But then, for all
            $x\in\mathcal{S}$, $\alpha(x)$ is a subset
            of $A$. But for all $x\in\mathcal{S}$, there
            is an $n\in\mathbb{N}$ such that
            $a_{n}=x$. Thus, let $S$ be the following:
            \begin{equation}
                S=\bigcup_{n=1}^{\infty}\alpha(a_{n})
            \end{equation}
        \end{proof}
        \begin{table}[H]
            \captionsetup{type=table}
            \centering
            \begin{tabular}{ccccc}
                $u_{11}$&$u_{12}$&$u_{13}$
                &$u_{14}$&$\hdots$\\
                $u_{21}$&$u_{22}$&$u_{23}$
                &$u_{24}$&$\hdots$\\
                $u_{31}$&$u_{32}$&$u_{33}$
                &$u_{34}$&$\hdots$\\
                $u_{41}$&$u_{42}$&$u_{43}$
                &$u_{44}$&$\hdots$\\
                $\vdots$&$\vdots$&$\vdots$
                &$\vdots$&$\ddots$
            \end{tabular}
            \caption{Construction of a Bijection on the
                     Countable Union of Countably Infinite
                     Sets.}
            \label{table:Countable_Union_of_Countable}
        \end{table}
        Where $u_{nm}$ is the $m^{th}$ element of
        $\mathcal{U}_{n}$.
        Using the \textit{diagonal argument},
        we obtain:
        In the absence of the requirement that
        $a\cap{b}=\emptyset$ for all pairs in $\mathcal{U}$,
        we still have that the union is, at most, countable.
        The mapping we found would be a
        \textit{surjection}, rather than a bijection.
        The union is then either finite or countable. The
        Cantor-Schr\"{o}der-Bernstein Theorem can often be
        used to help identify the size of a set. This says
        that if $A$ and $B$ are sets such that there exists
        a surjective function $f:A\rightarrow{B}$ and a
        surjective function $g:B\rightarrow{A}$, then there
        is a bijective function $h:A\rightarrow{B}$. The
        requirement that $f$ and $g$ both be surjective
        can be replaced with the requirement that they both
        be injective. This is similar to saying that if
        $\Card(A)\leq\Card(B)$ and $\Card(B)\leq\Card(A)$,
        then $\Card(A)=\Card(B)$. Here, $\Card(A)$ denotes
        the \textit{cardinality} of the set $A$.
        \begin{theorem}
            Equivalence has the following properties:
            \begin{enumerate}
                \item   $A\sim A$ for any set $A$.
                \item   If $A\sim B$, then $B\sim A$.
                \item   If $A\sim B$ and $B\sim C$, then $A\sim C$.
            \end{enumerate}
        \end{theorem}
        \begin{proof}
        In order,
        \begin{enumerate}
        \item   For let $f$ be the identity mapping. That is, for all
                $x\in A$, $f(x)=x$. This is bijective and thus $A\sim A$.
        \item   If $A\sim B$, there is a bijective function $f:A\rightarrow B$.
                Then $f^{-1}:B\rightarrow A$ is bijective, and $B\sim A$.
        \item   Let $f:A\rightarrow B$ and $g:B\rightarrow C$ be bijections.
                Then $g\circ f:A\rightarrow C$ is a bijection,
                and thus $A\sim C$.
        \end{enumerate}
        \end{proof}
        \begin{theorem}
            If $A\sim{C}$ and $B\sim{D}$, where $A,B$ and $C,D$ are disjoint,
            then $A\cup{B}\sim{C}\cup{D}$.
        \end{theorem}
        \begin{proof}
            Let $f:A\rightarrow C$ and $g:B\rightarrow D$ be isomorphisms.
            Let $h:A\cup{B}\rightarrow{C}\cup{D}$ be defined by:
            \begin{equation}
                h(x)=
                \begin{cases}
                    f(x),&x\in{A}\\
                    g(x),&x\in{B}
                \end{cases}
            \end{equation}
            As $A$ and $B$ are disjoint, this is indeed a function and it is
            bijective as $C$ and $D$ are disjoint. Therefore, etc.
        \end{proof}
        \begin{definition}
        A set $A$ is a said to be finite if and only if there is some $n\in \mathbb{N}$ such
        that there is a bijection $f:\mathbb{Z}_n \rightarrow A$.
        \end{definition}
        \begin{definition}
        If $A$ is a set that is equivalent to $\mathbb{Z}_n$ for some $n\in \mathbb{N}$,
        then the cardinality of $A$, denoted $|A|$, is $n$.
        \end{definition}
        \begin{theorem}
        For two finite sets $A$ and $B$, $A\sim B$ if and only if $|A|=|B|$.
        \end{theorem}
        \begin{proof}
        $[|A|=|B|=n]\Rightarrow[A\sim \mathbb{Z}_n]\land[B\sim \mathbb{Z}_n]\Rightarrow [A\sim B]$.
        $[A\sim B]\Rightarrow [\exists \underset{Bijective}{f:A\rightarrow B}]\Rightarrow [f(A) = B]\Rightarrow [|A|=|B|]$.
        \end{proof}
        \begin{definition}
        A set $A$ is said to be infinite if and only if there is a proper subset $B\underset{Proper}\subset A$ such that $B\sim A$.
        \end{definition}
        \begin{theorem}
        Infinite sets are not finite.
        \end{theorem}
        \begin{proof}
        Suppose not. Let $A$ be an infinite set and suppose there is an $n\in \mathbb{N}$ such
        that $A\sim \mathbb{Z}_n$. But as $A$ is an infinite set, there is a proper subset $B$
        such that $B\sim A$. But then $B\sim \mathbb{Z}_n$. But as $B$ is a proper subset,
        there is at least one point in $A$ not contained in $B$. But then $|B|<n$, a contradiction. Thus $A$ is not finite.
        \end{proof}
        \begin{theorem}
            If $A$ is an infinite set, then for every $n\in \mathbb{N}$
            there is a subset $B\subset A$ such that $B\sim \mathbb{Z}_n$.
        \end{theorem}
        \begin{proof}
        Suppose not. Then there is a least $n\in \mathbb{N}:B\subset A\Rightarrow |B|<n$.
        But then $A$ has at most $n$ elements, a contradiction.
        \end{proof}
        \begin{definition}
            A set $A$ is called countable if and only if $A\sim \mathbb{N}$.
        \end{definition}
        \begin{theorem}
        A set $A$ is infinite if and only if it contains a proper subset $B$ such that $B\sim \mathbb{N}$.
        \end{theorem}
        \begin{proof}
        If $A$ has a proper subset $B$ such that $B\sim \mathbb{N}$, then $A$ is not finite and is thus infinite.
        If $A$ is infinite, then for all $n\in \mathbb{N}$ there is a set $A_n\subset A$ such that
        $A_n \sim \mathbb{Z}_n$. Let $B = \{a_n: a_n \in A_n, a_n \notin A_{n-1}\}$.
        Note that $a_{n} = a_{m}$ if and only if $m= n$. Let $f:\mathbb{N} \rightarrow B$
        be defined by $n\mapsto a_n$. This is bijective, and thus $B\sim \mathbb{N}$.
        \end{proof}
        This shows that $\mathbb{N}$ is, in a sense, the "Smallest,"
        infinite set. $|\mathbb{N}|$ is denoted $\aleph_0$.
        \begin{definition}
        A set is called uncountable if and only if it is infinite and not countable.
        \end{definition}
        \begin{theorem}
            If $B\subset A$, $f:A\rightarrow B$ is injective, then there
            is a bijection $g:A\rightarrow B$
        \end{theorem}
        \begin{proof}
            Let $Y = A\setminus B$, and inductively define
            $f^{k+1}(Y)=f(f^{k}(Y))$. Let
            $X=Y\cup(\cup_{k=0}^{\infty}f^{k}(Y))$. As  $Y\cap{B}=\emptyset$,
            then $f(Y)\cap Y= \emptyset$. As $f$ is an injection,
            $f(f(Y))\cap f(Y)=\emptyset$,
            and similarly $f(f(Y))\cap Y = \emptyset$. Inductively,
            $f^{n}(Y)\cap f^{m}(Y)=\emptyset$,
        for $n\ne m$. It then also follows that $f(X) = \cup_{k=1}^{\infty} f^{k}(Y)$.
        Thus $A\setminus X = [B\cup Y]\setminus [Y\cup f(X)] = B\setminus f(X)$.
        Let $g(x) = \begin{cases} f(x), & x\in X \\ x, & x \in B\setminus f(X)\end{cases}$.
        This is a bijections from $A$ to $B$.
        \end{proof}
        \begin{theorem}[Cantor-Schr\"{o}der-Bernstein Theorem]
        If $A_1 \subset A$, $B_1 \subset B$, and $A\sim B_1$, $B \sim A_1$, then $A\sim B$.
        \end{theorem}
        \begin{proof}
        Let $f:A\rightarrow B_1$ and $g:B\rightarrow A_1$ be bijections.
        Then $(g\circ f):A\rightarrow A_1$ is an injection from $A$ into $A_1$.
        Thus, there is a bijection $h:A\rightarrow A_1$. Thus, $A\sim A_1 \sim B\Rightarrow A\sim B$.
        \end{proof}
        \begin{theorem}
            $\mathbb{N}\times \mathbb{N}$ is countable.
        \end{theorem}
        \begin{proof}
        For $f:\mathbb{N} \rightarrow \mathbb{N}\times \mathbb{N}$ defined by $f(n) = (0,n)$
        shows there is a subset $N_1$ of $\mathbb{N} \times \mathbb{N}$ such that
        $\mathbb{N}\sim N_1$. And $g:\mathbb{N}\times \mathbb{N} \rightarrow \mathbb{N}$
        defined by $g(n,m) =n+2^{n+m}$ shows that there is a subset $M_1 \subset \mathbb{N}$
        such that $\mathbb{N} \times \mathbb{N} \sim M_1$. By the Cantor-Schr\"{o}der-Bernstein Theorem,
        $\mathbb{N} \sim \mathbb{N}\times \mathbb{N}$.
        \end{proof}
        \begin{theorem}
            If $A$ is infinite and $f:A\rightarrow\mathbb{N}$ is injective,
            then $A$ is countable.
        \end{theorem}
        \begin{proof}
        As $A$ is infinite and $A\sim f(A)$, $f(A)$ is infinite.
        But as $f(A)\subset \mathbb{N}$ and $f(A)$ is infinite,
        $f(A)\sim \mathbb{N}$. Thus, $A\sim \mathbb{N}$.
        \end{proof}
        \begin{definition}
            If $A$ and $B$ are sets, we say that $|A|<|B|$ if there is an
            injective function $f:A\rightarrow B$, yet no bijection.
        \end{definition}
        \begin{theorem}[Cantor's Theorem]
            For a set $M$, $|M|<|\mathcal{P}(M)|$.
        \end{theorem}
        \begin{proof}
        For let $M$ be a set with cardinality $|M|$. Let $U_m \subset M$ such that $U_m \sim M$.
        Such a set exists, for example, the singletons of $\mathcal{P}(M)$. Thus, $M$ is split into
        two distinct sets $Class\ I=\{x\in M: \textrm{There is a subset } X\subset U_m\textrm{ such that }x\in X\}$,
        and $Class\ II=M-Class\ I$. Let $L = Class\ II$. $L\subset M$, and thus $L\in \mathcal{P}(M)$. However,
        $L \notin U_m$ for if it were, then the element $m_1$ paired with it in $M$ is of Class II
        (For it cannot be of Class I as $m_1$ would not appear in $L$). If $m_1$ were in Class II,
        then by definition $m_1 \notin L$. But as $m_1 \in L$, we see that $L\notin U_m$. Thus,
        $|U_m| <|\mathcal{P}(M)|$, and therefore $|M|<|\mathcal{P}(M)|$.
        \end{proof}
        \begin{theorem}
            The set $R=\{x\in \mathbb{R}:0<x<1\}$ is equivalent to
            $\mathcal{P}(\mathbb{N})$.
        \end{theorem}
        \begin{proof}
            For every real number has a binary representation (Proof of this
            is omitted). That is, for every real number $r$, there is a sequence
            $a:\mathbb{N}\rightarrow\{0,\,1\}$ such that:
            \begin{equation}
                r=\sum_{n=0}^{\infty}\frac{a_{n}}{2^{n}}
            \end{equation}
        As $0<x<1$, this sum is just $\sum_{n=1}^{\infty} \frac{a_n}{2^n}$.
        Let $f:\mathcal{P}(\mathbb{N})\rightarrow R$ be defined by the
        following: If $N\subset \mathcal{P}(\mathbb{N})$ and $n\in N$, then
        $a_{n}=1$, other wise $n=0$. Then every real number is matched to a
        subset of $\mathcal{P}(\mathbb{N})$, moreover this is done bijectively.
        Thus, $\mathcal{P}(\mathbb{N})\sim R$.
        \end{proof}
        \begin{theorem}
            $\mathbb{R}\sim\mathcal{P}(\mathbb{N})$.
        \end{theorem}
        \begin{proof}
            It suffices to show that $(0,1)\sim\mathbb{R}$.
            Let $f:(0,1)\rightarrow\mathbb{R}$ be defined by:
            \begin{equation}
                f(x)=
                \begin{cases}
                    \frac{x(1-x)}{2x-1},&x\ne\frac{1}{2}\\
                    0,&x=\frac{1}{2}
                \end{cases}
            \end{equation}
        \end{proof}
        \begin{theorem}
            The following are true:
            \begin{enumerate}
                \item $\Card(A)=0$ if and only if $A=\emptyset$.
                \item If ${A}\sim{\mathbb{Z}_{n}}$, then $\Card(A)=n$.
            \end{enumerate}
        \end{theorem}
        \begin{definition}
            A finite cardinal number is a cardinal
            number of a finite set.
        \end{definition}
        \begin{definition}
            The standard ordering on the finite cardinal
            number is $0<1<\hdots<n<n+1<\hdots$
        \end{definition}
        Thus, if $A$ and $B$ are finite sets, then we write $\Card(A)<\Card(B)$ if
        $A$ is equivalent to a subset of $B$ but not equivalent to $B$. We take this
        notion and generalize to all sets. For $A$ and $B$, we write
        $\Card(A)<\Card(B)$ if $A$ is equivalent to a subset of $B$ but is not
        equivalent to $B$. This is the same as saying $A$ is equivalent to a subset
        of $B$, but $B$ is not equivalent to a subset of $A$. We write that
        $\Card(A)\leq\Card(B)$ is $A$ is equivalent to a subset of $B$.
        \begin{theorem}
            The following are true:
            \begin{enumerate}
                \item If $\Card(A)\leq\Card(B)$ and
                      $\Card(B)\leq\Card(A)$, then
                      $\Card(A)\leq\Card(C)$.
                \item If $\Card(A)\leq\Card(B)$, then
                      $\Card(A)+\Card(C)\leq\Card(B)+\Card(C)$
            \end{enumerate}
        \end{theorem}
        \begin{theorem}
            If ${A}\subset{B}\subset{C}$, and
            $\Card(A)=\Card(C)$, then $\Card(B)=\Card(C)$.
        \end{theorem}
        \begin{theorem}
            If $f:{X}\rightarrow{Y}$ is a function,
            then $\Card(f(X))\leq\Card(X)$.
        \end{theorem}
        \begin{proof}
            Note that $f^{-1}(\{y\})$ creates a set of mutually disjoint
            subsets of $X$. By the axiom of choice there is a function
            $F:{f(X)}\rightarrow{X}$ such that for all ${y}\in{f(X)}$,
            ${F(y)}\in{f^{-1}(\{y\})}$. But since these sets are disjoint,
            $F$ is injective. Thus, $f(X)$ is equivalent to a subset of $X$.
            Therefore, $\Card(f(X))\leq\Card(X)$.
        \end{proof}
        The Schr\"{o}der-Bernstein theorem can be restated equivalently as
        ``If $A$ is equivalent to a subset of $B$ and $B$ is equivalent to a
        subset of $A$, then $A$ is equivalent to $B$.'' Addition and
        multiplication of finite cardinals follows directly from the standard
        arithmetic for the natural numbers. For cardinals of infinite sets,
        the arithmetic becomes a little more complicated.
        \begin{definition}
            The sum of two cardinal numbers is the cardinality of the union of two
            disjoint sets $A$ and $B$. That is, if ${A}\cap{B}=\emptyset$, then
            $\Card(A)+\Card(B)=\Card({A}\cup{B})$.
        \end{definition}
        \begin{theorem}
            If $a$ and $b$ are distinct cardinal numbers, then there exists sets $A$
            and $B$ such that ${A}\cap{B}=\emptyset$, $\textrm{Card}(A)=a$, and
            $\textrm{Card}(B)=b$.
        \end{theorem}
        \begin{theorem}
            If $A,B,C,$ and $D$ are sets such that $\Card(A)=\Card(C)$,
            $\Card(B)=\Card(D)$, and if ${A}\cap{B}=\emptyset$ and
            ${C}\cap{D}=\emptyset$, then
            $\Card({A}\cup{B})=\Card({C}\cup{D})$.
        \end{theorem}
        \begin{theorem}
            If $x,y,$ and $z$ are cardinal numbers, then
            $x+y=y+x$ and $x+(y+z)=(x+y)+z$.
        \end{theorem}
        The carinality of the set of natural numbers is denoted $\aleph_{0}$.
        That is, $\Card(\mathbb{N})=\aleph_{0}$
        \begin{example}
            Find the cardinal sum of $2$ and $5$. Let $N_{2}=\{1,2\}$ and
            $N_{5}=\{3,4,5,6,7\}$. Then $N_{2}$ and $N_{5}$ are disjoint,
            $\Card(N_{2})=2$ and $\Card(N_{5})=5$. Therefore
            $2+5=\Card(N_{2}\cup{N_{5}})$. But ${N_{2}}\cup{N_{5}}$ is just
            $\mathbb{Z}_{7}$, and $\Card(\mathbb{Z}_{7})=7$. Thus, $2+5=7$.
        \end{example}
        \begin{theorem}
            If $n$ and $m$ are finite cardinalities, then the cardinal sum of $n$
            and $m$ is the integer $n+m$, where $+$ is the usual arithmetic
            addition.
        \end{theorem}
        \begin{example}
            Compute the cardinal sum $\aleph_{0}+\aleph_{0}$. Let $\mathbb{N}_{e}$
            be the set of even natural numbers, and let $\mathbb{N}_{o}$ be the set
            of odd natural numbers. Then $\Card(\mathbb{N}_{e})=\aleph_{0}$,
            $\Card(\mathbb{N}_{o})=\aleph_{0}$, and
            ${\mathbb{N}_{o}}\cap{\mathbb{N}_{e}}=\emptyset$. Thus:
            \begin{equation}
                \aleph_{0}+\aleph_{0}=\Card({\mathbb{N}_{o}}\cup{\mathbb{N}_{e}})
            \end{equation}
            But ${\mathbb{N}_{o}}\cup{\mathbb{N}_{e}}=\mathbb{N}$ and
            $\Card(\mathbb{N})=\aleph_{0}$. Therefore,
            $\aleph_{0}+\aleph_{0}=\aleph_{0}$.
        \end{example}
        \begin{example}
            Find $n+\aleph_{0}$, where $n\in\mathbb{N}$. We have that
            $\Card(\mathbb{Z}_{n}z)=n$ and
            $\Card(\mathbb{N}\setminus\mathbb{Z}_{n})=\aleph_{0}$
            But then
            $n+\aleph_{0}=\Card(\mathbb{Z}_{n}\cup%
             \mathbb{N}\setminus\mathbb{Z}_{n})=\Card(\mathbb{N})=\aleph_{0}$.
            Therefore, $n+\aleph_{0}=\aleph_{0}$.
        \end{example}
        \begin{definition}
            The cardinality of the continuum, denoted $\mathfrak{c}$, is the
            cardinality of the set of real numbers. That is,
            $\mathfrak{c}=\Card(\mathbb{R})$.
        \end{definition}
        \begin{theorem}
            $\mathfrak{c}+\aleph_{0}=\mathfrak{c}$.
        \end{theorem}
        \begin{proof}
            We have $\Card((0,1))=\mathfrak{c}$ and $\Card(\mathbb{N})=\aleph_{0}$.
            But $(0,1)\cap\mathbb{N}=\emptyset$, and thus
            $\aleph_{0}+\mathfrak{c}=\Card((0,1)\cup\mathbb{N})$.
            But $\mathbb{R}\sim(0,1)$ and $\mathbb{N}\cup(0,1)\subset\mathbb{R}$. By
            the Schr\"{o}der-Bernstein theorem, $\mathbb{N}\cup(0,1)\sim\mathbb{R}$.
            Therefore, etc.
        \end{proof}
        \begin{definition}
            The product of two cardinal numbers $a$ and $b$ is the cardinality of
            the cartesian product of two set $A$ and $B$ such that $\Card(A)=a$ and
            $\Card(B)=b$. That is, ${a}\times{b}=\Card({A}\times{B})$.
        \end{definition}
        \begin{theorem}
            The following are true of cardinal numbers:
            \begin{enumerate}
                \item $xy=yx$
                \item $x(yz)=(xy)z$
                \item $x(y+z)=xy+xz$
            \end{enumerate}
        \end{theorem}
        \begin{proof}[Proof of Part 3]
            Let $A,B,$ and $C$ be disjoint.
            Then
            ${A}\times{({B}\cup{C})}%
             =({A}\times{B})\cup({A}\times{C})$, and thus
            $\Card({A}\times{({B}\cup{C})})%
             =\Card(({A}\times{B})\cup({A}\times{C}))$.
            But ${A}\times{B}$ and ${A}\times{C}$ are disjoint.
            Thus we have
            $\Card(({A}\times{B})\cup({A}\times{C}))%
             =\Card({A}\times{B})+\Card({A}\times{C})$.
            Therefore, etc.
        \end{proof}
        \begin{theorem}
            If $\Card(T)=x$ and $F:{T}\rightarrow{\mathcal{P}(T)}$ is a
            set-valued mapping such that for all ${t}\in{T}$ we have that
            $\Card(F(t))=y$ and for all ${t}\ne{t}$,
            ${F(t)}\cap{F(t')}=\emptyset$, then $\Card(\cup_{t=1}^{N}F(t))=xy$
        \end{theorem}
        \begin{example}
            Let $f:{\mathbb{N}^{2}}\rightarrow{\mathbb{N}}$
            be defined by $f(n,m)=2^{n}3^{m}$.
            Then $f$ is injective, since $2$ and $3$
            are coprime. Therefore,
            $\aleph_{0}\times\aleph_{0}=\aleph_{0}$.
        \end{example}
        \begin{example}
            Show that $\mathbb{R}^{2}\sim\mathbb{R}$.
            Let $f:\mathbb{R}^{2}\rightarrow\mathbb{R}$
            be the rather bizarre function defined by the image
            $f(x_{0}.x_{1}x_{2}\hdots,y_{0}.y_{1}y_{2}\hdots)%
             =x_{0}y_{0}.x_{0}y_{0}x_{1}y_{1}\hdots$ Then
            $f$ is inective. But the mapping
            $g:\mathbb{R}\rightarrow\mathbb{R}^{2}$
            defined by $g(x)=(x,0)$ is also injective.
            By Schr\"{o}der-Bernstein,
            $\mathbb{R}^{2}\sim\mathbb{R}$.
        \end{example}
        \begin{definition}
           Order isomorphic set are two sets $A$ and $B$
           with well orders $<_{A}$ and $<_{B}$ such that
           there exists a bijection $f:{B}A\rightarrow{B}$
           such that for all $a_{1},a_{2}\in{A}$ such that
           $a_{1}<_{A}a_{2}$, $f(a_{1})<_{B}f(a_{2})$.
        \end{definition}
        \begin{theorem}
           Order-Isomorphism is an equivalence relation.
        \end{theorem}
        To every well ordered set, an ordinal number is
        assigned, denoted $\Ord(A,<_{A})$. Conversely,
        for every ordinal number there is a set with a
        well order corresponding to it. Two ordinal numbers
        are equal if and only if the well-ordered sets
        corresponding to them are order isomorphic.
        That is,
        $\Ord(A,<_{A})=\Ord(B,<_{B})$ if and only if
        $(A,<_{A})$ and $(B,<_{B})$ are order isomorphic.
        \begin{theorem}
           If $(A,<_{A})$ and $(B,<_{B})$ are well ordered
           sets, and if $\Card(A)=\Card(B)$, then
           $(A,<_{A})$ and $(B,<_{B})$ are order
           isomorphic.
        \end{theorem}
        The ordinal number of the empty set is $0$. The
        ordinal number of a finite set of $n$ elements with
        a well ordering is denoted $n\in\mathbb{N}$.
        The ordinal for the natural numbers $\mathbb{N}$
        with their usual well-ordering is denoted $\omega$.
        A given well-ordered set has only one cardinal number,
        but it is possible for it to have two ordinal numbers.
        \begin{definition}
            An ordinal number $\alpha$ is less than or equal to an ordinal
            number $\beta$ if there are well-ordered sets $(A,<_{A})$ and
            $(B,<_{B})$ such that $\alpha=\Ord((A,<_{A}))$ and
            $\beta=\Ord(B,<_{B})$, and $(A,<_{B})$ is order isomorphic to
            subset of $(B,<_{B})$.
        \end{definition}
        \begin{theorem}
            The only order isomorphism from a well ordered set $(A,<_{A})$ to
            itself is the identity isomorphism.
        \end{theorem}
        \begin{theorem}
            If $\alpha$ and $\beta$ are ordinal numbers and
            ${\alpha}\leq{\beta}$ and ${\beta}\leq{\alpha}$,
            then $\alpha=\beta$.
        \end{theorem}
        \begin{theorem}
            If $\alpha$ and $\beta$ are ordinal numbers, either
            ${\alpha}\leq{\beta}$, or ${\beta}\leq{\alpha}$.
        \end{theorem}
        \begin{theorem}
            If $\alpha$ and $\beta$ are ordinal numbers, either
            $\alpha<\beta$, $\beta<\alpha$, or $\alpha=\beta$.
        \end{theorem}
        \begin{definition}
            The total ordering relation of a well-ordered set $(A,<_{A})$
            with respect
           to a well-ordered set $(B,<_{B})$ is the ordering
           on the set ${A}\cup{B}$ defined as: For all
           $a_{1},a_{2}\in{A}$ such that $a_{1}<_{A}a_{2}$,
           $a_{1}<_{*}a_{2}$, for all $b_{1},b_{2}\in{B}$
           such that $b_{1}<_{B}b_{2}$, $b_{1}<_{*}b_{2}$,
           and for all ${a}\in{A}$ and ${b}\in{B}$,
           ${a}<_{*}{b}$.
        \end{definition}
        \begin{theorem}
           The total ordering relation $<_{*}$ on the set
           ${A}\cup{B}$ is a well-ordering.
        \end{theorem}
        \begin{definition}
            The ordinal sum of two ordinal numbers $\Ord((A,<_{A}))$ and
            $\Ord((B,<_{B}))$, where $A$ and $B$ are disjoint, is the ordinal
            number $\Ord(({A}\cup{B},<_{*}))$.
        \end{definition}
        \begin{theorem}
           The following are true of ordinal numbers:
           \begin{enumerate}
                \item $\alpha<\beta\Rightarrow\alpha+\gamma<\beta+\gamma$
                \item $(\alpha+\beta)+\gamma=\alpha+(\beta+\gamma)$
                \item $\alpha+\beta=\alpha+\gamma\Rightarrow\beta=\gamma$
           \end{enumerate}
        \end{theorem}
        \begin{definition}
            The lexicographic ordering on the cartesianproduct of well
            ordered set $(A,<_{A})$ and $(B,<_{B})$ is the ordering on
            ${A}\times{B}$ defined by: If ${a}<_{A}{x}$, then
            $(a,b)<_{*}(x,y)$ for all $b,y\in{B}$, and if $a=x$ and
            $b<_{B}y$, then $(a,b)<_{*}(x,y)$.
        \end{definition}
        \begin{theorem}
            If $(A,<_{A})$ and $(B,<_{B})$ are well ordered sets, then the
            lexicographic ordering on ${A}\times{B}$ is a well ordering.
        \end{theorem}
        \begin{definition}
            The ordinal product of two ordinal numbers
            $\Ord((A,<_{A}))$ and $\Ord((B,<_{B}))$,
            is $\Ord(({A}\times{B},<_{*}))$
        \end{definition}
        \begin{theorem}
            The following are true of ordinal numbers:
            \begin{enumerate}
                \item $\alpha(\beta\gamma)=(\alpha\beta)\gamma$
                \item $\alpha(\beta+\gamma)=\alpha\beta+\alpha\gamma$
            \end{enumerate}
        \end{theorem}
        \begin{definition}
           Relatively prime integers are integers
           $a,b\in\mathbb{N}$ such that $\gcd(a,b)=1$.
        \end{definition}
        \begin{theorem}
           If $p$ is prime and $a\in\mathbb{N}$ is
           such that $p$ does not divide $a$, then $a$ and $p$
           are relatively prime.
        \end{theorem}
        \begin{theorem}
           There are infinitely many prime numbers.
        \end{theorem}
        \begin{theorem}
           If $a\in\mathbb{N}$, $a>1$, then either
           $a$ is a prime number, or $a$ is the product
           of finitely many primes.
        \end{theorem}
        \begin{theorem}
           If $a\in\mathbb{N}$, $a>1$, and if $a$ is not
           prime, then the prime expansion of $a$ is
           unique.
        \end{theorem}
        \begin{definition}
           A diophantine equation is an equation whose
           solutions are required to be integers.
        \end{definition}
        \begin{definition}
           A linear diophantine equation in two variables
           $x$ and $y$ is an equation
           $ax+by=c$, where $a,b,c\in\mathbb{Z}$.
        \end{definition}
        \begin{theorem}
           If $a,b,c\in\mathbb{Z}$ $d=\gcd(a,b)$, and if $d$ does not
           divide $c$, then $ax+by=c$ has no integral solutions.
        \end{theorem}
        \begin{theorem}
           If $a,b,c\in\mathbb{Z}$ $d=\gcd(a,b)$, and if $d$ divides $c$,
           then $ax+by=c$ has infinitely many solutions.
        \end{theorem}
        \chapter{Cardinality}
    \chapter{Order Theory}
        \documentclass[crop=false,class=book,oneside]{standalone}                      %
%----------------------------------Preamble------------------------------------%
%---------------------------Packages----------------------------%
\usepackage{geometry}
\geometry{b5paper, margin=1.0in}
\usepackage[T1]{fontenc}
\usepackage{graphicx, float}            % Graphics/Images.
\usepackage{natbib}                     % For bibliographies.
\bibliographystyle{agsm}                % Bibliography style.
\usepackage[french, english]{babel}     % Language typesetting.
\usepackage[dvipsnames]{xcolor}         % Color names.
\usepackage{listings}                   % Verbatim-Like Tools.
\usepackage{mathtools, esint, mathrsfs} % amsmath and integrals.
\usepackage{amsthm, amsfonts, amssymb}  % Fonts and theorems.
\usepackage{tcolorbox}                  % Frames around theorems.
\usepackage{upgreek}                    % Non-Italic Greek.
\usepackage{fmtcount, etoolbox}         % For the \book{} command.
\usepackage[newparttoc]{titlesec}       % Formatting chapter, etc.
\usepackage{titletoc}                   % Allows \book in toc.
\usepackage[nottoc]{tocbibind}          % Bibliography in toc.
\usepackage[titles]{tocloft}            % ToC formatting.
\usepackage{pgfplots, tikz}             % Drawing/graphing tools.
\usepackage{imakeidx}                   % Used for index.
\usetikzlibrary{
    calc,                   % Calculating right angles and more.
    angles,                 % Drawing angles within triangles.
    arrows.meta,            % Latex and Stealth arrows.
    quotes,                 % Adding labels to angles.
    positioning,            % Relative positioning of nodes.
    decorations.markings,   % Adding arrows in the middle of a line.
    patterns,
    arrows
}                                       % Libraries for tikz.
\pgfplotsset{compat=1.9}                % Version of pgfplots.
\usepackage[font=scriptsize,
            labelformat=simple,
            labelsep=colon]{subcaption} % Subfigure captions.
\usepackage[font={scriptsize},
            hypcap=true,
            labelsep=colon]{caption}    % Figure captions.
\usepackage[pdftex,
            pdfauthor={Ryan Maguire},
            pdftitle={Mathematics and Physics},
            pdfsubject={Mathematics, Physics, Science},
            pdfkeywords={Mathematics, Physics, Computer Science, Biology},
            pdfproducer={LaTeX},
            pdfcreator={pdflatex}]{hyperref}
\hypersetup{
    colorlinks=true,
    linkcolor=blue,
    filecolor=magenta,
    urlcolor=Cerulean,
    citecolor=SkyBlue
}                           % Colors for hyperref.
\usepackage[toc,acronym,nogroupskip,nopostdot]{glossaries}
\usepackage{glossary-mcols}
%------------------------Theorem Styles-------------------------%
\theoremstyle{plain}
\newtheorem{theorem}{Theorem}[section]

% Define theorem style for default spacing and normal font.
\newtheoremstyle{normal}
    {\topsep}               % Amount of space above the theorem.
    {\topsep}               % Amount of space below the theorem.
    {}                      % Font used for body of theorem.
    {}                      % Measure of space to indent.
    {\bfseries}             % Font of the header of the theorem.
    {}                      % Punctuation between head and body.
    {.5em}                  % Space after theorem head.
    {}

% Italic header environment.
\newtheoremstyle{thmit}{\topsep}{\topsep}{}{}{\itshape}{}{0.5em}{}

% Define environments with italic headers.
\theoremstyle{thmit}
\newtheorem*{solution}{Solution}

% Define default environments.
\theoremstyle{normal}
\newtheorem{example}{Example}[section]
\newtheorem{definition}{Definition}[section]
\newtheorem{problem}{Problem}[section]

% Define framed environment.
\tcbuselibrary{most}
\newtcbtheorem[use counter*=theorem]{ftheorem}{Theorem}{%
    before=\par\vspace{2ex},
    boxsep=0.5\topsep,
    after=\par\vspace{2ex},
    colback=green!5,
    colframe=green!35!black,
    fonttitle=\bfseries\upshape%
}{thm}

\newtcbtheorem[auto counter, number within=section]{faxiom}{Axiom}{%
    before=\par\vspace{2ex},
    boxsep=0.5\topsep,
    after=\par\vspace{2ex},
    colback=Apricot!5,
    colframe=Apricot!35!black,
    fonttitle=\bfseries\upshape%
}{ax}

\newtcbtheorem[use counter*=definition]{fdefinition}{Definition}{%
    before=\par\vspace{2ex},
    boxsep=0.5\topsep,
    after=\par\vspace{2ex},
    colback=blue!5!white,
    colframe=blue!75!black,
    fonttitle=\bfseries\upshape%
}{def}

\newtcbtheorem[use counter*=example]{fexample}{Example}{%
    before=\par\vspace{2ex},
    boxsep=0.5\topsep,
    after=\par\vspace{2ex},
    colback=red!5!white,
    colframe=red!75!black,
    fonttitle=\bfseries\upshape%
}{ex}

\newtcbtheorem[auto counter, number within=section]{fnotation}{Notation}{%
    before=\par\vspace{2ex},
    boxsep=0.5\topsep,
    after=\par\vspace{2ex},
    colback=SeaGreen!5!white,
    colframe=SeaGreen!75!black,
    fonttitle=\bfseries\upshape%
}{not}

\newtcbtheorem[use counter*=remark]{fremark}{Remark}{%
    fonttitle=\bfseries\upshape,
    colback=Goldenrod!5!white,
    colframe=Goldenrod!75!black}{ex}

\newenvironment{bproof}{\textit{Proof.}}{\hfill$\square$}
\tcolorboxenvironment{bproof}{%
    blanker,
    breakable,
    left=3mm,
    before skip=5pt,
    after skip=10pt,
    borderline west={0.6mm}{0pt}{green!80!black}
}

\AtEndEnvironment{lexample}{$\hfill\textcolor{red}{\blacksquare}$}
\newtcbtheorem[use counter*=example]{lexample}{Example}{%
    empty,
    title={Example~\theexample},
    boxed title style={%
        empty,
        size=minimal,
        toprule=2pt,
        top=0.5\topsep,
    },
    coltitle=red,
    fonttitle=\bfseries,
    parbox=false,
    boxsep=0pt,
    before=\par\vspace{2ex},
    left=0pt,
    right=0pt,
    top=3ex,
    bottom=1ex,
    before=\par\vspace{2ex},
    after=\par\vspace{2ex},
    breakable,
    pad at break*=0mm,
    vfill before first,
    overlay unbroken={%
        \draw[red, line width=2pt]
            ([yshift=-1.2ex]title.south-|frame.west) to
            ([yshift=-1.2ex]title.south-|frame.east);
        },
    overlay first={%
        \draw[red, line width=2pt]
            ([yshift=-1.2ex]title.south-|frame.west) to
            ([yshift=-1.2ex]title.south-|frame.east);
    },
}{ex}

\AtEndEnvironment{ldefinition}{$\hfill\textcolor{Blue}{\blacksquare}$}
\newtcbtheorem[use counter*=definition]{ldefinition}{Definition}{%
    empty,
    title={Definition~\thedefinition:~{#1}},
    boxed title style={%
        empty,
        size=minimal,
        toprule=2pt,
        top=0.5\topsep,
    },
    coltitle=Blue,
    fonttitle=\bfseries,
    parbox=false,
    boxsep=0pt,
    before=\par\vspace{2ex},
    left=0pt,
    right=0pt,
    top=3ex,
    bottom=0pt,
    before=\par\vspace{2ex},
    after=\par\vspace{1ex},
    breakable,
    pad at break*=0mm,
    vfill before first,
    overlay unbroken={%
        \draw[Blue, line width=2pt]
            ([yshift=-1.2ex]title.south-|frame.west) to
            ([yshift=-1.2ex]title.south-|frame.east);
        },
    overlay first={%
        \draw[Blue, line width=2pt]
            ([yshift=-1.2ex]title.south-|frame.west) to
            ([yshift=-1.2ex]title.south-|frame.east);
    },
}{def}

\AtEndEnvironment{ltheorem}{$\hfill\textcolor{Green}{\blacksquare}$}
\newtcbtheorem[use counter*=theorem]{ltheorem}{Theorem}{%
    empty,
    title={Theorem~\thetheorem:~{#1}},
    boxed title style={%
        empty,
        size=minimal,
        toprule=2pt,
        top=0.5\topsep,
    },
    coltitle=Green,
    fonttitle=\bfseries,
    parbox=false,
    boxsep=0pt,
    before=\par\vspace{2ex},
    left=0pt,
    right=0pt,
    top=3ex,
    bottom=-1.5ex,
    breakable,
    pad at break*=0mm,
    vfill before first,
    overlay unbroken={%
        \draw[Green, line width=2pt]
            ([yshift=-1.2ex]title.south-|frame.west) to
            ([yshift=-1.2ex]title.south-|frame.east);},
    overlay first={%
        \draw[Green, line width=2pt]
            ([yshift=-1.2ex]title.south-|frame.west) to
            ([yshift=-1.2ex]title.south-|frame.east);
    }
}{thm}

%--------------------Declared Math Operators--------------------%
\DeclareMathOperator{\adjoint}{adj}         % Adjoint.
\DeclareMathOperator{\Card}{Card}           % Cardinality.
\DeclareMathOperator{\curl}{curl}           % Curl.
\DeclareMathOperator{\diam}{diam}           % Diameter.
\DeclareMathOperator{\dist}{dist}           % Distance.
\DeclareMathOperator{\Div}{div}             % Divergence.
\DeclareMathOperator{\Erf}{Erf}             % Error Function.
\DeclareMathOperator{\Erfc}{Erfc}           % Complementary Error Function.
\DeclareMathOperator{\Ext}{Ext}             % Exterior.
\DeclareMathOperator{\GCD}{GCD}             % Greatest common denominator.
\DeclareMathOperator{\grad}{grad}           % Gradient
\DeclareMathOperator{\Ima}{Im}              % Image.
\DeclareMathOperator{\Int}{Int}             % Interior.
\DeclareMathOperator{\LC}{LC}               % Leading coefficient.
\DeclareMathOperator{\LCM}{LCM}             % Least common multiple.
\DeclareMathOperator{\LM}{LM}               % Leading monomial.
\DeclareMathOperator{\LT}{LT}               % Leading term.
\DeclareMathOperator{\Mod}{mod}             % Modulus.
\DeclareMathOperator{\Mon}{Mon}             % Monomial.
\DeclareMathOperator{\multideg}{mutlideg}   % Multi-Degree (Graphs).
\DeclareMathOperator{\nul}{nul}             % Null space of operator.
\DeclareMathOperator{\Ord}{Ord}             % Ordinal of ordered set.
\DeclareMathOperator{\Prin}{Prin}           % Principal value.
\DeclareMathOperator{\proj}{proj}           % Projection.
\DeclareMathOperator{\Refl}{Refl}           % Reflection operator.
\DeclareMathOperator{\rk}{rk}               % Rank of operator.
\DeclareMathOperator{\sgn}{sgn}             % Sign of a number.
\DeclareMathOperator{\sinc}{sinc}           % Sinc function.
\DeclareMathOperator{\Span}{Span}           % Span of a set.
\DeclareMathOperator{\Spec}{Spec}           % Spectrum.
\DeclareMathOperator{\supp}{supp}           % Support
\DeclareMathOperator{\Tr}{Tr}               % Trace of matrix.
%--------------------Declared Math Symbols--------------------%
\DeclareMathSymbol{\minus}{\mathbin}{AMSa}{"39} % Unary minus sign.
%------------------------New Commands---------------------------%
\DeclarePairedDelimiter\norm{\lVert}{\rVert}
\DeclarePairedDelimiter\ceil{\lceil}{\rceil}
\DeclarePairedDelimiter\floor{\lfloor}{\rfloor}
\newcommand*\diff{\mathop{}\!\mathrm{d}}
\newcommand*\Diff[1]{\mathop{}\!\mathrm{d^#1}}
\renewcommand*{\glstextformat}[1]{\textcolor{RoyalBlue}{#1}}
\renewcommand{\glsnamefont}[1]{\textbf{#1}}
\renewcommand\labelitemii{$\circ$}
\renewcommand\thesubfigure{%
    \arabic{chapter}.\arabic{figure}.\arabic{subfigure}}
\addto\captionsenglish{\renewcommand{\figurename}{Fig.}}
\numberwithin{equation}{section}

\renewcommand{\vector}[1]{\boldsymbol{\mathrm{#1}}}

\newcommand{\uvector}[1]{\boldsymbol{\hat{\mathrm{#1}}}}
\newcommand{\topspace}[2][]{(#2,\tau_{#1})}
\newcommand{\measurespace}[2][]{(#2,\varSigma_{#1},\mu_{#1})}
\newcommand{\measurablespace}[2][]{(#2,\varSigma_{#1})}
\newcommand{\manifold}[2][]{(#2,\tau_{#1},\mathcal{A}_{#1})}
\newcommand{\tanspace}[2]{T_{#1}{#2}}
\newcommand{\cotanspace}[2]{T_{#1}^{*}{#2}}
\newcommand{\Ckspace}[3][\mathbb{R}]{C^{#2}(#3,#1)}
\newcommand{\funcspace}[2][\mathbb{R}]{\mathcal{F}(#2,#1)}
\newcommand{\smoothvecf}[1]{\mathfrak{X}(#1)}
\newcommand{\smoothonef}[1]{\mathfrak{X}^{*}(#1)}
\newcommand{\bracket}[2]{[#1,#2]}

%------------------------Book Command---------------------------%
\makeatletter
\renewcommand\@pnumwidth{1cm}
\newcounter{book}
\renewcommand\thebook{\@Roman\c@book}
\newcommand\book{%
    \if@openright
        \cleardoublepage
    \else
        \clearpage
    \fi
    \thispagestyle{plain}%
    \if@twocolumn
        \onecolumn
        \@tempswatrue
    \else
        \@tempswafalse
    \fi
    \null\vfil
    \secdef\@book\@sbook
}
\def\@book[#1]#2{%
    \refstepcounter{book}
    \addcontentsline{toc}{book}{\bookname\ \thebook:\hspace{1em}#1}
    \markboth{}{}
    {\centering
     \interlinepenalty\@M
     \normalfont
     \huge\bfseries\bookname\nobreakspace\thebook
     \par
     \vskip 20\p@
     \Huge\bfseries#2\par}%
    \@endbook}
\def\@sbook#1{%
    {\centering
     \interlinepenalty \@M
     \normalfont
     \Huge\bfseries#1\par}%
    \@endbook}
\def\@endbook{
    \vfil\newpage
        \if@twoside
            \if@openright
                \null
                \thispagestyle{empty}%
                \newpage
            \fi
        \fi
        \if@tempswa
            \twocolumn
        \fi
}
\newcommand*\l@book[2]{%
    \ifnum\c@tocdepth >-3\relax
        \addpenalty{-\@highpenalty}%
        \addvspace{2.25em\@plus\p@}%
        \setlength\@tempdima{3em}%
        \begingroup
            \parindent\z@\rightskip\@pnumwidth
            \parfillskip -\@pnumwidth
            {
                \leavevmode
                \Large\bfseries#1\hfill\hb@xt@\@pnumwidth{\hss#2}
            }
            \par
            \nobreak
            \global\@nobreaktrue
            \everypar{\global\@nobreakfalse\everypar{}}%
        \endgroup
    \fi}
\newcommand\bookname{Book}
\renewcommand{\thebook}{\texorpdfstring{\Numberstring{book}}{book}}
\providecommand*{\toclevel@book}{-2}
\makeatother
\titleformat{\part}[display]
    {\Large\bfseries}
    {\partname\nobreakspace\thepart}
    {0mm}
    {\Huge\bfseries}
\titlecontents{part}[0pt]
    {\large\bfseries}
    {\partname\ \thecontentslabel: \quad}
    {}
    {\hfill\contentspage}
\titlecontents{chapter}[0pt]
    {\bfseries}
    {\chaptername\ \thecontentslabel:\quad}
    {}
    {\hfill\contentspage}
\newglossarystyle{longpara}{%
    \setglossarystyle{long}%
    \renewenvironment{theglossary}{%
        \begin{longtable}[l]{{p{0.25\hsize}p{0.65\hsize}}}
    }{\end{longtable}}%
    \renewcommand{\glossentry}[2]{%
        \glstarget{##1}{\glossentryname{##1}}%
        &\glossentrydesc{##1}{~##2.}
        \tabularnewline%
        \tabularnewline
    }%
}
\newglossary[not-glg]{notation}{not-gls}{not-glo}{Notation}
\newcommand*{\newnotation}[4][]{%
    \newglossaryentry{#2}{type=notation, name={\textbf{#3}, },
                          text={#4}, description={#4},#1}%
}
%--------------------------LENGTHS------------------------------%
% Spacings for the Table of Contents.
\addtolength{\cftsecnumwidth}{1ex}
\addtolength{\cftsubsecindent}{1ex}
\addtolength{\cftsubsecnumwidth}{1ex}
\addtolength{\cftfignumwidth}{1ex}
\addtolength{\cfttabnumwidth}{1ex}

% Indent and paragraph spacing.
\setlength{\parindent}{0em}
\setlength{\parskip}{0em}                                                           %
%---------------------------------tikz Path------------------------------------%
\makeatletter                                                                  %
    \def\input@path{{../../../tikz/}}                                          %
\makeatother                                                                   %
%----------------------------------GLOSSARY------------------------------------%
\makeglossaries                                                                %
\loadglsentries{glossary}                                                      %
\loadglsentries{acronym}                                                       %
%--------------------------------Main Document---------------------------------%
\begin{document}
    \ifx\ifmain\undefined
        \pagenumbering{roman}
        \title{Graph Theory}
        \author{Ryan Maguire}
        \date{\vspace{-5ex}}
        \maketitle
        \tableofcontents
        \clearpage
        \chapter*{Graph Theory}
        \addcontentsline{toc}{chapter}{Graph Theory}
        \markboth{}{GRAPH THEORY}
        \vspace{10ex}
        \setcounter{chapter}{1}
        \pagenumbering{arabic}
    \else
        \chapter{Graph Theory}
    \fi
    \section{Graph Theory I}
        \begin{definition}
            A graph is a set of points called vertices and a set
            of lines called edges that connect pairs of vertices.
            The vertex set is denoted $V(G)$, and the edge
            set is denoted $E(G)$.
        \end{definition}
        \begin{definition}
            If $u$ and $v$ are edges of a set, and if there
            is an edge connecting them, then they
            are said to be adjacent. 
        \end{definition}
        \begin{definition}
            A vertex that has no adjacent
            vertices is called isolated.
        \end{definition}
        \begin{definition}
            The edge connecting vertices $u$ and
            $v$ is said to be incident with them.
        \end{definition}
        \begin{definition}
            The size of $V(G)$ is written as $|V(G)|$, and
            is the number of vertices in the graph.
            Similarly with $|E(G)|$.
        \end{definition}
        \begin{definition}
            If $G$ is a graph, then a subgraph $G'$ of
            $G$ is a subset of vertices $V(G')\subset{V(G)}$
            together with a subset of edges
            $E(G')\subset{E(G)}$, such that $V(G')$
            and $E(G')$ are themselves a graph.
        \end{definition}
        \begin{definition}
            Given a graph $G$ with vertices $V(G)$ and
            edges $V(G)$, the complimentary graph is the
            graph with vertices $V(G)$ and edges $E(V)^C$
            consisting of all of the edges not contained
            in $E(V)$. The complimentary graph
            is denoted $G^C$.
        \end{definition}
        \begin{theorem}
            For any graph $G$, $(G^C)^{C}=G$.
        \end{theorem}
        \begin{proof}
            For let $G$ be a graph and consider $G^C$.
            From the definition, either an edge is contained
            in $G$ or it is contained in $G^C$. If the edge
            is contained in $G^C$, then it is not contained
            in $(G^C)^C$, and vice versa. But if an edge is
            not contained in $G^C$, then it is contained in
            $G$, and thus edges in $(G^C)^C$ are contained in
            $G$. Similarly, if an edge is contained in $G$, then
            it is not contained in $G^C$, and thus it is contained
            in $(G^C)^C$. Thus all edges in $G$ are contained
            in $(G^C)^C$. But it was just proved that all edges
            in $(G^C)^{C}$ are contained in $G$.
            Thus $(G^C)^{C}=G$.
        \end{proof}
        \begin{definition}
        The number of edges incident with a vertex is called the degree of that vertex. The degree of a vertex $v$ is denoted $\deg(v)$.
        \end{definition}
        \begin{theorem}
        For any finite graph $G$, $\sum_{v\in V(G)} \deg(v) = 2|E(G)|$.
        \end{theorem}
        \begin{proof}
        We prove by induction. If there are zero edges, then the degree of each vertex is zero and $\sum_{v\in V(G)}\deg(v) = 0 = 2|E(G)|$, as $|E(G)| = 0$. If there is only one edge, suppose incident on the vertices $u$ and $v$, then $\deg(u) = \deg(v) = 1$, and thus $\sum_{v\in V(G)} \deg(v) = 1+1 = 2 = 2|E(G)|$, as $|E(G)| = 1$. Now suppose that if there are $n$ edges then $\sum_{v\in V(G)}\deg(v) = 2|E(G)|$. It suffices to show that this implies that if there $n+1$ edges this result remains valid. Let $G$ be a graph with $|E(G)| = n+1$. Let $uv$ be some edge in $E(V)$ incident to the vertices $u$ and $v$, and let $G'$ be the subgraph with vertices $V(G') = V(G)$ and edges $E(G')=E(G)\setminus \{uv\}$. Then $|E(G')| = n$, and therefore $\sum_{v\in V(G')}\deg(v) = 2n$. But $\sum_{v\in V(G)} \deg(v) - \sum_{v\in V(G')}\deg(v) = 2$, as the vertices $u$ and $v$ each have one more degree in $G$ than in $G'$, and all other vertices are the same. Thus $\sum_{v\in V(G)}\deg(v) = 2 + \sum_{v\in V(G')}\deg(v) = 2(n+1) = 2|E(G)|$.
        \end{proof}
        Euler's proof of the same theorem is as follows:
        \begin{theorem}[Handshaking Theorem]
        For every finite graph $G$, $\sum_{v\in V(G)}\deg(v) = 2|E(V)|$.
        \end{theorem}
        \begin{proof}
        If $|E(V)| = 0$, then no two vertices are adjacent and thus $\sum_{v\in V(G)}\deg(v) = 0$. Otherwise, each edge is incident on exactly two vertices. As the degree of a vertex is the number of edges which are incident on it, we see that each edge contributes to the degree of 2 vertices. Thus, adding up all of degrees of the vertices is just twice the total number of edges.
        \end{proof}
        \begin{corollary}
        For any finite graph $G$, $\sum_{v\in V(G)}\deg(v)$ is even.
        \end{corollary}
        \begin{proof} As $\sum_{v\in V(G)}\deg(v) = 2|E(G)|$, we have that the sum is twice the value of an integer, and thus is even.
        \end{proof}
        \begin{corollary}
        The number of vertices with an odd degree is even.
        \end{corollary}
        \begin{proof}
        For suppose not. Suppose there is an odd number of vertices with odd degree. Let $V_0(G)$ be the vertices with even degree, and $V_1(G)$ be the vertices with odd degree. But $\sum_{v\in V_0(G)}\deg(v) + \sum_{v\in V_1(G)} \deg(v) = \sum_{v\in V(G)}\deg(v) = 2|E(G)|$, which is even. But we have that $\sum_{v\in V_0(G)}\deg(v)$ is even, as for all $v\in V_{0}(G)$, $\deg(v)$ is even. Thus $\sum_{v\in V(G)}\deg(v) - \sum_{v\in V_0(G)}\deg(v)$ is even. But $\sum_{v\in V_1(G)}\deg(v) = \sum_{v\in V(G)}\deg(v) - \sum_{v\in V_0(G)}\deg(v)$, and is therefore even. But for all $v\in V_1(G)$, $\deg(v)$ is odd. And the sum of and odd number of odd numbers is odd. Thus, there cannot be an odd number of vertices with odd degree. Therefore there is an even number of such vertices.
        \end{proof}
        \begin{definition}
        A vertex with an odd degree is called odd, a vertex with even degree is called even.
        \end{definition}
        \begin{definition}
        Two graphs $G$ and $H$ are said to be isomorphic if and only if there is a bijective function $f:G\rightarrow H$ such that $\{u,v\}\in E(G)$ if and only if $\{f(u),f(v)\}\in E(H)$. We write $G \cong H$.
        \end{definition}
        \begin{lemma}
        If $G$ is a graph, and $w\in V(G)$, then the function $\chi_{\{v,w\}}^G = \begin{cases} 0 & \{v,w\} \notin V(G) \\ 1 & \{v,w\} \in V(G) \end{cases}$ satisfies $\deg(v) = \sum_{\underset{v\ne w}{w\in V(G)}} \chi_{\{v,w\}}^G$
        \end{lemma}
        \begin{proof}
        For $\deg(v)$ is the number of edges incident to it. As an edge is incident to exactly two vertices, for each edge connected to $v$ there is a $w\in V(G)$ such that $\{v,w\} \in E(G)$. Moreover, there are $\deg(v)$ vertices connected to it. Thus we have $\sum_{\underset{\{v,w\}\in E(G)}{w\in V(G)}} \chi_{\{v,w\}}^G = \sum_{\underset{\{v,w\}\in E(G)}{w\in V(G)}}1 = \deg(v)$. And finally $\sum_{\underset{v\ne w}{w\in V(G)}}\chi_{\{v,w\}}^G = \sum_{\underset{\{v,w\}\in E(G)}{w\in V(G)}}\chi_{\{v,w\}}^G+\sum_{\underset{\{v,w\}\notin E(G)}{w\in V(G)}}\chi_{\{v,w\}}^G = \deg(v) + 0 = \deg(v)$.
        \end{proof}
        \begin{theorem}
        The vertices of isomorphic graphs have the same degree.
        \end{theorem}
        \begin{proof}
        For let $G$ and $H$ be isomorphic with isomorphism $f$ and define $\chi_{\{v,w\}}^G = \begin{cases} 0 & \{v,w\} \notin V(G) \\ 1 & \{v,w\} \in V(G) \end{cases}$. Then $\deg(v) = \sum_{\underset{v\ne w}{w\in V(G)}}\chi_{\{v,w\}}^G =\sum_{\underset{v\ne w}{w\in V(G)}}\chi_{\{f(v),f(w)\}}^H = \deg(f(v))$. Thus, $\deg(v) = \deg(f(v))$.
        \end{proof}
        \begin{lemma}
        Define  $\chi_{\{v,w\}}^G = \begin{cases} 0 & \{v,w\} \notin V(G) \\ 1 & \{v,w\} \in V(G) \end{cases}$. Then $\chi_{\{v,w\}}^{G^C} = |1-\chi_{\{v,w\}}^G|$
        \end{lemma}
        \begin{proof}
        From the definition, $\chi_{\{v,w\}}^{G^C} = \begin{cases} 0 & \{v,w\} \notin V(G^C) \\ 1 & \{v,w\} \in V(G^C) \end{cases}$. But if $\{v,w\} \in V({G^C})$, then $\{v,w\}\notin V(G)$, and thus $\chi_{\{v,w\}}^G = 0$, $\chi_{\{v,w\}}^{G^C} = 1$. By reversing this argument we set that $\chi_{\{v,w\}}^{G} = 1$ when $\chi_{\{v,w\}}^{G^C} = 0$. The result immediately follows.
        \end{proof}
        \begin{corollary}
        If $G$ is a graph with $n$ vertices and $v$ is a vertex, then the sum of the degree of $v$ in $G$ and $G^C$ is $n-1$.
        \end{corollary}
        \begin{proof}
        This is because $\deg(v)_G+\deg(v)_{G^C} = \sum_{\underset{v\ne w}w\in V(G)}\chi_{\{v,w\}}^G + \sum_{\underset{v\ne w}{w\in V(G^C)}}\chi_{\{v,w\}}^{G^C} = \sum_{\underset{v\ne w}{w\in V(G)}} 1$. As there are $n-1$ elements of $V(G)$ not equal to $v$, we have that the sum is $n-1$.
        \end{proof}
        \begin{theorem}
        Graphs $G$ and $H$ are isomorphic if and only if $G^C$ and $H^C$ are isomorphic.
        \end{theorem}
        \begin{proof}
        For let $f:G\rightarrow H$ be an isomorphic function and suppose $\{v,w\}\notin E(G)$. Then, as $f$ is an isomorphism from $G$ to $H$, $\{f(v),f(w)\}\notin E(H)$. Thus, $\{v,w\}\in G^C$ and $\{f(v),f(w)\}\in H^C$. Similarly for any pair $\{V,W\}\in E(H^C)$, $\{f^{-1}(V),f^{-1}(W)\} \in E(G^C)$. Thus, $f^{-1}:G^C \rightarrow H^C$ is an isomorphism. Therefore if $G\cong H$, then $G^C \cong H^C$. By an identical argument, the converse is true.
        \end{proof}
        \begin{definition}
        If all pairs of vertices are adjacent, then the graph is called complete. The complete graph of $n$ vertices is denoted $K_n$. $K_1$ is called the trivial graph of one point.
        \end{definition}
        \begin{definition}
        The empty graph on $n$ vertices is the graph in which $|V(G)| = n$ and $|E(G)| = 0$. That is, it is the graph that has no connections and no two vertices are adjacent.
        \end{definition}
        \begin{corollary}
        For any $n\in \mathbb{N}$, $(K_n)^C$ is the empty graph on $n$ vertices.
        \end{corollary}
        \begin{proof}
        For $K_n$ contains all possible edges, and therefore $(K_n)^C$ contains no edges. But this is merely the empty graph on $n$ vertices.
        \end{proof}
        \begin{corollary}
        The degree of any vertex of a $K_n$ graph is $n-1$.
        \end{corollary}
        \begin{proof}
        For $\deg(v)_{K_n} + \deg(v)_{K_n^c}=n-1$ and $\deg(v)_{K_n^c} = 0$. Therefore, etc.
        \end{proof}
        \begin{corollary}
        For any graph $G$ of $n$ vertices, $|E(G)|\leq \frac{n^2-n}{2}$.
        \end{corollary}
        \begin{proof}
        For suppose $G$ is a complete graph $K_n$. Then $\sum_{v\in V(G)}\deg(v) = n(n-1) = 2|E(G)| \Rightarrow |E(G)| = \frac{n^2-n}{2}$. If $G$ is not the complete graph $K_n$, then it has fewer edges than $K_n$ and the $|E(G)| <|E(K_n)|= \frac{n^2-n}{2}$. Thus, for and arbitrary graph of $n$ elements, $|E(G)|\leq \frac{n^2-n}{2}$.
        \end{proof}
        \begin{definition}
        A graph is called regular if all of its vertices have the same degree. If the common degree is $k$, $G$ is called $k$-regular.
        \end{definition}
        \begin{corollary}
        $G$ is regular if and only if $G^C$ is regular. 
        \end{corollary}
        \begin{proof}
        For let $G$ be $k$-regular and suppose $G$ has $n$ vertices. Then given $v$ in $G^C$, it must have $n-1-k$ edges as $\deg(v)_G + \deg(v)_{G^C} = n-1$. But as $v$ is arbitrary, all vertices must have $n-1-k$ edges. Thus $G^C$ is $n-k-1$ regular
        \end{proof}
        \begin{corollary}
        $k$-regular graphs of odd degree have an even number of vertices.
        \end{corollary}
        \begin{proof}
        Suppose not and let $G$ be a $k$-regular graph with $n$ vertices, where $k$ and $n$ are odd. Then $\sum_{v\in V(G)}\deg(v) = 2|E(G)|$ from the handshaking theorem, and is thus even. But if $\sum_{v\in V(G)}\deg(v) = n\cdot k$, which is odd. A contradiction.
        \end{proof}
        \begin{definition}
        The degree sequence of a graph $G$ is a sequence of degrees of vertices of the graph in descending order.
        \end{definition}
        \begin{theorem}
        The degree sequence of a graph $G$ must contain a repeated degree.
        \end{theorem}
        \begin{proof}
        For suppose $G$ has $n$ vertices. Then $\max\{\deg(v)\} = n-1$. Thus, the range of degrees is $0$ to $n-1$. If there is no vertex with degree zero, then there are $n$ vertex and $n-1$ possible degrees, and therefore there must be at least one repeat. If there is a vertex with degree zero, then none of the vertices can have degree $n-1$. Thus the range of the vertices is $0$ to $n-2$. Thus there must be a repeat. 
        \end{proof}
        \begin{theorem}
        There exists graphs with the same degree sequence that are not
        isomorphic.
        \end{theorem}
        \begin{proof}
        For let $G$ be the graph consisting of two triangles, with no edge joining either of them. That is, $u_1\rightarrow u_2 \rightarrow u_3\rightarrow u_1$, and $u_4\rightarrow u_5 \rightarrow u_6 \rightarrow u_4$. The degree sequence is $2,2,2,2,2,2$. Take the hexagon as $H$, $v_1\rightarrow v_2 \rightarrow \hdots \rightarrow v_6 \rightarrow v_1$. $H$ and $G$ cannot be isomorphic for if they were we have that there is an edge between the two triangles in $G$. As no such edge exists, there is no isomorphism.
        \end{proof}
        \begin{definition}
        A $u-v$ walk is a sequence of vertices and edges beginning at $u$ and ending at $v$ such that vertices and edges alternate and each edge is incident with the vertices preceding and following it. We denote a walk by the vertices, as the edges are implied.
        \end{definition}
        \begin{definition}
        A closed walk begins and ends at the same vertex.
        \end{definition}
        \begin{definition}
        The length of a walk is the number of edges traversed. If an edge is traversed more than once, each time is counted.
        \end{definition}
        \begin{definition}
        A trail is a walk in which no edge is traversed more than once.
        \end{definition}
        \begin{definition} A circuit is a closed trail.
        \end{definition}
        \begin{definition}
        A path is a trail in which no vertex is repeated.
        \end{definition}
        \begin{definition}
        The shortest $u-v$ path is called a geodesic.
        \end{definition}
        \begin{definition}
        The length of a geodesic from $u$ to $v$ is called the distance between $u$ and $v$. It is denoted $d(u,v)$.
        \end{definition}
        \begin{definition}
        A graph that contains a pair of vertices with no path between them is called disconnected.
        \end{definition}
        \begin{definition}
        A graph that is not disconnected is called connected.
        \end{definition}
        \begin{corollary}
        If $G$ is a connected graph and $u$ and $v$ are points in $G$, then there is a path between them.
        \end{corollary}
        \begin{proof}
        For suppose not. But if there is no such path, then $G$ is disconnected, a contradiction. Therefore, etc.
        \end{proof}
        \begin{definition}
        A connected component of a graph is a subgraph such that every pair of vertices has a path between them.
        \end{definition}
        \begin{theorem}
        If $G$ is a graph, then either $G$ is connected or $G^C$ is connected, or both. 
        \end{theorem}
        \begin{proof}
        If both $G$ and $G^C$ are connected, we are done. Suppose $G$ is not connected. Then there are two elements $u$ and $v$ such that there is no path from $u$ to $v$. Then $\{u,v\}\in E(G^C)$. Let $r$ be any arbitrary point in $G$. Now either there is a path from $r$ to $v$, there is a path from $r$ to $u$, or there is a path to neither. There can not be a path to both as this would imply $u$ and $v$ are connected, but they are not. Thus in $G^C$ there must be an edge from $u$ to $r$ or $v$ to $r$, and as $u$ and $v$ are connected in $G^C$, $r$ must be connected to both vertices as well. But $r$ is arbitrary. Thus $G^C$ is connected.
        \end{proof}
        \begin{definition}
        A path on $n$ vertices is denoted $P_n$.
        \end{definition}
        \begin{corollary}
        If $G = P_n$, for some $n\in \mathbb{N}$, then the longest geodesic has length $n-1$.
        \end{corollary}
        \begin{proof}
        Let $G=P_n$ with vertices $v_k$, $k=1,2,\hdots, n$ and edges $\{v_k,v_{k+1}\}$, for $k=1,2,\hdots n-1$. First we show that such a geodesic starts and ends at the endpoints of $P_n$. For suppose not, and let such a geodesic begin at $v_k$. As paths retrace neither edges nor vertices, the next vertex is either $v_{k+1}$ or $v_{k-1}$. Suppose it is $v_{k+1}$. Suppose this geodesic terminates at $v_{k+N} \ne v_{n}$. But then the geodesic from $v_{k}$ to $v_{n}$ is longer than $v_{k+N}$. And similarly the geodesic from $v_1$ to $v_n$ is longer the geodesic from $v_k$ to $v_n$. Thus, the longest geodesic starts and ends at the endpoints of $P_n$. Next we compute the length of this geodesic. But as there are $n$ vertices, there are $n-1$ edges from $v_1$ to $v_n$, and thus the length is $n-1$.
        \end{proof}
        \begin{definition}
        The diameter of a connected graph is the length of the longest geodesic. That is, if $G$ is a connected graph, then the diameter of $G$ is $\max\{d(u,v): u,v\in V(G)\}$. The diameter is denoted $d(G)$.
        \end{definition}
        \begin{corollary}
        The two definitions of the previous definition are equivalent.
        \end{corollary}
        \begin{proof}
        For let $G$ be a connected graph and suppose the length of the longest geodesic is $d(G)\in \mathbb{N}$. Then for any such $u,v\in V(G)$, $d(u,v) \leq d(G)$, for if $d(u,v)> d(G)$ then the geodesic from $u$ to $v$ would be longer than $d(G)$, a contradiction. Thus the diameter is $\max\{d(u,v):u,v\in V(G)\}$. Now suppose $d(G) = \max\{d(u,v):u,v \in V(G)\}$. Suppose there is a geodesic from points $p$ and $q$ that is longer than $d(G)$. But then $d(p,q)>\max\{d(u,v):u,v\in V(G)\}$. A contradiction. Thus the diameter is the longest geodesic.
        \end{proof}
        \begin{corollary}
        For $n>1$, the diameter of any $K_n$ graph is $1$.
        \end{corollary}
        \begin{proof}
        If $n=1$, then $d(K_1) = 0$, as there are no geodesics in the graph. Thus, suppose $n>1$ and let $u$ and $v$ be arbitrary vertices of $K_n$. Then $d(u,v)=1$, as $K_n$ is a complete graph and thus there exists an edge between $u$ and $v$. But $u$ and $v$ are arbitrary. Thus $\max\{d(u,v):u,v\in V(G)\} = 1$. Therefore $d(K_n) = 1$.
        \end{proof}
        \begin{corollary}
        The diameter of any $P_n$ graph is $n-1$.
        \end{corollary}
        \begin{proof}
        For the diameter is the longest geodesic, and from corollary 1.10 this is $n-1$. Thus $d(P_n) = n-1$.
        \end{proof}
        \begin{corollary}
        For any connected graph $G$ where $|V(G)| = n$, $d(G) \leq n-1$.
        \end{corollary}
        \begin{proof}
        For let $G$ be a graph on $n$ vertices. Suppose $d(G)>n-1$. But as there are only $n$ elements, a path of length greater than $n-1$ must traverse some vertex more than once. But then this is not a path, and thus not a geodesic. So there is no geodesic of length greater than $n-1$. Thus $d(G)\leq n-1$.
        \end{proof}
        \begin{theorem}
        If a walk contains no repeated vertex, then it contains no repeated edge.
        \end{theorem}
        \begin{proof}
        Suppose not. Let $uv$ be an edge that is repeated. Then both $u$ and $v$ must be repeated, a contradiction. Thus if no edge is repeated, no edge is repeated.
        \end{proof}
        \begin{corollary}
        The degree sequence of $P_n$ is $1,1,2,\hdots, 2$. There are two $1's$, and $n-2$ $2's$.
        \end{corollary}
        \begin{proof}
        For if $v\in V(P_n)$ is an endpoint, then $\deg(u) = 1$. If not, then $\deg(u) = 2$. As there are only two endpoints, there must be $n-2$ vertices that are not endpoints. Thus the degree sequence is $1,1,2,\hdots,2$.
        \end{proof}
        \begin{corollary}
        The degree sequence of $K_n$ is $n-1,n-1,\hdots, n-1$.
        \end{corollary}
        \begin{proof}
        For given a point in $K_n$, there is an edge from $u$ to every other vertex in $K_n$. Thus there are $n-1$ such edges from such 
        a vertex, and therefore $\deg(u) = n-1$. As $u$ is arbitrary, for each $u\in V(K_n)$, $\deg(u) = n-1$. Thus the degree sequence is $n-1,n-1,\hdots, n-1$.
        \end{proof}
        \begin{definition}
        A cycle is a closed trail in which no vertex is repeated, with the the exception of the first vertex. That is, a cycle is a trail with one and only one repeated vertex (The start/end point).
        \end{definition}
        \begin{remark}
        A cycle is sometimes called a closed path.
        \end{remark}
        \begin{corollary}
        $2-regular$ graphs form cycles.
        \end{corollary}
        \begin{proof}
        For suppose there are $n$ vertices, and label them $v_1,\hdots, v_n$. As $\deg(v_1)=2$, there must be two vertices connected to it, let $v_2$ be such a vertex. Again, as $\deg(v_2)=2$ there must be two vertices connected to it. But $v_1$ is one such vertex, let $v_3$ be another. Continue in the manner until $v_n$. $v_n$ is connected to $v_{n-1}$, and for all $k=2,3,\hdots, n-1$, there are already two connections. Thus, $v_n$ must be adjacent to $v_1$. But this is a cycle. Therefore, etc.
        \end{proof}
        \begin{definition}
        A cycle on $n$ vertices is denoted $C_n$.
        \end{definition}
        \begin{remark}
        $C_3$ is usually called a triangle.
        \end{remark}
        \begin{corollary}
        For $n>2$, the degree sequence of $C_n$ is $2,2,\hdots, 2$.
        \end{corollary}
        \begin{proof}
        For given a point $u\in V(C_n)$, there are 2 edges incident to $u$, and thus $\deg(u) = 2$. But $u$ is arbitrary, and thus for any $u\in V(C_n)$, $\deg(u) = 2$. The degree sequence is therefore $2,2,\hdots, 2$.
        \end{proof}
        \begin{theorem}
        For a $C_n$ graph, $n>2$, $d(C_n) = \frac{n}{2}$ if $n$ is even, and $\frac{n-1}{2}$ is $n$ is odd.
        \end{theorem}
        \begin{proof}
        For suppose $n$ is odd and let $u$ be an arbitrary point in a $C_n$ graph. Then, for $v\ne u$ there are two independent paths $uu_1\hdots u_k v$ and $vu_{k+1}\hdots u_{n-2}u$. The length of the first being $k+1$ and the length of the second being $n-k-1$, the geodesic being the smaller of the two. When $k= \frac{n-3}{2}$ we have $k+1 = \frac{n-1}{2}$ and $n-k-1 = \frac{n+1}{2}$. When $k=\frac{n-1}{2}$ we have $k+1 = \frac{n+1}{2}$ and $n-k-1 = \frac{n-1}{2}$. Thus, the geodesic maximized at either of these points and is equal to $\frac{n-1}{2}$. As $u$ is arbitrary, $d(G) = \frac{n-1}{2}$. A similar argument follows when $n$ is even.
        \end{proof}
        \begin{definition}
        A graph that is isomorphic to its complement is called self-complementary.
        \end{definition}
        \begin{theorem}
        If $C_n$ is a cycle on n points, and $(C_n)^C$ is a cycle on $n$ points, then $n = 5$.
        \end{theorem}
        \begin{proof}
        For suppose $n=3$. Then $C_n = K_n$, and thus $(C_n)^C$ is the empty graph, and thus not a cycle. If $n = 4$, then given a vertex $u\in V(C_4)$ there is one and only one vertex $v$ such that $\{u,v\} \notin E(C_4)$. Thus $\{u,v\}$ is the only edge containing both $u$ and $v$ in $E((C_4)^C)$, thus $(C_4)^C$ is disconnected and therefore not a cycle. For $n=5$, $C_5$ and $(C_5)^C$ are both cycles. Given $n>5$, the degree of any point in $C_{n}$ is $2$, but the degree of any point in $(C_n)^C$ is $n-3$ as $\deg(u)_G + \deg(u)_{G^c} = n-1$. But then for $n>5, \deg(u)_{u\in (C_n)^C} > 2$ and thus cannot be a cycle. $C_5$ is the only cycle such that $(C_n)^C$ is also a cycle.
        \end{proof}
        \begin{theorem}
        Every circuit contains a cycle.
        \end{theorem}
        \begin{proof}
        For let $G$ be a circuit with $n$ vertices. If no vertices are repeated, with the exception of the starting point, then this is a cycle and we are done. Let the circuit be denoted by listing the vertices $v_1 \rightarrow v_n$. If a vertex is repeated in this sequence, remove all points in between. That is, if the sequence is $v_1,\hdots, v_k,\hdots, v_k, \hdots$, remove all of the points that lie between $v_k$ and itself. Our new sequence is $v_1, \hdots, v_k, \hdots$. Continue this refinement until we have returned to $v_1$ (Which will eventually happen as this is a circuit). Any point on this newly developed circuit is crossed once and only once from the refinement. But then this is a cycle.
        \end{proof}
        \begin{theorem}
        Every circuit is either a cycle or contains two cycles in it.
        \end{theorem}
        \begin{proof}
        Let $G$ be a proper circuit. That is, at least one vertex is repeated. Suppose this is $v_k$. Let the sequence of points traversed be listed as $v_1,\hdots, v_k, \hdots,v_k, \hdots, v_n$. Consider all of the points that lie between $v_k$ and itself. If we repeated the refinement done on the entire circuit from the previous theorem on just this subgraph, we again obtain a cycle. This cycle is different from the original one obtained as it contains edges not on the original. Thus, there are at least two cycles.
        \end{proof}
        \begin{definition}
        Two $u-v$ paths are said to be independent if the only common vertices are $u$ and $v$.
        \end{definition}
        \begin{theorem}
        If $u$ and $v$ are vertices that lie in the same cycle of some graph $G$, then there are at least 2 independent $u-v$ paths.
        \end{theorem}
        \begin{proof}
        For let $u$ and $v$ lie in the cycle $vv_1 \hdots v_k u v_{k+1}\hdots v_n v$. Then the path $v v_1 \hdots v_k u$ is a $u-v$ path, and the path $u v_{k+1} \hdots v_n v$ is a $u-v$ path, and moreover they are independent as the only common points are $u$ and $v$. And $u$ and $v$ are arbitrary points in the cycle. Thus every pair of vertices in the same cycle have at least two independent paths.
        \end{proof}
        \begin{definition}
        A connected graph without cycles is called a tree.
        \end{definition}
        \begin{definition}
        An $(n,e)$ connected graph is a graph on $n$ vertices with $e$ edges.
        \end{definition}
        \begin{theorem}
        If $u$ and $v$ are vertices of a tree, then there is only one $u-v$ path.
        \end{theorem}
        \begin{proof}
        Let $u$ and $v$ be points on a tree. As a tree is connected, there is at least one $u-v$ path. Suppose there is another. Let $a$ be the first point where the two paths diverge (This must happen at least once as the paths are not equal) and let $b$ be the first point where the paths converge (This must happen at least once as both paths end at the same point). But then there are two independent paths from $a$ to $b$, and thus a cycle. But trees do not have cycles. Therefore, etc.
        \end{proof}
        \begin{corollary}
        A tree with more than one point contains at least two elements with degree $1$.
        \end{corollary}
        \begin{proof}
        For let $G$ be a tree and $u,v\in G$ such that $d(u,v) = d(G)$. Then $\deg(u) = \deg(v) = 1$. For suppose not. Let $w$ be a point such that $vw\in E(G)$ but not on the path between $u$ and $v$. Then the only path between $u$ and $w$ contains the entirety of the $uv$ path, as there is only one path between two points in a tree. But then $d(u,w)>d(u,v)$, a contradiction. Thus the point $w$ does not exist and $\deg(v)=1$. Similarly, $\deg(u)=1$.
        \end{proof}
        \begin{corollary}
        The longest geodesic on a tree starts and ends at endpoints.
        \end{corollary}
        \begin{proof}
        For given any two points on a tree that are not endpoints we may append to them adjacent points and construct a longer geodesic. Thus, the longest such geodesic starts and ends at endpoints.
        \end{proof}
        \begin{lemma}
        A graph with $n$ vertices and $e$ edges has at least $n-e$ connected components (Or at least 1 if $n<e$).
        \end{lemma}
        \begin{proof}
        Let $G$ be a graph with $n$ vertices, and keep $n$ fixed throughout. We prove by induction on $e$. If $e=0$, there are $n$ connected components. Let $0<e < n$. By hypothesis there are at least $n-e$ connected components. If $u$ and $v$ are not adjacent and lie in the same connected component, then the new graph formed by appending $uv$ to $E(G)$ still has at least $n-e$ connected components. Let $u$ be in one and $v$ in another. If we Append to $E(G)$ the edge $uv$ then there $e+1$ edges at least $n-(e+1)$ connected components. Therefore, etc.
        \end{proof}
        \begin{corollary}
        A connected graph on $n$ vertices has at least $n-1$ edges.
        \end{corollary}
        \begin{proof}
        For the number of connected components $N$ of any graph with $n$ vertices and $e$ edges is at least $n-e$. That is $N\leq n-e$. But $N=1$ for connected graphs, so $n-e\leq 1 \Rightarrow n-1\leq e$.
        \end{proof}
        \begin{theorem}
        If a tree has $n$ vertices, it has $n-1$ edges.
        \end{theorem}
        \begin{proof} For as a tree is connected, there must at least be $n-1$ edges. If there are more than $n-1$ edges, then there must be a cycle as $\sum_{v\in V(G)} \deg(v) = 2e \geq 2n\Rightarrow e\geq n$. Thus, $e=n-1$. 
        \end{proof}
        \begin{definition}
        An induced subgraph $H$ of a graph $G$ is a set of vertices $V(H) \subset V(G)$ such that for all $u,v\in V(H)$ if $uv\in E(G)$ then $uv \in E(H)$.
        \end{definition}
        \begin{remark}
        That is, the induced subgraph is the graph produced by deleting points and only deleting the edges incident on said points.
        \end{remark}
        \begin{definition}
        If $H$ is a subgraph of $G$ and $V(H)=V(G)$, the $H$ is called a spanning graph of $G$.
        \end{definition}
        \begin{definition}
        If $H$ is a spanning graph of $G$ and $H$ is a tree, then it is called a spanning tree of $G$.
        \end{definition}
        \begin{corollary}
        The induced spanning subgraph of any graph $G$ is $G$.
        \end{corollary}
        \begin{proof}
        Let $H$ be an induced spanning subgraph of some graph $G$. As $H$ is a spanning graph, $V(H)=V(G)$. But as $H$ is an induced subgraph, $E(H)=E(G)$. But then $H=G$.
        \end{proof}
        \begin{theorem}
        All of the spanning trees of $C_n$ are isomorphic to each other.
        \end{theorem}
        \begin{proof}
        For let $C_n$ be the graph characterized by the walk $v_1\hdots v_n$ ($v_1$ and $v_n$ are adjacent). The spanning tree is produced by removing one and only one edge. Let one tree be the removal of $v_{k}v_{k+1}$ and another be $v_{j}v_{j+1}$. Define the isomorphism $f$ as $f(v_l) = v_{k-j+l \mod(n)}$. Then if $l=j+1$, $f(v_{j+1}) = v_{k+1}$, and thus is adjacent only to $f(v_{j+2 \mod(n)}) = v_{k+2\mod(n)}$. If $l=j$, then $f(v_j) = v_k$ and thus is only adjacent to $f(v_{j-1}\mod(n)) = v_{k-1}\mod(n)$. This is an isomorphism.
        \end{proof}
        \begin{remark}
        The isomorphism described above has the effect of "Rotating," one spanning tree to match the other.
        \end{remark}
        \begin{theorem}
        Disconnected sets do not have spanning trees.
        \end{theorem}
        \begin{proof}
        For let $G$ be a disconnected graph and suppose $u,v\in G$ are such that no path lie between them. Suppose $G$ has a spanning tree $H$ where $V(H)=V(G)$. Then, as $H$ is a tree, there is a path from $u$ to $v$. But as $E(H)\subset E(G)$, then there is a path from $u$ to $v$ in $G$, a contradiction. Therefore, etc.
        \end{proof}
        \begin{theorem}
        Given an $(n,e)$ connected graph $G$, the spanning tree of $G$ is a deletion of $e-n+1$ edges.
        \end{theorem}
        \begin{proof}
        For the number of edges in a tree is $n-1$, then $e-x = n-1$, where $x$ is the number of edges that are to be deleted. Solving for this yields $x=e-n+1$.
        \end{proof}
        \begin{definition}
        In a graph $G$, a bridge is an edge in $E(G)$ that is not contained in a cycle.
        \end{definition}
        \begin{corollary}
        Every edge of a tree is a bridge.
        \end{corollary}
        \begin{proof}
        For as trees contain no cycles, no edge lies within a cycle, and thus every edge is a bridge.
        \end{proof}
        \begin{corollary}
        If every edge of a graph $G$ is a bridge, then $G$ is a tree.
        \end{corollary}
        \begin{proof}
        For suppose not. Suppose $G$ contains a cycle in it. Then there is an edge contained within a cycle, and therefore an edge that is not a bridge. But every edge is a bridge, a contradiction. Therefore, etc.
        \end{proof}
        \begin{theorem}
        If $G$ is a connected graph with at least one cycle, then there exists a spanning tree.
        \end{theorem}
        \begin{proof}
        Let $G$ be a connected graph and let $v_1 \rightarrow v_n\rightarrow v_1$ be a cycle. Remove any edge in this cycle. The resulting subgraph is still connected, for given a point $u$ and a point $v$, either their geodesic contains the removed edge or it doesn't. If it doesn't, we are done. If not, there is another path from $u$ to $v$ as every cycle has two independent paths between all of its points. Thus $u$ and $v$ are still connected. As $u$ and $v$ are arbitrary, the graph $G$ is still connected. But now $v_1\rightarrow v_n \rightarrow v_1$ is no longer a cycle. For all other cycles, remove a single edge. The resulting subgraph is still connected. But then this final subgraph is a spanning graph containing no cycles and is connected, and is thus a spanning tree.
        \end{proof}
        \begin{corollary}
        If $G$ is a connected graph with at least one cycle, then it has more than one spanning tree.
        \end{corollary}
        \begin{proof}
        In the construction above, in any given cycle remove a different edge to obtain a new spanning tree.
        \end{proof}
        \begin{theorem}
        Given a graph $G$, an edge of $G$ is a bridge if and only if its deletion increases the number of components of $G$.
        \end{theorem}
        \begin{proof}
        For suppose the deletion of some edge $uv$ increases the number of components of $G$. Then $uv$ can not be the member of a cycle, otherwise there would be a second independent path from $u$ to $v$, and thus $u$ and $v$ would lie in the same component. Thus $uv$ is a bridge. Suppose $uv$ is an edge in $G$. Then if deleted, $u$ and $v$ would lie in two different components as otherwise $uv$ would be a cycle, but it is not. Thus the deletion of a bridge creates two new components.
        \end{proof}
        \begin{corollary}
        If $G$ is a connected graph and $v\in V(G)$ does not lie in a cycle, then the degree of $v$ in any spanning tree of $G$ is $\deg(v)$.
        \end{corollary}
        \begin{proof}
        It suffices to show that for any spanning tree of $G$, no edge incident on $v$ is removed. Let $u$ be some adjacent vertex. Then $uv$ is a bridge, for if not then it is contained in a cycle and thus $v$ is contained in a cycle, but it is not. Thus the deletion of $uv$ creates two connected components, and thus no spanning tree may be obtained. Thus, no edges may be deleted from $v$ in a spanning tree of $G$.
        \end{proof}
        \begin{theorem}
        Given a $K_n$ graph, there exists a $P_n$ spanning tree.
        \end{theorem}
        \begin{proof}
        For let $G$ be a $K_n$ graph and let $v\in V(G)$ be arbitrary. Let $u\ne v$ be arbitrary, and remove from $v$ all edges other than $uv$. The edge $uv$ exists as $G$ is complete. From $u$, let $w\ne u, w\ne v$ be arbitrary and remove from $u$all edges other than $uv$ and $uw$. Continue in this manner until the last vertex. The degree of every vertex is $2$, with the exception of $v$ and the last vertex, who have degree $1$. Thus, this is a $P_n$ graph.
        \end{proof}
        \begin{definition}
        The eccentricity of a vertex $v$ of a connected graph $G$ is the maximum value of $d(v,x)$ for all vertices $x\in V(G)$. It is denoted $e(v)$.
        \end{definition}
        \begin{definition}
        The center of a graph $G$ is the set of vertices $v\in G$ such that $e(v) \leq e(w)$ for all $w\in G$. This set is denoted $C(G)$.
        \end{definition}
        \begin{definition}
        The eccentricity of the center of a graph $G$ is called the radius of $G$, denoted $r(G)$.
        \end{definition}
        \begin{corollary}
        $\max\{e(v):v\in G\}= d(G)$.
        \end{corollary}
        \begin{proof}
        For $\max\{e(v):v\in G\} = \max\{d(v,x), x\in G:v\in G\} = d(G)$.
        \end{proof}
        \begin{corollary}
        The center of a $P_n$ graph $v_1\rightarrow v_n$ is $\{v_\frac{n}{2}\}$ if $n$ is even, and $\{v_{\frac{n-1}{2}},v_{\frac{n+1}{2}}\}$ if $n$ is odd.
        \end{corollary}
        \begin{proof}
        For $\max\{d(v_k,x):x\in G\} = \min\{n-k,k\}$. This is minized, when $n$ is even, for $k= \frac{n}{2}$, and is minimized for odd $n$ when $k=\frac{n\pm 1}{2}$.
        \end{proof}
        \begin{corollary}
        There exists graphs $G$ such that every element $v\in V(G)$ is also in $C(G)$.
        \end{corollary}
        \begin{proof}
        For let $G$ be a $K_n$ graph. Then $e(v) = 1$ for all $v\in V(G)$, and thus $C(G) = V(G)$.
        \end{proof}
        \begin{corollary}
        The eccentricities of all vertices within a tree that are not themselves endpoints are lowered by 1 if all endpoints of the tree are removed.
        \end{corollary}
        \begin{proof}
        For let $G$ be a tree, $v\in G$, and suppose $v$ is not an endpoint. Then $d(v,u)$ is maximized at an endpoint $x$. Thus removing all endpoints from $G$ reduces $e(v)$. Moreover, it does so by $1$, for the $d(v,u)$ will now be maximized at a point adjacent to $x$.
        \end{proof}
        \begin{theorem}[Jordan's Center Theorem]
        The center of a tree consists either of $1$ point of $2$.
        \end{theorem}
        \begin{proof}
        Let $G$ be a tree. Remove the endpoints from $G$ to create the subgraph $G'$. This is again a tree. From $G'$, remove the endpoints to create the subgraph $G''$. Continuing in the fashion we are left either with two adjacent points of equal eccentricity or with one point. Thus, the center contains either $1$ point or $2$.
        \end{proof}
        \begin{lemma}
        For a connected graph, $r(G) \leq d(G)$.
        \end{lemma}
        \begin{proof}
        For $r(G) = \min\{\max\{d(x,v):x\in G\} v\in G\} \leq \max\{d(x,v):x,v\in G\}=d(G)$.
        \end{proof}
        \begin{theorem}
        For any connected graph, $r(G) \leq d(G) \leq 2r(G)$.
        \end{theorem}
        \begin{proof}
        From the lemma, $r(G) \leq d(G)$. Let $u,v\in G$ be such that $d(u,v)=d(G)$ and let $w\in G$ be such that $e(w) = r(G)$. As $G$ is connected, $d(u,w)$ and $d(v,w)$ exists. But then $d(u,v)\leq d(u,w)+d(v,w) \leq r(G)+r(G)=2r(G)$. Thus, $r(G)\leq d(G)\leq 2r(G)$.
        \end{proof}
        \begin{corollary}
        There exists graphs $G$ such that $r(G) = d(G)$.
        \end{corollary}
        \begin{proof}
        For consider $G=P_2$. Then $r(G) = d(G) = 2$.
        \end{proof}
        \begin{corollary}
        There exists graphs $G$ such that $d(G) = 2r(G)$.
        \end{corollary}
        \begin{proof}
        For consider $G=P_5$. Then $d(G) = 2r(G) = 4$.
        \end{proof}
        \begin{definition}
        A graph $G$ is said to be bipartite if and only if there are two sets $V_1$ and $V_2$ such that $V(G) = V_1\cup V_2$ where $V_1\cap V_2 = \emptyset$, and if $uv \in E(G)$, then $u$ and $v$ are not both contained in one of the $V_i$.
        \end{definition}
        \begin{theorem}
        Every tree is bipartite.
        \end{theorem}
        \begin{proof}
        For let $G$ be a tree and $v\in G$ be an endpoint. Add $v$ to the set $V_1$. The vertex adjacent to $v$, call it $u$, append to $V_2$. For all element adjacent $u$ append to $V_1$. Continue in such a manner, oscillating between $V_1$ and $V_2$, until all elements are in either $V_1$ or $V_2$. Then $V_1 \cap V_2 = \emptyset$ and $V_1 \cup V_2 = V(G)$.
        \end{proof}
        \begin{theorem}
        $C_n$ is bipartite if and only if $n$ is even. 
        \end{theorem}
        \begin{proof}
        Let $G=C_n$, $n$ be even, and let $v_1\rightarrow v_n$ be a path on $n$ points in $C_n$. Append $v_1$ to $V_1$, $v_2$ to $v_2$, and so on. Then odd numbers are in $V_1$ and even numbers are in $V_2$, so $V_1\cup V_2 = V(G)$ and $V_1 \cap V_2 = \emptyset$. But as $n$ is even, $v_n \in V_2$. Thus, $v_1 v_n \in E(G)$ and $v_1$ and $v_n$ are not in the same partition of $V(G)$. Thus $C_n$ is bipartite. Now suppose $C_n$ is bipartite and suppose $n$ is odd. Let $v_1 \rightarrow v_n$ be a path on the $n$ points. Suppose $v_1 \in V_1$. Then $v_2 \in V_2$, and $v_n \in V_2$. But as $v_n$ is odd, $v_{n} \in V_1$, a contradiction. Thus if $C_n$ is bipartite, then $n$ is even.
        \end{proof}
        \begin{definition}
        If $G$ is a bipartite graph partition by $V_1$ and $V_2$, where $|V_1| = n$ and $|V_2| = m$, and if ever element of $V_1$ is adjacent to every element of $V_2$, then $G$ is called complete and is denoted $K_{n,m}$.
        \end{definition}
        \begin{corollary}
        $r(K_{1,n}) = 1$.
        \end{corollary}
        \begin{proof}
        For let $v\in V_1$ be the unique point. As $K_{1,n}$ is complete, every element of $V_2$ is adjacent to $V$. Thus, $e(v) = 1$. Therefore, $r(K_{1,n}) =1$.
        \end{proof}
        \begin{corollary}
        $C(K_{1,n})$ is the unique element of $V_1$.
        \end{corollary}
        \begin{proof}
        For if $v\in V_1$, then $e(v) = 1 = r(K_{1,n})$.
        \end{proof}
        \begin{corollary}
        $d(K_{1,n})=2$ for $n>1$.
        \end{corollary}
        \begin{proof}
        For let $v\in V_1$ be the unique point in $V_1$, and let $u_1$ and $u_2 \in V_2$ be arbitrary. Then $u_1$ and $u_2$ are not adjacent, and thus the only path between them is $u_1 v u_2$. Thus, $d(u_1,u_2) = 2$. But $u_1$ and $u_2$ are arbitrary. Thus, $d(K_{1,n}) = 2$.
        \end{proof}
        \begin{corollary}
        If $n\geq1$, $m> 1$ then $d(K_{m,n}) = 2$.
        \end{corollary}
        \begin{proof}
        For let $G=K_{m,n}$, $n\geq1$, $m> 1$. If $u\in V_1$ and $v\in V_2$, then $d(u,v) = 1$. If $u,v\in V_1$, then, for any $w\in V_2$, $uwv$ is a path from $u$ to $v$, and thus $d(u,v) = 2$. Thus, $d(K_{n.m}) = 2$.
        \end{proof}
        \begin{corollary}
        If $n,m\geq 1$, then $r(K_{m,n}) = 1$.
        \end{corollary}
        \begin{proof}
        For $r(K_{n,m}) = \min\{ \max\{d(u,v):u\in G\}: v\in G\} = 1$ (That is, take $u\in V_1$ and $v\in V_2$, and thus $d(u,v)=1$).
        \end{proof}
        \begin{corollary}
        $K_{m,n}^C$ contains a $P_m$ and a $P_n$ graph.
        \end{corollary}
        \begin{proof}
        For let $u_1,\hdots, u_m$ be in $V_1$, and $v_1,\hdots v_n$ be in $V_2$. Then $u_1 \rightarrow u_m$ and $v_1 \rightarrow v_n$ are paths in $K_{m,n}^C$.
        \end{proof}
        \begin{theorem}
        $K_{m,n}^C$ is disconnected.
        \end{theorem}
        \begin{proof}
        For let $u\in V_1$ and $v\in V_2$ be arbitrary. As $K_{m,n}$ is complete, $uv$ is an edge and thus $uv$ is not an edge in $K_{m,n}^C$. But as $u$ and $v$ were arbitrarily chosen, $V_1$ and $V_2$ are disconnected components in $K_{m,n}^C$.
        \end{proof}
        \begin{theorem}
        A $K_{m,n}$ graph is a tree if and only if $n$ or $m$ is equal to $1$.
        \end{theorem}
        \begin{proof}
        For suppose $m=1$. Then there are no cycles in $K_{1,n}$, as let $v\in V_1$ and $u_1,u_2\in V_2$ be arbitrary. Then the only path between these points is $u_1 v u_2$, and therefore there can be no cycles. No suppose $K_{n,m}$ is a tree. Then $\sum_{v\in V(K_{m,n})}\deg(v) = 2e=2(m+n-1)$, as this is a tree, and $\sum_{v\in V(K_{m,n})}\deg(v) = 2nm$. Thus, $n+m-1=mn$, so $n(m-1)=m-1$, and therefore either $m=1$ or $n=1$.
        \end{proof}
        \begin{theorem}
        If $G$ is such that $|V(G)|=6$, then either $G$ or $G^C$ contains a triangle as a subgraph.
        \end{theorem}
        \begin{proof}
        Let $k$ be a vertex of $G$. Let $k$ be a point with at least $3$ adjacent vertices. If no such point exists, then it must exist in $G^C$ as $\deg(k)_G+\deg(k)_{G^C}=5$. Let $r,t$ and $s$ be the points adjacent to $k$. If any of them are adjacent to one another, then together with $k$ they form a triangle. If no, then $r,t,s$ are adjacent in $G^C$ and thus there is a triangle.
        \end{proof}
        \begin{corollary}
        If $G$ is a graph and $|V(G)|\geq 6$, then either $G$ or $G^C$ contains a triangle subgraph.
        \end{corollary}
        \begin{proof}
        For let $k\in G$ be a vertex with at least $3$ adjacent vertices. If no such point exists, then it must exist in $G^C$ as $\deg(k)_G+\deg(k)_{G^C} = n-1 \geq 5$. Let $r,t,s$ be such adjacent points. If any of them are also adjacent, we are done. If not, then they are adjacent in $G^C$ and we are done.
        \end{proof}
        \begin{theorem}
        There exists graphs on $5$ vertices such that neither the graph nor the complement contain a triangle.
        \end{theorem}
        \begin{proof}
        For let the vertices be labelled $(1),\hdots,(5)$. Consider the "House," $(1)(2)(3)(4)(5)(1)$. Its complement is the "Star," $(1)(3)(5)(2)(4)(1)$. Neither contain triangles.
        \end{proof}
        \begin{definition}
        A graph is called planar if and only if it can be drawn in the plane with no edges intersecting. A graph that is not planar is called non-planar.
        \end{definition}
        \begin{definition}
        If a planar graph is drawn in a manner without intersecting edges, it is called a plane graph.
        \end{definition}
        \begin{corollary}
        All graphs with fewer than $5$ edges are planar.
        \end{corollary}
        \begin{proof}
        For $K_4$ is planar, and thus all smaller graphs are. To show this, let $(1)(2)(3)$ be the edges of a triangle. Place the fourth point $(4)$ in the center and then connect $(1)(4)$, $(2)(4)$, and $(3)(4)$. This is a plane graph of $K_4$ and thus $K_4$ is planar.
        \end{proof}
        \begin{theorem}
        All trees are planar.
        \end{theorem}
        \begin{proof}
        By induction. If $|V(G)| = 1$, we are done. Suppose it is true for $|V(G)| = n$. Let $G$ be a tree such that $|V(G)| = n+1$ and let $p$ be an endpoint adjacent to $u$. Deleting this creates a subtree on $n$ points. But then this is planar. Appending $p$ in a small enough neighborhood about $u$ shows that $G$ is planar.
        \end{proof}
        \begin{theorem}
        For all $n$, $K_{1,n}$ and $K_{2,n}$ are planar.
        \end{theorem}
        \begin{proof}
        For $K_{1,n}$ is simply a "Star," with $n$ endpoints whose sole adjacent vertex is the center, and is thus planar. For $K_{2,n}$, let $u_1,u_2 \in V_1$ and $v_1,\hdots, v_n \in V_2$. Place $u_1$ at $(-1,0)$ and $u_2$ at $(1,0)$. Place $v_k$ at $(k,0)$. Connect the straight lines from $v_k$ to $u_1$ and $u_2$. This is $K_{2,n}$ and is a plane graph. Thus, $K_{2,n}$ is planar.
        \end{proof}
        \begin{definition}
        A cutpoint of a graph is a point whose deletion increases the number of components of the graph.
        \end{definition}
        \begin{definition}
        A block of a graph is a maximal connected subgraph which contains no cutpoints relative to itself.
        \end{definition}
        \begin{theorem}
        A vertex $v$ on a tree is a cutpoint if and only if $\deg(v) \geq 2$.
        \end{theorem}
        \begin{proof}
        For suppose $v$ is a cutpoint. Then it is not an endpoint, as otherwise its deletion would not create disconnected components. But if it is not an endpoint, then $\deg(v)>1$. Thus, $\deg(v)\geq 2$. Now suppose $\deg(v) \geq 2$. Let $u$ and $w$ be two vertices adjacent to $v$ and not equal to each other. If we delete $v$, there is no path from $u$ to $w$ as if there were then there would be two independent paths from $u$ to $w$ and thus a cycle. But trees have no cycles. Thus, the deletion of $v$ disconnects the graph and $v$ is a cutpoint.
        \end{proof}
        \begin{theorem}
        Blocks of Trees.
        \end{theorem}
        \begin{corollary}
        A $C_n$ graph contains no cut points for $n>2$.
        \end{corollary}
        \begin{proof}
        For let $v$ be an arbitrary vertex in a $C_n$ graph and let $u$ and $w$ be other vertices ($n>2$, so such vertices exist). Then, as $C_n$ is a cycle, there are two independent paths from $u$ to $w$, one of which contains $v$ and the other not containing $v$. Thus, removing $v$ does not disconnect $C_n$. As $v$ is arbitrary, $C_n$ has no cutpoints for $n>2$.
        \end{proof}
        \begin{theorem}
        In a graph $G$ with $u,v\in V(G)$, if $uv\in E(G)$ and $u$ and $v$ are cutpoints, then $uv$ is a bridge.
        \end{theorem}
        \begin{proof}
        For let $u$ and $v$ be cutpoints of $G$ and suppose $uv\in E(G)$. Let $s$ and $t$ be vertices in two of the separate components that are formed by removing $v$. Thus, there is no path from $s$ to $t$. But then $uv$ cannot be part of a cycle, as then there would be two independent paths from $u$ to $v$, a contradiction. Thus $uv$ is a bridge.
        \end{proof}
        \begin{theorem}
        If $uv$ is a bridge, $\deg(u),\deg(v)>1$, then $u$ and $v$ are cutpoints.
        \end{theorem}
        \begin{proof}
        For let $uv$ be a bridge, let $s$ be adjacent to $u$ and $t$ adjacent to $v$. Any path from $s$ to $t$ must contain $uv$, as otherwise $uv$ would be contained in a cycle, but $uv$ is a bridge and thus is not contained in a cycle. Then the removal of $uv$ separates $s$ from $t$, and we now have disconnected components. Thus, $u$ and $v$ are cutpoints.
        \end{proof}
        \begin{corollary}
        If $uv$ is a bridge, and either $\deg(v)>1$ or $\deg(u)>1$ (But not necessarily both), then either $u$ or $v$ is a cutpoint.
        \end{corollary}
        \begin{proof}
        For let $uv$ be a bridge and suppose $\deg(v)>1$. Let $s$ be adjacent to $v$ and not equal to $u$. As $uv$ is a bridge, any path from $s$ to $u$ must contain $uv$. Thus, the removal of $v$ separates $u$ from $s$. That is, $v$ is a cutpoint.
        \end{proof}
        \begin{corollary}
        There exist graphs $G$ such that $u$ is a cutpoint of $G$, $ux \in E(G)$, and yet $ux$ is not a bridge.
        \end{corollary}
        \begin{proof}
        For let $K_3$ be the complete graph on vertices $t_1,t_2,t_3$, and append to it a single vertex and $v$ and connect it only to $t_1$. Then $t_1$ is a cutpoint, for it seperates $v$ from $t_2$ and $t_3$, $t_1t_2\in E(G)$, yet $t_1 t_2$ is not a bridge as it is contained within a cycle.
        \end{proof}
        \begin{definition}
        A coloring of a graph is an assignment of labels, denoted by colors, to vertices of the graph.
        \end{definition}
        \begin{definition}
        The chromatic number of a graph is the minimum number of unique colors that are needed to color a graph such that no two adjacent vertices have the same color. This is denoted $N(G)$.
        \end{definition}
        \begin{theorem}
        Even cycles $C_{2k}$ are $2-colorable$. That is, they have chromatic number $2$.
        \end{theorem}
        \begin{proof}
        For let $v_1$ be arbitrary, and call it blue. Characterize $C_{2k}$ by $v_1 v_2 \hdots v_{2k-1}v_{2k} v_1$. If $n$ is even, let $v_n$ be red, otherwise let it be blue. Thus, if $1<n<2k$, then $v_n$ and $v_{n+1}$ have different colors. For $n=2k$, $v_{2k-1}$ and $v_1$ are blue, and $v_{2k}$ is red. Thus, this is $2-colorable$.
        \end{proof}
        \begin{theorem}
        A graph $G$ has chromatic number $N(G)=2$ if and only if $G$ is bipartite.
        \end{theorem}
        \begin{proof}
        For let $G$ be bipartite, being the disjoint union of $V_1$ and $V_2$. Then, if $v\in V_1$ call it blue and if $v\in V_2$ call it red. Then no two adjacent vertices of $G$ are colored the same way, as elements of $V_1$ connect only to elements of $V_2$ and vice-versa. Thus, $N(G)=2$. If $N(G)=2$, then for $v\in G$ such that $v$ is blue, append to $V_1$. For $v\in G$ colored red, append to $V_2$. Then $V_1\cup V_2 = G$ and $V_1\cap V_2 = \emptyset$ as all elements of $G$ are either blue or red, exclusively. But as $N(G)=2$, if $v\in V_1$ and $u\in V_1$, then $uv\notin E(G)$. Similarly for $V_2$. Thus, $G$ is bipartite.
        \end{proof}
        \begin{theorem}
        If $G$ is a graph on $n$ vertices, and $N(G) = n$, then $G= K_n$.
        \end{theorem}
        \begin{proof}
        For let $G$ be a graph on $n$ vertices and let $N(G) = n$. Let $v,u\in G$ be arbitrary and suppose $v$ is blue. If $uv\notin E(G)$, then we may label $v$ freely as blue. But then, at most, $N(G)=n-1$, a contradiction. Thus $uv\in E(G)$. As $u$ and $v$ are arbitrary, all vertices are adjacent. Thus $G=K_n$.
        \end{proof}
        \begin{corollary}
        If $N(G) = 1$, then $G$ is totally disconnected $E(G)=\emptyset$, or $G$ contains one point.
        \end{corollary}
        \begin{proof}
        For if $G$ contains one point, we are done. Suppose not and let $u,v\in G$. If $uv \in E(G)$ then $u$ and $v$ cannot be colored the same way, and thus $N(G)>1$, a contradiction. Thus $uv\notin E(G)$. As $u$ and $v$ are arbitrary, $E(G) = \emptyset$.
        \end{proof}
        \begin{lemma}
        Trees are bipartite.
        \end{lemma}
        \begin{proof}
        By induction. A tree on two vertices is bipartite. Suppose a tree on $n$ vertices is bipartite. Let $G$ be a tree on $n+1$ vertices, and let $v\in G$ be an endpoint. Deleting $v$ creates a tree on $n$ vertices, and is thus bipartite. Appending $v$ back in, let $u$ be the unique vertices adjacent to $v$. Place $v$ in the partition not containing $u$. Thus $G$ is bipartite.
        \end{proof}
        \begin{corollary}
        Trees are two colorable.
        \end{corollary}
        \begin{proof}
        For as trees are bipartite, they are two colorable.
        \end{proof}
        \begin{corollary}
        There exist two-colorable graphs that are not trees.
        \end{corollary}
        \begin{proof}
        For consider $K_{3,3}$. It is a bipartite graph, and is therefore two-colorable. However it is not a tree as it contains cycles.
        \end{proof}
        \begin{definition}
        If a planar graph is drawn with no intersecting edges, it divides the plane into regions called the faces of the graph.
        \end{definition}
        \begin{definition}
        The outer face of a plane graph is the face which is unbounded in the plane.
        \end{definition}
        \begin{lemma}
        If $G$ is a plane connected graph with $F$ faces, and if a vertex $v$ is appended to $G$, if $v$ is made adjacent to $k$ vertices in $G$ and is still plane, then this new graph has $F+k-1$ faces,
        \end{lemma}
        \begin{proof}
        For let $G$ be a plane graph with $F$ faces. We prove by induction. Let $v$ be an appended vertex. If we add $1$ edge, we create no new faces as otherwise there is an intersection and the result is no longer planar. Suppose an addition of $k$ edges yields $k-1$ new faces. From $v$, add a new edge to some vertex $u$. As $G$ is connected, and as $G$ with $v$ is also connected, there is a path from $u$ to $v$ that does not contain $uv$. But then $uv$ creates a new region, that is the cycle on $uv$. Thus there is $k$ new faces.
        \end{proof}
        \begin{theorem}[Euler's Characteristic Theorem]
        For planar graphs $G$, if $G$ is represent as plane then $V-E+F=2$, where $V = |V(G)|$, $E=|E(G)|$, and $F$ is the number of faces.
        \end{theorem}
        \begin{proof}
        We prove by induction on the number of vertices. If $V=1$, then $F=1$ and $E=0$, thus $V-E+F=1-0+1=2$. Now, suppose on $n$ vertices, $V-E+F=2$. Then, for $n+1$ vertices with $k$ new vertices we have $(V+1)-(E+k)+(F+k-1) = V+1-E-k+F+k+1 = V-E+F=2$.
        \end{proof}
        \begin{lemma}
        $K_5$ is non-planar.
        \end{lemma}
        \begin{proof}
        Suppose not, and let it be represent in the plane as planar. Then $V=5$, $E = 10$, and thus $V-E = -5$, meaning $F = 7$. But each region is bounded by, at least, $3$ edges. As each edge is the boundary of at least $2$ regions, $\frac{3F}{2} \leq E$. But $\frac{3F}{2} = 10.5$ and $E=10$, a contradiction. Thus $K_5$ is non-planar.
        \end{proof}
        \begin{lemma}
        $K_{3,3}$ is non-planar.
        \end{lemma}
        \begin{proof}
        For suppose not. Then $V-E+F=2$. We have that $V=6$, $E= 9$, and thus $F=5$. But $F$ is contained within at least 4 edges, for $K_{3,3}$ has no cycles on $3$ vertices as it is Bipartite. But then, $\frac{4F}{2} = 2F \leq E = 9$. But $2F = 10$, a contradiction. Thus $K_{3,3}$ is nonplanar.
        \end{proof}
        \begin{lemma}
        If $G$ is nonplanar and $v$ is appended to $G$, then the result is nonplanar.
        \end{lemma}
        \begin{proof}
        For suppose not. That is $G$ with $v$ contains no intersecting vertices. Then deleting $v$ means that $G$ has no intersecting vertices, and is thus planar. But it is not, a contradiction. Thus the resulting graph is non-planar.
        \end{proof}
        \begin{theorem}
        For $n>4$, $K_n$ is non-planar.
        \end{theorem}
        \begin{proof}
        By induction. For $n=5$, we are done. Suppose it is true for $n$. Appending a new vertices results in yet again a non-planar graph. Thus $K_n$ is non-planar for $n>4$.
        \end{proof}
        \begin{theorem}
        For $n,m\geq 3$, $K_{n,m}$ is non-planar.
        \end{theorem}
        \begin{proof}
        For let $n,m\geq 3$. Choose $3$ elements in $V_1$ and $3$ in $V_2$. As $K_{n,m}$ is the complete bipartite graph, there are edges from the three chosen points in $V_1$ to the three chosen points in $V_2$, and vice-versa. But this is the $K_{3,3}$ graph, and is thus non-planar. Appending to this graph vertices also results in a non-planar graph. Thus $K_{n,m}$ is non-planar.
        \end{proof}
        \begin{lemma}
        The number of faces on a plane graph is maximized when each face is contained by a triangle, or is the outer face.
        \end{lemma}
        \begin{proof}
        For suppose not. Suppose a face is contained by $n>3$ sides and that this is the maximum number of faces on the graph. But as $n>3$, at least two vertices which contain the face are not adjacent and thus may be connected forming two new faces. But we say the graph had the maximal number of faces, a contradiction. Thus all faces form triangles.
        \end{proof}
        \begin{theorem}
        A planar graph on $n$ vertices has at most $3n-6$ edges.
        \end{theorem}
        \begin{proof}
        When a graph is maximized, the faces are contained within triangles. Then $\frac{2}{3}V= F$. But from Euler's Characteristic Formula, $V-E+F=2$. So, $n-V+\frac{2}{3}V = 2$, or $n-\frac{1}{3}V = 2$. So, $V=3n-6$. If the graph does not contain the maximal number of faces, it is less than this number. Thus, if a graph is planar, $E\leq 3V-6$.
        \end{proof}
        \begin{theorem}
        If $G$ is a planar connected graph, then it can be drawn such that any face is the outer face.
        \end{theorem}
        \begin{theorem}
        Euler's Characteristic Theorem may be modified for $k$ disconnected components.
        \end{theorem}
        \begin{corollary}
        For trees, $E=V-1$.
        \end{corollary}
        \begin{proof}
        For as tress have no cycle, $F=1$. But trees are connected and planar, and thus $V-E+F = 2$. Therefore $E=V-1$.
        \end{proof}
        \begin{corollary}
        For cycles $C_n$, $E=V$.
        \end{corollary}
        \begin{proof}
        For $F=2$. Thus, $V-E+2=2\Rightarrow V=E$.
        \end{proof}
        \begin{definition}
        If a planar graph can be drawn such that all of its vertices are touching the outer face it is called outer planar.
        \end{definition}
        \begin{corollary}
        Trees are outer planar.
        \end{corollary}
        \begin{proof}
        For trees, $F=1$, and thus every vertices touches the outer (Only) face.
        \end{proof}
        \begin{lemma}
        If $G$ is a connected graph that is not outer planar, then there exists a vertex $\deg(v)$ such that $\deg(v)>2$.
        \end{lemma}
        \begin{proof}
        For let $u \in G$ be such that $u$ does not touch the outer edge and let $w$ be a vertex that does. As $G$ is connected, there is a path from $u$ to $w$. Let $v$ be the first point on this path that touches the outer face. Then $\deg(v)>2$. For suppose not. $\deg(v)$ is at least $2$, as there is a path to $u$ and a path to $w$. But if there is no other adjacent vertex then $u$ and $w$ are adjacent to the same face. A contradiction. Thus $\deg(v)>2$.
        \end{proof}
        \begin{corollary}
        All cycles are outer planar.
        \end{corollary}
        \begin{proof}
        For all $v\in C_n$, $\deg(v)=2$. Thus $C_n$ cannot possible be not outer planar, and is therefore planar.
        \end{proof}
        \begin{definition}
        The point-connectivity of a graph is the minimum number of points whose deletion yields either a disconnected graph or a trivial one. For a graph $G$, this is denoted $k(G)$.
        \end{definition}
        \begin{corollary}
        A disconnected graph $G$ has $k(G) = 0$.
        \end{corollary}
        \begin{proof}
        For deleting zero points yields a disconnected graph.
        \end{proof}
        \begin{corollary}
        If $G$ is connected and has a cutpoint, then $k(G) = 1$.
        \end{corollary}
        \begin{proof}
        For the deletion of a cutpoint increases the number of disconnected components, and thus if $G$ is connected the deletion of a cutpoint makes it disconnected.
        \end{proof}
        \begin{corollary}
        If $G$ is connected and contains a bridge $uv$, then $k(G)=1$.
        \end{corollary}
        \begin{proof}
        For delete $u$ or $v$ from $G$. As $uv$ is a bridge, $G$ is now disconnected. Thus, $k(G) = 1$>
        \end{proof}
        \begin{corollary}
        $k(C_n) =2$
        \end{corollary}
        \begin{proof}
        For if delete some point $v$ from $C_n$. This is still connected as there are two independent paths from any two points in $C_n$. Let $u$ be a point not adjacent to $v$ (If $n=3$, delete another point and we arrive at the trivial graph). Deleting this point then disconnects the graph for there is now no path from the points adjacent to $v$. Thus $k(C_n)=2$.
        \end{proof}
        \begin{corollary}
        $k(K_n) = n-1$.
        \end{corollary}
        \begin{proof}
        For suppose not, and suppose $k(K_n)<n-1$. Delete $k<n-1$ points from $K_n$ and let $u,v$ be arbitrary points that remain (There are at least $2$ as $k<n-1$). But as the graph is $K_n$, $uv\in E(K_n)$. As $u$ and $v$ are arbitrary, the graph is still connected. Thus $k=n-1$ and we are left with the trivial graph.
        \end{proof}
        \begin{corollary}
        $k(K_{m,n}) = \min\{m,n\}$.
        \end{corollary}
        \begin{proof}
        For suppose not. Suppose $k(K_{m,n})<\min\{m,n\}$. Remove less than $\min\{m,n\}$ vertices from $K_{m,n}$ and let $u,v$ be arbitrary. If $u\in V_1$ and $v\in V_2$, we are done as $uv\in E(K_{m,n})$. Suppose $u,v\in V_1$. As less then $\min\{m,n\}$ points have been removed, there is still a $w\in V_2$. But then $uw$ and $wv$ exists. Thus there is a path from $u$ to $v$. As $u$ and $v$ are arbitrary, the graph is still connected. Finally, there exist a deletion of $\min\{n,m\}$ points that disconnected $K_{m,n}$. For suppose $|V_1| = m$ and $|V_2| = n$, and let $n<m$. Delete all points from $V_1$. This is disconnected as no points in $V_2$ are adjacent. Thus $k(K_{m,n})=\min\{m,n\}$.
        \end{proof}
        \begin{definition}
        A wheel on $n$ vertices $W_n$ is a cycle $C_{n-1}$ with a vertex $v$ appended that is adjacent to all elements of $C_{n-1}$.
        \end{definition}
        \begin{corollary}
        $r(W_n) = 1$.
        \end{corollary}
        \begin{proof}
        For recall $r(W_n) = \min\{\max\{d(v,x):x\in W_n\}v\in W_n\}$. Let $v$ be the appended vertex to $C_{n-1}$. Then $d(v,x) = 1$ for all $x\in W_n$. Thus $r(W_n) = 1$.
        \end{proof}
        \begin{corollary}
        $d(W_n) = 2$
        \end{corollary}
        \begin{proof}
        For $d(W_n) = \max\{\max\{d(v,x):x\in W_n\}v\in W_n\}$. Let $u$ be an element in the $C_{n-1}$ subgraph and $w$ be arbitrary. If $w= v$, the midpoint, then $d(u,w) = 1$. If $w$ is adjacent to $u$, then $d(u,w) = 1$. Otherwise $u$ and $w$ are connected by the path $uvw$. Thus, $d(u,w) = 2$. Therefore $d(W_n) = 2$>
        \end{proof}
        \begin{corollary}
        The center $C(W_n)$ is the appended vertex $v$.
        \end{corollary}
        \begin{proof}
        For $d(v,x) = 1 = r(W_n)$ for all other $x$.
        \end{proof}
        \begin{corollary}
        The radii of the maximum spanning tree of $W_n$ is equal to the radii of $P_n$.
        \end{corollary}
        \begin{proof}
        For the points on $C_{n-1}$, label $c_1,\hdots, c_{n-1}$, and let $v$ be the appended vertex. Define the spanning tree $vc_1\cdots c_{n-1}$. This is a $P_n$ graph and is thus the longest possible radii on $n$ points.
        \end{proof}
        \begin{corollary}
        The diameter of the maximum spanning tree of $W_n$ is the diameter of $P_n$>
        \end{corollary}
        \begin{proof}
        From the previous construction, there is a $P_n$ spanning tree and thus this is the maximal diameter on $n$ points.
        \end{proof}
        \begin{corollary}
        The minimum radii of any spanning tree of $W_n$ is $1$.
        \end{corollary}
        \begin{proof}
        For let $c_k$, $k=1,\hdots,n-1$ be the points on the $C_{n-1}$ subgraph and append $v$. Remove all edges incident on $c_k$ except that which lie on incident on $v$ as well. This is a tree. Moreover, for all $x\ne v$, $d(x,v) = 1$.
        \end{proof}
        \begin{corollary}
        The minimum diameter of any spanning tree of $W_n$ is $2$.
        \end{corollary}
        \begin{proof}
        The construction from the previous corollary has diameter $2$. There is no spanning tree of smaller diameter, for any two vertices $u$ and $w$ that are not adjacent in $W_n$ will not be adjacent in the spanning tree, and thus $d(u,w)>1$.
        \end{proof}
        \begin{definition}
        For a graph $G$ with vertices $v_1,\hdots, v_n$, the adjacency matrix $A(G)$ is a $n\times n$ binary matrix (A matrix whose elements are $0$ or $1$) such that the entry $a_{ij}= 1$ if $v_i v_j \in E(G)$, and $a_{ij}=0$ otherwise.
        \end{definition}
        \begin{corollary}
        For any graph $G$, if $a_{ij}$ is the entry of the $i^{th}$ row and $j^{th}$ column of $A(G)$, then the diagonal $a_{ii} = 0$.
        \end{corollary}
        \begin{proof}
        For as no vertex is adjacent to itself, $a_{ii}=0$.
        \end{proof}
        \begin{corollary}
        $A(K_n) = \begin{bmatrix} 0 & 1 &  \hdots & 1 & 1 \\ 1 & 0 & \hdots & 1 & 1 \\ \vdots & \ddots & \ddots & \vdots & \vdots \\ \vdots & \ddots & \ddots & \vdots & \vdots \\ 1 & 1 & \hdots & 1 & 0 \end{bmatrix}$. That is $a_{ij} = \begin{cases} 0, & i=j \\ 1, & i\ne j\end{cases}$
        \end{corollary}
        \begin{proof}
        For if $i\ne j$, then $v_iv_j \in E(K_n)$, and thus $a_{ij}=1$.
        \end{proof}
        \begin{corollary}
        For a graph $G$ with entries $a_{ij}$ of $A(G)$, $a_{ij} = a_{ji}$.
        \end{corollary}
        \begin{proof}
        For if $v_iv_j \in E(G)$, then $v_j v_i \in E(G)$. Thus, $a_{ij} = a_{ji}$.
        \end{proof}
        \begin{definition}
        A directed graph $G$ is a set of vertices $V(G)$ together with directed edges $E(G)$ that form ordered pairs $(a,b)$ between vertices. If $a$ and $b$ are vertices and there is a directed edge from $a$ to $b$ we write $ab \in E(G)$. These are also called digraphs.
        \end{definition}
        \begin{remark}
        It is not necessarily true that $(a,b) \in G$ implies $(b,a) \in G$. Indeed, it is not true that $(a,b) = (b,a)$, as these are ordered pairs.
        \end{remark}
        \begin{definition}
        A graph with multiple edges between vertices is called a multigraph.
        \end{definition}
        \begin{definition}
        In a multigraph, a loop is a connection from a vertex to itself.
        \end{definition}
        \begin{definition}
        If $G$ is a multigraph with vertices $v_1,\hdots, v_n$, then the adjacency matrix $A(G)$ is defined by the entries $a_{ij} = n$, where $n$ is the number of edges from $v_i$ to $v_j$. 
        \end{definition}
        \begin{remark}
        For a multigraph it is not necessarily true that $a_{ii}=0$, as loops may exist.
        \end{remark}
        \begin{corollary}
        If $G$ is a graph with vertices $v_1,\hdots, v_n$, and $a_{ij}$ are the entries of $A(G)$, then $\deg(v_i) = \sum_{j=1}^{n} a_{ij}$.
        \end{corollary}
        \begin{proof}
        For $\deg(v_i) = \sum_{\underset{v_j\ne v_i}{v_j\in V(G)}}\chi_{\{v_i,v_j\}}^G$. But $\chi_{\{v_i,v_j\}}^G = \begin{cases} 1, & v_iv_j\in E(G)\\ 0, & v_iv_j \notin E(G)\end{cases}$. Thus, $\chi_{\{v_i,v_i\}}^G = a_{ij}$, and $\deg(v_i) = \sum_{j=1}^{n} a_{ij}$
        \end{proof}
        \begin{theorem}
        For a graph $G$, it is not necessarily true that $A(G)^2$ is a binary matrix.
        \end{theorem}
        \begin{proof}
        For take $A(G) = \begin{bmatrix} 0 & 0 & 1 \\ 1 & 0 & 1 \\ 1& 1 & 0 \end{bmatrix}$. Then $A(G)^2 = \begin{bmatrix} 1 & 1 & 0 \\ 1 & 1 & 1 \\ 1 & 0 & 2 \end{bmatrix}$
        \end{proof}
        \begin{theorem}
        If $G$ is a graph on $n$ vertices with adjacency matrix $A(G)$ with elements $a_{ij}$, and if $a^2_{ij}$ are the elements of $A^2(G)$, then they represent the number of $v_{i}-v_{j}$ walks of length $2$.
        \end{theorem}
        \begin{proof}
        For $a^2_{ij} = \sum_{k=1}^{n} a_{ik}a_{kj}$. Now $a_{ik}a_{kj}$ is $1$ if and only if both $v_iv_k \in E(G)$ and $v_kv_j \in E(G)$ and zero otherwise.. But then there is a walk of length $2$ from $v_i$ to $v_j$ in the form of $v_i v_k v_j$. Thus, $a^2_{ij}$ is the number walks of length $2$ from $v_i$ to $v_j$.
        \end{proof} 
        \begin{theorem}
        If $G$ is a graph on $n$ vertices with adjacency matrix $A(G)$ with elements $a_{ij}$, and if $a^m_{ij}$ are the elements of $A^m(G)$, then the elements $a^m_{ij}$ are the number of walks of length $m$ from $v_i$ to $v_j$.
        \end{theorem}
        \begin{proof}
        By induction. The base case is solved by the previous theorem. Suppose the elements $b{ij}$ of $A^m(G)$ represent the number of walks of length $k$ from $v_i$ to $v_k$. Then $A^{m+1}(G)=A(G)A^m(G)$. Thus the elements are $c_{ij} = \sum_{k=1}^{n} a_{ik}b_{kj}$. But $a_{ik}b_{kj} = \begin{cases} 0, & v_i v_k \notin E(G) \\ b_{kj}, & v_i v_k \in E(G)\end{cases}$. Thus $c_{ij} = \underset{v_i v_k \in E(G)}\sum b_{kj}$. But $b_{kj}$ is the number of walks from $v_k$ to $v_j$ of length $m$. But $v_i v_k \in E(G)$. But then there is a walk from $v_i$ to $v_j$ of length $m+1$ as $v_iv_k \in E(G)$. And this is a sum over all possible $v_k$ such that $v_iv_k \in E(G)$. Thus $c_{ij}$ is the number of walks from $v_i$ to $v_j$ of length $m+1$.
        \end{proof}
        \begin{theorem}
        If $G$ is a graph on $n$ vertices with adjacency matrix $A(G)$ and $k$ is the smallest number such that $\sum_{i=0}^{k} A^i(G)$ is non-zero, then $d(G) = k$.
        \end{theorem}
        \begin{proof}
        For suppose $\sum_{i=0}^{k}A^i(G)$ has no zero elements, and suppose $\sum_{i=0}^{k-1}A^i(G)$ has a zero element, say $a_{ij}$. Then there are no walk of lengths $1,2 \hdots, k-1$ from $v_i$ to $v_j$. As $\sum_{i=0}^{k} A^i(G)$ has no zeroes, there is a path from $v_{i}$ to $v_{j}$ of length $k$. Moreover, this is the shortest possible path. But as there are no zeros in this sum, every pair $v_k$ and $v_{\ell}$ has a path of at most length $k$. Thus $d(G)=k$.
        \end{proof}
        \begin{theorem}
        If $G$ is a graph with adjacency matrix $A(G)$ and no such $k\in \mathbb{G}$ exists such that $\sum_{i=0}^{k}A^i(G)$ has no zeros, then $G$ is disconnected.
        \end{theorem}
        \begin{proof}
        For the diameter of a connected graph is less than the number of vertices of the graph. Thus $G$ is disconnected.
        \end{proof}
    \section{Graph Theory II}
        \subsection{Spanning Trees}
        \subsection{Product Graphs}
        \subsection{Distance Properties of Graphs}
    \section{Additional Material}
        \subsection{F\'{a}ry's Theorem}
        \subsection{The Art Gallery Theorem}
        \subsection{F\'{a}ry's Theorem}
        \subsection{Kuratowski's Theorem}
        \subsection{Notes}
        \begin{theorem}
        If a graph $G$ has n vertices, and each vertex has more than $\frac{n}{2}$ edges, then it is connected.
        \end{theorem}
        \begin{proof}
        For suppose not. Suppose it can be disconnected into at least two graphs. Then the first graph must have more than $\frac{n}{2}$ vertices, as a given vertices has more than $\frac{n}{2}$ edges. But this must hold as well for any other connected graph. And the sum of these disconnected graphs has more than $n$ vertices, an impossibility. Therefore it is connected.
        \end{proof}
        \begin{definition}
        Two graphs $G$ and $H$ are isomorphic if there exists a bijective function $f:G\rightarrow H$ such that for all $v,w\in V(G)$, $\{v,w\}\in E(G)$ if and only if $\{f(v)<f(w)\}\in E(H)$.
        \end{definition}
        \begin{theorem} If $G$ and $H$ are isomorphic, then $\deg{v} = \deg(f(v))$.
        \end{theorem}
        \begin{proof}
        Define $\xi_{\{v,w\}}^{G}:E(G)\rightarrow \{0,1\}$ by $\xi_{\{v,w\}}^{G} = 0$ if $\{v,w\} \notin E(G)$ and 1 if $\{v,w\}\in E(G)$. Then $\deg(v)=\sum_{w\in V(G),w\ne v}\xi_{\{v,w\}}^{G}$. But $\xi_{\{f(v),f(w)\}}^{H} = \xi_{\{v,w\}}^{G}$, and from this we have:
        \begin{equation*}
            \deg(f(v))=\sum_{f(v)\in V(H),f(v)\ne f(w)}\xi_{\{f(v),f(w)\}}^{H}=\sum_{w\in V(G),w\ne v}\xi_{\{v,w\}}^{G}=\deg(v)
        \end{equation*}
        \end{proof}
        \begin{remark}
        The previous function is an example of a characteristic function.
        \end{remark}
        \begin{problem}
        Given that a graph has diamater $d(G)\geq 3$, what can be said about $d(G^C)$, given $G^C$ is connected.
        \end{problem}
        \subsubsection{Notes on Trees}
            \begin{definition}
                A circuit free graph is a graph $G$ that contains
                no circuits.
            \end{definition}
            \begin{definition}
                A tree is a connected circuit free graph.
            \end{definition}
            \begin{remark}
                A forest is a circuit free graph. That is, a forest
                is the union of trees (Hence the name).
            \end{remark}
            \begin{definition}
                A trivial tree is a graph consisting of a single vertex.
            \end{definition}
            \begin{definition}
                An empty tree is a graph containing no vertices.
            \end{definition}
            \begin{remark}
                Trees appear in computer science often as they are
                useful for solving sorting problems, and lend themselves
                easily to algorithms. All trees are simple graphs, and
                contain no loops.
            \end{remark}
            \begin{definition}
                A leaf in a graph is a vertex of degree 1.
            \end{definition}
            \begin{definition}
                An internal vertex in a graph $G$ is a vertex $v$ such
                that $\deg(v)>1$.
            \end{definition}
            \begin{remark}
                A leaf is also called a \textit{terminal vertex}.
            \end{remark}
            \begin{theorem}
                If $G=(V,E)$ is a tree, and
                $e\in{E}$, then $(V,E\setminus\{e\})$ is
                disconnected.
            \end{theorem}
            \begin{proof}
                For suppose not. Let $e=\{a,b\}$. If $(V,E\setminus\{v\})$
                is connected, then there is a path from $a$ to $b$,
                $x_{1}\rightarrow\cdots\rightarrow{x_{n}}$. But then
                $a\rightarrow{x_{1}}\rightarrow\cdots%
                 \rightarrow{x_{n}}\rightarrow{b}\rightarrow{a}$ is a
                cycle in $(V,E)$, a contradiction as $G$ is a graph
                and thus contains no cycles. Therefore, etc.
            \end{proof}
            \begin{theorem}
                If $G$ is a non-trivial and non-empty tree, then
                it contains at least two leaves.
            \end{theorem}
            \begin{proof}
                By induction. The base case is $n=2$ which is
                two vertices with an edge between them, and therefore
                both vertices have degree 1. Suppose it is true for $n$.
                If $G=(V,E)$ is a tree on $n+1$ edge, then there is
                an edge $e\in{E}$. Let $e=\{a,b\}$. Let
                $(V_{1},E_{1})$ be the subgraph formed from
                $(V,E\setminus\{e\})$ by all of the vertices that are
                connected to $a$, and let $(V_{1},E_{1})$ be the
                subgraph of all vertices connected to $b$. As
                $(V,E\setminus\{e\})$ is disconnected,
                no element of $(V_{1},E_{1})$ is connected to an element
                of $(V_{2},E_{2})$. Moreover,
                $(V,E)=(V_{1},E_{1})\cup(V_{2},E_{2})\cup(\{a,b\},e\}$.
                But also $(V_{1},E_{1})$ and $(V_{2},E_{2})$ are trees
                and $|V_{1}|<n+1$, $|V_{2}|<n+1$. But then there are
                two leaves in $(V_{1},E_{1})$ and two leaves in
                $(V_{2},E_{2})$. Let $v_{1},v_{2}$ be such leaves that
                are not equal to $a$ and $b$. Then
                $v_{1}$ is not connected to $b$ in $(V,E)$, and
                $v_{2}$ is not connected to $a$ in $(V,E)$, and therefore
                the degrees of $v_{1}$ and $v_{2}$ do not change.
                Therefore $\deg(v_{1})=\deg(v_{2})=1$, and they are
                leaves. Therefore, etc.
            \end{proof}
            \begin{theorem}
                If $G$ is a tree with $n$ vertices, then it has
                $n-1$ edges.
            \end{theorem}
            \begin{proof}
                By induction. The base case is trivial on one vertex.
                Suppose it is true for $n$. If $G$ is a graph on
                $n+1$ vertices, then there is a leaf $v$.
                Then then subgraph formed by
                $V\setminus\{v\}$ is a tree on $n$ vertices. But then,
                by induction, there are $n-1$ vertices. As $v$ is a leaf,
                $G$ has $(n-1)+1=n$ vertices.
            \end{proof}
            \begin{theorem}
                If $G$ is a connected graph with $n$ vertices and
                $n-1$ edges, then $G$ is a tree.
            \end{theorem}
            \begin{proof}
                Suppose not. Suppose $G$ is a connected graph
                on $n$ vertices with $n-1$ edges that contains a
                circuit. Let $e$ be an edge in the circuit.
                Then $(V,E\setminus\{e\})$ is connected. Continue removing
                edges from circuits until we obtain a spanning tree.
                At no point did we ever remove vertices, so we have
                a tree on $n$ vertices with less than $n-1$ edges,
                a contradiction. Thus, $G$ is a tree.
            \end{proof}
            \begin{definition}
                A rooted tree is a tree with distinct vertex
                called the root of the tree.
            \end{definition}
            \begin{definition}
                The level of a vertex in a rooted tree is the number
                of edges between the vertex and the root.
            \end{definition}
            \begin{definition}
                The height of a tree is the maximum level of any
                vertex in the tree.
            \end{definition}
            \begin{definition}
                The children of an internal vertex in a rooted tree
                are the adjacent vertices that are one level
                higher.
            \end{definition}
            \begin{definition}
                The parent of a vertex in a rooted tree is
                the adjacent vertex one level below.
            \end{definition}
            \begin{definition}
                An ancestor of a vertex in a rooted tree is a vertex
                that lies on the path between the vertex and the root.
            \end{definition}
            \begin{definition}
                A descendant of a vertex is a vertex that is of a higher
                and whose path does not contain the root.
            \end{definition}
            \begin{definition}
                A binary tree is a rooted tree in which each internal
                vertex has at most two children.
            \end{definition}
            \begin{definition}
                A full binary tree is a binary tree in which every
                internal vertex has exactly two children.
            \end{definition}
            \begin{remark}
                The children in a full binary tree can be labelled
                ``left child'' and ``right child.'' From this there
                is a notion of ``left subtree'' and ``right subtree''
                with respect to the root of the tree.
            \end{remark}
            \begin{theorem}
                If $n\in\mathbb{N}$ and $T$ is a full binary tree with
                $n$ internal vertices,
                then $T$ has $n+1$ leaves and $2n+1$ vertices.
            \end{theorem}
            \begin{proof}
                If $T$ is a full binary tree with $n$ internal vertices,
                then every internal vertex has 2 children. Thus, the
                number of vertices that have a parent is equal to twice
                the number of internal vertices, which is $2n$. But the
                only vertex without a parent is the root. Thus, the
                total number of vertices is $2n+1$. Moreover, the number
                of points is equal to the sum of the number of leaves
                with the number of internal vertices. But there are
                $2n+1$ points and $n$ vertices, and therefore there
                are $n+1$ leaves.
            \end{proof}
            \begin{theorem}
                If $T$ is a binary tree with $\ell$ leaves and height
                $h$, then $\ell\leq2^{h}$.
            \end{theorem}
            \begin{proof}
                By induction on $h$. The base case case is $h=0$, in
                which the tree is trivial. Suppose it is true for
                $h=n$. Let $G$ be a rooted tree of heigh $n+1$.
                Removing the root creates two rooted subtrees,
                each of height $n$. But then there are, at most
                $2^{n}$ leaves in each subtree. Connecting the root
                removes two leaves (The roots of the two subtrees),
                and adds one (The original root). Thus we have
                $\ell\leq{2^{n}+2^{n}-2+1}=2^{n+1}-1$.
            \end{proof}
            \begin{definition}
                A spanning tree of a graph $G$ is a subgraph
                $H$ of $G$ such that $H$ is a tree and every
                vertex in $G$ is a vertex in $H$.
            \end{definition}
            \begin{theorem}
                Every connected graph has a spanning tree.
            \end{theorem}
            \begin{proof}
                By induction on the number of edges. The base cases
                is $n=0$ or $n=1$, in both cases the graph $G$ is its
                own spanning tree. Suppose it true for all graphs with
                $n$ vertices. Let $G$ be a graph with $n+1$ vertices.
                If $G$ is a tree, we are done. Suppose not. Then there
                is a circuit in $G$. Let $e$ be an edge in the circuit.
                Then $(V,E\setminus\{e\})$ is connected and a graph
                with $n$ edges. But then there is a spanning tree. But
                then this is a tree containing all vertices of $G$,
                and is thus a spanning tree. Therefore, etc.
            \end{proof}
            \begin{theorem}
                If $G$ is a connected graph with $n$ vertices, and if
                $H_{1}$ and $H_{2}$ are spanning trees,
                then the number of edges in $H_{1}$ is equal to
                the number of edges in $H_{2}$.
            \end{theorem}
            \begin{proof}
                If $H_{1}$ and $H_{2}$ are spanning trees of
                $G$, then they are both trees with $n$ vertices and
                thus have $n-1$ edges.
            \end{proof}
            \begin{definition}
                A weighted graph is a graph $G$ with a function
                $w:E\rightarrow\mathbb{R}$, we $E$ is the edge set.
            \end{definition}
            \begin{remark}
                A weighted graph is a graph $G$ such that every
                edge of $G$ has a weight attached to it, some
                real number.
            \end{remark}
            \begin{definition}
                The total weight of a weighted graph $G$ is the
                sum of all of the weights of all of the edges in $G$.
            \end{definition}
            \begin{definition}
                A minimal spanning tree of a weighted graph $G$
                is a spanning tree of $G$ with the least possible
                total weight.
            \end{definition}
            There are two algorithm's for computing the minimal
            spanning tree of a weight graph $G$.
            \begin{theorem}[Kruskal's Algorithm]
                Every connected weighted graph $G$
                has a minimal spanning tree.
            \end{theorem}
            \begin{proof}
                Let $G=(V,E)$ be a connected graph on $n$ vertices.
                Define $T_{0}=(V,\emptyset)$. That is, the completely
                disconnected graph on $n$ vertices. Let $E_{0}=E$,
                and let $e_{0}\in{E}$ be an element of minimal weight.
                Let $T_{1}=(V,\{e\})$, $E_{1}=\setminus\{e\}$.
                For $m<n$, inductively define $e_{m-1}$ to be a minimal
                element of $E_{m-1}$, $T_{m}=T_{m-1}\cup\{e_{m}\}$,
                and $E_{m}=E_{m-1}\setminus\{e_{m-1}\}$.
                Then $T_{n}$ is a minimal spanning tree of $G$.
            \end{proof}
            \begin{theorem}[Prim's Algorithm]
                Every connected weighted graph $G$ has a minimal
                spanning tree.
            \end{theorem}
            \begin{proof}
                Let $v$ be a vertex in $V$.
                Let $V_{0}=V\setminus\{v\}$,
                $U_{0}=\{v\}$, and $F_{0}=\emptyset$.
                Let $T_{0}$ be the graph $(U_{0},F_{0})$.
                For $i=1,\hdots,n-1$, find a minimum weight
                $e\in{E_{i}}$ so that $e=\{u,w\}$, where
                $u\notin{V_{i}}$ and $w\in{V_{i}}$.
                Let $U_{i+1}=U_{i}\cup\{w\}$,
                $F_{i+1}=F_{i}\cup\{e\}$,
                $V_{i+1}=V_{i}\setminus\{w\}$, and
                $E_{i+1}=E_{i}\setminus\{e\}$.
                Then $(U_{n},F_{n})$ is a minimal spanning tree of
                $G$.
            \end{proof}
\end{document}
        \section{Relations of Order}
    \begin{definition}
        Given a set $A$, a total order on $A$ is a relation $\leq$ with the
        following properties: For all $a,b,c\in A$,
        \begin{enumerate}
            \item   $a\leq b$ and $b\leq a$ if and only if $a=b$.
                    \hfill[Antisymmetry]
            \item   If $a\leq{b}$ and $b\leq c$, then $a\leq{c}$.
                    \hfill[Transitivity]
            \item Either $a\leq b$, or $b\leq a$, or both.\hfill[Totality]
        \end{enumerate}
    \end{definition}
    If $a\leq b$, we may also write $b\geq a$.
    \begin{definition}
        Given a set $A$, a strict relation of order is a relation $<$ with the
        following properties: For all $a,b,c\in A$,
        \begin{enumerate}
            \item   Precisely one of the following is true: $a<b$, $b<a$, $a=b$.
                    \hfill [Trichotomy]
            \item   If $a<b$ and $b<c$, then $a<c$.
                    \hfill[Transitivity]
        \end{enumerate}
    \end{definition}
    \begin{definition}
        An ordered field is a field $\langle F,+,\cdot \rangle$ with a total
        order $\leq$ with the following properties: For all $a,b,c\in F$,
        \begin{enumerate}
            \item   If $a\leq b$, then $a+c\leq b+c$
            \item   If $0 \leq a$ and $0\leq b$, then $0\leq a\cdot b$
            \item   $0\leq 1$
        \end{enumerate}
    \end{definition}
    If $a\leq b$ and $a\ne b$, we write $a<b$.
    \begin{theorem}
        In a field, $(ab)^{2}=a^{2}b^{2}$.
    \end{theorem}
    \begin{proof}
        For $(ab)^{2}=(ab)(ab)=(a)(b)(a)(b)=(a)(a)(b)(b)=a^{2}b^{2}$.
    \end{proof}
    \begin{theorem}
        In an ordered field, if $0\leq a$ and $0\leq b$, then $0\leq a+b$.
    \end{theorem}
    \begin{proof}
        For as $0\leq a$, $0+b\leq a+b$. But $0+b = b$ and $0\leq b$. From
        transitivity, $0\leq a+b$.
    \end{proof}
    \begin{theorem}
        In an ordered field, if $0\leq x$, then $-x\leq 0$.
    \end{theorem}
    \begin{proof}
        For $0\leq x$, and thus $(-x)=0+(-x)\leq x+(-x)=0$. From transitivity,
        $(-x)\leq 0$.
    \end{proof}
    \begin{theorem}
        In a field, $(-1)^2 = 1$.
    \end{theorem}
    \begin{proof}
        For $(-1)^2 +(-1) = (-1)(-1+1) = (-1)\cdot 0 = 0$. As additive inverses
        are unique, $(-1)^2 = 1$.
    \end{proof}
    \begin{theorem}
        In an ordered field, $0\leq x^2$.
    \end{theorem}
    \begin{proof}
        If $0 \leq x$, we are done. Suppose $x\leq 0$. Then
        $0\leq(-x)=(-1)x$, and thus $0\leq (-1)^2 x^2=x^2$.
    \end{proof}
    \begin{theorem}
        In an ordered field, $a\leq b$ if and only if $0 \leq b-a$
    \end{theorem}
    \begin{proof}
        For suppose $a\leq b$. Then $0=a+(-a)\leq b-a\Rightarrow 0 \leq b-a$.
        If $0\leq b-a$, then $a=0+a \leq (b-a)+a = b\Rightarrow a\leq b$.
    \end{proof}
    \begin{theorem}
        If $a\leq b$, then $-b\leq -a$.
    \end{theorem}
    \begin{proof}
        For then $0 \leq b-a$, and thus $-(b-a)=a-b\leq 0$, and therefore
        $-b\leq-a$.
    \end{proof}
    \begin{theorem}
        In an ordered field, if $a\leq b$ and $c\leq d$, then $a+c \leq b+d$.
    \end{theorem}
    \begin{proof}
        For $0\leq b-a$ and $0\leq d-c$. Thus,
        $0\leq(b-a)+(d-c)=(b+d)-(a+c)$, and therefore $a+c \leq b+d$.
    \end{proof}
    \begin{theorem}
        In an ordered field, if $0\leq a$ and $b\leq 0$, then $ab\leq 0$.
    \end{theorem}
    \begin{proof}
        For as $b\leq 0$, $0\leq -b$, and thus $0\leq -ba$, and therefore
        $-(-ba) = ba \leq 0$.
    \end{proof}
    \begin{theorem}
        If $0< a$, then $0<\frac{1}{a}$.
    \end{theorem}
    \begin{proof}
        For $\frac{1}{a}\ne 0$ as it is invertible, and $0$ is not. But
        $0\leq1=a\cdot \frac{1}{a}$ and $0<a$ and thus $\frac{1}{a} \not <0$.
        Therefore $0<\frac{1}{a}$.
    \end{proof}
    \begin{theorem}
        In an ordered field, if $0<a\leq b$, then
        $0<\frac{1}{b}\leq\frac{1}{a}$.
    \end{theorem}
    \begin{proof}
        As $a\leq b$:
        \begin{equation}
            \frac{1}{b}
            =a\cdot\frac{1}{ba}\leq{b}\cdot\frac{1}{ba}
            =\frac{1}{a}
        \end{equation}
        Thus, $0< \frac{1}{b}\leq \frac{1}{a}$.
    \end{proof}
    \begin{theorem}
        In an ordered field, if $0 \leq a \leq b$, then $a^2 \leq b^2$.
    \end{theorem}
    \begin{proof}
        For as $0\leq{a}\leq{b}$, $a\cdot{a}\leq{b}\cdot{a}$. Thus,
        $a^{2}\leq{b}\cdot{a}$. But also $a\cdot{b}\leq{b}\cdot{b}$. Thus,
        $a\cdot{b}\leq{b}^{2}$. By transitivity, $a^{2}\leq{b}^{2}$.
    \end{proof}
    \begin{theorem}
        If $1\leq a$, then $a \leq a^2$. If $0\leq a \leq 1$, then
        $a^2 \leq a$.
    \end{theorem}
    \begin{proof}
        For as $1\leq a$, $a=1\cdot a \leq a^2$. If $0\leq a \leq 1$, then
        $a^2 \leq 1\cdot a = a$.
    \end{proof}
        \section{Zorn's Lemma}
    A relation on a set $X$ is a subset $R\subseteq{X}\times{X}$. Given an
    element $(x,y)\in{R}$, we often write $xRy$ to denote this. Here we'll write
    $x\leq{y}$.
    \begin{ldefinition}{Ordered Sets}{Ordered_Set}
        An ordered set is a set $X$ and a relation $\leq$ on $X$, denoted
        $(X,\leq)$ such that the following are true:
        \begin{enumerate}
            \item   For all $x\in{X}$, $x\leq{x}$.
            \item   For all $x,y\in{X}$ such that $x\leq{y}$ and $y\leq{x}$, it
                    is true that $x=y$.
            \item   For all $x,y,z\in{X}$ such that $x\leq{y}$ and $y\leq{z}$,
                    it is true that $x\leq{z}$.
        \end{enumerate}
    \end{ldefinition}
    \begin{ldefinition}{Majorants in Ordered Sets}{Majorant_in_Ord_Set}
        A majorant of a subset $Y\subseteq{X}$ of and ordered set $(X,\leq)$ is
        an element $x\in{X}$ such that, for all $y\in{Y}$, it is true that
        $y\leq{x}$.
    \end{ldefinition}
    \begin{lexample}{}{}
        Let $(X,d)$ be a metric space, and let $x_{0}\in{X}$. Define the
        following:
        \begin{equation}
            \mathscr{N}(x_{0})=
            \big\{\mathcal{V}\subseteq{X}:\mathcal{V}
                \textrm{ is a neighborhood of $x_{0}$}\big\}
        \end{equation}
        We can order $\mathscr{N}$ by reverse containment. That is, We have the
        following relation:
        \begin{equation}
            \leq=\big\{(\mathcal{U},\mathcal{V})\in
                \mathscr{N}(x_{0})\times\mathscr{N}(x_{0})
                :\mathcal{V}\subseteq\mathcal{U}\big\}
        \end{equation}
        That is, we write $\mathcal{U}\leq\mathcal{V}$ if $\mathcal{V}$ is a
        subset of $\mathcal{U}$. Note that, for all $x_{0}$, $\mathscr{x_{0}}$
        has a least element, or a minorant, but $X$ is such an element. But, if
        $\{x_{0}\}$ is not open, then there is no majorant.
    \end{lexample}
    \begin{ldefinition}{Totally Ordered Sets}{Tot_Ord_Set}
        A totally ordered set is an ordered set $(X,\leq)$ such that, for all
        $x,y\in{X}$, either $x\leq{y}$ or $y\leq{x}$.
    \end{ldefinition}
    \begin{ldefinition}{Maximal Element}{Maximal_Element}
        A maximal element of a subset $Y\subseteq{X}$ of a totally ordered set
        $(X,\leq)$ is an element $y$ such that:
        \begin{equation}
            \{y'\in{Y}:y\leq{y}'\}=\{y\}
        \end{equation}
        Note that $y$ is not necessary a majorant for $Y$ nor is $y$ necessarily
        unique.
    \end{ldefinition}
    \begin{ldefinition}{Inductively Ordered Sets}{Induct_Ordered_Set}
        An inductively ordered set is an ordered set $)X,\leq)$ such that, for
        all totally ordered subsets $S\subseteq{X}$, there is a majorant
        $x\in{X}$ of $S$.
    \end{ldefinition}
    That is, there exists $x\in{X}$ such that, for all $y\in{S}$, $y\leq{x}$.
    Let $X=\mathbb{R}$ and consider the set $\mathscr{N}(0)$. Then
    $\mathscr{N}(0)$ is not inductively ordered.
    \begin{ltheorem}{Zorn's Lemma}{Zorns_Lemma}
        If $(X,\leq)$ is an inductively ordered set, then there is a maximal
        element $x\in{X}$.
    \end{ltheorem}

    \renewcommand{\PATH}{\OLDPATH}
\endgroup
        \part{Category Theory}
            \chapter{Category Theory}
        \part{Model Theory}
            \documentclass[crop=false,class=book,oneside]{standalone}
%----------------------------Preamble-------------------------------%
%---------------------------Packages----------------------------%
\usepackage{geometry}
\geometry{b5paper, margin=1.0in}
\usepackage[T1]{fontenc}
\usepackage{graphicx, float}            % Graphics/Images.
\usepackage{natbib}                     % For bibliographies.
\bibliographystyle{agsm}                % Bibliography style.
\usepackage[french, english]{babel}     % Language typesetting.
\usepackage[dvipsnames]{xcolor}         % Color names.
\usepackage{listings}                   % Verbatim-Like Tools.
\usepackage{mathtools, esint, mathrsfs} % amsmath and integrals.
\usepackage{amsthm, amsfonts, amssymb}  % Fonts and theorems.
\usepackage{tcolorbox}                  % Frames around theorems.
\usepackage{upgreek}                    % Non-Italic Greek.
\usepackage{fmtcount, etoolbox}         % For the \book{} command.
\usepackage[newparttoc]{titlesec}       % Formatting chapter, etc.
\usepackage{titletoc}                   % Allows \book in toc.
\usepackage[nottoc]{tocbibind}          % Bibliography in toc.
\usepackage[titles]{tocloft}            % ToC formatting.
\usepackage{pgfplots, tikz}             % Drawing/graphing tools.
\usepackage{imakeidx}                   % Used for index.
\usetikzlibrary{
    calc,                   % Calculating right angles and more.
    angles,                 % Drawing angles within triangles.
    arrows.meta,            % Latex and Stealth arrows.
    quotes,                 % Adding labels to angles.
    positioning,            % Relative positioning of nodes.
    decorations.markings,   % Adding arrows in the middle of a line.
    patterns,
    arrows
}                                       % Libraries for tikz.
\pgfplotsset{compat=1.9}                % Version of pgfplots.
\usepackage[font=scriptsize,
            labelformat=simple,
            labelsep=colon]{subcaption} % Subfigure captions.
\usepackage[font={scriptsize},
            hypcap=true,
            labelsep=colon]{caption}    % Figure captions.
\usepackage[pdftex,
            pdfauthor={Ryan Maguire},
            pdftitle={Mathematics and Physics},
            pdfsubject={Mathematics, Physics, Science},
            pdfkeywords={Mathematics, Physics, Computer Science, Biology},
            pdfproducer={LaTeX},
            pdfcreator={pdflatex}]{hyperref}
\hypersetup{
    colorlinks=true,
    linkcolor=blue,
    filecolor=magenta,
    urlcolor=Cerulean,
    citecolor=SkyBlue
}                           % Colors for hyperref.
\usepackage[toc,acronym,nogroupskip,nopostdot]{glossaries}
\usepackage{glossary-mcols}
%------------------------Theorem Styles-------------------------%
\theoremstyle{plain}
\newtheorem{theorem}{Theorem}[section]

% Define theorem style for default spacing and normal font.
\newtheoremstyle{normal}
    {\topsep}               % Amount of space above the theorem.
    {\topsep}               % Amount of space below the theorem.
    {}                      % Font used for body of theorem.
    {}                      % Measure of space to indent.
    {\bfseries}             % Font of the header of the theorem.
    {}                      % Punctuation between head and body.
    {.5em}                  % Space after theorem head.
    {}

% Italic header environment.
\newtheoremstyle{thmit}{\topsep}{\topsep}{}{}{\itshape}{}{0.5em}{}

% Define environments with italic headers.
\theoremstyle{thmit}
\newtheorem*{solution}{Solution}

% Define default environments.
\theoremstyle{normal}
\newtheorem{example}{Example}[section]
\newtheorem{definition}{Definition}[section]
\newtheorem{problem}{Problem}[section]

% Define framed environment.
\tcbuselibrary{most}
\newtcbtheorem[use counter*=theorem]{ftheorem}{Theorem}{%
    before=\par\vspace{2ex},
    boxsep=0.5\topsep,
    after=\par\vspace{2ex},
    colback=green!5,
    colframe=green!35!black,
    fonttitle=\bfseries\upshape%
}{thm}

\newtcbtheorem[auto counter, number within=section]{faxiom}{Axiom}{%
    before=\par\vspace{2ex},
    boxsep=0.5\topsep,
    after=\par\vspace{2ex},
    colback=Apricot!5,
    colframe=Apricot!35!black,
    fonttitle=\bfseries\upshape%
}{ax}

\newtcbtheorem[use counter*=definition]{fdefinition}{Definition}{%
    before=\par\vspace{2ex},
    boxsep=0.5\topsep,
    after=\par\vspace{2ex},
    colback=blue!5!white,
    colframe=blue!75!black,
    fonttitle=\bfseries\upshape%
}{def}

\newtcbtheorem[use counter*=example]{fexample}{Example}{%
    before=\par\vspace{2ex},
    boxsep=0.5\topsep,
    after=\par\vspace{2ex},
    colback=red!5!white,
    colframe=red!75!black,
    fonttitle=\bfseries\upshape%
}{ex}

\newtcbtheorem[auto counter, number within=section]{fnotation}{Notation}{%
    before=\par\vspace{2ex},
    boxsep=0.5\topsep,
    after=\par\vspace{2ex},
    colback=SeaGreen!5!white,
    colframe=SeaGreen!75!black,
    fonttitle=\bfseries\upshape%
}{not}

\newtcbtheorem[use counter*=remark]{fremark}{Remark}{%
    fonttitle=\bfseries\upshape,
    colback=Goldenrod!5!white,
    colframe=Goldenrod!75!black}{ex}

\newenvironment{bproof}{\textit{Proof.}}{\hfill$\square$}
\tcolorboxenvironment{bproof}{%
    blanker,
    breakable,
    left=3mm,
    before skip=5pt,
    after skip=10pt,
    borderline west={0.6mm}{0pt}{green!80!black}
}

\AtEndEnvironment{lexample}{$\hfill\textcolor{red}{\blacksquare}$}
\newtcbtheorem[use counter*=example]{lexample}{Example}{%
    empty,
    title={Example~\theexample},
    boxed title style={%
        empty,
        size=minimal,
        toprule=2pt,
        top=0.5\topsep,
    },
    coltitle=red,
    fonttitle=\bfseries,
    parbox=false,
    boxsep=0pt,
    before=\par\vspace{2ex},
    left=0pt,
    right=0pt,
    top=3ex,
    bottom=1ex,
    before=\par\vspace{2ex},
    after=\par\vspace{2ex},
    breakable,
    pad at break*=0mm,
    vfill before first,
    overlay unbroken={%
        \draw[red, line width=2pt]
            ([yshift=-1.2ex]title.south-|frame.west) to
            ([yshift=-1.2ex]title.south-|frame.east);
        },
    overlay first={%
        \draw[red, line width=2pt]
            ([yshift=-1.2ex]title.south-|frame.west) to
            ([yshift=-1.2ex]title.south-|frame.east);
    },
}{ex}

\AtEndEnvironment{ldefinition}{$\hfill\textcolor{Blue}{\blacksquare}$}
\newtcbtheorem[use counter*=definition]{ldefinition}{Definition}{%
    empty,
    title={Definition~\thedefinition:~{#1}},
    boxed title style={%
        empty,
        size=minimal,
        toprule=2pt,
        top=0.5\topsep,
    },
    coltitle=Blue,
    fonttitle=\bfseries,
    parbox=false,
    boxsep=0pt,
    before=\par\vspace{2ex},
    left=0pt,
    right=0pt,
    top=3ex,
    bottom=0pt,
    before=\par\vspace{2ex},
    after=\par\vspace{1ex},
    breakable,
    pad at break*=0mm,
    vfill before first,
    overlay unbroken={%
        \draw[Blue, line width=2pt]
            ([yshift=-1.2ex]title.south-|frame.west) to
            ([yshift=-1.2ex]title.south-|frame.east);
        },
    overlay first={%
        \draw[Blue, line width=2pt]
            ([yshift=-1.2ex]title.south-|frame.west) to
            ([yshift=-1.2ex]title.south-|frame.east);
    },
}{def}

\AtEndEnvironment{ltheorem}{$\hfill\textcolor{Green}{\blacksquare}$}
\newtcbtheorem[use counter*=theorem]{ltheorem}{Theorem}{%
    empty,
    title={Theorem~\thetheorem:~{#1}},
    boxed title style={%
        empty,
        size=minimal,
        toprule=2pt,
        top=0.5\topsep,
    },
    coltitle=Green,
    fonttitle=\bfseries,
    parbox=false,
    boxsep=0pt,
    before=\par\vspace{2ex},
    left=0pt,
    right=0pt,
    top=3ex,
    bottom=-1.5ex,
    breakable,
    pad at break*=0mm,
    vfill before first,
    overlay unbroken={%
        \draw[Green, line width=2pt]
            ([yshift=-1.2ex]title.south-|frame.west) to
            ([yshift=-1.2ex]title.south-|frame.east);},
    overlay first={%
        \draw[Green, line width=2pt]
            ([yshift=-1.2ex]title.south-|frame.west) to
            ([yshift=-1.2ex]title.south-|frame.east);
    }
}{thm}

%--------------------Declared Math Operators--------------------%
\DeclareMathOperator{\adjoint}{adj}         % Adjoint.
\DeclareMathOperator{\Card}{Card}           % Cardinality.
\DeclareMathOperator{\curl}{curl}           % Curl.
\DeclareMathOperator{\diam}{diam}           % Diameter.
\DeclareMathOperator{\dist}{dist}           % Distance.
\DeclareMathOperator{\Div}{div}             % Divergence.
\DeclareMathOperator{\Erf}{Erf}             % Error Function.
\DeclareMathOperator{\Erfc}{Erfc}           % Complementary Error Function.
\DeclareMathOperator{\Ext}{Ext}             % Exterior.
\DeclareMathOperator{\GCD}{GCD}             % Greatest common denominator.
\DeclareMathOperator{\grad}{grad}           % Gradient
\DeclareMathOperator{\Ima}{Im}              % Image.
\DeclareMathOperator{\Int}{Int}             % Interior.
\DeclareMathOperator{\LC}{LC}               % Leading coefficient.
\DeclareMathOperator{\LCM}{LCM}             % Least common multiple.
\DeclareMathOperator{\LM}{LM}               % Leading monomial.
\DeclareMathOperator{\LT}{LT}               % Leading term.
\DeclareMathOperator{\Mod}{mod}             % Modulus.
\DeclareMathOperator{\Mon}{Mon}             % Monomial.
\DeclareMathOperator{\multideg}{mutlideg}   % Multi-Degree (Graphs).
\DeclareMathOperator{\nul}{nul}             % Null space of operator.
\DeclareMathOperator{\Ord}{Ord}             % Ordinal of ordered set.
\DeclareMathOperator{\Prin}{Prin}           % Principal value.
\DeclareMathOperator{\proj}{proj}           % Projection.
\DeclareMathOperator{\Refl}{Refl}           % Reflection operator.
\DeclareMathOperator{\rk}{rk}               % Rank of operator.
\DeclareMathOperator{\sgn}{sgn}             % Sign of a number.
\DeclareMathOperator{\sinc}{sinc}           % Sinc function.
\DeclareMathOperator{\Span}{Span}           % Span of a set.
\DeclareMathOperator{\Spec}{Spec}           % Spectrum.
\DeclareMathOperator{\supp}{supp}           % Support
\DeclareMathOperator{\Tr}{Tr}               % Trace of matrix.
%--------------------Declared Math Symbols--------------------%
\DeclareMathSymbol{\minus}{\mathbin}{AMSa}{"39} % Unary minus sign.
%------------------------New Commands---------------------------%
\DeclarePairedDelimiter\norm{\lVert}{\rVert}
\DeclarePairedDelimiter\ceil{\lceil}{\rceil}
\DeclarePairedDelimiter\floor{\lfloor}{\rfloor}
\newcommand*\diff{\mathop{}\!\mathrm{d}}
\newcommand*\Diff[1]{\mathop{}\!\mathrm{d^#1}}
\renewcommand*{\glstextformat}[1]{\textcolor{RoyalBlue}{#1}}
\renewcommand{\glsnamefont}[1]{\textbf{#1}}
\renewcommand\labelitemii{$\circ$}
\renewcommand\thesubfigure{%
    \arabic{chapter}.\arabic{figure}.\arabic{subfigure}}
\addto\captionsenglish{\renewcommand{\figurename}{Fig.}}
\numberwithin{equation}{section}

\renewcommand{\vector}[1]{\boldsymbol{\mathrm{#1}}}

\newcommand{\uvector}[1]{\boldsymbol{\hat{\mathrm{#1}}}}
\newcommand{\topspace}[2][]{(#2,\tau_{#1})}
\newcommand{\measurespace}[2][]{(#2,\varSigma_{#1},\mu_{#1})}
\newcommand{\measurablespace}[2][]{(#2,\varSigma_{#1})}
\newcommand{\manifold}[2][]{(#2,\tau_{#1},\mathcal{A}_{#1})}
\newcommand{\tanspace}[2]{T_{#1}{#2}}
\newcommand{\cotanspace}[2]{T_{#1}^{*}{#2}}
\newcommand{\Ckspace}[3][\mathbb{R}]{C^{#2}(#3,#1)}
\newcommand{\funcspace}[2][\mathbb{R}]{\mathcal{F}(#2,#1)}
\newcommand{\smoothvecf}[1]{\mathfrak{X}(#1)}
\newcommand{\smoothonef}[1]{\mathfrak{X}^{*}(#1)}
\newcommand{\bracket}[2]{[#1,#2]}

%------------------------Book Command---------------------------%
\makeatletter
\renewcommand\@pnumwidth{1cm}
\newcounter{book}
\renewcommand\thebook{\@Roman\c@book}
\newcommand\book{%
    \if@openright
        \cleardoublepage
    \else
        \clearpage
    \fi
    \thispagestyle{plain}%
    \if@twocolumn
        \onecolumn
        \@tempswatrue
    \else
        \@tempswafalse
    \fi
    \null\vfil
    \secdef\@book\@sbook
}
\def\@book[#1]#2{%
    \refstepcounter{book}
    \addcontentsline{toc}{book}{\bookname\ \thebook:\hspace{1em}#1}
    \markboth{}{}
    {\centering
     \interlinepenalty\@M
     \normalfont
     \huge\bfseries\bookname\nobreakspace\thebook
     \par
     \vskip 20\p@
     \Huge\bfseries#2\par}%
    \@endbook}
\def\@sbook#1{%
    {\centering
     \interlinepenalty \@M
     \normalfont
     \Huge\bfseries#1\par}%
    \@endbook}
\def\@endbook{
    \vfil\newpage
        \if@twoside
            \if@openright
                \null
                \thispagestyle{empty}%
                \newpage
            \fi
        \fi
        \if@tempswa
            \twocolumn
        \fi
}
\newcommand*\l@book[2]{%
    \ifnum\c@tocdepth >-3\relax
        \addpenalty{-\@highpenalty}%
        \addvspace{2.25em\@plus\p@}%
        \setlength\@tempdima{3em}%
        \begingroup
            \parindent\z@\rightskip\@pnumwidth
            \parfillskip -\@pnumwidth
            {
                \leavevmode
                \Large\bfseries#1\hfill\hb@xt@\@pnumwidth{\hss#2}
            }
            \par
            \nobreak
            \global\@nobreaktrue
            \everypar{\global\@nobreakfalse\everypar{}}%
        \endgroup
    \fi}
\newcommand\bookname{Book}
\renewcommand{\thebook}{\texorpdfstring{\Numberstring{book}}{book}}
\providecommand*{\toclevel@book}{-2}
\makeatother
\titleformat{\part}[display]
    {\Large\bfseries}
    {\partname\nobreakspace\thepart}
    {0mm}
    {\Huge\bfseries}
\titlecontents{part}[0pt]
    {\large\bfseries}
    {\partname\ \thecontentslabel: \quad}
    {}
    {\hfill\contentspage}
\titlecontents{chapter}[0pt]
    {\bfseries}
    {\chaptername\ \thecontentslabel:\quad}
    {}
    {\hfill\contentspage}
\newglossarystyle{longpara}{%
    \setglossarystyle{long}%
    \renewenvironment{theglossary}{%
        \begin{longtable}[l]{{p{0.25\hsize}p{0.65\hsize}}}
    }{\end{longtable}}%
    \renewcommand{\glossentry}[2]{%
        \glstarget{##1}{\glossentryname{##1}}%
        &\glossentrydesc{##1}{~##2.}
        \tabularnewline%
        \tabularnewline
    }%
}
\newglossary[not-glg]{notation}{not-gls}{not-glo}{Notation}
\newcommand*{\newnotation}[4][]{%
    \newglossaryentry{#2}{type=notation, name={\textbf{#3}, },
                          text={#4}, description={#4},#1}%
}
%--------------------------LENGTHS------------------------------%
% Spacings for the Table of Contents.
\addtolength{\cftsecnumwidth}{1ex}
\addtolength{\cftsubsecindent}{1ex}
\addtolength{\cftsubsecnumwidth}{1ex}
\addtolength{\cftfignumwidth}{1ex}
\addtolength{\cfttabnumwidth}{1ex}

% Indent and paragraph spacing.
\setlength{\parindent}{0em}
\setlength{\parskip}{0em}
%----------------------------GLOSSARY-------------------------------%
\makeglossaries
\loadglsentries{../../glossary}
\loadglsentries{../../acronym}
%--------------------------Main Document----------------------------%
\begin{document}
    \ifx\ifmain\undefined
        \pagenumbering{roman}
        \title{Model Theory}
        \author{Ryan Maguire}
        \date{\vspace{-5ex}}
        \maketitle
        \tableofcontents
        \clearpage
        \chapter*{Model Theory}
        \addcontentsline{toc}{chapter}{Model Theory}
        \markboth{}{MODEL THEORY}
        \vspace{10ex}
        \setcounter{chapter}{1}
        \pagenumbering{arabic}
    \else
        \chapter{Model Theory}
    \fi
\end{document}
    \addtocontents{toc}{\protect\newpage}
    \clearpage

    \setcounter{endpage}{\thepage}
    \pagenumbering{gobble}
    \book{Algebra}
        \renewcommand{\PATH}{\TOPPATH/Algebra}
        \pagenumbering{arabic}
        \setcounter{page}{\value{endpage}}
        \part{Group Theory}
            %------------------------------------------------------------------------------%
\begingroup
    \ifcsname\PATH\endcsname
        \newcommand{\PATH}{books/Algebra/Group_Theory}
        \newcommand{\OLDPATH}{\PATH}
    \else
        \newcommand{\OLDPATH}{\PATH}
        \renewcommand{\PATH}{books/Algebra/Group_Theory}
    \fi
    \chapter{Elementary Group Theory}
        One of the most fundamental structures studied in algebra, and more
        generally in mathematics, is that of a \gls{group}\index{Group}.
        Recalling some notions from Book~\ref{book:Foundations}, we define a
        group to be a \gls{set}\index{Set} $G$ together with a
        \gls{binary operation}\index{Binary Operation} $*$ that satisfies a few
        properties. There are three properties $*$ must satisfy,
        \glslink{associative operation}{associativity}%
        \index{Binary Operation!Associative},
        \glslink{unital element}{identity}\index{Unital Element}, and
        \glslink{invertible element}{invertibility}\index{Invertible Element},
        all of which were briefly discussed in
        Chapt.~\ref{chapt:Function_Theory}. Groups then seem to be very simple
        objects, and indeed the standard arithmetic that one is
        familiar with in \gls{mathbbR} has far more structure (it is a
        \gls{field}\index{Field}). However, in another sense, perhaps groups
        have too much structure. We can certainly strip away the requirement of
        invertibility, leaving associativity and identity in tact, and this
        results in a \gls{monoid}\index{Monoid}. We can further rid of the
        existence of an identity, maintaining only associativity, and it may be
        reasonable that such an object has its uses. Furthermore we can take
        away the requirement that we have a binary operation, and replace this
        with a partial function\index{Partial Function}. This latter object
        gives rise to the notion of a groupoid\index{Groupoid}, which has
        applications in geometry and analysis. We'll begin by considering
        ordinary binary operations, but only require the operation to be
        associative. Such a structure is called a
        \gls{semigroup}\index{Semigroup}.
        \section{Semigroups}
    The bulk of group theory should discuss groups. A group is a set with a
    binary operation\index{Binary Operation} that satisfies three particular
    properties. Many theorems that can be proved about groups do no need all
    three of these properties and thus it becomes natural to generalize groups
    to slightly weaker structures. The first two objects to discuss are
    semigroups and monoids. Developing monoids is particularly useful for when
    we wish to develop rings, which is a set with two binary operations.
    \begin{fdefinition}{Semigroup}{Semigroup}
        A \gls{semigroup} is a set $G$ and an \gls{associative operation} $*$
        on $G$.
    \end{fdefinition}
    Associativity is usually a crucial operation to have, otherwise we've no
    idea how to combine three elements to get a fourth.
    \begin{fexample}{Example of a Semigroup}{Example_of_a_Semigroup}
        Let $X$ be a set with several distinct elements and let $\mathscr{F}$ be
        the set of all functions $f:X\rightarrow{X}$ to such that $f$ is a
        constant mapping. That is, there is some $c\in{X}$ such that for all
        $x\in{X}$ it is true that $f(x)=c$. In other words, the image of $X$ is
        $\{c\}$: $f(X)=\{c\}$. Let $\circ$ denote function composition. We know
        that function composition is associative. Moreover, $\circ$ takes
        elements of $\mathscr{F}$ to $\mathscr{F}$. For if $f,g\in\mathscr{F}$
        then there are $c_{f},c_{g}\in{X}$ such that $f(X)=\{c_{f}\}$ and
        $g(X)=\{c_{g}\}$. But then, for all $x\in{X}$ we have:
        \begin{equation}
            (g\circ{f})(x)=g(f(x))=g(c_{f})=c_{g}
        \end{equation}
        and thus $g\circ{f}$ is a constant mapping as well. Therefore $\circ$ is
        an associative binary operation on $\mathscr{F}$ and
        $(\mathscr{F},\circ)$ is a semigroup.
    \end{fexample}
    The example shown in Ex.~\ref{ex:Example_of_a_Semigroup} is missing most
    algebraic properties. Most notably, there is no identity element. For since
    we chose $X$ to have at least several distinct elements, for any two
    distinct functions $f,g\in\mathscr{F}$, we have that
    $g\circ{f}\ne{f}\circ{g}$, and thus there can be no unital element. There
    can also be no left or right unital element,
    and as such there can be no invertible elements. Moreover, as this previous
    expression shows, the operation is not commutative. Thus
    $(\mathscr{F},\circ)$ is a semigroup but can't possible be any of the nicer
    objects like monoids or groups. While such examples may be trivial, this
    does show that it may be worth while studying the structure of these weaker
    algebraic systems.
        \section{Groups}
    While monoids are structurally simpler than groups, the notion of a group
    is one of the most basic and fundamental algebraic structures that one can
    consider.
    \begin{fdefinition}{Group}{Group}
        A group is a \gls{monoid} $(G,*)$ such that for all $g\in{G}$, $g$ is
        an \gls{invertible element} with respect to $*$. That is:\index{Group}
        \begin{align}
            &\label{def:Group_Assoc}
            \textrm{The binary operation }*\textrm{ is associative}\tag{1}\\
            &\label{def:Group_Unit}
            \textrm{There exists a unital element $e\in{G}$}\tag{2}\\
            &\label{def:Group_Inverse}
            \textrm{For every element }a\in{A}
            \textrm{ there is an inverse element }a^{\minus{1}}\tag{3}
        \end{align}
    \end{fdefinition}
    Note that it is not necessarily true that $a*b=b*a$. Such groups are called
    Abelian. There are several examples of groups that one is likely familiar
    with.
    \begin{example}
        If $+$ denotes the usual addition on $\mathbb{Z}$, then
        $(\mathbb{Z},+)$ is a group. The unital element is 0, and for all
        $n\in\mathbb{Z}$, $\minus{n}$ is an inverse element of $n$. That is,
        $n+(\minus{n})=0$, which is a unital element. Moreover, addition is
        associative and therefore $(\mathbb{Z},+)$ is a group.
    \end{example}
    \begin{example}
        If $\mathbb{R}^{+}$ denotes the positive real numbers, and if $\cdot$
        denotes the usual multiplication of real numbers, then
        $(\mathbb{R},\cdot)$ is a group. For this group, 1 is the unital
        element for all $r\in\mathbb{R}^{+}$, $1/r$ is the inverse element.
        Lastly, the operation is indeed associative.
    \end{example}
    \begin{example}
        Consider $\mathbb{Z}_{4}$ with it's modular addition $+$. Then
        $(\mathbb{Z}_{4},+)$ is a group. We can represent the operation $+$ with
        the following table:
        \begin{table}[H]
            \centering
            \captionsetup{type=table}
            \begin{tabular}{c|cccc}
                $+$&0&1&2&3\\
                \hline
                0&0&1&2&3\\
                1&1&2&3&0\\
                2&2&3&0&1\\
                3&3&0&1&2
            \end{tabular}
            \caption{The Group Structure of $\mathbb{Z}_{4}$}
        \end{table}
    \end{example}
    \begin{example}
        More generally, consider $\mathbb{Z}_{n}$ with modular addition $+$.
        Then $(\mathbb{Z}_{n},+)$ is a group. The unital element is $0$ since
        for all $k\in\mathbb{Z}_{n}$, $k+0=k$. The inverse of
        $k\in\mathbb{Z}_{n}$ is the unique integer $m\in\mathbb{Z}_{n}$ such
        that $k+m=n$. That is, $m=n-k$. For then $k+m=n$, and $n$ is equivalent
        to zero in $\mathbb{Z}_{n}$. Since modular addition is commutative, it
        follows that $0$ is a left and right unital element and that each
        element is left and right invertible, and thus $(\mathbb{Z}_{n},+)$ is a
        group.
    \end{example}
    \begin{theorem}
        If $G$ is a set, if $*$ is an associative binary operation on $G$, if
        $e\in{G}$ is a unique right unital element, and if for all $g\in{G}$ it
        is true that $g$ is weakly right invertible, then $G$ is a group.
    \end{theorem}
    \begin{proof}
        For by
        Thm.~\ref{thm:existence_of_weak_left_and_weak_right_implies_unital},
        $e$ is a unital element. And by
        Thm.~\ref{thm:unqiue_right_unit_and_weak_r_inv_implies_inv}, for all
        $g\in{G}$ it is true that $g$ is invertible. Therefore, $(G,*)$ is a
        group (Def.~\ref{def:Group}).
    \end{proof}
    Thus it suffices to check that there is a unique right identity and that all
    elements are weakly right intertible. The unital element of a group is
    unique (Thm.~\ref{thm:Unital_Elements_are_Unique}), as are the inverses of
    elements (Thm.~\ref{thm:Assoc_Op_Inverses_are_Unique}). Moreover, the
    inverse of a unital element is itself
    (Thm.~\ref{thm:Unital_Elements_Are_Invertible}).
    \begin{theorem}
        If $n\in\mathbb{N}^{+}$, if $\tilde{+}$ is addition modulo $n$,
        then $\monoid[][\tilde{+}]{\mathbb{Z}/n\mathbb{Z}}$ is a group.
    \end{theorem}
    \begin{proof}
        For $\tilde{+}$ is associative (Thm.~\ref{thm:Mod_Add_is_Assoc}),
        $[0]$ is a unital element
        (Thm.~\ref{thm:Equiv_Class_of_0_is_Add_Identity_Mod_n}), and
        for all $x\in\mathbb{Z}/n\mathbb{Z}$ there is a
        $y\in\mathbb{Z}/n\mathbb{Z}$ such that $x\tilde{y}$ is a unital
        element (Thm.~\ref{thm:Additive_Inv_Mod_n}). Therefore,
        $\monoid[][\tilde{+}]{\mathbb{Z}/n\mathbb{Z}}$ is a group.
    \end{proof}
    \begin{theorem}
        If $p\in\mathbb{N}^{+}$ is prime, and if $\tilde{\cdot}$ is the
        restriction of multiplication modulo $n$ to
        $\mathbb{Z}/n\mathbb{Z}\setminus\{[0]\}$, then
        $\monoid[][\tilde{\cdot}]{\mathbb{Z}/n\mathbb{Z}\setminus\{[0]\}}$
        is a group.
    \end{theorem}
    \begin{proof}
        For $[1]$ is a unital element
        (Thm.~\ref{thm:Equiv_Class_of_1_is_Mult_Identity_Mod_n}),
        $\tilde{\cdot}$ is associative (Thm.~\ref{thm:Mod_Mult_is_Assoc}).
        But $p$ is prime, and hence for all non-zero
        $x\in\mathbb{Z}/p\mathbb{Z}$ there is a multiplicative inverse
        (Thm.~\ref{thm:Invertible_Mod_n_iff_Relatively_Prime}). Hence,
        $\monoid[][\tilde{\cdot}]{\mathbb{Z}/p\mathbb{Z}\setminus\{[0]\}}$
        is a group.
    \end{proof}
    \begin{example}
        If $G\subseteq\mathbb{C}$ is defined by:
        \begin{equation}
            G=\{\,z\in\mathbb{C}\;|\;
                \textrm{ There exists }n\in\mathbb{N}^{+}
                \textrm{ such that }z^{n}=1\,\}
        \end{equation}
        then $G$ is a group under multiplication. For these are just the
        \textit{roots of unity} and using the polar representation of a
        complex number we have $z=\exp(2\pi{i}m/n)$ for some
        $m,n\in\mathbb{N}$. Multiplying $z\cdot{w}$ yields:
        \begin{equation}
            \exp\Big(\frac{2\pi{i}m}{n}\Big)
                \exp\Big(\frac{2\pi{i}j}{k}\Big)
            =\exp\Big(\frac{2\pi{i}(mk+nj)}{nk}\Big)
        \end{equation}
        showing that $G$ is closed to multiplication. The identity element
        is contained in here since $1=1^{1}$, hence $1\in{G}$. Lastly,
        inverses: If $z=\exp(2\pi{i}m/n)$, let
        $z^{\minus{1}}=\exp(\minus{2}\pi{i}m/n)$. However, $G$ is not a
        group under addition since $1+1$ is not contained in $G$.
        That is, $1+1=2\exp(2\pi{i})$ in polar form and hence any power of
        this will not result in one since:
        \begin{equation}
            \norm{\big(2\exp(2\pi{i})\big)^{n}}
            =2^{n}\norm{\exp(2\pi{i}n)}=2^{n}
        \end{equation}
        and this is not equal to 1 for all $n\in\mathbb{N}^{+}$.
    \end{example}
    \begin{example}
        Define $G\subseteq\mathbb{R}$ by:
        \begin{equation}
            G=\{\,a+b\sqrt{2}\;|\;a,b\in\mathbb{Q}\,\}
        \end{equation}
        Then $G$ is a group under addition. Given $x,y\in{G}$ we have:
        \begin{equation}
            x+y=(a+b\sqrt{2})+(c+d\sqrt{2})=(a+c)+(b+d)\sqrt{2}
        \end{equation}
        and hence $G$ is closed under addition. The identity element is
        contained in $G$ since $0=0+0\sqrt{2}$, and moreover so are
        additive inverses: Let $\minus{x}=\minus{a}-b\sqrt{2}$. Since
        addition is also associative, we have that $G$ is a group under
        addition. If we consider $G\setminus\{0\}$ with multiplication, then
        this too is a group. Give $x,y\in{G}$ we have:
        \begin{equation}
            x\cdot{y}=(a+b\sqrt{2})(c+d\sqrt{2})
                =(ac+2bd)+(ad+bd)\sqrt{2}
        \end{equation}
        since $ad+bd\in\mathbb{Q}$ and $ac+2bd\in\mathbb{Q}$, we have that
        $x\cdot{y}$ is again an element of $\mathbb{Q}$. The identity is in
        $G$, setting $a=1$ and $b=0$. Lastly, if $x\in{G}$ then by
        definition $x$ is not zero, and hence if $x=a+b\sqrt{2}$ then either
        $a$ is non-zero or $b$ is non-zero. But then $a^{2}-2b^{2}$ is
        non-zero since $\sqrt{2}$ is irrational. Let
        $x^{\minus{1}}=(a-b\sqrt{2})/(a^{2}-2b^{2})$. Then:
        \begin{equation}
            x\cdot{x}^{\minus{1}}=
            \big(a+b\sqrt{2}\big)\big(\frac{a-b\sqrt{2}}{a^{2}-2b^{2}}\big)
            =\frac{(a+b\sqrt{2})(a-b\sqrt{2})}{a^{2}-2b^{2}}
            =\frac{a^{2}-2b^{2}}{a^{2}-2b^{2}}=1
        \end{equation}
    \end{example}
    \begin{theorem}
        If $\monoid{G}$ is a group, and if $g\in{G}$ has order
        $n\in\mathbb{N}$, then $g^{\minus{1}}=g^{n-1}$.
    \end{theorem}
    \begin{proof}
        For $e=g^{n}=g\cdot{g}^{n-1}$, so the uniqueness of inverses
        $g^{n-1}=g^{\minus{1}}$.
    \end{proof}
    \begin{theorem}
        If $\monoid{G}$ is a group, if $x,y\in{G}$, and if $x$ and $y$
        commute, then $y^{\minus{1}}*x*y=x$ and
        $x^{\minus{1}}*y^{\minus{1}}*x*y=e$.
    \end{theorem}
    \begin{proof}
        Since $x$ and $y$ commute, $x*y=y*x$. Apply the cancellation laws.
    \end{proof}
    \begin{theorem}[Power Laws]
        If $\monoid{G}$ is a group, $a\in{G}$, $n,m\in\mathbb{Z}$, then:
        \begin{align}
            a^{n}*a^{m}&=a^{n+m}\\
            (a^{\minus{1}})^{n}&=(a^{n})^{\minus{1}}
        \end{align}
    \end{theorem}
    \begin{theorem}
        If $x\in{G}$ has odd order, there is a $n\in\mathbb{Z}$ such that
        $x=(x^{2})^{n+1}$.
    \end{theorem}
    \begin{proof}
        For $e=x^{2n+1}=x^{2n}x$ and hence by cancellation
        $x=x^{2n+2}=(x^{2})^{n+1}$.
    \end{proof}
    \begin{theorem}
        If $\monoid{G}$ is a group, $a\in{G}$ has order $n$, $g\in{G}$,
        then $g^{\minus{1}}ag$ has order $n$. Also, the order of $a*b$ is
        the order of $b*a$.
    \end{theorem}
    \begin{proof}
        We show $(g^{\minus{1}}ag)^{n}=g^{\minus{1}}a^{n}g$ by induction:
        \begin{equation}
            (g^{\minus{1}}ag)^{n+1}=(g^{\minus{1}}ag)(g^{\minus{1}}ag)^{n}
            =(g^{\minus{1}}ag)(g^{\minus{1}}a^{n}g)
            =g^{\minus{1}}a^{n+1}g
        \end{equation}
        hence if $a$ has order $n$, applying this shows that
        $g^{\minus{1}}ag$ has order $n$.
    \end{proof}
    \begin{theorem}
        If $a,b$ commute, then $(ab)^{n}=a^{n}b^{n}$.
    \end{theorem}
    \begin{theorem}[Cyclic Subgroup]
        If $\monoid{G}$ is a group, if $x\in{G}$, and if $H\subseteq{G}$ is
        defined by:
        \begin{equation}
            H=\{\,x^{n}\;|\;n\in\mathbb{Z}\,\}
        \end{equation}
        Then $\monoid{H}$ is a group.
    \end{theorem}
    \begin{proof}
        It contains the identity since $x^{0}=e$ by definition. It contains
        inverses since $(x^{n})^{\minus{1}}=x^{\minus{n}}$. Lastly, it
        is closed under $*$ since $x^{n}*x^{m}=x^{n+m}$. Hence, it is a
        group.
    \end{proof}
    \begin{theorem}
        If $\monoid[A]{A}$ and $\monoid[B]{B}$ are groups, if
        $\monoid{A\times{B}}$ is the direct product of $A$ and $B$, then
        $\monoid{A\times{B}}$ is Abelian if and only if
        $\monoid[A]{A}$ and $\monoid[B]{B}$ are Abelian.
    \end{theorem}
    \begin{proof}
        For if $\monoid[A]{A}$ and $\monoid[B]{B}$ are Abelian,
        $(a,b),(c,d)\in{A}\times{B}$, then:
        \begin{equation}
            (a,b)*(c,d)=(a*_{A}c,b*_{B}d)=
            (c*_{A}a,d*_{B}b )=(c,d)*(a,b)
        \end{equation}
        and hence $\monoid{A\times{B}}$ is Abelian. The other direction is
        similar.
    \end{proof}
    \begin{theorem}
        If $\monoid[A]{A}$ and $\monoid[B]{B}$ are groups, if $e_{A}\in{A}$
        is the unital element of $A$, if $e_{B}\in{B}$ is the unital element
        of $B$, if $a\in{A}$, if $b\in{B}$, and if $\monoid{A\times{B}}$
        is the direct product of $A$ and $B$, then $(a,e_{B})$ and
        $(e_{A},b)$ commute in $A\times{B}$.
    \end{theorem}
    \begin{proof}
        For
        \begin{equation}
            (a,e_{B})*(e_{A},b)=(a*e_{A},e_{B}*b)
            =(e_{A}*a,b*e_{B})=(e_{A},b)*(a,e_{B})
        \end{equation}
    \end{proof}
    \begin{theorem}
        If $\monoid[A]{A}$ and $\monoid[B]{B}$ are groups, if $a\in{A}$ has
        order $n\in\mathbb{N}^{+}$, if $b\in{B}$ has order
        $m\in\mathbb{N}^{+}$, and if $\monoid{A\times{B}}$ is the direct
        product of $A$ and $B$, then $(a,b)$ has order $\LCM(n,m)$ in
        $A\times{B}$.
    \end{theorem}
    \begin{proof}
        For:
        \begin{equation}
            (a,b)^{N}=\big((a,e_{B})*(e_{A},b)\big)^{N}
            =(a,e_{B})^{N}*(e_{A},b)^{N}
            =(a^{N},e_{B})*(e_{A},b^{N})
        \end{equation}
        and hence the least $N$ that makes $a^{N}=e$ and $b^{N}=e$ is the
        least $N$ such that $n|N$ and $m|N$. But this is just $\LCM(n,m)$.
    \end{proof}
    DF 1.1.32
    \begin{theorem}
        If $\monoid{G}$ is a group, $x\in{G}$ has order $n\in\mathbb{N}$,
        and if $f:\mathbb{Z}_{n}\rightarrow{G}$ is defined by
        $f(k)=x^{k}$, then $f$ is injective.
    \end{theorem}
    \begin{proof}
        For if not then there are distinct $k_{1},k_{2}\in\mathbb{Z}_{n}$
        such that $f(k_{1})=f(k_{2})$. Suppose $k_{1}<k_{2}$. Then
        $x^{k_{2}}=x^{k_{1}}$ and hence $x^{k_{2}-k_{1}}=e$, a contradiction
        since $k_{2}-k_{2}<n$ and $n$ is the least such element in
        $\mathbb{N}^{+}$ with $x^{n}=e$.
    \end{proof}
    \begin{theorem}
        If $\monoid{G}$ is a group, if $x\in{G}$ has infinite order, and if
        $f:\mathbb{Z}\rightarrow{G}$ is defined by $f(n)=x^{n}$, then
        $f$ is injective.
    \end{theorem}
    \begin{proof}
        For suppose not. Then $x^{m}=x^{n}$ which implies $x^{m-n}=e$,
        a contradiction since $x$ has infinite order. Hence, $m-n=0$ and
        thus $m=n$.
    \end{proof}
    \begin{theorem}
        If $\monoid{G}$ is a group, if $n\in\mathbb{N}^{+}$, if $x\in{G}$
        has order $n$, and if $N\in\mathbb{N}$, then there is a
        $K\in\mathbb{Z}_{n}$ such that $x^{k}=x^{N}$.
    \end{theorem}
    \begin{proof}
        For either $N=0$ or $N\ne{0}$. But if $N=0$, then since
        $0\in\mathbb{Z}_{n}$ we have that $x^{N}=x^{0}$ and we're done.
        Suppose $N\ne{0}$. But then $N,n\in\mathbb{Z}\setminus\{0\}$ and
        hence by Euclid's division algorithm there exists $q\in\mathbb{Z}$
        and $r\in\mathbb{N}$ such that $r<n$ and $N=qn+r$. But then:
        \begin{equation}
            x^{N}=x^{qn+r}=x^{qn}x^{r}=(x^{n})^{q}x^{r}
            =e^{q}x^{r}=x^{r}
        \end{equation}
        proving the claim.
    \end{proof}
    \begin{theorem}
        \label{thm:Group_Inverse_of_Product}%
        If $(G,*)$ is a group and $a,b\in G$, then:
        \begin{equation}
            (a*b)^{\minus{1}}=b^{\minus{1}}*a^{\minus{1}}
        \end{equation}
    \end{theorem}
    \begin{proof}
        For if $(G,*)$ is a group, and if $a,b\in{G}$, then $*$ is associative
        and $a$ and $b$ are invertible (Def.~\ref{def:Group}). But then by
        Thm.~\ref{thm:assoc_op_prod_of_inv_is_inv},
        $a*b=b^{\minus{1}}*a^{\minus{1}}$.
    \end{proof}
    \begin{theorem}
        \label{thm:Group_Inverse_of_Inverse}%
        If $(G,*)$ is a group and $a\in{G}$, then:
        \begin{equation}
            (a^{\minus{1}})^{\minus{1}}=a
        \end{equation}
    \end{theorem}
    \begin{proof}
        For if $e$ is the unital element of $G$, then:
        \begin{align}
            a^{\minus{1}}*(a^{\minus{1}})^{\minus{1}}
            &=(a^{\minus{1}}* a)^{\minus{1}}
            \tag{Thm.~\ref{thm:Group_Inverse_of_Product}}\\
            &=e
            \tag{Inverse Property}
        \end{align}
        From the uniqueness of inverses
        (Thm.~\ref{thm:Assoc_Op_Inverses_are_Unique}),
        $(a^{\minus{1}})^{\minus{1}}=a$.
    \end{proof}
    One of the fundamental aspects of elementary algebra is adding and
    subtracting like expressions from two sides of an equation. That is, if we
    are given $4x+1=9$, we solve this by adding $\minus{1}$ to both sides, and
    then dividing both sides by $4$ yielding $x=2$. Such procedures are valid
    because of the \textit{cancellation laws}\index{Cancellation Law} of real
    number arithmetic. These laws hold for groups, as well.
    \begin{ltheorem}{Left Cancellation Law}{Group_Left_Cancellation_Law}
        \index{Cancellation Law!Left}%
        If $(G,*)$ is a group, if $a,b,c\in{G}$, and if $a*b=a*c$, then $b=c$.
    \end{ltheorem}
    \begin{proof}
        For let $e$ be the unital element of $G$. Then:
        \par\vspace{-2.5ex}
        \begin{minipage}[t]{0.49\textwidth}
            \centering
            \begin{align}
                b&=e*b
                \tag{Identity}\\
                &=(a^{\minus{1}}*a)*b
                \tag{Inverse}\\
                &=a^{\minus{1}}*(a*b)
                \tag{Associativity}\\
                &=a^{\minus{1}}*(a*c)
                \tag{Hypothesis}
            \end{align}
        \end{minipage}
        \hfill
        \begin{minipage}[t]{0.49\textwidth}
            \centering
            \begin{align}
                &=(a^{\minus{1}}*a)*c
                \tag{Associativity}\\
                &=c*e
                \tag{Inverse}\\
                &=c
                \tag{Identity}
            \end{align}
        \end{minipage}
        \par\vspace{2.5ex}
        Thus by the transitivity of equality
        (Thm.~\ref{thm:Transitivity_of_Equality}), $b=c$.
    \end{proof}
    \begin{ltheorem}{Right Cancellation Law}{Group_Right_Cancellation_Law}
        \index{Cancellation Law!Right}%
        If $(G,*)$ is a group, if $a,b,c\in{G}$, and if $b*a=c*a$, then $b=c$.
    \end{ltheorem}
    \begin{proof}
        For let $e$ be the unital element of $G$. Then:
        \par\vspace{-2.5ex}
        \begin{minipage}[t]{0.49\textwidth}
            \centering
            \begin{align}
                b&=b*e
                \tag{Identity}\\
                &=b*(a*a^{\minus{1}})
                \tag{Inverse}\\
                &=(b*a)*a^{\minus{1}}
                \tag{Associativity}\\
                &=(c*a)*a^{\minus{1}}
                \tag{Hypothesis}
            \end{align}
        \end{minipage}
        \hfill
        \begin{minipage}[t]{0.49\textwidth}
            \centering
            \begin{align}
                &=c*(a*a^{\minus{1}})
                \tag{Associativity}\\
                &=e*c
                \tag{Inverse}\\
                &=c
                \tag{Identity}
            \end{align}
        \end{minipage}
        \par\vspace{2.5ex}
        And therefore by transitivity (Thm.~\ref{thm:Transitivity_of_Equality}),
        $b=c$.
    \end{proof}
    With this we can perform the basic substitution operations from elementary
    algebra.
    \begin{example}
        Let $(G,*)$ be a group with unital element $e\in{G}$, and suppose
        $x\in{G}$ and $a\in{G}$ are elements satisfying the following:
        \par
        \begin{subequations}
            \begin{minipage}[b]{0.49\textwidth}
                \centering
                \begin{equation}
                    x^{2}=b
                \end{equation}
            \end{minipage}
            \hfill
            \begin{minipage}[b]{0.49\textwidth}
                \begin{equation}
                    x^{7}=e
                \end{equation}
            \end{minipage}
        \end{subequations}
        \par\vspace{2.5ex}
        where $x^{n}=x^{n-1}*x$. We can use the cancellation laws to solve for
        $x$. We have:
        \begin{equation}
            b^{3}*x=(x^{2})^{3}*x=x^{6}*x=x^{7}=e
        \end{equation}
        And thus we have $b^{3}*x=e$. But $b^{3}*(b^{3})^{\minus{1}}=e$, and
        from the left cancellation law
        (Thm.~\ref{thm:Group_Left_Cancellation_Law}) we obtain
        $x=(b^{3})^{\minus{1}}$.
    \end{example}
    \begin{example}
        As a more complicated example, let $(G,*)$ be a group with unital
        element $e\in{G}$, and $a,b,x\in{G}$ satisfying the following:
        \par
        \begin{subequations}
            \begin{minipage}[b]{0.49\textwidth}
                \centering
                \begin{equation}
                    a*x^{2}=b
                \end{equation}
            \end{minipage}
            \hfill
            \begin{minipage}[b]{0.49\textwidth}
                \centering
                \begin{equation}
                    x^{3}=e
                \end{equation}
            \end{minipage}
        \end{subequations}
        \par\vspace{2.5ex}
        We can solve for $x$ in terms of the other variables. Note that if we
        multiply both sides of the first equation on the right by $x$ we
        obtain:
        \begin{equation}
            a*x^{3}=b*x
        \end{equation}
        But $x^{3}=e$, and thus we can simplify this to $a=b*x$. One solution to
        this is $b^{\minus{1}}*a$, and applying the right cancellation law
        (Thm.~\ref{thm:Group_Right_Cancellation_Law}) we see that this is the
        only solution. Thus, $x=b^{\minus{1}}*a$.
    \end{example}
    \begin{theorem}
        \label{thm:Group_Simplifying_on_Left}%
        If $(G,*)$ is a group, if $a,b,c\in{G}$, and if $a*b=c$, then
        $b=a^{\minus{1}}*c$.
    \end{theorem}
    \begin{proof}
        For $a*(a^{\minus{1}}*c)=(a*a^{\minus{1}})*c=e*c=c$, by associativity
        and identity. But $a*b=c$, and thus by the left cancellation law
        (Thm.~\ref{thm:Group_Left_Cancellation_Law}), $b=a^{\minus{1}}*c$.
    \end{proof}
    \begin{theorem}
        \label{thm:Group_Simplifying_on_Right}%
        If $(G,*)$ is a group, if $a,b,c\in{G}$, and if $a*b=c$, then
        $a=c*b^{\minus{1}}$.
    \end{theorem}
    \begin{proof}
        For $(c*b^{\minus{1}})*b=c*(b^{\minus{1}}*b)=c*e=c$, by associativity
        and identity. But $a*b=c$, and thus by the right cancellation law
        (Thm.~\ref{thm:Group_Right_Cancellation_Law}), $a=c*b^{\minus{1}}$.
    \end{proof}
    \begin{theorem}
        If $(G,*)$ is a group and if $a\in{G}$ is idempotent, then $a$ is the
        unital element.
    \end{theorem}
    \begin{proof}
        If $(G,*)$ is a group, then there is a unital element $e\in{G}$
        (Def.~\ref{def:Group}). But if $a$ is idempotent, then $a*a=a$
        (Def.~\ref{def:Idempotent}). But $a*e=a$, and thus by the left
        cancellation law (Thm.~\ref{thm:Group_Left_Cancellation_Law}) $a=e$.
    \end{proof}
    Note that this does not say that $a^{2}=e$ implies that $a=e$, for consider
    the group $(\mathbb{Z}_{2},+)$. Then $1+1=0$, which is the identity element,
    but $1\ne{0}$. Real valued arithmetic does not have this property. If $a$ is
    a real number such that $a+a=0$, then $a=0$. For multiplication, if
    $a\cdot{a}=1$, then we can conclude that either $a=1$ or $a=\minus{1}$. In a
    general group, however, there may be infinitely many elements such that
    $a*a=e$, yet $a\ne{e}$. Groups such that every element has the property that
    $a^{2}=e$ are called \textit{Boolean} groups\index{Boolean Group}.
    \par\hfill\par
    Groups may also lack \textit{square roots}\index{Square Root}. That is,
    given a group $(G,*)$ and an element $a\in{G}$, there may not be an element
    $b\in{G}$ such that $a=b*b$. One need only think of the group
    $(\mathbb{Q}^{+},\cdot\,)$, where $\cdot$ denotes usual multiplication.
    This is a group, and $2\in\mathbb{Q}^{+}$, but there is no element
    $x\in\mathbb{Q}^{+}$ such that $x^{2}=2$. Groups do, however, possess the
    Latin square property\index{Latin Square Property}.
    \begin{ltheorem}{Latin Square Property of Groups}{Group_Latin_Square}
        If $(G,*)$ is a group, $a,b\in{G}$, then there is a unique $x\in{G}$
        such that $a*x=b$.\index{Latin Square Property!of a group}
    \end{ltheorem}
    \begin{proof}
        For since $(G,*)$ is a group, there is a unital element $e\in{G}$ and an
        element $a^{\minus{1}}\in{G}$ such that $a*a^{\minus{1}}=e$
        (Def.~\ref{def:Group}). Let $x=a^{\minus{1}}*b$. Then:
        \begin{equation}
            a*x=a*(a^{\minus{1}}*b)=(a*a^{\minus{1}})*b=e*b=b
        \end{equation}
        Moreover, if $a*y=b$, then by the left cancellation law
        \ref{thm:Group_Left_Cancellation_Law}, $y=x$. Therefore, $x$ is unique.
    \end{proof}
    \begin{theorem}
        \label{thm:Group_Commuting_Elements_Have_Commuting_Inverses}%
        If $(G,*)$ is a group, if $a,b\in{G}$ are commuting elements, and if
        $a^{\minus{1}}$ and $b^{\minus{1}}$ are the inverses of $a$ and $b$,
        respectively, then $a^{\minus{1}}$ and $b^{\minus{1}}$ are commuting
        elements.
    \end{theorem}
    \begin{proof}
        For:
        \begin{align}
            b^{\minus{1}}*a^{\minus{1}}
            &=(a*b)^{\minus{1}}
            \tag{Thm.~\ref{thm:Group_Inverse_of_Product}}\\
            &=(b*a)^{\minus{1}}
            \tag{Hypothesis}\\
            &=a^{\minus{1}}*b^{\minus{1}}
            \tag{Thm.~\ref{thm:Group_Inverse_of_Product}}
        \end{align}
        By transitivity,
        $a^{\minus{1}}*b^{\minus{1}}=b^{\minus{1}}*a^{\minus{1}}$. Thus,
        $a^{\minus{1}}$ and $b^{\minus{1}}$ commute
        (Def.~\ref{def:Commuting_Elements}).
    \end{proof}
    We will later define the center of a group to be the subset of all commuting
    elements. The above theorem will the be used to show that the center is a
    \textit{subgroup} of the original group.
    \begin{theorem}
        \label{thm:Group_Comm_Elems_Have_Comm_Powers_a}%
        If $(G,*)$ is a group, if $a,b\in{G}$ commute, and if $n\in\mathbb{N}$,
        then $a^{n}$ and $b$ commute.
    \end{theorem}
    \begin{proof}
        For by induction, suppose not. Then by the well ordering of $\mathbb{N}$
        there is a least $n\in\mathbb{N}$ such that $a^{n}$ and $b$ do not
        commute. But $a$ and $b$ commute, and thus $n>1$. But then $a^{n-1}$
        commutes with $b$. Thus:
        \par
        \begin{minipage}[t]{0.52\textwidth}
            \centering
            \begin{align}
                a^{n}*b&=(a^{n-1}*a)*b
                \tag{Definition of $a^{n}$}\\
                &=a^{n-1}*(a*b)
                \tag{Associativity}\\
                &=a^{n-1}*(b*a)
                \tag{Commutativity}\\
                &=(a^{n-1}*b)*a
                \tag{Associativity}
            \end{align}
            \hfill
        \end{minipage}
        \begin{minipage}[t]{0.47\textwidth}
            \centering
            \begin{align}
                &=(b*a^{n-1})*a
                \tag{Commutativity}\\
                &=b*(a^{n-1}*a)
                \tag{Associativity}\\
                &=b*a^{n}
                \tag{Definition of $a^{n}$}
            \end{align}
        \end{minipage}
        \par\vspace{2.5ex}
        By transitivity, $a^{n}*b=b*a^{n}$, a contradiction. Thus, for all
        $n\in\mathbb{N}$ it is true that $a^{n}$ and $b$ commute.
    \end{proof}
    \begin{theorem}
        \label{thm:Group_Commuting_Elements_Have_Commuting_Powers_Part_b}%
        If $(G,*)$ is a group, if $a,b\in{G}$ commute, and if $n\in\mathbb{N}$,
        then $a^{n}$ and $b^{n}$ commute.
    \end{theorem}
    \begin{proof}
        For by Thm.~\ref{thm:Group_Comm_Elems_Have_Comm_Powers_a},
        $b$ commutes with $a^{n}$. But again, if $a^{n}$ commutes with
        $b$, then $a^{n}$ commutes with $b^{n}$
        (Thm.~\ref{thm:Group_Comm_Elems_Have_Comm_Powers_a}), completing the
        proof.
    \end{proof}
    \begin{theorem}
        \label{thm:Group_Equiv_Def_of_Commute_a}%
        If $(G,*)$ is a group, if $a,b\in{G}$ commute, and if $b^{\minus{1}}$
        is the inverse element of $b$, then $a=b*a*b^{\minus{1}}$
    \end{theorem}
    \begin{proof}
        For:
        \begin{equation}
            a=a*(b*b^{\minus{1}})
            =(a*b)*b^{\minus{1}}
            =(b*a)*b^{\minus{1}}
        \end{equation}
        By transitivity, $a=b*a*b^{\minus{1}}$.
    \end{proof}
    \begin{theorem}
        \label{thm:Group_Equiv_Def_of_Commute_b}%
        If $(G,*)$ is a group, if $a,b\in{G}$ are such that
        $a=b*a*b^{\minus{1}}$, then $a$ and $b$ commute.
    \end{theorem}
    \begin{proof}
        For:
        \begin{equation}
            a*b=(b*a*b^{\minus{1}})*b
            =b*(a*b*b^{\minus{1}})
            =b*a
        \end{equation}
        By transitivity, $a*b=b*a$, and thus $a$ and $b$ commute
        (Def.~\ref{def:Commuting_Elements}).
    \end{proof}
    \begin{theorem}
        \label{thm:Group_Commuting_Elements_Commute_with_Inverses}%
        If $(G,*)$ is a group, if $a,b\in{G}$ commute, and if $b^{\minus{1}}$ is
        the inverse of $b$, then $a$ and $b^{\minus{1}}$ commute.
    \end{theorem}
    \begin{proof}
        For let $e\in{G}$ be the unital element. Then:
        \par
        \begin{minipage}[t]{0.56\textwidth}
            \centering
            \begin{align}
                b^{\minus{1}}*a
                &=b^{\minus{1}}*(b*a*b^{\minus{1}})
                \tag{Thm.~\ref{thm:Group_Equiv_Def_of_Commute_a}}\\
                &=(b^{\minus{1}}*b)*(a*b^{\minus{1}})
                \tag{Associativity}
            \end{align}
        \end{minipage}
        \hfill
        \begin{minipage}[t]{0.42\textwidth}
            \centering
            \begin{align}
                &=e*(a*b^{\minus{1}})
                \tag{Inverse}\\
                &=a*b^{\minus{1}}
                \tag{Identity}
            \end{align}
        \end{minipage}
        \par\vspace{2.5ex}
        By transitivity, $a*b^{\minus{1}}=b^{\minus{1}}*a$, and thus $a$ and
        $b^{\minus{1}}$ commute (Def.~\ref{def:Commuting_Elements}).
    \end{proof}
    \begin{theorem}
        If $(G,*)$ is a group, if $e\in{G}$ is the unital element, and if
        $a,b\in{G}$ commute, then $a*b*a^{\minus{1}}*b^{\minus{1}}=e$.
    \end{theorem}
    \begin{proof}
        For $(a*b)^{\minus{1}}=b^{\minus{1}}*a^{\minus{1}}$
        (Thm.~\ref{thm:Group_Inverse_of_Product}). But if $a$ and $b$ commute,
        and $a^{\minus{1}}$ and $b^{\minus{1}}$ commute
        (Thm.~\ref{thm:Group_Commuting_Elements_Have_Commuting_Inverses}),
        and thus $b^{\minus{1}}*a^{\minus{1}}=a^{\minus{1}}*b^{\minus{1}}$.
        But then:
        \begin{equation}
            e=(a*b)*(b^{\minus{1}}*a^{\minus{1}})
            =(a*b)*(a^{\minus{1}}*b^{\minus{1}})
        \end{equation}
        Completing the proof.
    \end{proof}
    \begin{theorem}
        If $(G,*)$ is a group, if $a,g\in{G}$, and if $n\in\mathbb{N}$, then"
        \begin{equation}
            (g*a*g^{\minus{1}})^{n}=g*a^{n}*g^{\minus{1}}
        \end{equation}
    \end{theorem}
    \begin{proof}
        For by induction, suppose not and let $n\in\mathbb{N}$ be the least
        $n$ such that the proposition fails. But by the definition of
        $(g*a*g^{\minus{1}})^{n}$, we have that:
        \begin{equation}
            (g*a*g^{\minus{1}})^{1}=g*a*g^{\minus{1}}=g*a^{1}*g^{\minus{1}}
        \end{equation}
        and therefore $n>1$. But then the proposition holds for $n-1$. But then:
        \begin{align}
            (g*a*g^{\minus{1}})^{n}
            &=(g*a*g^{\minus{1}})^{n-1}*(g*a*g^{\minus{1}})
            \tag{Definition of $x^{n}$}\\
            &=(g*a^{n-1}*g^{\minus{1}})*(g*a*g^{\minus{1}})
            \tag{Inductive Hypothesis}\\
            &=(g*a^{n-1})*(g^{\minus{1}}*g)*(a*g^{\minus{1}})
            \tag{Associativity}\\
            &=(g*a^{n-1})*e*(a*g^{\minus{1}})
            \tag{Inverse}\\
            &=(g*a^{n-1})*(a*g^{\minus{1}})
            \tag{Identity}\\
            &=g*(a^{n-1}*a)*g^{\minus{1}}
            \tag{Associativity}\\
            &=g*a^{n}*g^{\minus{1}}
            \tag{Definition of $x^{n}$}
        \end{align}
        A contradiction. Therefore, the proposition is true for all
        $n\in\mathbb{N}$.
    \end{proof}
    \begin{theorem}
        If $(G,*)$ is a group, if $a,b\in{G}$ are commuting elements, and if
        $n\in\mathbb{N}$, then $(a*b)^{n}=a^{n}*b^{n}$.
    \end{theorem}
    \begin{proof}
        For by induction, suppose not and let $n\in\mathbb{N}$ be the least
        element such that the proposition fails. But $(a*b)^{1}=a*b=a^{1}*b^{1}$
        and therefore $n>1$. But then the proposition is true for $n-1$. But
        then:
        \begin{align}
            (a*b)^{n}&=(a*b)^{n-1}*(a*b)
            \tag{Definition of $x^{n}$}\\
            &=(a^{n-1}*b^{n-1})*(a*b)
            \tag{Inductive Hypothesis}\\
            &=\big((a^{n-1}*b^{n-1})*a\big)*b
            \tag{Associativity}\\
            &=\big(a^{n-1}*(b^{n-1}*a)\big)*b
            \tag{Associativity}\\
            &=\big(a^{n-1}*(a*b^{n-1})\big)*b
            \tag{Thm.~\ref{thm:Group_Comm_Elems_Have_Comm_Powers_a}}\\
            &=\big((a^{n-1}*a)*b^{n-1}\big)*b
            \tag{Associativity}\\
            &=(a^{n}*b^{n-1})*b
            \tag{Definition of $x^{n}$}\\
            &=a^{n}*(b^{n-1})*b
            \tag{Associativity}\\
            &=a^{n}*b^{n}
            \tag{Definition of $x^{n}$}
        \end{align}
        A contradiction. Thus, the proposition holds for all $n\in\mathbb{N}$.
    \end{proof}
    \begin{theorem}
        \label{thm:Group_bab_eq_e_implies_abab_eq_a}%
        If $(G,*)$ is a group, $a,b\in{G}$, and $a*b=b^{\minus{1}}$, then
        $(a*b)^{2}=a$.
    \end{theorem}
    \begin{proof}
        For:
        \begin{equation}
            (a*b)^{2}=(a*b)*(a*b)
                =a*\big(b*(a*b)\big)
                =a*(b*b^{\minus{1}})
                =a*e
                =a
        \end{equation}
        Completing the proof.
    \end{proof}
    \begin{theorem}
        If $(G,*)$ is a group, if $a,b\in{G}$, if $a*b=b^{\minus{1}}$, and if
        $n\in\mathbb{N}$, then $(a*b)^{2n}=a^{n}$.
    \end{theorem}
    \begin{proof}
        For by induction, suppose not. Let $n\in\mathbb{N}$ be the least
        integer such that the proposition fails. But by
        Thm.~\ref{thm:Group_bab_eq_e_implies_abab_eq_a}, $n>1$. Thus the
        proposition is true for $n-1$. But:
        \begin{align}
            (a*b)^{2n}&=(a*b)^{2(n-1)}*(a*b)^{2}
            \tag{Definition of $x^{n}$}\\
            &=a^{n-1}*(a*b)^{2}
            \tag{Inductive Hypothesis}\\
            &=a^{n-1}*a
            \tag{Thm.~\ref{thm:Group_bab_eq_e_implies_abab_eq_a}}\\
            &=a^{n}
            \tag{Definition of $x^{n}$}
        \end{align}
        A contradiction. Thus, the proposition is true for all $n\in\mathbb{N}$.
    \end{proof}
    \begin{fdefinition}{Square Root}{Square_Root}
        A square root of an element $a$ in a \gls{group} $(G,*)$ is an element
        $b\in{G}$ such that $b*b=a$. We write $b=\sqrt{a}$.
    \end{fdefinition}
    This is the same as the usual notion of squares roots with real numbers, but
    we've now generalized the concept to abstract groups.
    \begin{example}
        If we consider $(R^{+},\cdot\,)$, then every element
        $r\in\mathbb{R}^{+}$ has a square root. This comes from the
        \textit{completeness} of $\mathbb{R}$.
    \end{example}
    \begin{example}
        If we consider $(\mathbb{C}\setminus\{0\},\cdot\,)$, then again, every
        element $z\in\mathbb{C}\setminus\{0\}$ has a square root. To see this,
        let $z=r\exp(i\theta)$, where $r\in\mathbb{R}^{+}$ and
        $\theta\in[0,2\pi)$. This is the polar representation of a complex
        number. Since $r>0$, by the previous example we know that it has a
        square root, $\sqrt{r}$. We then see that $\sqrt{r}\exp(i\theta/2)$ is a
        square root of $z$. Thus, every element of $\mathbb{C}\setminus\{0\}$
        has a square root. Zero has a square root as well (itself), but
        $(\mathbb{C},\cdot\,)$ is not a group since $0$ has no inverse element,
        and hence we excluded it from this example.
    \end{example}
    \begin{example}
        The multiplicative group $(G,\cdot\,)$ does contain square roots for all
        of it's elements. One need only consider $2\in\mathbb{Q}^{+}$, where it
        is well known that $\sqrt{2}$ is irrational, and thus
        $\sqrt{2}\notin\mathbb{Q}^{+}$..
    \end{example}
    \begin{theorem}
        If $(G,*)$ is a group, if $a\in{G}$, if $e\in{G}$ is the unital element,
        and if $a^{3}=e$, then $a$ has a square root in $G$.
    \end{theorem}
    \begin{proof}
        For if $a^{3}=e$, then $a^{2}*a=e$, and thus by the left cancellation
        law (Thm.~\ref{thm:Group_Left_Cancellation_Law}), $a^{2}=a^{\minus{1}}$.
        But then again by the left cancellation law, $a=(a^{\minus{1}})^{2}$.
        Thus, $\sqrt{a}=a^{\minus{1}}$ (Def.~\ref{def:Square_Root}).
    \end{proof}
    \begin{theorem}
        If $(G,*)$ is a group, if $a,b,g\in{G}$, and if $g*a*g=b$, then $a*b$
        has a square root in $G$.
    \end{theorem}
    \begin{proof}
        For $a*b=a*(g*a*g)=(a*g)*(a*g)=(a*g)^{2}$, and thus
        $\sqrt{a*b}=a*g$
    \end{proof}
    \begin{theorem}
        If $(G,*)$ is a group, if $a\in{G}$, and if $a^{\minus{1}}$ has a
        square root in $G$, then $a$ has a square root in $G$.
    \end{theorem}
    \begin{proof}
        For if $a^{\minus{1}}$ has a square root, there is a $b\in{G}$ such that
        $b^{2}=a^{\minus{1}}$. But $(a^{\minus{1}})^{\minus{1}}=a$
        (Thm.~\ref{thm:Group_Inverse_of_Inverse}) and thus
        $a=(b^{2})^{\minus{1}}$. But $(b^{2})^{\minus{1}}=(b^{\minus{1}})^{2}$
        (Thm.~\ref{thm:Group_Inverse_of_Product}). Thus,
        $\sqrt{a}=b^{\minus{1}}$.
    \end{proof}
    Similarly, we can define cube roots.
    \begin{fdefinition}{Cube Root}{Cube_Root}
        A cube root of an element $a$ in a \gls{group} $(G,*)$ is an element
        $b\in{G}$ such that $b^{3}=a$. We write $b=\sqrt[3]{a}$.
    \end{fdefinition}
    \begin{theorem}
        \label{thm:Group_inv_has_cube_root_implies_a_has_cube_root}%
        If $(G,*)$ is a group, if $a\in{G}$, and if $a^{\minus{1}}$ has a cube
        root, then $a$ has a cube root.
    \end{theorem}
    \begin{proof}
        For if $a^{\minus{1}}$ has a cube root, there exists a $b\in{G}$ such
        that $b^{3}=a^{\minus{1}}$. But $(a^{\minus{1}})^{\minus{1}}=a$
        (Thm.~\ref{thm:Group_Inverse_of_Inverse}), and thus
        $a=(b^{3})^{\minus{1}}$. But $(b^{3})^{\minus{1}}=(b^{\minus{1}})^{3}$
        (Thm.~\ref{thm:Group_Inverse_of_Product}), and thus
        $a=(b^{\minus{1}})^{3}$. Thus, $\sqrt[3]{a}=b^{\minus{1}}$
        (Def.~\ref{def:Cube_Root}).
    \end{proof}
    \begin{theorem}
        \label{thm:Group_ggag_eq_inv_a_implies_gagg_eq_inv_a}%
        If $(G,*)$ is a group, if $a,g\in{G}$, and if $g^{2}*a*g=a^{\minus{1}}$,
        then $g*a*g^{2}=a^{\minus{1}}$.
    \end{theorem}
    \begin{proof}
        For:
        \begin{align}
            g^{2}*a&=a^{\minus{1}}*g^{\minus{1}}
            \tag{Thm.~\ref{thm:Group_Simplifying_on_Right}}\\
            \Rightarrow
            g^{2}&=a^{\minus{1}}*g^{\minus{1}}*a^{\minus{1}}
            \tag{Thm.~\ref{thm:Group_Simplifying_on_Right}}
        \end{align}
        But then:
        \begin{align*}
            g*a*g^{2}&=g*a*(a^{\minus{1}}*g^{\minus{1}}*a^{\minus{1}})\\
            &=g*(a*a^{\minus{1}})*(g^{\minus{1}}*a^{\minus{1}})
            \tag{Associativity}\\
            &=(g*e)*(g^{\minus{1}}*a^{\minus{1}})
            \tag{Inverse}\\
            &=g*(g^{\minus{1}}*a^{\minus{1}})
            \tag{Identity}\\
            &=(g*g^{\minus{1}})*a^{\minus{1}}
            \tag{Associativity}\\
            &=e*a^{\minus{1}}
            \tag{Inverse}\\
            &=a^{\minus{1}}
            \tag{Identity}
        \end{align*}
        Completing the proof.
    \end{proof}
    \begin{theorem}
        If $(G,*)$ is a group, if $a,g\in{G}$, and if $g^{2}*a*g=a^{\minus{1}}$,
        then $a$ has a cube root.
    \end{theorem}
    \begin{proof}
        For if $g^{2}*a*g=a^{\minus{1}}$, then:
        \begin{align}
            (g*a*g)^{3}
            &=(g*a*g)*(g*a*g)*(g*a*g)
            \tag{Definition of $x^{3}$}\\
            &=(g*a*g^{2})*\big(a*(g^{2}*a*g)\big)
            \tag{Associativity}\\
            &=(g*a*g^{2})*(a*a^{\minus{1}})
            \tag{Hypothesis}\\
            &=(g*a*g^{2})*e
            \tag{Inverse}\\
            &=g*a*g^{2}
            \tag{Identity}
        \end{align}
        But if $g^{2}*a*g=a^{\minus{1}}$, then $g*a*g^{2}=a^{\minus{1}}$
        (Thm.~\ref{thm:Group_ggag_eq_inv_a_implies_gagg_eq_inv_a}).
        Therefore, $a^{\minus{1}}$ has a cube root (Def.~\ref{def:Cube_Root}).
        But if $a^{\minus{1}}$ has a cube root, then $a$ has a cube root
        (Thm.~\ref{thm:Group_inv_has_cube_root_implies_a_has_cube_root}). Thus,
        $a$ has a cube root.
    \end{proof}
    \begin{fdefinition}{Abelian Group}{Abelian_Group}
        An \gls{Abelian group} is a \gls{group} $(G,*)$ such that $*$ is
        a \gls{commutative operation}.\index{Group!Abelian}
    \end{fdefinition}
    \begin{lexample}{The Dihedral Group $D_{6}$}{Dihedral_Group_D_6}
        Not every group is Abelian, and a classic non-Abelian group is the group
        of symmetries on an equilateral triangle. This is the dihedral group
        $D_{6}$. It is formed by considering all of the distinct ways one can
        rearrange the three points on an equilateral triangle by means of
        rotation by $60^{\circ}$ and by reflection across the $y$ axis, as well
        as any combination of these two (see Fig.~\ref{fig:Dihedral_Group_D_6}).
        \begin{figure}[H]
            \centering
            \captionsetup{type=figure}
            \begin{tikzpicture}[>=Latex]
    \coordinate (A) at ( 0.0,     1.0);
    \coordinate (B) at ( 1.1547, -1.0);
    \coordinate (C) at (-1.1547, -1.0);

    \draw[fill=Apricot,opacity=0.8] (A) to (B) to (C) to cycle;

    \draw[densely dashed, draw=red] (-2, -1.4880) to (2,  0.8213);
    \draw[densely dashed, draw=red] (-2,  0.8213) to (2, -1.4880);
    \draw[densely dashed, draw=red] ( 0, -1.8000) to (0,  1.8000);

    \draw[fill=black] (A) circle (0.1);
    \draw[fill=black] (B) circle (0.1);
    \draw[fill=black] (C) circle (0.1);
\end{tikzpicture}
            \caption{The Dihedral Group $D_{6}$}
            \label{fig:Dihedral_Group_D_6}
        \end{figure}
        It turns out there are 6 distinct such moves, but by considering $*$ to
        be the \textit{successor} operation, $(D_{6},*)$ forms a group. The
        successor operation means that if $r$ denotes rotation and $a$ denotes
        reflection, then $r*a$ denotes \textit{rotate and then reflect}. By
        studying the triangle we get the \textit{Cayley table}
        (Tab.~\ref{tab:Cayley_Table_D_6}) of this operation. A few things to
        note, the identity of our group is the \textit{do nothing} symmetry.
        That is, we neither rotate nor reflect. Also note that reflecting twice
        in a row or rotating three times in a row is equivalent to doing
        nothing. The last thing to note is that reflection, rotation, then
        reflecting again is the same as rotating \textit{backwards}. In other
        words, we have the following constraints:
        \begin{equation}
            r^{3}=e
            \quad\quad
            a^{2}=e
            \quad\quad
            (a*r)*(a*r)=e
        \end{equation}
        We can verify this last statement via pictures.
        Fig.~\ref{fig:Restraints_on_D_6} shows that $a*r*a*r$ is equivalent to
        doing nothing, as claimed.
        \begin{figure}[H]
            \centering
            \captionsetup{type=figure}
            \resizebox{\textwidth}{!}{%
                \begin{tikzpicture}[%
    >=Latex,
    p_arc/.style args={#1:#2:#3}{
        insert path={+ (#1:#3) arc (#1:#2:#3)},->
    },
    semithick
]
    \newcommand*{\defcoords}{%
        \coordinate (O) at ( 0.0,    -0.333333);
        \coordinate (A) at ( 0.0,     1.0);
        \coordinate (B) at ( 1.1547, -1.0);
        \coordinate (C) at (-1.1547, -1.0);
    }
    \begin{scope}[xshift=-7.0cm]
        \defcoords;
        \draw (A) to (B) to (C) to cycle;
        \draw (O) [p_arc=-140:-40:1.333];
        \draw (0,-1.6) [p_arc=40:140:1.2];
        \draw[densely dashed,thin,red] (0, 1) to (0,-1.0);
        \node at (A) [above]       {$A$};
        \node at (B) [below right] {$B$};
        \node at (C) [below left]  {$C$};
    \end{scope}

    \begin{scope}[xshift=-2.333cm]
        \defcoords;
        \draw (A) to (B) to (C) to cycle;
        \draw (O) [p_arc=80:-20:1.333];
        \draw (O) [p_arc=-40:-140:1.333];
        \draw (O) [p_arc=200:100:1.333];
        \node at (A) [above]       {$A$};
        \node at (B) [below right] {$C$};
        \node at (C) [below left]  {$B$};
    \end{scope}

    \begin{scope}[xshift=2.333cm]
        \defcoords;
        \draw (A) to (B) to (C) to cycle;
        \draw (O) [p_arc=-140:-40:1.333];
        \draw (0,-1.6) [p_arc=40:140:1.2];
        \draw[densely dashed,thin,red] (0, 1) to (0,-1.0);
        \node at (A) [above]       {$B$};
        \node at (B) [below right] {$A$};
        \node at (C) [below left]  {$C$};
    \end{scope}

    \begin{scope}[xshift=7.0cm]
        \defcoords;
        \draw (A) to (B) to (C) to cycle;
        \draw (O) [p_arc=80:-20:1.333];
        \draw (O) [p_arc=-40:-140:1.333];
        \draw (O) [p_arc=200:100:1.333];
        \node at (A) [above]       {$B$};
        \node at (B) [below right] {$C$};
        \node at (C) [below left]  {$A$};
    \end{scope}

    \begin{scope}[xshift=11.666cm]
        \defcoords;
        \draw (A) to (B) to (C) to cycle;
        \node at (A) [above]       {$A$};
        \node at (B) [below right] {$B$};
        \node at (C) [below left]  {$C$};
    \end{scope}
    \draw[draw=blue,->,thick] (-5.16, 0) to node[above] {$a$} (-4.16,  0);
    \draw[draw=blue,->,thick] (-0.50, 0) to node[above] {$r$} ( 0.50,  0);
    \draw[draw=blue,->,thick] ( 4.16, 0) to node[above] {$a$} ( 5.16,  0);
    \draw[draw=blue,->,thick] ( 9.16, 0) to node[above] {$a$} ( 10.16, 0);
\end{tikzpicture}
            }
            \caption{Restraints on the Dihedral Group $D_{6}$}
            \label{fig:Restraints_on_D_6}
        \end{figure}
        This tells us that rotation and reflection has an inverse notion
        (rotate backwards and reflect again, respectively). Since the group is
        determined by these two operations, we need only check that
        $r*(ar)=(r*a)*r$ and $a*(r*a)=(a*r)*a$ to determine the associative of
        the rest of the group. We can do this by examing the triangle
        (see Fig.~\ref{fig:Assoc_of_Dihedral_Group_D6}). That is, if we
        rotate, and the follow by reflecting and then rotating, it's the same
        thing as rotating and then reflecting, following by rotating again.
        Similarly, if we reflect, and then follow by rotating and then
        reflecting, this is equivalence to reflecting and then rotating, and
        then following with another reflection.
        \begin{figure}[H]
            \centering
            \captionsetup{type=figure}
            \resizebox{\textwidth}{!}{%
                \begin{tikzpicture}[%
    >=Latex,
    p_arc/.style args={#1:#2:#3}{
        insert path={+ (#1:#3) arc (#1:#2:#3)},->
    },
    semithick
]
\newcommand*{\defcoords}{%
    \coordinate (O) at ( 0.0,    -0.333333);
    \coordinate (A) at ( 0.0,     1.0);
    \coordinate (B) at ( 1.1547, -1.0);
    \coordinate (C) at (-1.1547, -1.0);
}

    \begin{scope}[xshift=-7.0cm]
        \defcoords;
        \draw (A) to (B) to (C) to cycle;
        \draw (O) [p_arc=80:-20:1.333];
        \draw (O) [p_arc=-40:-140:1.333];
        \draw (O) [p_arc=200:100:1.333];
        \node at (A) [above]       {$A$};
        \node at (B) [below right] {$B$};
        \node at (C) [below left]  {$C$};
    \end{scope}

    \begin{scope}[xshift=-2.333cm]
        \defcoords;
        \draw (A) to (B) to (C) to cycle;
        \draw (O) [p_arc=-140:-40:1.333];
        \draw (0,-1.6) [p_arc=40:140:1.2];
        \draw[densely dashed,thin,red] (0, 1) to (0,-1.0);
        \node at (A) [above]       {$C$};
        \node at (B) [below right] {$A$};
        \node at (C) [below left]  {$B$};
    \end{scope}

    \begin{scope}[xshift=2.333cm]
        \defcoords;
        \draw (A) to (B) to (C) to cycle;
        \draw (O) [p_arc=80:-20:1.333];
        \draw (O) [p_arc=-40:-140:1.333];
        \draw (O) [p_arc=200:100:1.333];
        \node at (A) [above]       {$C$};
        \node at (B) [below right] {$B$};
        \node at (C) [below left]  {$A$};
    \end{scope}
    \begin{scope}[xshift=7.0cm]
        \defcoords;
        \draw (A) to (B) to (C) to cycle;
        \node at (A) [above]       {$C$};
        \node at (B) [below right] {$A$};
        \node at (C) [below left]  {$B$};
    \end{scope}
    \draw[draw=blue,->,thick] (-5.16, 0) to node[above] {$r$} (-4.16, 0);
    \draw[draw=blue,->,thick] (-0.50, 0) to node[above] {$a$} ( 0.50, 0);
    \draw[draw=blue,->,thick] ( 4.16, 0) to node[above] {$r$} ( 5.16, 0);

    \begin{scope}[yshift=-4cm]
        \begin{scope}[xshift=-7.0cm]
            \defcoords;
            \draw (A) to (B) to (C) to cycle;
            \draw (O) [p_arc=-140:-40:1.333];
            \draw (0,-1.6) [p_arc=40:140:1.2];
            \draw[densely dashed,thin,red] (0, 1) to (0,-1.0);
            \node at (A) [above]       {$A$};
            \node at (B) [below right] {$B$};
            \node at (C) [below left]  {$C$};
        \end{scope}
    
        \begin{scope}[xshift=-2.333cm]
            \defcoords;
            \draw (A) to (B) to (C) to cycle;
            \draw (O) [p_arc=80:-20:1.333];
            \draw (O) [p_arc=-40:-140:1.333];
            \draw (O) [p_arc=200:100:1.333];
            \node at (A) [above]       {$A$};
            \node at (B) [below right] {$C$};
            \node at (C) [below left]  {$B$};
        \end{scope}
    
        \begin{scope}[xshift=2.333cm]
            \defcoords;
            \draw (A) to (B) to (C) to cycle;
            \draw (O) [p_arc=-140:-40:1.333];
            \draw (0,-1.6) [p_arc=40:140:1.2];
            \draw[densely dashed,thin,red] (0, 1) to (0,-1.0);
            \node at (A) [above]       {$B$};
            \node at (B) [below right] {$A$};
            \node at (C) [below left]  {$C$};
        \end{scope}

        \begin{scope}[xshift=7.0cm]
            \defcoords;
            \draw (A) to (B) to (C) to cycle;
            \node at (A) [above]       {$B$};
            \node at (B) [below right] {$C$};
            \node at (C) [below left]  {$A$};
        \end{scope}
        \draw[draw=blue,->,thick] (-5.16, 0) to node[above] {$a$} (-4.16, 0);
        \draw[draw=blue,->,thick] (-0.50, 0) to node[above] {$r$} ( 0.50, 0);
        \draw[draw=blue,->,thick] ( 4.16, 0) to node[above] {$a$} ( 5.16, 0);
    \end{scope}
    \let\defcoords\undefined
\end{tikzpicture}
            }
            \caption{Associativity of the Dihedral Group $D_{6}$}
            \label{fig:Assoc_of_Dihedral_Group_D6}
        \end{figure}
        With this we can compute the table (Tab.~\ref{tab:Cayley_Table_D_6}).
        We now see that $(D_{6},*)$ is not an Abelian group since the successor
        operation is not commutative. That is, $r*a=a*r^{2}\ne{a}*r$, and thus
        $a*r\ne{r}*a$.
    \end{lexample}
    \begin{table}[H]
        \centering
        \captionsetup{type=table}
        \begin{tabular}{c|cccccc}
            $*$&$e$&$r$&$r^{2}$&$a$&$a*r$&$a*r^{2}$\\
            \hline
            $e$&$e$&$r$&$r^{2}$&$a$&$a*r$&$a*r^{2}$\\
            $r$&$r$&$r^{2}$&$e$&$a*r^{2}$&$a$&$a*r$\\
            $r^{2}$&$r^{2}$&$e$&$r$&$a*r$&$a*r^{2}$&$a$\\
            $a$&$a$&$a*r$&$a*r^{2}$&$e$&$r$&$r^{2}$\\
            $a*r$&$a*r$&$a*r^{2}$&$a$&$r^{2}$&$e$&$r$\\
            $a*r^{2}$&$a*r^{2}$&$a$&$a*r$&$r$&$r^{2}$&$e$
        \end{tabular}
        \caption{Cayley Table of $D_{6}$}
        \label{tab:Cayley_Table_D_6}
    \end{table}
    \begin{example}
        The trivial group is the set $G=\{\emptyset\}$, although we usually
        label $G=\{e\}$ or $G=\{0\}$ when considering the trivial group, and we
        combine this with the operation $*:G\times{G}\rightarrow{G}$ defined by
        $e*e=e$. This makes $(G,*)$ a group, but moreover it is an Abelian group
        since $*$ is trivially commutative.
    \end{example}
    \begin{example}
        Addition in $\mathbb{Z}$ and multiplication in $\mathbb{Q}^{+}$ are
        commutative operations, and thus $(\mathbb{Z},+)$ and
        $(\mathbb{Q}^{+},\cdot\,)$ are Abelian groups.
    \end{example}
    \begin{example}
        Addition of complex numbers $\mathbb{C}$ is commutative and associative,
        as is the multiplication of non-zero complex numbers. Thus
        $(\mathbb{C},+)$ and $(\mathbb{C}\setminus\{0\},\cdot\,)$ are
        Abelian groups. The identities are $0=0+i0$ and $1=1+i0$, respectively.
        Recall that we developed the arithmetic of $\mathbb{C}$ based on the
        structure of $\mathbb{R}^{2}$ together with the arithmetic of
        $\mathbb{R}$. We define:
        \begin{equation}
            a+ib=(a,\,b)
            \quad\quad
            a,b\in\mathbb{R}
        \end{equation}
        And defined addition and multiplication as follows:
        \begin{subequations}
            \begin{align}
                (a+ib)+(c+id)&=(a+c)+i(b+d)\\
                (a+ib)\cdot(c+id)&=(ac-bd)+i(bc+ad)
            \end{align}
        \end{subequations}
        The commutativity and associativity of these operations stems from the
        arithmetic of $\mathbb{R}$, as does the fact that 0 and 1 are identities
        fo these operations.
    \end{example}
    \begin{example}
        There are other Abelian group structures we can place on
        $\mathbb{R}^{2}$. Much the way the arithemtic of $\mathbb{Q}$ was
        developed by building from the arithmetic of $\mathbb{Z}$ and putting
        a structure on $\mathbb{Z}^{2}$, we can do the same of $\mathbb{R}^{2}$.
        Let $*$ be the binary operation on
        $\mathbb{R}\times(\mathbb{R}\setminus\{0\})$ defines as follows:
        \begin{equation}
            (a,\,b)*(c,\,d)=(ad+bc,\,bd)
        \end{equation}
        where $ad$ denotes the uusual real valued multiplication of $a$ and $b$,
        and where $+$ is the usual addition in $\mathbb{R}$. This operation
        will be clearer if we write it out as:
        \begin{equation}
            \frac{a}{b}*\frac{c}{d}
            =\frac{ad+bc}{bd}
        \end{equation}
        Now we see why we required the second entry to be non-zero, this is the
        usually additive operation for fractions of real numbers. As such is is
        associative and commutative. Moreover, there is an identity $(0,1)$, and
        an inverse $(\minus{a},b)$. Thus, we have an Abelian group.
    \end{example}
    \begin{lexample}{Group Operation on Power Set}{Group_Operation_on_Power_Set}
        When studying Boolean algebras we saw that on a set $A$ the
        structure $(\mathcal{P}(A),\cup,\cap)$ does not yield inverse elements.
        That is, given $B,C\subseteq{A}$, if $A\cup{C}=\emptyset$ then
        $A=C=\emptyset$, and if $B\cap{C}=A$, then $A=B=C$. So Boolean algebras
        do not have an underlying group structure. That is, neither
        $(\mathcal{P}(A),\cup)$ not $(\mathcal{P}(A),\cap)$ are groups. We can
        place a group structure on $\mathcal{P}(A)$ by considering another
        familiar operation: The symmetric
        difference\index{Symmetric Difference}, $\ominus$. Recall that this is
        defined as:
        \begin{equation}
            B\ominus{C}=(B\cup{C})\setminus(B\cap{C})
        \end{equation}
        The symmetric difference is both associative and commutative, and
        moreover there is a unital element since $B\ominus\emptyset=B$. Lastly,
        there is an inverse element, since $B\ominus{B}=\emptyset$. Thus,
        $(\mathcal{P}(A),\ominus)$ forms an Abelian group.
    \end{lexample}
    \begin{lexample}{Reflections on a Square}{Group_Reflection_on_a_Square}
        We now consider the reflections on a square, but do not consider
        rotations. We can visualize this by consider a point on one of the
        four verticies of a square and allowing it to move diagonally,
        horizontally, or vertically. Our operation is again the
        \textit{successor} operation. Let's label $h$ as horizontal, and
        similarly $d$ and $v$ for diagonal and vertical. Let $e$ denote the
        \textit{do nothing} reflection. Note that each of these
        \textit{generators} is it's own inverse. Reflecting across the
        diagonal twice is equivalent to doing nothing, as are reflecting twice
        vertically or horizontally. We thus obtain the following restrictions:
        \begin{equation}
            h*h=e
            \quad\quad
            v*v=e
            \quad\quad
            d*d=e
        \end{equation}
        Using these constraints, and the diagram below
        (Fig.~\ref{fig:Group_Reflections_on_a_Square}), we can once again
        compute the Cayley table for this operation. The table is given by:
        \begin{table}[H]
            \centering
            \captionsetup{type=table}
            \begin{tabular}{c|cccc}
                $e$&$e$&$h$&$v$&$d$\\
                \hline
                $e$&$e$&$h$&$v$&$d$\\
                $h$&$h$&$e$&$d$&$v$\\
                $v$&$v$&$d$&$e$&$h$\\
                $d$&$d$&$v$&$h$&$e$
            \end{tabular}
            \caption{Cayley Table for Reflection on a Square}
            \label{tab:Cayley_Table_Reflection_on_Square}
        \end{table}
        This group is equivalent to the group structure that can be placed on
        $\mathbb{Z}_{2}\times\mathbb{Z}_{2}$ by considering pointwise modular
        addition. For example, we have:
        \begin{equation}
            (0,1)+(1,1)=(0+1,1+1)=(1,0)
        \end{equation}
        and simillary for the other elements. This group structure, called the
        \textit{direct product} of $(\mathbb{Z}_{2},+)$ with itself, is the same
        as the group structure we've described here. By looking at the Cayley
        table we see that the group of reflections on the square is an Abelian
        group. This is contrary to $D_{6}$ where we allowed both rotations and
        reflections and saw that the group is not Abelian.
    \end{lexample}
    \begin{figure}[H]
        \centering
        \captionsetup{type=figure}
        \begin{tikzpicture}[>=Latex]
            \coordinate (A) at (0, 0);
            \coordinate (B) at (4, 0);
            \coordinate (C) at (4, 4);
            \coordinate (D) at (0, 4);
            \draw[densely dashed, draw=red] (-1.0, -1.0) to (5.0, 5.0);
            \draw[densely dashed, draw=red] ( 2.0, -1.0) to (2.0, 5.0);
            \draw[densely dashed, draw=red] (-1.0,  2.0) to (5.0, 2.0);
            \draw[densely dashed, semithick]
                (A) to (B) to (C) to (D) to cycle;
            \draw[->, thick] (A) to (0.0, 2.0);
            \draw[->, thick] (A) to (2.0, 0.0);
            \draw[->, thick] (A) to (1.5, 1.5);
            \draw[fill=black] (A) circle (0.1);
            \draw[fill=black] (B) circle (0.1);
            \draw[fill=black] (C) circle (0.1);
            \draw[fill=black] (D) circle (0.1);
            \node at (A) [left]  {$A$};
            \node at (B) [right] {$B$};
            \node at (C) [right] {$C$};
            \node at (D) [left]  {$D$};
        \end{tikzpicture}
        \caption{Reflections on a Square}
        \label{fig:Group_Reflections_on_a_Square}
    \end{figure}
    \begin{lexample}{Group of Two Coins}{Group_of_Coins}
        Consider two coins located at points $A$ and $B$. We group generated by
        the following operations:
        \begin{itemize}
            \item $e$: Do nothing.
            \item $f$: Flip the coin at $A$.
            \item $s$: Swap the coins $A$ and $B$.
        \end{itemize}
        From this we get that there are 8 moves total that can be generated from
        the \textit{successor} operation, which generated by the following
        constraints:
        \begin{equation}
            f^{2}=e
            \quad\quad
            s^{2}=e
        \end{equation}
        That is, flipping the coin at $A$ twice does nothing, nor does swapping
        the two coins twice. We also need one more restraint, and this tells us
        how to flip both coins without swapping them. We can achieve this by
        doing flip-swap-flip-swap, or equivalently by swap-flip-swap-flip. Thus,
        we have one more restraint:
        \begin{equation}
            fsfs=sfsf
        \end{equation}
        This is equivalent to requiring $(fsfs)^{2}=e$. We can now compute the
        table (Tab.~\ref{tab:Group_Formed_on_Two_Coins}) finding that this group
        is non-Abelian.
    \end{lexample}
    \begin{table}[H]
        \centering
        \captionsetup{type=table}
        \begin{tabular}{c|cccccccc}
            $*$&$e$&$f$&$s$&$fs$&$sf$&$fsf$&$sfs$&$sfsf$\\
            \hline
            $e$&$e$&$f$&$s$&$fs$&$sf$&$fsf$&$sfs$&$sfsf$\\
            $f$&$f$&$e$&$fs$&$s$&$fsf$&$sf$&$sfsf$&$sfs$\\
            $s$&$s$&$sf$&$e$&$sfs$&$f$&$sfsf$&$fs$&$fsf$\\
            $fs$&$fs$&$fsf$&$f$&$sfsf$&$e$&$sfs$&$s$&$sf$\\
            $sf$&$sf$&$s$&$sfs$&$e$&$sfsf$&$f$&$fsf$&$fs$\\
            $fsf$&$fsf$&$fs$&$sfsf$&$f$&$sfs$&$e$&$sf$&$s$\\
            $sfs$&$sfs$&$sfsf$&$sf$&$fsf$&$s$&$fs$&$e$&$f$\\
            $sfsf$&$sfsf$&$sfs$&$fsf$&$sf$&$fs$&$s$&$f$&$e$
        \end{tabular}
        \caption{Cayley Table of Group Formed on Two Coins}
        \label{tab:Group_Formed_on_Two_Coins}
    \end{table}
    \begin{theorem}
        If $(G,*)$ is a group, if $a,b\in{G}$ commute, and if $g\in{G}$, then
        $g*a*g^{\minus{1}}$ and $g*b*g^{\minus{1}}$ commute.
    \end{theorem}
    \begin{proof}
        For:
        \begin{align}
            (g*a*g^{\minus{1}})*(g*b*g^{\minus{1}})
            &=(g*a)*(g^{\minus{1}}*g)*(b*g^{\minus{1}})
            \tag{Associativity}\\
            &=(g*a)*e*(b*g^{\minus{1}})
            \tag{Inverse}\\
            &=(g*a)*(b*g^{\minus{1}})
            \tag{Identity}\\
            &=g*(a*b)*g^{\minus{1}}
            \tag{Associativity}\\
            &=g*(b*a)*g^{\minus{1}}
            \tag{Commutativity}\\
            &=g*\big((b*e)*a\big)*g^{\minus{1}}
            \tag{Identity}\\
            &=g*\Big(\big(b*(g^{\minus{1}}*g)\big)*a\Big)*g^{\minus{1}}
            \tag{Inverse}\\
            &=g*\Big(\big((b*g^{\minus{1}})*g\big)*a)*g^{\minus{1}}
            \tag{Associativity}\\
            &=g*(b*g^{\minus{1}})*(g*a)*g^{\minus{1}}
            \tag{Associativity}\\
            &=(g*b*g^{\minus{1}})*(g*a*g^{\minus{1}})
            \tag{Associativity}
        \end{align}
        And thus $g*a*g^{\minus{1}}$ and $g*b*g^{\minus{1}}$ commute
        (Def.~\ref{def:Commuting_Elements}).
    \end{proof}
    \begin{theorem}
        If $(G,*)$ is a group, if $e$ is the unital element of $G$, and if
        $a,b\in{G}$ are such that $a*b=a$, then $b=e$.
    \end{theorem}
    \begin{proof}
        For since $e$ is the unital element, $a*e=a$
        (Def.~\ref{def:Unital_Element}). But then by the left cancellation
        law (Thm.~\ref{thm:Group_Left_Cancellation_Law}), $b=e$.
    \end{proof}
    \begin{theorem}
        If $(G,*)$ is a group, if $e$ is the unital element of $G$, and if
        $a,b\in{G}$ are such that $a*b=b$, then $a=e$.
    \end{theorem}
    \begin{proof}
        For since $e$ is the unital element, $e*b=b$
        (Def.~\ref{def:Unital_Element}). But then by the right cancellation
        law (Thm.~\ref{thm:Group_Right_Cancellation_Law}), $a=e$.
    \end{proof}
    These two theorems state some slightly stronger than the uniqueness of
    the unital element, and show that to check that an element $e\in{G}$ is the
    unital element it suffices to see that $a*e=a$ for at least one $a\in{G}$.
    \begin{theorem}
        \label{thm:Group_Mult_by_Const_is_Surj_Func}%
        If $(G,*)$ is a group, if $a\in{G}$, and if $f:G\rightarrow{G}$ is
        defined by $f(g)=a*g$ for all $g\in{G}$, then $f$ is surjective.
    \end{theorem}
    \begin{proof}
        For suppose not. Then there is a $y\in{G}$ such that, for all $x\in{G}$
        it is true that $f(x)\ne{y}$. But by the Latin square property
        (Thm.~\ref{thm:Group_Latin_Square}), if $a,y\in{G}$ then there is a
        unique $x\in{G}$ such that $a*x=y$. But then $f(x)=y$, a contradiction.
        Therefore, $f$ is surjective.
    \end{proof}
    \begin{theorem}
        \label{thm:Group_Mult_by_Const_is_Inj_Func}%
        If $(G,*)$ is a group, if $a\in{G}$, and if $f:G\rightarrow{G}$ is
        defined by $f(g)=a*g$ for all $g\in{G}$, then $f$ is injective.
    \end{theorem}
    \begin{proof}
        For suppose not. Then there exists $x,y\in{G}$ such that $x\ne{y}$ and
        $f(x)=f(y)$. But then $a*x=a*y$, and thus by the left cancellation law
        (Thm.~\ref{thm:Group_Left_Cancellation_Law}), $x=y$, a contradiction.
        Therefore, $f$ is injective.
    \end{proof}
    \begin{theorem}
        If $(G,*)$ is a group, if $a\in{G}$, and if $f:G\rightarrow{G}$ is
        defined by $f(g)=a*g$ for all $g\in{G}$, then $f$ is a permutation.
    \end{theorem}
    \begin{proof}
        For by Thm.~\ref{thm:Group_Mult_by_Const_is_Surj_Func}, $f$ is
        surjective. And by Thm.~\ref{thm:Group_Mult_by_Const_is_Inj_Func},
        $f$ is injective. But then $f$ is a bijection
        (Def.~\ref{def:Bijective_Function}), and therefore $f$ is a bijective
        function from $G$ to itself, and is therefore a permutation
        (Def.~\ref{def:Permutation}).
    \end{proof}
    The function presented in these three theorems is the main tool used in
    proving Cayley's Theorem\index{Cayley's Theorem}, which is one of the
    classic results of group theory.
    \begin{fdefinition}{Boolean Group}{Boolean_Group}
        A Boolean group is a group $(G,*)$ such that for all $a\in{G}$ it is
        true that $a^{2}=e$, where $e$ is the unital element of $G$.
        \index{Group!Boolean}
    \end{fdefinition}
    \begin{theorem}
        If $(G,*)$ is a Boolean group, then it is Abelian.
    \end{theorem}
    \begin{proof}
        For suppose not. Then there are $a,b\in{G}$ such that $a*b\ne{b}*a$.
        But:
        \begin{align}
            a*b&=
            (a*b)*e
            \tag{Identity}\\
            &=(a*b)*(b*a)^{2}
            \tag{Boolean Property}\\
            &=\big((a*b)*(b*a)\big)*(b*a)
            \tag{Associativity}\\
            &=\big(a*(b*b)*a\big)*(b*a)
            \tag{Associativity}\\
            &=(a*e*a)*(b*a)
            \tag{Boolean Property}\\
            &=(a*a)*(b*a)
            \tag{Identity}\\
            &=e*(b*a)
            \tag{Boolean Property}\\
            &=b*a
            \tag{Identity}
        \end{align}
        A contradiction. Thus, $(G,*)$ is Abelian.
    \end{proof}
    \subsection{Direct Product}
        \begin{fdefinition}{Group Direct Product}{Group_Direct_Product}
            The direct product of two groups $(G,*)$ and $(G',\circ)$ is the
            Cartesian product $G\times{G}$ together with the binary operation
            $\star:(G\times{G}')\times(G\times{G}')\rightarrow{G}\times{G}'$
            defined by:\index{Group!Direct Product of}
            \begin{equation*}
                (a,a')\star(b,b')=(a*b,a'\circ{b}')
            \end{equation*}
        \end{fdefinition}
        \begin{theorem}
            If $(G,*)$ and $(G',\circ)$ are groups, and if $(G\times{G}',\star)$
            is their direct product, then it is a group.
        \end{theorem}
        \begin{theorem}
            If $(G,*)$ and $(G',\circ)$ are Abelian groups, and if
            $(G\times{G}',\star)$ is their direct product, then it is an Abelian
            group.
        \end{theorem}
        \begin{theorem}
            The direct product of Boolean groups is Boolean.
        \end{theorem}
        \begin{example}
            As we will soon see, taking the direct product of two groups may not
            produce anything new and exciting. For example, the direct product
            of $\mathbb{Z}_{2}$ with $\mathbb{Z}_{3}$ is equivalent (isomorphic)
            to $\mathbb{Z}_{6}$. That is,
            $\mathbb{Z}_{2}\times\mathbb{Z}_{3}\simeq\mathbb{Z}_{6}$. We've
            already seen examples where this is not the case, and indeed
            $\mathbb{Z}_{4}$ and $\mathbb{Z}_{2}\times\mathbb{Z}_{2}$ are
            different groups.
        \end{example}
    \subsection{Subgroups}
        Subgroups are the group analog of subsets and subspaces. Given a group
        $(G,*)$, we consider a subset $H\subseteq{G}$ such that $H$ is
        \textit{closed} under the group operation $*$, and such that it is
        closed to inverses. We can phrase this precisely.
        \begin{fdefinition}{Subgroup}{Subgroup}
            A subgroup of a \gls{group} $(G,*)$ is a
            \glslink{non-empty set}{non-empty} \gls{subset} $H\subseteq{G}$ such
            that for all $a\in{H}$ it is true that $a^{\minus{1}}\in{G}$, and
            for all $a,b\in{H}$ it is true that $a*b\in{G}$.
            \index{Group!Subgroup}
        \end{fdefinition}
        \begin{example}
            If we consider $(\mathbb{Z},+)$ and let $\mathbb{Z}_{e}$ be the even
            integers, then $\mathbb{Z}_{e}$ is a subgroup of $\mathbb{Z}$. For
            the sum of two even integers is again even, and the negative of an
            even integer is even, and thus $\mathbb{Z}_{e}$ is closed under
            addition and under negation.
        \end{example}
        \begin{example}
            Let $(\mathbb{R},\cdot)$ denote the multiplicative group of positive
            real numbers. Then $\mathbb{Q}^{+}$ is a subgroup. That is, the
            product of rational numbers is rational numbers, and the inverse
            of $p/q$ (with $p,q>0$) is $q/p$, which is rational. Thus
            $\mathbb{Q}^{+}$ is closed under multiplicative and inverses and is
            thus a subgroup of $\mathbb{R}^{+}$.
        \end{example}
        \begin{example}
            Let $(\mathbb{Z}_{4},+)$ denote the group of 4 elements under modulo
            arithmetic. There is a $\mathbb{Z}_{2}$ subgroup hiding in here, for
            let $H=\{0,2\}$. We can check case by case that this is a subgroup:
            \begin{equation}
                0+0=0
                \quad\quad
                0+2=2
                \quad\quad
                2+0=2
                \quad\quad
                2+2=0
            \end{equation}
            and thus $H$ is closed under modular addition. Finally, $0$ is its
            own inverse, and since $2+2=0$ we see that $2$ is its own inverse as
            well, and thus $H$ is closed under inversion. $H$ is a subgroup of
            $\mathbb{Z}_{4}$. Our claim that this is a $\mathbb{Z}_{2}$ in
            disguised can be realized by the function
            $f:H\rightarrow\mathbb{Z}_{2}$ defined by $f(0)=2$ and $f(2)=1$.
            This shows that $H$ and $\mathbb{Z}_{2}$ are essentially
            relabellings of the same structure.
        \end{example}
        \begin{example}
            If $\mathcal{F}(\mathbb{R},\mathbb{R})$ is the set of all functions
            $f:\mathbb{R}\rightarrow\mathbb{R}$, and if $\boldsymbol{+}$ denotes
            function addition:
            \begin{equation}
                (f\boldsymbol{+}g)(x)=f(x)+g(x)
                \quad\quad
                x\in\mathbb{R}
            \end{equation}
            Then $(\mathcal{F}(\mathbb{R},\mathbb{R}),\boldsymbol{+})$ is a
            group. The identity is the zero function, $\mathscr{O}(x)=0$ for all
            $x\in\mathbb{R}$, and associativity stems from the associativity of
            addition in $\mathbb{R}$. To see that there are inverses, let
            $\minus{f}$ denote the function:
            \begin{equation}
                (\minus{f})(x)=\minus{f}(x)
                \quad\quad
                x\in\mathbb{R}
            \end{equation}
            If we let $\mathcal{C}(\mathbb{R},\mathbb{R})$ denote the set of all
            \textit{continuous} functions, then this is a subgroup. Similarly,
            if $\mathcal{C}^{1}(\mathbb{R},\mathbb{R})$ denotes the set of all
            differentiable functions, then this too is a subgroup. In general,
            if $k\in\mathbb{N}$ and if $\mathcal{C}^{k}(\mathbb{R},\mathbb{R})$
            denotes the set of all $k$ times differentiable function
            $f:\mathbb{R}\rightarrow\mathbb{R}$, then this is a subgroup of
            $(\mathcal{F}(\mathbb{R},\mathbb{R}))$.
        \end{example}
        \begin{example}
            Again letting $\mathcal{F}(\mathbb{R},\mathbb{R})$, the subset of
            all periodic functions $f:\mathbb{R}\rightarrow\mathbb{R}$ with
            period $T>0$ is again a subgroup
            $(\mathcal{F}(\mathbb{R},\mathbb{R}),\boldsymbol{+})$.
        \end{example}
        \begin{example}
            If $A$ and $B$ are sets, and if $A\subseteq{B}$, then
            $\mathcal{P}(A)$ is a subgroup of $(\mathcal{P}(B),\ominus)$, where
            $\ominus$ denotes the symmetric difference operation, and
            $\mathcal{P}(B)$ is the power set of $B$.
        \end{example}
        \begin{theorem}
            If $(G,*)$ is a group, and if $e\in{G}$ is the unital element, then
            $\{e\}$ is a subgroup of $(G,*)$.
        \end{theorem}
        \begin{proof}
            For since $e*e=e$, $\{e\}$ is closed under $*$ and to inverses.
            Thus, $\{e\}$ is a subgroup (Def.~\ref{def:Subgroup}).
        \end{proof}
        This is called the trivial subgroup of a group. At the other extreme,
        the entire group is a subgroup of itself.
        \begin{theorem}
            If $(G,*)$ is a group, then $G$ is a subgroup of $(G,*)$.
        \end{theorem}
        \begin{proof}
            For since $(G,*)$ is a group, $G$ is closed to $*$ and inverses
            (Def.~\ref{def:Group}). Thus, $G$ is a subgroup
            (Def.~\ref{def:Subgroup}).
        \end{proof}
        \begin{theorem}
            \label{thm:Subgroup_Contains_Identity}%
            If $(G,*)$ is a group, if $H\subseteq{G}$ is a subgroup, and if
            $e\in{G}$ is the unital element, then $e\in{H}$.
        \end{theorem}
        \begin{proof}
            For if $H$ is a subgroup of $(G,*)$ then it is non-empty
            (Def.~\ref{def:Subgroup}). But if $H$ is non-empty, then there is an
            $a\in{H}$ (Def.~\ref{def:Non_Empty_Set}). But $H$ is a subgroup, and
            thus it is true that $a^{\minus{1}}\in{H}$
            (Def.~\ref{def:Subgroup}) and since subgroups are closed under $*$
            we have that $a*a^{\minus{1}}\in{H}$. But $a*a^{\minus{1}}$ is a
            unital element, and unital elements are unique
            (Thm.~\ref{thm:Unital_Elements_are_Unique}). Thus, $e\in{H}$.
        \end{proof}
        \begin{theorem}
            \label{thm:Restriction_of_bi_op_to_subgroup_is_bi_op}*
            If $(G,*)$ is a group, if $H\subseteq{G}$ is a subgroup, and if
            $*|_{H}$ is the restriction of $*$ to $H$, then $*$ is a binary
            operation on $H$.
        \end{theorem}
        \begin{proof}
            Since the restriction of a function is a function, it suffices to
            show that the range of $*|_{H}$ is $H$. For suppose not. Then there
            exists $a,b\in{H}$ such that $a*b\notin{H}$. But $H$ is a subgroup
            and thus if $a,b\in{H}$, then $a*b\in{H}$ (Def.~\ref{def:Subgroup}),
            a contradiction. Thus, $*|_{H}$ is a binary operation on $H$
            (Def.~\ref{def:Binary_Operation}).
        \end{proof}
        \begin{theorem}
            If $(G,*)$ is a group, if $H\subseteq{G}$ is a subgroup, and if
            $*|_{H}$ is the restriction of $*$ to $H$, then $(H,*|_{H})$ is a
            group.
        \end{theorem}
        \begin{proof}
            For $*|_{H}$ is a binary operation on $H$
            (Thm.~\ref{thm:Restriction_of_bi_op_to_subgroup_is_bi_op}). Moreover
            since $H$ is a subgroup, if $a\in{H}$ then $a^{\minus{1}}\in{H}$
            (Def.~\ref{def:Subgroup}) and thus for all $a\in{H}$ it is true that
            $a$ is invertible. Lastly, there is a unital element of $(H,*_{H})$
            (Thm.~\ref{thm:Subgroup_Contains_Identity}). Therefore,
            $(H,*|_{H})$ is a group (Def.~\ref{def:Group}).
        \end{proof}
        \begin{theorem}
            If $(G,*)$ is an Abelian group, and if $H\subseteq{G}$ is defined
            by:
            \begin{equation}
                H=\big\{\,a\in{G}\;|\;a^{2}=e\,\big\}
            \end{equation}
            then $H$ is a subgroup of $G$.
        \end{theorem}
        \begin{proof}
            Since $(G,*)$ is Abelian, $*$ is commutative
            (Def.~\ref{def:Abelian_Group}). Thus, if $a,b\in{G}$, then:
            \begin{align}
                (a*b)^{2}&=(a*b)*(a*b)
                \tag{Definition of $x^{n}$}\\
                &=\big(a*(b*a)\big)*b
                \tag{Associativity}\\
                &=\big(a*(a*b)\big)*b
                \tag{Commutativity}\\
                &=\big((a*a)*b\big)*b
                \tag{Associativity}\\
                &=(e*b)*b
                \tag{Hypothesis}\\
                &=b*b
                \tag{Identity}\\
                &=e
                \tag{Hypothesis}
            \end{align}
            and thus $a*b\in{H}$. Moreover, if $a\in{H}$ then:
            \begin{align}
                (a^{\minus{1}})^{2}
                &=a^{\minus{1}}*a^{\minus{1}}
                \tag{Definition of $x^{n}$}\\
                &=(a^{\minus{1}}*e)*a^{\minus{1}}
                \tag{Identity}\\
                &=\big(a^{\minus{1}}*(a*a)\big)*a^{\minus{1}}
                \tag{Hypothesis}\\
                &=\big((a^{\minus{1}}*a)*a\big)*a^{\minus{1}}
                \tag{Associativity}\\
                &=(e*a)*a^{\minus{1}}
                \tag{Inverse}\\
                &=a*a^{\minus{1}}
                \tag{Identity}\\
                &=e
                \tag{Inverse}
            \end{align}
        \end{proof}
        \begin{theorem}
            If $(G,*)$ is an Abelian group, if $H\subseteq{G}$ is defined by:
            \begin{equation}
                H=\{\,a\in{G}\;|\;\textrm{There exists }b\in{G}
                    \textrm{ such that }b^{2}=a\,\}
            \end{equation}
            Then $H$ is a subgroup of $(G,*)$.
        \end{theorem}
        \begin{proof}
            For if $a,b\in{H}$, then there exists $c,d\in{G}$ such that
            $c^{2}=a$ and $d^{2}=b$. But $G$ is Abelian, and thus $*$ is
            commutative (Def.~\ref{def:Abelian_Group}). Thus:
            \begin{equation}
                a*b=c^{2}*d^{2}=(c*d)*(c*d)=(c*d)^{2}
            \end{equation}
            and thus $a*b\in{H}$. Moreover $a^{\minus{1}}\in{H}$.
        \end{proof}
        \begin{theorem}
            If $(G,*)$ is a group, if $H,K\subseteq{G}$ are subgroups of
            $(G,*)$, and $H\subseteq{K}$, then $H$ is a subgroup of
            $(K,*|_{K})$, where $*|_{K}$ is the restriction of $*$ to $K$.
        \end{theorem}
        \begin{proof}
            For if $H\subseteq{K}$, then for all $a\in{H}$ it is true that
            $a\in{K}$ (Def.~\ref{def:Subsets}). But since $H$ is a subgroup of
            $(G,*)$, for all $a,b\in{H}$ it is true that $a*b\in{H}$
            (Def.~\ref{def:Subgroup}). But then $a*b\in{K}$ and thus
            $a*b=a*|_{K}b$, and thus $H$ is closed under $*|_{K}$. Moreover, if
            $H$ is a subgroup of $G$, then for all $a\in{H}$ it is true that
            $a^{\minus{1}}\in{H}$. But if $a^{\minus{1}}\in{H}$, then
            $a^{\minus{1}}\in{K}$ since $H\subseteq{K}$. Thus, $H$ is a subgroup
            of $(K,*|_{K})$.
        \end{proof}
        \begin{fdefinition}{Center of a Group}{Center_of_Group}
            The center of a group $(G,*)$ is the set $Z(G,*)$ defined by:
            \index{Group!Center of}
            \begin{equation*}
                Z(G,*)=\{\,a\in{G}\;|\;a*b=b*a\textrm{ for all }b\in{G}\,\}
            \end{equation*}
        \end{fdefinition}
        Note that the center of a group is non-empty since the unital element
        of a group commutes with everything. Moreover, the center of a group is
        a subgroup.
        \begin{theorem}
            If $(G,*)$ is a group and if $Z(G,*)$ is the center of $(G,*)$, then
            $Z(G,*)$ is a subgroup of $(G,*)$.
        \end{theorem}
        \begin{theorem}
            If $(G,*)$ is a finite group, and if $S\subseteq{G}$ is closed under
            $*$, then $S$ is a subgroup of $G$. That is, $S$ is closed under
            inverses as well.
        \end{theorem}
        The set all \textit{periods} of a function $f:G\rightarrow{G}$ from a
        group $(G,*)$ to itself forms a subgroup of $G$. That is, the set of
        all elements $g\in{G}$ such that, for all $x\in{G}$ it is true that
        $f(g*x)=f(x)$.
        \begin{theorem}
            If $(G,*)$ is a group, if $f:G\rightarrow{G}$, and if
            $H\subseteq{G}$ is defined by:
            \begin{equation}
                H=\{\,g\in{G}\;|\;\textrm{For all }x\in{G},f(g*x)=f(g)\,\}
            \end{equation}
            then $H$ is a subgroup of $(G,*)$.
        \end{theorem}
        \begin{theorem}
            If $(G,*)$ is a group, if $(G',\circ)$ is a group, if $e\in{G}$ is
            the unital element, and if $H$ is defined by:
            \begin{equation}
                H=\{(e,g')\in{G}\times{G}'\;|\;g'\in{G}'\}
            \end{equation}
            Then $H$ is a subgroup of the direct product $G\times{G}'$.
        \end{theorem}
        \begin{theorem}
            If $(G,*)$ is a group, if $(G',\circ)$ is a group, if $e'\in{G}'$ is
            the unital element, and if $H$ is defined by:
            \begin{equation}
                H=\{(g,e')\in{G}\times{G}'\;|\;g\in{G}\}
            \end{equation}
            Then $H$ is a subgroup of the direct product $G\times{G}'$.
        \end{theorem}
        \begin{theorem}
            If $(G,*)$ is a group, if $\mathcal{P}(G)$ is the power set of $G$,
            if $\mathcal{G}\subseteq\mathcal{P}(G)$ is such that for all
            $H\in\mathcal{G}$ it is true that $H$ is a subgroup of $(G,*)$, then
            the set $N$ defined by:
            \begin{equation}
                N=\bigcap_{H\in\mathcal{G}}H
            \end{equation}
            is a subgroup of $(G,*)$.
        \end{theorem}
        \begin{proof}
            For $N$ is not empty since for all $H\in\mathcal{G}$, $e\in{H}$
            (Thm.~\ref{thm:Subgroup_Contains_Identity}). Suppose there exists
            $a,b\in{N}$ such that $a*b\notin{N}$. But if $a,b\in{N}$, then for
            all $H\in\mathcal{G}$ it is true that $a,b\in{H}$
            (Def.~\ref{def:Intersection_Over_a_Collection}). But since $H$ is a
            subgroup of $(G,*)$, $a*b\in{H}$. But if $a*b\in{H}$ for all
            $H\in\mathcal{G}$, then $a*b\in{N}$, a contradiction. Thus $N$ is
            closed under $*$. Suppose there is an $a\in{N}$ such that
            $a^{\minus{1}}\notin{N}$. But then $a\in{H}$ for all
            $H\in\mathcal{G}$ and thus $a^{\minus{1}}\in{H}$ for all $H$, since
            $H$ is a subgroup (Def.~\ref{def:Subgroup}). But then
            $a^{\minus{1}}\in{a}$
            (Def.~\ref{def:Intersection_Over_a_Collection}), a contradiction.
            Therefore, $N$ is closed to inverses. Thus, $N$ is a subgroup.
        \end{proof}
        This theorem allows us to define the subgroup of a group $(G,*)$
        \textit{generated} from some set $S\subseteq{G}$.
        \begin{theorem}
            If $(G,*)$ is a group, and if $S\subseteq{G}$, then there exists a
            set $\mathcal{G}$ such that $\mathcal{G}$ is non-empty and such that
            for all $H\in\mathcal{G}$ it is true that $S\subseteq{H}$ and $H$ is
            a subgroup of $(G,*)$.
        \end{theorem}
        \begin{proof}
            Apply the axiom schema of specification
            (Ax.~\ref{ax:Axiom_Schema_of_Specification}) to the proposition
            $P$ where $P(H)$ is true if and only if $S\subseteq{H}$ and $H$ is
            a subgroup of $G$ to the power set $\mathcal{P}(G)$. That is:
            \begin{equation}
                \mathcal{G}=\big\{\,H\in\mathcal{P}(G)\;|\;P(H)\,\big\}
            \end{equation}
            This completes the proof.
        \end{proof}
        This theorems means the following is well-defined.
        \begin{fdefinition}{Generated Subgroup}{Generated_Subgroup}
            The subgroup of a group $(G,*)$ generated by a subset
            $S\subseteq{G}$ is the set:
            \begin{equation*}
                \langle{S}\rangle=\bigcap_{H\in\mathcal{G}}H
            \end{equation*}
            where $\mathcal{G}$ is the collection of all subgroups of $(G,*)$
            that contain $S$.\index{Group!Generated Subgroup}
        \end{fdefinition}
        \begin{example}
            Consider $\mathbb{Z}_{6}$ with usual modular addition. The entire
            group can be considered as the group generated by the element 1.
            That is, $2=1+1$, $3=1+1+1$, and so on, up until $0=1+1+1+1+1+1$.
            Thus we can write $(\mathbb{Z}_{6},+)=\langle{1}\rangle$. Such
            groups are called finitely generated.
        \end{example}
        \begin{example}
            The Dihedral group $D_{6}$ discussed before can be generated from
            two elements $r,f$ with the requirements that $r^{3}=f^{2}=e$ and
            that $(rf)^{2}=e$. We can write this as:
            \begin{equation}
                D_{6}=\langle{r},f\;|\;r^{3},f^{2},(rf)^{2}\rangle
            \end{equation}
            This is called a \textit{presentation}\index{Group!Presentation} of
            $D_{6}$. Similarly, the dihedral group on $4$ points, $D_{8}$, has
            the presentation:
            \begin{equation}
                D_{8}=\langle{r},f\;|\;r^{4},f^{2},(rf)^{2}\rangle
            \end{equation}
            The quaternion group $Q$, which is another non-Abelian group with 8
            elements, has the presentation:
            \begin{equation}
                Q=\langle\minus{1},i,j,k\;|\;(\minus{1}^{2})=1,
                    i^{2}=j^{2}=k^{2}=ijk=\minus{1}\rangle
            \end{equation}
        \end{example}
    \subsection{Cayley Diagrams}
        Cayley diagrams can be used to represent finite groups and a formed from
        a directed graph. The points on the graph correspond to the elements of
        the group $(G,*)$, and the director arrows correspond to multiplying by
        generators of the group.
        \begin{lexample}{Cayley Diagram of the Dihedral Group $D_{6}$}
                        {Cayle_Diagram_of_D6}
            Consider the Dihedral group $D_{6}$, defined as the group on six
            points generated by $r,f\in{G}$ such that $r^{3}=f^{2}=(rf)^{2}=e$.
            That is, the group of rotational and reflectional symmetries on a
            triangle. It is thus not surprising that the Cayley diagram of
            $D_{6}$ consists of triangle. The 6 points are:
            \begin{equation}
                G=\{\,e,\,r,\,f,\,r^{2},\,rf,\,r^{2}f\,\}
            \end{equation}
            We'll draw a dotted arrow from $g$ to $h$ meaning that $g*f=h$, and
            a solid arrow for $g*r=h$.
            \begin{figure}[H]
                \centering
                \captionsetup{type=figure}
                %--------------------------------Dependencies----------------------------------%
%   tikz                                                                       %
%       arrows.meta                                                            %
%       decorations.markings                                                   %
%-------------------------------Main Document----------------------------------%
\begin{tikzpicture}[%
    ->-/.style={%
        decoration={%
            markings,
            mark=at position .55 with \arrow{Stealth}
        },
        postaction={decorate}
    }
]
    % The coordinates for the outter triangle.
    \coordinate (e)   at (210.0:3.0);
    \coordinate (r)   at (330.0:3.0);
    \coordinate (r2)  at (90.00:3.0);

    % The coordinates for the inner triangle.
    \coordinate (f)   at (210.0:1.5);
    \coordinate (fr2) at (330.0:1.5);
    \coordinate (fr)  at (90.00:1.5);

    % Dots for the coordinates.
    \foreach\x in {e,r,r2,f,fr,fr2}{%
        \draw[fill=black] (\x) circle (0.05);
    }

    % Solid arrows for rotations.
    \draw[->-] (e)   to (r);
    \draw[->-] (r)   to (r2);
    \draw[->-] (r2)  to (e);
    \draw[->-] (f)   to (fr);
    \draw[->-] (fr)  to (fr2);
    \draw[->-] (fr2) to (f);

    % Dashed arrows for reflections.
    \draw[densely dashed] (e)  to (f);
    \draw[densely dashed] (r)  to (fr2);
    \draw[densely dashed] (r2) to (fr);

    % Label the nodes.
    \node at (e)   [below left]  {$e$};
    \node at (r)   [below right] {$r$};
    \node at (r2)  [above]       {$r^{2}$};
    \node at (f)   [above left]  {$f$};
    \node at (fr)  [right]       {$fr$};
    \node at (fr2) [above right] {$fr^{2}$};
\end{tikzpicture}
                \caption{Cayley Diagram of $D_{6}$}
                \label{fig:Cayley_Diagram_D6}
            \end{figure}
            Note that since $f^{2}=e$, multiplying by $f$ or by $f^{\minus{1}}$
            results in the same move, and thus arrows are redundant. Hence in
            the Cayley diagram, the dotted lines which represent reflections
            have no arrows. The Cayley diagram contains all of the information
            about the group. If we wish to compute $(fr)*(fr^{2})$, we need only
            follow the arrows. That is, start at $fr$ and then apply
            associativity to unwrap this expression. Applying $f$ to $fr$, we
            move to $r^{2}$. We than apply $r$ and end up at $e$. Applying $r$
            again, we arrive at $r$, and thus $(fr)*(fr^{2})=r$. We can also
            compute inverses. The inverse of $fr^{2}$ is the path that leads
            back to $e$. We see that $(fr^{2})*(r)*(f)=e$< and thus the inverse
            is $rf=fr^{2}$.
        \end{lexample}
        We can form the Cayley diagram of any finite group, and indeed we can
        form this for infinite groups as well, though the resulting graph is
        infinite as well. In particular, the \textit{free group}%
        \index{Group!Free Group} on 2 generators can be expressed nicely via an
        infinite Cayley graph. Note that a directed graph is a Cayley graph
        for some group with generators $a_{k}$ if and only if for all points
        $g$ on the graph there is an arrow for $a_{k}$ which starts at $g$ and
        an arrow that ends on $g$.
        \begin{lexample}{Cayley Diagram on $\mathbb{Z}_{2}\times\mathbb{Z}_{2}$}
                        {Cayley_Diagram_Z2_X_Z2}
            Consider the Cayley diagram on the group
            $(\mathbb{Z}_{2}\times\mathbb{Z}_{2},+)$, where $+$ is pointwise
            modular addition. That is, the arithmetic that stems from the
            direct product of $\mathbb{Z}_{2}$ with itself. We arrive at
            Fig.~\ref{fig:Cayley_Diagram_Z2_x_Z2}. The dashed lines denote
            $+(0,1)$ and the solid lines denote $+(1,0)$. Since
            $(1,0)+(1,0)=(1+1,0+0)=(0,0)$, and similarly for
            $(0,1)$, there's no need to draw arrows on the diagram.
        \end{lexample}
        \begin{figure}[H]
            \centering
            \captionsetup{type=figure}
            \begin{tikzpicture}
    % Coordinates for the points.
    \coordinate (OO) at (0, 0);
    \coordinate (OI) at (0, 3);
    \coordinate (IO) at (3, 0);
    \coordinate (II) at (3, 3);

    % Dashed lines for +(0,1)
    \draw[densely dashed] (OO) to (OI);
    \draw[densely dashed] (IO) to (II);

    % Solid lines for +(1,0)
    \draw (OO) to (IO);
    \draw (OI) to (II);

    % Dots for the points.
    \draw[fill=black] (OO) circle (0.05);
    \draw[fill=black] (OI) circle (0.05);
    \draw[fill=black] (IO) circle (0.05);
    \draw[fill=black] (II) circle (0.05);

    % Labels for the points.
    \node at (OO) [below left]  {$(0,0)$};
    \node at (OI) [above left]  {$(0,1)$};
    \node at (IO) [below right] {$(1,0)$};
    \node at (II) [above right] {$(1,1)$};
\end{tikzpicture}
            \caption{Cayley Diagram of $\mathbb{Z}_{2}\times\mathbb{Z}_{2}$}
            \label{fig:Cayley_Diagram_Z2_x_Z2}
        \end{figure}
        \begin{lexample}{Cayley Diagram of $S_{3}$}{Cayley_Diagram_S3}
            Let $n\in\mathbb{N}$ and $G=\mathbb{Z}_{n}$. Consider the set of all
            \textit{permutations}\index{Permutation} on $\mathbb{Z}_{n}$. That
            is, the set of all bijective functions from $\mathbb{Z}_{n}$ to
            itself. This is a group under function composition and is called the
            symmetric group $S_{n}$. In particular, $S_{n}$ has $n!$ points.
            Now consider $S_{3}$. We thus have that $S_{3}$ has $3!=6$ points,
            and it can be seen that it is non-Abelian. As it turns out, $S_{3}$
            is equivalent to $D_{6}$, and indeed there are only two unique
            groups on $6$ points ($\mathbb{Z}_{6}$ and $D_{6}$), this result
            will be proven later. But for now we can compute the Cayley diagram
            of $S_{3}$. $S_{3}$ has the following presentation:
            \begin{equation}
                S_{3}=\langle{a,b}\;|\;a^{2}=b^{2}=(ab)^{3}=e\rangle
            \end{equation}
            To see that this is the same thing as the dihedral group $D_{6}$,
            let $a=f$ and let $b=fr$. Thus we have a different presentation that
            produces the same group, however this will produce a different
            Cayley diagram.
            \begin{figure}[H]
                \centering
                \captionsetup{type=figure}
                %--------------------------------Dependencies----------------------------------%
%   tikz                                                                       %
%       arrows.meta                                                            %
%       decorations.markings                                                   %
%-------------------------------Main Document----------------------------------%
\begin{tikzpicture}[%
    ->-/.style={%
        decoration={%
            markings,
            mark=at position .55 with \arrow{Stealth}
        },
        postaction={decorate}
    }
]
    % Coordinates for the points.
    \foreach\x in {0,60,120,180,240,300}{%
        \coordinate (\x) at (\x:2);

        % Add dots for each point.
        \draw[fill=black] (\x) circle (0.05);

        % Label the points by their angle on the unit circle.
        \node at (\x:2.4) {$\x$};
    }

    % Dashed lines for acting by a.
    \draw[densely dashed] (0)   to (60);
    \draw[densely dashed] (120) to (180);
    \draw[densely dashed] (240) to (300);

    % Solid lines for acting by b.
    \draw[->-] (60)  to (120);
    \draw[->-] (180) to (240);
    \draw[->-] (300) to (0);
\end{tikzpicture}
                \caption{Cayley Diagram for $S_{3}$}
                \label{fig:Cayley_Diagram_S3}
            \end{figure}
            While this Cayley diagram certainly looks different from the one
            presented in Ex.~\ref{ex:Cayle_Diagram_of_D6}, they both represent
            the same group $D_{6}$.
        \end{lexample}
        \begin{lexample}{The Dihedral Group $D_{8}$}{Dihedral_Group_D8}
            The dihedral group on $4$ points, $D_{8}$, is the group of
            symmetries on the square. If we use as generators rotation and
            reflection, with the constraints $r^{4}=f^{2}=e$, and that
            $(rf)^{2}=e$, we arrive at the Cayley Diagram shown in
            Fig.~\subref{fig:Std_Cayley_Diagram_D8}. Rather than representing
            the symmetries of a square by rotation and reflection, we can
            represent it by two different reflections. That is, let $d$ denote
            diagonal reflection, and $h$ represent horizontal reflection. We
            have the following presentation:
            \begin{equation}
                h^{2}=d^{2}=(hd)^{4}=e
            \end{equation}
            While this is equivalent to $D_{8}$, it presents a different, but
            equivalent, Cayley diagram
            (Fig.~\subref{fig:Alt_Cayley_Diagram_D8}).
        \end{lexample}
        \begin{figure}[H]
            \centering
            \captionsetup{type=figure}
            \begin{subfigure}[b]{0.49\textwidth}
                \resizebox{\textwidth}{!}{%
                    %--------------------------------Dependencies----------------------------------%
%   tikz                                                                       %
%       arrows.meta                                                            %
%       decorations.markings                                                   %
%-------------------------------Main Document----------------------------------%
\begin{tikzpicture}[%
    ->-/.style={%
        decoration={%
            markings,
            mark=at position .55 with \arrow{Stealth}
        },
        postaction={decorate}
    }
]
    % Coordinates for the outter square.
    \coordinate (e)  at (225.0:3.0);
    \coordinate (r)  at (315.0:3.0);
    \coordinate (r2) at (45.00:3.0);
    \coordinate (r3) at (135.0:3.0);

    % Coordinates for inner square.
    \coordinate (fr2) at (45.00:1.5);
    \coordinate (fr)  at (135.0:1.5);
    \coordinate (f)   at (225.0:1.5);
    \coordinate (fr3) at (315.0:1.5);

    % Dots for the coordinates.
    \foreach\x in {e,r,r2,r3,f,fr,fr2,fr3}{%
        \draw[fill=black] (\x) circle (0.05);
    }

    % Solid arrows for rotations.
    \draw[->-] (e)   to (r);
    \draw[->-] (r)   to (r2);
    \draw[->-] (r2)  to (r3);
    \draw[->-] (r3)  to (e);
    \draw[->-] (f)   to (fr);
    \draw[->-] (fr)  to (fr2);
    \draw[->-] (fr2) to (fr3);
    \draw[->-] (fr3) to (f);

    % Dashed arrows for reflections.
    \draw[densely dashed] (e) to (f);
    \draw[densely dashed] (r) to (fr3);
    \draw[densely dashed] (r2) to (fr2);
    \draw[densely dashed] (r3) to (fr);

    % Label the points.
    \node at (e)   [below left]  {$e$};
    \node at (r)   [below right] {$r$};
    \node at (r2)  [above right] {$r^{2}$};
    \node at (r3)  [above left]  {$r^{3}$};
    \node at (f)   [above right] {$f$};
    \node at (fr3) [above left]  {$fr^{3}$};
    \node at (fr2) [below left]  {$fr^{2}$};
    \node at (fr)  [below right] {$fr$};
\end{tikzpicture}}
                \subcaption{Standard Diagram for $D_{8}$}
                \label{fig:Std_Cayley_Diagram_D8}
            \end{subfigure}
            \begin{subfigure}[b]{0.49\textwidth}
                \resizebox{\textwidth}{!}{%
                    \begin{tikzpicture}
    % Coordinates for the points.
    \foreach\x in {0,45,90,135,180,225,270,315}{%
        \coordinate (\x) at (\x:2);
        \draw[fill=black] (\x) circle (0.05);
        \node at (\x:2.3) {$\x$};
    }

    % Dashed lines for acting by d.
    \draw[densely dashed] (0)   to (45);
    \draw[densely dashed] (90)  to (135);
    \draw[densely dashed] (180) to (225);
    \draw[densely dashed] (270) to (315);

    % Dashed lines for acting by d.
    \draw(45)  to (90);
    \draw(135) to (180);
    \draw(225) to (270);
    \draw(315) to (0);
\end{tikzpicture}}
                \subcaption{Alternate Diagram for $D_{8}$}
                \label{fig:Alt_Cayley_Diagram_D8}
            \end{subfigure}
            \caption{Cayley Diagrams for $D_{8}$}
            \label{fig:Cayley_Diagrams_D8}
        \end{figure}
        \subsection{Comments from Meeting with David}
            Abelianization (UMP, quotient out commutator)
            $D_{\infty}$, $D_{2n}=\mathbb{Z}_{n}\rtimes\mathbb{Z}_{2}$. Chinese
            remainder theorem for proving Euler totient is multiplicative.
            $D_{\infty}=\mathbb{Z}\rtimes\mathbb{Z}_{2}$. We can write:
            \begin{equation}
                D_{\infty}=\langle{r,s}\;|\;
                    s^{2}=e,rsr^{\minus{1}}=s^{\minus{1}}\rangle
                =\langle{a,b}\;|\;a^{2}=e=b^{2}\rangle
            \end{equation}
            Set $b=s$ and $a=rs$.
        \section{Group Morphisms}
    \subsection{Homomorphisms}
        \begin{fdefinition}{Group Homomorphism}{Group_Homomorphism}
            A \gls{group homomorphism} from a \gls{group} $(G,*)$ to a group
            $(G',\circ)$ is a \gls{bijective function}
            $\varphi:G\rightarrow{G}'$ such that for all $a,b\in{G}$ it is true
            that:
            \begin{equation*}
                \varphi(a*b)=\varphi(a)\circ\varphi(b)
            \end{equation*}
        \end{fdefinition}
        \begin{fdefinition}{Group Isomorphism}{Group_Isomorphism}
            A \gls{group isomorphism} from a \gls{group} $(G,*)$ to a group
            $(G',\circ)$ is a \glslink{bijective function}{bijective}
            \gls{group homomorphism} $\varphi:G\rightarrow{G}'$.
        \end{fdefinition}
        \begin{theorem}
            If $(G,*)$ and $(G',\circ)$ are isomorphic with identities $e_{*}$
            and $e_{\circ}$ are the identities, then $f(e_{*})=e_{\circ}$.
        \end{theorem}
        \begin{proof}
            $\forall a\in G,\ f(a)=f(a* e_*) = f(a)\circ f(e_*)$ as $f$ is
            an isomorphism. As identities are unique, $f(e_*)=e_{\circ}$.
        \end{proof}
        \begin{theorem}
            If $(G,*)$ and $(G',\circ)$ are isomorphic, with isomorphism $f$,
            and if $a\in{G}$, then $f(a^{\minus{1}})=f(a)^{\minus{1}}$.
        \end{theorem}
        \begin{proof}
            For:
            \begin{equation}
                e_{\circ}=f(e_*)
                        =f(a*a^{-1})
                        =f(a^{-1}*a)
                        =f(a)\circ f(a^{-1})
                        =f(a^{-1})\circ f(a)
            \end{equation}
            As inverses are unique, $f(a^{-1})=f(a)^{-1}$.
        \end{proof}
        \begin{definition}
            A permutation group on $n$ elements is a group whose elements are
            permutations of $n$ elements.
        \end{definition}
        \begin{definition}
            The symmetric group on $n$ elements, denoted $S_{n}$, is the group
            formed by permuting $n$ elements.
        \end{definition}
        \begin{definition}
            A homomorphism from a group $(G,*)$ to a group $(H,\circ)$ is a
            function $h:{G}\rightarrow{H}$ such that for all
            ${a,b}\in{G}$, $h(a*b)={h(a)}\circ{h(b)}$
        \end{definition}
        \begin{definition}
            An epimorphism from a group $(G,*)$ to a group $(H,\circ)$ is a
            homomorphism $h:{G}\rightarrow{H}$ such that $h$ is surjective.
        \end{definition}
        \begin{definition}
            A monomorphism from a group $(G,*)$ to a group $(H,\circ)$ is a
            homomorphism $h:{G}\rightarrow{H}$ such that $h$ is injective.
        \end{definition}
    \chapter{Finite Groups}
        Finite groups are of fundamental interest not only to mathematicians,
        but throughout many of the other sciences. Indeed, chemists and
        physicists make regular use of the theory of finite groups, and its
        application can be found in general relativity, quantum mechanics, and
        studying the lattice structure of organic molecules. A finite group is
        exactly what it sounds like: A group $(G,*)$ where $G$ is a finite set.
        \begin{fdefinition}{Finite Group}{Finite_Group}
            A finite group is a \gls{group} $(G,*)$ such that $G$ is a
            finite set.
        \end{fdefinition}
        One of the fundamental problems of group theory is a combinatorial one.
        Given an integer $n\in\mathbb{N}$, how many groups with $n$ elements are
        there (up to isomorphism)? This challenging problem can be aided by the
        theorems of Cayley, Cauchy, Lagrange, and Sylow, and it is our aim to
        develop this theory.
        \section{Permutation Groups}
    Recall that a permutation on a set $A$ is a bijective function 
    $f:A\rightarrow{A}$. That is, $f$ is a rearranging of $A$. Under the
    operation of function composition, given a non-empty set $A$, the set of all
    permutation on $A$ together with function composition $\circ$ have a
    group structure.
    \begin{theorem}
        \label{thm:Symmetric_Group_is_a_Group}%
        If $A$ is a non-empty set, if $S_{A}$ is the set of all permutations of
        $A$, and if $\circ$ denotes function composition, then
        $S_{A},\circ)$ is a group.
    \end{theorem}
    \begin{proof}
        For $\circ$ is indeed a binary operation on $S_{A}$. There also
        exists a unital element, since $\textrm{Id}_{A}$ is a permutation on
        $A$. Lastly, if $f\in{S}_{A}$, then it is a bijection and thus there
        exists an inverse function $g:A\rightarrow{A}$. But the inverse of a
        permutation is a permutation, and thus $g\in{S}_{A}$. Thus,
        $(S_{A},\circ)$ is closed to inverses and is therefore a group
        (Def.~\ref{def:Group}).
    \end{proof}
    We will be most interested in case when $A=\mathbb{Z}_{n}$ for some
    $n\in\mathbb{N}$. The set of all permutations on a set $A$ is called the
    \textit{symmetric group}\index{Symmetric Group} of $A$.
    \begin{fdefinition}{Symmetric Group}{Symmetric_Group}
        The symmetric group of a set $A$ is the group $(S_{A}\circ)$
        of all permutations of $A$ under function composition $\circ$.
    \end{fdefinition}
    By Thm.~\ref{thm:Symmetric_Group_is_a_Group}, the symmetric group is a
    group. The reason we required the underlying set to be non-empty is because
    the set of perumtations of the empty set is empty, and thus $S_{\emptyset}$
    cannot be a group since groups are required to have a unital element, and
    thus cannot be empty.
    \begin{lexample}{Symmetric Group $S_{3}$}{Symmetric_Group_S3}
        We've seen the symmetric group for $\mathbb{Z}_{3}$ before and noted
        that it is isomorphic to $D_{6}$. Given a permutation $f\in{S}_{3}$, we
        can describe $f$ via the following matrix:
        \begin{equation}
            f=
            \begin{pmatrix}
                0&1&2\\
                1&0&2
            \end{pmatrix}
        \end{equation}
        The first row of the matrix is the input, and the second row is the
        output. This matrix tells us that $f$ can be defined as follows:
        \begin{equation}
            f(n)=
            \begin{cases}
                1,&n=0\\
                0,&n=1\\
                2,&n=2
            \end{cases}
        \end{equation}
        We have that there are $3!=6$ permutations on $\mathbb{Z}_{3}$, and we
        can list them as follows:
        \par
        \begin{subequations}
            \begin{minipage}[b]{0.49\textwidth}
                \centering
                \begin{align}
                    \textrm{Id}_{\mathbb{Z}_{3}}&=
                    \begin{pmatrix}
                        0&1&2\\
                        0&1&2
                    \end{pmatrix}\\
                    \alpha&=
                    \begin{pmatrix}
                        0&1&2\\
                        0&2&1
                    \end{pmatrix}\\
                    \beta&=
                    \begin{pmatrix}
                        0&1&2\\
                        2&0&1
                    \end{pmatrix}
                \end{align}
            \end{minipage}
            \hfill
            \begin{minipage}[b]{0.49\textwidth}
                \centering
                \begin{align}
                    \gamma&=
                    \begin{pmatrix}
                        0&1&2\\
                        1&0&2
                    \end{pmatrix}\\
                    \delta&=
                    \begin{pmatrix}
                        0&1&2\\
                        1&2&0
                    \end{pmatrix}\\
                    \epsilon&=
                    \begin{pmatrix}
                        0&1&2\\
                        2&1&0
                    \end{pmatrix}
                \end{align}
            \end{minipage}
        \end{subequations}
        \par\vspace{2.5ex}
        We can use this to compute compositions of permutations.
        \begin{equation}
            \beta\circ\delta=
            \begin{pmatrix}
                0&1&2\\
                2&0&1
            \end{pmatrix}
            \begin{pmatrix}
                0&1&2\\
                1&2&0
            \end{pmatrix}=
            \begin{pmatrix}
                0&1&2\\
                0&1&2
            \end{pmatrix}
            =\textrm{Id}_{\mathbb{Z}_{3}}
        \end{equation}
        This should be read as \textit{0 goes to 1 and 1 goes to 0},
        \textit{so 0 goes to 0}. That is, we read the $\delta$ matrix first,
        and then feed this result to the $\beta$ matrix. Similarly,
        1 goes to 2 and 2 goes to 1, so 1 goes to 1. Lastly, 2 goes to 0 and 0
        goes to 2, so 2 goes to 0. The resulting permutation is the identity
        permutation. Note that we are \textbf{not} performing matrix
        multiplication. On the one hand, we've yet to define matrix
        multiplication at this point, and on the other matrix multiplication is
        \textit{undefined} for matrices of these sizes. That is, we cannot
        multiply a $2\times{3}$ matrix by a $2\times{3}$ matrix in the usual
        fashion. This representation is simply to aid in ones understanding of
        groups of permutations. The symmetric group is non-Abelian, as is
        $D_{6}$ We can see that it is non-Abelian by considering
        $\alpha\circ\beta$ and $\beta\circ\alpha$. In $\alpha\circ\beta$ we have
        that 0 goes to 2 and 2 goes to 1, so 0 goes to 1. However in
        $\beta\circ\alpha$ we see that 0 goes to 0, and then 0 goes to 2, so 0
        goes to 2. Thus $\alpha\circ\beta\ne\beta\circ\alpha$, so
        $(S_{3},\circ)$ is not Abelian. We can form a new representation of
        $S_{3}$ to show that it is isomorphic to $D_{6}$. In fact, there are
        only two groups with 6 elements (up to isomorphism).
    \end{lexample}
    We do not have to consider \textit{all} permutations on a given group, and
    the more general \textit{group of permutations} is formed by considering
    subgroups of a symmetric.
    \begin{fdefinition}{Permutation Group}{Permutation_Group}
        A permutation group of a set $A$ is a subgroup of the symmetric group
        $(S_{A},\circ)$ on $A$.
    \end{fdefinition}
    What's remarkable is that \textit{every} group is a permutation group for
    some set $A$. This result is known as Cayley's
    Theorem\index{Cayley's Theorem} and will be proved shortly.
    \subsection{Finite Permutations}
        When dealing with permutations on a finite set, it is convenient to
        break up a given permutation into disjoint \textit{cycles}. For example,
        suppose we have the following permutation on $\mathbb{Z}_{8}$:
        \begin{equation}
            f=
            \begin{pmatrix}
                0&1&2&3&4&5&6&7\\
                2&5&6&4&0&7&3&1
            \end{pmatrix}
        \end{equation}
        We can draw this as two disjoint cycles as follows:
        \begin{figure}[H]
            \centering
            \captionsetup{type=figure}
            \begin{tikzpicture}[%
                ->-/.style={%
                    decoration={%
                        markings,
                        mark=at position .55 with \arrow{Stealth}
                    },
                    postaction={decorate}
                }
            ]
                \coordinate (0) at (0.000:2);
                \coordinate (2) at (72.00:2);
                \coordinate (6) at (144.0:2);
                \coordinate (3) at (216.0:2);
                \coordinate (4) at (288.0:2);

                \draw[fill=black] (0) circle (0.05);
                \draw[fill=black] (2) circle (0.05);
                \draw[fill=black] (6) circle (0.05);
                \draw[fill=black] (3) circle (0.05);
                \draw[fill=black] (4) circle (0.05);

                \node at (0.000:2.3) {$0$};
                \node at (72.00:2.3) {$2$};
                \node at (144.0:2.3) {$6$};
                \node at (216.0:2.3) {$3$};
                \node at (288.0:2.3) {$4$};

                % Draw the first cycle (its a pentagon)
                \draw[->-] (0) to (2);
                \draw[->-] (2) to (6);
                \draw[->-] (6) to (3);
                \draw[->-] (3) to (4);
                \draw[->-] (4) to (0);

                \begin{scope}[xshift=6cm]
                    \coordinate (1) at (90.00:2);
                    \coordinate (5) at (210.0:2);
                    \coordinate (7) at (330.0:2);

                    \draw[fill=black] (1) circle (0.05);
                    \draw[fill=black] (5) circle (0.05);
                    \draw[fill=black] (7) circle (0.05);

                    \node at (90.00:2.3) {$1$};
                    \node at (210.0:2.3) {$5$};
                    \node at (330.0:2.3) {$7$};

                    \draw[->-] (1) to (5);
                    \draw[->-] (5) to (7);
                    \draw[->-] (7) to (1);
                \end{scope}
            \end{tikzpicture}
            \caption{Cycle Diagram for a Permutation}
            \label{fig:Cycle_Diagram_of_Permutation}
        \end{figure}
        We can write this simply as the product of two cycle of the permutation:
        \begin{equation}
            f=(02634)(157)
        \end{equation}
        First we need to rigorously define a cycle.
        \begin{fdefinition}{Cycle Permutation}{Cycle_Permutation}
            A cycle permutation on a finite set $A$ is a permutation
            $f\in{S}_{A}$ such that there exists two disjoint subsets
            $M,N\subseteq{A}$ such that $M\cup{N}=A$,
            $f|_{N}=\textrm{Id}_{A}|_{N}$, and such that there 
        \end{fdefinition}
        \begin{theorem}
            Disjoint cycles commute.
        \end{theorem}
        \begin{theorem}
            If $A$ is a set and if $f$ is a permutation on $A$, then $f$ is
            the product of finitely many disjoint cycles. The product is unique.
        \end{theorem}
        \begin{fdefinition}{Transposition}{Transposition}
            A transposition is a cycle of length 2.
        \end{fdefinition}
        \begin{theorem}
            Every cycle is the product of transpositions.
        \end{theorem}
        \begin{theorem}
            If $\alpha$ is a cycle of length $s$, and if $\alpha^{2}$ is a
            cycle, then $s$ is odd.
        \end{theorem}
        \begin{theorem}
            If $\alpha$ is a permutation, then $\alpha^{2}$ is an even
            permutation.
        \end{theorem}
        The decomposition of a cycle into transpositions need not be unique, and
        even the number of transpositions in such a decomposition can vary.
        \begin{theorem}
            The identity is always the product of an even number of
            transpositions.
        \end{theorem}
        \begin{theorem}
            The number of transpositions in the decomposition of a permutation
            is either always odd or always even.
        \end{theorem}
        This always us to define the alternating group.
        \begin{fdefinition}{Alternating Group}{Alternating_Group}
            The alternating group on a set $B$ is the subgroup of $S_{B}$
            consisting of all even permutations. It is denoted $A_{B}$.
        \end{fdefinition}

        \section{Dihedral Groups}
    Pretty pictures, presentation, etc. If $k\in\mathbb{Z}_{k}$ then
    $fr^{k}=r^{\minus{k}}f$. Every element has unique representation
    $f^{j}r^{k}$ with $j=0$ or $j=1$ and $k\in\mathbb{Z}_{n}$.
    \begin{fdefinition}{Generator of a Group}{Generator_of_a_Group}
        A generator of a group $\monoid{G}$ is a subset $S\subseteq{G}$
        such that for all $g\in{G}$ there exists an $n\in\mathbb{N}$ and
        a sequence $a:\mathbb{Z}_{n}\rightarrow{S}$ such that:
        \begin{equation*}
            g=\prod_{k\in\mathbb{Z}_{n}}a_{k}
        \end{equation*}
    \end{fdefinition}
    Note that sequences allow for repetition and have a notion of order,
    and so this respect the potential non-commutativity of a group.
    Another way of interpreting this definition is that every element of
    $G$ can be written as the finite product of elements in $S$.
    \begin{example}
        For any dihedral group $(D_{2n},\circ)$, the set
        $S=\{r,f\}$ containing the rotation and the reflection element
        is a generator for $D_{2n}$. To tell two such dihedral groups
        apart we introduce \textit{relations}. For example, $r^{n}=e$
        and $f^{2}=e$, and these are the least such positive integers
        with these properties. Moreover, $rf=fr^{\minus{1}}$.
    \end{example}
    \begin{example}
        Consider $G$ with the presentation:
        \begin{equation}
            G=\langle{a},b\;|\;a^{n}=e,b^{2}=e,ab=ba^{2}\rangle
        \end{equation}
        Let's see if we can determine what this group is. We have:
        \par\vspace{-2.5ex}
        \begin{subequations}
            \begin{minipage}[t]{0.49\textwidth}
                \begin{align}
                    a&=ae\\
                    &=ab^{2}\\
                    &=(ab)b\\
                    &=(ba^{2})b\\
                    &=(ba)(ab)
                \end{align}
            \end{minipage}
            \hfill
            \begin{minipage}[t]{0.49\textwidth}
                \begin{align}
                    &=(ba)(ba^{2})\\
                    &=(bab)a^{2}\\
                    &=b(ba^{2})a^{2}\\
                    &=b^{2}a^{4}\\
                    &=a^{4}
                \end{align}
            \end{minipage}
        \end{subequations}
        \par\vspace{2.5ex}
        and hence by the cancellation law we conclude that $a^{3}=e$.
        Hence, we may take $n$ to be either 1, 2, or 3. If $n=1$ we are
        left with the group presented by a single variable $b$ such that
        $b^{2}=e$, and this is just $\mathbb{Z}/2\mathbb{Z}$. If $n=2$
        then from $ab=ba^{2}$ we conclude $ab=b$, and hence $a=e$, again
        leading us to $\mathbb{Z}/2\mathbb{Z}$. Finally, with $n=3$ we
        have $ab=ba^{2}$ implying that $ab=ba^{\minus{1}}$, and this is
        precisely the presentation for the dihedral group $D_{6}$.
    \end{example}
    \begin{example}
        Consider $G$ defined by:
        \begin{equation}
            G=\langle{a,b}\;|\;a^{4}=e,b^{3}=e,ab=b^{2}a^{2}\rangle
        \end{equation}
        Let's show that $G$ is just the trivial group. We have:
        \begin{equation}
            bab=b(ab)=b(b^{2}a^{2})=b^{3}a^{2}=ea^{2}=a^{2}
        \end{equation}
        and hence:
        \begin{equation}
            e=a^{4}=(bab)^{2}=(bab)(bab)=bab^{2}ab=
            bab^{2}b^{2}a^{2}=bab^{3}ba^{2}=baba^{2}
        \end{equation}
        and since $b^{3}=e$, we apply the cancellation law to obtain
        $b^{2}=aba^{2}$. But then:
        \begin{equation}
            a=ae=ab^{3}=(ab)b^{2}=b^{2}a^{2}(aba^{2})
            =b^{2}a^{3}ba^{2}=b^{\minus{1}}a^{\minus{1}}ba^{2}
        \end{equation}
        and hence $aba=ba^{2}$, and thus $ab=ba$. So the operation
        commutes. But then:
        \begin{equation}
            ab=b^{2}a^{2}=a^{2}b^{2}
        \end{equation}
        and so applying cancellation we have $ab=e$ and hence
        $a=b^{\minus{1}}$. But $b^{\minus{1}}=b^{2}$ since $b^{3}=e$
        and thus $a=b^{2}$. But then:
        \begin{equation}
            a^{3}=(b^{2})^{3}=(b^{3})^{2}=e
        \end{equation}
        But $a^{4}=e$ and therefore by the cancellation law $a=e$.
        And since $b=a^{\minus{1}}$ we have that $b=e$. Hence, $G$ is
        the trivial group.
    \end{example}
    The presentation of the general dihedral group $D_{2n}$ is:
    \begin{equation}
        D_{2n}=\langle{r,f}\;|\;r^{n}=e,f^{2}=e,rf=fr^{\minus{1}}\rangle
    \end{equation}
    \begin{theorem}
        If $D_{2n}$ is the dihedral group with rotational generator $r$
        and reflectional generator $f$, and if $x\in{D}_{2n}$ is not a
        power of $r$, then $rx=xr^{\minus{1}}$.
    \end{theorem}
    \begin{proof}
        By induction. Suppose $a:\mathbb{Z}_{k}\rightarrow\{r,f\}$ is
        a least sequence such that the product is $x$. In the base case
        of $k=1$, since $x$ is not a power of $r$ we simply have that
        $x=f$. But $rf=fr^{\minus{1}}$, so we are done. Suppose it is
        true of $k\in\mathbb{N}$ and let
        $a:\mathbb{Z}_{k+1}\rightarrow\{r,f\}$ be a sequence whose
        product equals $x$. Then either $a_{k}=r$ or $a_{k}=f$. Suppose
        $a_{k}=r$. Since $x$ is not a power of $r$, there is and
        $i\in\mathbb{Z}_{k}$ such that $a_{i}\ne{r}$. Define $y$ by:
        \begin{equation}
            y=\prod_{j\in\mathbb{Z}_{n}}a_{j}
        \end{equation}
        Then since $y$ is not a power of $r$, $ry=yr^{\minus{1}}$. But
        then:
        \begin{equation}
            rx=r(yr)=(ry)r=(yr^{\minus{1}})r=y(r^{\minus{1}}r)=ye=y
        \end{equation}
        but $x=yr$, and hence $y=xr^{\minus{1}}$. Thus
        $rx=x^{\minus{1}}$. If $a_{k}=f$, let $y$ be defined similarly.
        If $y=r^{k}$, then:
        \begin{equation}
            rx=ryf=rr^{k}f=r^{k}rf=r^{k}fr^{\minus{1}}=xr^{\minus{1}}
        \end{equation}
        if $y$ is not a power of $r$, then by the induction hypothesis
        we have that $ry=yr^{\minus{1}}$. But then:
        \begin{equation}
            rx=r(yr)=(ry)r=(yr^{\minus{1}})r=y(r^{\minus{1}}r)=ye=y
        \end{equation}
        But $x=yr$ and hence $y=xr^{\minus{1}}$, so $rx=xr^{\minus{1}}$,
        as claimed.
    \end{proof}
    \begin{theorem}
        In $D_{2n}$ every element that is not a power of $r$ has order
        2.
    \end{theorem}
    \begin{proof}
        For every such element can be written $fr^{k}$ and hence
        $x^{2}=(fr^{k})(fr^{k})=(fr^{k})(r^{\minus{k}}f)=f^{2}=e$.
    \end{proof}
    \begin{theorem}
        If $n\in\mathbb{N}^{+}$, if $D_{2n}$ is the dihedral group with
        rotational element $r$, then the order of $r$ is $n$.
    \end{theorem}
    \begin{proof}
        By definition of $D_{2n}$, $r^{n}$ is the identity element.
    \end{proof}
    \begin{example}
        We can also consider the dihedral group on two points. We have
        the presentation:
        \begin{equation}
            D_{4}=\langle{r,f}\;|\;
                r^{2}=e,f^{2}=e,rf=fr^{\minus{1}}\rangle
        \end{equation}
        This is isomorphic to
        $\mathbb{Z}/2\mathbb{Z}\times\mathbb{Z}/2\mathbb{Z}$. To see
        this, note that the presentation allows us to show that $D_{4}$
        is commutative. Since $r^{2}=e$ we know that $r^{\minus{1}}=r$
        and thus the last relation shows that $rf=fr$, so $D_{4}$ is
        commutative. Let
        $\phi:\mathbb{Z}/2\mathbb{Z}\times\mathbb{Z}/2\mathbb{Z}%
         \rightarrow{D}_{4}$ be defined by:
        \begin{equation}
            \phi(x)=
            \begin{cases}
                e,&x=(0,0)\\
                r,&x=(1,0)\\
                f,&x=(0,1)\\
                rf,&x=(1,1)
            \end{cases}
        \end{equation}
        Then $\phi$ is injective. Moreover, it is surjective. For if
        $y\in{D}_{4}$ then since $D_{4}$ is commutative we know that
        $y=r^{j}f^{k}$. But we only care about $j$ and $k$ mod 2, and
        so we have four possibilies, each of which is mapped onto by
        $\phi$. Moreover, it is an isomorphism.
    \end{example}
\section{Polyhedra Groups}
    DF 1.2 Problems 9-13.

    \renewcommand{\PATH}{\OLDPATH}
\endgroup
        \part{Ring Theory}
            \begingroup
    \ifcsname\PATH\endcsname
        \newcommand{\PATH}{books/Algebra/Ring_Theory}
        \newcommand{\OLDPATH}{\PATH}
    \else
        \newcommand{\OLDPATH}{\PATH}
        \renewcommand{\PATH}{books/Algebra/Ring_Theory}
    \fi
    \chapter{Rings}
        We now add more structure by considering a set with two operations.
        Everything we've studied so far (semigroups, quasigroups, monoids,
        groups) has had only one operation associated to it, but in the most
        fundamental forms of arithmetic there are two. The only structure we've
        encountered with two operations so far has been Boolean algebras
        (see Book~\ref{book:Foundations}), but as we will see when we study
        topology, there is essentialy only one type of Boolean algebra and thus
        this study is, in a sense, complete. If we are going to axiomitize some
        algebraic structure it is then wise to avoid recreating Boolean algebras
        and so instead we try to model the arithmetic of the real numbers. The
        most fundamental properties can be stated quite succintly:
        $\monoid[][+]{\mathbb{R}}$ is an Abelian group and
        $\monoid[][\cdot]{\mathbb{R}}$ is a monoid. We cannot just leave it
        there, however, since we've no way of knowing how $+$ and $\cdot$ play
        together. As presented, we have two potentially unrelated binary
        operations and thus cannot procede any further. To complete our
        structure, we add \glslink{distributive operation}{distributivity}.
        \section{Definitions}
    \begin{fdefinition}{Fields}{Fields}
        A field is a commutative ring $(\mathbb{F},+,\cdot\,)$ such that, for
        all $a\in\mathbb{F}$ such that $a$ is not the unital element of
        $(\mathbb{F},+)$, it is true that $a$ is an invertible element of
        $(\mathbb{F},\cdot\,)$.
    \end{fdefinition}
    \begin{fdefinition}{Subfield}{Subfield}
        A subfield of a field $(F,+,\cdot)$ is a set $K\subset F$, such that
        $(K,+,\cdot)$ is a field.
    \end{fdefinition}
    Given an element $a\in\mathbb{F}$, if $b$ is such that
    $a+b=0$ then we write $b=\minus{a}$. Subtraction of two elements
    $a$ and $c$, denoted $a-c$, is defined as $a+(\minus{c})$. The
    structure $(\mathbb{F},+)$ forms an Abelian group. From this we have
    that the identity is unique, as are additive inverses.
    It is common in the definition of a field to require that
    $0\ne{1}$. This is because if $0=1$ then we have $\mathbb{F}=\{0\}$.
    This comes from the following.
    \begin{ltheorem}{Multiplication by Zero}{Multiplication_by_Zero}
        If $(\mathbb{F},\,+,\,\cdot\,)$ is a field, and if $a\in\mathbb{F}$,
        then $a\cdot{0}=0$.
    \end{ltheorem}
    \begin{proof}
        For we have:
        \begin{equation}
            0=a\cdot(0)-a\cdot(0)=a\cdot(0-0)=a\cdot{0}
        \end{equation}
        This simply combines the distributive law with the additive
        property of zero, completing the proof.
    \end{proof}
    \begin{theorem}
        If $(\mathbb{F},\,+,\,\cdot\,)$ is a field, and if $0=1$, then
        $\mathbb{F}=\{0\}$.
    \end{theorem}
    \begin{proof}
        For suppose not, and let $a\in\mathbb{F}$ be such that $a\ne{0}$.
        But then, by Thm.~\ref{thm:Multiplication_by_Zero}:
        \begin{equation}
            a=a\cdot{1}=a\cdot{0}=0
        \end{equation}
        And thus $a=0$, a contradiction. Therefore,
        $\mathbb{F}$ is trivial.
    \end{proof}
    It is thus common to either call such a field a trivial field, or
    to require that $0\ne{1}$.
    \begin{lexample}{Examples of Fields}{Examples_of_Fields}
        There are several fields that should be familiar to the reader.
        If we let $\mathbb{R}$ denote the real numbers and $+$ and $\cdot$
        be the usual notations of addition and multiplication, then
        $(\mathbb{R},\,+,\,\cdot\,)$ is a field. Similarly, letting
        $\mathbb{Q}$ denote the rational numbers and $\mathbb{C}$ denote
        the complex numbers, $(\mathbb{Q},\,+,\,\cdot\,)$ is a field, as
        is $(\mathbb{C},\,+,\,\cdot\,)$. There are finite fields as well.
        Let $\mathbb{F}_{2}=\{0,\,1\}$ and define multiplication and
        addition as follows:
        \par\hfill\par
        \begin{table}[H]
            \centering
            \captionsetup{type=table}
            \parbox{.45\linewidth}{%
                \centering
                \begin{tabular}{c|cc}
                    $+$&0&1\\
                    \hline
                    0&0&1\\
                    1&1&0
                \end{tabular}
            }
            \parbox{.45\linewidth}{%
                \centering
                \begin{tabular}{c|cc}
                    $\cdot$&0&1\\
                    \hline
                    0&0&0\\
                    1&0&1
                \end{tabular}
            }
            \caption{The Arithmetic of $\mathbb{F}_{2}$}
        \end{table}
        $(\mathbb{F}_{2},\,+,\,\cdot)$ forms a field. Finally, if
        $p\in\mathbb{N}$ is prime, and if $+$ and $\cdot$ are addition
        and multiplication mod $p$, respectively, then
        $(\mathbb{Z}_{p},\,+,\,\cdot\,)$ is a field.
    \end{lexample}
    \begin{fdefinition}{Vector Space}{Vector_Space}
        A vector space over a field $(\mathbb{F},\,+,\,\cdot\,)$ is a
        set $V$ and a function
        $\boldsymbol{\cdot}:\mathbb{F}\times{V}\rightarrow{V}$ and
        a binary operation $\boldsymbol{+}$ on $V$, usuall called
        scalar multiplication and vector addition, respectively, 
        such that for all $\mathbf{x},\mathbf{y},\mathbf{z}\in{V}$,
        and all $a,b\in\mathbf{F}$, the following is true:
        \begin{enumerate}
            \item $\mathbf{x}\boldsymbol{+}%
                   (\mathbf{y}\boldsymbol{+}\mathbf{z})=%
                   (\mathbf{x}\boldsymbol{+}\mathbf{y})%
                   \boldsymbol{+}\mathbf{z}$
                  \hfill[Associative of Vector Addition]
            \item $\mathbf{x}\boldsymbol{+}\mathbf{y}=%
                   \mathbf{y}\boldsymbol{+}\mathbf{x}$
                  \hfill[Commutativity of Vector Addition]
            \item There is a $\mathbf{0}\in{V}$ such that
                  $\mathbf{0}\boldsymbol{+}\mathbf{x}=\mathbf{x}$
                  \hfill[Existence of Zero Vector]
            \item For all $\mathbf{x}$ there is a $\mathbf{y}$ such that
                  $\mathbf{x}\boldsymbol{+}\mathbf{y}=\mathbf{0}$
                  \hfill[Additive Inverses]
            \item $(a\cdot{b})\boldsymbol{\cdot}\mathbf{x}=%
                    a\boldsymbol{\cdot}(b\boldsymbol{\cdot}\mathbf{x})$
                  \hfill[Compatibility of Multiplication]
            \item $(a+b)\boldsymbol{\cdot}\mathbf{x}=%
                   (a\boldsymbol{\cdot}\mathbf{x})\boldsymbol{+}%
                   (b\boldsymbol{\cdot}\mathbf{x})$
                  \hfill[Distributive Law for Field Addition]
            \item $a\boldsymbol{\cdot}(\mathbf{x}\boldsymbol{+}\mathbf{y})=%
                   (a\boldsymbol{\cdot}\mathbf{x})\boldsymbol{+}%
                   (a\boldsymbol{\cdot}\mathbf{y})$
                  \hfill[Distributive Law for Vector Addition]
        \end{enumerate}
    \end{fdefinition}
    It is quite common not to distinguish between scalar multiplication
    $\boldsymbol{\cdot}$ and field multiplication $\cdot$, which may cause
    confusion. It is also common to drop the use of a symbol altogether and
    simply representation multiplication by concatenation of the the
    two variables, for example $a\mathbf{x}$ or $ab$, which represents
    scalar multiplication and field multiplication, respectively.
    \begin{example}
        If we let $\mathbb{F}=\mathbb{R}$ and let
        $V=\mathbb{R}^{n}$, where addition, multiplication, scalar
        multiplication, and vector addition are defined in their usual
        manner, then this forms a vector space. Similarly, the space
        $C([a,b])$ of continuous functions forms a vector space over
        $\mathbb{R}$, as does $L^{2}(\mathbb{R})$, the space of
        square integrable functions.
    \end{example}
    \begin{fdefinition}{Bilinear Operations}{Bilinear_Operations}
        A bilinear operation on a vector space
        $(V,\,\boldsymbol{+},\,\boldsymbol{\cdot}\,)$ over a field
        $(\mathbf{F},\,+,\,\cdot\,)$ is a function
        $[\,]:V\times{V}\rightarrow{V}$ such that, for all
        $\mathbf{x},\mathbf{y},\mathbf{z}\in{V}$, and for all
        $a,b\in\mathbf{F}$, the following is true:
        \begin{enumerate}
            \item $[\mathbf{x}\boldsymbol{+}\mathbf{y}, \mathbf{z}]=%
                   [\mathbf{x},\mathbf{z}]\boldsymbol{+}%
                   [\mathbf{y},\mathbf{z}]$
                  \hfill[Right Distributive Law]
            \item $[\mathbf{x},\mathbf{y}\boldsymbol{+}\mathbf{z}]=%
                   [\mathbf{x},\mathbf{y}]\boldsymbol{+}%
                   [\mathbf{x},\mathbf{z}]$
                  \hfill[Left Distributive Law]
            \item $[a\boldsymbol{\cdot}\mathbf{x},%
                    b\boldsymbol{\cdot}\mathbf{y}]=%
                   (a\cdot{b})\boldsymbol{\cdot}[\mathbf{x},\mathbf{y}]$
                  \hfill[Compatibility with Scalars]
        \end{enumerate}
    \end{fdefinition}
    \begin{lexample}{Examples of Bilinear Operations}
                    {Examples_of_Bilinear_Operation}
        The quintessential example of a bilinear operation is the
        cross product that one encounters in a multivariable calculus
        course. That is, for any three vectors
        $\mathbf{x},\mathbf{y},\mathbf{z}$, we have:
        \begin{equation}
            \mathbf{x}\times(\mathbf{y}+\mathbf{z})=
            \mathbf{x}\times\mathbf{y}+\mathbf{x}\times\mathbf{z}
        \end{equation}
        Similarly for right sided multiplication. The compatibility of
        the cross product with scalar multiplication is also true:
        \begin{equation}
            (a\mathbf{x})\times(b\mathbf{y})=ab(\mathbf{x}\times\mathbf{y})
        \end{equation}
        This serves somewhat as a motivating example for bilinear
        operations. If we think of the field of invertible matrices,
        then multiplication forms a bilinear operation as well, with
        scalar multiplication being the usual entry wise operation that
        is done on matrices. Lastly, if $\langle\,\rangle$ is an inner
        product on $\mathbb{R}$ or $\mathbb{C}$, then this is a bilinear
        operation, the vector space being the underlying field itself.
    \end{lexample}
    \begin{fdefinition}{Algebra over a Field}{Algebra_over_a_Field}
        An algebra of a field $(\mathbf{F},\,+,\,\cdot\,)$ is a
        vector space $(\mathbf{V},\,\boldsymbol{+},\,\boldsymbol{\cdot}\,)$
        and a bilinear operation $[\,]:V\times{V}\rightarrow{V}$.
    \end{fdefinition}
    \begin{fdefinition}{Associative Algebra over a Field}
                       {Associative_Algebra_over_a_Field}
        An associative algebra over a field $(\mathbb{F},\,+,\,\cdot\,)$
        is an algebra $(V,[\,])$ over $\mathbb{F}$ such that, for all
        $r\in\mathbb{F}$ and for all $\mathbf{x},\mathbf{y}\in{V}$,
        the following is true:
        \begin{equation}
            r[\mathbf{x},\,\mathbf{y}]=[r\mathbf{x},\,\mathbf{y}]
                                      =[\mathbf{x},\,r\mathbf{y}]
        \end{equation}
    \end{fdefinition}
    \begin{fdefinition}{Derivation on an Algebra}{Derivation_on_an_Algebra}
        A derivation on an algebra $(V,\,[\,])$ is a function
        $D:V\rightarrow{V}$ such that for all $\mathbf{x},\mathbf{y}\in{V}$,
        the following (Liebniz's Rule) is true:
        \begin{equation}
            D([\mathbf{x},\mathbf{y}])
            =[\mathbf{x},D(\mathbf{y})]+[D(\mathbf{x}),\mathbf{y}]
        \end{equation}
    \end{fdefinition}
    \begin{theorem}
        In a field, $0$ and $1$ are unique.
    \end{theorem}
    \begin{proof}
        For suppose not, and let $0'$ and $1'$ be other identities.
        Then $1'=1'\cdot 1 = 1$ and $0'=0'+0=0$.
    \end{proof}
    \begin{theorem}
        For any field $\langle{F},+,\cdot\rangle$ and $a\in{F}$, $a\cdot{0}=0$.
    \end{theorem}
    \begin{proof}
        For:
        \begin{equation}
            0=a\cdot{0}+(\minus{a}\cdot{0})
             =a\cdot(0+0)+(\minus{a}\cdot{0})
             =a\cdot{0}+a\cdot{0}+(\minus{a}\cdot{0})
             =a\cdot 0
        \end{equation}
        Thus, $a\cdot{0}=0$.
    \end{proof}
    If $1=0$, then $a=a\cdot{1}=a\cdot{0}=0$, and thus every element is
    zero. A very boring field.
    \begin{theorem}
        In a field $\langle F, +,\cdot \rangle$, if $0\ne 1$, then $0$ has no
        inverse.
    \end{theorem}
    \begin{proof}
        For let $a$ be such an inverse. Then $a\cdot{0}=1$. But for any element
        of $F$, $a\cdot{0}=0$. But $0\ne{1}$, a contradiction.
    \end{proof}
    \begin{theorem}
        If $a+b=0$, then $b=(\minus{1})\cdot{a}$ where $(\minus{1})$ is the
        solution to $1+(\minus{1})=0$.
    \end{theorem}
    \begin{proof}
        $a+(\minus{1})a=a(1+(\minus{1}))=a\cdot{0}=0$. From uniqueness,
        $b=(\minus{1})a$. We may thus write additive inverses as $\minus{a}$.
    \end{proof}
    \begin{definition}
        Given two fields $(F,+,\cdot)$ and $(F',+',\times)$, a bijection
        function $f:F\rightarrow{F}'$ is said to be a field isomorphism if and
        only if for allelements $a,b\in{F}$, $f(a+b)=f(a)+'f(b)$, and
        $f(a\cdot{b})=f(a)\times{f}(b)$
    \end{definition}
    \begin{definition}
        $(F,+,\cdot)$ and $(F',+',\times)$, are said to be isomorphic if and
        only if they have an isomorphism.
    \end{definition}
    \begin{theorem}
        Given an ismorphism between two fields $(F,+,\cdot)$ and
        $(F', +',\times)$, $f(1)=1'$ and $f(0)=0'$.
    \end{theorem}
    \begin{proof}
        For let $x\in{F}$. Then $f(x)=f(x\cdot 1)=f(x)\times{f}(1)$, and
        $f(x)=f(x+0)=f(x)+'f(0)$. Therefore, etc.
    \end{proof}
    \begin{theorem}
        In a field $(F,+,\cdot)$, $(a+b)^{2}=a^{2}+2ab+b^{2}$
        ($2$ being the solution to $1+1$).
    \end{theorem}
    \begin{proof}
        For:
        \begin{align}
            (a+b)^{2}&=(a+b)(a+b)\\
                     &=a(a+b)+b(a+b)\\
                     &=a^{2}+ab+ba+b^{2}\\
                     &=a^{2}+ab(1+1)+b^{2}\\
                     &=a^{2}+2ab+b^{2}
        \end{align}
    \end{proof}
        \section{Ring Morphisms}
    \begin{fdefinition}{Ring Homomorphism}{Ring_Homomorphism}
        A \gls{ring homomorphism}\index{Ring Homomorphism} from a \gls{ring}
        $(R_{1},\,+,\,\cdot\,)$ to a ring $(R_{2},\,+',\,*\,)$ is a
        \gls{function} $f:R_{1}\rightarrow{R}_{2}$ such that, for all
        $x,y\in{R}_{1}$, the following are true:
        \begin{align}
            f(x+y)&=f(x)+'f(y)
            \tag{Preservation of Addition}\\
            f(x\cdot{y})&=f(x)*f(y)
            \tag{Preservation of Multiplication}\\
            f(1_{R_{1}})&=1_{R_{2}}
            \tag{Preservation of Identities}
        \end{align}
        Where $1_{R_{1}}$ is the unital element of $R_{1}$ and
        $1_{R_{2}}$ is the unital element of $R_{2}$.
    \end{fdefinition}
    There's a special name for a homomorphism from a ring $(R,\,+,\,\cdot\,)$
    to itself.
    \begin{fdefinition}{Ring Endomorphisms}{Ring_Endomorphisms}
        A \gls{ring endomorphism}\index{Ring Endomorphism} on a \gls{ring}
        $(R,\,+,\,\cdot\,)$ is a \gls{ring homomorphism} from
        $(R,\,+,\,\cdot\,)$ to itself. That is, a ring homomorphism
        $f:R\rightarrow{R}$.
    \end{fdefinition}
    \begin{fnotation}{Set of Ring Endomorphisms}{Ring_Endomorphisms}
        The set of ring endomorphisms on a ring $\mathcal{R}=(R,\,+,\,\cdot)$
        is denoted $\textrm{End}(\mathcal{R})$.
    \end{fnotation}
    \chapter{Fields}
        \section{Definitions}
    \begin{fdefinition}{Fields}{Fields}
        A field is a commutative ring $(\mathbb{F},+,\cdot\,)$ such that, for
        all $a\in\mathbb{F}$ such that $a$ is not the unital element of
        $(\mathbb{F},+)$, it is true that $a$ is an invertible element of
        $(\mathbb{F},\cdot\,)$.
    \end{fdefinition}
    \begin{fdefinition}{Subfield}{Subfield}
        A subfield of a field $(F,+,\cdot)$ is a set $K\subset F$, such that
        $(K,+,\cdot)$ is a field.
    \end{fdefinition}
    Given an element $a\in\mathbb{F}$, if $b$ is such that
    $a+b=0$ then we write $b=\minus{a}$. Subtraction of two elements
    $a$ and $c$, denoted $a-c$, is defined as $a+(\minus{c})$. The
    structure $(\mathbb{F},+)$ forms an Abelian group. From this we have
    that the identity is unique, as are additive inverses.
    It is common in the definition of a field to require that
    $0\ne{1}$. This is because if $0=1$ then we have $\mathbb{F}=\{0\}$.
    This comes from the following.
    \begin{ltheorem}{Multiplication by Zero}{Multiplication_by_Zero}
        If $(\mathbb{F},\,+,\,\cdot\,)$ is a field, and if $a\in\mathbb{F}$,
        then $a\cdot{0}=0$.
    \end{ltheorem}
    \begin{proof}
        For we have:
        \begin{equation}
            0=a\cdot(0)-a\cdot(0)=a\cdot(0-0)=a\cdot{0}
        \end{equation}
        This simply combines the distributive law with the additive
        property of zero, completing the proof.
    \end{proof}
    \begin{theorem}
        If $(\mathbb{F},\,+,\,\cdot\,)$ is a field, and if $0=1$, then
        $\mathbb{F}=\{0\}$.
    \end{theorem}
    \begin{proof}
        For suppose not, and let $a\in\mathbb{F}$ be such that $a\ne{0}$.
        But then, by Thm.~\ref{thm:Multiplication_by_Zero}:
        \begin{equation}
            a=a\cdot{1}=a\cdot{0}=0
        \end{equation}
        And thus $a=0$, a contradiction. Therefore,
        $\mathbb{F}$ is trivial.
    \end{proof}
    It is thus common to either call such a field a trivial field, or
    to require that $0\ne{1}$.
    \begin{lexample}{Examples of Fields}{Examples_of_Fields}
        There are several fields that should be familiar to the reader.
        If we let $\mathbb{R}$ denote the real numbers and $+$ and $\cdot$
        be the usual notations of addition and multiplication, then
        $(\mathbb{R},\,+,\,\cdot\,)$ is a field. Similarly, letting
        $\mathbb{Q}$ denote the rational numbers and $\mathbb{C}$ denote
        the complex numbers, $(\mathbb{Q},\,+,\,\cdot\,)$ is a field, as
        is $(\mathbb{C},\,+,\,\cdot\,)$. There are finite fields as well.
        Let $\mathbb{F}_{2}=\{0,\,1\}$ and define multiplication and
        addition as follows:
        \par\hfill\par
        \begin{table}[H]
            \centering
            \captionsetup{type=table}
            \parbox{.45\linewidth}{%
                \centering
                \begin{tabular}{c|cc}
                    $+$&0&1\\
                    \hline
                    0&0&1\\
                    1&1&0
                \end{tabular}
            }
            \parbox{.45\linewidth}{%
                \centering
                \begin{tabular}{c|cc}
                    $\cdot$&0&1\\
                    \hline
                    0&0&0\\
                    1&0&1
                \end{tabular}
            }
            \caption{The Arithmetic of $\mathbb{F}_{2}$}
        \end{table}
        $(\mathbb{F}_{2},\,+,\,\cdot)$ forms a field. Finally, if
        $p\in\mathbb{N}$ is prime, and if $+$ and $\cdot$ are addition
        and multiplication mod $p$, respectively, then
        $(\mathbb{Z}_{p},\,+,\,\cdot\,)$ is a field.
    \end{lexample}
    \begin{fdefinition}{Vector Space}{Vector_Space}
        A vector space over a field $(\mathbb{F},\,+,\,\cdot\,)$ is a
        set $V$ and a function
        $\boldsymbol{\cdot}:\mathbb{F}\times{V}\rightarrow{V}$ and
        a binary operation $\boldsymbol{+}$ on $V$, usuall called
        scalar multiplication and vector addition, respectively, 
        such that for all $\mathbf{x},\mathbf{y},\mathbf{z}\in{V}$,
        and all $a,b\in\mathbf{F}$, the following is true:
        \begin{enumerate}
            \item $\mathbf{x}\boldsymbol{+}%
                   (\mathbf{y}\boldsymbol{+}\mathbf{z})=%
                   (\mathbf{x}\boldsymbol{+}\mathbf{y})%
                   \boldsymbol{+}\mathbf{z}$
                  \hfill[Associative of Vector Addition]
            \item $\mathbf{x}\boldsymbol{+}\mathbf{y}=%
                   \mathbf{y}\boldsymbol{+}\mathbf{x}$
                  \hfill[Commutativity of Vector Addition]
            \item There is a $\mathbf{0}\in{V}$ such that
                  $\mathbf{0}\boldsymbol{+}\mathbf{x}=\mathbf{x}$
                  \hfill[Existence of Zero Vector]
            \item For all $\mathbf{x}$ there is a $\mathbf{y}$ such that
                  $\mathbf{x}\boldsymbol{+}\mathbf{y}=\mathbf{0}$
                  \hfill[Additive Inverses]
            \item $(a\cdot{b})\boldsymbol{\cdot}\mathbf{x}=%
                    a\boldsymbol{\cdot}(b\boldsymbol{\cdot}\mathbf{x})$
                  \hfill[Compatibility of Multiplication]
            \item $(a+b)\boldsymbol{\cdot}\mathbf{x}=%
                   (a\boldsymbol{\cdot}\mathbf{x})\boldsymbol{+}%
                   (b\boldsymbol{\cdot}\mathbf{x})$
                  \hfill[Distributive Law for Field Addition]
            \item $a\boldsymbol{\cdot}(\mathbf{x}\boldsymbol{+}\mathbf{y})=%
                   (a\boldsymbol{\cdot}\mathbf{x})\boldsymbol{+}%
                   (a\boldsymbol{\cdot}\mathbf{y})$
                  \hfill[Distributive Law for Vector Addition]
        \end{enumerate}
    \end{fdefinition}
    It is quite common not to distinguish between scalar multiplication
    $\boldsymbol{\cdot}$ and field multiplication $\cdot$, which may cause
    confusion. It is also common to drop the use of a symbol altogether and
    simply representation multiplication by concatenation of the the
    two variables, for example $a\mathbf{x}$ or $ab$, which represents
    scalar multiplication and field multiplication, respectively.
    \begin{example}
        If we let $\mathbb{F}=\mathbb{R}$ and let
        $V=\mathbb{R}^{n}$, where addition, multiplication, scalar
        multiplication, and vector addition are defined in their usual
        manner, then this forms a vector space. Similarly, the space
        $C([a,b])$ of continuous functions forms a vector space over
        $\mathbb{R}$, as does $L^{2}(\mathbb{R})$, the space of
        square integrable functions.
    \end{example}
    \begin{fdefinition}{Bilinear Operations}{Bilinear_Operations}
        A bilinear operation on a vector space
        $(V,\,\boldsymbol{+},\,\boldsymbol{\cdot}\,)$ over a field
        $(\mathbf{F},\,+,\,\cdot\,)$ is a function
        $[\,]:V\times{V}\rightarrow{V}$ such that, for all
        $\mathbf{x},\mathbf{y},\mathbf{z}\in{V}$, and for all
        $a,b\in\mathbf{F}$, the following is true:
        \begin{enumerate}
            \item $[\mathbf{x}\boldsymbol{+}\mathbf{y}, \mathbf{z}]=%
                   [\mathbf{x},\mathbf{z}]\boldsymbol{+}%
                   [\mathbf{y},\mathbf{z}]$
                  \hfill[Right Distributive Law]
            \item $[\mathbf{x},\mathbf{y}\boldsymbol{+}\mathbf{z}]=%
                   [\mathbf{x},\mathbf{y}]\boldsymbol{+}%
                   [\mathbf{x},\mathbf{z}]$
                  \hfill[Left Distributive Law]
            \item $[a\boldsymbol{\cdot}\mathbf{x},%
                    b\boldsymbol{\cdot}\mathbf{y}]=%
                   (a\cdot{b})\boldsymbol{\cdot}[\mathbf{x},\mathbf{y}]$
                  \hfill[Compatibility with Scalars]
        \end{enumerate}
    \end{fdefinition}
    \begin{lexample}{Examples of Bilinear Operations}
                    {Examples_of_Bilinear_Operation}
        The quintessential example of a bilinear operation is the
        cross product that one encounters in a multivariable calculus
        course. That is, for any three vectors
        $\mathbf{x},\mathbf{y},\mathbf{z}$, we have:
        \begin{equation}
            \mathbf{x}\times(\mathbf{y}+\mathbf{z})=
            \mathbf{x}\times\mathbf{y}+\mathbf{x}\times\mathbf{z}
        \end{equation}
        Similarly for right sided multiplication. The compatibility of
        the cross product with scalar multiplication is also true:
        \begin{equation}
            (a\mathbf{x})\times(b\mathbf{y})=ab(\mathbf{x}\times\mathbf{y})
        \end{equation}
        This serves somewhat as a motivating example for bilinear
        operations. If we think of the field of invertible matrices,
        then multiplication forms a bilinear operation as well, with
        scalar multiplication being the usual entry wise operation that
        is done on matrices. Lastly, if $\langle\,\rangle$ is an inner
        product on $\mathbb{R}$ or $\mathbb{C}$, then this is a bilinear
        operation, the vector space being the underlying field itself.
    \end{lexample}
    \begin{fdefinition}{Algebra over a Field}{Algebra_over_a_Field}
        An algebra of a field $(\mathbf{F},\,+,\,\cdot\,)$ is a
        vector space $(\mathbf{V},\,\boldsymbol{+},\,\boldsymbol{\cdot}\,)$
        and a bilinear operation $[\,]:V\times{V}\rightarrow{V}$.
    \end{fdefinition}
    \begin{fdefinition}{Associative Algebra over a Field}
                       {Associative_Algebra_over_a_Field}
        An associative algebra over a field $(\mathbb{F},\,+,\,\cdot\,)$
        is an algebra $(V,[\,])$ over $\mathbb{F}$ such that, for all
        $r\in\mathbb{F}$ and for all $\mathbf{x},\mathbf{y}\in{V}$,
        the following is true:
        \begin{equation}
            r[\mathbf{x},\,\mathbf{y}]=[r\mathbf{x},\,\mathbf{y}]
                                      =[\mathbf{x},\,r\mathbf{y}]
        \end{equation}
    \end{fdefinition}
    \begin{fdefinition}{Derivation on an Algebra}{Derivation_on_an_Algebra}
        A derivation on an algebra $(V,\,[\,])$ is a function
        $D:V\rightarrow{V}$ such that for all $\mathbf{x},\mathbf{y}\in{V}$,
        the following (Liebniz's Rule) is true:
        \begin{equation}
            D([\mathbf{x},\mathbf{y}])
            =[\mathbf{x},D(\mathbf{y})]+[D(\mathbf{x}),\mathbf{y}]
        \end{equation}
    \end{fdefinition}
    \begin{theorem}
        In a field, $0$ and $1$ are unique.
    \end{theorem}
    \begin{proof}
        For suppose not, and let $0'$ and $1'$ be other identities.
        Then $1'=1'\cdot 1 = 1$ and $0'=0'+0=0$.
    \end{proof}
    \begin{theorem}
        For any field $\langle{F},+,\cdot\rangle$ and $a\in{F}$, $a\cdot{0}=0$.
    \end{theorem}
    \begin{proof}
        For:
        \begin{equation}
            0=a\cdot{0}+(\minus{a}\cdot{0})
             =a\cdot(0+0)+(\minus{a}\cdot{0})
             =a\cdot{0}+a\cdot{0}+(\minus{a}\cdot{0})
             =a\cdot 0
        \end{equation}
        Thus, $a\cdot{0}=0$.
    \end{proof}
    If $1=0$, then $a=a\cdot{1}=a\cdot{0}=0$, and thus every element is
    zero. A very boring field.
    \begin{theorem}
        In a field $\langle F, +,\cdot \rangle$, if $0\ne 1$, then $0$ has no
        inverse.
    \end{theorem}
    \begin{proof}
        For let $a$ be such an inverse. Then $a\cdot{0}=1$. But for any element
        of $F$, $a\cdot{0}=0$. But $0\ne{1}$, a contradiction.
    \end{proof}
    \begin{theorem}
        If $a+b=0$, then $b=(\minus{1})\cdot{a}$ where $(\minus{1})$ is the
        solution to $1+(\minus{1})=0$.
    \end{theorem}
    \begin{proof}
        $a+(\minus{1})a=a(1+(\minus{1}))=a\cdot{0}=0$. From uniqueness,
        $b=(\minus{1})a$. We may thus write additive inverses as $\minus{a}$.
    \end{proof}
    \begin{definition}
        Given two fields $(F,+,\cdot)$ and $(F',+',\times)$, a bijection
        function $f:F\rightarrow{F}'$ is said to be a field isomorphism if and
        only if for allelements $a,b\in{F}$, $f(a+b)=f(a)+'f(b)$, and
        $f(a\cdot{b})=f(a)\times{f}(b)$
    \end{definition}
    \begin{definition}
        $(F,+,\cdot)$ and $(F',+',\times)$, are said to be isomorphic if and
        only if they have an isomorphism.
    \end{definition}
    \begin{theorem}
        Given an ismorphism between two fields $(F,+,\cdot)$ and
        $(F', +',\times)$, $f(1)=1'$ and $f(0)=0'$.
    \end{theorem}
    \begin{proof}
        For let $x\in{F}$. Then $f(x)=f(x\cdot 1)=f(x)\times{f}(1)$, and
        $f(x)=f(x+0)=f(x)+'f(0)$. Therefore, etc.
    \end{proof}
    \begin{theorem}
        In a field $(F,+,\cdot)$, $(a+b)^{2}=a^{2}+2ab+b^{2}$
        ($2$ being the solution to $1+1$).
    \end{theorem}
    \begin{proof}
        For:
        \begin{align}
            (a+b)^{2}&=(a+b)(a+b)\\
                     &=a(a+b)+b(a+b)\\
                     &=a^{2}+ab+ba+b^{2}\\
                     &=a^{2}+ab(1+1)+b^{2}\\
                     &=a^{2}+2ab+b^{2}
        \end{align}
    \end{proof}
    \renewcommand{\PATH}{\OLDPATH}
\endgroup
        \part{Fields}
            %------------------------------------------------------------------------------%
\begingroup
    \ifcsname\PATH\endcsname
        \newcommand{\PATH}{books/Algebra/Fields}
        \newcommand{\OLDPATH}{\PATH}
    \else
        \newcommand{\OLDPATH}{\PATH}
        \renewcommand{\PATH}{books/Algebra/Fields}
    \fi
    \chapter{Elementary Field Theory}
    Fields are logically the next step in adding structure to algebra. Most of
    the problems that arise in ring theory stem from the fact that non-zero
    elements may not be invertible, and also that zero divisors may exist. Field
    theory rids us of these problems and establishes the 9 familiar properties
    of arithmetic that one is familiar with from elementary school. While the
    structure of a field is certainly nice, there can still be some problems
    that may appear when one let's their intuition get in the way (for example,
    in the field $\ring{\mathbb{Z}_{2}}$ we have $1+1=0$).
    Fields are logically the next step in adding structure to algebra. Most of
the problems that arise in ring theory stem from the fact that non-zero
elements may not be invertible, and also that zero divisors may exist. Field
theory rids us of these problems and establishes the 9 familiar properties
of arithmetic that one is familiar with from elementary school. While the
structure of a field is certainly nice, there can still be some problems
that may appear when one let's their intuition get in the way (for example,
in the field $\ring{\mathbb{Z}_{2}}$ we have $1+1=0$).
\section{What's the Point?}
    From the historical perspective, we can start with numbers. There's
    the standard inclusion:
    \begin{equation}
        \mathbb{N}^{+}\subseteq\mathbb{N}\subseteq
        \mathbb{Z}\subseteq\mathbb{Q}\subseteq\mathbb{R}
        \subseteq\mathbb{C}
    \end{equation}
    Suppose we are given the equation $2+x=1$. If we only know about
    the positive integers, then we cannot solve this equation. We thus
    need to introduce negative integers. Next we could write $2x=1$,
    and we are now forced to introduce the rational numbers. In ancient
    Greece, the solution to $x^{2}=2$ was proved to be irrational, and
    thus we must go beyond $\mathbb{Q}$ and develop the real numbers
    (or at the very least, the algebraic numbers $\mathbb{A}$). Pushing
    beyong this, polynomial equations such as $x^{2}+1=0$ were studied
    in Italy during the Renaissance era. It was known that there are no
    real solutions to this, as one can see from the graph of $x^{2}+1$
    (it never crosses the $x$ axis). To solve such equations one must
    invent $\mathbb{C}$. The complex numbers are the set $\mathbb{C}$ of
    the form:
    \begin{equation}
        \mathbb{C}=\{\,x+iy\;|\;x,y\in\mathbb{R}\,\}
    \end{equation}
    where $i^{2}=\minus{1}$, by definition, which is the solution to the
    equation $z^{2}+1=0$. This equation has no solutions in $\mathbb{R}$
    and so $i$ is not a real number, and hence is called the
    \textit{imaginary} unit. We can picture complex numbers by use of
    the plane $\mathbb{R}^{2}$. But there's nothing too special about
    the equation $z^{2}+1=0$, and we can consider $z^{2}+z+1=0$ and
    again we can ask if this has real solutions. Unlike the first
    equation, it's not so obvious that this has no real solution. We
    can look at the quadratic formula, and in particular the discriment,
    obtaining:
    \begin{equation}
        \Delta=b^{2}-4ac=1-4=\minus{2}
    \end{equation}
    Since this is negative, there are no real solutions, and hence
    $z^{2}+z+1$ has no solution in $\mathbb{R}$. It does have roots in
    $\mathbb{C}$:
    \twocolumneq{\omega=\minus\frac{1}{2}+\frac{\sqrt{3}}{2}i}
                {\overline{\omega}=\minus\frac{1}{2}-\frac{\sqrt{3}}{2}i}
    We can further consider the set $\mathbb{R}[w]$ defined by:
    \begin{equation}
        \mathbb{R}[\omega]=\{\,x+y\omega\;|\;x,y\in\mathbb{R}\,\}
    \end{equation}
    This has a nice field structure, like $\mathbb{C}$, and indeed this
    is equal to $\mathbb{C}$. That is, $\mathbb{R}[\omega]=\mathbb{C}$.
    We can see this since $\mathbb{R}[\omega]$ is a subspace of
    $\mathbb{C}$ with a basis consisting of two elements:
    $\{1,\omega\}$, and thus has the same dimension as $\mathbb{C}$.
    Hence, it is equal to the whole thing. We can be even more explicit:
    \begin{equation}
        x+y\omega
            =x+y\big(\minus\frac{1}{2}+\frac{\sqrt{3}}{2}i\big)
            =\big(x-\frac{1}{2}y\big)+\big(\frac{\sqrt{3}}{2}y\big)i
    \end{equation}
    And this is of the form $x'+y'i$, where:
    \twocolumneq{x'=x-\frac{1}{2}y}{y'=\frac{\sqrt{3}}{2}y}
    Since this is always solvable for both $(x,y)$ and $(x',y')$, the
    two spaces are the same. And indeed, we can generalize. If
    $f(x)=ax^{2}+bx+c$, with $a,b,c\in\mathbb{R}$ such that
    $b^{2}-4ac<0$, then defining:
    \begin{equation}
        \alpha=\frac{\minus{b}+\sqrt{b^{2}-4ac}}{2a}
    \end{equation}
    which is a complex root of $f$, then
    $\mathbb{R}[\alpha]=\mathbb{C}$. This shows there's nothing too
    special about $i$: extending $\mathbb{R}$ with any complex root of
    a quadratic gives the entirety of $\mathbb{C}$, we need not only
    choose $z^{2}+1=0$. Even if we were to stick with this polynomial,
    we could still choose $\minus{i}$, since this too is a solution.
    Choosing $i$ over $\minus{i}$ seems to purely be an accident of
    history. Going from one choice to another is an
    $\mathbb{C}$ automorphism: $x+iy\mapsto{x}-iy$. An $\mathbb{R}$
    automorphism is a bijective ring homomoprhism
    $f:\mathbb{C}\rightarrow\mathbb{C}$. That is, an isomorphism from
    $\mathbb{C}$ to itself:
    \twocolumneq[\par]{f(z_{1}+z_{2})=f(z_{1})+f(z_{2})}
                      {f(z_{1}z_{2})=f(z_{1})f(z_{2})}
    And also requiring:
    \begin{equation}
            f(1)=1
    \end{equation}
    The automorphism $x+iy\mapsto{x}-iy$ is called complex conjugation.
    If we don't like $i$, and have a complex number such as $\omega$,
    we can still take as an $\mathbb{R}$ automorphism the function
    $\sigma:\mathbb{C}\rightarrow\mathbb{C}$ where
    $x+y\omega\mapsto{x}+y\overline{\omega}$. As it turns out, this is
    the same as the automorphism $x+iy\mapsto{x}-iy$ since we can write:
    \begin{equation}
        i=\frac{1+2\omega}{\sqrt{3}}
    \end{equation}
    This is the object we wish to stress as the important part of the
    theory of complex numbers. Neither $i$ nor $\omega$ are too
    important, but rather the notion of complex conjugation is.
    The group of $\mathbb{R}$ automorphisms of $\mathbb{C}$ is equal to:
    \begin{equation}
        \textrm{Aut}_{\mathbb{R}}(\mathbb{C})
            =\{\identity{\mathbb{R}},\sigma\}
    \end{equation}
    Where $\sigma$ is complex conjugation. That is, $\sigma$ is the
    unique non-trivial $\mathbb{C}$ automorphism that has the property
    that it exchanges the roots of any $f(x)=ax^{2}+bx+c$ with
    $b^{2}-4ac<0$. The group structure comes from function composition.
    Since function composition is associative, the identity map is an
    automorphism, and since bijections have inverse functions,
    this is indeed a group. We can summarize all of this as follows:
    The roots of any real polynomial are either real or come in complex
    conjugate pairs.
    \par\hfill\par
    Looking at the numerology of the problem, there seems to be
    something special about the number two. This is the size of the
    automorphism group $\textrm{Aut}_{\mathbb{R}}(\mathbb{C})$, and
    this is also the dimension of $\mathbb{C}$, and lastly it is the
    degree of $\mathbb{C}$ over $\mathbb{R}$: $[\mathbb{C}:\mathbb{R}]$.
    More generally, consider any field $\mathbb{F}$ with characteristic
    not equal to 2 (that is, $1+1\ne{0}$), and any function
    $f(x)=ax^{2}+bx+c$, $a,b,c\in\mathbb{F}$ such that $f(x)=0$ has no
    solutions in $\mathbb{F}$. For example, $\mathbb{R}$ with
    $f(x)=x^{2}+1$, of $\mathbb{Q}$ with $f(x)=x^{2}-2$. If we have
    such conditions, then there is a field $\mathbb{K}$ and an inclusion
    $\mathbb{F}\subseteq\mathbb{K}$ making $\mathbb{F}$ a subfield,
    such that $f(x)=a(x-\alpha)(x-\beta)$, where
    $\alpha,\beta\in\mathbb{K}$. Moreover,
    $\mathbb{K}=\mathbb{F}[\alpha]$. That is:
    \begin{equation}
        \mathbb{K}=\{\,x+y\alpha\;|\;x,y\in\mathbb{F}\}
    \end{equation}
    Similarly, $\mathbb{K}=\mathbb{F}[\beta]$. Lastly, the automorphism
    group is
    \begin{equation}
        \textrm{Aut}_{\mathbb{F}}(\mathbb{K})
        =\{\,\textrm{id}_{\mathbb{F}},\sigma\}
    \end{equation}
    where $\sigma$ is the unique automorphism such that
    $\sigma(\alpha=\beta)$. The proof is simply an application of the
    quadratic formula, where we invoke the fact that $2\ne{0}$ in a
    field whose characteristic is not 2.
\subsection{Cubic Equations and Higher}
    In the $16^{th}$ century the Italians were able to solve the cubic
    equation:
    \begin{equation}
        x^{3}+px-q=0
    \end{equation}
    This may not look like the general cubic, but since we are
    interested in roots we may always divide off by the leading
    coefficient of $x^{3}$, and the quadratic term may be
    absorbed by completing the square, and thus any cubic can be
    written in such a form. The solution is much less elegant than the
    quadratic formula:
    \begin{equation}
        x=
    \end{equation}
    By the $18^{th}$ century the Italians were able to solve the general
    quartic equation. The next natural question is the solution to the
    quintic, but this was shown not to exist. The Abel-Ruffini theorem
    shows that the general quintic equation can not be solved using
    nested radicals. Galois went to prove that a polynomial has a root
    that can be written in terms of nested radicals if and only if
    $K/F$, the splitting field, has an automorphism group
    $\textrm{Aut}_{F}(K))$ that is solveable.
\subsection{Some Reminders}
    \begin{definition}
        A field $(\mathbb{F},+,\cdot)$ is an Abelian group
        $(\mathbb{F},+)$ such that $(F^{*}\setminus\{0\},\cdot)$ is an
        Abelian group as well. This is the group of \textit{units}.
    \end{definition}
    \begin{example}
        The classic exmaples are $\mathbb{Q}$, $\mathbb{R}$, and
        $\mathbb{C}$, as well as the finite fields $\mathbb{F}_{p}$,
        also commonly denoted $\mathbb{Z}/p\mathbb{Z}$ or simply
        $\mathbb{Z}_{p}$.
    \end{example}
    \begin{definition}
        A field extension of a field $F$ is a field $K$ such that
        $F\subseteq{K}$. We may also say that $F$ is a subfield of $K$.
    \end{definition}
    We often denote that $K$ is a field extension of $F$ by writing
    $K/F$. This is not to denote a quotient or anything of that manner
    and is simply to denote that $F$ sis a subfield of $K$.
    \begin{example}
        $\mathbb{C}$ is a field extension of $\mathbb{R}$ since both are
        fields and $\mathbb{R}\subseteq\mathbb{C}$. We can go backwards,
        thinking of $\mathbb{R}$ as a field extension $\mathbb{R}$.
    \end{example}
    Also important, if $K$ is a field extension of $F$, $K/F$, then
    $K$ has the structure of an $F$ vector space. That is, $K$ can be
    seen as a vector space over $F$. One thing that we write is this
    bracket notation $[K:F]$, which again is not to be confused with
    the notation found in groups about the cardinality of certain
    things. $[K:F]$ is the simply the dimesnion of the vector space
    $K$ over $F$:
    \begin{equation}
        [K:F]=\textrm{dim}_{K}(F)
    \end{equation}
    This is also called the degree of the extension $K/F$. If the
    dimension is finite, $[K:F]<\infty$, we say that $K/F$ is a finite
    extension.
    \begin{example}
        $\mathbb{C}$ is a two dimensional vector space over $\mathbb{R}$
        and thus $[\mathbb{C},\mathbb{R}]=2$. To see this, use
        $\{1,i\}$ as a basis.
    \end{example}
    \begin{theorem}
        Any countable dimensional vector space over a countable field is
        also countable.
    \end{theorem}
    \begin{example}
        Using this theorem shows that $\mathbb{R}$, as a vector space
        over $\mathbb{Q}$, in not only an infinite dimensional vector
        space, but also has an uncountably infinite basis.
        Thus, $[\mathbb{R}:\mathbb{Q}]$ is uncountably infinite.
    \end{example}
    \begin{example}
        Consider $\mathbb{Q}[\sqrt{2}]$, defined by:
        \begin{equation}
            \mathbb{Q}[\sqrt{2}]=\{x+y\sqrt{2}\;|\;x,y\in\mathbb{Q}\,\}
        \end{equation}
        This is a subfield of $\mathbb{R}$,
        $\mathbb{Q}[\sqrt{2}]\subseteq\mathbb{R}$. Addition and
        multiplication are easy enough to see, and 0 and 1 are contained
        in there, we need only check multiplicative inverses. But:
        \begin{equation}
            (x+\sqrt{2}y)^{\minus{1}}=\frac{x-\sqrt{2}y}{x^{2}-2y^{2}}
        \end{equation}
        And $x^{2}-2y^{2}$ is only zero when $x=y=0$, since if
        $x^{2}-2y^{2}=0$, then rearrange this to obtain $q^{2}=2$. But
        by the arguments of the ancient Greeks, there is no rational
        number whose square is 2, and thus the denominator is never
        zero for non-zero rational ordered pairs.
    \end{example}
    \begin{example}
        $\mathbb{R}/\mathbb{Q}[\sqrt{2}]$ is uncountably infinite, but
        $\mathbb{Q}[\sqrt{2}]/\mathbb{Q}$ has degree 2 with a basis
        $\{1,\sqrt{2}\}$.
    \end{example}
\subsection{Polynomials}
    We use $F[x]$ to denote the ring of polynomials with coefficients in
    $F$. For example:
    \begin{equation}
        f(x)=a_{n}x^{n}+a_{n-1}x^{n-1}+\cdots+a_{1}x+x_{0}
        \quad\quad
        a_{k}\in{F}
    \end{equation}
    Then $f\in{F}[x]$. The degree of a polynomial is the largest power
    of the polynomial with non-zero coefficient. Some things can be said
    about the degree of polynomials:
    \begin{align}
        \textrm{deg}(f+g)&\leq
            \textrm{max}\{\textrm{deg}(f),\textrm{def}(g)\}\\
        \textrm{deg}(fg)&=\textrm{deg}(f)+\textrm{deg}(g)
        \quad\quad
        f,g\ne{0}
    \end{align}
    The degree of a polynomial is zero if and only if the polynomial is
    constant. Since $F[x]$ has a ring structure, $F[x]^{*}$ can be seen
    as the set of all non-zero constant polynomials.
    \begin{theorem}
        $F[x]$ is a Euclidean domain. That is, for any polynomial
        $f\in{F}[x]$ and for any non-zero $g\in{F}[x]$, there exist
        unique polynomials $r,q\in{F}[x]$ such that $f=qg+r$ where
        either $r=0$ or $\textrm{deg}(r)<\textrm{deg}(g)$.
    \end{theorem}
    \begin{theorem}
        The polynomial ring $F[x]$ is a principal ideal domain. That is,
        every ideal $I\subseteq{F}[x]$ is principal. That is, every
        ideal is generated by a single element.
    \end{theorem}
    \begin{theorem}
        Every Euclidean domain is a principle ideal.
    \end{theorem}
    Thus, there is a bijection between ideals $I\subseteq{F}[x]$ and
    monic polynomials in $F[x]$. Recall that if $R$ is a commutatie ring
    with unity, then $r\in{R}$ is called irreducible if $r\ne{0}$, $r$
    not a unit, and if $r=ab$, then either $a$ or $b$ is a unit. We take
    this definition to exclude some trivialities. For example, in
    $\mathbb{Z}$, 3 is irreducible, however
    $3=(\minus{1})\cdot(\minus{3})$. We don't care about this product,
    since $\minus{1}$ is a unit. Moreover, an element $r\in{R}$ is
    prime if $(r)\subseteq{R}$ is a prime ideal. That is if
    $r$ divides $ab$, then either $r$ divides $a$ or $r$ divides $b$.
    By divides, $r|a$, we mean that $a=r\cdot{s}$ for some $s\in{R}$.
    \begin{example}
        If $F$ is a field, $f\in{F}[x]$, then $f$ is irreducible if and
        only if $f$ is not the product of two polynomials with degrees
        strictly less than $f$. That is, if $f=gh$, then one of these
        must be a constant.
    \end{example}
    \begin{example}
        In $\mathbb{Z}$, prime if and only if irreducible.
    \end{example}
    \begin{theorem}
        If $R$ is a integral domain, and if $r$ is prime, then it is
        irreducible. That is, if there are no zero divisors then prime
        implies irreducible.
    \end{theorem}
    \begin{theorem}
        If $R$ is a principal ideal domain and if $r$ is irreducible,
        then the ideal generated by $r$ is maximal.
    \end{theorem}
    \begin{theorem}
        A maximal ideal is a prime ideal.
    \end{theorem}
    Recale that an ideal is called prime if $R/I$ is a domain. That is,
    if $ab\in{I}$, then either $a\in{I}$ or $b\in{I}$. A maximal ideal
    is and ideal that has no proper ideals between it and the entire
    ring. Another way to say this is that $R/I$ is a field. In other
    words, if $I\subseteq{J}\subseteq{R}$, then $I=J$. Using this we see
    that a maximal ideal is prime since $R/I$ will be a field, which is
    certainly an integral domain.
    \begin{theorem}
        The fourth isomorphism theorem says that if $I\subseteq{R}$ is
        an ideal, then there is a bijection between ideals containing
        $I$, $I\subseteq{J}\subseteq{R}$, and ideals of $R/I$.
    \end{theorem}
    \begin{theorem}
        $f\in{F}[x]$ is irreducible if and only if $F[x]/(f)$ is a
        field.
    \end{theorem}
    Note that $F$ can been seen as a subfield of $F[x]/(f)$ since
    $F$ can be identitified with all constant polynomials, which can
    further be seen tolive inside of $F[x]/(f)$.
    \begin{theorem}
        If $\overline{g}\in{F}[x]/(f)$ then there exists a unique
        $g_{0}\in{F}[x]$ such that$\textrm{deg}(g_{0})<\textrm{deg}(f)$,
        with $\overline{g_{0}}=\overline{g}$.
    \end{theorem}
    If $n$ is the degree of $f$, then the set
    $\{\overline{1},\overline{x},\dots,\overline{x^{n-1}}\}$ is a basis
    for $F[x]/(f)$ over $F$.
    \begin{theorem}
        If $f$ is irreducible of degree $n$, then
        $F[x]/(f)$ is a field extension of $F$ of degree $n$.
    \end{theorem}
    \begin{example}
        In $\mathbb{R}$, the polynomial $f(x)=x^{2}+1$ is irreducible
        since it cannot be factors any further. Thus
        $\mathbb{R}[x]/(x^{2}+1)$ is a field extension of $\mathbb{R}$
        of degree 2.
    \end{example}
    \begin{theorem}
        $\mathbb{R}[x]/(x^{2}+1)$ is isomorphic to $\mathbb{C}$.
    \end{theorem}
    \begin{proof}
        For since $\{\overline{1},\overline{x}\}$ is a basis, we have:
        \begin{equation}
            \mathbb{R}[x]/(x^{2}+1)=
                \{a\overline{1}+b\overline{x}\;|\;a,b\in\mathbb{R}\}
        \end{equation}
        So we trivial map $a\overline{1}+b\overline{x}$ to $a+bi$.
    \end{proof}
    \begin{example}
        Consider now $\mathbb{Q}[x]$ with $x^{2}-2$. This is irreducible
        since it cannot be factor ($\sqrt{2}$ is irrational). Then
        $\mathbb{Q}[x]/(x^{2}-2)$ is isomorphic to
        $\mathbb{Q}[\sqrt{2}]$.
    \end{example}
\subsection{Review of Previous Lecture}
    If $F$ is a field, and if $f\in{F}[x]$ is irreducible, then
    $F[x]/(f)$ is a field extension of $F$ of degree $\textrm{deg}(f)$.
    Also, $\overline{x}=x+(f)\in{F}[x]/(f)$ is a root of $f(x)$ in this
    field $F[x]/(f)$. That is, $f(\overline{x})=\overline{f(x)}$, and
    this maps to zero. The fact that $f(\overline{x})=\overline{f(x)}$
    is simply the statement that the quotient ring is well defined. 
\endgroup
        \part{Modules}
            %------------------------------------------------------------------------------%
\begingroup
    \ifcsname\PATH\endcsname
        \newcommand{\PATH}{books/Algebra/Modules}
        \newcommand{\OLDPATH}{\PATH}
    \else
        \newcommand{\OLDPATH}{\PATH}
        \renewcommand{\PATH}{books/Algebra/Modules}
    \fi
    \chapter{Modules}
        Modules generalize the notion of a vector space that one might find in a
        linear algebra or multivariable calculus course. More than being
        a generalization for the sake of generalizing, they are found
        abundantly in the wild and are used in both geometry and algebra. This
        chapter is devoted to developing the basics of modules, providing
        definitions, theorems, and plenty of examples.
        \section{Definitions}
    \begin{fdefinition}{Fields}{Fields}
        A field is a commutative ring $(\mathbb{F},+,\cdot\,)$ such that, for
        all $a\in\mathbb{F}$ such that $a$ is not the unital element of
        $(\mathbb{F},+)$, it is true that $a$ is an invertible element of
        $(\mathbb{F},\cdot\,)$.
    \end{fdefinition}
    \begin{fdefinition}{Subfield}{Subfield}
        A subfield of a field $(F,+,\cdot)$ is a set $K\subset F$, such that
        $(K,+,\cdot)$ is a field.
    \end{fdefinition}
    Given an element $a\in\mathbb{F}$, if $b$ is such that
    $a+b=0$ then we write $b=\minus{a}$. Subtraction of two elements
    $a$ and $c$, denoted $a-c$, is defined as $a+(\minus{c})$. The
    structure $(\mathbb{F},+)$ forms an Abelian group. From this we have
    that the identity is unique, as are additive inverses.
    It is common in the definition of a field to require that
    $0\ne{1}$. This is because if $0=1$ then we have $\mathbb{F}=\{0\}$.
    This comes from the following.
    \begin{ltheorem}{Multiplication by Zero}{Multiplication_by_Zero}
        If $(\mathbb{F},\,+,\,\cdot\,)$ is a field, and if $a\in\mathbb{F}$,
        then $a\cdot{0}=0$.
    \end{ltheorem}
    \begin{proof}
        For we have:
        \begin{equation}
            0=a\cdot(0)-a\cdot(0)=a\cdot(0-0)=a\cdot{0}
        \end{equation}
        This simply combines the distributive law with the additive
        property of zero, completing the proof.
    \end{proof}
    \begin{theorem}
        If $(\mathbb{F},\,+,\,\cdot\,)$ is a field, and if $0=1$, then
        $\mathbb{F}=\{0\}$.
    \end{theorem}
    \begin{proof}
        For suppose not, and let $a\in\mathbb{F}$ be such that $a\ne{0}$.
        But then, by Thm.~\ref{thm:Multiplication_by_Zero}:
        \begin{equation}
            a=a\cdot{1}=a\cdot{0}=0
        \end{equation}
        And thus $a=0$, a contradiction. Therefore,
        $\mathbb{F}$ is trivial.
    \end{proof}
    It is thus common to either call such a field a trivial field, or
    to require that $0\ne{1}$.
    \begin{lexample}{Examples of Fields}{Examples_of_Fields}
        There are several fields that should be familiar to the reader.
        If we let $\mathbb{R}$ denote the real numbers and $+$ and $\cdot$
        be the usual notations of addition and multiplication, then
        $(\mathbb{R},\,+,\,\cdot\,)$ is a field. Similarly, letting
        $\mathbb{Q}$ denote the rational numbers and $\mathbb{C}$ denote
        the complex numbers, $(\mathbb{Q},\,+,\,\cdot\,)$ is a field, as
        is $(\mathbb{C},\,+,\,\cdot\,)$. There are finite fields as well.
        Let $\mathbb{F}_{2}=\{0,\,1\}$ and define multiplication and
        addition as follows:
        \par\hfill\par
        \begin{table}[H]
            \centering
            \captionsetup{type=table}
            \parbox{.45\linewidth}{%
                \centering
                \begin{tabular}{c|cc}
                    $+$&0&1\\
                    \hline
                    0&0&1\\
                    1&1&0
                \end{tabular}
            }
            \parbox{.45\linewidth}{%
                \centering
                \begin{tabular}{c|cc}
                    $\cdot$&0&1\\
                    \hline
                    0&0&0\\
                    1&0&1
                \end{tabular}
            }
            \caption{The Arithmetic of $\mathbb{F}_{2}$}
        \end{table}
        $(\mathbb{F}_{2},\,+,\,\cdot)$ forms a field. Finally, if
        $p\in\mathbb{N}$ is prime, and if $+$ and $\cdot$ are addition
        and multiplication mod $p$, respectively, then
        $(\mathbb{Z}_{p},\,+,\,\cdot\,)$ is a field.
    \end{lexample}
    \begin{fdefinition}{Vector Space}{Vector_Space}
        A vector space over a field $(\mathbb{F},\,+,\,\cdot\,)$ is a
        set $V$ and a function
        $\boldsymbol{\cdot}:\mathbb{F}\times{V}\rightarrow{V}$ and
        a binary operation $\boldsymbol{+}$ on $V$, usuall called
        scalar multiplication and vector addition, respectively, 
        such that for all $\mathbf{x},\mathbf{y},\mathbf{z}\in{V}$,
        and all $a,b\in\mathbf{F}$, the following is true:
        \begin{enumerate}
            \item $\mathbf{x}\boldsymbol{+}%
                   (\mathbf{y}\boldsymbol{+}\mathbf{z})=%
                   (\mathbf{x}\boldsymbol{+}\mathbf{y})%
                   \boldsymbol{+}\mathbf{z}$
                  \hfill[Associative of Vector Addition]
            \item $\mathbf{x}\boldsymbol{+}\mathbf{y}=%
                   \mathbf{y}\boldsymbol{+}\mathbf{x}$
                  \hfill[Commutativity of Vector Addition]
            \item There is a $\mathbf{0}\in{V}$ such that
                  $\mathbf{0}\boldsymbol{+}\mathbf{x}=\mathbf{x}$
                  \hfill[Existence of Zero Vector]
            \item For all $\mathbf{x}$ there is a $\mathbf{y}$ such that
                  $\mathbf{x}\boldsymbol{+}\mathbf{y}=\mathbf{0}$
                  \hfill[Additive Inverses]
            \item $(a\cdot{b})\boldsymbol{\cdot}\mathbf{x}=%
                    a\boldsymbol{\cdot}(b\boldsymbol{\cdot}\mathbf{x})$
                  \hfill[Compatibility of Multiplication]
            \item $(a+b)\boldsymbol{\cdot}\mathbf{x}=%
                   (a\boldsymbol{\cdot}\mathbf{x})\boldsymbol{+}%
                   (b\boldsymbol{\cdot}\mathbf{x})$
                  \hfill[Distributive Law for Field Addition]
            \item $a\boldsymbol{\cdot}(\mathbf{x}\boldsymbol{+}\mathbf{y})=%
                   (a\boldsymbol{\cdot}\mathbf{x})\boldsymbol{+}%
                   (a\boldsymbol{\cdot}\mathbf{y})$
                  \hfill[Distributive Law for Vector Addition]
        \end{enumerate}
    \end{fdefinition}
    It is quite common not to distinguish between scalar multiplication
    $\boldsymbol{\cdot}$ and field multiplication $\cdot$, which may cause
    confusion. It is also common to drop the use of a symbol altogether and
    simply representation multiplication by concatenation of the the
    two variables, for example $a\mathbf{x}$ or $ab$, which represents
    scalar multiplication and field multiplication, respectively.
    \begin{example}
        If we let $\mathbb{F}=\mathbb{R}$ and let
        $V=\mathbb{R}^{n}$, where addition, multiplication, scalar
        multiplication, and vector addition are defined in their usual
        manner, then this forms a vector space. Similarly, the space
        $C([a,b])$ of continuous functions forms a vector space over
        $\mathbb{R}$, as does $L^{2}(\mathbb{R})$, the space of
        square integrable functions.
    \end{example}
    \begin{fdefinition}{Bilinear Operations}{Bilinear_Operations}
        A bilinear operation on a vector space
        $(V,\,\boldsymbol{+},\,\boldsymbol{\cdot}\,)$ over a field
        $(\mathbf{F},\,+,\,\cdot\,)$ is a function
        $[\,]:V\times{V}\rightarrow{V}$ such that, for all
        $\mathbf{x},\mathbf{y},\mathbf{z}\in{V}$, and for all
        $a,b\in\mathbf{F}$, the following is true:
        \begin{enumerate}
            \item $[\mathbf{x}\boldsymbol{+}\mathbf{y}, \mathbf{z}]=%
                   [\mathbf{x},\mathbf{z}]\boldsymbol{+}%
                   [\mathbf{y},\mathbf{z}]$
                  \hfill[Right Distributive Law]
            \item $[\mathbf{x},\mathbf{y}\boldsymbol{+}\mathbf{z}]=%
                   [\mathbf{x},\mathbf{y}]\boldsymbol{+}%
                   [\mathbf{x},\mathbf{z}]$
                  \hfill[Left Distributive Law]
            \item $[a\boldsymbol{\cdot}\mathbf{x},%
                    b\boldsymbol{\cdot}\mathbf{y}]=%
                   (a\cdot{b})\boldsymbol{\cdot}[\mathbf{x},\mathbf{y}]$
                  \hfill[Compatibility with Scalars]
        \end{enumerate}
    \end{fdefinition}
    \begin{lexample}{Examples of Bilinear Operations}
                    {Examples_of_Bilinear_Operation}
        The quintessential example of a bilinear operation is the
        cross product that one encounters in a multivariable calculus
        course. That is, for any three vectors
        $\mathbf{x},\mathbf{y},\mathbf{z}$, we have:
        \begin{equation}
            \mathbf{x}\times(\mathbf{y}+\mathbf{z})=
            \mathbf{x}\times\mathbf{y}+\mathbf{x}\times\mathbf{z}
        \end{equation}
        Similarly for right sided multiplication. The compatibility of
        the cross product with scalar multiplication is also true:
        \begin{equation}
            (a\mathbf{x})\times(b\mathbf{y})=ab(\mathbf{x}\times\mathbf{y})
        \end{equation}
        This serves somewhat as a motivating example for bilinear
        operations. If we think of the field of invertible matrices,
        then multiplication forms a bilinear operation as well, with
        scalar multiplication being the usual entry wise operation that
        is done on matrices. Lastly, if $\langle\,\rangle$ is an inner
        product on $\mathbb{R}$ or $\mathbb{C}$, then this is a bilinear
        operation, the vector space being the underlying field itself.
    \end{lexample}
    \begin{fdefinition}{Algebra over a Field}{Algebra_over_a_Field}
        An algebra of a field $(\mathbf{F},\,+,\,\cdot\,)$ is a
        vector space $(\mathbf{V},\,\boldsymbol{+},\,\boldsymbol{\cdot}\,)$
        and a bilinear operation $[\,]:V\times{V}\rightarrow{V}$.
    \end{fdefinition}
    \begin{fdefinition}{Associative Algebra over a Field}
                       {Associative_Algebra_over_a_Field}
        An associative algebra over a field $(\mathbb{F},\,+,\,\cdot\,)$
        is an algebra $(V,[\,])$ over $\mathbb{F}$ such that, for all
        $r\in\mathbb{F}$ and for all $\mathbf{x},\mathbf{y}\in{V}$,
        the following is true:
        \begin{equation}
            r[\mathbf{x},\,\mathbf{y}]=[r\mathbf{x},\,\mathbf{y}]
                                      =[\mathbf{x},\,r\mathbf{y}]
        \end{equation}
    \end{fdefinition}
    \begin{fdefinition}{Derivation on an Algebra}{Derivation_on_an_Algebra}
        A derivation on an algebra $(V,\,[\,])$ is a function
        $D:V\rightarrow{V}$ such that for all $\mathbf{x},\mathbf{y}\in{V}$,
        the following (Liebniz's Rule) is true:
        \begin{equation}
            D([\mathbf{x},\mathbf{y}])
            =[\mathbf{x},D(\mathbf{y})]+[D(\mathbf{x}),\mathbf{y}]
        \end{equation}
    \end{fdefinition}
    \begin{theorem}
        In a field, $0$ and $1$ are unique.
    \end{theorem}
    \begin{proof}
        For suppose not, and let $0'$ and $1'$ be other identities.
        Then $1'=1'\cdot 1 = 1$ and $0'=0'+0=0$.
    \end{proof}
    \begin{theorem}
        For any field $\langle{F},+,\cdot\rangle$ and $a\in{F}$, $a\cdot{0}=0$.
    \end{theorem}
    \begin{proof}
        For:
        \begin{equation}
            0=a\cdot{0}+(\minus{a}\cdot{0})
             =a\cdot(0+0)+(\minus{a}\cdot{0})
             =a\cdot{0}+a\cdot{0}+(\minus{a}\cdot{0})
             =a\cdot 0
        \end{equation}
        Thus, $a\cdot{0}=0$.
    \end{proof}
    If $1=0$, then $a=a\cdot{1}=a\cdot{0}=0$, and thus every element is
    zero. A very boring field.
    \begin{theorem}
        In a field $\langle F, +,\cdot \rangle$, if $0\ne 1$, then $0$ has no
        inverse.
    \end{theorem}
    \begin{proof}
        For let $a$ be such an inverse. Then $a\cdot{0}=1$. But for any element
        of $F$, $a\cdot{0}=0$. But $0\ne{1}$, a contradiction.
    \end{proof}
    \begin{theorem}
        If $a+b=0$, then $b=(\minus{1})\cdot{a}$ where $(\minus{1})$ is the
        solution to $1+(\minus{1})=0$.
    \end{theorem}
    \begin{proof}
        $a+(\minus{1})a=a(1+(\minus{1}))=a\cdot{0}=0$. From uniqueness,
        $b=(\minus{1})a$. We may thus write additive inverses as $\minus{a}$.
    \end{proof}
    \begin{definition}
        Given two fields $(F,+,\cdot)$ and $(F',+',\times)$, a bijection
        function $f:F\rightarrow{F}'$ is said to be a field isomorphism if and
        only if for allelements $a,b\in{F}$, $f(a+b)=f(a)+'f(b)$, and
        $f(a\cdot{b})=f(a)\times{f}(b)$
    \end{definition}
    \begin{definition}
        $(F,+,\cdot)$ and $(F',+',\times)$, are said to be isomorphic if and
        only if they have an isomorphism.
    \end{definition}
    \begin{theorem}
        Given an ismorphism between two fields $(F,+,\cdot)$ and
        $(F', +',\times)$, $f(1)=1'$ and $f(0)=0'$.
    \end{theorem}
    \begin{proof}
        For let $x\in{F}$. Then $f(x)=f(x\cdot 1)=f(x)\times{f}(1)$, and
        $f(x)=f(x+0)=f(x)+'f(0)$. Therefore, etc.
    \end{proof}
    \begin{theorem}
        In a field $(F,+,\cdot)$, $(a+b)^{2}=a^{2}+2ab+b^{2}$
        ($2$ being the solution to $1+1$).
    \end{theorem}
    \begin{proof}
        For:
        \begin{align}
            (a+b)^{2}&=(a+b)(a+b)\\
                     &=a(a+b)+b(a+b)\\
                     &=a^{2}+ab+ba+b^{2}\\
                     &=a^{2}+ab(1+1)+b^{2}\\
                     &=a^{2}+2ab+b^{2}
        \end{align}
    \end{proof}

    \renewcommand{\PATH}{\OLDPATH}
\endgroup
    %     \part{Unsorted Stuff}
    %         \chapter{Combinatorics}
    \section{Introduction}
        We use the following notation:
        \begin{equation}
            [n]=\mathbb{Z}_{n}=\{1,2,\hdots,n\}
        \end{equation}
        In combinatorics we study functions of the form
        $f:[n]\rightarrow[m]$. These functions can be
        one-to-one, onto, or both. A permutation of the
        elements of $[n]$ is simply a bijection
        $f:[n]\rightarrow[n]$. A partial permutation is a
        permutation of length $k\leq{n}$ of the set
        $[n]$. That is, a permutation on some subset of
        $\mathbb{Z}_{n}$. We all study sets. In particular,
        the power set of $[n]$ and subsets from $k$ chosen
        elements of $[n]$. Another topic of study is that of
        lattice paths on $\mathbb{Z}\times\mathbb{Z}$. That
        it, paths from $(i,j)$ to $(n,m)$ using a prescribed
        set of rules. For example, how many ways can you get
        from $(0,0)$ to $(21,7)$ if you're only allowed to move
        North and East. Restricted paths impose more rules,
        for example the number of moves east must be greater
        than the number of moves north. There are also things
        called Catalan paths and Motzkin paths.
        \par\hfill\par
        Another topic in combinatorics is that of words. An
        alphabet is a set $[n]$, and we wish to study the number
        of words of length $k$ in $[n]$. Binary words are words
        when the alphabet is $\{0,1\}$. There are also words
        with a prescribed number for each letter.
        There are also circular arrangements of the elemnts
        in $[n]$, and the idea of multi-sets. Multi-sets
        are sets that allow for repetition. From elementary
        set theory, we have:
        \begin{equation}
            \{a,b,c\}=\{a,a,b,c\}
        \end{equation}
        Sets are uniquely defined by the elements they contain.
        Multi-sets allow for repetition, and this distringuishes
        two different sets. We write:
        \begin{equation}
            A=\{\{1,1,1,2,3,3\}\}
        \end{equation}
        Note that $A\ne\{\{1,2,3\}\}$. The multiplicty of
        an elements $a\in{A}$ is the number of times the
        element $a$ occurs in the multi-set $A$. To know how
        many multi-sets chosen from $[n]$ with $k$ elements,
        we wish to study the following equation:
        \begin{equation}
            \sum_{i=1}^{n}m_{i}=k
        \end{equation}
        Where $m_{i}$ is the multiplicity of the $i^{th}$
        element of $[n]$. Where wish to find integer solutions
        to this equation, in particular solutions with
        non-negative integers. We can also place restrictions
        on the multi-sets, for example by requiring that
        each element occurs at least once. Thus we'd have:
        \begin{equation}
            \sum_{i=1}^{n}m_{i}=k
            \quad\quad
            m_{i}\geq{1}
        \end{equation}
        A partition of $[n]$ is a collection of sets
        $B_{1},b_{2},\dots,B_{k}$ such that:
        \begin{equation}
            \cup_{i=1}^{k}B_{i}=[n]
            \quad\quad
            B_{i}\cap{B}_{j}=\emptyset
            \quad{i}\ne{j}
        \end{equation}
        The $B_{i}$ are called \textrm{blocks}. We also
        study partitions of numbers. Given $n\geq{0}$, we
        want $\lambda=(\lambda_{1},\dots,\lambda_{\ell})$ such
        that:
        \begin{equation}
            \lambda_{i+1}\leq\lambda_{i},
            \quad\quad
            i=1,2,\dots,\ell-1
        \end{equation}
        And such that:
        \begin{equation}
            \sum_{i=1}{\ell}\lambda_{i}=n
        \end{equation}
        Another commonly studied object is a graph. Labelled
        tress, colorings of graphs, and spanning trees. Finally
        we study tableaux's. These are fillings of arrays
        of boxes with objects, such as numbers, sets, and
        multi-sets.
    \section{Counting Techniques}
        \subsection{Basic Numbers}
            \begin{equation}
                n^{k}=\big|\{f:[k]\rightarrow[n]\}\big|=
                \textrm{The number of words over the alphabet }
                [n]
            \end{equation}
            \begin{equation}
                \binom{n}{k}=
                \frac{n!}{k!(n-k)!}=
                \textrm{The number of ways to choose }k
                \textrm{ objects from }[n]
            \end{equation}
            Stirling's number of the second kind, denoted
            $S(k,n)$, is the number of ways to partition
            $[n]$ into $n$ blocks. $P(n)$ is the number of
            partitions of the integer $n$. And lastly,
            $n!$ is the number of bijections from $[n]$ into
            $[n]$. That is, $n!$ is the number of permutations
            on $[n]$.
        \subsection{Basic Counting Principles}
            Sum Rule (Divide and Conquer): If you cannot
            count the set, divide it into pieces and count
            the pieces.
            \begin{align}
                S&=\bigcup_{i=1}^{k}S_{i}
                \quad\quad
                S_{i}\cap{S}_{j}=\emptyset,
                \quad{i}\ne{j}\\
                |S|&=\sum_{i=1}^{k}|S_{i}|
            \end{align}
            Application: Classify or partition the elements
            in $S$ according to a set of mutually disjoint
            properties $p_{1},\dots,p_{k}$ and let:
            \begin{equation}
                S_{k}=\{x\subseteq{S}:p_{k}(x)\}
            \end{equation}
            We often use such a scheme to prove recurrences.
            For example Pascal's triangle:
            \begin{equation}
                \binom{n}{k}=
                \binom{n-1}{k-1}+\binom{n-1}{k}
            \end{equation}
            We can prove this by plugging in the formula and
            simplifying, but we wish to give a combinatorial
            proof. Let $S\subseteq[n]$, where
            $\Card(S)=k\leq{n}$. Define:
            \begin{align}
                S_{1}&=\{A\subseteq{S}:n\in{A}\}\\
                S_{2}=\{A\subseteq{S}:n\notin{A}\}
            \end{align}
            Then $S_{1}$ and $S_{2}$ partition $S$, and thus
            $\Card(S)=\Card(S_{1})+\Card(S_{2})$. But:
            \begin{align}
                \Card(S_{1})&=\binom{n-1}{k-1}\\
                \Card(S_{2})&=\binom{n-1}{k}
            \end{align}
            This completes the proof. Next is the difference
            rule (Count the opposite):
            \begin{align}
                S&\subseteq\mathcal{U}\\
                \overline{S}&=\mathcal{U}\setminus{S}\\
                \Card(S)&=\Card(\mathcal{U})
                    -\Card(\overline{S})
            \end{align}
            For example, how many permutations on $[n]$ are
            there so that 1 and 2 are not next to each other?
            We count 1 and 2 being together as 12 or 21.
            It's thus easier to think of this as one element.
            So we're counting the number of permutation on
            $[n-1]$ by consider $12$ as one element, and
            again by consider $21$ as one element. This
            gives $2(n-1)!$. By taking the difference, we
            have:
            \begin{equation}
                \Card(S)=n!-2(n-1)!=(n-1)!(n-2)
            \end{equation}
    \section{Posets}
        Let $n\in\mathbb{N}$, $X=2^{[n]}=\mathcal{P}(\mathbb{Z}_{n})$, and
        let consider $(X,\subseteq)$, where $\subseteq$ is the partial ordering
        of inclusion. If $\Card(A)=\Card(B)$, then either $A=B$, or
        $A$ and $B$ are not comparable, and hence $\subseteq$ is a partial
        order.
        \begin{theorem}
            For any $n\geq{1}$, $2^{[n]}$ has a SCD.
        \end{theorem}
        \begin{proof}
            Let $x\in{2}^{[n]}$. Then $x$ has a binary representation. We
            want to create 1-0 pairs as if they were parentheses. For example,
            suppse $x=\{1,5,7\}$. Then $x=0110001$. We match this up to
            $)(()))(\leftrightarrow0(())01$. To get the chain containing
            $x$, we need to describe how to go up and how to go down
            in the chain. To go up, take the right-most unpaired zero
            and change it to a one. To go down, take the left-most one
            and change it to a zero. For the example of $x=\{1,5,6\}$, we have:
            \begin{equation*}
                0110000\rightarrow0110001\rightarrow0110011\rightarrow1110011
            \end{equation*}
            The size of the smallest set in the chain the the number of
            parenthesizations. If this number is $i$, then the largest set
            has size $(n-2i)+i=n-i$. By construction, the chain is saturated.
            At every step we only add one element. Thus, the constructions
            produce symmetric chains. Notice that we never produce new
            1-0 pairings in the algorithm. Thus, all of the sets in the chain
            have the same pairings. So two sets $X$ and $Y$ produce either
            the same chain or disjoint chains. For example, consider
            $2^{[4]}$. The chain is:
            \begin{align*}
                0000\rightarrow0001\rightarrow
                0011\rightarrow0111\rightarrow1111\\
                0010\rightarrow0110\rightarrow1110\\
                0100\rightarrow0101\rightarrow1101\\
                1000\rightarrow1001\rightarrow1011
            \end{align*}
        \end{proof}
        \begin{ltheorem}{Sperner's Theorem}
            Any anti-chain of $2^{[n]}$ elements has at most
            $\binom{n}{\floor{n/2}}$ subsets.
        \end{ltheorem}
        \begin{proof}
            Notice that in the diagram of $2^{[n]}$ each level contains
            $\binom{n}{k}$ elements. An SCD partitions $2^{[n]}$. Let
            $G$ be an $SCD$. with $m$ chains. Then the maximum number of
            incomparable elements is $m$. therwise, two imcomparable elements
            are in the same chain. Thus, $m\geq\binom{n}{\floor{n/2}}$. Also,
            each chain will intersect a level in the graph of the poset at most
            once. Therefore, $m\leq\binom{n}{\floor{n/2}}$. Thus, etc.
        \end{proof}
        The existence of a symmetric chain decomposition gives an elegant
        combinatorial proof that the sequence
        $\binom{n}{k}$, $k=0,1,\dots,n$, is unimodal. To prove unimodality, it
        would suffice to show that, for $k\leq{n/2}$, there exists an
        injection from $k$ subsets to $k+1$ subsets. If we have a SCD, map
        a $k$ subset to it's successor in the chain. This gives the injection.
    \section{Binomial Coefficients and Multi-Sets}
        Recall that a multi-set is similar to a set, except that repetitions
        are allowed. For example, if we consider $[3]$, then a multi-set of
        of this could be:
        \begin{equation}
            \{\{1,1,2,2,3\}\}
        \end{equation}
        This has 5 elements, and is a mutli-set of size 5.
        \begin{theorem}
            The number of $k$ multisets of an $n$ element set is:
            \begin{equation}
                \frac{n^{\overline{k}}}{k!}
                =\frac{n(n+1)\cdots(n+k-1)}{k!}
                =\binom{n+k-1}{k}
            \end{equation}
        \end{theorem}
        \begin{proof}
            For let $S$ be the set of multi-sets of size $k$ of elements
            of an $n$ element set, and let $T$ be subsets of size $k$ in
            $[n+k-1]$. We need to produce a map $f:S\rightarrow[T]$. Let:
            \begin{equation}
                M=\{\{1\leq{a}_{1}\leq{a}_{2}\leq\cdots\leq{a}_{k}\leq{n}\}\}
            \end{equation}
            This maps to the set:
            \begin{equation}
                A=\{1\leq{a}_{1}<a_{2}+1<a_{3}+2<\cdots<a_{k}+k-1\leq{n}+k-1\}
            \end{equation}
            This mapping is reversible. Therefore, etc.
        \end{proof}
        \begin{example}
            Let $n=3$, and $k=5$. Also, define:
            \begin{equation}
                M=\{\{1,1,2,2,3\}\}
            \end{equation}
            Then:
            \begin{equation}
                A=\{1,2,4,5,7\}\subseteq[7]
            \end{equation}
        \end{example}
        Multi-sets can be seen as binary sequences (Stars and bars). For
        example, let $M=\{3,3,4,7,7\}$. We can write this as
        $||**|*|||**$. This helps count out the repetitions of various elements
        in the multi-set. $\binom{n}{k}$ can be seen as the number of
        functions that map $k$ elements to $0$ and $n-k$ elements to $1$.
        We can generalize to functions $[n]\rightarrow[m]$. Let
        $k_{1},k_{2},\dots,x_{m}$ be such that:
        \begin{equation}
            \sum_{i=1}^{m}k_{i}=n
        \end{equation}
        And such that, for all $i$, $k_{i}\geq{0}$. Then
        $\binom{n}{k_{1},\dots,k_{m}}$ is the number of ways to map
        $[n]\rightarrow[m]$ such that $k_{i}$ elements map to $i$, where:
        \begin{align}
            \binom{n}{k_{1},k_{2},\dots,k_{m}}&=
            \binom{n}{k_{1}}\binom{n-k_{1}}{k_{2}}
            \binom{n-k_{1}-k_{2}}{k_{3}}\cdots\binom{k_{m}}{k_{m}}\\
            &=\frac{n!}{k_{1}!k_{2}!\cdots{k}_{m}!}
        \end{align}
        \begin{ltheorem}{Multinomial Theorem}{Multinomial_Theorem}
            \begin{equation}
                (x_{1}+x_{2}+\cdots+x_{m})^{n}=
                \sum_{k_{1}+\cdots+k_{m}=n}
                \binom{n}{k_{1},\dots,k_{m}}x_{1}^{k_{1}}\cdots{x}_{m}^{k_{m}}
            \end{equation}
        \end{ltheorem}
        \subsection{Lattice Paths}
            Let $\mathbb{Z}^{d}$ be an integer lattice of dimension $d$, where
            $d\in\mathbb{N}$ and $d\geq{1}$.
            \begin{ldefinition}{Lattice Path}
                A lattice path in $\mathbb{Z}^{d}$ with $k$ steps in
                $S\subseteq\mathbb{Z}^{d}$ is a subset
                $L\subseteq\mathbb{Z}^{d}$ such that $L=\{v_{1},\dots,v_{k}\}$
                such that, for all $i=1,2,\dots,k-1$, $v_{i+1}-v_{i}\in{S}$.
            \end{ldefinition}
            \begin{example}
                If $d=2$, $S=\{(0,1),(1,0)\}$, then there are 6 paths
                from $(0,0)$ to $(2,2)$.
            \end{example}
            \begin{theorem}
                if $v=(a_{1},\cdots,a_{d})\in\mathbb{Z}^{d}$ and if $e_{i}$ is
                the $i^{th}$ unit vector in $\mathbb{Z}^{d}$, then the number
                of lattice paths in $\mathbb{Z}^{d}$ from the origin to
                $v$ with steps in $\{e_{i}:i\in\mathbb{d}\}$ is given by
                the multinomial coefficient
                $\binom{\norm{v}_{1}}{a_{1},\dots,a_{d}}$.
            \end{theorem}
            \begin{proof}
                For let $v_{0},\cdots,v_{k}$ be a path. Then
                $v_{1}-v_{0},v_{2}-v_{1},\dots,v_{k}-v_{k-1}$ consist of
                the $e_{i}$. Thus there are $a_{1}$ $e_{1}$'s,
                $a_{2}$ $e_{2}$'s, and so on. The total number is thus the
                multinomial coefficient.
            \end{proof}
            \begin{theorem}
                The number of lattice paths from $(0,0)$ to $(n,m)$ with
                steps in $\{(0,1),(1,0)\}$ is $\binom{n+m}{n}$.
            \end{theorem}
        \subsection{The Involution Principle}
            \begin{theorem}
                The number of lattice paths from $(i,j)$ to $(m,n)$ using
                steps $(1,0)$ and $(0,1)$ is $\binom{m-i+n-j}{m-i}$.
            \end{theorem}
            Given a set $S$ and a partition
            $S=S^{+}\cup{S}^{-}$ into a negative part $S^{-}$ and a
            positive part $S^{+}$, then $S$ is called a signed set. We
            are interested in computing $\Card(S^{+})-\Card(S^{-})$.
            \begin{ldefinition}{Sign Reversing Involution}
                A sign reversing involution is an involution
                $\psi:S\rightarrow{S}$ such that for all $x\in{S}$
                such that $\psi(x)\ne{x}$, then
                $\psi(x)\in{S}^{+}$ for all $x\in{S}^{-}$ and
                $\psi(x)\in{S}^{-}$ for all $x\in{S}^{+}$.
            \end{ldefinition}
            \begin{theorem}
                If $\psi$ is a sign reversing involution, if
                $F^{+}$ are the fixed points of $\psi$ in $S^{+}$,
                and $F^{-}$ are the fixed points are $\psi$ in
                $S^{-}$, then:
                \begin{align}
                    \Card(S^{+}\setminus{F}^{+})&=
                    \Card(S^{-}\setminus{F}^{-})\\
                    \Card(S^{+})-\Card(S^{-})&=
                    \Card(F^{+})-\Card(F^{-})
                \end{align}
            \end{theorem}
            Suppose we are given a set $X$ and we want to compute
            $\Card(X)$. Embed $X$ into a signed set $S=S^{+}\cup{S}^{-}$
            such that for all $x\in{X}$ there is a corresponding
            $s\in{S^{+}}$.
            \begin{ldefinition}{Catalan Path}
                A Catalan path is a lattice path from $(0,0)$ to
                $(n,n)$ using steps $(0,1)$ and $(1,0)$ such that the
                path never crosses the line $x=y$.
            \end{ldefinition}
            We are interested in counting the number of Catalan paths
            for a given $n\in\mathbb{N}$. The first few numbers are
            $1,2,5,14,42,\dots$ and occur frequently in mathematics.
            Let $S^{+}$ be the set of paths from $(1,0)$ to $(n+1,n)$
            and $S^{-}$ be the set of paths from $(0,1)$ to $(n+1,n)$.
            Using the previous theorem:
            \begin{align}
                \Card(S^{+})&=\binom{2n}{n}\\
                \Card(S^{-})&=\binom{2n}{n-1}
            \end{align}
            Now we need to embed the Catalan paths into $S^{+}$.
            The embedding comes from shifting the graphs 1 unit to the
            right. Note that the image never touches the line $x=y$.
            Define a sign reversing involution $\psi:S\rightarrow{S}$
            by letting $P$ be any path in $S$ that does not touch
            $x=y$, and defining $\psi(P)=P$. If $P$ touches or crosses
            $x=y$, let $p_{0}$ be the first such crossing. Let
            $\psi(P)$ be the path from $(1,0)$
            (Respectively, from $(0,1)$), such that the points from
            $(0,0)$ to $p_{0}$ are reflected, and the points from
            $p$ to $(n+1,n)$ stay the same. Now $F^{-}$ is empty, since
            given any path from $(1,0)$ to $(n+1,n)$, it must cross
            the line $y=x$. Thus there are no fixed points in $S^{-}$.
            But then:
            \begin{equation}
                \Card(F^{+})=\Card(S^{+})-\Card(S^{-})
            \end{equation}
            But the Catalan number $C_{n}$ is equal to the size of
            $F^{+}$, and thus we have:
            \begin{equation}
                C_{n}=\binom{2n}{n}-\binom{2n}{n-1}
                =\frac{1}{n+1}\binom{2n}{n}
            \end{equation}
        \subsection{Diagonal Lattice Paths}
            \begin{ldefinition}{Diagonal Lattice Paths}
                A diagonal lattice path is a lattice path with steps
                $(1,1)$ and $(1,-1)$.
            \end{ldefinition}
            \begin{lexample}
                Consider all diagonal paths from $(0,0)$ to $(4,0)$.
                Since any step increasing the $x$ coordinate by 1,
                there must be 4 steps in the lattice path. But since
                the path must end at 0, there must be an equal number
                of steps that go up as there are steps that go down.
                So, we must have two up steps and two down steps.
                The total number of diagonal lattice paths is thus
                $\binom{4}{2}=6$. In general, the total number of
                lattice paths from $(0,0)$ to $(2n,0)$ is
                $\binom{2n}{n}$.
            \end{lexample}
            These diagonal lattice paths can be seen as binary words with
            $d=(1,-1)$ and $(u=1,1)$ such that the number of occurences
            of $d$ is equal to the number of occurences of $u$. We can
            establish a correspondence between diagonal lattice paths and
            Catalan paths by considering as the bijection a reflection
            about the $x=y$ axis, and then a rotation by $45^{\circ}$.
            \begin{ldefinition}{Dyck Paths}
                A Dyck is a diagonal lattice path that never goes below
                it's starting point.
            \end{ldefinition}
    \section{q-Analogues}
        In combinatorics, a $q$ analogue of a counting function, such
        as $n!$, is typically a polynomial in $q$ which evaluates to
        the function if we set $q=1$, and if not a polynomial we take
        the limit as $q\rightarrow{1}$. We want the q-analogue to
        preserve the same reccurence properties that the counting
        function has.
        \begin{lexample}
            A q-Analogue of a real number $x\in\mathbb{R}$ could be:
            \begin{equation}
                [x]_{q}=\frac{1-q^{x}}{1-q}
            \end{equation}
            Taking the limit as $q\rightarrow{1}$, we see that this
            expression evaluates to $x$ by using L'H\^{o}pital's Rule.
            If $x=n\in\mathbb{N}$, then:
            \begin{equation}
                \frac{1-q^{n}}{1-q}=1+q+\cdots+q^{n-1}
            \end{equation}
            This allows us to construct a q-Analogue of $n!$:
            \begin{equation}
                [n]_{q}!=[1]_{q}[2]_{q}\cdots[n]_{q}
            \end{equation}
            This can be used to put statistics on sets.
        \end{lexample}
        \begin{ldefinition}{Statistic on a Finite Set}
            A statistic on a finite set $S$ is a function
            $f:S\rightarrow\mathbb{N}_{0}$
        \end{ldefinition}
        Let $S_{n}$ denote the symmetric group, which is the set of
        permutations of $1,2,\dots,n$ under the operation of composition.
        Then:
        \begin{equation}
            \Card(S_{n})=n!
        \end{equation}
        \begin{ldefinition}{Inversion of a Word}
            An inversion of a word $\sigma$ is a pair $(i,j)$, where
            $1\leq{i}<j\leq{n}$, where $\sigma_{i}>\sigma_{j}$.
        \end{ldefinition}
        \begin{lexample}
            Let $\sigma=(132)(45)(6)(7)$. Then $(1,3)$ is an inversion,
            since $\sigma_{1}=3>\sigma_{3}=2$. 
        \end{lexample}
        \begin{ldefinition}{Inversion Statistic}
            The inversion statistic on $S_{n}$ is the number of
            inversions of $\sigma\in{S}_{n}$.
        \end{ldefinition}
        \begin{theorem}
            If $S_{n}$ is the permutation group, then:
            \begin{equation}
                \sum_{\sigma\in{S}_{n}}q^{\textrm{inv}(\sigma)}
                =[n]_{q}!
            \end{equation}
        \end{theorem}
    \section{Lecture 6}
        Last week we introduced q-Analogs, and proved the following
        identities:
        \begin{equation}
            \sum_{\sigma\in{S}_{n}}q^{inv(\sigma)}=
            \sum_{\sigma\in{S}_{n}}q^{maj(\sigma)}
            =[n]_{q}!]
        \end{equation}
        Where $inv(\sigma)$ is the number of inversions, and
        $maj(\sigma)$ is the number of descents. We now want a
        q-Analog of $\binom{n}{k}$. Recall the inversion tables:
        \begin{equation}
            \mathcal{I}_{n}=
                \{(a_{1},\dots,a_{n}):0\leq{a}_{i}\leq{i}\}
        \end{equation}
        We can write this as:
        \begin{equation}
            \mathcal{I}_{n}=
                \{0\}\times\{0,1\}\times\{0,1,2\}\times\cdots
                \times\{0,1,2,\dots,n-1\}
        \end{equation}
        From this we obtain:
        \begin{equation}
            \Card(\mathcal{I}_{n})=n!
        \end{equation}
        We define the function
        $\Psi_{1}:\mathcal{I}_{n}\rightarrow{S}_{n}$ by mapping:
        \begin{equation}
            \Psi_{1}(a_{1},\dots,a_{n})=\sigma
        \end{equation}
        Where $\sigma$ is the permutation with inversions
        $a_{1},\dots,a_{n}$, and $a_{i}$ is the number inversions
        created by $i$. We also define
        $\Psi_{2}:\mathcal{I}_{n}\rightarrow{S}_{n}$ and
        re-interpreted $a_{i}$ to be the contribution of $i$ to the
        major index. Then $\Psi=\Psi_{2}\circ\Psi_{1}^{\minus{1}}$
        is a bijection from $S_{n}$ to itself. We now want to find
        a good q-Analog for $\binom{n}{k}$ that would satisfy
        similar properties as the binomial coefficient. One nice
        property is Pascal's Identity:
        \begin{equation}
            \binom{n}{k}=\binom{n-1}{k-1}+\binom{n-1}{k}
        \end{equation}
        Perhaps the obvious choice is to choose:
        \begin{equation}
            \binom{n}{k}_{q}=
            \frac{[n]_{q}!}{[k]_{q}![n-k]_{q}!}
        \end{equation}
        These are called the Gaussian polynomials, and it seems
        surprising that these are polynomials in the first place,
        since it appears to be a rational function. However, we
        can see just by plugging in that:
        \begin{equation}
            \binom{n+1}{k}_{q}\ne
            \binom{n}{k}_{q}+\binom{n}{k-1}_{q}
        \end{equation}
        And thus this is not a good q-Analog for the binomial
        coefficients. Let:
        \begin{equation}
            \mathcal{R}(1^{k}0^{n-k})=\{
            \textrm{Set of binary words of length $n$ with $k$ 1's}
            \}
        \end{equation}
        Then:
        \begin{equation}
            \Card\Big(\mathcal{R}(1^{k}0^{n-k})\Big)=
            \binom{n}{k}
        \end{equation}
        This is equivalent to saying:
        \begin{equation}
            k!(n-k)!\Card\Big(\mathcal{R}(1^{k}0^{n-k})\Big)
            =n!
        \end{equation}
        But the left-hand side of this equation s the cardinality
        of $S_{k}\times{S}_{n-k}\times\mathcal{R}(1^{k}0^{n-k})$,
        and the right-hand side is the cardinality of
        $S_{n}$. We need to define a function:
        \begin{equation}
            f:S_{k}\times{S}_{n-k}\times\mathcal{R}(1^{k}0^{n-k})
        \end{equation}
        Add $n-k$ to the numbers in the permutation $S_{k}$.
        For example consider:
        \begin{equation}
            (132,14523,10011000)\mapsto
            (687,14523,10011000)
        \end{equation}
        Send the left-most number in order from left to right
        to the 1's in the binary word (The third entry). Send the
        second entry to the 0's in the binary word.
        So, finally we have:
        \begin{equation}
            (132,14523,10011000)\mapsto
            (61487523)
        \end{equation}
        We now want to show that:
        \begin{equation}
            \sum_{r\in\mathcal{R}(1^{k}0^{n-k})}q^{inv(r)}
            =\binom{n}{k}_{q}
            =\frac{[n]_{q}!}{[k]_{q}![n-k]_{q}!}
        \end{equation}
        We can do this in a similar manner as before. We need a
        function $f$ from
        $S_{n-k}\times{S}_{k}\times\mathcal{R}(1^{k}0^{n-k})$ that
        is bijective. Let's use the one defined previously. We
        now need to show that $f$ preserves inversions.
        \begin{theorem}
            \begin{equation}
                \binom{n+1}{k}_{q}=
                q^{k}\binom{n}{k}_{q}+
                \binom{n}{k-1}_{q}
            \end{equation}
        \end{theorem}
        \begin{ltheorem}{Foata's Theorem}{Foatas_Theorem}
            \begin{equation}
                \sum_{r\in\mathcal{R}(1^{k}0^{n-k})}q^{maj(r)}
                =\binom{n}{k}_{q}
            \end{equation}
        \end{ltheorem}
        \subsection{Lattice Paths and Gaussian Polynomials}
            \begin{ldefinition}{Partitions of Integers}
                A partition of $\mathbb{Z}_{n}$, $n\in\mathbb{N}$,
                is a weakly decreasing sequence
                $\lambda=(\lambda_{1},\dots,\lambda_{\ell})$,
                such that:
                \begin{equation}
                    |\lambda|=\sum_{k=1}^{\ell}\lambda_{k}=n
                \end{equation}
                $|\lambda|$ is called the weight of $\lambda$ and
                $\lambda_{i}$ are called the parts of $\lambda$.
                The length of $\lambda$ is the number of non-zero
                parts.
            \end{ldefinition}
            A young diagram is a graphical representation of a
            partition $\lambda=(\lambda_{1},\dots,\lambda_{\ell})$.
            The conjugate of a partition $\lambda$ is obtained by
            transposing the Young diagram of $\lambda$. For example:
            \begin{equation}
                (3,3,1)\mapsto(3,2,2)
            \end{equation}
            We denote the conjugate by $\lambda'$. Recall that
            $\binom{n+m}{n}$ is the number of lattice paths from
            $(0,0)$ to $(n,m)$ using steps $(1,0)$ and $(0,1)$.
            \begin{theorem}
                There exists a bijection between the set of lattice
                paths from $(0,0)$ to $(m,n)$ and the set of
                partitions of such that $\lambda_{1}\leq{m}$ and
                $\ell(\lambda)\leq{n}$.
            \end{theorem}
            If $p(m,n)$ is the number of partitions that fit in
            the $m\times{n}$ rectangle, that is
            $\ell(\lambda)\leq{N}$ and $\lambda_{1}\leq{m}$, then
            $p(m,n)=\binom{n+m}{n}$. A statistic on partitions is
            given by the weight of $\lambda$, $|\lambda|$.
            \begin{theorem}
                For $m,n\in\mathbb{N}$:
                \begin{equation}
                    \binom{n}{m}_{q}=
                    \sum_{\lambda\subseteq[m^{n}]}q^{|\lambda|}
                \end{equation}
            \end{theorem}
            \begin{proof}
                We can show this by proving that the sum satisfies
                the same recurrence relation and initial conditions
                as the q binomial.
            \end{proof}
    \section{Lecture 7 (I Think)}
        We're currently discussing q-analogues. We want to extend
        the q-analogue defined for the factorial function to the
        binomial coefficient. The Gaussian polynomials are one
        such attempt at this:
        \begin{equation}
            \binom{n}{k}_{q}=
            \frac{[n]_{q}!}{[k]_{q}![n-k]_{q}!}
        \end{equation}
        Another such attempt was to sum over all binary words of
        length $n$ with $k$ one's, and obtain:
        \begin{equation}
            \binom{n}{k}_{q}=
            \sum_{r\in\mathcal{R}(1^{k},0^{n-k})}
                q^{inv(r)}
        \end{equation}
        Using this definition, we obtained the following equation:
        \begin{equation}
            \binom{n+1}{k}_{q}=
            q^{k}\binom{n}{k}_{q}+
            \binom{n}{k-1}_{q}
        \end{equation}
        Then we discussed lattice paths $L(m,n)$, which are paths
        from $(0,0)$ to $(m,n)$ using steps in $(1,0)$ and $(0,1)$.
        Next we discuessed partitions of numbers.
        \begin{table}[H]
            \centering
            \captionsetup{type=table}
            \begin{tabular}{|c|c|}
                \hline
                $n$&Partitions\\
                \hline
                0&$\emptyset$\\
                \hline
                1&$(1)$\\
                \hline
                2&$(2),(1,1)$\\
                \hline
                3&$(3),(2,1),)1,1,1)$\\
                \hline
            \end{tabular}
            \caption{Caption}
            \label{tab:my_label}
        \end{table}
        Given such a partition, we assign the weight to be:
        \begin{equation}
            |\lambda|=\sum_{k=1}^{\ell}\lambda_{k}
        \end{equation}
        Then we define:
        \begin{equation}
            P(m,n)=
            \{(\lambda_{1},\dots,\lambda_{\ell}:
                \lambda_{1}\leq{m},\ell(\lambda)\leq{n}\}
        \end{equation}
        We showed that there is a bijection between
        $L(m,n)$ and $P(m,n)$. Thus, we have:
        \begin{equation}
            \Card\Big(P(n,m)\Big)=\binom{m+n}{m}
        \end{equation}
        \begin{theorem}
            If $m,n\in\mathbb{N}$, then:
            \begin{equation}
                \binom{m+n}{m}_{q}=
                \sum_{\lambda\in{P}(m,n)}q^{|\lambda|}
            \end{equation}
        \end{theorem}
        \begin{proof}
            The strategy of the proof is to show that this sum
            satisfies the same initial conditions and the same
            recurrence as the original definition. We have that
            $p(m,0)=1=q^{0}$ since there is only the empty
            partition in the rectangle $m\times{0}$, and similarly
            $p(0,n)=1=q^{0}$ since there is only the empty
            partition in the rectangle $0\times{n}$. Moreover,
            $p(m,m)=1=q^{0}$ since:
            \begin{equation}
                \binom{m}{m}_{q}=
                \frac{[m]_{q}}{[0]_{q}[m]_{q}}=
                \frac{[m]_{q}}{[m]_{q}}=1
            \end{equation}
            We now must show that the recurrence relation is
            satisfied. We want:
            \begin{equation}
                p(m,n)=q^{m}p(n-1,m)+p(n,m-1)
            \end{equation}
            We have that:
            \begin{align}
                p(m,n)=&
                \sum_{\lambda\in{P}(m,n)}q^{|\lambda|}\\
                &=\sum_{\lambda_{1}=m}q^{|\lambda|}+
                \sum_{\lambda_{1}<m}q^{|\lambda|}\\
                &=q^{m}\sum_{\lambda\in{P}(m,n-1)}q^{|\lambda|}
                +\sum_{\lambda\in{P}(m-1,n)}q^{|\lambda|}
            \end{align}
            This completes the proof.
        \end{proof}
        \begin{ldefinition}{$x$ Factorization}
            Let $w\in{X}^{*}$ be a word in the alphabet $X$.
            Let $x\in{X}$ and suppose $w=vy$, where $v$ is a word
            and $y$ is a letter in $X$. That is, $y$ is the last
            letter of $w$. Then the factorization is:
            \begin{equation}
                w=v_{1}y_{1}\dots{v}_{k}y_{k}
            \end{equation}
            Where, if $y>x$ ($X$ is totally ordered):
            \begin{equation}
                y_{i}>x,
                \quad\quad
                y_{i}\in{X}
            \end{equation}
            \begin{equation}
                v_{i}\in{L}_{x}^*
            \end{equation}
            Where:
            \begin{equation}
                L_{x}=\{a:a\leq{x}\}
            \end{equation}
            If $y\leq{x}$, then:
            \begin{equation}
                y_{i}\leq{x}\quad\quad
                y_{i}\in{X}
            \end{equation}
            and:
            \begin{equation}
                v_{i}\in{R}_{x}^{*}
            \end{equation}
            Where:
            \begin{equation}
                R_{x}=\{a:a>x\}
            \end{equation}
        \end{ldefinition}
        \begin{lexample}
            Let $x=3$ and let:
            \begin{equation}
                w=125312641237
            \end{equation}
            Then $y=7$, and thus $y>x$. Then we can write:
            \begin{equation}
                w=|12|5|312|6||4|123|7
            \end{equation}
            We allow for empty words. As another example,
            consider:
            \begin{equation}
                w=135712136412
            \end{equation}
            Then $y=2$, and thus $y<2$. We obtain:
            \begin{equation}
                w=1|3|57|2|1|3|641|2
            \end{equation}
        \end{lexample}
        \begin{ltheorem}{Foata's Theorem}
            The following is true:
            \begin{equation}
                \sum_{r\in\mathcal{R}(1^{k},0^{n-k})}q^{maj(r)}
                =\binom{n}{k}_{q}
            \end{equation}
        \end{ltheorem}
        \begin{proof}
            We want to define a bijection
            $\varphi$ from $\mathcal{R}(1^{k},0^{n-k})$ to itself
            such that, for any $r$, we have $maj(r)=inv(\varphi(r))$.
            Let $X\subseteq\mathbb{N}$. Let $X^{*}$ be the set
            of all words over $X$. For example if $X=\{0,1\}$, then
            $X^{*}$ is the set of all binary words. Define
            $\varphi:X^{*}\rightarrow{X}^{*}$ be such that
            $maj(w)=inv(\varphi(w))$ for any $w\in{X}^{*}$.
            Note that inversions and descents are defined in the
            same way as for permutations. This is why we required
            the set to be totally ordered, $\mathbb{N}$ in our
            case. Define $\gamma_{x}:X^{*}\rightarrow{X}^{*}$ by:
            \begin{equation}
                \gamma_{x}(w)=
                \begin{cases}
                    \emptyset,&w=\emptyset\\
                    y_{1}v_{1}\dots{y}_{k}v_{k},&
                    w=v_{1}y_{1}\dots{v}_{k}y_{k}
                \end{cases}
            \end{equation}
            We define $\varphi$ as follows:
            \begin{equation}
                \varphi(w)=
                \begin{cases}
                    \emptyset,&w=\emptyset\\
                    w,&w\in{X}\\
                    \gamma_{x}(\varphi(v)),&
                    w=vx,x\in{X}
                \end{cases}
            \end{equation}
            This is a recursive definition. For example, let:
            \begin{equation}
                w=121314
            \end{equation}
            Then $\varphi(1)=1$, and thus
            $\varphi(12)=\gamma_{2}(\phi(1))2=12$. The first
            interesting case is with three elements. We have:
            \begin{equation}
                \varphi(121)=
                \gamma_{1}(\varphi(12))1=
                \gamma_{1}(12)1=211
            \end{equation}
            Note that $inv(211)=2$ and $maj(121)=2$. This function
            works since, if $w\in{X}^{*}$, then let
            $r_{x}$ be the number of letters in $w$ that are
            greater than $x$, and let $\ell_{x}$ be the number
            of letters in $w$ that are less than or equal to $x$.
            Let $v\in{X}^{*}$ and $x\in{X}$. Then:
            \begin{equation}
                inv(vx)=inv(v)+r_{x}(v)
            \end{equation}
            Also, when the last letter of $v$ is less than or
            equal to $x$, we have:
            \begin{equation}
                inv(\gamma_{x}(v))=inv(v)-r_{x}(v)
            \end{equation}
            And otherwise we have:
            \begin{equation}
                inv(\gamma_{x}(v))=inv(v)+\ell_{x}(v)
            \end{equation}
            Moreover, if the last letter $v$ is less than or
            equal to $x$, then:
            \begin{equation}
                maj(vx)=maj(v)
            \end{equation}
            And otherwise:
            \begin{equation}
                maj(vx)+|v|
            \end{equation}
        \end{proof}
    \section{Lecture 9}
        As a summary, we we studying q-Analogues. We have shown:
        \begin{equation}
            \sum_{\sigma\in{S}_{n}}q^{inv(\sigma)}=
            \sum_{\sigma\in{S}_{n}}q^{maj(\sigma)}=
            [n]_{q}!
        \end{equation}
        Also:
        \begin{equation}
            \sum_{w\in\mathcal{R}(1^{k},0^{n-k})}
            q^{inv(w)}=
            \sum_{w\in\mathcal{R}(1^{k},0^{n-k})}
            q^{maj(w)}=
            \binom{n}{k}_{q}
        \end{equation}
        \begin{ltheorem}{q-Binomial Theorem}
            The following is true:
            \begin{equation}
                \prod_{i=1}^{n}
                (x+q^{i}y)=
                \sum{q}^{\binom{n-k+1}{2}}
                \binom{n}{k}_{q}x^{k}y^{n-k}
            \end{equation}
        \end{ltheorem}
        For all $n,k\in\mathbb{N}$, we have:
        \begin{equation}
            \binom{n+k}{k}_{q}=
            \sum_{\lambda\in{P}(n,k)}q^{|\lambda|}
        \end{equation}
        We can use rising factorials to define
        $\binom{x}{k}_{q}$ for all $x\in\mathbb{R}$ and
        $k\in\mathbb{N}$. That is:
        \begin{equation}
            \binom{x}{k}_{q}=
            \frac{(1-q^{x-k+1})(1-q^{x-k+1})}{(1-q)(1-q^{2})\dots}
        \end{equation}
        \subsection{q-Catalan Analogue}
            Recall that $C_{n}$ is the number of lattice paths
            from $(0,0)$ to $(n,n)$ that do not go below or
            above the line $x=y$ in the plane. We showed earlier
            that:
            \begin{equation}
                C_{n}=\frac{1}{n+1}\binom{2n}{n}
            \end{equation}
            Recall that $L(m,n)$ is the number of lattice paths
            from $(0,0)$ to $(m,n)$. We define $L^{+}(m,n)$ to be
            the set of lattice paths from $(0,0)$ to $(m,n)$ that
            do not go below the line $y=\frac{n}{m}x$. In
            particular, $L^{+}(n,n)=C_{n}$. The Catalan numbers
            satisfy the following recurrence:
            \begin{equation}
                C_{n}=\sum_{k=1}^{n}C_{k-1}C_{n-k}
            \end{equation}
            Recall that
            $\omega:L(m,n)\rightarrow\mathcal{R}(0^{n},1^{m}$,
            where $\omega$ maps $N\rightarrow{0}$ and
            $E$ to 1, north and east. Let
            $\mathcal{R}^{+}(0^{n}1^{n})$ be the elements f
            $\mathcal{R}(1^{n}0^{n})$ that correspond to
            Catalan paths. From a previous observation, the words
            corresponding to the Catalan paths are characterized
            by alwas have more 0's than 1's for any initial word.
            \begin{ltheorem}{MacMahor's Theorem}
                The following is true:
                \begin{equation}
                    \sum_{p\in{L}^{+}(n,n)}q^{maj(w(p))}=
                    \frac{1}{[n+1]_{q}}\binom{2n}{n}_{q}
                \end{equation}
            \end{ltheorem}
            \begin{proof}
                Define the following:
                \begin{align}
                    \mathcal{R}^{\minus}(0^{n}1^{n})
                        &=\mathcal{R}(0^{n}1^{n})
                        -\mathcal{R}^{+}(0^{n}1^{n})\\
                    L^{\minus}(n,n)=L(n,n)-L^{+}(n,n)
                \end{align}
                Given a path $P$ in $L^{\minus}(n,n)$, let $A$ be
                the lattice point with the smallest $x$ coordinate
                among all the lattice points $(i,j)$ with
                $i-j$ maximized, whose distance from the
                $x=y$ line in the south east direction is maximized.
                Let $B$ be the lattice point just before $A$.
                Notice that the step $B\rightarrow{A}$ must be
                an east step. Create a new path as follows. Change
                the east step to a north step, and then take the
                remaining path from $A$ to $(n,n)$ and shift it
                up one and two the left one. This path ends on
                $(n-1,n+1)$. Let $\varphi$ denote the new path.
                Then:
                \begin{equation}
                    maj(w(\varphi(p)))=maj(w(p))-1
                \end{equation}
                For suppose $B\ne(0,0)$. The the step that goes to
                $B$ must be an east step. For if not, then $A$ does
                not have the smallest $x$ coordinate with maximal
                distance to the line $y=x$. If $B=(0,0)$, then the
                first position goes away. Moreover, the algorithm
                is reversible. For let $P'$ be a lattice path from
                $(0,0)$ to $(n,m)$, and let $A'$ be the point with
                maximal $x$ coordinate such that $i-j$ is maximized.
                This point corresponds to the $B$ in the previous
                path. Therefore:
                \begin{equation}
                    \sum_{w\in\mathcal{R}(1^{n}0^{n})}q^{maj(w)}=
                    \sum_{w\in\mathcal{R}(1^{n+1}0^{n-1})}
                        q^{maj(w)+1}=
                    q\binom{2n}{n+1}
                \end{equation}
            \end{proof}
            There is another q-analogue for $C_{n}$ due to
            Carlitz and Riordan. Let $p\in{L}^{+}(n,n)$ and
            define $a_{i}(p)$ to be the number of complete
            squares between the path and the $x=y$ line in row
            $i$. The number $a_{i}(p)$ is called the length of
            the $i^{th}$ row of $p$ and the sequence
            $(a_{1}(p),\dots,a_{n}(p))$ is called the
            co-area vector of $p$. The co-area statistic on
            $p$ is defined as:
            \begin{equation}
                Coarea(p)=\sum_{i=1}^{n}a_{i}(p)
            \end{equation}
            \begin{ltheorem}{Carlitz-Riordan Theorem}
                The following is true:
                \begin{equation}
                    C_{n}(q)=\sum_{p\in{L}^{+}(n,n)}q^{Coarea(p)}
                \end{equation}
            \end{ltheorem}
        Using the Carlitz-Riordan theorem, we can show the
        following result.
        \begin{theorem}
            The following is true:
            \begin{equation}
                C_{n}(q)=\sum_{k=1}^{n}q^{k-1}
                    C_{k-1}(q)C_{n-k}(q)
            \end{equation}
        \end{theorem}
        If we set $x\mapsto{q}^{i}x$, and $y\mapsto{1}$ in the
        q-binomial theorem, then we obtain:
        \begin{equation}
            (\minus{x};q)=\prod_{k=0}^{n-1}(q^{i}x+1)=
            \sum_{k=0}^{n}q^{\binom{k}{2}}\binom{n}{k}_{q}x^{k}
        \end{equation}
        Using this q-binomial, we get the following.
        \begin{theorem}
            If $h,n,m\in\mathbb{N}$, then:
            \begin{equation}
                \sum_{k=0}^{n}q^{(n-k)(h-k)}\binom{n}{k}_{q}
                \binom{m}{n-k}_{q}=
                \binom{m+n}{h}_{q}
            \end{equation}
        \end{theorem}
    \section{Generating Functions}
        Given a q-analogue and a set $S$, we can then define a
        statistic, $\lambda$. We then have:
        \begin{equation}
            \sum_{s\in{S}}q^{\lambda(s)}=
            \sum_{i=0}^{n}a_{i}q^{i}
        \end{equation}
        Where $a_{i}$ is the number of elements in $S$ with
        statistic value $i$. Thus we can think of a
        statistic $:S\rightarrow\mathbb{N}$, called the
        value function.
        \begin{lexample}
            Let $n\in\mathbb{N}$ and consider
            $S=\mathcal{P}(\mathbb{Z}_{n})$. One easy statistic
            we can place on $S$ is the cardinality function. That is,
            we define $f:S\rightarrow\mathbb{N}$ by:
            \begin{equation}
                f(\omega)=\Card(\omega)
                \quad\quad
                \omega\in\mathcal{P}(\mathbb{Z}_{n})
            \end{equation}
            Let's compute this a different way. Given a set
            $A\subseteq\mathcal{P}(\mathbb{Z}_{n})$, either
            $1\in{A}$ or $1\notin{A}$. Similarly, either
            $2\in{A}$ or $2\notin{A}$. For all $k\in\mathbb{Z}$,
            either $k\in{A}$ or $k\notin{A}$. Thus, we have:
            \begin{equation}
                (q^{1}+q^{0})\cdots(q^{1}+q^{0})=
                \prod_{k=1}^{n}(q^{1}+q^{0})
                =[2]_{q}^{n}
                =\sum_{k=0}^{n}\binom{n}{k}q^{k}
            \end{equation}
        \end{lexample}
        Next we want to consider $S$ being infinite. To get
        a generating function we require that $a_{i}$ is equal
        to the number of elements in $S$ with value $i$ being finite.
        We obtain the following power series:
        \begin{equation}
            a_{0}q^{0}+a_{1}q^{1}+\dots
            =\sum_{i=0}^{\infty}a_{i}q^{i}
        \end{equation}
        Let $\mathbb{C}[[q]]$ denote the ring of formal power
        series. This is a ring. For let:
        \begin{subequations}
            \begin{align}
                A(q)&=\sum_{i=0}^{\infty}a_{i}q^{i}\\
                B(q)&=\sum_{i=0}^{\infty}b_{i}q^{i}
            \end{align}
        \end{subequations}
        Then the sum is well defined, and we have:
        \begin{equation}
            A(q)+B(q)=
            \sum_{i=0}^{\infty}(a_{i}+b_{i})q^{i}
        \end{equation}
        We can also define the product by using the convolution
        product, or Cauchy sums:
        \begin{equation}
            A(q)B(q)=\sum_{i=0}^{\infty}c_{i}q^{i}
        \end{equation}
        Where:
        \begin{equation}
            c_{k}=\sum_{i=0}^{k}a_{i}b_{k-i}
        \end{equation}
        Some properties of $\mathbb{C}[[q]]$ is that it is a
        commutative ring. This is because we considered the
        coefficients to be over $\mathbb{C}$. Moreover, it is an
        integral domain. That is, $\mathbb{C}[[q]]$ has no zero
        divisors. The units, or invertible elements, are formal
        power series such that $a_{o}\ne{0}$. Indeed, this is a
        necessary and sufficient condition for an element to be
        invertible. For example, consider:
        \begin{equation}
            A(q)=\sum_{i=0}^{\infty}q^{i}
        \end{equation}
        This is a geometric sum, and we can show that for
        $|q|<1$, this formal sum is a convergent sum and evaluates
        to $(1-q)^{\minus{1}}$. However, for all formal sums, this
        formal power series has an inverse, and the inverse is
        indeed $(1-q)^{\minus{1}}$. Ivan Niven has a nice article
        on $\mathbb{C}[qq]]$ in the American Mathematical Monthly,
        1969. This has applications in counting partitions of
        numbers. See Andrews Theory of Partitions. Let $p(n)$ be
        the number of partitions on $n$. Then:
        \begin{equation}
            \sum_{n=0}^{\infty}p(n)q^{n}=
            1+q+2q^{2}+3q^{3}+5q^{4}+7q^{5}+11q^{6}+15q^{7}+\dots
        \end{equation}
        The value function is thus the weight of the partion
        $|\lambda|=\lambda_{1}+\dots+\lambda_{m}$.
        \begin{ltheorem}{Euler's Thoemre}
            If $P$ is the set of partitions, then:
            \begin{equation}
                \sum_{\lambda\in{P}}q^{|\lambda|}=
                \prod_{k=1}^{\infty}\frac{1}{q-q^{i}}
            \end{equation}
        \end{ltheorem}
        Give an arbitrary partition $\lambda$, consider the parts
        of size one. $\lambda$ does not have a part of size 1 or
        $\lambda$ has a part of size one, or $\lambda$ has a part of
        size two, and so on. Now do the same for each of the parts
        of size $k$, in general.
        \begin{ltheorem}{Euler's Other Theorem}
            The number of partitions with $n$ distinct parts
            is equal to the number of partitions of $n$
            with only odd parts.
        \end{ltheorem}
        \begin{proof}
            For:
            \begin{equation}
                \sum_{n=0}^{\infty}a_{n}q^{n}=
                \prod_{i=1}^{\infty}(1+q^{i})=
                \prod_{i=1}^{\infty}\frac{(q+q^{i})(1-q^{i})}{1-q^{i}}
            \end{equation}
            We can then simplify:
        \end{proof}
        \begin{ldefinition}{Durfee Square}
            The largest square that fits into a partition
            $\lambda$ is called the Durfee square of $\lambda$.
        \end{ldefinition}
        \begin{ldefinition}{Self-Conjugate Partition}
            A self-conjugate partition is a partition
            $\lambda$ such that $\lambda=\lambda'$, where
            $\lambda'$ is the conjugate of $\lambda$.
        \end{ldefinition}
        \begin{ltheorem}{Euler's Other-Other Theorem}
            The following is true:
            \begin{equation}
                \sum_{\lambda=\lambda'}q^{|\lambda|}=
                \prod_{n=0}^{\infty}(1+q^{2n+1})
            \end{equation}
            Where $\lambda=\lambda'$ are all of the self-conjugate
            partitions.
        \end{ltheorem}
        The product is the generating function for partitions with
        distinct odd parts. Euler's theorem then says that the
        generating function for this set is the equal to the sum
        over all of the self-conjugate partitions.
    \section{Euler's Theorem}
        \begin{equation}
            \sum_{n\in\mathbb{N}}p(n)q^{n}=
            \prod_{i=1}^{\infty}\frac{1}{1-q^{i}}
        \end{equation}
        Where $p(n)$ is the number of partitations of $n$.
        We also define the Euler function, not to be confused
        with the Euler totient function, as:
        \begin{equation}
            \phi(q)=\prod_{i=1}^{\infty}(1-q^{i})
        \end{equation}
        Let's try to simplify this:
        \begin{equation}
            \phi(q)=\prod_{i=1}^{\infty}(1-q^{i})=
            \sum_{k=0}^{\infty}b_{k}q^{k}
        \end{equation}
        We want to find the $b_{k}$. Multiplying through by
        the original series from Euler's theorem, we get:
        \begin{equation}
            \Big(\sum_{i=1}^{\infty}b_{i}q^{i}\Big)
            \Big(\sum_{j=0}^{\infty}p(j)q^{j}\Big)=1
        \end{equation}
        Using the convolution product, we have for all $k\geq{1}$:
        \begin{equation}
            \sum_{j=0}^{k}b_{j}p(k-j)=0
        \end{equation}
        This gives a recursion for $p(k)$. This gives us
        Euler's Pentagonal Number Theorem.
        \begin{ltheorem}{Euler's Pentagonal Number Theorem}
              {Euler_Pentagonal_Number_Theorem}
            \begin{subequations}
                \begin{align}
                    \phi(q)&=\prod_{i=1}^{\infty}(1-q^{i})\\
                    &=1+\sum_{m=1}^{\infty}(\minus{1})^{m}
                    \Big(q^{\frac{m(3m-2)}{2}}+
                        q^{\frac{m(3m+1)}{2}}\Big)\\
                    &=\sum_{m=\minus\infty}^{\infty}
                        (\minus{1})^{m}q^{\frac{m(3m-1)}{2}}
                \end{align}
            \end{subequations}
        \end{ltheorem}
        Let $p_{e}(d,n)$ denote the number of partitions $n$ with
        distinct parts and even length. Similarly, define
        $p_{o}(d,x)$ for odd length.
        \begin{theorem}
            \begin{equation}
                p_{e}(d,n)-p_{o}(d,n)=
                \begin{cases}
                    (\minus{1})^{n},&n=\frac{m(3m\pm{1})}{2}\\
                    0,&\textrm{Otherwise}
                \end{cases}
            \end{equation}
        \end{theorem}
        After some reflection, it should be easy to see that
        the first case is the inverse of the second case. Moreover,
        cases 1, 2, and 3 cover all partitions with distinct
        parts. For any partition in case three only one of
        $a$ or $b$ is true. The bijection constructed proves
        Euler's Pentagonal Theorem. Now we can compute $p$:
        \begin{subequations}
            \begin{align}
                p(0)&=1\\
                p(1)&=1\\
                p(2)&=2\\
                p(3)&=3\\
                p(4)&=p(3)+p(2)=5\\
                p(5)&=p(4)+p(3)-p(0)=7\\
                p(6)&=p(5)+p(4)-p(1)=11\\
                p(7)&=p(6)+p(5)-p(2)-p(0)=15
            \end{align}
        \end{subequations}
        And in general:
        \begin{equation}
            p(n)=p(n-1)+p(n-2)-p(n-5)-p(n-7)+\dots
        \end{equation}
        Where we add and subtrack over the pentagonal numbers.
        Gauss then turned to the question of computing powers
        of $\phi(q)$.
        \begin{ltheorem}{Gauss's Pentagonal Theorem}
              {Gauss_Pentagonal_Theorem}
            \begin{equation}
                \phi(q)^{3}=\prod_{i=1}^{\infty}(1-q^{i})^{3}
                =\sum_{r=0}^{\infty}(\minus{1})^{r}(2r+1)
                    q^{\frac{r(r+1)}{2}}
            \end{equation}
        \end{ltheorem}
        This identity occurs in many different areas of mathematics,
        such as homological algebra, complex analysis, and
        hyperbolic geometry. The proof comes from jacobi's
        Triple Product Identity.
        \begin{ltheorem}{Jacobi's Triple Product Identity}
              {Jacobi_Triple_Product_Identity}
            \begin{equation}
                \sum_{n=\minus\infty}^{\infty}z^{n}q^{n^{2}}=
                    \prod_{n=0}^{\infty}(1-q^{2n+2})
                        (1+zq^{2n+1})(1+z^{\minus{1}}q^{2n+1})
            \end{equation}
        \end{ltheorem}
        The proof of Gauss' identity then uses Sylvester's
        bijection. To get this from Jacobi, do a shift of
        index starting from $n=0$ to $n=1$. Differentiate both
        sides with respect to $q$, and then put
        $z=\minus{q}$. Finally, map $q^{2}$ to $q$.
        \par\hfill\par
        Felix Klein computed $\phi(q)^{8}$. In the theory of
        modular forms there is something called the $\tau$
        function, due to Ramanujan. This has the property that:
        \begin{equation}
            \sum_{n=1}^{\infty}\tau(n)q^{n-1}=
            \phi(q)^{24}=\prod_{m=1}^{\infty}(1-q^{m})^{24}
        \end{equation}
        Freeman Dyson also had some contributions to this subject
        and came up with nice formula for $\phi(q)^{d}$ when
        $d=3,8,10,14,15,21,24,26,28,35,36,\dots$ With the
        exception of $26$, these are the dimensions of the Lie
        algebras. Ian McDonald, working on the same problem, saw
        this as well. He came up with the following:
        \begin{equation}
            \phi(q)^{n^{2}-1}=\sum\varepsilon(k_{1},\dots,k_{n})
                \prod_{i=1}^{n}\binom{k_{i}}{n-i}q^{k_{n}}
        \end{equation}
        Where this is summed over all tuples $(k_{1},\dots,k_{n})$
        of non-negative integers such that:
        \begin{equation}
            \sum_{i=1}^{n}k_{i}^{2}=\sum_{i=1}^{k}k_{i}+
            \sum_{i=1}^{n-1}k_{i}k_{i+1}
            +k_{n}k_{1}
        \end{equation}
        And where $\varepsilon(k_{1},\dots,k_{m})=\pm{1}$.
    \section{Generating Function for Multisets}
        Recall that a multiset is a colection with repetition.
        For example:
        \begin{equation}
            A=\{\{1,1,1,3,3,4\}\}
        \end{equation}
        This is different from the set
        $B=\{1,1,1,3,3,4\}$, since sets cannot account for
        repetition. That is, $B$ can be reduced down to
        $B=\{1,3,4\}$. Note that partitions are multisets and
        we have shown that $\binom{k+(n-1)}{k}$ is the
        number of multi-sets of size $k$ chosen from the
        set $[n]$. We want to compute the generating function
        for the number of multi-sets of size $k$:
        \begin{equation}
            f(M)=\sum_{M}q^{|M|}
        \end{equation}
        Where $M$ is a multi-set of elements in $[n]$, and
        $|M|$ denotes the number of elements in $M$, with
        repetitions included. If $M$ is an arbitrary multi-set,
        then either $1\ne{M}$, or $1\in{M}$, or $1,1\in{M}$,
        and so on. So, in general, we get:
        \begin{equation}
            \sum_{k=0}^{\infty}q^{k}=\frac{1}{1-q}
        \end{equation}
        In general:
        \begin{equation}
            \sum_{M}q^{|M|}=\frac{1}{(1-q)^{n}}
            =\sum_{k=0}^{\infty}\binom{n-1+k}{k}q^{k}
        \end{equation}
        From the binomial theorem, we get:
        \begin{equation}
            (1+q)^{n}=
            \sum_{k=0}^{n}\binom{n}{k}q^{k}
        \end{equation}
        And thus, we can define:
        \begin{equation}
            \binom{\minus{n}}{k}=
            \binom{n-1+k}{k}(\minus{1})^{k}
        \end{equation}
        Extending the binomial coefficient to all
        $n\in\mathbb{Z}$. Next, recall the q-binomial theorem.
        \begin{equation}
            \sum_{k=0}^{n}\binom{n}{k}_{q}q^{\binom{k}{2}}x^{k}
            =\prod_{k=1}^{n-1}(1+q^{k}x)
        \end{equation}
        And also:
        \begin{equation}
            \sum_{k=0}^{\infty}\binom{n+k}{k}_{q}x^{k}=
            \prod_{i=0}^{n}(1-q^{i}x)^{\minus{1}}
        \end{equation}
        See Stanley and MacDonald.
    \section{Symmetric Functions}
        \subsection{Symmetric Polynomials}
            Let $x_{1},\dots,x_{N}$ commute, and let
            $f\in\mathbb{C}[x_{1},\dots,x_{N}]$. Then $f$ is
            symmetric if, for all $\sigma\in{S}_{N}$, then:
            \begin{equation}
            f(x_{1},\dots,x_{N})
            =f(x_{\sigma(1)},\dots,x_{\sigma(N)})
            \end{equation}
            Where $S_{N}$ is the symmetric group, and
            $\sigma$ is any permutation. It is convenient to
            work with infinitely many variables. We impose the
            requirements that there are countable many variables,
            so that we may list them, and that the commute. In
            this case we have power series instead of polynomials.
            We thus get sums of the form:
            \begin{equation}
                f=\sum_{\alpha}C_{\alpha}x^{\alpha}
            \end{equation}
            Where $\alpha$ is a sequence of non-negative integers.
            We require that the sum over $\alpha$ be finite, and
            thus this implies that all of the monomials are of
            finite degree. $f$ is said to be homogeneous if all
            of the monomials have the same degree. We require that
            $f$ be invariant under any permutation
            $\sigma:\mathbb{N}\rightarrow\mathbb{N}$.
            \begin{ldefinition}{Monomial Symmetric Basis}
                A monomial symmetric basis is:
                \begin{equation}
                    m_{\lambda}=m_{\lambda}(x)+
                    \sum_{\alpha}x^{\alpha}
                \end{equation}
                Where $\alpha$ is a rearrangement of $\lambda$.
            \end{ldefinition}
            That is, $m_{\lambda}$ is the sum of all monomials
            in $x_{i}$ whose exponents are the parts of $\lambda$.
            \begin{example}
                \begin{subequations}
                    \begin{align}
                        m_{1,1}=\sum_{i<j}x_{i}x_{j}&=
                            x_{1}x_{2}+x_{1}x_{3}+\dots
                            +x_{2}x_{3}+\dots\\
                        m_{2,1,1}(x_{1},x_{2},x_{3})&=
                            x_{1}^{2}x_{2}x_{2}+x_{1}x_{2}^{2}x_{3}+
                                x_{1}x_{2}x_{3}^{2}\\
                        m_{2}(x)=\sum_{i=1}^{\infty}x_{i}^{2}
                    \end{align}
                \end{subequations}
            \end{example}
            \begin{ldefinition}{Elementary Symmetric Function}
                The elementary symmetric function is defined as:
                \begin{equation}
                    e_{k}=m_{1^{k}}=
                    \sum_{i_{1}<i_{2}<\dots<i_{k}}
                        x_{i_{1}}x_{i_{2}}\cdots{x}_{i_{k}}
                \end{equation}
                For $k\in\mathbb{N}$.
            \end{ldefinition}
            Note that:
            \begin{equation}
                \prod_{i=1}^{\infty}(1+zx_{i})=
                \sum_{n=0}^{\infty}e_{n}z^{n}
            \end{equation}
            \begin{ldefinition}{Power Symmetric Function}
                The power symmetric function is defined as:
                \begin{equation}
                    P_{k}=m_{k}=
                    \sum_{i=1}^{\infty}x_{i}^{k}
                \end{equation}
                For $k\in\mathbb{N}$. That is, $k\geq{1}$.
            \end{ldefinition}
            \begin{ldefinition}
                  {Complete Homogeneous Symmetric Function}
                The complete homogeneous symmetric function is
                defined as:
                \begin{equation}
                    h_{k}=\sum_{\lambda+k}m_{\lambda}
                \end{equation}
                That is, the sum of all monomials of degree $k$.
            \end{ldefinition}
            \begin{theorem}
                The generating function for the homogeneous
                symmetric function is:
                \begin{equation}
                    \prod_{i=1}^{\infty}\frac{1}{1-zx_{i}}=
                    \sum_{n=0}^{\infty}h_{n}z^{n}
                \end{equation}
                Where $h_{0}$ is defined as $h_{0}=1$.
            \end{theorem}
            An endomorphism is a function such that
            $w(fg)=w(f)w(g)$, $w(f+g)=w(f)+w(g)$, and
            $w(cf)=cw(f)$.
            \begin{ldefinition}{$\omega$ Involution}
                The $\omega$ involution is the endomorphism defined
                by:
                \begin{equation}
                    \omega(p_{k})=(\minus{1})^{k=1}P_{k}
                \end{equation}
                And:
                \begin{equation}
                    \omega(p_{\lambda})=
                    (\minus{1})^{n-\ell(\lambda)}P_{\lambda}
                \end{equation}
            \end{ldefinition}
    %         \chapter{Discrete Structure}
    \section{Set Theory}
        A set is a collection of objects. The objects in a
        set are called the elements of the set. The emptyset
        $\emptyset$ is the set containing no elements. A set
        is finite if there is a bijection between it
        and $\mathbb{Z}_{n}=\{1,2,\hdots,n\}$ for some
        $n\in\mathbb{N}$. A set is infinite if it is
        not finite and non-empty.
        A set is countably infinite if there is a bijection
        between it and $\mathbb{N}$. A set is uncountably
        infinite if it is infinite and not
        countably infinite. Sets can be represented by
        Venn diagrams, which are essentially blobs in
        the plane. Two sets are equal if and only if they
        contain exactly the same elements. A subset $A$ of
        a set $B$, denoted $A\subset{B}$, is a set such that
        for all $x\in{A}$, $x\in{B}$ as well. That is, every
        element of $A$ is also an element of $B$. For any
        set $A$, $A\subset{A}$. A proper subset is a susbet
        $A\subset{B}$ such that $A\ne{B}$. That is, there is
        an $x\in{B}$ such that $x\notin{A}$. Disjoint sets
        are sets that have no common elements.
        \begin{theorem}
            If $A\subset{B}$ and $B\subset{C}$,
            then $A\subset{C}$.
        \end{theorem}
        \begin{theorem}
            If $A\subset{B}$ and $B\subset{A}$, then
            $A=B$.
        \end{theorem}
        \begin{theorem}
            If $A$ is a set, then $\emptyset\subset{A}$.
        \end{theorem}
        The universe set $U$ is the set of all objects under
        consideration. If $A$ is a subset of a universe $U$,
        then the complement of $A$, denoted $A^{C}$, is the
        set of all elements in $U$ that are NOT contained
        in $A$. This is also known as set difference:
        $A^{C}=U\setminus{A}$. The power set of a set $A$,
        denoted $\mathcal{P}(A)$ is the set of all subsets
        of $A$.
        \begin{definition}
            A partial ordering on a set $A$ is a
            relation $\leq$ such that if
            $A\leq{B}$ and $B\leq{C}$, then
            $A\leq{C}$, and if $A\leq{B}$ and
            $B\leq{A}$, then $A=B$.
        \end{definition}
        Set inclusion is a partial ordering on the power
        set of a set. The union of two sets $A$ and $B$
        is the set $A\cup{B}$ containing all of the elements
        of $A$ and all of the elements of $B$.
        The intersection of $A$ and $B$
        is the set $A\cap{B}$ containing only the elements
        that are in both $A$ and $B$.
        \begin{theorem}
            The following are true:
            \begin{enumerate}
                \begin{multicols}{3}
                    \item $A\cup{A}=A$
                    \item $A\cup{B}=B\cup{A}$
                    \item $A\cup\emptyset=A$
                    \item $(A\cup{B})\cup{C}=A\cup(B\cup{C})$
                    \item $A\cup{A}=A$
                    \item $A\subset{B}\Rightarrow{A=A\cup{B}}$
                    \item $A\subset{A\cup{B}}$
                    \item $A\cap{B}=B\cap{A}$
                    \item $A\cap\emptyset=\emptyset$
                    \item $(A\cap{B})\cap{C}=A\cap(B\cap{C})$
                    \item $A\cap{B}\subset{A}$
                    \item $A\subset{B}\Rightarrow{A=A\cap{B}}$
                \end{multicols}
            \end{enumerate}
        \end{theorem}
        \begin{theorem}
            If $U$ is a universe set, and $A\subset{U}$, then:
            \begin{enumerate}
                \begin{multicols}{5}
                    \item $\emptyset^{C}=U$
                    \item $U^{C}=\emptyset$
                    \item $(A^{C})^{C}=A$
                    \item $A\cup{A^{C}}=U$
                    \item $A\cap{A^{C}}=\emptyset$
                \end{multicols}
            \end{enumerate}
        \end{theorem}
        \begin{theorem}[DeMorgan's Theorem]
            If $U$ is a universe and $A,B\subset{U}$,
            then:
            \begin{enumerate}
                \begin{multicols}{2}
                    \item $(A\cup{B})^{C}=A^{C}\cap{B^{C}}$
                    \item $(A\cap{B})^{C}=A^{C}\cup{B^{C}}$
                \end{multicols}
            \end{enumerate}
        \end{theorem}
        \begin{theorem}[Distributive Laws]
            If $A$ and $B$ are sets, then:
            \begin{enumerate}
                \begin{multicols}{2}
                    \item $A\cup(B\cap{C})=(A\cup{B})\cap(A\cup{C})$
                    \item $A\cap(B\cup{C})=(A\cap{B})\cup(A\cap{C})$
                \end{multicols}
            \end{enumerate}
        \end{theorem}
        \begin{definition}
            The difference of two sets $A$ and $B$
            is: $A\setminus{B}=\{x\in{A}:x\notin{B}\}$
        \end{definition}
        \begin{theorem}
            If $U$ is a universe, and $A,B\subset{U}$,
            then $A\setminus{B}=A\cap{B^{C}}$
        \end{theorem}
        \begin{definition}
            The symmetric difference of two set
            $A$ and $B$ is
            $A\oplus{B}=(A\cup{B})\setminus(A\cap{B})$
        \end{definition}
        The symmetric difference is the set of all elements
        that are in either $A$ or in $B$, but not contained
        in both. A function or a mapping from a set $A$ to
        a set $B$ is a rule which assigns to every element
        $a\in{A}$ a unique element $b\in{B}$. An injective
        mapping, or a one-to-one mapping, is a function
        $f:A\rightarrow{B}$ such that
        $f(a_{1})=f(a_{2})$ if and only if $a_{1}=a_{2}$.
        A surjective mapping, or a correspondence, or an
        onto mapping, is a function $f:A\rightarrow{B}$
        such that for all $b\in{B}$, there is an
        $a\in{A}$ such that $f(a)=b$. A bijection is a
        function $f:A\rightarrow{B}$ that is both
        injective and surjective. Two sets are said to
        have the same size, or the same cardinality,
        if there is a bijection between them.
        \begin{theorem}
            The rational numbers $\mathbb{Q}$ are countable.
            That is, there is a bijection
            $f:\mathbb{N}\rightarrow\mathbb{Q}$.
        \end{theorem}
        \begin{theorem}
            The real numbers $\mathbb{R}$ are uncountable.
        \end{theorem}
    \section{Combinatorics}
        If one event can happen $n$ ways, and another independent
        event can happen $m$ ways, then the total number
        of possibilities is $nm$. If there are 5 entrees
        and 3 sides at a restaurant, then there are
        15 total possible meals.
        \begin{definition}
            The factor of a positive integer
            $n$, denoted $n!$, is
            $n!=n\cdot(n-1)\cdots{2}\cdot{1}$
            We define $0!=1$.
        \end{definition}
        The factorial of a number $n$ is the number
        of ways to permute $n$ objects. The number of ways
        to permute $k$ objects from a set of $n$ is
        $P(n,k)=n!/(n-k)!$, and the number of ways to
        choose $k$ objects from a set of $n$ objects
        (Taking the order into account) is the
        binomial coefficient $\binom{n}{k}$. The number
        of permutations of $n$ objects into groups
        $n_{1},n_{2},\hdots,n_{N}$, where
        $n_{1}+n_{2}+\hdots+n_{N}=n$, is
        $n!/(n_{1}!n_{2}!\hdots{n_{N}!})$
        Stirling's Approximation says that, for large
        $n$, the factorial can be approximated as follows:
        \begin{equation*}
            n!\approx\sqrt{2\pi{n}}n^{n}e^{-n}
        \end{equation*}
        \begin{example}
            What is the probability of rolling
            four 4's out six tosses of a six sided
            dice? The number of ways to choose
            two numbers that are not 4 is
            $\binom{6}{2}=15$. The probability of
            an event with four 4's and two numbers
            that aren't 4 is
            $(\frac{1}{6})^{4}(\frac{5}{6})^{2}$.
            So the probability of rolling four 4's is
            $15(\frac{1}{6})^{4}(\frac{5}{6})^{2}=0.008$.
        \end{example}
    \section{Exams}
    \subsection{Practice Exam I}
    \begin{problem}
    Let $S = \{1,2,3,4,5,6,7,8\}$.
    \begin{enumerate}
        \item How many subsets of $S$ are there with exactly three elements?
        \item How many subsets of $S$ contain exactly one even number and two odd numbers?
        \item How many subsets of $S$ contain exactly three elements, at most one of which is even?
        \item How many subsets of $S$ are there with exactly three elements satisfying the condition that the subset contains the numbers $1$ or $8$ (Or both)?
        \item How many subsets of $S$ are there with exactly three elements which satisfy the property that two of its elements sum to $9$?
    \end{enumerate}
    \end{problem}
    \begin{proof}[Solution]
    \
    \begin{enumerate}
    \begin{multicols}{3}
        \item $\binom{8}{3} = \frac{8!}{3!(8-3)!} = 56$
        \item $\binom{4}{1}\binom{4}{2} = 24$
        \item $\binom{4}{1} \binom{4}{2} + \binom{4}{0} \binom{4}{3} = 28$.
    \end{multicols}
        \item $\binom{7}{2}$ contain $1$, $\binom{7}{2}$ contain $8$, and $\binom{6}{1}$ contain both. $\binom{7}{2}+\binom{7}{2}-\binom{6}{1}=36$
        \item $8+1=7+2=6+3=5+4=9$. $4$ pairs with $\binom{6}{1}$ subsets per pair. We have $4\cdot 6 = 24$.
    \end{enumerate}
    \end{proof}
    \begin{problem}
    A club has $20$ members.
    \begin{enumerate}
        \item In how many different ways can the club select a president, vice-president, and secretary?
        \item In how many different ways can a social committee with four members be elected?
        \item The club contains 12 men and 8 women. In how many different ways can a committee of four people be selected if the committee must have two men and two women?
        \item In how many different ways can a four person committee be selected from the club members if one person is designated as the leader of the committee?
    \end{enumerate}
    \end{problem}
    \begin{proof}[Solution]
    \vspace{-\topsep}
    \
    \begin{enumerate}
    \begin{multicols}{2}
        \item $P(20,3) = \frac{20!}{(20-3)!} = 6840$
        \item $\binom{20}{4} = \frac{20!}{4!(20-4)!} = 4845$
        \item $\binom{12}{2}\binom{8}{2} = 1848$
        \item $\binom{19}{3} = 969$
    \end{multicols}
    \end{enumerate}
    \end{proof}
    \clearpage
    \begin{problem}
    Let $A = \{1,2\}$, $B = \{2,4,5\}$, and $U = \{1,2,3,4,5,6,7\}$. Compute the following:
    \begin{enumerate}
    \begin{multicols}{4}
        \item $A\times B$
        \item $A^3$
        \item $\mathcal{P}(B)$.
        \item $A\setminus B$
        \item $A \oplus B$
        \item $A\cap B^c$
        \item $|\{(x,y)\in U^2:x\ne y\}|$
        \item $A\cup B$.
    \end{multicols}
    \end{enumerate}
    \end{problem}
    \begin{proof}
    \vspace{-\topsep}
    \
    \begin{enumerate}
        \item $\{(1,2),(1,4),(1,5),(2,2),(2,4),(2,5)\}$
        \item $\{(1,1,1),(1,1,2),(1,2,1),(1,2,2),(2,1,1),(2,1,2),(2,2,1),(2,2,2)\}$
        \item $\big\{\emptyset,\{2\},\{4\},\{5\},\{2,4\},\{2,5\},\{4,5\},\{2,4,5\}\big\}$
    \begin{multicols}{5}
        \item $\{1\}$
        \item $\{1,4,5\}$
        \item $\{1\}$
        \item $7^2-7 = 42$.
        \item $\{1,2,4,5\}$
    \end{multicols}
    \end{enumerate}
    \end{proof}
    \begin{problem}
    Calculate the following:
    \begin{enumerate}
    \begin{multicols}{3}
        \item $\sum_{i=0}^{2}\sum_{j=1}^{3}(3i-j)$
        \item $\prod_{i=1}^{n} \frac{2i}{i+1}$
        \item $\cup_{i=1}^{n} \{i,i+1\}$
    \end{multicols}
    \end{enumerate}
    \end{problem}
    \begin{proof}[Solution]
    \vspace{-\topsep}
    \
    \begin{enumerate}
        \item $\sum_{i=0}^{2}\sum_{j=1}^{3}(3i-j) = \sum_{i=0}^{2}\big((3i-1)+(3i-2)+(3i-3)\big) = \sum_{i=0}^{2}\big(9i-6\big) = -6+3+12 = 9$.
        \item $\prod_{i=1}^{n} \frac{2i}{i+1}= \frac{2^n}{n+1}$. We prove by induction. The base case is trivial. Suppose it is true for $n\in \mathbb{N}$. Then $\prod_{i=1}^{n+1} \frac{2i}{i+1} = \frac{2(n+1)}{(n+1)+1}\prod_{i=1}^{n}\frac{2i}{i+1}=\frac{2n}{(n+1)+1}\frac{2^n}{n+1} = \frac{2^{n+1}}{(n+1)+n}$. Therefore $\prod_{i=1}^{n+1} \frac{2i}{i+1} = \frac{2^{n+1}}{(n+1)+1}$.
        \item $\cup_{i=1}^{n} A_i = \mathbb{Z}_{n+1}$. We prove by induction. The base case is trivial. Suppose it is true for $n\in \mathbb{N}$. Then $\cup_{i=1}^{n+1}A_i = \big(\cup_{i=1}^{n}A_{i}\big) \cup A_{n+1}\mathbb{Z}_{n+1}\cup\{n+1,n+2\}=\mathbb{Z}_{n+2}$. Thus $\cup_{i=1}^{n+1} A_{i} = \mathbb{Z}_{n+2}$.
    \end{enumerate}
    \end{proof}
    \begin{problem}
    \
    \begin{enumerate}
    \begin{multicols}{2}
        \item Compute the binary representation of $75$.
        \item Convert $1100101_2$ to decimal.
    \end{multicols}
    \end{enumerate}
    \end{problem}
    \begin{proof}[Solution]
    \vspace{-\topsep}
    \
    \begin{enumerate}
        \item $75=2^{6}+2^{3}+2^{1}+2^{0}=1001011_{2}$
        \item $1100101_{2}=2^{0}+2^{2}+2^{5}+2^{6}=1+4+32+64=101$
    \end{enumerate}
    \end{proof}
    \begin{problem}
    A woman has eight friends. Answer the following:
    \begin{enumerate}
        \item In how many different ways can she invite four of her friends to dinner?
        \item Two of her friends dislike each other. If she invites one friend, she can't invite the other. How many ways can she invite her friends?
        \item Five are her friends are men and three are women. How many ways can she invite four of her friends if she wants two men and two women.
        \item The eight friends consist of two married couples and four single people. How many ways can she invite four friends under the condition that if she invites one spouse she must invite the other?
    \end{enumerate}
    \end{problem}
    \begin{proof}[Solution]
    \vspace{-\topsep}
    \
    \begin{enumerate}
    \begin{multicols}{4}
        \item $\binom{8}{4} = 70$
        \item $2\binom{6}{3}+\binom{6}{4}=55$
        \item $\binom{5}{2}\binom{3}{2} = 30$
        \item $\binom{4}{0}+\binom{4}{2}+\binom{4}{2}+\binom{4}{4}=14$
    \end{multicols}
    \end{enumerate}
    \end{proof}
    \begin{problem}
    Expand $(2x-5y)^4$.
    \end{problem}
    \begin{proof}[Solution]
    \vspace{-0.5\topsep}
    $(2x-5y)^4=\sum_{k=0}^{4}\binom{n}{k}(2x)^{4-k}(-5y)^{k}=16x^4-160x^3y+600x^2y^2-1000xy^3+625y^4$
    \end{proof}
    \begin{problem}
    Find the coefficient of $x^3y^6$ in $(x-10y)^9$.
    \end{problem}
    \begin{proof}[Solution]
    \vspace{-0.5\topsep}
    $(x-10y)^9 = \sum_{k=0}^{9} \binom{9}{k}x^{n-k}(-10y)^{k}$. For $k=6$ we have $\binom{9}{6}(-10)^6 = 84,000,000$
    \end{proof}
    \clearpage
    \begin{problem}
    A bit string is a sequence of numbers consisting of $0's$ and $1's$. Answer the following:
    \begin{enumerate}
        \item How many strings of length $5$ are there?
        \item How many strings of length $6$ have an even number of $1$'s?
        \item How many strings of length 5 begin with $0$ or end with $1$ (Or both)?
        \item How many strings of length $6$ contain exactly three $1$'s?
        \item How many strings of length $6$ contain at least three $1$'s?
        \item How many strings of length $6$ are palindromic (Same from left to right as from right to left)?
    \end{enumerate}
    \end{problem}
    \begin{proof}[Solution]
    \vspace{-\topsep}
    \
    \begin{enumerate}
    \begin{multicols}{3}
        \item $2^{5}=32$.
        \item $\binom{6}{0}+\binom{6}{2}+\binom{6}{4}+\binom{6}{6}=32$
        \item $2^{3}+2^{3}+2^{3}=24$
        \item $\binom{6}{3} = 20$
        \item $\binom{6}{3}+\binom{6}{4}+\binom{6}{5}+\binom{6}{6} = 42$
        \item $2^{3}=8$
    \end{multicols}
    \end{enumerate}
    \end{proof}
    \begin{problem}
    Let $A$ and $B$ be be sets, $|A\cup B| = 50$, $|A| = 37$, and $|A\cap B| = 20$. Calculate:
    \begin{enumerate}
    \begin{multicols}{4}
        \item $|B|$
        \item $|A\setminus B|$
        \item $|B\setminus A|$
        \item $|A\oplus B|$
    \end{multicols}
    \end{enumerate}
    \end{problem}
    \begin{proof}[Solution]
    \vspace{-\topsep}
    \
    \begin{enumerate}
    \begin{multicols}{2}
        \item $|B| = |A\cup B|-|A|+|A\cap B| = 33$.
        \item $|A\setminus B| = |A|-|A\cap B| = 17$.
        \item $|B\setminus A| = |B| - |A\cap B| = 13$.
        \item $|A\oplus B| = |A\cup B|- |A\cap B| = 30$.
    \end{multicols}
    \end{enumerate}
    \end{proof}
    \subsection{Exam I}
    \begin{problem}
    Let $A = \{1,3,5\}$, $B = \{3,5,6,7\}$, $C = \{2,7\}$, $U = \{1,2,3,4,5,6,7,8\}$. Calculate:
    \begin{enumerate}
    \begin{multicols}{4}
        \item $\mathcal{P}(C)$
        \item $A^c \setminus C$
        \item $A\oplus B$
        \item $A\cup B$
    \end{multicols}
    \end{enumerate}
    \end{problem}
    \begin{proof}[Solution]
    \vspace{-\topsep}
    \
    \begin{enumerate}
    \begin{multicols}{4}
        \item $\big\{\emptyset, \{2\}, \{7\}, \{2,7\}\big\}$
        \item $\{4,6,8\}$
        \item $\{1,6,7\}$
        \item $\{1,3,5,6,7\}$
    \end{multicols}
    \end{enumerate}
    \end{proof}
    \begin{problem}
    A deck of cards contains $28$ cards. Each card is either red, yellow, green, or blue, and there are seven cards for each color labelled with integers $1$ to $7$. Answer the following:
    \begin{enumerate}
        \item How many ways can a hand of four cards be selected such that each card is a different color.
        \item How many ways can four cards be selected so that no cards cards are the same color or number.
        \item How many ways can a hand of four cards be selected so there is one red and three yellow cards. 
        \item How many ways can four cards be drawn such that all cards are the same color?
    \end{enumerate}
    \end{problem}
    \begin{proof}[Solution]
    \vspace{-\topsep}
    \
    \begin{enumerate}
    \begin{multicols}{4}
        \item $\frac{28 \cdot 21 \cdot 14 \cdot 7}{4!} = 2401$
        \item $\frac{28\cdot 18 \cdot 10 \cdot 4}{4!} = 840$
        \item $\binom{7}{1}\binom{7}{3} = 245$
        \item $4\binom{7}{4} = 140$
    \end{multicols}
    \end{enumerate}
    \end{proof}
    \begin{problem}
    A true false quiz has $6$ questions. Answer the following:
    \begin{enumerate}
        \item How many ways are there to fill out the quiz so that there are exactly four true answer?
        \item How many different ways are there to fill out the quiz so that there are at most two false answers?
    \end{enumerate}
    \end{problem}
    \begin{proof}[Solution]
    \vspace{-\topsep}
    \
    \begin{enumerate}
    \begin{multicols}{2}
        \item $\binom{6}{2} = \frac{6!}{2!(6-4)!}=15$
        \item $\binom{6}{4}+\binom{6}{5}+\binom{6}{6} = 22$
    \end{multicols}
    \end{enumerate}
    \end{proof}
    \begin{problem}
    Find the coefficient of $x^2y^3$ in $(7x-10y)^5$.
    \end{problem}
    \begin{proof}[Solution]
    \vspace{-0.5\topsep}
    $(7x-10y)^5 = \sum_{k=0}^{5}\binom{5}{k}(7x)^{5-k}(-10y)^{k}$. For $k = 3$, $\binom{5}{3}7^2(-10)^3=-490,000$.
    \end{proof}
    \clearpage
    \begin{problem}
    Let $A,B$ be finite sets such that $|A\cap B| =10$, $|A| = 22$, and $|B| = 15$. Calculate:
    \begin{enumerate}
    \begin{multicols}{4}
        \item $|A\cup B|$
        \item $|A\setminus B|$
        \item $|A\oplus B|$
        \item $|A\times B|$
    \end{multicols}
    \end{enumerate}
    \end{problem}
    \begin{proof}[Solution]
    \vspace{-\topsep}
    \
    \begin{enumerate}
    \begin{multicols}{2}
        \item $|A\cup B| = |A|+|B|-|A\cap B| = 27$
        \item $|A\setminus B| = |A|-|A\cap B| = 12$
        \item $|A\oplus B| = |A\cup B|-|A\cap B| = 17$
        \item $|A\times B| = |A||B| = 330$.
    \end{multicols}
    \end{enumerate}
    \end{proof}
    \begin{problem}
    Expand and simplify $\sum_{i=0}^{3}\sum_{j=1}^{2}(2x^i-x^j)$
    \end{problem}
    \begin{proof}[Solution]
    \vspace{-0.5\topsep}
    $\sum_{i=0}^{3}\sum_{j=1}^{2}(2x^i-x^j)=\sum_{i=0}^{3}(4x^{i}-x^{1}-x^{2})=4+4x^{1}+4x^{2}+4x^{3}-4x^{1}-4x^{2}=4+4x^3$
    \end{proof}
    \begin{problem}
    Find a formula for $|A\cup B\cup C|$, where $A,B,C$ are finite sets.
    \end{problem}
    \begin{proof}[Solution]
    \vspace{-0.5\topsep}
    $|A\cup B\cup C| = |A|+|B|+|C|-|A\cap B|-|A\cap C|-|B\cap C| +|A\cap B \cap C|$
    \end{proof}
    \subsection{Practice Exam II}
    \begin{problem}
    \label{problem:discrete_structures_contrapos_of_a+b_less_than_1}
    Write the contrapositive of: If $a<\frac{1}{2}$ and $b< \frac{1}{2}$, then $a+b<1$. 
    \end{problem}
    \begin{proof}[Solution]
    \vspace{-0.5\topsep}
    If $a+b \geq 1$, then $a\geq \frac{1}{2}$ or $b\geq \frac{1}{2}$.
    \end{proof}
    \begin{remark}
    The contrapositive is logically equivalent to the original statement.
    \end{remark}
    \begin{problem}
    Is the converse of the statement in problem \ref{problem:discrete_structures_contrapos_of_a+b_less_than_1} true?
    \end{problem}
    \begin{proof}[Solution]
    \vspace{-0.5\topsep}
    The converse is: If $a+b<1$, then $a<\frac{1}{2}$ and $b<\frac{1}{2}$. This is false: $2-2=0<1$, but $2\not<\frac{1}{2}$
    \end{proof}
    \begin{problem}
    \label{problem:discrete_structures_practice_exam_2_problem_3}
    Make a true table for $p,q,p\land q, p\lor q, (p\land q)\lor (\neg p\land q)$
    \end{problem}
    \begin{proof}[Solution]
    \vspace{-\topsep}
    \
    \begin{table}[H]
        \centering
        \captionsetup{type=table}
        \begin{tabular}{c c c c c} 
            \hline
            $p$ & $q$ & $p\land q$ & $p\lor q$ & $(p\land q)\lor(\neg p\land q)$ \\ [0.5ex] 
            \hline
            0 & 0 & 0 & 0 & 0\\ 
            0 & 1 & 0 & 1 & 1\\
            1 & 0 & 0 & 1 & 0\\
            1 & 1 & 1 & 1 & 1\\
            \hline
        \end{tabular}
        \caption{Truth Table for Problem \ref{problem:discrete_structures_practice_exam_2_problem_3}}
        \label{tab:discrete_structures_practice_exam_2_problem_3}
    \end{table}
    \end{proof}
    \begin{problem}
    Prove that $(p\land q)\lor(\neg p\land q)$ is equivalent to $q$.
    \end{problem}
    \begin{proof}[Solution]
    \vspace{-0.5\topsep}
    By the distributive law, $(p\land q)\lor(\neg p\land q) \Leftrightarrow (p\lor \neg p)\land q \Leftrightarrow 1\land q \Leftrightarrow q$.
    \end{proof}
    \begin{problem}
    Prove that if $n$ is an integer, then $n^2+6n$ is even if and only if $n$ is even.
    \end{problem}
    \begin{proof}[Solution]
    \vspace{-0.5\topsep}
    If $n$ is even, then $n=2k$ for some integer $k$. Thus $n^2+6n = 4k^2+12k = 2\big(2k^2+6k)$, so $n^2+6n$ is even. If $n^2+6n$ is even $n$ is odd, then $n=2k+1$ for some integer $k$. Thus $n^2+6n = 2\big(k^2+8k)+1$, which is odd. A contradiction. Therefore $n$ is even.
    \end{proof}
    \begin{problem}
    Prove by contradiction that if $7n^2+2n + 1$ is even, then $n$ odd.
    \end{problem}
    \begin{proof}
    \vspace{-0.5\topsep}
    If $n$ is even, then $n=2k$ for some integer $k$. Thus $7n^2+2n+1 = 28k^2+14k+1 = 2(14k^2+7k)+1$, which is odd, a contradiction. Therefore $n$ is odd.
    \end{proof}
    \begin{problem}
    Prove the following: $\sum_{k=1}^{n} k^2 = \frac{n(n+1)(2n+1)}{6}$
    \end{problem}
    \begin{proof}[Solution]
    \vspace{-0.5\topsep}
    We prove by induction. The base case is true. Suppose it is true for some $n\in\mathbb{N}$. Then $\sum_{k=1}^{n+1}k^2=(n+1)^{2}+\sum_{k=1}^{n}k^{2}= (n+1)^2+\frac{n(n+1)(2n+1)}{6}=\frac{(n+1)(n+2)(2n+3)}{6}=\frac{(n+1)((n+1)+1)(2(n+1)+1)}{6}$
    \end{proof}
    \begin{problem}
    Does $\neg p\land (p\rightarrow q)$ imply $\neg q$?
    \end{problem}
    \begin{proof}[Solution]
    \vspace{-0.5\topsep}
    No, it does not.
    \end{proof}
    \clearpage
    \begin{problem}
    Express the following in English. Determine whether it is true or false.
    \begin{enumerate}
    \begin{multicols}{4}
        \item $\exists_{x\in \mathbb{Z}}\scriptsize{\textrm{($x$ is an even prime)}}$
        \item $\forall_{n\in \mathbb{P}}(n^2>0)$
        \item $\forall_{x\in \mathbb{P}}\exists_{y\in \mathbb{Q}}(xy=1)$
        \item $\exists_{y\in \mathbb{Q}}\forall_{x\in \mathbb{P}}(xy=1)$
    \end{multicols}
    \end{enumerate}
    \end{problem}
    \begin{proof}[Solution]
    \vspace{-\topsep}
    \
    \begin{enumerate}
        \item There exists an integer $x$ such that $x$ is even and prime. This is true for $2$ is such an integer.
        \item For all positive integers $n$, $n^2>0$. This is true of all real numbers.
        \item For all positive integers $x$ there is a rational number $y$ such that $xy=1$. This is true for let $y=\frac{1}{x}$.
        \item There exists a rational number $y$ such that for all positive integers $x$, $xy=1$. This is false. For if $x_1 y = 1$ and $x_2 y = 1$, we have $(x_1-x_2)y = 0$. But $x_1 \ne x_2$, so $y=0$. A contradiction as $x_1y=1$.
    \end{enumerate}
    \end{proof}
    \begin{problem}
    Let $U = \mathcal{P}(\{1,2,3,4,5\})$, and consider the following propositions over $U$:
    \begin{align*}
        p(A):A\cap\{1,2,3\} &=\emptyset & q(A):A&\subset\{1,2,5\} & r(A):|A| &=2
    \end{align*}
    Find the truth sets of the following $p,r, q\land r, q\lor p$.
    \end{problem}
    \begin{proof}[Solution]
    \vspace{-\topsep}
    \
    \begin{enumerate}
        \item $T_{p} = \mathcal{P}(\{4,5\})$
        \item $T_{r} = \{\{1,2\},\{1,3\},\{1,4\},\{1,5\},\{2,3\},\{2,4\},\{2,5\},\{3,4\},\{3,5\},\{4,5\}\}$.
        \item $T_{q\land r} = \{1,3\},\{1,5\},\{3,5\}$.
        \item $T_{q\lor p} = \{\emptyset, \{1\},\{3\},\{5\},\{1,3\},\{1,5\},\{3,5\},\{4,\},\{4,5\},\{1,3,5\}$.
    \end{enumerate}
    \end{proof}
    \begin{problem}
    Consider the following propositions:
    \begin{align*}
        p(n)&:n^{2}=100 & q(n)&:2n-20=0 & r(n)&:n^{4}=10,000
    \end{align*}
    \begin{enumerate}
        \item Suppose $p,q,r$ are over $\mathbb{R}$. Which are equivalent? Which propositions imply another?
        \item Suppose $p,q,r$ are over $\mathbb{C}$. Which are equivalent? Which propositions imply another?
    \end{enumerate}
    \end{problem}
    \begin{proof}[Solution]
    \vspace{-\topsep}
    \
    \begin{enumerate}
        \item Over $\mathbb{R}$, we have $T_{p} = \{-10,10\}$, $T_{q} = \{10\}$, and $T_{r} = \{-10,10\}$. So $p$ and $r$ are equivalent since $T_{p} = T_{r}$. Also $q\Rightarrow p$ and $q\Rightarrow r$ since $T_{q} \subset T_{p}$ and $T_{q}\subset T_{r}$.
        \item Over $\mathbb{C}$, we have $T_{p} = \{-10,10\}$, $T_{q} = \{10\}$, $T_{r} = \{-10i,10i,-10,10\}$. So none of the propositions are equivalent, however $q\Rightarrow p \Rightarrow r$.
    \end{enumerate}
    \end{proof}
    \begin{problem}
    \label{problem:discrete structures_make_a_truth_table_for_p_and_1_and_more}
    Make a truth table for $p\land 1$, $p\rightarrow p\lor q$, $p\lor q \rightarrow q$, $p\lor 0 \leftrightarrow \neg p$. Determine which are tautologies, contradictions, or neither.
    \end{problem}
    \begin{proof}[Solution]
    $p\rightarrow p\lor q$ is a tautology, $p\lor 0 \leftrightarrow \neg p$ is a contradiction, both $p\land 1$ and $p\lor q \rightarrow q$ are neither. 
    \begin{table}[H]
        \centering
        \captionsetup{type=table}
        \begin{tabular}{c c c c c c c c c c c c c} 
             \hline 
             $p$ & $q$ & $\neg p$ & $\neg q$ & $0$ & $1$ & $p\lor q$ & $p \lor 0$ & $p\land 1$ & $p\rightarrow p\lor q$ & $p\lor q \rightarrow q$ & $p\lor 0 \leftrightarrow \neg p$.\\ [0.5ex] 
             \hline
             0 & 0 & 1 & 1 & 0 & 1 & 0 & 0 & 0 & 1 & 1 & 0 \\
             1 & 0 & 0 & 1 & 0 & 1 & 1 & 1 & 1 & 1 & 0 & 0 \\
             0 & 1 & 1 & 0 & 0 & 1 & 1 & 0 & 0 & 1 & 1 & 0 \\
             1 & 1 & 0 & 0 & 0 & 1 & 1 & 1 & 1 & 1 & 1 & 0 \\
             \hline
        \end{tabular}
        \caption{Truth Table for Problem \ref{problem:discrete structures_make_a_truth_table_for_p_and_1_and_more}}
    \end{table}
    \end{proof}
    \clearpage
    \begin{problem}
    Consider the following propositions over $\mathbb{Z}$:
    \begin{align*}
        p(n)&:n\textrm{ is a square.} & q(n)&:n\textrm{ is even.} & r(n)&:n\textrm{ is divisible by 4.} & s(n)&:n<0
    \end{align*}
    \begin{enumerate}
        \item Express ``Every integer that is an even square is divisible by 4" symbolically.
        \item Write $\neg(\exists_{n\in \mathbb{Z}}(p\land s))$ in English.
        \item Rewrite $2$ using the $\forall$ symbol.
    \end{enumerate}
    \end{problem}
    \begin{proof}[Solution]
    \vspace{-\topsep}
    \
    \begin{enumerate}
    \begin{multicols}{2}
        \item $\forall_{n\in \mathbb{Z}}(p\land q \rightarrow r)$
        \item There is no negative square integer.
        \item $\forall_{n\in \mathbb{R}}(\neg(p\land s))$
        \end{multicols}
    \end{enumerate}
    \end{proof}
    \subsection{Exam II}
    \begin{problem}
    Let $n$ be an integer. Prove the $n$ is odd if and only if $n^2+2n+5$ is even.
    \end{problem}
    \begin{proof}[Solution]
    \vspace{-0.5\topsep}
    Suppose $n$ is odd. Then there is a $k\in \mathbb{Z}$ such that $n = 2k+1$. So $n^2+2n+5 = (2k+1)^2+2(2k+1)+5 = 4k^2+4k+1+4k+2+5 = 4k^2+8k+8 = 2(2k^2+4k+4)$, which is even. Thus if $n$ is odd, then $n^2+2n+5$ is even. Let $n^2+2n+5$ be even and suppose $n$ is even. Then $n = 2k$. But then $n^2+2n+5 = 4k^2+4k+5 = 2(2k^2+2k+2)+1$. But this is an odd number, a contradiction. Therefore, $n$ is not an even number.
    \end{proof}
    \begin{problem}
    Prove that for all $n\in \mathbb{N}$, $\sum_{k=1}^{n} 3k(k+1) = n(n+1)(n+1)$.
    \end{problem}
    \begin{proof}[Solution]
    \vspace{-0.5\topsep}
    By induction. The base case is trivial. Suppose it is true for $n\in \mathbb{N}$. Then $\sum_{k=1}^{n+1}3k(k+1)=3(n+1)(n+2)+\sum_{k=1}^{n}3k(k+1)=3(n+1)(n+2)+n(n+1)(n+2)=(n+3)(n+1)(n+2)=(n+1)((n+1)+1)((n+1)+2)$.
    \end{proof}
    \begin{problem}
    \label{problem:discrete_structures_exam_2_p_and_q_iff_not_p_or_not_q}
    Consider $(p\land q) \leftrightarrow (\neg q \lor \neg p)$.
    \begin{enumerate}
        \item Write down the truth table for $(p\land q)\leftrightarrow (\neg q \lor \neg p)$ and $\neg(q\lor p)$
        \item Determine whether this proposition is a tautology, contradiction, or neither.
        \item Is $\neg(q\lor p)$ equivalent to $\neg q \lor \neg p$
    \end{enumerate}
    \end{problem}
    \begin{proof}[Solution]
    \vspace{-\topsep}
    The Proposition is a tautology. Also, $\neg q \lor \neg p$ is not equivalent to $\neg(q \lor p)$.
    \begin{table}[H]
        \centering
        \captionsetup{type=table}
        \begin{tabular}{c c c c c c c c c} 
             \hline
             $p$ & $q$ & $p\land q$ & $p \lor q$ & $\neg q$ & $\neg p$ & $\neg q\lor \neg p$ & $\neg(q\lor p)$ & $(p\land q)\leftrightarrow(\neg q \lor \neg p)$ \\ [0.5ex] 
             \hline
             0 & 0 & 0 & 0 & 1 & 1  &   1 & 1  & 0	\\ 
             0 & 1 & 0 & 1 & 0 & 1  &   1 & 0  & 0	\\
             1 & 0 & 0 & 1 & 1 & 0  &   1 & 0  & 0	\\
             1 & 1 & 1 & 1 & 0 & 0  &	0 & 0  & 0	\\
             \hline
        \end{tabular}
        \caption{Truth Table for Problem \ref{problem:discrete_structures_exam_2_p_and_q_iff_not_p_or_not_q}}
        \label{tab:discrete_structures_Exam_II_Problem_3}
    \end{table}
    \end{proof}
    \begin{problem}
    Let $n$ be an integer.
    \begin{enumerate}
        \item Write the contrapositive of : If $n^2 \geq 1$, then $n\geq 1$. Is this true?
        \item Write the converse of: If $n^2 \geq 1$, then $n\geq 1$. Is this true?
    \end{enumerate}
    \end{problem}
    \begin{proof}[Solution]
    \vspace{-\topsep}
    \
    \begin{enumerate}
        \item If $n < 1$, then $n^2 <1$. This is false. If $n = -5$, then $n<1$, yet $n^2 = 25 >1$.
        \item If $n\geq 1$, then $n^2 \geq 1$. This is true, for is $n\geq 1$, then squaring both sides we get $n^2 \geq 1^2 = 1$.
    \end{enumerate}
    \end{proof}
    \clearpage
    \begin{problem}
    Let $U$ be a set with $n$ elements. How many different propositions of $U$ can you list without listing two that are equivalent?
    \end{problem}
    \begin{proof}[Solution]
    \vspace{-0.5\topsep}
    Two propositions $p$ and $q$ are equivalent if $T_p = T_q$. There are $|\mathcal{P}(U)| = 2^n$ possible truth sets. So there are $2^n$ non-equivalent propositions over $U$.
    \end{proof}
    \begin{problem}
    Consider the following propositions of $\mathbb{N}$.
    \begin{equation*}
        p(n): 0\leq n \leq 5 \quad\quad\quad q(n): n\textrm{ is a prime greater than 7} \quad\quad\quad r(n): n^2\leq 9 \quad\quad\quad s(m,n): 2m+n = 4
    \end{equation*}
    \begin{enumerate}
        \item Translate the following statement into English: $\forall_{n\in \mathbb{Z}}\exists_{m\in \mathbb{Z}}(s(m,n))$
        \item Translate the following symbolically: There exists an integer $n$ so that $n^2>9$ or $0\leq n \leq 5$.
        \item The Down the truth sets for $p(n),q(n),r(n)$.
        \item Does one of the propositions imply another?
        \item Is the statement in part $1$ true?
    \end{enumerate}
    \end{problem}
    \begin{proof}[Solution]
    \vspace{-\topsep}
    \
    \begin{enumerate}
        \item For all integers $n$ there exists an integer $m$ so that $2m+n = 4$. 
        \item $\exists_{n\in \mathbb{Z}}(\neg r(n)\lor p(n))$
        \item   \begin{enumerate}
                    \item $T_{p} = \{0,1,2,3,4,5\}$
                    \item $T_{q} = \{2,3,5\}$
                    \item $\{,3,2,1,0,1,2,3\}$
                \end{enumerate}
        \item Yes, $q(n)\Rightarrow P(n)$ because $T_{q}\subset T_{p}$.
        \item False. If $n = 1$ then $2m+1$ is odd for all $n\in \mathbb{Z}$, yet $4$ is even.
    \end{enumerate}
    \end{proof}
    \subsection{Practice Exam III}
    \begin{problem}
    Let $A= \{1,2,3\}$, $B = \{3,4,5,6\}$, and $C = \{1,2,4,6\}$. Let the universe set be $U = \{1,2,3,4,5,6,7\}$. 
    \begin{enumerate}
        \item Find all minsets generated by $A,B,C$.
        \item Show that the nonempty minsets form a partition of $U$.
        \item How many different sets in $\mathcal{P}(U)$ can be generated by $A,B,C$ via any combination of union, intersection, and complement?
        \item Express $\{1,2,3,5,7\}$ and $\{5,6,7\}$ in minset normal form, if possible. If not, explain why.
    \end{enumerate}
    \end{problem}
    \begin{proof}[Solution]
    The non-empty sets form a partition of $U$ for their union is $U$, and there are no overaps. There are $5$ non-empty minsets, so $2^{5} = 32$ subsets of $\mathcal{P}(U)$ can be generated from the minsets. $\{1,2,3,5,7\} = (A^c\cap B^c \cap C^c)\cup (A^c\cap B\cap C^c)\cup (A\cap B^c\cap C)\cup (A\cap B\cap C^c)$. $\{5,6,7\}$ cannot be expressed as a union of minsets, for $6$ lies in the minset $\{4,6\}$, and thus $4$ would need to be included, but it is not.
    \begin{table}[H]
        \centering
        \captionsetup{type=table}
        \begin{tabular}{c c c c c}
            \hline
            $A$ & $B$ & $C$ & &  \\ [0.5ex] 
            \hline
            $0$ & $0$ & $0$ & $A^c\cap B^c\cap C^c$ & $\{7\}$\\
            $0$ & $0$ & $1$ & $A^c\cap B^c\cap C$ & $\emptyset$\\
            $0$ & $1$ & $0$ & $A^c\cap B\cap C^c$ & $\{5\}$\\
            $0$ & $1$ & $1$ & $A^c\cap B \cap C$ & $\{4,6\}$\\
            $1$ & $0$ & $0$ & $A\cap B^c\cap C^c$ & $\emptyset$\\
            $1$ & $0$ & $1$ & $A\cap B^c\cap C$ & $\{1,2\}$\\
            $1$ & $1$ & $0$ & $A\cap B\cap C^c$ & $\{3\}$\\
            $1$ & $1$ & $1$ & $A\cap B \cap C$ & $\emptyset$\\
            \hline
        \end{tabular}
        \caption{The Minsets of $A,B,$ and $C$.}
        \label{tab:discrete_structures_practice_exam_III_Problem_1}
    \end{table}
    \end{proof}
    \clearpage
    \begin{problem}
    Let $A$ and $B$ be subsets of $U$. 
    \begin{enumerate}
        \item List all minsets generated by $A, B$.
        \item Express the sets $A$ and $A^c\cup B$ in minset normal form.
    \end{enumerate}
    \end{problem}
    \begin{proof}[Solution]
    \begin{align*}
        A &= A\cap U = A\cap(B\cup B^c) = (A\cap B)\cup(A\cap B^c)\\
        A^{c}\cup B &= (A^{c}\cap U)\cup (B\cap U)=(A^{c}\cap (B\cup B^{c}))\cup(B\cap (A\cup A^{c}))=(A^c\cap B)\cup(A^c\cap B^c)\cup (B\cap A)
    \end{align*}
    \begin{table}[H]
        \centering
        \captionsetup{type=table}
        \begin{tabular}{c c c}
            \hline
            $A$ & $B$ & \\ [0.5ex]
            \hline
            $0$ & $0$ & $A^c \cap B^c$\\
            $0$ & $1$ & $A^c \cap B$\\
            $1$ & $0$ & $A\cap B^c$\\
            $1$ & $1$ & $A\cap B$\\
            \hline
        \end{tabular}
        \caption{Minsets of $A$ and $B$.}
        \label{tab:discrete_structures_exam_II_problem_blah}
    \end{table}
    \end{proof}
    \begin{problem}
    Let:
    \begin{align*}
        A&=\begin{bmatrix*}[r]1&4\\0&5\\2&-3\end{bmatrix*} & B&=\begin{bmatrix*}[r]2&10\\-1&0\\5&-1\end{bmatrix*} & C&=\begin{bmatrix*}[r]1&3\\-2&4\end{bmatrix*} & D&=\begin{bmatrix*}[r]3&0\\0&2\end{bmatrix*} & F&=\begin{bmatrix*}1&3\\-2&-6\end{bmatrix*}
    \end{align*}
    Compute the following, if possible.
    \begin{enumerate}
    \begin{multicols}{6}
        \item $AB$
        \item $3A-2B$
        \item $7A + 2C$
        \item $A^2$
        \item $C^2$
        \item $D^4$
        \item $AC+BC$
        \item $CA$
        \item $C^{-1}$
        \item $C^{-2}$
        \item $F^{-1}$.
        \item $XC=A$
    \end{multicols}
    \end{enumerate}
    \end{problem}
    \begin{proof}
    \vspace{-\topsep}
    \
    \begin{enumerate}
        \item $AB$ is not possible because $A$ and $B$ are both $3\times 2$ matrices. The number of columns of $A$ must be the same as the number of rows of $B$.
        \item $3A-2B = \begin{pmatrix}-1 & -8\\ 2 & 15\\ -4 & -7 \end{pmatrix}$
        \item $7A+2C$ is not possible because $A$ and $C$ are of different dimensions.
        \item $A^2$ is not possible because $A$ is not a square matrix.
    \begin{multicols}{2}
        \item $C^2 = \begin{pmatrix} -5 & 15 \\ -10 & 10\end{pmatrix}$
        \item $D^4 = \begin{pmatrix} 81 & 0 \\ 0 & 16\end{pmatrix}$
    \end{multicols}
        \item $AC+BC = (A+B)C = \begin{pmatrix} 3 & 14 \\ -1 & 5 \\ 7 & -4 \end{pmatrix} \begin{pmatrix} 1 & 3 \\ -2 & 4 \end{pmatrix} = \begin{pmatrix} -25 & 65 \\ -11 & 17 \\ 15 & 5 \end{pmatrix}$
        \item $C$ is a $2\times 2$ and $A$ is a $3\times 2$, so multiplication is undefined.
    \begin{multicols}{2}
        \item $\det(C) = 10$, so $C^{-1} = \frac{1}{10}\begin{pmatrix} 4 & -3 \\ 2 & 1 \end{pmatrix}$
        \item $C^{-2} = (C^2)^{-1} = \frac{1}{100}\begin{pmatrix}10 & -15 \\ 10 & -5\end{pmatrix}$
        \item $\det(F) = 0$, so $F^{-1}$ does not exist. 
        \item $X = AC^{-1} = \frac{1}{10}\begin{pmatrix}12 & 1\\10 & 5\\ 2 & -9\end{pmatrix}$
    \end{multicols}    
    \end{enumerate}
    \end{proof}
    \clearpage
    \begin{problem}
    Convert the following into $AX = B$, and then solve.
    \begin{align*}
        4x - y &= 10\\
        3x - 2y &= 3
    \end{align*}
    \end{problem}
    \begin{proof}[Solution]
    \vspace{-\topsep}
    \begin{equation*}
        \begin{bmatrix*}[r]4&-1\\3&-2\end{bmatrix*}\begin{bmatrix}x\\y\end{bmatrix}=\begin{bmatrix}10\\3\end{bmatrix}\Rightarrow X=A^{-1}B\Rightarrow X=-\tfrac{1}{5}\begin{bmatrix*}[r]-2&-3\\1&4\end{bmatrix*}\begin{bmatrix*}[r]29\\-22\end{bmatrix*}\Rightarrow X=\frac{1}{5}\begin{bmatrix}17\\18\end{bmatrix}
    \end{equation*}
    \end{proof}
    \begin{problem}
    Prove or disprove the following for $n\times n$ matrices $A, B$.
    \begin{enumerate}
        \item $AB = BA$
        \item $(A+B)(A-B) = A^2 - B^2$
        \item If $A^2 = AB$, and $\det(A) = 4$, then $A=B$.
        \item If $AB = 0$, then $A=0$ or $B=0$
    \end{enumerate}
    \end{problem}
    \begin{proof}[Solution]
    \vspace{-\topsep}
    \
    \begin{enumerate}
        \item False, for $\begin{bmatrix} 0 & 1 \\ 0 & 1\end{bmatrix} \begin{bmatrix} 1 & 1 \\ 0 & 0 \end{bmatrix} = \begin{bmatrix} 0 & 0 \\ 0 & 0\end{bmatrix}$, but $\begin{bmatrix} 1 & 1 \\ 0 & 0 \end{bmatrix}\begin{bmatrix} 0 & 1 \\ 0 & 1\end{bmatrix}  = \begin{bmatrix} 0 & 2 \\ 0 & 0 \end{bmatrix}$.
        \item False, for $(A+B)(A-B) = A(A-B) + B(A-B) = A^2-AB+BA - B^2 =d (A^2-B^2) + (BA-AB)$. Since $AB$ may not necessarily be equal to $BA$, $BA-AB$ may not zero.
        \item True. If $\det(A) = 4$, then $A^{-1}$ exists. Then $A^2 = AB$, and thus $A^{-1}A^2  = A^{-1}AB \Leftrightarrow A = B$.
        \item False. For $\begin{bmatrix} 1 & 0 \\ 0 & 0 \end{bmatrix} \begin{bmatrix} 0 & 0 \\ 0 & 1 \end{bmatrix} = 0$.
    \end{enumerate}
    \end{proof}
    \subsection{Exam III}
    \begin{problem}
    Prove that for sets $A,B,C$, $(A\cap B)\times C \subset B\times C$.
    \end{problem}
    \begin{proof}[Solution]
    \vspace{-0.5\topsep}
    For let $(x,y) \in (A\cap B)\times C$. Then $x\in A\cap B$ and $y\in C$. But if $x\in A\cap B$, then $x\in B$. But if $x\in B$ and $y\in C$, then $(x,y) \in B\times C$. Therefore, $(A\cap B)\times C \subset B\times C$. 
    \end{proof}
    \begin{problem}
    For set $A,B\subset U$, prove that $A\cup (A\cap B)^c = U$, and find the dual of this.
    \end{problem}
    \begin{proof}[Solution]
    \vspace{-0.5\topsep}
    For Let $x\in U$. If $x\in A$, then $x\in A\cup (A\cap B)^c$. Suppose not. Then $x\in A^c$. But if $x\in A^c$, then $x\notin A\cap B$. But if $x\notin A\cap B$, then $x\in (A\cap B)^c$. But then $x\in A\cup (A\cap B)^c$. Therefore $U\subset A\cup (A\cap B)^c$. But $A\cup (A\cap B)^c \subset U$, as $U$ is the universe set. Therefore $A\cup (A\cap B)^c = U$. The dual is $A\cap (A\cup B)^c = \emptyset$
    \end{proof}
    \begin{problem}
    Convert the following into the form $AX = B$. Find $A^{-1}$, and then solve for $X$:
    \begin{align*}
        4x-6y &= 5 \\
        3x-7y &= -7
    \end{align*}
    \end{problem}
    \begin{proof}[Solution]
    \vspace{-\topsep}
    \begin{equation*}
        \begin{bmatrix*}[r]4&-6\\3&-7\end{bmatrix*}\begin{bmatrix*}x\\y\end{bmatrix*}=\begin{bmatrix*}[r]5\\-7\end{bmatrix*}\Rightarrow X=A^{-1}B= -\tfrac{1}{10}\begin{bmatrix}-7&6\\-3&4\end{bmatrix}\begin{bmatrix*}[r]5\\-7\end{bmatrix*}=\tfrac{1}{10}\begin{bmatrix}83\\47 \end{bmatrix}
    \end{equation*}
    \end{proof}
    \clearpage
    \begin{problem}
    Let $A = \begin{pmatrix} 1 & 0 & -2 \\ 5 & 3 & 0 \end{pmatrix}, B = \begin{pmatrix} 0 & 2 \\ 4 & -10 \\ 8 & -6 \end{pmatrix}, C = \begin{pmatrix} 7 & -1 \\ -2 & 5 \\ -4 & 3 \end{pmatrix}, E = \begin{pmatrix} 1 & -3 \\ 2 & 5 \end{pmatrix}$. Compute the following, if possible:
    \begin{enumerate}
    \begin{multicols}{6}
        \item $3A+5C$
        \item $2B-3C$
        \item $CE$
        \item $EC$
        \item $E^2$
        \item $BA+2CA$
    \end{multicols}
    \end{enumerate}
    \end{problem}
    \begin{proof}[Solution]
    \vspace{-\topsep}
    \
    \begin{enumerate}
        \item $A$ and $C$ do not have the same dimensions, so this can't be done.
    \begin{multicols}{2}
        \item $2B - 3C = \begin{pmatrix} -21 & 7 \\ 14 & -35 \\ 28 & -21 \end{pmatrix}$
        \item $CE = \begin{pmatrix} 5 & -26 \\ 8 & 31 \\ 2 & 27 \end{pmatrix}$
    \end{multicols}
        \item $EC$ can't be done as $E$ is a $2\times 2$ and $C$ is a $3\times 2$.
    \begin{multicols}{2}
        \item $E^2 = \begin{pmatrix} -5 & -18 \\ 12 & 19 \end{pmatrix}$
        \item $BA + 2CA = \begin{pmatrix} 13 & 0 & -28 \\ 0 & 0 & 0 \\ 0 & 0 & 0 \end{pmatrix}$
    \end{multicols}
    \end{enumerate}
    \end{proof}
    \begin{problem}
    Let $A = \{1,3,5\}$, $B = \{2,3,4,5\}$, and $U = \{1,2,3,4,5,6\}$.
    \begin{enumerate}
        \item Find all minsets generated by $A$ and $B$.
        \item How many different sets in the power set of $U$ can be generated by $A,B$ by any combination of union, intersection, and complement?
        \item Express $\{2,4,6\}$ in minset normal form.
        \item Find the maxsets generated by $A$ and $B$.
        \item Express $\{1,3,5\}$ in maxset normal form.
    \end{enumerate}
    \end{problem}
    \begin{proof}[Solution]
    \vspace{-\topsep}
    \
    \begin{enumerate}
        \item $A^c\cap B^c = \{6\}$, $A\cap B^c = \{1\}$, $A^c \cap B = \{2,4\}$, $A\cap B = \{3,5\}$.
        \item There are $4$ non-empty minsets, so $2^4 = 16$. 
        \item $\{2,4,6\} = (A^c \cap B)\cup (A^c \cap B^c)$.
        \item $A\cup B = \{1,2,3,4,5\}, A^c \cup B = \{2,3,4,5,6\}, A\cup B^c = \{1,3,5,6\}, A^c \cup B^c = \{1,2,4,6\}$.
        \item $\{1,3,5\} = (A\cup B) \cap (A\cup B^c)$.
    \end{enumerate}
    \end{proof}
    \subsection{Final Exam}
    \begin{problem}
    Let $A = \{1,2\}$, $B = \{2,3,4,6\}$, $C = \{4,6,7\}$, $U = \{1,2,3,4,5,6,7,8\}$. Compute the following:
    \begin{enumerate}
    \begin{multicols}{4}
        \item $A^c \cap B$
        \item $A\cup C$
        \item $B\oplus C$
        \item $B\setminus C$
        \item $A\times C$
        \item $A^3$
        \item $\mathcal{P}(C)$
    \end{multicols}
    \end{enumerate}
    \end{problem}
    \begin{proof}[Solution]
    \vspace{-\topsep}
    \
    \begin{enumerate}
    \begin{multicols}{2}
        \item $A^c \cap B = \{3,4,6\}$.
        \item $A \cup C = \{1,2,4,6,7\}$
        \item $B\oplus C = \{2,3,7\}$
        \item $B\setminus C = \{2,3\}$
    \end{multicols}
        \item $A\times C = \{(1,4),(1,6),(1,7),(2,4),(2,6),(2,7)\}$
        \item $A^3 = \{(1,1,1),(1,1,2),(1,2,1),(1,2,2),(2,1,1),(2,1,2),(2,2,1),(2,2,2)\}$.
        \item $\mathcal{P}(C) = \{\emptyset, \{4\},\{6\},\{7\},\{4,6\},\{4,7\},\{6,7\},\{4,6,7\}\}$.
    \end{enumerate}
    \end{proof}
    \begin{problem}
    Prove the following is false: If $A\cap B = A \cap C$, then $B = C$.
    \end{problem}
    \begin{proof}[Solution]
    \vspace{-0.5\topsep}
    For let $A = \{1\}$, $B = \{1,2\}$, and $C = \mathbb{R}$. Then $A\cap B = \{1\}$, $A\cap C = \{1\}$, but $B \ne C$.
    \end{proof}
    \begin{problem}
    Ten students are competing for a scholarship.
    \begin{enumerate}
        \item If there are three scholarships worth $\$2000$, how many ways can they be distributed?
        \item If there are two scholarships worth $\$5000$ and three worth $\$2000$, how many ways can they be distributed?
        \item Suppose that the group of ten students consists of six freshmen and four sophomores. In how many different ways can four equal scholarships be distributed if at least two of the scholarships should be awarded to freshmen?
        \item Suppose the group of ten students consists of six freshmen and four sophomores. In how many different ways can two scholarships of $\$5000$ and two scholarships of $\$2000$ be distributed if at least three of the scholarships will be awarded to freshmen?
    \end{enumerate}
    \end{problem}
    \begin{proof}[Solution]
    \vspace{-\topsep}
    \
    \begin{enumerate}
    \begin{multicols}{2}
        \item $\binom{10}{3} = \frac{10!}{3!(10-3)!} = 120$
        \item $\binom{10}{2}\binom{8}{3} = 2520$
    \end{multicols}
        \item If $2$ scholarships are awarded to freshmen, we have $\binom{6}{2}\binom{4}{2} = 90$. If $3$ scholarships are awards to freshmen, we have $\binom{6}{3}\binom{4}{1} = 80$. If $4$ scholarships are awarded to freshmen, we have $\binom{6}{4}\binom{4}{0} = 15$. Adding them together, we get $185$.
        \item If $2$ $\$5000$ scholarships are awarded to freshmen, and $1$ $\$2000$ scholarship is awarded to a freshman, then there are $\binom{6}{3}\binom{4}{1}=80$. Simiarly if $2$ $\$2000$ scholarships are awarded to freshmen and $1$ $\$5000$ scholarship is awarded to a freshamn. Finally, there are $\binom{6}{4}=15$ ways to give all scholarships to freshmen. In total, there are $175$ total possible outcomes.
    \end{enumerate}
    \end{proof}
    \section{Quizzes}
    \subsection{Quiz I}
    \begin{problem}
    Let $A = \{1,2,3,7,8\}$, $B = \{1,3,5\}$, $C = \{2,4,8\}$, and let $U = \{1,2,3,4,5,6,7,8,9,10\}$. Evaluate the following:
    \begin{enumerate}
    \begin{multicols}{6}
        \item $A\cup C$
        \item $A\cap B$
        \item $A \oplus C$
        \item $A^c$
        \item $A\setminus B$
        \item $B\times C$
    \end{multicols}
    \end{enumerate}
    \end{problem}
    \begin{proof}[Solution]
    \vspace{-\topsep}
    \
    \begin{enumerate}
    \begin{multicols}{5}
        \item $\{1,2,3,4,7,8\}$
        \item $\{1,3\}$
        \item $\{1,3,4,7\}$
        \item $\{4,5,6,9,10\}$
        \item $A\setminus B = \{2,7,8\}$
    \end{multicols}
        \item $B \times C = \{(1,2),(1,4),(1,8),(3,2),(2,4),(3,8),(5,2),(5,4),(5,8)\}$
    \end{enumerate}
    \end{proof}
    \begin{problem}
    Let $S = \{1,3,5\}$. Find $\mathcal{P}(S)$.
    \end{problem}
    \begin{proof}[Solution]
    \vspace{-0.5\topsep}
    $\mathcal{P}(S) = \{\emptyset,\{1\},\{3\},\{5\},\{1,3\},\{1,5\},\{3,5\},\{1,3,5\}\}$
    \end{proof}
    \begin{problem}
    Let $S = \{7k-3: k \in \mathbb{N}, k < 5\}$. List all of the elements of $S$.
    \end{problem}
    \begin{proof}[Solution]
    \vspace{-0.5\topsep}
    $S = \{-3,4,11,18,25\}$
    \end{proof}
    \begin{problem}
    Express $49$ in binary.
    \end{problem}
    \begin{proof}[Solution]
    \vspace{-\topsep}
    \
    \begin{align*}
        49 &= 2\cdot24+1 & 6&=2\cdot3+0\\
        24&= 2\cdot12+0 & 3&=2\cdot1+1\\
        12&= 2\cdot6\phantom{2}+0 & 1&=2\cdot0+1
    \end{align*}
    So, $49 = 110001_{2}$
    \end{proof} 
    \subsection{Quiz II}
    \begin{problem}
    Calculate $\sum_{k=-1}^{3} (2^k+1)$.
    \end{problem}
    \begin{proof}[Solution]
    \vspace{-0.5\topsep}
    $\sum_{k=-1}^{3}(2^k+1) = (2^{-1}+1) + (2^0+1) + (2^1+1)+(2^2+1) + (2^3+1) = 20 +\frac{1}{2} = \frac{41}{2}$.
    \end{proof}
    \begin{problem}
    Three men and three women are to be seated in a row.
    \begin{enumerate}
        \item How many different ways can the six people be seated?
        \item How many different ways can the six people be seated if it is required that the genders alternate.
    \end{enumerate}
    \end{problem}
    \begin{proof}[Solution]
    \vspace{-\topsep}
    \
    \begin{enumerate}
        \item $6! = 6\cdot 5 \cdot 5 \cdot 4 \cdot 3 \cdot 2 \cdot 1 = 720$
        \item It is $MWMWMW$, so $3\cdot 3 \cdot 2 \cdot 2 \cdot 1 \cdot 1 = 72$. Or, there are $6$ ways to seat the first person, $3$ ways to seat the second person, $2$ ways to seat the third person, $2$ ways to seat the fourth person, and $1$ way to seat the last two. So, $6\cdot 3 \cdot 2 \cdot 2 \cdot 1 \cdot 1 = 72$.
    \end{enumerate}
    \end{proof}
    \begin{problem}
    Calculate $P(7;3)$
    \end{problem}
    \begin{proof}[Solution]
    \vspace{-0.5\topsep}
    $P(7;3) = \frac{7!}{(7-3)!} = 7\cdot 6 \cdot 5 = 210$.
    \end{proof}
    \begin{problem}
    Let $A$ be a set such that $|A| = n$.
    \begin{enumerate}
    \begin{multicols}{3}
        \item Calculate $|A^4|$
        \item Calculate $|\{\{a,b,c,d\}\subset A:\textrm{Each Term is Different}\}|$
    \end{multicols}
    \end{enumerate}
    \end{problem}
    \begin{proof}[Solution]
    \vspace{-\topsep}
    \
    \begin{enumerate}
    \begin{multicols}{2}
        \item $|A^4| = |A\times A \times A \times A| = n^4$
        \item $n \cdot (n-1)\cdot (n-2)\cdot (n-3) = P(n;4)$
    \end{multicols}
    \end{enumerate}
    \end{proof}
    \subsection{Quiz III}
    \begin{problem}
    Let $p,q,r$ be the following propositions:
    \begin{align*}
        p(x)&:x=1 & p(x)&:x=-1 & r(x)&:x^{2}=1
    \end{align*}
    \begin{enumerate}
        \item Express ``If $x^2 = 1$, then $x=1$ and $x=-1$," in symbolic form.
        \item Write the converse of this in English, and symbolically.
        \item Express $\neg p \land \neg r$ in English.
        \item Express $r\leftrightarrow (q\lor p)$ in English.
    \end{enumerate}
    \end{problem}
    \begin{proof}[Solution]
    \vspace{-\topsep}
    \
    \begin{enumerate}
        \item $r\rightarrow (p\land q)$
        \item $(p\land q) \rightarrow r$. If $x=1$ and $x=-1$, then $x^2 = 1$.
        \item $x\ne = 1$ and $x^2 \ne 1$>
        \item $x^2 = 1$ if and only if $x=1$ or $x=-1$.
    \end{enumerate}
    \end{proof}
    \begin{problem}
    \label{discrete_structures_quiz_3_problem_2}
    Make a truth table for $(p\lor \neg q)\land r$.
    \end{problem}
    \begin{proof}[Solution]
    \vspace{-\topsep}
    \
    \begin{table}[H]
        \centering
        \captionsetup{type=table}
        \begin{tabular}{c c c c c c}
            \hline
            $p$ & $q$ & $r$ & $\neg q$ & $p\lor \neg q$ & $(p\lor \neg q)\land r$ \\ [0.5ex]
            \hline
            $0$ & $0$ & $0$ & $1$ & $1$ & $0$\\
            $0$ & $0$ & $1$ & $1$ & $1$ & $1$\\
            $0$ & $1$ & $0$ & $0$ & $0$ & $0$\\
            $0$ & $1$ & $1$ & $0$ & $0$ & $0$\\
            $1$ & $0$ & $0$ & $1$ & $1$ & $0$\\
            $1$ & $0$ & $1$ & $1$ & $1$ & $1$\\
            $1$ & $1$ & $0$ & $0$ & $1$ & $0$\\
            $1$ & $1$ & $1$ & $0$ & $1$ & $1$\\
            \hline
        \end{tabular}
        \caption{Truth Table for Problem \ref{discrete_structures_quiz_3_problem_2}}
        \label{tab:discrete_structures_final_exam_problem}
    \end{table}
    \end{proof}
    \subsection{Quiz IV}
    \begin{problem}
    Prove directly that $a\rightarrow b, \neg c\rightarrow \neg b, \neg c \Rightarrow \neg a$.
    \end{problem}
    \begin{proof}[Solution]
    \vspace{-0.5\topsep}
    For if $a\rightarrow b$, then $\neg b \rightarrow \neg a$. But $\neg c \rightarrow \neg b$. But if $\neg c \rightarrow \neg b$ and $\neg b \rightarrow \neg a$, then $\neg c \rightarrow \neg a$. Thus $a\rightarrow b, \neg c \rightarrow \neg b, \neg c \Rightarrow \neg a$.
    \end{proof}
    \begin{problem}
    Prove indirectly that $a\rightarrow b, \neg c \rightarrow \neg b, \neg c \Rightarrow \neg a$.
    \end{problem}
    \begin{proof}[Solution]
    \vspace{-0.5\topsep}
    For if $\neg c \rightarrow \neg b$, then $b\rightarrow c$. But if $a\rightarrow b$ and $b\rightarrow c$, then $a\rightarrow c$. Therefore $a\rightarrow c$. But if $a\rightarrow c$, then $\neg c \rightarrow \neg a$. Therefore, $a\rightarrow b, \neg c \rightarrow \neg b, \neg c \Rightarrow \neg a$.
    \end{proof}
    \begin{problem}
    Let $U = \{1,2,3,4,5,6,7,8,9,10\}$. Consider the following propositions over $U$:
    \begin{align*}
        p(n)&:n\textrm{ is prime} & q(n)&:n\textrm{ is odd.} & r(n)&:n\leq 7
    \end{align*}
    \begin{enumerate}
        \item Find the truth sets for $p,q,r$.
        \item Which of these propositions implies one of the others?
        \item Find the truth set of $q\land r$.
    \end{enumerate}
    \end{problem}
    \begin{proof}[Solution]
    \vspace{-\topsep}
    \
    \begin{enumerate}
        \item $T_{p} = \{2,3,5,7\}$, $T_{q} = \{1,3,5,7,9\}$, $T_{r} = \{1,2,3,4,5,6,7\}$
    \begin{multicols}{2}
        \item $p\Rightarrow r$, since $T_{p}\subset T_{r}$.
        \item $T_{q\land r} = \{1,3,5,7\}$.
    \end{multicols}
    \end{enumerate}
    \end{proof}
    \subsection{Quiz V}
    \begin{problem}
    Let $p,q,r$ be the following propositions over $\mathbb{N}$:
    \begin{align*}
        p(n)&:n^{2}+3n=1 & q(n)&:n\textrm{ is prime.} & r(n)&:n\textrm{ is even.} & s(m,n)&:m|n
    \end{align*}
    \begin{enumerate}
        \item Express ``There exists a solution for $n^2+3n = 10$ that is prime," symbolically.
        \item Express $\forall_{n\in \mathbb{Z}}(q\rightarrow \neg r)$ in English.
        \item Express $\forall_{n\in \mathbb{N}}\exists_{m\in \mathbb{N}}(s(m,n))$ in English.
    \end{enumerate}
    \end{problem}
    \begin{proof}[Solution]
    \vspace{-\topsep}
    \
    \begin{enumerate}
        \item $\exists_{n\in T_{q\land p}}$.
        \item For every integer $n$, if $n$ is prime, then $n$ is not an even number.
        \item For every positive integer $n$, there exists a positive integer $m$ such that $m$ divides $n$. 
    \end{enumerate}
    \end{proof}
    \begin{problem}
    Prove $\sum_{k=1}^{n} 10k = 5n(n+1)$ using mathematical induction. 
    \end{problem}
    \begin{proof}[Solution]
    \vspace{-0.5\topsep}
    The base case is $n=1$, so $10 = 5\cdot 1(1+1) = 5\cdot 2 = 10$, which is true. Suppose this is true for some $n\in \mathbb{N}$. Then $\sum_{k=1}^{n+1} 10k = 10(n+1) + \sum_{k=1}^{n} 10k$. By hypothesis, $\sum_{k=1}^{n} 10k = 5n(n+1)$, so $\sum_{k=1}^{n+1}10k = 10(n+1)+5n(n+1) = (n+1)(10+5n) = 5(n+1)(n+2) = 4(n+1)((n+1)+1)$. This proves the induction step.
    \end{proof}
    %         \chapter{Linear Algebra}
\section{Linear Algebra}
    \begin{lexample}
        If $V$ and $W$ are $2-$dimensional subspaces in $\mathbb{R}^{4}$, what
        are the possible dimensions of $V\cap W$. If $V$ and $W$ are subspaces,
        then ${V}\cap{W}$ is subspace, and
        $\dim({V}\cap{W}\leq\min(\{\dim(V),\dim(W)\})$ we have in our problem
        that $\dim\{V\cap W\}\leq 2$. We now must show that $V\cap W$ can have
        dimensions 0,1, or 2. If $V=\{(x,y,0,0):x,y\in\mathbb{R}\}$ and
        $W=\{(0,0,z,w):z,w\in \mathbb{R}\}$, then ${V}\cap{W}=\{(0,0,0,0)\}$
        which has dimension $0$. If $V=\{(x,y,0,0):x,y\in\mathbb{R}\}$ and
        $W=\{(0,y,z,0):y,z\in\mathbb{R}\}$, then
        ${V}\cap{W}=\{(0,y,0,0):y\in\mathbb{R}\}$ which has dimension $1$.
        Finally, if $V=W$ then ${V}\cap{W}=V$, which has dimension $2$. So, the
        only possibilities are $0,1$, or $2$.
    \end{lexample}
    A system of linear equations can be written using matrices. Suppose we have
    the following equations:
    \begin{align}
        a_{0,0}x_{0}+a_{0,1}x_{1}+a_{0,2}x_{2}&=b_{0}\\
        a_{1,0}x_{0}+a_{1,1}x_{1}+a_{1,2}x_{2}&=b_{1}\\
        a_{2,0}x_{0}+a_{2,1}x_{1}+a_{2,2}x_{2}&=b_{2}\\
    \end{align}
    where are all the variables belong to some field $(\mathbb{F},+,\cdot\,)$.
    We can express this system in terms of matrices as follows:
    \begin{equation}
        \begin{bmatrix}
            a_{0,0}&a_{0,1}&a_{0,2}\\
            a_{1,0}&a_{1,1}&a_{1,2}\\
            a_{2,0}&a_{2,1}&a_{2,2}
        \end{bmatrix}
        \begin{bmatrix}
            x_{0}\\
            x_{1}\\
            x_{2}
        \end{bmatrix}
        =
        \begin{bmatrix}
            b_{0}\\
            b_{1}\\
            b_{2}
        \end{bmatrix}
    \end{equation}
    Better yet, if we let $\mathbf{x},\mathbf{b}\in\mathbb{F}^{3}$ be the points
    such that $\mathbf{x}(k)=x_{k}$ and $\mathbf{b}(k)=b_{k}$, for
    $k\in\mathbb{Z}_{3}$, and if we let
    $\mathbf{A}:\mathbf{F}^{3}\rightarrow\mathbf{F}^{3}$ be the linear operator
    defined by this matrix, we can write:
    \begin{equation}
        \mathbf{A}(\mathbf{x})=\mathbf{b}
    \end{equation}
    Matrices can also be written as $\mathbf{A}=(a_{ij})$. The following rules
    are used to define matrix arithmetic.
    \begin{enumerate}
        \item Addition: To add two matrices, add their
            corresponding elements. That is, if
            $\mathbf{A}=(a_{ij})$ and $\mathbf{B}=(b_{ij})$,
            then $\mathbf{A}+\mathbf{B}=(a_{ij}+b_{ij})$.
            Matrix addition is only defined on matrices of
            the same size.
        \item Scale multiplication: To multiply a
            matrix by a real or complex number $c$,
            multiply this number to every element. That is,
            if $\mathbf{A}=(a_{ij})$, then
            $c\mathbf{A}=({c}\cdot{a_{ij}})$
        \item Matrix Multiplication: The product of
            and ${M}\times{N}$ matrix with an
            ${N}\times{P}$ matrix is defined by
            $\mathbf{C}=\mathbf{A}\mathbf{B}$, where
            $(c_{ij})=(\sum_{k=1}^{N}a_{ik}b_{kj})$. Note
            that it is possible for
            $\mathbf{A}\mathbf{B}\ne\mathbf{B}\mathbf{A}$.
            Indeed, it is possible for
            $\mathbf{A}\mathbf{B}$ to be defined, whereas
            $\mathbf{B}\mathbf{A}$ is undefined.
    \end{enumerate}
    \begin{example}
        Let the following be true:
        \begin{align*}
            A&=
            \begin{bmatrix}
                1&2\\
                3&4
            \end{bmatrix}
            &
            B&=
            \begin{bmatrix}
                5&6\\
                7&8
            \end{bmatrix}
        \end{align*}
        Then, using the defined rules, we have:
        \begin{align*}
            A+B&=
            \begin{bmatrix}
                6&8\\
                10&12
            \end{bmatrix}
            &
            5A&=
            \begin{bmatrix}
                5&10\\
                15&20
            \end{bmatrix}
            \\
            AB&=
            \begin{bmatrix}
                19&22\\
                43&50
            \end{bmatrix}
            &
            BA&=
            \begin{bmatrix}
                23&34\\
                31&46
            \end{bmatrix}
        \end{align*}
    Note that even in this trivial example,
    ${AB}\ne{BA}$.
    \end{example}
    \begin{definition}
        The ${n}\times{n}$ identity matrix is the matrix
        $I_{n}=(I_{ij})$, where
        $I_{ij}=%
        \begin{cases}%
         0,&{i}\ne{j}\\%
         1,&{i}={j}%
        \end{cases}$
    \end{definition}
    \begin{definition}
        An inverse matrix of an ${n}\times{n}$ matrix
        $A$ is a matrix $A^{-1}$ such that
        $AA^{-1}=A^{-1}A=I_{n}$
    \end{definition}
    Not every matrix has an inverse matrix. If one
    does exists, there are many properties it contains.
    \begin{theorem}
        The following are true:
        \begin{enumerate}
            \item If $\mathbf{A}$ and $\mathbf{B}$
                are invertible ${n}\times{n}$ matrices,
                then $\mathbf{A}\mathbf{B}$ is invertible
                and
                $\mathbf{A}\mathbf{B}^{-1}%
                 =\mathbf{B}^{-1}\mathbf{A}^{-1}$
            \item If $\mathbf{A}$ is an invertible matrix,
                then $\mathbf{A}^{-1}$ is an invertible
                matrix and
                $(\mathbf{A}^{-1})^{-1}=\mathbf{A}$
        \end{enumerate}
    \end{theorem}
    \begin{definition}
        The trace of an ${n}\times{n}$ matrix
        $A$ is the sum of
        it's diagonal: $\Tr(A)=\sum_{i=1}^{n}a_{ii} $
    \end{definition}
    \begin{example}
        \begin{equation*}
            \Tr\Bigg(
            \begin{bmatrix}
                4&5&6\\
                1&2&3\\
                8&8&3
            \end{bmatrix}
            \Bigg)
            =4+2+3=9
        \end{equation*}
    \end{example}
    \begin{definition}
        The determinant of a ${2}\times{2}$ matrix
        $A=%
         \begin{bmatrix}%
            a&b\\%
            c&d%
         \end{bmatrix}$
        is $\det(A)=ad-bc$
    \end{definition}
    \begin{definition}
        The minor of the $i^{th}$ row and $j^{th}$
        column of an ${n}\times{n}$ matrix $\mathbf{A}$,
        denoted $M_{ij}$, is the determinant of the
        ${(n-1)}\times{(n-1)}$ matrix formed by
        removing the $i^{th}$ row and $j^{th}$ column
        from $\mathbf{A}$.
    \end{definition}
    \begin{definition}
        The cofactor of the minor $M_{ij}$ of an
        ${n}\times{n}$ matrix $\mathbf{A}$,
        denoted $C_{ij}$, is $(-1)^{i+j}M_{ij}$.
    \end{definition}
    \begin{example}
        \begin{align*}
            A&=
            \begin{bmatrix}
                7&1&3\\
                1&3&5\\
                17&4&20
            \end{bmatrix}
            &
            M_{11}
            &=
            \det\Bigg(\begin{bmatrix}
                     3&5\\
                     4&20
                 \end{bmatrix}
                \Bigg)
            =40
            &
            C_{11}
            &=(-1)^{1+1}M_{11}=40
        \end{align*}
    \end{example}
    \begin{definition}
        The determinant of an ${n}\times{n}$ matrix
        $\mathbf{A}$ is
        $\det(A)=\sum_{j=1}^{n}a_{1j}C_{1j}$
    \end{definition}
    \begin{theorem}
        If $\mathbf{A}$ is an ${n}\times{n}$ matrix
        and ${1}\leq{i}\leq{n}$, then
        $\det(A)=\sum_{j=1}^{n}a_{ij}C_{ij}$
    \end{theorem}
    \begin{definition}
        The transpose of an ${n}\times{m}$ matrix
        $\mathbf{A}$, denoted $\mathbf{A}^{T}$,
        is the ${m}\times{n}$ matrix formed by
        swapping the rows and columns of $\mathbf{A}$
        with each other. That is $(a_{ij})^{T}=(a_{ji})$.
    \end{definition}
    \begin{definition}
        A symmetric matrix is a matrix $\mathbf{A}$
        such that $\mathbf{A}^{T}=\mathbf{A}$
    \end{definition}
    \begin{theorem}
        If $\mathbf{A}$ is an ${n}\times{m}$ matrix and
        $\mathbf{B}$ is an ${m}\times{p}$ matrix, then
        the following are true:
        \begin{enumerate}
            \begin{multicols}{2}
                \item $(\mathbf{A}^{T})^{T}=\mathbf{A}$
                \item $(\mathbf{A}+\mathbf{B})^{T}%
                       =\mathbf{A}^{T}+\mathbf{B}^{T}$
                \item $(k\mathbf{A})^{T}=k\mathbf{A}^{T}$
                \item $(\mathbf{A}\mathbf{B})^{T}%
                       =\mathbf{B}^{T}\mathbf{A}^{T}$
            \end{multicols}
        \end{enumerate}
    \end{theorem}
    \begin{definition}
        The adjoint of an ${n}\times{n}$ matrix
        $\mathbf{A}$, denoted $\adjoint(\mathbf{A})$,
        is the matrix $(C_{ij})^{T}$.
    \end{definition}
    \begin{theorem}
        If ${\det(\mathbf{A})}\ne{0}$, then $\mathbf{A}$
        is invertible and
        $\mathbf{A}^{-1}=%
         \frac{1}{\det(\mathbf{A})}\adjoint(\mathbf{A})$
    \end{theorem}
    \begin{theorem}
        If $\mathbf{A}$ and $\mathbf{B}$ are
        ${n}\times{n}$ matrices, then the following
        are true:
        \begin{enumerate}
            \begin{multicols}{3}
                \item $\det(\mathbf{A})%
                       =\det(\mathbf{A}^{T})$
                \item $\det(k\mathbf{A})%
                       =k^{n}\det(\mathbf{A})$
                \item $\det(\mathbf{A}\mathbf{B})%
                       =\det(\mathbf{A})\det(\mathbf{B})$
            \end{multicols}
        \end{enumerate}
    \end{theorem}
    \begin{theorem}
        A matrix $\mathbf{A}$ is invertible if and only
        if ${\det(\mathbf{A})}\ne{0}$
    \end{theorem}
    \begin{theorem}
        If $\mathbf{A}$ is invertible, then
        $\det(\mathbf{A}^{-1})=\frac{1}{\det(\mathbf{A})}$
    \end{theorem}
    The differential equation
    $\sum_{k=0}^{n}a_{k}y^{(k)}(x)$ Can be expression
    in terms of the characteristic polynomial
    $\sum_{k=0}^{n}a_{k}D^{k}$. Factoring this linear
    operator into $\Pi_{k=0}^{n}(D-r_{k})$,
    the general solution is
    $y(x)=\sum_{k=1}^{n}c_{k}e^{r_{k}x}$. If some of the
    $r_{k}$ repeat, we have $c_{k}x^{m_{k}-1}e^{r_{k}x}$,
    where $m_{k}$ is the number of repetitions.
    In general, if we have
    $\Pi_{k=0}^{n}(D-r_{k})^{m_{k}}$, the general
    solution is
    $y(x)=%
     \sum_{k=1}^{n}c_{k}e^{r_{k}x}%
     (\sum_{j=0}^{m_{k}-1}x^{j})$
    \begin{example}
        \
        \begin{enumerate}
            \item $y'''-4y''+4y'=0$ has the characteristic
                polynomial $D(D-2)^{2}$, so
                $y(x)=c_{1}+c_{2}e^{2x}+c_{3}xe^{2x}$
        \end{enumerate}
    \end{example}
    In linear algebra, the determinant
    $\det(\mathbf{A}-\lambda{I})$ is the characteristic
    polynomial of the square matrix $\mathbf{A}$.
    \begin{definition}
        A vector space $V$ over a Field (Set of scalars)
        $F$ is a set $V$ with two operations
        $+$ and $\cdot$
        such that the following are true:
        \begin{enumerate}
            \begin{multicols}{3}
                \item $\forall_{{\mathbf{a},%
                                 \mathbf{b}}\in{V}}$
                      ${\mathbf{a}+\mathbf{b}}\in{V}$
                \item $\mathbf{a}+\mathbf{b}%
                       =\mathbf{b}+\mathbf{a}$
                \item $\mathbf{a}+(\mathbf{b}+\mathbf{c})%
                       =(\mathbf{a}+\mathbf{b})+\mathbf{c}$
                \item $\forall_{\mathbf{a}\in{V}}%
                       \exists_{\mathbf{b}\in{V}}:%
                       \mathbf{a}+\mathbf{b}=\mathbf{0}$
                \item $\forall_{{k}\in{F},\mathbf{a}\in{V}}$
                      $k\mathbf{a}\in{V}$
                \item $k(\mathbf{a}+\mathbf{b})%
                       =k\mathbf{a}+k\mathbf{b}$
                \item $(k_{1}+k_{2})\mathbf{a}%
                       =k_{1}\mathbf{a}+k_{2}\mathbf{a}$
                \item $1\mathbf{a}=\mathbf{a}$
                \item $k_{1}(k_{2}\mathbf{a})%
                       =(k_{1}k_{2})\mathbf{a}$
            \end{multicols}
        \end{enumerate}
    \end{definition}
    \begin{theorem}
        If $V$ is a vector space, then there is a
        $\mathbf{0}\in{V}$ such that for all
        $\mathbf{a}\in{V}$,
        $\mathbf{a}+\mathbf{0}=\mathbf{a}$
    \end{theorem}
    \begin{definition}
        A linearly dependent subset of a vector space
        $V$ (Over $\mathbb{R}$)
        is a subset ${S}\subset{V}$ such that
        there exists an $N\in\mathbb{N}$, a non-zero
        $a_{n}:\mathbb{Z}_{N}\rightarrow\mathbb{R}$
        and an injective function
        $\mathbf{v}_{n}:\mathbb{Z}_{N}\rightarrow{V}$
        such that
        $\sum_{k=1}^{N}a_{n}\mathbf{v}_{n}=\mathbf{0}$
    \end{definition}
    \begin{definition}
        A linearly independent subset of a vector space
        $V$ is a subset ${S}\subset{V}$ that is not
        linearly dependent.
    \end{definition}
    \begin{theorem}
        If $V\subset\mathbb{R}^{n}$ has more than
        $n$ vectors, then $V$ is linearly dependent.
    \end{theorem}
    \begin{definition}
        The rank of a matrix is the number
        of linearly independent columns of
        the matrix.
    \end{definition}
    \begin{example}
        Let
        $\mathbf{A}=[A_{1}\ A_{2}]$
        where $A_{1}=(1,2)^{T}$ and
        $A_{2}=(2,4)^{T}$. So
        $2A_{1}+(-1)A_{2}=(0,0)^{T}=\mathbf{0}$. Therefore
        $\{A_{1},A_{2}\}$ is a linearly independent
        subset. Thus, $\rk(\mathbf{A})=1$.
    \end{example}
    \begin{definition}
        A matrix with full rank is a square
        ${n}\times{n}$ matrix $\mathbf{A}$ such that
        $\rk(\mathbf{A})=n$.
    \end{definition}
    \begin{theorem}
        If $\mathbf{A}$ is a square matrix and
        $\det(\mathbf{A})\ne{0}$, then $\mathbf{A}$
        has full rank.
    \end{theorem}
    \begin{theorem}
        If $\mathbf{A}$ is a square matrix with full rank, then it is
        invertible.
    \end{theorem}
    \begin{definition}
        A finite basis of a vector space $V$ is a
        linearly independent subset ${S}\subset{V}$
        where
        $S=\{\mathbf{v}_{k}\}_{k=0}^{n}$
        and for all
        $\mathbf{x}\in{V}$ there is an
        $a_{n}:\mathbb{Z}_{n}\rightarrow\mathbb{R}$
        such that
        $\mathbf{x}=\sum_{k=1}^{n}a_{k}\mathbf{v}_{k}$
    \end{definition}
    \begin{theorem}
        All bases of a vector space $V$ have the
        same number of elements.
    \end{theorem}
    \begin{definition}
        The dimension of a vector space $V$ is the number of elements in any
        basis of $V$.
    \end{definition}
    \begin{definition}
        An eigenvector of an ${n}\times{n}$ matrix
        $\mathbf{A}$ is a vector
        $\mathbf{x}\in\mathbb{R}^{n}$
        such that there exists a $\lambda\in\mathbb{R}$
        such that
        $\mathbf{A}\mathbf{x}=\lambda\mathbf{x}$
    \end{definition}
    \begin{definition}
        An eigenvalue of an ${n}\times{n}$ matrix
        $\mathbf{A}$ is a real number
        $\lambda\in\mathbb{R}$ such that there is
        a vector $\mathbf{x}\in\mathbf{R}^{n}$ such
        that $\mathbf{A}\mathbf{x}=\lambda\mathbf{x}$
    \end{definition}
    \begin{definition}
        The characteristic equation, or
        the characteristic polynomial, of an
        ${n}\times{n}$ matrix $\mathbf{A}$
        is $\det(\lambda{I}-\mathbf{A})=0$
    \end{definition}
    \begin{definition}
        A diagonalizable matrix is an
        ${n}\times{n}$ matrix
        $\mathbf{A}$ such that there exists
        an invertible matrix $\mathbf{B}$
        such that
        $\mathbf{A}=\mathbf{B}^{-1}\mathbf{A}\mathbf{B}$
    \end{definition}
    \begin{theorem}
        The following are true:
        \begin{enumerate}
            \item If $\mathbf{A}$ is an ${n}\times{n}$
                diagonable matrix, then $\mathbf{A}$
                has $n$ linearly independent
                eigenvectors.
            \item If $\mathbf{A}$ is an ${n}\times{n}$
                matrix with $n$ linearly independent
                eigenvectors, then $\mathbf{A}$
                is diagonalizable.
            \item A symmetric matrix has all real
                eigenvalues.
        \end{enumerate}
    \end{theorem}
\section{Miscellaneous Lecture Notes}
    \subsection{Orthogonal Projections}
    \begin{definition}
    The span of
    $\mathcal{W}=\{X_{i}\}_{1}^{k}\subset\mathbb{R}^n$
    is the set
    $\Span(\mathcal{W})=\{\sum_{i=1}^{k}a_{i}X_{i}:a_{i}\in \mathbb{R}\}$.
    \end{definition}
    \begin{definition}
    A linearly dependent subset of $\mathbb{R}^{n}$ is a subset $S\subset\mathbb{R}^{n}$ such that there exists a finite subset $\{X_{i}\}_{i=1}^{k}\subset S$ and a subset $\{a_{i}\}_{i=1}^{k}\subset \mathbb{R}\setminus \{0\}$ such that $\sum_{i=1}^{k}a_{i}X_{i}=\mathbf{0}$
    \end{definition}
    \begin{definition}
    A linearly independent subset of $\mathbb{R}^{n}$ is a subset $S\subset \mathbb{R}^{n}$ that is not linearly dependent.
    \end{definition}
    \begin{theorem}
    If $S\subset\mathbb{R}^{n}$ is linearly independent, then $|S|\leq n$.
    \end{theorem}
    \begin{theorem}
    If $\mathcal{W}\subset\mathbb{R}^{n}$ is linearly independent and $|\mathcal{W}| = k$, then $\Span(\mathcal{W})$ is a $k$ dimensional subspace of $\mathbb{R}^n$.
    \end{theorem}
    If we have a linearly independent subset $\mathcal{W}=\{X_1,\hdots, X_k\}\subset\mathbb{R}^n$, and some other vector $Y$, we may wish to consider the orthogonal projection of $Y$ onto the $k$ dimensional subspace spanned by $\mathcal{W}$. That is, we may wish to find a vector $Y'\in\Span(X_1,\hdots, X_k)$ such that $Y-Y'$ is orthogonal to $\Span(X_1,\hdots, X_k)$.
    \begin{theorem}
    If $\{X_1,\hdots, X_k\}\subset\mathbb{R}^n$ is linearly independent, $\mathcal{W} = \Span(X_1,\hdots, X_n)$, and if $Y\in \mathbb{R}^n$ is such that $Y\perp X_i$ for $i=1,2,\hdots, k$, then $\forall_{Z\in \mathcal{W}}$, $Y\perp Z$.
    \end{theorem}
    \begin{proof}
    For let $Y\in \mathbb{R}^n$ be such that for $i=1,2,\hdots, k$, $Y\perp X_i$. Let $Z\in \mathcal{W}$. Then $Z= \sum_{i=1}^{k} a_i X_i$, where $a_i\in \mathbb{R}$. But then $\langle Y, Z\rangle = \sum_{i=1}^{k} a_i \langle Y, X_i\rangle = \sum_{i=1}^{k} a_i\cdot 0 = 0$. 
    \end{proof}
    \begin{theorem}
        If $P$ is an $n\times k$ matrix whose columns are linearly independent,
        then $P^TP$ is invertible.
    \end{theorem}
    \begin{proof}
    If $P^TPX = 0$, then $PX$ is orthogonal to the columns of $P$. But $PX$ is a linear combination of the columns of $P$, and thus $PX$ is orthogonal to itself. Thus, $PX = 0$. But the columns of $P$ are linearly independent, if $PX = 0$, then $X=0$. Thus $P^TPX = 0$ if and only if $X=0$. $P^TP$ is invertible.
    \end{proof}
    We need $X_{i}^T(Y-Y')=0$. Let $X_{i}=(x_{1i},x_{2i},\hdots,x_{ni})^{T},P=(x_{ij})$. Then $P^T(Y-Y') = 0$, so $P^TY = P^T Y'$. But $Y'\in \mathcal{W}$, so $Y' = \sum_{i=1}^{k} c_i X_i = P(c_1,\hdots, c_k)^T = PC$. Thus, $C = (P^TP)^{-1}P^TY$. Therefore $Y'=P(P^TP)^{-1}P^T Y$.
    \begin{definition}
    The projection matrix of $\Span(X_{1},\hdots,X_{k})$ is $P(P^TP)^{-1}P^T$, where $P=(x_{ij})$.
    \end{definition}
    \begin{theorem}
    If $Q = P(P^TP)^{-1}P^T$ is a projection matrix for a subspace $\mathcal{W}$, then $Q^T =Q$.
    \end{theorem}
    \begin{proof}
    $Q^T=(P(P^{T}P)^{-1}P^{T})^{T}=(P^{T})^{T}(P(P^{T}P)^{-1})^{T}= P((P^{T}P)^{-1}g)^{T}P^{T}=P(P^{T}P)^{-1}P^{T}=Q$
    \end{proof}
    \begin{theorem}
    If $Q = P(P^TP)^{-1}P^T$ is a projection matrix for a subspace $\mathcal{W}$, then $Q^2 = Q$.
    \end{theorem}
    \begin{proof}
    $Q^{2}=P(P^TP)^{-1}P^TP(P^TP)^{-1}P^T= P((P^{T}P)^{-1}(P^{T}P))(P^{T}P)^{-1}P^{T}=P(P^{T}P)^{-1}P^{T}=Q$
    \end{proof}
    \begin{theorem}
    If $Q$ is an $n\times n$ matrix, $Q = Q^{2}$, and $Q=Q^{T}$, then there is a subspace $\mathcal{W}\subset \mathbb{R}^{n}$ such that $Q$ is the projection matrix of $\mathcal{W}$.
    \end{theorem}
    \subsection{Reflections}
    Let $\mathcal{W}$ be a plane passing through the origin, and suppose we want to reflect a vector $v$ across this plane. Let $u$ be a unit vector along $\mathcal{W}^{\perp}$. That is, $u$ is normal to the plane. The projection of $v$ along the line through $u$ is then given by $\hat{v} = Proj_{u}(v) = u(u^Tu)^{-1}u^Tv$. But $u$ is a unit vector, and therefore $u^Tu = 1$, so $\hat{v} = uu^T v$. Let $Q_u = uu^T$. The definition of the reflection of $v$ across $\mathcal{W}$ is the vector $\Refl_{\mathcal{W}}(v)$ such that has the same magnitude as $v$ lying on the opposite side of $\mathcal{W}$. Thus $v-\Refl_{\mathcal{W}}(v) = 2Q_u v$, and so we have:
    \begin{equation*}
        \Refl_{\mathcal{W}}(v) = v-2Q_u v = (I-2Q_u)v =(I-2uu^T)v
    \end{equation*}
    \begin{definition}
    $H_{\mathcal{W}} = I-2uu^T$ is called the Reflection (Householder) Matrix for $\mathcal{W}$.
    \end{definition}
    \begin{definition}
    An orthogonal matrix is a matrix $P$ such that $P^TP = I$.
    \end{definition}
    \subsection{Lecture Notes on Orthogonal Matrices}
    \begin{definition}
    An orthoganal matrix is a $n\times n$ matrix $A$ such that $A^{T}A = I$.
    \end{definition}
    \begin{theorem}
    If $A$ is an orthogonal matrix, then $A^T = A^{-1}$.
    \end{theorem}
    \begin{proof}
    For $A^TA = I$, and inverses are unique. Thus $A^T = A^{-1}$.
    \end{proof}
    If we let $A_{i} = Ae_{i}$, then $A^TA = (A_{i}^{T}A_{j}) = I$. Therefore $A_i^TA_j = \delta_{ij}$.
    \begin{theorem}
    If $A$ is an $n\times n$ real-valued matrix and $A_i = Ae_i$, $i=1,2,\hdots, n$, then $A$ is orthogonal if and only if $\{A_1,\hdots, A_n\}$ is an orthonormal set of vectors.
    \end{theorem}
    \begin{proof}
    If $A$ is orthogonal, then $A_{i}^{T}A_{j} = \delta_{ij}$, and from this we have orthonormality. If $\{A_1,\hdots, A_n\}$ is orthonormal, then $A^TA = I$ and is therefore orthogonal.
    \end{proof}
    \begin{theorem}
    The following statements are equivalent:
    \begin{enumerate}
    \begin{multicols}{3}
        \item $A$ is orthogonal
        \item $\forall_{X\in\mathbb{R}^{n}}$, $\norm{AX} = \norm{X}$
        \item $\forall_{X,Y\in\mathbb{R}^{n}}$, $\langle AX, AY\rangle = \langle X, Y\rangle$
    \end{multicols}
    \end{enumerate}
    \end{theorem}
    \begin{proof}
    We show $1\Rightarrow 2 \Rightarrow 3 \Rightarrow 1$.
    \begin{enumerate}
        \item If $A^TA = I$, then $\norm{AX}^{2}=(AX)^{T}AX=X^{T}A^{T}AX=X^{T}X=\norm{X}^{2}$. Therefore, $\norm{AX}=\norm{X}$.
        \item If $A$ is a square matrix such that $\forall_{X\in\mathbb{R}^{n}}$, $\norm{AX} = \norm{X}$, then:
        \begin{equation*}
            \norm{X+Y}^{2}=(X+Y)^T(X+Y)=X^TX+X^TY+Y^TX+Y^TY=\norm{X}^2+2X^TY+\norm{Y}^2
        \end{equation*}
        But:
        \begin{equation*}
            \norm{A(X+Y)}^{2}=\norm{AX+AY}^{2}=\norm{AX}^{2}+2(AX)^{T}AY+\norm{AY}^2=\norm{X}^{2}+2(AX)^{T}AY+\norm{Y}^2
        \end{equation*}
        Therefore $(AX)^TAY = X^TY$. That is, $\langle AX, AY\rangle = \langle X, Y\rangle$.
        \item If $A$ is a square matrix such that $\forall_{X,Y\in \mathbb{R}^n}$, $\langle AX, AY\rangle = \langle X, Y\rangle$, then $\langle Ae_{i}, Ae_{j}\rangle=\langle e_i,e_j\rangle=\delta_{ij}$
        Therefore, $A$ is orthogonal.
    \end{enumerate}
    \end{proof}
    \begin{theorem}
    If $A$ and $B$ are $n\times n$ orthogonal matrices, then $AB$ is orthogonal.
    \end{theorem}
    \begin{proof}
    For if $A^{T}A = I$ and $B^{T}B = I$, then $(AB)^{T}AB = B^{T}A^{T}AB = B^{T}IB = B^{T}B = I$. $AB$ is orthogonal.
    \end{proof}
    \begin{theorem}
    \label{thm:ortho_matrices_have_det_pm_1}%
    If $A$ is an $n\times{n}$ orthogonal matrix, then $\det(A)\pm{1}$.
    \end{theorem}
    \begin{proof}
    For $\det(I) = \det(A^TA) = \det(A^T)\det(A) = \det(A)^2$. Thus, $\det(A) = \pm 1$.
    \end{proof}
    The converse of Thm.~\ref{thm:ortho_matrices_have_det_pm_1} is false.
    Recall that if $u\in\mathbb{R}^{n}$ is a unit vector and $W=u^{\perp}$,
    then $H=2uu^{T}$ is the reflection matrix for $W$. Reflections preserve
    distance, and therefore $H$ must be orthogonal.
    \newpage
    \begin{theorem}
    If $A$ is an $n\times n$ orthogonal matrix, then there exist $k$ $n\times n$ reflection matrices $H_1,\hdots, H_k$, $0\leq k \leq n$, such that $A = \prod_{i=1}^{j}H_i$.
    \end{theorem}
    \begin{proof}
    By induction. The base case is trivial. Suppose it holds for $n-1$. Let $z = Ae_n$, and let $H$ be the reflection matrix that exchanges $z$ and $e_n$. Then $HAe_n = Hz = e_n$, so $HA$ fixes $e_n$. But $HA$ is an orthogonal matrix, and thus preserves distances and angles. Thus $HA$ maps $\mathbb{R}^{n-1}$ onto itself, and thus by induction there are $H_2,\hdots, H_k$ such that $HA = \prod_{i=2}^{k} H_i$. Letting $H_{1}=H$, we have $A = HHA = \prod_{i=1}^{k}H_i$.
    \end{proof}
    \begin{theorem}
    If $H$ is a reflection matrix, then $\det(H) = -1$.
    \end{theorem}
    \begin{theorem}
    If $A$ is an orthogonal matrix and $A=\prod_{i=1}^{k} H_i$, then $\det(A) = (-1)^k$.
    \end{theorem}
    If $A$ is an orthogonal $2\times 2$ matrix, then we know that columns must be unit vectors that are also orthogonal (Orthonormal). That is, the two columns must lie on the unit circle about the origin. So we may express the first column as $(\cos(\theta),\sin(\theta))$ for some angle $\theta$. There are then two options for the seconds column: $(-\sin(\theta),\cos(\theta))$ or $(\sin(\theta),-\cos(\theta))$. The first is the rotation matrix which rotates $\mathbb{R}^2$ counterclockwise around the origin, and the second is the reflection matrix that makes a reflection across the line that makes an angle $\frac{\theta}{2}$ with the $x-$axis. 
    \begin{theorem}
    If $A$ is a $3\times 3$ orthogonal matrix and $\det(A) = 1$, then $A$ is a rotation matrix.
    \end{theorem}
    \begin{proof}
    $A$ must be the product of $0,1,2,$ or $3$ reflection matrices. If $\det(A) = 1$, then $A$ is the product of an even number of reflections, and thus either $A=I$ or $A$ is the product of two reflections, and is thus a rotation matrix.
    \end{proof}
    \begin{theorem}
    If $A$ is a $3\times 3$ orthogonal matrix, $\det(A) = -1$, and $A=A^T$, then either $A=-I$ or $A$ is a reflection matrix.
    \end{theorem}
    \begin{proof}
    If $\det(A)=-1$, then $A$ is the product of an odd number of reflections, either $1$ or $3$. If $A$ is a single reflection, then $A=H$ for some Householder matrix $H$. Thus $A^T = A$. Conversely, if $A = A^T$ and $\det(A) = -1$, then $\det(-A) = 1$, and $-A^T = -A = -A^{-1}$. Therefore $-A$ is a rotation whose square is the identity. If $A\ne I$, then $A$ must be a rotation of $\pi/2$ around some axis, and thus $A$ is a reflection.
    \end{proof}
    \begin{theorem}
    If $A$ is a $3\times 3$ orthogonal matrix, $\det(A) = -1$, and $A\ne A^T$, then $A$ is the product of three reflections.
    \end{theorem}
    \begin{proof}
    If $\det(A) = -1$, and $A\ne A^T$, then $A$is not a rotation or a pure reflection, and is thus a product of $3$ reflection matrices.
    \end{proof}
    \begin{theorem}
    If $A$ and $B$ are $3\times 3$ rotation matrices, then $AB$ is a rotation matrix.
    \end{theorem}
    \begin{proof}
    For $A$ and $B$ must be orthogonal, and thus $AB$ is orthogonal. But $\det(AB) = \det(A)\det(B) = 1\cdot 1 = 1$, and thus $AB$ is an orthogonal matrix with determinant equal to $1$, and is therefore a rotation matrix.
    \end{proof}
    \subsection{Rotations}
    The $2\times 2$ matrix $A_{\theta}$ rotates the plane $\mathbb{R}^2$ counterclockwise by $\theta$ around the origin. The question that then arises is, ``Is there a similar way to do this for $\mathbb{R}^3$?" The simple case would be rotating by an angle $\theta$ about the $z-$axis, analogous the rotating the Earth by $\theta$ about the North Pole. This fixes the $z-$axis and acts on the $xy$ plane only. This can be represented by the matrix $S_{\theta}$.
    \begin{equation*}
     A_{\theta}=\begin{bmatrix*}[r]\cos(\theta) & -\sin(\theta) \\ \sin(\theta) & \cos(\theta)\end{bmatrix*}\quad\quad\quad\quad S_{\theta}=\begin{bmatrix*}\cos(\theta) & -\sin(\theta) & \phantom{sin}0 \\ \sin(\theta) & \phantom{-}\cos(\theta) & \phantom{sin}0\\ 0 & \phantom{-}0 & \phantom{sin}1 \end{bmatrix*}   
    \end{equation*}
    $S_{\theta}$ is an orthogonal matrix. That is, $S_{\theta} S_{\theta}^T = I$, and therefore $S_{\theta}^T = S_{\theta}^{-1}$. Suppose we want to rotate by an angle $\theta$ about a different axis. Let $\mathbf{u}$ be a unit vector pointing in the direction of the axis of rotation and let $R_{\theta,\mathbf{u}}$ be the new rotation matrix. To compute $R_{\theta,\mathbf{u}}$ choose any unit vector $\mathbf{v}$ that is orthogonal to $\mathbf{u}$. Let $\mathbf{w} = \mathbf{u}\times \mathbf{v}$. Then $\{\mathbf{u},\mathbf{v},\mathbf{w}\}$ is an orthonormal basis of $\mathbb{R}^3$ such that $\mathbf{v}\times \mathbf{w} = \mathbf{u}$. Let:
    \begin{equation*}
        P = \begin{bmatrix} v_1 & w_1 & u_1 \\ v_2 & w_2 & u_2 \\ v_3 & w_3 & u_3 \end{bmatrix}
    \end{equation*}
    The columns of $P$ form an orthonormal set, and therefore $P$ is orthogonal. In particular:
    \begin{align*}
        P^{T}\mathbf{v}&=e_{1}&P^{T}\mathbf{w}&=e_{2}&P^{T}\mathbf{u}=e_{3}
    \end{align*}
    \begin{theorem}
    If $\theta \in [0,2\pi]$ and $\mathbf{u}\in \mathbb{R}^3$ is a unit vector, then $R_{\theta, \mathbf{u}} = PS_{\theta}P^T$.
    \end{theorem}
    \begin{proof}
    For $PS_{\theta}P^T\mathbf{u}=\mathbf{u}$, $PS_{\theta}P^{T}\mathbf{v}=\cos(\theta)\mathbf{v}+\sin(\theta)\mathbf{w}$, and $PS_{\theta}P^{T}\mathbf{w}=-\sin(\theta)\mathbf{v}+\cos(\theta)\mathbf{w}$
    Thus, if $X = a\mathbf{v}+b\mathbf{w}+c\mathbf{u}$, then $PS_{\theta}P^TX=a(\cos(\theta)\mathbf{v}+\sin(\theta)\mathbf{w})+b(-\sin(\theta)\mathbf{v}+\cos(\theta)\mathbf{w})+c\mathbf{u}=R_{\theta,\mathbf{u}}X$
    \end{proof}
    From the orthogonality of $P$ and $S_{\theta}$ we have that $R_{\theta,\mathbf{u}}$ is also orthogonal.
    \begin{theorem}
    \label{thm:Rot_Matrix_is_Ortho_Matrix_with_Det_1}
    A rotation matrix $R$ is an orthogonal matrix with determinant $1$.
    \end{theorem}
    \begin{proof}
    For $R^{T}R=(PS_{\theta}P^{T})^{T}PS_{\theta}P^{T}=PS_{\theta}^{T}P^{T}PS_{\theta}P^{T}=PS_{\theta}^{T}S_{\theta}P^{T}=PP^{T}=I$. But also we have $\det(R)=\det(PS_{\theta}P^T)=\det(P)\det(S_{\theta})\det(P^T)=\det(P)\det(P^{-1})=1$
    \end{proof}
    The converse of Thm.~\ref{thm:Rot_Matrix_is_Ortho_Matrix_with_Det_1} is
    also true. We now turn to the question of how to compute the rotation of
    $\mathbb{R}^3$ represented by a given orthogonal matrix. If $R$ is an
    orthogonal matrix such that $\det(R)=1$, how do we compute the angle of
    rotation? First recall that the trace of a matrix is the sum of the
    diagonal components, $\Tr(A)=\sum_{i=1}^{n}a_{ii}$.
    \begin{theorem}
    If $A$ and $B$ are $n\times n$ matrices, then $\Tr(AB) = \Tr(BA)$
    \end{theorem}
    \begin{theorem}
    If $R$ is a rotation matrix of angle $\theta$, then $\cos(\theta) = \frac{\Tr(R) - 1}{2}$.
    \end{theorem}
    \begin{proof}
    For $\Tr(R)=\Tr(PS_{\theta}P^{-1})=\Tr(PP^{-1}S_{\theta}) = \Tr(S_{\theta})=1+2\cos(\theta)\Rightarrow \cos(\theta)=\tfrac{\Tr(R)-1}{2}$
    \end{proof}
    This doesn't tell us everything, as we still don't know $\mathbf{u}$, and $\cos(\theta) = \cos(-\theta)$, so we still don't know the sign of $\theta$. Since $R$ is an orthogonal matrix, $R^T = R^{-1}$. So if $\mathbf{u}$ lies on the axis of rotation, then $(R-R^T)\mathbf{u} = (R-R^{-1})\mathbf{u} = 0$. We can find the axis of rotation by determining the null space of $R-R^T$. 
    \begin{equation*}
        R = \begin{bmatrix*}[c] r_{11} & r_{12} & r_{13} \\ r_{21} & r_{22} & r_{23} \\ r_{31} & r_{32} & r_{33} \end{bmatrix*} \Rightarrow R-R^{T} = \begin{bmatrix*}[c] 0 & r_{12} - r_{21} & r_{13} - r_{31} \\ r_{21} - r_{12} & 0 & r_{23}-r_{32} \\ r_{31} - r_{13} & r_{32} - r_{23} & 0 \end{bmatrix*}\equiv \begin{bmatrix*}[r] 0 & \alpha & \beta \\ -\alpha & 0 & \gamma \\ -\beta & -\gamma & \phantom{-}0 \end{bmatrix*}
    \end{equation*}
    This suggests that $\mathbf{u}$ is parallel to $(-\gamma, \beta, -\alpha)^{T} = (r_{32}-r_{23}, r_{13}-r_{31}, r_{21}-r_{12})^{T}$.
    \begin{theorem}
    If $R$ is a rotation matrix such that $R\ne R^T$, then the axis of rotation of $R$ is parallel to $\mathbf{q}=(-\gamma, \beta, -\alpha)^{T} = 2\sin(\theta)\mathbf{u}$, where $\mathbf{u}$ is a unit vector about the axis of rotation.
    \end{theorem}
    \begin{proof}
    Let $R = PS_{\theta}P^T$. Then:
    \begin{equation*}
        R-R^{T}=PS_{\theta}P^{T}-PS_{\theta}^{T}P^{T}=P(S_{\theta}-S_{\theta}^{T})P^{T}=2P\begin{bmatrix}0 & -\sin(\theta) & 0 \\ \sin(\theta) & 0 & 0 \\ 0 & 0 & 0 \end{bmatrix}P^{T}= 2\sin(\theta)(\mathbf{w}\mathbf{v}^{T} - \mathbf{v}\mathbf{w}^{T})
    \end{equation*} 
    Where $\mathbf{v}$ is orthogonal to $\mathbf{u}$ and $\mathbf{w} = \mathbf{u}\times \mathbf{v}$. Thus, $\mathbf{q}=(-\gamma,\beta,-\alpha)^{T}=2\sin(\theta)\mathbf{v}\times\mathbf{w}=2\sin(\theta)\mathbf{u}$
    \end{proof}
    What about the case when $R-R^T = 0$? When this happens either $\theta = 0$ or $\theta = \pi$. If $\theta = 0$, then this is the identity rotation and thus $R = I$, and we are done. If $R\ne I$, then $\theta = \pi$. To find out the axis of rotation, we have that:
    \begin{equation*}
        R = PS_{\pi}P^T = \begin{bmatrix*}[r]-1 & 0 & \phantom{-}0 \\ 0 & -1 & 0 \\ 0 & 0 & 1 \end{bmatrix*} = -\mathbf{v}\mathbf{v}^T-\mathbf{w}\mathbf{w}^T +\mathbf{u}\mathbf{u}^T    
    \end{equation*}
    But $\mathbf{v},\mathbf{w},$ and $\mathbf{u}$ form an orthonormal basis,
    and therefore $\mathbf{v}\mathbf{v}^T + \mathbf{w}\mathbf{w}^T+\mathbf{u}\mathbf{u}^T = I$. Thus, $R = -I+2\mathbf{u}\mathbf{u}^T$, so $\mathbf{u} \mathbf{u}^T = \frac{1}{2}(R+I)$. But the columns of $\mathbf{u}\mathbf{u}^T$ are parallel to $\mathbf{u}$, and therefore we can determine $\mathbf{u}$ by normalizing one of the columns of $\frac{1}{2}(R+I)$.
    \subsection{The Matrix Exponential}
    \begin{definition}
    If $A$ is an $n\times n$ matrix, then the exponential of $A$ is $e^{A} =\sum_{k=0}^{\infty} \frac{A^k}{k!}$.
    \end{definition}
    Notationally, we write $A^0 = I$. For any complex-valued matrix $A$ of
    finite dimension, it can be shown that this sum converges.
    \begin{theorem}
    If $A$ and $P$ are complex $n\times n$ matrices and $P$ invertible, then $e^{P^{-1}AP} = P^{-1}e^{A}P$.
    \end{theorem}
    \begin{proof}
    For all $m\in \mathbb{N}$, $(P^{-1}AP)^{m} = P^{-1}A^mP$. Thus:
    \begin{equation*}
        e^{P^{-1}AP} = \sum_{k=0}^{\infty} P^{-1}\frac{A^k}{k!}P = P^{-1}\big(\sum_{k=0}^{\infty} \frac{A^k}{k!}\big)P = P^{-1}e^A P
    \end{equation*}
    \end{proof}
    \begin{theorem}
    If $0$ is the zero matrix, then $e^0 = I$.
    \end{theorem}
    \begin{theorem}
    If $A$ is an $n\times n$ matrix and $m\in \mathbb{N}$, then $A^{m} e^{A} = e^{A} A^{m}$.
    \end{theorem}
    \begin{proof}
    For $A^{m} e^{A} = A^{m} \sum_{k=0}^{\infty} \frac{A^{k}}{k!} = \sum_{k=0}^{\infty} \frac{A^{k+m}}{k!} = \big(\sum_{k=0}^{\infty} \frac{A^k}{k!}\big)A^{m}$.
    \end{proof}
    \begin{theorem}
    If $A$ is an $n\times n$ matrix, then $e^{A^{T}} = (e^{A})^{T}$.
    \end{theorem}
    \begin{proof}
    For $e^{A^{T}} = \sum_{k=0}^{\infty} \frac{(A^{T})^{k}}{k!} = \sum_{k=0}^{\infty} \frac{(A^{k})^{T}}{k!} = \big(\sum_{k=0}^{\infty} \frac{A^{k}}{k!}\big)^{T} = (e^{A})^{T}$.
    \end{proof}
    \begin{theorem}
    If $A$ and $B$ are $n\times n$ matrices and if $AB = BA$, then $Ae^{B} = e^{B} A$.
    \end{theorem}
    \begin{proof}
    For $Ae^{B} = A\sum_{k=0}^{\infty} \frac{B^{k}}{k!} = \sum_{k=0}^{\infty} A\frac{B^{k}}{k!} = \sum_{k=0}^{\infty} \frac{B^{k}}{k!}A = \big(\sum_{k=0}^{\infty} \frac{B^{k}}{k!}\big)A = e^{B}A$.
    \end{proof}
    \begin{theorem}
    If $A$ and $B$ are $n\times n$ matrices and $AB = BA$, then $e^{A}e^{B} = e^{B}e^{A}$.
    \end{theorem}
    \begin{proof}
    For:
    \begin{align*}
        e^A e^B &= e^A\sum_{k=0}^{\infty}\frac{B^k}{k!}=\sum_{k=0}^{\infty} e^A\frac{B^k}{k!}= \sum_{k=0}^{\infty} \big(\sum_{j=0}^{\infty} \frac{A^j}{j!}\big) \frac{B^k}{k!}= \sum_{k=0}^{\infty}\big(\sum_{j=0}^{\infty} \frac{A^j}{j!}\frac{B^k}{k!}\big)\\
        &=\sum_{k=0}^{\infty}\big(\sum_{j=0}^{\infty} \frac{B^k}{k!}\frac{A^j}{j!}\big)=\sum_{k=0}^{\infty}\big(\sum_{j=0}^{\infty} \frac{B^k}{k!}\big)\frac{A^j}{j!}= \sum_{k=0}^{\infty}\frac{B^k}{k!}\sum_{j=0}^{\infty}\frac{A^j}{j!}=e^{B}e^{A}
    \end{align*}
    \end{proof}
    It is NOT true that $e^{A+B}=e^{A}e^{B}$, in general. Matrix exponentiation lacks this feature.
    \begin{theorem}
    If $A$ is an $n\times n$ matrix and $s,t\in \mathbb{C}$, then $e^{A(s+t)} = e^{As}e^{At}$.
    \end{theorem}
    \begin{proof}
    For $e^{As}e^{At} = \sum_{j=0}^{\infty} \sum_{k=0}^{\infty} \frac{A^{j+k}s^jt^k}{j!k!}$. Letting $n = j+k$, so $j = n-k$, we have:
    \begin{equation*}
        \sum_{n=0}^{\infty} \sum_{k=0}^{\infty} \frac{A^n}{n!}\frac{n!}{k!(n-k)!}s^{n-k}t^k = \sum_{n=0}^{\infty}\frac{A^n}{n!}\big(\sum_{k=0}^{\infty} \frac{n!}{k!(n-k)!}s^{n-k}t^k\big)    
    \end{equation*}
    From the binomial theorem, the expression inside the parenthesis is equal to $(s+t)^n$. So we have $e^{As}e^{At}=\sum_{n=0}^{\infty} \frac{A^n(t+s)^n}{n!} = e^{A(s+t)}$.
    \end{proof}
    \begin{theorem}
    If $A$ is an $n\times n$ matrix, then $e^A$ is invertible and $(e^A)^{-1} = e^{-A}$.
    \end{theorem}
    \begin{proof}
    For $I = e^{0} = e^{A(1-1)} = e^Ae^{-A}$. Thus $(e^{A})^{-1} = e^{-A}$.
    \end{proof}
    \begin{theorem}
    If $A$ is an $n\times n$ matrix and $t\in \mathbb{R}$, then $\frac{d}{dt}\big(e^{At}\big) = Ae^{At}$.
    \end{theorem}
    \begin{proof}
    For $\underset{h\rightarrow 0}\lim \frac{e^{A(t+h)}-e^{At}}{h} = e^{At}\underset{h\rightarrow 0}\lim \frac{e^{Ah}-I}{h} = e^{At}\underset{h\rightarrow 0}\lim\big[A+\frac{A^2}{2!}h+\hdots\big] = e^{At}A = Ae^{At}$.
    \end{proof}
    \begin{theorem}
    If $A$ and $B$ are $n\times n$ matrices and $AB=BA$, then $e^{A+B} = e^{A}e^{B}$.
    \end{theorem}
    \begin{proof}
    For let $g(t) = e^{(A+B)t}e^{-Bt}e^{-At}$. Then from commutativity of $A$ and $B$, $g'(t) = 0$. But then $g(t)$ is a constant. From the definition, $g(0) = I$, and thus $g(t) = I$. So $e^{(A+B)t}e^{-Bt}e^{-At} = I$, and therefore $e^{(A+B)t} = e^{At}e^{Bt}$.
    \end{proof}
    \begin{theorem}
    If $A^{2} = 0$, then $e^{A} = I+A$.
    \end{theorem}
    \begin{proof}
    For $e^{A} = I+A+A^{2}\big(\frac{I}{2!}+\frac{A}{3!}+\hdots\big) = I+A+0 = I+A$.
    \end{proof}
    \subsection{Linear Systems of Ordinary Differential Equations}
    Consider the equation $y' = ky$, where $k$ is some constant. We can solve this via calculus using separation of variables:
    \begin{equation*}
        \frac{y'}{y} = k\Rightarrow \int \frac{y'}{y}dx = \int kdx \Rightarrow \ln(y) = kx+c \Rightarrow y = e^c e^{kx}    
    \end{equation*}
    Setting $x=0$, we have $e^c = y_0$. So $y = y_0e^{kx}$. Let us solve this a different way: Let $F(x) = e^{-kx}y$, and let $y'=kx$. Differentiating we have:
    \begin{equation*}
        F'(x)=-ke^{kx}y+e^{-kx}y'=-kye^{-kx}+e^{-kx}ky=0    
    \end{equation*}
    So $F'(x) = 0$, and therefore $F(x)$ is a constant. Setting $x=0$, we have $F(x) = y_0$. So $y = y_0e^{kx}$. This shows us that $y_0e^{kx}$ is the $only$ solution to this problem. Let:
    \begin{equation*}
        Y(t) = \begin{bmatrix} y_1(t) \\ y_2(t)\end{bmatrix}    
    \end{equation*}
    Consider $Y'(t) = AY(t)$, where $A$ is an $n\times n$ matrix. Let $F(t) = e^{-At}Y(t)$. Then $F'(t) = 0$, and $Y(t) = Y_0 e^{At}$.
    \begin{theorem}
    If $Y:\mathbb{R}\rightarrow \mathbb{R}^n$ is a differentiable function such that $Y'(t) = AY(t)$, where $A$ is a diagonalizable matrix with eigenvalues $\lambda_1,\hdots, \lambda_n$ and eigenvectors $v_1,\hdots, v_n$, then $Y(t) = \sum_{k=1}^{n} \lambda_k e^{\lambda_k t}v_k$
    \end{theorem}
\section{Problem Sets}
    \subsection{Problem Set I}
    \begin{problem}
    Find the point on the line $y=4x$ which is closest to the point $(2,5)$.
    \end{problem}
    \begin{proof}[Solution 1]
    Given a vector $\mathbf{v}$ that is parallel to the line $y$, we know that the vector $\mathbf{w}$ from $(2,5)$ to the point $(x,y)$ that minimizes the distance from $y=4x$ to the point $(2,5)$ will satisfy $\langle \mathbf{v}, \mathbf{w}\rangle = 0$. That is:
    \begin{equation*}
        \big\langle (1,4), (2-x,5-y)\big\rangle = 0\Rightarrow 2-x+4(5-y) = 0 \Rightarrow 22 - x - 4 y = 0    
    \end{equation*}
    But $y = 4x$, and thus $22-17x = 0 \Rightarrow x= \frac{22}{17}$. The point of least distance is $\frac{22}{17}(1,4)$.
    \end{proof}
    \begin{proof}[Solution 2]
    This point is the projection of the vector $(2,5)^T$ onto $(1,4)^T$. That is:
    \begin{equation*}
        \mathbf{P} = \frac{\begin{bmatrix}2 & 5 \end{bmatrix} \begin{bmatrix} 1 \\ 4 \end{bmatrix}}{\begin{bmatrix} 1 & 4 \end{bmatrix} \begin{bmatrix} 1 \\ 4 \end{bmatrix}} \begin{bmatrix} 1 \\ 4 \end{bmatrix} = \frac{22}{17} \begin{bmatrix} 1 \\ 4\end{bmatrix}
    \end{equation*}
    \end{proof}
    \begin{problem}
    Show that $\mathbf{x}\mathbf{y}^T + \mathbf{y}\mathbf{x}^T$ is symmetric.
    \end{problem}
    \begin{proof}[Solution]
    Recall that a matrix is symmetric if it is equal to its transpose. Thus, we must show $A = A^T$. But for any $n\times n$ matrices $A$ and $B$, $(A+B)^T = A^T + B^T$, and $(AB)^T = B^T A^T$, and $(A^T)^T = A$. Thus, given our matrix $A= \mathbf{x}\mathbf{y}^T + \mathbf{y}\mathbf{x}^T$, we have that $A^T = (\mathbf{x}\mathbf{y}^T + \mathbf{y}\mathbf{x}^T)^T = (\mathbf{x}\mathbf{y}^T)^T + (\mathbf{y}\mathbf{x}^T)^T = (\mathbf{y}^T)^T\mathbf{x}^T + (\mathbf{x}^T)^T\mathbf{y}^T = \mathbf{y}\mathbf{x}^T + \mathbf{x}\mathbf{y}^T = \mathbf{x}\mathbf{y}^T + \mathbf{y}\mathbf{x}^T = A$
    \end{proof}
    \begin{problem}
    Compute the product $\begin{bmatrix*}[r] 2 & -1 \\ 3 & 1\end{bmatrix*} \begin{bmatrix*}[r] -1 & \phantom{-}2 & \phantom{-}3 & \phantom{-}1 \\ 2 & -2 & 1 & -1 \end{bmatrix*}$
    \end{problem}
    \begin{proof}[Solution]
    \begin{align*}
        \begin{bmatrix*}[r] 2 & -1 \\ 3 & 1\end{bmatrix*} \begin{bmatrix*}[r] -1 & \phantom{-}2 & \phantom{-}3 & \phantom{-}1 \\ 2 & -2 & 1 & -1 \end{bmatrix*}&=\begin{bmatrix} 2(-1)+(-1)2 & 2\cdot 2 + (-1)(-2) & 2\cdot 3 + (-1)1 & 2\cdot 1 + (-1)(-1) \\ 3(-1)+1\cdot 2 & 3\cdot 2 + 1(-2) & 3\cdot 3 + 1\cdot 1 & 3\cdot 1 + 1(-1)\end{bmatrix}\\
        &=\begin{bmatrix} -4 & 6 & 5 &3 \\ -1 & 4 & 10 & 2\end{bmatrix}
    \end{align*}
    \end{proof}
    \begin{problem}
    Find the equation of the plane that contains $P_{1}(2,2,1),P_{2}(2,3,2)$, and $P_{3}(-1,3,1)$.
    \end{problem}
    \begin{proof}[Solution]
    It suffices to find a vector normal to this plane. We have that:
    \begin{equation*}
        \overrightarrow{P_1P_2} = (0,1,1)^T \quad\quad\quad\quad \overrightarrow{P_1P_3} = (-3,1,0)^T
    \end{equation*}
    Then both vectors are parallel to the plane, and thus $\overrightarrow{P_1P_2}\times \overrightarrow{P_1P_3}=(-1,3,3)^T$ is perpendicular to the plane. Suppose $Q=(x,y,z)$ is a point in the plane. Then the relative position vector $P_1 Q = (x-2,y-2,z-1)^T$ is orthogonal to $(-1,3,3)^T$. Thus:
    \begin{align*}
        (x-2,y-2,z-1)(-1,3,3)^T &= 0\\
        \Rightarrow 2-x+3y-6+3z-3 &= 0\\
        \Rightarrow x-3y-3z +7 &= 0   
    \end{align*}
    This is the equation of the plane.
    \end{proof}
    \begin{problem}
    Let $S=\Span(\mathbf{x}_1,\mathbf{x}_{2})$, where $\mathbf{x}_{1}=(1,-1,2)^{T}$, $\mathbf{x}_{2}=(-1,2,2)^{T}$. Find a basis for $S^{\perp}$
    \end{problem}
    \begin{proof}[Solution]
    We seek a vector in $\mathbf{x}_3\in\mathbb{R}^3$ such that $\langle \mathbf{x}_3, \mathbf{x}_{i}\rangle = 0$, $i=1,2$. That is:
    \begin{equation*}
        \begin{bmatrix*}[r] 1 & -1 & \phantom{-}2 \\ 0 & 1 & 4 \end{bmatrix*}\begin{bmatrix} x_1 \\ x_2 \\ x_3 \end{bmatrix} = 0    
    \end{equation*}
    Solving gives us $x_2 = -4x_3$, $x_1=-6x_3$. $\{(-6,-4,1)\}$ is a basis.
    \end{proof}
    \begin{problem}
    For the matrix $A = \begin{bmatrix} 1 & 2 & 2 \\ -1 & -1 & 0 \end{bmatrix}$, find a basis for the following:
    \begin{enumerate}
    \begin{multicols}{4}
        \item $R(A^T)$
        \item $N(A)$
        \item $R(A)$
        \item $N(A^T)$
    \end{multicols}
    \end{enumerate}
    \end{problem}
    \begin{proof}[Solution]
    The row-echelon form of $A$ and $A^{T}$ are given below:
    \begin{align*}
        A'&=\begin{bmatrix}1&1&0\\0&1&2\end{bmatrix} & (A^{T})'&=\begin{bmatrix*}[r]1&0\\0&-1\\0&0\end{bmatrix*}
    \end{align*}
    \begin{enumerate}
        \item The rows of $A'$ give us a basis for $R(A^T)$ of $\{(1,1,0),(0,1,2)\}$
        \item $N(A) = \{x\in \mathbb{R}^3: Ax = 0\}$. Solving $A'x=0$ gives us a basis of $\{(2,-2,1)\}$
        \item The non-zero rows of $(A^{T})'$ give us a basis of $\{(1,0), (0,-1)\}$.
        \item $N(A^T)= \{x\in \mathbb{R}^2: A^T x = 0\}$. $A'x = 0$ gives us $x_1 = 0$ and $-x_2 = 0$. $N(A^T) = \{(0,0)\}$.
    \end{enumerate}
    \end{proof}
    \subsection{Problem Set II}
    \begin{problem}
    Find a point on the line $y=5x$ that is closest to the point $(1,3)$.
    \end{problem}
    \begin{proof}[Solution]
    Pick a point on the line, say $\mathbf{w} = (1,5)^T$. The point $P$ is the projection of $\mathbf{v} = (1,3)^T$ onto the line $y=5x$, and thus:
    \begin{equation*}
        P = \frac{v^T w}{w^T w} = \frac{\begin{bmatrix}1 & 5 \end{bmatrix}\begin{bmatrix}1 \\ 5\end{bmatrix}}{\begin{bmatrix}1 & 5 \end{bmatrix}\begin{bmatrix}1 \\ 5\end{bmatrix}}(1,5)^T = \frac{8}{13}(1,5)^T
    \end{equation*}
    \end{proof}
    \begin{problem}
    Is $A = xy^T - yx^T$ symmetric? ($x$ and $y$ are $n\times 1$ vectors)
    \end{problem}
    \begin{proof}[Solution]
    In general, no. For if it were, then $A-A^T = 0$. But:
    \begin{align*}
        0&=A-A^T=xy^{T}-yx^{T}-(xy^{T}-yx^{T})^{T}=xy^{T}-yx^{T}-[(xy^{T})^{T}-(yx^{T})^{T}]\\
        &=xy^T - yx^T - [yx^T - xy^T]=2xy^T-2yx^T=2A\Rightarrow xy^{T}-yx^{T}=0\Rightarrow xy^{T}=yx^{T} 
    \end{align*}
    As this is not, in general, true, $A$ is not necessarily symmetric.
    \end{proof}
    \begin{problem}
    Compute the product $\begin{bmatrix*}[r] -1 & \phantom{-}3 \\ 4 & 2 \end{bmatrix*} \begin{bmatrix*}[r] -1 & \phantom{-}1 & \phantom{-}2 & -2 \\ 2 & 3 & 1 & 1 \end{bmatrix*}$
    \end{problem}
    \begin{proof}[Solution]
    \begin{align*}
        \begin{bmatrix*}[r] -1 & \phantom{-}3 \\ 4 & 2 \end{bmatrix*} \begin{bmatrix*}[r] -1 & \phantom{-}1 & \phantom{-}2 & -2 \\ 2 & 3 & 1 & 1 \end{bmatrix*}=\begin{bmatrix*}[r] \phantom{-}1+6 & -1+9 & -2+3 & \phantom{-}2+3 \\ -4+4 & \phantom{-}4+6 & \phantom{-}8+2 & -8+2 \end{bmatrix*}=\begin{bmatrix*}[r] 7 & 8 & 1 & 5 \\ 0 & 10 & 10 & -6 \end{bmatrix*}
    \end{align*}
    \end{proof}
    \begin{problem}
    Find the equation of the plane that passes through $P_1(2,2,2), P_2(2,3,4), P_3(-1,3,3)$.
    \end{problem}
    \begin{proof}[Solution]
    $\overrightarrow{P_1P_2} = (0,1,2)^{T}$, $\overrightarrow{P_1 P_3} = (-3,1,1)^{T}$. So:
    \begin{equation*}
        \overrightarrow{N} = \begin{vmatrix*}[r] \hat{\mathbf{i}} & \hat{\mathbf{j}} & \hat{\mathbf{k}} \\ 0 & 1 & 2 \\ -3 & \phantom{-}1 & \phantom{-}1 \end{vmatrix*} = \hat{\mathbf{i}}(1-2) + \hat{\mathbf{j}}(0+6) + \hat{\mathbf{k}}(0+3)=\begin{bmatrix*}[r]-1 \\ -6 \\ 3\end{bmatrix*}   
    \end{equation*}
    For a point $P=(x,y,z)$ in the plane, $\langle \overrightarrow{P_1P}, \overrightarrow{N}\rangle = 0$. Thus, $x + 6y - 3z =0$
    \end{proof}
    \begin{problem}
    Let $S=\Span(\{(2,1,2)^T, (-2,-1,3)^T\})$. Find a basis for $S^{\perp}$.
    \end{problem}
    \begin{proof}[Solution]
    Let $A$ and it's row-echelon form be the matrices shown below. Then $S^{\perp} = N(A)$.
    \begin{align*}
        A&=\begin{bmatrix*}[r] 2 & 1 & 2 \\ -2 & -1 & \phantom{-}3\end{bmatrix*} & A'&=\begin{bmatrix} 2 & 1 & 2 \\ 0 & 0 & 5 \end{bmatrix}
    \end{align*}
    Solving for $A'x = 0$ gives us a basis of $\{(1,-2,0)\}$
    \end{proof}
    \begin{problem}
    For the matrix $A = \begin{bmatrix*}[r] 2 & 3 & 4 \\ -2 & -2 & \phantom{-}0 \end{bmatrix*}$, find a basis for the following:
    \begin{enumerate}
    \begin{multicols}{4}
        \item $R(A^T)$
        \item $N(A)$
        \item $R(A)$
        \item $N(A^T)$
    \end{multicols}
    \end{enumerate}
    \end{problem}
    \begin{proof}[Solution]
    $A$ and $A^{T}$ have the following row-echelon forms:
    \begin{align*}
        A'&=\begin{bmatrix}1&1&0\\0&1&4\end{bmatrix} & (A^{T})'=&\begin{bmatrix}1&0\\0&1\\0&0\end{bmatrix}
    \end{align*}
    \begin{enumerate}
        \item Putting $A$ into row-echelon  form and reading off the rows, we obtain the basis $\{(1,1,0),(0,1,4)\}$
        \item $N(A) = \{x\in \mathbb{R}^3:  Ax = 0\}$. This gives us a basis of $\{(4,-4,1)\}$
        \item The non-zero rows of $(A^{T})'$ give us a basis of $\{(1,0),(0,1)\}$
        \item $N(A^T) = \{x\in \mathbb{R}^2: A^Tx = 0\}$. Solving $A'^{T}x=0$ gives us $x_1 = 0$ and $x_2 = 0$. $N(A^T) = \{(0,0)\}$
    \end{enumerate}
    \end{proof}
    \subsection{Problem Set III}
    \begin{problem}
    Let $A,B,C$ be $n\times n$ matrices. Is $A = BC^T + CB^T$ symmetric?
    \end{problem}
    \begin{proof}[Solution]
    A matrix is symmetric if $A = A^T$. If $A = BC^T+CB^T$, then:
    \begin{align*}
        A^{T}=(BC^{T}+CB^{T})^{T}=(BC^{T})^{T}+(CB^{T})^{T}=(C^{T})^{T}B^{T}+(B^{T})^{T}C^{T}=CB^{T}+BC^{T}=A
    \end{align*}
    $A$ is symmetric.
    \end{proof}
    \begin{problem}
    Compute $\norm{x}_1, \norm{x}_2, \norm{x}_3$ for $x = (2,-3,1)^T$
    \end{problem}
    \begin{proof}[Solution]
    By definition, for $x\in \mathbb{R}^n$, $\norm{x}_p = (\sum_{k=1}^{n}|x_k|^p)^{1/p}$. So we have the following:
    \begin{enumerate}
        \item $\norm{x}_1 = |2|+|-3|+|1| = 2+3+1 = 6$
        \item $\norm{x}_2 = (|2|^2+|-3|^2+|1|^2)^{1/2} = (4+9+1)^{1/2} = \sqrt{14}$
        \item $\norm{x}_3 = (|2|^3+|-3|^3+|1|^3)^{1/3} = (8+27+1)^{1/3} = \sqrt[3]{36}$
    \end{enumerate}
    \end{proof}
    \newpage
    \begin{problem}
    For the matrix $A = \begin{bmatrix*}[r] \phantom{-}2 & -2 & \phantom{-}4 \\ -1 & 1 & -2 \end{bmatrix*}$, find a basis for the following:
    \begin{enumerate}
    \begin{multicols}{4}
        \item $R(A^T)$
        \item $N(A)$
        \item $R(A)$
        \item $N(A^T)$
    \end{multicols}
    \end{enumerate}
    \end{problem}
    \begin{proof}[Solution]
    $A$ and $A^{T}$ have the following row-echelon forms:
    \begin{align*}
       A'&=\begin{bmatrix*}[r]1&-1&2\\0&0&0\end{bmatrix*} & (A^{T})'&=\begin{bmatrix*}[r]-2&1\\0&0\\0&0\end{bmatrix*}
    \end{align*}
    \begin{enumerate}
        \item The non-zero rows of $A'$ give a basis of $\{(1,-1,2)\}$
        \item $N(A) = \{x\in \mathbb{R}^3: Ax = 0\}$. Solving $A'x=0$ gives a basis of $\{(1,1,0),(-2,0,1)\}$
        \item The non-zero rows of $(A^{T})'$ give a basis of $\{(-2,1)\}$
        \item $N(A^T) = \{x\in \mathbb{R}^2: A^T x = 0\}$. Solving $(A^{T})'x=0$ gives a basis of $\{(1,2)\}$
    \end{enumerate}
    \end{proof}
    \begin{problem}
    Find the least-squares solution to the following system:
    \begin{align*}
        x_{1}-x_{2}\phantom{2} &=2\\
        x_{1}+x_{2}\phantom{2} &= 0\\
        x_{1}+2x_{2} &=-1
    \end{align*}
    \end{problem}
    \begin{proof}[Solution]
    We want the solution to $A^T A x = A^T b$. We have:
    \begin{align*}
        A&= \begin{bmatrix*}[r]1&-1\\1&1\\1&2\end{bmatrix*} & b&=\begin{bmatrix*}[r]2\\0\\-1\end{bmatrix*} & A^{T}Ax&=A^{T}b\Rightarrow\begin{bmatrix}3&1\\1&9\end{bmatrix} \begin{bmatrix}x_{1}\\x_{2}\end{bmatrix}=\begin{bmatrix*}[r]1\\-6\end{bmatrix*}
    \end{align*}
    The solution is $x = \frac{1}{26}(15,-19)^T$
    \end{proof}
    \begin{problem}
    Let $\theta\in\mathbb{R}$ and let $\mathbf{x}_1 = (\cos(\theta), \sin(\theta))^{T}$, $\mathbf{x}_2 = (-\sin(\theta), \cos(\theta))^{T}$. Show that $\{\mathbf{x}_1,\mathbf{x}_2\}$ is an orthonormal basis for $\mathbb{R}^2$. Write $\mathbf{y}=(-2,3)^{T}$ as a linear combination $\mathbf{y}=c_{1} \mathbf{x}_{1}+c_{2}\mathbf{x}_{2}$
    \end{problem}
    \begin{proof}[Solution]
    They are orthonormal for $\mathbf{x}_1^T \mathbf{x}_2 = -\cos(\theta)\sin(\theta) + \cos(\theta)\sin(\theta) = 0$, and since $\norm{\mathbf{x}_1}=\norm{\mathbf{x}_2}= (\sin^2(\theta)+\cos^2(\theta))^{1/2}=1$. Let $c_{1}=\langle\mathbf{y},\mathbf{x}_{1}\rangle$ and $c_{2}=\langle\mathbf{y},\mathbf{x}_{2}\rangle$. Then $c_1 = -2\cos(\theta)+3\sin(\theta)$ and $c_2 = 2\sin(\theta)+3\cos(\theta)$. Therefore, $\mathbf{y}=(-2\cos(\theta)+3\sin(\theta))\mathbf{x}_1+(2\sin(\theta)+3\cos(\theta)\mathbf{x}_2)$
    \end{proof}
    \subsection{Problem Set IV}
    \begin{problem}
    Find the eigenvalues and associated eigenspaces of $A = \begin{bmatrix}4 & 5 \\ 2 & 1 \end{bmatrix}$
    \end{problem}
    \begin{proof}[Solution]
    We need to compute $\det(A-\lambda I)=0$. This gives us:
    \begin{equation*}
        \begin{vmatrix} 4-\lambda & 5 \\ 2 & 1-\lambda \end{vmatrix} = (4-\lambda)(1-\lambda)-10 = 0
    \end{equation*}
    The solutions to this are $\lambda_1 = 6, \lambda_2 = -1$. Solving $Ax = \lambda x$ yields the eigenspaces. We have:
    \begin{equation*}
        \begin{bmatrix} 4 & 5 \\ 2 & 1 \end{bmatrix} \begin{bmatrix} x_1 \\ x_2 \end{bmatrix}=-\begin{bmatrix} x_1 \\ x_2 \end{bmatrix}\quad\quad\quad\quad\begin{bmatrix} 4 & 5 \\ 2 & 1 \end{bmatrix} \begin{bmatrix} x_1 \\ x_2 \end{bmatrix}=6\begin{bmatrix} x_1 \\ x_2 \end{bmatrix}
    \end{equation*}
    These give solutions $x_2(-1,1)^T$ and $x_2 (\frac{5}{2},1)^T$, where $x_2$ is a free variable.
    \end{proof}
    \begin{problem}
    Show that for a $2\times 2$ matrix $A$, $\lambda^2 - \Tr(A)\lambda + \det(A) = 0$, where $\lambda$ is an eigenvalue of $A$.
    \end{problem}
    \begin{proof}[Solution]
    For we have that $\det(A-\lambda I) = 0$. But:
    \begin{equation*}
        \det(A-\lambda I)=\begin{vmatrix} a-\lambda & b \\ c & d-\lambda \end{vmatrix}=(a-\lambda)(d-\lambda)-bc=\lambda^2-(a+d)\lambda+ad-bc=\lambda^{2}-\Tr(A)\lambda+\det(A)
    \end{equation*}
    Therefore, if $\lambda$ is an eigenvalue of $A$, then $\lambda^2 - \Tr(A) \lambda + \det(A) = 0$.
    \end{proof}
    \begin{problem}
    Find the eigenvalues and associated eigenspaces for $A = \begin{bmatrix} 1 & 1 & 1 \\ 0 & 2 & 1 \\ 0 & 0 & 3\end{bmatrix}$
    \end{problem}
    \begin{proof}[Solution]
    Recall that the determinant expansion can be done along any row. Thus:
    \begin{align*}
        \det(A-\lambda I) &= \begin{vmatrix} 1-\lambda & 1 & 1 \\ 0 & 2-\lambda & 1 \\ 0 & 0 & 3-\lambda \end{vmatrix}=0\begin{vmatrix} 1 & 1 \\ 2-\lambda & 1 \end{vmatrix}-0 \begin{vmatrix} 1-\lambda & 1 \\ 0 & 1 \end{vmatrix} + (3-\lambda)\begin{vmatrix} 1-\lambda & 1 \\ 0 & 2-\lambda\end{vmatrix}\\
        &= (3-\lambda)(1-\lambda)(2-\lambda)    
    \end{align*}
    The solutions are $\lambda_1 = 1,\ \lambda_2 = 2,\ \lambda_3 = 3$. The eigenspaces correspond to the solutions of the equation $Ax = \lambda x$. Thus we get:
    \begin{equation*}
        \begin{bmatrix} 1 & 1 & 1 \\ 0 & 2 & 1 \\ 0 & 0 & 3 \end{bmatrix}\begin{bmatrix} x \\ y \\ z \end{bmatrix} = \lambda \begin{bmatrix}x \\ y \\ z\end{bmatrix}    
    \end{equation*}
    This gives 3 different equations for each value of $\lambda$.
    \begin{equation*}
        Ax=x\Rightarrow x=(1,0,0)^{T}\quad\quad\quad\quad Ax=2x\Rightarrow x=(1,1,0)^{T}\quad\quad\quad\quad Ax=3x\Rightarrow x=(1,1,1)^{T}
    \end{equation*}
    \end{proof}
    \subsection{Problem Set V}
    \begin{problem}
    Factor $\begin{bmatrix} 4 & 2 \\ 2 & 1 \end{bmatrix}$ into the form $PDP^T$, where $D$ is a diagonal and $P$ is orthogonal.
    \end{problem}
    \begin{proof}[Solution]
    The eigenvalues of $A$ are the solutions to $(4-\lambda)(1-\lambda)-4=0$: $\lambda_1 = 0$, $\lambda_2 = 5$. The eigenvectors are solutions to:
    \begin{equation*}
        \begin{bmatrix} 4 & 2 \\ 2 & 1 \end{bmatrix} \begin{bmatrix} x \\ y \end{bmatrix} = \lambda \begin{bmatrix} x \\ y \end{bmatrix}
    \end{equation*}
    Which gives us $\frac{1}{\sqrt{5}}(2,1)^T$ and $\frac{1}{\sqrt{5}}(-1,2)^T$. Thus:
    \begin{equation*}
        P = \frac{1}{\sqrt{5}}\begin{bmatrix} -1 & 2 \\ 2 & 1 \end{bmatrix}\quad\quad D = \begin{bmatrix} 0 & 0 \\ 0 & 5 \end{bmatrix}\quad\quad P^{T} = \frac{1}{\sqrt{5}}\begin{bmatrix} -1 & 2 \\ 2 & 1 \end{bmatrix}
    \end{equation*}
    \end{proof}
    \begin{problem}
    Solve the differential equation $Y'(t) = \begin{bmatrix} 4 & 2 \\ 2 & 1 \end{bmatrix} Y(t)$ with $Y(0) = \begin{bmatrix} -1 \\ 4 \end{bmatrix}$
    \end{problem}
    \begin{proof}[Solution]
    We know from the previous problem that the eigenvalues and eigenvectors are distinct, and thus $Y(t) = \alpha V_1 e^{\lambda_1 t} + \beta V_2 e^{\lambda_2 t}$ where $\lambda_{i}$ are the distinct eigenvalues, and $V_{i}$ are the distinct eigenvectors. Solving for the initial condition:
    \begin{equation*}
        \frac{1}{\sqrt{5}}\begin{bmatrix} 2 & -1 \\ 1 & 2 \end{bmatrix}\begin{bmatrix} \alpha \\ \beta \end{bmatrix}=\begin{bmatrix} -1 \\ 4 \end{bmatrix}\Rightarrow \begin{bmatrix} \alpha \\ \beta \end{bmatrix}=\frac{1}{\sqrt{5}}\begin{bmatrix} -1 & 2 \\ 2 & 1 \end{bmatrix}\begin{bmatrix} -1 \\ 4 \end{bmatrix}=\frac{1}{\sqrt{5}} \begin{bmatrix} 9 \\ 2 \end{bmatrix}
    \end{equation*}
    Thus, $Y(t) = \frac{9}{5}(-1,2)^T + \frac{2}{5} (2,1)^T e^{5t}$ 
    \end{proof}
    \begin{problem}
    Solve the following:
    \begin{enumerate}
        \item Let $A$ be an $n\times n$ complex Hermitian matrix such that $A^4=I$. What are the possible eigenvalues of $A$?
        \item If $A$ is an $n\times n$ complex matrix and $A^4 = I$, what are the possible eigenvalues?
    \end{enumerate}
    \end{problem}
    \begin{problem}
    Using least squares, find the line in $\mathbb{R}^2$ that best fits $\{(2,1),\ (3,2),\ (4,2),\ (5,3)\}$
    \end{problem}
    \begin{proof}[Solution]
    We want a line $y=mx+b$ that best fits the points. Setting up the problem, we get:
    \begin{equation*}
        \begin{bmatrix} 1 & 2 \\ 1 & 3 \\ 1 & 4 \\ 1 & 5 \end{bmatrix} \begin{bmatrix} b \\ m \end{bmatrix} = \begin{bmatrix} 1 \\ 2 \\ 2 \\ 3\end{bmatrix}   
    \end{equation*}
    This has no solution. Let $A$ be the left-most matrix. Then:
    \begin{equation*}
        A^T = \begin{bmatrix} 1 & 1 & 1 & 1 \\ 2 & 3 & 4 & 5 \end{bmatrix}\Rightarrow A^{T}A = \begin{bmatrix} 4 & 14 \\ 14 & 54 \end{bmatrix}
    \end{equation*}
    We now solve $A^{T}AX$:
    \begin{equation*}
        \begin{bmatrix} 4 & 14 \\ 14 & 54 \end{bmatrix} \begin{bmatrix} b \\ m \end{bmatrix} =  A^T \begin{bmatrix} 1 \\ 2 \\ 2 \\ 3 \end{bmatrix} = \begin{bmatrix} 8 \\ 31 \end{bmatrix}   
    \end{equation*}
    The solution gives us $y = \frac{3}{5}x-\frac{1}{10}$
    \end{proof}
    \begin{problem}
    Find the projection matrix $P$ that projects $\mathbb{R}^4$ onto the line through the origin spanned by the vector $(2,1,-1,-1)$.
    \end{problem}
    \begin{problem}
    Consider the rotation matrix $R$ shown below. Compute the axis vector $\textbf{u}$ and both the sine and cosine of the counterclockwise angle $\theta$ such that $R = R_{\theta,\textbf{u}}$
    \begin{equation*}
        R = \begin{bmatrix*}[r] -\frac{4}{9} & -\frac{7}{9} & \frac{4}{9} \\ \frac{1}{9} & \frac{4}{9} & \frac{8}{9} \\ -\frac{8}{9} & \frac{4}{9} & -\frac{1}{9} \end{bmatrix*}
    \end{equation*}
    \end{problem}
    \begin{problem}
    Find an orthonormal basis for the column space of the matrix:
    \begin{equation*}
        A = \begin{bmatrix*}[r] 1 & 1 & 1 \\ 0 & 3 & 1 \\ 2 & 2 & 2 \\ 2 & 4 & 3 \\ -1 & \phantom{-}2 & \phantom{-}0 \end{bmatrix*}
    \end{equation*}
    \end{problem}
    \begin{proof}[Solution]
    We use Gram-Schmidt to do this. Let $v_{1}=(1,0,2,2,-1)$. Normalizing gives us:
    \begin{equation*}
        e_{1} = \frac{1}{\sqrt{10}}(1,0,2,2,-1)^T    
    \end{equation*}
    We then compute:
    \begin{align*}
        (1,3,2,4,2)^T-\tfrac{(1,3,2,4,2)^T(1,0,2,2,-1)}{(1,0,2,2,-1)^T (1,0,2,2,-1)}(1,0,2,2,-1)^{T}&=(1,3,2,4,2)^{T}-\tfrac{11}{10}(1,0,2,2,-1)^{T}\\
        &=(-\tfrac{1}{10},3,-\tfrac{2}{10},\tfrac{18}{10},\tfrac{33}{10})^{T}=\tfrac{1}{10}(-1,30,-2,18,33)^{T}
    \end{align*}
    Thus:
    \begin{equation*}
        e_{2}=\tfrac{\frac{1}{10}(-1,30,-2,18,33)}{\norm{\frac{1}{10}(-1,30,-2,18,33)}}=\frac{1}{\sqrt{2318}}(-1,30,-2,18,33)
    \end{equation*}
    Finishing off, we compute:
    \begin{equation*}
        \mathbf{v}_{3}=(1,1,2,3,0)^{T}-\tfrac{(1,1,2,3,0)^T(1,0,2,2,-1)}{10}(1,0,2,2,-1)^T-\tfrac{(1,1,2,3,0)^T(1,3,2,4,2)}{34}(1,3,2,4,2,0)^{T}
    \end{equation*}
    Finally, $e_3=\frac{\textbf{v}_{3}}{\norm{\textbf{v}_{3}}}$
    \end{proof}
    \begin{problem}
    Eliminate crossterms and classify the conic section $6x^2 - 4xy+3y^2 = 1$
    \end{problem}
    \newpage
    \subsection{Problem Set VI}
    \begin{problem}
    Let $\begin{bmatrix*}[r] 1 & 0 & 3 & \vline & 1 \\ 0 & \phantom{-}1 & -2 & \vline & 3 \\ 1 & 2 & 0 & \vline & 0 \end{bmatrix*}$ be an augmented matrix.
    \begin{enumerate}
        \item Solve the system using Gaussian elimination.
        \item Express $(1,3,0)^{T}$ as a linear combination of the column vectors of the coefficient matrix.
        \item Use elementary matrices to find the LU decomposition of the coefficient matrix.
    \end{enumerate}
    \end{problem}
    \begin{proof}[Solution]
    In order,
    \begin{enumerate}
        \item 
        \begin{align*}
            \begin{bmatrix*}[r] 1 & 0 & 3 & \vline & 1 \\ 0 & \phantom{-}1 & -2 & \vline & 3 \\ 1 & 2 & 0 & \vline & 0 \end{bmatrix*} &\underset{r_{2}\leftrightarrow r_{3}\phantom{2}}{\longrightarrow} \begin{bmatrix*}[r] 1 & 0 & 3 & \vline & \phantom{-}1 \\ 1 & 2 & 0 & \vline & 0 \\ 0 & \phantom{-}1 & -2 & \vline & 3 \end{bmatrix*} \underset{r_{2}-r_{1}\phantom{3}}{\longrightarrow} \begin{bmatrix*}[r] 1 & 0 & 3 & \vline & 1 \\ 0 & \phantom{-} 2 & -3 & \vline & -1 \\ 0 & 1 & -2 & \vline & 3 \end{bmatrix*}\\
            &\underset{r_{2}\div 2\phantom{2_{2}}}{\longrightarrow} \begin{bmatrix*}[r] 1 & 0 & 3 & \vline & 1 \\ 0 & 1 & -\tfrac{3}{2} & \vline & -\tfrac{1}{2} \\ 0 & \phantom{-}1 & -2 & \vline & 3\end{bmatrix*} \underset{r_{3}-r_{2}\phantom{3}}{\longrightarrow} \begin{bmatrix*}[r] 1 & 0 & 3 & \vline & 1 \\ 0 & 1 & -\tfrac{3}{2} & \vline & -\tfrac{1}{2} \\ 0 & \phantom{-}0 & -\tfrac{1}{2} & \vline & \tfrac{7}{2} \end{bmatrix*}\\
            &\underset{r_{3}\cdot(-2)}{\longrightarrow} \begin{bmatrix*}[r] 1 & 0 & 3 & \vline & 1 \\ 0 & \phantom{-}1 & -\tfrac{3}{2} & \vline & -\tfrac{1}{2} \\ 0 & 0 & 1 & \vline & -7 \end{bmatrix*} \underset{r_{1}-3r_{3}}{\longrightarrow} \begin{bmatrix*}[r] 1 & 0 & 0 & \vline & 22 \\ 0 & 1 & -\tfrac{3}{2} & \vline & -\tfrac{1}{2} \\ 0 & \phantom{-}0 & \phantom{-}1 & \vline & -7 \end{bmatrix*}\\
            &\underset{r_{2}+\frac{3}{2}r_{3}}{\longrightarrow} \begin{bmatrix*}[r] 1 & 0 & 0 & \vline & 22 \\ 0 & 1 & 0 & \vline & -11 \\ 0 & 0 & 1 & \vline & -7 \end{bmatrix*}
        \end{align*}
        \item $(1,3,0)^{T}=22(1,0,1)^{T}-11(0,1,2)^{T}-7(3,-2,0)^{T}$
        \item
        \begin{equation*}
            A = \begin{bmatrix*}[r] 1 & 0 & 0 \\ 0 & 1 & 0 \\ 1 & 2 & 1 \end{bmatrix*} \begin{bmatrix*}[r] 1 & 0 & 3 \\ 0 & \phantom{-}1 & -2 \\ 0 & 0 & 1 \end{bmatrix*}
        \end{equation*}
    \end{enumerate}
    \end{proof} 
    \begin{problem}
    Let $A = \begin{bmatrix*}[r] 1 & 0 & 0 \\ 2 & 1 & 0 \\ 3 & 4 & 1 \end{bmatrix*}$, $B=\begin{bmatrix*}[r]1 & 0 & 0 \\ -2 & 1 & \phantom{-}0 \\ 5 & -4 & 1 \end{bmatrix*}$, and $C = \begin{bmatrix*}[r] 2 & 3 \\ -1 & 0 \\ 1 & 1 \end{bmatrix*}$. 
    \begin{enumerate}
    \begin{multicols}{3}
        \item Solve $AC+BC$
        \item Solve $AB$
        \item Does $A = B^{-1}$?
    \end{multicols}
    \end{enumerate}
    \end{problem}
    \begin{proof}[Solution]
    In order,
    \begin{enumerate}
        \item $AC+BC = (A+B)C = \begin{bmatrix*}[r] 2 & 0 & 0 \\ 0 & 2 & 0 \\ 8 & 0 & 2 \end{bmatrix*} \begin{bmatrix*}[r] 2 & 3 \\ -1 & 0 \\ 1 & 1 \end{bmatrix*} = \begin{bmatrix*}[r] 4 & 6 \\ -2 & 0 \\ 18 & 26 \end{bmatrix*}$
        \item $AB = \begin{bmatrix*}[r] 1 & 1 & 0 \\ 0 & -15 & \phantom{-}0 \\ 0 & 0 & 1 \end{bmatrix*}$
        \item No, for if $A=B^{-1}$ then $AB=I$, but this is not true.
    \end{enumerate}
    \end{proof}
    \begin{problem}
    If $A$ and $B$ are $n\times n$ invertible matrices, what is $(AB)^{-1}$?
    \end{problem}
    \begin{proof}[Solution]
    As $A^{-1}$ and $B^{-1}$ exist, and as $A$ and $B$ are of the same dimension, $B^{-1}A^{-1}$ exists. But $(B^{-1}A^{-1})(AB) = B^{-1}(A^{-1}A)B = B^{-1}IB = B^{-1}B = I$. As inverses are unique, $(AB)^{-1} = B^{-1}A^{-1}$.
    \end{proof}
    \begin{problem}
    If $A$ and $B$ are $n\times n$ matrices, what is $(A+B)^2$?
    \end{problem}
    \begin{proof}[Solution]
    $(A+B)^2 =(A+B)(A+B) = A(A+B)+B(A+B)=A^2+AB+BA+B^2$. Note: It is not true in general that $AB=BA$, and thus we cannot simplify further.
    \end{proof}
    \begin{problem}
    If $A$ and $A^T$ are $n\times n$ invertible matrices, show that $(A^T)^{-1} = (A^{-1})^T$
    \end{problem}
    \begin{proof}[Solution]
    For $A^T(A^{-1})^T = (A^{-1}A)^T = I^T = I$. As inverses are unique, $(A^T)^{-1} = (A^{-1})^T$
    \end{proof}
    \begin{problem}
    What are the solutions of:
    \begin{enumerate}
    \begin{multicols}{2}
        \item $\begin{bmatrix*}[r] 1 & 1 & 0 & 0 & \vline & -1 \\ 0 & 1 & 0 & 0 & \vline & 3 \\ 0 & 0 & 1 & 1 & \vline & 2 \\ 0 & 0 & 1 & 1 & \vline & 1 \end{bmatrix*}$
        \item $\begin{bmatrix*}[r] 1 & 1 & 0 & 0 & \vline & -1 \\ 0 & 1 & 0 & 0 & \vline & 3 \\ 0 & 0 & 1 & 1 & \vline & 1 \\ 0 & 0 & 1 & 1 & \vline & 1 \end{bmatrix*}$
    \end{multicols}
    \end{enumerate}
    \end{problem}
    \begin{proof}[Solution]
    In order,
    \begin{enumerate}
        \item No solution as the bottom two rows say $x_3 + x_4 = 2$ and $x_3 + x_4 = 1$. An impossibility.
        \item The entire space $S = \{(-4,3,x,1-x):x\in \mathbb{R}\}$.
    \end{enumerate}
    \end{proof}
    \begin{problem}
    If $A,B,$ and $C$ are $n\times n$ invertible matrices, then solve the following equations for $X$:
    \begin{enumerate}
    \begin{multicols}{3}
        \item $XA+B=C$
        \item $AX+B=X$
        \item $XA+C=X$
    \end{multicols}
    \end{enumerate}
    \end{problem}
    \begin{proof}
    In order,
    \begin{enumerate}
        \item $XA +B=C\Rightarrow XA = C-B \Rightarrow X = (C-B)A^{-1}$
        \item $AX+B = X\Rightarrow AX-X=-B \Rightarrow (A-I)X=-B \Rightarrow X = -(A-I)^{-1}B$
        \item $XA+C = X \Rightarrow XA-X = -C \Rightarrow X(A-I) = -C \Rightarrow X = -C(A-I)^{-1}$
    \end{enumerate}
    \end{proof}
    \subsection{Problem Set VII}
    \begin{problem}
    Determine the basis of the given vector space over the given field.
    \begin{enumerate}
    \begin{multicols}{3}
        \item $V=\mathbb{R}$ over $K=\mathbb{R}$
        \item $V=\mathbb{C}$ over $K=\mathbb{C}$
        \item $V=\mathbb{C}$ over $K=\mathbb{R}$
    \end{multicols}
    \end{enumerate}
    \end{problem}
    \begin{proof}[Solution]
    In order,
    \begin{enumerate}
        \item The set $\{1\}$ is a basis. Let $r \in \mathbb{R}$. Then $r=1\cdot r$.
        \item The set $\{(1,0)\}$ is a basis. Let $z\in \mathbb{Z}$. Then $z\cdot(1,0) = z$
        \item The set $\{(1,0),(0,1)\}$ is a basis. Let $z=a+bi\in \mathbb{Z}$. Then $z = a(1,0)+b(0,1)$.
    \end{enumerate}
    \end{proof}
    \begin{problem}
    What is the nullspace of an $n\times n$ matrix $A$ with real entries?
    \end{problem}
    \begin{proof}[Solution]
    The nullspace is the set $N(A) = \{X\in \mathbb{R}^n: AX = 0\}$
    \end{proof}
    \begin{problem}
    A matrix $A$ and its row reduced form $A'$ are shown below. What is the rank of $A$?
    \begin{equation*}
        A=\begin{bmatrix*}[r] 1 & 2 & 3 & 4 \\ -1 & -1 & -4 & -2 \\ 3 & 4 & 11 & 8 \end{bmatrix*} \quad\quad\quad\quad A' = \begin{bmatrix} 1 & 0 & 5 & 0 \\ 0 & 1 & -1 & 2 \\ 0 & 0 & 0 & 0 \end{bmatrix}
    \end{equation*}
    \end{problem}
    \begin{proof}[Solution]
    The rank is the dimension of the space spanned by the column vectors of the matrix. Using the row-reduced form, we see that these columns span $\mathbb{R}^2$ and thus the matrix has rank $2$.
    \end{proof}
    \begin{problem}
    What is the rank-nullity theorem?
    \end{problem}
    \begin{proof}[Solution]
    For an $n\times n$ matrix $A$, $\rk(A)+\nul(A) = n$.
    \end{proof}
    \newpage
    \subsection{Problem Set VIII}
    \begin{problem}
    Let $T:\mathbb{R}^3\rightarrow \mathbb{R}^2$ be defined by $T\begin{bmatrix} x_1 \\ x_2 \\ x_3 \end{bmatrix} = \begin{bmatrix} x_3 \\ x_1+x_2 \end{bmatrix}$.
    \begin{enumerate}
        \item Determine $\ker(T)$.
        \item Determine the dimensions of $\ker(T)$.
        \item Using the Nullity Theorem, determine the dimension of im$(T)$.
    \end{enumerate}
    \end{problem}
    \begin{proof}[Solution]
    In order,
    \begin{enumerate}
        \item If $T(x_{1},x_{2},x_{3})^{T} = 0$, then $x_3=0$ and $x_{1}+x_{2}=0$. $\ker(T)=\{(x,-x,0):x\in \mathbb{R}\}$
        \item This is a line through the origin, so the dimension is $1$ 
        \item The Nullity Theorem states that $\dim(\ker(T))+\dim(im(T)) = \dim(\mathbb{R}^3) = 3$. Thus $\dim(im(T)) = 2$.
    \end{enumerate}
    \end{proof}
    \begin{problem}
    Find the matrix representation of $T$ (Previous problem) in the standard basis of $\mathbb{R}^3$.
    \end{problem}
    \begin{proof}[Solution]
    $Te_1 = (0,1)^T$, $T e_2 = (0,1)^T$, and $Te_3 = (1,0)^T$. The matrix representation is $T=\begin{bmatrix} 0 & 0 & 1 \\ 1 & 1 & 0 \end{bmatrix}$
    \end{proof}
    \begin{problem}
    Let $P_n$ be the set of all polynomials with real coefficients of degree less than $n$. The standard basis is $\{1,x,\hdots, \ x^{n-1}\}$. Let $D:P_3 \rightarrow P_2$ be defined by $D(p) = 5\frac{dp}{dx}$. Determine the matrix representation of $D$ with respect to the standard basis.
    \end{problem}
    \begin{proof}[Solution]
    We need only check how $D$ acts on the basis vectors. $D(1) = 0+0x$, $D(x) = 1+0x$, $D(x^2) = 0+2x$. So, we have $D = \begin{bmatrix} 0 & 2 & 0 \\ 1 & 0 & 0 \end{bmatrix}$
    \end{proof}
    \begin{problem}
    Let $V$ be a vector space over $\mathbb{R}$ and let $S$ be a subspace of $V$.
    \begin{enumerate}
    \begin{multicols}{2}
        \item Define $S^{\perp}$.
        \item If $S=\Span\{ (1,2,1)^T, (1-1,2)^T\}$, what is $S^{\perp}$?
    \end{multicols}
    \end{enumerate}
    \end{problem}
    \begin{proof}[Solution]
    In order,
    \begin{enumerate}
        \item $S^{\perp} = \{x\in V: \forall y\in S, x^T y = 0\}$.
        \item Using the definition, the equations below give us $S^{\perp}=\{x_{3}(-\frac{5}{3},\frac{1}{3},1):x_{3}\in \mathbb{R}\}$
        \begin{equation*}
            \begin{bmatrix}1&2&1\\1&-1&2\end{bmatrix}\begin{bmatrix}x_1\\x_2\\x_3\end{bmatrix}=\begin{bmatrix}0\\0\end{bmatrix}\Leftrightarrow\begin{bmatrix}1&0&\frac{5}{3} \\0&1&\frac{-1}{3}\end{bmatrix}\begin{bmatrix}x_1\\x_2\\x_3\end{bmatrix}=\begin{bmatrix}0\\0\end{bmatrix}
        \end{equation*}
    \end{enumerate}
    \end{proof}
    \begin{problem}
    \
    \begin{enumerate}
        \item Let $V$ be a vector space over $\mathbb{R}$. Define an inner product.
        \item What is the difference between the standard dot product in $\mathbb{R}^n$ and an inner product? Can a vector space have more than one inner product?
        \item If $\langle x,y \rangle = xy$, what is $\norm{x}$?
    \end{enumerate}
    \end{problem}
    \begin{proof}[Solution]
    In order,
    \begin{enumerate}
        \item   An inner product is a generalization of the standard dot
                product. The dot product is itself an inner product, but not
                all inner products are dot products. There are infinitely many
                inner products for $\mathbb{R}$. Let $n\in \mathbb{N}$ be
                arbitrary, then $\langle{x}|y\rangle=nxy$ is an inner product.
        \item   $\norm{x}=\sqrt{\langle{x}|x\rangle}=\sqrt{x^2}= |x|$.
    \end{enumerate}
    \end{proof}
    \begin{problem}
    Let $V = C[-1,1]$ and let $\langle f,g\rangle = \int_{-1}^{1} f(x)g(x)dx$.
    \begin{enumerate}
        \item Show that $f(x)=1$ and $g(x) = x$ are orthogonal with respect to this inner product.
        \item Determine $\norm{f}$ and $\norm{g}$.
        \item Show that $f$ and $g$ satisfy the Pythagorean Law.
    \end{enumerate}
    \end{problem}
    \begin{proof}[Solution]
    In order,
    \begin{enumerate}
    \begin{multicols}{2}
        \item $\langle 1,x\rangle=\int_{-1}^{1}xdx=0$
        \item $\norm{1} = \sqrt{ \int_{-1}^{1} dx} = \sqrt{2}$, $\norm{x} = \sqrt{\int_{-1}^{1}x^2dx} = \sqrt{\frac{2}{3}}$
    \end{multicols}
        \item $\norm{1+x}^2 = \langle 1+x,1+x\rangle = \langle 1,1\rangle + 2\langle 1,x \rangle + \langle x,x\rangle = \norm{1}^2 + \norm{x}^2$
    \end{enumerate}
    \end{proof}
    \begin{problem}
    Let $V$ be any inner product space. State and prove the Pythagorean Theorem for inner product spaces.
    \end{problem}
    \begin{proof}[Solution]
    The Pythagorean Theorem for Inner Product Spaces state that if $V$ is an inner product space with inner product $\langle, \rangle$, and if $\langle x,y\rangle = 0$, then $\norm{x}^2+\norm{y}^2 = \norm{x+y}^2$. For $\norm{x+y}^2 = \langle x+y,x+y\rangle = \langle x,x\rangle + 2\langle x,y\rangle +\langle y,y\rangle$. But as $x$ and $y$ are orthogonal, $\langle x,y \rangle = 0$. Thus $\norm{x+y}^2 = \langle x,x\rangle + \langle y,y\rangle = \norm{x}^2+\norm{y}^2$. $\norm{x+y}^2 =\norm{x}^2+\norm{y}^2$.
    \end{proof}
    \begin{problem}
    Prove that if $V$ is an inner product space and $S$ is a subspace of $V$, then $S^{\perp}$ is a subspace of $V$.
    \end{problem}
    \begin{proof}[Solution]
    We must check that $0\in S^{\perp}$ and that $S^{\perp}$ is closed under addition and scalar multiplication.
    \begin{enumerate}
        \item For all $x\in S$, $\langle 0,x \rangle = 0$, and thus $0\in S^{\perp}$.
        \item If $x,y\in S^{\perp}$ and $z\in S$, then $\langle x+y,z\rangle = \langle x,z\rangle + \langle y,z\rangle = 0+0=0$. Thus $x+y\in S^{\perp}$.
        \item If $x\in S^{\perp}$, $y\in S$, and $\alpha$ is a scalar, then $\langle \alpha x,y \rangle = \alpha \langle x,y \rangle = \alpha \cdot 0 = 0$. Thus $\alpha x \in S^{\perp}$. $S^{\perp}$ is a subspace.
    \end{enumerate}
    \end{proof}
    %         \chapter{Number Theory}
    \section{Exams from UML 92.413: Spring 2017}
        \subsection{Exam I}
            \begin{problem}
                Find an integer $n$ such that $\gcd(n,4)=2$ and
                $\gcd(n,6)=3$, or prove that no such integer exists.
            \end{problem}
            \begin{proof}[Solution 1]
                If $\gcd(n,4)=2$, then ${2}\vert{n}$, and thus
                $\exists_{k\in\mathbb{Z}}:n=2k$. But
                $\gcd(n,6)=\gcd(2k,2\cdot 3)=2\gcd(k,3)$. But
                $\gcd(n,6)=3$, and therefore $2\gcd(k,3)=3$, a
                contradiction as $3$ is odd. No such $n$ exists.
            \end{proof}
            \begin{proof}[Solution 2]
                If $\gcd(n,4)=2$, then ${2}\vert{n}$, and thus
                $\exists_{j\in\mathbb{Z}}:n=2j$. If $\gcd(n,6)=3$,
                then ${3}\vert{n}$. Therefore
                $\exists_{k\in\mathbb{Z}}:n=3k$. But then $2j=3k$.
                As $3$ is odd, $k$ must be even. Therefore,
                $\exists_{m\in\mathbb{Z}}:k=2m$. But then
                $n=3k=3(2m)=6m$. Thus, ${6}\vert{n}$. But then
                $\gcd(n,6)=6$, a contradiction as $\gcd(n,6)=3$.
            \end{proof}
            \begin{proof}[Solution 3]
                If $\gcd(n,4)=2$, then ${2}\vert{n}$, and thus
                $\exists_{k\in\mathbb{Z}}:n=2k$. But $\gcd(n,6)=3$,
                and therefore $\exists_{x,y\in\mathbb{Z}}:nx+6y=3$.
                But $nx+6y=2kx+6y=2(kx+3y)$, and $nx+6y=3$, and
                therefore $2(nx+3y)=3$, a contradiction as $3$ is
                odd. No such $n$ exists.
            \end{proof}
            \begin{problem}
                Prove or disprove the following:
                \begin{enumerate}
                    \begin{multicols}{2}
                        \item ${20}\vert{300}$
                        \item If $a>0$, then ${a}\vert{1}$
                        \item $\forall_{a,b>0}$, either
                            ${a}\vert{b}$ or ${b}\vert{a}$
                        \item $\forall_{a,b,c>0}$, if ${a}\vert{b}$
                            and ${a}\vert{(b+c)}$,
                            then ${a}\vert{(c-b)}$
                        \item $\forall_{a,b,c>0}$, if ${a}\vert{b}$
                            and ${a}\vert{c}$, then 
                            ${a}\vert{(b^{2}+c^{2})}$
                        \item $\forall_{a,b,c>0}$, if ${a}\vert{b}$
                            and $a\vert{(b^{2}+c^{2})}$, then
                            ${a}\vert{c}$
                        \item $\forall_{a,b,c>0}$, if ${a}\vert{b}$
                            and ${b}\vert{c}$, then $a\leq c$
                        \item If $a,b,c>0$, then
                            $\gcd(a,bc)\geq\gcd(a,b)$
                        \item If $a,b,c>0$, then
                            $\gcd(a,c-a)=\gcd(a+c,c)$
                        \item If $p$ is prime and
                            ${p^{3}}\vert{abc}$, then ${p}\vert{a}$
                        \item If $a+b$ is prime, then $ab$ is even.
                        \item If $a$ and $b$ are composite, then
                            $a+b$ is composite.
                        \item If $p$ is prime and ${p}\vert{a^{2}}$,
                            then $p^{2}\vert{a^{2}}$
                        \item If $0<b<a$, then $a^{2}-b^{2}$ is
                            composite.
                    \end{multicols}
                \end{enumerate}
            \end{problem}
            \begin{proof}[Solution]
                \
                \begin{enumerate}
                    \item True, for $300=20\cdot 15$
                    \item False, for $2>0$, but $2$ does not divide
                        $1$
                    \item False, for $5>0$ and $7>0$ but $5$ does
                        not divide $7$ and $7$ does not
                        divide $5$ for they are prime.
                    \item True. If ${a}\vert{b}$, then
                        $\exists_{n\in\mathbb{Z}}:b=na$. If
                        ${a}\vert{(b+c)}$, then
                        $\exists_{m\in\mathbb{Z}}:b+c=ma$. But we
                        have that $c=ma-b=ma-na=a(m-n)$,
                        and therefore ${a}\vert{c}$. But then
                        $b-c=a(2n-m)$, so ${a}\vert{(b-c)}$
                    \item True. If ${a}\vert{b}$ then
                        $\exists_{n\in\mathbb{Z}}:b=an$.
                        If ${a}\vert{c}$, then
                        $\exists_{m\in\mathbb{Z}}:c=am$. But then
                        $b^{2}+c^{2}=a^{2}n^{2}+a^{2}m^{2}%
                         =a(an^{2}+am^{2})$, and therefore
                        ${a}\vert{(b^{2}+c^{2})}$
                    \item False. Let $a=4$, $b=8$, and $c=6$.
                        Then $b=2a$, $b^{2}+c^{2}=25a$, but $4$
                        does not divide $6$.
                    \item True. If $a,b,c>0$ and ${a}\vert{b}$,
                        then $\exists_{n\in\mathbb{N}}:b=na$,
                        and therefore $a\leq b$. If
                        ${b}\vert{c}$, then
                        $\exists_{m\in\mathbb{N}}:c=mb$. But then
                        $b\leq c$. But $a\leq b$, and therefore
                        $a\leq c$
                    \item True. If ${n}\vert{a}$ and ${n}\vert{b}$,
                        then ${n}\vert{a}$ and ${n}\vert{bc}$, and
                        therefore $\gcd(a,b)\leq\gcd(a,bc)$
                    \item True. If ${n}\vert{a}$ and
                        ${n}\vert{(c-a)}$, then ${n}\vert{c}$. But
                        then ${n}\vert{(a+c)}$. If ${n}\vert{c}$
                        and ${n}\vert{(a+c)}$, then ${n}\vert{c}$.
                        But then ${n}\vert{(c-a)}$, and therefore
                        $\gcd(a,c-a)=\gcd(a+c,c)$
                    \item False. Let $a=6$ and $c=10$. Then
                        $\gcd(a,b)=\gcd(6,10)=2$, and
                        $\gcd(a+c,c-a)=\gcd(16,4)=4$.
                    \item False. Let $p=5$, $a=2$, $b=5$, and $c=25$.
                        Then $p$ is prime, ${p^{3}}\vert{abc}$, but
                        $5$ does not divide $2$
                    \item False. Let $a=b=1$. Then $a+b=2$, which
                        is prime, but $ab=1$, which is odd.
                    \item False. Let $a=9$, and $b=8$. Then $a$ and
                        $b$ are composite, but $a+b=17$,
                        which is prime.
                    \item True. If ${p}\vert{a^{2}}$, then
                        $\exists_{n\in\mathbb{Z}}:a^{2}=np$. But, as
                        $p$ is prime, $a$ does not divide $p$, and
                        therefore $a=\frac{n}{a}p$. That is,
                        ${p}\vert{a}$. Therefore, ${p}\vert{a^{2}}$
                    \item False. Let $a=9$ and $b=8$. Then
                        $9^{2}-8^{2}=81-64=17$, which is prime.
                \end{enumerate}
            \end{proof}
            \begin{problem}
                Use Euclid's Algorithm to compute $\gcd(201,62)$.
            \end{problem}
            \begin{proof}[Solution]
                \begin{align*}
                    201&=62\cdot 3+15\\
                    62&=15\cdot 5+2\\
                    15&=2\cdot 7+1\\
                    2&=1\cdot 2+0
                \end{align*}
                $\gcd(201,62)=1$
            \end{proof}
            \begin{problem}
                Find all integer solutions to $201x+62y=1$
            \end{problem}
            \begin{proof}[Solution 1]
                From the previous problem, we have:
                \begin{equation*}
                    3+\frac{1}{4+\frac{1}{7}}=\frac{94}{29}
                \end{equation*}
                So $201(29)+62(-94)=1$. The general solution
                is therefore $x=29+62k$ and $y=-94-201k$ for
                all $k\in\mathbb{Z}$.
            \end{proof}
            \begin{proof}[Solution 2]
                From the previous problem, we have:
                \begin{align*}
                    1&=15-2\cdot7&
                    &=201\cdot(1+28)+62\cdot(-3-7-84)\\
                    &=(201-63\cdot3)-(62-15\cdot4)\cdot7&
                    &=201\cdot29+62\cdot(-94)\\
                    &=(201-62\cdot3)-(62-(201-62\cdot3)\cdot4)\cdot7
                \end{align*}
                The general solution is $x=29+62k$ and $y=-94-201k$
            \end{proof}
            \begin{problem}
                Solve the following:
                \begin{enumerate}
                    \begin{multicols}{2}
                        \item ${300^{3}+400^{4}}\mod{6}$
                        \item ${300^{3}+400^{4}}\mod{5}$
                        \item ${3^{1}}\mod{10}$
                        \item Last digit of $333^{222}$
                        \item ${1212^{11}}\mod{13}$
                        \item If $m$ is odd and $66\equiv{4}\mod{m}$,
                            what is $m$?
                        \item ${(21)(34)+765}\mod{9}$
                        \item ${48^{237}}\mod{4}$
                        \item ${3+3^{3}+3^{5}+3^{7}+3^{9}}\mod{8}$
                        \item If $2x\equiv{5}\mod{21}$, what is
                            ${x}\mod{21}$?
                    \end{multicols}
                \end{enumerate}
            \end{problem}
            \begin{proof}[Solution]
                \par\hfill\par
                \begin{enumerate}
                    \item We have
                        ${6}\vert{300}\Rightarrow%
                         300^{3}\equiv{0}\mod{6}$.
                        Also
                        $400\equiv{4}\mod{6}\Rightarrow%
                         400^{4}\equiv{4^{4}}\mod{6}%
                         ={256}\mod{6}\equiv{4}$
                    \item
                        ${5}\vert{300}\Rightarrow{300^{3}}%
                         \equiv{0}\mod{5}$,
                        ${5}\vert{400}\Rightarrow{400^{4}}%
                         \equiv{0}\mod{5}$.
                        ${300^{3}+400^{4}}\equiv{0}\mod{5}$
                    \item
                        ${3}\cdot{7}={21}\equiv{1}\mod{10}%
                         \Rightarrow{3^{-1}}\equiv{7}\mod{10}$
                    \item
                        ${333}\equiv{3}\mod{10}\Rightarrow%
                         {333^{222}}\equiv{3^{222}}\mod{10}$. But
                        $3^{222}=9(3^{2})^{110}$, and
                        $9^{110}={81^{55}}\equiv{1}\mod{10}$.
                        So, ${333^{222}}\equiv{9}\mod{10}$
                    \item
                        ${1212}\equiv{3}\mod{13}$, and
                        $3^{11}=9\cdot((3^{3})^{3}={9}\cdot{27}^{3}$.
                        But ${27}\equiv{1}\mod{13}$. So
                        ${1212^{11}}\equiv{9}\mod{13}$
                    \item ${62}\equiv{0}\mod{m}$. But
                        $62={31}\cdot{2}$. $m=31$
                    \item ${21}\equiv{3}\mod{9}$,
                        ${34}\equiv{7}\mod{9}$, and
                        ${765}\equiv{0}\mod{9}$. So we have
                        ${3}\cdot{7}={21}\equiv{3}\mod{9}$
                    \item ${48}\equiv{0}\mod{4}$.
                    \item $3^{2}\equiv{1}\mod{8}$,
                        $3^{5}\equiv{{3}\cdot{3^{4}}}\mod{8}%
                         \equiv{3}\mod{8}$,
                        $3^{7}\equiv{{3}\cdot{3^{6}}}\mod{8}%
                         \equiv{3}\mod{8}$, and finally
                        ${3^{9}}\equiv{{3}\cdot{3^{8}}}\mod{8}%
                         \equiv{3}\mod{8}$. So we have
                        $3+3+3+3+3={15}\equiv{7}\mod{8}$
                    \item If ${2x}\equiv{5}\mod{21}$, then
                        $x\equiv{{5}\cdot{2^{-1}}}\mod{21}$.
                        But ${2^{-1}}\equiv{11}\mod{21}$, so
                        ${x}\equiv{{5}\cdot{11}}\mod{21}$ and
                        ${5}\cdot{11}={55}\equiv{13}\mod{21}$.
                \end{enumerate}
            \end{proof}
            \begin{problem}
                Find all integers $n,m\geq{0}$ such that
                $5^{n}-4^{m}=1$.
            \end{problem}
            \begin{proof}[Solution]
                $n=m=1$ is a solution since
                $5-4=1$. Suppose there is another solution.
                Note that $5^{0}-4^{0}=1-1=0$,
                $5^{1}-4^{0}=5-1=4$, and $5^{0}-4^{1}=1-4=-3$.
                If $m\geq{1}$ and $n\geq{2}$, we have
                $5^{n}-4^{m}>5^{n}-1\geq25-4=21>1$. If $m\geq{2}$,
                then $4^{m}$ is divisible by 8, and thus
                $4^{m}\mod{8}=0$. If $(n,m)$ is a solution, then
                $1=5^{n}-4^{n}\equiv{5^{n}}\mod{8}$, and thus
                $5^{n}\equiv{1}\mod{8}$. If $n$ is even, then
                $5^{2k}=25^{k}\equiv{1}\mod{8}$. If $n$ is odd, then
                $5^{2k+1}\equiv{5}\mod{8}$. Thus $n$ must be even if it
                is a solution. But if $5^{n}-4^{m}=1$,
                then $5^{n}-4^{m}\equiv{1}\mod{3}$. But
                $5^{n}-4^{m}\equiv{(-1)^{n}-(1)^{m}}\mod{3}$. But $n$ is
                even, and thus $5^{n}-4^{m}\equiv{0}\mod{8}$. But then
                $1\equiv{0}\mod{3}$, a contradiction. Thus, there is
                no other solution. $n=m=1$ is the only solution.
            \end{proof}
    %         \chapter{Algebraic Geometry}
    \section{Notes on Cox, Little, and O'Shea}
                \begin{theorem}
                    If $p$ is prime, then
                    $\mathbb{Z}_p\setminus \{0\}$ is a group
                    under multiplication modulo $p$.
                \end{theorem}
            \subsubsection{Fields and Rings}
                We usually omit the multiplication symbol $\cdot$ and just
                write $ab$ instead of $a\cdot b$
                \begin{theorem}
                    If $-1$ is the additive inverse of $1$,
                    then $(-1)^2=1$
                \end{theorem}
                In rings and fields, $+$ is usually called addition and
                $\cdot$ is usually called multiplication.
                \begin{theorem}
                    If $R$ is a ring and $a\in R$, then $a\cdot 0=0\cdot a=0$
                \end{theorem}
                \begin{definition}
                    An integral domain is a commutative
                    ring such that $ab=0\Rightarrow a=0$ or $b=0$
                \end{definition}
                \begin{definition}
                    A divisor of zero in a ring $R$ is an element $a\in R$
                    such that $\exists_{b\in R\setminus\{0\}}:ab=0$
                \end{definition}
                \begin{theorem}
                    $a$ divisor of zero in a ring $R$ if and only if
                    $f:R\rightarrow R$, $f(x)=ax$ is not injective.
                \end{theorem}
                \begin{theorem}
                    Any field $k$ is an integral domain.
                \end{theorem}
                \begin{definition}
                    An ideal of a commutative ring is
                    a set $I\subset R$ such that:
                    \begin{enumerate}
                        \item $0\in I$
                              \hfill[Existence of Additive Inverse]
                        \item $\forall_{a,b\in I}$,
                              $a+b\in I$
                              \hfill[Closure Under Addition]
                        \item $\forall_{a\in I,b\in R}$,
                              $a b \in I$
                              \hfill[Absorption Property]
                    \end{enumerate}
                \end{definition}
            \subsubsection{Determinants}
                The elementary definitions from linear algebra
                are presumed. The set of all permutations of
                $\mathbb{Z}_{n}$ is denoted $S_n$. $S_{n}$ is a
                group under composition,
                $\langle S_{n},\circ\rangle$
                \begin{definition}
                    The permutation matrix of $\sigma \in S_{n}$,
                    denoted $P_{\sigma}$, is the matrix formed
                    by the image of the identity matrix $I_{n}$
                    under the mapping
                    $(a_{ij})\mapsto (a_{i\sigma(j)})$
                \end{definition}
                \begin{example}
                    Consider the permutation on $\mathbb{Z}_3$
                    defined by the cycle
                    $1\rightarrow 3\rightarrow 2\rightarrow 1$.
                    We can make this a matrix equation as follows:
                    \begin{equation*}
                        \begin{bmatrix}
                            0&0&1\\
                            1&0&0\\
                            0&1&0
                        \end{bmatrix}
                        \begin{bmatrix}
                            1\\
                            2\\
                            3
                        \end{bmatrix}
                        =
                        \begin{bmatrix}
                            3\\
                            1\\
                            2
                        \end{bmatrix}    
                    \end{equation*}
                    The leftmost matrix is obtained by
                    permuting the columns of the identity
                    matrix $I_{3}$ by $\sigma$.
                \end{example}
                \begin{definition}
                    The sign of a permutation $\sigma\in S_{n}$
                    is $\sgn(\sigma) = \det(P_{\sigma})$.
                \end{definition}
                From the way $P_{\sigma}$ is defined,
                $\sgn(\sigma)=\det(P_{\sigma})=\pm 1$,
                depending on $\sigma$.
                \begin{theorem}
                    If $A=(a_{ij})$ is an $n\times n$ matrix, then
                    $\det(A)=\underset{\sigma\in S_n}%
                     \sum\sgn(\sigma)\prod_{k=1}^{n}a_{k\sigma(k)}$.
                \end{theorem}
        \subsection{Geometry, Algebra, and Algorithms}
            This section introduces the basic ideas. Affine
            varieties and ideals in the polynomial ring
            $k[x_1,\hdots,x_n]$ are studied. Finally, polynomials
            in one variable are studied to introduce the role
            of algorithms.
            \subsubsection{Polynomials and Affine Space}
                To link algebra and geometry, we will study
                polynomials over a field. Fields are important
                because linear algebra works over any field $k$.
                There are three particular fields that will
                be used the most:
                \begin{enumerate}
                    \item $\mathbb{Q}$: This field is used
                          for computer examples.
                    \item $\mathbb{R}$: This field is used
                          for drawing pictures of curves and surfaces.
                    \item $\mathbb{C}$: This field is used
                          for proving many theorems.
                \end{enumerate}
                \begin{definition}
                    A monomial in $x_1,\hdots,x_n$ is a product
                    $\prod_{i=1}^{n}x_{i}^{\alpha_{i}}$, where
                    $\alpha_{1},\hdots,\alpha_{n}\in\mathbb{N}_0$
                \end{definition}
                \begin{definition}
                    The total degree of a monomial
                    $x_1^{\alpha_1}\cdots x_n^{\alpha_n}$ is
                    the sum $\sum_{i=1}^{n}\alpha_{i}$
                \end{definition}
                For $\alpha_1,\hdots,\alpha_n\in\mathbb{N}_0$, let
                $\alpha=(\alpha_1,\hdots,\alpha_n)$. We write
                $\prod_{i=1}^{n}x_{i}^{\alpha_{i}}=x^{\alpha}$
                \begin{definition}
                    A polynomial $f$ in $x_1,\hdots, x_n$ is afinite linear
                    combination of monomials over $k$.
                \end{definition}
                The set of all polynomials in $n$ variables with coefficients in
                $k$ is denoted $k[x_1,\hdots ,x_n]$
                \begin{definition}
                    For a polynomial
                    $f=\sum_{\alpha}a_{\alpha}x^{\alpha}\in{k}[x_1,\hdots ,x_n]$
                    $a_\alpha$ is called the coefficient of $x^{\alpha}$
                \end{definition}
                \begin{definition}
                    A term of
                    $f=\sum_{\alpha}a_{\alpha}x^{\alpha}%
                       \in k[x_1,\hdots ,x_n]$ is a product
                    $a_{\alpha}x^{\alpha}$ where
                    $a_{\alpha}\ne{0}$
                \end{definition}
                \begin{definition}
                    The total degree of
                    $f=\sum_{\alpha}a_\alpha x^\alpha$,
                    denoted $\deg(f)$, is
                    $\deg(f)=\max\{|\alpha|:a_\alpha\ne{0}\}$
                \end{definition}
                \begin{definition}
                    The zero polynomial is the
                    polynomial with all zero coefficients.
                \end{definition}
                \begin{theorem}
                    The sum and product of polynomials in
                    $k[x_1,\hdots ,x_n]$ is a polynomial
                    in $k[x_1,\hdots,x_n]$
                \end{theorem}
                \begin{definition}
                    A divisor of $f\in k[x_{1},\hdots,x_{n}]$,
                    is a $g\in k[x_{1},\hdots,x_{n}]$ such that
                    $\exists_{h\in k[x_{1},\hdots,x_{n}]}:f=gh$
                \end{definition}
                \begin{theorem}
                    For all $n\in\mathbb{N}$,
                    $k[x_1,\hdots ,x_n]$ is a commutative ring.
                \end{theorem}
                Because of this we call $k[x_1,\hdots ,x_n]$
                a polynomial ring.
                \begin{definition}
                    The $n-$dimensional affine space over $k$ is the set
                    $k^{n}=\{(a_1,\hdots,a_n):a_1,\hdots,a_n \in k\}$
                \end{definition}
                A polynomial defines a function $f:k^{n}\rightarrow{k}$.
                \begin{definition}
                    A zero function $f:k^{n}\rightarrow{k}$
                    is a function such that $f(x)=0$
                    for all $x\in{k^{n}}$
                \end{definition}
                A zero function and the zero polynomial are not necessarily
                the same thing. That is, there are fields $k$ with non-zero
                polynomials that evaluate to zero at every point.
                \begin{theorem}
                    There exists fields $k$, $f\in k[x]$ such
                    that $f$ is a non-zero polynomial and
                    $\forall_{a\in k},f(a)=0$
                \end{theorem}
                \begin{theorem}
                    If $k$ is an infinite field,
                    $f\in k[x_1,\hdots ,x_n]$, then $f$ is a
                    zero function if and only if it is
                    the zero polynomial.
                \end{theorem}
                \begin{theorem}
                    If $k$ is an infinite field and
                    $f,g\in k[x_1,\hdots,x_n]$, then $f=g$ if and
                    only if $f:k^n\rightarrow k$ and
                    $g:k^n \rightarrow k$ give
                    the same function.
                \end{theorem}
                There is a special property for polynomials
                over the complex numbers $\mathbb{C}$.
                \begin{theorem}
                    Every non-constant polynomial
                    $f\in\mathbb{C}[x]$ has a root in $\mathbb{C}$.
                \end{theorem}
                \begin{definition}
                    An algebraically closed field is a field
                    such that for non-constant $f$,
                    $\exists_{x\in k}:f(x)=0$.
                \end{definition}
            \subsubsection{Affine Varieties}
                \begin{definition}
                    The affine variety of
                    $f_{1},\hdots,f_{s}\in{k}[x_{1},\hdots,x_{n}]$
                    is
                    $\{x\in{k^{n}}:\forall_{1\leq{i}\leq{s}},f_i(x)=0\}$
                \end{definition}
                The affine variety of
                $f_{1},\hdots,f_{s}\in k[x_{1},\hdots,x_{n}]$ is denoted
                $\mathbf{V}(f_1,\hdots, f_s)$ The affine variety of a finite set
                of polynomials is the solution set of the system of equations
                $f_{i}(x)=0$
                \begin{example}
                    $\mathbf{V}(x^2+y^2-1)\subset\mathbb{R}^2$
                    is the set of solutions to $x^2+y^2-1 = 0$: The unit circle.
                \end{example}
                \begin{example}
                    The conic sections
                    (Circles, ellipses, parabolas, and hyperbolas)
                    are affine varieties. The graphs of rational
                    functions are also affine varieties.
                    For if $y = \frac{P(x)}{Q(x)}$, where
                    $P,Q\in \mathbb{R}[x]$, then
                    $\mathbf{V}\big(yQ(x)-P(x)\big)$ is an
                    affine variety equivalent to that graph.
                \end{example}
                \begin{example}
                    The surfaces the represent affine varieties
                    need not be smooth everywhere. Indeed,
                    $\mathbf{V}(z^2-x^2-y^2)$ is the graph of a cone
                    with its apex at the origin. As such, the surface
                    obtained is not smooth at the origin. Such points
                    are called singular points.
                \end{example}
                \begin{example}
                    The twisted cubic is $\mathbf{V}(y-x^2,z-x^3)$,
                    with the parametrization $\{(t,t^2,t^3):t\in\mathbb{R}\}$
                \end{example}
                The notion of dimension is very subtle.
                In previous examples, if we have $m$ polynomials
                in $\mathbb{R}^n$, we expect a surface of $n-m$
                dimension. This is not always the case, however.
                \begin{example}
                    $\mathbf{V}(xz,yz)$ is the set of solutions
                    to $xy=yz=0$. If $z=0$, then any point
                    $(x,y,0)\in \mathbb{R}^3$ satisfies this.
                    If $z\ne 0$, then $x=y=0$ and thus and point
                    $(0,0,z)\in \mathbb{R}^3$ is a solution.
                    Thus, $\mathbf{V}(xz,yz)$ is the union of the $xy$
                    plane and the $z$ axis. So $\mathbf{V}(xz,yz)$
                    is two dimensional, not one.
                \end{example}
                \begin{definition}
                    A linear variety is an affine variety
                    in which the defining polynomials are linear.
                \end{definition}
                \begin{example}
                    Let $k$ be a field and consider
                    the following polynomials:
                    \begin{align*}
                        a_{11}x_{1}+\hdots+a_{1n}x_{n}
                        &=b_{1}\\
                        \vdots&\\
                        a_{m1}x_{1}+\hdots+a_{mn}x_{n}
                        &=b_{m}
                    \end{align*}
                    From linear algebra we know that the
                    method of Gaussian Elimination and
                    row-reduction gives us the solution set of the
                    system of equations. We also know that the
                    dimension of the solutions set is $n-r$, where $r$
                    is the number of independent equations
                    (Also known as the rank of the coefficient matrix).
                    The dimension of an affine variety is also
                    determined by the number of independent equations,
                    however the term ``Independent," is much more subtle.
                \end{example}
                \begin{example}
                    Find the maximum of $f(x,y,z)=x^{3}+2xyz-z^{2}$
                    subject to $g(x,y,z) = x^2+y^2+z^2=1$.
                    From multivariable calculus, specifically
                    the method of Lagrange Multipliers, we know
                    this occurs when $\nabla(f)=\lambda\nabla(g)$,
                    for some $\lambda\in\mathbb{R}$.
                    This gives us the following:
                    \begin{align*}
                        x^{2}+2yz
                        &=2\lambda{x}&2xy-2z
                        &=2\lambda{z}\\
                        2xz
                        &=2\lambda{y}&x^{2}+y^{2}+z^2
                        &=1
                    \end{align*}
                    Solving this via algebraic means can
                    be a nightmare.
                    Various algorithms exist, however.
                \end{example}
                It is possible for an affine variety to be the empty set.
                Let $k=\mathbb{R}$, and $f=x^{2}+y^{2}+1$. Then
                $\mathbf{V}(f)=\emptyset$. That is, there is no real
                solution to $x^{2}+y^{2}=-1$.
                \begin{example}
                    Consider a robot arm. The ``Armpit,'' is at the origin,
                    and the ``Elbow,'' is at the point
                    $(x,y)\in \mathbb{R}^2$ where $x^{2}+y^{2}=r^{2}$
                    ($r$ is the length of ``Bicep.'') The ``Hand,'' will
                    then be at $(z,w)\in\mathbb{R}^{2}$ where
                    $(x-z)^{2}+(y-w)^{2}=\ell^{2}$
                    ($\ell$ is the length of the ``Forearm.")
                    Not every point $(x,y,z,w)\in\mathbb{R}^{4}$
                    represents a possible position of the
                    robot arm, there are the following constraints:
                    \begin{align*}
                        x^{2}+y^{2}&=r^{2}\\ 
                        (x-z)^{2}+(y-z)^{2}&=\ell^{2}
                    \end{align*}
                    The solution set defines an affine variety
                    in $\mathbb{R}^4$. For arms in $\mathbb{R}^3$,
                    the solution set would be in $\mathbb{R}^6$.
                \end{example}
                \begin{theorem}
                    If $V,W\subset k^n$ are affine varieties,
                    then so are $V\cup W$ and $V\cap W$. Moreover:
                    \begin{enumerate}
                        \item $\mathbf{V}(f_{1},\hdots,f_{s})%
                               \cap\mathbf{V}(g_{1},\hdots,g_{s})%
                               =\mathbf{V}(f_{1},%
                               \hdots,f_{s},g_{1},%
                               \hdots,g_{t})$
                        \item $\mathbf{V}(f_{1},\hdots,f_{s})%
                               \cup\mathbf{V}(g_{1},\hdots,g_{s})%
                               =\mathbf{V}(%
                               f_{i}g_{j}:1\leq i\leq s,1\leq j\leq t)$
                    \end{enumerate}
                \end{theorem}
                \begin{example}
                    $\mathbf{V}(xz,yz)%
                     =\mathbf{V}(xy)\cup\mathbf{V}(z)$.
                    $\mathbf{V}(xz,yz)$ is the union
                    of the $xy$ plane
                    and the $z$ axis.
                \end{example}
                \begin{example}
                    For the twisted cubic:
                    $\mathbf{V}(y-x^{2},z-x^{3})%
                     =\mathbf{V}(y-x^{2})\cap\mathbf{V}(z-x^{3})$
                \end{example}
                Several problems arise concerning affine varieties:
                \begin{enumerate}
                    \item Can we determine if
                          $\mathbf{V}(f_{1},\hdots,f_{s})\ne\emptyset$?
                          \hfill[Consistency]
                    \item Can we determine if
                          $\mathbf{V}(f_{1},\hdots,f_{s})$ is finite?
                          \hfill[Finiteness]
                    \item Can we determine the ``Dimension,'' of
                          $\mathbf{V}(f_{1},\hdots,f_{s})$?
                \end{enumerate}
                The answer to these questions is yes,
                although we must be careful in choosing
                the field we work with. 
            \subsubsection{Parametrizations of Affine Varieties}
                We now arrive at the problem of describing
                all of the points in an affine variety. 
                \begin{example}
                    Consider the system in $\mathbb{R}[x,y,z]$:
                    \begin{align*}
                        x+y+z&=1\\
                        x+2y-z&=3
                    \end{align*}
                    From linear algebra we get the
                    row echelon matrix:
                    \begin{equation*}
                        \begin{bmatrix*}[r]
                            1&0&3&\vline&-1\\
                            0&\phantom{-}1&-2&\vline&2
                        \end{bmatrix*}    
                    \end{equation*}
                    Letting $z=t$, we get $x=-3t-1$
                    and $y=2+2t$. The parametrization of the
                    affine variety is thus
                    $\{(-3t-1,2t+2,t):t\in\mathbb{R}\}$.
                    We call $t$ a parameter,
                    and $(-3t-1,2t+2,t)$ a parametrization.
                \end{example}
                \begin{example}
                    One way to parametrize the unit circle uses
                    trigonometric functions: $(\cos(t),\sin(t))$.
                    A rational way to do this is
                    $\big(\frac{1-t^2}{1+t^2},\frac{2t}{1+t^2}\big)$.
                    This parametrizes the entire unit circle,
                    with the exception of the point $(-1,0)$.
                    This point is
                    $\underset{t\rightarrow\infty}{\lim}%
                     \big(\frac{1-t^{2}}{1+t^{2}},\frac{2t}{1+t^{2}}\big)$.
                    So in a sense, $(-1,0)$ is a ``Point at infinity.''
                \end{example}
                \begin{definition}
                    A parametrization of an affine variety
                    $\mathbf{V}(f_1,\hdots, f_s)\subset k^n$ is a set of
                    $j$ equations $x_{j}=f_{j}(t_{1},\hdots,t_{m})$,
                    whose solution set $S$ is such that
                    $S\subset\mathbf{V}(f_{1},\hdots,f_{s})$,
                    and for all $g_{1},\hdots,g_{t}$ such that
                    $S\subset\mathbf{V}(g_{1},\hdots,g_{t})$,
                    $\mathbf{V}(f_{1},\hdots,f_{s})%
                     \subset\mathbf{V}(g_{1},\hdots,g_{s})$
                \end{definition}
                Solutions to $x_{k}=f_{k}(t_{1},\hdots,t_{m})$ lie in
                $\mathbf{V}(f_1,\hdots, f_s)$ and
                $\mathbf{V}(f_1,\hdots, f_s)$ is the smallest 
                affine variety containing these points.
                \begin{definition}
                    A rational function in $x_1,\hdots, x_n$
                    is a quotient
                    $\frac{P(x)}{Q(x)}:P,Q\in k[x_1,\hdots ,x_n],Q\ne 0$.
                \end{definition}
                \begin{definition}
                    $k(x_1,\hdots ,x_n)$ is the set of all
                    rational functions over a field $k$ in
                    $x_{1},\hdots,x_{n}$.
                \end{definition}
                \begin{definition}
                    Equal rational functions are functions
                    $\frac{P_1}{Q_1},5%
                     \frac{P_2}{Q_2}\in k(x_1,\hdots,x_n)$
                    where $P_{1}Q_{2}=P_{2}Q_{1}$.
                \end{definition}
                \begin{theorem}
                    If $k$ is a field,
                    then $k(x_1,\hdots ,x_n)$ is a field.
                \end{theorem}
                \begin{definition}
                    A rational representation of an
                    affine variety is a rational parametrization.
                \end{definition}
                \begin{definition}
                    A polynomial representation of an
                    affine variety is a polynomial parametrization.
                \end{definition}
                Writing out an affine variety as
                $V=\mathbf{V}(f_{1},\hdots,f_{s})$ is called an implicit
                representation. There are two questions that arise from
                parametrization:
                \begin{enumerate}
                    \item Does every affine variety have a
                          rational parametric representation?
                    \item Given a parametric representation of
                          an affine variety, can we find the
                          implicit representation?
                \end{enumerate}
                The answers are: No to the first question, yes to the
                second. Indeed, most affine varieties cannot be
                parametrized by rational functions.
                \begin{example}
                    Find the affine variety parametrized by:
                    \begin{align*}
                        x&=1+t\\
                        y&=1+t^{2}
                    \end{align*}
                    We have that $t=x-1$, and thus
                    $y=1+(x-1)^{2}=x^{2}-2x+2$.
                \end{example}
                The process described above involved eliminating the
                variable $t$ and creating a polynomial in $x$ and $y$. This
                illustrates the role played by elimination theory.
                \begin{example}
                    Let's parametrize the unit circle in a rational manner.
                    Let $(x,y)$ be a point on the unit circle and draw a
                    line from the point $(-1,0)$ to $(x,y)$. This line
                    intersects the $y-$axis at some point $(0,t)$. We have
                    that the slope of this line is
                    $m=\frac{t-0}{0-(-1)}=\frac{y-0}{x-(-1)}=\frac{y}{x+1}$.
                    So $y=t(x+1)$. But $x^{2}+y^{2}=1$, so
                    $x^2+t^{2}(x+1)^{2}%
                     =1\Leftrightarrow x^{2}+x\frac{2t^{2}}{1+t^{2}}%
                     =\frac{1-t^2}{1+t^2}%
                     \Leftrightarrow(x+\frac{t^{2}}{1+t^{2}})^{2}%
                     =\frac{1}{(1+t^{2})^{2}}%
                     \Leftrightarrow x=\frac{-t^{2}\pm 1}{1+t^{2}}$.
                    But $x\in [-1,1]$, and thus we get
                    $x=\frac{1-t^2}{1+t^2}$. But $y=t(x+1)$, and thus
                    $y=\frac{2t}{1+t^2}$.
                    $(x,y)=(\frac{1-t^{2}}{1+t^{2}},\frac{2t}{1+t^{2}})$.
                \end{example}
                \begin{definition}
                    The tangent surface of a smooth curve
                    $\Gamma:\mathbb{R}\rightarrow\mathbb{R}^n$
                    is $\{\Gamma(t)+u\Gamma'(t):t,u\in \mathbb{R}\}$
                \end{definition}
                The tangent surface is obtained by taking the union of all
                of the tangent lines to every point on the curve. $t$
                tells us which point on the curve we are one, and $u$
                tells us how far along the tangent line we are.
                \begin{example}
                    The twisted cubic is the curve defined by
                    $\mathbf{r}(t)=(t,t^2,t^3)$. It's tangent surface is
                    $\mathbf{r}+u\mathbf{r}'(t)%
                     =(t,t^2,t^3)+u(1,2t,3t^3)%
                     =(t+u,t^2+2ut,t^3+3ut^2)$.
                    One question that arises is ``Is this an affine variety?
                    If so, what are the defining polynomials.''
                    The answer for this particular surface is yes.
                    The graph of this surface is equal to
                    $\mathbf{V}(-4x^3z+3x^2y^2-4y^3+6xyz-z^2)$.
                \end{example}
                An application of this is in the design of
                complex objects such as automobile hoods and
                airplane wings. Engineers need curves and surfaces
                that are easy to describe, quick to draw, and varied
                in shape. Polynomials and rational functions satisfy
                this criteria. Complicated curves are usually
                formed by joining together simpler curves. Suppose
                a design engineer needs to draw a curve in the plane.
                The curves in question need to join smoothly, and
                thus the tangent directions need to match at the
                endpoints. The engineer must control the following:
                \begin{enumerate}
                    \item The starting and ending
                          points of the curve.
                    \item The tangent directions
                          at the starting and ending points.
                \end{enumerate}
                The B\'{e}zier Cubic does this.
                \begin{definition}
                    The B\'{e}zier Cubic in $\mathbb{R}^2$
                    is defined by:
                    \begin{align*}
                        x&=(1-t)^{3}x_0
                          +3t(1-t)^2x_1
                          +3t^2(1-t)x_2+t^3x_3\\
                        y&=(1-t)^{2}y_0
                          +3t(1-t)^2y_1
                          +3t^2(1-t)y_2+t^3y_3
                    \end{align*}
                    Where $x_0,x_1,x_2,x_3,y_0,y_1,y_2,y_3$
                    are input parameters.
                \end{definition}
                When $t=0$, we have $x = x_0, y=y_0$.
                Thus $(x_0,y_0)$ is the starting point.
                Similarly $(x_3,y_3)$ is the end point.
                The derivatives are:
                \begin{align*}
                    x'&=-3(1-t)^2x_0
                       +3(1-t)(1-3t)x_1
                       +3t(2-3t)x_2
                       +3t^2x_3 \\
                    y'&=-3(1-t)^2y_0
                       +3(1-t)(1-3t)y_1
                       +3t(2-3t)y_2
                       +3t^2y_3
                \end{align*}
                So $(x'(0),y'(0))=\big(3(x_1-x_0),3(y_1-y_0)\big)$
                and $(x'(1),y'(1))=\big(3(x_3-x_2),3(y_3-y_2)\big)$.
                Hence, choosing $x_1,x_2$ and $y_1,y_2$ carefully
                allows that designer to control the tangent of the
                curve at the endpoints. Moreover, choosing the point
                $(x_1,y_1)$ makes the tangent at $(x(0),y(0))$ point
                in the same direction as the line from $(x_0,y_0)$
                to $(x_1,y_1)$. Similarly, choosing $(x_2,y_2)$ makes
                the tangent at $(x(1),y(1))$ point in the same
                direction as the line from
                $(x_2,y_2)$ to $(x_3,y_3)$.
                \begin{definition}
                    The control polygon of a B\'{e}zier Cubic
                    in $\mathbb{R}^2$ is the polygon formed
                    by the lines
                    $(x_0,y_0)\rightarrow(x_1,y_1)%
                     \rightarrow(x_2,y_2)\rightarrow(x_3,y_3)%
                     \rightarrow (x_0,y_0)$.
                \end{definition}
                Interestingly enough, the B\'{e}zier Cubic always
                lies inside the control polygon. The final thing
                to control is the length of the tangents at the
                endpoint. But from equations $1.3$ and $1.4$, the
                lengths are three times the distance from
                $(x_0,y_0)$ to $(x_1,y_1)$ and $(x_2,y_2)$ to
                $(x_3,y_3)$, respectively. 
            \subsubsection{Ideals}
                \begin{definition}
                    An ideal of a polynomial ring
                    $k[x_1,\hdots ,x_n]$ is a set
                    $I\subset k[x_1,\hdots ,x_n]$ such that:
                    \begin{enumerate}
                        \begin{multicols}{3}
                            \item $0\in I$
                            \item $\forall_{f,g\in I}, f+g\in I$ 
                            \item $\forall_{f\in I, h\in k[x_1,\hdots ,x_n]},%
                                   hf\in I$
                        \end{multicols}
                    \end{enumerate}
                \end{definition}
                \begin{definition}
                    The ideal generated by a set
                    $\{f_1,\hdots, f_s\}\subset k[x_1,\hdots ,x_n]$
                    is the set
                    $\langle f_1,\hdots,f_s\rangle%
                     =\{\sum_{i=1}^{s} h_i f_i:%
                        h_1,\hdots,h_s\in k[x_1,\hdots ,x_n]\}$.
                \end{definition}
                \begin{theorem}
                    If $f_1,\hdots, f_s\in k[x_1,\hdots ,x_n]$,
                    then $\langle f_1,\hdots, f_s\rangle$ is an ideal.
                \end{theorem}
                The ideal $\langle f_1,\hdots, f_s\rangle$ has a nice
                    interpretation. If $x\in k$ such that
                $f_1(x)=\hdots=f_s(x)=0$, then for any set of polynomials
                $h_1,\hdots, h_s$, we have
                $h_1(x)f_1(x)=\hdots=h_s(x)f_s(x)=0$, and adding the
                equations we get
                $h_1(x)f_1(x)+\hdots+h_s(x)f_s(x)=0$. Thus we can think of
                $\langle f_1,\hdots,f_s\rangle$ as the set of all
                ``Polynomial consequences,'' of the equations
                $f_1=\hdots=f_s=0$.
                \begin{example}
                    Consider the following system:
                    \begin{align*}
                        x&=1+t&y&=1+t^{2}
                    \end{align*}
                    We can eliminate $t$ to obtain $y=x^2-2x+2$.
                    To see this, write
                    \begin{align*}
                        x-1-t&=0&-y+1+t^{2}&=0
                    \end{align*}
                    Multiplying this first by by $x-1+t$ and adding, we get
                    $(x-1)^2-y+1=0$. Thus $y=x^2-2x+2$.
                \end{example}
                \begin{definition}
                    A finitely generated ideal is an ideal
                    such that
                    $\exists_{f_1,\hdots, f_s}:I=\langle f_1,\hdots, f_s\rangle$.
                \end{definition}
                \begin{definition}
                    A basis of an ideal is a set
                    $\{f_1,\hdots, f_s\}\subset k[x_1,\hdots ,x_n]$
                    such that $I=\langle f_{1},\hdots,f_{s}\rangle$
                \end{definition}
                Hilbert's Basis Theorem, to be proved later, states
                that every ideal in $k[x_{1},\hdots,x_{n}]$ is finitely
                generated. An ideal in $k[x_{1},\hdots,x_{n}]$ is similar
                to a subspace in linear algebra. Both must be closed
                under multiplication and addition, except that in a
                subspace we multiply by scalars and in an ideal
                we multiply by polynomials. 
                \begin{theorem}
                    If $\langle f_1,\hdots,f_s\rangle=\langle g_1,\hdots,g_t\rangle$,
                    then $\mathbf{V}(f_1,\hdots, f_s)=\mathbf{V}(g_1,\hdots, g_s)$.
                \end{theorem}
                \begin{example}
                    $\langle2x^2+3y^2-11,x^2-y^2-3\rangle%
                     =\langle x^2-4,y^2-1\rangle$.
                    So,
                    $\mathbf{V}(2x^2+3y^2-11,x^2-y^2-3)%
                     =\{(2,1),(-2,1),(2,-1),(-2,-1)\}$.
                    Changing basis simplifies the problem.
                \end{example}
                \begin{definition}
                    The ideal of an affine variety
                    $V\subset k^n$ is
                    $\mathbf{I}(V)%
                     =\{f\in k[x_1,\hdots ,x_n]:\forall_{x\in V}f(x)=0\}$
                \end{definition}
                \begin{theorem}
                    If $V\subset k^n$ is an affine variety,
                    then $\mathbf{I}(V)$ is an ideal of
                    $k[x_1,\hdots ,x_n]$.
                \end{theorem}
                \begin{theorem}
                    For any field $k$,
                    $\mathbf{I}\big(\{(0,0)\}\big)=\langle x,y\rangle$.
                \end{theorem}
                \begin{theorem}
                    For any infinite field $k$,
                    $\mathbf{I}(k^n)=\{0\}$.
                \end{theorem}
                \begin{theorem}
                    If $V=\mathbf{V}(y-x^2,z-x^3)\subset\mathbb{R}^3$,
                    $f\in \mathbf{I}(V)$,
                    then $\exists_{h_1,h_2,r(x)\in\mathbb{R}[x,y,z]}$
                    such that $f=h_1(y-x^2)+h_2(z-x^3)+r$.
                \end{theorem}
                \begin{theorem}
                    If $V=\mathbf{V}(y-x^2,z-x^3)\subset\mathbb{R}^3$,
                    then $\mathbf{I}(V)=\langle y-x^2,z-x^3\rangle$
                \end{theorem}
                It is not always true that
                $\mathbf{I}(\mathbf{V}(f_1,\hdots, f_s))%
                 =\langle f_1,\hdots, f_s\rangle$.
                \begin{theorem}
                    If $f_1,\hdots, f_s \in k[x_1,\hdots ,x_n]$, then
                    $\langle f_1,\hdots,f_s\rangle%
                     \subset\mathbf{I}(\mathbf{V}(f_1,\hdots, f_s))$.
                \end{theorem}
                \begin{theorem}
                    There exists fields $k$ and polynomials
                    such that
                    $\langle f_1,\hdots,f_s\rangle%
                     \ne\mathbf{I}(\mathbf{V}(f_1,\hdots, f_s))$
                \end{theorem}
                \begin{theorem}
                    If $k$ is a field, $V,W\subset k^n$ are affine
                    varieties, then $V\subset W$ if and only if
                    $\mathbf{I}(W)\subset \mathbf{I}(V)$.
                \end{theorem}
                \begin{theorem}
                    If $k$ is a field, $W,W\subset k^n$ are
                    affine varieties, then $V=W$ if
                    and only if $\mathbf{I}(W)=\mathbf{I}(V)$.
                \end{theorem}
                Three questions arise concerning
                ideals in $k[x_1,\hdots ,x_n]$.
                \begin{enumerate}
                    \item Can every ideal $I\subset k[x_1,\hdots ,x_n]$
                          be written as $\langle f_1,\hdots, f_s\rangle$
                          for some $f_1,\hdots, f_s \in k[x_1,\hdots ,x_n]$?
                    \item If $f_1,\hdots, f_s\in k[x_1,\hdots ,x_n]$,
                          $f\in k[x_1,\hdots ,x_n]$, is there an
                          algorithm to see if $f\in\langle f_{1},\hdots,f_{s}\rangle$?
                    \item Is there a relation between
                          $\langle f_1,\hdots, f_s\rangle$ and
                          $\mathbf{I}(\mathbf{V}(f_1,\hdots, f_s))$?
                \end{enumerate}
            \subsubsection{Polynomials in One Variable}
                This section studies the division algorithm of polynomials in one variable.
                \begin{definition}
                    The leading term of $f=\sum_{k=1}^{n}a_kx^{k}\in k[x]$,
                    where $a_{n}\ne 0$, is $\LT(f)=a_nx^{n}$.
                \end{definition}
                \begin{example}
                    If $f=2x^{3}-4x+3$, then $\LT(f)=2x^{3}$.
                \end{example}
                \begin{theorem}
                    If $k$ is a field and $g\in k[x]\setminus\{0\}$,
                    then $\forall_{f\in k[x]},\exists_{q,r\in k[x]}:f=qg+r$,
                    where either $r=0$ or $\deg(r)<\deg(g)$.
                    Furthermore, $q$ and $r$ are unique.
                \end{theorem}
                From the uniqueness of $r$, we call $r$ the remainder of
                $f$ with respect to $g$.
                \begin{theorem}
                    If $k$ is a field and $f\in k[x]$ is a non-zero
                    polynomial, then $f$ has at most $\deg(f)$ roots.
                \end{theorem}
                \begin{definition}
                    A principal ideal is an ideal generated by a
                    single element.
                \end{definition}
                \begin{theorem}
                    If $k$ is a field, then every ideal of $k[x]$
                    is principal.
                \end{theorem}
                \begin{theorem}
                    If $\langle f\rangle=\langle g$ are ideals
                    in $k[x]$, then there is a constant $h$
                    such that $f=hg$.
                \end{theorem}
                \begin{definition}
                    A greatest common divisor of $f,g\in k[x]$
                    is a polynomial $h\in k[x]$ such that $h$
                    divides $f$ and $g$ and $\forall_{p\in k[x]}$
                    such that $p$ divides $f$ and $g$, $p$ divides $h$. 
                \end{definition}
                \begin{theorem}
                    If $f,g\in k[x]$, then there is a
                    greatest common divisor of $f$ and $g$.
                \end{theorem}
                \begin{theorem}
                    If $f,g\in k[x]$, and $h_{1},h_{2}$ are greatest
                    common divisors of $f$ and $g$, then there is a
                    constant $c\in k$ such that $h_{1}=ch_{2}$.
                \end{theorem}
                The Euclidean Algorithm is used for computational purposes
                to compute the greatest common divisor of two polynomials.
                Let $f,g\in k[x]$.
                \begin{enumerate}
                    \item Let $h_{1}=f$
                    \item Let $s_{1}=g$
                    \item While $s_{n}\ne 0$, do the following:
                    \begin{enumerate}
                        \item $r_{n}=remainder(h_{n},s_{n})$
                        \item $h_{n+1}=s_{n}$
                        \item $s_{n+1}=r_{n}$
                    \end{enumerate}
                \end{enumerate}
                There is an $N\in \mathbb{N}$ such that for all $n>N$,
                $h_{n}=h_{N}$. Letting $h=h_{N}$, this is the greatest
                common divisor of $f$ and $g$. This comes from
                $\GCD(f,g)=\GCD(f-qg,g)=\GCD(r,g)$ and the fact
                that $\deg(r)<\deg(g)$. So $\deg(r_{n+1})<\deg(r_{n})$,
                and eventually $\deg(r_{N})=0$.
                \begin{definition}
                    A greatest common divisor of polynomials
                    $f_1,\hdots, f_s \in k[x]$ is a polynomial
                    $h\in k[x]$ such that $h$ divides $f_1,\hdots, f_s$
                    and if $p\in k[x]$ such that $p$ divides
                    $f_1,\hdots, f_s$, then $p$ divides $h$.
                \end{definition}
                \begin{theorem}
                    If $f_{1},\hdots, f_{s}\in k[x]$, then there is
                    a polynomial $h\in k[x]$ that is a greatest
                    common divisor of $f_{1},\hdots,f_{s}$.
                \end{theorem}
                \begin{theorem}
                    If $f_{1},\hdots,f_{s}\in k[x]$, and if $h$
                    is a GCD of $f_{1},\hdots, f_{s}$, then
                    $\langle h\rangle=\langle f_{1},\hdots, f_{s}\rangle$
                \end{theorem}
        \subsection{Elimination Theory}
            \subsubsection{The Elimination and Extension Theorems}
                    \begin{definition}
                        If
                        $I=\langle{f_{1},\hdots,f_{s}}\rangle\subset%
                        k[x_1,\hdots ,x_n]$,
                        the $\ell-$th elimination ideal, denoted $I_{\ell}$,
                        is the ideal defined as
                        $I_{\ell}=I\cap{k}[x_{\ell+1},\hdots,x_{n}]$.
                    \end{definition}
                    \begin{theorem}
                        For $\ell\in\mathbb{Z}_{n-1}$, if
                        $I=\langle{f_{1},\hdots,f_{s}}\rangle\subset%
                         k[x_1,\hdots ,x_n]$
                        is an ideal, then $I_{\ell}$ is
                        an ideal of $k[x_1,\hdots ,x_n]$.
                    \end{theorem}
                    \begin{theorem}[The Elimination Theorem]
                        If $I\subset k[x_1,\hdots ,x_n]$ is an ideal and
                        $G$ is a Groebner Basis of $I$ with respect to the
                        lexicographic ordering $x_1>x_2>\hdots > x_n$, then for
                        all $\ell\in\mathbb{Z}_{n}$,
                        $G_{\ell}=G\cap{k}[x_{\ell+1},\hdots,x_{n}]$
                        is a Groebner Basis of $I_{\ell}$.
                    \end{theorem}
                    \begin{theorem}[The Extension Theorem]
                        If
                        $I=\langle{f_{1},\hdots,g_{s}}\rangle%
                         \subset\mathbb{C}[x_{1},\hdots,x_{n}]$,
                        and if $I_{1}$ is the first elimination ideal of $I$,
                        and if for all $i\in\mathbb{Z}_s$
                        $f_{i}=g(x_{2},\hdots,x_{n})x_{1}^{N_i}+h$,
                        where the degree of the $x_{1}$ component of $h$ is
                        less than $N_{i}$, and if
                        $(a_{2},\hdots,a_{n})\notin\textbf{V}(g_{1},\hdots,g_{s})$,
                        then there is an $a_{1}\in\mathbb{C}$ such that
                        $(a_{1},\hdots,a_{n})\in\textbf{V}(I)$.
                    \end{theorem}
                    The requirement that we work in $\mathbb{C}$ is crucial.
                    This theorem does not hold in $\mathbb{R}$. 
                    \begin{theorem}
                        If
                        $I=\langle{f_{1},\hdots,f_{s}}\rangle\subset%
                         \mathbb{C}[x_{1},\hdots,x_{n}]$
                        if for some $i$, $f_{i}$ is of the form
                        $f_{i}=cx_{1}^{N}+g(x_{1},\hdots,x_{n})$,
                        where the degree of the $x_{1}$ term in $g$ is
                        less than $N$, and $c\ne{0}$, and if
                        $(a_{2},\hdots,a_{n})\in\textbf{V}(I_{1})$,
                        then there is an $a_{1}\in\mathbb{C}$ such that
                        $(a_{1},\hdots,a_{n})\in\textbf{V}(I)$.
                    \end{theorem}
            \subsubsection{The Geometry of Elimination}
                \begin{definition}
                        The projectiom map
                        $\pi_{\ell}:\mathbb{C}^{n}\rightarrow\mathbb{C}^{n-\ell}$
                        is defined as
                        $\pi_{\ell}(a_{1},\hdots,a_{n})=(a_{\ell+1},\hdots,a_{n})$.
                \end{definition}
                \begin{theorem}
                        If $V=\mathbf{V}(f_{1},\hdots,f_{s})\subset\mathbb{C}^{n}$,
                        and $I_{\ell}$ is the $\ell^{th}$ elimination ideal of
                        $\langle{f_{1},\hdots,f_{s}}\rangle$,
                        then $\pi_{\ell}(V)\subset\textbf{V}(I_{\ell})$
                \end{theorem}
                \begin{theorem}
                        If $V=\mathbf{V}(f_{1},\hdots,f_{s})\subset\mathbb{C}^{n}$,
                        and $G_{\ell}$ is as defined in the extension theorem,
                        then $\textbf{V}(I_{\ell})=\pi_{\ell}(V)\cup{G_{\ell}}$
                \end{theorem}
                \begin{theorem}[The First Closure Theorem]
                        If $V=\mathbf{V}(f_{1},\hdots,f_{s})\subset\mathbb{C}^{n}$
                        and $I_{\ell}$ is the $\ell^{th}$ elimination ideal of
                        $\langle{f_{1},\hdots,f_{s}}\rangle$, then
                        $\textbf{V}(I_{\ell})$ is the smallest affine variety
                        containing $\pi_{\ell}(V)\subset\mathbb{C}^{n-\ell}$.
                \end{theorem}
                \begin{theorem}[The Second Closure Theorem]
                If $V = \mathbf{V}(f_1,\hdots, f_s) \subset \mathbb{C}^n$, $V\ne \emptyset$, and if $I_{\ell}$ is the $\ell-$th elimination ideal of $\langle f_1,\hdots,f_s\rangle$, then there is an affine variety $W\underset{Proper}{\subset} \textbf{V}(I_{\ell})$ such that $\textbf{V}(I_{\ell})\setminus W \subset \pi_{\ell}(V)$.
                \end{theorem}
                \begin{theorem}
                If $V = \mathbf{V}(f_1,\hdots, f_s)\subset \mathbb{C}^n$ and if for some $i$, $f_i$ is of the form $f_i = cx_1^N + g$, where the $x_1$ terms in $g$ are of degreeless than $N$, and $c\ne 0$, then $\pi_{1}(V) = \textbf{V}(I_{1})$.
                \end{theorem}
            \subsubsection{Implicitization}
                \begin{definition}
                    A polynomial parametrization is a finite set of equations
                    $x_k = f_k(t_1,\hdots, t_m)\in k[t_1,\hdots, t_m]$.
                    The function $F:k^m\rightarrow k^n$ is the image defined by
                    $(t_1,\hdots, t_m)\mapsto (x_1,\hdots, x_n)$
                \end{definition}
                \begin{theorem}[The Polynomial Implicitization Theorem]
                    If $k$ is an infinite field and $F:k^m\rightarrow k^n$ is a
                    function determined by some polynomial parametrization,
                    and if $I$ is an ideal
                    $I=\langle{x_{1}-f_{1},\hdots,x_{n}-f_{n}}\rangle%
                     \subset k[t_{1},\hdots,t_{m},x_{1},\hdots,x_{n}]$,
                    then $\textbf{V}(I_{m})$ is the smallest variety in
                    $k^{n}$ containing $F(k^{n})$, where $I_{m}$ is the
                    $m^{th}$ elimination ideal.
                \end{theorem}
                \begin{definition}
                    A rational parametrization is a finite set of equations
                    $x_k = f_k(t_1,\hdots, t_m)\in k(t_1,\hdots, t_m)$
                \end{definition}
                \begin{theorem}[Rational Implicitization]
                    If $k$ is an infinite field, $f_k, g_k, k=1,2,\hdots, n$ are
                    a rational parametrization, $W = \mathbf{V}(g_1,\hdots, g_s)$,
                    and if $F:k^m\setminus W \rightarrow k^n$ is the function
                    determined by the rational parametrization, if
                    $J=\langle g_1 x_1-g_1,\hdots,g_n x_n-g_n,1-gy\rangle\subset%
                     k[y,t_1,\hdots, g_m, x_1,\hdots, x_n]$,
                    where $g = g_1\cdots g_n$, and if $J_{m+1}$ is the
                    $(m+1)^{th}$ elimination ideal, then $\textbf{V}(J_{m+1})$
                    is the smallest variety in $k^n$
                    containing $F(K^m\setminus W)$.
                \end{theorem}
            \subsubsection{Singular Points and Envelopes}
                    \begin{definition}
                    A singular point on an affine variety $\mathbf{V}(f)$ is a point $x\in k$ such that there exists no tangent line at $x$.
                    \end{definition}
                    For curves in the plane, this usually happens when either the curve intersects itself or has a kink in it.
                    \begin{definition}
                    If $k\in \mathbb{N}$, if $(a,b)\in \mathbf{V}(f)$, and if $L$ is a line through $(a,b)$, then $L$ meets $\mathbf{V}(f)$ with multiplicity $k$ at $(a,b)$ if$L$ can be linearly parametrized in $x$ and $y$ so that $t=0$ is a root of multiplicity $k$ of the polynomial $g(t) = f(a+ct,b+dt)$.
                    \end{definition}
                    \begin{theorem}
                    If $f\in k[x,y]$, $(a,b) \in \mathbf{V}(f)$, and if $\nabla f(a,b) \ne (0,0)$, then there is a unique line through $(a,b)$ which meets $\mathbf{V}(f)$ withmultiplicity $k\geq 2$.
                    \end{theorem}
                    \begin{theorem}
                    If $f\in k[x,y]$, $(a,b) \in \mathbf{V}(f)$, and if $\nabla f(a,b) = 0$, then every line through $(a,b)$ meets $\mathbf{V}(f)$ with multiplicity $k \geq 2$.
                    \end{theorem}
                    \begin{definition}
                    If $f\in k[x,y]$, $(a,b) \in \mathbf{V}(f)$, and if $\nabla f(a,b) \ne (0,0)$, then the tangent line of $\mathbf{V}(f)$ at $(a,b)$ is the unique line through$(a,b)$ with multiplicity $k\geq 2$. We say that $(a,b)$ is a non-singular point of $\mathbf{V}(f)$.
                    \end{definition}
                    \begin{definition}
                    If $f\in k[x,y]$, $(a,b) \in \mathbf{V}(f)$, and if $\nabla f(a,b) = (0,0)$, then we say that $(a,b)$ is a singular point of $\mathbf{V}(f)$.
                    \end{definition}
                    \begin{definition}
                    If $\mathbf{V}(F_t)$ is a family of curves in $\mathbb{R}^2$, its envelope consists of all points $(x,y) \in \mathbb{R}^2$ such that $F(x,y,t) = 0$ and$\frac{\partial}{\partial t}F(x,y,t) = 0$ for some $t\in \mathbb{R}$.
                    \end{definition}
            \subsubsection{Unique Factorization and Resultants}
                \begin{definition}
                If $k$ is a field, then a polynomial $f\in k[x_1,\hdots ,x_n]$ is said to be irreducible if $f$ is non-constant and is not the product of two non-constantpolynomials in $k[x_1,\hdots ,x_n]$.
                \end{definition}
                \begin{theorem}
                Every non-constant polynomial $f\in k[x_1,\hdots ,x_n]$ can be written as a product of polynomials which are irreducible over $k$
                \end{theorem}
                \begin{theorem}
                If $f,g\in k[x_1,\hdots ,x_n]$ have positive degree in $x_1$, then $f$ and $g$ have a common factor in $k[x_1,\hdots ,x_n]$ of positive degree in $x_1$ if andonly if they have a common factor in $k(x_2,\hdots, x_n)[x_1]$
                \end{theorem}
                \begin{theorem}
                Every non-constant $f\in k[x_1,\hdots ,x_n]$ can be written as a product $f = f_1\cdots f_r$ of irreducibles of $k$. Furthermore, if $f = g_1\cdots g_s$, wherethe $g_k$ are irreducible, then $r=s$ and there are constants $\alpha_1,\hdots, \alpha_n$ such that $\{f_1,\hdots, f_r\} = \{\alpha_1 g_1, \hdots, \alpha_rg_r\}$.
                \end{theorem}
                \begin{theorem}
                If $f,g \in k[x]$ are polynomials of degree $\ell>0$ and $m>0$,
                respectively, then $f$ and $g$ have a common factor if and only
                if there are polynomials $A,B\in{k}[x]$ such that $A$ and $B$
                are not both zero, $A$ has degree at most $m-1$ and $B$ has
                degree at most $\ell-1$, and $Af+Bg = 0$.
                \end{theorem}
                \begin{definition}
                If $f = a_0 x^{\ell} +\hdots + a_{\ell}$ and $g = b_0 x^m + \hdots b_m$, then the Sylvester Matrix is:
                \begin{equation*}
                    \begin{pmatrix} a_0 & 0 & 0 & 0 & b_0 & 0 & 0 & 0 \\ a_1 & a_0 & 0 & 0 & b_1 & b_0 & 0 & 0 \\ \vdots & \vdots & \ddots & 0 & \vdots & \vdots & \ddots & 0 \\\vdots & \vdots & \ddots & a_{0} & \vdots & \vdots & \ddots & b_0 \\ a_{\ell} & \hdots & \hdots & a_{1} & b_{m} & \hdots & \hdots & 0 \\ 0 & a_{\ell} & \hdots& \vdots & 0 & b_{m} & \hdots & \vdots\\ 0 & 0 & \ddots & 0 & 0 & \hdots & \ddots & 0 \\ 0 & \hdots & \hdots & a_{\ell} & 0 & \hdots & \hdots & b_{m}\end{pmatrix}
                \end{equation*}
                \end{definition}
                \begin{theorem}
                If $f,g \in k[x]$, then the resultant of $f$ and $g$ is the determinant of the Sylvester matrix of $f$ and $g$.
                \end{theorem}
                \begin{theorem}
                If $f,g\in k[x]$ are polynomials of positive degree, then the resultant of $f$ and $g$ is an integer polynomial in the coefficients of $f$ and $g$.
                \end{theorem}
                \begin{theorem}
                If $f,g\in k[x]$ are polynomials of positive degree, then $f$ and $g$ have a common factor if and only if their resultant is zero.
                \end{theorem}
                \begin{theorem}
                If $f,g\in k[x]$ are of positive degree, then there are polynomials $A,B \in k[x]$ such that $Af + Bg = Resultant(f,g)$
                \end{theorem}
        \subsection{Groebner Bases}
            \subsubsection{Introduction}
                There are three problems we wish to address:
                \begin{enumerate}
                    \item Does every Ideal
                          $I\subset k[x_1,\hdots ,x_n]$
                          have a finite generating set?
                    \item Given $f\in k[x_1,\hdots ,x_n]$,
                          and $I=\langle f_1,\hdots, f_s\rangle$,
                          can we determine if $f\in I?$
                    \item For $f_1,\hdots,f_{s}\in{k}[x_{1},\hdots,x_{n}]$,
                          can we determine what
                          $\mathbf{V}(f_1,\hdots, f_s)$ is?
                \end{enumerate}
                We've already solved this in the case of one
                variable, $n=1$. The case of $n\in\mathbb{N}$
                where $f_1,\hdots,f_s$ are linear functions is the
                subject of linear algebra. Both the Eucldiean
                algorithm and the methods of linear algebra require
                a notion of ordering of terms. In the case of one
                variable, if $n>m$ we write $x^n>x^m$. In the case of
                linear algebra we usually write
                $x_n>x_{n-1}>\hdots>x_{2}>x_{1}$. 
            \subsubsection{Orderings on the Monomials in
                           \texorpdfstring{$k[x_1,\hdots ,x_n]$}{kx}}
                \begin{definition}
                    A monomial ordering on $k[x_1,\hdots, x_n]$
                    is any relation $\succ$ on $\mathbb{N}^n$ such that:
                \begin{enumerate}
                    \item $\succ$ is a total ordering.
                    \item If $\alpha \succ \beta$ and
                          $\gamma\in\mathbb{N}^n$, then
                          $\alpha+\gamma \succ \beta + \gamma$.
                    \item $\succ$ is a well-ordering on $\mathbb{N}^n$. 
                \end{enumerate}
                \end{definition}
                \begin{theorem}
                    An ordering $\prec$ on $\mathbb{N}^n$ is
                    a well-ordering if and only if for any
                    monotonically decreasing sequence
                    $\{a_n\}_{n=1}^{\infty}$, there is an
                    $N\in\mathbb{N}$ such that for all $n>N$,
                    $a_{n}=a_{N}$.
                \end{theorem}
                \begin{proof}
                    For if $\prec$ is a well ordering, then
                    $\{a_n\}_{n=1}^{\infty}$ contains a least element
                    $x$. Suppose $a_n$ contains a strictly decreasing
                    subsequence. But $\prec$ is a well ordering, and
                    therefore $\{a_n\}_{n=1}^{\infty}$ contains a least
                    element $x$. But again $\prec$ is a well ordering,
                    and thus $\{a_n\}_{n=1}^{\infty} \setminus \{x\}$
                    contains a least element $y$. But then $x\prec y$,
                    and $x$ is the least element of
                    $\{a_n\}_{n=1}^{\infty}$. Therefore there is an
                    $a_n$ such that $x\preceq a_{n}\preceq y$.
                    But $a_n$ is strictly decreasing, and therefore
                    $a_{n+1}\preceq x$, and thus $a_{n+2}\prec x$.
                    But $x$ is the least element of $\{a_n\}_{n=1}^{\infty}$,
                    a contradiction. Therefore $a_n$ contains no
                    strictly increasing subsequence. Suppose every
                    decreasing sequence eventually terminates. Let
                    $E\subset\mathbb{N}^n$. Suppose there is no
                    least element. Then we can construct a strictly
                    decreasing sequence. But every decreasing sequence
                    eventually terminates, a contradiction. Therefore, etc.
                \end{proof}
                \begin{definition}
                    If $\alpha,\beta \in \mathbb{N}^n$, then $\alpha$
                    is said to be lexicographically greater than
                    $\beta$, denoted $\underset{Lex}{>}$,
                    if the left-most entry of
                    $\alpha-\beta$ is positive.
                \end{definition}
                \begin{theorem}
                    The Lexicographic Ordering
                    is a monomial ordering.
                \end{theorem}
                \begin{definition}
                    The graded lexicographic ordering
                    $\underset{GrLex}{>}$ on $\mathbb{N}^n$ is an
                    ordering on $\mathbb{N}^n$ such that
                    $\alpha\underset{GrLex}{>}\beta$ if and only if
                    either $|\alpha|>|\beta|$, or
                    $|\alpha|=|\beta|$ and
                    $\alpha\underset{Lex}{>}\beta$.
                \end{definition}
                \begin{theorem}
                    The graded lexicographic
                    ordering is a monomial ordering.
                \end{theorem}
                \begin{definition}
                    For
                    $f=\sum_{\alpha}%
                       a_{\alpha}x^{\alpha}\in{k}[x_{1},\hdots,x_{n}]$,
                    and $\prec$ a monomial ordering,
                    the multidegree of $f$ is
                    $\multideg(f)%
                     =\max\{\alpha\in\mathbb{N}^n:a_{\alpha}\ne{0}\}$.
                \end{definition}
                \begin{definition}
                    For
                    $f=\sum_{\alpha}%
                       a_\alpha{x}^\alpha\in k[x_{1},\hdots,x_{n}]$
                    and monomial order $>$, the leading
                    coefficient of $f$ is
                    $LC(f)=a_{\multideg(f)}\in{k}$
                \end{definition}
                \begin{definition}
                    For
                    $f=\sum_{\alpha}%
                       a_{\alpha}x^{\alpha}\in{k}[x_{1},\hdots,x_{n}]$,
                    and $\prec$ a monomial ordering, the leading
                    monomial of $f$ is $\LM(f)=x^{\multideg(f)}$
                \end{definition}
                \begin{definition}
                    For
                    $f=\sum_{\alpha}%
                       a_{\alpha}x^{\alpha}\in{k}[x_{1},\hdots,x_{n}]$,
                    and $\prec$ a monomial ordering,
                    the leading term of $f$ is
                    $\LT(f)=\LC(f)\cdot\LM(f)$.
                \end{definition}
                \begin{theorem}
                    If $f,g\in k[x_1,\hdots ,x_n]$ are non-zero,
                    then $\multideg(fg)=\multideg(f)+\multideg(g)$
                \end{theorem}
                \begin{theorem}
                    If $f,g\in k[x_1,\hdots ,x_n]$ are non-zero,
                    and if $f+g\ne 0$, then
                    $\multideg(f+g)\leq\max\{\multideg(f),\multideg(g)\}$.
                \end{theorem}
                \begin{theorem}
                    If $f,g\in k[x_1,\hdots ,x_n]$ are non-zero,
                    $f+g\ne 0$, and if
                    $\multideg(f)\ne\multideg(g)$, then
                    $\multideg(f+g)=\max\{\multideg(f),\multideg(g)\}$.
                \end{theorem}
                \begin{theorem}
                    If $>$ is a monomial ordering on $\mathbb{N}^n$,
                    and $F=(f_1,\hdots,f_s)$ is an ordered $s-$tuple
                    of polynomials in $k[x_1,\hdots ,x_n]$,
                    then every $f\in k[x_1,\hdots ,x_n]$ can be
                    written as $f=r+\sum_{k=1}^{s}a_{k}f_{k}$,
                    where $a_{k},r\in{k}[x_{1},\hdots,x_{n}]$,
                    and either $r=0$ or $r$ is a linear combination,
                    with coefficients in $k$, of monomials, none of
                    which is divisible by any of
                    $\LT(f_{1}),\hdots,\LT(f_{s})$.
                    We call $r$ the remainder of $f$
                    with respect to $F$.
                \end{theorem}
                \begin{definition}
                    An ideal $I\subset k[x_1,\hdots ,x_n]$ is a
                    monomial ideal if there is a subset
                    $A\subset\mathbb{N}^{n}$ such that $I$ consists
                    of all polynomials which are finite sums of
                    the form $\sum_{\alpha} h_{\alpha}x^{\alpha}$,
                    where $h_{\alpha}\in {k}[x_{1},\hdots,x_{n}]$. 
                \end{definition}
                \begin{theorem}
                    If $I=\langle x^\alpha: \alpha \in A\}$ is
                    a monomial ideal, then a monomial $x^\beta$ lies
                    in $I$ if and only if $x^\beta$ is divisible by
                    $x^\alpha$ for some $\alpha \in A$.
                \end{theorem}
                \begin{theorem}
                    If $I$ is a monomial ideal, and
                    $f\in{k}[x_{1},\hdots,x_{n}]$, then
                    the following are equivalent:
                    \begin{enumerate}
                            \item $f\in I$
                            \item Every term of $f$ lies in $I$.
                            \item $f$ is a $k-$linear combination
                                  of the monomials in $I$.
                    \end{enumerate}
                \end{theorem}
                \begin{theorem}[Dickson's Lemma]
                    If $I=\langle{x}^{\alpha}:\alpha\in{A}\rangle$
                    is a monomial ideal, then $I$ can be written as
                    $\langle{x}^{\alpha(1)},\hdots,x^{\alpha(s)}\rangle$,
                    where $\alpha(1),\hdots,\alpha(s)\in{A}$. 
                \end{theorem}
                \begin{theorem}
                    If $>$ is a relation on $\mathbb{N}^n$
                    such that $>$ is a total ordering and for
                    $\alpha>\beta$ and $\gamma\in\mathbb{N}^n$,
                    $\alpha+\gamma>\beta+\gamma$, then $>$ is a
                    well-ordering if and only if for all
                    $\alpha\in\mathbb{N}^n$, $\alpha\geq{0}$.
                \end{theorem}
            \subsubsection{The Hilbert Basis Theorem and Groebner Bases}
                \begin{definition}
                    For a non-zero ideal $I\subset k[x_1,\hdots ,x_n]$,
                    $\LT(I)$ is the set of leading terms of elements
                    of $I$. $\langle \LT(I)\rangle$ is the ideal
                    generated by this set.
                \end{definition}
                \begin{theorem}
                    If $I\subset k[x_1,\hdots ,x_n]$ is an ideal,
                    then $\langle \LT(I)\rangle$ is a monomial ideal.
                \end{theorem}
                \begin{theorem}
                    If $I\subset k[x_1,\hdots ,x_n]$ is an ideal,
                    then there are $g_1,\hdots, g_t\in I$ such that
                    $\langle\LT(I)\rangle%
                     =\langle\LT(g_1),\hdots,\LT(g_t)\rangle$
                \end{theorem}
                \begin{theorem}[Hilbert Basis Theorem]
                    Every ideal $I\subset k[x_1,\hdots,x_n]$
                    has a finite generating set.
                \end{theorem}
                \begin{definition}
                    For a monomial order $>$, a finite subset
                    $G=\{g_1,\hdots, g_t\}$ of an ideal $I$ is
                    said to be a Groebner Basis if
                    $\langle\LT(g_1),\hdots,\LT(g_t)\rangle%
                     =\langle \LT(I)\rangle$
                \end{definition}
                \begin{theorem}
                    If $>$ is a monomial order, then every non-zero
                    ideal $I\subset k[x_1,\hdots,x_n]$ has a Groebner basis.
                \end{theorem}
                \begin{theorem}
                    If $I\subset k[x_1,\hdots ,x_n]$ is a non-zero
                    ideal and $G$ is a Groebner Basis, then $G$
                    is also a generated set of $I$.
                \end{theorem}
                \begin{theorem}[The Ascending Chain Condition]
                    If $I_n$ is a sequence of ideals such that
                    $I_{n}\subset I_{n+1}$, then there is an
                    $N\in\mathbb{N}$ such that for all $n>N$,
                    $I_n=I_N$.
                \end{theorem}
                \begin{definition}
                    If $I\subset k[x_1,\hdots ,x_n]$ is an ideal,
                    then $\textbf{V}(I)$ is the set
                    $\{\alpha\in k^n:\forall_{f\in I},f(\alpha)=0\}$
                \end{definition}
                \begin{theorem}
                    If $I\subset k[x_1,\hdots ,x_n]$ is an ideal,
                    then $\textbf{V}(I)$ is an affine variety.
                \end{theorem}
                \begin{theorem}
                    If $I=\langle f_1,\hdots, f_s\rangle$,
                    then $\textbf{V}(I)=\mathbf{V}(f_1,\hdots,f_s)$.
                \end{theorem}
            \subsubsection{Properties of Groebner Bases}
                \begin{theorem}
                    If $G=\{g_1,\hdots, g_t\}$ is a Groebner basis 
                    of $I\subset k[x_1,\hdots ,x_n]$ and
                    $f\in k[x_1,\hdots ,x_n]$, then there is a
                    unique $r\in k[x_1,\hdots ,x_n]$ such that $r$
                    is not divisible by any of
                    $\LT(g_1),\hdots,\LT(g_t)$, and there is a
                    $g\in I$ such that $f=g+r$. 
                \end{theorem}
                We write $\overline{f}^{F}$ for the remainder on division of
                $f$ by $F=(f_{1},\hdots,f_{s})$
                \begin{definition}
                    If $f,g\in k[x_1,\hdots ,x_n]$ are
                    non-zero polynomials,
                    $\multideg(f)=\alpha$, $\multideg(g)=\beta$,
                    and if $\gamma=(\gamma_1,\hdots, \gamma_n)$,
                    where $\gamma_k=\max\{\alpha_k,\beta_k\}$,
                    then $x^y$ is the least common multiple of
                    $\LM(f)$ and $\LM(g)$, denoted
                    $x^{y}=\LCM(\LM(f),\LM(g))$.
                \end{definition}
                \begin{definition}
                    If $f,g\in k[x_1,\hdots ,x_n]$ are non-zero,
                    then the $S-$polynomial of $f$ and $g$ is
                    $S(f,g)=\frac{x^y}{\LT(f)}f-\frac{x^y}{\LT(g)}g$
                \end{definition}
                \begin{theorem}[Buchberger's Criterion]
                    If $I$ is a polynomial ideal, then a basis
                    $G=\{g_1,\hdots, g_t\}$ for $I$ is a
                    Groebner basis for $I$ if and only if for
                    all pairs $i\ne j$, the remainder on
                    division of $S(g_i,g_j)$ by $G$ is zero.
                \end{theorem}
        \subsection{The Algebra-Geometry Dictionary}
            \subsubsection{Hilbert's Nullstellensatz}
                \begin{theorem}[The Weak Nullstellensatz Theorem]
                    If $k$ is an algebraically closed field,
                    $I\subset k[x_1,\hdots ,x_n]$ is an ideal,
                    and $\mathbf{V}(I)=\emptyset$,
                    then $I=k[x_1,\hdots ,x_n]$.
                \end{theorem}
                \begin{theorem}[Hilbert's Nullstellensatz]
                    If $k$ is an algebraically closed,
                    $f_{1},\hdots,f_{s}\in k[x_{1},\hdots,x_{n}]$,
                    and if
                    $f\in\textbf{I}\big(\mathbf{V}(f_1,\hdots,f_s)\big)$,
                    then $\exists_{m\in\mathbb{N}}$ such that
                    $f^m \in \langle f_1,\hdots, f_s \rangle$.
                \end{theorem}
            \subsubsection{Radical Ideals and the Ideal-Variety Correspondence}
                \begin{theorem}
                    If $V$ is an affine variety, and if
                    $f\in \textbf{I}(V)$, then $f^m\in \textbf{I}(V)$.
                \end{theorem}
                \begin{definition}
                    An ideal $I$ is said to be radical $f^m \in I$
                    implies $f\in I$ for some $m\geq 1$.
                \end{definition}
                \begin{theorem}
                    If $V$ is an affine variety,
                    then $\textbf{I}(V)$ is a radical ideal.
                \end{theorem}
                \begin{definition}
                    The radical of an ideal
                    $I\subset k[x_{1},\hdots,x_{n}]$ is the set
                    $\sqrt{I}=\{f:f^{m}\in I,m\in\mathbb{N}\}$.
                \end{definition}
                \begin{theorem}
                    If $I\subset k[x_1,\hdots ,x_n]$ is an ideal,
                    then $\sqrt{I}$ is an ideal.
                \end{theorem}
                \begin{theorem}[The Strong Nullstellensatz]
                    If $k$ is an algebraically closed,
                    and $I\subset k[x_1,\hdots ,x_n]$ is an ideal,
                    then $\textbf{I}(\mathbf{V}(I))=\sqrt{I}$.
                \end{theorem}
                \begin{theorem}[The Ideal-Variety Correspondence]
                    If $k$ is a field, then the maps
                    $\textrm{affine varieties}%
                     \overset{\textbf{I}}\rightarrow\textrm{ideals}$
                    and
                    $\textrm{ideals}%
                     \overset{\mathbf{V}}\rightarrow\textrm{affine varieties}$
                    are inclusion reversing and for any
                    afffine variety $V$,
                    $\mathbf{V}\big(\textbf{I}(V)\big)=V$.
                \end{theorem}
                \begin{theorem}[Radical Membership Theorem]
                    If $k$ is a field and
                    $I=\langle f_1,\hdots,f_s\rangle\subset k[x_1,\hdots,x_n]$
                    is an ideal, then $f\in \sqrt{I}$ if and only if
                    the constant polynomial $1$ belongs to
                    $\langle f_1,\hdots, f_s, 1-yf\rangle$.
                \end{theorem}
                \begin{theorem}
                    If $f\in k[x_1,\hdots ,x_n]$, and
                    $I=\langle f\rangle$, and if
                    $f=f_1^{\alpha_1}\cdots f_s^{\alpha_s}$,
                    then $\sqrt{I}=\langle f_1\cdots f_s\rangle$.
                \end{theorem}
                \begin{definition}
                    The reduction of a polynomial
                    $f\in k[x_1,\hdots ,x_n]$ is the polynomial
                    $f_{red}$ such that
                    $\langle f_{red}\rangle=\sqrt{\langle f\rangle}$.
                \end{definition}
                \begin{definition}
                    A square free polynomial is a polynomial
                    $f\in k[x_1,\hdots ,x_n]$ such that $f=f_{red}$.
                \end{definition}
                \begin{definition}
                    If $f,g\in k[x_1,\hdots ,x_n]$, then
                    $h\in k[x_1,\hdots ,x_n]$ is said to be the
                    greatest common divisor of $f$ and $g$ if $f$
                    divides $f$ and $g$, and if $p$ is any polynomial
                    that divides $f$ and $g$, then $p$ divides $h$.
                \end{definition}
                \begin{theorem}
                    If $k$ is a field such that $\mathbb{Q}\subset k$,
                    and $I=\langle f\rangle$ for some
                    $f\in k[x_1,\hdots ,x_n]$, then
                    $\sqrt{I}=\langle f_{red}\rangle$,
                    where
                    $f_{red}=\frac{f}{GCD%
                         \big(%
                             f,%
                             \frac{\partial f}{\partial x_1},%
                             \frac{\partial f}{\partial x_2},%
                             \hdots,%
                             \frac{\partial f}{\partial x_n}%
                         \big)}$
                \end{theorem}
            \subsubsection{Sums, Products, and Intersections of Ideals}
                \begin{definition}
                    If $I$ and $J$ are ideals of a the ring
                    $k[x_1,\hdots ,x_n]$, then the sum of $I$ and $J$,
                    denoted $I+J$, is the set
                    $I+J=\{f+g: f\in I, g\in J\}$.
                \end{definition}
                \begin{theorem}
                    If $I$ and $J$ are ideals in $k[x_1,\hdots ,x_n]$,
                    then $I+J$ is also an ideal in $k[x_1,\hdots ,x_n]$.
                \end{theorem}
                \begin{theorem}
                    If $I$ and $J$ are ideals in $k[x_1,\hdots ,x_n]$,
                    then $I+J$ is the smallest ideal containing $I$ and $J$.
                \end{theorem}
                \begin{theorem}
                    If $f_1,\hdots, f_r \in k[x_1,\hdots ,x_n]$,
                    then
                    $\langle f_1,\hdots, f_r\rangle%
                     =\sum_{k=1}^{r}\langle f_k\rangle$
                \end{theorem}
                \begin{theorem}
                    If $I$ and $J$ are ideals in
                    $k[x_1,\hdots ,x_n]$, then
                    $\mathbf{V}(I+J)=\mathbf{V}(I)\cap\mathbf{V}(J)$.
                \end{theorem}
                \begin{definition}
                    If $I$ and $J$ are two ideals in
                    $k[x_1,\hdots ,x_n]$, then their product,
                    denoted $I\cdot J$, is defined to be the ideal
                    generated by all polynomials $f\cdot g$,
                    where $f\in I$, and $g\in J$.
                \end{definition}
                \begin{theorem}
                    If $I = \langle f_1,\hdots, f_r\rangle$ and
                    $J = \langle g_1,\hdots, g_s\rangle$, then
                    $I \cdot J$ is generated by the set of all
                    products
                    $\{f_ig_j:1\leq i\leq r, 1\leq j \leq s\}$
                \end{theorem}
                \begin{theorem}
                    If $I,J\subset k[x_1,\hdots ,x_n]$
                    are ideals, then
                    $\mathbf{V}(I\cdot J)=\mathbf{V}(I)\cup\mathbf{V}(J)$.
                \end{theorem}
                \begin{definition}
                    If $I,J\subset k[x_1,\hdots ,x_n]$ are ideals,
                    then the intersection of $I$ and $J$,
                    denoted $I\cap J$, is the set of polynomials
                    in both $I$ and $J$.
                \end{definition}
                \begin{theorem}
                    If $I,J\subset k[x_1,\hdots ,x_n]$ are ideals,
                    then $I\cap J$ is an ideal.
                \end{theorem}
            \subsubsection{Zariski Closure and Quotients of Ideals}
                \begin{theorem}
                        If $S\subset k^n$, then the affine variety
                        $\mathbf{V}\big(\textbf{I}(S)\big)$ is
                        the smallest affine variety that contains $S$.
                \end{theorem}
                \begin{definition}
                    The Zariski Closure of a subset $S$,
                    denoted $\overline{S}$, of an affine space
                    is the smallest affine algebraic variety
                    containing the set. 
                \end{definition}
                \begin{theorem}
                    If $k$ is an algebraically closed field
                    and $V=\mathbf{V}(f_1,\hdots, f_s)\subset k^n$,
                    then $\mathbf{V}(I_{\ell})$ is the Zariski Closure
                    of $\pi_{\ell}(V)$.
                \end{theorem}
                \begin{theorem}
                    If $V$ and $W$ are varieties such that
                    $V\subset W$,
                    then $W=V\cup \overline{\big(W\setminus V\big)}$.
                \end{theorem}
                \begin{definition}
                    If $I,J\subset k[x_1,\hdots ,x_n]$ are ideals,
                    then $I:J$ is the set,
                    $\{f\in k[x_1,\hdots ,x_n]: fg \in I\ \forall_{g\in J}\}$
                    and is called the ideal quotient of $I$ by $J$.
                \end{definition}
                \begin{theorem}
                    If $I,J\subset k[x_1,\hdots ,x_n]$ are ideals,
                    then $I:J$ is an ideal.
                \end{theorem}
                \begin{theorem}
                    If $I,J\subset k[x_1,\hdots ,x_n]$ are ideals,
                    then
                    $\overline{\mathbf{V}(I)\setminus%
                     \mathbf{V}(J)}\subset\mathbf{V}(I:J)$.
                \end{theorem}
                \begin{theorem}
                    If $I,J\subset k^n$ are affine varieties,
                    then $\textbf{I}(V):\textbf{I}(W)=\textbf{I}(V\setminus)$
                \end{theorem}
                \begin{theorem}
                    If $I,J,K\subset k[x_1,\hdots ,x_n]$,
                    then $I:k[x_1,\hdots ,x_n]=I$.
                \end{theorem}
                \begin{theorem}
                    If $I,J,K \subset k[x_1,\hdots ,x_n]$ are ideals,
                    then $I\cdot J\subset K$ if and only if $I\subset K:J$
                \end{theorem}
                \begin{theorem}
                    If $I,J,K\subset k[x_1,\hdots ,x_n]$ are ideals,
                    then $J\subset I$ if and only if
                    $I:J=k[x_1,\hdots ,x_n]$
                \end{theorem}
                \begin{theorem}
                    If $I$ is an ideal, $g\in k[x_1,\hdots ,x_n]$,
                    and if $\{h_1,\hdots, h_p\}$ is a basis of the
                    ideal $I\cap \langle g \rangle$, then
                    $\{h_1/g,\hdots, h_p/g\}$ is a basis of
                    $I:\langle g\rangle$.
                \end{theorem}
            \subsubsection{Irreducible Varieties and Prime Ideals}
                \begin{definition}
                    An affine variety $V\subset k^n$ is irreducible
                    if there are no affine varieties $V_1, V_2$,
                    such that $V = V_1\cup V_2$, $V_1,V_2\ne \emptyset$,
                    and $V_1 \ne V, V_2 \ne V$.
                \end{definition}
                \begin{definition}
                    An ideal $I\subset k[x_1,\hdots ,x_n]$ is
                    said to be prime if whenever
                    $f,g\in k[x_1,\hdots ,x_n]$ and $fg\in I$,
                    either $f\in I$ or $g\in I$.
                \end{definition}
                \begin{theorem}
                    If $V\subset k^n$ is an affine variety,
                    then $V$ is irreducible if and only if
                    $\textbf{I}(V)$ is a prime ideal.
                \end{theorem}
                \begin{definition}
                    An ideal $I\subset k[x_1,\hdots ,x_n]$ is
                    said to be maximal if $I \ne k[x_1,\hdots ,x_n]$
                    and any ideal $J$ containing $I$ is such that
                    either $J=I$ or $J=k[x_1,\hdots ,x_n]$.
                \end{definition}
                \begin{definition}
                    An ideal $I\subset k[x_1,\hdots,x_n]$
                    is called proper if $I$ is not equal to
                    $k[x_1,\hdots ,x_n]$.
                \end{definition}
                \begin{theorem}
                    If $k$ is a field and
                    $I=\langle x_1-a_1,\hdots,x_n-a_n\rangle$
                    is and ideal where $a_1,\hdots, a_n \in k$,
                    then $I$ is maximal.
                \end{theorem}
                \begin{theorem}
                    If $k$ is a field, then any maximal
                    ideal is also a prime ideal.
                \end{theorem}
                \begin{theorem}
                    If $k$ is an algebraically closed field,
                    then every maximal ideal of $k[x_1,\hdots ,x_n]$
                    is of the form
                    $\langle x_1-a_1,\hdots, x_n-a_n\rangle$
                    for some $a_1,\hdots, a_n\in k$.
                \end{theorem}
                \begin{definition}
                    A primary decomposition of an ideal $I$ is
                    an expression of $I$ as an intersection of
                    primary ideals $I=\cap_{i=1}^{r} Q_{i}$.
                \end{definition}
                \begin{definition}
                    A primary decomposition of an ideal $I$
                    is said to be minimal $\sqrt{Q_i}$ are all
                    distinct and
                    $\cap_{j\ne i}Q_j\not\subset Q_i$
                \end{definition}
                \begin{theorem}
                    If $I,J$ are primary and
                    $\sqrt{I}=\sqrt{J}$,
                    then $I\cap J$ is primary.
                \end{theorem}
                \begin{theorem}[Lasker-Noether Theorem]
                    Every ideal $I \subset k[x_1,\hdots ,x_n]$
                    has a minimal primary decomposition.
                \end{theorem}
        \subsection{Polynomials and Rational Functions on a Variety}
            \subsubsection{Polynomial Mappings}
                \begin{definition}
                    If $V\subset k^m$, $W\subset k^n$ are affine
                    varieties, a function $\phi:V\rightarrow W$ is
                    said to be a polynomial mapping if there exist
                    polynomials $f_1,\hdots, f_n\in k[x_1,\hdots, x_m]$
                    such that
                    $\phi(a_1,\hdots, a_m)%
                     =\big(%
                         f_1(a_1,\hdots,a_m),\hdots,f_n(a_1,\hdots, a_m)%
                      \big)$
                    for all $(a_1,\hdots, a_m) \in V$. We say that
                    $(f_1,\hdots, f_n)$ represents $\phi$.
                \end{definition}
                \begin{theorem}
                    If $V\subset k^m$ is an affine variety,
                    then $f,g\in k[x_1,\hdots, x_m]$ represent the
                    same polnyomial on $V$ if and only if
                    $f-g\in\textbf{I}(V)$.
                \end{theorem}
                \begin{theorem}
                    If $V\subset k^m$ is an affine variety, then
                    $(f_1,\hdots, f_n)$ and $(g_1,\hdots, g_n)$
                    represent the same polynomial mapping if and only
                    if $f_i-g_i \in \textbf{I}(V)$ for $1\leq i \leq n$.
                \end{theorem}
                The set of polynomial mappings from $V$ to $k$ is denoted
                $k[V]$.
                \begin{theorem}
                    If $V\subset k^n$ is an affine variety,
                    the the following are equivalent:
                    \begin{enumerate}
                        \begin{multicols}{3}
                            \item $V$ is irreducible.
                            \item $\textbf{I}(V)$ is a prime ideal.
                            \item $k[V]$ is an integral domain.
                        \end{multicols}
                    \end{enumerate}
                \end{theorem}
            \subsubsection{Quotients of Polynomial Rings}
                \begin{definition}
                    If $I\subset k[x_{1},\hdots,x_{n}]$ is an ideal,
                    if $f,g\in k[x_{1},\hdots,x_{n}]$, then $f$ and
                    $g$ are congruent modulo $I$, denoted
                    $f\equiv g\mod I$, if $f-g\in I$.
                \end{definition}
                \begin{theorem}
                    If $I\subset k[x_1,\hdots ,x_n]$ is an ideal,
                    then the congruence modulo $I$ is an equivalence
                    relation on $k[x_1,\hdots ,x_n]$.
                \end{theorem}
                \begin{theorem}
                    There exists a bijection from the set of distinct
                    polynomial functions $\phi:V\rightarrow k$ and the
                    set of equivalence classes of polynomials under
                    congruence modulo $\textbf{I}(V)$.
                \end{theorem}
                \begin{definition}
                    The quotient of $k[x_1,\hdots ,x_n]$ modulo $I$,
                    denoted $k[x_1,\hdots ,x_n]/I$, is the set of
                    equivalence classes for congruence modulo $I$.
                \end{definition}
                \begin{theorem}
                    If $I\subset k[x_1,\hdots ,x_n]$ is an ideal,
                    then $k[x_1,\hdots ,x_n]/I$ is a commutative ring
                    under the sum and product operations.
                \end{theorem}
                \begin{definition}
                    A ring isomorphism of rings $R$ and $S$ is a
                    bijective function $\phi:R\rightarrow S$ such that:
                    \begin{enumerate}
                        \begin{multicols}{2}
                            \item For all $a,b\in R$,
                                  $\phi(a+b)=\phi(a)+\phi(b)$
                            \item For all $a,b\in R$,
                                  $\phi(ab)=\phi(a)\phi(b)$
                        \end{multicols}
                    \end{enumerate}
                \end{definition}
                \begin{theorem}
                    If $I\subset k[x_1,\hdots ,x_n]$ is an ideal,
                    then there is a bijection between the ideals
                    in the quotient ring $k[x_1,\hdots ,x_n]/I$
                    and the ideals of $k[x_1,\hdots ,x_n]$
                    that contain $I$.
                \end{theorem}
                \begin{theorem}
                    If $I\subset k[x_1,\hdots ,x_n]$ is an ideal,
                    then every ideal of $k[x_1,\hdots ,x_n]/I$
                    is finitely generated.
                \end{theorem}
            \subsubsection{The Coordinate Ring of an Affine Variety}
                \begin{definition}
                    The coordinate ring of an affine variety
                    $V\subset k^n$ is the ring $k[V]$.
                \end{definition}
                \begin{definition}
                    If $V\subset k^n$ is an affine variety, and if
                    $J=\langle\phi_1,\hdots,\phi_s\rangle\subset k[V]$,
                    then
                    $\mathbf{V}_{V}(J)%
                     =\{x\in V:\forall_{\phi \in J},\phi(x)=0\}$
                    is called the subvariety of $V$.
                \end{definition}
                \begin{theorem}
                    If $V\subset k^n$ is an affine variety and if
                    $J\subset k[V]$ is an ideal, then
                    $W=\mathbf{V}_{V}(J)$ is an affine variety
                    in $k^n$ contained in $V$.
                \end{theorem}
                \begin{theorem}
                    If $V\subset k^n$ is an affine variety,
                    and if $W\subset V$, then $\mathbf{V}_{V}(W)$
                    is an ideal of $k[V]$.
                \end{theorem}
                \begin{definition}
                    If $V$ is an irreducible variety in $k^n$,
                    then the function field, denoted $QF(k[V])$,
                    on $V$ is the quotient field of $k[V]$.
                \end{definition}
                \begin{definition}
                    If $V\subset k^m$ and $W\subset k^n$ are irreducible
                    affine varieties, then a rational mapping is a
                    function $\phi$ such that
                    $\phi(x_1,\hdots, x_m)%
                     =\bigg(%
                          \frac{f_1(x_1,\hdots, x_m)}%
                               {g_1(x_1,\hdots, x_m)},%
                          \hdots,%
                          \frac{f_n(x_1,\hdots, x_m)}%
                               {g_n(x_1,\hdots, x_m)}%
                      \bigg)$.
                \end{definition}
                \begin{theorem}
                    Two rational mappings $\phi,\psi:V\rightarrow W$
                    are equal if and only if there is a proper subvariety
                    $V'\subset V$ such that $\phi$ and $\psi$ are
                    defined on $V\setminus V'$ and $\phi(p)=\psi(p)$ for
                    all $p\in{V}\setminus{V'}$.
                \end{theorem}
                \begin{theorem}[The Closure Theorem]
                    If $k$ is an algebraically closed field,
                    $V=\mathbf{V}(I)$, $V\ne\emptyset$, then there
                    is an affine variety
                    $W\underset{Proper}\subset\mathbf{V}(I_{\ell})$
                    such that
                    $\mathbf{V}(I_{\ell})\setminus%
                     W\subset\pi_{\ell}(V)$.
                \end{theorem}
    \section{Miscellaneous Notes}
        \subsection{Groebner Bases}
            \begin{definition}
                A ring is a set $R$ with two binary operations $+$
                and $\cdot$, called addition and multiplication,
                such that the following are true:
                \begin{enumerate}
                    \begin{multicols}{3}
                        \item $(R,+)$ is an Abelian Group
                        \item $(a\cdot{b})\cdot{c}=a\cdot(b\cdot{c})$
                        \item $a\cdot(b+c)=(a\cdot b)+(a\cdot c)$
                    \end{multicols}
                \end{enumerate}
            \end{definition}
            \begin{definition}
                A commutative ring is a ring $R$ such that
                $\forall_{a,b\in R},a\cdot{b}=b\cdot{a}$
            \end{definition}
            \begin{definition}
                A ring with identity is a ring $R$ such that
                $\exists_{1_{R}\in R}:\forall_{a\in R}, 1_{R}\cdot a=a\cdot 1_{R}=a$
            \end{definition}
            \begin{definition}
                A subring of a ring with identity $R$ is a set
                $S\subset R$ such that $1_{R}\in S$, and $S$ is
                closed under the ring operations.
            \end{definition}
            \begin{definition}
                A monomial in variables $x_1,\hdots, x_n$ over a
                ring $R$ is a product
                $x^\alpha=\prod_{k=1}^{n} x_1^{\alpha_1}$,
                where $(\alpha_1,\hdots,\alpha_n)\in \mathbb{N}^n$.
            \end{definition}
            The set of monomials in $n$ variables over $R$ is denoted
            $\Mon_{R}(x_1,\hdots, x_n)$.
            \begin{definition}
                If $\alpha,\beta \in \mathbb{N}^n$ such that
                $\alpha_i \leq \beta_i$, then $x^{\alpha}$ is said
                to divide $x^\beta$, denoted $x^\alpha \vert x^\beta$,
                if $x^\beta = x^\alpha \cdot x^\gamma$ for some
                $\gamma\in\mathbb{N}^n$.
            \end{definition}
            \begin{definition}
                A term is a monomial multiplied by a coefficient in $R$.
            \end{definition}
            \begin{definition}
                A polnyomial over $R$ is a finite $R-$linear
                combination of monomials,
                $f=\sum_{\alpha} a_{\alpha}\cdot x^{\alpha}$.
            \end{definition}
            The set of all polynomials in $n$ variables over a ring $R$ is
            denoted $R[x_1,\hdots, x_n]$.
            \begin{theorem}
                If $R$ is a commutative ring with identity,
                then $R[x_1,\hdots, x_n]$ is a commutative
                ring with identity.
            \end{theorem}
            \begin{definition}
                A polynomial $f\in R[x_1,\hdots, x_n]$ is
                called a constant polynomial if $f\in R$.
            \end{definition}
            \begin{definition}
                A field $k$ is a commutative ring with identity
                such that for all $a\in k$, $a\ne 0$, there is a
                $b\in k$ such that $a\cdot b=1$
            \end{definition}
            We usually work with fields and consider polynomial rings of the
            form $k[x_1,\hdots ,x_n]$.
            \begin{definition}
                A total ordering on a set $A$ is a relation
                $>$ such that $\forall_{a,b\in A}$, precisely one
                of the following truee:
                \begin{enumerate}
                    \begin{multicols}{3}
                        \item $a<b$
                        \item $a=b$
                        \item $b<a$
                    \end{multicols}
                \end{enumerate}
            \end{definition}
            \begin{definition}
                A relation $\sim$ on a set $A$ is said to be
                transitive if for all $a,b,c\in A$, if $a\sim b$ and
                $b\sim c$, then $a\sim c$.
            \end{definition}
            \begin{definition}
                A well ordering on a set $A$ is a relation $<$
                such that for every subset $E\subset A$, there is an
                element $x\in E$ such that for all $y\in E$, $y\ne x$,
                we have $x<y$.
            \end{definition}
            Equivalently, a well ordering on a set $A$ is a relation $<$ such
            that for every monotonically decreasing sequence $\alpha_n$, there
            is an $N\in \mathbb{N}$ such that for all $n>N$,
            $\alpha_{n}=\alpha_{N}$. That is, decreasing sequences terminate.
            \begin{definition}
                A monomial ordering on $\mathbb{N}^n$ is a relation
                $>$ such that $>$ is total, transitive, well
                ordering. A well ordering on $k[x_1,\hdots ,x_n]$
                is a well ordering on
                $\alpha=(\alpha_1,\hdots,\alpha_n)\in\mathbb{N}^n$.
            \end{definition}
            \begin{definition}
                The lexicographic ordering on $\mathbb{N}^n$ is
                defined as
                $(\alpha_1,\hdots,\alpha_n)\underset{Lex}{>}%
                 (\beta_1,\hdots,\beta_n)$
                if the left-most non-zero entry of
                $(\alpha_1-\beta_1,\hdots, \alpha_n-\beta_n)$
                is positive.
            \end{definition}
            \begin{theorem}
                The lexicographic ordering is a monomial ordering.
            \end{theorem}
            \begin{definition}
                The graded lexicographic ordering is defined as
                $(\alpha_1,\hdots,\alpha_n)\underset{GrLex}{>}%
                 (\beta_1,\hdots, \beta_n)$
                if $|\alpha|>|\beta|$ or $|\alpha|=|\beta|$
                and $\alpha\underset{Lex}{>}\beta$.
            \end{definition}
            \begin{theorem}
                The graded lexicographic ordering is a monomial ordering.
            \end{theorem}
            \begin{theorem}[The Division Algorithm]
                If $f_1,\hdots, f_s\in k[x_1,\hdots ,x_n]$ are
                non-zero polynomials and if $>$ is a monomial ordering,
                then there are $r,q_1,\hdots, q_n\in k[x_1,\hdots ,x_n]$
                such that the following are true:
                \begin{enumerate}
                    \item $f=q_{1}f_{1}+\hdots+q_{s}f_{s}+r$
                    \item No term of $r$ is divisible by
                          any of $\LT(f_{1}),\hdots,\LT(f_{s})$.
                    \item $\LT(f)=\max_{>}\{\LT(q_{i})%
                           \cdot\LT(f_i):q_i\ne{0}\}$
                \end{enumerate}
            \end{theorem}
            \begin{definition}
                An ideal
                $I=\langle{x}^{\alpha}:\alpha\in{A}\rangle%
                  =\{\sum_{\alpha}h_{\alpha}x^\alpha,h_{\alpha}%
                   \in k[x_1,\hdots ,x_n]\}$
                is called a monomial ideal.
            \end{definition}
            \begin{theorem}
                If $I=\langle{x}^\alpha:\alpha\in{A}\rangle$
                is a monomial ideal,
                $\beta\in\mathbb{N}^n$, then $x^\beta\in{I}$
                if and only if there is an $\alpha\in{A}$
                such that $x^{\alpha}$ divides $x^{\beta}$.
            \end{theorem}
            \begin{theorem}
                If $I$ is a monomial ideal,
                $f\in{k}[x_1,\hdots ,x_n]$,
                then the following are equivalent:
                \begin{enumerate}
                    \item $f\in I$
                    \item Every term of $f$ lies in $I$.
                    \item $f$ is a $k-$linear combination of
                          monomials in $I$.
                \end{enumerate}
            \end{theorem}
            \begin{theorem}[Dickson's Lemma]
                Every monomial ideal of $k[x_{1},\hdots,x_{n}]$
                is finitely generated.
            \end{theorem}
            \begin{theorem}[Hilbert's Basis Theorem]
                Every ideal $I\subset{k}[x_{1},\hdots,x_{n}]$
                is finitely generated.
            \end{theorem}
            \begin{definition}
                If $>$ is a monomial ordering on $k[x_{1},\hdots,x_{n}]$,
                then a Groebner Basis of $I\subset k[x_{1},\hdots,x_{n}]$
                is a set $G=\{g_{1},\hdots,g_{s}\}$ such that
                $\langle\LT(I)\rangle%
                 =\langle \LT(g_1),\hdots,\LT(g_s)\rangle$
            \end{definition}
            \begin{theorem}
                Every non-zero ideal
                $I\subset{k}[x_{1},\hdots,x_{n}]$
                has a Groebner Basis.
            \end{theorem}
        \subsection{Elimination Theory}
            \begin{definition}
                If $I\subset{k}[x_1,\hdots,x_{n}]$ is an ideal,
                then the $i^{th}$ elimination ideal of $I$,
                denoted $I_{i}$, is the set
                $I_{i}=I\cap{k}[x_{i+1},\hdots,x_{n}]$,
                where $1\leq{i}\leq{n}$, and $I_{0}=I$.
            \end{definition}
            \begin{theorem}[The Elimination Theorem]
                If $I\subset k[x_1,\hdots ,x_n]$ is an ideal and
                $G$ is a Groebner Basis of $I$ with respect to the
                lexicographic ordering, and $x_1>\hdots > x_n$, then
                for all $i=0,1,\hdots,n$, the set
                $G_{i}\cap{k}[x_1,\hdots,x_n]$ is a Groebner Basis of
                the $i^{th}$ elimination ideal $I_{i}$.
            \end{theorem}
            Using the lexicographic ordering, and for some ideal
            $I=\langle{f_{1}},\hdots,f_{s}\rangle%
               \subset{k}[x_1,\hdots ,x_n]$,
            to compute all elimination ideals $I_{i}$:
            \begin{enumerate}
                \item Compute a Groebner Basis $G$ for $I$ with
                      respect to the lex order on $k[x_1,\hdots,x_n]$.
                \item For all $i$, the elements $g\in G$ with
                      $\LT(g)\in{k}[x_{i+1},\hdots,x_{n}]$ form a
                      Groebner basis $I_{i}$ with respect to
                      the lexicographic ordering on
                      $k[x_{i+1},\hdots,x_n]$.
            \end{enumerate}å
            \begin{definition}
                A monomial order on
                $k[x_{1},\hdots,x_{n},y_{1},\hdots,y_{m}]$
                is an elimination order with respect to
                $x_{1},\hdots,x_{n}$ if the following holds for
                all $f\in{k}[x_{1},\hdots,x_{n},y_{1},\hdots,y_{m}]$:
                $L(f)\in{k}[y_{1},\hdots,y_{m}]%
                 \Rightarrow{f}\in{k}[y_{1},\hdots,y_{m}]$
            \end{definition}
            \begin{theorem}[The Extension Theorem]
                If $k$ is an algebraically closed field,
                $I=\langle{f_{1}},\hdots,f_{s}\rangle$, $I_{1}$ is
                the first elimination ideal of $I$, and if
                $f_{i}=g_{i}(x_{2},\hdots,x_{n})x_{1}^{N_i}+r_{i}$,
                where $r_{i}$ contains only terms where the degree
                of $x_{1}$ is less than $N_{i}$, and if
                $(a_{2},\hdots,a_{n})\in{k}^{n-1}$ such that
                $(a_{2},\hdots,a_{n})\notin%
                 \mathbf{V}(g_{1},\hdots,g_{s})$,
                 then there is an $a_1 \in k$ such that
                 $(a_{1},\hdots,a_{n})\in\mathbf{V}(I_1)$.
            \end{theorem}
            \begin{definition}
                The $k^{th}$ projection map on $k^{n}$ is
                $\pi_{k}:k^{n}\rightarrow k^{n-k}$ defined by
                $(a_{1},\hdots,a_{n})=(a_{k+1},\hdots,a_{n})$
            \end{definition}
            If $I\subset{k}[x_{1},\hdots,x_{n}]$ is an ideal,
            $X=\mathbf{V}(I)$, and $f\in{I_{k}}$, then $f(X)=0$.
            Thus $f\big(\pi_{k}(X)\big)=0$, and therefore
            $\pi_{k}(X)\subset\mathbf{V}(I_k)$.
            Also $\pi_{k}(X)$ may NOT be Zariski closed.
            \begin{theorem}
                If $k$ is algebraically closed,
                then $\overline{\pi_k(X)}=\mathbf{V}(I_k)$.
            \end{theorem}
            \begin{theorem}
                If $k$ is an infinite field,
                $F:k^{m}\rightarrow{k^{n}}$ a function determined
                by some parametrization
                $x_{j}=f_{j}(t_{1},\hdots,t_{m})$, and if
                $I=\langle{x_{1}-f_{1}},\hdots,x_{n}-f_{n}\rangle$,
                then $\mathbf{V}(I_m)$ is the smallest algebraic
                set in $k^{n}$ containing $F(k^{m})$.
            \end{theorem}
            $V(I_{m})$ is the Zariski closure of $F(k^{m})$.
        \subsection{\'{E}tale Cohomology}
            \subsubsection{Review of Schemes}
                Limitations of Affine Varieties:
                \begin{itemize}
                    \item   One would like to construct spaces by gluing
                            together simpler pieces, like in geometry and
                            topology.
                    \item Difficult over non-algebraically
                          closed fields.
                    \item Keeping track of multiplicities.
                \end{itemize}
                Grothendieck's Theory of Schemes gives solutions to
                these problems. Should $x^{2}+y^{2}=-1$ and
                $x^{2}+y^{2}=3$ be regarded as the same over
                $\mathbb{A}_{\mathbb{R}}^{2}$? They both have no
                solution. The answer is no. An isomorphism should
                be given by an invertible transformation. In general,
                the affine variety $X\subset\mathbb{A}_{R}^n$ is
                completely determined by the coordinate ring
                $S=\mathcal{O}(X)%
                  =R[x_{1},\hdots,x_{n}]/(f_{1},\hdots,f_{N})$.
                Given a compact Hausdorff space $X$, let $C(X)$
                denote the set of continous complex valued functions.
                This is a commutative ring with identity. With the
                supremum norm, it becomes a unital $C^{*}-$algebra.
                \begin{theorem}
                    The map $X\rightarrow\max\{C(X)\}$
                    is a homeomorphism.
                \end{theorem}
                Given a continuous map of spaces $f:X\rightarrow Y$,
                we get a homomorphism $C(Y)\rightarrow C(X)$ given
                by $g\mapsto{g}\circ f$. Thus $C(X)$ can be
                regarded as a contravariant functor. 
                \begin{theorem}[Gelfand]
                    The functor $X\mapsto C(X)$ induces an
                    equivalence between the category of compact
                    Hausdorff spaces and the opposite category of
                    commutative unital $C^{*}-$algebras.
                \end{theorem}
                \begin{definition}
                    The spectrum of $R$, denoted $\Spec(R)$,
                    is the set of prime ideals of $R$.
                \end{definition}
                \begin{theorem}
                    The Zariski topology on $\Spec(R)$
                    contains open sets
                    $D(f)=\{p\in\Spec(R):f\notin{p}\}$
                \end{theorem}
                A function $f:\mathbb{R}^{n}\rightarrow\mathbb{R}$
                is $C^{\infty}$ if and only if its restriction to
                the neighborhood of every point is $C^{\infty}$.
                That is, $f\in{C}^{\infty}(X)$ if and only if
                for any open cover
                $\{U_{i}\},f\big|_{U_{i}}\in{C}^{\infty}(U_i)$.
                \begin{definition}
                    If $X$ is a topological space, a presheaf
                    of sets $\mathcal{F}$ is a collection of
                    sets $\mathcal{F}(U)$ for each open set
                    $U\subset X$ together with maps
                    $\rho_{UV}:\mathcal{F}(U)\rightarrow \mathcal{F}(V)$
                    for each pair $U\subset V$ such that $\rho_{UU}=id$
                    and $\rho_{WV}\circ\rho_{VU}=\rho_{WU}$
                    whenever $U\subset{V}\subset{W}$.
                \end{definition}
                \begin{definition}
                    A sheaf is a presheaf such that for any open
                    cover $\{U_i\}$ of an open $U\subset{X}$ and
                    section $f_{i}\in\mathcal{F}(U_{i})$ such that
                    $F_{i}\big|_{U_{i}\cap I_{j}}%
                     =f_{j}\big|_{U_{i}\cap U_{j}}$,
                    there is a unique $f\in\mathcal{F}(U)$
                    such that $f_{i}=f\big|_{U_{i}}$.
                \end{definition}
                \begin{definition}
                    A ringed space is a pair $(X,\mathcal{O}_{X})$,
                    where $X$ is a topological space and
                    $\mathcal{O}_{X}$ is a sheaf of commutative rings.
                \end{definition}
                The collection of presheaves of a topological space
                form a category, denoted $Sh(X)$. 
                \begin{definition}
                    A scheme is a ringed space $(X,\mathcal{O}_X)$
                    which is locally an affine space.
                \end{definition}
                \begin{theorem}
                    A property of commutative rings extends
                    to schemes if it is local.
                \end{theorem}
            \subsubsection{Differential Calculus of Schemes}
                \begin{definition}
                    The tangent space of an affine variety
                    $X=V(f_{1},\hdots,f_{N})\subset\mathbb{A}_{k}^{n}$,
                    denoted $T_{X,p}$, is the set of points
                    $v\in{k}^{n}$ such that
                    $\sum\frac{\partial{f_{j}}}%
                              {\partial{x_{i}}}p)v_{i})%
                     =0$
                \end{definition}
                \begin{definition}
                    A domain
                    $R=k[x_{1},\hdots,x_{n}]/(f_{1},\hdots,f_{N})$
                    or $\Spec(R)$ is smooth if and only if the rank of
                    $\big(\frac{\partial{f_{j}}}{\partial{x_{i}}}(p)\big)$
                    is $n=\dim(R)$ for all $p\in\max(R)$. 
                \end{definition}
                \begin{definition}
                    An \'{e}tale $R$ algebra is smooth of relative
                    dimension 0, where
                    $\det(\frac{\partial{f_{i}}}{\partial{x_{j}}})$
                    is a unit in $S$.
                \end{definition}
                \begin{theorem}
                    If $k$ is a field, then an algebra over $k$
                    is \'{e}tale if and only if it is a finite Cartesian
                    product of separable field extensions. 
                \end{theorem}
                \begin{theorem}
                    The tensor product of two
                    \'{e}tale algebras is \'{e}tale.
                \end{theorem}
                \begin{theorem}
                    If $S$ is \'{e}tale over $R$ and $T$ is
                    \'{e}tale over $S$, then $T$ is \'{e}tale over $R$.
                \end{theorem}
                \begin{definition}
                    If $R$ is a commutative ring and $S$ is an
                    $R$ algebra and $M$ is an $S$ module,
                    then an $R$ linear derivation from $S$ to $M$ is
                    a map $\delta:S\rightarrow M$ such that
                    $\delta(s_{1}+s_{2})=\delta(s_{1})+\delta(s_{2})$,
                    $\delta(s_{1}s_{2})%
                     =s_{1}\delta(s_{2})+s_{2}\delta(s_{1})$,
                     and $\delta(r)=0$ for all $r\in{R}$.
                \end{definition}
                \begin{theorem}
                    There exists an $S$ module $\Omega_{S/R}$
                    with a universal $R$ linear derivation
                    $d:S\rightarrow \Omega_{S/R}$.
                \end{theorem}
                \begin{theorem}
                    If $M$ is a finitely generated module over a
                    Noetherian ring $R$, then these are equivalent:
                    \begin{enumerate}
                        \begin{multicols}{3}
                            \item $M$ is locally free.
                            \item $\forall_{p\in\Spec(R)},R_{p}\otimes M$
                                  is free.
                            \item $M$ is projective.
                        \end{multicols}
                    \end{enumerate}
                \end{theorem}
                \begin{definition}
                    If $M$ is an $S-$module, then $M$ is called
                    flat if $M\otimes{i}$ is injective for any $i$.
                \end{definition}
                \begin{theorem}
                    If $S$ is an $R$ algebra and $M$ is an $S$ module,
                    $f\in S$ is an element such that multiplication
                    by $f$ is injective on $M\otimes k(m)$ for all
                    $m\in\max(R)$, and if $M$ is flat over $R$,
                    then $M/fM$ is flat over $R$.
                \end{theorem}
                \begin{theorem}
                    A smooth algebra is flat.
                \end{theorem}
                \begin{theorem}
                    If $R$ is a Noetherian ring, then a homomorphism
                    $R\rightarrow S$ is \'{e}tale if and only if:
                    \begin{enumerate}
                        \item $S$ is finitely generated as an algebra.
                        \item $S$ is flat as an $R$-module.
                        \item $\Omega_{S/R}=0$.
                    \end{enumerate}
                \end{theorem}
                \begin{definition}
                    A sheaf on a scheme is quasi-coherent if
                    it is with respect to some affine open cover.
                \end{definition}
                \begin{theorem}
                    If $f:X\rightarrow Y$ is a morphism, there
                    exists a quasi-coherent sheaf $\Omega_{X/Y}$
                    such that
                    $\Omega_{X/Y}\big|_{\Spec(S_{ij})}=\Omega_{S_{ij}/R_i}$
                    for open affine covers $\Spec(R_i)=U_{i}$.
                \end{theorem}
            \subsubsection{The Fundamental \'{E}tale Group}
                \begin{definition}
                    A topological group is a topological space
                    $(X,\tau)$ with a group structure $(X,*)$ such
                    that $*:X\times X\rightarrow X$ is a continuous
                    function with respect to the product topology.
                \end{definition}
                \begin{theorem}
                    A topological space is profinite if and only
                    if it is compact Hausdorff and totally disconnected.
                \end{theorem}
                \begin{definition}
                    The topological fundamental group of a
                    topological space $X$, denoted $\pi_1(X)$,
                    is the group of homotopy classes of loops in $X$
                    with a given base point.
                \end{definition}
                \begin{theorem}
                    Any \'{e}tale morphism $Y\rightarrow X$ is
                    a finite to one covering space of $X$ with
                    the usual topology.
                \end{theorem}
                \begin{theorem}[Grothendieck's Theorem]
                    If $X$ is a scheme of finite type of
                    $\mathbb{C}$, then $\pi_{1}^{et}(X)$ is the
                    profinite completion of $\pi_{1}{X}$.
                \end{theorem}
            \subsubsection{\'{E}tale Topology}
                Given a topological space $(X,\tau)$, the topology $\tau$
                (That is, the collection of open sets) forms a partially ordered
                set with respect to set inclusion. There also exists a notion of
                open covering $U=\cup U_{i}$.
                \begin{definition}
                    A Groethendieck Topology on a category $C$ with
                    fibre products is a collection of families of
                    morphisms $U_{i}\rightarrow U$ such that:
                    \begin{enumerate}
                        \item The family consisting of a single
                              isomorphism $\{U\sim{U}\}$ is a covering.
                        \item If $\{U_{i}\rightarrow{U}\}$ and
                              $\{V_{ij}\rightarrow{U_{i}}\}$ are coverings,
                              then so is the composition
                              $\{V_{ij}\rightarrow{U}\}$.
                    \end{enumerate}
                \end{definition}
                \begin{definition}
                    A site is a category with a Grothendieck Topology.
                \end{definition}
        \subsection{The Zariski Topology}
            \subsubsection{The Zariski Topology}
                \begin{definition}
                    A subset of $k^{n}$ is is closed in the
                    Zariski Topology if it is an algebraic set.
                    The Zariski Topology is formed by
                    considering all such sets.
                \end{definition}
                \begin{definition}
                    A topological space $X$ is called irreducible if
                    for any closed subsets $X_1,X_2\subset X$ such that
                    $X=X_{1}\cup{X_{2}}$, either $X=X_{1}$ or
                    $X=X_{2}$. A topological space that is
                    not irreducible is called reducible.
                \end{definition}
                \begin{definition}
                    A subset $Y\subset X$ of a topological space
                    is said to be irreducible if $Y$ is irreducible
                    with respect to the inherited,
                    or the induced topology.
                \end{definition}
                \begin{definition}
                    A topological space $X$ is said to be disconnected
                    if there are two non-empty closed subsets $X_1,X_2$
                    such that $X=X_1\cup X_2$,
                    and $X_1\cap X_2 = \emptyset$.
                \end{definition}
                \begin{theorem}
                    If $X$ is disconnected, then it is reducible.
                \end{theorem}
                \begin{proof}
                    For if $X$ is disconnected, there are two
                    non-empty closed sets $X_1,X_2\subset X$ such
                    that $X_1\cap X_2 = \emptyset$ and
                    $X=X_1\cup X_2$. But if $X_1$ and $X_2$
                    are non-empty and disjoint, then
                    $X_1\ne X$ and $X_2 \ne X$.
                    Therefore $X$ is reducible.
                \end{proof}
                \begin{definition}
                    An algebraic affine variety is an
                    irreducible closed subset of $k^n$.
                \end{definition}
                \begin{definition}
                    An open subset of an affine variety
                    is called a quasi-affine variety.
                \end{definition}
                \begin{definition}
                    If $X\subset k^n$ is an algebraic set,
                    then $k[x_1,\hdots ,x_n]/\mathbb{I}(X)$
                    is called the coordinate ring of $X$.
                \end{definition}
                \begin{definition}
                    A set $Y$ in a topological space $X$ is
                    said to be dense in $X$ if for every
                    non-empty open set $\mathcal{O}$,
                    $\mathcal{O}\cap Y\ne \emptyset$.
                \end{definition}
                \begin{theorem}
                    A topological space $X$ is irreducible if
                    and only if every non-empty open set is dense.
                \end{theorem}
                \begin{definition}
                    An irreducible component of $X$ is a
                    maximal irreducible subset of $X$.
                \end{definition}
                \begin{theorem}
                    If $X$ is a closed topological space,
                    then any irreducible subset $Y\subset X$ is
                    contained in a maximal component.
                \end{theorem}
                \begin{theorem}
                    If $X$ is a topological space,
                    then it is the union of irreducible components.
                \end{theorem}
                \begin{definition}
                    A topological space $X$ is called Noetherian
                    if every descending chain $X_n \subset X_{n+1}$
                    of closed subsets stabilizes.
                \end{definition}
                \begin{theorem}
                    If $X$ is a Noetherian Space,
                    then every subset $Y\subset X$ can be
                    written as a finite union of irreducible
                    closed subsets.
                \end{theorem}
                \begin{theorem}
                    Every algebraic set in $k^n$ can be expressed
                    uniquely as a union of varieties.
                \end{theorem}
                \begin{theorem}
                    If $R$ is an Noetherian ring,
                    then $k[x_1,\hdots ,x_n]$ is Noetherian.
                \end{theorem}
                \begin{theorem}
                    A ring $R$ is Noetherian if and only if every
                    non-empty set of ideals in $R$ has a maximal element.
                \end{theorem}
                \begin{theorem}[Hilbert's Nullstellensats]
                    If $k$ is an algebraically closed field,
                    $I\subset R = k[x_1,\hdots ,x_n]$ is an ideal,
                    and $f\in R$ is a polynomial which vanishes on
                    $\mathbf{V}(I)$, then there is an $n\in \mathbb{N}$
                    such that $f^{n}\in{I}$.
                \end{theorem}
                \begin{definition}
                    The dimension of a topological space $X$ is the
                    supremum of all $n\in \mathbb{N}$ such that
                    there is a chain
                    $Z_0\subset Z_1\subset\hdots\subset Z_n$
                    of distinct irreducible closed
                    subsets of $X$.
                \end{definition}
                \begin{theorem}
                    If $k$ is a field, and $B$ is an integral domain
                    which is finitely generated by a $k-$algebra,
                    then the dimension of $B$ is equal to the
                    transcendence degree of the quotient field $k(B)$
                    of $B$ over $k$.
                \end{theorem}
                \begin{theorem}
                    The dimension of $k^{n}$ is $n$.
                \end{theorem}
                \begin{theorem}
                    If $Y$ is a quasi-affine variety,
                    then $\dim(Y)=\dim(\overline{Y})$.
                \end{theorem}
            \subsubsection{Problems}
                \begin{problem}
                    Let $f\in k[x]$ be a non-constant polynomial
                    in one variable over a field $k$. $f$ is called
                    irreducible if $f\notin k$ and if it is not
                    the product of two polynomials of strictly smaller
                    degree. Prove the following are equivalent:
                    \begin{enumerate}
                        \item $k[x]/\langle f\rangle$ is a field.
                        \item $k[x]/\langle f\rangle$ is an
                              integral domain.
                        \item $f$ is irreducible.
                    \end{enumerate}
                \end{problem}
                \begin{proof}[Solution]
                    If $k[x]/\langle f\rangle$ is a field,
                    then it is an integral domain. If $f$ is
                    irreducible, then $\langle f\rangle$ is
                    maximal and thus $k[x]/\langle f\rangle$ is
                    a field. Finally, if $k[x]/\langle f\rangle$ is
                    an integral domain, then $\langle f\rangle$ is
                    prime. But if $\langle f\rangle$ is prime,
                    then it is maximal. And if $\langle f\rangle$
                    is maximal, then $f$ is irreducible. 
                \end{proof}
                \begin{problem}
                    Show every prime ideal is radical.
                \end{problem}
                \begin{proof}[Solution]
                    Let $I$ be a prime ideal. Then if $fg\in I$,
                    either $f\in I$ or $g\in I$. Suppose $f^n \in I$
                    for some $f\in R$. Then $f^{n-1}f \in R$.
                    But then either $f^{n-1} \in I$ or $f\in I$.
                    If $f\in I$, we are done. If not, by induction
                    $f^{n-k} \in I$ and we obtain $f\in I$.
                \end{proof}
                \begin{problem}
                    Show that any Noetherian
                    Topological Space $X$ is compact.
                \end{problem}
                \begin{proof}[Solution]
                    If $X$ is Noetherian, then every ascending
                    chain terminates. Suppose $X$ is not compact.
                    Then there is an open cover $\Delta$ with no
                    finite subcover. Let $\mathcal{O}_1$ be a finite
                    subcover. Then
                    $\cup_{\mathcal{U}\in \mathcal{O}_1}\mathcal{U}$
                    is not all of $X$, otherwise $X$ would be compact.
                    Thus there is an open subcover $\mathcal{O}_2$
                    such that $\mathcal{O}_1 \subset \mathcal{O}_2$.
                    Inductively, we have a sequence
                    $\mathcal{O}_n\subset \mathcal{O}_{n+1}$. Let
                    $A_{n}=\cup_{k=1}^{n}\cup_{\mathcal{U}\in \mathcal{O}_k}\mathcal{U}$.
                    Then $A_{n}\subset A_{n+1}$.
                    But by the Noetherian property,
                    this chain must stabilize.
                    But then there is an $N\in \mathbb{N}$
                    such that $\mathcal{O}_{N+1}=\mathcal{O}_N$,
                    a contradiction as we said $X$ is not compact.
                    Therefore, etc.
                \end{proof}
                This proof subtly requires the axiom of choice in the
                construction of such $\mathcal{O}'s$.
        \subsection{Notes on Varieties}
            \subsubsection{Affine Varieties}
                Let $k$ denote an algebraically closed field.
                $\textbf{A}_{k}^n$ is the affine $k-$space in
                $n$ dimensions. An element $a=(a_1,\hdots, a_n)$
                is called a point, and $a_i$ is called a coordinate.
                \begin{definition}
                    The zero set of a set of polynomials
                    $T=\{f_{1},\hdots,f_{s}\}$ is the set
                    $Z(T)%
                     =\{p\in\textbf{A}_{k}^{n}|f_{i}(p)=0,%
                        i=1,\hdots,s\}$.
                \end{definition}
                The set of polynomials in $n$ variables over
                $\textbf{A}_{k}^{n}$ is denoted $A$.
                \begin{definition}
                    A subset $Y\subset\textbf{A}_{k}^{n}$ is an
                    algebraic set if there exists a subset
                    $T\subset{A}$ such that $Z(T)=Y$.
                \end{definition}
                \begin{theorem}
                    The union of two algebraic
                    sets is algebraic.
                \end{theorem}
                \begin{theorem}
                    The intersection of two algebraic
                    sets is algebraic.
                \end{theorem}
                \begin{definition}
                    The Zariski topology $\mathcal{Z}$ on
                    $\textbf{A}_{k}^{n}$ is the set of compliments
                    of algebraic sets. That is,
                    algebraic sets are closed.
                \end{definition}
                \begin{definition}
                    A non-empty subset $Y$ of a topological space
                    $X$ is irreducible if it cannot be expressed
                    as the union $Y={Y_{1}}\cup{Y_{2}}$ of
                    two proper subsets, each on of which is
                    closed in $Y$.
                \end{definition}
                \begin{definition}
                    An affine algebraic variety is an irreducible
                    subset of $\textbf{A}_{k}^{n}$ with respect
                    to the induced topology.
                \end{definition}
                \begin{definition}
                    An open subset of an affine variety
                    is called a quasi-affine variety.
                \end{definition}
                If $Y\subset\textbf{A}_{k}^{n}$,
                $I(Y)=\{f\in A:\forall_{p\in Y},f(p)=0\}$.
                \begin{theorem}
                    \
                    \begin{enumerate}
                        \item If $T_1\subset T_2$,
                              the $Z(T_2)\subset{Z}(T_1)$
                        \item If
                              $Y_{1}\subset{Y_{2}}\subset%
                              \textbf{A}_{k}^{n}$,
                              then $I(Y_{2})\subset{I}(Y_{1})$
                        \item $I(Y_{1}\cup{Y_{2}})%
                               =I(Y_{1})\cap{I}(Y_{2})$
                        \item If $a\subset A$,
                              then $I(Z(a))=\sqrt{a}$
                              (The radical of $a$)
                        \item If $Y\subset\textbf{A}_{k}^{n}$,
                              then $Z(I(Y))=\overline{Y}$
                              (The closure of $Y$)
                    \end{enumerate}
                \end{theorem}
                \begin{theorem}[Hilbert's Nullstellensatz]
                    If $k$ is an algebraically closed field,
                    $a\subset{A}=k[x_{1},\hdots,x_{n}]$
                    is an ideal, and if $f\in{A}$ is a polynomial
                    which vanishes on $Z(a)$, then there is an
                    $r\in\mathbb{N}$ such that $f^{r}\in{a}$.
                \end{theorem}
                \begin{definition}
                    The affine coordinate ring of an affine
                    algebraic set $Y\subset\textbf{A}_{k}^{n}$
                    is $A/I(Y)$.
                \end{definition}
                \begin{definition}
                    A topological space $X$ is called Noetherian
                    if it satisfies the descending chain condition
                    for closed subsets.
                \end{definition}
                \begin{theorem}
                    A Noetherian Topological Space is compact.
                \end{theorem}
                \begin{definition}
                    If $A$ is a ring, then height of a prime
                    ideal $p$ is the supremum of all integers $n$
                    such that there is a chain
                    $p_{0}\subset\hdots\subset{p_{n}}=p$
                    of distinct prime ideals.
                \end{definition}
                \begin{definition}
                    The Krull dimension of a ring $A$ is the
                    supremum of the height of all ideals.
                \end{definition}
                \begin{theorem}[Krull's Hauptidealsatz]
                    If $A$ is a Noetherian Ring, and $f\in A$
                    has neither a zero divisor nor a unit,
                    then every minimal prime ideal $p$
                    containing $f$ has height $1$.
                \end{theorem}
                \begin{theorem}
                    The dimension of $\textbf{A}_{k}^{n}$ is $n$.
                \end{theorem}
            \subsubsection{Projective Varieties}
                \begin{definition}
                    A subset $Y$ of $P^n$ is an algebraic
                    set if there is a set $T$ of homogeneous
                    elements of $S$ such that $Y=Z(T)$.
                \end{definition}
                \begin{definition}
                    The Zariski Topology on $P^n$ is defined
                    as the complements of algebraic sets.
                    That is, algebraic sets are closed.
                \end{definition}
                \begin{definition}
                    A projective algebraic variety is an
                    irreducible algebraic set in $P^{n}$.
                \end{definition}
            \subsubsection{More Notes on Projective Varieties}
                \begin{definition}
                    The projective $n-$space over $\mathbb{A}$,
                    denoted $\mathbb{P}^{n}$, is the set of all
                    one-dimensional linear subspaces of the vector
                    space $\mathbb{A}^{n+1}$.
                \end{definition}
                Equivalently, it is the set of all lines in $\mathbb{A}^{n+1}$
                through the origin.
                \begin{definition}
                    The projective $n$ space $\mathbb{P}^{n}$ over $k$
                    is the set of all equivalence classes
                    $\mathbb{A}^{n+1}/\{0\}$, where
                    $(a_{1},\hdots,a_{n})\sim(b_{1},\hdots,b_{n})$
                    if and only if there is a
                    $\lambda\in\mathbb{A}\setminus\{0\}$
                    such that $b_{i}=\lambda{a_{i}}$.
                \end{definition}
                Elements of $\mathbb{P}^{n}$ are called points.
                \begin{definition}
                    A homogenous polynomial of degree $d$
                    is a polynomial $f$ such that
                    $f(\lambda a_1,\hdots,\lambda a_n)%
                     =\lambda^d f(a_1,\hdots, a_n)$.
                \end{definition}
                \begin{theorem}
                    If $I\subset k[x_1,\hdots ,x_n]$ is an ideal,
                    then the following are equivalent:
                    \begin{enumerate}
                        \item $I$ can be generated by
                              homogeneous polynomials.
                        \item For every $f\in I$, the degree
                              $d$ part of $f$ in contained in $I$
                    \end{enumerate}
                \end{theorem}
                \begin{definition}
                    If $I\subset k[x_1,\hdots ,x_n]$ is a
                    homogeneous ideal, then
                    $\mathbf{V}(I)%
                     =\{(a_1:\hdots:a_{n})\in\mathbb{P}^{n}:%
                     f(a_{1},\hdots,a_{n})=0,f\in I\}$.
                \end{definition}
                \begin{definition}
                    An algebraic subset of $\mathbb{P}^{n}$ is a
                    set of the form $\mathbf{V}(I)$.
                    These are called projective algebraic sets.
                \end{definition}
                \begin{theorem}
                    Every projective algebraic set can be
                    written as the zero set of finitely many
                    homogeneous polynomials of the same degree.
                \end{theorem}
                \begin{definition}
                    The projective close of and algebraic set
                    $X\subset\mathbb{A}^n$ is the Zariski closure
                    in $\mathbb{P}^{n}$ under the mapping
                    $\mathbb{A}^{n}\rightarrow\mathbb{P}^n$
                    by $(x_{1},\hdots,x_{n})\mapsto(1:x_1,\hdots, x_n)$.
                \end{definition}
                \begin{theorem}
                    If $f$ is the sum of forms $f=\sum_{d}f^{(d)}$,
                    if $P\in \mathbb{P}^n$ and $f(x_1,\hdots, x_n)=0$
                    for every choice of homogeneous coordinates,
                    then for each $d$, $f^{(d)}(x_1,\hdots, x_n)=0$.
                \end{theorem}
                \begin{definition}
                    If $F\in \mathbb{A}[x_1,\hdots, x_n]$ is homogeneous
                    of degree $d$, then its de-homogenization is the
                    polynomial $f(x_1,\hdots, x_n)=F(1,x_1,\hdots, x_n)$.
                \end{definition}
                \begin{theorem}
                    Let $X\subset \mathbb{A}^n$ be an affine
                    algebraic set, $\overline{X}$ the projective closure. Then
                    $\mathbb{I}(\overline{X})\subset\mathbb{A}[x_1,\hdots,x_n]$
                    is generated by the homogenization of all
                    elements of $\mathbb{I}(X)$.
                \end{theorem}
                \begin{theorem}
                    An algebraic set $X$ is irreducible
                    if and only if the ideal $\mathbb{I}(X)$ is prime.
                \end{theorem}
                \begin{definition}
                    An affine algebraic set $X\subset \mathbb{A}^{n+1}$
                    is called a cone if it is not empty, and if for all
                    $\lambda\in{k}$,
                    $(x_1,\hdots, x_n)%
                     \in{X}\Rightarrow(\lambda{x_{1}},\hdots,\lambda{x_{n}})%
                     \in{X}$.
                \end{definition}
                \begin{theorem}[The Projective Nullstellensatz]
                    \
                    \begin{enumerate}
                        \item If $X_1\subset X_2$ are algebraic
                              set in $\mathbb{P}^{n}$,
                              then $I(X_{2})\subset{I}(X_{1})$.
                        \item For any algebraic set
                              $X\subset\mathbb{P}^{n}$, we have
                              $\mathbf{V}(I(X))=X$.
                        \item For any homogeneous ideal
                              $I\subset{k}[x_{1},\hdots,x_{n}]$ such
                              that $\mathbf{V}(I)\ne\emptyset$,
                              we have
                              $\mathbb{I}(\mathbf{V}(I))=\sqrt{I}$.
                    \end{enumerate}
                \end{theorem}
    \addtocontents{toc}{\protect\newpage}
    \clearpage

    % \setcounter{endpage}{\thepage}
    % \pagenumbering{gobble}
    % \book{Topology}
    %     \renewcommand{\PATH}{\TOPPATH/Topology/}
    %     \pagenumbering{arabic}
    %     \setcounter{page}{\value{endpage}}
    %     \part{Point-Set Topology}
    %         \documentclass[crop=false,class=book,oneside]{standalone}
%----------------------------Preamble-------------------------------%
%---------------------------Packages----------------------------%
\usepackage{geometry}
\geometry{b5paper, margin=1.0in}
\usepackage[T1]{fontenc}
\usepackage{graphicx, float}            % Graphics/Images.
\usepackage{natbib}                     % For bibliographies.
\bibliographystyle{agsm}                % Bibliography style.
\usepackage[french, english]{babel}     % Language typesetting.
\usepackage[dvipsnames]{xcolor}         % Color names.
\usepackage{listings}                   % Verbatim-Like Tools.
\usepackage{mathtools, esint, mathrsfs} % amsmath and integrals.
\usepackage{amsthm, amsfonts, amssymb}  % Fonts and theorems.
\usepackage{tcolorbox}                  % Frames around theorems.
\usepackage{upgreek}                    % Non-Italic Greek.
\usepackage{fmtcount, etoolbox}         % For the \book{} command.
\usepackage[newparttoc]{titlesec}       % Formatting chapter, etc.
\usepackage{titletoc}                   % Allows \book in toc.
\usepackage[nottoc]{tocbibind}          % Bibliography in toc.
\usepackage[titles]{tocloft}            % ToC formatting.
\usepackage{pgfplots, tikz}             % Drawing/graphing tools.
\usepackage{imakeidx}                   % Used for index.
\usetikzlibrary{
    calc,                   % Calculating right angles and more.
    angles,                 % Drawing angles within triangles.
    arrows.meta,            % Latex and Stealth arrows.
    quotes,                 % Adding labels to angles.
    positioning,            % Relative positioning of nodes.
    decorations.markings,   % Adding arrows in the middle of a line.
    patterns,
    arrows
}                                       % Libraries for tikz.
\pgfplotsset{compat=1.9}                % Version of pgfplots.
\usepackage[font=scriptsize,
            labelformat=simple,
            labelsep=colon]{subcaption} % Subfigure captions.
\usepackage[font={scriptsize},
            hypcap=true,
            labelsep=colon]{caption}    % Figure captions.
\usepackage[pdftex,
            pdfauthor={Ryan Maguire},
            pdftitle={Mathematics and Physics},
            pdfsubject={Mathematics, Physics, Science},
            pdfkeywords={Mathematics, Physics, Computer Science, Biology},
            pdfproducer={LaTeX},
            pdfcreator={pdflatex}]{hyperref}
\hypersetup{
    colorlinks=true,
    linkcolor=blue,
    filecolor=magenta,
    urlcolor=Cerulean,
    citecolor=SkyBlue
}                           % Colors for hyperref.
\usepackage[toc,acronym,nogroupskip,nopostdot]{glossaries}
\usepackage{glossary-mcols}
%------------------------Theorem Styles-------------------------%
\theoremstyle{plain}
\newtheorem{theorem}{Theorem}[section]

% Define theorem style for default spacing and normal font.
\newtheoremstyle{normal}
    {\topsep}               % Amount of space above the theorem.
    {\topsep}               % Amount of space below the theorem.
    {}                      % Font used for body of theorem.
    {}                      % Measure of space to indent.
    {\bfseries}             % Font of the header of the theorem.
    {}                      % Punctuation between head and body.
    {.5em}                  % Space after theorem head.
    {}

% Italic header environment.
\newtheoremstyle{thmit}{\topsep}{\topsep}{}{}{\itshape}{}{0.5em}{}

% Define environments with italic headers.
\theoremstyle{thmit}
\newtheorem*{solution}{Solution}

% Define default environments.
\theoremstyle{normal}
\newtheorem{example}{Example}[section]
\newtheorem{definition}{Definition}[section]
\newtheorem{problem}{Problem}[section]

% Define framed environment.
\tcbuselibrary{most}
\newtcbtheorem[use counter*=theorem]{ftheorem}{Theorem}{%
    before=\par\vspace{2ex},
    boxsep=0.5\topsep,
    after=\par\vspace{2ex},
    colback=green!5,
    colframe=green!35!black,
    fonttitle=\bfseries\upshape%
}{thm}

\newtcbtheorem[auto counter, number within=section]{faxiom}{Axiom}{%
    before=\par\vspace{2ex},
    boxsep=0.5\topsep,
    after=\par\vspace{2ex},
    colback=Apricot!5,
    colframe=Apricot!35!black,
    fonttitle=\bfseries\upshape%
}{ax}

\newtcbtheorem[use counter*=definition]{fdefinition}{Definition}{%
    before=\par\vspace{2ex},
    boxsep=0.5\topsep,
    after=\par\vspace{2ex},
    colback=blue!5!white,
    colframe=blue!75!black,
    fonttitle=\bfseries\upshape%
}{def}

\newtcbtheorem[use counter*=example]{fexample}{Example}{%
    before=\par\vspace{2ex},
    boxsep=0.5\topsep,
    after=\par\vspace{2ex},
    colback=red!5!white,
    colframe=red!75!black,
    fonttitle=\bfseries\upshape%
}{ex}

\newtcbtheorem[auto counter, number within=section]{fnotation}{Notation}{%
    before=\par\vspace{2ex},
    boxsep=0.5\topsep,
    after=\par\vspace{2ex},
    colback=SeaGreen!5!white,
    colframe=SeaGreen!75!black,
    fonttitle=\bfseries\upshape%
}{not}

\newtcbtheorem[use counter*=remark]{fremark}{Remark}{%
    fonttitle=\bfseries\upshape,
    colback=Goldenrod!5!white,
    colframe=Goldenrod!75!black}{ex}

\newenvironment{bproof}{\textit{Proof.}}{\hfill$\square$}
\tcolorboxenvironment{bproof}{%
    blanker,
    breakable,
    left=3mm,
    before skip=5pt,
    after skip=10pt,
    borderline west={0.6mm}{0pt}{green!80!black}
}

\AtEndEnvironment{lexample}{$\hfill\textcolor{red}{\blacksquare}$}
\newtcbtheorem[use counter*=example]{lexample}{Example}{%
    empty,
    title={Example~\theexample},
    boxed title style={%
        empty,
        size=minimal,
        toprule=2pt,
        top=0.5\topsep,
    },
    coltitle=red,
    fonttitle=\bfseries,
    parbox=false,
    boxsep=0pt,
    before=\par\vspace{2ex},
    left=0pt,
    right=0pt,
    top=3ex,
    bottom=1ex,
    before=\par\vspace{2ex},
    after=\par\vspace{2ex},
    breakable,
    pad at break*=0mm,
    vfill before first,
    overlay unbroken={%
        \draw[red, line width=2pt]
            ([yshift=-1.2ex]title.south-|frame.west) to
            ([yshift=-1.2ex]title.south-|frame.east);
        },
    overlay first={%
        \draw[red, line width=2pt]
            ([yshift=-1.2ex]title.south-|frame.west) to
            ([yshift=-1.2ex]title.south-|frame.east);
    },
}{ex}

\AtEndEnvironment{ldefinition}{$\hfill\textcolor{Blue}{\blacksquare}$}
\newtcbtheorem[use counter*=definition]{ldefinition}{Definition}{%
    empty,
    title={Definition~\thedefinition:~{#1}},
    boxed title style={%
        empty,
        size=minimal,
        toprule=2pt,
        top=0.5\topsep,
    },
    coltitle=Blue,
    fonttitle=\bfseries,
    parbox=false,
    boxsep=0pt,
    before=\par\vspace{2ex},
    left=0pt,
    right=0pt,
    top=3ex,
    bottom=0pt,
    before=\par\vspace{2ex},
    after=\par\vspace{1ex},
    breakable,
    pad at break*=0mm,
    vfill before first,
    overlay unbroken={%
        \draw[Blue, line width=2pt]
            ([yshift=-1.2ex]title.south-|frame.west) to
            ([yshift=-1.2ex]title.south-|frame.east);
        },
    overlay first={%
        \draw[Blue, line width=2pt]
            ([yshift=-1.2ex]title.south-|frame.west) to
            ([yshift=-1.2ex]title.south-|frame.east);
    },
}{def}

\AtEndEnvironment{ltheorem}{$\hfill\textcolor{Green}{\blacksquare}$}
\newtcbtheorem[use counter*=theorem]{ltheorem}{Theorem}{%
    empty,
    title={Theorem~\thetheorem:~{#1}},
    boxed title style={%
        empty,
        size=minimal,
        toprule=2pt,
        top=0.5\topsep,
    },
    coltitle=Green,
    fonttitle=\bfseries,
    parbox=false,
    boxsep=0pt,
    before=\par\vspace{2ex},
    left=0pt,
    right=0pt,
    top=3ex,
    bottom=-1.5ex,
    breakable,
    pad at break*=0mm,
    vfill before first,
    overlay unbroken={%
        \draw[Green, line width=2pt]
            ([yshift=-1.2ex]title.south-|frame.west) to
            ([yshift=-1.2ex]title.south-|frame.east);},
    overlay first={%
        \draw[Green, line width=2pt]
            ([yshift=-1.2ex]title.south-|frame.west) to
            ([yshift=-1.2ex]title.south-|frame.east);
    }
}{thm}

%--------------------Declared Math Operators--------------------%
\DeclareMathOperator{\adjoint}{adj}         % Adjoint.
\DeclareMathOperator{\Card}{Card}           % Cardinality.
\DeclareMathOperator{\curl}{curl}           % Curl.
\DeclareMathOperator{\diam}{diam}           % Diameter.
\DeclareMathOperator{\dist}{dist}           % Distance.
\DeclareMathOperator{\Div}{div}             % Divergence.
\DeclareMathOperator{\Erf}{Erf}             % Error Function.
\DeclareMathOperator{\Erfc}{Erfc}           % Complementary Error Function.
\DeclareMathOperator{\Ext}{Ext}             % Exterior.
\DeclareMathOperator{\GCD}{GCD}             % Greatest common denominator.
\DeclareMathOperator{\grad}{grad}           % Gradient
\DeclareMathOperator{\Ima}{Im}              % Image.
\DeclareMathOperator{\Int}{Int}             % Interior.
\DeclareMathOperator{\LC}{LC}               % Leading coefficient.
\DeclareMathOperator{\LCM}{LCM}             % Least common multiple.
\DeclareMathOperator{\LM}{LM}               % Leading monomial.
\DeclareMathOperator{\LT}{LT}               % Leading term.
\DeclareMathOperator{\Mod}{mod}             % Modulus.
\DeclareMathOperator{\Mon}{Mon}             % Monomial.
\DeclareMathOperator{\multideg}{mutlideg}   % Multi-Degree (Graphs).
\DeclareMathOperator{\nul}{nul}             % Null space of operator.
\DeclareMathOperator{\Ord}{Ord}             % Ordinal of ordered set.
\DeclareMathOperator{\Prin}{Prin}           % Principal value.
\DeclareMathOperator{\proj}{proj}           % Projection.
\DeclareMathOperator{\Refl}{Refl}           % Reflection operator.
\DeclareMathOperator{\rk}{rk}               % Rank of operator.
\DeclareMathOperator{\sgn}{sgn}             % Sign of a number.
\DeclareMathOperator{\sinc}{sinc}           % Sinc function.
\DeclareMathOperator{\Span}{Span}           % Span of a set.
\DeclareMathOperator{\Spec}{Spec}           % Spectrum.
\DeclareMathOperator{\supp}{supp}           % Support
\DeclareMathOperator{\Tr}{Tr}               % Trace of matrix.
%--------------------Declared Math Symbols--------------------%
\DeclareMathSymbol{\minus}{\mathbin}{AMSa}{"39} % Unary minus sign.
%------------------------New Commands---------------------------%
\DeclarePairedDelimiter\norm{\lVert}{\rVert}
\DeclarePairedDelimiter\ceil{\lceil}{\rceil}
\DeclarePairedDelimiter\floor{\lfloor}{\rfloor}
\newcommand*\diff{\mathop{}\!\mathrm{d}}
\newcommand*\Diff[1]{\mathop{}\!\mathrm{d^#1}}
\renewcommand*{\glstextformat}[1]{\textcolor{RoyalBlue}{#1}}
\renewcommand{\glsnamefont}[1]{\textbf{#1}}
\renewcommand\labelitemii{$\circ$}
\renewcommand\thesubfigure{%
    \arabic{chapter}.\arabic{figure}.\arabic{subfigure}}
\addto\captionsenglish{\renewcommand{\figurename}{Fig.}}
\numberwithin{equation}{section}

\renewcommand{\vector}[1]{\boldsymbol{\mathrm{#1}}}

\newcommand{\uvector}[1]{\boldsymbol{\hat{\mathrm{#1}}}}
\newcommand{\topspace}[2][]{(#2,\tau_{#1})}
\newcommand{\measurespace}[2][]{(#2,\varSigma_{#1},\mu_{#1})}
\newcommand{\measurablespace}[2][]{(#2,\varSigma_{#1})}
\newcommand{\manifold}[2][]{(#2,\tau_{#1},\mathcal{A}_{#1})}
\newcommand{\tanspace}[2]{T_{#1}{#2}}
\newcommand{\cotanspace}[2]{T_{#1}^{*}{#2}}
\newcommand{\Ckspace}[3][\mathbb{R}]{C^{#2}(#3,#1)}
\newcommand{\funcspace}[2][\mathbb{R}]{\mathcal{F}(#2,#1)}
\newcommand{\smoothvecf}[1]{\mathfrak{X}(#1)}
\newcommand{\smoothonef}[1]{\mathfrak{X}^{*}(#1)}
\newcommand{\bracket}[2]{[#1,#2]}

%------------------------Book Command---------------------------%
\makeatletter
\renewcommand\@pnumwidth{1cm}
\newcounter{book}
\renewcommand\thebook{\@Roman\c@book}
\newcommand\book{%
    \if@openright
        \cleardoublepage
    \else
        \clearpage
    \fi
    \thispagestyle{plain}%
    \if@twocolumn
        \onecolumn
        \@tempswatrue
    \else
        \@tempswafalse
    \fi
    \null\vfil
    \secdef\@book\@sbook
}
\def\@book[#1]#2{%
    \refstepcounter{book}
    \addcontentsline{toc}{book}{\bookname\ \thebook:\hspace{1em}#1}
    \markboth{}{}
    {\centering
     \interlinepenalty\@M
     \normalfont
     \huge\bfseries\bookname\nobreakspace\thebook
     \par
     \vskip 20\p@
     \Huge\bfseries#2\par}%
    \@endbook}
\def\@sbook#1{%
    {\centering
     \interlinepenalty \@M
     \normalfont
     \Huge\bfseries#1\par}%
    \@endbook}
\def\@endbook{
    \vfil\newpage
        \if@twoside
            \if@openright
                \null
                \thispagestyle{empty}%
                \newpage
            \fi
        \fi
        \if@tempswa
            \twocolumn
        \fi
}
\newcommand*\l@book[2]{%
    \ifnum\c@tocdepth >-3\relax
        \addpenalty{-\@highpenalty}%
        \addvspace{2.25em\@plus\p@}%
        \setlength\@tempdima{3em}%
        \begingroup
            \parindent\z@\rightskip\@pnumwidth
            \parfillskip -\@pnumwidth
            {
                \leavevmode
                \Large\bfseries#1\hfill\hb@xt@\@pnumwidth{\hss#2}
            }
            \par
            \nobreak
            \global\@nobreaktrue
            \everypar{\global\@nobreakfalse\everypar{}}%
        \endgroup
    \fi}
\newcommand\bookname{Book}
\renewcommand{\thebook}{\texorpdfstring{\Numberstring{book}}{book}}
\providecommand*{\toclevel@book}{-2}
\makeatother
\titleformat{\part}[display]
    {\Large\bfseries}
    {\partname\nobreakspace\thepart}
    {0mm}
    {\Huge\bfseries}
\titlecontents{part}[0pt]
    {\large\bfseries}
    {\partname\ \thecontentslabel: \quad}
    {}
    {\hfill\contentspage}
\titlecontents{chapter}[0pt]
    {\bfseries}
    {\chaptername\ \thecontentslabel:\quad}
    {}
    {\hfill\contentspage}
\newglossarystyle{longpara}{%
    \setglossarystyle{long}%
    \renewenvironment{theglossary}{%
        \begin{longtable}[l]{{p{0.25\hsize}p{0.65\hsize}}}
    }{\end{longtable}}%
    \renewcommand{\glossentry}[2]{%
        \glstarget{##1}{\glossentryname{##1}}%
        &\glossentrydesc{##1}{~##2.}
        \tabularnewline%
        \tabularnewline
    }%
}
\newglossary[not-glg]{notation}{not-gls}{not-glo}{Notation}
\newcommand*{\newnotation}[4][]{%
    \newglossaryentry{#2}{type=notation, name={\textbf{#3}, },
                          text={#4}, description={#4},#1}%
}
%--------------------------LENGTHS------------------------------%
% Spacings for the Table of Contents.
\addtolength{\cftsecnumwidth}{1ex}
\addtolength{\cftsubsecindent}{1ex}
\addtolength{\cftsubsecnumwidth}{1ex}
\addtolength{\cftfignumwidth}{1ex}
\addtolength{\cfttabnumwidth}{1ex}

% Indent and paragraph spacing.
\setlength{\parindent}{0em}
\setlength{\parskip}{0em}
%----------------------------GLOSSARY-------------------------------%
\makeglossaries
\loadglsentries{../../glossary}
\loadglsentries{../../acronym}
%--------------------------Main Document----------------------------%
\begin{document}
    \ifx\ifmathcourses\undefined
        \pagenumbering{roman}
        \title{Point-Set Topology}
        \author{Ryan Maguire}
        \date{\vspace{-5ex}}
        \maketitle
        \tableofcontents
        \clearpage
        \chapter*{Point-Set Topology}
        \addcontentsline{toc}{chapter}{Point-Set Topology}
        \markboth{}{POINT-SET TOPOLOGY}
        \vspace{10ex}
        \setcounter{chapter}{1}
        \pagenumbering{arabic}
    \else
        \chapter{Point-Set Topology}
    \fi
    \subsection{Old Notes}
        \begin{definition}
            A topology $\tau$ on a set $X$ is a subset
            of $\mathcal{P}(X)$ such that:
            \begin{enumerate}
                \item $\emptyset\subset\tau$ and $X\subset\tau$
                \item For any subset $\mathcal{O}\subset\tau$, the
                      union of all elements of $\mathcal{O}$ is an
                      element of $\tau$.
                \item For any finite subset of $\tau$, the
                      intersection is also an element of $\tau$.
            \end{enumerate}
        \end{definition}
        \begin{definition}
            A topological space, denoted $(X,\tau)$, is a set
            $X$ and a topology $\tau$ on $X$.
        \end{definition}
        Elements of a topological space are called points.
        Elements of the topology are called open subset of $X$.
        A neighborhood of a point $x$ is a set $V$ that contains
        an open subste $U$ such that $x\in{U}$. An open neighborhood
        of $x$ is an open set $U$ such that $x\in{U}$.
        \begin{example}
            The chaotic topology on a set
            $X$ is the set $\tau=\{\emptyset,X\}$.
        \end{example}
        \begin{example}
            The discrete topology on a set $X$ is
            $\tau=\mathcal{P}(X)$.
        \end{example}
        \begin{example}
            The Sierpinski topology
            on $\{0,1\}$ is the set
            $\{\emptyset,\{0\},\{0,1\}\}$.
        \end{example}
        \begin{theorem}
            If $T_{\omega}$ is a set of topologies on
            a topological space $X$, then
            $\bigcap{T_{\omega}}$ is a topology on $X$.
        \end{theorem}
        However, the union of topologies may not be a
        topology. A topology $\tau_{1}$ is set to
        be finer than a topology $\tau_{0}$ if
        $\tau_{0}\subset\tau_{1}$. An accumulation point
        of a set $A$ is a point $x$ such that, for
        all open neighborhoods $U$ of $A$,
        $U\cap{A}\ne\emptyset$.
        \begin{theorem}[Bolzano-Weierstrass Theorem]
            If $X$ is a bounded, infinite subset of
            $\mathbb{R}$, then $X$ has at least
            one accumulation point.
        \end{theorem}
        \begin{definition}
            The Euclidean topology on
            $\mathbb{R}$ is the set of
            all open sets in the sense that
            $U$ is open if, for all $x\in{U}$,
            there is an $\varepsilon>0$ such
            that $(x-\varepsilon,x+\varepsilon)\subset{U}$.
        \end{definition}
        \begin{definition}
            A closed subset of a topological space
            $(X,\tau)$ is a set $A$ such that
            $A^{C}\in\tau$.
        \end{definition}
        \begin{theorem}
            The intersection of an arbitrary collection of
            closed sets is closed. The union of finitely
            many closed sets is closed.
        \end{theorem}
        \begin{proof}
            Apply DeMorgan's theorem to the properties
            of a topological space $\tau$.
        \end{proof}
        \begin{definition}
            The closure of a set $A$,
            denoted $\overline{A}$, is the
            intersection of all closed sets
            containing $A$.
        \end{definition}
        \begin{theorem}
            If $A$ is a set in a topological space
            $(X,\tau)$, then $A\subset\overline{A}$.
        \end{theorem}
        There's also something called the derived
        set of a set $A$. The interior of $A$
        is the union of all open subset of $A$.
        The boundary of $A$ is the set difference
        of the closure of $A$ and the interior of
        $A$.
        \begin{theorem}
            If $A$ is a set, then
            the interior of $A$ is equal to
            $(\overline{A^{C}})^{C}$
        \end{theorem}
        \begin{definition}
            A dense subset of a topological
            space $(X,\tau)$ is a set $A$
            such that $\overline{A}=X$.
        \end{definition}
        \begin{definition}
            The neighborhood system of a point
            $x$ in a topological space $(X,\tau)$
            is the set of all neighborhoods of
            $x$.
        \end{definition}
        \begin{definition}
            A sequence in a topological space
            $a_{n}$ converges to a point $a$ if,
            for all open neighborhoods $U$ of $a$,
            there is an $N\in\mathbb{N}$ such that,
            for all $n>N$, $a_{n}\in{U}$.
        \end{definition}
        Limits of sequences in topological spaces are NOT
        necessarily unique. This is different from convergence
        in $\mathbb{R}$, where convergence is always unique.
        \begin{definition}
            The relative topology of a
            topological space $(X,\tau)$ with
            respect to a subset $A\subset{X}$
            is $\tau_{A}=\{A\cap{U}:U\in\tau\}$
        \end{definition}
        \begin{theorem}
            If $(X,\tau)$ is a topological space and
            $A\subset{X}$, then
            $(A,\tau_{A})$ is a topological space.
        \end{theorem}
        $(A,\tau_{A})$ is called a subspace of
        $(X,\tau)$.
        \begin{definition}
            A basis of a topological space
            $(X,\tau)$ is a subset $B$ of
            $\tau$ such that every element
            of $\tau$ is the union of some of the
            elements of $B$.
        \end{definition}
        \begin{theorem}
            A subset $B\subset\tau$ is a basis
            for $\tau$ if and only if for all
            $U\in\tau$ and all $x\in{U}$, there is
            a $V\in{B}$ such that
            $x\in{V}\subset{U}$.
        \end{theorem}
        \begin{theorem}
            If $B$ is a basis of $\tau$, then
            $U$ is open if and only if for all
            $x\in{U}$ there is a $V\in{B}$ such that
            $x\in{V}\subset{U}$.
        \end{theorem}
        \begin{theorem}
            $\mathbb{R}$ has a countable basis.
        \end{theorem}
        \begin{proof}
            For the set of open intervals
            $(p,q)$, where $p$ and $q$ are rational
            numbers, forms a basis for the standard
            topology on $\mathbb{R}$. Moreover, this
            is countable.
        \end{proof}
        \begin{definition}
            If $(X,\tau)$ is a topological space
            and $S\subset\tau$, then $S$ is a subbase
            if a finite intersection of elements of $S$
            forms a base of $\tau$.
        \end{definition}
        \begin{definition}
            A local base for a point
            $x$ in a topological space $(X,\tau)$
            is a set of open neighborhods $B_{x}$ of
            $x$ such that for all open neighborhoods $G$
            of $x$, there is a $G_{x}\in{B_{x}}$ such that
            $x\in{G_{x}}\subset{G}$.
        \end{definition}
        \begin{theorem}
            If $(X,\tau)$ is a topological space, $x\in{X}$,
            and if $B$ is a base for $\tau$, then
            the set of elements $G_{x}$ in $B$ such that
            $x\in{G_{x}}$ is a local base for $x$.
        \end{theorem}
    \section{Set Theory}
        
\end{document}
    % \addtocontents{toc}{\protect\newpage}
    % \clearpage

    % \setcounter{endpage}{\thepage}
    % \pagenumbering{gobble}
    % \book{Analysis}
    %     \renewcommand{\PATH}{\TOPPATH/Analysis/}
    %     \pagenumbering{arabic}
    %     \setcounter{page}{\value{endpage}}
    %     \part{Measure Theory}
    %     \chapter{Real Analysis}
    \section{A Review of Real Analysis}
        \subsection{Completeness}
            One of the fundamental properties of $\mathbb{R}$ is
            that is is \textit{complete}. This property is
            fundamental to many theorems involved in a
            standard calculus or real analysis course. For
            example, the concepts of differentiation and
            convergence rely on completeness, and the
            intermediate value theorem may fail without it.
            On the other hand, $\mathbb{Q}$ is not complete. The
            rationals are, however, \textit{dense} in the reals.
            That is, elements of $\mathbb{R}$ can be
            approximated arbitrarily well by elements of
            $\mathbb{Q}$. $\mathbb{R}$ is also something
            called a \textit{field}. From algebra,
            a field is just a set with two operations
            (Usually called addition and multiplication)
            that are defined in such a way as to give rise
            to the usual notions of addition, subtraction,
            multiplication, and non-zero division,
            and to give them the basic properties of
            associativity, commutativity,
            and the distributive law that is found in
            arithmetic. $\mathbb{Q}$ is also a field.
            Moreover, $\mathbb{R}$ and $\mathbb{Q}$ are
            \textit{ordered fields} with respect to
            their standard ordering. What makes
            $\mathbb{R}$ special is that it is a
            complete ordered field. In fact, $\mathbb{R}$
            is the \textit{only} complete ordered field
            (Up to isomorphism). Completeness in
            $\mathbb{R}$ can be stated by fact that the real
            numbers have the least upper bound property.
            \begin{definition}
                A bounded above subset of $\mathbb{R}$
                is a nonempty subset $S\subseteq{\mathbb{R}}$
                such that there exists an $M\in\mathbb{R}$
                such that for all $x\in{S}$, $x\leq{M}$.
            \end{definition}
            \begin{definition}
                An upper bound of a bounded above
                subset $S\subseteq\mathbb{R}$ is a real
                number $M\in\mathbb{R}$
                such that for all $x\in{S}$, $x\leq{M}$.
            \end{definition}
            If $S\subseteq\mathbb{R}$ is bounded above,
            then there exists infinitely many bounds.
            Completeness says that every bounded above subset
            has a smallest upper bound.
            \begin{theorem}[Least Upper Bound Theorem]
                \label{thm:Func_Least_Upper_Bound_Theorem}
                If $S\subseteq{\mathbb{R}}$ is bounded above,
                then there exists an $s\in\mathbb{R}$,
                called the least upper bound, such that $s$
                and an upper bound and for all upper bounds
                $M$ of $S$, $s\leq{M}$.
            \end{theorem}
            The proof of
            Thm.~\ref{thm:Func_Least_Upper_Bound_Theorem}
            depends on how one defines the real numbers. This is
            often done via Dedekind cuts or equivalence
            classes of Cauchy sequences in $\mathbb{Q}$.
            \begin{theorem}
                There exist bounded above sets
                $S\subset\mathbb{Q}$ such that for all
                upper bounds $M$ there exists an
                $s\in\mathbb{Q}$ such that $s$ is an upper
                bound of $S$ and $s<M$.
            \end{theorem}
            \textit{Sketch of Proof.}
            For let $S=\{x\in\mathbb{Q}:x^{2}\leq{2}\}$.
            This set has no least upper bound. Loosely
            speaking this is because the
            least upper bound wants to be $\sqrt{2}$,
            but $\sqrt{2}$ is not a rational number. Thus
            there is no rational number to fill the gap.
            \par\hfill\par
            The least upper bound property gives rise
            to many theorems, many of which are equivalent
            to this axiom. Recall that a sequence is a
            function $x:\mathbb{N}\rightarrow{X}$. That is,
            a sequence is a function whose domain is the
            natural numbers, and whose image lies in some
            set $X$. A sequence of real numbers is thus a
            function $x:\mathbb{N}\rightarrow\mathbb{R}$,
            and a sequence of rational numbers is a function
            $x:\mathbb{N}\rightarrow\mathbb{Q}$.
            Often times sequences are denoted $x_{n}$,
            but also the image of $n$ is usually
            denoted $x(n)=x_{n}$ which may be a cause
            for confusion. That is, when we write $x_{n}$
            we often mean $x(n)$, so $x_{0}$, $x_{1}$,
            $x_{2}$ can be written as $x(0)$, $x(1)$,
            $x(2)$ but nobody does this. Similarly,
            we may mean $x_{n}=x$ since nobody writes
            a sequence as $x$. For consistency, we will.
            \begin{definition}
                A sequence in a set $X$ is a function
                $x:\mathbb{N}\rightarrow{X}$.
                We write the image of $n\in\mathbb{N}$
                as $x(n)=x_{n}$.
            \end{definition}
            The notion of \textit{convergence} of a sequence
            in $\mathbb{R}$ is defined as follows.
            \begin{definition}
                A convergent sequence in $S\subseteq\mathbb{R}$
                is a sequence $x:\mathbb{N}\rightarrow{S}$
                such that there exists an $a\in\mathbb{R}$
                such that $|a-x_{n}|\rightarrow{0}$ as
                $n\rightarrow\infty$. We write
                $x_{n}\rightarrow{a}$.
            \end{definition}
            \begin{definition}
                A limit of a sequence $x$
                in a subset $S\subseteq\mathbb{R}$ is an
                element $a\in\mathbb{R}$ such that
                $|a-x_{n}|\rightarrow{0}$.
            \end{definition}
            \begin{theorem}
                If $S\subseteq\mathbb{R}$ and $a$ and $b$ are
                limits of $x:\mathbb{N}\rightarrow{S}$,
                then $a=b$.
            \end{theorem}
            \begin{proof}
                Suppose not. Then $|a-b|>0$. But as $a$ is a
                limit of $x$, there is an $N_{1}\in\mathbb{N}$
                such that, for all $n>N_{1}$,
                $|a-x_{n}|<|a-b|/4$. But, as $b$ is a limit
                of $x$, there is an $N_{2}\in\mathbb{N}$
                such that for all $n>N_{2}$,
                $|b-x_{n}|<|a-b|/4$. Let $N=\max\{N_{1},N_{2}\}+1$.
                But from the triangle inequality:
                $|a-b|\leq|a-x_{N}|+|b-x_{N}|<|a-b|/2$, a
                contradiction. Therefore, $a=b$.
            \end{proof}
            The next notion to discuss is that of
            \textit{subsequences}. There are two ways to define
            a subsequence rigorously. A subsequence of a sequence
            $x:\mathbb{N}\rightarrow{X}$ is a sequence 
            $y:\mathbb{N}\rightarrow{X}$ such that there exists
            a strictly increasing sequence
            $k:\mathbb{N}\rightarrow\mathbb{N}$ such that, for all
            $n\in\mathbb{N}$, $y_{n}=(x\circ{k})(n)$. Here,
            $(x\circ{k})$ is the \textit{composition} of
            the two functions $x$ and $k$. This is
            often written $x_{k_{n}}$, but this can occasionally
            be confusing. We can also just define a subsequence
            to be any strictly increasing sequence
            $k:\mathbb{N}\rightarrow\mathbb{N}$. Given a sequence
            $x:\mathbb{N}\rightarrow{X}$, since $k$ is strictly
            increasing the ordering of $x\circ{k}$
            remains the same, and we've simply skipped over
            some points. Recall that
            monotonic sequences are sequences such
            that, for all $n\in\mathbb{N}$, either
            $x_{n+1}\leq{x_{n}}$ (Monotonically decreasing),
            or $x_{n}\leq{x_{n+1}}$ (Monotonically increasing).
            Strictly monotonic means either $x_{n+1}<x_{n}$
            or $x_{n}<x_{n+1}$ for all $n\in\mathbb{N}$.
            \begin{definition}
                A subsequence is a strictly increasing sequence
                $k:\mathbb{N}\rightarrow\mathbb{N}$
            \end{definition}
            \begin{definition}
                A convergent subsequence of a sequence
                $x:\mathbb{N}\rightarrow{S}$ in
                a subset $S\subseteq\mathbb{R}$ is a
                subsequence $k$ such that
                $x\circ{k}$ is a convergent sequence in $S$.
            \end{definition}
            \begin{definition}
                A monotonic subsequence of a sequence
                $x:\mathbb{N}\rightarrow{S}$ in a subset
                $S\subseteq\mathbb{R}$ is a subsequence
                $k:\mathbb{N}\rightarrow\mathbb{N}$ such
                that $x\circ{k}$ is a monotonic sequence.
            \end{definition}
            \begin{example}
                If $x:\mathbb{N}\rightarrow\mathbb{N}$ is
                the sequence defined by $x_{n}=n$, and if
                $k:\mathbb{N}\rightarrow\mathbb{N}$ is the
                subsequence defined by
                $k_{n}=2n$, then $x_{k_{n}}=2n$. This is the
                subsequence of all even numbers.
                If $k_{n}=2n-1$, then $x_{k_{n}}=2n-1$. This
                is the subsequence of all odd numbers. As a
                boring example, let $k_{n}=n$. Then
                $x_{k_{n}}=n$. This is the identity subsequence.
            \end{example}
            \begin{theorem}
                If $S\subseteq\mathbb{R}$,
                $x:\mathbb{N}\rightarrow\mathbb{R}$ is a
                convergent sequence, and if
                $k:\mathbb{N}\rightarrow\mathbb{N}$ is a
                subequence, then $x\circ{k}$ is a convergent
                sequence.
            \end{theorem}
            \begin{proof}
                For let $\varepsilon>0$.
                As $x$ is a convergent sequence there is
                an $a\in\mathbb{R}$ such that
                $x_{n}\rightarrow{a}$. Thus, there is an
                $N\in\mathbb{N}$ such that, for all
                $n>N$, $|a-x_{n}|<\varepsilon$. But
                $k$ is a subsequence and is therefore
                strictly increasing, so
                for all $n\in\mathbb{N}$, $k_{n}\geq{n}$.
                But then for all $n>N$, $k_{n}>N$, and thus
                $|x_{k_{n}}-a|<\varepsilon$. Therefore,
                $x_{k_{n}}\rightarrow{a}$.
            \end{proof}
            There is an important theorem about
            subsequences of bounded sequences called the
            Bolzano-Weierstrass theorem. It states that
            every bounded sequence has a convergent subsequence,
            and is an equivalent definition of the
            completeness of $\mathbb{R}$. There are a few theorems
            needed before we can prove it.
            \begin{theorem}
                \label{th:funct:bounded_monotone_%
                       sequences_converge}
                If $x:\mathbb{N}\rightarrow\mathbb{R}$
                is a bounded monotonic sequence,
                then $x$ is a convergent sequence.
            \end{theorem}
            \begin{proof}
                Let $x$ be a bounded monotonic sequence that
                is increasing in $\mathbb{R}$.
                If $x$ is decreasing, we replace the least
                upper bound with the greatest lower
                bound and the proof is symmetric.
                Then $S=\{x_{n}:n\in\mathbb{N}\}$ is a
                non-empty subset of $\mathbb{R}$. But $x$ is
                a bounded sequence, and therefore $S$ is a
                bounded subset of $\mathbb{R}$. By the least
                upper bound property there exists a least
                upper bound $s\in\mathbb{R}$ of $S$.
                We now show that $x_{n}\rightarrow{s}$.
                Let $\varepsilon>0$ be given. Since $s$ is
                the least upper bound, $s-\varepsilon$
                is not an upper bound of $S$, since
                $s-\varepsilon<s$. Therefore there exists
                an $N\in\mathbb{N}$ such that
                $s-\varepsilon<x_{N}$. But $x$ is
                monotonically increasing, and therefore
                for all $n>N$, $x_{N}\leq{x_{n}}$.
                But, as $s$ is a least upper
                bound of $S$, $x_{n}\leq{s}$. But then,
                for all $n>N$,
                $0\leq{s-x_{n}}\leq{s-x_{N}}<\varepsilon$.
                Therefore, $x_{n}\rightarrow{s}$.
            \end{proof}
            The least upper bound is, in a sense, the
            reason why decimal expansions of
            real numbers work. For example, let $x$ be the
            sequence 3, 3.1, 3.14, 3.141, 3.1415, 3.14159,
            and so forth. This sequence, which is
            the decimal expansion of $\pi$, is bounded by $4$.
            Therefore it has a least upper bound.
            We define the number $\pi$
            to be the least upper bound of this sequence.
            Completeness is a very important property
            but so far it relies on the ordering
            of the real numbers.
            We want to find an equivalent definition
            of completeness that does not rely on ordering
            so that we may speak of complete spaces,
            or sets, which have no notion of
            order. We start with a different definition
            for the completeness of $\mathbb{R}$.
            \begin{definition}
                A Cauchy sequence in a subset
                $S\subseteq\mathbb{R}$ is a
                sequence $x:\mathbb{N}\rightarrow{S}$
                such that for all $\varepsilon>0$ there
                is an $N\in\mathbb{N}$ such that for all
                $n,m>N$, $|x_{n}-x_{m}|<\varepsilon$.
                That is:
                \begin{equation}
                    \label{thm:Func_Def_Cauchy_Sequence}
                    \forall_{\varepsilon>0}
                    \exists_{N\in\mathbb{N}}:
                    n,m>N\Rightarrow
                    |x_{n}-x_{m}|<\varepsilon
                \end{equation}
            \end{definition}
            \begin{theorem}
                \label{FUNCTIONAL_ANALYSIS:CONVERGENT_%
                       SEEQUENCES_ARE_CAUCHY_SEQUENCES}
                If $S\subseteq\mathbb{R}$ and if
                $x:\mathbb{N}\rightarrow{S}$
                is a convergent sequence, then it
                is a Cauchy sequence.
            \end{theorem}
            \begin{proof}
                For let $x$ be a convergent sequence and
                let $a$ be it's limit.
                Let $\varepsilon>0$ be given. Then, as
                $x_{n}\rightarrow{a}$, there is an
                $N\in\mathbb{N}$ such that for all $n>N$,
                $|x_{n}-a|<\varepsilon/2$.
                But by the triangle inequality,
                for all $n,m>N$:
                \begin{equation}
                    |x_{n}-x_{m}|\leq
                    |x_{n}-a|+|x_{m}-a|<
                    \frac{\varepsilon}{2}+
                    \frac{\varepsilon}{2}
                    =\varepsilon
                \end{equation}
                Therefore, $x$ is a Cauchy sequence.
            \end{proof}
            The converse of
            Thm.~\ref{FUNCTIONAL_ANALYSIS:CONVERGENT_%
                      SEEQUENCES_ARE_CAUCHY_SEQUENCES}
            turns out to be a more general notion
            of completeness. That is, we can apply
            this to spaces that do not have
            a notion of order, but do have a notion
            of completeness. There are Cauchy sequences
            $x:\mathbb{N}\rightarrow\mathbb{Q}$ that do
            not converge. This is again related to the fact
            that $\mathbb{Q}$ is not complete. For sequences
            $x:\mathbb{N}\rightarrow\mathbb{R}$,
            if $x$ is Cauchy then it must converge.
            \begin{theorem}
                \label{THM:FUNCTIONAL_ANALYSIS:%
                       SUBSEQ_OF_CAUCHY_IS_CAUCHY}
                If $S\subseteq\mathbb{R}$,
                $x:\mathbb{N}\rightarrow{S}$ is a Cauchy sequence,
                and if $k:\mathbb{N}\rightarrow\mathbb{N}$
                is a subsequence, then
                $x\circ{k}$ is a Cauchy sequence.
            \end{theorem}
            \begin{proof}
                For let $\varepsilon>0$. As $x$ is a Cauchy
                sequence, there is an $N\in\mathbb{N}$ such that,
                for all $n,m>N$, $|x_{n}-x_{m}|<\varepsilon$.
                But, as $k$ is a subsequence it is strictly
                increasing, and thus for all $n\in\mathbb{N}$,
                $k_{n}\geq{n}$. But then for all $n,m>N$,
                $k_{n},k_{m}>N$, and thus
                $|x_{k_{n}}-x_{k_{m}}|<\varepsilon$. Thus,
                $x\circ{k}$ is Cauchy.
            \end{proof}
            \begin{theorem}
                If $S\subseteq\mathbb{R}$ and if
                $x:\mathbb{N}\rightarrow{S}$ is a Cauchy sequence,
                then $x$ is a bounded sequence.
            \end{theorem}
            \begin{proof}
                For as $x$ is a Cauchy sequence there is an
                $N\in\mathbb{N}$ such that, for all $n,m>N$,
                $|x_{n}-x_{m}|<1$. Then, for all $n>N+1$,
                $x_{N+1}-1<x_{n}<x_{N+1}+1$. Let
                $M=\max\{|x_{0}|,|x_{1}|,\hdots,|x_{N+1}|+1\}$.
                Then for all $n\in\mathbb{N}$,
                $|x_{n}|\leq{M}$.
            \end{proof}
            We cannot replace the requirement that,
            for all $n,m>N$, $|x_{n}-x_{m}|<\varepsilon$
            with $n,n+k$ for some fixed $k\in\mathbb{N}$.
            There are sequences such that
            $x_{n+1}-x_{n}\rightarrow{0}$,
            yet $x$ is not Cauchy. Indeed, there are such sequences
            that are bounded.
            \begin{example}
                \begin{subequations}
                    There are unbounded sequences $x$ such that
                    $x_{n+1}-x_{n}\rightarrow{0}$. For let
                    $x:\mathbb{N}\rightarrow\mathbb{R}$
                    be the sequence defined
                    by $x_{n}=\sqrt{n}$. Then:
                    \begin{equation}
                        |x_{n+1}-x_{n}|=|\sqrt{n+1}-\sqrt{n}|
                        =\frac{1}{\sqrt{n+1}+\sqrt{n}}
                        <\frac{1}{2\sqrt{n}}
                        \rightarrow{0}
                    \end{equation}
                    But $\sqrt{n}\rightarrow\infty$.
                    Moreover, there are bounded sequences $x$
                    such that $x_{n+1}-x_{n}\rightarrow{0}$,
                    yet $x$ is not Cauchy. For let
                    $x:\mathbb{N}\rightarrow\mathbb{R}$
                    be defined by
                    $x_{n}=\cos(\pi\sqrt{n})$.
                    Then $x$ is bounded, and:
                    \begin{align}
                        x_{n+1}-x_{n}
                        &=\cos\big(\pi\sqrt{n+1})
                            -\cos(\pi\sqrt{n}\big)\\
                        &=-2\sin\Big(\pi
                            \frac{\sqrt{n+1}-\sqrt{n}}{2}\Big)
                            \sin\Big(\pi
                                \frac{\sqrt{n+1}+\sqrt{n}}{2}\Big)
                    \end{align}
                    But we saw from the previous example that
                    $\sqrt{n+1}-\sqrt{n}\rightarrow{0}$, and
                    therefore $x_{n+1}-x_{n}\rightarrow{0}$.
                    $x$ is not Cauchy, however, for let
                    $k:\mathbb{N}\rightarrow\mathbb{N}$ be
                    the subsequence defined by $k_{n}=n^{2}$. Then:
                    \begin{equation}
                        x_{k_{n}}=\cos(\pi{n})=(-1)^{n}
                    \end{equation}
                    And this is not a Cauchy sequence. By
                    Thm.~\ref{THM:FUNCTIONAL_ANALYSIS:%
                              SUBSEQ_OF_CAUCHY_IS_CAUCHY},
                    $x$ is not a Cauchy sequence.
                \end{subequations}
            \end{example}
            \begin{theorem}
                \label{th:funct:sequences_have_%
                       monotonic_subsequence}
                Every sequence in $\mathbb{R}$
                has a monotonic subsequence.
            \end{theorem}
            \begin{proof}
                Let $x$ be a sequence in $\mathbb{R}$.
                Call $n$ a ``peak point'' if
                $x_{n}\geq{x_{m}}$ for all
                ${m}\geq{n}$. If there are infinitely many
                of these peak points, then we have obtained
                a decreasing sequence, since the $n^{th}$
                peak point will be greater than or equal to
                the $(n+1)^{th}$ peak point.
                We have thus obtained
                a monotonically decreasing subsequence.
                If there are finitely many,
                there are either zero or there is a last one,
                $x_{n_{0}}$. Then $x_{n_{0}+1}$ is not a
                peak point. But then there is a
                $k\in\mathbb{N}$ such that $k>n_{0}+1$ and
                $x_{k}\geq{x_{n_{0}+1}}$, for otherwise
                $x_{n_{0}+1}$ would be a peak point. But
                $x_{k}$ is also not a peak point, and so
                there is a $k_{1}$ such that $k_{1}>k$ and
                $x_{k_{1}}\geq{x_{k}}$. This pattern
                continues, and we thus have a monotonically
                increasing subsequence. If there are zero
                peak points, repeat the argument above
                argument with $x_{n_{0}}=x_{1}$.
            \end{proof}
            There's probably some axiom of choice stuff
            going on here.
            \begin{theorem}[Bolzano-Weierstrass Theorem]
                If $x:\mathbb{N}\rightarrow\mathbb{R}$
                is a bounded sequence, then there is
                a convergent subsequence
                $k:\mathbb{N}\rightarrow\mathbb{N}$
                of $x$.
            \end{theorem}
            \begin{proof}
                By Thm.~\ref{th:funct:sequences_%
                             have_monotonic_subsequence},
                if $x:\mathbb{N}\rightarrow\mathbb{R}$ is a
                sequence, then there is a monotonic subsequence
                $k:\mathbb{N}\rightarrow\mathbb{N}$.
                But by Thm.~\ref{th:funct:bounded_%
                                 monotone_sequences_converge},
                bounded monotonic sequences converge.
                Thus $x\circ{k}$ converges.
                Therefore $k$ is a convergent
                subsequence of $x$.
            \end{proof}
            This notion is so important it has a name.
            A sequentially compact space is a space such that
            every sequence has a convergent subsequence. The
            Bolzano-Weierstrass Theorem is equivalent
            to saying that every closed and bounded subset
            of $\mathbb{R}$ is sequentially
            compact. The general notion of \textit{compactness}
            is a topological one, but as it turns out
            sequential compactness and compactness are
            identical concepts in a \textit{metric space}.
            Metric spaces will be one of the primary
            subjects of study in functional analysis.
            In $\mathbb{R}^{n}$ there is another equivalent,
            and perhaps more intuitive,
            definition of compactness. A subset of
            $\mathbb{R}^{n}$ is compact if and only if it
            is closed and bounded. A set
            $S\subseteq\mathbb{R}$ is closed if for
            all convergent sequences
            $x:\mathbb{N}\rightarrow{S}$,
            the limit also lies in $S$.
            Compactness will be discussed later in the
            context of continuous functions on a compact set.
            \begin{example}
                \begin{subequations}
                    Find a subsequence $k$ of the identity
                    $x:\mathbb{N}\rightarrow\mathbb{R}$
                    defined by $x_{n}=n$ for which
                    both $\sin(x\circ{k})$ and $\cos(x\circ{k})$
                    converge. First note that for any subsequence
                    $k$, $(x\circ{k})(n)=k_{n}$.
                    In degrees this is simple:
                    \begin{equation}
                        k_{n}=360n+45
                        \Rightarrow
                        \sin(k_{n})=\cos(k_{n})
                        =\frac{1}{\sqrt{2}}
                    \end{equation}
                    In radians we need to be a little more careful.
                    Let $y:\mathbb{N}\rightarrow\mathbb{R}$
                    be defined by $y_{n}=\sin(n)$.
                    Then $y$ is bounded and
                    by the Bolzano-Weierstrass theorem,
                    there is a convergent subsequence $k$.
                    Let $z:\mathbb{N}\rightarrow\mathbb{R}$
                    be defined by $z_{n}=\cos(k_{n})$. Then $z$
                    is bounded and by the
                    Bolzano-Weierstrass theorem there is a
                    convergent subsequence $j$. Let $k_{j}$
                    denote the subsequence $k\circ{j}$. But
                    any subsequequence of a convergent sequence
                    converges to the same limit, and therefore
                    $\sin(k_{j})$ is a convergent sequence. Thus,
                    $\sin(k_{j})$ and $\cos(k_{j})$ are
                    convergent sequences. It's also
                    possible to make them converge to the same
                    limit. We need to know that
                    $\{n\mod\alpha:n\in\mathbb{N}\}$ is dense
                    in $(0,\alpha)$ when $\alpha$ is irrational.
                    Thus there is a subsequence such that
                    $k_{n}\mod2\pi\rightarrow\pi/4$.
                    Then $\sin(k_{n})$ and $\cos(k_{n})$
                    both converge to $1/\sqrt{2}$.
                    Let's first try to find a subsequence such that
                    $\cos(k_{n})\rightarrow{1}$. If we can
                    do that, we simply need to modify the
                    argument so that
                    $\cos(k_{n})\rightarrow{1}/\sqrt{2}$.
                    Let $k$ be a sequence of integers
                    such that $0<n-2\pi{k_{n}}<2\pi$.
                    Let $\varepsilon>0$ and let $N\in\mathbb{N}$
                    be such that $N>\frac{2\pi}{\varepsilon}$.
                    Now consider the set:
                    \begin{equation}
                        A_{N}=\{n-2\pi{k_{n}}:n=1,2,\hdots,N+1\}
                    \end{equation}
                    Then $A_{N}$ has $N+1$ elements and by the
                    pidgeon-hole principle there are
                    elements that are within
                    $2\pi/\frac{2\pi}{\varepsilon}$ of each other.
                    Let $n_{1}$ and $n_{2}$ be such numbers.
                    Then:
                    \begin{align}
                        \cos(n_{2}-n_{1})
                        &=\cos(n_{2}-n_{1}-2\pi(k_{2}-k_{1}))\\
                        &=\cos((n_{2}-2\pi{k}_{2})
                               -(n_{1}-2\pi{k_{1}}))\\
                        &=\cos(\xi)
                    \end{align}
                    Where $\xi$ is a number such that
                    $0<|\xi|<\varepsilon$. But then
                    $|1-\cos(\xi)|<\frac{\varepsilon^{2}}{2}$.
                    And $n_{2}-n_{1}$ is a natural number,
                    so we can find a subsequence $k$ such
                    that $\cos(k_{n})\rightarrow{1}$. Modifying
                    this with $\pi/4$
                    and $1/\sqrt{2}$ gives the result.
                \end{subequations}
            \end{example}
            \begin{theorem}
                If $x:\mathbb{N}\rightarrow\mathbb{R}$ is
                a Cauchy sequence, then it converges.
            \end{theorem}
            \begin{proof}
                If $x$ is Cauchy, then it is bounded.
                By the Bolzano-Weiestrass theorem there
                is a convergent subsequence $k$. But then there
                is an $a\in\mathbb{R}$ such that
                $x_{k_{n}}\rightarrow{a}$. We now must show that
                $x_{n}\rightarrow{a}$. Let $\varepsilon>0$
                be given. As $x_{k_{n}}\rightarrow{a}$,
                there is an $N_{1}\in\mathbb{N}$ such
                that for all $n>N_{1}$,
                $|x_{k_{n}}-a|<\frac{\varepsilon}{2}$.
                But as $x$ is a Cauchy sequence, there
                is an $N_{2}$ such that for all $n,m>N_{2}$, 
                $|x_{n}-x_{m}|<\frac{\varepsilon}{2}$. Let
                $N=\max\{N_{1},N_{2}\}$. 
                But $k$ is a subsequence, and thus for all
                $n>N$, $k_{n}>N$. But then if $n>N$,
                $|x_{k_{n}}-x_{n}|<\frac{\varepsilon}{2}$.
                By the triangle inequality,
                    $|a-x_{n}|\leq
                     |a-x_{k_{n}}|+|x_{k_{n}}-x_{n}|\leq
                     \frac{\varepsilon}{2}+
                     \frac{\varepsilon}{2}%
                     =\varepsilon$.
            \end{proof}
            Real numbers can be constructed by considering
            \textit{equivalence classes} of Cauchy sequences of
            rational numbers. Two Cauchy sequences $x_{n}$ and
            $y_{n}$ are equivalent if $x_{n}-y_{n}\rightarrow{0}$.
            By considering the set
            of all such equivalent sequences, we can give a more
            rigorous construction of the real numbers.
            \begin{example}
                \begin{subequations}
                    Let $x:\mathbb{N}\rightarrow\mathbb{Q}$
                    be the sequence:
                    \begin{equation}
                        x_{n}=\frac{2n+3}{n}
                    \end{equation}
                    Let $\varepsilon>0$ and let
                    $N=\ceil{6/\varepsilon}+1$.
                    Then, for $n,m>N$, we have:
                    \begin{equation}
                        |x_{n}-x_{m}|=
                        \Big|\frac{2n+3}{n}-\frac{2m+3}{m}\Big|
                        =3\Big|\frac{n-m}{nm}\Big|
                        <\frac{6}{\min\{n,m\}}
                        <\frac{6}{N}<\varepsilon
                    \end{equation}
                    Therefore $x$ is a Cauchy sequence of rational
                    numbers. It has a standard representation
                    of 2 since $x_{n}\rightarrow{2}$. To see this:
                    \begin{equation}
                        \Big|2-\frac{2n+3}{n}\Big|
                        =\Big|\frac{3}{n}\Big|\rightarrow{0}
                    \end{equation}
                    There are other elements of the equivalence
                    class for 2. Indeed the sequence
                    $x:\mathbb{N}\rightarrow\mathbb{Q}$ defined
                    by $x_{n}=2$ for all $n\in\mathbb{N}$ is
                    such an example. The equivalence classes
                    also define the irrational numbers as well.
                    For let $x:\mathbb{N}\rightarrow\mathbb{Q}$
                    be defined by:
                    \begin{equation}
                        x_{n}=\sum_{k=0}^{n}\frac{(-1)^{k}}{n!}
                    \end{equation}
                    The ratio test, or the alternating series
                    test, can be applied to show that this
                    converges. Convergent sequences are Cauchy
                    sequence, and thus $x$ can be used to
                    represent some real number. The number it
                    represents is $e^{-1}$, which is irrational.
                    If one recalls the history of $e$, we know
                    that the equivalence class for $e^{-1}$ also
                    contains the following sequence:
                    \begin{equation}
                        x_{n}=\Big(1-\frac{1}{n}\Big)^{n}
                    \end{equation}
                \end{subequations}
            \end{example}
            We have seen that the least upper bound axiom,
            together with the ordered field structure that
            $\mathbb{R}$ possesses, implies that
            Cauchy sequence in $\mathbb{R}$ converge. This can
            be reversed, showing that we now have two equivalent
            definitions of completeness.
            \begin{theorem}
                If $x:\mathbb{N}\rightarrow\mathbb{R}$
                is a bounded monotonic sequence, then
                $x$ is a Cauchy sequence.
            \end{theorem}
            \begin{proof}
                For suppose not. Suppose $x$ is monotonically
                increasing. If $x$ is not Cauchy
                then there exists an $\varepsilon>0$ such
                that, for all $N\in\mathbb{N}$ there exists
                $n,m>N$ such that
                $|x_{n}-x_{m}|\geq\varepsilon$. But if
                $x$ is bounded, there is an $s$ such that,
                for all $n\in\mathbb{N}$, $|x_{n}|\leq{s}$.
                From the Archimedean principle, as
                $\varepsilon>0$ there is an $N_{1}\in\mathbb{N}$
                such that $x_{1}+N_{1}\varepsilon>s$.
                Let $X=\{x_{n}:n\in\mathbb{N}\}$.
                For all $N\in\mathbb{N}$, $N\geq{2}$,
                there exists a function
                $z:\mathbb{Z}_{N}\rightarrow{X}$ such that, for
                all $n\in\mathbb{Z}_{N-1}$,
                $z_{n}<z_{n+1}$, and
                $\min(\{|z_{n}-z_{m}|:n,m\in\mathbb{Z}_{N}\})%
                 \geq\varepsilon$. By induction,
                let $z_{1}=x_{1}$. As $x$ is not Cauchy, there
                are $n,m>1$ such that
                $|x_{n}-x_{m}|\geq\varepsilon$. But from
                monotonicity, $x_{m}\geq{x}_{1}$, and thus
                $|x_{n}-x_{1}|\geq\varepsilon$.
                Let $z_{2}=x_{n}$. Suppose it is true for
                $N\in\mathbb{N}$. Let
                $z:\mathbb{Z}_{N}\rightarrow{X}$ be such a
                function. As $x$ is not Cauchy and
                monotonic, there is an $n>N$ such that
                $|x_{n}-z_{N}|\geq\varepsilon$.
                Let $z':\mathbb{Z}_{N+1}\rightarrow{X}$
                be defined by:
                \begin{subequations}
                    \begin{equation}
                        z'_{k}=
                        \begin{cases}
                            z_{k},&1\leq{k}\leq{N}\\
                            x_{n},&k=N+1
                        \end{cases}
                    \end{equation}
                    From monotonicity, for all
                    $n\in\mathbb{Z}_{N}$,
                    $z'_{N+1}-z'_{n}\geq\varepsilon$. Moreover,
                    $z'_{N+1}>z'_{N}$. Thus $z'$
                    satisfies the criterion.
                    Thus, there is a function
                    $z:\mathbb{Z}_{N_{1}+1}\rightarrow{X}$
                    such that $z$ is increasing and
                    $\min(\{|z_{n}-z_{m}|:%
                            n,m\in\mathbb{Z}_{N_{1}}\})%
                     \geq\varepsilon$.
                     But then:
                    \begin{equation}
                        z_{N_{1}+1}-z_{1}=
                        \sum_{n=1}^{N_{1}}(z_{n+1}-z_{n})
                        \geq{N}_{1}\varepsilon
                    \end{equation}
                    But then:
                    \begin{equation}
                        z_{N_{1}+1}>z_{1}+N_{1}\varepsilon
                    \end{equation}
                    But $z_{1}\in{X}$, and thus
                    $z_{1}\geq{x}_{1}$. But then
                    $z_{N_{1}+1}>x_{1}+N\varepsilon$. But
                    $s<x_{1}+N\varepsilon$, a contradiction
                    as $s\geq{x}_{n}$ for all $n\in\mathbb{N}$.
                    Therefore, $x$ is Cauchy.
                \end{subequations}
            \end{proof}
            This shows that the monotone convergence theorem
            can be proved without the least upper bound principle.
            The proof becomes messier, however. We now prove
            the equivalence of completeness and the least upper
            bound axiom.
            \begin{theorem}
                If every Cauchy sequence in $\mathbb{R}$
                is a convergent sequence, then every
                bounded above subset of $\mathbb{R}$ has a
                least upper bound.
            \end{theorem}
            \begin{proof}
                For if $L\subseteq\mathbb{R}$ is non-empty and
                bounded then there is an $a\in{L}$ and an
                $s\in\mathbb{R}$ such that, for all
                $y\in{L}$, $y\leq{s}$. If $s\in{L}$, then
                $s$ is a least upper bound of $L$. Suppose not.
                Let
                $S=\{y\in\mathbb{R}:\forall_{x\in{L}}x\leq{y}\}$.
                Then $S$ is non-empty, as $s\in{S}$.
                Suppose $s$ is not the least upper bound of $L$
                and define the following:
                \begin{subequations}
                    \begin{equation}
                        A_{k}
                        =\Big\{s-\frac{n}{2^{k}}:
                               n\in\mathbb{N}\Big\}
                        \bigcap{S}
                    \end{equation}
                    There exists an $N\in\mathbb{N}$ such that, for
                    all $k>N$, $A_{k}\ne\emptyset$, for otherwise
                    $s$ would be a least upper bound. Moreover, for
                    all $k>N$, $A_{k}$ is finite for by the
                    Archimedean property there is an
                    $n\in\mathbb{N}$ such that $s-n/2^{k}<x$,
                    and thus for all $m>n$,
                    $s-m/2^{k}\notin{A_{k}}$.
                    Lastly, $A_{k}\subseteq{A_{k+1}}$. Let
                    $x:\mathbb{N}\rightarrow\mathbb{R}$
                    be defined by:
                    \begin{equation}
                        x_{n}=\min(A_{n+N})
                    \end{equation}
                \end{subequations}
                This is well defined since, for all
                $n>N$, $A_{n}$ is finite. Then, since
                $A_{n}\subseteq{A_{n+1}}$ for all
                $n>N$, $x$ is a monotonically decreasing
                sequence. But then $x$ is monotonic and
                bounded, and is therefore Cauchy. But Cauchy
                sequences converge, and thus there is a
                $c\in\mathbb{R}$ such that
                $x_{n}\rightarrow{c}$. For all $y\in{L}$,
                $y\leq{c}$. For if there is a
                $y\in{L}$ such that $c<y$, then there is an
                $N$ such that $x_{N}<y$, a contradiction as
                $x_{N}\in{S}$. Thus, $c$ is an upper bound of $L$.
                Suppose $c$ is not
                the least upper bound, and thus there is a
                $d\in{S}$ such that $d<c$. But then there is an
                $k\in\mathbb{N}$ such that $c-d<2^{-k}$. But
                then $x_{k+1}<c$, a contradiction as $x$
                is monotonically decreasing and
                $x_{n}\rightarrow{c}$. Thus, $c$ is the least
                upper bound.
            \end{proof}
            We've now used the Archimedean property twice. This
            says that for any $\varepsilon>0$ and any
            $x>0$, there is an $N\in\mathbb{N}$ such that
            $N\varepsilon>x$. It is equivalent to saying the
            real numbers have no ``infinitesimals.'' We now
            have two ways to talk about the completeness of
            $\mathbb{R}$. The monotone convergence theorem
            can also be taken as axiom, and shown that it
            implies completeness, as well as the
            Bolzano-Weierstrass theorem. Lastly, there is the
            Nested Interval Theorem which will be proved later
            in the context of more general metric spaces.
        \subsection{Continuity}
            We now discuss continuity, uniform continuity, and
            related theorems.
            \begin{definition}
                A function $f:S\rightarrow\mathbb{R}$
                on a subset $S\subseteq\mathbb{R}$ continuous
                at a point $x\in{S}$ is a function such that
                for all $\varepsilon>0$ there is a $\delta>0$
                such that for all $x_{0}\in{S}$,
                $|x-x_{0}|<\delta$ implies
                $|f(x)-f(x_{0})|<\varepsilon$. That is:
                \begin{equation}
                    \forall_{\varepsilon>0}\exists_{\delta>0}:
                    x\in{S},|x-x_{0}|<\delta
                    \Rightarrow|f(x)-f(x_{0})|<\varepsilon
                \end{equation}
            \end{definition}
            \begin{theorem}
                If $S\subseteq\mathbb{R}$, $x\in{S}$,
                $f:S\rightarrow\mathbb{R}$
                is continuous at $x$, and if
                $a:\mathbb{N}\rightarrow{S}$
                is a convergent sequence such that
                $a_{n}\rightarrow{x}$, then
                $f(a_{n})\rightarrow{f(x)}$.
            \end{theorem}
            \begin{proof}
                For let $\varepsilon>0$. As $f$ is
                continuous there is a $\delta>0$ such that,
                for all $x_{0}\in{S}$
                such that $|x-x_{0}|<\delta$,
                $|f(x)-f(x_{0})|<\varepsilon$.
                But $a_{n}\rightarrow{x}$, and thus there is an
                $N\in\mathbb{N}$ such that, for all $n>N$,
                $|x-a_{n}|<\delta$. But then, for all $n>N$,
                $|f(x)-f(a_{n})|<\varepsilon$. Therefore,
                $f(a_{n})\rightarrow{f(x)}$.
            \end{proof}
            The converse of this theorem is true, giving us
            an equivalent definition of continuity.
            \begin{theorem}
                If $S\subseteq\mathbb{R}$, $x\in{S}$ and
                $f:S\rightarrow\mathbb{R}$
                is a function such that for all sequences
                $a:\mathbb{N}\rightarrow\mathbb{R}$ such that
                $a_{n}\rightarrow{x}$,
                $f(a_{n})\rightarrow{f(x)}$,
                then $f$ is continuous at $x$.
            \end{theorem}
            \begin{proof}
                For suppose not. Then there is an
                $\varepsilon>0$ such that, for all $\delta>0$,
                there is an $x_{0}\in{S}$ such that
                $|x-x_{0}|<\delta$ and
                $|f(x)-f(x_{0})|\geq\varepsilon$.
                Let $a:\mathbb{N}\rightarrow\mathbb{R}$
                be a sequence such that, for all
                $n\in\mathbb{N}$, $|a_{n}-x|<1/n$, but
                $|f(x)-f(a_{n})|\geq\varepsilon$.
                But then $a_{n}\rightarrow{x}$. But for all
                sequences $a$ such that $a_{n}\rightarrow{x}$,
                $f(a_{n})\rightarrow{f(x)}$. But, for all $n$,
                $|f(x)-f(a_{n})|\geq\varepsilon$,
                a contradiction.
                Therefore, $f$ is continuous at $x$.
            \end{proof}
            \begin{theorem}
                If $x\in\mathbb{R}$ and
                $a:\mathbb{N}\rightarrow\mathbb{R}$
                is a convergent sequence such that
                $a_{n}\rightarrow{x}$ and for all
                $n\in\mathbb{N}$, $a_{n}\geq{0}$,
                then $x\geq{0}$.
            \end{theorem}
            \begin{proof}
                For suppose not. Suppose $x<0$. Let
                $\varepsilon=|x|/2$. Then, as $\varepsilon>0$, there
                is an $N\in\mathbb{N}$ such that for all $n>N$,
                $|x-a_{n}|<\varepsilon$. But then
                $a_{N+1}<x+\varepsilon=x/2<0$, a contradiction as
                $a_{N+1}\geq{0}$.
            \end{proof}
            \begin{theorem}
                \label{thm:Funct:Continuous_Limit_%
                       of_Pos_Sequ_is_nonneg}
                If $S\subseteq\mathbb{R}$, $x\in{S}$,
                $f:S\rightarrow\mathbb{R}$ is continuous at $x$, and if
                $a:\mathbb{N}\rightarrow\mathbb{R}$ is a sequence such
                that  $a_{n}\rightarrow{x}$ and $f(a_{n})>0$
                for all $n\in\mathbb{N}$, then $f(x)\geq{0}$.
            \end{theorem}
            \begin{proof}
                For suppose not. Let $r=f(x)<0$, and let
                $\varepsilon=|r|/2$. Then $\varepsilon>0$. But
                from continuity, there is a $\delta>0$ such that
                for all $x_{0}\in{S}$ such that $|x-x_{0}|<\delta$,
                $|f(x)-f(x_{0})|<\varepsilon$. But
                $a_{n}\rightarrow{x}$, and thus there is an
                $N\in\mathbb{N}$ such that for all $n>N$,
                $|x-a_{n}|<\delta$. Thus
                $|f(x)-f(a_{N+1})|<\varepsilon$. But then
                $f(a_{n})<f(x)+\varepsilon=f(x)/2<0$,
                a contradiction as $f(a_{N+1})>0$. Therefore, etc.
            \end{proof}
            \begin{theorem}
                If $x\in\mathbb{R}$,
                $f:\mathbb{R}\rightarrow\mathbb{R}$ is
                continuous at $x$, and if $f(x)>0$,
                then there is an open interval
                $\mathcal{U}$ such that $x\in\mathcal{U}$,
                and for all $y\in\mathcal{U}$, $f(y)>0$.
            \end{theorem}
            \begin{proof}
                For let $\varepsilon=f(x)/2$. Then
                $\varepsilon>0$, and thus there is a $\delta>0$
                such that for all $x_{0}\in\mathbb{R}$
                such that $|x-x_{0}|<\delta$,
                $|f(x)-f(x_{0})|<\varepsilon$. Let
                $\mathcal{U}=(x-\delta,x+\delta)$.
                Then $\mathcal{U}$ is an open intervals and if
                $y\in\mathcal{U}$, then $|x-y|<\delta$,
                and therefore:
                \begin{equation}
                    |f(y)-f(x)|<\varepsilon
                    \Rightarrow
                    f(y)>f(x)-\varepsilon
                    =\frac{f(x)}{2}>0
                \end{equation}
                Thus, for all $y\in\mathcal{U}$, $f(y)>0$.
            \end{proof}
            \begin{definition}
                A continuous function on
                $S\subseteq\mathbb{R}$ is a function
                $f:S\rightarrow\mathbb{R}$ such that
                $f$ is continuous at all $x\in{S}$. That is:
                \begin{equation}
                    \forall_{x\in{S}}\forall_{\varepsilon>0}
                    \exists_{\delta>0}:x_{0}\in{S},
                    |x-x_{0}|<\delta
                    \Rightarrow|f(x)-f(x_{0})|<\varepsilon
                \end{equation}
            \end{definition}
            This definition comes from the fact that
            continuity is a point-wise property, and not a
            ``curve'' property. Continuous functions are
            functions that have point-wise continuity at
            every point. The statement ``A continuous function
            is a curve that you can draw,'' which many have
            heard in calculus, is slightly misleading. There
            are functions that are continuous at one point and
            nowhere else. There are functions that are
            continuous on the irrationals and discontinuous
            on the rationals. For example, if $x$ is
            rational write it as $x=p/q$ where $p$ and
            $q$ are integers and relatively prime. Define $f$
            as follows:
            \begin{equation}
                f(x)=
                \begin{cases}
                    \frac{1}{q},&x\in\mathbb{Q}\\
                    0,&x\notin\mathbb{Q}
                \end{cases}
            \end{equation}
            This function, which is known as
            Dirichlet's Function, but also as the Popcorn
            Function, or Thomae's Function, is continuous at every
            irrational number and discontinuous at every
            rational number.
            \begin{figure}[H]
                \captionsetup{type=figure}
                \centering
                \documentclass[crop,class=article]{standalone}
%----------------------------Preamble-------------------------------%
\usepackage{tikz}                       % Drawing/graphing tools.
\usetikzlibrary{arrows.meta}            % Latex arrows.
%--------------------------Main Document----------------------------%
\begin{document}
    \begin{tikzpicture}[scale=8,>=latex]
        \draw[->] (-0.1,0) -- (1.1,0)
            node[above left] {$x$};
        \draw[->] (0,-0.1) -- (0,0.6)
            node[right] {$f(x)$};
        \draw (0.02,1/2) -- (-0.02,1/2)
            node[left]{$\frac{1}{2}$};
        \draw (0.02,1/3) -- (-0.02,1/3)
            node[left]{$\frac{1}{3}$};
        \draw (0.02,1/4) -- (-0.02,1/4)
            node[left]{$\frac{1}{4}$};
        \foreach\X[%
            evaluate=\X as \Ymax using {int(\X-1)}]
            in {25,24,...,2}{%
                \foreach\Y in {1,...,\Ymax}{%
                    \ifnum\X<5
                        \draw
                        (\Y/\X,0.02) -- (\Y/\X,-0.02)
                        node[below,fill=white]
                            {$\frac{\Y}{\X}$};
                    \else
                        \draw[ultra thin]
                        (\Y/\X,0.01) to (\Y/\X,-0.01);
                    \fi
                    \pgfmathtruncatemacro{\TST}
                        {gcd(\X,\Y)}
                    \ifnum\TST=1
                        \fill ({\Y/\X},1/\X) 
                            circle (0.2pt); 
                    \fi
                }
        }
        \foreach\X in {0,1,...,80}
        {\fill (\X/80,0) circle(0.2pt);}
    \end{tikzpicture}
\end{document}
                \caption{Dirichlet's Function is Continuous on
                         $\mathbb{R}\setminus\mathbb{Q}$ and
                         Discontinuous on $\mathbb{Q}$.}
                \label{fig:Funct:Dirichlet_Thomae_Function}
            \end{figure}
            There is no ``reverse,'' of this
            function. That is, there is no function which is
            continuous on $\mathbb{Q}$ and discontinuous at
            every irrational number. Uniform continuity is a
            property of all points in the domain of a function.
            Point-wise continuity says that given a point $x$
            and a positive number
            $\varepsilon$, one can find a $\delta$ satisfying a
            certain property. The key part is that the point $x$ must
            be specified first. That is, the $\delta$
            may be dependent on $x$.
            Uniform continuity occurs when a $\delta>0$ can be
            chosen regardless of $x$. $\delta$ is
            only dependent on $\varepsilon$.
            \begin{definition}
                A uniformly continuous function on a subset
                $S\subseteq\mathbb{R}$
                is a function $f:S\rightarrow\mathbb{R}$ such that:
                \begin{equation*}
                    \forall_{\varepsilon>0}\exists_{\delta>0}
                    \forall_{x\in{S}}:\forall_{x_{0}\in{S}},
                    |x-x_{0}|<\delta
                    \Rightarrow|f(x)-f(x_{0})|<\varepsilon    
                \end{equation*}
            \end{definition}
            Continuity is a point-wise property. There are
            functions that are continuous at one point
            and nowhere else. Uniform continuity, however,
            is a set property. You can't have uniform
            continuity at a single point,
            but rather on a set of points. Unless, of course,
            your domain $S$ is a single point. But that's rather boring.
            \begin{theorem}
                \label{thm:Funct:equiv_def_of_uni_cont}
                A function $f:S\rightarrow\mathbb{R}$
                is uniformly continuous if and only if
                for all sequences $x,y:\mathbb{N}\rightarrow\mathbb{R}$
                such that $x_{n}-y_{n}\rightarrow{0}$,
                $f(x_{n})-f(y_{n})\rightarrow{0}$.
            \end{theorem}
            \begin{proof}
                Let $\varepsilon>0$. If $f$ is uniformly continuous,
                then there is a $\delta>0$ such that for all
                $x$, $x_{0}\in{S}$ such that $|x-x_{0}|<\delta$,
                we have that $|f(x)-f(x_{0})|<\varepsilon$. But if
                $x_{n}-y_{n}\rightarrow{0}$, then there is an
                $N\in\mathbb{N}$ such that for all $n>N$,
                $|x_{n}-y_{n}|<\delta$. But then, for all $n>N$,
                $|f(x_{n})-f(y_{n})|<\varepsilon$. Therefore,
                $f(x_{n})-f(y_{n})\rightarrow{0}$. Proving the
                converse, suppose not. If $f$ is not uniformly
                continuous, then there exists $\varepsilon>0$
                such that for all $\delta>0$ there exists
                $x$, $x_{0}\in{S}$ such that
                $|x-x_{0}|<\delta$ and yet
                $|f(x)-f(x_{0})|\geq{\varepsilon}$. Let
                $x_{n}$ and $y_{n}$ be points such that
                $|x_{n}-y_{n}|<\frac{1}{n}$ and yet
                $|f(x_{n})-f(y_{n})|\geq\varepsilon$. Then
                $x_{n}-y_{n}\rightarrow{0}$. But if
                $x_{n}-y_{n}\rightarrow{0}$, then
                $f(x_{n})-f(y_{n})\rightarrow{0}$. But for all
                $n$, $|f(x_{n})-f(y_{n})|\geq{\varepsilon}$,
                a contradiction.
            \end{proof}
            The requirement of uniform continuity is crucial.
            Let $f:(0,1)\rightarrow\mathbb{R}$ be defined by
            $f(x)=x^{-1}$. Then $f$ is continuous, but
            not uniformly continuous. Let $x_{n}=n^{-1}$
            and $y_{n}=2n^{-1}$. Then
            $|y_{n}-x_{n}|=n^{-1}\rightarrow{0}$, but
            $|f(y_{n})-f(x_{n})|=n/2$, which diverges.
            Point-wise continuity says
            $f(x_{n})-f(x)\rightarrow{0}$, whereas
            uniform continuity allows the target to
            vary as well. Point-wise continuity can
            not guarantee this. The set under consideration is
            also crucial to uniform continuity. Indeed,
            the function $f(x)=x^{-1}$ \textit{is} uniformly
            continuous on $(1,\infty)$. For if $x,y\in(1,\infty)$:
            \begin{equation}
                |f(x)-f(y)|=\Big|\frac{1}{x}-\frac{1}{y}\Big|
                =\Big|\frac{x-y}{xy}\Big|\leq|x-y|
            \end{equation}
            Choosing $\delta=\varepsilon/2$ gives the result.
            \begin{theorem}
                \label{thm:FUNCTIONAL_ANALYSIS:CONT_ON_CLOSED_INTERVAL}
                If $f:[a,b]\rightarrow\mathbb{R}$ is continuous,
                then $f$ is uniformly continuous.
            \end{theorem}
            \begin{proof}
                Suppose not. Then, by
                Thm.~\ref{thm:Funct:equiv_def_of_uni_cont},
                there are sequences
                $x,y:\mathbb{N}\rightarrow[a,b]$ such
                that $x_{n}-y_{n}\rightarrow{0}$, yet there is
                an $\varepsilon>0$ such that, for all
                $N\in\mathbb{N}$, there is an $n>N$ such that
                $|f(x_{n})-f(y_{n})|\geq\varepsilon$. Let
                $k:\mathbb{N}\rightarrow\mathbb{N}$
                be a subsequence such that, for all
                $n\in\mathbb{N}$,
                $|f(x_{k_{n}})-f(y_{k_{n}})|\geq\varepsilon$.
                By the Bolzano-Weierstrass theorem there is a
                convergent subsequence
                $j:\mathbb{N}\rightarrow\mathbb{N}$ of
                $x\circ{k}$. Let $\alpha$
                be the limit. But for all $n\in\mathbb{N}$:
                \begin{subequations}
                    \begin{equation}
                        y_{j_{k_n}}=x_{j_{k_{n}}}
                        -(y_{j_{k_{n}}}-x_{j_{k_{n}}})
                        \Rightarrow
                        y_{j_{k_{n}}}\rightarrow\alpha
                    \end{equation}
                    Let $X,Y:\mathbb{N}\rightarrow[a,b]$
                    be sequences
                    defined by $X_{n}=x_{j_{k_{n}}}$ and
                    $Y_{n}=y_{j_{k_{n}}}$, respectively.
                    Then we have:
                    \begin{equation}
                        f(X_{n})-f(Y_{n})
                        =(f(X_{n})-f(\alpha))-(f(Y_{n}-f(\alpha))
                    \end{equation}
                \end{subequations}
                From continuity, $f(X_{n})\rightarrow{f(\alpha)}$
                and $f(Y_{n})\rightarrow{f(\alpha)}$, and thus
                $f(X_{n})-f(Y_{n})\rightarrow{0}$. But for
                all $n$,
                $|f(x_{k_{n}})-f(y_{k_{n}})|\geq\varepsilon$,
                a contradiction. Therefore, etc.
            \end{proof}
            The above theorem relies on the fact that
            $[a,b]$ is closed and bounded. Indeed, this is
            the only thing it relies on, the fact that it's
            an interval (Or connected) is unnecessary. We can
            write a more general result.
            \begin{definition}
                A closed subset of $\mathbb{R}$ is a subset
                $S\subseteq\mathbb{R}$ such that for all
                convergent sequences
                $x:\mathbb{N}\rightarrow{S}$, the limit of
                $x$ is an element of $S$.
            \end{definition}
            \begin{definition}
                A compact subset of $\mathbb{R}$ is a subset
                that is closed and bounded.
            \end{definition}
            \begin{theorem}
                If $S\subseteq\mathbb{R}$ is a compact subset
                of $\mathbb{R}$ and if
                $f:S\rightarrow\mathbb{R}$ is continuous,
                then $f$ is uniformly continuous.
            \end{theorem}
            Proving this more general result requires
            the equivalence of sequential compactness and
            regular compactness in $\mathbb{R}$. This will be
            shown to be true for any \textit{metric space}.
            We can lessen the the requirement that
            the subset be compact to being a
            half-open interval $[a,\infty)$ provided that
            the limit of $f(x)$ exists as
            $x\rightarrow\infty$.
            \begin{theorem}
                If $a\in\mathbb{R}$ and
                $f:[a,\infty)\rightarrow\mathbb{R}$ is
                a continuous function such that
                $\lim_{x\rightarrow\infty}f(x)$ exists,
                then $f$ is uniformly continuous.
            \end{theorem}
            \begin{proof}
                \begin{subequations}
                    For let $\varepsilon>0$. 
                    As $\lim_{x\rightarrow\infty}f(x)$ exists,
                    there is a $c\in\mathbb{R}$ such that,
                    for all $\varepsilon>0$ there is an
                    $x_{0}\in[a,\infty)$ such that, for all
                    $x>x_{0}$, $|f(x)-c|<\varepsilon/2$.
                    Let $b=x_{0}+1$. By
                    Thm.~\ref{thm:FUNCTIONAL_ANALYSIS:%
                              CONT_ON_CLOSED_INTERVAL}
                    $f$ is uniformly continuous on
                    $[a,b]$, and thus there is a $\delta>0$
                    such that, for all
                    $x_{1},x_{2}\in[a,b]$ such that
                    $|x_{1}-x_{2}|<\delta$,
                    $|f(x_{1})-f(x_{2})|<\varepsilon/2$.
                    But for all $x_{1},x_{2}\in(b,\infty)$:
                    \begin{equation}
                        |f(x_{1})-f(x_{2})|
                        \leq|f(x_{1})-c|+|f(x_{2})-c|<\varepsilon
                    \end{equation}
                    And if $x_{1}\in=[a,b]$ and
                    $x_{2}\in(b,\infty)$ are such that
                    $|x_{1}-x_{2}|<\delta$, then:
                    \begin{equation}
                        |f(x_{1})-f(x_{2})|\leq
                        |f(x_{1})-f(b)|+|f(x_{2})-f(b)|
                        <\varepsilon
                    \end{equation}
                    Thus, $f$ is uniformly continuous.
                \end{subequations}
            \end{proof}
            \begin{theorem}[Intermediate Value Theorem]
                If $f:[a,b]\rightarrow\mathbb{R}$
                is continuous and
                $f(a)<f(b)$, then for all $z\in\mathbb{R}$
                such that $f(a)<z<f(b)$,
                there is a $c\in(a,b)$ such that $f(c)=z$.
            \end{theorem}
            \begin{proof}
                For if $z\in\mathbb{R}$, let
                $g[a,b]\rightarrow\mathbb{R}$ be defined by
                $g(x)=z-f(x)$ for all $x\in[a,b]$. Then,
                since $f(a)<z$, $g(z)<0$. But then there
                is an $\varepsilon>0$ such that, for
                all $x\in[a,a+\varepsilon)$,
                $g(x)<0$. Define the following:
                \begin{equation}
                    \mathcal{U}=
                    \{r>0:\forall_{s<r}g(a+s)<0\}
                \end{equation}
                Then $\mathcal{U}$ is non-empty, for
                $\varepsilon\in\mathcal{U}$. But
                $\mathcal{U}$ is bounded above for
                $b-a\notin\mathcal{U}$, for
                $f(b)>0$, and thus for all
                $r>b-a$, $r\notin\mathcal{U}$. But then
                $\mathcal{U}$ is a non-empty bounded
                above subset, and by the least upper
                bound property, there exists a least
                upper bound $c$ of $\mathcal{U}$.
                As $c\leq{b-a}$, $a+c\in[a,b]$.
                By trichotomy, either $g(a+c)<0$,
                $g(a+c)=0$, or $g(a+c)>0$. Suppose
                $g(a+c)>0$. Then there is a
                $\varepsilon_{1}>0$ such that,
                for all $x\in(a+c-\varepsilon,a+c]$,
                $g(x)>0$. But this is a contradiction,
                as $c$ is the least upper bound of
                $\mathcal{U}$. Suppose $g(a+c)<0$.
                Then there is a $\varepsilon_{2}>0$
                such that, for all
                $x\in[a+c,a+c+\varepsilon_{2})$,
                $g(x)<0$. Again, this is a contradiction
                as $c$ is the least upper bound of
                $\mathcal{U}$. Therefore, $g(a+c)=0$,
                ad thus $f(a+c)=z$.
            \end{proof}
            Another way that is commonly used to prove this
            theorem is the method of bisection. Start with
            $x_{1}=a$, $x_{2}=b$, and then let
            $x_{3}=\tfrac{1}{2}(x_{1}+x_{2})$. Check whether
            $f(x_{3})=z$ or not. If $f(x_{3})<z$,
            let $x_{4}=\tfrac{1}{2}(x_{2}+x_{3})$, otherwise
            let $x_{4}=\tfrac{1}{2}(x_{1}+x_{3})$. Continuing
            dividing the region of interest in half, obtaining
            a Cauchy sequence $x$. The final part
            is to show that the limit $c$ of $x$ is such that
            $f(c)=z$. This theorem fails in $\mathbb{Q}$, for it
            relies on the completeness of $\mathbb{R}$. For
            example, $f(x)=x^{2}$ defined on $[0,4]$.
            Then $2\in[0,4]$, but there is no
            rational such that $x^{2}=2$.
            Another way to phrase this, in a more topological
            sense, is that the image of $[a,b]$, which is an
            interval, or a connected subset of $\mathbb{R}$,
            is again an interval, or a connected subset of
            $\mathbb{R}$. The proof that continuous
            functions take connected sets (Intervals)
            to connected sets (Again, intervals) is a lot
            easier than the one presented
            here, but relies on notions from topology.
            We'll revisit this when we
            discuss the topology of metric spaces.
            Another commonly used theorem in calculus
            is the extreme value theorem.
            The extreme value is used to proved Rolle's theorem,
            which says that if $f$ is differentiable on $(a,b)$
            and if $f(a)=f(b)$, then there is a point
            $c\in(a,b)$ such that $f'(c)=0$. This is used to
            prove the mean value theorem, which says that
            if $f$ is differentiable on $(a,b)$,
            then there is a point $c\in(a,b)$ such that
            $f'(c)=\frac{f(b)-f(a)}{b-a}$. This is in
            turned used to prove the Fundamental Theorem
            of Calculus. First we prove that continuous functions
            on closed and bounded sets (That is,
            compact sets) are bounded. We stick to
            closed intervals for now.
            \begin{theorem}
                If $f:[a,b]\rightarrow\mathbb{R}$ is
                continuous, then it is bounded.
            \end{theorem}
            \begin{proof}
                Suppose not. Then for all $n\in\mathbb{N}$, 
                there is an $\alpha\in[a,b]$ such that
                $f(\alpha)>n$. Invoking choice and using the
                sequence $x:\mathbb{N}\rightarrow[a,b]$ such
                that $f(x_{n})>n$, we have that
                $x$ is a bounded sequence, and thus by
                Bolzano-Weierstrass there is a convergent
                subsequence $k$ of $x$. Let $a$ be the limit of
                $x\circ{k}$.
                But then $f(x_{k_{n}})\rightarrow{f(a)}$.
                But $f(x_{k_{n}})\rightarrow\infty$,
                a contradiction. Therefore, etc.
            \end{proof}
            \begin{theorem}[Exreme Value Theorem]
                If $f:[a,b]\rightarrow\mathbb{R}$ is
                continuous, then there exists $c\in[a,b]$
                such that for all $x\in[a,b]$,
                $f(x)\leq{f(c)}$
            \end{theorem}
            \begin{proof}
                By the previous theorem,
                $\{f(x):x\in[a,b]\}$ is bounded.
                By completeness, there is a least upper
                bound. Let $s$ be such
                a bound. If $s$ is the least upper bound,
                then for all $n\in\mathbb{N}$, $s-\frac{1}{n}$
                is not the least upper bound. Thus, for
                all $n\in\mathbb{N}$ there is an $\alpha\in[a,b]$
                such that
                $s-\frac{1}{n}<f(\alpha)$. Invoking choice and
                choosing a sequence $x:\mathbb{N}\rightarrow[a,b]$ such
                that, for all $n\in\mathbb{N}$, $s-\frac{1}{n}<f(x_{n})$.
                But then $x$
                is a bounded sequence, and bounded
                sequences have a convergent subsequence.
                Let $a$ be the limit of this subsequence.
                From continuity,
                $f(a)=\lim_{n\rightarrow\infty}f(x_{k_{n}})$.
                But $s-\frac{1}{n}\leq{f(x_{k_{n}})}\leq{s}$,
                and therefore $f(x_{k_{n}})\rightarrow{s}$.
                Thus, $f(a)=s$.
            \end{proof}
            Much the way the intermediate value theorem can be
            generalized to say that the continuous image of
            connected sets is connected, the extreme value
            theorem can be generalized to say that the
            continuous image of a compact set is compact.
            The proof is rather easy, but again
            requires topology, so we'll return to this later.
            The requirement of these previous theorems on
            continuity is crucial. Without continuity, functions
            on $[a,b]$ need not be bounded and functions on
            $(a,b)$ can just ``jump,'' right over other points.
        \subsection{Sequences of Functions}
            \begin{definition}
                A sequence of functions from a
                set $X$ to a set $Y$ is a function
                $F:\mathbb{N}\times{X}\rightarrow{Y}$.
                We often write the image of
                $(n,x)\in\mathbb{N}\times{X}$ as
                $F(n,x)=F_{n}(x)$.
            \end{definition}
            \begin{definition}
                A point-wise convergent
                sequence of real-valued functions on
                a subset $S\subseteq\mathbb{R}$
                is a function
                $F:\mathbb{N}\times{S}\rightarrow\mathbb{R}$
                such that there exists a function
                $f:S\rightarrow\mathbb{R}$ such that,
                for all $\varepsilon>0$ and for all
                $x\in{S}$, there exists an
                $N\in\mathbb{N}$ such that, for all
                $n>N$, $|F_{n}(x)-f(x)|<\varepsilon$.
                That is:
                \begin{equation}
                    \label{eqn:FUNCTIONAL_ANALYSIS:POINTWISE_CONV_DEF}
                    \forall_{\varepsilon>0}
                    \forall_{x\in{S}}
                    \exists_{N\in\mathbb{N}}:
                    n>N\Rightarrow
                    |f(x)-F_{n}(x)|<\varepsilon
                \end{equation}
            \end{definition}
            That is, a sequence $F$ converges point-wise
            to $f$ if, for all $x\in\mathbb{R}$,
            $F_{n}(x)\rightarrow{f(x)}$.
            Uniform continuity requires that all of the
            points of the domain converge to $f(x)$ at
            the same speed. That is, given any $\varepsilon>0$
            there is an $N\in\mathbb{N}$ that works for
            all points. Point-Wise convergence may not
            have this property.
            \begin{definition}
                A uniformly convergent
                sequence of real-valued functions on
                a subset $S\subseteq\mathbb{R}$
                is a function
                $F:\mathbb{N}\times{S}\rightarrow\mathbb{R}$
                such that there exists a function
                $f:S\rightarrow\mathbb{R}$ such that,
                for all $\varepsilon>0$ there exists
                an $N\in\mathbb{N}$ such that, for all
                $x\in{S}$ and for all
                $n>N$, $|F_{n}(x)-f(x)|<\varepsilon$.
                That is:
                \begin{equation}
                    \label{eqn:FUNCTIONAL_ANALYSIS:UNIFORM_CONV_DEF}
                    \forall_{\varepsilon>0}
                    \exists_{N\in\mathbb{N}}
                    \forall_{x\in{S}}:
                    n>N\Rightarrow
                    |f(x)-F_{n}(x)|<\varepsilon
                \end{equation}
            \end{definition}
            That is, $F_{n}\rightarrow{f}$
            point-wise if for all $x$,
            $|F_{n}(x)-f(x)|\rightarrow{0}$ and
            $F_{n}\rightarrow{f}$ uniformly if
            $\sup\{|F_{n}(x)-f(x)|\}\rightarrow{0}$.
            It is worthwhile spotting the very subtle
            difference between
            expressions~\ref{eqn:FUNCTIONAL_ANALYSIS:POINTWISE_CONV_DEF}
            and~\ref{eqn:FUNCTIONAL_ANALYSIS:UNIFORM_CONV_DEF}.
            \begin{definition}
                A limit function on a subset
                $S\subseteq\mathbb{R}$ of a convergent sequence
                of real-valued functions
                $F:\mathbb{N}\times{S}\rightarrow\mathbb{R}$
                is a function $f:S\rightarrow\mathbb{R}$ such that,
                for all $x\in{S}$, $F_{n}(x)\rightarrow{f(x)}$.
            \end{definition}
            \begin{theorem}
                If $S\subseteq\mathbb{R}$, if
                $F:\mathbb{N}\times{S}\rightarrow\mathbb{R}$
                is a convergent sequence of real-valued function,
                and if $f,g:S\rightarrow\mathbb{R}$ are limit
                functions of $F$, then $f=g$.
            \end{theorem}
            \begin{proof}
                For suppose not. Suppose there is an
                $x\in{S}$ such that $f(x)\ne{g(x)}$.
                But $F_{n}(x)\rightarrow{f(x)}$ and
                $F_{n}(x)\rightarrow{g(x)}$. From the
                uniqueness of limits, $f(x)=g(x)$, 
                a contradiction. Therefore, etc.
            \end{proof}
            \begin{example}
                Let
                $F:\mathbb{N}\times[0,1]\rightarrow\mathbb{R}$
                be defined by $F_{n}(x)=nx\exp(-nx)$.
                $F_{n}(x)\rightarrow{0}$ for all $x\in[0,1]$,
                and therefore $F$ converges point-wise to zero.
                Note that $F_{n}'(x)=(n-n^{2}x)\exp(-nx)$.
                This has a zero at $x=n^{-1}$, so
                $F_{n}(x)$ has a maximum of $e^{-1}$. But then
                $\sup|F_{n}(x)-f(x)|=\sup|F_{n}(x)|=e^{-1}$.
                So $F_{n}(x)$ does not converge
                \textit{uniformly} to $0$. The
                convergence is only \textit{point-wise}.
            \end{example}
            \begin{example}
                Let $F_{n}(x)=n^{2}x\exp(-nx)$. Then
                $F_{n}(x)\rightarrow{0}$ for all
                $x\geq{0}$. But $F_{n}(x)$ has
                a maximum of $ne^{-1}$ at $x=n^{-1}$.
                Thus $F_{n}(n^{-1})\rightarrow\infty$.
                It is possible for a sequence
                of functions to converge point-wise to zero
                and for there to be a sequence such that
                $F_{n}(x_{n})\rightarrow\infty$. Uniform
                convergence does not allow this.
            \end{example}
            \begin{definition}
                A sequence of continuous real-valued functions
                on a subset $S\subseteq\mathbb{R}$ is a
                function
                $F:\mathbb{N}\times{S}\rightarrow\mathbb{R}$
                such that, for all $n\in\mathbb{N}$, the function
                $g:S\rightarrow\mathbb{R}$ defined by
                $g(x)=F_{n}(x)$ for all $x\in{S}$,
                is continuous.
            \end{definition}
            \begin{theorem}
                If $S\subseteq\mathbb{R}$ and if
                $F:\mathbb{N}\times{S}\rightarrow\mathbb{R}$
                is a uniformly convergent sequence of
                real-valued continuous functions and if
                ${f}$ is the limit function of $F$, then
                $f$ is continuous.
            \end{theorem}
            \begin{proof}
                For let $x\in{S}$ and let
                $\varepsilon>0$. As $F$ converges
                uniformly to $f$, there is an
                $N_{0}\in\mathbb{N}$ such that, for all
                $n>N_{0}$ and for all $x_{0}\in{S}$,
                $|F_{n}(x_{0})-f(x_{0})|<\varepsilon/3$.
                Let $N=N_{0}+1$.
                But, for all $n\in\mathbb{N}$,
                $F_{n}$ is a continuous function, and
                thus there is a $\delta>0$ such that,
                for all $x_{0}\in{S}$ such that
                $|x-x_{0}|<\delta$,
                $|F_{N}(x)-F_{N}(x_{0})|<\varepsilon/3$.
                But then, from the triangle inequality,
                for all $x_{0}\in{S}$ such that
                $|x-x_{0}|<\delta$:
                \begin{align}
                    \nonumber
                    |f(x)-f(x_{0})|&\leq
                    |f(x)-F_{N}(x)|
                    +|F_{N}(x)-F_{N}(x_{0})|
                    +|F_{N}(x_{0})-f(x_{0})|\\
                    &<\frac{\varepsilon}{3}+
                    \frac{\varepsilon}{3}+
                    \frac{\varepsilon}{3}
                    =\varepsilon
                \end{align}
            \end{proof}
            The word ``uniformly,'' is crucial.
            This theorem is not necessarily true of
            point-wise converging functions. Let $F$ be
            defined on $[0,1]$ as
            $F_{n}(x)=x^{n}$. Then $F$ converges to
            $0$ if $x\ne{1}$, and $1$ if $x=1$. Thus,
            the limit function is discontinuous. This is
            possible because the convergence is
            point-wise and not uniform.
            \begin{theorem}
                    \label{thm:funct:Weak_Weierstrass_%
                           Approx_Theorem}
                    If $f:[0,1]\rightarrow\mathbb{R}$
                    is continuous,
                    and if $f(0)=f(1)=0$,
                    then there is a sequence of polynomials
                    $F$ such that $F_{n}\rightarrow{f}$
                    uniformly on $[0,1]$.
                \end{theorem}
            \begin{proof}
                \begin{subequations}
                    Extend $f$ to be zero outside of $[0,1]$. Let
                    $Q_{n}(x)=c_{n}(1-x^{2})^{n}$ on $[-1,1]$,
                    and choose $c_{n}$ such that
                    $\int_{-1}^{1}Q_{n}(x)\diff{x}=1$. So we have:
                    \begin{align}
                        c_{n}\int_{-1}^{1}(1-x^{2})^{n}\diff{x}
                        &=2c_{n}\int_{0}^{1}(1-x^{2})^{n}\diff{x}\\
                        &=2c_{n}\int_{0}^{1}(1-x)^{n}(1+x)^{n}\diff{x}\\
                        &\geq{2c_{n}}\int_{0}^{1}(1-x)^{n}\diff{x}\\
                        &=\frac{2}{n+1}c_{n}
                    \end{align}
                    From this we have that $c_{n}\leq{n+1}$. Let
                    $f_{n}(x)=\int_{0}^{1}f(t)Q_{n}(x-t)\diff{x}$.
                    Then $f_{n}(x)$ is a polynomial. Note that
                    $f(t)Q_{n}(x-t)$ is roughly zero when $t$ differs
                    from $x$ and $n$ is large enough. So we have:
                    \begin{equation}
                        f_{n}(x)=
                        \int_{0}^{1}f(t)Q_{n}(x-t)\diff{t}
                        \approx{f(x)}\int_{0}^{1}Q_{n}(x-t)\diff{t}
                        =f(x)
                    \end{equation}
                    The remainder of the proof is to quantify this.
                    Since $f$ is zero outside of $[0,1]$, if
                    we let $s=t-x$, then:
                    \begin{align}
                        f_{n}(x)
                        &=\int_{-x}^{1-x}f(s+x)Q_{n}(s)\diff{s}\\
                        &=\int_{-1}^{1}f(s+x)Q_{n}(s)\diff{s}
                    \end{align}
                    Using this, we obtain:
                    \begin{align}
                        |f_{n}(x)-f(x)|
                        &=\bigg|\int_{-1}^{1}f(x+t)Q_{n}(t)\diff{t}
                        -\int_{-1}^{1}f(x)Q_{n}(t)\diff{t}\bigg|\\
                        &\leq\int_{-1}^{1}|f(x+t)-f(x)|Q_{n}(t)\diff{t}
                    \end{align}
                    This comes for the fact
                    that $\int_{-1}^{1}Q_{n}(t)=1$
                    and from the integral version of the triangle
                    inequaility.
                    Suppose $\varepsilon>0$. Since $f$ is continuous
                    on $[0,1]$, it is uniformly continuous. But
                    if $f$ is uniformly continuous then there exists
                    a $\delta>0$ such that
                    $|f(x+t)-f(x)|<\frac{\varepsilon}{2}$ for all
                    $t<\delta$. So we have:
                    \begin{equation}
                        \begin{split}
                        |f_{n}(x)-f(x)|
                        &\leq
                        \int_{-1}^{-\delta}|f(x+t)-f(x)|
                        Q_{n}(t)\diff{t}\\
                        &+\int_{-\delta}^{\delta}
                        |f(x+t)-f(x)|Q_{n}(t)\diff{t}\\
                        &+\int_{\delta}^{1}
                        |f(x+t)-f(x)|Q_{n}(t)\diff{t}
                        \end{split}
                    \end{equation}
                    But $f$ is continuous on a closed and bounded
                    set and therefore $f$ is bounded. Let $M$ be
                    such a bound. Then $|f(x+t)-f(x)|\leq{2M}$.
                    We have:
                    \begin{equation}
                        |f_{n}(x)-f(x)|\leq
                        2M\int_{-1}^{-\delta}Q_{n}(t)\diff{t}
                        +\frac{\varepsilon}{2}
                        \int_{-\delta}^{\delta}Q_{n}(t)\diff{t}
                        +2M\int_{\delta}^{1}Q_{n}(t)\diff{t}
                    \end{equation}
                    But for all $t\in[-1,-\delta]$,
                    $Q_{n}(t)\leq{Q_{n}(-\delta)}$. Similarly for
                    $t$ in $[\delta,1]$. Since $Q_{n}(t)$
                    is an even function:
                    \begin{equation}
                        |f_{n}(x)-f(x)|\leq
                        4MQ_{n}(\delta)+
                        \frac{\varepsilon}{2}
                        \int_{-1}^{1}Q_{n}(t)\diff{t}
                        =4MQ_{n}(\delta)+\frac{\varepsilon}{2}
                    \end{equation}
                    But since $\delta>0$, $Q_{n}(\delta)\rightarrow0$.
                    Therefore, there is an $N\in\mathbb{N}$ such that
                    for all $n>N$,
                    $|Q_{n}(\delta)|<\frac{\varepsilon}{8M}$.
                    But then $4MQ_{n}(\delta)<\frac{\varepsilon}{2}$.
                    Therefore, etc.
                \end{subequations}
            \end{proof}
            Another way to put this is that if $f$ is continuous
            on $[a,b]$ and if $\varepsilon>0$, then there is
            a polynomial $P$ such that for all $x\in[0,1]$,
            $|f(x)-P(x)|<\varepsilon$. There is a generalization
            of this and the set of functions need not be
            polynomials. The set needs to be closed
            under addition, multiplication, and scalar
            multiplication, it must separate points,
            and must not take every point to zero.
            This is the Stone-Weierstrass theorem.
            This shows that continuous functions on compact sets
            can be approximated arbitrarily well by polynomials.
            Furthermore, any continuous function on a compact
            set can be approximated arbitrarily well by
            polynomials with rational coefficients. To see this,
            let $f[0,1]\rightarrow\mathbb{R}$ be continuous,
            and let $\varepsilon>0$. Then there is a
            polynomial $P$ such that
            $\sup|P(x)-f(x)|<\varepsilon/2$.
            Suppose $P$ is of degree $n$.
            As $\mathbb{Q}$ is dense
            in $\mathbb{R}$, for each coefficient
            $c_{k}$ of $P$ there is a $d_{k}\in\mathbb{Q}$
            such that
            $|c_{k}-d_{k}|<\frac{\varepsilon}{2n}$.
            Let $Q(x)=\sum{d_{k}x^{k}}$. Then:
            \begin{equation}
                    |P(x)-Q(x)|\leq
                    \sum_{k=0}^{n}|c_{k}-d_{k}||x|^{n}
                    <\frac{\varepsilon}{2}
                \end{equation}
            Thus, by the triangle inequality:
            $\sup|Q(x)-f(x)|<\varepsilon$. A set is called
            \textit{separable} if it
            contains a countable dense subset. $\mathbb{R}$
            is separable since $\mathbb{Q}$ is dense in
            $\mathbb{R}$, and $\mathbb{Q}$ is countable.
            The set of all continuous functions from
            $[0,1]$ to $\mathbb{R}$, which we label as
            $C(I,\mathbb{R})$, is also separable.
            Since any continuous function can be approximated
            arbitrarily well by a polynomial with rational
            coefficients, we can say the set of polynomials
            with rational coefficients is \textit{dense} in
            $C(I,\mathbb{R})$. But the set of polynomials with
            rational coefficients is countable. For all
            $N\in\mathbb{N}$, define $P_{N}$ as:
            \begin{equation}
                    P_{N}=\Big\{\sum_{k=0}^{N}
                    q_{k}x^{k}:q_{k}\in\mathbb{Q},
                    q_{N}\ne{0}\Big\}
                \end{equation}
            This is the set of all rational polynomials
            of degree $N$. It is countable since there is
            a one-to-one correspondence with
            the set $\mathbb{Q}^{N}$, and $\mathbb{Q}^{N}$
            is countable for all $N\in\mathbb{N}$. But the
            set of all rational polynomials is simply the
            union over all $P_{N}$. And the countable union
            of countably many disjoint sets is countable.
            Therefore, the set of all polynomials with
            rational coefficients is countable. Thus
            $C(I,\mathbb{R})$ is \textit{separable}. We need
            to be careful when we say \textit{dense} and
            \textit{separable}, for we are implicitly speaking
            of some sort of notion of \textit{closeness} on the
            sets. This all comes from the notion of
            \textit{metrics} and \textit{metric spaces},
            and the more general \textit{topological space}.
            \begin{theorem}
                \label{thm:Funct:Weierstrass_%
                       Approx_on_unit_interval}
                If $f:[0,1]\rightarrow\mathbb{R}$ is
                continuous, then there is a sequence
                of polynomials $F$ such that
                $F_{n}\rightarrow{f}$ uniformly on $[0,1]$.
            \end{theorem}
            \begin{proof}
                If $f:[0,1]\rightarrow\mathbb{R}$ be
                continuous, let
                $g(x)=xf(1)+(1-x)f(0)$. Then
                $h(x)=f(x)-g(x)$ is a continuous function
                such that $h(0)=h(1)=0$ and thus by
                Thm.~\ref{thm:funct:Weak_Weierstrass_%
                          Approx_Theorem}
                there is a sequence of polynomials
                $P_{n}(x)$ such that
                $P_{n}(x)\rightarrow{h(x)}$
                uniformly on $[a,b]$.
                But $g(x)$ is a polynomial and
                $f(x)=h(x)+g(x)$. Therefore
                $F_{n}(x)=P_{n}(x)+g(x)$ is a sequence
                of polynomials and
                $F_{n}(x)\rightarrow{f(x)}$
                uniformly on $[0,1]$.
            \end{proof}
            \begin{theorem}[Weierstrass Approximation Theorem]
                If $f:[a,b]\rightarrow\mathbb{R}$ is a
                continuous function, then there is a sequence
                of polynomials $P$ such that
                $P_{n}\rightarrow{f}$ uniformly.
            \end{theorem}
            \begin{proof}
                If $f:[a,b]\rightarrow\mathbb{R}$ is
                continuous, define
                $g:[0,1]\rightarrow\mathbb{R}$ by
                $g(x)=f(\frac{x-a}{b-a})$. Then, since
                the composition of continuous functions
                is continuous, $g$ is a continuou function
                on $[0,1]$. But by the Weierstrass
                approximation theorem there is a sequence
                of polynomials $P_{n}(x)$ such that
                $P_{n}(x)\rightarrow{g(x)}$. Let
                $F_{n}(x)=P_{n}(bx+(1-x)a)$. Then
                $F_{n}(x)$ is a sequence of polynomials
                on $[a,b]$, and $F_{n}(x)\rightarrow{f(x)}$.
            \end{proof}
            An application of this is in the uniform
            approximation of continuous periodic functions
            by Cosines.
        \begin{theorem}
            If $f\in{C[0,\pi]}$ and $\varepsilon>0$,
            then there exists
            $a_{0},\hdots,a_{n}\in\mathbb{R}$ such that, for all
            $x\in[0,\pi]$:
            \begin{equation}
                |f(x)-\sum_{k=0}^{n}a_{k}\cos(kx)|
                <\varepsilon
            \end{equation}
        \end{theorem}
        \begin{proof}
            \begin{subequations}
                For $\cos(x)$ is a bijective function on
                the interval $[0,\pi]$. Thus we can consider the
                function $f(\cos^{-1}(x))$. But since $\cos(x)$ is
                continuous on $[0,\pi]$, $\cos^{-1}(x)$ is
                continuous on $[-1,1]$. And the composition of
                continuous functions is continuous. So
                $f(\cos^{-1}(x))$ is continuous. By the
                Weierstrass Approximation Theorem, there is a
                sequence of polynomials $P$ such that
                $P_{n}(x)\rightarrow{f(\cos^{-1}(x))}$. But then
                $P_{n}(\cos(x))\rightarrow{f(x)}$. But $P_{n}(x)$
                is a polynomial of the form
                $\sum_{k=0}^{n}a_{k}x^{k}$, and thus:
                \begin{equation}
                    P_{n}(\cos(x))=\sum_{k=0}^{n}a_{k}\cos^{k}(x)    
                \end{equation}
                It now suffices to show that
                $\cos^{k}(x)=\sum_{m=0}^{N}c_{m}\cos(mx)$ for
                suitable $c_{m}$. We prove by induction.
                The base case is trivial. Suppose it holds
                for some $k\in\mathbb{N}$.
                Then:
                \begin{equation}
                    \cos^{k+1}(x)=\cos(x)\cos^{k}(x)
                    =\cos(x)\sum_{k=0}^{N}c_{k}\cos(kx)
                \end{equation}
                Note that
                $\cos(x)\cos(kx)%
                 =\frac{1}{2}\cos((k-1)x)+\frac{1}{2}\cos((k+1)x)$.
                So we have:
                \begin{equation}
                    \cos^{k+1}(x)
                    =\frac{1}{2}\sum_{k=0}^{N}c_{k}
                    \Big(
                        \cos\big((k+1)x\big)+
                        \cos\big((k-1)x\big)
                    \Big)
                \end{equation}
                This completes the theorem.
            \end{subequations}
        \end{proof}
        \subsection{Inequalities}
            \begin{definition}
                H\"{o}lder Conjugates are non-zero real numbers
                $p$, $q\in\mathbb{R}$ where:
                \begin{equation}
                    p^{-1}+q^{-1}=1
                \end{equation}
            \end{definition}
            \begin{theorem}[Young's Inequality]
                If $x,y\geq{0}$, $p>1$, and if $p$ and $q$
                are H\"{o}lder Conjugates, then:
                \begin{equation}
                    xy\leq{\frac{1}{p}x^{p}+\frac{1}{q}y^{q}}
                \end{equation}
            \end{theorem}
            \begin{proof}
                If $x$ or $y$ are zero, then we are done.
                Suppose $x,y>0$. Let $t=p^{-1}$. As $p$ and
                $q$ are H\"{o}lder Conjugates,
                $1-t=q^{-1}$. But then, as $p>1$,
                $t$ and $1-t$ are positive, and thus:
                \begin{equation}
                    \ln(tx^{p}+(1-t)y^{q})
                    \geq{t}\ln(x^{p})+(1-t)\ln(y^{q})
                    =\ln(x)+\ln(y)=\ln(xy)
                \end{equation}
                Where the inequality comes from the fact
                that $\ln$ is a concave function.
                Exponentiating completes the proof.
            \end{proof}
            There is another way to prove this without using
            the concavity of the logarithmic function.
            Let $y>0$ and define $f:(0,\infty)$ by:
            \begin{subequations}
                \begin{equation}
                    f(x)=p^{-1}x^{p}-xy
                \end{equation}
                Then, differentiating, we have:
                \begin{equation}
                    f'(x)=x^{p-1}-y
                \end{equation}
                This has an extremum at
                $x=y^{\frac{1}{p-1}}$. We also have:
                \begin{equation}
                    f''(x)=(p-1)x^{p-2}
                \end{equation}
                And this is positive for all $x\in(0,\infty)$,
                and thus $y^{\frac{1}{p-1}}$ is a global minimum.
                Using the fact that $p$ and $q$ are
                H\"{o}lder Conjugates, we have:
                \begin{equation}
                    \frac{1}{p-1}=q-1
                \end{equation}
                Applying some algebra obtains the result. We
                also see that equality happens when
                $x^{p}=y^{q}$.
            \end{subequations}
            For $p=q=2$ this is easy, for:
            \begin{equation*}
                0\leq\frac{(x-y)^{2}}{2}=
                \frac{x^{2}+y^{2}}{2}-xy
            \end{equation*}
            Using this, we can prove the Peter-Paul inequality:
            \begin{theorem}[Peter-Paul Inequality]
                If $x,y\in\mathbb{R}$ and $\varepsilon>0$,
                then:
                \begin{equation}
                    ab\leq
                    \frac{x^{2}}{2\varepsilon}+
                    \frac{\varepsilon{y}^{2}}{2}
                \end{equation}
            \end{theorem}
            \begin{proof}
                For we have:
                \begin{equation}
                    0\leq\Big(\frac{x}{\sqrt{\varepsilon}}-
                    \varepsilon{y}\Big)^{2}
                    =\frac{x^{2}}{\varepsilon}-2xy+
                    \varepsilon{y}^{2}
                \end{equation}
                Bringing $2xy$ to the left side and dividing
                by 2 completes the proof.
            \end{proof}
            \begin{theorem}[H\"{o}lder's Inequality]
                If $a:\mathbb{N}\rightarrow\mathbb{R}$ and
                $b:\mathbb{N}\rightarrow\mathbb{R}$
                are nonnegative
                sequences, $p>1$, and if $p$ and $q$ are
                H\"{o}lder Conjugates, then:
                \begin{equation}
                    \sum_{n=1}^{\infty}a_{n}b_{n}\leq
                    \bigg(\sum_{n=1}^{\infty}a_{n}^{p}\bigg)^{1/p}
                    \bigg(\sum_{n=1}^{\infty}b_{n}^{q}\bigg)^{1/q}
                \end{equation}
            \end{theorem}
            \begin{proof}
                If $\sum_{n=1}^{\infty}a_{n}b_{n}=0$, then
                $a$ and $b$ are both the zero sequence and
                we are done. Suppose the sum is positive.
                If either $\sum_{n=1}^{\infty}a_{n}^{p}$
                or $\sum_{n=1}^{\infty}b_{n}^{q}$ diverges,
                then the must diverge to $+\infty$ since
                $a$ and $b$ are non-negative sequences, and
                we would again be done. Suppose they both
                converge. Define the following:
                \par\hfill\par
                \begin{subequations}
                    \begin{minipage}[b]{0.49\textwidth}
                        \begin{equation}
                            A=\bigg(\sum_{n=1}^{\infty}
                                a_{n}^{p}\bigg)^{1/p}
                        \end{equation}
                    \end{minipage}
                    \hfill
                    \begin{minipage}[b]{0.49\textwidth}
                        \begin{equation}
                            B=\bigg(\sum_{n=1}^{\infty}
                                b_{n}^{q}\bigg)^{1/q}
                        \end{equation}
                    \end{minipage}
                    \par\hfill\par
                    Then, by Young's inequality,
                    for all $n\in\mathbb{N}$:
                    \begin{equation}
                        \frac{a_{n}b_{n}}{AB}
                        \leq
                        \frac{1}{p}\Big(\frac{a_{n}}{A}\Big)^{p}+
                        \frac{1}{q}\Big(\frac{b_{n}}{B}\Big)^{q}
                    \end{equation}
                    Summing both sides, we have:
                    \begin{equation}
                        \frac{1}{AB}\sum_{n=1}^{\infty}a_{n}b_{n}
                        \leq\frac{1}{p}+\frac{1}{q}=1
                    \end{equation}
                    As $p$ and $q$ are H\"{o}lder Conjugates.
                    Multiplying by $AB$ proves the result.
                \end{subequations}
            \end{proof}
            When $p=q=2$ this is often
            called the Cauchy-Schwartz inequality.
            That is,
            $|\mathbf{a}\cdot\mathbf{b}|%
             \leq\norm{\mathbf{a}}\norm{\mathbf{b}}$.
            It holds for the integrals of continuous
            functions, as well as for sequences.
            \begin{theorem}[Minkowski's Inequality]
                If $a:\mathbb{N}\rightarrow\mathbb{R}$
                and $b:\mathbb{N}\rightarrow\mathbb{R}$ are
                non-negative sequences, and if $p>1$,
                then:
                \begin{equation*}
                    \bigg(
                        \sum_{n=1}^{\infty}(a_{n}+b_{n})
                    \bigg)^{1/p}
                    \leq
                    \bigg(
                        \sum_{n=1}^{\infty}a_{n}^{p}
                    \bigg)^{1/p}
                    +
                    \bigg(
                        \sum_{n=1}^{\infty}b_{n}^{p}
                    \bigg)^{1/p}
                \end{equation*}
            \end{theorem}
    \section{Notes from Rosenlicht}
        \subsection{Sets}
            Give a function $f:X\rightarrow{Y}$, the dinstinction
            between the image of a subset $S\subseteq{X}$ and a
            point $x\in{X}$ is:
            \begin{equation}
                f(\{x\})=\{f(x)\}
            \end{equation}
            Similarly for the pre-image:
            \begin{equation}
                \{f^{\minus{1}}(y)\}=f^{\minus{1}}(\{y\})
            \end{equation}
            One definition of an infinite set is that it contains
            a bijection between itself and a proper subset. Such
            sets are called Dedekind infinite, and countable choice is
            needed here. The following are true:
            \begin{align}
                (A^{C})^{C}&=A\\
                A\cup{A}&=A\cap{A}=A\cup\emptyset=A\\
                A\cap\emptyset&=\emptyset\\
                A\times\emptyset&=\emptyset
            \end{align}
            In addition, there are De Morgan's laws and the distributive
            laws. Some more identities:
            \begin{align}
                (A\setminus{B})\cap{C}&=(A\cap{C})\setminus{B}\\
                (A\cup{B})\setminus(A\cap{B})
                    &=(A\setminus{B})\cup(B\setminus{A})\\
                (A\setminus(B\setminus{C}))
                    &=(A\setminus{B})\cup(A\cap{B}\cap{C})\\
                (A\setminus{B})\times{C}
                    &=(A\times{C})\setminus(B\times{C})
            \end{align}
            Given any collection of sets $X_{i}$, $i\in{I}$, and a
            set $B$, we have:
            \begin{align}
                B\cap\Big(\bigcup_{i\in{I}}A_{i}\Big)
                    &=\bigcup_{i\in{I}}\Big(B\cap{A_{i}}\Big)
            \end{align}
            Composition is a commutative operation. That is, given
            $f:X\rightarrow{Y}$, $g:Y\rightarrow{Z}$, and
            $h:Z\rightarrow{W}$, we have:
            \begin{equation}
                h\circ(g\circ{f})=(h\circ{h})\circ{f}
            \end{equation}
            The following is also true of functions:
            \begin{subequations}
                \begin{align}
                    f(A\cup{B})&=f(A)\cup{f}(B)\\
                    f(A\cap{B})&\subseteq{f}(A)\cap{f}(B)\\
                    f^{\minus{1}}(A\cup{B})
                        &=f^{\minus{1}}(A)\cup{f}^{\minus{1}}(B)\\
                    f^{\minus{1}}(A\cap{B})
                        &=f^{\minus{1}}(A)\cap{f}^{\minus{1}}(B)\\
                    A\subseteq{f}^{\minus{1}}(f(A))\\
                    f(f^{\minus{1}}(A)\subseteq{A}
                \end{align}
            \end{subequations}
            \begin{theorem}
                If $f:X\rightarrow{Y}$ is injective, then:
                \begin{subequations}
                    \begin{align}
                        f^{\minus{1}}(f(A))&=A\\
                        f(A\cap{B})&=f(A)\cap{f}(B)
                    \end{align}
                \end{subequations}
            \end{theorem}
            \begin{theorem}
                If $f:X\rightarrow{Y}$ is surjective, then:
                \begin{equation}
                    f(f^{\minus{1}}(A))=A
                \end{equation}
            \end{theorem}
        \subsection{The Real Number System}
            The real numbers are a set $\mathbb{R}$ with several
            properties. These properties make $\mathbb{R}$ a
            complete ordered field, and indeed the only complete
            ordered field. That is, the real numbers are unique
            up to \textit{isomorphism}. There are two functions
            $+,\cdot:\mathbb{R}^{2}\rightarrow\mathbb{R}$, called
            addition and multiplication, respectively, that satisfy
            the following \textit{field axioms}:
            \begin{align}
                a+b&=b+a&
                a\cdot{b}&=b\cdot{a}
                \tag{Commutativity}\\
                a+(b+c)&=(a+b)+c&
                a\cdot(b\cdot{c})&=(a\cdot{b})\cdot{c}
                \tag{Associativity}\\
                a\cdot(b+c)&=a\cdot{b}+a\cdot{c}
                \tag{Distributive Law}\\
                \exists_{0\in\mathbb{R}}:0+a&=a&
                \exists_{1\in\mathbb{R}}:a\cdot{1}&=a
                \tag{Neutral Elements}\\
                \forall_{a\in\mathbb{R}}\exists_{b\in\mathbb{R}}:
                a+b&=0&
                \forall_{a\in\mathbb{R},a\ne{0}}
                \exists_{a^{\minus{1}}}:
                a\cdot{a}^{\minus{1}}&=1
                \tag{Inverse Elements}
            \end{align}
            By inductively using the associative laws and the
            commutative laws, we see that adding $n$ elements
            does not depend on the order in which they are
            added. Similarly for multiplication. For a general
            field, we write $(F,+,\cdot)$.
            \begin{theorem}
                If $(F,+,\cdot)$ is a field, and if $a\in{F}$, then
                the additive inverse of $a$ is unique.
            \end{theorem}
            \begin{proof}
                For suppose $b$ and $b'$ are additive inverses. Then:
                \begin{equation}
                    b=b+0=b+(a+b')=(b+a)+b'=0+b'=b'
                \end{equation}
                And therefore $b$ is unique.
            \end{proof}
            We denote the additive inverse of an element $a$ by
            writing $\minus{a}$.
            \begin{theorem}
                If $(F,+,\cdot)$ is a field, if $a,b\in{F}$, then
                there is a unique $x\in{F}$ such that
                $x+a=b$.
            \end{theorem}
            \begin{proof}
                For let $x=a-b$. Then:
                \begin{equation}
                    x+a=(b-a)+a
                    =b+(-a+a)
                    =b+0
                    =b
                \end{equation}
                Moreover, of $x'$ is a solution, then:
                \begin{equation}
                    x'=x'+0=x'+(a+(\minus{a}))=
                    (x'+a)+(\minus{a})=b+(\minus{a})=x
                \end{equation}
                Thus, $x'=x$.
            \end{proof}
            Instead of writing $b+(\minus{a})$, we
            denote this by $b-a$. This new operation is called
            subtraction. Note that it is not commutative, nor
            is it associative. Indeed, for any $a,b\in\mathbb{R}$,
            suppose $a-b=b-a$, and let $y=a-b$. Then we have that
            $y=\minus{y}$, and thus $y+y=2y=0$. This is only possible
            in $\mathbb{R}$ if $y=0$, and thus we'd require that
            $a=b$. So subtraction is not commutative in $\mathbb{R}$.
            There are fields such that $y+y=0$ and such that
            $y\ne{0}$, but such fields can't have a notion of
            \textit{order} on them. We'll discuss these later.
            Note that the notion is not associative either. Again,
            let $a=2$ and $b=c=1$. Then $a-(b-c)=2$, but
            $(a-b)-c=0$. Again we come to the conclusion that either
            $2=0$, or subtraction is not associative. In an ordered
            field, which is what $\mathbb{R}$ is, we cannot have
            $2=0$. In finite fields, this is possible.
            \begin{theorem}
                If $(F,+,\cdot)$ is a field and if $a\in{F}$
                is non-zero, then the multiplicative inverse
                of $a$ is unique.
            \end{theorem}
            \begin{proof}
                For suppose $b$ and $b'$ are multiplicative inverses
                of $a$. Then:
                \begin{equation}
                    b=b\cdot{1}=b\cdot(a\cdot{b}')=
                    (b\cdot{a})\cdot{b}'=1\cdot{b}'=b'
                \end{equation}
                And therefore $b$ is unique.
            \end{proof}
            We write the multiplicative inverse of a non-zero element
            by $a^{\minus{1}}$.
            \begin{theorem}
                If $(F,+,\cdot)$ is a field, if $a,b\in{F}$, and if
                $a\ne{0}$, then there is a unique $x\in{F}$ such that
                $x\cdot{a}=b$.
            \end{theorem}
            \begin{proof}
                For let $x=b\cdot{a}^{\minus{1}}$. Then:
                \begin{equation}
                    x\cdot{a}=(b\cdot{a}^{\minus{1}})=
                    b\cdot(a^{\minus{1}}\cdot{a})=
                    b\cdot{1}=b
                \end{equation}
                Moreoever, if $x'$ is a solution, then:
                \begin{equation}
                    x'=x'\cdot{1}=x'\cdot(a\cdot{a^{\minus{1}}})
                    =(x'\cdot{a})\cdot{a^{\minus{1}}}=
                    b\cdot{a}^{\minus{1}}=x
                \end{equation}
                Thus, $x'=x$.
            \end{proof}
            We define division by non-zero numbers by writing
            $\frac{a}{b}=a\cdot{b}^{\minus{1}}$. Other symbols are
            used for this, like $a\div{b}$, or simply $a/b$. Similar
            to subtraction, division is neither commutative nor
            associative.
            \begin{theorem}
                If $(F,+,\cdot)$ is a field, if $a,b,c\in{F}$, and if
                $a+c=b+c$, then $a=b$.
            \end{theorem}
            \begin{proof}
                For:
                \begin{equation}
                    a=a+0=a+(c-c)=(a+c)-c=(b+c)-c=b+(c-c)=0
                \end{equation}
                Therefore, etc.
            \end{proof}
            \begin{theorem}
                If $(F,+,\cdot)$ is a field, $a,b,c\in{F}$, if
                $c\ne{0}$, and if $a\cdot{c}=b\cdot{c}$, then
                $a=b$.
            \end{theorem}
            \begin{proof}
                For:
                \begin{equation}
                    a=a\cdot{1}=a\cdot(c\cdot{c}^{\minus{1}})=
                    (a\cdot{c})\cdot{c}^{\minus{1}}=
                    (b\cdot{c})\cdot{c}^{\minus{1}}=
                    b\cdot(c\cdot{c}^{\minus{1}})=
                    b\cdot{1}=b
                \end{equation}
                Therefore, etc.
            \end{proof}
            \begin{theorem}
                If $(F,\cdot,+)$ is a field, and if $a\in{F}$, then
                $a\cdot{0}=0$.
            \end{theorem}
            \begin{proof}
                For:
                \begin{equation}
                    a\cdot{0}+a\cdot{0}=a\cdot(0+0)=
                    a\cdot{0}=a\cdot{0}+0
                \end{equation}
                And therefore from the cancellation laws,
                $a\cdot{0}=0$.
            \end{proof}
            \begin{theorem}
                If $(F,+,\cdot)$ is a field, and $a\in{F}$, then
                $\minus{a}=(\minus{1})\cdot{a}$
            \end{theorem}
            \begin{proof}
                For:
                \begin{equation}
                    (\minus{1})\cdot{a}+a=
                    (\minus{1}+1)\cdot{a}=
                    0\cdot{a}=0
                \end{equation}
                From the uniqueness of inverses,
                $\minus{a}=(\minus{1})\cdot{a}$.
            \end{proof}
            \begin{theorem}
                If $(F,+,\cdot)$ is a field and $a\in{F}$, then
                $\minus(\minus{a})=a$.
            \end{theorem}
            \begin{proof}
                For:
                \begin{equation}
                    \minus(\minus{a})+(\minus{a})=
                    (\minus{1})\cdot(\minus{a})+(\minus{a})
                    =(\minus{1}+1)\cdot(\minus{a})
                    =0\cdot(\minus{a})=0
                \end{equation}
                From the uniqueness of inverses, etc.
            \end{proof}
            \begin{theorem}
                If $(F,+,\cdot)$ is a field, and if $a,b\in{F}$ are
                non-zero, then $(a\cdot{b})^{\minus{1}}=%
                                b^{\minus{1}}\cdot{a}^{\minus{1}}$.
            \end{theorem}
            \begin{proof}
                For:
                \begin{equation}
                    (a\cdot{b})
                    \cdot(b^{\minus{1}}\cdot{a}^{\minus{1}})
                    =a\cdot
                    (b\cdot{b}^{\minus{1}})\cdot{a}^{\minus{1}}
                    =a\cdot{1}\cdot{a}^{\minus{1}}=
                    a\cdot{a}^{\minus{1}}=1
                \end{equation}
                From the uniqueness of inverses, etc.
            \end{proof}
            \begin{theorem}
                If $(F,+,\cdot)$ is a field, $a\in{F}$ is non-zero,
                then $(a^{\minus{1}})^{\minus{1}}=a$.
            \end{theorem}
            \begin{proof}
                For:
                \begin{equation}
                    (a^{\minus{1}})^{\minus{1}}\cdot{a}^{\minus{1}}=
                    (a\cdot{a}^{\minus{1}})^{\minus{1}}=
                    1^{\minus{1}}=1
                \end{equation}
                From uniquness, etc.
            \end{proof}
            \begin{theorem}
                If $(F,+,\cdot)$ is a field, and $a,b,c,d\in{F}$, and
                if $b,d\ne{0}$, then:
                \begin{equation}
                    (a\cdot{b}^{\minus{1}})+(c\cdot{d}^{\minus{1}})=
                    (a\cdot{d}+b\cdot{c})\cdot(b\cdot{d})^{\minus{1}}
                    =\frac{ad+bc}{bd}
                \end{equation}
            \end{theorem}
            As stated before, the axioms of a field are not enough
            to uniquely define the real numbers. Indeed, the rational
            numbers $\mathbb{Q}$ define a field, as do the complex
            numbers $\mathbb{C}$. To see a finite field, consider the
            set $\mathbb{F}_{2}=\{0,1\}$, and consider the following
            arithmetic:
            \par
            \begin{minipage}[b]{0.49\textwidth}
                \centering
                \begin{table}[H]
                    \centering
                    \captionsetup{type=table}
                    \begin{tabular}{c|cc}
                        $+$&0&1\\
                        \hline
                        0&0&1\\
                        1&1&0
                    \end{tabular}
                    \caption{Addition in $\mathbb{F}_{2}$}
                    \label{tab:Real_Analysis_Add_in_F_2_Field}
                \end{table}
            \end{minipage}
            \hfill
            \begin{minipage}[b]{0.49\textwidth}
                \begin{table}[H]
                    \centering
                    \captionsetup{type=table}
                    \begin{tabular}{c|cc}
                        $\cdot$&0&1\\
                        \hline
                        0&0&0\\
                        1&0&1
                    \end{tabular}
                    \caption{Multiplication in $\mathbb{F}_{2}$}
                    \label{tab:Real_Analysis_Mult_in_F_2_Field}
                \end{table}
            \end{minipage}
            Then $(F,+,\cdot)$ is a field. It's a very strange field,
            since we have $1+1=0$, but alas it satisfies all of the
            properties of a field, and all of the theorem's we have
            proved still apply. Interesting, it is the only field
            with two elements. We have no choice in deciding what
            $a\cdot{b}$ means in the field, since multiplication by
            zero must give zero, and multiplication by one must give
            back the original number. Similarly for addition.
            Adding zero must not change anything, and so all
            we are left with is deciding what $1+1$ equals. But
            to be a field, there must be an additive inverse
            element. Thus we are forced to set $1+1=0$. There is
            also a field with three elements. For let
            $\mathbb{F}_{3}=\{0,1,2\}$ and define:
            \par\hfill\par
            \begin{minipage}[b]{0.49\textwidth}
                \centering
                \begin{table}[H]
                    \centering
                    \captionsetup{type=table}
                    \begin{tabular}{c|ccc}
                        $+$&0&1&2\\
                        \hline
                        0&0&1&2\\
                        1&1&2&0\\
                        2&2&0&1
                    \end{tabular}
                    \caption{Addition in $\mathbb{F}_{3}$}
                    \label{tab:Real_Analysis_Add_in_F_3_Field}
                \end{table}
            \end{minipage}
            \hfill
            \begin{minipage}[b]{0.49\textwidth}
                \begin{table}[H]
                    \centering
                    \captionsetup{type=table}
                    \begin{tabular}{c|ccc}
                        $\cdot$&0&1&2\\
                        \hline
                        0&0&0&0\\
                        1&0&1&2\\
                        2&0&2&1
                    \end{tabular}
                    \caption{Multiplication in $\mathbb{F}_{3}$}
                    \label{tab:Real_Analysis_Mult_in_F_3_Field}
                \end{table}
            \end{minipage}
            Intuition tells us that $1+1>1>0$, and thus $1+1$ cannot
            be equal to zero. Thus, to exclude finite fields we
            need to introduce the notion of order.
            \begin{enumerate}
                \item There is a subset $\mathbb{R}^{+}$
                      of $\mathbb{R}$ such that, for all
                      $a,b\in\mathbb{R}^{+}$, we
                      have $a\cdot{b}\in\mathbb{R}^{+}$ and
                      $a+b\in\mathbb{R}^{+}$.
                \item For all $a\in\mathbb{R}$, one and only one of
                      the following statements is true:
                      \begin{itemize}
                          \item $a\in\mathbb{R}^{+}$
                          \item $a=0$
                          \item $\minus{a}\in\mathbb{R}^{+}$
                      \end{itemize}
            \end{enumerate}
            $\mathbb{R}^{+}$ is called the set of positive numbers,
            and the elements such that $\minus{a}\in\mathbb{R}^{+}$
            are called negative. We define less than by writing
            $a<b$ if $b-a\in\mathbb{R}^{+}$. Similarly, we define
            greater than by writing $a>b$ is $a-b\in\mathbb{R}^{+}$.
            The less than or equal to and greater than or equal to
            symbols, denoted $\leq$ and $\geq$, respectively,
            are such that $a\leq{b}$ if $a<b$ or $a=b$, and
            similarly $a\geq{b}$ if $a>b$ or $a=b$. This defines
            $\mathbb{R}$ to be on ordered field.
            \begin{theorem}
                If $a,b\in\mathbb{R}$, then either $a=b$, $a<b$, or
                $a>b$.
            \end{theorem}
            \begin{proof}
                For either $a-b\in\mathbb{R}^{+}$, $a-b=0$, or
                $\minus(a-b)\in\mathbb{R}^{+}$. If
                $a-b\in\mathbb{R}^{+}$, then $a>b$. If $a-b=0$, then
                $a=b$. Finally, if $\minus(a-b)\in\mathbb{R}^{+}$,
                then $b-a\in\mathbb{R}^{+}$, and thus $b>a$.
            \end{proof}
            \begin{theorem}
                If $a,b,c\in\mathbb{R}$, if $a<b$, and if $b<c$, then
                $a<c$.
            \end{theorem}
            \begin{proof}
                For if $a<b$, then $b-a\in\mathbb{R}^{+}$. But if
                $b<c$, then $c-b\in\mathbb{R}^{+}$. But then:
                \begin{equation}
                    c-a=(c-b)+(b-a)\in\mathbb{R}^{+}
                \end{equation}
                Therefore, etc.
            \end{proof}
            \begin{theorem}
                If $a,b,c,d\in\mathbb{R}$, if $a<b$, and if
                $c\leq{d}$, then $a+c<b+d$.
            \end{theorem}
            \begin{proof}
                If $a<b$, then $b-a\in\mathbb{R}^{+}$. If $c\leq{d}$,
                then either $d-c\in\mathbb{R}^{+}$, or $d-c=0$. Thus:
                \begin{equation}
                    (b+d)-(a+c)=(b-a)+(d-c)\in\mathbb{R}^{+}
                \end{equation}
                Therefore, etc.
            \end{proof}
            \begin{theorem}
                If $a,b,c,d\in\mathbb{R}^{+}$, if $a<b$, and if
                $c\leq{d}$, then $a\cdot{c}<b\cdot{d}$.
            \end{theorem}
            \begin{proof}
                For if $a<b$, then $b-a\in\mathbb{R}^{+}$. But if
                $c\leq{d}$, then $d-c\in\mathbb{R}^{+}$, or
                $d-c=0$. But then
                $b\cdot{c}-a\cdot{c}=c\cdot(b-a)\in\mathbb{R}^{+}$.
                Similarly,
                $a\cdot{d}-a\cdot{c}=a\cdot(d-c)$, and thus this is
                either positive of zero. Therefore:
                \begin{equation}
                    bd-ac=(bd-ad)+(ad-ac)=
                    d(b-d)+a(d-c)\in\mathbb{R}^{+}
                \end{equation}
            \end{proof}
            \begin{theorem}
                If $a,b\in\mathbb{R}$ are negative, then $a+b$ is
                negative.
            \end{theorem}
            \begin{proof}
                For if $a$ and $b$ are negative, then
                $\minus{a}$ and $\minus{b}$ are positive. But then
                $(\minus{a})+(\minus{b})\in\mathbb{R}^{+}$. But:
                \begin{equation}
                    (\minus{a})+(\minus{b})=
                    (\minus{1})\cdot{a}+(\minus{1})\cdot{b}
                    =(\minus{1})\cdot(a+b)
                    =\minus(a+b)\in\mathbb{R}^{+}
                \end{equation}
                Thus, $a+b$ is negative.
            \end{proof}
            \begin{theorem}
                If $a,b\in\mathbb{R}$, if $a$ is positive, and if
                $b$ is negative, than $a\cdot{b}$ is negative.
            \end{theorem}
            \begin{proof}
                For if $b$ is negative, then $\minus{b}$ is
                positive, and thus:
                \begin{equation}
                    \minus(a\cdot{b})
                    =a\cdot(\minus{b})\in\mathbb{R}^{+}
                \end{equation}
                Thus, $a\cdot{b}$ is negative.
            \end{proof}
            \begin{theorem}
                If $a,b\in\mathbb{R}$ are negative, then $a\cdot{b}$
                is positive.
            \end{theorem}
            \begin{proof}
                For if $a$ and $b$ are negative, then
                $\minus{a},\minus{b}\in\mathbb{R}^{+}$. But then:
                \begin{equation}
                    a\cdot{b}=1\cdot(a\cdot{b})
                    =\big((\minus{1})\cdot(\minus{1})\big)
                    \cdot(a\cdot{b})
                    =(\minus{a})\cdot(\minus{b})\in\mathbb{R}^{+}
                \end{equation}
                And thus $a\cdot{b}$ is positive.
            \end{proof}
            \begin{theorem}
                If $a\in\mathbb{R}$, then $a^{2}\geq{0}$.
            \end{theorem}
            \begin{proof}
                For if $a$ is positive, then $a\cdot{a}$ is positive.
                If $a$ is zero, then $a\cdot{a}=0$. Finally, from the
                previous theorem, the product of two negative numbers
                is positive, and therefore if $a$ is negative, then
                $a\cdot{a}$ is positive.
            \end{proof}
            From this we have that $1=1^{1}>0$. This generalized to
            the sum of any number of squares.
            \begin{theorem}
                If $a>0$, then $a^{\minus{1}}>0$.
            \end{theorem}
            \begin{proof}
                Suppose not. Then either $a^{\minus{1}}$ is negative
                or it is zero. But it is not zero, for zero has no
                multiplicative inverse, and $a$ is an inverse of
                $a^{\minus{1}}$. Thus $a^{\minus{1}}$ is negative.
                But $a\cdot{a}^{\minus{1}}=1>0$, a contradiction.
                Therefore, $a^{\minus{1}}$ is positive.
            \end{proof}
            \begin{theorem}
                If $0<a<b$, then $0<b^{\minus{1}}<a^{\minus{1}}$.
            \end{theorem}
            \begin{proof}
                For:
                \begin{equation}
                    0<a<b\Longrightarrow
                    0<a\cdot(a^{\minus{1}}b^{\minus{1}})<
                    b\cdot(a^{\minus{1}}b^{\minus{1}})
                    \Longrightarrow
                    0<b^{\minus{1}}<a^{\minus{1}}
                \end{equation}
            \end{proof}
            \begin{theorem}
                If $a<b<0$, then $b^{\minus{1}}<a^{minus{1}}$.
            \end{theorem}
            \begin{proof}
                For if $a<b<0$, then $0<b-a$ and
                $0<a\cdot{b}$. But then
                $0<a^{\minus{1}}\cdot{b}^{\minus{1}}$.
                Thus:
                \begin{equation}
                    0<(b-a)\cdot{a}^{\minus{1}}b^{\minus{1}}=
                    a^{\minus{1}}-b^{\minus{1}}
                \end{equation}
                And thus $b^{\minus{1}}<a^{\minus{1}}$.
            \end{proof}
            \begin{theorem}
                If $a,b,c\in\mathbb{R}$, then
                $\minus(a-b)=b-a$.
            \end{theorem}
            \begin{proof}
                For:
                \begin{equation}
                    (b-a)+(a-b)=b+(\minus{a}+a)-b=
                    b+0-b=b-b=0
                \end{equation}
                From the uniqueness of inverses, etc.
            \end{proof}
            \begin{theorem}
                If $a,b,c,d\in\mathbb{R}$, then:
                \begin{equation}
                    (a-b)\cdot(c-d)
                    =(a\cdot{c}+b\cdot{d})-(a\cdot{d}+b\cdot{c})
                \end{equation}
            \end{theorem}
            \begin{proof}
                For:
                \begin{subequations}
                    \begin{align}
                        (a-b)\cdot(c-d)&=
                        a\cdot(c-d)-b\cdot(c-d)\\
                        &=(a\cdot{c}-a\cdot{d})-
                            (b\cdot{c}-b\cdot{d})\\
                        &=(a\cdot{c}-a\cdot{d})+
                            (b\cdot{d}-b\cdot{c})\\
                        &=(a\cdot{c}+b\cdot{d})-
                            (a\cdot{d}+b\cdot{c})
                    \end{align}
                \end{subequations}
                Therefore, etc.
            \end{proof}
            We thus have a way to distinguish $\mathbb{R}$
            from finite fields. We define the natural numbers to by
            $2=1+1$, $3=2+1$, $4=3+1$, and so on. Order also excludes
            the complex numbers, $\mathbb{C}$, since the complex
            numbers are not ordered. However, the rational numbers,
            $\mathbb{Q}$, still satisfy all of these properties and
            are too an ordered field. We need another property to
            distinguish $\mathbb{Q}$ from $\mathbb{R}$. First, a
            discussion of exponentiation and the absolute value
            function. Given a positive integer $n$, we define the
            exponentiation of a real number $r$ by
            $r^{n}=r\cdots{r}$, where multiplication is
            carried out $n$ times. From this, we get:
            \begin{subequations}
                \begin{align}
                    a^{n}\cdot{a}^{m}&=a^{n+m}\\
                    (a^{m})^{n}&=a^{mn}\\
                    (ab)^{n}&=a^{n}b^{n}
                \end{align}
            \end{subequations}
            The absolute value of a real number is defined as:
            \begin{equation}
                |a|=
                \begin{cases}
                    a,&a\geq{0}\\
                    \minus{a},&a<-
                \end{cases}
            \end{equation}
            \begin{theorem}
                If $a\in\mathbb{R}$, then $|a|\geq{0}$.
            \end{theorem}
            \begin{theorem}
                If $a,b\in\mathbb{R}$, then
                $|a\cdot{b}|=|a|\cdot|b|$.
            \end{theorem}
            \begin{theorem}
                If $a\in\mathbb{R}$, then $a^{2}=|a|^{2}$.
            \end{theorem}
            \begin{ltheorem}{Triangle Inequality}
                If $a,b\in\mathbb{R}$, then
                $|a+b|\leq|a|+|b|$.
            \end{ltheorem}
            \begin{ltheorem}{Reverse Triangle Inequality}
                If $a,b\in\mathbb{R}$, then
                $|a-b|\geq\big||a|-|b|\big|$
            \end{ltheorem}
            \begin{theorem}
                If $a,b\in\mathbb{R}$, then:
                \begin{equation}
                    \max\{a,b\}=\frac{a+b+|a-b|}{2}
                \end{equation}
            \end{theorem}
            \begin{proof}
                If $a=b$, then we are done. If $a<b$, then
                $|a-b|=b-a$, and thus:
                \begin{equation}
                    \frac{a+b+|a-b|}{2}=\frac{a+b+b-a}{2}=b
                \end{equation}
                And this is the max of $a$ and $b$. similarly
                if $b<a$.
            \end{proof}
            \begin{theorem}
                If $a,b\in\mathbb{R}$, then:
                \begin{equation}
                    \min\{a,b\}=\frac{a+b-|a-b|}{2}
                \end{equation}
            \end{theorem}
            \begin{proof}
                For if $a=b$, then we are done. If
                $a<b$, then $|a-b|=b-a$, and thus:
                \begin{equation}
                    \frac{a+b-|a-b|}{2}=
                    \frac{a+b-(b-a)}{2}=a
                \end{equation}
                And this is the minimum of $a$ and $b$. Similarly
                for $b<a$.
            \end{proof}
            \begin{theorem}
                If $a,b,x,y\in\mathbb{R}$, if $a<x<b$< and if
                $a<y<b$< then:
                \begin{equation}
                    |x-y|<b-a
                \end{equation}
            \end{theorem}
            \begin{proof}
                For:
                \begin{equation}
                    a-b=\minus(b-a)<
                    x-b<x-y<b-y<ba
                \end{equation}
                And therefore:
                \begin{equation}
                    \minus(b-a)<x-y<b-a
                \end{equation}
                Therefore, etc.
            \end{proof}
            Note that $|x-a|<\varepsilon$ implies that
            $\varepsilon-a<x<\varepsilon+a$. Thus, the solution set
            to this inequality is all of the points that lie in the
            interval $(a-\varepsilon,a+\varepsilon)$. Now, to
            separate $\mathbb{R}$ from $\mathbb{Q}$ we need
            to introduce the idea of \textit{completeness}.
            We will do this in the form of the Least Upper
            Bound axiom.
            \begin{definition}
                An upper bound for a subset $S\subseteq\mathbb{R}$
                is a real number $r$ such that, for all $x\in{S}$,
                we have $x\leq{r}$.
            \end{definition}
            A bounded above subset is a subset with an upper bound.
            \begin{definition}
                A least upper bound for a subset
                $S\subseteq\mathbb{R}$ is a real number $r$ such
                that $r$ is an upper bound
                for $S$, and for all upper bounds $s$, we have
                $r\leq{s}$.
            \end{definition}
            From this definition we have that least upper bounds are
            unique for a given bounded above set.
            \begin{theorem}
                If $S$ is a subset of $\mathbb{R}$, if $s$ is a
                least upper bound of $S$, and if $x\in\mathbb{R}$
                is such that $x<s$, then there is a $y\in{S}$
                such that $x<y$.
            \end{theorem}
            \begin{proof}
                For suppose not. Then $x$ is an upper bound of $S$,
                a contradiction as $s$ is the least upper bound.
            \end{proof}
            Any non-empty finite subset will have a least
            upper bound. Infinite subsets need not have a least
            upper bound, and indeed $\mathbb{R}$ does not have
            one. If the least upper bound of $S$ exists, it may
            not belong to $S$. For example, the set of all
            negative numbers has zero as its least upper bound,
            but zero is not a negative number. The real
            numbers satisfy the following property:
            \begin{enumerate}
                \item For any non-empty set of real numbers that
                      is bounded from above, there is a least
                      upper bound.
            \end{enumerate}
            This axiom distringuishes the rational numbers from the
            real numbers. That is, there are bounded above subsets
            of $\mathbb{Q}$ with no least upper bound.
            We can justify the least upper bound axiom by
            considering the decimal expansion of real numbers. That
            is, we write out
            $x=n+0.x_{1}x_{2}x_{3}\dots$ where $n$ is an integer,
            and $x_{i}$ is an integer between zero and nine. If
            $S$ is bounded above, then there is a least integer
            $n$ such that, for all $x\in{S}$, $x\leq{n}$. This
            simply comes from the Archimedean principle and the
            well-ordering principle of the real numbers.
            But then there is a least $x_{1}$ such that $x_{1}$ is
            and integer between zero and nine and such that, for
            all $x\in{S}$, $x\leq{n}.x_{1}$ where this indicates
            the usual representation of
            $n+x_{1}\times{10}^{\minus{1}}$. We can continue on
            for $x_{2}$ and so on, and this decimal expansion
            will represent the least upper bound of $S$. The
            least upper bound of a set $S$ is often denoted
            $\sup{S}$, where $\sup$ denotes the latin word
            \textit{supremum}. Similarly, the greatest lower bound
            of a set is denoted $\inf{S}$, where $\inf$ stands
            for \textit{infinum}.
            \begin{theorem}
                If $S\subseteq\mathbb{R}$ is bounded from below,
                then there exists a greatest lower bound of $S$.
            \end{theorem}
            \begin{proof}
                For is $S$ is bounded below, then
                $\minus{S}=\{\minus{x}:x\in{S}\}$ is bounded
                from above. But sets that are bounded above have
                a least upper bound. Let $s$ be the least upper
                bound of $\minus{S}$. Then $\minus{s}$ is the
                greatest lower bound of $S$. Therefore, etc.
            \end{proof}
            The real numbers have the property that any real
            number can be approximated arbitrarily well by a
            rational number. The rational numbers, however, have
            certain gaps that are filled in by the real numbers.
            In a sense, the real numbers are \textit{complete}
            whereas the rational numbers are not.
            \begin{ltheorem}{The Archimedean Property}
                If $x$ is a real number, then there is an integer
                $n$ such that $x<n$.
            \end{ltheorem}
            \begin{proof}
                For suppose not. Then there is an $x\in\mathbb{R}$
                such that, for all $n\in\mathbb{N}$, $n\leq{x}$.
                But then $\mathbb{N}$ is bounded above, and then
                there exists a least upper bound. Let $s$ be such
                a bound. But if $s$ is a bound, then for all
                $n\in\mathbb{N}$, $n\leq{s}$. But if
                $n\in\mathbb{N}$, then $n+1\in\mathbb{N}$ and thus
                $n+1\leq{s}$. But then, for all $n\in\mathbb{N}$,
                $n\leq{s-1}$, a contradiction as $s$ is a least
                upper bounded, and $s-1<s$. Therefore, etc.
            \end{proof}
            \begin{theorem}
                If $\varepsilon>0$, then there is an
                $n\in\mathbb{N}$ such that
                $n^{\minus{1}}<\varepsilon$.
            \end{theorem}
            \begin{proof}
                Since $\varepsilon>0$ $\varepsilon^{\minus{1}}$ is
                well defined and positive. But then there is an
                $n\in\mathbb{N}$ such that
                $n>\varepsilon^{\minus{1}}$. But then
                $n^{\minus{1}}<\varepsilon$. Therefore, etc.
            \end{proof}
            \begin{theorem}
                If $x\in\mathbb{R}$, then there is an integer
                $n\in\mathbb{Z}$ such that
                $n\leq{x}<n+1$.
            \end{theorem}
            \begin{proof}
                For if $x\in\mathbb{R}^{+}$, there is an
                $N\in\mathbb{N}$ such that $x<N$. But then, from
                the well-ordering of $\mathbb{N}$, there is a
                least $k\in\mathbb{N}$ such that
                $x<k$. Let $n=k-1$. But then $n\in\mathbb{Z}$ and
                $n\leq{x}<n+1$. If $\minus{x}\in\mathbb{R}^{+}$,
                negate this and repeat the process. If $x=0$, let
                $n=0$.
            \end{proof}
            \begin{theorem}
                If $x\in\mathbb{R}$ and $N\in\mathbb{N}$, and there
                is an $n\in\mathbb{Z}$ such that:
                \begin{equation}
                    \frac{n}{N}\leq{x}<\frac{n+1}{N}
                \end{equation}
            \end{theorem}
            \begin{proof}
                For let $y=N\cdot{x}$. Then there is an
                $n\in\mathbb{Z}$ such that
                $n\leq{N}\cdot{x}<n+1$. Dividing by $N$ proves
                the result.
            \end{proof}
            \begin{theorem}
                If $\varepsilon>0$ and $r\in\mathbb{R}$, then
                there is a $q\in\mathbb{Q}$ such that
                $|r-q|<\varepsilon$.
            \end{theorem}
            \begin{proof}
                For let $\varepsilon>0$. Then there is an
                $N\in\mathbb{N}$ such that
                $N^{\minus{1}}<\varepsilon$. But then there is
                an $n\in\mathbb{Z}$ such that
                $n\leq{N}\cdot{r}<n+1$. Let
                $q=n\cdot{N}^{\minus{1}}$. Then
                $|q-r|<\varepsilon$.
            \end{proof}
            This final theorem shows that any real number can
            be approximated arbitrarily well by any rational number,
            as was claimed. Let's return to the discussion of
            the decimal expansion of real numbers. First, we
            consider finite decimals. Let $n\in\mathbb{N}$ and
            let $a{1},\dots,a_{n}$ be a sequence of integers between
            zero and nine. Let $a_{0}$ be any integer. If
            $m<n$, then:
            \begin{equation}
                \begin{split}
                    a_{0}.a_{1}a_{2}\dots{a}_{m}&\leq
                    a_{0}.a_{1}a_{2}\dots{a}_{m}a_{m+1}
                    \dots{a}_{n}\\
                    &\leq
                    a_{0}.a_{1}a_{2}\dots{a}_{m}
                    +9\times{10}^{\minus(m+1)}+\cdots
                    +9\times{10}^{\minus{n}}
                \end{split}
            \end{equation}
            If we add $10^{\minus{n}}$, this reduces to the
            following:
            \begin{equation}
                a_{0}.a_{1}a_{2}\dots{a}_{m}\leq
                a_{0}.a_{1}a_{2}\dots{a}_{n}\leq
                a_{0}.a_{1}a_{2}\dots{a}_{m}+10^{\minus{m}}
            \end{equation}
            We can thus view an \textit{infinite decimal} as a
            sequence $a:\mathbb{N}\rightarrow\mathbb{Z}$ such that
            $a_{1}\in\mathbb{Z}$, and for all $k>1$,
            $a_{k}$ is an integer between zero and nine. Using the
            decimal expansion we can find real numbers that are
            not rational. For let
            $x=0.101001000100001000001\dots$ This can't be
            rational since $Nx$ is not an integer for any
            positive integer $N$. Another classic example of
            a real number that is not rational is $\sqrt{2}$.
            \begin{theorem}
                If $r>0$, then there is a unique number $a>0$,
                called the square root of $r$, such that
                $a^{2}=r$.
            \end{theorem}
            \begin{proof}
                For uniqueness, first note that if
                $0<a<b$, then $0<a^{2}<b^{2}$, and thus
                any positive real number can have, at most, one
                positive square root. Define the following:
                \begin{equation}
                    S=\{x\in\mathbb{R}^{+}:x^{2}\leq{r}\}
                \end{equation}
                Then $S$ is bounded above, since $\max\{1,r\}$
                is such a bound. Let $a$ be the least upper bound
                of $S$. First, note that $s>0$ since:
                \begin{equation}
                    (\min\{1,r\})^{2}\leq\min\{1,r\}\cdot{1}
                    =\min\{1,r\}\leq{r}
                \end{equation}
                And therefore, $\min\{1,r\}\leq{s}$. Given
                $\varepsilon>0$, we have:
                \begin{equation}
                    (s-\varepsilon)^{2}<r<
                    (s+\varepsilon)^{2}
                \end{equation}
                And therefore:
                \begin{equation}
                    |s^{2}-r|<4s\varepsilon
                \end{equation}
                But $\varepsilon$ is arbitrary, and thus this
                difference is zero. Therefore $r=s^{2}$.
            \end{proof}
            The value $s$ is called the square root of $r$, and
            we denote it by $s=\sqrt{r}$. Note that, for any
            positive real number, there are two square roots:
            $\pm\sqrt{r}$. When we write $\sqrt{r}$, we mean the
            positive value. This theorem shows that positive real
            numbers are the squares of non-zero real numbers.
            Thus, the set $\mathbb{R}^{+}$ described earlier is
            unique, further justifying the use of this set to
            order the real numbers. The real numbers are the only
            arithmetic system, up to isomorphism, that is a
            complete ordered field. Here, complete means that the
            least upper bound axiom holds. If
            $(\mathbb{R},+,\cdot)$ and $(\mathbb{R}',+',\cdot')$
            are complete ordered fields, we may as well consider
            them to be the exact same object. They are essentially
            a relabelling of each other.
            \begin{lexample}
                Find the greatest lower bound and least upper bound
                of the set:
                \begin{equation}
                    A=\{\frac{1}{n}:n\in\mathbb{N}\}
                \end{equation}
                The least upper bound is 1, since for all
                $n\geq{1}$, we have $1\leq{n}^{\minus{1}}$. There
                is no bound less, since $1\in{A}$. The greatest
                lower bound is zero. It is indeed a bound, since
                for all $n>0$, $n^{\minus{1}}>0$. Moreover, if
                $s>0$, there is an $N\in\mathbb{N}$ such that
                $N^{\minus{1}}<s$, and thus $s$ cannot be a lower
                bound. Consider the set:
                \begin{equation}
                    B=\{\frac{1}{3},\frac{4}{9},
                        \frac{13}{27},\frac{40}{81},\dots\}
                \end{equation}
                The denominator's of this set are powers of
                three, and the numerators are sums of powers of
                three. That is, we can write:
                \begin{equation}
                    B=\Big\{\frac{1}{3^{n}}\sum_{k=0}^{n-1}3^{k}:
                        n\in\mathbb{N}\Big\}
                \end{equation}
                We can use the geometric series to simplify the
                sum, noting that:
                \begin{equation}
                    \frac{1}{3^{n}}\sum_{k=0}^{n-1}3^{k}=
                    \frac{1}{3^{n}}\frac{1-3^{n}}{1-3}=
                    \frac{3^{n}-1}{2\cdot{3}^{n}}
                \end{equation}
                Splitting this into two parts, we get:
                \begin{equation}
                    \frac{1}{3^{n}}\sum_{k=0}^{n-1}3^{k}=
                    \frac{1}{2}-\frac{1}{2\cdot{3}^{n}}
                \end{equation}
                And this decays to zero. Thus, we see that the
                least upper bound is $\frac{1}{2}$, since for all
                $\varepsilon>0$ there is an $N\in\mathbb{N}$ such
                that $(2\cdot{3^{N}})^{\minus{1}}<\varepsilon$,
                and thus there are elements of the set that are
                arbitrarily close to $\frac{1}{2}$. Moreover,
                $\frac{1}{2}$ is an upper bound, since every
                element of the set is strictly less than it.
                Using the final equation, we see that the elements
                are strictly increasing as $n$ increasing, and
                thus the greatest lower bound is simply the
                first element, $\frac{1}{3}$. Lastly, find the
                greatest lower bound and least upper bound for:
                \begin{equation}
                    C=\{\sqrt{2},\sqrt{2+\sqrt{2}},
                        \sqrt{2+\sqrt{2+\sqrt{2}}},\dots\}
                \end{equation}
                We see that the pattern is:
                \begin{equation}
                    x_{n+1}=\sqrt{2+\sqrt{x_{n}}}
                \end{equation}
                Thus, this sequence is strictly increasing as
                $n$ increases. From this we know that
                $\sqrt{2}$ is the greatest lower bound. Now we
                must show that the set has a least upper bound.
                Let $x$ be the solution to the equation
                $x=\sqrt{2+\sqrt{x}}$. We know such a solution
                exists since this equation simplifies to a
                quartic polynomial with roots. Then:
                \begin{align}
                    |x-x_{n+1}|&=
                    |\sqrt{2+\sqrt{x}}-\sqrt{2+\sqrt{x_{n}}}|\\
                    &=\Big|\frac{\sqrt{x}-\sqrt{x_{n}}}
                        {\sqrt{2+\sqrt{x}}+\sqrt{2+\sqrt{x_{n}}}}
                    \Big|\\
                    &<\Big|\frac{\sqrt{x}-\sqrt{x_{n}}}{2\sqrt{2}}
                        \Big|\\
                    &=\Big|\frac{x-x_{n}}
                        {2\sqrt{2}(\sqrt{x}+\sqrt{x_{n}})}\Big|\\
                \end{align}
                But we not that
                $\sqrt{x}+\sqrt{x_{n}}>2\sqrt{x_{n}}>2\sqrt{2}$,
                and obtain:
                \begin{equation}
                    |x-x_{n+1}|<\frac{1}{8}|x-x_{n}|
                \end{equation}
                Similarly:
                \begin{equation}
                    |x-x_{n+2}|<\frac{1}{8}|x-x_{n+1}|<
                    \frac{1}{8^{2}}|x-x_{n}|
                \end{equation}
                By induction:
                \begin{equation}
                    |x-x_{n+k}|<\frac{1}{8^{k}}|x-x_{n}|
                \end{equation}
                And this tends to zero, and therefore
                $x_{n}\rightarrow{x}$. Thus, $x$ is the least
                upper bound.
            \end{lexample}
    \section{Old Notes}
        The real line, or real number system, is a complete ordered
        field. That is, it is complete in the sense that all
        Cauchy sequences converge, has a total order structure
        on it, and has a field structure (That of addition,
        multiplication, subtraction, and division).
        An open subset of the real line is a set $S$ such that
        for all $x\in{S}$ there is an $\varepsilon>0$ such that
        $(x-\varepsilon,x+\varepsilon)\subset{S}$. The entire
        space $\mathbb{R}$ is open, as is the empty set
        $\emptyset$. The union of
        an arbitrary collection of open sets is open, and the
        intersection of finitely many open sets is open. The
        intersection of infinitely many open sets may not be
        open, however. A set is closed if its complement is
        open. The Euclidean plane is the set of all ordered
        pairs $(a,b)$. That is,
        $\mathbb{R}^{2}=\mathbb{R}\times\mathbb{R}$. Euclidean
        space, or 3-space, is
        $\mathbb{R}^{3}=\mathbb{R}\times\mathbb{R}\times\mathbb{R}$.
        This is the set of all ordered triplets $(x,y,z)$. Similarly,
        $n$ dimensional Euclidean space is the set of all
        $n$ tuples. This is denoted $\mathbb{R}^{n}$. The distance
        between two points $\mathbf{x}$ and $\mathbf{y}$ is defined
        by the generalized Pythagorean Theorem:
        \begin{equation*}
            d(\mathbf{x},\mathbf{y})=
            \sqrt{\sum_{k=1}^{n}(x_{k}-y_{k})^{2}}
        \end{equation*}
        \begin{definition}
            A metric on a set $X$ is a function
            $d:X\times{X}\rightarrow\mathbb{R}$ such that:
            \begin{enumerate}
                \item $d(x,y)\geq{0}$ for all $x,y\in{X}$.
                \item $d(x,y)=0$ if and only if $x=y$.
                \item $d(x,y)=d(y,x)$ for all $x,y\in{X}$.
                \item $d(x,z)\leq{d(x,y)+d(y,z)}$
                      for all $x,y,z\in{X}$.
            \end{enumerate}
        \end{definition}
        There are two types of integrals defined for functions
        of a real variable: Riemann Integration and Lebesgue Integration.
        Lebesgue integration requires the notion of \textit{measure}.
    \subsection{Definitions}
        \begin{definition}
                The tangent line of a differentiable function
                $y:\mathbb{R}\rightarrow\mathbb{R}$ at a point
                $x_{0}\in\mathbb{R}$ is the function
                $y_{T}:\mathbb{R}\rightarrow\mathbb{R}$ defined by
                $y_{T}(x)=y'(x_0)(x-x_0)+y(x_0)$ 
            \end{definition}
        \begin{definition}
            If $\Gamma(t)=\big(x(t),y(t)\big)$, for $a\leq t\leq b$,
            and $\Gamma'(t)=\big(x'(t),y'(t)\big)$ exists for
            $a<t<b$, then the length of $\Gamma$ from $a$ to $b$ is:
            \begin{equation}
                L=\int_{a}^{b}\sqrt{
                    \bigg(\frac{dx}{dt}\bigg)^{2}+
                    \bigg(\frac{dy}{dt}\bigg)^{2}
                }dt
            \end{equation}
        \end{definition}
        \begin{definition}
            The dimension of a vector space is the cardinality of
            any basis of the space. 
        \end{definition}
        By the Dimension Theorem, all bases of a vector space
        have the same cardinality.
    \subsection{Theorems}
    \begin{theorem}[Mean Value Theorem]
        If $f:(a,b)\rightarrow\mathbb{R}$ is continuous and
        bounded, and if $x\in(a,b)$, then there is a $c\in(a,x)$
        such that $\int_{a}^{x}f=(x-a)f(c)$.
    \end{theorem}
    \begin{theorem}
        [Generalized Fundamental Theorem of Calculus]
        If $\mathcal{U}$ is an open non-empty subset of
        $\mathbb{R}$, $a\in\mathcal{U}$, and if
        $f:\mathcal{U}\rightarrow\mathbb{R}$
        is bounded and continuous, then
        $F:\mathcal{U}\rightarrow\mathbb{R}$
        defined by $F(x)=\int_{\mathcal{U}\cap (a,x)}f$ is
        differentiable and $F'(x)=f(x)$
    \end{theorem}
    \begin{proof}
        For let $x\in\mathcal{U}$. Let
        $\{x_n\}_{n=1}^{\infty}\subset\mathcal{U}$
        be a sequence such that $x_{n}\rightarrow x$,
        $x\notin\{x_{n}\}_{n=1}^{\infty}$.
        As $\mathcal{U}$ is open and $x\in\mathcal{U}$,
        there is an $\varepsilon>0$ such that
        $B_{\varepsilon}(x)\subset\mathcal{U}$. But, as
        $x_{n}\rightarrow x$, there is an $N\in \mathbb{N}$ such
        that for all $n>N$, $x_{n}\in B_{\varepsilon}(x)$.
        But then for all $n>N$:
        \begin{equation}
            \int_{\mathcal{U}\cap(a,x)}f-%
            \int_{\mathcal{U}\cap(a,x_{n})}f=%
            \int_{x_{n}}^{x}f
        \end{equation}
        But, as $f$ is continuous, by the mean value theorem for
        all $n>N$ there is a $c_{n}\in(x_n,x)$ such that
        $\int_{x_{n}}^{x}f=(x-x_{n})f(c_{n})$. But then 
        \begin{equation}
            \Big|\frac{\int_{x_{n}}^{x}f}{x-x_{n}}-f(x)\Big|
            =|f(c_{n})-f(x)|
        \end{equation}
        But $c_{n}\in(x_{n},x)$, and $x_{n}\rightarrow x$, and
        therefore $c_{n} \rightarrow x$. But $f$ is continuous,
        and therefore $f(c_{n})\rightarrow f(x)$. Therefore, by
        the definition of the derivative of $F$ at $x$,
        $F'(x)=f(x)$. 
    \end{proof}
    \begin{theorem}
        If $V$ is a vector space and $A,B\subset V$ are
        subspaces, then $A\cap B$ is a subspace and
        $\dim(A\cap B)\leq\min\{\dim(A),\dim(B)\}$
    \end{theorem}
    \begin{theorem}
        If $f:\mathbb{R}\rightarrow \mathbb{R}$
        is differentiable
        and $f'(x)>0$ for all $x$,
        then $f$ is strictly increasing.
    \end{theorem}
    \begin{theorem}
        If $f:(a,b)\rightarrow\mathbb{R}$ is continuous and
        $f(a)<0<f(b)$, then there is a $c\in (a,b)$ such that
        $f(c)=0$.
    \end{theorem}
    \begin{theorem}
        If $f$ is integrable on $(a,b)$, and if $c\in(a,b)$, then
        $\int_{a}^{b}f=\int_{a}^{c}f+\int_{c}^{b}f$
    \end{theorem}
        \subsection{Metric Spaces}
            \begin{definition}
            A metric space is a set $X$ with a function $d:X\times X\rightarrow \mathbb{R}$ with the following properties:
            \begin{enumerate}
            \item For all $x,y\in X$, $d(x,y) = 0\Leftrightarrow x=y$. \hfill [Identity of Indiscernables]
            \item For all $x,y,z\in X$, $d(x,y) \leq d(x,z)+d(y,z)$\hfill [Modified Triangle Inequality]
            \end{enumerate}
            They are denoted $(X,d)$. $d$ is called a
            \textit{metric} or \textit{distance}== function.
            \end{definition}
            \begin{theorem}
            A metric space $(X,d)$ has the following properties:
            \begin{enumerate}
                \item $d(x,y) = 0 \Leftrightarrow x=y$ \hfill [Identity of Indiscernibles]
                \item $d(x,y) = d(y,x)$ \hfill [Symmetry]
                \item $d(x,y) \geq 0$ \hfill [Positivity]
                \item $d(x,y) \leq d(x,z)+d(z,y)$ \hfill [Triangle Inequaility]
            \end{enumerate}
            \end{theorem}
            \begin{proof}
            In order,
            \begin{enumerate}
                \item This is part of the definition.
                \item For $d(x,y) \leq d(x,x)+d(y,x) = d(y,x)$. But $d(y,x) \leq d(y,y)+d(x,y) = d(x,y)$. Thus $d(x,y)\leq d(y,x)$ and $d(y,x) \leq d(x,y)$, and therefore $d(x,y) = d(y,x)$
                \item For $0=d(x,x) \leq d(x,y)+d(y,x) = 2d(x,y)$. Thus, $0\leq d(x,y)$
                \item $d(x,y)\leq d(x,z)+d(y,z) = d(x,z)+d(z,y)$
            \end{enumerate}
            \end{proof}
            It is most common, almost universal, that textbooks state theorem
            1.8.1 as the definition of a metric space. However, when proving
            something is a metric space, it is nicer to prove two things rather
            than four.
            \begin{theorem}
            If $V$ is a vector space with a norm and $d$ is the induced metric, then $(V,d)$ is a metric space.
            \end{theorem}
            \begin{proof}
            In order,
            \begin{enumerate}
            \item $\norm{x-y} = 0$ if and only if $x-y = 0$. Thus $d(x,y) = 0 \Leftrightarrow x=y$.
            \item $d(x,y) = \norm{x-y}\leq \norm{x-z}+\norm{y-z} = d(x,z)+d(y,z)$
            \end{enumerate}
            \end{proof}
            \begin{definition}
            If $(X,d)$ is a metric space, $x\in X$, then the open ball of radius $r>0$ is $B_{r}(x) = \{y\in x: d(x,y)<r\}$.
            \end{definition}
            \begin{definition}
            In a metric space, $\mathcal{U}$ is metrically open if and only if for all $x\in \mathcal{U}$ there is an $r>0$ such that $B_{r}(x)\subset \mathcal{U}$.
            \end{definition}
            For metric spaces, metrically open and topologically open are the
            same thing, as we will see.
            \begin{theorem}
            The empty set is open.
            \end{theorem}
            \begin{proof}
            For suppose not. Then there is some $x\in \emptyset$ such that for all $r>0$, $B_{r}(x)\not\subset \emptyset$. A contradiction. Therefore, etc.
            \end{proof}
            \begin{theorem}
            The whole space $X$ is open.
            \end{theorem}
            \begin{proof}
            For let $x\in X$ and $r>0$. Then $B_{r}(x) = \{y\in X:d(x,y)<r\}$, and thus $B_{r}(x)\subset X$. Therefore, etc.
            \end{proof}
            \begin{theorem}
            If $\mathcal{U}\subset X$ is open, then it is the union of open balls.
            \end{theorem}
            \begin{proof}
            For let $\mathcal{U} \subset X$ be open. Then, for all $x\in \mathcal{U}$ there is a $r(x)>0$ such that $B_{r(x)}(x) \subset \mathcal{U}$. But then $\cup_{x\in \mathcal{U}}B_{r(x)}(x)\subset \mathcal{U}$. But, as for all $y\in \mathcal{U}$, $y\in \cup_{x\in \mathcal{U}}B_{r(x)}(x)$, $\mathcal{U} \subset \cup_{x\in \mathcal{U}}B_{r(x)}(x)$. Thus, $\mathcal{U}= \cup_{x\in \mathcal{U}}B_{r(x)}(x)$.
            \end{proof}
            \begin{definition}
            If $(X,d)$ is a metric space, then the metric space topology is the set $\tau = \{\mathcal{U}:\mathcal{U}\underset{Open}\subset X\}$>
            \end{definition}
            \begin{theorem}
            The metric space topology is a topology.
            \end{theorem}
            \begin{proof}
            In order,
            \begin{enumerate}
            \item $\emptyset, X \in \tau$
            \item Let $\mathcal{U}_{\alpha}$ be a family of open sets and let $x\in \mathcal{U}_{\alpha}$ be arbitrary. Then there is an open set $\mathcal{U} \in \{\mathcal{U}_{\alpha}:\alpha\in A\}$ such that $x\in \mathcal{U}$. But then there is an $r>0$ such that $B_{r}(x)\subset\mathcal{U}$. But then $B_{r}(x) \subset \cup_{\alpha \in A}\mathcal{U}_{\alpha}$.
            \item Let $\mathcal{U}_{k}, 1\leq k \leq n$ be open sets, and let $x\in \cap_{k=1}^{n} \mathcal{U}_k$. Then, for each $\mathcal{U}_k$ there is an $r_{k}$ such that $B_{r_k}(x)\subset \mathcal{U}_{k}$. Let $r = \min\{r_k:1\leq k \leq n\}$. Then $B_{r}(x) \in \cap_{k=1}^{n}\mathcal{U}_k$.
            \end{enumerate}
            \end{proof}
            \begin{theorem}
            If $(X,d_X)$ and $(Y,d_Y)$ are metric space, then $f:X\rightarrow Y$ is a continuous function (With respect to their metric space topologies) if and only if $\forall \varepsilon>0,\ \forall x\in X,\ \exists \delta>0:y\in B_{\delta}(x)\Rightarrow f(y) \in B_{\varepsilon}(x)$.
            \end{theorem}
            \begin{proof}
            For let $x\in X$ and $\varepsilon>0$ be given. As $B_{\varepsilon}(f(x))$ is open and $f$ is continuous, the preimage is open. But as $x\in f^{-1}(B_{\varepsilon}(f(x))$, there is a $\delta>0$ such that $B_{\delta}(x)\in f^{-1}(B_{\varepsilon}(f(x)))$. Thus, for all $y \in B_{\delta}(x)$, $f(y) \in B_{\varepsilon}(f(x))$. Now suppose for all $x\in X$ and for all $\varepsilon>0$, there is a $\delta>0$ such that $y\in B_{\delta}(x)\Rightarrow f(x) \in B_{\varepsilon}(f(x))$. Let $\mathcal{U}$ be open in $f(X)$. If $f^{-1}(\mathcal{U})$ is empty, we are done. Suppose not. Let $x\in f^{-1}(\mathcal{U})$. As $\mathcal{U}$ is open, there is a $\varepsilon>0$ such that $B_{\varepsilon}(f(x))$ is open in $\mathcal{U}$. But then there is a $\delta>0$ such that if $y\in B_{\delta}(x)$, then $f^{-1}(B_{\varepsilon}(f(y))$. But then $f^{-1}(\mathcal{U})$ is open. Therefore, etc.
            \end{proof}
            \begin{definition}
            If $S\subset (X,d)$, then $x$ is said to be a limit point of $S$ if and only if for all $\varepsilon>0$, $B_{\varepsilon}(x)\cap S \ne \emptyset$.
            \end{definition}
            \begin{definition}
            If $S\subset (X,d)$, then the closure of $S$, denoted $\overline{S}$, is the set of all limit points of $S$.
            \end{definition}
            \begin{definition}
            If $S\subset (X,d)$, then $x\in S$ is an interior point if and only if $\exists r>0:B_{r}(x)\subset S$.
            \end{definition}
            \begin{definition}
            If $S\subset (X,d)$, the interior of $S$, denoted Int$(S)$, is the set of all interior points.
            \end{definition}
            \begin{definition}
            If $S\subset (X,d)$, the relative interior of $S$ is $\textrm{ri}(S)= \{x\in S:\exists \varepsilon>0:B_{\varepsilon}(x)\cap \textrm{aff}(S)\subset S\}$.
            \end{definition}
            \begin{definition}
            The boundary of $S\subset V$ is $S\setminus \textrm{ri}(S)$.
            \end{definition}
            \begin{theorem}
            A subset $S$ of a metric space $(X,d)$ is closed if and only if every limit point of $S$ is in $S$.
            \end{theorem}
            \begin{proof}
            For let $S$ be closed and let $x$ be a limit point of $S$. Suppose $x\in S^c$. But $S^c$ is open, as $S$ is closed, and thus there is a $r>0$ such that $B_{r}(x)\subset S^c$. But then $B_{r}(x)\cap S = \emptyset$, a contradiction. Thus, $x\in S$. Now suppose $S$ contains all of its limit points. Suppose $S^c$ is not open. Then there is a $y\in S^c$ such that for all $r>0$, $B_{r}(y)\not \subset S^c$. Then for all $r>0$, $B_{r}(y)\cap S \ne \emptyset$. But then $y\in S$, as $S$ contains all of its limit points. Thus $S^c$ is open, and therefore $S$ is closed.
            \end{proof}
            \begin{theorem}
            Metric spaces, with the metric space topology, are $T_4$ spaces.
            \end{theorem}
            \begin{proof}
            Let $(X,d)$ be a metric space and let $\tau$ be the metric space topology. $(X,\tau)$ is $T_1$, for let $x,y\in X$, $x\ne y$, and let $r= \frac{d(x,y)}{2}$. Then $x\in B_{r}(x)$ and $y\notin B_{r}(x)$. $(X,\tau)$ is normal, for let $E$ and $V$ be closed, nonempty, disjoint subsets of $X$. As $V$ is closed, and as $E$ and $V$ are disjoint, for all $x\in E$ there is an $r(x)>0$ such that $B_{r(x)}(x)\cap V = \emptyset$ (Otherwise $x$ is a limit point of $V$, and thus in $V$). Similarly, for all $y\in V$ there is an $r(y)>0$ such that $B_{r(y)}(y)\cap E = \emptyset$. Let $\mathcal{U} = \cup_{x\in E}B_{\frac{r(x)}{4}}(x)$ and $\mathcal{V} = \cup_{y\in V}B_{\frac{r(y)}{4}}(y)$. Then $E\subset \mathcal{U}$ and $E\subset \mathcal{V}$, and $\mathcal{U}$ and $\mathcal{V}$ are disjoint. For suppose not. Let $z\in \mathcal{U}\cap \mathcal{V}$. Then, there is an $x\in E$ and a $y\in V$ such that $d(x,z)\leq \frac{r(x)}{4}$ and $d(y,z)\leq \frac{r(y)}{4}$. But then $d(x,y) \leq d(x,z)+d(y,z) = \frac{r(x)+r(y)}{4} \leq \frac{\max\{r(x),r(y)\}}{2}$. Thus $x\in B_{r(y)}(y)$, or $y\in B_{r(x)}(x)$, a contradiction. Therefore, etc.
            \end{proof}
            \begin{definition}
            A subset $S\subset (X,d)$ is said to be bounded if and only if $\exists M\in \mathbb{R}:x,y\in S\Rightarrow d(x,y)\leq M$.
            \end{definition}
            \begin{definition}
                A Cauchy Sequence in a metric space is a sequence
                $x_n:\forall \varepsilon>0,\exists N\in \mathbb{N}:n,m>N\Rightarrow d(x_n,x_m)<\varepsilon$.
            \end{definition}
            \begin{theorem}
                Convergence in a metric space is unique.
            \end{theorem}
            \begin{proof}
                As metric spaces are $T_4$, they are Hausdorff, and thus
                limits are unique.
            \end{proof}
            \begin{theorem}
            In a metric space $(X,d)$, a sequence $x_n\rightarrow x$ if and only if $\forall\varepsilon>0,\exists N\in \mathbb{N}:n>N\Rightarrow d(x,x_n)<\varepsilon$.
            \end{theorem}
            \begin{proof}
            For any open set $\mathcal{U}$, $x\in \mathcal{U}$, there is an $N\in \mathbb{N}:n>N\Rightarrow x_n \in \mathcal{U}$. Let $\mathcal{U}=B_{\varepsilon}(x)$. Then $n>N\Rightarrow d(x,x_n)<\varepsilon$. Now, let $\mathcal{U}$ be open and $x\in \mathcal{U}$. $\exists\varepsilon>0:B_{\varepsilon}(x)\subset \mathcal{U}$. But $\exists N\in \mathbb{N}:n>N\Rightarrow d(x,x_n)<\varepsilon$. Thus, $n>N\Rightarrow x_n\in \mathcal{U}$.
            \end{proof}
            \begin{theorem}
            $f:(X,d_X)\rightarrow (Y,d_Y)$ is continuous if and only if for all $x\in X$, $x_n\rightarrow x \Rightarrow f(x_n)\rightarrow f(x)$.
            \end{theorem}
            \begin{proof}
            $\forall \varepsilon>0,\forall x\in X,\exists \delta>0:d_X(x,x_0)<\delta \Rightarrow d_Y(f(x),f(x_0))<\varepsilon$. Let $x_n \rightarrow x$. Then, $\exists N\in \mathbb{N}:n>N \Rightarrow d_X(x_n,x)<\delta$. But then $d_Y(f(x),f(x_n)) < \varepsilon$. Thus $f(x_n)\rightarrow f(x)$. Now suppose $x_n\rightarrow x \Rightarrow f(x_n)\rightarrow f(x)$ for all such sequences, and suppose $f$ is discontinuous. Then there is a $\varepsilon>0$ such that for all $n\in \mathbb{N}$, there is an $x_{n_k} \in B_{\frac{1}{k}}(x)$ such that $d_Y(f(x),f(x_0))>\varepsilon$. But then $d_X(x,x_{n_k})\rightarrow 0$, and thus $d_Y(f(x),f(x_n))\rightarrow 0$, a contradiction. Therefore, etc.
            \end{proof}
            \begin{definition}
            A metric space $(X,d)$ is said to be complete if and only if every Cauchy sequence in $X$ converges.
            \end{definition}
            \begin{definition}
            An inner product space is called a Hilbert Space if and only if it is complete (Induced Metric).
            \end{definition}
            \begin{definition}
            A normed space is called a Banach Space if and only if it is complete (Induced Metric).
            \end{definition}
            \begin{definition}
            A subset $S$ of a metric space $(X,d)$ is said to be sequentially compact if and only if every sequence $x_n$ in $S$ has a convergent subsequence whose limit is in $S$.
            \end{definition}
            \begin{definition}
            If $x_n$ is a sequence in a metric space $(X,d)$, then $x$ is said to be an accumulation point of $x_n$ if and only if for all $\varepsilon>0$, $B_{\varepsilon}(x)\cap \{x_n\}_{n=1}^{\infty}$ is infinite.
            \end{definition}
            \begin{definition}
            A subset $S$ of a metric space $(X,d)$ is said to be limit point compact if and only if every sequence in $S$ has an accumulation point in $S$.
            \end{definition}
            \begin{theorem}
            A subset $S$ of $(X,d)$ is sequentially compact if and only if it is limit point compact.
            \end{theorem}
            \begin{proof}
            For suppose $S$ is sequentially compact, and let $x_n$ be a sequence. As $S$ is sequentially compact there is a convergent subsequence with a limit in $S$. But then this limit is an accumulation point in $S$. Now, suppose $S$ is limit point compact. Let $x_n$ be a sequence in $S$. As $S$ is limit point compact, there is an accumulation point of $x_n$, call it $x$. But then for all $n\in \mathbb{N}$, $B_{\frac{1}{n}}(x)\cap \{x_k\}_{k=1}^{\infty}\ne \emptyset$. By the axiom of choice, we may construct a subsequence $x_{n_k}$ of points contained in each open ball. But then $d(x_{n_k},x)\rightarrow 0$. Thus, there is a convergent subsequence.
            \end{proof}
            \begin{definition}
            A subset $S$ of a metric space is said to be totally bounded if and only if for all $r>0$ there are finitely many points $x_k$ such that $S\subset \cup_{k=1}^{n} B_{r}(x_k)$.
            \end{definition}
            \begin{theorem}
            If $(X,d)$ is a metric space and $S\subset X$ is sequentially compact, then it is closed and bounded.
            \end{theorem}
            \begin{proof}
            \item Suppose it is unbounded. That is, if $x\in S$ and $n\in \mathbb{N}$, there is a $y\in S$ such that $d(x,y)>n$. Let $x_n$ be a such a sequence such that $d(x,x_n)>n$ for all $n\in \mathbb{N}$ (The existence of such a sequence requires the axiom of choice. My apologies). This sequence has no convergent subsequence, as suppose it does, say $s\in S$. But $d(x,x_n) \leq d(s,x)+d(s,x_n)$, and thus $d(x,x_n)-d(s,x)\leq d(s,x_n)$. Thus $d(s,x_n) \not\rightarrow 0$. Thus $S$ is not unbounded, and is therefore bounded. Suppose it is not closed. Then there is a point $x$ such that $x$ is a limit point of $S$ but $x\notin S$. Let $x_n$ be a sequence that converges to $S$. (Such a sequence exists as $x$ is a limit point, and the axiom of choice). But then $x_n \rightarrow x$, and thus $x\in S$. Therefore $S$ is closed.
            \end{proof}
            \begin{theorem}
            Every subset of a totally bounded space is totally bounded.
            \end{theorem}
            \begin{proof}
            For let $S\subset X$, and suppose $X$ is totally bounded. Let $r>0$. As $X$ is totally bounded, there are finitely many points such that $\cup_{k=1}^{n} B_{\frac{r}{2}}(x_k)$ contains all of $X$. Let $s_k$ be the points such that $S\subset \cup_{k=1}^{m} B_{\frac{r}{2}}(s_k)$ and $S\cap B_{\frac{r}{2}}(s_k) \ne \emptyset$. If the $s_k$ are in $S$, we are done. Suppose not. As $s_k \notin S$ and $B_{r}(s_k)\cap S \ne \emptyset$, there is an $\ell_k \in S$ such that $d(\ell_k,s_k)< \frac{r}{2}$. But then $S\subset \cup_{k=1}^{m} B_{r}(\ell_k)$. Therefore, etc.
            \end{proof}
            \begin{theorem}
            If $(X,d)$ is a metric space and $S\subset X$ is sequentially compact, then $S$ is totally bounded.
            \end{theorem}
            \begin{proof}
            For let $r>0$ and suppose that $S$ is not totally bounded. Let $s_1\in S$. There must be a point $s_2$ such that $s_2 \notin B_{r}(s_1)$, as $S$ is not contained inside the entire open disc. Similarly, there is a point $s_3\in S$ such that $s_3 \notin B_r(s_1)\cup B_r(s_2)$. In this manner we obtain $s_1, s_2, \hdots s_n, \hdots$, such that $s_n \in S$ and $s_n \notin \cup_{k=1}^{n-1}B_r(s_k)$. But as $s_n \notin B_r(s_{n-1})$, $d(s_n, s_{n-1})\geq r$. But as $s_n$ is a sequence in $S$, and as $S$ is compact, the sequence must have a convergent subsequence whose limit is some $s\in S$. Thus the ball $B_{\frac{r}{2}}(s)\cap \{x_n\}_{n=1}^{\infty}$ is infinite. But then there are points $s_{l}$ and $s_{m}$ such that $d(s_l,s_m)<r$, a contradiction. Thus $S$ is totally bounded.
            \end{proof}
            \begin{theorem}[Heine-Borel-Lebesgue Theorem]
            In a metric space $(X,d)$, with the metric space topology, if $S\subset X$ then the following are equivalent:
            \begin{enumerate}
            \item $S$ is compact.
            \item $S$ is complete and totally bounded.
            \item $S$ is sequentially compact.
            \item $S$ is limit point compact.
            \end{enumerate}
            \end{theorem}
            \begin{proof}
            We have seen that $(3)$ and $(4)$ imply each other. We now show $(1)\Rightarrow (2)\Rightarrow (3) \Rightarrow (1)$.
            \begin{enumerate}
            \item Suppose $S$ is compact. Let $r>0$ be arbitrary. Then $\cup_{x\in S}B_{r}(x)$ is an open cover of $S$, and thus there is a finite subcover. Thus $S$ is contained in a finite collection of open balls of radius $r$. Let $x_n$ be a Cauchy sequence in $S$. Suppose it does not converge. Then, for all $x\in S$ there is a $\varepsilon(x)>0$ such that $B_{\varepsilon(x)}(x)\cap \{x_n\}_{n=1}^{\infty}$ is finite. But $\cup_{x\in S}B_{\varepsilon(x)}(x)$ is a cover of $S$, and thus there is a finite subcover, suppose with $N$ open sets. But then $\{x_n\}_{n=1}^{\infty}\subset \cup_{k=1}^{N}B_{\varepsilon(x_k)}(x_k)$, a contradiction as each $B_{\varepsilon(x_k)}(x_k)$ contains but finitely many elements of $\{x_k\}_{n=1}^{\infty}$, and thus a finite union cannot contain all of $\{x_n\}_{n=1}^{\infty}$. Thus, $x_n$ converges. Therefore $S$ is complete.
            \item Let $x_n$ be a sequence in $S$. As $S$ is totally bounded, for all $n\in \mathbb{N}$ there is a finite set of points $a(n)$ such that $S\subset \underset{a(n)}\cup B_{\frac{1}{n+1}}(a(n))$. Thus there is a point $a(1)$ such that $B_{1/2}(a(1))\cap \{x_n\}_{n=1}^{\infty}$ is infinite. As $B_{1/2}(a(1))\subset S$, it is totally bounded. Thus there is a covering of finitely many points of $B_{1/2}(a(1))$ of radius $1/3$. By induction, for all $n\in \mathbb{N}$ there is a point $a(n+1)$ such that $B_{\frac{1}{n+1}}(a(n+1))\subset B_{\frac{1}{2^{n}}}(a(n))$, and $B_{\frac{1}{2^{n+1}}}(a(n+1))\cap \{x_n\}_{n=1}^{\infty}$ is infinite. By the axiom of choice, we may choose a subsequence of points $x_{n_k}$ that lie in each of these sets. But then this set is Cauchy, as for $\varepsilon>0$ there is an $N\in \mathbb{N}$ such that $n>N$ implies $\frac{1}{n}<\frac{\varepsilon}{2}$, and thus for $j,k>N$, $d(x_{n_j},x_{n_k})\leq d(x_{n_j},a(n+1))+d(x_{n_k},a(n+1))<\varepsilon$. But Cauchy sequences converge as $S$ is complete. Therefore, etc.
            \item Let $\mathcal{O}$ be an open cover and suppose no finite subcover exist. But as $S$ is sequentially compact, it is totally bounded and thus there are finitely many points such that $S\subset \underset{k}\cup B_{1}(a_k(1))$. But then one of these open balls must have no finite subcover, as the entirety of $S$ has no finite subcover. Let $a(1)$ be the center of such a set. But as $B_{1}(a(1))\cap S \subset S$, it is totally bounded as well. Thus there are finitely many points such that $B_{1}(a(1))\subset \underset{k}\cup B_{1/2}(a_k(2))$, and again there is at least one open ball that has no finite subcover, as $B_{1}(a(1))$ has no finite subcover. Inductive, we obtain a sequence of points $a(n)$ such that $B_{\frac{1}{n+1}}(a(n+1))\subset B_{\frac{1}{n}}(a(n))$ and $B_{\frac{1}{n}}(a(n))$ has no finite subcover of $\mathcal{O}$. By the axiom of choice, we may choose a sequence $a(n)$ of points of in the ball. But as $S$ is sequentially compact, there is a convergent subsequence $a(n_k)$ with some limit in $S$, call it $x$. But as $x\in S$, $x$ is covered by $\mathcal{O}$, and thus there is some open set such that $x\in \mathcal{U}$. But as $\mathcal{U}$ is open, there is an $\varepsilon>0$ such that $B_{r}(x)\subset \mathcal{U}$. But as the subsequence converges, there is an $N\in \mathbb{N}$ such that for all $k>N$, $d(a(n_k)<x) < \varepsilon$. But then for any point $y\in B_{\frac{1}{n_{N+1}}}(a(n_{N+1})$, $d(x,y) \leq d(x,a(n_{N+1}))+d(y,a(n_{N+1}))$. But then $B_{\frac{1}{n_{N+1}}}\in \mathcal{U}$, a contradiction as $B_{\frac{1}{n_{N+1}}}$ has no finite subcover. Thus, $S$ is compact. 
            \end{enumerate}
            \end{proof}
            \begin{theorem}[Heine-Borel Theorem]
            A set $S\subset \mathbb{R}$ is compact if and only if it is closed and bounded.
            \end{theorem}
            \begin{proof}
            As $S$ is compact, it is sequentially compact and thus closed and bounded. Suppose $S\subset \mathbb{R}$ is closed and bounded and let $\mathcal{O}$ be an open cover. Suppose no finite subcover exists. Denote $\Delta$ as the set of elements $x\in S$ such that for all elements $s<x$ and $s\in S$, there are indeed finitely many open sets in $\mathcal{O}$ that cover them. This set is not empty, as the greatest lower bound of $S$ is contained in it. It is also bounded by the least upper bound of $S$. Let $r$ be the least upper bound of $\Delta$. Suppose, $r\ne l.u.b.(S)$. As $r\in S$, there must be some open set $\mathcal{U}_1\in \mathcal{O}$ such that $r\in \mathcal{U}_1$. But then there is an $\varepsilon>0$ such that $B_{\varepsilon}(r) \subset \mathcal{U}_1$. As $r$ is the least upper bound of $\Delta$, $[r,r+\varepsilon)\cap S = \emptyset$. Let $r' = g.l.b.\{x\in S: x>r\}$. Then $r'\in S$ and there is a set $\mathcal{U}_2 \in \mathcal{O}$ such that $r\in \mathcal{U}_2$. But then $r' \in \Delta$ and $r'>r$, a contradiction. Thus, $r=b$. But then every element of $S$ is covered by finitely many elements of $\mathcal{O}$. Therefore every open cover of $S$ has a finite subcover.
            \end{proof}
            \begin{theorem}
            A subset of $\mathbb{R}^n$ is compact if and only if it is closed and bounded.
            \end{theorem}
            \begin{proof}
            For if $\mathcal{U}\subset \mathcal{R}^n$ is continuous, then $\pi_{j}(\mathcal{U})$ is compact for all $1\leq j \leq n$. But then $\mathcal{U}$ is the product closed and bounded spaces and is thus closed and bounded. If $\mathcal{R}^n$ is closed and bounded, then $\pi_j(\mathcal{U})$ is as well and is thus compact. But the product of compact spaces is compact. Therefore, etc.
            \end{proof}
            \begin{definition}
            The unit sphere $\mathbb{S}^{n-1}$ is defined as $\mathbb{S}^{n-1} = \{x\in \mathbb{R}^n: \norm{x}=1\}$
            \end{definition}
            The set of all compact subset of $\mathbb{R}^n$ is denoted
            $\mathscr{C}_{n}$
            \begin{theorem}
            If $S$ is a metric space $T\subset S$ is compact, and $f:T\rightarrow \mathbb{R}$ is continuous, then $f$ attains its maximum in $T$.
            \end{theorem}
            \begin{proof}
            As $f$ is continuous and $T$ is compact, $f(T)$ is compact and therefore $f(T)$ is bounded. Let $r$ be its least upper bound. For $n\in \mathbb{N}$, let $x_n$ be a point such that $|r-f(x_n)|< \frac{1}{n}$. Such a point exist as otherwise $r$ is not a least upper bound. As $T$ is compact it is limit point compact and thus there is an accumulation point in $x\in T$. From continuity, $f(x) = r$.
            \end{proof}
            \begin{definition}
            A subset $S\subset X$ of a topological space $(X,\tau)$ is said to be path-connected if and only if for every pair of points $x,y\in S$, there is a continuous function $f:[0,1]\rightarrow S$ such that $f(0)=x$ and $f(1)=y$.
            \end{definition}
            \begin{theorem}
            A path-connected set is connected.
            \end{theorem}
            \begin{proof}
            For suppose not. Let $S$ be path-connected and suppose $T\subset S$ is both open and closed and non-empty. Let $x\in T$ and $y\in S/T$. Let $f:[0,1]\rightarrow S$ be a continuous path. Let $A =\{0\leq x \leq 1: f(x) \in T\}$. This set is non-empty as $0\in A$. As it is bounded, it has a least upper bound, call it $r$. Either $f(r)\in T$ or $f(r)\in T^c$. If $f(r)\in T$, then there is a ball $B_{\varepsilon}(f(r))$ that is contained in $T$. But then from continuity of $f$, $r$ is not the least upper bound of $A$. Thus $r\notin T$. In a similar manner, $f(r)\notin T^c$, a contradiction. Thus, $S$ is connected.
            \end{proof}
        \subsection{The Real Numbers}
            We construct the "God-Given" positive integers $\mathbb{N}$, then the whole numbers $\mathbb{Z}$, rational numbers $\mathbb{Q}$, and real numbers $\mathbb{R}$.
            \begin{definition}[Peano's Axioms]
            $\mathbb{N}$ is a set with equality, a total order $\leq$, and a successor function $s$ such that:
            \begin{enumerate}
            \item $1\in \mathbb{N}$
            \item For all $n\in \mathbb{N}$, $1\leq n < s(n)$.
            \item If $n,m\in \mathbb{N}$ and $n\leq m \leq s(n)$, then either $m=n$ or $m=s(n)$.
            \item Given any set $K$, if $1\in K$ and $s(n)\in K$ for all $n\in \mathbb{N}$, then $\mathbb{N}\subset K$.
            \end{enumerate}
            \end{definition}
            \begin{theorem}
            There is no element $n\in \mathbb{N}$ such that $s(n) =1$.
            \end{theorem}
            \begin{proof}
            $[s(n) = 1]\Rightarrow [1\leq n < s(n)=1]\Rightarrow[1<1]$, a contradiction.
            \end{proof}
            \begin{theorem}
            If $n<m$, then $s(n)< s(m)$.
            \end{theorem}
            \begin{proof}
            $[n<m]\Rightarrow [s(n)\leq m] \Rightarrow [s(n) < s(m)]$.
            \end{proof}
            \begin{theorem}
            For $n,m\in \mathbb{N}$, $s(n)=s(m)$ if and only if $n=m$.
            \end{theorem}
            \begin{proof}
            $[n=m]\Rightarrow [s(n)=s(m)]$. $\big[[s(n)=s(m)]\land [n<m]\big] \Rightarrow [s(n)<s(m)]$, a contradiction.
            \end{proof}
            The successor function $s$ is the $+1$ function, $s(n)=n+1$.
            We freely write $2=1+1$, $3=1+2$, $\hdots$
            \begin{theorem}
            Every nonempty subset of $\mathbb{N}$ has a least element.
            \end{theorem}
            \begin{proof}
            Suppose not. Let $E\subset \mathbb{N}$, $E\ne\emptyset$. $[n\in E]\Rightarrow [1\leq n]\Rightarrow [1\in E^c]$. $[k\in E^c]\Rightarrow [s(k)\in E^c]\Rightarrow [\mathbb{N} \subset E^c]\Rightarrow [E = \emptyset]$.
            \end{proof}
            \begin{theorem}[Principle of Mathematical Induction]
            If $P$ is a proposition on the positive integers, if $P(1)$ is true and the truthfulness of $P(n)$ implies the truthfulness of $P(n+1)$, then $P(n)$ is true for all $n\in \mathbb{N}$.
            \end{theorem}
            \begin{proof}
            For suppose not. Then there is a least element $n$ such that $P(n)$ is false. As $P(1)$ is true, $n\ne 1$. But then $P(n-1)$ is true. But the truthfulness of $P(n-1)$ implies the truthfulness of $P(n)$. Thus $P(n)$ is true. A contradiction.
            \end{proof}
            \begin{definition}
            An $n-tuple$ is inductively defined by $(a_1,\hdots,a_{n+1}) = (a_1,\hdots, a_n)\cup \{a_1,\hdots,a_{n+1}\}$.
            \end{definition}
            \begin{definition}
            We now inductively define for any set $A$, $A^{n} = \underset{n-times}{A\times \cdots \times A}$ by $A^{n+1} = A^n \times A$.
            \end{definition}
            \begin{definition}
            The whole numbers $\mathbb{Z}$ are a group with operation $+$ with the following properties.
            \begin{enumerate}
            \item $0$ is the identity element.
            \item $\mathbb{N}\subset \mathbb{Z}$.
            \item If $0<n$, $n\in \mathbb{N}$.
            \end{enumerate}
            \end{definition}
            We have thus added all of the negative integers and $0$. The
            whole numbers are also called integers.
            \begin{definition}
            The rational numbers $\mathbb{Q}$ are an ordered field with operations $+$ and $\cdot$ such that $\mathbb{Z}\subset \mathbb{Q}$.
            \end{definition}
            This gives us all of the fractions. If $x\in \mathbb{Q}$ we may write $x= \frac{p}{q}:p,q\in \mathbb{Z}, q\ne 0$.
            \begin{definition}
            The greatest common divisor of two positive integers
            $p,q\in\mathbb{N}$ is the smallest positive integer, $r$, such
            that there are integers $n$ and $m$ such that $n\cdot{r}=p$ and
            $m\cdot{r}=q$. This number is denoted $g.c.d.(p,q)$.
            \end{definition}
            \begin{theorem}
            If $x\in \mathbb{Q}$ is positive, then there are unique positive
            integers $p, q$ such that $g.c.d.(p,q)=1$ and $x=\frac{p}{q}$.
            \end{theorem}
            \begin{proof}
            This is proved via application of the fundamental theorem of arithmetic and will be one of the few omissions.
            \end{proof}
            \begin{definition}
            A subset $A$ of $\mathbb{Q}$ is said to be bounded above if and only if $\exists M\in \mathbb{Q}: \forall x\in A,x \leq M$.
            \end{definition}
            \begin{definition}
            A subset $A$ of $\mathbb{Q}$ is said to be bounded below if and only if $\exists M\in \mathbb{Q}:\forall x\in A,M\leq x$. 
            \end{definition}
            \begin{definition}
            A subset $A$ of $\mathbb{Q}$ is said to be bounded if and only if it is both bounded below and bounded above.
            \end{definition}
            \begin{definition}
            If $A\subset \mathbb{Q}$ is bounded above, then $r$ is said to be a least upper bound of $A$ if and only if $r$ is an upper bound for $A$ and for all $s<r$ there is an element $x\in A$ such that $s<x$. We write $l.u.b.(A)$.
            \end{definition}
            \begin{definition}
            If $A\subset \mathbb{Q}$ is bounded below, then $r$ is said to be a greatest lower bound of $A$ if and only if $r$ is a lower bound for $A$ and for all $r<s$ there is an element $x\in A$ such that $x<s$. We write $g.l.b.(A)$.
            \end{definition}
            \begin{definition}
            A number $n\in \mathbb{N}$ is said to be even if and only if there is a number $k\in \mathbb{N}$ such that $n=2k$. A number $m\in \mathbb{N}$ is said to be odd if and only if there is a number $k\in \mathbb{N}$ such that $m=2k-1$.
            \end{definition}
            \begin{theorem}
            If $n\in \mathbb{N}$ and $n^2$ is even, then $n$ is even.
            \end{theorem}
            \begin{proof}
            $\big[[n^2\ even]\land [n\ odd]\big]\Rightarrow [\exists k\in \mathbb{N}:n=2k-1]\Rightarrow [n^2 = 4k(k-1)+1]\Rightarrow [n^2\ odd]$, a contradiction. Thus, $n$ is even.
            \end{proof}
            \begin{theorem}
            There is no rational number $q$ such that $q^2 = 2$.
            \end{theorem}
            \begin{proof}
            $\big[[x\in \mathbb{Q}]\land [x^2=2]\big]\Rightarrow [x= \frac{p}{q}:g.c.d.(p,q)=1]\Rightarrow [\frac{p^2}{q^2}= 2]\Rightarrow [p^2 = 2q^2]\Rightarrow [p\ even]\Rightarrow [\exists k\in \mathbb{N}:p=2k]\Rightarrow [\frac{4k^2}{q^2}=2]\Rightarrow [q^2 = 2k^2]\Rightarrow [q\ even]\Rightarrow [g.c.d.(p,q)\geq 2]$, a contradiction.
            \end{proof}
            \begin{theorem}
            There exist bounded subsets of $\mathbb{Q}$ that contain no least upper bound.
            \end{theorem}
            \begin{proof}
            Let $E=\{x\in \mathbb{Q}:x^2 < 2\}$. It is bounded above by $2$. Suppose $s\in \mathbb{Q}$ is the least upper bound of $E$. Let $x = s - \frac{s^2-2}{s+2}$. $[x\in \mathbb{Q}] \land [x^2 = 2\frac{s^2-2}{(s+2)^2}+2]$. $[s^2<2]\Rightarrow \big[[x^2<2 ]\Rightarrow [x\in E]\big]\land [s<x]$, a contradiction as $s$ is an upper bound of $E$. $[s^2>2]\Rightarrow \big[[2<x^2 ]\land [x<s]\big]\Rightarrow$, a contradiction as $s$ is the least upper bound. Therefore, etc.
            \end{proof}
            \begin{definition}
            $\mathbb{R}$ is an ordered field, $\mathbb{Q}\subset \mathbb{R}$ : every nonempty, bounded above subset has a least upper bound.
            \end{definition}
            \begin{theorem}
            Least upper bounds are unique.
            \end{theorem}
            \begin{proof}
            If $A$ is a bounded set, $r\ne r'$ are least upper bounds, then either $r<r'$ or $r'<r$, a contradiction.
            \end{proof}
            \begin{theorem}
            If $A$ is a bounded below set, then there is a greatest lower bound.
            \end{theorem}
            \begin{proof}
            Let $-M<0$ be a bound and define $-A = \{-x: x\in A\}$. $[-x\in -A]\Rightarrow [x\in A]\Rightarrow [-x\leq M]\Rightarrow [-A\ is\ bounded\ above]\\ \Rightarrow [\exists l.u.b.(-A)]$. $[-x\in -A]\Rightarrow [x\in A]\Rightarrow [x\leq -l.u.b.(A)]\Rightarrow [-l.u.b.(A)\leq -x]\Rightarrow [g.l.b.(-A)=-l.u.b.(A)]$
            \end{proof}
            \begin{theorem}[The Archimedean Principle]
            For every $x\in \mathbb{R}$ there is a least $n\in \mathbb{N}$ such that $x<n$. 
            \end{theorem}
            \begin{proof}
            If $x\leq1$, let $n=1$. Let $x>1$ and $E=\{i \in \mathbb{Z}: 0 \leq i \leq x\}$. $[0\in E]\Rightarrow [E\ne \emptyset]$. $[i\in E]\Rightarrow [i\leq x]\Rightarrow [\exists l.u.b.(E)]$. Let $l.u.b.(E)=s$. $[s-1<s]\Rightarrow [\exists i \in E:s-1 \leq i \leq s]$. $[i< s]\Rightarrow[\exists m\in E: i < m \leq s]\Rightarrow [0 < m-i \leq s-1 < 1]$. But $[m-i \in \mathbb{Z}]\Rightarrow [0<m-i<1\ is\ false]\Rightarrow [i = s]\Rightarrow s\in \mathbb{N}$. If $x=s$, $n = s+1$. Otherwise, $n=s$.
            \end{proof}
            \begin{theorem}
            For every $x\in \mathbb{R}$ there is a least $n\in \mathbb{N}$ such that $-n<x$.
            \end{theorem}
            \begin{proof}
            There is a least $n\in \mathbb{N}$ such that $(-x)<n$. But then $-n <-(-x) = x$. 
            \end{proof}
            \begin{theorem}
            If $x,y\in \mathbb{R}$ and $x>0$, then there is an
            $n\in \mathbb{N}$ such that $nx>y$.
            \end{theorem}
            \begin{proof}
            If $y\leq 0$, $n=1$. If $y>1$, let $r = \frac{y}{x}$. $[x,y>0]\Rightarrow [\frac{y}{x}>0]\Rightarrow [\exists n\in \mathbb{N}:n>r]\Rightarrow [nx > rx = \frac{y}{x}x = y]\Rightarrow[nx>y]$.
            \end{proof}
            \begin{theorem}
            If $0\leq y$, there is a a unique number $x>0$ such that $x^2 = y$.
            \end{theorem}
            \begin{proof}
            $\big[[x^2=y]\land [x'^2=y]\land [x\ne x']\big] \Rightarrow \big[[x<x']\lor[x'<x]\big] \Rightarrow \big[[2=x^2<xx'<x'^2=2]\lor[2=x'^2<x'x<x^2=2]\big]$, a contradiction. Thus, uniqueness is proved. For existence, $[y=0]\Rightarrow[x=0]$.$[y=1]\Rightarrow [x=1]$. Let $0 < y < 1$ and define $A = \{x\geq0:x^2 \leq y\}$. $[0\in A]\Rightarrow[A\ne \emptyset]$. $[y<1]\Rightarrow [A\ is\ bounded\ above]$. Let $r$ be the least upper bound. Suppose $r^2\ne y$.
            \begin{enumerate}
            \item $[y<r^2]\Rightarrow[\frac{r^2-y}{2}>0]\Rightarrow [r-\frac{r^2-y}{2}<r]\land[(r-\frac{r^2-y}{2})^2= r^2 - (r^2-y)+\big(\frac{r^2-y}{2}\big)^2 = y + \big(\frac{r^2-y}{2}\big)^2 < y]$. A contradiction.
            \item $[r^2 <y]\Rightarrow [0<\frac{y-r^2}{2r+1}<1]\Rightarrow [r^2 + 2r\frac{y-r^2}{2r+1}+\big(\frac{y-r^2}{2r+1}\big)^2\leq r^2 + 2r\frac{y-r^2}{2r+1}+\frac{y-r^2}{2r+1} = r^2+\frac{y-r^2}{2r+1}(2r+1)=y]$. A contradiction.
            \end{enumerate}
            Thus, $r^2 = y$.
            \end{proof}
            \begin{definition}
            If $x>0$, then $\sqrt{x}$ is the unique positive number such that $(\sqrt{x})^2 = x$. This is the $square-root$ of $x$.
            \end{definition}
            \begin{theorem}
            $1<\sqrt{2}$
            \end{theorem}
            \begin{proof}
                For $\sqrt{2} \ne 1$, as $1^2 = 1\ne 2$. If $\sqrt{2}<1$,
                then $2=(\sqrt{2})^2 <1$, a contradiction. Thus $1<\sqrt{2}$.
            \end{proof}
            \begin{definition}
                An irrational number is a real number that is not rational.
            \end{definition}
            \begin{theorem}
                $\frac{1}{\sqrt{2}}$ is irrational. 
            \end{theorem}
            \begin{proof}
                For if $\frac{1}{\sqrt{2}}=\frac{p}{q}$, $p,q\in \mathbb{N}$,
                then $\sqrt{2}=\frac{q}{p}$, a contradiction.
            \end{proof}
            \begin{theorem}
            If $q$ is a rational number not equal to zero, and $r$ is irrational, then $rq$ is irrational.
            \end{theorem}
            \begin{proof}
            As $q\ne 0$, let $q = \frac{n}{m}$ be in reduced form. Suppose $rq = \frac{x}{y}\in \mathbb{Q}$. Then $r=\frac{xm}{yn}$, a contradiction.
            \end{proof}
            \begin{theorem}
            Given a rational number $q$, and for any $\varepsilon>0$, there is an irrational number $r$ such that $|r-q|<\varepsilon$.
            \end{theorem}
            \begin{proof}
            $[\varepsilon>0]\Rightarrow [\frac{1}{\varepsilon}>]0\Rightarrow [\exists N\in \mathbb{N}:\frac{1}{\varepsilon}<N]\Rightarrow [\frac{1}{\varepsilon} < \sqrt{2}N]\Rightarrow [\frac{1}{\sqrt{2}N}< \varepsilon]$. $[r \equiv q+\frac{1}{\sqrt{2}{N}}]\Rightarrow [r\notin \mathbb{Q}]\land [|r-q| = |\frac{1}{\sqrt{2}N}| < \varepsilon]$.
            \end{proof}
            \begin{theorem}
            If $r$ is an irrational number, and $\varepsilon>0$, then there is a rational number $q$ such that $|r-q|<\varepsilon$.
            \end{theorem}
            \begin{proof}
            $[0<r]\Rightarrow \big[[\exists n\in \mathbb{N}: \frac{1}{n} < \varepsilon]\land[\exists m\in \mathbb{N}: m-1\leq nr \leq m]\big]\Rightarrow[|r-\frac{m}{n}| \leq \frac{1}{n} < \varepsilon]$. Similarly if $r<0$.
            \end{proof}
            \begin{definition}
            The absolute value function is defined on $\mathbb{R}$ as $|x| = \begin{cases} x, & 0 \leq x \\ -x, & x<0 \end{cases}$
            \end{definition}
            \begin{theorem}
            $|x| = \sqrt{x^2}$.
            \end{theorem}
            \begin{proof}
            If $0 \leq x$, we are done. If $x<0$, then $|x| = (-x) = \sqrt{(-x)^2} = \sqrt{x^2}$.
            \end{proof}
            \begin{theorem}
            For $x,y \in \mathbb{R}$, $|x+y|\leq |x|+|y|$
            \end{theorem}
            \begin{proof}
            $[0\leq x,y]\Rightarrow [|x+y| = x+y = |x|+|y|]$. $[x,y\leq 0]\Rightarrow [|x+y| = -(x+y) = (-x)+(-y)=|x|+|y|]$. $[x\leq 0 \leq y]\Rightarrow [0\leq x+y]\Rightarrow [|x+y| = x+y \leq (-x)+y=|x|+|y|]$. Similarly for $y\leq 0 \leq x$.
            \end{proof}
            \begin{theorem}[Triangle Inequality]
            If $x,y,z\in \mathbb{R}$, then $|x-y| \leq |x-z|+|y-z|$.
            \end{theorem}
            \begin{proof}
            For $|x-y| = |x+ 0 - y| = |x-z+z-y| = |(x-z)+(z-y)| \leq |x-z|+|z-y| = |x-y|+|y-z|$.
            \end{proof}
            \begin{theorem}
            If $\varepsilon >0$ and $|x|<\varepsilon$, then $-\varepsilon < x < \varepsilon$.
            \end{theorem}
            \begin{proof}
            For if $0\leq x$, then $-\varepsilon< x=|x|< \varepsilon$. If $x\leq 0$, then $x\leq 0<\varepsilon$ and $(-x)=|x|<\varepsilon$, thus $-\varepsilon < -(-x) = x$.
            \end{proof}
            \begin{definition}
            A sequence in $A$ is a function $x_n$ is a function $x_n:\mathbb{N}\rightarrow A$. We write the image of $n\in \mathbb{N}$ as $n\mapsto x_n$.
            \end{definition}
            \begin{definition}
            If $x_n$ is a sequence, a subsequence is a subset of $x_n$, denoted $x_{n_k}$, where $n_k\in \mathbb{N}$ is strictly increasing.
            \end{definition}
            \begin{definition}
            A sequence $x_n$ converges to $x$ if and only if $\forall \varepsilon>0,\ \exists N\in \mathbb{N}: n>N\Rightarrow |x-x_n|<\varepsilon$. We write $x_n \rightarrow x$.
            \end{definition}
            \begin{theorem}
            Convergence in $\mathbb{R}$ is unique.
            \end{theorem}
            \begin{proof}
            For suppose not. Let $x_n \rightarrow x$ and $x_n \rightarrow x'$ and suppose $x\ne x'$. Let $\varepsilon = \frac{|x-x'|}{2}$. $[x\ne x']\Rightarrow [\varepsilon>0]\Rightarrow [\exists N\in\mathbb{N}:|x_n-x|<\varepsilon\land |x_n-x'| <\varepsilon]\Rightarrow \big[|x-x'|=|x-x_n+x_n-x'|\leq |x_n-x|+|x_n-x'|<2\varepsilon = |x-x'|\big]$, a contradiction.
            \end{proof}
            \begin{definition}
            A sequence is Cauchy if and only if $\forall \varepsilon>0,\ \exists N\in \mathbb{N}: n,m>N\Rightarrow |x_n-x_m|<\varepsilon$.
            \end{definition}
            \begin{definition}
            A closed interval $[a,b]$ is a subset of $\mathbb{R}$ defined as $[a,b] = \{x\in\mathbb{R}:a\leq x\leq b\}$. 
            \end{definition}
            \begin{definition}
            A sequence is said to be monotonically increasing if and only if for all $n\in \mathbb{N}$, $x_n \leq x_{n+1}$, monotonically decreasing if and only if for all $n\in \mathbb{N}$, $x_{n+1} \leq x_{n}$, and monotonic if and only if it is either monotonically decreasing or monotonically increasing. Strictly increasing or strictly decreasing means $x_{n}<x_{n+1}$ and $x_{n+1}<x_n$, respectively.
            \end{definition}
            \begin{theorem}
            Every sequence in $\mathbb{R}$ has a monotonic subseqence.
            \end{theorem}
            \begin{proof}
            If $n\in \mathbb{N}:n<m\Rightarrow x_m \leq x_n$, call $n$ a peak point. If there is a sequence of peak points $n_k$, then the subsequence $x_{n_k}$ is a sequence of peak points and is thus monotonic. If none such subsequence exist, there is a greatest peak point, call it $n_1$. Then there is an $n_2\in \mathbb{N}$ such that $n_1 < n_2$ and $x_{n_1}< x_{n_2}$, otherwise there is a peak point greater than $n_1$. Inductively, there is a strictly increasing sequence $n_k$ such that $x_{n_k}< x_{n_{k+1}}$. Therefore, etc.
            \end{proof}
            \begin{definition}
            A Dedekind Cut is a combination of two sets $A$ and $B$ such that for all $x\in A$ and all $y\in B$, $x< y$, $A\cap B=\emptyset$, and $\mathbb{Q} \subset A\cup B$. A real number $r$ is said to produce a Dedekind cut if and only if $\forall a\in A\land \forall b\in B, a\leq r\leq b$.
            \end{definition}
            \begin{theorem}
            The following are equivalent characterizations of the completeness of $\mathbb{R}$.
            \begin{enumerate}
                \item Dedekind Cuts are produced by a unique real number. \hfill [Dedekind Completeness]
                \item Bounded monotonic sequences converge. \hfill [Monotone Convergence Theorem]
                \item If $x_n$ is a bounded sequence, then there exist a convergent subsequence. \hfill [Bolzano-Weierstrass Theorem]
                \item Cauchy Sequences Converge. \hfill [Cauchy Completeness]
                \item If $I_n = [a_n,b_n]\subset [a_{n+1},b_{n+1}]=I_{n+1}$ are a sequence of non-empty closed intervals and $b_n-a_n \rightarrow 0$, then there is a unique point $x$ that is contained in all intervals $I_n$. \hfill [Cantor's Nested Intervals Theorem]
            \end{enumerate}
            \end{theorem}
            \begin{proof}
            We show that $(1)\Rightarrow (2) \Rightarrow (3)\Rightarrow (4)\Rightarrow (5)\Rightarrow (1)$, where $\Rightarrow$ means implies.
            \begin{enumerate}
            \item For let $A$ and $B$ be a Dedekind Cut of $\mathbb{Q}$. It suffices to show that the least upper bound of $A$ is equal to the greatest lower bound of $B$. Let $r$ be the least upper bound of $A$ and $s$ the greatest lower bound of $B$. If $x\in \mathbb{Q}$ and $x<s$, then $x\notin B$. But as $A\cup B = \mathbb{Q}$, $x\in A$. Similarly, if $r<x$, then $x\in B$. Thus $s=r$.
            \item For let $x_n$ be a bounded monotonic sequence, suppose increasing, in $\mathbb{R}$ and let $A=\{M\in \mathbb{R}:x_n \leq M\ \textrm{for all } n\in \mathbb{N}\}$. Then $A$ and $A^c$ form a Dedekind cut of $\mathbb{Q}$ and is thus produced by a real number, call it $r$. Then $r$ is a greatest lower bound of $A$. Let $\varepsilon>0$ be arbitrary. Then, as $r$ is a greatest lower bound, there is an $N\in \mathbb{N}$ such that $r-\varepsilon < x_n$. But as $x_n$ is monotonic, for all $k\in \mathbb{N}$, $r-\varepsilon < x_{n+k}\leq r$. Thus, $x_n \rightarrow r$
            \item For let $x_n$ be a bounded sequence. As all sequence have a monotonic subsequence, let $x_{n_k}$ be such a subsequence. But then $x_{n_k}$ is a bounded monotonic subsequence and thus converges.
            \item Let $x_n$ be a Cauchy sequence and let $\varepsilon>0$ be arbitrary. Then there is an $N\in \mathbb{N}$ such that for all $n,m>N$, $|x_n-x_m|<\frac{\varepsilon}{2}$. Then $-\varepsilon < x_n - x_{N+1} < \varepsilon$, and thus $x_{N+1}-\varepsilon < x_n < \varepsilon + x_{N+1}$. Thus, $x_n$ is a bounded sequence. But bounded sequences have a convergent subsequence $x_{n_k}$. Let $x$ be the limit. Then, for $n>N$, $|x-x_n| \leq |x-x_{n_k}|+|x_{n_k}-x_n| < \frac{\varepsilon}{2} + \frac{\varepsilon}{2} = \varepsilon$.
            \item Let $x_n = \begin{cases} a_n, & n\ \textrm{is even}. \\ b_{n}, & n\ \textrm{is odd}. \end{cases}$. As $b_n - a_n \rightarrow 0$ and $a_n$ and $b_n$ are monotonic (As $I_n \subset I_{n+1}$), $x_n$ is a Cauchy sequence and thus converges. Let $x$ be the limit. But then $a_n \leq x \leq b_n$ for all $n\in \mathbb{N}$, and for any $x'\ne x$, let $\varepsilon = \frac{|x-x'|}{2}$. As $b_n - a_n \rightarrow 0$, there is an $N\in \mathbb{N}$ such that for all $n>N$, $|b_n-a_n|<\varepsilon$, thus $x' \notin [a_{N+1},b_{N+1}]$. $x$ is unique.
            \item Finally, $(5)\Rightarrow (1)$. Let $A$ and $B$ be a Dedekind Cut of $\mathbb{Q}$. Let $x_1 \in A$ be arbitrary and $x_2 \in B$ be arbitrary and defined $x_3 = \frac{x_1+x_2}{2}$. Define\\ $x_n = \begin{cases} \frac{x_{n-1}+x_{n-2}}{2}, \textrm{The Previous Two Terms are in Different Cuts}\\\frac{x_{n-1}+x_{n-3}}{2}, \textrm{The Previous Two Terms are in the Same Cut}\end{cases}$
            For all $n\in \mathbb{N}$, define the following:
            \begin{paracol}{2}
            \begin{enumerate}
            \item $a_n = \begin{cases} x_n, & x_n \in A \\ a_{n-1}, & x_n \notin A\end{cases}$
            \switchcolumn
            \item $b_n = \begin{cases} x_n, & x_n \in B \\ b_{n-1}, & x_n \notin B\end{cases}$
            \end{enumerate}
            \end{paracol}
            Then $I_{n+1} = [a_{n+1},b_{n+1}] \subset [a_n,b_n]=I_n$, and $b_n-a_n \rightarrow 0$. Thus there is a unique point $x\in I_n$ for all $n\in \mathbb{N}$. This produces the Dedekind cut, as for all $a\in A$, $a\leq x$ and for all $b\in B$, $x\leq b$. Therefore, etc.
            \end{enumerate}
            \end{proof}
        \subsection{Vector Spaces and Euclidean Spaces}
            \begin{definition}
            A vector space is a set $V$ of vectors and a field $K$ of scalars with the following properties: For all $a,b\in K$, $\mathbf{u,v,w}\in V$:
            \begin{enumerate}
                \item $a\mathbf{v} \in V$. \hfill [Closure of Scalar Multiplication]
                \item $\mathbf{v}+\mathbf{u} \in V$. \hfill [Closure of Vector Addition]
                \item $\mathbf{u}+(\mathbf{v}+\mathbf{w}) = (\mathbf{u}+\mathbf{v})+\mathbf{w}$ \hfill [Vector Addition is Associative]
                \item $\mathbf{u}+\mathbf{v}=\mathbf{v}+\mathbf{u}$ \hfill [Vector Addition is Commutative]
                \item There exists a $\mathbf{0}\in V$ such that $\mathbf{0}+\mathbf{v}=\mathbf{v}$ for all $\mathbf{v}\in V$. \hfill [Existence of Zero Vector]
                \item $a(b\mathbf{v}) = (ab)\mathbf{v}$. \hfill [Associativity of Scalar Multiplication]
                \item $1 \mathbf{v} = \mathbf{v}$. \hfill[Multiplication by Scalar Identity]
                \item $a(\mathbf{v}+\mathbf{u}) = a\mathbf{v}+a\mathbf{u}$. \hfill [Scalar Multiplication Distributes of Vector Addition]
                \item $(a+b)\mathbf{v}= a\mathbf{v}+b\mathbf{v}$. \hfill [Scalar Multiplication Distributes over Field Addition]
            \end{enumerate}
            \end{definition}
            \begin{theorem}
            $\mathbf{0}$ is unique.
            \end{theorem}
            \begin{proof}
            For $\mathbf{0}'=\mathbf{0}'+\mathbf{0}=\mathbf{0}$.
            \end{proof}
            \begin{theorem}
            $0\mathbf{v} = \mathbf{0}$.
            \end{theorem}
            \begin{proof}
            For $\mathbf{v}+0\mathbf{v} = (1+0)\mathbf{v} = 1\mathbf{v} = \mathbf{v}$. As $\mathbf{0}$ is unique, $0\mathbf{v}=\mathbf{0}$.
            \end{proof}
            \begin{theorem}
            For all $\mathbf{v}\in V$, there exists a $\mathbf{u}$ such that $\mathbf{v}+\mathbf{u}=0$. That is, additive inverses exist.
            \end{theorem}
            \begin{proof}
            For let $\mathbf{u} = (-1)\mathbf{v}$. Then $\mathbf{v}+\mathbf{u} = \mathbf{v}+(-1)\mathbf{v} = (1+(-1))\mathbf{v} = 0\mathbf{v} = \mathbf{0}$.
            \end{proof}
            \begin{theorem}
            Inverses are unique.
            \end{theorem}
            \begin{proof}
            For $-\mathbf{v}'=-\mathbf{v}'+\mathbf{0}=-\mathbf{v}'+\mathbf{v}-\mathbf{v}=- \mathbf{v}$
            \end{proof}
            \begin{definition}
            A subspace of a vector space $V$ over a field $K$ is a vector space $W$ with the following properties:
            \begin{enumerate}
                \item $\mathbf{0} \in W$
                \item If $\mathbf{u,v}\in W$, then $\mathbf{u}+\mathbf{v} \in W$
                \item For all $a\in K$ and $\mathbf{V} \in W$, $a\mathbf{v} \in W$
            \end{enumerate}
            \end{definition}
            \begin{definition}
            An affine subspace of a vector space $V$ over a field $K$ is a subset $\xi\subset V$ such that $\xi = \{v+w:w\in W\}$, where $v$ is a fixed vector in $V$, and $W$ is a fixed subspace of $V$. That is, they are translations of subspaces.
            \end{definition}
            \begin{definition}
            An inner product on a vector space $V$ over a subfield $K$ of $\mathbb{R}$ is a function $\langle , \rangle:V\times V\rightarrow \mathbb{R}$ with the following properties: For all $x,y,z \in V,$ and $\alpha \in K$,
            \begin{enumerate}
                \item $\langle x,y \rangle = \langle y,x \rangle$ \hfill [Symmetry]
                \item $\langle \alpha x, y \rangle = \alpha \langle x,y \rangle$ \hfill [Linearity]
                \item $\langle x+y,z \rangle = \langle x,z\rangle + \langle y,z \rangle$ \hfill [Linearity]
                \item  If $x\ne \mathbf{0}$, then $\langle x,x\rangle >0$ \hfill [Positiveness]
            \end{enumerate}
            \end{definition}
            \begin{definition}
            An inner product space is a vector space with an inner product.
            \end{definition}
            \begin{theorem}
            $\langle x,y+z\rangle=\langle x,y\rangle+\langle x,z\rangle$
            \end{theorem}
            \begin{proof}
            For $\langle x,y+z\rangle=\langle y+z,x\rangle=\langle y,x\rangle+\langle z,x\rangle=\langle x,y\rangle+\langle x,z\rangle$
            \end{proof}
            \begin{theorem}
            $\langle x,x \rangle = 0$ if and only if $x= \mathbf{0}$.
            \end{theorem}
            \begin{proof}
            For $\langle \mathbf{0}, \mathbf{0} \rangle = \langle 0\mathbf{0},\mathbf{0} \rangle = 0 \langle \mathbf{0},\mathbf{0}\rangle = 0$. Suppose $\langle x,x \rangle =0$ but $x\ne \mathbf{0}$. But then $\langle x,x \rangle >0$, a contradiction.
            \end{proof}
            \begin{theorem}
            For all $x\in V$, $\langle \mathbf{0},x \rangle = 0$
            \end{theorem}
            \begin{proof}
            For $\langle \mathbf{0}, x\rangle = \langle 0\mathbf{0},x \rangle = 0\langle \mathbf{0},x\rangle = 0$.
            \end{proof}
            \begin{theorem}[Cauchy-Bunyakovsky-Schwarz Inequality]
            In an inner product space $V$, $x,y\in V\Rightarrow \langle x,y \rangle^2 \leq \langle x,x \rangle \langle y,y \rangle$
            \end{theorem}
            \begin{proof}
            $[y=\mathbf{0}]\Rightarrow [\langle x,y\rangle = 0]$. Suppose $y\ne \mathbf{0}$, and let $\lambda = \frac{\langle x,y \rangle}{\langle y,y \rangle}$. Then $[0 \leq \langle x-\lambda y, x-\lambda y\rangle = \langle x,x \rangle - 2\lambda \langle x,y \rangle + \lambda^2 \langle y,y \rangle]\Rightarrow [0\leq \langle x,x \rangle - 2\frac{\langle x,y \rangle ^2 }{\langle y,y \rangle} + \frac{\langle x,y \rangle^2}{\langle y,y \rangle} = \langle x,x \rangle - \frac{\langle x,y \rangle^2}{\langle y,y \rangle}]\Rightarrow [\frac{\langle x,y \rangle ^2}{\langle y,y \rangle} \leq \langle x,x \rangle]\Rightarrow [\langle x,y \rangle^2 \leq \langle x,x \rangle \langle y,y \rangle]$.
            \end{proof}
            \begin{definition}
            A norm on a vector space $V$ over a subfield $K$ of $\mathbb{R}$ is a function $\norm{}:V\rightarrow \mathbb{R}$ with the following properties: For all $x \in V$ and $\alpha \in K$:
            \begin{enumerate}
            \item $\norm{\alpha x} = |\alpha| \norm{x}$ \hfill [Absolute Homogeneity]
            \item $\norm{x+y} \leq \norm{x}+\norm{y}$ \hfill [Triangle Inequality]
            \item $\norm{x} = 0$ if and only if $x = \mathbf{0}$. \hfill [Definiteness]
            \end{enumerate}
            \end{definition}
            \begin{definition}
            A normed space is a vector space with a norm.
            \end{definition}
            \begin{theorem}
            If $V$ is a normed space, then for all $x\in V$, $0\leq \norm{x}$
            \end{theorem}
            \begin{proof}
            For $0=\norm{0} = \norm{\frac{x-x}{2}} \leq \norm{\frac{x}{2}}+\norm{-\frac{x}{2}} = \frac{1}{2}\norm{x} + \frac{1}{2}\norm{x} = \norm{x}$.
            \end{proof}
            \begin{definition}
            If $V$ is an inner product space, then the induced norm is $\norm{x}=\sqrt{\langle x,x \rangle}$.
            \end{definition}
            \begin{theorem}
            The induced norm is a norm.
            \end{theorem}
            \begin{proof}
            In order,
            \begin{enumerate}
            \item $\norm{\alpha x} = \sqrt{\langle \alpha x, \alpha x \rangle} = \sqrt{\alpha^2 \langle x,x\rangle} = |\alpha| \sqrt{\langle x,x \rangle} = |\alpha| \norm{x}$
            \item $\norm{x+y}^2= \langle x,x \rangle + 2\langle x,y \rangle + \langle y,y \rangle = \norm{x}^2 + 2\langle x,y \rangle + \norm{y}^2 \leq \norm{x}^2 +2\norm{x}\norm{y} + \norm{y}^2 = (\norm{x}+\norm{y})^2\Rightarrow \norm{x+y}\leq \norm{x}+\norm{y}$
            \item If $x= \mathbf{0}$, then $\sqrt{\langle x,x \rangle} = \sqrt{0} = 0$. If $\sqrt{\langle x,x \rangle} = 0$ then $\langle x,x \rangle = 0$, and thus $x = \mathbf{0}$.
            \end{enumerate}
            \end{proof}
            \begin{theorem}[Cauchy-Schwarz Inequality]
            If $V$ is an inner product space, then for all $x,y \in V$, $|\langle x,y\rangle| \leq \norm{x} \norm{y}$.
            \end{theorem}
            \begin{proof}
            For $\langle x,y \rangle^2 \leq \langle x,x \rangle \langle y,y \rangle = \norm{x}^2 \norm{y}^2 = (\norm{x}\norm{y})^2$. Thus, $|\langle x,y \rangle| \leq \norm{x}\norm{y}$
            \end{proof}
            \begin{definition}
            Euclidean $n$-space is defined as $\mathbb{R}^n=\{(x_1,\hdots, x_n):x_1,\hdots, x_n \in \mathbb{R}\}$, and has the following arithmetic: For all $x,y\in \mathbb{R}^n$, $\alpha \in \mathbb{R}$,
            \begin{enumerate}
            \item $x+y = (x_1+y_1,\hdots, x_n+y_n)$
            \item $\alpha x = (\alpha x_1,\hdots, \alpha x_n)$.
            \end{enumerate}
            \end{definition}
            \begin{theorem}
            $\mathbb{R}^n$, with its usual arithmetic, is a vector space over $\mathbb{R}$.
            \end{theorem}
            \begin{proof}
            In order (Laboriously): Let $\alpha, \beta \in \mathbb{R}$, $x,y,z\in \mathbb{R}^n$,
            \begin{enumerate}
            \item $\alpha x = \alpha(x_1,\hdots,x_n) = (\alpha x_1,\hdots, \alpha x_n)$. As $\alpha x_i \in \mathbb{R}$, $\alpha x \in \mathbb{R}^n$.
            \item $x+y = (x_1+y_1,\hdots,x_n+y_n)$. As $x_i+y_i \in \mathbb{R}$, $x+y\in \mathbb{R}^n$.
            \item $x+(y+z) = (x_1+(y_1+z_z),\hdots, x_n+(y_n+z_n)) = ((x_1+y_1)+z_1,\hdots, (x_n+y_n)+z_n) = (x+y)+z$
            \item $x+y = (x_1+y_1,\hdots,x_n+y_n) = (y_1+x_1,\hdots, y_n+x_n)=y+x$
            \item $\mathbf{0}+x = (0+x_1,\hdots, 0+x_n) = (x_1,\hdots, x_n) = x$.
            \item $\alpha(\beta x) = \alpha(\beta x_1,\hdots, \beta x_n) = \alpha \beta (x_1,\hdots, x_n) = (\alpha \beta) x$
            \item $1 x = (x_1,\hdots, x_n) = x$.
            \item
                \begin{align*}
                    \alpha(x+y) &= \alpha(x_1+y_1,\hdots, x_n+y_n) & &= (\alpha x_1, \hdots, \alpha x_n) + (\alpha y_1,\hdots, \alpha y_n)\\
                    &= (\alpha(x_1+y_n),\hdots, \alpha(x_n+y_n)) & &= \alpha(x_1,\hdots, x_n)+\alpha(y_1,\hdots, y_n)\\
                    &= (\alpha x_1+\alpha y_1,\hdots, \alpha x_n + \alpha y_n) & &= \alpha x + \alpha y
                \end{align*} 
            \item
                \begin{align*}
                    (\alpha + \beta)x &= ((\alpha+\beta)x_1,\hdots, (\alpha+\beta)x_n) & &= \alpha (x_1, \hdots, x_n)+\beta (x_1, \hdots, x_n)\\
                    &= (\alpha x_1 + \beta x_1,\hdots, \alpha x_n + \beta x_n) & &= \alpha x+\beta x\\
                    &= (\alpha x_1,\hdots, \alpha x_n) + (\beta x_1,\hdots, \beta x_n)
                \end{align*}
            \end{enumerate}
            \end{proof}
            \begin{definition}
            The dot product is a function $\cdot:\mathbb{R}^n \times \mathbb{R}^n \rightarrow \mathbb{R}$ defined as $x\cdot y = \sum_{i=1}^{n} x_iy_i$
            \end{definition}
            \begin{theorem}
            The dot product is an inner product on $\mathbb{R}^n$.
            \end{theorem}
            \begin{proof}
            In order,
            \begin{enumerate}
            \item $x\cdot y = \sum_{i=1}{n} x_i y_i = \sum_{i=1}^{n} y_i x_i = y\cdot x$
            \item $\alpha x\cdot y = \sum_{i=1}^{n} \alpha x_i y_i = \alpha \sum_{i=1}^{n} x_i y_i = \alpha x_i \cdot y_i$
            \item $(x+y)\cdot z = \sum_{i=1}^{n} (x_i+y_i)z_i = \sum_{i=1}^{n} (x_iz_i +y_i z_i)=\sum_{i=1}^{n}x_i z_i+\sum_{i=1}^{n} y_i z_i = x\cdot z + y\cdot z$
            \item $x\cdot x = \sum_{i=1}^{n} x_i^2 \geq 0$. Indeed, if $x\cdot x = 0$, then $x_i = 0$ for all $i=1,\hdots, n$, and thus $x=\mathbf{0}$.
            \end{enumerate}
            \end{proof}
            The induced norm is thus $\norm{x} = \sqrt{\sum_{i=1}^{n} x_i^2}$.
            \begin{definition}
            A linear combination of vectors $\mathbf{v_i}$ in a vector space $V$ over a field $K$ is a sum $\sum_{i=1}^{n} a_i \mathbf{v_i}$, $a_i \in K$.
            \end{definition}
            \begin{definition}
            If $V$ is a vector space over $K$, and if $\mathbf{v_1},\hdots, \mathbf{v_n}\in V$, then they are said to be linearly dependent if and only if there are scalars $a_1,\hdots, a_n$, not all equal to zero, such that $\sum_{i=1}^{n} a_i \mathbf{v}_i = 0$.
            \end{definition}
            \begin{definition}
            If $V$ is a vector space over $K$, and if $\mathbf{v_1},\hdots, \mathbf{v_n}\in V$, then they are said to be linearly independent if and only if they are not linearly dependent.
            \end{definition}
            \begin{definition}
            A set of a vectors $\mathbf{v_1},\hdots, \mathbf{v_n}$ is said to span a vector space $V$ if and only if every element $x\in V$ can be written as a linear combination $x=\sum_{i=1}^{n} a_i \mathbf{v_i}$ for some scalars $a_i \in K$.
            \end{definition}
            \begin{definition}
            A basis of a vector space $V$ of a field $K$ is a set of linearly independent vectors that span $V$.
            \end{definition}
            \begin{definition}
            A vector space $V$ is said to be finite if it has a basis of finitely many elements.
            \end{definition}
            \begin{definition}
            The dimension of a vector space $V$, denoted $\dim(V)$, is the cardinality of the smallest basis of $V$.
            \end{definition}
            \begin{theorem}[The Dimension Theorem]
            If $V$ is a vector space and $\dim(V)=n$, then every basis of $V$ has $n$ vectors.
            \end{theorem}
            \begin{definition}
            If $V$ is a normed space, then an affine combination of vectors is $\sum_{k=1}^{n} \lambda_k v_k$, where $\lambda_k \in K$, $v_k \in V$, and $\sum_{k=1}^{n} \lambda_k = 1$.
            \end{definition}
            \begin{definition}
            A set of $n$ vectors $\{v_k:k\in \mathbb{Z}_n\}$ is said to be affinely dependent if and only if there exists scalars $\lambda_k \in K$, not all equal to zero, such that $\sum_{k=1}^{n} \lambda_k v_k = \mathbf{0}$ and $\sum_{k=1}^{n} \lambda_k = 0$.
            \end{definition}
            \begin{definition}
            A set of vectors is said to be affinely independent if and only if there are not affinely dependent.
            \end{definition}
            \begin{theorem}
            A set $\{v_k:k\in \mathbb{Z}_n\}$ is affinely independent if and only if $\{v_k-v_1:k\in \mathbb{Z}_n, k>1\}$ is linearly independent.
            \end{theorem}
            \begin{proof}
            Suppose the latter set is linearly independent. Then $\sum_{k=1}^{n} \lambda_k(v_k-v_1) \ne 0$ if at least one $\lambda_k \ne 0$. Let $\lambda_k$ be any sequence such that $\sum_{k=1}^{n} \lambda_k = 0$, but not all $\lambda_k$ are zero. Then $\sum_{k=1}^{n} \lambda_k(v_k-v_1)\ne 0$. Let this sum be $c_{\lambda}$. Then $\sum_{k=1}^{n} \lambda_k v_k = \sum_{k=1}^{n} \lambda_k v_1 + c_\lambda = c_{\lambda}$. Thus, the set $v_k$ is affinely independent. Suppose the former set is affinely independent. Then $\sum_{k=1}^{n} \lambda_k v_k = \mathbf{0} \Rightarrow \sum_{k=1}^{n} \lambda \ne 0$. But then $\sum_{k=1}^{n}\lambda_k (v_k-v_1) = - \sum_{k=1}^{n} \lambda_k v_1 \ne 0$. Thus, the latter set is linearly independent.
            \end{proof}
            \begin{theorem}
            For any $n-dimensional$ vector space $V$, any set of affinely independent vectors has at most $n+1$ vectors.
            \end{theorem}
            \begin{proof}
            If $v_k$ are affinely independent, then $v_k-v_1$ is linearly independent. The latter has at most $n$ vectors. Thus, etc.
            \end{proof}
            \begin{theorem}
            If $v_k$ are affinely independent and
            $\sum_{k=1}^{n}\lambda_{k}v_{k}=\sum_{k=1}^{n}\sigma_{k}v_{k}$,
            then $\lambda_{k}=\sigma_{k}$ for all $k$.
            \end{theorem}
            \begin{proof}
            $[\sum_{k=1}^{n}(\lambda_k - \sigma_k)v_k = 0]\land[\sum_{k=1}{^n}(\lambda_k-\sigma_k) = 0]\Rightarrow [\lambda_k-\sigma_k = 0]$. Therefore, etc.
            \end{proof}
            \begin{definition}
            The Affine Hull of a set $S\subset V$ of some normed space $V$ is $\textrm{aff}(S) = \{\sum_{i=1}^{m}\lambda_i x_i: x_i \in S\land \sum_{i=1}^{m}\lambda_i =1\}$.
            \end{definition}
            \begin{definition}
            If $V$ is a normed space, a convex combination is $\sum_{i=1}^{n}|\lambda_i| v_i$, where $v_i\in V$, $\sum_{i=1}^{n}|\lambda_i| = 1$.
            \end{definition}
            \begin{definition}
            For a normed space $V$, the Convex Hull of $S\subset V$ is $\textrm{conv}(S)=\{\sum_{i=1}^{n}|\lambda_i| x_i:x_i\in S\land \sum_{i=1}^{n} |\lambda_i| = 1 \}$.
            \end{definition}
            \begin{definition}
            In an inner product space $V$, $v,w\in V$ are said to be orthogonal, $v\perp w$, if and only if $\langle v,w \rangle = 0$.
            \end{definition}
            In an inner product space $V$, $W\subset V$, $x\in V$, and
            $y\in W \Rightarrow \langle x,y\rangle = 0$, we write $x\perp W$.
            Similarly for orthogonal subsets $W$ and $U$ of $V$, we write
            $W\perp U$.
            \begin{definition}
            A line in $\mathbb{R}^n$ containing the points $x,y\in \mathbb{R}^n$ is the set $\{\lambda y + (1-\lambda)x: \lambda \in \mathbb{R}\}$.
            \end{definition}
            \begin{definition}
            A line segment in $\mathbb{R}^n$ that begins at $x$ and terminates at $y$ is the set $\{\lambda y + (1-\lambda)x: 0\leq \lambda \leq 1 \}$.
            \end{definition}
            \begin{definition}
            If $W\underset{Subspace}\subset\mathbb{R}^n$, $K \subset \mathbb{R}^n$, then the orthogonal projection is $K_{W}\equiv\{x\in W: \exists y\in K: y-x \perp W\}$.
            \end{definition}
            \begin{theorem}
            If $W$ is a subspace of $\mathbb{R}^n$, $K \subset \mathbb{R}^n$, and $x\in K_{W}$, then there is a line through $x$ and a point $y\in K$ such that, for any $\alpha, \beta$ contained on said line, $\beta-\alpha \perp W$.
            \end{theorem}
            \begin{proof}
            For let $x\in W$. Then there is a $y\in K$ such that, for all $z\in W$, $\langle y-x,z\rangle = 0$. Let $\Gamma$ be the line $\lambda y + (1-\lambda)x$. If $\alpha,\beta \in \Gamma$, there are values $\lambda_1$ and $\lambda_2$ such that $\alpha = \lambda_1 y+ (1-\lambda_1)x$ and $\beta = \lambda_2 y +(1-\lambda_2)x$. But then $\beta-\alpha = y(\lambda_2-\lambda_1)-x(\lambda_2-\lambda_1) = (\lambda_2-\lambda_1)(x-y)$. But then $\langle \beta - \alpha,z\rangle = \langle (\lambda_2-\lambda_1)(x-y),z\rangle = (\lambda_2-\lambda_1)\langle x-y,z \rangle = 0$.
            \end{proof}
            From the Cauchy-Schwarz inequality, we have that
            $|\langle x,y \rangle| \leq \norm{x}\norm{y}$, and thus
            $|\frac{\langle x,y \rangle}{\norm{x}\norm{y}}| \leq 1$. We
            define the $angle\ \theta$ between two non-zero vectors
            $x,y\in \mathbb{R}^n$ as $\theta=\cos^{-1}\big(\frac{\langle x,y \rangle}{\norm{x}\norm{y}}\big)$. We omit rigorous definition of
            the cosine function.
            \begin{definition}
            If $\mathcal{U},\mathcal{V}\subset V$, then $\mathcal{U}+\mathcal{V} = \{x+y:x\in \mathcal{U},y\in \mathcal{V}\}$.
            \end{definition}
            \begin{definition}
            If $V$ is a vector space over $K$, $\mathcal{U}\subset V$, and $\alpha \in K$, then $\alpha \mathcal{U} = \{\alpha x:x\in \mathcal{U}\}$.
            \end{definition}
    \section{Definitions and Theorems}
        \subsection{Definitions}
            \begin{definition}
                \label{Definition:MathEnc:Analysis:Sum:Sets}
                A set is a collection of elements, none of which
                are the set itself.
            \end{definition}
            \begin{definition}
                \label{Definition:MathEnc:Analysis:Sum:EmptySet}
                The empty set is the set $\emptyset$
                such that $\forall_{x}$, $x\notin A$.
            \end{definition}
            \begin{definition}
                \label{Definition:MathEnc:Analysis:Sum:Subsets}
                A subset of a set $A$ is a set $B$, denoted
                $B\subset A$, such that $\forall_{x\in B}$,
                $x\in A$.
            \end{definition}
            \begin{definition}
                \label{Definition:MathEnc:Analysis:Sum:Equality}
                Equal sets are sets $A$ and $B$, denoted $A=B$,
                such that $A\subset B$ and $B\subset A$.
            \end{definition}
            \begin{definition}
                \label{Definition:MathEnc:Analysis:Sum:OrderedPair}
                An ordered pair of an element $a$ with respect
                to an element $b$, denoted $(a,b)$,
                is the set $(a,b)=\{\{a\},\{a,b\}\}$
            \end{definition}
            \begin{definition}
                \label{%
                    Definition:MathEnc:Analysis:%
                    Sum:CartesianProduct%
                }
                The Cartesian Product of a set $A$ with respect
                to a set $B$, denoted $A\times B$, is the set
                $A\times B=\{(a,b):a\in A,b\in B\}$
            \end{definition}
            \begin{definition}
                \label{%
                    Definition:MathEnc:Analysis:%
                    Sum:BinaryRelation%
                }
                A binary relation on a set $X$ is
                a subset $R$ of $X\times X$.
            \end{definition}
            \begin{definition}
                \label{%
                    Definition:MathEnc:Analysis:%
                    Sum:ComparableElements%
                }
                Comparable elements in a set $X$ with respect
                to a binary relation $R$
                are elements $x,y\in X$ such that either
                $(x,y)\in R$, denoted $xRy$,
                or $(y,x)\in R$, denoted $yRx$.
            \end{definition}
            \begin{definition}
                \label{%
                    Definition:MathEnc:Analysis:%
                    Sum:ConnexRelation%
                }
                A connex relation on a set $X$ is a
                binary relation $R$ on $X$ such that
                $\forall_{x,y\in X}$, either $xRy$ or $yRx$.
            \end{definition}
            \begin{definition}
                \label{%
                    Definition:MathEnc:Analysis:%
                    Sum:TransitiveRelation%
                }
                A transitive relation on a set $X$
                is a binary relation $R$ on $X$ such
                that $\forall_{x,y,z\in X}$,
                $xRy\land yRz\Rightarrow xRz$.
            \end{definition}
            \begin{definition}
                \label{%
                    Definition:MathEnc:Analysis:%
                    Sum:AntisymmetricRelation%
                }
                An antisymmetric relation is a binary relation
                $R$ on a set $X$ such that
                $\forall_{x,y\in X}$, $xRy\land yRx\Rightarrow x=y$.
            \end{definition}
            \begin{definition}
                \label{Definition:MathEnc:Analysis:Sum:TotalOrder}
                A total order on a set $X$ is a binary relation $R$
                on $X$ that is a transitive relation, an
                antisymmetric relation, and a connex relation.
            \end{definition}
            \begin{definition}
                \label{%
                    Definition:MathEnc:Analysis:%
                    Sum:TrichotomousRelation%
                }
                A trichotomous relation on a set $X$ is a
                binary relation $R$ on $X$ such that
                $\forall_{x,y\in X}$ precisely one of the
                following are true: $xRy$, $yRx$, or $x=y$.
            \end{definition}
            \begin{definition}
                \label{Definition:MathEnc:Analysis:Sum:Function}
                A function $f$ from a set $A$ to a set $B$,
                denoted $f:A\rightarrow B$,
                is a subset $f\subset A\times B$ such that
                $\forall_{a\in A}$ there is
                a unique $b\in B$ such that $(a,b)\in f$.
            \end{definition}
            \begin{definition}
                \label{Definition:MathEnc:Analysis:Sum:Image}
                The image of an element $x\in A$ with respect to a
                function $f:A\rightarrow B$, denoted $f(x)$, is the
                unique element $b\in B$ such that $(a,b)\in f$.
            \end{definition}
            \begin{definition}
                \label{%
                    Definition:MathEnc:Analysis:Sum:BinaryOperation%
                }
                A binary operation on a set $A$ is a function
                $*:A\times A\rightarrow A$
            \end{definition}
            \begin{definition}
                \label{Definition:MathEnc:Analysis:Sum:Product}
                The product of elements $a,b\in A$ with respect
                to a binary operation
                $*$ on $A$, denoted $a*b$, is the image of $(a,b)$
                with respect to $*$.
            \end{definition}
            \begin{definition}
                \label{%
                    Definition:MathEnc:Analysis:%
                    Sum:CommunativeOperation%
                }
                A commutative operation on a set $A$
                is a binary operation $*$ on $A$
                such that $\forall_{x,y\in A}$, $a*b=b*a$.
            \end{definition}
            \begin{definition}
                \label{%
                    Definition:MathEnc:Analysis:%
                        Sum:AssociativeOperation%
                }
                An associative operation on a set $A$ is a
                binary operation $*$ on $A$
                such that $\forall_{a,b,c\in A}$, $a*(b*c)=(a*b)*c$.
            \end{definition}
            \begin{definition}
                \label{%
                    Definition:MathEnc:Analysis:%
                    Sum:RightDistribute%
                }
                A binary operation that right distributes
                over a binary operation $+$ on
                a set $A$ is a binary operation $\cdot$ on $A$
                such that $\forall_{a,b,c\in A}$,
                $a\cdot (b+c)=(a\cdot b)+(a\cdot c)$.
            \end{definition}
            \begin{definition}
                \label{%
                    Definition:MathEnc:Analysis:%
                    Sum:LeftDistribute%
                }
                A binary operation that left distributes
                over a binary operation $+$ on
                a set $A$ is a binary operation $\cdot$ on $A$
                such that $\forall_{a,b,c\in A}$,
                $(b+c)\cdot a=(b\cdot a)+(c\cdot a)$.
            \end{definition}
            \begin{definition}
                \label{Definition:MathEnc:Analysis:Sum:Distribute}
                A binary operation that distributes over a binary
                operation $+$ on a set $A$ is a binary operation
                $\cdot$ on $A$ such that $\cdot$ both left and right
                distributes over $+$.
            \end{definition}
            \begin{definition}
                \label{Definition:MathEnc:Analysis:Sum:LeftIdentity}
                A left identity in a set $A$ with respect
                to a binary operation $*$ is
                an element $e\in A$ such that $\forall_{a\in A}$,
                $e*a=a$.
            \end{definition}
            \begin{definition}
                \label{Definition:MathEnc:Analysis:Sum:RightIdentity}
                A right identity in a set $A$ with respect to a
                binary operation $*$ is an element $e\in A$
                such that $\forall_{a\in A}$, $a*e=a$.
            \end{definition}
            \begin{definition}
                \label{%
                    Definition:MathEnc:Analysis:%
                    Sum:IdentityElement%
                }
                An identity element in a set $A$ with
                respect to a binary operation $*$
                on $A$ is an element $e\in A$ that is
                both a right and a left identity.
            \end{definition}
            \begin{definition}
                \label{Definition:MathEnc:Analysis:Sum:LeftInverse}
                A left inverse of an element $a\in A$ with
                respect to a binary operation $*$
                on $A$ and an identity $e\in A$ is an element
                $b\in A$ such that $b*a=e$.
            \end{definition}
            \begin{definition}
                \label{Definition:MathEnc:Analysis:Sum:RightInverse}
                A right inverse of an element $a\in A$ with
                respect to a binary operation $*$
                on $A$ and an identity $e\in A$ is an element
                $b\in A$ such that $a*b=e$.
            \end{definition}
            \begin{definition}
                \label{Definition:MathEnc:Analysis:Sum:Inverse}
                An inverse of an element $a\in A$ with respect to a
                binary operation $*$ on $A$ and an identity
                $e\in A$ is an element $b\in A$ that is both a
                right and a left inverse of $a$.
            \end{definition}
            \begin{definition}
                \label{Definition:MathEnc:Analysis:Sum:Semigroup}
                A semigroup is an ordered pair $(A,*)$ such that
                $A$ is a set and $*$ is an associative binary
                operation on $A$.
            \end{definition}
            \begin{definition}
                \label{Definition:MathEnc:Analysis:Sum:Monoid}
                A monoid is a semigroup $(A,*)$ such that
                $\exists e\in A$ such that $e$ is
                and identity element with respect to $*$.
            \end{definition}
            \begin{definition}
                \label{Definition:MathEnc:Analysis:Sum:Group}
                A group is a monoid $(G,*)$, denoted
                $\langle G,*\rangle$, with an identity
                element $e$ such that
                $\forall_{a\in A}\exists_{b\in A}:a*b=e$ 
            \end{definition}
            \begin{definition}
                \label{Definition:MathEnc:Analysis:Sum:AbelianGroup}
                An Abelian Group is a group $\langle G,*\rangle$
                such that $*$ is a
                commutative binary operation on $G$.
            \end{definition}
            \begin{definition}
                \label{Definition:MathEnc:Analysis:Sum:3Tuple}
                A 3-tuple of an element $a$ with respect
                to an ordered pair $(b,c)$ is the
                set $(a,b,c)=\{\{a\},\{a,\{b,c\}\}\}$
            \end{definition}
            \begin{definition}
                \label{Definition:MathEnc:Analysis:Sum:Ring}
                A ring is a 3-tuple $(A,\cdot,+)$ such that
                $\langle A,+\rangle$ is an
                \hyperref[%
                    Definition:MathEnc:Analysis:%
                    Sum:AbelianGroup%
                ]{Abelian Group},
                $(A,\cdot)$ is a
                \hyperref[%
                    Definition:MathEnc:Analysis:%
                    Sum:Semigroup%
                ]{semigroup},
                and $*$
                \hyperref[%
                    Definition:MathEnc:Analysis:%
                    Sum:Distribute%
                ]{distributes}
                over $+$.
            \end{definition}
            \begin{definition}
                \label{Definition:MathEnc:Analysis:Sum:RingUnit}
                A ring with unity is a
                \hyperref[Definition:MathEnc:Analysis:Sum:Ring]{ring}
                $(A,\cdot,+)$ such that $(A,\cdot)$ is a
                \hyperref[%
                    Definition:MathEnc:%
                    Analysis:Sum:Monoid%
                ]{monoid}.
            \end{definition}
            \begin{definition}
                \label{Definition:MathEnc:Analysis:Sum:ComRing}
                A commutative ring is a
                \hyperref[Definition:MathEnc:Analysis:Sum:Ring]{ring}
                $(A,\cdot,+)$ such that $\cdot$ is a
                \hyperref[%
                    Definition:MathEnc:%
                    Analysis:Sum:CommunativeOperation%
                ]{commutative}
                binary operation over $A$.
            \end{definition}
            \begin{definition}
                \label{Definition:MathEnc:Analysis:Sum:ComRingUnit}
                A commutative ring with unity is a 
                \hyperref[%
                    Definition:MathEnc:%
                    Analysis:Sum:RingUnit%
                ]{ring with unity}
                such that $\cdot$ is a
                \hyperref[%
                    Definition:MathEnc:%
                    Analysis:Sum:CommunativeOperation%
                ]{commutative}
                binary operation over $A$.
            \end{definition}
            \begin{definition}
                \label{Definition:MathEnc:Analysis:Sum:Field}
                A field is a 
                \hyperref[%
                    Definition:MathEnc:%
                    Analysis:Sum:ComRingUnit%
                ]{commutative ring with unity}
                $(A,\cdot,+)$ such that $\forall_{a\in A}$
                such that $a$ is not an identity with respect
                to $\cdot$, $\exists_{b\in A}$ such that $b$ is an
                \hyperref[%
                    Definition:MathEnc:Analysis:%
                    Sum:Inverse%
                ]{inverse} of $a$ with respect to $\cdot$.
            \end{definition}
        \subsection{Theorems}
            \begin{theorem}
                \label{%
                    Theorem:MathEnc:Analysis:Sum:SqurPresIneqPosNum%
                }
                If $a,b\in\mathbb{R}^{+}$ and $a<b$,
                then $a^{2}<b^{2}$
            \end{theorem}
            \begin{proof}
                If $a<b$, and $0<a$, then $a\cdot a<a\cdot b$
                \hfill
                (Multiplicative Property of Ordered Fields)\par
                But $a\cdot a = a^{2}$, and thus $a^{2}<a\cdot b$
                \hfill
                (Definition of Exponents)\par
                But if $a<b$ and $0<b$, then $a\cdot b<b\cdot b$
                \hfill
                (Multiplicative Property of Ordered Fields)\par
                Therefore $a\cdot b<b^{2}$
                \hfill
                (Definiiton of Exponents)\par
                But if $a^{2}<a\cdot b$ and $a\cdot b<b^{2}$,
                then $a^{2}<b^{2}$
                \hfill
                (Transitive Property of Inequalities)\par
                Therefore, $a^{2}<b^{2}$
            \end{proof}
            \begin{theorem}
                If $a,b\in\mathbb{R}^{+}$ and $a^{2}=b^{2}$,
                then $a=b$.
            \end{theorem}
            \begin{proof}
                If $a^{2}=b^{2}$, then $b^{2}-a^{2}=0$\hfill
                (Additive Property of Ordered Fields)\par
                If $a<b$, then $0<b-a$\hfill
                (Additive Property of Ordered Fields)\par
                If $a,b\in\mathbb{R}^{+}$, then $0<b+a$
                \hfill
                (Closure of Addition in $\mathbb{R}^{+}$)\par
                If $0<b-a$ and $0<b+a$, then $0<(b-a)\cdot (b+a)$
                \hfill
                (Closure of Multiplication in $\mathbb{R}^{+}$)\par
                But $(b-a)\cdot(b+a)=b^{2}-a^{2}$
                \hfill
                (Elementary Algebra)\par
                But $b^{2}-a^{2}=0$, and thus $0\not<b^{2}-a^{2}$
                \hfill
                (Trichotomous Property of Inequalities)\par
                Therefore, $a\not<b$. Similarly, $b\not<a$
                \hfill
                (Proof by Contradiction)\par
                But if $a\not<b$ and $b\not<a$, then $a=b$
                \hfill
                (Trichotomous Property of Inequalities)\par
                Therefore, $a=b$
            \end{proof}
            \begin{theorem}
                If $y\in(0,1)$, then there is an $x\in(0,1)$
                such that $y=x^{2}$.
            \end{theorem}
            \begin{proof}
                Let $A=\{\xi\in\mathbb{R}^{+}:\xi^{2}\leq y\}$.\par
                But $y\in(0,1)$ and therefore $y<1$\hfill
                (Definition of $(0,1)$)\par
                Therefore $A$ is bounded above by $1$.\par
                Thus there exists a
                least upper bound $x$ of $A$\hfill
                (Completeness of $\mathbb{R}$)\par
                If $x^{2}>y$, then $\frac{x^{2}-y}{2}>0$.\hfill
                (Additive Property of Ordered Fields)\par
                Then
                $(x-\frac{x^{2}-y}{2})^{2}=%
                 a+(\frac{x^{2}-y}{2})^{2}>y$\hfill
                (Additive Property of Ordered Fields)\par
                Thus $x-\frac{x^{2}-y}{2}$ is an
                upper bound of A.\hfill
                (Definition of Upper Bounds)\par
                But $x$ is the least upper bound,
                a contradiction.\par
                Therefore $x^{2}\not>a$.\hfill
                (Proof by Contradiction)\par
                If $x^{2}<y$, then $0<\frac{y-x^{2}}{2x+1}$\hfill
                (Additive Property of Ordered Fields)\par
                Let $\epsilon=\min\{\frac{y-x^{2}}{2x+1},1\}$\par
                Then
                $(x+\epsilon)^{2}=x^{2}+2x\epsilon+\epsilon^{2}%
                 <x^{2}+2x\epsilon+\epsilon$\hfill
                (Thm.~\ref{%
                    Theorem:MathEnc:Analysis:%
                    Sum:SqurPresIneqPosNum%
                })\par
                But
                $x^{2}+2x\epsilon+\epsilon=x^{2}+(2x+1)\epsilon=y$
                \hfill (Elementary Algebra)\par
                Therefore $(x+\epsilon)^{2}<y$\hfill
                (Transitive Property of Total Orders)\par
                But $\epsilon>0$, and thus $x+\epsilon>x$.\hfill
                (Additive Property of Ordered Fields)\par
                Thus $x+\epsilon\in A$, a contradiction as
                $x$ is a least upper bound of $A$.\par
                Therefore $x^{2}\not<y$.\hfill
                (Proof by Contradiction)\par
                Therefore $x^{2}=y$\hfill
                (Trichotomous Property of Inequalities)\par
            \end{proof}
    %     \chapter{Measurable Spaces}
    \ifcsname\PATH\endcsname
        \newcommand{\PATH}{books/Analysis/Measure_Theory}
    \else
        \renewcommand{\PATH}{books/Analysis/Measure_Theory}
    \fi

    %------------------------------------------------------------------------------%
\section{Set Rings}
    Given a set $\Omega$, $\mathcal{P}(\Omega)$ is the set of all subsets of
    $\Omega$. Often this is too much, and too difficult to handle. Indeed,
    even $\mathcal{P}(\mathbb{R})$ is quite large and hard to get a grasp on.
    We wish to speak of collections of sets that have some structure on them.
    The first thing we will talk about is a set ring.
    \begin{fdefinition}{Set Ring}
        A set ring of a set $\Omega$ is a nonempty subset
        $\mathcal{R}\subseteq\mathcal{P}(\Omega)$ such that
        for all $A,B\in\mathcal{R}$, $A\cup{B}\in\mathcal{R}$, and
        $A\setminus{B}\in\mathcal{R}$.
    \end{fdefinition}
    \begin{fexample}{Example of Set Rings}{Example_of_Set_Rings}
        If $\Omega$ is a set, then $\mathcal{P}(\Omega)$ is a set ring of
        $\Omega$. So is the set $R=\{\emptyset$. For any $A\subset\Omega$,
        the set $R=\{A\}$ is also a set ring. If $\Omega=\{1,2,3\}$, then
        $R=\{\emptyset,\{1\},\{2,3\},\{1,2,3\}\}$ is
        a set ring on $\Omega$.
    \end{fexample}
    \begin{lexample}
        If $\Omega$ is an infinite set, and if
        $\mathcal{E}=\big\{\{x\}:x\in\Omega\big\}$, then the smallest set
        ring that contains $\mathcal{E}$ is the set of all finite subsets of
        $\Omega$. For the union of two finite sets is finite, as is the set
        difference of two finite sets, and thus this satisfies a set ring.
        Moreover, if $\mathcal{R}$ is a set ring that contains $\mathcal{E}$
        then it contains the union of any finite collection of elements in
        $\mathcal{E}$. But $\mathcal{E}$ is the set of all of the singletons
        of $\Omega$, and any finite subset of $\Omega$ can be written as the
        union of finitely many singletons. Thus, $\mathcal{R}$ is the
        smallest set ring that contains $\mathcal{E}$.
    \end{lexample}
    \begin{theorem}
        If $\Omega$ is a set, if $R$ is a set ring on $\Omega$, and if $A$ is
        a finite subset of $R$, then $\cup_{\alpha\in{A}}\alpha$ is an
        element of $R$.
    \end{theorem}
    \begin{proof}
        Apply induction to the closure of unions.
    \end{proof}
    \begin{theorem}
        If $X$ is a set, if $R$ is a set ring on $X$, and if
        $A,B\in{R}$, then $A\cup{B}\in{R}$.
    \end{theorem}
    \begin{proof}
        For $A\cap{B}=A\setminus(A\setminus{B})$, and from the closure of set
        difference, $A\cap{B}\in{R}$.
    \end{proof}
    \begin{theorem}
        If $X$ is a set, if $R$ is a set ring on $X$, and if $A$ is a finite
        subset of $R$, then $\cap_{\alpha\in{A}}\alpha$ is an element of $R$.
    \end{theorem}
    \begin{proof}
        Apply induction to the closure of intersections.
    \end{proof}
    \begin{theorem}
        If $\Omega$ is a set, if $R$ is a set ring on
        $\Omega$, if $A,B\subset\Omega$, and if
        $A\setminus{B}$, $B\setminus{A}$, and
        $A\cap{B}$ are elements of $R$, then
        $A,B\in{R}$.
    \end{theorem}
    Thus, the set ring generated by the set $\{A,B\}$ and
    the set ring generated by
    $\{A\setminus{B},B\setminus{A},A\cap{B}\}$ are the
    same.
    \begin{theorem}
        If $\Omega$ is a set and $R$ is a set ring
        of $\Omega$, then $\emptyset\in{R}$.
    \end{theorem}
    \begin{proof}
        For as $R$ is non-empty, there is an element
        $A\in{R}$. If $A=\emptyset$, then we are done.
        If not, as $R$ is closed under set difference,
        $A\setminus{A}\in{R}$. But
        $A\setminus{A}=\emptyset$.
    \end{proof}
    From this, if we have a collection $R$ of subsets of
    $\Omega$ and we wish to check if $R$ is a set ring
    of $\Omega$, there are several redundant operations
    we don't need to check. Since, for any set $A$,
    we have:
    \begin{align}
        A\setminus\emptyset&=A\\
        A\setminus{A}&=\emptyset\\
        \emptyset\setminus{A}&=\emptyset\\
        A\cup{A}&=A\\
        A\cup\emptyset&=A\\
        \emptyset\cup\emptyset&=\emptyset
    \end{align}
    Using our previous example $\Omega=\{1,2,3\}$,
    we can check laboriously that
    $R=\{\emptyset,\{1\},\{2,3\},\{1,2,3\}\}$ is a
    set ring on $\Omega$. The set
    $\{\emptyset,\{1\},\{2\},\{1,2,3\}\}$ is not
    a set ring, for $\{1,2\}=\{1\}\cup\{2\}$ is not
    an element.
    \begin{theorem}
        If $\Omega$ is a set, and if $A$ and $B$ are
        disjoint subsets of $\Omega$, then
        $R=\{\emptyset,A,B,A\cup{B}\}$ is a set ring
        on $\Omega$.
    \end{theorem}
    \begin{theorem}
        If $\Omega$ is a set, if $A$ and $B$ are
        disjoint subsets of $\Omega$, and if
        $R$ is a set ring such that $A,B\in{R}$,
        then $\{emptyset,A,B,A\cup{B}\}\subset{R}$.
    \end{theorem}
    As such, the set ring $\{\emptyset,A,B,A\cup{B}\}$
    is called the set ring generated by $A$ and $B$. We
    can continue and consider the case of three mutually
    disjoint subsets.
    \begin{theorem}
        If $\Omega$ is a set, and $A_{1},A_{2},A_{3}$ are
        mutually disjoint subsets of $\Omega$, then:
        \begin{equation}
            R=\{\emptyset,A_{1},A_{2},A_{3},
                A_{1}\cup{A}_{2},A_{1}\cup{A}_{3},
                A_{2}\cup{A}_{3},
                A_{1}\cup{A}_{2}\cup{A}_{3}\}
        \end{equation}
        is a set ring on $\Omega$.
    \end{theorem}
    Indeed, we may generalize further.
    \begin{theorem}
        If $\Omega$ is a set and if
        $A$ is a subset of $\mathcal{P}(\Omega)$ of
        $n$ elements such that, for all
        $a,b\in{A}$, $a\cap{B}=\emptyset$, then:
        \begin{equation}
            R=\{\cup_{i\in{I}}A_{i}:
            I\in\mathcal{P}(\mathbb{Z}_{n})\}
        \end{equation}
        Is a set ring on $\Omega$.
    \end{theorem}
    \begin{theorem}
        If $\Omega$ is a set, then the set of all
        finite subsets of $\Omega$ is a set ring on
        $\Omega$.
    \end{theorem}
    A left semi-interval of $\mathbb{R}$ is an interval
    of the form $[a,b)$ where $a\leq{b}$. If $a=b$, this
    is the empty set. The set of all left semi-intervals
    is not a set ring on $\mathbb{R}$ since the union
    of two semi-intervals need not be a semi-interval.
    We need to add more sets to allow this to be a
    set ring. The collection of all finite unions of
    semi-intervals of $\mathbb{R}$ is a set ring.
    First, note the following:
    \begin{equation}
        \Big(\bigcup_{n=1}^{N}[a_{n},b_{n})\Big)
        \setminus[c,d)=\bigcup_{n=1}^{N}
        \Big([a_{n},b_{n})\setminus[c,d)]
    \end{equation}
    This is again the finite union of intervals. By
    induction we see that this collection is a ring on
    $\mathbb{R}$. We have seen that a set ring is
    closed to unions and set differences, and this
    implies that rings are closed under intersections and
    closed under symmetric differences. As it turns out,
    this is an equivalent definition of a set ring.
    \begin{theorem}
        If $\Omega$ is a set and
        $R\subset\mathcal{P}(\Omega)$, then $R$ is
        a set ring of $\Omega$ if and only if $R$ is
        closed under symmetric differences and
        intersections.
    \end{theorem}
    If $R$ is a set ring on $\Omega$, and if
    $A\in{R}$, let $\chi_{A}:\Omega\rightarrow[0,1]$ be
    the indicator function defined as follows:
    \begin{equation}
        \chi_{A}(\omega)=
        \begin{cases}
            0,&\omega\notin{A}\\
            1,&\omega\in{A}
        \end{cases}
    \end{equation}
    Then we have:
    \begin{align}
        \chi_{A\cap{B}}(\omega)
        &=\chi_{A}(\omega)\chi_{B}(\omega)\\
        \chi_{A\ominus{B}}&=
        \big(\chi_{A}(\omega)+\chi_{B}(\omega)\big)
        \mod{2}
    \end{align}
    These two operations satisfy the axioms of a ring.
    That is, a ring in the algebraic sense of the word:
    A set with two operations that behave certain
    properties. It is because of this that set rings
    have earned their name.
    %        %------------------------------------------------------------------------------%
\begingroup
    \ifcsname\PATH\endcsname
        \newcommand{\PATH}{books/Analysis/Complex_Analysis}
        \newcommand{\OLDPATH}{\PATH}
    \else
        \newcommand{\OLDPATH}{\PATH}
        \renewcommand{\PATH}{books/Analysis/Complex_Analysis}
    \fi
    \chapter{Complex Numbers}
        The theory of complex analysis extends the study of calculus of a
        single real variable to that of a \textit{complex} variable. The
        complex numbers have many interesting and counter-intuitive properties,
        many of which are used regularly in physics.
        \section{Complex Numbers}
    A \gls{complex number} is a point in the plane $z=(x,\,y)$, but we often
    write:
    \begin{equation}
        z=x+iy
    \end{equation}
    and call $i$ the \textit{imaginary unit}. We call $x$ the \textit{real part}
    and $y$ the \textit{imaginary part}, denoted $\Re(z)$ and $\Im(z)$,
    respectively. The planar representation is shown in
    Fig.~\ref{fig:Cart_Rep_of_Comp_Num}. The arithmetic goes as follows:
    \begin{subequations}
        \begin{align}
            \label{eqn:Complex_Addition}%
            (a+ib)+(c+id)&=(a+c)+i(b+d)\\
            \label{eqn:Complex_Multiplication}%
            (a+ib)\cdot(c+id)&=(ac-bd)+i(bc+ad)
        \end{align}
    \end{subequations}
    We'd hope this definition preserves the arithmetic of the \textit{real}
    numbers, and indeed it does. Setting $b$ and $d$ to zero, we see that
    elementary arithmetic is recovered.
    \par\hfill\par
    The arithmetic of the complex numbers arises when one studies equation
    like $y(x)=x^{2}+1$. For a real variable $x$, there is no root to this
    equation. That is, there is no real number $x$ such that $x^{2}+1=0$.
    We can invent such a number and give that the property that it's square
    is $\minus{1}$. This is what the imaginary unit does.
    \begin{figure}[H]
        \centering
        \captionsetup{type=figure}
        %--------------------------------Dependencies---------------------------------%
%   amssymb                                                                   %
%   tikz                                                                      %
%       arrows.meta                                                           %
%-------------------------------Main Document---------------------------------%
\begin{tikzpicture}[%
    >=Latex,
    line width=0.2mm,
    line cap=round,
    scale=1.7
]

    % Coordinates for the various points.
    \coordinate (O)   at (0.0, 0.0);
    \coordinate (z)   at (2.3, 2.1);
    \coordinate (z_x) at (2.3, 0.0);
    \coordinate (z_y) at (0.0, 2.1);
    \coordinate (C)   at (3.0, 3.0);

    % Axes:
    \begin{scope}[thick]
        \draw[->] (-0.5,  0.0) to (4.4, 0.0) node [above] {$\Re\{z\}$};
        \draw[->] ( 0.0, -0.5) to (0.0, 4.4) node [right] {$\Im\{z\}$};
    \end{scope}

    % Axes labels:
    \foreach\n in {1,2,3,4}{%
        \draw (\n, 3pt) to (\n, -3pt) node [below] {$\n$};
        \draw (3pt, \n) to (-3pt, \n) node [left] {$\n{i}$};
    }

    % Draw a line from the origin to the point z.
    \draw (O) to (z);

    % Scope for dashed lines.
    \begin{scope}[densely dashed, thin]
        \draw (z_x) to (z);
        \draw (z_y) to (z);
    \end{scope}

    % Draw a point to denote z.
    \draw[fill=black] (2.3, 2.1) circle (0.4mm);

    % Nodes for labeling.
    \node at (C)          {\Large{$\mathbb{C}$}};
    \node at (z) [above]  {$z=(x,\,y)$};
\end{tikzpicture}

        \caption{Cartesian Representation of Complex Numbers}
        \label{fig:Cart_Rep_of_Comp_Num}
    \end{figure}
    It should be clear  that addition and multiplication are commutative
    operations ($z+w=w+z$ and $z\cdot{w}=w\cdot{z}$). That addition is
    associative is also straight forward. What is not obvious is the
    associativity of multiplication.
    \begin{theorem}
        \label{thm:Complex_Multiplication_Associative}%
        If $z$, $w$, and $v$ are complex numbers, then:
        \begin{equation}
            z\cdot(w\cdot{v})=(z\cdot{w})\cdot{v}
        \end{equation}
        That is, complex multiplication is associative.
    \end{theorem}
    \begin{proof}
        For let $z=a+ib$, $w=c+id$, and $v=e+if$. Then:
        \begin{subequations}
            \begin{align}
                z\cdot(w\cdot{v})
                    &=(a+ib)\cdot\Big((c+id)\cdot(e+if)\Big)\\
                    &=(a+ib)\cdot\Big((ce-df)+i(cf+de)\Big)\\
                    &=a(ce-df)-b(cf+de)+i\big(a(cf+de)+b(ce-df)\big)\\
                    &=(ace-adf-bcf-bde)+i(acf+ade+bce-bdf)\\
                    &=(ac-bd)e-(ad+bc)f+i\big((ad+bc)e+(ac-bd)f\big)\\
                    &=\Big((a+ib)\cdot(c+id)\Big)\cdot(e+if)
            \end{align}
        \end{subequations}
        This completes the proof.
    \end{proof}
    \begin{theorem}
        If $i$ is the imaginary unit, then $i^{2}=\minus{1}$.
    \end{theorem}
    \begin{proof}
        For $i=0+1i$, and thus by Eqn.~\ref{eqn:Complex_Multiplication}:
        \begin{equation}
            i^{2}=(0+1i)\cdot(0+1i)
                 =(0\cdot{0}-1\cdot{1})+i(1\cdot{0}+0\cdot{1})
                 =\minus{1}+i\cdot{0}
                 =\minus{1}
        \end{equation}
        This completes the proof.
    \end{proof}
    The complex numbers are \textit{algebraically closed}:
    Every non-constant polynomial has a \textit{root}, or a zero.
    Moreover, given a polynomial of degree $n$ there are at most
    $n$ roots. This result is called the
    \textit{Fundamental Theorem of Algebra}. The real numbers lack this
    feature, for consider the graph of $y(x)=x^{2}+1$. Many attempts at
    proving this theorem were made between 1608 and 1799, and the likes of
    Euler, Lagrange, Laplace, Gauss, and d'Alambert failed in their
    attempts. In 1806 Jean Robert-Argand published a rigorous proof, and
    due to this the complex plane is occasionally called the Argand plane.
    \par\hfill\par
    There are two fundamental notions worth mentioning: The complex
    conjugate and the modulus of a complex number.
    \begin{ldefinition}{Complex Conjugate}{Complex_Conjugate}
        The \gls{complex conjugate} a complex number $z=x+iy$ is:
        \begin{equation}
            \overline{z}=x-iy
        \end{equation}
        That is, the reflection of $z$ across the $x$ axis.
    \end{ldefinition}
    A visual for the complex conjugate of a complex number is given in
    Fig.~\ref{fig:Conj_and_Mod_of_Com_Num}. There are various
    arithmetic properties of the complex conjugate that ease the
    process of computation.
    \begin{theorem}
        If $z$ is a complex number, then $z\cdot\overline{z}$ is
        a non-negative real number.
    \end{theorem}
    \begin{proof}
        For let $z=x+iy$, where $x$ and $y$ are real numbers. Then, by
        the definition of the complex conjugate
        (Def.~\ref{def:Complex_Conjugate}) and of
        complex multiplication (Eqn.~\ref{eqn:Complex_Multiplication}):
        \begin{equation}
            z\cdot\overline{z}=(x+iy)\cdot(x-iy)
                              =x^{2}+y^{2}
        \end{equation}
        This is the sum of the squares of two real numbers, and is
        therefore a real and non-negative number. Therefore, etc.
    \end{proof}
    \begin{figure}[H]
        \centering
        \captionsetup{type=figure}
        %--------------------------------Dependencies----------------------------------%
%   amssymb                                                                    %
%   pgfplots                                                                   %
%       compat=1.9                                                             %
%   tikz                                                                       %
%       arrows.meta                                                            %
%   Unary minus sign.                                                          %
%       \DeclareMathSymbol{\minus}{\mathbin}{AMSa}{"39}                        %
%-------------------------------Main Document----------------------------------%
\begin{tikzpicture}[%
    >=Latex,
    line width=0.2mm,
    line cap=round,
    scale=1.7
]

    % Coordinates for the various points used.
    \coordinate (O)       at (0.0,  0.0);
    \coordinate (z_x)     at (3.6,  0.0);
    \coordinate (z_y)     at (0.0,  1.6);
    \coordinate (z)       at (3.6,  1.6);
    \coordinate (z_bar)   at (3.6, -1.6);
    \coordinate (z_bar_y) at (0.0, -1.6);
    \coordinate (C)       at (2.0,  2.0);

    % Axes:
    \begin{scope}[thick]
        \draw[->]  (-0.5,  0.0) to (4.4, 0) node [above] {$\Re\{z\}$};
        \draw[<->] ( 0.0, -2.2) to (0, 2.2) node [right] {$\Im\{z\}$};
    \end{scope}

    % Axes labels:
    \foreach\n in {1, 2}{%
        \draw (\n, 3pt)  to (\n, -3pt)  node [below] {$\n$};
        \draw (3pt, \n)  to (-3pt, \n)  node [left]  {$\n{i}$};
        \draw (3pt, -\n) to (-3pt, -\n) node [left]  {$\minus\n{i}$};
    }

    % More labels for the x-axis.
    \foreach\n in {3, 4}{%
        \draw (\n, 3pt) to (\n, -3pt) node [below] {$\n$};
    }

    % Draw a line from the origin to the point (x, y).
    \draw (O) to node [above left=2mm and -2mm] {$|z|$} (z);

    % Draw a brace denoting the magnitude of z.
    \draw[decorate, decoration={brace, amplitude=5pt},thin] (O) to (z);

    % Draw a line from the origin to the conjugate of z.
    \draw (O) to (z_bar);

    % Scope for dashed lines.
    \begin{scope}[densely dashed, thin]
        % Draw dashed lines for z.
        \draw[densely dashed, thin] (z_y) to (z);
        \draw[densely dashed, thin] (z_x) to (z);

        % Draw dashed lines for the conjugate of z.
        \draw[densely dashed, thin] (z_bar_y) to (z_bar);
        \draw[densely dashed, thin] (z_x)     to (z_bar);
    \end{scope}

    % Points for z and z_bar.
    \draw[fill=black] (z)     circle (0.4mm);
    \draw[fill=black] (z_bar) circle (0.4mm);

    % Nodes for labelling.
    \node at (C)              {\Large{$\mathbb{C}$}};
    \node at (z)     [above]  {$z=(x,\,y)$};
    \node at (z_bar) [below]  {$\overline{z}=(x,\,\minus{y})$};
\end{tikzpicture}
        \caption{Modulus and Conjugate of a Complex Number}
        \label{fig:Conj_and_Mod_of_Com_Num}
    \end{figure}
    \begin{theorem}
        If $z$ and $w$ are complex numbers, then:
        \begin{equation}
            \overline{z+w}=\overline{z}+\overline{w}
        \end{equation}
    \end{theorem}
    \begin{proof}
        For let $z=a+ib$ and $w=c+id$. Then, by
        Eqn.~\ref{eqn:Complex_Addition}, we have:
        \begin{equation}
            \overline{z+w}=\overline{(a+ib)+(c+id)}
                          =\overline{(a+c)+i(b+d)}
        \end{equation}
        Invoking the definition of complex conjugate
        (Def.~\ref{def:Complex_Conjugate}), we obtain:
        \begin{equation}
            \overline{z+w}=(a+c)-i(b+d)
                          =(a-ib)+(c-id)
                          =\overline{z}+\overline{w}
        \end{equation}
        Therefore, etc.
    \end{proof}
    \begin{theorem}
        If $z$ and $w$ are complex numbers, then:
        \begin{equation}
            \overline{z\cdot{w}}=\overline{z}\cdot\overline{w}
        \end{equation}
    \end{theorem}
    \begin{proof}
        For let $z=a+ib$ and $w=c+id$. Then, by
        Eqn.~\ref{eqn:Complex_Multiplication}, we obtain:
        \begin{equation}
            \overline{z\cdot{w}}=\overline{(a+ib)\cdot(c+id)}
                                =\overline{(ac-bd)+i(ad+bc)}
        \end{equation}
        Invoking Def.~\ref{def:Complex_Conjugate}, we have:
        \begin{equation}
            \overline{z\cdot{w}}=(ac-bd)-i(ad+bc)
                                =(a-ib)\cdot(c-id)
                                =\overline{z}\cdot\overline{w}
        \end{equation}
        Therefore, etc.
    \end{proof}
    From the geometry shown in Fig.~\ref{fig:Conj_and_Mod_of_Com_Num},
    one would expect adding a complex number to its conjugate would
    eliminate the imaginary component, and subtracting would eliminate
    the real part. This is indeed true.
    \begin{theorem}
        \label{thm:Sum_with_Conj_is_Real}%
        If $z$ is a complex number, then:
        \begin{equation}
            z+\overline{z}=2\Re(z)
        \end{equation}
    \end{theorem}
    \begin{proof}
        For let $z=x+iy$. Then, by Def.~\ref{def:Complex_Conjugate}
        and Eqn.~\ref{eqn:Complex_Addition}:
        \begin{equation}
            z+\overline{z}=(x+iy)+(x-iy)
                          =(x+x)+i(y-y)
                          =2x
                          =2\Re(z)
        \end{equation}
        Therefore, etc.
    \end{proof}
    \begin{theorem}
        If $z$ is a complex number, then:
        \begin{equation}
            z-\overline{z}=2i\Im(z)
        \end{equation}
    \end{theorem}
    \begin{proof}
        For let $z=x+iy$. Then, by Def.~\ref{def:Complex_Conjugate}
        and Eqn.~\ref{eqn:Complex_Addition}:
        \begin{equation}
            z-\overline{z}=(x+iy)-(x-iy)
                          =(x-x)+i(y+y)
                          =2iy
                          =2i\Im(z)
        \end{equation}
        Therefore, etc.
    \end{proof}
    Lastly, taking the complex conjugate twice is equivalent to performing
    two reflection across the $x$ axis and thus should result in
    no change.
    \begin{theorem}
        \label{thm:Conj_of_Conj}%
        If $z$ is a complex number, then $\overline{\overline{z}}=z$.
    \end{theorem}
    \begin{proof}
        For let $z=x+iy$. Then:
        \begin{equation}
            \overline{\overline{z}}=\overline{\overline{(x+iy)}}
                                   =\overline{(x-iy)}
                                   =x+iy
                                   =z
        \end{equation}
        Therefore, etc.
    \end{proof}
    The complex conjugate can be used to define the modulus, or
    absolute value, of a complex number by simply taking the (positive)
    square root of $z\overline{z}$.
    \begin{ldefinition}{Modulus of a Complex Number}{Modulus_of_Comp_Num}
        The \gls{modulus} of a complex number $z=x+iy$ is:
        \begin{equation}
            |z|=\sqrt{x^{2}+y^{2}}
        \end{equation}
        We can also write $|z|=\sqrt{z\overline{z}}$, where
        $\overline{z}$ is the complex conjugate of $z$.
    \end{ldefinition}
    This is the size, or magnitude, of a complex number in the plane,
    using the Euclidean notion of distance: We compute the length via
    the Pythagorean formula.
    \begin{theorem}
        \label{thm:Mod_of_z_is_mod_of_conj}%
        If $z$ a complex number, then $|z|=|\overline{z}|$.
    \end{theorem}
    \begin{proof}
        For let $z=x+iy$. Then:
        \begin{equation}
            |z|=\sqrt{x^{2}+y^{2}}=\sqrt{x+(\minus{y})^{2}}=|\overline{z}|
        \end{equation}
        Therefore, etc.
    \end{proof}
    There is one particular theorem that is vital to all areas of
    mathematical analysis which dates back to Euclid: The Triangle
    Inequality. To prove this we will need a few results about the
    modulus of a complex number. Firstly, it is preserved by products,
    and secondly that the modulus of the real part of complex number
    is not greater than the entire modulus. That is, the projection of
    a complex number $z$ onto the $x$ axis is less than or equal to the
    magnitude of $z$.
    \begin{theorem}
        \label{thm:Mod_Preserves_Products}%
        If $z$ and $w$ are complex numbers, then:
        \begin{equation}
            |z\cdot{w}|=|z|\cdot|w|
        \end{equation}
    \end{theorem}
    \begin{proof}
        For let $z=a+ib$ and $w=c+id$. By
        Eqn.~\ref{eqn:Complex_Multiplication}, we have:
        \begin{equation}
            |z\cdot{w}|=|(a+ib)\cdot(c+id)|
                       =|(ac-bd)+i(ad+bc)|
        \end{equation}
        Using Def.~\ref{def:Modulus_of_Comp_Num}, we obtain:
        \begin{equation}
            |z\cdot{w}|=\sqrt{(ac-bd)^{2}+(ad+bc)^{2}}
                       =\sqrt{(ac)^{2}+(bd)^{2}+(ad)^{2}+(bc)^{2}}
        \end{equation}
        Factoring this gives us the result:
        \begin{equation}
            |z\cdot{w}|=\sqrt{(a^{2}+b^{2})(c^{2}+d^{2})}
                       =\sqrt{a^{2}+b^{2}}\sqrt{c^{2}+d^{2}}
                       =|z|\cdot|w|
        \end{equation}
        Therefore, etc.
    \end{proof}
    \begin{theorem}
        \label{thm:Mod_of_Real_Part_LEQ_Mod}%
        If $z$ is a complex number, then:
        \begin{equation}
            |\Re(z)|\leq|z|
        \end{equation}
    \end{theorem}
    \begin{proof}
        For let $z=a+ib$. Using Def.\ref{def:Modulus_of_Comp_Num},
        we have:
        \begin{equation}
            |\Re(z)|^{2}=|\Re(a+ib)|^{2}
                        =|a|^{2}
                        \leq|a|^{2}+|b|^{2}
                        =|z|^{2}
        \end{equation}
        Taking the square root of both sides completes the proof.
    \end{proof}
    \begin{ltheorem}{The Triangle Inequality}{Triangle_Inequality}
        If $z$ and $w$ are complex numbers, then $|z+w|\leq|z|+|w|$.
    \end{ltheorem}
    \begin{proof}
        Invoking Def.~\ref{def:Modulus_of_Comp_Num},
        Thms.~\ref{thm:Sum_with_Conj_is_Real}, \ref{thm:Conj_of_Conj},
        \ref{thm:Mod_of_z_is_mod_of_conj},
        \ref{thm:Mod_Preserves_Products}, and
        \ref{thm:Mod_of_Real_Part_LEQ_Mod}, we obtain:
        \par
        \begin{subequations}
            \begin{minipage}[b]{0.55\textwidth}
                \centering
                \begin{align}
                    |z+w|^{2}&=(z+w)\cdot\overline{(z+w)}\\
                             &=z\overline{z}+z\overline{w}+\overline{z}w
                                            +w\overline{w}\\
                             &=|z|^{2}+z\overline{w}
                                      +\overline{z}w+|w|^{2}\\
                             &=|z|^{2}+z\overline{w}
                                      +\overline{z\overline{w}}+|w|^{2}
                \end{align}
            \end{minipage}
            \hfill
            \begin{minipage}[b]{0.44\textwidth}
                \centering
                \begin{align}
                    &=|z|^{2}+2\Re(z\overline{w})+|w|^{2}\\
                    &\leq|z|^{2}+2|z||\overline{w}|+|w|^{2}\\
                    &=|z|^{2}+2|z||w|+|w|^{2}\\
                    &=(|z|+|w|)^{2}
                \end{align}
            \end{minipage}
        \end{subequations}
        \par
        \vspace{2.5ex}
        Taking the square root of both sides completes the proof.
    \end{proof}
    It would be nonsensical to call something the triangle inequality
    if triangles weren't involved. In Euclid's \textit{Elements} he
    proves that, given any triangle, the length of one side is less
    than the sum of the other two. This can be realized in the
    complex plane by thinking of $z$, $w$, and $z+w$ as points on a
    triangle (Fig.~\ref{fig:Triangle_Inequality}). The triangle inequality
    states that it is shorter to walk from the origin to the point $z+w$,
    than it is to walk from the origin to $z$, and then $z$ to $z+w$.
    \begin{figure}[H]
        \centering
        \captionsetup{type=figure}
        %----------------------------Dependencies-------------------------------%
%   amssymb                                                             %
%   pgfplots                                                            %
%       compat=1.9                                                      %
%   tikz                                                                %
%       arrows.meta                                                     %
%----------------------------Main Document------------------------------%
\begin{tikzpicture}[%
    >=Latex,
    line width=0.2mm,
    line cap=round,
    scale=1.7
]

    % Coordinates for the various points.
    \coordinate (O)   at (0.0, 0.0);
    \coordinate (z)   at (2.7, 1.5);
    \coordinate (sum) at (3.6, 3.6);

    % Axes:
    \begin{scope}[thick]
        \draw[->] (-0.5,  0.0) to (4.4, 0.0) node [above] {$\Re\{z\}$};
        \draw[->] ( 0.0, -0.5) to (0.0, 4.4) node [right] {$\Im\{z\}$};
    \end{scope}

    % Axes labels:
    \foreach\n in {1, 2, 3, 4}{%
        \draw (\n, 3pt) to (\n, -3pt) node [below] {$\n$};
        \draw (3pt, \n) to (-3pt, \n) node [left]  {$\n{i}$};
    }

    \draw (O) to (z);
    \draw (z) to node [below right] {$w$}   (sum);
    \draw (O) to node [above left]  {$z+w$} (sum);

    % Draw filled in circles to indicate points.
    \draw[fill=black] (z)   circle (0.4mm);
    \draw[fill=black] (sum) circle (0.4mm);

    % Nodes for labelling.
    \node at (1.0, 3.6) {\Large{$\mathbb{C}$}};
    \node at (1.5, 0.6) {$z$};
\end{tikzpicture}

        \caption{Visual Representation of the Triangle Inequality}
        \label{fig:Triangle_Inequality}
    \end{figure}
    The complex conjugate and the modulus of a complex number can
    combine to form the \textit{inverse} of a non-zero complex number.
    That, the complex numbers form something called a \textit{field}. A
    field is a set with two operations, usually called addition
    and multiplication, such that the operations are commutative,
    associative, and such that multiplication \textit{distributes}
    over addition. There is also the requirement of the existence of an
    \textit{additive identity} and a \textit{multiplicative identity}.
    Lastly, every number needs an \textit{additive inverse}, and every
    non-zero number needs a \textit{multiplicative inverse}. It is not
    difficult to see that the first eight properties are satisfied by
    the complex numbers, given $z=x+iy$,
    $\minus{z}=(\minus{x})+i(\minus{y})$ serves as the additive inverse.
    The last property is tricky, but vital for computations.
    \begin{theorem}
        \label{thm:Inverses_Are_Unique_In_Group}%
        If $G$ is a set, if $*$ is an associative operation on $G$ with an
        identity element $e$, and if $x$ has an inverse $x^{\minus{1}}$,
        then $x^{\minus{1}}$ is unique.
    \end{theorem}
    \begin{proof}
        For suppose $x'^{\minus{1}}$ is a different inverse. Then:
        \begin{equation}
            x'^{\minus{1}}=x'^{\minus{1}}*e
                          =x'^{\minus{1}}*(x*x^{\minus{1}})
                          =(x'^{\minus{1}}*x)*x^{\minus{1}}
                          =e*x^{\minus{1}}
                          =x^{\minus{1}}
        \end{equation}
        And thus $x'^{\minus{1}}=x^{\minus{1}}$. Therefore,
        the inverse is unique.
    \end{proof}
    From uniquness, once we've found a candidate for an inverse, we know
    that this is indeed the inverse. We now prove
    that non-zero complex numbers have multiplicative inverses.
    \begin{theorem}
        \label{thm:Complex_Inverse}%
        If $z$ is a non-zero complex number, then there is a unique
        $z^{\minus{1}}$ such that $z\cdot{z}^{\minus{1}}=1$.
        The inverse of $z=a+ib$ is:
        \begin{equation}
            \label{eqn:Mult_Inv_of_Complex}%
            z^{\minus{1}}=\frac{a-ib}{a^{2}+b^{2}}
        \end{equation}
    \end{theorem}
    \begin{proof}
        If $a+ib\ne{0}$, then $a^2+b^2\ne{0}$, so
        $(a-ib)/(a^2+b^2)$ is well defined. But:
        \begin{equation}
            (a+ib)\cdot\frac{a-ib}{a^2+b^2}=\frac{(a+ib)(a-ib)}{a^2+b^2}
                                           =\frac{a^2+b^2}{a^2+b^2}=1
        \end{equation}
        The uniqueness of inverses
        (Thm.~\ref{thm:Inverses_Are_Unique_In_Group}) gives us our result.
    \end{proof}
    If $|z|$ is the modulus of $z$, and $\overline{z}$ is it's complex
    conjugate, $z^{\minus{1}}$ can be written as:
    \begin{equation}
        \label{eqn:Complex_Addition_Alt}%
        z^{\minus{1}}=\frac{\overline{z}}{|z|^{2}}
    \end{equation}
    We will make use of these formulae often, so they are
    good to keep in mind.
    \begin{lexample}{}{Inverse_of_i}
        Consider the complex number $z=i$. We can use
        Eqn.~\ref{eqn:Mult_Inv_of_Complex} to compute it's
        multiplicative inverse, and we obtain $i^{\minus{1}}=\minus{i}$.
        We can also see this since $i^{\minus{1}}\cdot{i}=1$, and we
        know that $i^{2}=\minus{1}$. Multiplying by $\minus{1}$,
        we have $\minus{i}^{2}=i^{\minus{1}}\cdot{i}$. Dividing by $i$
        obtains the result again.
    \end{lexample}
    \begin{lexample}{}{Inverse_of_Comp_Num}
        Now let $z=(1+i)/F$, where $F$ is a non-zero real number.
        The multiplicative inverse of this is:
        \begin{equation}
            \Big(\frac{1+i}{F}\Big)^{\minus{1}}=F(1+i)^{\minus{1}}
                                               =F\frac{1-i}{2}
        \end{equation}
        Invoking Pythagoras, we see that $|z|=\sqrt{2}/F$.
        Using Eqn.~\ref{eqn:Complex_Addition_Alt}, we obtain:
        \begin{equation}
            z^{\minus{1}}=\frac{\overline{z}}{|z|^{2}}
                         =\frac{\frac{1-i}{F}}{\frac{2}{F^{2}}}
                         =F\frac{1-i}{2}
        \end{equation}
        In agreement with our previous calculation.
    \end{lexample}
\subsection{Polar Representation of Complex Numbers}
    In Walter Rudin's classic text on real and
    complex analysis, he opens with a prologue on the
    \textit{exponential} function and calls it
    ``Undoubtedly the most important function in
    mathematics.'' We take a moment to study this function.
    \begin{ldefinition}{Exponential Function}{Exp_Func}
        The complex exponential function is the function
        $\exp:\mathbb{C}\rightarrow\mathbb{C}$ defined by:
        \begin{equation}
            \exp(z)=\sum_{n=0}^{\infty}\frac{z^{n}}{n!}
        \end{equation}
        where $n!$ denotes the factorial of $n$,
        $n!=n\cdot(n-1)!$, and where $0!\equiv{1}$.
    \end{ldefinition}
    One useful result about the exponential function is that it relates
    multiplication and addition in a convenient way. We will need
    Cauchy's product theorem.
    \begin{ltheorem}{Cauchy's Product Theorem}{Cauchy_Product_Theorem}
        If $a:\mathbb{N}\rightarrow\mathbb{C}$ and
        $b:\mathbb{N}\rightarrow\mathbb{C}$ are sequences, if
        $\sum{a}_{n}$ converge absolutely, and if $\sum{b}_{n}$
        converges, then:
        \begin{equation}
            \Big(\sum_{j=0}^{\infty}a_{j}\Big)
            \Big(\sum_{k=0}^{\infty}b_{k}\Big)
                =\sum_{n=0}^{\infty}\sum_{m=0}^{n}a_{m}b_{n-m}
        \end{equation}
    \end{ltheorem}
    \begin{proof}
        Since the two sums $\sum{a}_{n}$ and $\sum{b}_{n}$ converge,
        let $A$ and $B$ be their limits, respectively. For all
        $n\in\mathbb{N}$, define the following partial sums:
        \par
        \begin{subequations}
            \begin{minipage}[b]{0.49\textwidth}
                \centering
                \begin{equation}
                    A_{n}=\sum_{k=0}^{n}a_{k}
                \end{equation}
            \end{minipage}
            \hfill
            \begin{minipage}[b]{0.49\textwidth}
                \centering
                \begin{equation}
                    B_{n}=\sum_{k=0}^{n}b_{k}
                \end{equation}
            \end{minipage}
        \end{subequations}
        \par\vspace{2.5ex}
        Furthermore, let $c_{n}$ be the Cauchy product and $C_{n}$ be
        the partial sums:
        \par
        \begin{subequations}
            \begin{minipage}[b]{0.49\textwidth}
                \centering
                \begin{equation}
                    c_{n}=\sum_{k=0}^{n}a_{k}b_{n-k}
                \end{equation}
            \end{minipage}
            \hfill
            \begin{minipage}[b]{0.49\textwidth}
                \centering
                \begin{equation}
                    C_{n}=\sum_{m=0}^{n}c_{m}
                \end{equation}
            \end{minipage}
        \end{subequations}
        \par\vspace{2.5ex}
        And finally, let $\beta_{n}=B_{n}-B$. Then, for all
        $n\in\mathbb{N}$:
        \begin{equation}
            C_{n}=\sum_{j=0}^{n}\sum_{k=0}^{j}a_{k}b_{j-k}
                 =\sum_{j=0}^{n}a_{j}B_{n-j}
                 =A_{n}B+\sum_{j=0}^{n}a_{j}\beta_{n-j}
        \end{equation}
        Let $d_{n}$ be the remainder term. That is:
        \begin{equation}
            d_{n}=\sum_{j=0}^{n}a_{j}\beta_{n-j}
        \end{equation}
        Since $\sum{a}_{n}$ is absolutely convergent, and thus
        $\sum|a_{n}|$ converges. Let $A'$ be the limit. Also, by the
        definition of $\beta_{n}$, $\beta_{n}$ converges to zero. That
        is, given any $\varepsilon>0$ there is an $N\in\mathbb{N}$ such
        that, for all $n>N$, we have $|\beta_{n}|<\varepsilon$.
        But then:
        \begin{equation}
            |d_{n}|=\Big|\sum_{j=0}^{N}a_{j}\beta_{n-j}+
                         \sum_{j=N+1}^{n}a_{j}\beta_{n-j}\Big|
                \leq\Big|\sum_{j=0}^{N}a_{j}\beta_{n-j}\Big|
                   +\Big|\sum_{j=N+1}^{n}a_{j}\beta_{n-j}\Big|
        \end{equation}
        Where this last step comes from the triangle inequality.
        Simplifying, we have:
        \begin{equation}
            |d_{n}|<|\sum_{j=0}^{N}a_{j}\beta_{n-j}|+\varepsilon{A}'
        \end{equation}
        The first term can be made small since $\beta_{n-j}$ is small
        for large $n$ (And since $j<N$), and the second term can also
        be made small since $\varepsilon$ is arbitrary. So we see that
        $d_{n}$ converges to zero. Thus:
        \begin{equation}
            \underset{n\rightarrow\infty}{\lim}C_{n}
            =\underset{n\rightarrow\infty}{\lim}A_{n}B+d_{n}
            =AB
        \end{equation}
        This completes the proof.
    \end{proof}
    The immediate application of this is the power
    rule for the exponential function. We will need the
    \textit{binomial theorem}. Let $\binom{n}{k}$ (Which reads as
    $n$ \textit{choose} $k$) denote the \textit{binomial coefficient}:
    \begin{equation}
        \binom{n}{k}=\frac{n!}{k!(n-k)!}
    \end{equation}
    The binomial theorem then says the following:
    \begin{ltheorem}{The Binomial Theorem}{Binomial_Theorem}
        If $x$ and $y$ are real numbers, and if $n\in\mathbb{N}$, then:
        \begin{equation}
            (x+y)^{n}=\sum_{k=0}^{n}\binom{n}{k}x^{n-k}y^{k}
        \end{equation}
        Where $\binom{n}{k}$ denotes the binomial coefficient.
    \end{ltheorem}
    \begin{proof}
        We prove by induction. When $n=0$ or $n=1$ we can evaluate the
        validity of this by hand. When $n=2$ this is commonly known
        as the FOIL rule. Suppose it is true for $n\in\mathbb{N}$. We must
        now show this implies it is true for $n+1$. We have:
        \begin{equation}
            (x+y)^{n+1}=(x+y)(x+y)^{n}
                       =(x+y)\sum_{k=0}^{n}\binom{n}{k}x^{n-k}y^{k}
        \end{equation}
        We can further simplify, and perform a shift of index, to obtain:
        \begin{subequations}
            \begin{align}
                (x+y)^{n+1}&=(x+y)\sum_{k=0}^{n}\binom{n}{k}x^{n-k}y^{k}\\
                           &=x\sum_{k=0}^{n}\binom{n}{k}x^{n-k}y^{k}
                            +y\sum_{k=0}^{n}\binom{n}{k}x^{n-k}y^{k}\\
                           &=\sum_{k=0}^{n}\binom{n}{k}x^{n+1-k}y^{k}
                            +\sum_{k=1}^{n+1}\binom{n}{k-1}
                                x^{n+1-k}y^{k}\\
                    \label{eqn:Binomial_Theorem_Pascal_Identity}%
                    &=x^{n+1}+\sum_{k=1}^{n}
                        \Big[\binom{n}{k}+\binom{n}{k-1}\Big]
                        x^{n+1-k}y^{k}+y^{n+1}
            \end{align}
        \end{subequations}
        The sum of these two binomial coefficients is known as
        Pascal's Identity. We have:
        \begin{subequations}
            \begin{align}
                \binom{n}{k}+\binom{n}{k-1}
                    &=\frac{n!}{k!(n-k)!}+\frac{n!}{(k-1)!(n+1-k)!}\\
                    &=n!\frac{(n+1-k)+k}{k!(n+1-k)!}\\
                    &=\frac{(n+1)!}{k!(n+1-k)!}\\
                    &=\binom{n+1}{k}
            \end{align}
        \end{subequations}
        Thus, returning to
        Eqn.~\ref{eqn:Binomial_Theorem_Pascal_Identity}, we obtain:
        \begin{subequations}
            \begin{align}
                (x+y)^{n+1}
                    &=x^{n+1}+\sum_{k=1}^{n}\binom{n+1}{k}x^{n+1-k}y^{k}
                     +y^{n+1}\\
                    &=\sum_{k=0}^{n+1}\binom{n+1}{k}x^{n+1-k}y^{k}
            \end{align}
        \end{subequations}
        This completes the proof.
    \end{proof}
    \begin{theorem}
        \label{thm:Expo_Product_Formula}%
        If $a$ and $b$ are complex numbers, then:
        \begin{equation}
            \exp(a+b)=\exp(a)\exp(b)
        \end{equation}
    \end{theorem}
    \begin{proof}
        Invoking the \textit{binomial theorem}
        (Thm.~\ref{thm:Binomial_Theorem}), we have:
        \begin{equation}
            \exp(a+b)=\sum_{n=0}^{\infty}\frac{(a+b)^{n}}{n!}
                     =\sum_{n=0}^{\infty}\frac{1}{n!}
                      \sum_{k=0}^{n}\frac{n!}{k!(n-k)!}a^{k}b^{n-k}
        \end{equation}
        But by Cauchy's product theorem
        (Thm.~\ref{thm:Cauchy_Product_Theorem}), this last double
        sum can be written as the product of two sums of the form:
        \begin{equation}
            \sum_{n=0}^{\infty}\sum_{k=0}^{n}
                \frac{1}{k!(n-k)!}a^{k}b^{n-k}
                =\Big(\sum_{j=0}^{\infty}\frac{a^{j}}{j!}\Big)
                 \Big(\sum_{m=0}^{\infty}\frac{b^{m}}{m!}\Big)
        \end{equation}
        But this is just the product of $\exp(a)$
        and $\exp(b)$. Therefore, etc.
    \end{proof}
    We next prove one of the most important theorems of complex analysis:
    Euler's Theorem. This is a crucial part of the theory and allows
    one to define the \textit{polar representation} of a complex number.
    It relates the exponential function to the trigonometric functions.
    \newpage
    \begin{ftheorem}{Euler's Exponential Formula}{Euler_Expo_Formula}
        If $\theta$ is a real number, then:
        \begin{equation}
            \exp(i\theta)=\cos(\theta)+i\sin(\theta)
        \end{equation}
    \end{ftheorem}
    \begin{bproof}
        Using the definition of the exponential function
        (Def.~\ref{def:Exp_Func}) and evaluating $i\theta$ into this
        equation, we obtain:
        \begin{equation}
            \label{def:Exp_Func_With_it}%
            \exp(i\theta)=\sum_{n=0}^{\infty}i^{n}\frac{\theta^{n}}{n!}
        \end{equation}
        But $i^{n}$ cycles between $i$, $\minus{1}$, $\minus{i}$, and $1$.
        So we may split this sum into two parts, a real part and an
        imaginary part, to obtain:
        \begin{equation}
            \sum_{n=0}^{\infty}\exp(i\theta)
                =\sum_{n=0}^{\infty}(\minus{1})^{n}\frac{x^{2n}}{(2n)!}
               +i\sum_{n=0}^{\infty}(\minus{1})^{n}
                    \frac{x^{2n+1}}{(2n+1)!}
        \end{equation}
        But the left sum is the Taylor expansion for $\cos(\theta)$,
        and the right sum is the Taylor expansion for $i\sin(\theta)$.
        This completes the proof.
    \end{bproof}
    Euler's Theorem can also be proved by showing the two expressions
    satisfy the same initial value problem: $\ddot{z}+z=0$, $z(0)=1$,
    $\dot{z}(0)=i$. A corollary of this is often hailed as the most
    beautiful result in mathematics. This is Euler's Identity:
    \begin{equation}
        e^{i\pi}+1=0
    \end{equation}
    Combining Euler's Exponential Formula and the product rule for the
    exponential function, we see that given any complex number $z=a+ib$,
    the following holds:
    \begin{equation}
        \exp(z)=\exp(a)\big(\cos(b)+i\sin(b)\big)
    \end{equation}
    Many of the identities from trigonometry are short corollaries of this
    theorem, rendering memorization of these formulae redundant.
    \begin{ltheorem}{DeMoivre's Theorem}{DeMoivres_Theorem}
        If $n\in\mathbb{N}$ and if $\theta\in\mathbb{R}$, then:
        \begin{equation}
            \big(\cos(\theta)+i\sin(\theta)\big)^{n}
            =\cos(n\theta)+i\sin(n\theta)
        \end{equation}
    \end{ltheorem}
    \begin{proof}
        By Euler's formula (Thm.~\ref{thm:Euler_Expo_Formula}), and by
        Thm.~\ref{thm:Expo_Product_Formula}, we have:
        \begin{equation}
            \big(\cos(\theta)+i\sin(\theta)\big)^{n}
                =\big(\exp(i\theta)\big)^{n}
                =\exp(in\theta)
                =\cos(n\theta)+i\sin(n\theta)
        \end{equation}
        Therefore, etc.
    \end{proof}
    Combining the binomial theorem (Thm.~\ref{thm:Binomial_Theorem})
    and DeMoivre's Theorem allows one to quickly compute any
    trigonometric identity one might need. Letting $n=2$, we obtain the
    double-angle formula:
    \begin{equation}
        \cos(2\theta)+i\sin(2\theta)=\cos^{2}(\theta)-\sin^{2}(\theta)
                                    +2i\cos(\theta)\sin(\theta)
    \end{equation}
    Equating real and imaginary parts, we have:
    \par
    \begin{subequations}
        \begin{minipage}[b]{0.49\textwidth}
            \centering
            \begin{equation}
                \cos(2\theta)=\cos^{2}(\theta)-\sin^{2}(\theta)
            \end{equation}
        \end{minipage}
        \hfill
        \begin{minipage}[b]{0.49\textwidth}
            \centering
            \begin{equation}
                \sin(2\theta)=2\cos(\theta)\sin(\theta)
            \end{equation}
        \end{minipage}
    \end{subequations}
    \par
    \vspace{2.5ex}
    The important thing is that we can now define the polar form of a
    complex number.
    \begin{theorem}
        \label{thm:Polar_Form_Comp_Num}%
        If $z$ is a complex number, then there is a unique real number
        $r\geq{0}$ and a real number $\theta\in[0,2\pi)$ such that:
        \begin{equation}
            \label{eqn:Polar_Form_Comp_Num}%
            z=r\exp(i\theta)
        \end{equation}
    \end{theorem}
    \begin{proof}
        Let $z=x+iy$. Define $r$ and $\theta$ as:
        \begin{subequations}
            \begin{equation}
                r=\sqrt{x^{2}+y^{2}}
            \end{equation}
            \begin{equation}
                \theta=
                \begin{cases}
                    \arctan\big(\frac{y}{x}\big),&x>0,y\geq{0}\\
                    \frac{\pi}{2}+\arctan\big(\frac{y}{|x|}\big),
                        &x<0,y\geq{0}\\
                    \pi+\arctan\big(\frac{y}{x}\big),
                        &x<0,y\leq{0}\\
                    \frac{3\pi}{2}+\arctan\big(\frac{|y|}{x}\big),
                        &x<0,y\geq{0}\\
                    \frac{\pi}{2}\sgn(y),&x=0
                \end{cases}
            \end{equation}
        \end{subequations}
        Here $\sgn(y)$ is the sign of $y$. Euler's Theorem completes
        the proof. Uniqueness of $r$ comes from the fact that
        $|\exp(i\theta)|=1$, so if $z=r_{1}\exp(i\theta_{1})$
        and $z=r_{2}\exp(i\theta_{2})$, then $|r_{1}|=|r_{2}|$. But
        $r_{1}$ and $r_{2}$ are non-negative, and thus $r_{1}=r_{2}$.
    \end{proof}
    Eqn.~\ref{eqn:Polar_Form_Comp_Num} is the definition of the polar
    form of a complex number. This gives geometrical interpretations of
    many aspects of complex arithmetic. Multiplication can be seen
    as rotations and scaling in the complex plane. For if
    $z=r_{1}\exp(i\theta_{1})$ and if $w=z_{2}\exp(i\theta_{2})$,
    then we have:
    \begin{equation}
        z\cdot{w}=r_{1}r_{2}\exp\big(i(\theta_{1}+\theta_{2})\big)
    \end{equation}
    That is, multiplying $z$ by $w$ scales $z$ by the magnitude of $w$,
    and rotates it in the plane by the angle $\theta_{2}$. This also
    allows us to define square roots. We define the $n^{th}$ root of a
    complex number to be:
    \begin{equation}
        \label{eqn:Sqrt_of_Comp_Num}%
        \sqrt[n]{z}=\sqrt[n]{r}\exp\Big(\frac{i\theta}{n}\Big)
    \end{equation}
    This is well defined for all complex numbers since the $n^{th}$
    root of a non-negative real number $r$ is well defined, and
    $\exp(i\theta/n)$ is well defined for all real $\theta$. Thus we
    have avoided the messiness of square roots that occurs in the real
    world. By $\sqrt[n]{r}$, we still mean the positive root.
    So $\sqrt{4}=2$, and not $\minus{2}$.
    \begin{figure}[H]
        \centering
        \captionsetup{type=figure}
        %--------------------------------Dependencies----------------------------------%
%   amssymb                                                                    %
%   tikz                                                                       %
%       arrows.meta                                                            %
%       angles                                                                 %
%       quotes                                                                 %
%-------------------------------Main Document----------------------------------%
\begin{tikzpicture}[%
    >=Latex,
    line width=0.2mm,
    line cap=round,
    scale=1.7
]

    % Coordinates for various points.
    \coordinate (O)   at (0.0, 0.0);
    \coordinate (z)   at (3.3, 1.6);
    \coordinate (z_x) at (3.3, 0.0);
    \coordinate (C)   at (4.0, 4.0);

    % Axes:
    \begin{scope}[thick]
        \draw[->] (-0.5,  0.0) to (4.4, 0.0) node [above] {$\Re\{z\}$};
        \draw[->] ( 0.0, -0.5) to (0.0, 4.4) node [right] {$\Im\{z\}$};
    \end{scope}

    % Axes labels:
    \foreach\n in {1, 2, 3, 4}{%
        \draw (\n, 3pt) to (\n, -3pt) node [below] {$\n$};
        \draw (3pt, \n) to (-3pt, \n) node [left] {$\n{i}$};
    }

    % Draw a line from the origin to the point z.
    \draw (O) to node [above] {$r$} (z);

    % Draw a dot marking the point z.
    \draw[fill=black] (z) circle (0.4mm);

    % Nodes to label various things.
    \node at (C) {\Large{$\mathbb{C}$}};
    \node at (z) [above] {$z=(r,\theta)$};
    \pic[%
        draw=black,
        "{${\theta}$}",
        angle eccentricity=1.2,
        angle radius=1.3cm
    ]   {angle = z_x--O--z};
\end{tikzpicture}

        \caption{Polar Representation of a Complex Number}
        \label{fig:Comp_Num_Polar}
    \end{figure}
    \begin{lexample}{}{Square_Root_of_i}
        Consider the square root of $i$. Using Euler's formula, we
        have that $i=\exp(i\pi/2)$. Using
        Eqn.~\ref{eqn:Sqrt_of_Comp_Num}, we obtain:
        \begin{equation}
            \sqrt{i}=\exp\big(\frac{i\pi}{4}\big)
                    =\cos\big(\frac{\pi}{4}\big)
                        +i\sin\big(\frac{\pi}{4}\big)
                    =\frac{1+i}{\sqrt{2}}
        \end{equation}
        We can check this solution by squaring:
        \begin{equation}
            \Big(\frac{1+i}{\sqrt{2}}\Big)^{2}=\frac{(1-1)+i(1+1)}{2}
                                              =\frac{2i}{2}
                                              =i
        \end{equation}
        in agreement with the definition of square roots.
    \end{lexample}
    We must be careful when evaluating square roots. We define the polar
    representation as $z=r\exp(i\theta)$, where
    $0\leq\theta<2\pi$. Problems can occur if we allow $\theta$ to be
    any real number. For note that $\exp(2\pi{i})=1=\exp(0i)$. Thus, we
    may naively perform the following computation:
    \begin{equation}
        1=\sqrt{1}=\exp(2\pi{i}/2)=\exp(\pi{i})=\minus{1}
    \end{equation}
    The angle $\theta\in[0,2\pi)$ that we use to represent $z$ is called
    the \textit{principal value of the argument}, and is often denoted
    $\mathrm{Arg}(z)$.
    \begin{lexample}{}{Roots_of_Unity}
        Let $f(z)=z^{n}-1$. This has a trivial root
        at $z=1$, and by the Fundamental Theorem of Algebra there are
        at most $n$ roots. The roots of this polynomial are called the
        \textit{roots of unity}. The real solutions are $1$ for odd $n$,
        and $\pm{1}$ for even $n$. In the complex world, there are
        always $n$ solutions. Interestingly enough, these points form an
        $n\textrm{-gon}$ around the origin of the complex plane. The
        solutions are:
        \begin{equation}
            z_{k}=\exp\Big(\frac{2\pi{i}{k}}{n}\Big)
            \quad\quad
            k=0,\,1,\,2,\,\dots,\,n-1
        \end{equation}
        Let's plot these solutions for various $n$.
        \begin{figure}[H]
            \centering
            \captionsetup{type=figure}
            %--------------------------------Dependencies----------------------------------%
%   amssymb                                                                    %
%   pgfplots                                                                   %
%       compat=1.9                                                             %
%   tikz                                                                       %
%       arrows.meta                                                            %
%   Unary minus sign.                                                          %
%       \DeclareMathSymbol{\minus}{\mathbin}{AMSa}{"39}                        %
%-------------------------------Main Document----------------------------------%
\begin{tikzpicture}[%
    >=Latex,
    line width=0.2mm,
    line cap=round,
    scale=2.0
]

    % Draw the cubed roots of unity.
    \begin{scope}[xshift=-0.6in]

        % Coordinates for the three roots and origin.
        \coordinate (O)  at (0.0, 0.0);
        \coordinate (Z1) at (1.0, 0.0);
        \coordinate (Z2) at (-0.5, 0.866);
        \coordinate (Z3) at (-0.5, -0.866);

        % Axes:
        \begin{scope}[thick]
            \draw[<->] (-1.3, 0) to (1.3, 0) node[above] {$\Re\{z\}$};
            \draw[<->] (0, -1.3) to (0, 1.3) node[right] {$\Im\{z\}$};
        \end{scope}

        % Axes labels:
        \draw (1, 3pt)  to (1, -3pt)  node [below] {$1$};
        \draw (-1, 3pt) to (-1, -3pt) node [below] {$\minus{1}$};
        \draw (3pt, 1)  to (-3pt, 1)  node [left]  {$1{i}$};
        \draw (3pt, -1) to (-3pt, -1) node [left]  {$\minus{i}$};

        % Draw the cubed roots of unity.
        \draw[blue] (O) to (Z1);
        \draw[blue] (O) to (Z2);
        \draw[blue] (O) to (Z3);

        % Connect the three roots with dashed lines.
        \draw[densely dashed, thin] (Z1) to (Z2) to (Z3) to cycle;

        % Draw dots to mark the points.
        \draw[fill=blue] (Z1) circle (0.3mm);
        \draw[fill=blue] (Z2) circle (0.3mm);
        \draw[fill=blue] (Z3) circle (0.3mm);
    \end{scope}

    % Draw the fourth roots of unity.
    \begin{scope}[xshift=0.6in]

        % Coordinates for the four roots.
        \coordinate (O)  at ( 0.0,  0.0);
        \coordinate (Z1) at ( 1.0,  0.0);
        \coordinate (Z2) at ( 0.0,  1.0);
        \coordinate (Z3) at (-1.0,  0.0);
        \coordinate (Z4) at ( 0.0, -1.0);

        % Axes:
        \begin{scope}[thick]
            \draw[<->] (-1.3, 0) to (1.3, 0) node [above] {$\Re\{z\}$};
            \draw[<->] (0, -1.3) to (0, 1.4) node [right] {$\Im\{z\}$};
        \end{scope}

        % Axes labels:
        \draw (1, 3pt)  to (1, -3pt)  node [below] {$1$};
        \draw (-1, 3pt) to (-1, -3pt) node [below] {$\minus{1}$};
        \draw (3pt, 1)  to (-3pt, 1)  node [left]  {$1{i}$};
        \draw (3pt, -1) to (-3pt, -1) node [left]  {$\minus{i}$};

        % Draw the quartic roots of unity.
        \draw[blue] (O) to (Z1);
        \draw[blue] (O) to (Z2);
        \draw[blue] (O) to (Z3);
        \draw[blue] (O) to (Z4);

        % Draw dashed lines connecting the roots.
        \draw[densely dashed, thin] (Z1) to (Z2) to (Z3) to (Z4) to cycle;

        % Draw dots to mark the points.
        \draw[fill=blue] (Z1) circle (0.3mm);
        \draw[fill=blue] (Z2) circle (0.3mm);
        \draw[fill=blue] (Z3) circle (0.3mm);
        \draw[fill=blue] (Z4) circle (0.3mm);
    \end{scope}

    % Draw the fifth roots of unity.
    \begin{scope}[xshift=-0.6in, yshift=-1.2in]

        % Coordinates for the five roots.
        \coordinate (O)  at (0, 0);
        \coordinate (Z1) at (1, 0);
        \coordinate (Z2) at (0.309, 0.951);
        \coordinate (Z3) at (-0.809, 0.587);
        \coordinate (Z4) at (-0.809, -0.587);
        \coordinate (Z5) at (0.309, -0.951);

        % Axes:
        \begin{scope}[thick]
            \draw[<->] (-1.4, 0) to (1.4, 0) node[above] {$\Re\{z\}$};
            \draw[<->] (0, -1.4) to (0, 1.4) node[right] {$\Im\{z\}$};
        \end{scope}

        % Axes labels:
        \draw (1, 3pt)  to (1, -3pt)  node [below] {$1$};
        \draw (-1, 3pt) to (-1, -3pt) node [below] {$\minus{1}$};
        \draw (3pt, 1)  to (-3pt, 1)  node [left]  {$1{i}$};
        \draw (3pt, -1) to (-3pt, -1) node [left]  {$\minus{i}$};

        % Draw the fifth roots of unity.
        \draw[blue] (O) to (Z1);
        \draw[blue] (O) to (Z2);
        \draw[blue] (O) to (Z3);
        \draw[blue] (O) to (Z4);
        \draw[blue] (O) to (Z5);

        % Draw dashed lines connecting the roots.
        \draw[densely dashed, thin] (Z1) to (Z2) to (Z3)
                                         to (Z4) to (Z5) to cycle;

        % Draw circles to mark the five roots.
        \draw[fill=blue] (Z1) circle (0.3mm);
        \draw[fill=blue] (Z2) circle (0.3mm);
        \draw[fill=blue] (Z3) circle (0.3mm);
        \draw[fill=blue] (Z4) circle (0.3mm);
        \draw[fill=blue] (Z5) circle (0.3mm);
    \end{scope}

    % Draw the sixth roots of unity.
    \begin{scope}[xshift=0.6in, yshift=-1.2in]

        % Coordinates for the six roots.
        \coordinate (O)  at (0, 0);
        \coordinate (Z1) at (1, 0);
        \coordinate (Z2) at (0.5, 0.866);
        \coordinate (Z3) at (-0.5, 0.866);
        \coordinate (Z4) at (-1, 0);
        \coordinate (Z5) at (-0.5, -0.866);
        \coordinate (Z6) at (0.5, -0.866);

        % Axes:
        \begin{scope}[thick]
            \draw[<->] (-1.4, 0) to (1.4, 0) node [above] {$\Re\{z\}$};
            \draw[<->] (0, -1.4) to (0, 1.4) node [right] {$\Im\{z\}$};
        \end{scope}

        % Axes labels:
        \draw (1, 3pt) to (1, -3pt)   node [below] {$1$};
        \draw (-1, 3pt) to (-1, -3pt) node [below] {$\minus{1}$};
        \draw (3pt, 1)  to (-3pt, 1)  node [left]  {$1{i}$};
        \draw (3pt, -1) to (-3pt, -1) node [left]  {$\minus{i}$};

        % Draw the sixth roots of unity.
        \draw[blue] (O) to (Z1);
        \draw[blue] (O) to (Z2);
        \draw[blue] (O) to (Z3);
        \draw[blue] (O) to (Z4);
        \draw[blue] (O) to (Z5);
        \draw[blue] (O) to (Z6);

        % Draw dashed lines connecting the roots.
        \draw[densely dashed, thin] (Z1) to (Z2) to (Z3) to (Z4)
                                         to (Z5) to (Z6) to cycle;

        % Draw dots marking the roots.
        \draw[fill=blue] (Z1) circle (0.3mm);
        \draw[fill=blue] (Z2) circle (0.3mm);
        \draw[fill=blue] (Z3) circle (0.3mm);
        \draw[fill=blue] (Z4) circle (0.3mm);
        \draw[fill=blue] (Z5) circle (0.3mm);
        \draw[fill=blue] (Z6) circle (0.3mm);
    \end{scope}
\end{tikzpicture}

            \caption{Roots of Unity for Degrees 3 to 6.}
            \label{fig:Comp_Roots_Unity}
        \end{figure}
        It should be clear from the definition of $f$ that the roots lie
        on the unit circle centered at the origin.
        While this is certainly an interesting and aesthetically
        appealing bit of mathematics, it also spells trouble for
        certain methods of numerical analysis. We'll return to
        this later when we discuss root finding algorithms.
    \end{lexample}
\subsection{Analytic Functions}
    We take a brief moment to talk about what it means
    to be analytic, the Cauchy-Riemann Equations, and
    Green's Theorem. The results here are counter-intuitive, and
    it is easy to apply certain results where they do not hold.
    \begin{ldefinition}{Entire Function}{Entire_Func}
        An entire function is a function
        $f:\mathbb{C}\rightarrow\mathbb{C}$ such that
        for all $z_{0}\in\mathbb{C}$, the following limit
        exists:
        \begin{equation}
            f'(z_{0})=\underset{z\rightarrow{z_{0}}}{\lim}
                      \frac{f(z)-f(z_{0})}{z-z_{0}}
        \end{equation}
        where $f'$ is called the derivative of $f$.
    \end{ldefinition}
    An entire function is simply a complex function that
    is \textit{differentiable} at every point in the
    complex plane. We can weaken this definition to include
    only some parts of the complex plane, and these are
    called \textit{holomorphic} functions.
    A function $f$ is analytic about the point $z_{0}$
    if its Taylor Series converges for all $z$ to $f(z)$ in some
    neighborhood of $z_{0}$:
    \begin{equation}
        f(z)=\sum_{n=0}^{\infty}\frac{f^{(n)}(z_{0})}{n!}(z-z_{0})^{n}
    \end{equation}
    \begin{lexample}{}{Exp_Func_Is_Entire}
        The exponential function is analytic, since we've  defined it
        as a power series. It is indeed entire as well, since:
        \begin{equation}
            \underset{z\rightarrow{z_{0}}}{\lim}
                \frac{\exp(z)-\exp(z_{0})}{z-z_{0}}
            =\exp(z_{0})\underset{z\rightarrow{z_{0}}}{\lim}
             \frac{\exp(z-z_{0})-1}{z-z_{0}}
        \end{equation}
        Letting $w=z-z_{0}$, we have:
        \begin{subequations}
            \begin{align}
                \underset{z\rightarrow{z_{0}}}{\lim}
                    \frac{\exp(z)-\exp(z_{0})}{z-z_{0}}
                &=\exp(z_{0})\underset{w\rightarrow{0}}{\lim}
                  \frac{\exp(w)-1}{w}\\
                &=\exp(z_{0})\underset{w\rightarrow{0}}{\lim}
                  \Big(1+\sum_{n=2}^{\infty}\frac{w^{n-1}}{n!}\Big)\\
                &=\exp(z_{0})
            \end{align}
            This proves $\exp$ is differentiable at every point
            $z_{0}\in\mathbb{C}$, and is thus entire.
        \end{subequations}
    \end{lexample}
    The remarkable fact of entire functions is that they are automatically
    analytic. This is certainly not true for real valued functions. One
    only need consider the example $f(x)=x|x|$. The derivative is
    $f'(x)=2|x|$, and this has no derivative at the
    origin. Similarly, there are functions with two derivatives,
    but not three. In the real world, having $n$ derivatives does
    not imply having $n+1$ derivatives. For complex functions, one
    derivative implies \textit{all} higher derivatives exist.
    \begin{lexample}{}{Smooth_Non_Analytic}%
        It is often believed that \textit{most} functions of a
        real variable are analytic, but the opposite is true. Real valued
        functions can be quite messy, and we need not construct overly
        pathological examples to show this. For consider the following:
        \begin{equation}
            f(x)=
            \begin{cases}
                \exp\big(\!\minus\!\frac{1}{x^{2}}\big),&x\ne{0}\\
                0,&x=0
            \end{cases}
        \end{equation}
        This is a function that most students of calculus can understand,
        and is everywhere \textit{smooth}: For all $x_{0}\in\mathbb{R}$,
        and for all $n\in\mathbb{N}$, the $n^{th}$ derivative
        $f^{(n)}(x_{0})$ exists.
        \begin{figure}[H]
            \centering
            \captionsetup{type=figure}
            %--------------------------------Dependencies----------------------------------%
%   amssymb                                                                    %
%   pgfplots                                                                   %
%       compat=1.9                                                             %
%   tikz                                                                       %
%       arrows.meta                                                            %
%   Unary minus sign.                                                          %
%       \DeclareMathSymbol{\minus}{\mathbin}{AMSa}{"39}                        %
%-------------------------------Main Document----------------------------------%
\begin{tikzpicture}[%
    >=Latex,
    line width=0.2mm,
    line cap=round,
    scale=3
]

    % Axes:
    \begin{scope}[thick]
        \draw[<->] (-1.6, 0) to (1.6, 0) node [above] {$x$};
        \draw[<->] (0, -0.2) to (0, 1.2) node [right] {$y$};
    \end{scope}

    % Axes Labels:
    \begin{scope}[font=\large]
        \draw (1, 0.05)  to (1, -0.05)  node [below] {$1$};
        \draw (-1, 0.05) to (-1, -0.05) node [below] {$\minus{1}$};
        \draw (0.05, 1)  to (-0.05, 1)  node [left]  {1};

        % Label for the function.
        \node at (0.7, 0.9) {$f(x)=\exp\big(\!\minus\!\frac{1}{x^{2}}\big)$};
    \end{scope}

    % Draw the function (Points calculated before hand).
    \draw[blue] (-1.49, 0.6373539279680577)
             to (-1.48, 0.6334731785079655)
             to (-1.47, 0.6295373200237058)
             to (-1.46, 0.625545525546842)
             to (-1.45, 0.6214969616146686)
             to (-1.44, 0.617390788765911)
             to (-1.43, 0.6132261620873393)
             to (-1.42, 0.6090022318149272)
             to (-1.41, 0.6047181439934188)
             to (-1.40, 0.6003730411984056)
             to (-1.39, 0.5959660633252755)
             to (-1.38, 0.5914963484496641)
             to (-1.37, 0.5869630337643256)
             to (-1.36, 0.5823652565976396)
             to (-1.35, 0.5777021555192884)
             to (-1.34, 0.5729728715389651)
             to (-1.33, 0.568176549404326)
             to (-1.32, 0.5633123390047627)
             to (-1.31, 0.5583793968879508)
             to (-1.30, 0.5533768878965255)
             to (-1.29, 0.5483039869326549)
             to (-1.28, 0.5431598808587065)
             to (-1.27, 0.5379437705426525)
             to (-1.26, 0.5326548730573174)
             to (-1.25, 0.52729242404305)
             to (-1.24, 0.521855680243893)
             to (-1.23, 0.5163439222278197)
             to (-1.22, 0.51075645730212)
             to (-1.21, 0.5050926226355373)
             to (-1.20, 0.4993517885992776)
             to (-1.19, 0.49353336233953554)
             to (-1.18, 0.4876367915946986)
             to (-1.17, 0.4816615687708995)
             to (-1.16, 0.47560723529008375)
             to (-1.15, 0.4694733862252225)
             to (-1.14, 0.4632596752377421)
             to (-1.13, 0.45696581983263557)
             to (-1.12, 0.4505916069470661)
             to (-1.11, 0.4441368988885388)
             to (-1.10, 0.4376016396389122)
             to (-1.09, 0.4309858615406031)
             to (-1.08, 0.42428969238130754)
             to (-1.07, 0.4175133628933708)
             to (-1.06, 0.41065721468358135)
             to (-1.05, 0.4037217086085994)
             to (-1.04, 0.3967074336104235)
             to (-1.03, 0.38961511602520493)
             to (-1.02, 0.3824456293773025)
             to (-1.01, 0.3752000046686704)
             to (-1.00, 0.36787944117144394)
             to (-0.99, 0.36048531772883974)
             to (-0.98, 0.35301920456618285)
             to (-0.97, 0.3454828756099071)
             to (-0.96, 0.3378783213076685)
             to (-0.95, 0.33020776193715917)
             to (-0.94, 0.32247366138471106)
             to (-0.93, 0.3146787413672178)
             to (-0.92, 0.3068259960621399)
             to (-0.91, 0.29891870710026475)
             to (-0.90, 0.29096045886431193)
             to (-0.89, 0.28295515402326293)
             to (-0.88, 0.27490702921726423)
             to (-0.87, 0.2668206707909444)
             to (-0.86, 0.25870103045382237)
             to (-0.85, 0.2505534407249795)
             to (-0.84, 0.24238362999516153)
             to (-0.83, 0.234197737012785)
             to (-0.82, 0.2260023245708207)
             to (-0.81, 0.21780439213909428)
             to (-0.80, 0.2096113871510995)
             to (-0.79, 0.20143121461595628)
             to (-0.78, 0.19327224468472104)
             to (-0.77, 0.18514331775604914)
             to (-0.76, 0.17705374665950327)
             to (-0.75, 0.16901331540606748)
             to (-0.74, 0.16103227394534206)
             to (-0.73, 0.15312132831839462)
             to (-0.72, 0.14529162554556097)
             to (-0.71, 0.13755473254124842)
             to (-0.70, 0.12992260830506097)
             to (-0.69, 0.1224075686029256)
             to (-0.68, 0.11502224232652866)
             to (-0.67, 0.10777951870813635)
             to (-0.66, 0.10069248457540325)
             to (-0.65, 0.09377435086251001)
             to (-0.64, 0.08703836765622475)
             to (-0.63, 0.08049772715543974)
             to (-0.62, 0.07416545406842356)
             to (-0.61, 0.06805428317217875)
             to (-0.60, 0.062176524022117395)
             to (-0.59, 0.056543913137095655)
             to (-0.58, 0.05116745440357744)
             to (-0.57, 0.046057248950990025)
             to (-0.56, 0.04122231635355694)
             to (-0.55, 0.03667040971370454)
             to (-0.54, 0.03240782797472983)
             to (-0.53, 0.028439229684312707)
             to (-0.52, 0.02476745336398339)
             to (-0.51, 0.02139335059811582)
             to (-0.50, 0.018315638888734703)
             to (-0.49, 0.015530782160000365)
             to (-0.48, 0.013032907448509774)
             to (-0.47, 0.01081376667096577)
             to (-0.46, 0.00886275228811561)
             to (-0.45, 0.007166975037612688)
             to (-0.44, 0.005711410539145645)
             to (-0.43, 0.004479119346177029)
             to (-0.42, 0.003451541830562694)
             to (-0.41, 0.002608865109376468)
             to (-0.40, 0.001930454136227812)
             to (-0.39, 0.0013953333115509971)
             to (-0.38, 0.00098269893511076)
             to (-0.37, 0.000672437155990288)
             to (-0.36, 0.00044561759559204756)
             to (-0.35, 0.00028493048887659066)
             to (-0.34, 0.00017503597545462341)
             to (-0.33, 0.00010279884345300595)
             to (-0.32, 5.739088873947436e-05)
             to (-0.31, 3.025566061372087e-05)
             to (-0.30, 1.4945338524783204e-05)
             to (-0.29, 6.854491542533225e-06)
             to (-0.28, 2.8875503621934562e-06)
             to (-0.27, 1.1030614309402012e-06)
             to (-0.26, 3.762923728763693e-07)
             to (-0.25, 1.1253517471928151e-07)
             to (-0.24, 2.8851290572494142e-08)
             to (-0.23, 6.16984770536148e-09)
             to (-0.22, 1.064078123060734e-09)
             to (-0.21, 1.419229286794676e-10)
             to (-0.20, 1.388794386496925e-11)
             to (-0.19, 9.325710764248232e-13)
             to (-0.18, 3.943204710722736e-14)
             to (-0.17, 9.386620866567126e-16)
             to (-0.16, 1.084855264043724e-17)
             to (-0.15, 4.989109392799343e-20)
             to (-0.14, 6.95213617457717e-23)
             to (-0.13, 2.0049413023827448e-26)
             to (-0.12, 6.928847118341716e-31)
             to (-0.11, 1.2820178072520857e-36)
             to (-0.10, 0.0)
             to (-0.09, 0.0)
             to (-0.08, 0.0)
             to (-0.07, 0.0)
             to (-0.06, 0.0)
             to (-0.05, 0.0)
             to (-0.04, 0.0)
             to (-0.03, 0.0)
             to (-0.02, 0.0)
             to (-0.01, 0.0)
             to (0.00, 0.0)
             to (0.01, 0.0)
             to (0.02, 0.0)
             to (0.03, 0.0)
             to (0.04, 0.0)
             to (0.05, 0.0)
             to (0.06, 0.0)
             to (0.07, 0.0)
             to (0.08, 0.0)
             to (0.09, 0.0)
             to (0.10, 0.0)
             to (0.11, 1.2820178072520857e-36)
             to (0.12, 6.928847118341716e-31)
             to (0.13, 2.0049413023827448e-26)
             to (0.14, 6.95213617457717e-23)
             to (0.15, 4.989109392799343e-20)
             to (0.16, 1.084855264043724e-17)
             to (0.17, 9.386620866567126e-16)
             to (0.18, 3.943204710722736e-14)
             to (0.19, 9.325710764248232e-13)
             to (0.20, 1.388794386496925e-11)
             to (0.21, 1.419229286794676e-10)
             to (0.22, 1.064078123060734e-09)
             to (0.23, 6.16984770536148e-09)
             to (0.24, 2.8851290572494142e-08)
             to (0.25, 1.1253517471928151e-07)
             to (0.26, 3.762923728763693e-07)
             to (0.27, 1.1030614309402012e-06)
             to (0.28, 2.8875503621934562e-06)
             to (0.29, 6.854491542533225e-06)
             to (0.30, 1.4945338524783204e-05)
             to (0.31, 3.025566061372087e-05)
             to (0.32, 5.739088873947436e-05)
             to (0.33, 0.00010279884345300595)
             to (0.34, 0.00017503597545462341)
             to (0.35, 0.00028493048887659066)
             to (0.36, 0.00044561759559204756)
             to (0.37, 0.000672437155990288)
             to (0.38, 0.00098269893511076)
             to (0.39, 0.0013953333115509971)
             to (0.40, 0.001930454136227812)
             to (0.41, 0.002608865109376468)
             to (0.42, 0.003451541830562694)
             to (0.43, 0.004479119346177029)
             to (0.44, 0.005711410539145645)
             to (0.45, 0.007166975037612688)
             to (0.46, 0.00886275228811561)
             to (0.47, 0.01081376667096577)
             to (0.48, 0.013032907448509774)
             to (0.49, 0.015530782160000365)
             to (0.50, 0.018315638888734703)
             to (0.51, 0.02139335059811582)
             to (0.52, 0.02476745336398339)
             to (0.53, 0.028439229684312707)
             to (0.54, 0.03240782797472983)
             to (0.55, 0.03667040971370454)
             to (0.56, 0.04122231635355694)
             to (0.57, 0.046057248950990025)
             to (0.58, 0.05116745440357744)
             to (0.59, 0.056543913137095655)
             to (0.60, 0.062176524022117395)
             to (0.61, 0.06805428317217875)
             to (0.62, 0.07416545406842356)
             to (0.63, 0.08049772715543974)
             to (0.64, 0.08703836765622475)
             to (0.65, 0.09377435086251001)
             to (0.66, 0.10069248457540325)
             to (0.67, 0.10777951870813635)
             to (0.68, 0.11502224232652866)
             to (0.69, 0.1224075686029256)
             to (0.70, 0.12992260830506097)
             to (0.71, 0.13755473254124842)
             to (0.72, 0.14529162554556097)
             to (0.73, 0.15312132831839462)
             to (0.74, 0.16103227394534206)
             to (0.75, 0.16901331540606748)
             to (0.76, 0.17705374665950327)
             to (0.77, 0.18514331775604914)
             to (0.78, 0.19327224468472104)
             to (0.79, 0.20143121461595628)
             to (0.80, 0.2096113871510995)
             to (0.81, 0.21780439213909428)
             to (0.82, 0.2260023245708207)
             to (0.83, 0.234197737012785)
             to (0.84, 0.24238362999516153)
             to (0.85, 0.2505534407249795)
             to (0.86, 0.25870103045382237)
             to (0.87, 0.2668206707909444)
             to (0.88, 0.27490702921726423)
             to (0.89, 0.28295515402326293)
             to (0.90, 0.29096045886431193)
             to (0.91, 0.29891870710026475)
             to (0.92, 0.3068259960621399)
             to (0.93, 0.3146787413672178)
             to (0.94, 0.32247366138471106)
             to (0.95, 0.33020776193715917)
             to (0.96, 0.3378783213076685)
             to (0.97, 0.3454828756099071)
             to (0.98, 0.35301920456618285)
             to (0.99, 0.36048531772883974)
             to (1.00, 0.36787944117144394)
             to (1.01, 0.3752000046686704)
             to (1.02, 0.3824456293773025)
             to (1.03, 0.38961511602520493)
             to (1.04, 0.3967074336104235)
             to (1.05, 0.4037217086085994)
             to (1.06, 0.41065721468358135)
             to (1.07, 0.4175133628933708)
             to (1.08, 0.42428969238130754)
             to (1.09, 0.4309858615406031)
             to (1.10, 0.4376016396389122)
             to (1.11, 0.4441368988885388)
             to (1.12, 0.4505916069470661)
             to (1.13, 0.45696581983263557)
             to (1.14, 0.4632596752377421)
             to (1.15, 0.4694733862252225)
             to (1.16, 0.47560723529008375)
             to (1.17, 0.4816615687708995)
             to (1.18, 0.4876367915946986)
             to (1.19, 0.49353336233953554)
             to (1.20, 0.4993517885992776)
             to (1.21, 0.5050926226355373)
             to (1.22, 0.51075645730212)
             to (1.23, 0.5163439222278197)
             to (1.24, 0.521855680243893)
             to (1.25, 0.52729242404305)
             to (1.26, 0.5326548730573174)
             to (1.27, 0.5379437705426525)
             to (1.28, 0.5431598808587065)
             to (1.29, 0.5483039869326549)
             to (1.30, 0.5533768878965255)
             to (1.31, 0.5583793968879508)
             to (1.32, 0.5633123390047627)
             to (1.33, 0.568176549404326)
             to (1.34, 0.5729728715389651)
             to (1.35, 0.5777021555192884)
             to (1.36, 0.5823652565976396)
             to (1.37, 0.5869630337643256)
             to (1.38, 0.5914963484496641)
             to (1.39, 0.5959660633252755)
             to (1.40, 0.6003730411984056)
             to (1.41, 0.6047181439934188)
             to (1.42, 0.6090022318149272)
             to (1.43, 0.6132261620873393)
             to (1.44, 0.617390788765911)
             to (1.45, 0.6214969616146686)
             to (1.46, 0.625545525546842)
             to (1.47, 0.6295373200237058)
             to (1.48, 0.6334731785079655)
             to (1.49, 0.6373539279680577);
\end{tikzpicture}

            \caption{A Smooth Function That is Not Analytic at the Origin}
            \label{fig:Smooth_Not_Analytic_At_Origin}
        \end{figure}
        However, this function is not analytic at the origin. This
        function approaches zero so quickly at the origin that for all
        $n\in\mathbb{N}$ we have:
        \begin{equation}
            \frac{\diff^{n}f}{\diff{x}^{n}}(0)=0
        \end{equation}
        The Taylor expansion is thus zero, the radius of convergence
        is infinite, but $f$ is not the zero function. Thus $f$ is a
        function that is smooth but not analytic.
    \end{lexample}
    \begin{lexample}{}{Smooth_Nowhere_Analytic}
        Further study of Ex.~\ref{ex:Smooth_Non_Analytic} reveals that
        $f$ is analytic \textit{everywhere else}, so one might expect
        smooth functions must be \textit{somewhere} analytic, but
        this is false. Consider:
        \begin{equation}
            F(x)=\sum_{k=0}^{\infty}\exp(\minus\sqrt{2^{k}})\cos(2^{k}x)
        \end{equation}
        Application of the $M$ test from calculus shows that this sum
        converges, and that all of its derivatives exist. However, for
        all $x\in\mathbb{R}$, the Taylor series:
        \begin{equation}
            \sum_{n=0}^{\infty}
                F^{(n)}(x_{0})\frac{(x-x_{0})^{n}}{n!}
        \end{equation}
        diverges for all $x\ne{x}_{0}$. So this function is smooth
        and \textit{nowhere} analytic.
    \end{lexample}
    \begin{figure}[H]
        \centering
        %--------------------------------Dependencies----------------------------------%
%   amssymb                                                                    %
%   pgfplots                                                                   %
%       compat=1.9                                                             %
%   tikz                                                                       %
%       arrows.meta                                                            %
%   Unary minus sign.                                                          %
%       \DeclareMathSymbol{\minus}{\mathbin}{AMSa}{"39}                        %
%-------------------------------Main Document----------------------------------%
\begin{tikzpicture}[%
    >=Latex,
    line width=0.2mm,
    line cap=round,
    scale=3
]

    % Axes:
    \begin{scope}[thick]
        \draw[<->] (-1.6, 0) to (1.6, 0) node [above] {$x$};
        \draw[<->] (0, -0.2) to (0, 1.2) node [right] {$y$};
    \end{scope}

    % Axes Labels:
    \begin{scope}[font=\large]
        \draw (1, 0.05)  to ( 1, -0.05) node [below] {$1$};
        \draw (-1, 0.05) to (-1, -0.05) node [below] {$\minus{1}$};
        \draw (0.05, 1)  to (-0.05, 1)  node [left] {1};

        % Label for the function.
        \node at (0.8, 0.9) {$F(x)=\sum\exp(\minus\sqrt{2^{k}})\times$};
        \node at (1.2, 0.7) {$\cos(2^{k}x)$};
    \end{scope}

    % Draw the function (Points calculated before hand).
    \draw[blue] (-1.5, -0.02937852591712955)
             to (-1.49, -0.03254571141051943)
             to (-1.48, -0.036002260965522774)
             to (-1.47, -0.03969011307645282)
             to (-1.46, -0.043521828574620725)
             to (-1.45, -0.04736794272046125)
             to (-1.44, -0.05109899851116673)
             to (-1.43, -0.054631412130562905)
             to (-1.42, -0.05794359513738732)
             to (-1.41, -0.06105903647659064)
             to (-1.4, -0.06403290645258392)
             to (-1.39, -0.06695388810876417)
             to (-1.38, -0.06992283673300181)
             to (-1.37, -0.07300256680849901)
             to (-1.36, -0.0761637128321949)
             to (-1.35, -0.0792810663950719)
             to (-1.34, -0.08218537189688116)
             to (-1.33, -0.08471615792663889)
             to (-1.32, -0.08674714323781758)
             to (-1.31, -0.08818555628445766)
             to (-1.3, -0.08898331090739822)
             to (-1.29, -0.0891562813709484)
             to (-1.28, -0.08877259369850855)
             to (-1.27, -0.08790943709953121)
             to (-1.26, -0.08660801543571002)
             to (-1.25, -0.08487974541297444)
             to (-1.24, -0.08275245937436759)
             to (-1.23, -0.08030375319777651)
             to (-1.22, -0.07766017463727219)
             to (-1.21, -0.07497294223084021)
             to (-1.2, -0.07241040190041072)
             to (-1.19, -0.07015289901872526)
             to (-1.18, -0.0683578170539517)
             to (-1.17, -0.06710209770250775)
             to (-1.16, -0.0663396223494681)
             to (-1.15, -0.06592248742982568)
             to (-1.14, -0.06565953678681268)
             to (-1.13, -0.06536209078544927)
             to (-1.12, -0.06485910933292226)
             to (-1.11, -0.06399850116910182)
             to (-1.1, -0.0626685322020603)
             to (-1.09, -0.060811555856694705)
             to (-1.08, -0.058402513357751144)
             to (-1.07, -0.055402965406200266)
             to (-1.06, -0.05173342559824931)
             to (-1.05, -0.047303825980685776)
             to (-1.04, -0.042063101391724214)
             to (-1.03, -0.03602448682300089)
             to (-1.02, -0.029255252425168892)
             to (-1.01, -0.021857692165652084)
             to (-1.0, -0.013969884738666782)
             to (-0.99, -0.0057527285094008)
             to (-0.98, 0.0026542049752210674)
             to (-0.97, 0.011190683701628618)
             to (-0.96, 0.019898070354111237)
             to (-0.95, 0.028877991200081513)
             to (-0.94, 0.03823585129540363)
             to (-0.93, 0.048047111507936925)
             to (-0.92, 0.05835406693428543)
             to (-0.91, 0.06916133484494459)
             to (-0.9, 0.0804146315558443)
             to (-0.89, 0.09200279180321524)
             to (-0.88, 0.10379430579724938)
             to (-0.87, 0.11568743625847747)
             to (-0.86, 0.1276211808834097)
             to (-0.85, 0.13953439111635388)
             to (-0.84, 0.15132583110889183)
             to (-0.83, 0.16284524654866325)
             to (-0.82, 0.17391541360937604)
             to (-0.81, 0.18434629195917096)
             to (-0.8, 0.1939363267251311)
             to (-0.79, 0.20249892991060114)
             to (-0.78, 0.20991602557890587)
             to (-0.77, 0.21619067614121656)
             to (-0.76, 0.22144427474524062)
             to (-0.75, 0.22586189389471706)
             to (-0.74, 0.2296402911567279)
             to (-0.73, 0.23296402795236756)
             to (-0.72, 0.23600467998535166)
             to (-0.71, 0.23890478106886787)
             to (-0.7, 0.24175718390910717)
             to (-0.69, 0.24461689881928544)
             to (-0.68, 0.2475427126994567)
             to (-0.67, 0.25063567218731764)
             to (-0.66, 0.2540232726745881)
             to (-0.65, 0.25780880814457185)
             to (-0.64, 0.2620371630600785)
             to (-0.63, 0.2666956893971372)
             to (-0.62, 0.2717353796677231)
             to (-0.61, 0.27707374080701536)
             to (-0.6, 0.2825998440417076)
             to (-0.59, 0.28821048392649645)
             to (-0.58, 0.29386739543262386)
             to (-0.57, 0.29963473108500366)
             to (-0.56, 0.3056518522133886)
             to (-0.55, 0.3120750803584976)
             to (-0.54, 0.3190356929343689)
             to (-0.53, 0.326629368544421)
             to (-0.52, 0.334916076932714)
             to (-0.51, 0.3438999885265096)
             to (-0.5, 0.3535217264569611)
             to (-0.49, 0.3636863110991995)
             to (-0.48, 0.3743133607797781)
             to (-0.47, 0.38536346911427705)
             to (-0.46, 0.396806834564667)
             to (-0.45, 0.4085782302793931)
             to (-0.44, 0.42055822427933853)
             to (-0.43, 0.4325891801630923)
             to (-0.42, 0.44449526417196306)
             to (-0.41, 0.4560831959094872)
             to (-0.4, 0.4671598985839953)
             to (-0.39, 0.4775802026372402)
             to (-0.38, 0.4873046826596487)
             to (-0.37, 0.49641566000326337)
             to (-0.36, 0.505071151539558)
             to (-0.35, 0.5134483460818265)
             to (-0.34, 0.5217111174079181)
             to (-0.33, 0.5300063976816948)
             to (-0.32, 0.5384548012618016)
             to (-0.31, 0.5471266638884859)
             to (-0.3, 0.5560441594662361)
             to (-0.29, 0.5652168254705201)
             to (-0.28, 0.5746867610319384)
             to (-0.27, 0.5845298561965913)
             to (-0.26, 0.594808848411639)
             to (-0.25, 0.6055320259815571)
             to (-0.24, 0.6166444815328893)
             to (-0.23, 0.6280484581853614)
             to (-0.22, 0.6396126143269155)
             to (-0.21, 0.6511722657562067)
             to (-0.2, 0.6625574125875934)
             to (-0.19, 0.6736469832000488)
             to (-0.18, 0.684418153665471)
             to (-0.17, 0.6949371615641742)
             to (-0.16, 0.7053036782993417)
             to (-0.15, 0.7156023564178071)
             to (-0.14, 0.7258841249934336)
             to (-0.13, 0.7361681901759755)
             to (-0.12, 0.746426883941273)
             to (-0.11, 0.7565705204373665)
             to (-0.1, 0.7664663927972456)
             to (-0.09, 0.7759860996986521)
             to (-0.08, 0.7850447001274024)
             to (-0.07, 0.7935834148429975)
             to (-0.06, 0.8015228232971983)
             to (-0.05, 0.8087353972538599)
             to (-0.04, 0.8150532894797189)
             to (-0.03, 0.8202922160593791)
             to (-0.02, 0.8242553764743977)
             to (-0.01, 0.8267432190915989)
             to (0.0, 0.8275941133489709)
             to (0.01, 0.8267432190915985)
             to (0.02, 0.8242553764743968)
             to (0.03, 0.8202922160593776)
             to (0.04, 0.8150532894797173)
             to (0.05, 0.8087353972538581)
             to (0.06, 0.8015228232971963)
             to (0.07, 0.7935834148429955)
             to (0.08, 0.7850447001273999)
             to (0.09, 0.7759860996986497)
             to (0.1, 0.766466392797243)
             to (0.11, 0.7565705204373637)
             to (0.12, 0.7464268839412701)
             to (0.13, 0.7361681901759726)
             to (0.14, 0.7258841249934309)
             to (0.15, 0.7156023564178043)
             to (0.16, 0.705303678299339)
             to (0.17, 0.6949371615641714)
             to (0.18, 0.6844181536654682)
             to (0.19, 0.6736469832000459)
             to (0.2, 0.6625574125875903)
             to (0.21, 0.6511722657562036)
             to (0.22, 0.6396126143269125)
             to (0.23, 0.6280484581853583)
             to (0.24, 0.6166444815328863)
             to (0.25, 0.6055320259815542)
             to (0.26, 0.5948088484116362)
             to (0.27, 0.5845298561965883)
             to (0.28, 0.5746867610319357)
             to (0.29, 0.5652168254705175)
             to (0.3, 0.5560441594662335)
             to (0.31, 0.5471266638884834)
             to (0.32, 0.5384548012617992)
             to (0.33, 0.5300063976816924)
             to (0.34, 0.5217111174079159)
             to (0.35, 0.5134483460818242)
             to (0.36, 0.5050711515395556)
             to (0.37, 0.49641566000326104)
             to (0.38, 0.48730468265964627)
             to (0.39, 0.47758020263723755)
             to (0.4, 0.4671598985839925)
             to (0.41, 0.45608319590948415)
             to (0.42, 0.44449526417195995)
             to (0.43, 0.4325891801630891)
             to (0.44, 0.42055822427933526)
             to (0.45, 0.40857823027938983)
             to (0.46, 0.3968068345646639)
             to (0.47, 0.3853634691142741)
             to (0.48, 0.3743133607797752)
             to (0.49, 0.36368631109919675)
             to (0.5, 0.3535217264569585)
             to (0.51, 0.3438999885265074)
             to (0.52, 0.33491607693271164)
             to (0.53, 0.32662936854441876)
             to (0.54, 0.319035692934367)
             to (0.55, 0.312075080358496)
             to (0.56, 0.305651852213387)
             to (0.57, 0.29963473108500205)
             to (0.58, 0.2938673954326224)
             to (0.59, 0.2882104839264951)
             to (0.6, 0.2825998440417061)
             to (0.61, 0.27707374080701375)
             to (0.62, 0.2717353796677216)
             to (0.63, 0.26669568939713617)
             to (0.64, 0.26203716306007735)
             to (0.65, 0.25780880814457074)
             to (0.66, 0.2540232726745871)
             to (0.67, 0.2506356721873168)
             to (0.68, 0.2475427126994559)
             to (0.69, 0.24461689881928467)
             to (0.7, 0.24175718390910642)
             to (0.71, 0.23890478106886712)
             to (0.72, 0.23600467998535085)
             to (0.73, 0.23296402795236665)
             to (0.74, 0.22964029115672696)
             to (0.75, 0.225861893894716)
             to (0.76, 0.22144427474523923)
             to (0.77, 0.21619067614121493)
             to (0.78, 0.2099160255789041)
             to (0.79, 0.2024989299105992)
             to (0.8, 0.1939363267251286)
             to (0.81, 0.18434629195916805)
             to (0.82, 0.1739154136093731)
             to (0.83, 0.16284524654866053)
             to (0.84, 0.15132583110888867)
             to (0.85, 0.13953439111635055)
             to (0.86, 0.12762118088340657)
             to (0.87, 0.11568743625847454)
             to (0.88, 0.10379430579724627)
             to (0.89, 0.09200279180321187)
             to (0.9, 0.08041463155584122)
             to (0.91, 0.0691613348449419)
             to (0.92, 0.05835406693428262)
             to (0.93, 0.048047111507934)
             to (0.94, 0.03823585129540108)
             to (0.95, 0.02887799120007928)
             to (0.96, 0.019898070354108892)
             to (0.97, 0.011190683701626182)
             to (0.98, 0.002654204975218823)
             to (0.99, -0.00575272850940282)
             to (1.0, -0.013969884738668911)
             to (1.01, -0.021857692165654308)
             to (1.02, -0.02925525242517078)
             to (1.03, -0.036024486823002466)
             to (1.04, -0.04206310139172567)
             to (1.05, -0.04730382598068717)
             to (1.06, -0.051733425598250365)
             to (1.07, -0.0554029654062011)
             to (1.08, -0.058402513357751866)
             to (1.09, -0.060811555856695315)
             to (1.1, -0.0626685322020607)
             to (1.11, -0.06399850116910212)
             to (1.12, -0.0648591093329225)
             to (1.13, -0.0653620907854494)
             to (1.14, -0.06565953678681274)
             to (1.15, -0.06592248742982573)
             to (1.16, -0.06633962234946823)
             to (1.17, -0.06710209770250804)
             to (1.18, -0.06835781705395207)
             to (1.19, -0.07015289901872573)
             to (1.2, -0.07241040190041138)
             to (1.21, -0.07497294223084099)
             to (1.22, -0.07766017463727294)
             to (1.23, -0.08030375319777712)
             to (1.24, -0.08275245937436823)
             to (1.25, -0.08487974541297501)
             to (1.26, -0.08660801543571041)
             to (1.27, -0.08790943709953146)
             to (1.28, -0.08877259369850872)
             to (1.29, -0.08915628137094844)
             to (1.3, -0.08898331090739808)
             to (1.31, -0.08818555628445741)
             to (1.32, -0.08674714323781713)
             to (1.33, -0.08471615792663818)
             to (1.34, -0.08218537189688045)
             to (1.35, -0.07928106639507114)
             to (1.36, -0.07616371283219406)
             to (1.37, -0.07300256680849809)
             to (1.38, -0.06992283673300098)
             to (1.39, -0.0669538881087634)
             to (1.4, -0.06403290645258311)
             to (1.41, -0.0610590364765898)
             to (1.42, -0.057943595137386464)
             to (1.43, -0.05463141213056207)
             to (1.44, -0.051098998511165755)
             to (1.45, -0.047367942720460084)
             to (1.46, -0.043521828574619704)
             to (1.47, -0.03969011307645193)
             to (1.48, -0.036002260965521754)
             to (1.49, -0.03254571141051846);
\end{tikzpicture}

        \caption{A Smooth and Nowhere Analytic Function}
        \label{fig:Smooth_But_Non_Analytic}
    \end{figure}
    The difficulties shown in Ex.~\ref{ex:Smooth_Non_Analytic} and
    Ex.~\ref{ex:Smooth_Nowhere_Analytic} vanish when we study functions
    of a complex variable. Given a function
    $f:\mathbb{C}\rightarrow\mathbb{C}$, differentiable at $z_{0}$
    implies twice differentiable at $z_{0}$, which further implies
    smooth at $z_{0}$, and this implies analytic at $z_{0}$. There is
    one step that is missing here: Continuous does \textit{not} imply
    differentiable. And, unfortunately, there are functions that
    \textit{look} differentiable (Meaning the formula used to represent
    them would make us think at first glance that they are
    differentiable), but are not. Again, as we will see, we need not
    construct overly pathological examples to show this.
    \begin{lexample}{}{Comp_Conj_is_Continuous}
        Let $f:\mathbb{C}\rightarrow\mathbb{C}$ be defined by
        $f(z)=\overline{z}$. That is, $f$ maps $x+iy$ to $x-iy$.
        Then $f$ is continuous at all $z\in\mathbb{C}$. For let
        $\varepsilon>0$ be given, and let $\delta=\varepsilon/2$.
        Then, for all $z_{0}$ such that $|z-z_{0}|<\delta$, we have:
        \begin{subequations}
            \begin{align}
                |f(z)-f(z_{0})|&=|x-iy-(x_{0}-iy_{0})|\\
                               &=|(x-x_{0})+i(y_{0}-y)|\\
                               &\leq|x-x_{0}|+|y-y_{0}|\\
                               &<\frac{\varepsilon}{2}
                                +\frac{\varepsilon}{2}=\varepsilon
            \end{align}
        \end{subequations}
        where we have applied the triangle inequality
        (Thm.~\ref{thm:Triangle_Inequality}) and
        Thm.~\ref{thm:Mod_of_Real_Part_LEQ_Mod} to derive these
        inequalities. Thus $f$ is a continuous function.
        We will soon see that $f$ is \textit{nowhere} differentiable.
    \end{lexample}
    To reveal such functions we will need to present the
    \textit{Cauchy-Riemann} equations. This set of equations provides both
    a \textit{necessary} and a \textit{sufficient} condition for a
    function $f:\mathbb{C}\rightarrow\mathbb{C}$ to be entire.
    We will prove the easy part: Entire functions satisfy the
    Cauchy-Riemann equations.
    \newpage
    \begin{ftheorem}{Cauchy-Riemann Theorem}{Cauchy_Riemann}
        If $f:\mathbb{C}\rightarrow\mathbb{C}$ is an entire function
        defined by:
        \begin{equation}
            f(z)=u(x,\,y)+iv(x,\,y)
        \end{equation}
        Then:
        \par
        \begin{subequations}
            \begin{minipage}{0.49\textwidth}
                \centering
                \begin{equation}
                    \frac{\partial{u}}{\partial{x}}=
                    \frac{\partial{v}}{\partial{y}}
                \end{equation}
            \end{minipage}
            \hfill
            \begin{minipage}{0.49\textwidth}
                \centering
                \begin{equation}
                    \frac{\partial{u}}{\partial{y}}=
                    -\frac{\partial{v}}{\partial{x}}
                \end{equation}
            \end{minipage}
        \end{subequations}
        \par
        \vspace{2.5ex}
    \end{ftheorem}
    \begin{bproof}
        As $f$ is entire, its complex derivative exists at every point
        in the complex plane. Let $z_{0}=x_{0}+iy_{0}$, and let
        $z=x+iy_{0}$. Taking the limit, we have:
        \begin{subequations}
            \begin{align}
                f'(z_{0})&=\underset{x\rightarrow{x_{0}}}{\lim}
                    \frac{u(x,\,y_{0})+iv(x,\,y_{0})
                         -u(x_{0},\,y_{0})-iv(x_{0},\,y_{0})}{x-x_{0}}\\
                &=\underset{x\rightarrow{x_{0}}}{\lim}
                    \frac{\big(u(x,\,y_{0})-u(x_{0},\,y_{0})\big)
                        +i\big(v(x,\,y_{0})-iv(x_{0},\,y_{0})\big)}
                            {x-x_{0}}\\
                &=\underset{x\rightarrow{x_{0}}}{\lim}
                    \Big(\frac{u(x,\,y_{0})-u(x_{0},\,y_{0})}
                              {x-x_{0}}\Big)
                +i\underset{x\rightarrow{x_{0}}}{\lim}
                    \Big(\frac{v(x,\,y_{0})-v(x_{0},\,y_{0})}
                              {x-x_{0}}\Big)
                \end{align}
        \end{subequations}
        Using the definition of \textit{partial derivatives}, we obtain:
        \begin{equation}
            \label{eqn:Cauchy_Riemann_x_Limit}%
            f'(z_{0})=
            \frac{\partial{u}}{\partial{x}}+
            i\frac{\partial{v}}{\partial{x}}
        \end{equation}
        Next we evaluate the limit along the path $z=x_{0}+iy$. Since the
        function is complex differentiable, any path as
        $z\rightarrow{z_{0}}$ will give the same value. Therefore:
        \begin{subequations}
            \begin{align}
                f'(z_{0})&=\underset{y\rightarrow{y_{0}}}{\lim}
                \frac{u(x_{0},\,y)+iv(x_{0},\,y)-
                      u(x_{0},\,y_{0})-iv(x_{0},\,y_{0})}{i(y-y_{0})}\\
                &=\underset{y\rightarrow{y_{0}}}{\lim}
                \frac{\big(u(x_{0},\,y)-u(x_{0},\,y_{0})\big)+
                     i\big(v(x_{0},\,y)-iv(x_{0},\,y_{0})\big)}
                        {i(y-y_{0})}\\
                &=\frac{1}{i}\underset{y\rightarrow{y_{0}}}{\lim}
                    \Big(\frac{u(x_{0},\,y)-u(x_{0},\,y_{0})}
                              {i(y-y_{0})}\Big)
                +\underset{y\rightarrow{y_{0}}}{\lim}
                    \Big(\frac{v(x_{0},\,y)-v(x_{0},\,y_{0})}
                              {(y-y_{0})}\Big)
            \end{align}
        \end{subequations}
        Recalling our result from Thm.~\ref{thm:Complex_Inverse}, the
        inverse of $i$ is $\minus{i}$. Again using the definition of
        partial derivatives:
        \begin{equation}
            \label{eqn:Cauchy_Riemann_y_Limit}%
            f'(z_{0})=-i\frac{\partial{u}}{\partial{y}}+
                        \frac{\partial{v}}{\partial{y}}
        \end{equation}
        Thus, equating
        Eqn.~\ref{eqn:Cauchy_Riemann_x_Limit} and
        Eqn.~\ref{eqn:Cauchy_Riemann_y_Limit}, we obtain:
        \begin{equation}
              \frac{\partial{u}}{\partial{x}}
            +i\frac{\partial{v}}{\partial{x}}
            = \frac{\partial{v}}{\partial{y}}
            -i\frac{\partial{u}}{\partial{y}}
        \end{equation}
        Comparing real and imaginary parts completes the proof.
    \end{bproof}
    This theorem excludes many functions from being analytic.
    \begin{lexample}{}{Real_Valued_Analytic_Func_Is_Constant}
        Let $f:\mathbb{C}\rightarrow\mathbb{C}$ be an analytic function,
        and suppose for all $z\in\mathbb{C}$, $f(z)$ is a \textit{real}
        number. That is, $f$ is a real-valued function. Then $f$
        must be a constant. For:
        \begin{equation}
            f(z)=u(x,\,y)+0i
        \end{equation}
        And from the Cauchy-Riemann equations, we have:
        \par
        \begin{subequations}
            \begin{minipage}[b]{0.49\textwidth}
                \centering
                \begin{equation}
                    \frac{\partial{u}}{\partial{x}}=0
                \end{equation}
            \end{minipage}
            \hfill
            \begin{minipage}[b]{0.49\textwidth}
                \centering
                \begin{equation}
                    \frac{\partial{u}}{\partial{y}}=0
                \end{equation}
            \end{minipage}
        \end{subequations}
        \par\hfill\par
        And thus we conclude that $u(x,\,y)=const$.Given a non-constant
        real function $f:\mathbb{R}\rightarrow\mathbb{R}$ that is
        analytic (Has a Taylor series), the complex extension
        $F(x+iy)=f(x)$ is \textit{not} analytic. Indeed, it is nowhere
        differentiable. Lastly, consider the function $f(z)=z$. This
        is indeed complex analytic. However:
        \begin{equation}
            \overline{f(z)}=x-iy
        \end{equation}
        is \textit{nowhere-analytic}. Indeed, it is nowhere
        differentiable. This is counterintuitive and reveals the bizarre
        nature of complex functions. It is worth recalling
        Ex.~\ref{ex:Comp_Conj_is_Continuous} where we showed that
        $\overline{f}$ is continuous. Thus, we see that continuity
        does not imply differentiability, even for complex valued
        functions.
    \end{lexample}
    We will return to this later when we discuss convolutions and the
    Hilbert transform. The Cauchy-Riemann equations seem to give some
    information for free. If we know $f(z)=u(x,\,y)+iv(x,\,y)$ is
    analytic, and we know $u(x,\,y)$, then we can determine $v(x,\,y)$, up
    to an additive constant. In the theory of signal processing, given a
    real valued function $u:\mathbb{R}\rightarrow\mathbb{R}$, also called
    a \textit{signal}, it will often be the case that we seek a real
    valued function $v:\mathbb{R}\rightarrow\mathbb{R}$, called the
    harmonic conjugate of $u$, such that $u+iv$ is the boundary of some
    analytic function. Imposing certain criteria on $u$ reveals that $v$
    is unique. Thus, given a complex signal where the imaginary part has
    been lost but the real part exists, we can recover the imaginary
    component by computing the harmonic conjugate of $u$. This is the
    \textit{Hilbert Transform}, and we return to it in the section about
    Fourier Analysis.
\subsection{Contour Integrals}
    Next we introduce contour integrals. Throughout this section,
    integration is meant in the sense of the Riemann integral.
    We start by defining \textit{Jordan Curves}.
    \begin{ldefinition}{Jordan Curve}{Jordan_Curve}
        A Jordan Curve in the Complex Plane is a continuous function
        $\Gamma:[0,1]\rightarrow\mathbb{C}$ such that
        $\Gamma(0)=\Gamma(1)$, and there are no values $0<x_{1}<x_{2}<1$
        such that $\Gamma(x_{1})=\Gamma(x_{2})$.
    \end{ldefinition}
    A simple example of a Jordan curve is a circle. Jordan curves are
    \textit{closed}, meaning they start where they end, and do not
    self-intersect. A Figure-8 is thus \textbf{not} a Jordan curve, but
    an ellipse is. An example of a Jordan curve is given below in
    Fig.~\subref{fig:Ex_of_Smooth_Jordan_Curve}. Much the way
    the closed unit interval $[0,1]$ has an ordering on it, a Jordan
    curve has a direction associated with it. Given a Jordan curve
    $\Gamma(t)$, one may change directions by defining
    $\reflectbox{\ensuremath{\Gamma}}(t)=\Gamma(1-t)$.
    While this will plot out the same curve in the complex plane, the
    direction is different and thus it represents a different path. When
    evaluating contour integrals, the direction matters.
    \begin{figure}[H]
        \captionsetup{type=figure}
        \centering
        \begin{subfigure}[b]{0.49\textwidth}
            \centering
            \documentclass[crop,class=article]{standalone}
%----------------------------Preamble-------------------------------%
\usepackage{amssymb}        % For \mathbb{C}.
\usepackage{tikz}           % Drawing/graphing tools.
\usetikzlibrary{
    arrows.meta,            % Latex and Stealth arrows.
    decorations.markings,   % Adding arrows in the middle of a line.
}
%--------------------------Main Document----------------------------%
\begin{document}
    \begin{tikzpicture}[thick]
        % Axes:
        \draw[>=Latex, ->] (-0.5, 0) -- (4.4, 0) node[above] {$\Re\{z\}$};
        \draw[>=Latex, ->] (0, -0.5) -- (0, 4.4) node[right] {$\Im\{z\}$};
        % Axes labels:
        \foreach\n in {1,2,3,4}{%
            \draw (\n,3pt) -- (\n,-3pt) node [below] {$\n$};
            \draw (3pt,\n) -- (-3pt,\n) node [left] {$\n{i}$};
        }
        \begin{scope}[%
            >=Latex,
            ->-/.style={%
                decoration={%
                    markings,
                    mark=at position 0 with \arrow{>},
                    mark=at position .15 with \arrow{>},
                    mark=at position .4 with \arrow{>},
                    mark=at position .6 with \arrow{>},
                    mark=at position .8 with \arrow{>}
                },
                postaction={decorate}
            }
        ]
            \draw[->-]
                (1.5,3) to [out=70, in=140] (3,3)
                        to [out=-50, in=90] (4,1)
                        to [out=-90, in=-30] (2,1.5)
                        to [out=150, in=-90] (1,2)
                        to [out=90, in=-110] cycle;
        \end{scope}
        \node at (4,4) {\Large{$\mathbb{C}$}};
        \node at (3.5,3.2) {\normalsize{$\Gamma(t)$}};
    \end{tikzpicture}
\end{document}
            \subcaption{A Smooth Jordan Curve.}
            \label{fig:Ex_of_Smooth_Jordan_Curve}
        \end{subfigure}
        \begin{subfigure}[b]{0.49\textwidth}
            \centering
            %--------------------------------Dependencies----------------------------------%
%   amssymb                                                                    %
%   tikz                                                                       %
%       arrows.meta                                                            %
%       decorations.markings                                                   %
%-------------------------------Main Document----------------------------------%
\begin{tikzpicture}[
    >=Latex,
    ->-/.style={%
        decoration={%
            markings,
            mark=at position 0.24 with \arrow{>},
            mark=at position .52 with \arrow{>},
            mark=at position .84 with \arrow{>}
        },
        postaction={decorate}
    }
]

    % Coordinates for points on the curve.
    \coordinate (P1) at (0, 0);
    \coordinate (P2) at (3, 0);

    % Axes:
    \begin{scope}[thick]
        \draw[->] (-0.5, 0) to (4.4, 0) node[above] {$\Re\{z\}$};
        \draw[->] (0, -0.5) to (0, 4.4) node[right] {$\Im\{z\}$};
    \end{scope}

    % Axes Labels:
    \foreach\n in {1, 2, 3, 4}{%
        \draw (\n,3pt) to (\n,-3pt) node [below] {$\n$};
        \draw (3pt,\n) to (-3pt,\n) node [left]  {$\n{i}$};
    }

    % Draw the sector, shading the interior cyan.
    \draw[fill=cyan] (P1) to (P2) arc(0:45:3) to cycle;
    \draw[blue, ->-] (P1) to (P2) arc(0:45:3) to cycle;

    % Labels.
    \node at (4, 4)     {\Large{$\mathbb{C}$}};
    \node at (1, 1.7)   {\normalsize{$\Gamma(t)$}};
    \node at (1.8, 0.7) {\normalsize{$\textrm{Int}(\Gamma)$}};
\end{tikzpicture}

            \subcaption{An Example of a Sector.}
            \label{fig:Ex_of_Jordan_Curve_Sector}
        \end{subfigure}
        \caption{Jordan Curves in the Complex Plane}
        \label{fig:Ex_of_Jordan_Curve}
    \end{figure}
    For the sake of computation, we will stick to Jordan curves that are
    differentiable at all but finitely many points. A \textit{sector},
    which is the region contained within an arc of a circle, is an example
    of a Jordan curve that is differentiable at all but three points
    (Fig.~\subref{fig:Ex_of_Jordan_Curve_Sector}). We prove Green's
    Theorem for such curves, particularly curves that can be broken into a
    \textit{top} part and a \textit{bottom} part. While we wish to avoid
    presenting theorems without proof, some results are too difficult to
    include. We state the \textit{Jordan Curve Theorem}, but do not prove
    it. The proof can be found in a textbook on algebraic topology.
    \begin{theorem}
        If $\Gamma:\mathbb{R}\rightarrow\mathbb{R}^{2}$ is a Jordan curve,
        then $\Gamma$ separates the plane in to two disjoint parts: The
        interior, denoted $\interior[](\Gamma)$, and the exterior. The interior
        is bounded, the exterior is unbounded, and
        $\Gamma$ is their common boundary.
    \end{theorem}
    A quick look at Fig.~\subref{fig:Ex_of_Smooth_Jordan_Curve} can
    convince one of the validity of this statement. We use the fact that a
    Jordan curve has an interior to state Green's Theorem, which is useful
    for the evaluation of complex integrals. A student of electromagnetism
    will already understand the importance and usefulness of Green's
    Theorem. The Weak Green's Theorem applies to \textit{simple} regions.
    There are two types of simple regions: Horizontal and Vertical.
    \begin{definition}
        A vertically simple region is a subset $D$ of
        the plane $\mathbb{R}^{2}$ such that there are
        two functions
        $g_{1},g_{2}:[a,b]\rightarrow\mathbb{R}$ such that:
        \begin{equation}
            D=\{\;(x,\,y)\,:\,a\leq{x}\leq{b},\,
                              g_{1}(x)\leq{y}\leq{g}_{2}(x)\;\}
        \end{equation}
    \end{definition}
    \begin{definition}
        A horizontally simple region is a subset $D$
        of the plane
        $\mathbb{R}^{2}$ such that there are two functions
        $g_{1},g_{2}:[a,b]\rightarrow\mathbb{R}$ such that:
        \begin{equation}
            D=\{\;(x,\,y)\,:\,a\leq{y}\leq{b},\,
                              g_{1}(y)\leq{x}\leq{g}_{2}(y)\;\}
        \end{equation}
    \end{definition}
    A vertically simple region is a subset of the plane bounded by two
    vertical lines, whereas a horizontally simple region is
    bounded by two horizontal lines (Fig.~\ref{fig:Simply_Regions}).
    \begin{figure}[H]
        \centering
        \captionsetup{type=figure}
        \begin{subfigure}[b]{0.49\textwidth}
            \centering
            \captionsetup{type=figure}
            %--------------------------------Dependencies----------------------------------%
%   amssymb                                                                    %
%   tikz                                                                       %
%       arrows.meta                                                            %
%   Int DeclareMathOperator                                                    %
%       \DeclareMathOperator{\Int}{Int}                                        %
%-------------------------------Main Document----------------------------------%
\begin{tikzpicture}[%
    >=Latex,
    line width=0.2mm,
    line cap=round,
    scale=2.5
]

    % coordinates for the points that define the Jordan curve.
    \coordinate (P1) at (0.4, 0.4);
    \coordinate (P2) at (0.4, 1.4);
    \coordinate (P3) at (1, 1.1);
    \coordinate (P4) at (1.8, 1.1);
    \coordinate (P5) at (1.8, 0.3);
    \coordinate (P6) at (1, 0.5);

    % Axes:
    \begin{scope}[thick]
        \draw[->] (-0.2, 0) to (2, 0) node [above] {$\Re\{z\}$};
        \draw[->] (0, -0.2) to (0, 2) node [right] {$\Im\{z\}$};
    \end{scope}

    % Draw the Jordan curve and color the interior cyan.
    \draw[blue, fill=cyan, opacity=0.7] (P1) to (P2)
        to [out=30,in=150]  (P3)
        to [out=-30,in=110] (P4)
        to                  (P5)
        to [out=110,in=-40] (P6)
        to [out=140,in=30]  cycle;

    % Labels for the Jordan curve and its interior.
    \node at (1.2,0.8) {$\interior[](\Gamma)$};
    \node at (1.7,1.3) {$\Gamma$};
\end{tikzpicture}

            \subcaption{A Vertically Simple Region}
        \end{subfigure}
        \hfill
        \begin{subfigure}[b]{0.49\textwidth}
            \centering
            \captionsetup{type=figure}
            %--------------------------------Dependencies----------------------------------%
%   amssymb                                                                    %
%   tikz                                                                       %
%       arrows.meta                                                            %
%   Int DeclareMathOperator                                                    %
%       \DeclareMathOperator{\Int}{Int}                                        %
%-------------------------------Main Document----------------------------------%
\begin{tikzpicture}[%
    >=Latex,
    line width=0.2mm,
    line cap=round,
    scale=2.5
]

    % Coordinates for the points that define the frame of the figure.
    \coordinate (P1) at (0.3, 0.3);
    \coordinate (P2) at (1.8, 0.3);
    \coordinate (P3) at (1.9, 0.8);
    \coordinate (P4) at (1.8, 1.5);
    \coordinate (P5) at (1.8, 1.8);
    \coordinate (P6) at (0.5, 1.8);
    \coordinate (P7) at (0.7, 1.0);

    % Axes:
    \begin{scope}[thick]
        \draw[->] (-0.2, 0) to (2, 0) node [above] {$\Re\{z\}$};
        \draw[->] (0, -0.2) to (0, 2) node [right] {$\Im\{z\}$};
    \end{scope}

    % Draw the simple region.
    \draw[blue,fill=cyan,opacity=0.7] (P1) to (P2)
        to [out=30,in=-80]  (P3)
        to [out=100,in=-45] (P4)
        to [out=135,in=-30] (P5)
        to                  (P6)
        to [out=-80,in=90]  (P7)
        to [out=-90,in=90]  cycle;

    % Nodes to label the Jordan curve and its interior.
    \node at (0.5,1)   {$\Gamma$};
    \node at (1.2,1.2) {$\Int(\Gamma)$};
\end{tikzpicture}

            \subcaption{A Horizontally Simple Region}
        \end{subfigure}
        \caption{Examples of Simple Regions}
        \label{fig:Simply_Regions}
    \end{figure}
    We now prove special cases of Green's Theorem for vertically and
    horizontally simple regions, and then tie this together for the weak
    form of Green's Theorem.
    \begin{theorem}
        \label{thm:Greens_Theorem_Simple_t1_region}%
        If $M:\mathbb{R}^{2}\rightarrow\mathbb{R}$ is
        a differentiable function and if $\Gamma$ is a Jordan curve such
        that $\interior[](\Gamma)$ is vertically simple, then:
        \begin{equation}
            \iint_{\interior[](\Gamma)}\frac{\partial{M}}{\partial{y}}\diff{A}
            =-\oint_{\Gamma}M\diff{x}
        \end{equation}
    \end{theorem}
    \begin{proof}
        Since the interior of $\Gamma$ is a vertically simple region,
        there are two functions
        $g_{1}$, $g_{2}:[a,b]\rightarrow\mathbb{R}$ such that:
        \begin{equation}
            \Gamma=\{\;(x,y)\,:\,a\leq{x}\leq{b},\,
                                 g_{1}(x)\leq{y}\leq{g}_{2}(x)\;\}
        \end{equation}
        But then:
        \begin{subequations}
            \label{eqn:weak_type1_greens_theorem_eq1}%
            \begin{align}
                \iint_{\interior[](\Gamma)}
                    \frac{\partial{M}}{\partial{y}}\diff{A}
                &=\int_{a}^{b}\int_{g_{1}(x)}^{g_{2}(x)}
                    \frac{\partial{M}}{\partial{y}}
                    \diff{y}\diff{x}\\
                &=\int_{a}^{b}
                    \Big(M\big(x,g_{2}(x)\big)
                    -M\big(x,g_{1}(x)\big)\Big)\diff{x}\\
                &=-\int_{a}^{b}
                    \Big(M\big(x,g_{1}(x)\big)
                    -M\big(x,g_{2}(x)\big)\Big)\diff{x}
            \end{align}
        \end{subequations}
        But since $\interior[](\Gamma)$ is simple, the path at $x=a$ is either a
        point or a vertical straight line. But then the integral along
        this portion with respect to $x$ is zero. Similarly for
        $x=b$, and therefore:
        \begin{equation}
            \label{eqn:weak_type1_greens_theorem_eq2}%
            \oint_{\Gamma}M\diff{x}
            =\int_{a}^{b}\Big(M\big(x,g_{1}(x)\big)
                -M\big(x,g_{2}(x)\big)\Big)\diff{x}
        \end{equation}
        Comparing Eqn.~\ref{eqn:weak_type1_greens_theorem_eq1}
        and Eqn.~\ref{eqn:weak_type1_greens_theorem_eq2}
        completes the proof.
    \end{proof}
    \begin{theorem}
        \label{thm:Greens_Theorem_Simple_t2_region}%
        If $N:\mathbb{R}^{2}\rightarrow\mathbb{R}$ is a differentiable
        function and if $\Gamma$ is a Jordan curve such that
        $\interior[](\Gamma)$ is horizontally simple, then:
        \begin{equation}
            \iint_{\interior[](\Gamma)}\frac{\partial{M}}{\partial{x}}\diff{A}
            =\oint_{\Gamma}N\diff{y}
        \end{equation}
    \end{theorem}
    \begin{proof}
        The proof is a mimicry of the proof for
        Thm.~\ref{thm:Greens_Theorem_Simple_t1_region},
        but since the orientation of the path changes
        since we are now integrating with respect to $y$,
        we pick up a minus sign in the contour integral.
    \end{proof}
    \begin{theorem}[Weak Green's Theorem]
        If $M:\mathbb{R}^{2}\rightarrow\mathbb{R}$ and
        $N:\mathbb{R}^{2}\rightarrow\mathbb{R}$ are
        differentiable functions, and if
        $\Gamma$ is a Jordan curve such that the
        interior of $\Gamma$ is vertically and horizontally
        simple (A rectangular region), then:
        \begin{equation}
            \oint_{\Gamma}(M\diff{x}+N\diff{y})=
            \iint_{\interior[](\Gamma)}\Big(
            \frac{\partial{N}}{\partial{x}}-
            \frac{\partial{M}}{\partial{y}}\Big)\diff{A}
        \end{equation}
    \end{theorem}
    \begin{proof}
        Since $\interior[](\Gamma)$ is both vertically and
        horizontally simple, we may sum the results from
        Thm.~\ref{thm:Greens_Theorem_Simple_t1_region}
        and
        Thm.~\ref{thm:Greens_Theorem_Simple_t2_region},
        completing the proof.
    \end{proof}
    While this is not quite what we want, since most
    regions we wish to integrate over will not be simple,
    we can approximate the interior of a smooth Jordan
    curve arbitrarily well by a finite collection of
    simple regions. The full proof will get into the
    mechanics of this approximation, and show that in
    the \textit{limit} we obtain the result. We'll state
    Green's Theorem, but neglect the full proof.
    \begin{ltheorem}{Green's Theorem}{Greens_Theorem}
        If $M:\mathbb{R}^{2}\rightarrow\mathbb{R}$ and
        $N:\mathbb{R}^{2}\rightarrow\mathbb{R}$ are
        differentiable functions, and if
        $\Gamma$ is a Jordan curve that is differentiable at
        all but finitely many points, then:
        \begin{equation}
            \oint_{\Gamma}(M\diff{x}+N\diff{y})=
            \iint_{\interior[](\Gamma)}\Big(
            \frac{\partial{N}}{\partial{x}}-
            \frac{\partial{M}}{\partial{y}}\Big)\diff{A}
        \end{equation}
    \end{ltheorem}
    We return to complex analysis and prove one of the central results of
    the theory: Cauchy's Integral Theorem.
    \newpage
    \begin{ftheorem}{Cauchy's Integral Theorem}{Cauchy_Int_Theorem}
        If $f:\mathbb{C}\rightarrow\mathbb{C}$ is an entire function, and
        if $\Gamma:[0,1]\rightarrow\mathbb{C}$ is a Jordan curve
        differentiable at all but finitely many points, then:
        \begin{equation}
            \oint_{\Gamma}f(z)\diff{z}=0
        \end{equation}
    \end{ftheorem}
    \begin{bproof}
        For let $f(z)=u(x,\,y)+iv(x,\,y)$. Then:
        \begin{subequations}
            \begin{align}
                \oint_{\Gamma}f(z)\diff{z}
                &=\oint_{\Gamma}\big(u(x,\,y)+iv(x,\,y)\big)
                    \big(\diff{x}+i\diff{y}\big)\\
                \nonumber&=\oint_{\Gamma}
                    \big(u(x,\,y)\diff{x}-v(x,\,y)\diff{y}\big)\\
                &\hspace{2cm}+i\oint_{\Gamma}
                    \big(v(x,\,y)\diff{x}+u(x,\,y)\diff{y}\big)
            \end{align}
        \end{subequations}
        As $f$ is entire, $u$ and $v$ are differentiable.
        Applying Green's Theorem, we obtain:
        \begin{equation}
            \oint_{\Gamma}f(z)\diff{z}
            =\iint_{\interior[](\Gamma)}
            \Big(\frac{\partial{u}}{\partial{y}}+
                 \frac{\partial{v}}{\partial{x}}\Big)
                 \diff{A}+
            i\iint_{\interior[](\Gamma)}
            \Big(\frac{\partial{u}}{\partial{x}}-
                 \frac{\partial{v}}{\partial{y}}\Big)\diff{A}
        \end{equation}
        But since $f$ is entire, $u$ and $v$ satisfy
        the Cauchy-Riemann equations. That is:
        \par\hfill\par
        \begin{subequations}
            \begin{minipage}{0.49\textwidth}
                \centering
                \begin{equation}
                    \frac{\partial{u}}{\partial{x}}-
                    \frac{\partial{v}}{\partial{y}}=0
                \end{equation}
            \end{minipage}
            \hfill
            \begin{minipage}{0.49\textwidth}
                \centering
                \begin{equation}
                    \frac{\partial{u}}{\partial{y}}+
                    \frac{\partial{v}}{\partial{x}}=0
                \end{equation}
            \end{minipage}
        \end{subequations}
        \par\hfill\par
        Thus the integrands of both double integrals are zero,
        and hence the integrals are zero. This completes the proof.
    \end{bproof}
    Finally we prove Jordan's Lemma. This is used
    in conjunction with Cauchy's Integral Theorem to
    provide a powerful means of computing the integrals
    of difficult functions. In particular, this is used
    to evaluate the limits of the \textit{Fresnel Integrals}.
    \begin{theorem}[Jordan's Inequality]
        \label{thm:Jordan_Inequality}%
        If $x\in[0,\,\frac{\pi}{2}]$, then:
        \begin{equation}
            \frac{2}{\pi}x\leq\sin(x)
        \end{equation}
    \end{theorem}
    \begin{proof}
        For let $f:[0,\,\frac{\pi}{2}]\rightarrow\mathbb{R}$ be defined
        by $f(x)=\frac{2}{\pi}x$. Then $f(0)=\sin(0)$ and
        $f(\frac{\pi}{2})=\sin(\frac{\pi}{2})$. But since
        $\sin$ is concave down on the interval $[0,\,\frac{\pi}{2}]$,
        it is impossible for $\sin(x)<f(x)$ on the open interval
        $(0,\,\frac{\pi}{2})$, and therefore we have that
        $\sin(x)\geq{f}(x)$. Therefore, etc.
    \end{proof}
    This simple theorem is best understood by graphing the two functions.
    We can use this to prove Jordan's Lemma, and this will conclude our
    discussion of complex analysis.
    \begin{ltheorem}{Jordan's Lemma}{Jordans_Lemma}
        If $g:\mathbb{C}\rightarrow\mathbb{C}$ is a continuous function,
        if $\theta_{0}\in[0,\,\pi]$, if $R$ and $a$ are positive
        real numbers, and if $\gamma_{R}$ is the arc from
        $R$ to $R\exp(i\theta_{0})$, then:
        \begin{equation}
            \Big|\int_{\gamma_{R}}\exp(iaz)g(z)\diff{z}\Big|
            \leq\frac{\pi}{a}M_{R}
        \end{equation}
        Where $M_{R}=\max\,\{\;|g(z)|\,:\,z\in\gamma_{R}\;\}$.
    \end{ltheorem}
    \begin{proof}
        Applying the triangle inequality
        (Thm.~\ref{thm:Triangle_Inequality}) for integrals,
        Euler's Theorem (Thm.~\ref{thm:Euler_Expo_Formula}) and
        integrating in polar coordinates, we have:
        \begin{subequations}
            \begin{align}
                \Big|\int_{\gamma_{R}}\exp(iaz)g(z)\diff{z}\Big|
                &=\Big|\int_{\gamma_{R}}\exp(iaz)g(z)iR\exp(i\theta)
                    \diff{\theta}\Big|\\
                &\leq\int_{\gamma_{R}}\big|\exp(iaz)g(z)iR\exp(i\theta)
                    \big|\diff{\theta}\\
                &=R\int_{\gamma_{R}}\big|\exp(iaz)g(z)\big|\diff{\theta}\\
                &=R\int_{\gamma_{R}}\big|
                    \exp\big[iaR\big(\cos(\theta)+i\sin(\theta)\big)\big]
                    g(z)\big|\diff{\theta}\\
                &=R\int_{\gamma_{R}}\big|
                    \exp\big[aR\big(i\cos(\theta)-\sin(\theta)\big)\big]
                    g(z)\big|\diff{\theta}\\
                &=R\int_{\gamma_{R}}\big|
                    \exp\big(\!\minus\!aR\sin(\theta)\big)g(z)
                \big|\diff{\theta}\\
                &\leq{R}M_{R}\int_{\gamma_{R}}
                    \exp\big(\!\minus\!aR\sin(\theta)\big)\diff{\theta}
            \end{align}
        \end{subequations}
        Finally, applying Jordan's inequality
        (Thm.~\ref{thm:Jordan_Inequality}), we have:
        \begin{subequations}
            \begin{align}
                \Big|\int_{\gamma_{R}}\exp(iaz)g(z)\diff{z}\Big|
                &\leq{R}M_{R}\int_{\gamma_{R}}
                    \exp\Big(\!\minus\!\frac{2aR\theta}{\pi}\Big)
                \diff{\theta}\\
                &\leq\frac{\pi}{a}\big(1-\exp(\minus{a}R)\big)M_{R}\\
                &\leq\frac{\pi}{a}M_{R}
            \end{align}
        \end{subequations}
        Therefore, etc.
    \end{proof}
    \newpage
\section{Complex Variables}
    A complex function is a function whose argument is a complex
    variable $z=x+iy$, where $i$ is the imaginary unit. Complex
    functions can have the problem of being multi-valued, which
    is a cause for caution when dealing with them. For example,
    in the complex realm every non-zero complex number $z$
    has two square roots $\sqrt{z}$. So the square root
    function is multi-valued. Any complex function $f(z)$ can
    be written as $f(z)=u(x,y)+iv(x,y)$, where $u$ and $v$ are
    purely real functions. The function $w=f(z)$ can be seen
    as a mapping, or transformation, of the $z$ plane to
    the $w$ plane. That is, $f$ is a transformation of
    its domain onto its range, or image. A compound complex
    function is one of the form $F(z)=g(f(z))$. Since complex
    functions are functions of two variables, in a sense, one
    must be careful when considering limits of complex functions.
    \begin{example}
        What is the limit of $z/\overline{z}$ as $z\rightarrow{0}$?
        This is undefined. For:
        \begin{equation*}
            \frac{z}{\overline{z}}=\frac{x+iy}{x-iy}
        \end{equation*}
        Letting $x=0$ and taking the limit on $y$,
        we get:
        \begin{equation*}
            \frac{0+iy}{0-iy}=-1
        \end{equation*}
        Letting $y=0$ and taking the limit on $x$,
        we get:
        \begin{equation*}
            \frac{x+0i}{x-0i}=1
        \end{equation*}
        So the limit does not exist.
    \end{example}
    Continuity and the various properties of limits
    are defined similarly on $\mathbb{C}$ as for
    $\mathbb{R}$, with distance between points being
    defined by
    $d(z_{1},z_{2})=\sqrt{(x_{2}-x_{1})^{2}+(y_{2}-y_{1})^{2}}$.
    Differentiation is defined as:
    \begin{equation*}
        f'(z_{0})=\lim_{z\rightarrow{z_{0}}}\frac{f(z)-f(z_{0})}{z-z_{0}}
    \end{equation*}
    \begin{theorem}
        A complex function $f(z)=u(x,y)+iv(x,y)$ is
        differentiable if and only if it satisfies
        the Cauchy-Riemann equations:
        \begin{align*}
            \frac{\partial{u}}{\partial{x}}
            &=\frac{\partial{v}}{\partial{y}}
            &
            \frac{\partial{u}}{\partial{y}}
            &=-\frac{\partial{v}}{\partial{x}}
        \end{align*}
    \end{theorem}
    \begin{theorem}
        If $f(z)=u(x,y)+iv(x,y)$ is differentiable,
        then:
        \begin{equation*}
            f'(z)=u_{x}(x,y)+iv_{y}(x,y)
        \end{equation*}
    \end{theorem}
    \begin{definition}
        A complex function $f(z)$ is analytic,
        or holomorphic, at a point $z_{0}$ if
        it is differentiable in some neighborhood of
        $z_{0}$.
    \end{definition}
    \begin{definition}
        An entire function is a complex function
        $f(z)$ such that $f$ is analytic at every
        point $z\in\mathbb{C}$.
    \end{definition}
    \begin{definition}
        A harmonic function is a function
        $A(x,y)$ such that all of its second
        partial derivatives exists, and it
        satisfies the Laplace Equation:
        \begin{equation*}
            \nabla^{2}A
            =A_{xx}(x,y)+A_{yy}(x,y)
            =0
        \end{equation*}
    \end{definition}
    \begin{theorem}
        If $f(z)=u(x,y)+iv(x,y)$ is differentiable
        on a domain $D$, then $u$ and $v$ are
        harmonic on the domain.
    \end{theorem}
    \begin{theorem}
        A function $f(z)$ is analytic if and only if
        its real and complex parts are harmonic
        conjugates of each other.
    \end{theorem}
    \begin{definition}
        A level curve of a function $f(x,y)$ is
        a curve in $\mathbb{R}^{2}$ such that
        $f$ is constant on that curve.
    \end{definition}
    One of the most basic and fundamental results from
    complex variables is Euler's Formula:
    \begin{equation*}
        \exp(i\theta)=\cos(\theta)+i\sin(\theta)
    \end{equation*}
    \renewcommand{\PATH}{\OLDPATH}
\endgroup
    %        \documentclass[crop=false,class=book,oneside]{standalone}
%----------------------------Preamble-------------------------------%
%---------------------------Packages----------------------------%
\usepackage{geometry}
\geometry{b5paper, margin=1.0in}
\usepackage[T1]{fontenc}
\usepackage{graphicx, float}            % Graphics/Images.
\usepackage{natbib}                     % For bibliographies.
\bibliographystyle{agsm}                % Bibliography style.
\usepackage[french, english]{babel}     % Language typesetting.
\usepackage[dvipsnames]{xcolor}         % Color names.
\usepackage{listings}                   % Verbatim-Like Tools.
\usepackage{mathtools, esint, mathrsfs} % amsmath and integrals.
\usepackage{amsthm, amsfonts, amssymb}  % Fonts and theorems.
\usepackage{tcolorbox}                  % Frames around theorems.
\usepackage{upgreek}                    % Non-Italic Greek.
\usepackage{fmtcount, etoolbox}         % For the \book{} command.
\usepackage[newparttoc]{titlesec}       % Formatting chapter, etc.
\usepackage{titletoc}                   % Allows \book in toc.
\usepackage[nottoc]{tocbibind}          % Bibliography in toc.
\usepackage[titles]{tocloft}            % ToC formatting.
\usepackage{pgfplots, tikz}             % Drawing/graphing tools.
\usepackage{imakeidx}                   % Used for index.
\usetikzlibrary{
    calc,                   % Calculating right angles and more.
    angles,                 % Drawing angles within triangles.
    arrows.meta,            % Latex and Stealth arrows.
    quotes,                 % Adding labels to angles.
    positioning,            % Relative positioning of nodes.
    decorations.markings,   % Adding arrows in the middle of a line.
    patterns,
    arrows
}                                       % Libraries for tikz.
\pgfplotsset{compat=1.9}                % Version of pgfplots.
\usepackage[font=scriptsize,
            labelformat=simple,
            labelsep=colon]{subcaption} % Subfigure captions.
\usepackage[font={scriptsize},
            hypcap=true,
            labelsep=colon]{caption}    % Figure captions.
\usepackage[pdftex,
            pdfauthor={Ryan Maguire},
            pdftitle={Mathematics and Physics},
            pdfsubject={Mathematics, Physics, Science},
            pdfkeywords={Mathematics, Physics, Computer Science, Biology},
            pdfproducer={LaTeX},
            pdfcreator={pdflatex}]{hyperref}
\hypersetup{
    colorlinks=true,
    linkcolor=blue,
    filecolor=magenta,
    urlcolor=Cerulean,
    citecolor=SkyBlue
}                           % Colors for hyperref.
\usepackage[toc,acronym,nogroupskip,nopostdot]{glossaries}
\usepackage{glossary-mcols}
%------------------------Theorem Styles-------------------------%
\theoremstyle{plain}
\newtheorem{theorem}{Theorem}[section]

% Define theorem style for default spacing and normal font.
\newtheoremstyle{normal}
    {\topsep}               % Amount of space above the theorem.
    {\topsep}               % Amount of space below the theorem.
    {}                      % Font used for body of theorem.
    {}                      % Measure of space to indent.
    {\bfseries}             % Font of the header of the theorem.
    {}                      % Punctuation between head and body.
    {.5em}                  % Space after theorem head.
    {}

% Italic header environment.
\newtheoremstyle{thmit}{\topsep}{\topsep}{}{}{\itshape}{}{0.5em}{}

% Define environments with italic headers.
\theoremstyle{thmit}
\newtheorem*{solution}{Solution}

% Define default environments.
\theoremstyle{normal}
\newtheorem{example}{Example}[section]
\newtheorem{definition}{Definition}[section]
\newtheorem{problem}{Problem}[section]

% Define framed environment.
\tcbuselibrary{most}
\newtcbtheorem[use counter*=theorem]{ftheorem}{Theorem}{%
    before=\par\vspace{2ex},
    boxsep=0.5\topsep,
    after=\par\vspace{2ex},
    colback=green!5,
    colframe=green!35!black,
    fonttitle=\bfseries\upshape%
}{thm}

\newtcbtheorem[auto counter, number within=section]{faxiom}{Axiom}{%
    before=\par\vspace{2ex},
    boxsep=0.5\topsep,
    after=\par\vspace{2ex},
    colback=Apricot!5,
    colframe=Apricot!35!black,
    fonttitle=\bfseries\upshape%
}{ax}

\newtcbtheorem[use counter*=definition]{fdefinition}{Definition}{%
    before=\par\vspace{2ex},
    boxsep=0.5\topsep,
    after=\par\vspace{2ex},
    colback=blue!5!white,
    colframe=blue!75!black,
    fonttitle=\bfseries\upshape%
}{def}

\newtcbtheorem[use counter*=example]{fexample}{Example}{%
    before=\par\vspace{2ex},
    boxsep=0.5\topsep,
    after=\par\vspace{2ex},
    colback=red!5!white,
    colframe=red!75!black,
    fonttitle=\bfseries\upshape%
}{ex}

\newtcbtheorem[auto counter, number within=section]{fnotation}{Notation}{%
    before=\par\vspace{2ex},
    boxsep=0.5\topsep,
    after=\par\vspace{2ex},
    colback=SeaGreen!5!white,
    colframe=SeaGreen!75!black,
    fonttitle=\bfseries\upshape%
}{not}

\newtcbtheorem[use counter*=remark]{fremark}{Remark}{%
    fonttitle=\bfseries\upshape,
    colback=Goldenrod!5!white,
    colframe=Goldenrod!75!black}{ex}

\newenvironment{bproof}{\textit{Proof.}}{\hfill$\square$}
\tcolorboxenvironment{bproof}{%
    blanker,
    breakable,
    left=3mm,
    before skip=5pt,
    after skip=10pt,
    borderline west={0.6mm}{0pt}{green!80!black}
}

\AtEndEnvironment{lexample}{$\hfill\textcolor{red}{\blacksquare}$}
\newtcbtheorem[use counter*=example]{lexample}{Example}{%
    empty,
    title={Example~\theexample},
    boxed title style={%
        empty,
        size=minimal,
        toprule=2pt,
        top=0.5\topsep,
    },
    coltitle=red,
    fonttitle=\bfseries,
    parbox=false,
    boxsep=0pt,
    before=\par\vspace{2ex},
    left=0pt,
    right=0pt,
    top=3ex,
    bottom=1ex,
    before=\par\vspace{2ex},
    after=\par\vspace{2ex},
    breakable,
    pad at break*=0mm,
    vfill before first,
    overlay unbroken={%
        \draw[red, line width=2pt]
            ([yshift=-1.2ex]title.south-|frame.west) to
            ([yshift=-1.2ex]title.south-|frame.east);
        },
    overlay first={%
        \draw[red, line width=2pt]
            ([yshift=-1.2ex]title.south-|frame.west) to
            ([yshift=-1.2ex]title.south-|frame.east);
    },
}{ex}

\AtEndEnvironment{ldefinition}{$\hfill\textcolor{Blue}{\blacksquare}$}
\newtcbtheorem[use counter*=definition]{ldefinition}{Definition}{%
    empty,
    title={Definition~\thedefinition:~{#1}},
    boxed title style={%
        empty,
        size=minimal,
        toprule=2pt,
        top=0.5\topsep,
    },
    coltitle=Blue,
    fonttitle=\bfseries,
    parbox=false,
    boxsep=0pt,
    before=\par\vspace{2ex},
    left=0pt,
    right=0pt,
    top=3ex,
    bottom=0pt,
    before=\par\vspace{2ex},
    after=\par\vspace{1ex},
    breakable,
    pad at break*=0mm,
    vfill before first,
    overlay unbroken={%
        \draw[Blue, line width=2pt]
            ([yshift=-1.2ex]title.south-|frame.west) to
            ([yshift=-1.2ex]title.south-|frame.east);
        },
    overlay first={%
        \draw[Blue, line width=2pt]
            ([yshift=-1.2ex]title.south-|frame.west) to
            ([yshift=-1.2ex]title.south-|frame.east);
    },
}{def}

\AtEndEnvironment{ltheorem}{$\hfill\textcolor{Green}{\blacksquare}$}
\newtcbtheorem[use counter*=theorem]{ltheorem}{Theorem}{%
    empty,
    title={Theorem~\thetheorem:~{#1}},
    boxed title style={%
        empty,
        size=minimal,
        toprule=2pt,
        top=0.5\topsep,
    },
    coltitle=Green,
    fonttitle=\bfseries,
    parbox=false,
    boxsep=0pt,
    before=\par\vspace{2ex},
    left=0pt,
    right=0pt,
    top=3ex,
    bottom=-1.5ex,
    breakable,
    pad at break*=0mm,
    vfill before first,
    overlay unbroken={%
        \draw[Green, line width=2pt]
            ([yshift=-1.2ex]title.south-|frame.west) to
            ([yshift=-1.2ex]title.south-|frame.east);},
    overlay first={%
        \draw[Green, line width=2pt]
            ([yshift=-1.2ex]title.south-|frame.west) to
            ([yshift=-1.2ex]title.south-|frame.east);
    }
}{thm}

%--------------------Declared Math Operators--------------------%
\DeclareMathOperator{\adjoint}{adj}         % Adjoint.
\DeclareMathOperator{\Card}{Card}           % Cardinality.
\DeclareMathOperator{\curl}{curl}           % Curl.
\DeclareMathOperator{\diam}{diam}           % Diameter.
\DeclareMathOperator{\dist}{dist}           % Distance.
\DeclareMathOperator{\Div}{div}             % Divergence.
\DeclareMathOperator{\Erf}{Erf}             % Error Function.
\DeclareMathOperator{\Erfc}{Erfc}           % Complementary Error Function.
\DeclareMathOperator{\Ext}{Ext}             % Exterior.
\DeclareMathOperator{\GCD}{GCD}             % Greatest common denominator.
\DeclareMathOperator{\grad}{grad}           % Gradient
\DeclareMathOperator{\Ima}{Im}              % Image.
\DeclareMathOperator{\Int}{Int}             % Interior.
\DeclareMathOperator{\LC}{LC}               % Leading coefficient.
\DeclareMathOperator{\LCM}{LCM}             % Least common multiple.
\DeclareMathOperator{\LM}{LM}               % Leading monomial.
\DeclareMathOperator{\LT}{LT}               % Leading term.
\DeclareMathOperator{\Mod}{mod}             % Modulus.
\DeclareMathOperator{\Mon}{Mon}             % Monomial.
\DeclareMathOperator{\multideg}{mutlideg}   % Multi-Degree (Graphs).
\DeclareMathOperator{\nul}{nul}             % Null space of operator.
\DeclareMathOperator{\Ord}{Ord}             % Ordinal of ordered set.
\DeclareMathOperator{\Prin}{Prin}           % Principal value.
\DeclareMathOperator{\proj}{proj}           % Projection.
\DeclareMathOperator{\Refl}{Refl}           % Reflection operator.
\DeclareMathOperator{\rk}{rk}               % Rank of operator.
\DeclareMathOperator{\sgn}{sgn}             % Sign of a number.
\DeclareMathOperator{\sinc}{sinc}           % Sinc function.
\DeclareMathOperator{\Span}{Span}           % Span of a set.
\DeclareMathOperator{\Spec}{Spec}           % Spectrum.
\DeclareMathOperator{\supp}{supp}           % Support
\DeclareMathOperator{\Tr}{Tr}               % Trace of matrix.
%--------------------Declared Math Symbols--------------------%
\DeclareMathSymbol{\minus}{\mathbin}{AMSa}{"39} % Unary minus sign.
%------------------------New Commands---------------------------%
\DeclarePairedDelimiter\norm{\lVert}{\rVert}
\DeclarePairedDelimiter\ceil{\lceil}{\rceil}
\DeclarePairedDelimiter\floor{\lfloor}{\rfloor}
\newcommand*\diff{\mathop{}\!\mathrm{d}}
\newcommand*\Diff[1]{\mathop{}\!\mathrm{d^#1}}
\renewcommand*{\glstextformat}[1]{\textcolor{RoyalBlue}{#1}}
\renewcommand{\glsnamefont}[1]{\textbf{#1}}
\renewcommand\labelitemii{$\circ$}
\renewcommand\thesubfigure{%
    \arabic{chapter}.\arabic{figure}.\arabic{subfigure}}
\addto\captionsenglish{\renewcommand{\figurename}{Fig.}}
\numberwithin{equation}{section}

\renewcommand{\vector}[1]{\boldsymbol{\mathrm{#1}}}

\newcommand{\uvector}[1]{\boldsymbol{\hat{\mathrm{#1}}}}
\newcommand{\topspace}[2][]{(#2,\tau_{#1})}
\newcommand{\measurespace}[2][]{(#2,\varSigma_{#1},\mu_{#1})}
\newcommand{\measurablespace}[2][]{(#2,\varSigma_{#1})}
\newcommand{\manifold}[2][]{(#2,\tau_{#1},\mathcal{A}_{#1})}
\newcommand{\tanspace}[2]{T_{#1}{#2}}
\newcommand{\cotanspace}[2]{T_{#1}^{*}{#2}}
\newcommand{\Ckspace}[3][\mathbb{R}]{C^{#2}(#3,#1)}
\newcommand{\funcspace}[2][\mathbb{R}]{\mathcal{F}(#2,#1)}
\newcommand{\smoothvecf}[1]{\mathfrak{X}(#1)}
\newcommand{\smoothonef}[1]{\mathfrak{X}^{*}(#1)}
\newcommand{\bracket}[2]{[#1,#2]}

%------------------------Book Command---------------------------%
\makeatletter
\renewcommand\@pnumwidth{1cm}
\newcounter{book}
\renewcommand\thebook{\@Roman\c@book}
\newcommand\book{%
    \if@openright
        \cleardoublepage
    \else
        \clearpage
    \fi
    \thispagestyle{plain}%
    \if@twocolumn
        \onecolumn
        \@tempswatrue
    \else
        \@tempswafalse
    \fi
    \null\vfil
    \secdef\@book\@sbook
}
\def\@book[#1]#2{%
    \refstepcounter{book}
    \addcontentsline{toc}{book}{\bookname\ \thebook:\hspace{1em}#1}
    \markboth{}{}
    {\centering
     \interlinepenalty\@M
     \normalfont
     \huge\bfseries\bookname\nobreakspace\thebook
     \par
     \vskip 20\p@
     \Huge\bfseries#2\par}%
    \@endbook}
\def\@sbook#1{%
    {\centering
     \interlinepenalty \@M
     \normalfont
     \Huge\bfseries#1\par}%
    \@endbook}
\def\@endbook{
    \vfil\newpage
        \if@twoside
            \if@openright
                \null
                \thispagestyle{empty}%
                \newpage
            \fi
        \fi
        \if@tempswa
            \twocolumn
        \fi
}
\newcommand*\l@book[2]{%
    \ifnum\c@tocdepth >-3\relax
        \addpenalty{-\@highpenalty}%
        \addvspace{2.25em\@plus\p@}%
        \setlength\@tempdima{3em}%
        \begingroup
            \parindent\z@\rightskip\@pnumwidth
            \parfillskip -\@pnumwidth
            {
                \leavevmode
                \Large\bfseries#1\hfill\hb@xt@\@pnumwidth{\hss#2}
            }
            \par
            \nobreak
            \global\@nobreaktrue
            \everypar{\global\@nobreakfalse\everypar{}}%
        \endgroup
    \fi}
\newcommand\bookname{Book}
\renewcommand{\thebook}{\texorpdfstring{\Numberstring{book}}{book}}
\providecommand*{\toclevel@book}{-2}
\makeatother
\titleformat{\part}[display]
    {\Large\bfseries}
    {\partname\nobreakspace\thepart}
    {0mm}
    {\Huge\bfseries}
\titlecontents{part}[0pt]
    {\large\bfseries}
    {\partname\ \thecontentslabel: \quad}
    {}
    {\hfill\contentspage}
\titlecontents{chapter}[0pt]
    {\bfseries}
    {\chaptername\ \thecontentslabel:\quad}
    {}
    {\hfill\contentspage}
\newglossarystyle{longpara}{%
    \setglossarystyle{long}%
    \renewenvironment{theglossary}{%
        \begin{longtable}[l]{{p{0.25\hsize}p{0.65\hsize}}}
    }{\end{longtable}}%
    \renewcommand{\glossentry}[2]{%
        \glstarget{##1}{\glossentryname{##1}}%
        &\glossentrydesc{##1}{~##2.}
        \tabularnewline%
        \tabularnewline
    }%
}
\newglossary[not-glg]{notation}{not-gls}{not-glo}{Notation}
\newcommand*{\newnotation}[4][]{%
    \newglossaryentry{#2}{type=notation, name={\textbf{#3}, },
                          text={#4}, description={#4},#1}%
}
%--------------------------LENGTHS------------------------------%
% Spacings for the Table of Contents.
\addtolength{\cftsecnumwidth}{1ex}
\addtolength{\cftsubsecindent}{1ex}
\addtolength{\cftsubsecnumwidth}{1ex}
\addtolength{\cftfignumwidth}{1ex}
\addtolength{\cfttabnumwidth}{1ex}

% Indent and paragraph spacing.
\setlength{\parindent}{0em}
\setlength{\parskip}{0em}
\graphicspath{{../../../images/}}   % Path to Image Folder.
%--------------------------Main Document----------------------------%
\begin{document}
    \ifx\ifmathcourses\undefined
        \pagenumbering{roman}
        \title{Functional Analysis}
        \author{Ryan Maguire}
        \date{\vspace{-5ex}}
        \maketitle
        \tableofcontents
        \clearpage
        \chapter*{Functional Analyis}
        \addcontentsline{toc}{chapter}{Functional Analysis}
        \markboth{}{FUNCTIONAL ANALYSIS}
        \setcounter{chapter}{1}
        \pagenumbering{arabic}
    \else
        \chapter{Functional Analysis}
    \fi
    \section{Old Notes}
        \subsection{Summary of Lectures}
            \subsubsection{Metric Spaces}
                There are many kinds of metric spaces with
                various metrics. Popular ones are $p$ adic.
                Other examples include $s,c,c_{0},B(A)$, and more.
            \subsubsection{Baire's Catagory Theorem}
                The interior of a set $X$ is the set of a points
                in $X$ that can be enclosed by an open ball that
                is entirely contained in $X$. That is, the interior
                of $X$ is the ``largest,'' open subset of $X$.
                The interior of an open set is itself. A set is
                nowhere dense if its interior is empty. For example, the boundary of a circle
                in $\mathbb{R}^{2}$ is nowhere dense, with respect to
                the metric on $\mathbb{R}^{2}$. Any open ball about any point
                on the circle contains points not on the circle, and thus
                it has empty interior. A set is meager, or of first
                category, if it is the countable union of nowhere dense
                sets. A set is of second category, or non-meager, if
                it is not a meager set. Baire's Category Theorem says that
                a complete metric space is of second category. It also claims
                that given a countable collection of open and dense subsets
                of a complete metric space, the intersection of the collection
                is also dense (Though it may not be open). Incomplete
                metric spaces can be ``completed,'' by adding in enough
                points to make all Cauchy sequences converge. The
                completion of a metric space is unique up to
                \textit{isometry}. The Hilbert cube is an example of
                an infinite dimensional space that is compact.
                It is the set of all sequences
                $x_{n}:\mathbb{N}\rightarrow\mathbb{R}$ such that the
                $n^{th}$ entry, $x_{n}$ is less than $1/n$. That is,
                for all $n\in\mathbb{N}$, $x_{n}<1/n$.
                Something about $\varepsilon$ nets.
            \subsubsection{Normed Spaces and Banach Spaces}
                A vector space $V$ over a field $K$ is a set that has
                a notion of \textit{vector addition} and
                \textit{scalar multiplication}, that behaves all of the
                usual rules one finds in a linear algebra course. There are
                many examples, such as $\mathbb{R}^{n}$ and $C[a,b]$.
                There are also the familiar notions of subspace,
                linear combination, indepence, spanning, dimension, and
                basis. There's also the notion of a Hamel basis. A norm on
                a vector space is a function $\norm{}:V\rightarrow\mathbb{R}$
                that obeys the triangle inequality, positive definiteness,
                and homogeneity. There's also the notion of
                \textit{convexity}. Open and closed balls are convex.
                The induced metric on a normed space is the metric
                $d(x,y)=\norm{x-y}$. A Banach Space is a normed space
                that is complete with respect to the induced metric.
                A subspace of a Banach space is complete if and only if
                it is closed. There are things like series and Schauder basis.
                A Schauder basis implies the space is separable. If
                $\{x_{1},\hdots,x_{n}\}$ is independent, then there exists
                a $c>0$ such that, for all $\boldsymbol{\alpha}$,
                $|\boldsymbol{\alpha}\cdot\mathbf{x}|%
                 \geq{c}\norm{\boldsymbol{\alpha}}$.
                Finite dimensional subspaces are complete, as are closed
                subspaces. In finite dimensional normed spaces,
                a space is compact if and only if it is closed and bounded.
                Riesz's Lemma says that if $Z$ is a subspace of a normed
                space $X$, and if $Y$ is a proper closed subspace of
                $Z$, then there is a $z\in{Z}$ such that
                $\norm{z}=1$ and $D(z,Y)\geq{1/2}$. A corollary of this is
                that $B_{1}(0)$ is compact if and only if
                $X$ is finite dimensional.
            \subsubsection{Linear Operators}
                A linear operator is a function
                $T:X\rightarrow{Y}$ such that, for all
                $x,y\in{X}$, $\alpha,\beta\in\mathbb{R}$,
                $T(\alpha{x}+\beta{y})=\alpha{Tx}+\beta{Ty}$.
                Stuff about identity, zero, differentiation, and
                integration. 
                Domain of a linear operator, range of a linear operator,
                and the null space.
                Inverse of a linear operator is linear.
                $(ST)^{-1}=T^{-1}S^{-1}$.
                A bounded operator is a function $T:X\rightarrow{Y}$
                such that there is a $K\in\mathbb{R}$ such that, for all
                $x\in{X}$, $\norm{Tx}\leq{K}\norm{x}$. The norm of an operator
                is defined as
                $\norm{T}=\sup\{\norm{Tx}/\norm{x}:x\in{X},x\ne{0}\}$.
                This is equivalent to
                $\norm{T}=\sup\{\norm{Tx}:\norm{x}=1\}$. In finite dimension
                all linear operators are continuous. An operator is bounded
                if and only if it is continuous. If a linear operator is
                continuous at some point, then it is continuous everywhere.
                An operator is bounded if and only if its null space is closed.
                There is something called the extension of a bounded linear
                operator. $B(X,Y)$ is the set of bounder linear operators
                from $X$ to $Y$. This is complete if and only if
                $Y$ is complete. A functional is a mapping from a vector
                space $X$ into the real numbers $\mathbb{R}$. For continuous
                linear functional, continuity at $0$ implies continuity
                everywhere. There is something called the dual space
                $X'$, which is itself a Banach space. The dual
                of $\mathbb{R}$ is $\mathbb{R}$, and the dual of
                $\mathbb{R}^{n}$ is $\mathbb{R}^{n}$.
            \subsubsection{Inner Product and Hilbert Spaces}
                If $x_{n}\rightarrow{x}$ and $y_{n}\rightarrow{y}$,
                then
                $\langle{x_{n},y_{n}}\rangle\rightarrow\langle{x,y}\rangle$.
                There's a notion of orthogonal sets,
                and orthonormality. If $(e_{n})$ is orthonormal basis,
                then $x=\sum\langle{x,e_{k}}\rangle{e_{k}}$
                for all $x$.
                Bessel's inequality is a thing. So is the Gram-Schmidt
                procedure. $\sum\alpha_{k}e_{k}$
                converges if and only if
                $\sum|\alpha_{k}|^{2}$ converges. A set $M$ is total in
                a Hilbert space $H$ is the span of the closure
                of $M$ is equal
                to $H$. If $M$ is complete, then it is 
                total if and only if $M^{\perp}=0$. Parseval's theorem.
                Legendre, Hermite, and Laguerre polynomials are things.
                Self adjoint, unitary, and normal operators.
                $T^{*}=T$, $T^{*}=T^{-1}$, and $T^{*}T=TT^{*}$. If
                $X$ is a vector space over the complex numbers,
                and if $T$ is self adjoint, then
                $\langle{Tx,x}\rangle$ is a real number for all $x$.
            \subsubsection{Compact Linear Operators}
                If $T$ is compact and linear, then it is bounded and
                continuous. An operator is compact and linear if and only if
                for all bounded sequences $x_{n}$,
                $Tx_{n}$ has a convergent subsequence. Compact linear
                operators form a vector space. The rank of an
                operator is
                the dimension of its image. If $T$ is linear,
                bounded, and of finite rank, then it is compact.
                If $T_{n}$ is a sequence of compact linear
                operators, if $Y$ is
                complete, and if $\norm{T_{n}-T}\rightarrow{0}$, then
                $T$ is compact. A sequence $x_{n}$ converges weakly to
                $x$ if, for all $y$,
                $\langle{x_{n},y}\rangle\rightarrow\langle{x,y}\rangle$.
                If $x_{n}$ converges weakly to $x$, then
                and if $T$ is a compact linear operator, then
                $Tx_{n}\rightarrow{Tx}$. If $H$ is a Hilbert space,
                $T$ is a compact self-adjoint operator, and if
                $x_{n}$ converges weakly to $x$, then
                $\langle{Tx_{n},x_{n}}\rangle\rightarrow\langle{Tx,x}\rangle$.
                If $T:H\rightarrow{H}$ is compact and linear, then so
                is its adjoint. The Hilbert-Schmidt theorem says that
                compact self-adjoint operators on a Hilbert space $H$
                have an orthonormal basis of eigenvectors. All of this
                has applications to integral operators and
                Sturm-Liouville Theory.
            \subsubsection{Fundamental Theorems}
                Zorn's Lemma. Hahn-Banach Theorem. Sublinear functionals.
                If $X$ is a normed space, and $Z$ is a subspace, and if
                $f\in{Z'}$, then $f$ be extended to $X$ such that
                $\norm{f}_{X}=\norm{f}_{Z}$. This is immediately extended to
                Hilbert spaces by Riesz's theorem. If $X$ is a normed space
                and $x\ne{0}$, then there is an $f\in{X'}$ such that
                $\norm{f}=1$ and $f(x_{0})=\norm{x_{0}}$. For all $x$,
                $\norm{x}=\sup\{\norm{f(x)}/\norm{f}:f\in{X'},f\ne{0}\}$.
                There's a thing called bounded variation. If $x\in{X}$ and
                $g_{x}(f)=f(x)$ for $f\in{X'}$, then
                $g_{x}\in{X''}$ and $\norm{g_{x}}=\norm{x}$.
                The canonical map $g:X\rightarrow{X''}$,
                $C(x)=g_{x}$ is an isomorphism.
                A reflexive space is one such that
                $\mathscr{R}(X)=X''$.
                Reflexive implies complete. Finite and Hilbert implies
                reflexive.
                $X'$ separable implies $X$ is separable.
                $X$ separable and reflexive implies $X'$ is separable.
                There exist continuous functions whose Fourier series
                diverges at a point. $x_{n}\rightarrow{x}$ strongly if
                $\norm{x_{n}-x}\rightarrow{0}$. If $x_{n}\rightarrow{x}$
                strongly, then it converges weakly as well. The converse is
                not true. If $X$ is finite dimensional, then weak convergence
                implies strong convergence. Weak convergence implies
                $\norm{x_{n}}$ is bounded. If
                $x_{n}\rightarrow{x}$ weakly, and if
                $\norm{x_{n}}\rightarrow\norm{x}$, then
                $x_{n}\rightarrow{x}$ strongly. Open mapping theorem is
                a thing. As is the closed graph theorem. Differentiation
                is a closed operator on $C^{1}[a,b]\rightarrow{C[a,b]}$.
    \section{A Review of Real Analysis}
        \subsection{Sets, Functions, and Countability}
            There are many different sets that will be used
            in functional analysis, so let's begin with their
            notations:
            \begin{table}[H]
                \centering
                \begin{tabular}{|l|l|l|}
                    \hline
                    Notation:&Description:&Importance:\\
                    \hline
                    $\mathbb{N}$&The Natural Numbers
                    &Used a lot.\\
                    \hline
                    $\mathbb{Z}$&The integers.&Never used.\\
                    \hline
                    $\mathbb{Z}_{n}$&Integers from $1$ to $n$.
                    &Mentioned rarely.\\
                    \hline
                    $\mathbb{Q}$&The Rational Numbers&
                    Good for examples and counterexamples.\\
                    \hline
                    $\mathbb{R}$&The Real Numbers.&
                    The set we'll be primarily concerned with.\\
                    \hline
                    $\mathbb{C}$&The Complex Numbers&
                    Used rarely.\\
                    \hline
                \end{tabular}
            \end{table}
            From set theory, a function $f$ from a set $X$ to a
            set $Y$ is a subset of $X\times{Y}$ such that, for
            all $x\in{X}$, there is a unique $y\in{Y}$
            such that $(x,y)\in{f}$. We often called $X$ the
            domain, $Y$ the range or co-domain, and write
            $f:X\rightarrow{Y}$ to indicate this. The
            \textit{image} of $x\in{X}$ is often written
            as $y=f(x)$. Requiring that $y$ be unique for each
            $x$ is equivalent to the \textit{vertical line test}
            one might find in a calculus course.
            \begin{definition}
                The image of a subset $S\subset{X}$
                by a function $f:X\rightarrow{Y}$
                is the set:
                \begin{equation*}
                    f(S)=\{f(x):x\in{X}\}
                \end{equation*}
            \end{definition}
            That is, the image of a subset $S\subset{X}$
            is the set of all points in $Y$ that $S$ gets
            mapped to by $f$.
            \begin{definition}
                The pre-image of a subset
                $S\subset{Y}$ by a function
                $f:X\rightarrow{Y}$ is the set:
                \begin{equation*}
                    f^{-1}(S)
                    =\{x\in{X}:f(x)=y\}
                \end{equation*}
            \end{definition}
            There are special kinds of functions: Injective,
            surjective, and bijective. An injective
            function is a function $f:X\rightarrow{Y}$ such
            that, for all $x_{1},x_{2}\in{X}$,
            $f(x_{1})=f(x_{2})$ if and only if $x_{1}=x_{2}$.
            That is, the \textit{pre-image} $f^{-1}(\{y\})$
            is unique (Or empty). A
            surjective function is a function $f:X\rightarrow{Y}$
            such that $f(X)=Y$. That is, every point $y\in{Y}$
            gets mapped to by at least one point in $X$. Or, in
            more familiar notation, for all $y\in{Y}$ there is
            and $x\in{X}$ such that $y=f(x)$. A bijective
            function is a function that is both injective and
            surjective. Sets $X$ and $Y$ such that there
            exists a bijective function $f:X\rightarrow{Y}$ are
            called \textit{equivalent}. Such sets can be said
            to have the same size. We say that $X$ is strictly
            smaller than $Y$ if there is an injective function
            $f:X\rightarrow{Y}$, but no bijective function.
            \begin{definition}
                A finite set is a set $X$ such that there is
                an $n\in\mathbb{N}$ and a bijection
                $f:\mathbb{Z}_{n}\rightarrow{X}$.
            \end{definition}
            \begin{definition}
                A countable set is a set
                $X$ such that there is a bijection
                $f:\mathbb{N}\rightarrow{X}$.
            \end{definition}
            Being countable means you can write
            the elements out in a list, or a
            one-to-one correspondence with all of
            the positive integers. Many sets are countable,
            including the whole numbers, integers, rational
            numbers, and \textit{algebraic} numbers. The
            union of finitely many countable sets is also
            countable, as is the union of countably many
            countable sets.
            \begin{theorem}
                If $A$ is a countable set such that for all
                $\mathcal{U}\in{A}$, $\mathcal{U}$ is a
                countable set, and if for all $a,b\in{A}$,
                $a\cap{b}=\emptyset$, then
                $\bigcup_{\mathcal{U}\in{A}}\mathcal{U}$
                is countable set.
            \end{theorem}
            \begin{definition}
                An uncountable set is a set $X$ that is neither
                finite nor countable.
            \end{definition}
            \begin{theorem}
                $\mathbb{Q}$ is countable.
            \end{theorem}
            \begin{theorem}
                $\mathbb{R}$ is uncountable.
            \end{theorem}
            For a set $X$, we often write
            $\mathcal{P}(X)$ to denote the
            \textit{power set} of $X$. This is the
            set of all subsets of $X$.
            For any set $X$ you can show that $X$ is
            strictly smaller than $\mathcal{P}(X)$.
            For example, $\mathcal{P}(\mathbb{N})$
            can be shown to be equivalent to $\mathbb{R}$.
            Since $\mathbb{N}$ is stricly smaller than
            $\mathbb{R}$, one might ask if there exists
            a set $X$ such that $\mathbb{N}$ is strictly
            smaller than $X$, but $X$ is strictly smaller
            than $\mathbb{R}$. Continuing, you can ask the
            same thing about $\mathbb{R}$ and
            $\mathcal{P}(\mathbb{R})$, and so on.
            This is called the continuum hypothesis.
            It turns out to be independent of
            the standard axioms of mathematics.
        \subsection{Completeness}
            One of the fundamental properties of $\mathbb{R}$ is
            that is is \textit{complete}. This property is
            fundamental to many theorems involved in a
            standard calculus or real analysis course. For
            example, the concepts of differentiation and
            convergence rely on completeness, and the
            intermediate value theorem may fail without it.
            On the other hand, $\mathbb{Q}$ is not complete. The
            rationals are, however, \textit{dense} in the reals.
            That is, elements of $\mathbb{R}$ can be
            approximated arbitrarily well by element of
            $\mathbb{Q}$. $\mathbb{R}$ is also something
            called a \textit{field}. From algebra,
            a field is just a
            set with two operations (Usually called addition
            and multiplication) that are defined in such
            a way as to make addition, subtraction,
            multiplication, and non-zero division
            well defined concepts, and to give them the
            basic properties of associativity, commutativity,
            and the distributive law that is found in
            arithmetic. $\mathbb{Q}$ is also a field.
            Moreover, $\mathbb{R}$ and $\mathbb{Q}$ are
            \textit{ordered fields} with respect to
            their standard ordering. What makes
            $\mathbb{R}$ special is that it is a
            complete ordered field. In fact, $\mathbb{R}$
            is the \textit{only} complete ordered field
            (Up to isomorphism). Completeness in
            $\mathbb{R}$ can be stated by fact that the real
            numbers have the least upper bound property.
            \begin{definition}
                A bounded above subset of $\mathbb{R}$
                is a nonempty subset $S\subset{\mathbb{R}}$
                such that there exists an $M\in\mathbb{R}$
                such that for all $x\in{S}$, $x\leq{M}$.
            \end{definition}
            \begin{definition}
                An upper bound of a bounded above
                subset $S\subset\mathbb{R}$ is a real
                number $M\in\mathbb{R}$
                such that for all $x\in{S}$, $x\leq{M}$.
            \end{definition}
            If $S\subset\mathbb{R}$ is bounded above,
            then there exists infinitely many bounds.
            The real numbers have a special property that
            every set that is bounded above has a
            smallest upper bound.
            \begin{theorem}[Least Upper Bound Theorem]
                If $S\subset{\mathbb{R}}$ is bounded above,
                then there exists an $s\in\mathbb{R}$,
                called the least upper bound, such that $s$
                and an upper bound and for all upper bounds
                $M$ of $S$, $s\leq{M}$.
            \end{theorem}
            \begin{theorem}
                There exist subsets $S\subset\mathbb{Q}$
                such that $S$ is bounded above, yet
                for all upper bounds $M$ there exists
                an $s$ such that $s$ is an upper
                bound of $S$ and $s<M$.
            \end{theorem}
            \begin{example}
                As an example, consider the set
                $\{x\in\mathbb{Q}:x^{2}\leq{2}\}$.
                This set has no least upper bound. Loosely
                speaking this is because the
                least upper bound wants to be $\sqrt{2}$,
                but $\sqrt{2}$ is not a rational number. Thus
                there is no rational number to fill the gap.
                The rationals are incomplete.
            \end{example}
            The least upper bound property gives rise
            to many theorems, many of which are equivalent
            to this axiom. Recall that a sequence is a
            function $x:\mathbb{N}\rightarrow{X}$. That is,
            a sequence is a function whose domain is the
            natural numbers, and whose image lies in some
            set $X$. A sequence of real numbers is thus a
            function $x:\mathbb{N}\rightarrow\mathbb{R}$,
            and a sequence of rational numbers is a function
            $x:\mathbb{N}\rightarrow\mathbb{Q}$.
            Often times sequences are denoted $x_{n}$,
            but also the image of $n$ is usually
            denoted $x(n)=x_{n}$ which may be a cause
            for confusion. That is, when we write $x_{n}$
            we often mean $x(n)$, so $x_{0}$, $x_{1}$,
            $x_{2}$ can be written as $x(0)$, $x(1)$,
            $x(2)$ but nobody does this. Similarly,
            we may mean $x_{n}=x$ since nobody writes
            a sequence as $x$. We will.
            \begin{definition}
                A sequence in a set $X$ is a function
                $x:\mathbb{N}\rightarrow{X}$.
                We write the image as $x(n)=x_{n}$.
            \end{definition}
            The notion of \textit{convergence} of a sequence
            is defined as follows.
            \begin{definition}
                A convergent sequence in a subset
                $S\subset\mathbb{R}$ is a sequence
                $x:\mathbb{N}\rightarrow{S}$
                such that there exists an $a\in\mathbb{R}$ such
                that $|a-x_{n}|\rightarrow{0}$. We write
                $x_{n}\rightarrow{a}$.
            \end{definition}
            \begin{definition}
                A limit of a sequence $x$
                in a subset $S\subset\mathbb{R}$ is an element
                $a\in\mathbb{R}$ such that
                $|a-x_{n}|\rightarrow{0}$.
            \end{definition}
            \begin{theorem}
                If $S\subset\mathbb{R}$ and $a$ and $b$ are limits
                of $x:\mathbb{N}\rightarrow{S}$, then $a=b$.
            \end{theorem}
            \begin{proof}
                Suppose not. Then $|a-b|>0$. But as $a$ is a
                limit of $x$, there is an $N_{1}\in\mathbb{N}$
                such that, for all $n>N_{1}$,
                $|a-x_{n}|<|a-b|/4$. But, as $b$ is a limit
                of $x$, there is an $N_{2}\in\mathbb{N}$
                such that for all $n>N_{2}$,
                $|b-x_{n}|<|a-b|/4$. Let $N=\max\{N_{1},N_{2}\}+1$.
                But from the triangle inequality:
                $|a-b|\leq|a-x_{N}|+|b-x_{N}|<|a-b|/2$, a
                contradiction. Therefore, $a=b$.
            \end{proof}
            The next notion to discuss is that of
            \textit{subsequences}. A subsequence is a sequence
            $k:\mathbb{N}\rightarrow\mathbb{N}$ such that $k$
            is a strictly increasing sequence of positive
            integers. In terms of functions, if
            $x:\mathbb{N}\rightarrow{X}$ is a sequence,
            and $k:\mathbb{N}\rightarrow\mathbb{N}$ is
            strictly increasing, then the \textit{composition}
            $(x\circ{k})(n)=x(k(n))=x_{k_{n}}$ is a
            subsequence of $x$. Since $k$ is strictly
            increasing the ordering of the original sequence $x$
            remains the same, and we've simply skipped over
            some points (Or infinitely many, or none).
            \begin{definition}
                A subsequence is a strictly increasing sequence
                $k:\mathbb{N}\rightarrow\mathbb{N}$
            \end{definition}
            \begin{definition}
                A convergent subsequence of a sequence
                $x:\mathbb{N}\rightarrow{S}$ in
                a subset $S\subset\mathbb{R}$ is a
                subsequence $k:\mathbb{N}\rightarrow\mathbb{N}$
                such that there is an $a\in{S}$ such that
                $|a-x_{k_{n}}|\rightarrow{0}$.
            \end{definition}
            \begin{definition}
                A monotonic subsequence of a sequence
                $x:\mathbb{N}\rightarrow{S}$ in a subset
                $S\subset\mathbb{R}$ is a subsequence
                $k:\mathbb{N}\rightarrow\mathbb{N}$ such
                that $x\circ{k}$ is a monotonic sequence.
            \end{definition}
            \begin{example}
                If $x_{n}:\mathbb{N}\rightarrow\mathbb{N}$ is
                the sequence defined by $x_{n}=n$, and if
                $k_{n}=2n$, then $x_{k_{n}}=2n$. This is the
                subsequence of all even numbers. Similarly,
                if $k_{n}=2n-1$, then $x_{k_{n}}=2n-1$. This
                is the subsequence of all odd numbers. As a
                very boring example, take $k_{n}=n$. Then
                $x_{k_{n}}=n$. This is the identity subsequence.
                It does not change the original sequence.
            \end{example}
            There is an important theorem associated
            to subsequences of bounded sequences called the
            Bolzano-Weierstrass theorem. It states that
            every bounded sequence has a convergent subsequence,
            and is an equivalent definition of the
            completeness of $\mathbb{R}$.
            Monotonic sequences are sequences such
            that, for all $n\in\mathbb{N}$, either
            $x_{n+1}\leq{x_{n}}$ (Monotonically decreasing),
            or $x_{n}\leq{x_{n+1}}$ (Monotonically increasing).
            Strictly monotonic means either $x_{n+1}<x_{n}$
            or $x_{n}<x_{n+1}$ for all $n\in\mathbb{N}$.
            \begin{theorem}
                \label{th:funct:bounded_monotone_%
                       sequences_converge}
                Bounded monotonic sequences converge.
            \end{theorem}
            \begin{proof}
                Let $x$ be a bounded monotonic sequence that
                is increasing in $\mathbb{R}$.
                If $x$ is decreasing, we replace the least
                upper bound with the greatest lower
                bound and the proof is symmetric.
                Then $S=\{x_{n}:n\in\mathbb{N}\}$ is a
                non-empty subset of $\mathbb{R}$. But $x$ is
                a bounded sequence, and therefore $S$ is a
                bounded subset of $\mathbb{R}$. By the least
                upper bound property there exists a least
                upper bound $s\in\mathbb{R}$ of $S$.
                We now show that $x_{n}\rightarrow{s}$.
                Let $\varepsilon>0$ be given. Since $s$ is
                the least upper bound, $s-\varepsilon$
                is not an upper bound of $S$, since
                $s-\varepsilon<s$. Therefore there exists
                an $N\in\mathbb{N}$ such that
                $s-\varepsilon<x_{N}$. But $x$ is
                monotonically increasing, and therefore
                for all $n>N$, $x_{N}\leq{x_{n}}$.
                But, as $s$ is a least upper
                bound of $S$, $x_{n}\leq{s}$. But then,
                for all $n>N$,
                $0\leq{s-x_{n}}\leq{s-x_{N}}<\varepsilon$.
                Therefore, $x_{n}\rightarrow{s}$.
            \end{proof}
            The least upper bound is, in a sense, the
            reason why decimal expansions of
            real numbers work. For example, let $x$ be the
            sequence 3, 3.1, 3.14, 3.141, 3.1415, 3.14159, ...
            This sequence, which is the decimal
            expansion of $\pi$,
            is bounded by $4$. Therefore it has a least
            upper bound. We define the number $\pi$
            to be the least upper bound of this sequence.
            Completeness is a very important property
            but so far it relies on the ordering
            of the real numbers.
            We want to find an equivalent definition
            of completeness that does not rely on ordering
            so that we may speak of complete spaces,
            or sets, which have no notion of
            order. We start with a different definition
            for the completeness of $\mathbb{R}$.
            \begin{definition}
                A Cauchy sequence in a subset
                $S\subset\mathbb{R}$ is a
                sequence $x:\mathbb{N}\rightarrow{S}$
                such that for all $\varepsilon>0$ there
                is an $N\in\mathbb{N}$ such that for all
                $n,m>N$, $|x_{n}-x_{m}|<\varepsilon$.
                That is:
                \begin{equation*}
                    \forall_{\varepsilon>0}
                    \exists_{N\in\mathbb{N}}:
                    n,m>N\Rightarrow
                    |x_{n}-x_{m}|<\varepsilon
                \end{equation*}
            \end{definition}
            \begin{theorem}
                \label{FUNCTIONAL_ANALYSIS:CONVERGENT_%
                       SEEQUENCES_ARE_CAUCHY_SEQUENCES}
                If $S\subset\mathbb{R}$ and if
                $x:\mathbb{N}\rightarrow\mathbb{R}$
                is a convergent sequence, then it
                is a Cauchy sequence.
            \end{theorem}
            \begin{proof}
                For let $x$ be a convergent sequence and
                let $a$ be it's limit.
                Let $\varepsilon>0$ be given. Then, as
                $x_{n}\rightarrow{a}$, there is an
                $N\in\mathbb{N}$ such that for all $n>N$,
                $|x_{n}-a|<\frac{\varepsilon}{2}$.
                But by the triangle inequality,
                for all $n,m>N$:
                \begin{equation*}
                        |x_{n}-x_{m}|\leq
                        |x_{n}-a|+|x_{m}-a|<
                        \frac{\varepsilon}{2}+
                        \frac{\varepsilon}{2}
                        =\varepsilon
                    \end{equation*}
                Therefore, $x$ is a Cauchy sequence.
            \end{proof}
            The converse of
            Thm.~\ref{FUNCTIONAL_ANALYSIS:CONVERGENT_%
                      SEEQUENCES_ARE_CAUCHY_SEQUENCES}
            turns out to be a more general notion
            of completeness. That is, we can apply
            this to spaces that do not have
            a notion of order, but do have a notion
            of completeness. There are Cauchy sequences
            $x:\mathbb{N}\rightarrow\mathbb{Q}$ that do
            not converge. This is again related to the fact
            that $\mathbb{Q}$ is not complete. For sequences
            $x:\mathbb{N}\rightarrow\mathbb{R}$,
            if $x$ is Cauchy then it must converge.
            \begin{theorem}
                If $S\subset\mathbb{R}$ and if
                $x:\mathbb{N}\rightarrow{S}$
                is a Cauchy sequence, then it is bounded.
            \end{theorem}
            \begin{proof}
                For as $x$ is a Cauchy sequence there is an
                $N\in\mathbb{N}$ such that, for all $n,m>N$,
                $|x_{n}-x_{m}|<1$. Then, for all $n>N+1$,
                $x_{N+1}-1<x_{n}<x_{N+1}+1$. Let
                $M=\max\{|x_{0}|,|x_{1}|,\hdots,|x_{N+1}|+1\}$.
                Then for all $n\in\mathbb{N}$,
                $|x_{n}|<M$. Therefore, etc.
            \end{proof}
            We cannot replace that requirement that,
            for all $n,m>N$, $|x_{n}-x_{m}|<\varepsilon$
            with $n,n+k$ for some fixed $k\in\mathbb{N}$.
            There are sequences such that
            $x_{n+1}-x_{n}\rightarrow{0}$,
            yet $x$ is not Cauchy.
            \begin{example}
                There are unbounded sequences $x$ such that
                $x_{n+1}-x_{n}\rightarrow{0}$. For let
                $x:\mathbb{N}\rightarrow\mathbb{R}$
                be the sequence defined
                by $x_{n}=\sqrt{n}$. Then:
                \begin{equation*}
                    x_{n+1}-x_{n}=\sqrt{n+1}-\sqrt{n}
                    =\frac{1}{\sqrt{n+1}+\sqrt{n}}
                    <\frac{1}{2\sqrt{n}}
                    \rightarrow{0}
                \end{equation*}
                However, $\sqrt{n}\rightarrow\infty$.
                Moreover, there are bounded sequences $x$
                such that $x_{n+1}-x_{n}\rightarrow{0}$,
                yet $x$ is not Cauchy. For let
                $x:\mathbb{N}\rightarrow\mathbb{R}$
                be the sequence defined by
                $x_{n}=\sin(\sqrt{n})$.
                Then $x$ is bounded, and:
                \begin{equation*}
                    x_{n+1}-x_{n}=
                    \sin\big(\sqrt{n+1})-\sin(\sqrt{n}\big)=
                    2\sin\Big(\frac{\sqrt{n+1}-\sqrt{n}}{2}\Big)
                    \cos\Big(\frac{\sqrt{n+1}-\sqrt{n}}{2}\Big)
                \end{equation*}
                But we saw from the previous example that
                $\sqrt{n+1}-\sqrt{n}\rightarrow{0}$, and
                therefore $x_{n+1}-x_{n}\rightarrow{0}$.
                $x$ is not Cauchy, however, for let
                $k:\mathbb{N}\rightarrow\mathbb{N}$ be
                the subsequence defined by $k_{n}=n^{2}$. Then:
                \begin{equation*}
                    x_{k_{n+1}}-x_{k_{n}}
                    =\sin(n+1)-\sin(n)
                    =2\sin\Big(\frac{1}{2}\Big)
                    \cos\Big(\frac{2n+1}{2}\Big)
                \end{equation*}
                And this does not converge to zero.
                Therefore, $x$ is not Cauchy.
            \end{example}
            \begin{theorem}
                \label{th:funct:sequences_have_%
                       monotonic_subsequence}
                Every sequence in $\mathbb{R}$
                has a monotonic subsequence.
            \end{theorem}
            \begin{proof}
                Let $x$ be a sequence in $\mathbb{R}$.
                Call $n$ a ``peak point'' if
                $x_{n}\geq{x_{m}}$ for all
                ${m}\geq{n}$. If there are infinitely many
                of these peak points, then we have obtained
                a decreasing sequence, since the $n^{th}$
                peak point will be greater than or equal to
                the $(n+1)^{th}$ peak point.
                We have thus obtained
                a monotonically decreasing subsequence.
                If there are finitely many,
                there are either zero or there is a last one,
                $x_{n_{0}}$. Then $x_{n_{0}+1}$ is not a
                peak point. But then there is a
                $k\in\mathbb{N}$ such that $k>n_{0}+1$ and
                $x_{k}\geq{x_{n_{0}+1}}$, for otherwise
                $x_{n_{0}+1}$ would be a peak point. But
                $x_{k}$ is also not a peak point, and so
                there is a $k_{1}$ such that $k_{1}>k$ and
                $x_{k_{1}}\geq{x_{k}}$. This pattern
                continues, and we thus have a monotonically
                increasing subsequence. If there are zero
                peak points, repeat the argument above
                argument with $x_{n_{0}}=x_{1}$.
            \end{proof}
            There's probably some axiom of choice stuff
            going on here.
            \begin{theorem}[Bolzano-Weierstrass Theorem]
                    Bounded sequences have a
                    convergent subsequence.
                \end{theorem}
            \begin{proof}
                By Thm.~\ref{th:funct:sequences_%
                             have_monotonic_subsequence},
                all sequences have a monotonic subsequence.
                But by Thm.~\ref{th:funct:bounded_%
                                 monotone_sequences_converge},
                bounded monotonic sequences converge.
                Therefore there is a convergent subsequence.
            \end{proof}
            This notion is so important it has a name.
            A sequentially
            compact space is a space such that every bounded
            sequence has a convergent subsequence. The
            Bolzano-Weierstrass Theorem is equivalent
            to saying that $\mathbb{R}$ is sequentially
            compact. The general notion of \textit{compactness}
            is a topological one, but as it turns out
            sequential compactness and compactness are
            identical concepts in a \textit{metric space}.
            Metric spaces will be one of the primary
            subjects of study in functional analysis.
            In $\mathbb{R}^{n}$ there is
            another equivalent, and perhaps more intuitive,
            definition of compactness. A subset of
            $\mathbb{R}^{n}$ is compact if and only if it
            is closed and bounded. A set $S\subset\mathbb{R}$
            is closed if for all convergent sequences
            $x:\mathbb{N}\rightarrow{S}$,
            the limit also lies in $S$.
            Compactness will be discussed later in the
            context of continuous functions on a compact set.
            \begin{example}
                Find a subsequence of $n$ for which
                both $\sin(x_{k_{n}})$ and $\cos(x_{k_{n}})$
                converge. In degrees
                this is rather simple: Let $k$ be defined by
                $k_{n}=360+45n$. Then
                $\sin(x_{k_n})=\cos(x_{k_{n}})=1/\sqrt{2}$.
                In radians we need to be a little more careful.
                Let $y:\mathbb{N}\rightarrow\mathbb{R}$
                be defined by $y_{n}=\sin(n)$.
                Then $y$ is bounded and
                by the Bolzano-Weierstrass theorem,
                there is a convergent subsequence $k$.
                Let $z:\mathbb{N}\rightarrow\mathbb{R}$
                be defined by $z_{n}=\cos(x_{k_{n}})$. Then $z$
                is bounded and by the
                Bolzano-Weierstrass theorem there is a
                convergent subsequence $j$. But
                any subsequequence of a convergent sequence
                converges to the same limit, and therefore
                $\sin(x_{k_{j_{n}}})$ converges. Thus,
                $\sin(x_{k_{j_{n}}})$ and
                $\cos(x_{k_{j_{n}}})$ converge. It's also
                possible to make them converge to the same
                limit. We need to know that the set
                $\{n\mod\alpha:n\in\mathbb{N}\}$ is dense
                in $(0,\alpha)$ when $\alpha$ is irrational.
                Thus there is a subsequence such that
                $x_{k_{n}}\mod2\pi\rightarrow{\frac{\pi}{4}}$.
                Then $\sin(x_{k_{n}})$ and $\cos(x_{k_{n}})$
                both converge to $\frac{1}{\sqrt{2}}$.
                Let's first try to find a subsequence such that
                $\cos(k_{n})\rightarrow{1}$. If we can
                do that, we simply need to modify the
                argument so that
                $\cos(k_{n})\rightarrow{\frac{1}{\sqrt{2}}}$.
                Let $k$ be a sequence of integers
                such that $0<n-2\pi{k_{n}}<2\pi$.
                Let $\varepsilon>0$ and let $N\in\mathbb{N}$
                be such that $N>\frac{2\pi}{\varepsilon}$.
                Then the set
                $\{n-2\pi{k_{n}}:n=1,2,\hdots,N+1\}$
                has $N+1$ elements, and thus by the
                pidgeonhole principle there are two
                elements that are within
                $2\pi/\frac{2\pi}{\varepsilon}$ of each other.
                Let $n_{1}$ and $n_{2}$ be such numbers.
                Then:
                \begin{equation*}
                    \cos(n_{2}-n_{1})
                    =\cos(n_{2}-n_{1}-2\pi(k_{2}-k_{1}))
                    =\cos((n_{2}-2\pi{k}_{2})
                           -(n_{1}-2\pi{k_{1}}))
                    =\cos(\xi)
                \end{equation*}
                Where $\xi$ is a number such that
                $0<|\xi|<\varepsilon$. But then
                $|1-\cos(\xi)|<\frac{\varepsilon^{2}}{2}$.
                And $n_{2}-n_{1}$ is a natural number,
                so we can find a subsequence of $n$ such
                that $\cos(k_{n})\rightarrow{1}$. Modifying
                this with $\frac{\pi}{4}$
                and $\frac{1}{\sqrt{2}}$ gives the result.
            \end{example}
            \begin{theorem}
                If $x:\mathbb{N}\rightarrow\mathbb{R}$ is
                a Cauchy sequence, then it converges.
            \end{theorem}
            \begin{proof}
                If $x$ is Cauchy, then it is bounded.
                By the Bolzano-Weiestrass theorem there
                is a convergent subsequence $k$. But then there
                is an $a\in\mathbb{R}$ such that
                $x_{k_{n}}\rightarrow{a}$. We now must show that
                $x_{n}\rightarrow{a}$. Let $\varepsilon>0$
                be given. As $x_{k_{n}}\rightarrow{a}$,
                there is an $N_{1}\in\mathbb{N}$ such
                that for all $n>N_{1}$,
                $|x_{k_{n}}-a|<\frac{\varepsilon}{2}$.
                But as $x$ is a Cauchy sequence, there
                is an $N_{2}$ such that for all $n,m>N_{2}$, 
                $|x_{n}-x_{m}|<\frac{\varepsilon}{2}$. Let
                $N=\max\{N_{1},N_{2}\}$. 
                But as $k$ is a subsequence, and therefore
                $k_{n}>N$. Thus, for all $n>N$,
                $|x_{k_{n}}-x_{n}|<\frac{\varepsilon}{2}$.
                By the triangle inequality,
                    $|a-x_{n}|\leq
                     |a-x_{k_{n}}|+|x_{k_{n}}-x_{n}|\leq
                     \frac{\varepsilon}{2}+
                     \frac{\varepsilon}{2}%
                     =\varepsilon$.
               Therefore, etc.
            \end{proof}
            Real numbers can be constructed by considering
            \textit{equivalence classes} of Cauchy sequences of
            rational numbers. Two Cauchy sequences $x_{n}$ and
            $y_{n}$ are called \textit{equivalent} if
            $x_{n}-y_{n}\rightarrow{0}$. By considering the set
            of all such equivalent sequences, we can give a more
            rigorous construction of the real numbers.
        \subsection{Continuity}
            \begin{definition}
                A function $f:S\rightarrow\mathbb{R}$
                on a subset $S\subset\mathbb{R}$ continuous
                at a point $x\in{S}$ is a function such that
                for all $\varepsilon>0$ there is a $\delta>0$
                such that for all $x_{0}\in{S}$,
                $|x-x_{0}|<\delta$ implies
                $|f(x)-f(x_{0}|<\varepsilon$.
            \end{definition}
            \begin{theorem}
                If $S\subset\mathbb{R}$, $x\in{S}$,
                if $f:S\rightarrow\mathbb{R}$
                is continuous at $x$, and if
                $a:\mathbb{N}\rightarrow{S}$
                is a convergent sequence such that
                $a_{n}\rightarrow{x}$, then
                $f(a_{n})\rightarrow{f(x)}$.
            \end{theorem}
            \begin{proof}
                For let $\varepsilon>0$. As $f$ is
                continuous there is a $\delta>0$ such that,
                for all $x_{0}\in\mathbb{R}$
                such that $|x-x_{0}|<\delta$,
                $|f(x)-f(x_{0})|<\varepsilon$.
                But $a_{n}\rightarrow{x}$, and thus there is an
                $N\in\mathbb{N}$ such that, for all $n>N$,
                $|x-a_{n}|<\delta$. But then, for all $n>N$,
                $|f(x)-f(a_{n})|<\varepsilon$. Therefore,
                $f(a_{n})\rightarrow{f(x)}$.
            \end{proof}
            The converse of this theorem is true,
            and thus we have
            found an equivalent definition of continuity.
            \begin{theorem}
                If $x\in\mathbb{R}$ and if
                $f:\mathbb{R}\rightarrow\mathbb{R}$
                is a function such that, for all sequences
                $a:\mathbb{N}\rightarrow\mathbb{R}$ such that
                $a_{n}\rightarrow{x}$,
                $f(a_{n})\rightarrow{f(x)}$,
                then $f$ is continuous at $x$.
            \end{theorem}
            \begin{proof}
                For suppose not. Then there is an
                $\varepsilon>0$ such that, for all $\delta>0$,
                there is an $x_{0}\in\mathbb{R}$ such that
                $|x-x_{0}|<\delta$ and
                $|f(x)-f(x_{0})|\geq\varepsilon$.
                Let $a:\mathbb{N}\rightarrow\mathbb{R}$
                be a sequence such that, for all
                $n\in\mathbb{N}$, $|a_{n}-x|<1/n$, but
                $|f(x)-f(a_{n})|\geq\varepsilon$.
                But then $a_{n}\rightarrow{x}$. But for all
                sequences $a$ such that $a_{n}\rightarrow{x}$,
                $f(a_{n})\rightarrow{f(x)}$. But, for all $n$,
                $|f(x)-f(a_{n})|\geq\varepsilon$,
                a contradiction.
                Therefore, $f$ is continuous at $x$.
            \end{proof}
            \begin{theorem}
                \label{thm:Funct:Continuous_Limit_%
                       of_Pos_Sequ_is_nonneg}
                If $x\in\mathbb{R}$, if
                $f:\mathbb{R}\rightarrow\mathbb{R}$
                is continuous at $x$, and if
                $a:\mathbb{N}\rightarrow\mathbb{R}$ is a
                convergent sequence such that 
                $a_{n}\rightarrow{x}$ and $f(a_{n})>0$
                for all $n\in\mathbb{N}$, then $f(x)\geq{0}$.
            \end{theorem}
            \begin{proof}
                For suppose not. Let $r=f(x)<0$, and let
                $\varepsilon=|r|/2$. Then $\varepsilon>0$. But
                from continuity, there is a $\delta>0$ such that
                for all $x_{0}$ such that $|x-x_{0}|<\delta$,
                $|f(x)-f(x_{0})|<\varepsilon$. But
                $a_{n}\rightarrow{x}$, and thus there is an
                $N\in\mathbb{N}$ such that for all $n>N$,
                $|x-a_{n}|<\delta$. But then for all $n>N$,
                $|f(x)-f(a_{n})|<\varepsilon$. But then
                $f(a_{n})<f(x)+\varepsilon=f(x)/2<0$,
                a contradiction as $f(a_{n})>0$ for
                all $n\in\mathbb{N}$. Therefore, etc.
            \end{proof}
            \begin{theorem}
                If $x\in\mathbb{R}$ and
                $a:\mathbb{N}\rightarrow\mathbb{R}^{+}$
                is a convergent sequence such that
                $a_{n}\rightarrow{x}$, then $x\geq{0}$.
            \end{theorem}
            \begin{proof}
                For let $f:\mathbb{R}\rightarrow\mathbb{R}$
                be defined by $f(x)=x$. Then $f$ is continuous,
                and for all $n\in\mathbb{N}$,
                $f(a_{n})=a_{n}>0$.
                Thus, by Thm.~\ref{thm:Funct:Continuous_Limit_%
                                   of_Pos_Sequ_is_nonneg},
                $f(x)=x\geq{0}$.
            \end{proof}
            \begin{theorem}
                If $x\in\mathbb{R}$, if
                $f:\mathbb{R}\rightarrow\mathbb{R}$ is
                continuous at $x$, and if $f(x)>0$,
                then there is an open interval
                $\mathcal{U}$ such that $x\in\mathcal{U}$,
                and for all $y\in\mathcal{U}$, $f(y)>0$.
            \end{theorem}
            \begin{proof}
                For let $\varepsilon=f(x)/2$. Then
                $\varepsilon>0$, and thus there is a $\delta>0$
                such that for all $x_{0}\in\mathbb{R}$
                such that $|x-x_{0}|<\delta$,
                $|f(x)-f(x_{0})|<\varepsilon$. Let
                $\mathcal{U}=(x-\delta,x+\delta)$.
                If $y\in\mathcal{U}$,
                then $|x-y|<\delta$, and therefore:
                \begin{equation*}
                    |f(y)-f(x)|<\varepsilon
                    \Rightarrow
                    f(y)>f(x)-\varepsilon
                    =\frac{f(x)}{2}>0
                \end{equation*}
                Thus, for all $y\in\mathcal{U}$, $f(y)>0$.
            \end{proof}
            \begin{definition}
                A continuous function
                $f:S\rightarrow\mathbb{R}$ is a function
                that is continuous at all $x\in{S}$.
            \end{definition}
            This definition comes from the fact that
            continuity is a point-wise property, and not a
            ``curve'' property. Continuous functions are
            functions that have point-wise continuity at
            every point. The statement ``A continuous function
            is a curve that you can draw,'' which many have
            heard in calculus, is slightly misleading. There
            are functions that are continuous at one point and
            nowhere else. There are functions that are
            continuous on the irrationals and discontinuous
            on the rationals. For example, if $x$ is
            rational write it as $x=p/q$ where $p$ and
            $q$ are integers and relatively prime. Define
            $f(x)=1/q$. If $x$ is irrational, define
            $f(x)=0$. This function is continuous at every
            irrational number and discontinuous at every
            rational number. There is no ``reverse,'' of this
            function. That is, there is no function which is
            continuous on $\mathbb{Q}$ and discontinuous at
            every irrational number. Uniform continuity is a
            property of all points in the domain of a function.
            We can rewrite the defintion of continuity as
            follows:
            \begin{definition}
                A continuous function on a set $S$ is a
                function $f:S\rightarrow\mathbb{R}$ such that:
                \begin{equation*}
                    \forall_{x\in{S}}\forall_{\varepsilon>0}
                    \exists_{\delta>0}:\forall_{x_{0}\in{S}}:
                    |x-x_{0}|<\delta
                    \Rightarrow|f(x)-f(x_{0})|<\varepsilon
                \end{equation*}
            \end{definition}
            This says, give me a point $x$
            and a positive number
            $\varepsilon$ and I can find a $\delta$ satisfying
            this property. The key part is that you must
            specify the point first. That is, the $\delta$ I
            choose may be dependent on the $x$ you choose.
            Uniform continuity occurs when a $\delta>0$ can be
            chosen regardless of $x$. $\delta$ is
            only dependent on $\varepsilon$.
            \begin{definition}
                A uniformly continuous function on a set $S$
                is a function $f:S\rightarrow\mathbb{R}$ such that:
                \begin{equation*}
                    \forall_{\varepsilon>0}\exists_{\delta>0}
                    \forall_{x\in{S}}:\forall_{x_{0}\in{S}}:
                    |x-x_{0}|<\delta
                    \Rightarrow|f(x)-f(x_{0})|<\varepsilon    
                \end{equation*}
            \end{definition}
            Continuity is a point-wise property. There are functions
            that are continuous at one point and nowhere else.
            Uniform continuity, however, is a set property. You can't
            have uniform continuity at a single point, but rather on
            a set of points.
            \begin{theorem}
                \label{thm:Funct:equiv_def_of_uni_cont}
                A function $f:S\rightarrow\mathbb{R}$
                is uniformly continuous if and only if
                for all sequences $x_{n}$ and $y_{n}$ in
                $S$ such that $x_{n}-y_{n}\rightarrow{0}$,
                $f(x_{n})-f(y_{n})\rightarrow{0}$.
            \end{theorem}
            \begin{proof}
                Let $\varepsilon>0$. If $f$ is uniformly continuous,
                then there is a $\delta>0$ such that for all
                $x$, $x_{0}\in{S}$ such that $|x-x_{0}|<\delta$,
                we have that $|f(x)-f(x_{0})|<\varepsilon$. But if
                $x_{n}-y_{n}\rightarrow{0}$, then there is an
                $N\in\mathbb{N}$ such that for all $n>N$,
                $|x_{n}-y_{n}|<\delta$. But then, for all $n>N$,
                $|f(x_{n})-f(y_{n})|<\varepsilon$. Thefefore,
                $f(x_{n})-f(y_{n})\rightarrow{0}$. Proving the
                converse, suppose not. If $f$ is not uniformly
                continuous, then there exists $\varepsilon>0$
                such that for all $\delta>0$ there exists
                $x$, $x_{0}\in{S}$ such that
                $|x-x_{0}|<\delta$ and yet
                $|f(x)-f(x_{0})|\geq{\varepsilon}$. Let
                $x_{n}$ and $y_{n}$ be points such that
                $|x_{n}-y_{n}|<\frac{1}{n}$ and yet
                $|f(x_{n})-f(y_{n})|\geq\varepsilon$. Then
                $x_{n}-y_{n}\rightarrow{0}$. But if
                $x_{n}-y_{n}\rightarrow{0}$, then
                $f(x_{n})-f(y_{n})\rightarrow{0}$. But for all
                $n$, $|f(x_{n})-f(y_{n})|\geq{\varepsilon}$,
                a contradiction. Therefore, etc.
            \end{proof}
            The requirement of uniform continuity is crucial.
            Let $f:(0,1)\rightarrow\mathbb{R}$ be defined by
            $f(x)=\frac{1}{x}$. Then $f$ is continuous, but
            not uniformly continuous. Let $x_{n}=\frac{1}{n}$
            and $y_{n}=\frac{1}{2n}$. Then
            $y_{n}-x_{n}=\frac{1}{2n}\rightarrow{0}$, but
            $f(y_{n})-f(x_{n})=2n-n=n$, which diverges.
            Point-wise convergence gives us that
            $f(x_{n})-f(x)\rightarrow{0}$. In
            Uniform convergence allows the target to
            vary as well. Point-wise convergence can
            not guarantee this.
            \begin{theorem}
                If $f:[a,b]\rightarrow\mathbb{R}$ is continuous,
                then $f$ is uniformly continuous.
            \end{theorem}
            \begin{proof}
                Let $x_{n}$, $y_{n}\in[a,b]$ be sequences such
                that $x_{n}-y_{n}\rightarrow{0}$. By the
                Bolzano-Weierstrass theorem there is a
                convergent subsequence $x_{k_{n}}$. Let $x$
                be the limit. But:
                \begin{equation*}
                    y_{k_{n}}=x_{k_{n}}-(y_{k_{n}}-x_{k_{n}})
                    \Rightarrow
                    y_{k_{n}}\rightarrow{x}
                \end{equation*}
                Let $X_{n}=x_{k_{n}}$, and $Y_{n}=y_{k_{n}}$. Then:
                \begin{equation*}
                    f(X_{n})-f(Y_{n})=(f(X_{n})-f(x))-(f(Y_{n}-f(x))
                \end{equation*}
                But from continuity, $f(x_{n})\rightarrow{f(x)}$
                and $f(y_{n})\rightarrow{f(x)}$ and therefore
                $f(X_{n})-f(Y_{n})\rightarrow{0}$. Then by
                Thm.~\ref{thm:Funct:equiv_def_of_uni_cont}
                $f$ is uniformly continuous. Therefore, etc.
            \end{proof}
            \begin{theorem}
                If $f:[a,b]\rightarrow\mathbb{R}$ is
                continuous, then it is uniformly continuous.
            \end{theorem}
            The above theorem relies on the fact that
            $[a,b]$ is closed and bounded. Indeed, this is
            the only thing it relies on, the fact that it's
            an interval (Or connected) it unnecessary. We can
            write a more general result.
            \begin{theorem}
                    If $f:S\rightarrow\mathbb{R}$ is continuous
                    and if $S$ is compact, then $f$ is
                    uniformly continuous.
                \end{theorem}
            \begin{theorem}[Intermediate Value Theorem]
                    If $f:[a,b]\rightarrow\mathbb{R}$
                    is continuous and
                    if $f(a)<f(b)$, then for all $z\in(f(a),f(b))$,
                    there is a $c\in(a,b)$ such that $f(c)=z$.
                \end{theorem}
            \begin{proof}
                    Let $x_{1}=\frac{a+b}{2}$. By trichotomy,
                    which is one of the ordering properties, either
                    $f(x_{1})=z$, $f(x_{1})<z$, or $f(x_{1})>z$. If
                    $f(x_{1})=z$, we are done. If not, suppose
                    $f(x_{1})<z$. The proof is symmetric for
                    $f(x_{1})>z$. Let $x_{2}=\frac{x_{1}+b}{2}$.
                    We continue checking whether $f(x_{2})=z$,
                    and continue dividing the region in two.
                    If $f(x_{2})<z$,  we set
                    $x_{3}=\frac{x_{1}+x_{2}}{2}$, and if
                    $f(x_{2})>z$ we set $x_{3}=\frac{x_{2}+b}{2}$.
                    Note that $|x_{n+1}-x_{n}|=\frac{b-a}{2^{n+1}}$.
                    Moreover, $x_{n}$ converges. Suppose it
                    converges to $c$. Then $c\in[a,b]$. That is,
                    the limit of a function
                    $x_{n}:\mathbb{N}\rightarrow[a,b]$ is
                    contained in $[a,b]$. This is related to the
                    ``compactness'' of $[a,b]$. Moreover, it is
                    related to the ``closedness'' of $[a,b]$.
                    If there is an $N\in\mathbb{N}$ such that
                    $f(x_{N})=z$, then we are
                    done. Suppose not. Let $k_{n}$ be the
                    elements such that $f(x_{k_{n}})<z$ and
                    $\ell_{n}$ be the elements such that
                    $f(x_{\ell_{n}})>z$. Both of these must be
                    infinite. For suppose not.
                    Suppose there is a final $N$ such that
                    $f(x_{N})<z$. Then for all $n>N$, $f(x_{n})>z$.
                    But from how we've defined the sequence $x_{n}$,
                    we have that $x_{n}\rightarrow{x_{N}}$.
                    From the continuity of $f$, there is an open
                    interval about $X_{N}$ such that for all
                    elements $x$ inside that interval, we have
                    that$f(x)<z$. A contradiction, since eventually
                    some of the $x_{n}$ will be in this interval,
                    and thus $f(x_{n})<z$. So, both $k_{n}$
                    and $\ell_{n}$ are
                    infinite. From continuity, we have
                    $\lim_{n\rightarrow\infty}f(x_{k_{n}})\leq{z}$
                    and
                    $\lim_{n\rightarrow\infty}
                     f(x_{\ell_{n}})\geq{z}$.
                    Thus, $f(c)\leq{z}$ and $f(c)\geq{z}$,
                    and therefore $f(c)=z$.
                \end{proof}
            This theorem fails in $\mathbb{Q}$, for it relies
            on the completeness of $\mathbb{R}$. For example,
            $f(x)=x^{2}$ defined on $[0,4]$.
            Then $2\in[0,4]$, but there is no
            rational such that $x^{2}=2$.
            A cuter way to phrase this, in a more topological
            sense, is that the image of $[a,b]$, which is an
            interval, or a connected subset of $\mathbb{R}$,
            is again an interval, or a connected subset of
            $\mathbb{R}$. The proof that continuous
            functions take connected sets (Intervals)
            to connected sets (Again, intervals) is a lot
            easier than the one presented
            here, but relies on notions from topology.
            So we'll skip that.
            Another commonly used theorem in calculus
            (Again, usually not proved) is the extreme
            value theorem.
            The extreme value is used to proved Rolle's theorem,
            which says that if $f$ is differentiable on $(a,b)$
            and if $f(a)=f(b)$, then there is a point
            $c\in(a,b)$ such that $f'(c)=0$. This is used to
            prove the mean value theorem, which says that
            if $f$ is differentiable on $(a,b)$,
            then there is a point $c\in(a,b)$ such that
            $f'(c)=\frac{f(b)-f(a)}{b-a}$. This is in
            turned used to prove the Fundamental Theorem
            of Calculus. So some very important stuff going
            on here. First we prove that continuous functions
            on closed and bounded sets (That is,
            compact sets) are bounded. We stick to
            closed intervals for now.
            \begin{theorem}
                    If $f:[a,b]\rightarrow\mathbb{R}$ is
                    continuous, then it is bounded.
                \end{theorem}
            \begin{proof}
                    Suppose not. Then for all $n\in\mathbb{N}$, 
                    there is an $x_{n}\in[a,b]$ such that
                    $f(x_{n})>n$. But then
                    $x_{n}$ is a bounded sequence, and thus by
                    Bolzano-Weierstrass there is a convergent
                    subsequence. Let $x$ be the limit of this
                    convergent subsequence.
                    But then $f(x_{k_{n}})\rightarrow{f(x)}$,
                    this is the fundamental property of
                    continuous functions. This is indeed
                    equivalent to the standard
                    $\varepsilon-\delta$ definition of continuity.
                    But $f(x_{k_{n}})\rightarrow\infty$,
                    and $f(x)<\infty$, a contradiction.
                    Therefore, etc.
                \end{proof}
            \begin{theorem}[Exreme Value Theorem]
                    If $f:[a,b]\rightarrow\mathbb{R}$ is
                    continuous, then there exists $c\in[a,b]$
                    such that for all $x\in[a,b]$,
                    $f(x)\leq{f(c)}$
                \end{theorem}
            \begin{proof}
                    By the previous theorem,
                    $\{f(x):x\in[a,b]\}$ is bounded.
                    By completeness, there is a least upper
                    bound. Let $s$ be such
                    a bound. If $s$ is the least upper bound,
                    then $s-1$ is not a least upper bound.
                    Therefore this is an $x_{0}\in[a,b]$ such
                    that $s-1<f(x_{0})$. Similarly,
                    for all $n\in\mathbb{N}$, there is an
                    $x_{n}\in\mathbb{N}$ such that
                    $s-\frac{1}{n}<f(x_{n})$. But $x_{n}$
                    is a bounded sequence, and bounded
                    sequences have a convergent subsequence.
                    Let $x$ be the limit of this subsequence.
                    From continuity,
                    $f(x)=\lim_{n\rightarrow\infty}f(x_{k_{n}})$.
                    But $s-\frac{1}{n}\leq{f(x_{k_{n}})}\leq{s}$,
                    and therefore $f(x_{k_{n}})\rightarrow{s}$.
                    Thus, $f(x)=s$.
                \end{proof}
            Much the way the intermediate value theorem can be
            generalized to say that the continuous image of
            connected sets is connected, the extreme value
            theorem can be generalized to say that the
            continuous image of compact sets is compact.
            The proof is rather easy, but requires
            topology. So, we'll skip that too.
            The requirement of these previous theorems on
            continuity is crucial. Without continuity, functions
            on $[a,b]$ need not be bounded and functions on
            $(a,b)$ can just ``jump,'' right over other points.
            We end with a brief discussion on sequences of
            functions.
        \subsection{Sequences of Functions}
            \begin{definition}
                    A sequence of functions from a
                    set $X$ to a set $Y$ is a function
                    $F:\mathbb{N}\times{X}\rightarrow{Y}$.
                    We often write the image of
                    $(n,x)\in\mathbb{N}\times{X}$ as
                    $F(n,x)=F_{n}(x)$.
                \end{definition}
            \begin{definition}
                    A sequence of real-valued functions
                    $F$ converges point-wise to a function
                    $f$ if:
                    \begin{equation*}
                        \forall_{\varepsilon>0}
                        \forall_{x\in\mathbb{R}}
                        \exists{N\in\mathbb{N}}:
                        \forall_{n>N}\Rightarrow
                        |f(x)-F_{n}(x)|<\varepsilon
                    \end{equation*}
                \end{definition}
            That is, a sequence $F$ converges point-wise
            to $f$ if, for all $x\in\mathbb{R}$,
            $F_{n}(x)\rightarrow{f(x)}$.
            Uniform continuity requires that all of the
            points of the domain converge to $f(x)$ at
            the same speed. That is, given any $\varepsilon>0$
            there is an $N\in\mathbb{N}$ that works for
            all points. Point-Wise convergence does not
            have this property.
            \begin{definition}
                A sequence of real-valued functions $F$
                converges uniformly to a function
                $f$ if:
                \begin{equation*}
                    \forall_{\varepsilon>0}
                    \exists_{N\in\mathbb{N}}
                    \forall_{x\in{S}}:\forall_{n>N}
                    \Rightarrow
                    |f(x)-F_{n}(x)|<\varepsilon
                \end{equation*}
            \end{definition}
            That is, $f_{n}\rightarrow{f}$
            point-wise if for all $x$,
            $|f_{n}(x)-f(x)|\rightarrow{0}$ and
            $f_{n}\rightarrow{f}$ uniformly if
            $\sup\{|f_{n}(x)-f(x)|\}\rightarrow{0}$.
            \begin{example}
                Let
                $F:\mathbb{N}\times[0,1]\rightarrow\mathbb{R}$
                be defined by $F_{n}(x)=nx\exp(-nx)$.
                $F_{n}(x)\rightarrow{0}$ for all $x\in[0,1]$,
                and therefore $F$ converges point-wise to zero.
                Note that $F_{n}'(x)=(n-n^{2}x)\exp(-nx)$.
                This has a zero at $x=\frac{1}{n}$, so
                $F_{n}(x)$ has a maximum of $e^{-1}$. But then
                $\sup|F_{n}(x)-f(x)|=\sup|F_{n}(x)|=e^{-1}$.
                So $F_{n}(x)$ does not converge
                \textit{uniformly} to $0$. The
                convergence is only \textit{point-wise}.
            \end{example}
            \begin{example}
                Let $F_{n}(x)=n^{2}x\exp(-nx)$. Then
                $F_{n}(x)\rightarrow{0}$ for all
                $x\geq{0}$. But $F_{n}(x)$ has
                a maximum of $ne^{-1}$ at $x=\frac{1}{n}$.
                Thus $F_{n}(\frac{1}{n})\rightarrow\infty$.
                It is possible for a sequence
                of functions to converge point-wise to zero
                and for there to be a sequence such that
                $F_{n}(x_{n})\rightarrow\infty$. Uniform
                convergence does not allow this.
            \end{example}
            \begin{theorem}
                If $F$ is a sequence of
                real-valued continuous functions and if
                $F_{n}\rightarrow{f}$ uniformly, then
                $f$ is continuous.
            \end{theorem}
            The word ``uniformly,'' is crucial.
            This theorem is not necessarily true of
            point-wise converging functions. Let $F$ be
            defined on $[0,1]$ be
            $F_{n}(x)=x^{n}$. Then $F$ converges to
            $0$ if $x\ne{1}$, and $1$ if $x=1$. Thus,
            the limit function is discontinuous. This is
            possible because the convergence is
            point-wise and not uniform.
            \begin{theorem}
                    \label{thm:funct:Weak_Weierstrass_%
                           Approx_Theorem}
                    If $f:[0,1]\rightarrow\mathbb{R}$ is continuous,
                    and if $f(0)=f(1)=0$,
                    then there is a sequence of polynomials
                    $F$ such that $F_{n}\rightarrow{f}$
                    uniformly on $[0,1]$.
                \end{theorem}
            \begin{proof}
                    Extend $f$ to be zero outside of $[0,1]$. Let
                    $Q_{n}(x)=c_{n}(1-x^{2})^{n}$ on $[-1,1]$,
                    and choose $c_{n}$ such that
                    $\int_{-1}^{1}Q_{n}(x)dx=1$. So we have:
                    \begin{align*}
                        c_{n}\int_{-1}^{1}(1-x^{2})^{n}dx
                        &=2c_{n}\int_{0}^{1}(1-x^{2})^{n}dx
                        &
                        &\geq{2c_{n}}\int_{0}^{1}(1-x)^{n}dx\\
                        &=2c_{n}\int_{0}^{1}(1-x)^{n}(1+x)^{n}dx
                        &
                        &=\frac{2}{n+1}c_{n}
                    \end{align*}
                From this we have that $c_{n}\leq{n+1}$. Let
                $f_{n}(x)=\int_{0}^{1}f(t)Q_{n}(x-t)dx$.
                Then $f_{n}(x)$ is a polynomial. Note that
                $f(t)Q_{n}(x-t)$ is roughly zero when $t$ differs
                from $x$ and $n$ is large enough. So we have:
                \begin{equation*}
                    f_{n}(x)=\int_{0}^{1}f(t)Q_{n}(x-t)dt
                    \approx{f(x)}\int_{0}^{1}Q_{n}(x-t)dt=f(x)
                \end{equation*}
                The remainder of the proof is to quantify this.
                Since $f$ is zero outside of $[0,1]$, if
                we let $s=t-x$, then:
                \begin{align*}
                    f_{n}(x)&=\int_{-x}^{1-x}f(s+x)Q_{n}(s)ds
                    =\int_{-1}^{1}f(s+x)Q_{n}(s)ds\\
                    \Rightarrow|f_{n}(x)-f(x)|
                    &=\bigg|\int_{-1}^{1}f(x+t)Q_{n}(t)dt
                    -\int_{-1}^{1}f(x)Q_{n}(t)dt\bigg|\\
                    \Rightarrow|f_{n}(x)-f(x)|
                    &\leq\int_{-1}^{1}|f(x+t)-f(x)|Q_{n}(t)dt
                \end{align*}
                This comes for the fact that$\int_{-1}^{1}Q_{n}(t)=1$
                and from the integral version of the triangle
                inequaility.
                Suppose $\varepsilon>0$. Since $f$ is continuous
                on $[0,1]$, it is uniformly continuous. But
                if $f$ is uniformly continuous then there exists
                a $\delta>0$ such that
                $|f(x+t)-f(x)|<\frac{\varepsilon}{2}$ for all
                $t<\delta$. So we have:
                \begin{equation*}
                    |f_{n}(x)-f(x)|\leq
                    \int_{-1}^{-\delta}|f(x+t)-f(x)|Q_{n}(t)dt
                    +\int_{-\delta}^{\delta}|f(x+t)-f(x)|Q_{n}(t)dt
                    +\int_{\delta}^{1}|f(x+t)-f(x)|Q_{n}(t)dt
                \end{equation*}
                But $f$ is continuous on a closed and bounded
                set and therefore $f$ is bounded. Let $M$ be
                such a bound. Then $|(f(x+t)-f(x)|\leq{2M}$.
                We have:
                \begin{equation*}
                    |f_{n}(x)-f(x)|\leq
                    2M\int_{-1}^{-\delta}Q_{n}(t)dt
                    +\frac{\varepsilon}{2}
                    \int_{-\delta}^{\delta}Q_{n}(t)dt
                    +2M\int_{\delta}^{1}Q_{n}(t)dt
                \end{equation*}
                But for all $t\in[-1,-\delta]$,
                $Q_{n}(t)\leq{Q_{n}(-\delta)}$. Similarly for
                $t$ in $[\delta,1]$. Since $Q_{n}(t)$
                is an even function:
                \begin{equation*}
                    |f_{n}(x)-f(x)|\leq
                    4MQ_{n}(\delta)+
                    \frac{\varepsilon}{2}\int_{-1}^{1}Q_{n}(t)dt
                    =4MQ_{n}(\delta)+\frac{\varepsilon}{2}
                \end{equation*}
                But since $\delta>0$, $Q_{n}(\delta)\rightarrow0$.
                Therefore, there is an $N\in\mathbb{N}$ such that
                for all $n>N$,
                $|Q_{n}(\delta)|<\frac{\varepsilon}{8M}$.
                But then $4MQ_{n}(\delta)<\frac{\varepsilon}{2}$.
                Therefore, etc.
            \end{proof}
            Another way to put this is that if $f$ is continuous
            on $[a,b]$ and if $\varepsilon>0$, then there is
            a polynomial $P$ such that for all $x\in[0,1]$,
            $|f(x)-P(x)|<\varepsilon$. There is a generalization
            of this. The set of functions need not be
            polynomials. The set needs to be closed
            under addition, multiplication, and scalar
            multiplication, it must separate points,
            and must not take every point to zero.
            This is the Stone-Weierstrass theorem.
            This shows that continuous functions on compact sets
            can be approximated arbitrarily well by polynomials.
            Furthermore, any continuous function on a compact
            set can be approximated arbitrarily well by
            polynomials with rational coefficients. To see this,
            let $f[0,1]\rightarrow\mathbb{R}$ be continuous,
            and let $\varepsilon>0$. Then there is a
            polynomial $P$ such that
            $\sup|P(x)-f(x)|<\frac{\varepsilon}{2}$.
            Suppose $P$ is of degree $n$.
            As $\mathbb{Q}$ is dense
            in $\mathbb{R}$, for each coefficient
            $c_{k}$ of $P$ there is a $d_{k}\in\mathbb{Q}$
            such that
            $|c_{k}-d_{k}|<\frac{\varepsilon}{2n}$.
            Let $Q(x)=\sum{d_{k}x^{k}}$. Then:
            \begin{equation*}
                    |P(x)-Q(x)|<
                    \sum_{k=0}^{n}|c_{k}-d_{k}||x|^{n}
                    <\frac{\varepsilon}{2}
                \end{equation*}
            Thus, by the triangle inequality:
            $\sup|Q(x)-f(x)|<\varepsilon$. A set is called
            \textit{separable} if there if it
            contains a countable dense subset. $\mathbb{R}$
            is separable since $\mathbb{Q}$ is dense in
            $\mathbb{R}$, and $\mathbb{Q}$ is countable.
            The set of all continuous functions from
            $[0,1]$ to $\mathbb{R}$, which we label as
            $C(I,\mathbb{R})$, is also separable.
            Since any continuous function can be approximated
            arbitrarily well by a polynomial with rational
            coefficients, we can say the set of polynomials
            with rational coefficients is \textit{dense} in
            $C(I,\mathbb{R})$. But the set of polynomials with
            rational coefficients is countable. For all
            $N\in\mathbb{N}$, define $P_{N}$ as:
            \begin{equation*}
                    P_{N}=\Big\{\sum_{k=0}^{N}
                    q_{k}x^{k}:q_{k}\in\mathbb{Q},
                    q_{N}\ne{0}\Big\}
                \end{equation*}
            This is the set of all rational polynomials
            of degree $N$. It is countable since there is
            a one-to-one correspondence with
            the set $\mathbb{Q}^{N}$, and $\mathbb{Q}^{N}$
            is countable for all $N\in\mathbb{N}$. But the
            set of all rational polynomials is simply the
            union over all $P_{N}$. And the countable union
            of countably many disjoint sets is countable.
            Therefore, the set of all polynomials with
            rational coefficients is countable. Thus
            $C(I,\mathbb{R})$ is \textit{separable}. We need
            to be careful when we say \textit{dense} and
            \textit{separable}, for we are implicitly speaking
            of some sort of notion of \textit{closeness} on the
            sets. This all comes from the notion of
            \textit{metrics} and \textit{metric spaces}.
            \begin{theorem}
                    \label{thm:Funct:Weierstrass_%
                           Approx_on_unit_interval}
                    If $f:[0,1]\rightarrow\mathbb{R}$ is
                    continuous, then there is a sequence
                    of polynomials $F$ such that
                    $F_{n}\rightarrow{f}$ uniformly on $[0,1]$.
                \end{theorem}
            \begin{proof}
                    If $f:[0,1]\rightarrow\mathbb{R}$ be
                    continuous, let
                    $g(x)=xf(1)+(1-x)f(0)$. Then
                    $h(x)=f(x)-g(x)$ is a continuous function
                    such that $h(0)=h(1)=0$ and thus by
                    Thm.~\ref{thm:funct:Weak_Weierstrass_%
                              Approx_Theorem}
                    there is a sequence of polynomials
                    $P_{n}(x)$ such that
                    $P_{n}(x)\rightarrow{h(x)}$
                    uniformly on $[a,b]$.
                    But $g(x)$ is a polynomial and
                    $f(x)=h(x)+g(x)$. Therefore
                    $F_{n}(x)=P_{n}(x)+g(x)$ is a sequence
                    of polynomials and
                    $F_{n}(x)\rightarrow{f(x)}$
                    uniformly on $[0,1]$.
                \end{proof}
            \begin{theorem}[Weierstrass Approximation Theorem]
                If $f:[a,b]\rightarrow\mathbb{R}$ is a
                continuous function, then there is a sequence
                of polynomials $P$ such that
                $P_{n}\rightarrow{f}$ uniformly.
            \end{theorem}
            \begin{proof}
                If $f:[a,b]\rightarrow\mathbb{R}$ is
                continuous, define
                $g:[0,1]\rightarrow\mathbb{R}$ by
                $g(x)=f(\frac{x-a}{b-a})$. Then, since
                the composition of continuous functions
                is continuous, $g$ is a continuou function
                on $[0,1]$. But by the Weierstrass
                approximation theorem there is a sequence
                of polynomials $P_{n}(x)$ such that
                $P_{n}(x)\rightarrow{g(x)}$. Let
                $F_{n}(x)=P_{n}(bx+(1-x)a)$. Then
                $F_{n}(x)$ is a sequence of polynomials
                on $[a,b]$, and $F_{n}(x)\rightarrow{f(x)}$.
            \end{proof}
            An application of this is in the uniform
            approximation of continuous periodic functions
            by Cosines.
        \begin{theorem}
            If $f\in{C[0,\pi]}$ and $\varepsilon>0$,
            then there exists
            $a_{0},\hdots,a_{n}\in\mathbb{R}$ such that:
            \begin{equation*}
                |f(x)-\sum_{k=0}^{n}a_{k}\cos(kx)|
                <\varepsilon\quad{x\in[0,\pi]}
            \end{equation*}
        \end{theorem}
        \begin{proof}
            $\cos(x)$ is a bijective function when considered on
            the interval $[0,\pi]$. Thus we can consider the
            function $f(\cos^{-1}(x))$. But since $\cos(x)$ is
            continuous on $[0,\pi]$, $\cos^{-1}(x)$ is
            continuous on $[-1,1]$. And the composition of
            continuous functions is continuous. So
            $f(\cos^{-1}(x))$ is continuous. By the
            Weierstrass Approximation Theorem, there is a
            sequence of polynomials $P$ such that
            $P_{n}(x)\rightarrow{f(\cos^{-1}(x))}$. But then
            $P_{n}(\cos(x))\rightarrow{f(x)}$. But as $P_{n}(x)$
            is a polynomial, it is of the form
            $\sum_{k=0}^{n}a_{k}x^{k}$. But then:
            \begin{equation*}
                P_{n}(\cos(x))=\sum_{k=0}^{n}a_{k}\cos^{k}(x)    
            \end{equation*}
            It now suffices to show that
            $\cos^{k}(x)=\sum_{m=0}^{N}c_{m}\cos(mx)$ for
            suitable $c_{m}$. We prove by induction.
            The base case is trivial. Suppose it holds
            for some $k\in\mathbb{N}$.
            Then:
            \begin{equation*}
                \cos^{k+1}(x)=\cos(x)\cos^{k}(x)
                =\cos(x)\sum_{k=0}^{N}c_{k}\cos(kx)
            \end{equation*}
            Note that
            $\cos(x)\cos(kx)%
             =\frac{1}{2}\cos((k-1)x)+\frac{1}{2}\cos((k+1)x)$.
            So we have:
            \begin{equation*}
                \cos^{k+1}(x)
                =\frac{1}{2}\sum_{k=0}^{N}c_{k}
                \bigg(
                    \cos\Big((k+1)x\Big)+\cos\Big((k-1)x\Big)
                \bigg)
            \end{equation*}
            This completes the theorem.
        \end{proof}
        \subsection{Inequalities}
            \begin{theorem}[Young's Inequality]
                If $x$, $y>0$, $p>1$, and if
                $\frac{1}{p}+\frac{1}{q}=1$, then
                $xy\leq{\frac{1}{p}x^{p}+\frac{1}{q}y^{q}}$.
            \end{theorem}
            For $p=q=2$ this is easy, for:
            \begin{equation*}
                0\leq\frac{(x-y)^{2}}{2}=
                \frac{x^{2}+y^{2}}{2}-xy
            \end{equation*}
            A cute proof of the more general theorem comes from
            considering the area under the graph of
            $x^{p-1}$.
            \begin{definition}
                H\"{o}lder Conjugates are non-zero real numbers
                $p$, $q\in\mathbb{R}$ such that
                $p^{-1}+q^{-1}=1$
            \end{definition}
            \begin{theorem}[H\"{o}lder's Inequality]
                If $a:\mathbb{N}\rightarrow\mathbb{R}$ and
                $b:\mathbb{N}\rightarrow\mathbb{R}$
                are nonnegative
                sequences, if $p$ and $q$ are
                H\"{o}lder Conjugates, then:
                \begin{equation*}
                    \sum_{n=1}^{\infty}a_{n}b_{n}
                    \leq
                    \bigg(
                        \sum_{n=1}^{\infty}a_{n}^{p}
                    \bigg)^{1/p}
                    \bigg(
                        \sum_{n=1}^{\infty}b_{n}^{q}
                    \bigg)^{1/q}
                \end{equation*}
            \end{theorem}
            \begin{theorem}[Minkowski's Inequality]
                If $a:\mathbb{N}\rightarrow\mathbb{R}$
                and $b:\mathbb{N}\rightarrow\mathbb{R}$ are
                non-negative sequences, and if $p>1$,
                then:
                \begin{equation*}
                    \bigg(
                        \sum_{n=1}^{\infty}(a_{n}+b_{n})
                    \bigg)^{1/p}
                    \leq
                    \bigg(
                        \sum_{n=1}^{\infty}a_{n}^{p}
                    \bigg)^{1/p}
                    +
                    \bigg(
                        \sum_{n=1}^{\infty}b_{n}^{p}
                    \bigg)^{1/p}
                \end{equation*}
            \end{theorem}
            When $p=q=2$ this is often
            called the Cauchy-Schwartz inequality.
            That is,
            $|\mathbf{a}\cdot\mathbf{b}|%
             \leq\norm{\mathbf{a}}\norm{\mathbf{b}}$
    \section{Metric Spaces}
        \subsection{Basic Definitions}
            Functional analysis is concerned with normed spaces.
            This is a vector space $V$ with a function, called
            a norm, from $V$ to $[0,\infty)$. This is usually
            written $\norm{\mathbf{x}}$ for an element
            $\mathbf{x}\in{V}$. This norm must satisfy the
            folllowing for all $\mathbf{x}$, $\mathbf{y}\in{V}$:
            \begin{enumerate}
                \item $\norm{\mathbf{x}}=0$ if and only
                      if $\mathbf{x}=\mathbf{0}$
                      \hfill[Definiteness]
                \item $\norm{c\mathbf{x}}=|c|\norm{\mathbf{x}}$
                      for all $c\in\mathbb{R}$.
                      \hfill[Positiveness]
                \item $\norm{\mathbf{x}+\mathbf{y}}%
                       \leq\norm{\mathbf{x}}+\norm{\mathbf{y}}$
                      \hfill[Triangle Inequality]
            \end{enumerate}
            \begin{example}
                The following are normed spaces:
                \begin{enumerate}
                    \begin{multicols}{2}
                        \item $\mathbb{R}$ with $\norm{x}=|x|$
                        \item $\mathbb{R}^{n}$ with
                              $\norm{\mathbf{x}}%
                               =\sqrt{\sum_{k=1}^{n}x_{k}^{2}}$
                        \item $\mathbb{R}^{n}$ with
                              $\norm{\mathbf{x}}%
                               =(\sum_{k=1}^{n}x_{k}^{p})^{1/p}$
                        \item $\ell^{p}$: sequences $x_{n}$
                              such that
                              $\sum_{k=1}^{\infty}|x_{n}|^P%
                               <\infty$.
                        \item $\mathbb{R}^{n}$ with
                              $\norm{\mathbf{x}}%
                               =\max\{x_{1},\hdots,x_{n}\}$
                        \item $\ell^{\infty}$: Bounded sequences
                              with $\norm{v}=\sup\{v_{n}\}$
                    \end{multicols}
                \end{enumerate}
            \end{example}
            From the fact that $\ell^{\infty}$ is a normed space
            we have that the set of convergent sequences,
            again with the $\norm{}_{\infty}$ norm, is also
            a normed space. The set of null sequences, which
            is the set of sequences that converge to zero,
            is also a normed space. A stranger normed space
            is the set of all bounded continuous functions
            $f:S\rightarrow\infty$ with norm
            $\norm{f}=\sup\{|f(x)|\}$. Furthermore, the
            set of all integrable functions with
            bounded integrals, with norm
            $(\int_{S}|f|^{p})^{1/p}$. If you allow integral
            to mean Lebesgue Integrable, then this becomes
            a special space denoted $L^{p}(S)$. As a final
            example, the set of functions
            $f:[a,b]\rightarrow\mathbb{R}$ such that
            $(\int_{a}^{b}\sum_{k=0}^{n}|f^{(k)}|^{p})^{(1/p)}$
            exists, denoted $W^{n,p}([a,b])$ is called
            the Sobolev space.
            A lot of the things we wish to
            prove don't rely on the fact that all of these
            spaces are vector spaces. Really, we only care about
            the properties that the norm on the space has.
            What matters is that there's a set and a notion
            of distance on the set. This abstraction is the
            fundamental concept of a metric space.
            \begin{definition}
                A metric space is a set $S$ and a function
                $d:{X}\times{X}\rightarrow[0,\infty)$ such that:
                \begin{enumerate}
                    \item For all $x$, $y\in{X}$, $d(x,y)=0$
                          if and only if $x=y$.
                          \hfill[Definitenes]
                    \item For all $x$, $y\in{X}$,
                          $d(x,y)=d(y,x)$
                          \hfill[Symmetry]
                    \item For all $x$, $y\in{X}$,
                          $d(x,z)\leq{d(x,y)+d(y,z)}$
                          \hfill[Triangile Inequality]
                \end{enumerate}
            \end{definition}
            It turns out
            that we can actually write the following:
            \begin{definition}
                A metric space is a set $X$ and a function
                $d:{X}\times{X}\rightarrow\mathbb{R}$
                such that:
                \begin{enumerate}
                    \item $d(x,y)=0$ if and only if
                          $x=y$.
                    \item $d(x,z)\leq{d(x,y)+d(z,y)}$
                \end{enumerate}
            \end{definition}
            By writing the triangle inequality in this
            way, symmetry comes for free
            (The fact that $d(x,y)=d(y,x)$), as well
            as positivity (The fact that $d(x,y)\geq{0}$).
            Since it's easier to prove two things are
            true, rather than four things, it's nice to
            take this as the definition of a metric space,
            and then prove that the two definitions are
            equivalent.
            In a metric space $(X,d)$, $d$ is often called the
            \textit{distance function} or
            \textit{metric function}. It is meant to be an
            abstract mimicry of the absolute value function
            that is used with real numbers. Definiteness
            says the only point that is zero meters from a
            point $x$ is $x$ itself. Symmetry says the distance
            walking from $x$ to $y$ is the same as the distance
            walking from $y$ to $x$. The last rule stems from
            Euclidean geometry. It says walking from $x$ to $z$
            is shorter than (or equal to) walking from
            $x$ to $y$ and then $y$ to $z$. In Euclidean
            geometry equality is achieved only when
            $y$ lies between $x$ and $z$. In
            abstract metric spaces there may be no such
            thing as a \textit{line} between two points,
            so we need to be careful.
            \begin{example}
                $\mathbb{R}^{n}$ (for $1\leq{p}<\infty$):
                \begin{equation*}
                    d_{p}(\mathbf{x},\mathbf{y})=
                    \big(
                        \sum_{k=1}^{n}|x_{k}-y_{k}|
                    \big)^{1/p}
                    =\norm{\mathbf{x}-\mathbf{y}}_{p}
                \end{equation*}
            \end{example}
            \begin{example}
                In $\ell^{p}$, which are sequences for
                which
                $\sum_{k=1}^{\infty}|x_{k}|^{p}<\infty$,
                $d_{p}(x,y)$ forms a metric, as well
                as
                $d_{\infty}(x,y)=\sup\{|x_{k}-y_{k}|\}$,
                which is called the supremum norm.
            \end{example}
            \begin{example}
                $C(S,\mathbb{R})$, which is the
                set of continuous functions from
                $S$ to $\mathbb{R}$, letting
                $L^{p}(S)$ be the set of of functions
                such that:
                \begin{equation*}
                    \int_{S}|x(t)|^{p}<\infty
                \end{equation*}
                Then the following is a metric:
                \begin{equation*}
                    d_{p}(x,y)=
                    \bigg(
                        \int_{S}|x(t)-y(t)|^{p}dt
                    \bigg)^{1/p}
                \end{equation*}
                Also,
                $d_{\infty}(x,y)=\sup\{|x(t)-y(t)|\}$,
                which is called the supremum norm.
            \end{example}
            \begin{example}
                Let $C$ be the set of sequences such that
                $x_{n}\rightarrow{0}$. Then, with
                $d_{p}$, this forms a metric space.
                If $C_{0}$ is set of sequences with
                only finitely many non-zero terms,
                then
                $C_{0}\subset{C}\subset{\ell^{\infty}}$.
                Is there a sequence $x\in{C}$ such
                that, for all $1\leq{p}<\infty$,
                $x\notin{\ell^{p}}$.
            \end{example}
            Since the image of the metric function
            lies in $\mathbb{R}$,
            we may speak of \textit{convergence}
            in metric spaces.
            \begin{definition}
                A convergent sequence in a metric space
                $(X,d)$ is a sequence
                $x:\mathbb{N}\rightarrow{X}$ such that there
                is an $a\in{X}$ such that
                $d(a,x_{n})\rightarrow{0}$.
            \end{definition}
            \begin{definition}
                A limit of a sequence
                $x$ in a metric space $(X,d)$ is an
                $a\in{X}$ such that
                $d(x_{n},a)\rightarrow{0}$.
            \end{definition}
            Much like convergence in real numbers, limits
            in metric spaces are unique.
            \begin{theorem}
                \label{thm:Funct:Limit_of_Metric_Sequence_Unique}
                If $(X,d)$ is a metric space,
                $x:\mathbb{N}\rightarrow{X}$
                is a convergence sequence in $X$,
                and if $a$ and $b$ are limits of $x$,
                then $a=b$.
            \end{theorem}
            \begin{proof}
                For suppose not. As
                $(X,d)$ is a metric space, $d(a,b)>0$.
                Let $\varepsilon=\frac{d(a,b)}{4}$.
                Then, as $d(a,x_{n})\rightarrow{0}$
                and $\varepsilon>0$, there is an
                $N_{1}\in\mathbb{N}$ such that, for all
                $n>N_{1}$, $d(a,x_{n})<\varepsilon$. But,
                as $d(b,x_{n})\rightarrow{0}$ and
                $\varepsilon>0$, there is an $N_{2}$ such
                that, for all $n>N_{2}$,
                $d(b,x_{n})<\varepsilon$.
                Let $n=\max\{N_{1},N_{2}\}+1$.
                But then:
                \begin{equation*}
                    d(a,b)\leq{d(a,x_{n})+d(b,x_{n})}
                    <2\varepsilon=\frac{d(a,b)}{2}
                \end{equation*}
                A contradiction. Therefore, $a$ is unique.
            \end{proof}
            \begin{theorem}
                If $(X,d)$ be a metric space and if
                $x$, $y$, $z\in{X}$, then
                $|d(x,z)-d(y,z)|\leq{d(x,y)}$
            \end{theorem}
            \begin{proof}
                Suppose $d(x,z)\geq{d(y,z)}$.
                If $d(x,z)<d(y,z)$, the proof is
                symmetric. Thus we have:
                \begin{equation*}
                    |d(x,z)-d(y,z)|
                    =d(x,z)-d(y,z)
                    \leq{(d(x,y)+d(y,z))-d(y,z)}
                    =d(x,y)
                \end{equation*}
                Therefore,
                $|d(x,z)-d(y,z)|\leq{d(x,y)}$.
            \end{proof}
            \begin{theorem}
                If $(X,d)$ is a metric space
                and $x_{n}\rightarrow{a}$, then
                for all $b\in{X}$,
                $d(x_{n},b)\rightarrow{d(a,b)}$.
            \end{theorem}
            \begin{proof}
                For
                $|d(x_{n},b)-d(a,b)|\leq{d(x_{n},a)}%
                 \rightarrow{0}$.
            \end{proof}
            \begin{theorem}
                If $(V,\norm{})$ is a normed space
                and if $d:{V}\times{V}\rightarrow[0,\infty)$
                is defined by
                $d(\mathbf{x},\mathbf{y})%
                 =\norm{\mathbf{x}-\mathbf{y}}$,
                then $(V,d)$ is a metric space.
            \end{theorem}
            \begin{proof}
                In order:
                \begin{enumerate}
                    \item If $\norm{\mathbf{x}-\mathbf{y}}=0$,
                          then $\mathbf{x}=\mathbf{y}$.
                          Similarly,
                          $\norm{\mathbf{x}-\mathbf{x}}%
                           =\norm{\mathbf{0}}=0$.
                    \item $d(\mathbf{x},\mathbf{y})%
                           =\norm{\mathbf{x}-\mathbf{y}}%
                           =\norm{(-1)(\mathbf{y}-\mathbf{x})}%
                           =|-1|\norm{\mathbf{y}-\mathbf{x}}%
                           =\norm{\mathbf{y}-\mathbf{y}}%
                           =d(y,x)$
                    \item The triangle inequality follows
                          from the triangle inequality that
                          norms have.
                \end{enumerate}
            \end{proof}
            There are metric spaces that have nothing to do
            with vector spaces or norms. Metric spaces are
            a more abstract object. Every normed space
            has an associated metric space since there
            is the ``induced'' metric.
            \begin{example}
                Let $X$ be a set and let
                $d(x,y)=\begin{cases}%
                            0,&x=y\\%
                            1,&{x}\ne{y}%
                        \end{cases}$
                This is the discrete metric on $X$.
            \end{example}
            \begin{example}
                Let $X=\{a,b,c\}$, and
                $d(a,b)=1$, $d(b,c)=2$. What value
                must $d(a,c)$ have if $d$ is a metric on $X$?
                Consider the following table:
                \begin{table}[H]
                    \captionsetup{type=table}
                    \centering
                    \begin{tabular}{|c|c|c|c|}
                        \hline
                        $X$&a&b&c\\
                        \hline
                        a&0&1&?\\
                        \hline
                        b&1&0&2\\
                        \hline
                        c&?&2&0\\
                        \hline
                    \end{tabular}
                \end{table}
                This obeys everything except the triangle
                inequality. We must pick $d(a,c)$
                such that this is upheld.
                So we need the following:
                \begin{align*}
                    d(a,b)&\leq{d(a,c)+d(c,b)}&
                    d(a,c)&\leq{d(a,b)+d(b,c)}&
                    d(b,c)&\leq{d(b,a)+d(a,c)}\\
                    \Rightarrow{1}&\leq{2+d(a,c)}&
                    \Rightarrow{d(a,c)}&\leq{3}&
                    \Rightarrow{2}&\leq{1+d(a,c)}
                \end{align*}
                So we need $1\leq{d(a,c)}\leq{3}$.
                Pick $d(a,c)=2$.
                This makes $(X,d)$ a metric space.
            \end{example}
            \begin{example}
                Let $X=\mathbb{R}$ and $d(x,y)=|x-y|$.
                Then $(X,d)$ is a metric space.
            \end{example}
            \begin{example}
                $\mathbb{R}$ with
                $d(x,y)=|f(x)-f(y)|$, where
                $f:\mathbb{R}\rightarrow\mathbb{R}$
                is injective, is a metric space.
                Let $f$ be a real-valued function. Then
                from the triangle inequality
                \begin{equation*}
                    |f(x)-f(y)|\leq|f(x)-f(z)|+|f(z)-f(y)|
                \end{equation*}
                Therefore $d$ obeys the triangle inequality.
                It also obeys symmetry, for:
                \begin{equation*}
                    |f(x)-f(y)|=|(-1)(f(y)-f(x))|=|f(y)-f(x)|
                \end{equation*}
                The absolute value function is doing
                most of the work.
                But finally we require that
                $|f(x)-f(y)|=0$ if and only if
                $x=y$. But $|f(x)-f(y)|=0$ if and only
                if $f(x)=f(y)$. So we require that $f$
                is injective. If $f$ is not injective,
                then there exists $x_{1}$, $x_{2}$
                such that
                $x_{1}\ne{x_{2}}$ and yet
                $f(x_{1})=f(x_{2})$. But then
                $|f(x_{1})-f(x_{2})|=0$, contradicting the
                fact that this is a metric. If $f$ is
                injective, then this is a metric. Note
                injective functions need not be
                continuous, and can be very crazy.
            \end{example}
            \begin{example}
                $\mathbb{R}$ with
                $d(x,y)=|\tan^{-1}(x)-\tan^{-1}(y)|$ is a
                metric. Moreover, $d(x,y)<\pi$ for all
                $x,y\in\mathbb{R}$. Thus, we have found
                a metric that makes $\mathbb{R}$ a bounded
                set. As a fun fact, $x_{n}=n$ is a Cauchy
                sequence in this metric space, but
                this sequence does not converge to anything.
                Thus we've found a metric on
                $\mathbb{R}$ such that
                $(\mathbb{R},d)$ is not complete.
            \end{example}
            \begin{example}
            Can $d(x,y)=f(x-y)$ be a metric on $\mathbb{R}$
            if $f$ is differentiable? Not everywhere.
            $f$ can not be differentiable at the origin for
            $d(x,y)=f(x-y)$ to be a metric function, however
            $f$ can be differentiable everywhere else. Use
            $f(x)=|x|$ as an example.
            If $f(x-y)$ is a metric, $f$
            must be an even function. But
            then $f'(0)=0$. But $f(x-y)$ also must obey
            the triangle inequality. Therefore:
            \begin{equation*}
                f(2x)\leq{f(x)+f(x)}=2f(x)    
            \end{equation*}
            Define $h(x)$ by:
            \begin{equation*}
                h(x)=
                \left\{
                    \begin{array}{cr}
                    \frac{f(x)}{x},&x\ne{0}\\
                    0,&x=0
                    \end{array}\right.
            \end{equation*}
            Then, from
            the previous statement, $h(2x)\leq{h(x)}$.
            But then:
            \begin{equation*}
                h\Big(\frac{1}{2^{n}}\Big)\leq
                h\Big(\frac{1}{2^{n+1}}\Big)
            \end{equation*}
            But from L'H\^{o}pital's Rule,
            $h(x)\rightarrow{f'(0)}$ as $x\rightarrow{0}$.
            Therefore $h(1)\leq{f'(0)}$. But $h(1)>0$ since
            $f(x-y)$ is a metric, a contradiction.
            Therefore, $f$ can not be differentiable at
            the origin.
            \end{example}
        \subsection{Topology}
            \begin{definition}
                The open ball of radius $r>0$
                about a point $x$ in a metric space
                $(X,d)$ is the set
                $B_{r}(x)=\{y\in{X}:d(x,y)<r\}$
            \end{definition}
            The picture for this is a ``circle'' around the
            point $x$ or radius $r$. However, this circle
            can look very strange for weird metrics.
            \begin{example}
                If $X$ is a set and $d$ is the discrete metric,
                then $B_{r}(x)$ is either the point $x$
                (If $r\leq{1}$), or it is the entire set $X$.
            \end{example}
            \begin{example}
                With $X=\mathbb{R}$ and $d$ the standard metric
                $d(x,y)=|x-y|$, we have $B_{r}(x)$ is simply
                the open interval $(x-r,x+r)$.
            \end{example}
            \begin{example}
                \label{EXAMPLE:FUNCTIONAL:UNIT_BALLS_EXAMPLE}
                Let $X=\mathbb{R}^{2}$ and define
                $d_{p}(x,y)%
                 =(|x_{1}-y_{1}|^{p}+|x_{2}-y_{2}|^{p})^{1/p}$.
                For $p=2$, an open ball is a circle around
                the point $(x,y)$ of radius $r$. For $p=1$,
                we have ``diamonds'' around the point $x$.
                And for $p=\infty$ we have a square
                around $x$.
                Let $X=\mathbb{R}^{2}$ and let $d$ be the metric
                such that you can only travel parallel to the
                $y$ axis, or along the $x$ axis.
                Consider the unit balls in $(X,d)$
                about the following points:
                \begin{enumerate}
                    \begin{multicols}{4}
                        \item $(0,0)$
                        \item $(0,1)$
                        \item $(0, 0.5)$
                        \item $(0.5,0.5)$
                    \end{multicols}
                \end{enumerate}
                If $\mathbf{x}_{1}=(x_{1},y_{1})$ and
                $\mathbf{x}_{2}=(x_{2},y_{2})$, then we have:
                \begin{equation*}
                    d(\mathbf{x}_{1},\mathbf{x}_{2})=
                    \begin{cases}
                        |y_{2}-y_{1}|,&x_{1}=x_{2}\\
                        |x_{2}-x_{1}|+|y_{1}|+|y_{2}|,
                        &x_{1}\ne{x_{2}}
                    \end{cases}
                \end{equation*}
            About the point $(0,0)$, the unit ball
            is simply points
            $(x,y)$ such that $|x|+|y|<1$. This is a ``diamond.''
            About $(0,1)$, first note that to get to any point
            whose $x$ coordinate is not $0$, you first must travel
            the entirety of the $y$ axis. Since this length is
            already $1$, you can't go left or right
            on the $x$ axis.
            The unit ball is the line segment on the $y$ axis
            between $(0,0)$ and $(0,2)$. For the third one, if
            the $x$ coordinate changes, we have
            $0.5+|y|+|x|<1$, which implies
            $|y|+|x|<0.5$. This is again a diamond, but a
            smaller one. If the $x$ coordinate does not
            change, we have $|y-0.5|<1$. This is another
            line segment. Repeat the same arguments for the
            fourth coordinate. The diagrams are show in
            Fig.~\ref{FUNCTIONAL:HOMEWORK:2:PROBLEM:4:FIGURES}.
        \begin{figure}[H]
            \centering
            \captionsetup{type=figure}
            \subimport{../../../tikz/}
                      {Functional_Analysis_Fall_2018_%
                       HW_2_Problem_4}
            \caption{Figures for Example
                     \ref{EXAMPLE:FUNCTIONAL:UNIT_BALLS_EXAMPLE}.}
            \label{FUNCTIONAL:HOMEWORK:2:PROBLEM:4:FIGURES}
        \end{figure}
            \end{example}
            If you have a vector space and a norm on it,
            then the open balls about a point will have the
            property of convexity. Convexity is a vector space
            property, given two points the ``line'' between the
            two remains in the set. Metric spaces have no such
            notion. Since the balls of $\norm{}_{p}$ are not
            convex with $p<1$, we have that $\norm{}_{p}$ is
            a metric on $\mathbb{R}^{n}$
            if and only if $p\geq{1}$.
            \begin{definition}
                An open subset of a metric space
                $(X,d)$ is a set $S\subset{X}$ such that,
                for all $x\in{S}$, there is an
                $r>0$ such that
                $B_{r}(x)\subset{S}$.
            \end{definition}
            \begin{example}
                If $(X,d)$ is a metric space, then
                $X$ is open and $\emptyset$ is open
                (Vacuously true).
            \end{example}
            \begin{theorem}
                If $(X,d)$ is a metric space, $x\in{X}$,
                and $r>0$, then $B_{r}(x)$ is an open
                subset of $X$.
            \end{theorem}
            \begin{proof}
                If $z\in{B_{r}(x)}$, let $t=d(x,z)$.
                Then $0\leq{t}<r$. Let $r'=r-t$.
                But if $y\in{B_{r'}(z)}$, then
                $d(x,y)\leq{d(x,z)+d(y,z)}<t+r'=t+r-t=r$.
                Therefore $B_{r'}(z)\subset{B_{r}(x)}$.
            \end{proof}
            \begin{theorem}
                A finite intersection of open sets is open.
            \end{theorem}
            \begin{proof}
                If $\mathcal{U}_{1},\hdots,\mathcal{U}_{n}$
                are open and if
                $x\in\cap_{k=1}^{n}\mathcal{U}_{k}$, then there
                exists $r_{1},\hdots,r_{n}$ such that
                $B_{r_{i}}(x)\subset\mathcal{U}_{i}$. Let
                $r=\min\{r_{1},\hdots,r_{n}\}$. Then
                $B_{r}(x)\subset\cap_{k=1}^{n}\mathcal{U}_{i}$
            \end{proof}
            \begin{theorem}
                Arbitrary unions of open sets are open.
            \end{theorem}
            Infinite intersections need not be open.
            The proof above would fail since the
            $r_{i}$ can form a sequence tending to zero.
            But indeed, let $X=\mathbb{R}$ and let
            $d(x,y)=|x-y|$, and take
            $\mathcal{U}_{n}=(-\frac{1}{n},\frac{1}{n})$.
            Then all of the $\mathcal{U}_{n}$ are open,
            yet the intersection, which is the set $\{0\}$,
            is not open. All of this mumbo-jumbo creates
            the more general notion of a topological space.
            \begin{definition}
                A topological space is a set $X$ and a
                subset $\tau\subset\mathcal{P}(X)$ such that:
                \begin{enumerate}
                    \item $\emptyset,X\in\tau$
                    \item Finite intersections of sets in $\tau$
                          are also sets in $\tau$.
                    \item Arbitrary unions of sets in $\tau$
                          are also sets in $\tau$.
                \end{enumerate}
            \end{definition}
            Here, $\mathcal{P}(X)$ denotes the \textit{power set}
            of $X$. This is the set of all subsets of $X$.
            The notion of a topological space generalizes the
            notion of a metric space. There is no notion of
            distance in such spaces, and things can be weird.
            There are topological spaces that have no metric
            associated with them.
            \begin{definition}
                An open subset of a topological space
                $(X,\tau)$ is a set $\mathcal{U}\in\tau$.
            \end{definition}
            \begin{definition}
                A function from a metric space
                $(X,d_{X})$ to a metric space $(Y,d_{Y})$
                continuous at $x\in{X}$ is a function
                $f:X\rightarrow{Y}$ such that
                for all $\varepsilon>0$ there is
                a $\delta>0$ such that for all
                $x_{0}\in{X}$ such that
                $d_{X}(x,x_{0})<\delta$, we have
                $d_{Y}(f(x),f(x_{0})<\varepsilon$
            \end{definition}
            \begin{theorem}
                If $(X,d)$ is a metric space,
                $y\in{X}$, then
                $f:X\rightarrow\mathbb{R}$ defined by
                $f(x)=d(x,y)$ is uniformly continuous.
            \end{theorem}
            A surprising theorem, and the entire
            basis of the study of topology, goes as
            follows:
            \begin{theorem}
                If $(X,d_{x})$ and $(Y,d_{Y})$
                are metric spaces, then
                $f:X\rightarrow{Y}$ is continuous
                at $x\in{X}$ if and only if
                for all open subsets
                of $S\subset{Y}$ such that
                $f(x)\in{S}$, $f^{-1}(S)$ is an
                open subset of $X$.
            \end{theorem}
            This allows us to talk about continuous
            functions without a notion of metric.
            Thus, for topological spaces, this is
            the \textit{definition} of continuity.
            When the space we're discussing is a
            metric space, this theorem shows that the
            definition from topology and the defintition
            from real analysis are in fact equivalent.
            \begin{theorem}
                A function $f:X\rightarrow{Y}$ between
                metric spaces is continuous at a point
                $x\in{X}$ if and only if for all
                sequences $x_{n}$ such that
                $d_{X}(x,x_{n})\rightarrow{0}$, we have
                $d_{Y}(f(x),f(x_{n})\rightarrow{0}$.
            \end{theorem}
            We now have three different ways to talk
            about continuity. Topological spaces can be
            nastier, however. We saw in
            Thm.~\ref{thm:Funct:Limit_of_Metric_Sequence_Unique}
            that the limit of a convergent sequence in a
            metric space is unique.
            This is not true in a topological space and there
            are topological spaces with sequences
            which converge to every point in the
            space simultaneously. Indeed, it may be impossible
            to distinguish two points in a topological
            space. The ability to
            ``Separate,'' points is special.
            Hausdorff spaces can, but
            we'll save that for topology.
        \subsubsection{Closed Sets}
            \begin{definition}
                A limit point of a subset
                $S\subset{X}$ of a metric space
                $(X,d)$ is a point $a\in{X}$ such
                that there is a sequence
                $x:\mathbb{N}\rightarrow{S}$ such that
                $d(a,x_{n})\rightarrow{0}$.
            \end{definition}
            \begin{definition}
                A closed subset of a metric space $(X,d)$
                is a set $S$ such that for all $x\in{X}$ such
                that $x$ is a limit point of $S$, $x\in{S}$.
            \end{definition}
            This says that if $S$ is closed, and
            $x$ is a sequence in $S$ such
            that $x_{n}\rightarrow{a}$, then
            $a\in{S}$.
            \begin{example}
                In $\mathbb{R}$, with the standard
                metric, $(a,b)$ is open,
                $\mathbb{R}$ is open (and closed),
                $[a,b]$ is closed,
                $[a,\infty)$ is closed,
                $[a,b)$ is neither closed nor open.
            \end{example}
            \begin{example}
                If $X=(0,1)$, and
                $d(x,y)=|x-y|$, then
                $(0,1)$ is closed. This is because
                there is no sequence that converges
                to a point in the space whose limit
                is not in the space. There are no sequences
                in $X$ which converge to zero or one since,
                as far as $X$ is concerned,
                neither or these points exist.
            \end{example}
            \begin{theorem}
                If $(X,d)$ is a metric space,
                then a subset $S\subset{X}$ is open
                if and only if $X\setminus{S}$ is closed.
            \end{theorem}
            \begin{proof}
                Suppose $S$ is open, and let
                $x_{n}$ be a sequence in $S^{c}$.
                Suppose $x_{n}\rightarrow{x}$ and
                $x\in{S}$. But $S$ is open, and thus
                there is an $\varepsilon>0$ such that
                $B_{\varepsilon}(x)\subset{S}$.
                But $x_{n}\rightarrow{x}$, and thus
                this is an $N\in\mathbb{N}$ such that
                for all $n>N$, $d(x,x_{n})<\varepsilon$.
                But then for all $n>N$,
                $x_{n}\in{B_{\varepsilon}(x)}$. But
                $x_{n}\in{S^{c}}$, a contradiction.
                Therefore, $S^{c}$ is closed. On the
                other hand, if $S^{c}$ is closed
                and there is an $x\in{S}$ such that
                for all $r>0$,
                $B_{r}(x)\cap{S}\ne\emptyset$, then
                for all $n\in\mathbb{N}$ there is
                an $x_{n}\in{S^{c}}$ such that
                $d(x,x_{n})<\frac{1}{n}$. But then
                $x_{n}\rightarrow{x}$, and therefore
                $x\in{S^{c}}$. But $x\in{S}$,
                a contradiction. Thus, $S$ is open.
            \end{proof}
            In topology we take the definition of
            closed sets to be the compliment of open
            sets. This theorem shows that the
            topological definition is equivalent when we
            consider metric spaces.
            \begin{definition}
                The closure of a subset
                $S$ of a metric space
                $(X,d)$, denoted $\overline{S}$,
                is the set of all
                limit points of $S$.
            \end{definition}
            \begin{theorem}
                If $(X,d)$ is a metric space, if
                $S\subset{X}$, and if
                $\Delta$ is the set of all closed subsets
                $\mathcal{C}\subset{X}$ such that
                $S\subset\mathcal{C}$, then:
                $\overline{S}=
                 \bigcap_{\mathcal{C}\in\Delta}
                 \mathcal{C}$
            \end{theorem}
            Thus we may loosely say that
            the closure of a set $S$ is the
            ``Smallest,'' closed set that contains $S$.
            \begin{definition}
                The closed ball of radius $r>0$ about
                a point $x$ in a metric space
                $(X,d)$ is the set:
                \begin{equation*}
                    \overline{B}_{r}(x)=
                    \{y\in{X}:d(x,y)\leq{r}\}
                \end{equation*}
            \end{definition}
            There exists metric spaces $(X,d)$
            such that
            $\overline{B}_{r}(x)\ne\overline{B_{r}(x)}$.
            For take the discrete metric, $r=1$.
            Then the closure of $B_{1}(x)$ is simply
            the point $x$. However, the closed ball
            $\overline{B}_{1}(x)$ is the entire space.
            Metric spaces can be very weird like this.
            They have a property, that given a nested
            sequence of closed balls whose radius
            tends to zero, there is precisely one
            point that lies in the intersection. However,
            if the radius does not tend to zero it is
            possible that the intersection is empty.
            This is very counter-intuitive.
            \begin{definition}
                A dense subset of a metric space $(X,d)$
                is a set $S\subset{X}$ such that
                $\overline{S}=X$.
            \end{definition}
            A subset $S$ is dense in $X$ if every point
            in $X$ can be approximated arbitrarily well
            by points in $S$. For any point $a\in{X}$
            there is a sequence $x\in{S}$
            such that $x_{n}\rightarrow{a}$. The
            classic example is $\mathbb{Q}$ and
            $\mathbb{R}$. Every real number can be
            approximated arbitrary well by a rational
            number. To see this, just take the continued
            fraction of a real number and stop once
            the approximation is less than
            $\varepsilon$. When we say $\mathbb{Q}$ is
            dense in $\mathbb{R}$, we of course mean with
            respect to the standard metric on $\mathbb{R}$.
            $\mathbb{Q}$ is \textbf{not} dense in
            $\mathbb{R}$ with respect to the discrete metric.
            Indeed, if $d$ is the discrete metric on $X$,
            then $S\subset{X}$ is dense in $X$ if and only if
            $S=X$.
            \begin{example}
                $\mathbb{Q}$ is dense in $\mathbb{R}$
                with respect to $d_{p}$ for all
                $p\geq{1}$. This includes
                $d(x,y)=|x-y|$.
            \end{example}
            \begin{example}
                The set of polynomials on the interval
                $[a,b]$ are dense in the set of
                continuous functions on $[a,b]$ with
                respect to the $d_{\infty}$ metric.
                This comes from Weierstrass's Theorem.
            \end{example}
            \begin{example}
                The set of polynomials on $[a,b]$
                is dense in the set of continuous
                functions on $[a,b]$ with respect to
                the $d_{p}$ metric, for $p\geq{1}$. This
                is because:
                \begin{align*}
                    d_{p}(P,x)&=
                    \Big(
                        \int_{a}^{b}|P(t)-x(t)|^{p}\diff{t}
                    \Big)^{1/p}
                    &
                    &=\Big(
                        d_{\infty}(P,x)^{p}\int_{a}^{b}\diff{t}
                    \Big)^{1/p}\\
                    &\leq\Big(\int_{a}^{b}
                        |\max\{P(t)-x(t)\}|^{p}\diff{t}
                    \Big)^{1/p}
                    &
                    &=(b-a)^{1/p}d_{\infty}(P,x)
                \end{align*}
            \end{example}
            \begin{example}
                The continuous functions are not dense
                in the set of integrable functions,
                with respect to the supremum metric
                $d_{\infty}$. This is more or less
                because integrable functions can
                be discontinuous, or have jumps. This
                means, with respect to $d_{\infty}$,
                that no continuous functions could
                approximate such a discontinuous function
                arbitrary well.
            \end{example}
            \begin{definition}
                A separable metric space
                is a metric space $(X,d)$ with
                a countable dense subset $S$.
            \end{definition}
            \begin{example}
                $\mathbb{R}$ is separable, with
                the standard metric, since
                $\mathbb{Q}$ is countable and also
                dense in $\mathbb{R}$.
            \end{example}
            \begin{example}
                The set of continuous functions on
                $[a,b]$ is separable. For
                take the set of polynomials with
                rational coefficients. This can
                be seen as a countable union of
                countably many elements. For let
                $P_{N}$ be the set of polynomials
                of degree $N$ with rational
                coefficients. This is countable,
                and the set of all polynomials with
                rational coefficients is simply the
                union of $P_{N}$ over all $N$. This
                is dense in the set of polynomials,
                and the set of polynomials is dense
                in $C[a,b]$, and thus
                the set of polynomials with rational
                coefficients is dense in $C[a,b]$. Thus
                $C[a,b]$ is separable.
            \end{example}
            \begin{example}
                $\ell^{p}$ is separable with the
                $d_{p}$ metric, simply use elements
                with rational entries. That is,
                sequences of rational numbers.
            \end{example}
            \begin{example}
                $\ell^{p}$ with the $d_{\infty}$ metric
                is NOT separable. Consider the real
                numbers in $(0,1)$.
            \end{example}
        \subsection{Completeness}
            \begin{definition}
                A complete metric space is a metric
                space $(X,d)$ such that every
                Cauchy sequence $x_{n}$
                in $X$ converges to a point in $X$
                with respect to $d$.
            \end{definition}
            Recall that a sequence $x_{n}$ is Cauchy if
            $\forall_{\varepsilon>0}\exists_{N\in\mathbb{N}}:%
             \forall_{n,m>N},d(x_{n},x_{m})<\varepsilon$.
            Convergence with respect to $d$ means that
            $d(x,x_{n})\rightarrow{0}$.
            \begin{example}
                $\mathbb{R}$ with the standard metric
                $d(x,y)=|x-y|$ is complete.
            \end{example}
            \begin{example}
                $(\mathbb{R}^{n},d_{p})$ is also complete
                for all $n\in\mathbb{N}$.
            \end{example}
            Completeness is both a property of the set
            and the metric itself. It is not a topological
            property.
            \begin{example}
                $(\mathbb{R},d)$, where
                $d(x,y)=|\tan^{-1}(x)-\tan^{-1}(y)|$
                is \textit{not} complete. For let
                $x_{n}=n$. This is a Cauchy sequence,
                as one can see from the graph
                of $\tan^{-1}(x)$. That is, because
                $\tan^{-1}(x)\rightarrow{\pi/2}$,
                $x_{n}=n$ is a Cauchy sequence in this
                metric. Being even more rigorous, let
                $\varepsilon>0$ and
                $N=\ceil{\tan(\pi/2-\varepsilon)}$.
                Then, for all $n,m>N$,
                $d(x_{n},x_{m})%
                 =|\tan^{-1}(n)-\tan^{-1}(m)|%
                 <|\pi/2-\tan^{-1}(\min\{n,m\})|%
                 <|\pi/2-(\pi/2-\varepsilon)|%
                 =\varepsilon$. But $x_{n}$ does not
                converge. For suppose not.,
                Suppose $x_{n}=n\rightarrow{x}$.
                Then for $n>x+1$,
                $d(x_{n},x)=|\tan^{-1}(n)-\tan^{-1}(x)|%
                 <|\tan^{-1}(x+1)-\tan^{-1}(x)|$,
                so $d(x_{n},x)\not\rightarrow{0}$.
                The sequence does not converge.
            \end{example}
            Let $X=\mathbb{R}\cup\{-\infty,\infty\}$.
            Let $d:X\times{X}\rightarrow\mathbb{R}$
            be defined by
            \begin{align*}
                d(x,y)
                &=|\tan^{-1}(x)-\tan^{-1}(y)|
                &
                d(x,\infty)
                &=\frac{\pi}{2}-\tan^{-1}(x)\\
                d(-\infty,x)
                &=\frac{\pi}{2}+\tan^{-1}(x)
                &
                d(-\infty,\infty)&=\pi
            \end{align*}
            Then $d$ is a metric on $X$, and moreover
            $(X,d)$ is complete. The counterexample
            we found for $(\mathbb{R},d)$ has been
            ``filled,'' in a sense. The hole is
            no longer there. The sequence $x_{n}=n$
            now converges to $\infty$. Somewhat
            unsurpringly, $\mathbb{R}$ is
            dense in $X$, with respect to
            $d$. Every element in $X$ is the limit of
            a sequence of elements in $\mathbb{R}$.
            \begin{definition}
                A completion of a metric space
                $(X,d)$ is a complete metric space
                $(\tilde{X},\tilde{d})$
                such that
                $X\subset{\tilde{X}}$ and
                the restriction of
                $\tilde{d}$ onto $X$ is equal
                to $d$.
            \end{definition}
            \begin{theorem}
                Every metric space has
                a completion.
            \end{theorem}
            \begin{definition}
                An isometry between
                metric spaces
                $(X,d_{X})$ and
                $(Y,d_{Y})$ is a function
                $f:X\rightarrow{Y}$ such that
                $d_{X}(x,y)=d_{Y}(f(x),f(y))$
                for all $x,y\in{X}$.
            \end{definition}
            \begin{definition}
                Isometric metric spaces are metric spaces
                with an isometry between them.
            \end{definition}
            \begin{theorem}
                If $(X,d)$ is a metric space
                and $(\tilde{X}_{1},\tilde{d}_{1})$
                and $(\tilde{X}_{2},\tilde{d}_{2})$
                are completions of $(X,d)$, then
                $(\tilde{X}_{1},\tilde{d}_{1})$
                and $(\tilde{X}_{2},\tilde{d}_{2})$
                are isometric.
            \end{theorem}
            This says the completion of a metric space is
            unique up to isometry.
            The Lebesgue space $L^{p}(S)$
            can be defined to be the completion of
            $C(S)$ with respect to the $d_{p}$ metric.
            \begin{theorem}
                $(C(S),d_{\infty})$ is complete.
            \end{theorem}
            \begin{proof}
                Suppose $x_{n}$ is a Cauchy sequence
                and let $\varepsilon>0$. As $x_{n}$ is
                Cauchy, there exists $N\in\mathbb{N}$
                such that for all $n,m>N$,
                $\sup|x_{m}(t)-x_{n}(t)|<\frac{\varepsilon}{3}$.
                But then for all $t\in{S}$,
                $|x_{m}(t)-x_{n}(t)|<\frac{\varepsilon}{3}$,
                for all
                $n,m>N$. That is, if $x_{n}$ is
                a Cauchy sequence in $(C(S),d_{\infty})$,
                then it is a Cauchy sequence in
                $(\mathbb{R},d_{1})$. But
                $(\mathbb{R},d_{1})$ is complete, and
                therefore, for all $t\in{S}$, there is
                an $x(t)$ such that
                $x_{n}(t)\rightarrow{x(t)}$ with respect
                to the $d_{1}$ metric on $\mathbb{R}$. We
                now need to show that $x(t)$ is a continuous
                function. That is, that
                $x(t)\in{C(S)}$. Finally we need to show that
                $x_{n}\rightarrow{d}$ with respect to
                $d_{\infty}$. We need to show that
                for all $\varepsilon>0$ and all $t\in{S}$
                there is a $\delta>0$
                such that for all $|t-t_{0}|<\delta$,
                $|x(t)-x(t_{0})|<\varepsilon$. But for
                all $n,m>N$,
                $\sup\{x_{n}(t)-x_{m}(t)\}<\frac{\varepsilon}{3}$.
                Taking the limit on $m$, we have
                $|x(t)-x_{n}(t)|<\frac{\varepsilon}{2}$.
                But $x_{n}(t)$ is continuous, and thus
                there exists $\delta>0$ such that
                for all $|t-t_{0}|<\delta$,
                $|x_{n}(t)-x_{n}(t_{0})|<\frac{\varepsilon}{3}$.
                But
                $|x(t)-x(t_{0})|\leq%
                  |x(t)-x_{n}(t)|%
                 +|x_{n}(t)-x_{m}(t)|%
                 +|x(t_{0})-x_{n}(t_{0})$
                But
                $|x(t_{0})-x_{n}(t_{0})|<%
                 \sup\{|x(t)-x_{n}(t)|\}<\frac{\varepsilon}{3}$,
                and therefore
                $|x(t)-x(t_{0})|<\varepsilon$.
                So $x(t)$ is continuous.
            \end{proof}
            The Weierstrass Approximation Theorem says that,
            for closed finite intervals $S$,
            $(C(S),d_{\infty})$ is the completion
            of the set of polynomials with respect to
            the $d_{\infty}$ metric. On the other hand,
            $(C[0,1],d_{p})$ is not complete when
            $1\leq{p}<\infty$. For define the following:
            \begin{equation*}
                H(x)=
                \begin{cases}
                    0,&0\leq{x}\leq{\frac{1}{2}}\\
                    1,&\frac{1}{2}<x\leq{1}
                \end{cases}
            \end{equation*}
            This is discontinuous and cannot be
            approximated arbitrarily well
            by any continuous function. However, the
            \textit{area} underneath $H$ can be approximated
            arbitrarily well be continuous functions. For define:
            \begin{equation*}
                x_{n}(t)=
                \begin{cases}
                    0,&0\leq{x}\leq{\frac{1}{2}-\frac{1}{n}}\\
                    n(x-\frac{1}{2}+\frac{1}{n}),
                    &\frac{1}{2}-\frac{1}{n}\leq{x}
                     \leq{\frac{1}{2}}\\
                    1,&\frac{1}{2}<{x}\leq{1}
                \end{cases}
            \end{equation*}
            Then the area under $x_{n}(t)$
            is $\frac{1}{2}+\frac{1}{2n}$, and thus
            $d_{1}(x_{n}(t),x_{m}(t))%
             =|\frac{1}{2m}-\frac{1}{2n}|$,
            and therefore $x_{n}(t)$ is a Cauchy sequence.
            But $x_{n}(t)$ does not converge in
            $(C[0,1],d_{1})$. For suppose not, suppose
            $x_{n}(t)\rightarrow{x(t)}$, and
            $x(t)\in{C[0,1]}$.
            If $x(1/2)\geq{1/2}$, then, as $x(t)$ is
            continuous, there is a $\delta>0$ such that
            for all $|t-1/2|<\delta$,
            $x(t)>1/4$. But then
            $d(x_{n},x)=\int_{0}^{1}|x(t)-x_{n}(t)|dt%
            \geq\int_{1/2-\delta/2}^{1/2}|x(t)-x_{n}(t)|dt$.
            But $|x|=|(x-y)+y|\leq{|x-y|+|y|}$,
            and thus
            $|x|-|y|\leq{|x-y|}$. From this we have
            $d(x_{n}(t),x(t))\geq%
             \int_{1/2-\delta/2}^{1/2}(x(t)-x_{n}(t))dt%
             >\int_{1/2-\delta/2}^{1/2}\frac{1}{4}dt%
             -\int_{0}^{1/2}x_{n}(t)dt%
             =\frac{1}{4}\delta-\frac{1}{2n}%
             \rightarrow{\frac{1}{4}}\delta$.
            But then $d(x_{n}(t),x(t))\not\rightarrow{0}$.
            Therefore $x_{n}(t)$ does not converge.
            \begin{theorem}
                If $1\leq{p}<\infty$, then
                $(\ell^{p},d_{p})$ is complete.
            \end{theorem}
            \begin{proof}
                Let $x_{n}$ be a Cauchy sequence
                in $(e\ell^{p},d_{p})$,
                $x_{n}=x_{n}(1),x_{n}(2),\hdots,x_{n}(k),\hdots$
                Then, for $n,m\in\mathbb{N}$,
                $d_{p}(x_{n},x_{m})%
                 =(%
                    \sum_{k=0}^{\infty}|x_{n}(k)-x_{m}(k)|^{p}%
                  )^{1/p}$
                As $x_{n}$ is Cauchy, for all 
                $\varepsilon>0$ there is an $N\in\mathbb{N}$
                such that for all $n,m>N$,
                $d_{p}(x_{n},x_{m})<\varepsilon$.
                But then, for all $n,m>N$ and all
                $k\in\mathbb{N}$,
                $|x_{n}(k)-x_{m}(k)|^{p}<d_{p}(x_{n},x_{m})^{P}%
                 <\varepsilon^{p}$.
                But then
                $|x_{n}(k)-x_{m}(k)|<\varepsilon$. Therefore
                $x_{n}(k)$ is a Cauchy sequence in
                $(\mathbb{R},d)$, and this metric space is
                complete. Therefore, for all $k\in\mathbb{N}$,
                there is a $z_{k}$ such that
                $x_{n}(k)\rightarrow{z_{k}}$. We now need to
                show that $z_{k}$ is an element of
                $\ell^{p}$ and that
                $x_{n}\rightarrow{z_{k}}$ with respect to
                the $d_{p}$ metric. For let $N\in\mathbb{N}$.
                Then
                $\sum_{k=0}^{N}|x_{n}(k)-x_{m}(k)|^{p}%
                 \leq{\sum_{k=0}^{\infty}|x_{n}(k)-x_{m}(k)|^{p}}%
                 <\varepsilon^{p}$. Taking the limit on $m$,
                we have
                $\sum_{k=0}^{N}|z_{k}-x_{n}(k)|<\varepsilon^{p}$.
                The reason we have written a finite sum is to
                avoid getting into trouble with limits. An
                infinite sum is itself a limit, and taking
                limits of limits can get very messy very easily.
                For example,
                $f(n,m)=\frac{m}{n+m}$. Taking the limit on
                $m$ first results in $1$, whereas taking the
                limit on $n$ first gives you $0$.
                That is,
                $\lim_{n}\lim_{m}f(n,m)%
                 \ne\lim_{m}\lim_{n}f(n,m)$.
                You have to
                be careful when considering limits of limits.
                With this we have shown that
                $z_{k}-x_{n}(k)\in\ell^{p}$ for all
                $n\in\mathbb{N}$. But $x_{n}\in\ell^{p}$,
                and $\ell^{p}$ is closed under addition.
                Therefore $z_{k}\in\ell^{p}$. But also,
                for $n>N$, we have
                $d_{p}(x_{n},z)<\varepsilon$. Thus,
                $x_{n}$ converges.
            \end{proof}
            \begin{theorem}
                If $(X,d)$ is complete and $S$ is a closed
                subset of $X$, then $(S,d_{S})$ is complete,
                where $d_{S}$ is the restriction of
                $d$ onto $S$.
            \end{theorem}
            \begin{proof}
                Let $x_{n}$ be a Cauchy sequence in $S$. Then
                $x_{n}\rightarrow{x}$, $x\in{X}$,
                since $x_{n}$ is Cauchy in $X$
                and $X$ is complete. Since $S$ is closed,
                $x\in{S}$. Therefore, etc.
            \end{proof}
            \begin{theorem}
                If $(X,d)$ is complete and
                $S\subset{X}$ is not closed,
                then $(S,d_{S})$ is not complete.
            \end{theorem}
            \begin{proof}
                If $S$ is not closed then there
                is a convergent sequence $x_{n}\in{S}$
                whose limit it not in $S$. But
                then $x_{n}$ is a Cauchy sequence in
                $X$, and therefore is also a
                Cauchy sequence in $S$, but
                $x_{n}$ does not converge in $S$.
                Therefore $(S,d_{S})$ is not complete.
            \end{proof}
            Recall that $c_{0}$ is the set of sequences which
            tend to zero. That is, it is the set of
            null sequences.
            \begin{theorem}
                $c_{0}$ is a closed subset of
                $(\ell^{\infty},d_{\infty})$
            \end{theorem}
            \begin{proof}[proof 1]
                Let $x_{n}$ be a sequence in $c_{0}$
                that converges to $z\in\ell^{\infty}$
                with respect to $d_{\infty}$.
                Then
                $\sup\{|x_{n}(k)-z_{k}|\}\rightarrow{0}$.
                We need to show that $z\in{c_{0}}$.
                Let $\varepsilon>0$. Let $N_{1}\in\mathbb{N}$
                be such that
                $n>N$ implies
                $\sup\{|x_{n}(k)-z_{k}\}<\frac{\varepsilon}{2}$.
                But $x_{n}\in{c_{0}}$ for all $n$, and thus
                $x_{n}(k)\rightarrow{0}$ as $k\rightarrow\infty$.
                Thus, there is an $N_{2}\in\mathbb{N}$
                such that $n>N_{2}$ implies
                $|x_{n}(k)<\varepsilon$.
                But then for $n>\max\{N_{1},N_{2}\}$,
                $|z_{k}|\leq|z_{k}-x_{n}(k)|+|x_{n}(k)|%
                 <\varepsilon$.
            \end{proof}
            \begin{proof}[Proof 2]
                We can also show that
                $c_{0}^{C}$ is open.
                Let $x\in{c_{0}^{C}}$. Then there is
                an $r>0$ and a subsequence
                $x_{k_{n}}$ of $x$ such that
                $x_{k_{n}}>r$ for all $n$.
                But then $B_{r/2}(x)$ is
                an open ball contained in $c_{0}^{C}$.
                For if $y\in{B_{r/2}(x)}$, then
                $d_{\infty}(x,y)%
                 =\sup\{|x_{n}-y_{n}|\}<r<2$,
                and thus
                $|y_{k_{n}}-x_{k_{n}}|<r/2$,
                and there for $|y_{k_{n}}|>r/2$.
                Thus, $y$ is not a null sequence and
                $c_{0}^{C}$ is open. So
                $c_{0}$ is closed.
            \end{proof}
            Let $X$ be the set of sequences with only
            finitely many nonzero terms.
            Then $(X,d_{\infty}$ is not complete.
            Let $x_{1}=(1,0,0,\hdots)$,
            $x_{2}=(1,1/2,0,0,\hdots)$,
            $x_{n}=(1,1/2,\hdots,1/n,0,0,\hdots)$.
            Then
            $d_{\infty}(x_{n},x_{m})=1/\max\{n,m\}\rightarrow{0}$.
            But clearly
            $x_{n}\rightarrow(1,1/2,\hdots,1/n,\hdots)$, which
            is an element of $c_{0}$, but not an element
            of $X$. Thus $X$ is not closed, and therefore is
            not complete. Returning to $C[0,1]$, when we had
            that sequence of continuous functions that clearly
            converged to a discontinuous functions, we still
            needed to show that there is no continuous function
            that the $x_{n}(t)$ converged to. Here we've embedded
            $X$ into a bigger space, shown that the
            sequence converges to something outside of $X$,
            in our case an element of
            $c_{0}\setminus{X}$, and then used the uniqueness
            of limits to show that the limit does
            not converge in $X$.
        \subsection{Banach's Fixed Point Theorem}
            If $(X,d)$ is a complete metric space,
            and if $T:X\rightarrow{X}$ satisfies
            the property that, for all $x$ and $y$
            in $X$, $d(T(x),T(y))<kd(x,y)$ for
            some $k<1$, then $T$ has a unique
            point $x$, called a fixed point,
            such that $T(x)=x$.
            \begin{definition}
                A contraction of a metric
                space $(X,d)$ is a function
                $T:{X}\rightarrow{X}$ such that there
                exists a $k\in(0,1)$ such that
                for all $x,y\in{X}$,
                $d(T(x),T(y))<kd(x,y)$.
            \end{definition}
            \begin{definition}
                A fixed point of a function
                $f:X\rightarrow{X}$ is a point
                $x\in{X}$ such that
                $f(x)=x$.
            \end{definition}
            \begin{theorem}[%
                Banach's Fixed Point Theorem%
            ]
                If $(X,d)$ is a complete
                metric space and $T:X\rightarrow{X}$
                is a contraction, then there is
                a unique fixed point $x\in{X}$
                with respect to $T$.
            \end{theorem}
            \begin{definition}
                A Lipschitz continuous function is a
                function $f:[a,b]\rightarrow\mathbb{R}$
                such that there is an $L\in\mathbb{R}$
                such that
                $|f(x)-f(y)|<L|x-y|$ for all
                $x,y\in[a,b]$.
            \end{definition}
            This says that the slopes of the
            secant lines of the
            function are bounded. The square root
            function $y=\sqrt{x}$ is an example
            of a function that is not Lipschitz. The
            slopes of secant lines go to infinity
            as the points tend towards the origin.
            \begin{theorem}[Picard's Theorem]
                If $f:[a,b]\times\mathbb{R}%
                    \rightarrow\mathbb{R}$
                is Lipschitz continuous,
                Then there is a unique function
                $x:[a,b]\rightarrow\mathbb{R}$
                such that
                $\frac{dx}{dt}=f(t,x(t))$ and $x(a)=a$.
            \end{theorem}
            \begin{proof}
                We prove Picard by using the
                Banach Fixed Point Theorem. First
                we write the problem as an integral
                equation.
                If $\dot{x}=f(t,x(t))$, then:
                \begin{equation*}
                    x(t)
                    =\int_{a}^{t}\frac{dx}{dt}dt
                    =x_{0}+\int_{a}^{t}f(t,x(t))dt
                \end{equation*}
                Let $(X,d)$ be $C[a,b]$ with the
                supremum norm $d_{\infty}$. Then
                $(x,d)$ is a complete metric space.
                Let $T:{X}\rightarrow{X}$ be defined
                by:
                \begin{equation*}
                    Tx=x_{0}+\int_{a}^{t}f(t,x(t))dt
                \end{equation*}
                All we need to do is show that $T$ is
                a contraction. Applying the
                Banach Fixed Point theorem then
                shows that there is a unique
                fixed point of $T$, thus showing
                that there is a unique solution
                to our original initial value problem.
                If $x,y\in{X}$, then:
                \begin{align*}
                    d(Tx,Ty)
                    &=\sup\{|Tx(t)-Ty(t)|\}\\
                    &=\sup\{
                        (x_{0}+
                         \int_{a}^{t}f(t,x(t))dt)
                       -(x_{0}+
                         \int_{a}^{t}f(t,y(t))dt)
                    \}\\
                    &=\sup\{
                        \int_{a}^{t}f(t,x(t))dt)-
                        \int_{a}^{t}f(t,y(t))dt)
                    \}\\
                    &\leq\int_{a}^{t}|
                        f(t,x(t))-f(t,y(t))|dt
                \end{align*}
                But from the Lipschitz continuity
                of $f$, we have:
                \begin{align*}
                    d(Tx,Ty)&\leq
                    L\int_{a}^{t}|x(t)-y(t)|dt\\
                    &\leq{L}(t-a)d(x,y)\\
                    &\leq{L}(b-a)d(x,y)
                \end{align*}
                So $T$ is a contraction for
                $L(b-a)<1$. Usually we can
                extend this solution by taking
                $b$ as the initial condition and
                stepping forward one interval
                at a time. We'll take a different
                approach. We have that
                $d(Tx,Ty)\leq{L}(b-a)d(x,y)$. From
                this, we obtain:
                \begin{align*}
                    d(T^{2}x,T^{2}y)
                    &\leq{L}\int_{a}^{b}d(Tx,Ty)dt\\
                    &\leq{L}\int_{a}^{t}
                        L(t-a)d(x,t)dt\\
                    &=\frac{L^{2}}{2}(t-a)^{2}d(x,y)\\
                    &\leq
                    \frac{L^{2}}{2}(b-a)^{2}d(x,y)
                \end{align*}
                Applying induction, we have:
                \begin{equation*}
                    d(T^{n}x,T^{n}y)
                    \leq\frac{L^{n}}{n!}(b-a)^{n}
                \end{equation*}
                But this tends to zero, and thus
                there is an $N$ such that,
                for all $n>N$, $T^{n}$ is a
                contraction. But then, by the
                Banach Fixed Point Theorem, there
                is a unique point $x$ such that
                $T^{n}x=x$. But then
                $Tx=T^{n}(Tx)$, and thus
                $Tx$ is a fixed point of
                $T^{n}$. But the fixed point of
                $T^{n}$ is unique, and $x$ is a
                fixed point. Therefore
                $Tx=x$. Therefore, etc.
            \end{proof}
            Without Lipschitz continuous you may
            lose uniqueness, but you still have
            existence. This is Peano's theorem.
            An example is $\dot{x}=\sqrt{x}$
            with $x(0)=0$.
            This has solutions $x(t)=0$ and
            $(t)=t^{2}/4$. Now back to compactness.
            \subsubsection{Compactness}
                \begin{definition}
                    A metric space $(X,d)$ is
                    sequentially compact if every
                    sequence in $X$ has a convergent
                    subsequence.
                \end{definition}
                In topology there is a difference
                between sequential compactness
                and regular compactness, but in
                metric spaces they turn out
                to be the same.
                A subset of $S$ of $X$ is
                compact if every sequence in
                $S$ has a subsequence which converges.
                That is, $(S,d)$ is compact.
                \begin{theorem}
                    A subset $S$ of a compact
                    metric space $(X,d)$ is compact
                    if and only if $S$ is closed.
                \end{theorem}
                \begin{proof}
                    For let $x_{n}$ be a sequence
                    in $S$. Then $x_{n}$ is a
                    sequence in $X$ and thus there
                    is a convergent subsequence
                    $x_{k_{n}}$ with a limit $x$.
                    But $x_{k_{n}}$ is in $S$ and
                    $S$ is closed, and therefore
                    $x$ is in $S$. Thus, $S$
                    is compact. Conversely, if
                    $S$ is compact, suppose it is
                    not closed. Then there is a point
                    $y\in{X}$ such that $y$ is a
                    limit point of $S$ but not
                    contained in $S$. Let
                    $x_{n}$ be a sequence that
                    converges to $y$. Then, as
                    $S$ is compact, there is
                    a convergent subsequence. But
                    the limit of this subsequence
                    is $y$, a contradiction as
                    $y\notin{S}$. Therefore $S$
                    is closed.
                \end{proof}
                \begin{theorem}
                    If $(X,d)$ is a compact metric
                    space, then
                    $(X,d)$ is complete.
                \end{theorem}
                \begin{proof}
                    If $x_{n}$ is Cauchy in $X$,
                    then there is a convergent
                    subsequence $x_{k_{n}}$
                    in $X$. But if $x_{k_{n}}$
                    converges to $x$, then
                    $x_{n}$ converges to $x$ as
                    well, as $x_{n}$ is Cauchy.
                    Therefore, $(X,d)$ is complete.
                \end{proof}
                \begin{theorem}[Heine-Borel Theorem]
                    A subset of
                    $\mathbb{R}^{n}$ is
                    compact if and only if
                    it is closed and bounded.
                \end{theorem}
                \begin{example}
                    The closed unit ball
                    of $\ell^{p}$ is not compact,
                    if $1\leq{p}\leq{\infty}$.
                    Let $x_{n}(m)$ be the sequence
                    (of sequences) such that
                    $x_{n}(m)=1$ if $n=m$, and
                    zero otherwise. Then
                    $d_{p}(x_{n},x_{m})=2^{1/p}$,
                    so $x_{n}$ has no subsequence
                    which is Cauchy. But then there
                    is no convergent subsequence
                    either, and therefore
                    $\ell^{p}$ is not compact.
                \end{example}
                \begin{example}
                    The closed unit ball in
                    $(C[0,1],d_{\infty})$ is
                    not compact. For let
                    $x_{n}(t)=t^{2^{n}}$. Then
                    (Do some calculus) the maximum of
                    $d(x_{n},x_{n+1})$ is always
                    $1/4$. So this has no subsequence
                    which is Cauchy, and thus no
                    convergent subsequence exists.
                \end{example}
                \begin{definition}
                    A metric space $X$ is totally
                    bounded if for all
                    $\varepsilon>0$ there is a finite
                    number of points $x_{n}$ such
                    that $B_{\varepsilon}(x_{n})$
                    covers the entirety of $X$.
                \end{definition}
                \begin{theorem}
                    A compact metric space is
                    totally bounded.
                \end{theorem}
                \begin{proof}
                    Suppose not. Then there is an
                    $\varepsilon>0$ such that
                    no finite collection
                    $B_{\varepsilon}(x_{n})$
                    is a covering of $X$. Let
                    $x_{1}\in{X}$. Then
                    $B_{\varepsilon}(x_{1})$ is not
                    $X$. Thus there is an $x_{2}$
                    such that
                    $x_{2}\notin%
                     B_{\varepsilon}(x_{1})$.
                    But also
                    $B_{\varepsilon}(x_{1})\cup%
                     B_{\varepsilon}(x_{2})$ is
                    not the entirety of $X$.
                    Continuing we have that there
                    is a sequence $x_{n}$ such that,
                    for all $n\ne{m}$,
                    $d(x_{n},x_{m})\geq{\varepsilon}$.
                    So there is no convergent
                    subsequence. But $X$ is
                    compact, a contradiction.
                    Therefore, etc.
                \end{proof}
                There are metric spaces that are
                bounded but not totally bounded.
                For let
                $X=\mathbb{R}$ and $d$ be the
                discrete metric. Then, for
                $\varepsilon=1/2$, the is no
                finite covering. Every point needs
                it's own ball, so the covering is
                uncountable.
                \begin{theorem}
                    If $(X,d)$ is complete and
                    totally bounded, then it
                    is compact.
                \end{theorem}
                \begin{proof}
                    Let $x_{n}$ be a sequence
                    in $X$. Let $\varepsilon=1$. Then
                    there are finitely many points
                    $y_{k}$ such that
                    $B_{\varepsilon}(y_{k})$ covers
                    $X$. Then one of these
                    balls has infinitely many of
                    the $x_{n}$. Similarly, for
                    $\varepsilon=\frac{1}{n}$, there
                    is a finite number of points
                    $y_{k}$ such that
                    $B_{\frac{1}{n}}(y_{k})$ covers
                    $X$. Thus there is a point with
                    infinitely many of the $x_{n}$
                    in it. So, we can find a
                    subsequence such that, for
                    $n,m>N$,
                    $d(x_{k_{n}},x_{k_{m}})<%
                     \frac{1}{N}$. But $(X,d)$ is
                    complete, and therefore
                    $x_{k_{n}}$ converges. Therefore
                    $x_{n}$ has a convergent
                    subsequence. Thus, $(X,d)$ is
                    compact.
                \end{proof}
                \begin{theorem}
                    Compact spaces are separable.
                \end{theorem}
                \begin{proof}
                    If $X$ is compact, then
                    it is totally bounded. But
                    then, for $\varepsilon=1/n$
                    there is a finite covering of
                    $X$ with balls of radius
                    $\varepsilon$. Then,
                    taking all of the
                    centers of all of the points
                    for all $n$ (Countable union
                    of finite points is countable),
                    we obtain a countable dense
                    subset.
                \end{proof}
                \begin{example}
                    There are ``infinite dimension''
                    sets that are also compact. Two
                    in particular worth mentioning.
                    The first is the hilbert Cube.
                    It's a subset of $\ell^{2}$
                    whose elements are such that
                    $|x_{n}|<1/n$. That is, elements
                    are sequences whose $n^{th}$
                    elements are less than
                    $1/n$. This is compact.
                    Arzela-Ascoli. Peano.
                \end{example}
    \section{Normed Spaces and Inner Product Spaces}
        \subsection{Basic Definitions}
            We're finally going to put some structure on these
            sets, and talk about vector spaces. In a metric
            space, the only thing you can really talk about
            is the distance between points. In a vector space
            we have a lot more structure. We will start off
            with vector spaces over the reals $\mathbb{R}$.
            The main properties are that there is a
            $\mathbf{0}$ element, addition is well defined
            and is both associative and commutative,
            there is a notion of scalar multiplication that
            is associative, and the distributive law holds.
            \begin{example}
                $\mathbb{R}^{n}$, with it's usual notion
                of addition, and with scalar multiplication
                defined over $\mathbb{R}$, is a vector space.
            \end{example}
            \begin{definition}
                A norm on a vector space $X$ over $\mathbb{R}$
                is a function $\norm{}:X\rightarrow\mathbb{R}$
                such that:
                \begin{enumerate}
                    \item For all $\mathbf{x}\in{X}$,
                          $\norm{\mathbf{x}}\geq{0}$ and
                          $\norm{\mathbf{x}}=0$ if and only
                          if $\mathbf{x}=\mathbf{0}$.
                          \hfill[Positive Definiteness]
                    \item For all $\mathbf{x}\in{X}$ and
                          $c\in\mathbb{R}$,
                          $\norm{c\mathbf{x}}%
                           =|c|\norm{\mathbf{x}}$
                          \hfill[Homogeneity]
                    \item For all $\mathbf{x},\mathbf{y}\in{X}$,
                          $\norm{\mathbf{x}+\mathbf{y}}%
                           \leq\norm{\mathbf{x}}%
                           +\norm{\mathbf{y}}$
                          \hfill[Triangle Inequality]
                \end{enumerate}
            \end{definition}
            We have seen before that
            $d(\mathbf{x},\mathbf{y})%
             =\norm{\mathbf{x}-\mathbf{y}}$
            defines a metric, and thus $(X,d)$ is a metric space.
            Thus, for every vector space there is an associated
            metric space, the metric $d$ called the
            \textit{induced} metric.
            \begin{definition}
                A normed vector space is a vector space
                $X$ over $\mathbb{R}$ with a norm
                $\norm{}$ on $X$.
            \end{definition}
            \begin{example}
                $\mathbb{R}^{n}$ with
                $\norm{\mathbf{x}}_{p}$, for $p\geq{1}$,
                is a normed vector space.
            \end{example}
            \begin{example}
                $\ell^{p}$ with $\norm{x}_{p}$ is
                also a normed vector space.
            \end{example}
            \begin{example}
                $C[a,b]$ equipped with the supremum norm,
                $\norm{x(t)}_{\infty}$,
                is a normed vector space.
            \end{example}
        \subsubsection{Inner Product Spaces}
            \begin{definition}
                An inner product on a vector space
                $X$ over $\mathbb{R}$ is a function
                $\langle\rangle:X\rightarrow\mathbb{R}$
                such that:
                \begin{enumerate}
                    \item For all $x\in{X}$,
                          $\langle{\mathbf{x},\mathbf{x}}%
                           \rangle\geq{0}$
                          and
                          $\langle\mathbf{x},\mathbf{x}\rangle=0$
                          if and only
                          if $\mathbf{x}=\mathbf{0}$.
                          \hfill[Positive Definiteness]
                    \item For all $\mathbf{x},\mathbf{y}\in{X}$,
                          $\langle\mathbf{x},\mathbf{y}\rangle%
                           =\langle\mathbf{y},\mathbf{x}\rangle$
                          \hfill[Symmetry]
                    \item For all
                          $\mathbf{x},\mathbf{y},\mathbf{z}%
                           \in{X}$
                          and all $\alpha,\beta\in\mathbb{R}$,
                          $\langle\alpha\mathbf{x}%
                           +\beta\mathbf{y},\mathbf{z}\rangle%
                           =\alpha\langle\mathbf{x},\mathbf{z}%
                           \rangle+\beta\langle\mathbf{y},%
                           \mathbf{z}\rangle$
                          \hfill[Linearity]
                \end{enumerate}
            \end{definition}
            \begin{example}
                $\mathbb{R}^{2}$ with
                $\langle(x_{1},x_{2}),(y_{1},y_{2})\rangle%
                 =x_{1}y_{1}+x_{2}y_{2}$ is an inner product.
                Replacing this with $\mathbb{R}^{n}$ and doing
                $\sum_{k=1}^{n}x_{k}y_{k}$ is also an inner
                product. This is the usual dot product that one
                sees in a vector calculus course. In $\ell^{2}$,
                $\sum_{k=1}^{\infty}x_{k}y_{k}$ is an inner
                product as well. Note also that
                $\sum|x_{i}y_{i}|$ converges since
                $|x_{i}y_{i}|\leq\frac{1}{2}|x_{i}^{2}|%
                 +\frac{1}{2}|y_{i}|^{2}$.
            \end{example}
            \begin{example}
                In $C[a,b]$, let
                $\langle{x(t),y(t)}\rangle%
                 =\int_{a}^{b}x(t)y(t)dt$. This defines an
                inner product.
            \end{example}
            \begin{definition}
                An inner product space is a vector space
                $X$ over $\mathbb{R}$ with an inner product
                $\langle\rangle$.
            \end{definition}
            \begin{theorem}[Cauchy-Schwarz Inequality]
                If $X$ is an inner product space
                and $x,y\in{X}$, then
                $|\langle{x,y}\rangle<\norm{x}\norm{y}$
            \end{theorem}
            \begin{proof}
                For all $y\in\mathbb{R}$,
                $\langle{x+ty,x+ty}\rangle%
                 =\langle{x,x}\rangle%
                 +2t\langle{x,y}\rangle%
                 +t^{2}\langle{y,y}\rangle%
                 =\norm{x}^{2}+2t\langle{x,y}\rangle%
                 +t^{2}\norm{y}^{2}$. Thus we have a
                quadratic in $t$. But this is always positive,
                and thus the discriminant must be non-positive. Therefore
                $(2\langle{x,y})^{2}-4\norm{x}^{2}\norm{y}^{2}%
                 \leq{0}$
                and thus
                $|\langle{x,y})|\leq\norm{x}\norm{y}$.
            \end{proof}
            \begin{theorem}
                If $X$ is a vector space over $\mathbb{R}$
                and $\langle\rangle$ is an inner product,
                then
                $\norm{\mathbf{x}}%
                 =\sqrt{\langle\mathbf{x},\mathbf{y}\rangle}$
                is a norm on $X$.
            \end{theorem}
            \begin{proof}
                Positivity, homogeneity, and definiteness are
                pretty easy. The only tricky thing to check is
                the triangle inequality. We have that
                $\norm{x+y}=\langle{x+y,x+y}\rangle$,
                and this simplify to
                $\norm{x}^{2}+2\langle{x,y}\rangle+\norm{y}^{2}$.
                But from the Cauchy-Schwartz inequality, we
                have $\langle{x,y}\rangle\leq\norm{x}\norm{y}$.
                Thus
                $\norm{x+y}^{2}\leq\norm{x}^{2}%
                 +2\norm{x}\norm{y}+\norm{y}^{2}%
                 =(\norm{x}+\norm{y})^{2}$. Taking square roots
                 completes the theorem.
            \end{proof}
            In $\mathbb{R}^{n}$, the Cauchy-Schwartz inequality
            says that the dot product of two vectors is less
            than or equal to the product of the magnitude
            of the two vectors.
            This is obvious from the fact that the dot product
            of two vector is the product of the magnitudes and
            the \textit{cosine} of the angle between them.
            Since the cosine of a number is less than or equal
            to one, this would complete the theorem.
            In $\ell^{p}$ and $L^{p}$ spaces, this is the
            special case of the H\"{o}lder inequality for
            when $p=q=2$.
        \subsubsection{Convergence in Normed Spaces}
            In a metric space, convergence meant that
            $d(x_{n},x)\rightarrow{0}$. In a normed space
            we have the induced metric, and thus we may define
            convergence as $\norm{x_{n}-x}\rightarrow{0}$.
            \begin{definition}
                A convergent sequence in a normed space $X$
                is a sequence $x_{n}$ such that there is an
                $x\in{X}$ such that
                $\norm{x_{n}-x}\rightarrow{0}$.
            \end{definition}
            Since
            $\norm{y}=\norm{(y-x)+x}\leq\norm{y-x}+\norm{x}$,
            it follows that
            $|\norm{x}-\norm{y}|\leq\norm{x-y}$.
            But then if $x_{n}\rightarrow{x}$, then
            $|\norm{x_{n}}-\norm{x}|\leq\norm{x_{n}-x}$,
            and $\norm{x_{n}-x}\rightarrow{0}$. Therefore
            $\norm{x_{n}}\rightarrow\norm{x}$. That is,
            the norm function is a continuous function.
            Similarly, if $x_{n}\rightarrow{x}$, then
            $\langle{x_{n},y}\rangle\rightarrow%
             \langle{x,y}\rangle$.
            In fact, if $x_{n}\rightarrow{x}$ and
            $y_{n}\rightarrow{y}$, then
            $\langle{x_{n},y_{n}}\rangle%
             \rightarrow\langle{x,y}\rangle$. To see this, we
            have
            $\langle{x_{n},y_{n}}\rangle-\langle{x,y}\rangle%
             =\langle{x_{n}-x,y}\rangle+\langle{x,y-y_{n}}\rangle$
            and therefore
            $|\langle{x_{n},y_{n}}\rangle-\langle{x,y}\rangle%
             \leq\norm{x_{n}-x}\norm{y_{n}}%
             +\norm{x}\norm{y-y_{n}}$. But $\norm{x-x_{n}}\rightarrow{0}$
            and $\norm{y-y_{n}}\rightarrow{0}$. But also
            $\norm{y_{n}}=\norm{(y_{n}-y)+y}\leq\norm{y_{n}-y}+\norm{y}$,
            which is bounded. Therefore
            $\langle{x_{n},y_{n}}-\langle{x,y}\rangle\rightarrow{0}$.
            So inner product spaces and normed spaces are metric spaces
            and we can define everything we did for metric spaces and all
            of the previous results remain true. That is, the notions and
            theorems pertaining to convergence, completeness, compactness,
            the notion of open and closed. All of these still make sense in
            these new spaces.
        \subsubsection{Banach Spaces and Hilbert Spaces}
            \begin{definition}
                A Banach Space is a normed vector space $X$ that is
                complete with respect to the induced metric.
            \end{definition}
            \begin{definition}
                A Hilbert Space is an inner product space $X$ that is
                complete with respect to the induced metric.
            \end{definition}
        \subsubsection{Linear Operators}
            Let $X$ and $Y$ be normed spaces. A mapping
            $T:X\rightarrow{Y}$ is called a linear operator if, for
            all $x,y\in{X}$, and for all $\alpha,\beta\in\mathbb{R}$,
            $T(\alpha{x}+\beta{y})=\alpha{T(x)}+\beta{T(y)}$. Usually, with
            operators, we simply write $Tx$ and $Ty$. Similar to how
            we write matric multiplication over vectors. In $\mathbb{R}^{n}$,
            every $n\times{n}$ matrix defines a linear operator.
            \begin{definition}
                A linear operator from a normed vector space $X$ to
                a normed vector space $Y$ is a function
                $T:X\rightarrow{Y}$ such that, for all $x,y\in{X}$
                and for all $\alpha,\beta\in\mathbb{R}$,
                $T(\alpha{x}+\beta{y})=\alpha{Tx}+\beta{Ty}$.
            \end{definition}
            \begin{definition}
                A bounded linear operator from a normed vector space
                $X$ to a normed vector space $Y$ is a linear operator
                $T:X\rightarrow{Y}$ such that there is a $K\in\mathbb{R}$
                such that for all $x\in{X}$, $\norm{Tx}\leq{K}\norm{x}$
            \end{definition}
            In a just world, ``bounded'' would mean
            $\norm{Tx}\leq{K}$. However, the only linear mapping that does
            this is the zero mapping. For if $\norm{Tx}=1$,
            then $\norm{T(2x)}=2$, and so on, and thus no linear mapping
            is bounded (With the exception of the zero mapping).
            Boundedness of a norm $T:X\rightarrow{Y}$ depends on
            the norms of the space.
            \begin{theorem}
                Bounded linear operators are continuous.
            \end{theorem}
            \begin{proof}
                If $x_{n}\rightarrow{x}$, then
                $\norm{Tx_{n}-Tx}=\norm{T(x_{n}-x)}$. But
                $T$ is bounded, and thus there is a $K$ such that
                $\norm{T(x_{n}-x)}\leq{K}\norm{x_{n}-x}$. But
                $\norm{x_{n}-x}\rightarrow{0}$. Therefore, etc.
            \end{proof}
            The converse is also true.
            \begin{theorem}
                If $T$ is a continuous linear operator,
                than there exists a $\delta>0$ such that for
                all $x\in{B}_{\delta}(0)$,
                $\norm{Tx-T0}<1$. But from linearity,
                $T0=0$, and thus $\norm{Tx}<1$. Then for any
                $z\in{Z}$, we have
                $\norm{\frac{\delta}{2}\frac{z}{\norm{z}}}=\frac{\delta}{2}$,
                and thus $\norm{T(\frac{\delta}{2}\frac{z}{\norm{z}})}<1$.
                Letting $K=\delta$, we have
                $\norm{Tx}<K\norm{x}$. Thus, $T$ is bounded.
            \end{theorem}
            Continuity at 0 implies uniform continuity since
            if $x_{n}-y_{n}\rightarrow{0}$, then
            $\norm{Tx_{n}-Ty_{n}}=\norm{T(x_{n}-y_{n})}%
             \leq{K}\norm{x_{n}-y_{n}}\rightarrow{0}$.
            The set of bounded linear operators form a vector space,
            where addition is $(S+t)(x)=(Sx)+(Tx)$, and scalar multiplication
            is defined by $(\alpha{T})(x)=\alpha(Tx)$. We must show that
            when you add two bounded linear operators, the result is a
            bounded linear operator.
            \begin{theorem}
                If $T_{1}:X\rightarrow{Y}$ and $T_{2}:X\rightarrow{Y}$
                are bounded linear operators, then $T_{1}+T_{2}$ is a
                bounded linear operator.
            \end{theorem}
            \begin{proof}
                For let $T_{1}$ and $T_{2}$ be bounded. Then there are
                $K_{1},K_{2}$ such that, for all $x\in{X}$,
                $\norm{T_{1}x}\leq{K_{1}}\norm{x}$ and
                $\norm{T_{2}x}\leq{K_{2}}\norm{x}$. But then
                $\norm{(T_{1}+T_{2})x}=\norm{T_{1}x+T_{2}x}%
                 \leq\norm{T_{1}x}+\norm{T_{2}x}%
                 \leq{K_{1}}\norm{x}+K_{2}\norm{x}$. Let $K=K_{1}+K_{2}$.
            \end{proof}
            \begin{theorem}
                If $T:X\rightarrow{Y}$ is a bounded linear operator, and
                $\alpha\in\mathbb{R}$, then $\alpha{T}$ is a bounded
                linear operator.
            \end{theorem}
            \begin{proof}
                For
                $\norm{\alpha{Tx}}=|\alpha|\norm{Tx}%
                 \leq|\alpha|K\norm{x}=K\norm{\alpha{x}}$.
            \end{proof}
            We write $B(X,Y)$ to denote the set of bounded linear
            operators from $X$ to $Y$. That is, linear operators
            $T:X\rightarrow{Y}$.
            We can define a norm on $B(X,Y)$ as follows:
            $\norm{T}_{B}%
             =\sup_{x\in{X},x\ne{0}}\{\frac{\norm{Tx}}{\norm{x}}\}$.
            This is the ``Smallest $K$,'' used as a bounded for the linear
            operator $T$. This shows that
            $\norm{Tx}_{Y}\leq\norm{T}_{B}\norm{x}_{X}$.
    \subsection{Lecture 7: October 22, 2018}
        \subsubsection{Bounded Linear Operators}
            A bounded linear operator is a function
            $T:X\rightarrow{Y}$ between normed spaces
            $X$ and $Y$ such that $T$ is linear, and
            there exists a $K\in\mathbb{R}$ such that,
            for all $x\in{X}$,
            $\norm{Tx}_{Y}\leq{K}\norm{x}_{X}$. The
            norm of $T$, $\norm{T}$, is then defined
            as the smallest such $K$. Equivalently:
            \begin{equation*}
                \norm{T}=
                \sup\Big\{\frac{\norm{Tx}_{Y}}{\norm{x}_{X}}:
                          x\in{X},x\ne{0}\Big\}
                =\sup\{\norm{Tx}_{Y}:\norm{x}_{X}=1\}
            \end{equation*}
            The set of all bounded linear operators
            from a normed space $X$ to a normed space
            $Y$ is denoted $B(X,Y)$. This is a vector
            space with addition defined as
            $(T+S)x=(Tx)+(Sx)$ and $(aT)x=a(Tx)$.
            \begin{theorem}
                $\norm{T}$ defines a norm on
                $B(X,Y)$.
            \end{theorem}
            \begin{proof}
                For $\norm{T}\geq{0}$ and
                $\norm{Tx}=0$ if and only if
                $Tx=0$ for all $x\in{X}$, and thus
                $T$ is the zero operator. If
                $\alpha\in\mathbb{R}$, then:
                \begin{equation*}
                    \norm{\alpha{T}}
                    =\sup\Big\{
                        \frac{\norm{\alpha{T}x}_{Y}}{\norm{x}_{X}}:
                        x\in{X},x\ne{0}\Big\}
                    =|\alpha|\sup\Big\{
                        \frac{\norm{Tx}_{Y}}{\norm{x}_{X}}:
                        x\in{X},x\ne{0}\Big\}
                    =|\alpha|\norm{T}
                \end{equation*}
                Finally, if $S,T\in{B}(X,Y)$, then:
                \begin{align*}
                    \norm{S+T}&=\sup\Big\{
                        \frac{\norm{(S+t)x}_{Y}}{\norm{x}_{X}}:
                        x\in{X},x\ne{0}\}\\
                    &=\sup\Big\{
                        \frac{\norm{Sx+Tx}_{Y}}{\norm{x}_{X}}:
                        x\in{X},x\ne{0}\Big\}\\
                    &\leq\sup\Big\{
                        \frac{\norm{Sx}_{y}+\norm{Tx}_{Y}}
                             {\norm{x}_{X}}:
                        x\in{X},x\ne{0}\Big\}\\
                    &\leq\norm{T}+\norm{S}
                \end{align*}
            \end{proof}
            \begin{theorem}
                If $Y$ is a Banach space, and 
                if $X$ is a normed space, then
                $B(X,Y)$ is a Banach space.
            \end{theorem}
            \begin{proof}
                For let $T_{n}$ be a Cauchy sequence
                in $B(X,Y)$ and let $\varepsilon>0$.
                Then there exists $N_{0}\in\mathbb{N}$
                such that for all $n,m>N_{0}$,
                $\norm{T_{n}-T_{m}}<\varepsilon$. That is,
                for all $n,m>N_{0}$:
                \begin{align*}
                    \sup\Big\{
                        \frac{\norm{T_{n}x-T_{m}x}_{Y}}
                             {\norm{x}_{X}}:
                        x\in{X},x\ne{0}\Big\}
                    &\leq\varepsilon\\
                    \Rightarrow
                    \frac{\norm{T_{n}x-T_{m}y}_{Y}}
                         {\norm{x}_{X}}
                    &\leq\varepsilon
                \end{align*}
                That is, $T_{n}x$ is a Cauchy sequence
                in $Y$ for any fixed value $x\in{X}$.
                But $Y$ is a Banach space, and is therefore
                complete. But then if $T_{n}x$ is a Cauchy
                sequence in $Y$ it has a limit $y\in{Y}$.
                Let $Tx=\lim_{n\rightarrow\infty}T_{n}x$
                for all $x\in{X}$.
                Then $T\in{B(X,Y)}$. For:
                \begin{equation*}
                    T(x+y)
                    =\lim_{n\rightarrow\infty}T_{n}(x+y)
                    =\lim_{n\rightarrow\infty}(T_{n}x+T_{n}y)
                    =Tx+Ty
                \end{equation*}
                And similarly $(\alpha{T})x=\alpha{T}x$.
                Lastly, $T$ is bounded. For all $n,m>N$ we have
                $\norm{T_{n}x-T_{m}x}_{Y}/\norm{x}_{X}<\varepsilon$.
                Taking the limit on $m$, we have
                $\norm{Tx-T_{n}x}_{Y}/\norm{x}_{X}\leq\varepsilon$
                for all $n>N_{0}$. Thus,
                $\norm{T_{n}x-Tx}_{X}\leq\varepsilon\norm{x}_{X}$.
                But
                $\norm{Tx-T_{n}x}_{Y}=\norm{T_{n}x-(T_{n}-Tx)}_{Y}$,
                and therefore
                $\norm{Tx}\leq\varepsilon\norm{x}_{X}+\norm{T_{n}x}$,
                and $\norm{T_{n}x}\leq\norm{T_{n}}$, and therefore
                $\norm{Tx}\leq\varepsilon\norm{X}_{X}+%
                 \norm{T}\norm{x}_{X}$. But then
                $\norm{Tx}_{Y}\leq%
                 (\varepsilon+\norm{T_{n}})\norm{x}_{X}$.
                But $T_{n}$ is bounded, and therefore
                $T$ is bounded. Finally, we must show that
                $T_{n}\rightarrow{T}$ in $B(X,Y)$ with respect
                to the norm $\norm{T_{n}-T}$. That is, we must
                show that $\norm{T-T_{n}}\rightarrow{0}$. This
                follows since
                $\norm{Tx-T_{n}x}_{Y}/\norm{x}_{X}<\varepsilon$
                for $n>N_{0}$, and therefore
                $\norm{T-T_{n}}<\varepsilon$. Therefore, etc.
            \end{proof}
        \subsubsection{Dual Spaces}
            So if $Y$ is a Banach space, and $X$ is any normed
            space, then $B(X,Y)$ is a Banach space. One of the
            most important cases is $Y=(\mathbb{R},||)$, where
            $||$ is the normal absolute value ``norm.''
            $B(X,\mathbb{R})$ is a Banach space, and it is
            called the continuous dual space of $X$, written
            $X'$. Elements of $X'$ are called bounded linear
            functionals. These are bounded linear operators
            whose range of the operator is the real numbers.
            The characterization, or the representation, or
            realization, of these dual spaces is a major
            topic in functional analysis. A lot of these
            theorems are do to a mathematician by the name
            of Riesz.
            \begin{example}
                A functional takes an element of a normed
                space $X$ and spits out a real number. For
                example, if $X$ is the space of continuous
                functions, then the following are
                functionals:
                \begin{align*}
                    f_{1}(x)&=\int_{0}^{1}x(t)t^{2}\diff{t}
                    &
                    f_{2}(x)&=x(0.5)
                    &
                    f_{3}(x)&=0
                \end{align*}
            \end{example}
            Let $X=(\mathbb{R}^{2},\ell^{1}$. What does
            $X'$ look like? That is, what is the dual
            space of $X$? let $f:X\rightarrow\mathbb{R}$
            be defined by
            $f(x_{1},x_{2})=2x_{1}-5x_{2}$. Then
            $f\in{X'}$ and $\norm{f}=5$. More generally,
            every element of $\mathbb{R}^{2}$ defines
            and element of $X'$. Given
            $(a,b)\in\mathbb{R}^{2}$, we define
            $f(x_{1},x_{2})=ax_{1}+bx_{2}$. $f$ is then linear,
            and:
            \begin{align*}
                |f(x_{1},x_{2})|
                &=|ax_{1}+bx_{2}|\\
                &\leq|a||x_{1}|+|b||x_{2}|\\
                &\leq\max\{|a|,|b|\}(|x_{1}|+|x_{2}|)\\
                &=\norm{(a,b)}_{\infty}
                \norm{(x_{1},x_{2})}_{\ell^{1}}
            \end{align*}
            And therefore $f$ is bounded, as $\norm{(a,b)}_{\infty}$
            is a bound. That is, $\norm{f}\leq\norm{(a,b)}_{\infty}$.
            By choosing $x=(x_{1},x_{2})$, where $x_{1}=1$ and
            $x_{2}=0$ if $|b|\leq|a|$, and $x_{1}=0$ and
            $x_{2}=1$ otherwise, we ge
            $|f|=\max\{(a,b)\}=\norm{(a,b)}_{\infty}$.
            Therefore $\norm{f}=\norm{(a,b)}_{\infty}$. On
            the other hand, if $f\in{X'}$, let
            $a=f(1,0)$ and $b=f(0,1)$. Then, for all
            $(x_{1},x_{2})\in\mathbb{R}^{2}$:
            \begin{equation*}
                f(x_{1},x_{2})
                =f(x_{1}(1,0)+x_{2}(0,1))
                =x_{1}f(1,0)+x_{2}f(0,1)
                =ax_{1}+bx_{2}
            \end{equation*}
            So the dual of $(\mathbb{R}^{2},\ell^{1})$
            looks very much like $(\mathbb{R}^{2},\ell^{\infty})$.
            In fact, $(\mathbb{R}^{2},\ell^{1})'$ and
            $(\mathbb{R}^{2},\ell^{\infty})$ are isometric
            and isomorphic. That is, we really can't tell them
            apart and we can consider them as the same thing.
            More generally,
            $(\mathbb{R}^{n},\ell^{n})'=(\mathbb{R}^{n},\ell^{\infty})$.
            Even more general, if $p$ and $q$ are exponential
            conjugates of each other (That is,
            $\frac{1}{q}+\frac{1}{p}=1$), then
            $(\mathbb{R}^{n},\ell^{p})'=(\mathbb{R}^{n},\ell^{p})$
            for all $1\leq{p}\leq\infty$. Saying $p=\infty$ is
            equivalent to saying $q=1$. Setting $p=q=2$, we have
            $(\mathbb{R}^{n},\ell^{2})'=(\mathbb{R}^{n},\ell^{2})$.
            This is true of any Hilbert space: The dual of any
            Hilbert Space $\mathcal{H}$ is itself. That is,
            $\mathcal{H}'=\mathcal{H}$. This is one of the
            Riesz Representation Theorems. In infinite dimensions,
            $(\ell^{p})'=\ell^{q}$, where $p$ and $q$ are such that
            $\frac{1}{p}+\frac{1}{q}=1$, and $1\leq{p}<\infty$.
            Now, we cannot allow $p=\infty$. For
            $(\ell^{\infty})'$ is not equal to $\ell^{1}$.
            \begin{theorem}
                If $1\leq{p}<\infty$ and
                $\frac{1}{p}+\frac{1}{q}=1$, then
                $(\ell^{p})'=\ell^{q}$.
            \end{theorem}
            \begin{proof}
                If $(f_{1},f_{2},\hdots)\in\ell^{q}$, then let
                $f:\ell^{p}\rightarrow\mathbb{R}$ be defined by
                $f(x_{1},x_{2},\hdots)=\sum_{k=1}^{\infty}x_{k}f_{k}$.
                This converges from H\"{o}lder's inequality:
                \begin{equation*}
                    \sum_{k=1}^{\infty}|x_{k}f_{k}|
                    \leq
                    \Big(\sum_{k=1}^{\infty}f_{k}^{q}\Big)^{1/q}
                    \Big(\sum_{k=1}^{\infty}x_{i}^{p}\Big)^{1/p}
                \end{equation*}
                And therefore
                $|fx|=\norm{(f_{1},f_{2},\hdots)}_{q}%
                 \norm{(x_{1},x_{2},\hdots)}_{p}$. That is,
                Moreover $f$ is linear. Therefore
                $f\in(\ell^{p})'$ and
                $\norm{f}\leq\norm{(f_{1},f_{2},\hdots)}_{q}$.
                On the other hand, let
                $x_{i}=|f_{i}|^{q/p}\sgn(f_{i}$. Then
                \begin{equation*}
                    fx=\sum_{k=1}^{\infty}f_{k}x_{k}
                    =\sum_{k=1}^{\infty}|f_{k}|^{q/p+1}
                \end{equation*}
                But $\frac{1}{p}+\frac{1}{q}=1$, and thus
                $\frac{q}{p}+1=q$. Thus:
                \begin{align*}
                    |fx|
                    &=\sum_{k=1}^{\infty}|f_{k}|^{q}\\
                    &=\norm{(f_{1},f_{2},\hdots)}_{q}^{q}\\
                    &=\norm{(f_{1},f_{2},\hdots)}_{q}
                        \norm{(f_{1},f_{2},\hdots)}_{q}^{q-1}\\
                    &=\norm{(f_{1},f_{2},\hdots)}_{q}
                        \Big(\sum_{k=1}^{\infty}|f_{k}^{q}
                        \Big)^{\frac{q-1}{q}}
                    &=\norm{(f_{1},f_{2},\hdots)}_{q}
                        \Big(\sum_{k=1}^{\infty}|x_{k}|^{p}
                        \Big)^{1/p}\\
                    &=\norm{(f_{1},f_{2},\hdots)}_{q}\norm{x}_{p}
                \end{align*}
                Therefore, $\norm{f}=\norm{(f_{1},f_{2},\hdots)}_{q}$.
                Thus, for all $y\in\ell^{q}$ there is a bounded
                linear operator $f\in\ell^{p}$ such that
                $\norm{y}_{\ell^{q}}=\norm{f}_{(\ell^{p})'}$. That
                is, every $(f_{i})\in\ell^{q}$ defines an element
                of $(\ell^{p})'$ by
                $fx=\sum_{k=1}^{\infty}f_{k}x_{k}$, for any
                $(x_{i})\in\ell^{p}$. So $\ell^{q}$ can
                be \textit{embedded} into to $(\ell^{p})'$.
                Now we need to show that this embedding is
                the entirety of $(\ell^{p})'$. If
                $f\in(\ell^{p})'$, let
                $f_{i}=f(e_{i})$, where $e_{i}$ is the
                sequence $(0,0,\hdots,1,0,0,\hdots)$, where
                the 1 occurs in the $i^{th}$ spot. We need
                to show that $(f_{i})\in\ell^{q}$ and
                $fx=\sum_{k=1}^{\infty}f_{k}x_{k}$
                for all $x\in\ell^{p}$. If $x\in\ell^{p}$, then:
                \begin{align*}
                    x=\sum_{k=1}^{\infty}x_{k}e_{k}
                    \Rightarrow
                    fx=f\Big(\sum_{k=1}^{\infty}x_{k}e_{k}\Big)
                    =\sum_{k=1}^{\infty}f(x_{k}e_{k})
                    =\sum_{k=1}^{\infty}x_{k}f(x_{k})
                    =\sum_{k=1}^{\infty}x_{k}f_{k}
                \end{align*}
                Choosing $x_{k}=|f_{k}|^{q/p}\sgn(f_{k})$ and apply
                H\"{o}lder.
            \end{proof}
    \subsection{Lecture 8: October 29, 2018}
        \subsubsection{Review}
            If $X$ is a vector space, an inner product on
            $X$ is a mapping
            $\langle,\rangle:X\times{X}\rightarrow\mathbb{R}$
            such that:
            \begin{enumerate}
                \item $\langle{x,x}\rangle\geq{0}$ and
                      $\langle{x,x}\rangle=0$ if and only if
                      $x=0$
                \item $\langle{x,y}\rangle=\langle{y,x}\rangle$
                \item $\langle{ax+by,z}\rangle%
                       =a\langle{x,z}\rangle+b\langle{y,z}\rangle$
            \end{enumerate}
            Think of the dot product for vectors. This is a
            generalization of this concept. Every inner product
            on a vector space $V$ induce a norm on $V$:
            \begin{equation*}
                \norm{x}=\sqrt{\norm{x,x}}
            \end{equation*}
            An inner product that is complete with respect to
            the induced norm is called a Hilbert Space. A mapping
            $f:X\rightarrow\mathbb{R}$ is bounded if there is a
            $K\in\mathbb{R}$ such that, for all $x\in{X}$,
            $|f(x)|\leq{L\norm{x}}$. $f$ is linear if
            $f(ax+by)=af(x)+bf(y)$, for all $x,y\in{X}$ and all
            $a,b\in\mathbb{R}$. The smallest such $K$ that works
            is called the norm of $f$, denoted $\norm{f}$. For
            all $x\in{X}$, $|f(x)|\leq\norm{f}\norm{x}$. The
            vector space of all bounded linear functionals on
            $X$ is the dual space $X'$. This is also a Banach
            space with the functional norm $\norm{f}$. One
            question that arises is, how do we know that there
            are bounded linear functionals on a space $X$? In
            the case that $X$ is a Hilbert space, this is rather
            easy, but for a more general Banach space this is not
            that trivial. For any normed space $X$ we can at least
            one bounded linear functional because the zero mapping
            $f(x)=0$ is such a functional. The question is then
            does every normed space have a bounded linear functional
            on it? The answer is yes, and this is related to
            the Hahn-Banach Theorem. As said before, in the
            Hilbert case this is rather easy.
            \begin{theorem}
                If $X$ is a Hilbert space, then there is a
                non-trivial bounded linear functional
                $f:X\rightarrow\mathbb{R}$.
            \end{theorem}
            \begin{proof}
                If $X$ is an inner product and
                $z\in{X}$, let $f(x)=\langle{x,z}\rangle$ for
                all $x\in{X}$. Then $f$ is linear since the
                inner product is linear. But moreover, from
                Cauchy-Schwarz we have:
                \begin{equation*}
                    |f(x)|=|\langle{x,z}\rangle|
                    \leq\norm{x}\norm{z}
                \end{equation*}
                And thus $\norm{f}\leq\norm{z}$
                But $|f(z)|=\norm{z}$, so
                $\norm{f}=\norm{z}$. $f$ is a bounded
                linear functional.
            \end{proof}
            Riesz's Representation theorem says that this is it.
            All bounded linear functionals look like this. Thus,
            if $H$ is a Hilbert space, it's dual $H'$ is the space
            of all functions that look like
            $f(x)=\langle{x,y}\rangle$ for some $y\in{X}$. More
            precisely, if $H$ is a Hilbert space and $f\in{H'}$,
            then there is a $y\in{H}$ such that, for all
            $x\in{X}$, $f(x)=\langle{x,y\rangle}$.
        \subsubsection{Riesz's Representation Theorem}
            \begin{theorem}
                If $H$ is a Hilbert space, and
                $f:H\rightarrow\mathbb{R}$ is a bounded
                linear functional, then there is a unique
                $y\in{H}$ such that, for all $x\in{X}$,
                $f(x)=\langle{x,y}\rangle$. Moreover,
                $\norm{f}=\norm{y}$.
            \end{theorem}
            \begin{proof}
                Let $f\in{H'}$ and let $N=\nul(f)$. That is,
                $N$ is the null space of the functional $f$
                which is the set of all points
                $x\in{X}$ such that $f(x)=0$. The Null space
                actually defines a closed vector space, which
                is a subspace of $H$. If $N=H$, then
                $f(x)=0$, and thus let $y=0$. Otherwise, let $z$
                be a non-zero elements such that,
                for all $x\in{N}$, $\langle{x,y}\rangle=0$.
                For all $x,z$, $f(x)z-f(z)x\in{N}$, for:
                \begin{equation*}
                    f(f(x)z-f(z)x)
                    =f(f(x)z)-f(f(z)z)
                    =f(x)f(z)-f(z)f(z)=0
                \end{equation*}
                Therefore:
                \begin{align*}
                    \langle{f(x)z-f(z)x,z}\rangle&=0\\
                    \Rightarrow
                    |f(x)|\norm{z}^{2}-|fz|\langle{x,y}\rangle&=0\\
                    \Rightarrow
                    f(x)
                    &=\langle{x,\frac{f(z)z}{\norm{z}^{2}}}\rangle
                \end{align*}
                Therefore, let $y=\frac{f(z)}{\norm{z}^{2}}z$.
                This is unique since if for all $x\in{H}$,
                $\langle{x,y_{1}}\rangle=\langle{x,y_{2}}\rangle$,
                then $y_{1}=y_{2}$. Finally:
                \begin{equation*}
                    \norm{y}
                    =\frac{|f(z)|}{\norm{z}^{2}}\norm{z}
                    =\frac{|f(z)|}{\norm{z}}
                    \leq\norm{f}
                \end{equation*}
                But also:
                \begin{equation*}
                    |f(x)|=|\langle{x,z}\rangle|
                    \leq\norm{x}\norm{z}
                \end{equation*}
                Thus, $\norm{f}\leq\norm{z}$. But
                $\norm{z}\leq\norm{f}$. Therefore,
                $\norm{f}=\norm{z}$.
            \end{proof}
            Much like in $\mathbb{R}^{n}$, there is a notion of
            orthogonality in a general inner product space.
            \begin{definition}
                Orthognal elements in an inner product space $X$
                are elements $x,y\in{X}$ such that
                $\langle{x,y}\rangle=0$.
            \end{definition}
            There's also a notion of convexity for a general
            vector space.
            \begin{definition}
                A convex subset of a vector space $V$ space is a
                subset $S\subset{V}$ such that, for all
                $x,y\in{V}$ and for all $\lambda\in\mathbb{R}$,
                $\lambda{x}+(1-\lambda)y\in{S}$.
            \end{definition}
            \begin{theorem}
                If $S$ is a subset of $V$, then $S$ is convex.
            \end{theorem}
            Recall that for a general metric space $X$,
            if $S\subset{X}$, we defined
            $dist(x,S)=\inf\{d(x,s):s\in{S}\}$. We proved that,
            if $S$ is compact, then there is an $s\in{S}$ such
            that $dist(x,S)=d(x,s)$. We showed that, without
            compactness, this may not be true. Indeed, even
            complete spaces may lack this property. If
            $X$ is a Hilbert space, however, this property is
            guaranteed.
            \begin{theorem}
                If $H$ is a Hilbert space and if $S\subset{H}$
                is a closed convex subset of $H$, then there is
                a unique $s\in{S}$ such that
                $dist(x,S)=\norm{x-s}$.
            \end{theorem}
            \begin{proof}
                As $dist(x,S)=\inf\{d(x,s):s\in{S}\}$, there is
                a sequence $x_{n}\in{S}$ such that
                $\norm{x-x_{n}}\rightarrow{dist(x,S)}$. Then, by
                Appolonius:
                \begin{equation*}
                    \norm{x-x_{n}}^{2}+\norm{x-x_{m}}^{2}
                    =\frac{1}{2}\norm{x_{n}-x_{m}}^{2}
                    +\frac{1}{2}\norm{\frac{1}{2}(x_{n}+x_{m})-x}^{2}
                    \geq\frac{1}{2}\norm{x_{n}+x_{m}}^{2}
                    +2dist(x,S)
                \end{equation*}
                But $\norm{x-x_{n}}\rightarrow{dist(x,S)}$ and
                $\norm{x-x_{m}}\rightarrow{dist(x,S)}$, so:
                \begin{equation*}
                    \frac{1}{2}\norm{x_{n}-x_{m}}^{2}
                    \leq\norm{x-x-{m}}^{2}
                    +\norm{x-x_{m}}^{2}-2dist(x,S)
                \end{equation*}
                Which can be made arbitrarily small. Therefore,
                $x_{n}$ is Cauchy. But $H$ is a Hilbert space, and
                is therefore complete, and thus $x_{n}$ converges.
                Let $s$ be the limit. As $S$ is closed, $s\in{S}$.
                Moreover, from construction,
                $\norm{x-s}=dist(x,S)$. If there is another point
                $v$, then $\norm{x-s}=\norm{x-v}$. From
                Appolonius:
                \begin{equation*}
                    \norm{x-s}^{2}+\norm{x-v}^{2}
                    \geq\frac{1}{2}\norm{s-v}+2dist(x,S)^{2}
                \end{equation*}
                But $\norm{x-s}=\norm{x-v}=dist(x,S)$, and
                thus $\norm{s-v}=0$. Therefore, $s=v$.
            \end{proof}
            Is $S$ is a closed subspace of $H$, then it's
            automatically convex. In this case, $x-s\perp{S}$,
            where $z\perp{S}$ means that, for all
            $s\in{S}$, $\langle{s,z}\rangle=0$. For if
            $z\in{S}$, then $s+tz\in{S}$ for all $t$. Thus:
            \begin{equation*}
                \norm{s+tz-x}\geq{dist(x,S)}=\norm{s-x}^{2}
            \end{equation*}
            And therefore:
            \begin{equation*}
                \rangle{s-x,s-x}\rangle+2t\langle{s-x,z}
                +t^{2}\langle{z,z}\geq{s-x}^{2}
                \Rightarrow{t^{2}\norm{z}^{2}+2t\langle{s-x,z}}
                \geq{0}
            \end{equation*}
            Looking at the discriminant of this polynomial, we
            have:
            \begin{equation*}
                \langle{s-x,z}\rangle=0
            \end{equation*}
            Therefore, $s-x\perp{S}$. You obtain $s\in{S}$
            by ``dropping the perpendicular of $x$,'' onto
            $S$. That is, $s$ is the orthogonal projection
            of $x$ onto $S$. $s=P_{S}x$ where
            $P_{S}:H\rightarrow{H}$ is the orthogonal
            projection. This has a few nice properties:
            \begin{enumerate}
                \item It is idempotent: $P_{S}^{2}=P_{S}$.
                \item Self adjoint:
                      $\langle{P_{S}x,y}\rangle%
                       =\langle{x,P_{S}y}\rangle=$
                \item Linear.
                \item Bounded and $\norm{P_{S}}=1$.
            \end{enumerate}
            If $S$ is a subset of an inner product space $X$,
            we write $S^{\perp}=\{x\in{X}:\langle{x,s}\rangle=0\}$.
            This is often read aloud as ``$S$ perp'' or
            ``$S$ perpendicular.''
            \begin{theorem}
                If $S\subset{X}$, then $S^{\perp}$ is a
                closed subspace.
            \end{theorem}
            \begin{theorem}
                $S\subset(S^{\perp})^{\perp}$
            \end{theorem}
            The direct sum of two subsets of a Hilbert space
            $H$ is
            $S_{1}\oplus{S_{2}}=\{ax+by:x\in{S_{1}},y\in{S_{2}}\}$.
            \begin{theorem}
                If $H$ is a Hilbert space and $S$ is a closed
                subspace of $H$, then $H=S\oplus{S^{\perp}}$
            \end{theorem}
            \begin{proof}
                For $x=P_{S}(x)+(x-P_{S}(x))$, and thus there is
                an element in $S$ and an element in $S^{\perp}$
                such that $x$ is the sum of those two elements.
                This is the only representation. For if
                $x=s_{1}+s_{1}^{\perp}$ and
                $x=s_{2}+s_{2}^{\perp}$, then stuff.
            \end{proof}
            If $X$ and $Y$ are normed spaces, and if
            $f\in{B(X,Y)}$, then
            $\{x\in{X}:f(x)=0\in{Y}\}$ is called the null
            space of $f$.
            \begin{theorem}
                If $X$ and $Y$ are normed spaces, and if
                $f\in{B(X,Y)}$, then $\nul(f)$ is a closed
                linear subspace of $X$.
            \end{theorem}
            \begin{proof}
                Obvious since $f$ is linear and continuous.
            \end{proof}
            In a Hilbert space $H$, then
            $H=\nul(f)\oplus\nul(f)^{\perp}$. Thus, if
            $\nul(f)\ne{H}$, then $\nul(f)^{\perp}\ne\{0\}$.
            That is, there exists a $z\in\nul(f)^{\perp}$ that
            is non-zero. This is the $z$ we used to prove the
            Riesz representation theorem. Riesz's
            Theorem thus says that every Hilbert space is its own dual.
    \subsection{Lecture 9: November 5, 2018}
        \subsubsection{Adjoint}
            If $H$ is a Hilbert space and
            $f\in{H'}$, then there is a $z\in{H}$
            such that $f(x)=\langle{x,z}\rangle$ for
            all $x\in{H}$. Moreover, $\norm{f}=\norm{z}$.
            The adjoint of $T\in{B(H,H)}$ is an
            operator $T^{*}:H\rightarrow{H}$ such that
            $\langle{Tx,y}\rangle=\langle{x,T^{*}}\rangle$.
            There is always such an operator for any
            $T\in{B(H,H)}$. $T^{*}$ is also bounded and
            linear. By Riesz there is a $z=T^{*}y$ such
            that $f(x)=\langle{x,T^{*}y}\rangle$. Then
            $\norm{T^{*}y}=\norm{z}=\norm{f}\leq\norm{T}\norm{y}$.
            Thus, $\norm{T^{*}}\leq\norm{T}$. Therefore
            $T^{*}\in{B(H,H)}$. $T^{*}$ is called the
            ajdoint of $T$.
            \begin{example}
                Consider $\mathbb{R}^{n}$ with the usual
                inner product. Let $T$ be the matrix
                $(T_{ij})$. Then:
                \begin{equation*}
                    (Tx)_{i}=\sum_{j=1}^{n}T_{ij}x_{j}    
                \end{equation*}
                and:
                \begin{equation*}
                    \langle{Tx,y}\rangle
                    =\sum_{i=1}^{n}(Tx)_{i}y_{i}
                    =\sum_{i=1}^{n}\sum_{j=1}^{n}
                    T_{ij}x_{j}y_{i}
                    =\sum_{j=1}^{n}\sum_{i=1}^{n}
                    T_{ij}y_{i}x_{j}
                \end{equation*}
                If $T^{*}$ is the adjoint, then:
                \begin{equation*}
                    \langle{x,T^{*}y}\rangle
                    =\sum_{j=1}^{n}
                    \Big(\sum_{i=1}^{n}T^{*}_{ji}y_{i}\Big)x_{j}
                \end{equation*}
                And thus $T^{*}_{ji}=T_{ij}$. That is, the adjoint
                of $T$ is the transpose of $T$. If we were in
                $\mathbb{C}^{n}$ we would use the complex conjugate
                of the transpose of $T$. In general, if $T=T^{*}$
                we say that $T$ is \textit{self-adjoint}. This is
                also called symmetric or Hermitian.
            \end{example}
            \begin{example}
                As another example, consider $H=\ell^{2}$
                and let $T(x_{1},x_{2},\hdots)=(x_{2},x_{3},\hdots)$.
                This is linear, and:
                \begin{equation*}
                    \norm{T(x_{1},x_{2},\hdots)}
                    =\norm{(x_{2},x_{3},\hdots)}
                    =\sqrt{\sum_{n=2}^{\infty}x_{n}^{2}}
                    \leq
                    \sqrt{\sum_{n=1}^{\infty}x_{n}^{2}}
                    =\norm{(x_{1},x_{2},\hdots)}
                \end{equation*}
                Therefore $T$ is bounded and
                $\norm{T}\leq{1}$. But
                $T(0,1,0,0,\hdots)=(1,0,0,\hdots)$ showing that
                $\norm{T}\geq{1}$. Thus, $\norm{T}=1$. Then, from
                the definition of $T$:
                \begin{align*}
                    \langle{Tx,y}\rangle
                    &=\langle{(x_{2},x_{3},\hdots),
                              (y_{1},y_{2},\hdots)}\rangle\\
                    &=x_{2}y_{1}+x_{3}y_{2}+\hdots\\
                    &={x_{1}}\cdot{0}+x_{2}y_{1}+x_{3}y_{2}+\hdots\\
                    &=\langle{(x_{1},x_{2},\hdots),
                              (0,y_{1},y_{2},\hdots)}\rangle
                \end{align*}
                And therefore
                $T^{*}(y_{1},y_{2},\hdots)=(0,y_{1},y_{2},\hdots)$.
                Also $\norm{T^{*}}=1$. In general, if $T\in{B(H,H)}$
                then $\norm{T^{*}}=\norm{T}$.
            \end{example}
            \begin{theorem}
                If $T\in{B(H,H)}$, then
                $T^{**}=T$.
            \end{theorem}
            \begin{theorem}
                $\norm{T}=\norm{T^{*}}$
            \end{theorem}
            \begin{proof}
                For $\norm{T}\leq\norm{T^{*}}$ and
                $\norm{T^{*}}\leq\norm{T^{**}}$, but
                $T=T^{**}$, and therefore $\norm{T}=\norm{T^{*}}$.
            \end{proof}
            \begin{example}
                Let $x=C[0,1]$ and let
                $\langle{x,y}\rangle=\int_{0}^{1}x(t)y(t)\diff{t}$.
                Let $K:X\times{X}\rightarrow\mathbb{R}$ be continuous
                and define $T$ by:
                \begin{equation*}
                    Tx(t)=\int_{0}^{1}K(t,s)x(s)\diff{s}
                \end{equation*}
                Then for all $x\in{X}$, $Tx\in{X}$ as well,
                since $K$ is continuous. Moreover, from Cauchy-Schwarz:
                \begin{equation*}
                    \norm{Tx}^{2}
                    =\int_{0}^{1}
                    \Big[\int_{0}^{1}K(t,s)x(d)\diff{s}\Big]^{2}\diff{t}
                    \leq\int_{0}^{1}
                    \Big[\int_{0}^{1}K(t,s)^{2}\diff{s}
                    \int_{0}^{1}x(s)^{2}\diff{s}\Big]\diff{t}
                \end{equation*}
                But $\int_{0}^{1}x(s)^{2}\diff{s}=\norm{x}^{2}$. So:
                \begin{equation*}
                    \norm{Tx}^{2}\leq
                    \norm{x}^{2}\int_{0}^{1}\int_{0}^{1}
                    K(t,s)\diff{s}\diff{t}
                \end{equation*}
                Therefore $T$ is bounded and:
                \begin{equation*}
                    \norm{T}\leq
                    \sqrt{\int_{0}^{1}\int_{0}^{1}K(t,s)\diff{s}\diff{t}}
                \end{equation*}
                Computing the adjoint:
                \begin{align*}
                    \langle{Tx,y}\rangle
                    &=\int_{0}^{1}Tx(t)y(t)\diff{t}\\
                    &=\int_{0}^{1}\Big(
                    \int_{0}^{1}K(t,s)x(s)\diff{s}\Big)
                    y(t)\diff{t}\\
                    &=\int_{0}^{1}\Big(
                    \int_{0}^{1}K(t,s)y(t)\diff{t}\Big)
                    x(s)\diff{s}\\
                    &=\int_{0}^{1}\Big(
                    \int_{0}^{1}K(s,t)y(s)\diff{s}\Big)
                    x(t)\diff{t}\\
                    &=\int_{0}^{1}Ty(s)x(s)\diff{s}\\
                    &=\langle{Ty,x}\rangle
                \end{align*}
                We may swap the order of integration since $K$
                is continuous on a compact set.
            \end{example}
            \begin{theorem}
                $\norm{T^{*}T}=\norm{T}^{2}$
            \end{theorem}
            \begin{proof}
                For:
                \begin{equation*}
                    \norm{T^{*}Tx}\leq\norm{T^{*}}\norm{Tx}\leq
                    \norm{T^{*}}\norm{T}\norm{x}
                \end{equation*}
                And therefore $\norm{T^{*}T}\leq\norm{T}^{2}$.
                On the other hand:
            \end{proof}
            \begin{theorem}
                If $T$ is self-adjoint, then
                $\norm{T}=\sup\{|\langle{Tx,x}\rangle|:\norm{x}=1\}$.
            \end{theorem}
            \begin{proof}
                \label{thm:Funct:Norm_of_Self_Adjoint_Operator}
                Let
                $\alpha=\sup\{sup\{|\langle{Tx,x}\rangle|:\norm{x}=1\}$.
                Then:
                \begin{equation*}
                    |\langle{Tx,x}\rangle|
                    \leq\norm{Tx}\norm{x}\leq\norm{T}\norm{x}^{2}
                \end{equation*}
                Taking the supremum over $\norm{x}=1$, we have
                $\alpha\leq\norm{T}$. But if $\norm{x}=\norm{y}=1$,
                then:
                \begin{align*}
                    |\langle{Tx,y}\rangle|
                    &=|\frac{1}{4}\langle{T(x+y),x+y}\rangle
                    -\frac{1}{4}\langle{T(x-y),x-y}\rangle\\
                    &=|\frac{1}{4}\norm{x+y}^{2}\langle
                    T\frac{x+y}{\norm{x+y}},\frac{x+y}{\norm{x+y}}\rangle
                    -\frac{1}{4}\norm{x-y}\langle
                    T\frac{x-y}{\norm{x-y}},\frac{x-y}{\norm{x-y}}\rangle
                \end{align*}
                Since $\langle{Tx,y}\rangle=\langle{x,Ty}\rangle$,
                as $T$ is self-adjoint. And from the definition of
                $\alpha$:
                \begin{equation*}
                    |\langle{Tx,y}\rangle|
                    \leq\frac{\alpha}{4}
                    \big(\norm{x+y}^{2}+\norm{x-y}^{2}\big)
                    \leq\frac{\alpha}{4}
                    \big(2\norm{x}^{2}+2\norm{y}^{2}\big)
                    =\alpha
                \end{equation*}
                Let $y=Tx/\norm{Tx}$, we get:
                \begin{equation*}
                    \langle{Tx,\frac{Tx}{\norm{Tx}}}\rangle\leq\alpha
                \end{equation*}
                And therefore $\norm{T}\leq\alpha$. But also
                $\alpha\leq\norm{T}$. Thus, $\norm{T}=\alpha$.
            \end{proof}
            Thm.~\ref{thm:Funct:Norm_of_Self_Adjoint_Operator} can
            fail if $T$ is not self-adjoint. In $\mathbb{R}^{2}$, let
            $T(x_{1},x_{2})=(0,x_{1})$. Then:
            \begin{equation*}
                \norm{Tx}^{2}=x_{1}^{2}\leq{x_{1}^{2}+x_{2}^{2}}
                =\norm{x}^{2}
            \end{equation*}
            And therefore $\norm{T}\leq{1}$. But $T(1,0)=(0,1)$, and
            thus $\norm{T}=1$. But if $(x_{1},x_{2})$ lies on the
            unit circle, then $|x_{1}x_{2}|\leq0.5$. Thus:
            \begin{equation*}
                |\langle{Tx,x}\rangle|
                =|\langle(x_{1},x_{2}),(0,x_{1})\rangle
                =|x_{1}x_{2}|\leq\frac{1}{2}
            \end{equation*}
            Therefore $|\langle{Tx,x}\rangle|\leq{0.5}<\norm{T}$ for
            all $x\in\mathbb{R}^{2}$ such that $\norm{x}=1$.
        \subsubsection{Compact Operators}
            Compact operators can be defined in a more general spaces
            than that of Hilbert or Banach spaces. They can be defined
            on Topological spaces, but we won't go that far. For now
            we will simply define them on a general metric space.
            \begin{definition}
                A compact mapping from a metric space $X$ to a metric
                space $Y$ is a function $T:X\rightarrow{Y}$ such that
                for all bounded subsets $S$ of $X$, the image
                $T(S)$ is pre-compact in $Y$. That is,
                $\overline{T(S)}$ is compact
                (The closure of $T(S)$ is compact).
            \end{definition}
            \begin{theorem}
                If $T:X\rightarrow{Y}$ is a linear compact operator
                between normed spaces $X$ and $Y$, then $T$ is continuous.
            \end{theorem}
            \begin{proof}
                For let $S=\overline{B_{1}(\mathbf{0})}$.
                This is bounded, so
                $\overline{T(S)}$ is compact, and therefore bounded.
                Let $M$ be such a bound.
                Thus, for all $s\in\overline{S}$ such that $\norm{s}=1$,
                $\norm{Ts}\leq{M}$, and therefore $\norm{T}\leq{M}$.
                Thus $T$ is bounded and linear, and is therefore
                continuous.
            \end{proof}
            \begin{example}
                Every linear mapping
                $T:\mathbb{R}^{n}\rightarrow\mathbb{R}^{m}$ is compact.
                As a another example, let
                $X=C[0,1]$ and equip this with the supremum norm.
                Define $T$ as:
                \begin{equation*}
                    Tx(t)=\int_{0}^{1}K(t,s)x(s)\diff{s}
                \end{equation*}
                Where $K:[0,1]\times[0,1]\rightarrow\mathbb{R}$
                is continuous. This is a compact operator. For if
                $S$ is a bounded subset then there exists an $M$ such
                that for all $x\in{S}$, $\norm{x}\leq{M}$. Thus:
                \begin{equation*}
                    \norm{Tx}=\sup|Tx(t)|
                    =\sup|\int_{0}^{1}K(t,s)x(s)\diff{s}|
                    \leq\sup\int_{0}^{1}|K(t,s)||x(s)|\diff{s}
                    \leq\kappa\int_{0}^{1}|x(s)|\diff{s}
                    \leq\kappa\norm{x}
                \end{equation*}
                Where $\kappa=\sup|K(t,s)|$. $\kappa$ exists since
                $K(t,s)$ is continuous on a compact set and is therefore
                bounded. So $T(S)$ is uniformly bounded. To apply
                Arzela-Ascoli we need to show that
                $T(S)$ is equicontinuous. That is, for all
                $\varepsilon>0$ there is a $\delta>0$ such that,
                for all $x\in{S}$, if $|t_{2}-t_{1}|<\delta$
                then $|Tx(t_{2})-Tx(t_{1})|<\varepsilon$. If we
                can show that $T$ satisfies this, then
                $\overline{T(S)}$ is compact, and thus $T$ is compact.
                Let's show this. If $x\in{S}$, then:
                \begin{align*}
                    |Tx(t_{2})-Tx(t_{1})|
                    &=\Big|\int_{0}^{1}K(t_{1},s)x(s)\diff{s}
                    -\int_{0}^{1}K(t_{2},s)x(s)\diff{s}\Big|\\
                    &=\Big|\int_{0}^{1}(K(t_{2},s)-
                    K(t_{1},s))x(s)\diff{s}\Big|\\
                    &\leq
                    \int_{0}^{1}|K(t_{2},s)-K(t_{1},s)||x(s)|\diff{s}\\
                    &\leq{M}\int_{0}^{1}|K(t_{2},s)-K(t_{1},s)|\diff{s}
                \end{align*}
                But as $K$ is uniformly continuous, there is a $\delta>0$
                such that, for all $s\in[0,1]$,
                $|t_{2}-t_{1}|<\delta$ implies
                $|K(t_{2},s)-K(t_{1},s)|<\varepsilon/M$.
                Thus, $T(S)$ is equicontinuous. We can replace the
                supremum norm with $L^{2}$ and $T$ is still compact.
                Indeed, it is truee for $L^{p}$ if we replace the
                use of Cauchy-Schwarz with the more general
                H\"{o}lder's Inequality. From this we have that
                $T$ is a compact self-adjoint operator.
            \end{example}
\end{document}
    %        \documentclass[crop=false,class=book,oneside]{standalone}                      %
%----------------------------------Preamble------------------------------------%
%---------------------------Packages----------------------------%
\usepackage{geometry}
\geometry{b5paper, margin=1.0in}
\usepackage[T1]{fontenc}
\usepackage{graphicx, float}            % Graphics/Images.
\usepackage{natbib}                     % For bibliographies.
\bibliographystyle{agsm}                % Bibliography style.
\usepackage[french, english]{babel}     % Language typesetting.
\usepackage[dvipsnames]{xcolor}         % Color names.
\usepackage{listings}                   % Verbatim-Like Tools.
\usepackage{mathtools, esint, mathrsfs} % amsmath and integrals.
\usepackage{amsthm, amsfonts, amssymb}  % Fonts and theorems.
\usepackage{tcolorbox}                  % Frames around theorems.
\usepackage{upgreek}                    % Non-Italic Greek.
\usepackage{fmtcount, etoolbox}         % For the \book{} command.
\usepackage[newparttoc]{titlesec}       % Formatting chapter, etc.
\usepackage{titletoc}                   % Allows \book in toc.
\usepackage[nottoc]{tocbibind}          % Bibliography in toc.
\usepackage[titles]{tocloft}            % ToC formatting.
\usepackage{pgfplots, tikz}             % Drawing/graphing tools.
\usepackage{imakeidx}                   % Used for index.
\usetikzlibrary{
    calc,                   % Calculating right angles and more.
    angles,                 % Drawing angles within triangles.
    arrows.meta,            % Latex and Stealth arrows.
    quotes,                 % Adding labels to angles.
    positioning,            % Relative positioning of nodes.
    decorations.markings,   % Adding arrows in the middle of a line.
    patterns,
    arrows
}                                       % Libraries for tikz.
\pgfplotsset{compat=1.9}                % Version of pgfplots.
\usepackage[font=scriptsize,
            labelformat=simple,
            labelsep=colon]{subcaption} % Subfigure captions.
\usepackage[font={scriptsize},
            hypcap=true,
            labelsep=colon]{caption}    % Figure captions.
\usepackage[pdftex,
            pdfauthor={Ryan Maguire},
            pdftitle={Mathematics and Physics},
            pdfsubject={Mathematics, Physics, Science},
            pdfkeywords={Mathematics, Physics, Computer Science, Biology},
            pdfproducer={LaTeX},
            pdfcreator={pdflatex}]{hyperref}
\hypersetup{
    colorlinks=true,
    linkcolor=blue,
    filecolor=magenta,
    urlcolor=Cerulean,
    citecolor=SkyBlue
}                           % Colors for hyperref.
\usepackage[toc,acronym,nogroupskip,nopostdot]{glossaries}
\usepackage{glossary-mcols}
%------------------------Theorem Styles-------------------------%
\theoremstyle{plain}
\newtheorem{theorem}{Theorem}[section]

% Define theorem style for default spacing and normal font.
\newtheoremstyle{normal}
    {\topsep}               % Amount of space above the theorem.
    {\topsep}               % Amount of space below the theorem.
    {}                      % Font used for body of theorem.
    {}                      % Measure of space to indent.
    {\bfseries}             % Font of the header of the theorem.
    {}                      % Punctuation between head and body.
    {.5em}                  % Space after theorem head.
    {}

% Italic header environment.
\newtheoremstyle{thmit}{\topsep}{\topsep}{}{}{\itshape}{}{0.5em}{}

% Define environments with italic headers.
\theoremstyle{thmit}
\newtheorem*{solution}{Solution}

% Define default environments.
\theoremstyle{normal}
\newtheorem{example}{Example}[section]
\newtheorem{definition}{Definition}[section]
\newtheorem{problem}{Problem}[section]

% Define framed environment.
\tcbuselibrary{most}
\newtcbtheorem[use counter*=theorem]{ftheorem}{Theorem}{%
    before=\par\vspace{2ex},
    boxsep=0.5\topsep,
    after=\par\vspace{2ex},
    colback=green!5,
    colframe=green!35!black,
    fonttitle=\bfseries\upshape%
}{thm}

\newtcbtheorem[auto counter, number within=section]{faxiom}{Axiom}{%
    before=\par\vspace{2ex},
    boxsep=0.5\topsep,
    after=\par\vspace{2ex},
    colback=Apricot!5,
    colframe=Apricot!35!black,
    fonttitle=\bfseries\upshape%
}{ax}

\newtcbtheorem[use counter*=definition]{fdefinition}{Definition}{%
    before=\par\vspace{2ex},
    boxsep=0.5\topsep,
    after=\par\vspace{2ex},
    colback=blue!5!white,
    colframe=blue!75!black,
    fonttitle=\bfseries\upshape%
}{def}

\newtcbtheorem[use counter*=example]{fexample}{Example}{%
    before=\par\vspace{2ex},
    boxsep=0.5\topsep,
    after=\par\vspace{2ex},
    colback=red!5!white,
    colframe=red!75!black,
    fonttitle=\bfseries\upshape%
}{ex}

\newtcbtheorem[auto counter, number within=section]{fnotation}{Notation}{%
    before=\par\vspace{2ex},
    boxsep=0.5\topsep,
    after=\par\vspace{2ex},
    colback=SeaGreen!5!white,
    colframe=SeaGreen!75!black,
    fonttitle=\bfseries\upshape%
}{not}

\newtcbtheorem[use counter*=remark]{fremark}{Remark}{%
    fonttitle=\bfseries\upshape,
    colback=Goldenrod!5!white,
    colframe=Goldenrod!75!black}{ex}

\newenvironment{bproof}{\textit{Proof.}}{\hfill$\square$}
\tcolorboxenvironment{bproof}{%
    blanker,
    breakable,
    left=3mm,
    before skip=5pt,
    after skip=10pt,
    borderline west={0.6mm}{0pt}{green!80!black}
}

\AtEndEnvironment{lexample}{$\hfill\textcolor{red}{\blacksquare}$}
\newtcbtheorem[use counter*=example]{lexample}{Example}{%
    empty,
    title={Example~\theexample},
    boxed title style={%
        empty,
        size=minimal,
        toprule=2pt,
        top=0.5\topsep,
    },
    coltitle=red,
    fonttitle=\bfseries,
    parbox=false,
    boxsep=0pt,
    before=\par\vspace{2ex},
    left=0pt,
    right=0pt,
    top=3ex,
    bottom=1ex,
    before=\par\vspace{2ex},
    after=\par\vspace{2ex},
    breakable,
    pad at break*=0mm,
    vfill before first,
    overlay unbroken={%
        \draw[red, line width=2pt]
            ([yshift=-1.2ex]title.south-|frame.west) to
            ([yshift=-1.2ex]title.south-|frame.east);
        },
    overlay first={%
        \draw[red, line width=2pt]
            ([yshift=-1.2ex]title.south-|frame.west) to
            ([yshift=-1.2ex]title.south-|frame.east);
    },
}{ex}

\AtEndEnvironment{ldefinition}{$\hfill\textcolor{Blue}{\blacksquare}$}
\newtcbtheorem[use counter*=definition]{ldefinition}{Definition}{%
    empty,
    title={Definition~\thedefinition:~{#1}},
    boxed title style={%
        empty,
        size=minimal,
        toprule=2pt,
        top=0.5\topsep,
    },
    coltitle=Blue,
    fonttitle=\bfseries,
    parbox=false,
    boxsep=0pt,
    before=\par\vspace{2ex},
    left=0pt,
    right=0pt,
    top=3ex,
    bottom=0pt,
    before=\par\vspace{2ex},
    after=\par\vspace{1ex},
    breakable,
    pad at break*=0mm,
    vfill before first,
    overlay unbroken={%
        \draw[Blue, line width=2pt]
            ([yshift=-1.2ex]title.south-|frame.west) to
            ([yshift=-1.2ex]title.south-|frame.east);
        },
    overlay first={%
        \draw[Blue, line width=2pt]
            ([yshift=-1.2ex]title.south-|frame.west) to
            ([yshift=-1.2ex]title.south-|frame.east);
    },
}{def}

\AtEndEnvironment{ltheorem}{$\hfill\textcolor{Green}{\blacksquare}$}
\newtcbtheorem[use counter*=theorem]{ltheorem}{Theorem}{%
    empty,
    title={Theorem~\thetheorem:~{#1}},
    boxed title style={%
        empty,
        size=minimal,
        toprule=2pt,
        top=0.5\topsep,
    },
    coltitle=Green,
    fonttitle=\bfseries,
    parbox=false,
    boxsep=0pt,
    before=\par\vspace{2ex},
    left=0pt,
    right=0pt,
    top=3ex,
    bottom=-1.5ex,
    breakable,
    pad at break*=0mm,
    vfill before first,
    overlay unbroken={%
        \draw[Green, line width=2pt]
            ([yshift=-1.2ex]title.south-|frame.west) to
            ([yshift=-1.2ex]title.south-|frame.east);},
    overlay first={%
        \draw[Green, line width=2pt]
            ([yshift=-1.2ex]title.south-|frame.west) to
            ([yshift=-1.2ex]title.south-|frame.east);
    }
}{thm}

%--------------------Declared Math Operators--------------------%
\DeclareMathOperator{\adjoint}{adj}         % Adjoint.
\DeclareMathOperator{\Card}{Card}           % Cardinality.
\DeclareMathOperator{\curl}{curl}           % Curl.
\DeclareMathOperator{\diam}{diam}           % Diameter.
\DeclareMathOperator{\dist}{dist}           % Distance.
\DeclareMathOperator{\Div}{div}             % Divergence.
\DeclareMathOperator{\Erf}{Erf}             % Error Function.
\DeclareMathOperator{\Erfc}{Erfc}           % Complementary Error Function.
\DeclareMathOperator{\Ext}{Ext}             % Exterior.
\DeclareMathOperator{\GCD}{GCD}             % Greatest common denominator.
\DeclareMathOperator{\grad}{grad}           % Gradient
\DeclareMathOperator{\Ima}{Im}              % Image.
\DeclareMathOperator{\Int}{Int}             % Interior.
\DeclareMathOperator{\LC}{LC}               % Leading coefficient.
\DeclareMathOperator{\LCM}{LCM}             % Least common multiple.
\DeclareMathOperator{\LM}{LM}               % Leading monomial.
\DeclareMathOperator{\LT}{LT}               % Leading term.
\DeclareMathOperator{\Mod}{mod}             % Modulus.
\DeclareMathOperator{\Mon}{Mon}             % Monomial.
\DeclareMathOperator{\multideg}{mutlideg}   % Multi-Degree (Graphs).
\DeclareMathOperator{\nul}{nul}             % Null space of operator.
\DeclareMathOperator{\Ord}{Ord}             % Ordinal of ordered set.
\DeclareMathOperator{\Prin}{Prin}           % Principal value.
\DeclareMathOperator{\proj}{proj}           % Projection.
\DeclareMathOperator{\Refl}{Refl}           % Reflection operator.
\DeclareMathOperator{\rk}{rk}               % Rank of operator.
\DeclareMathOperator{\sgn}{sgn}             % Sign of a number.
\DeclareMathOperator{\sinc}{sinc}           % Sinc function.
\DeclareMathOperator{\Span}{Span}           % Span of a set.
\DeclareMathOperator{\Spec}{Spec}           % Spectrum.
\DeclareMathOperator{\supp}{supp}           % Support
\DeclareMathOperator{\Tr}{Tr}               % Trace of matrix.
%--------------------Declared Math Symbols--------------------%
\DeclareMathSymbol{\minus}{\mathbin}{AMSa}{"39} % Unary minus sign.
%------------------------New Commands---------------------------%
\DeclarePairedDelimiter\norm{\lVert}{\rVert}
\DeclarePairedDelimiter\ceil{\lceil}{\rceil}
\DeclarePairedDelimiter\floor{\lfloor}{\rfloor}
\newcommand*\diff{\mathop{}\!\mathrm{d}}
\newcommand*\Diff[1]{\mathop{}\!\mathrm{d^#1}}
\renewcommand*{\glstextformat}[1]{\textcolor{RoyalBlue}{#1}}
\renewcommand{\glsnamefont}[1]{\textbf{#1}}
\renewcommand\labelitemii{$\circ$}
\renewcommand\thesubfigure{%
    \arabic{chapter}.\arabic{figure}.\arabic{subfigure}}
\addto\captionsenglish{\renewcommand{\figurename}{Fig.}}
\numberwithin{equation}{section}

\renewcommand{\vector}[1]{\boldsymbol{\mathrm{#1}}}

\newcommand{\uvector}[1]{\boldsymbol{\hat{\mathrm{#1}}}}
\newcommand{\topspace}[2][]{(#2,\tau_{#1})}
\newcommand{\measurespace}[2][]{(#2,\varSigma_{#1},\mu_{#1})}
\newcommand{\measurablespace}[2][]{(#2,\varSigma_{#1})}
\newcommand{\manifold}[2][]{(#2,\tau_{#1},\mathcal{A}_{#1})}
\newcommand{\tanspace}[2]{T_{#1}{#2}}
\newcommand{\cotanspace}[2]{T_{#1}^{*}{#2}}
\newcommand{\Ckspace}[3][\mathbb{R}]{C^{#2}(#3,#1)}
\newcommand{\funcspace}[2][\mathbb{R}]{\mathcal{F}(#2,#1)}
\newcommand{\smoothvecf}[1]{\mathfrak{X}(#1)}
\newcommand{\smoothonef}[1]{\mathfrak{X}^{*}(#1)}
\newcommand{\bracket}[2]{[#1,#2]}

%------------------------Book Command---------------------------%
\makeatletter
\renewcommand\@pnumwidth{1cm}
\newcounter{book}
\renewcommand\thebook{\@Roman\c@book}
\newcommand\book{%
    \if@openright
        \cleardoublepage
    \else
        \clearpage
    \fi
    \thispagestyle{plain}%
    \if@twocolumn
        \onecolumn
        \@tempswatrue
    \else
        \@tempswafalse
    \fi
    \null\vfil
    \secdef\@book\@sbook
}
\def\@book[#1]#2{%
    \refstepcounter{book}
    \addcontentsline{toc}{book}{\bookname\ \thebook:\hspace{1em}#1}
    \markboth{}{}
    {\centering
     \interlinepenalty\@M
     \normalfont
     \huge\bfseries\bookname\nobreakspace\thebook
     \par
     \vskip 20\p@
     \Huge\bfseries#2\par}%
    \@endbook}
\def\@sbook#1{%
    {\centering
     \interlinepenalty \@M
     \normalfont
     \Huge\bfseries#1\par}%
    \@endbook}
\def\@endbook{
    \vfil\newpage
        \if@twoside
            \if@openright
                \null
                \thispagestyle{empty}%
                \newpage
            \fi
        \fi
        \if@tempswa
            \twocolumn
        \fi
}
\newcommand*\l@book[2]{%
    \ifnum\c@tocdepth >-3\relax
        \addpenalty{-\@highpenalty}%
        \addvspace{2.25em\@plus\p@}%
        \setlength\@tempdima{3em}%
        \begingroup
            \parindent\z@\rightskip\@pnumwidth
            \parfillskip -\@pnumwidth
            {
                \leavevmode
                \Large\bfseries#1\hfill\hb@xt@\@pnumwidth{\hss#2}
            }
            \par
            \nobreak
            \global\@nobreaktrue
            \everypar{\global\@nobreakfalse\everypar{}}%
        \endgroup
    \fi}
\newcommand\bookname{Book}
\renewcommand{\thebook}{\texorpdfstring{\Numberstring{book}}{book}}
\providecommand*{\toclevel@book}{-2}
\makeatother
\titleformat{\part}[display]
    {\Large\bfseries}
    {\partname\nobreakspace\thepart}
    {0mm}
    {\Huge\bfseries}
\titlecontents{part}[0pt]
    {\large\bfseries}
    {\partname\ \thecontentslabel: \quad}
    {}
    {\hfill\contentspage}
\titlecontents{chapter}[0pt]
    {\bfseries}
    {\chaptername\ \thecontentslabel:\quad}
    {}
    {\hfill\contentspage}
\newglossarystyle{longpara}{%
    \setglossarystyle{long}%
    \renewenvironment{theglossary}{%
        \begin{longtable}[l]{{p{0.25\hsize}p{0.65\hsize}}}
    }{\end{longtable}}%
    \renewcommand{\glossentry}[2]{%
        \glstarget{##1}{\glossentryname{##1}}%
        &\glossentrydesc{##1}{~##2.}
        \tabularnewline%
        \tabularnewline
    }%
}
\newglossary[not-glg]{notation}{not-gls}{not-glo}{Notation}
\newcommand*{\newnotation}[4][]{%
    \newglossaryentry{#2}{type=notation, name={\textbf{#3}, },
                          text={#4}, description={#4},#1}%
}
%--------------------------LENGTHS------------------------------%
% Spacings for the Table of Contents.
\addtolength{\cftsecnumwidth}{1ex}
\addtolength{\cftsubsecindent}{1ex}
\addtolength{\cftsubsecnumwidth}{1ex}
\addtolength{\cftfignumwidth}{1ex}
\addtolength{\cfttabnumwidth}{1ex}

% Indent and paragraph spacing.
\setlength{\parindent}{0em}
\setlength{\parskip}{0em}                                                           %
%---------------------------------tikz Path------------------------------------%
\makeatletter                                                                  %
    \def\input@path{{../../../tikz/}}                                          %
\makeatother                                                                   %
%----------------------------------GLOSSARY------------------------------------%
\makeglossaries                                                                %
\loadglsentries{glossary}                                                      %
\loadglsentries{acronym}                                                       %
%--------------------------------Main Document---------------------------------%
\begin{document}
    \ifx\ifmain\undefined
        \pagenumbering{roman}
        \title{Fourier Analysis}
        \author{Ryan Maguire}
        \date{\vspace{-5ex}}
        \maketitle
        \tableofcontents
        \clearpage
        \chapter*{Fourier Analysis}
        \addcontentsline{toc}{chapter}{Fourier Analysis}
        \markboth{}{FOURIER ANALYSIS}
        \vspace{10ex}
        \setcounter{chapter}{1}
        \pagenumbering{arabic}
    \else
        \chapter{Fourier Analysis}
    \fi
    \section{Stuff}
\end{document}
    %        \documentclass[crop=false,class=article,oneside]{standalone}
%----------------------------Preamble-------------------------------%
%---------------------------Packages----------------------------%
\usepackage{geometry}
\geometry{b5paper, margin=1.0in}
\usepackage[T1]{fontenc}
\usepackage{graphicx, float}            % Graphics/Images.
\usepackage{natbib}                     % For bibliographies.
\bibliographystyle{agsm}                % Bibliography style.
\usepackage[french, english]{babel}     % Language typesetting.
\usepackage[dvipsnames]{xcolor}         % Color names.
\usepackage{listings}                   % Verbatim-Like Tools.
\usepackage{mathtools, esint, mathrsfs} % amsmath and integrals.
\usepackage{amsthm, amsfonts, amssymb}  % Fonts and theorems.
\usepackage{tcolorbox}                  % Frames around theorems.
\usepackage{upgreek}                    % Non-Italic Greek.
\usepackage{fmtcount, etoolbox}         % For the \book{} command.
\usepackage[newparttoc]{titlesec}       % Formatting chapter, etc.
\usepackage{titletoc}                   % Allows \book in toc.
\usepackage[nottoc]{tocbibind}          % Bibliography in toc.
\usepackage[titles]{tocloft}            % ToC formatting.
\usepackage{pgfplots, tikz}             % Drawing/graphing tools.
\usepackage{imakeidx}                   % Used for index.
\usetikzlibrary{
    calc,                   % Calculating right angles and more.
    angles,                 % Drawing angles within triangles.
    arrows.meta,            % Latex and Stealth arrows.
    quotes,                 % Adding labels to angles.
    positioning,            % Relative positioning of nodes.
    decorations.markings,   % Adding arrows in the middle of a line.
    patterns,
    arrows
}                                       % Libraries for tikz.
\pgfplotsset{compat=1.9}                % Version of pgfplots.
\usepackage[font=scriptsize,
            labelformat=simple,
            labelsep=colon]{subcaption} % Subfigure captions.
\usepackage[font={scriptsize},
            hypcap=true,
            labelsep=colon]{caption}    % Figure captions.
\usepackage[pdftex,
            pdfauthor={Ryan Maguire},
            pdftitle={Mathematics and Physics},
            pdfsubject={Mathematics, Physics, Science},
            pdfkeywords={Mathematics, Physics, Computer Science, Biology},
            pdfproducer={LaTeX},
            pdfcreator={pdflatex}]{hyperref}
\hypersetup{
    colorlinks=true,
    linkcolor=blue,
    filecolor=magenta,
    urlcolor=Cerulean,
    citecolor=SkyBlue
}                           % Colors for hyperref.
\usepackage[toc,acronym,nogroupskip,nopostdot]{glossaries}
\usepackage{glossary-mcols}
%------------------------Theorem Styles-------------------------%
\theoremstyle{plain}
\newtheorem{theorem}{Theorem}[section]

% Define theorem style for default spacing and normal font.
\newtheoremstyle{normal}
    {\topsep}               % Amount of space above the theorem.
    {\topsep}               % Amount of space below the theorem.
    {}                      % Font used for body of theorem.
    {}                      % Measure of space to indent.
    {\bfseries}             % Font of the header of the theorem.
    {}                      % Punctuation between head and body.
    {.5em}                  % Space after theorem head.
    {}

% Italic header environment.
\newtheoremstyle{thmit}{\topsep}{\topsep}{}{}{\itshape}{}{0.5em}{}

% Define environments with italic headers.
\theoremstyle{thmit}
\newtheorem*{solution}{Solution}

% Define default environments.
\theoremstyle{normal}
\newtheorem{example}{Example}[section]
\newtheorem{definition}{Definition}[section]
\newtheorem{problem}{Problem}[section]

% Define framed environment.
\tcbuselibrary{most}
\newtcbtheorem[use counter*=theorem]{ftheorem}{Theorem}{%
    before=\par\vspace{2ex},
    boxsep=0.5\topsep,
    after=\par\vspace{2ex},
    colback=green!5,
    colframe=green!35!black,
    fonttitle=\bfseries\upshape%
}{thm}

\newtcbtheorem[auto counter, number within=section]{faxiom}{Axiom}{%
    before=\par\vspace{2ex},
    boxsep=0.5\topsep,
    after=\par\vspace{2ex},
    colback=Apricot!5,
    colframe=Apricot!35!black,
    fonttitle=\bfseries\upshape%
}{ax}

\newtcbtheorem[use counter*=definition]{fdefinition}{Definition}{%
    before=\par\vspace{2ex},
    boxsep=0.5\topsep,
    after=\par\vspace{2ex},
    colback=blue!5!white,
    colframe=blue!75!black,
    fonttitle=\bfseries\upshape%
}{def}

\newtcbtheorem[use counter*=example]{fexample}{Example}{%
    before=\par\vspace{2ex},
    boxsep=0.5\topsep,
    after=\par\vspace{2ex},
    colback=red!5!white,
    colframe=red!75!black,
    fonttitle=\bfseries\upshape%
}{ex}

\newtcbtheorem[auto counter, number within=section]{fnotation}{Notation}{%
    before=\par\vspace{2ex},
    boxsep=0.5\topsep,
    after=\par\vspace{2ex},
    colback=SeaGreen!5!white,
    colframe=SeaGreen!75!black,
    fonttitle=\bfseries\upshape%
}{not}

\newtcbtheorem[use counter*=remark]{fremark}{Remark}{%
    fonttitle=\bfseries\upshape,
    colback=Goldenrod!5!white,
    colframe=Goldenrod!75!black}{ex}

\newenvironment{bproof}{\textit{Proof.}}{\hfill$\square$}
\tcolorboxenvironment{bproof}{%
    blanker,
    breakable,
    left=3mm,
    before skip=5pt,
    after skip=10pt,
    borderline west={0.6mm}{0pt}{green!80!black}
}

\AtEndEnvironment{lexample}{$\hfill\textcolor{red}{\blacksquare}$}
\newtcbtheorem[use counter*=example]{lexample}{Example}{%
    empty,
    title={Example~\theexample},
    boxed title style={%
        empty,
        size=minimal,
        toprule=2pt,
        top=0.5\topsep,
    },
    coltitle=red,
    fonttitle=\bfseries,
    parbox=false,
    boxsep=0pt,
    before=\par\vspace{2ex},
    left=0pt,
    right=0pt,
    top=3ex,
    bottom=1ex,
    before=\par\vspace{2ex},
    after=\par\vspace{2ex},
    breakable,
    pad at break*=0mm,
    vfill before first,
    overlay unbroken={%
        \draw[red, line width=2pt]
            ([yshift=-1.2ex]title.south-|frame.west) to
            ([yshift=-1.2ex]title.south-|frame.east);
        },
    overlay first={%
        \draw[red, line width=2pt]
            ([yshift=-1.2ex]title.south-|frame.west) to
            ([yshift=-1.2ex]title.south-|frame.east);
    },
}{ex}

\AtEndEnvironment{ldefinition}{$\hfill\textcolor{Blue}{\blacksquare}$}
\newtcbtheorem[use counter*=definition]{ldefinition}{Definition}{%
    empty,
    title={Definition~\thedefinition:~{#1}},
    boxed title style={%
        empty,
        size=minimal,
        toprule=2pt,
        top=0.5\topsep,
    },
    coltitle=Blue,
    fonttitle=\bfseries,
    parbox=false,
    boxsep=0pt,
    before=\par\vspace{2ex},
    left=0pt,
    right=0pt,
    top=3ex,
    bottom=0pt,
    before=\par\vspace{2ex},
    after=\par\vspace{1ex},
    breakable,
    pad at break*=0mm,
    vfill before first,
    overlay unbroken={%
        \draw[Blue, line width=2pt]
            ([yshift=-1.2ex]title.south-|frame.west) to
            ([yshift=-1.2ex]title.south-|frame.east);
        },
    overlay first={%
        \draw[Blue, line width=2pt]
            ([yshift=-1.2ex]title.south-|frame.west) to
            ([yshift=-1.2ex]title.south-|frame.east);
    },
}{def}

\AtEndEnvironment{ltheorem}{$\hfill\textcolor{Green}{\blacksquare}$}
\newtcbtheorem[use counter*=theorem]{ltheorem}{Theorem}{%
    empty,
    title={Theorem~\thetheorem:~{#1}},
    boxed title style={%
        empty,
        size=minimal,
        toprule=2pt,
        top=0.5\topsep,
    },
    coltitle=Green,
    fonttitle=\bfseries,
    parbox=false,
    boxsep=0pt,
    before=\par\vspace{2ex},
    left=0pt,
    right=0pt,
    top=3ex,
    bottom=-1.5ex,
    breakable,
    pad at break*=0mm,
    vfill before first,
    overlay unbroken={%
        \draw[Green, line width=2pt]
            ([yshift=-1.2ex]title.south-|frame.west) to
            ([yshift=-1.2ex]title.south-|frame.east);},
    overlay first={%
        \draw[Green, line width=2pt]
            ([yshift=-1.2ex]title.south-|frame.west) to
            ([yshift=-1.2ex]title.south-|frame.east);
    }
}{thm}

%--------------------Declared Math Operators--------------------%
\DeclareMathOperator{\adjoint}{adj}         % Adjoint.
\DeclareMathOperator{\Card}{Card}           % Cardinality.
\DeclareMathOperator{\curl}{curl}           % Curl.
\DeclareMathOperator{\diam}{diam}           % Diameter.
\DeclareMathOperator{\dist}{dist}           % Distance.
\DeclareMathOperator{\Div}{div}             % Divergence.
\DeclareMathOperator{\Erf}{Erf}             % Error Function.
\DeclareMathOperator{\Erfc}{Erfc}           % Complementary Error Function.
\DeclareMathOperator{\Ext}{Ext}             % Exterior.
\DeclareMathOperator{\GCD}{GCD}             % Greatest common denominator.
\DeclareMathOperator{\grad}{grad}           % Gradient
\DeclareMathOperator{\Ima}{Im}              % Image.
\DeclareMathOperator{\Int}{Int}             % Interior.
\DeclareMathOperator{\LC}{LC}               % Leading coefficient.
\DeclareMathOperator{\LCM}{LCM}             % Least common multiple.
\DeclareMathOperator{\LM}{LM}               % Leading monomial.
\DeclareMathOperator{\LT}{LT}               % Leading term.
\DeclareMathOperator{\Mod}{mod}             % Modulus.
\DeclareMathOperator{\Mon}{Mon}             % Monomial.
\DeclareMathOperator{\multideg}{mutlideg}   % Multi-Degree (Graphs).
\DeclareMathOperator{\nul}{nul}             % Null space of operator.
\DeclareMathOperator{\Ord}{Ord}             % Ordinal of ordered set.
\DeclareMathOperator{\Prin}{Prin}           % Principal value.
\DeclareMathOperator{\proj}{proj}           % Projection.
\DeclareMathOperator{\Refl}{Refl}           % Reflection operator.
\DeclareMathOperator{\rk}{rk}               % Rank of operator.
\DeclareMathOperator{\sgn}{sgn}             % Sign of a number.
\DeclareMathOperator{\sinc}{sinc}           % Sinc function.
\DeclareMathOperator{\Span}{Span}           % Span of a set.
\DeclareMathOperator{\Spec}{Spec}           % Spectrum.
\DeclareMathOperator{\supp}{supp}           % Support
\DeclareMathOperator{\Tr}{Tr}               % Trace of matrix.
%--------------------Declared Math Symbols--------------------%
\DeclareMathSymbol{\minus}{\mathbin}{AMSa}{"39} % Unary minus sign.
%------------------------New Commands---------------------------%
\DeclarePairedDelimiter\norm{\lVert}{\rVert}
\DeclarePairedDelimiter\ceil{\lceil}{\rceil}
\DeclarePairedDelimiter\floor{\lfloor}{\rfloor}
\newcommand*\diff{\mathop{}\!\mathrm{d}}
\newcommand*\Diff[1]{\mathop{}\!\mathrm{d^#1}}
\renewcommand*{\glstextformat}[1]{\textcolor{RoyalBlue}{#1}}
\renewcommand{\glsnamefont}[1]{\textbf{#1}}
\renewcommand\labelitemii{$\circ$}
\renewcommand\thesubfigure{%
    \arabic{chapter}.\arabic{figure}.\arabic{subfigure}}
\addto\captionsenglish{\renewcommand{\figurename}{Fig.}}
\numberwithin{equation}{section}

\renewcommand{\vector}[1]{\boldsymbol{\mathrm{#1}}}

\newcommand{\uvector}[1]{\boldsymbol{\hat{\mathrm{#1}}}}
\newcommand{\topspace}[2][]{(#2,\tau_{#1})}
\newcommand{\measurespace}[2][]{(#2,\varSigma_{#1},\mu_{#1})}
\newcommand{\measurablespace}[2][]{(#2,\varSigma_{#1})}
\newcommand{\manifold}[2][]{(#2,\tau_{#1},\mathcal{A}_{#1})}
\newcommand{\tanspace}[2]{T_{#1}{#2}}
\newcommand{\cotanspace}[2]{T_{#1}^{*}{#2}}
\newcommand{\Ckspace}[3][\mathbb{R}]{C^{#2}(#3,#1)}
\newcommand{\funcspace}[2][\mathbb{R}]{\mathcal{F}(#2,#1)}
\newcommand{\smoothvecf}[1]{\mathfrak{X}(#1)}
\newcommand{\smoothonef}[1]{\mathfrak{X}^{*}(#1)}
\newcommand{\bracket}[2]{[#1,#2]}

%------------------------Book Command---------------------------%
\makeatletter
\renewcommand\@pnumwidth{1cm}
\newcounter{book}
\renewcommand\thebook{\@Roman\c@book}
\newcommand\book{%
    \if@openright
        \cleardoublepage
    \else
        \clearpage
    \fi
    \thispagestyle{plain}%
    \if@twocolumn
        \onecolumn
        \@tempswatrue
    \else
        \@tempswafalse
    \fi
    \null\vfil
    \secdef\@book\@sbook
}
\def\@book[#1]#2{%
    \refstepcounter{book}
    \addcontentsline{toc}{book}{\bookname\ \thebook:\hspace{1em}#1}
    \markboth{}{}
    {\centering
     \interlinepenalty\@M
     \normalfont
     \huge\bfseries\bookname\nobreakspace\thebook
     \par
     \vskip 20\p@
     \Huge\bfseries#2\par}%
    \@endbook}
\def\@sbook#1{%
    {\centering
     \interlinepenalty \@M
     \normalfont
     \Huge\bfseries#1\par}%
    \@endbook}
\def\@endbook{
    \vfil\newpage
        \if@twoside
            \if@openright
                \null
                \thispagestyle{empty}%
                \newpage
            \fi
        \fi
        \if@tempswa
            \twocolumn
        \fi
}
\newcommand*\l@book[2]{%
    \ifnum\c@tocdepth >-3\relax
        \addpenalty{-\@highpenalty}%
        \addvspace{2.25em\@plus\p@}%
        \setlength\@tempdima{3em}%
        \begingroup
            \parindent\z@\rightskip\@pnumwidth
            \parfillskip -\@pnumwidth
            {
                \leavevmode
                \Large\bfseries#1\hfill\hb@xt@\@pnumwidth{\hss#2}
            }
            \par
            \nobreak
            \global\@nobreaktrue
            \everypar{\global\@nobreakfalse\everypar{}}%
        \endgroup
    \fi}
\newcommand\bookname{Book}
\renewcommand{\thebook}{\texorpdfstring{\Numberstring{book}}{book}}
\providecommand*{\toclevel@book}{-2}
\makeatother
\titleformat{\part}[display]
    {\Large\bfseries}
    {\partname\nobreakspace\thepart}
    {0mm}
    {\Huge\bfseries}
\titlecontents{part}[0pt]
    {\large\bfseries}
    {\partname\ \thecontentslabel: \quad}
    {}
    {\hfill\contentspage}
\titlecontents{chapter}[0pt]
    {\bfseries}
    {\chaptername\ \thecontentslabel:\quad}
    {}
    {\hfill\contentspage}
\newglossarystyle{longpara}{%
    \setglossarystyle{long}%
    \renewenvironment{theglossary}{%
        \begin{longtable}[l]{{p{0.25\hsize}p{0.65\hsize}}}
    }{\end{longtable}}%
    \renewcommand{\glossentry}[2]{%
        \glstarget{##1}{\glossentryname{##1}}%
        &\glossentrydesc{##1}{~##2.}
        \tabularnewline%
        \tabularnewline
    }%
}
\newglossary[not-glg]{notation}{not-gls}{not-glo}{Notation}
\newcommand*{\newnotation}[4][]{%
    \newglossaryentry{#2}{type=notation, name={\textbf{#3}, },
                          text={#4}, description={#4},#1}%
}
%--------------------------LENGTHS------------------------------%
% Spacings for the Table of Contents.
\addtolength{\cftsecnumwidth}{1ex}
\addtolength{\cftsubsecindent}{1ex}
\addtolength{\cftsubsecnumwidth}{1ex}
\addtolength{\cftfignumwidth}{1ex}
\addtolength{\cfttabnumwidth}{1ex}

% Indent and paragraph spacing.
\setlength{\parindent}{0em}
\setlength{\parskip}{0em}
%--------------------------Main Document----------------------------%
\begin{document}
    \ifx\ifworkmasterswork\undefined
        \section*{Calculus on Normed Spaces}
        \setcounter{section}{1}
    \fi
    \subsection{Linear Transformations}
    \subsection{Integration}
    \subsection{Differentiation}
    \subsection{Real Analysis}
\end{document}
    %        \documentclass[crop=false,class=book,oneside]{standalone}
%----------------------------Preamble-------------------------------%
%---------------------------Packages----------------------------%
\usepackage{geometry}
\geometry{b5paper, margin=1.0in}
\usepackage[T1]{fontenc}
\usepackage{graphicx, float}            % Graphics/Images.
\usepackage{natbib}                     % For bibliographies.
\bibliographystyle{agsm}                % Bibliography style.
\usepackage[french, english]{babel}     % Language typesetting.
\usepackage[dvipsnames]{xcolor}         % Color names.
\usepackage{listings}                   % Verbatim-Like Tools.
\usepackage{mathtools, esint, mathrsfs} % amsmath and integrals.
\usepackage{amsthm, amsfonts, amssymb}  % Fonts and theorems.
\usepackage{tcolorbox}                  % Frames around theorems.
\usepackage{upgreek}                    % Non-Italic Greek.
\usepackage{fmtcount, etoolbox}         % For the \book{} command.
\usepackage[newparttoc]{titlesec}       % Formatting chapter, etc.
\usepackage{titletoc}                   % Allows \book in toc.
\usepackage[nottoc]{tocbibind}          % Bibliography in toc.
\usepackage[titles]{tocloft}            % ToC formatting.
\usepackage{pgfplots, tikz}             % Drawing/graphing tools.
\usepackage{imakeidx}                   % Used for index.
\usetikzlibrary{
    calc,                   % Calculating right angles and more.
    angles,                 % Drawing angles within triangles.
    arrows.meta,            % Latex and Stealth arrows.
    quotes,                 % Adding labels to angles.
    positioning,            % Relative positioning of nodes.
    decorations.markings,   % Adding arrows in the middle of a line.
    patterns,
    arrows
}                                       % Libraries for tikz.
\pgfplotsset{compat=1.9}                % Version of pgfplots.
\usepackage[font=scriptsize,
            labelformat=simple,
            labelsep=colon]{subcaption} % Subfigure captions.
\usepackage[font={scriptsize},
            hypcap=true,
            labelsep=colon]{caption}    % Figure captions.
\usepackage[pdftex,
            pdfauthor={Ryan Maguire},
            pdftitle={Mathematics and Physics},
            pdfsubject={Mathematics, Physics, Science},
            pdfkeywords={Mathematics, Physics, Computer Science, Biology},
            pdfproducer={LaTeX},
            pdfcreator={pdflatex}]{hyperref}
\hypersetup{
    colorlinks=true,
    linkcolor=blue,
    filecolor=magenta,
    urlcolor=Cerulean,
    citecolor=SkyBlue
}                           % Colors for hyperref.
\usepackage[toc,acronym,nogroupskip,nopostdot]{glossaries}
\usepackage{glossary-mcols}
%------------------------Theorem Styles-------------------------%
\theoremstyle{plain}
\newtheorem{theorem}{Theorem}[section]

% Define theorem style for default spacing and normal font.
\newtheoremstyle{normal}
    {\topsep}               % Amount of space above the theorem.
    {\topsep}               % Amount of space below the theorem.
    {}                      % Font used for body of theorem.
    {}                      % Measure of space to indent.
    {\bfseries}             % Font of the header of the theorem.
    {}                      % Punctuation between head and body.
    {.5em}                  % Space after theorem head.
    {}

% Italic header environment.
\newtheoremstyle{thmit}{\topsep}{\topsep}{}{}{\itshape}{}{0.5em}{}

% Define environments with italic headers.
\theoremstyle{thmit}
\newtheorem*{solution}{Solution}

% Define default environments.
\theoremstyle{normal}
\newtheorem{example}{Example}[section]
\newtheorem{definition}{Definition}[section]
\newtheorem{problem}{Problem}[section]

% Define framed environment.
\tcbuselibrary{most}
\newtcbtheorem[use counter*=theorem]{ftheorem}{Theorem}{%
    before=\par\vspace{2ex},
    boxsep=0.5\topsep,
    after=\par\vspace{2ex},
    colback=green!5,
    colframe=green!35!black,
    fonttitle=\bfseries\upshape%
}{thm}

\newtcbtheorem[auto counter, number within=section]{faxiom}{Axiom}{%
    before=\par\vspace{2ex},
    boxsep=0.5\topsep,
    after=\par\vspace{2ex},
    colback=Apricot!5,
    colframe=Apricot!35!black,
    fonttitle=\bfseries\upshape%
}{ax}

\newtcbtheorem[use counter*=definition]{fdefinition}{Definition}{%
    before=\par\vspace{2ex},
    boxsep=0.5\topsep,
    after=\par\vspace{2ex},
    colback=blue!5!white,
    colframe=blue!75!black,
    fonttitle=\bfseries\upshape%
}{def}

\newtcbtheorem[use counter*=example]{fexample}{Example}{%
    before=\par\vspace{2ex},
    boxsep=0.5\topsep,
    after=\par\vspace{2ex},
    colback=red!5!white,
    colframe=red!75!black,
    fonttitle=\bfseries\upshape%
}{ex}

\newtcbtheorem[auto counter, number within=section]{fnotation}{Notation}{%
    before=\par\vspace{2ex},
    boxsep=0.5\topsep,
    after=\par\vspace{2ex},
    colback=SeaGreen!5!white,
    colframe=SeaGreen!75!black,
    fonttitle=\bfseries\upshape%
}{not}

\newtcbtheorem[use counter*=remark]{fremark}{Remark}{%
    fonttitle=\bfseries\upshape,
    colback=Goldenrod!5!white,
    colframe=Goldenrod!75!black}{ex}

\newenvironment{bproof}{\textit{Proof.}}{\hfill$\square$}
\tcolorboxenvironment{bproof}{%
    blanker,
    breakable,
    left=3mm,
    before skip=5pt,
    after skip=10pt,
    borderline west={0.6mm}{0pt}{green!80!black}
}

\AtEndEnvironment{lexample}{$\hfill\textcolor{red}{\blacksquare}$}
\newtcbtheorem[use counter*=example]{lexample}{Example}{%
    empty,
    title={Example~\theexample},
    boxed title style={%
        empty,
        size=minimal,
        toprule=2pt,
        top=0.5\topsep,
    },
    coltitle=red,
    fonttitle=\bfseries,
    parbox=false,
    boxsep=0pt,
    before=\par\vspace{2ex},
    left=0pt,
    right=0pt,
    top=3ex,
    bottom=1ex,
    before=\par\vspace{2ex},
    after=\par\vspace{2ex},
    breakable,
    pad at break*=0mm,
    vfill before first,
    overlay unbroken={%
        \draw[red, line width=2pt]
            ([yshift=-1.2ex]title.south-|frame.west) to
            ([yshift=-1.2ex]title.south-|frame.east);
        },
    overlay first={%
        \draw[red, line width=2pt]
            ([yshift=-1.2ex]title.south-|frame.west) to
            ([yshift=-1.2ex]title.south-|frame.east);
    },
}{ex}

\AtEndEnvironment{ldefinition}{$\hfill\textcolor{Blue}{\blacksquare}$}
\newtcbtheorem[use counter*=definition]{ldefinition}{Definition}{%
    empty,
    title={Definition~\thedefinition:~{#1}},
    boxed title style={%
        empty,
        size=minimal,
        toprule=2pt,
        top=0.5\topsep,
    },
    coltitle=Blue,
    fonttitle=\bfseries,
    parbox=false,
    boxsep=0pt,
    before=\par\vspace{2ex},
    left=0pt,
    right=0pt,
    top=3ex,
    bottom=0pt,
    before=\par\vspace{2ex},
    after=\par\vspace{1ex},
    breakable,
    pad at break*=0mm,
    vfill before first,
    overlay unbroken={%
        \draw[Blue, line width=2pt]
            ([yshift=-1.2ex]title.south-|frame.west) to
            ([yshift=-1.2ex]title.south-|frame.east);
        },
    overlay first={%
        \draw[Blue, line width=2pt]
            ([yshift=-1.2ex]title.south-|frame.west) to
            ([yshift=-1.2ex]title.south-|frame.east);
    },
}{def}

\AtEndEnvironment{ltheorem}{$\hfill\textcolor{Green}{\blacksquare}$}
\newtcbtheorem[use counter*=theorem]{ltheorem}{Theorem}{%
    empty,
    title={Theorem~\thetheorem:~{#1}},
    boxed title style={%
        empty,
        size=minimal,
        toprule=2pt,
        top=0.5\topsep,
    },
    coltitle=Green,
    fonttitle=\bfseries,
    parbox=false,
    boxsep=0pt,
    before=\par\vspace{2ex},
    left=0pt,
    right=0pt,
    top=3ex,
    bottom=-1.5ex,
    breakable,
    pad at break*=0mm,
    vfill before first,
    overlay unbroken={%
        \draw[Green, line width=2pt]
            ([yshift=-1.2ex]title.south-|frame.west) to
            ([yshift=-1.2ex]title.south-|frame.east);},
    overlay first={%
        \draw[Green, line width=2pt]
            ([yshift=-1.2ex]title.south-|frame.west) to
            ([yshift=-1.2ex]title.south-|frame.east);
    }
}{thm}

%--------------------Declared Math Operators--------------------%
\DeclareMathOperator{\adjoint}{adj}         % Adjoint.
\DeclareMathOperator{\Card}{Card}           % Cardinality.
\DeclareMathOperator{\curl}{curl}           % Curl.
\DeclareMathOperator{\diam}{diam}           % Diameter.
\DeclareMathOperator{\dist}{dist}           % Distance.
\DeclareMathOperator{\Div}{div}             % Divergence.
\DeclareMathOperator{\Erf}{Erf}             % Error Function.
\DeclareMathOperator{\Erfc}{Erfc}           % Complementary Error Function.
\DeclareMathOperator{\Ext}{Ext}             % Exterior.
\DeclareMathOperator{\GCD}{GCD}             % Greatest common denominator.
\DeclareMathOperator{\grad}{grad}           % Gradient
\DeclareMathOperator{\Ima}{Im}              % Image.
\DeclareMathOperator{\Int}{Int}             % Interior.
\DeclareMathOperator{\LC}{LC}               % Leading coefficient.
\DeclareMathOperator{\LCM}{LCM}             % Least common multiple.
\DeclareMathOperator{\LM}{LM}               % Leading monomial.
\DeclareMathOperator{\LT}{LT}               % Leading term.
\DeclareMathOperator{\Mod}{mod}             % Modulus.
\DeclareMathOperator{\Mon}{Mon}             % Monomial.
\DeclareMathOperator{\multideg}{mutlideg}   % Multi-Degree (Graphs).
\DeclareMathOperator{\nul}{nul}             % Null space of operator.
\DeclareMathOperator{\Ord}{Ord}             % Ordinal of ordered set.
\DeclareMathOperator{\Prin}{Prin}           % Principal value.
\DeclareMathOperator{\proj}{proj}           % Projection.
\DeclareMathOperator{\Refl}{Refl}           % Reflection operator.
\DeclareMathOperator{\rk}{rk}               % Rank of operator.
\DeclareMathOperator{\sgn}{sgn}             % Sign of a number.
\DeclareMathOperator{\sinc}{sinc}           % Sinc function.
\DeclareMathOperator{\Span}{Span}           % Span of a set.
\DeclareMathOperator{\Spec}{Spec}           % Spectrum.
\DeclareMathOperator{\supp}{supp}           % Support
\DeclareMathOperator{\Tr}{Tr}               % Trace of matrix.
%--------------------Declared Math Symbols--------------------%
\DeclareMathSymbol{\minus}{\mathbin}{AMSa}{"39} % Unary minus sign.
%------------------------New Commands---------------------------%
\DeclarePairedDelimiter\norm{\lVert}{\rVert}
\DeclarePairedDelimiter\ceil{\lceil}{\rceil}
\DeclarePairedDelimiter\floor{\lfloor}{\rfloor}
\newcommand*\diff{\mathop{}\!\mathrm{d}}
\newcommand*\Diff[1]{\mathop{}\!\mathrm{d^#1}}
\renewcommand*{\glstextformat}[1]{\textcolor{RoyalBlue}{#1}}
\renewcommand{\glsnamefont}[1]{\textbf{#1}}
\renewcommand\labelitemii{$\circ$}
\renewcommand\thesubfigure{%
    \arabic{chapter}.\arabic{figure}.\arabic{subfigure}}
\addto\captionsenglish{\renewcommand{\figurename}{Fig.}}
\numberwithin{equation}{section}

\renewcommand{\vector}[1]{\boldsymbol{\mathrm{#1}}}

\newcommand{\uvector}[1]{\boldsymbol{\hat{\mathrm{#1}}}}
\newcommand{\topspace}[2][]{(#2,\tau_{#1})}
\newcommand{\measurespace}[2][]{(#2,\varSigma_{#1},\mu_{#1})}
\newcommand{\measurablespace}[2][]{(#2,\varSigma_{#1})}
\newcommand{\manifold}[2][]{(#2,\tau_{#1},\mathcal{A}_{#1})}
\newcommand{\tanspace}[2]{T_{#1}{#2}}
\newcommand{\cotanspace}[2]{T_{#1}^{*}{#2}}
\newcommand{\Ckspace}[3][\mathbb{R}]{C^{#2}(#3,#1)}
\newcommand{\funcspace}[2][\mathbb{R}]{\mathcal{F}(#2,#1)}
\newcommand{\smoothvecf}[1]{\mathfrak{X}(#1)}
\newcommand{\smoothonef}[1]{\mathfrak{X}^{*}(#1)}
\newcommand{\bracket}[2]{[#1,#2]}

%------------------------Book Command---------------------------%
\makeatletter
\renewcommand\@pnumwidth{1cm}
\newcounter{book}
\renewcommand\thebook{\@Roman\c@book}
\newcommand\book{%
    \if@openright
        \cleardoublepage
    \else
        \clearpage
    \fi
    \thispagestyle{plain}%
    \if@twocolumn
        \onecolumn
        \@tempswatrue
    \else
        \@tempswafalse
    \fi
    \null\vfil
    \secdef\@book\@sbook
}
\def\@book[#1]#2{%
    \refstepcounter{book}
    \addcontentsline{toc}{book}{\bookname\ \thebook:\hspace{1em}#1}
    \markboth{}{}
    {\centering
     \interlinepenalty\@M
     \normalfont
     \huge\bfseries\bookname\nobreakspace\thebook
     \par
     \vskip 20\p@
     \Huge\bfseries#2\par}%
    \@endbook}
\def\@sbook#1{%
    {\centering
     \interlinepenalty \@M
     \normalfont
     \Huge\bfseries#1\par}%
    \@endbook}
\def\@endbook{
    \vfil\newpage
        \if@twoside
            \if@openright
                \null
                \thispagestyle{empty}%
                \newpage
            \fi
        \fi
        \if@tempswa
            \twocolumn
        \fi
}
\newcommand*\l@book[2]{%
    \ifnum\c@tocdepth >-3\relax
        \addpenalty{-\@highpenalty}%
        \addvspace{2.25em\@plus\p@}%
        \setlength\@tempdima{3em}%
        \begingroup
            \parindent\z@\rightskip\@pnumwidth
            \parfillskip -\@pnumwidth
            {
                \leavevmode
                \Large\bfseries#1\hfill\hb@xt@\@pnumwidth{\hss#2}
            }
            \par
            \nobreak
            \global\@nobreaktrue
            \everypar{\global\@nobreakfalse\everypar{}}%
        \endgroup
    \fi}
\newcommand\bookname{Book}
\renewcommand{\thebook}{\texorpdfstring{\Numberstring{book}}{book}}
\providecommand*{\toclevel@book}{-2}
\makeatother
\titleformat{\part}[display]
    {\Large\bfseries}
    {\partname\nobreakspace\thepart}
    {0mm}
    {\Huge\bfseries}
\titlecontents{part}[0pt]
    {\large\bfseries}
    {\partname\ \thecontentslabel: \quad}
    {}
    {\hfill\contentspage}
\titlecontents{chapter}[0pt]
    {\bfseries}
    {\chaptername\ \thecontentslabel:\quad}
    {}
    {\hfill\contentspage}
\newglossarystyle{longpara}{%
    \setglossarystyle{long}%
    \renewenvironment{theglossary}{%
        \begin{longtable}[l]{{p{0.25\hsize}p{0.65\hsize}}}
    }{\end{longtable}}%
    \renewcommand{\glossentry}[2]{%
        \glstarget{##1}{\glossentryname{##1}}%
        &\glossentrydesc{##1}{~##2.}
        \tabularnewline%
        \tabularnewline
    }%
}
\newglossary[not-glg]{notation}{not-gls}{not-glo}{Notation}
\newcommand*{\newnotation}[4][]{%
    \newglossaryentry{#2}{type=notation, name={\textbf{#3}, },
                          text={#4}, description={#4},#1}%
}
%--------------------------LENGTHS------------------------------%
% Spacings for the Table of Contents.
\addtolength{\cftsecnumwidth}{1ex}
\addtolength{\cftsubsecindent}{1ex}
\addtolength{\cftsubsecnumwidth}{1ex}
\addtolength{\cftfignumwidth}{1ex}
\addtolength{\cfttabnumwidth}{1ex}

% Indent and paragraph spacing.
\setlength{\parindent}{0em}
\setlength{\parskip}{0em}
\graphicspath{{../images/}}   % Path to Image Folder.
%----------------------------GLOSSARY-------------------------------%
\makeglossaries
\loadglsentries{../glossary}
\loadglsentries{../acronym}
%--------------------------Main Document----------------------------%
    %        \documentclass[crop=false,class=book,oneside]{standalone}
%----------------------------Preamble-------------------------------%
%---------------------------Packages----------------------------%
\usepackage{geometry}
\geometry{b5paper, margin=1.0in}
\usepackage[T1]{fontenc}
\usepackage{graphicx, float}            % Graphics/Images.
\usepackage{natbib}                     % For bibliographies.
\bibliographystyle{agsm}                % Bibliography style.
\usepackage[french, english]{babel}     % Language typesetting.
\usepackage[dvipsnames]{xcolor}         % Color names.
\usepackage{listings}                   % Verbatim-Like Tools.
\usepackage{mathtools, esint, mathrsfs} % amsmath and integrals.
\usepackage{amsthm, amsfonts, amssymb}  % Fonts and theorems.
\usepackage{tcolorbox}                  % Frames around theorems.
\usepackage{upgreek}                    % Non-Italic Greek.
\usepackage{fmtcount, etoolbox}         % For the \book{} command.
\usepackage[newparttoc]{titlesec}       % Formatting chapter, etc.
\usepackage{titletoc}                   % Allows \book in toc.
\usepackage[nottoc]{tocbibind}          % Bibliography in toc.
\usepackage[titles]{tocloft}            % ToC formatting.
\usepackage{pgfplots, tikz}             % Drawing/graphing tools.
\usepackage{imakeidx}                   % Used for index.
\usetikzlibrary{
    calc,                   % Calculating right angles and more.
    angles,                 % Drawing angles within triangles.
    arrows.meta,            % Latex and Stealth arrows.
    quotes,                 % Adding labels to angles.
    positioning,            % Relative positioning of nodes.
    decorations.markings,   % Adding arrows in the middle of a line.
    patterns,
    arrows
}                                       % Libraries for tikz.
\pgfplotsset{compat=1.9}                % Version of pgfplots.
\usepackage[font=scriptsize,
            labelformat=simple,
            labelsep=colon]{subcaption} % Subfigure captions.
\usepackage[font={scriptsize},
            hypcap=true,
            labelsep=colon]{caption}    % Figure captions.
\usepackage[pdftex,
            pdfauthor={Ryan Maguire},
            pdftitle={Mathematics and Physics},
            pdfsubject={Mathematics, Physics, Science},
            pdfkeywords={Mathematics, Physics, Computer Science, Biology},
            pdfproducer={LaTeX},
            pdfcreator={pdflatex}]{hyperref}
\hypersetup{
    colorlinks=true,
    linkcolor=blue,
    filecolor=magenta,
    urlcolor=Cerulean,
    citecolor=SkyBlue
}                           % Colors for hyperref.
\usepackage[toc,acronym,nogroupskip,nopostdot]{glossaries}
\usepackage{glossary-mcols}
%------------------------Theorem Styles-------------------------%
\theoremstyle{plain}
\newtheorem{theorem}{Theorem}[section]

% Define theorem style for default spacing and normal font.
\newtheoremstyle{normal}
    {\topsep}               % Amount of space above the theorem.
    {\topsep}               % Amount of space below the theorem.
    {}                      % Font used for body of theorem.
    {}                      % Measure of space to indent.
    {\bfseries}             % Font of the header of the theorem.
    {}                      % Punctuation between head and body.
    {.5em}                  % Space after theorem head.
    {}

% Italic header environment.
\newtheoremstyle{thmit}{\topsep}{\topsep}{}{}{\itshape}{}{0.5em}{}

% Define environments with italic headers.
\theoremstyle{thmit}
\newtheorem*{solution}{Solution}

% Define default environments.
\theoremstyle{normal}
\newtheorem{example}{Example}[section]
\newtheorem{definition}{Definition}[section]
\newtheorem{problem}{Problem}[section]

% Define framed environment.
\tcbuselibrary{most}
\newtcbtheorem[use counter*=theorem]{ftheorem}{Theorem}{%
    before=\par\vspace{2ex},
    boxsep=0.5\topsep,
    after=\par\vspace{2ex},
    colback=green!5,
    colframe=green!35!black,
    fonttitle=\bfseries\upshape%
}{thm}

\newtcbtheorem[auto counter, number within=section]{faxiom}{Axiom}{%
    before=\par\vspace{2ex},
    boxsep=0.5\topsep,
    after=\par\vspace{2ex},
    colback=Apricot!5,
    colframe=Apricot!35!black,
    fonttitle=\bfseries\upshape%
}{ax}

\newtcbtheorem[use counter*=definition]{fdefinition}{Definition}{%
    before=\par\vspace{2ex},
    boxsep=0.5\topsep,
    after=\par\vspace{2ex},
    colback=blue!5!white,
    colframe=blue!75!black,
    fonttitle=\bfseries\upshape%
}{def}

\newtcbtheorem[use counter*=example]{fexample}{Example}{%
    before=\par\vspace{2ex},
    boxsep=0.5\topsep,
    after=\par\vspace{2ex},
    colback=red!5!white,
    colframe=red!75!black,
    fonttitle=\bfseries\upshape%
}{ex}

\newtcbtheorem[auto counter, number within=section]{fnotation}{Notation}{%
    before=\par\vspace{2ex},
    boxsep=0.5\topsep,
    after=\par\vspace{2ex},
    colback=SeaGreen!5!white,
    colframe=SeaGreen!75!black,
    fonttitle=\bfseries\upshape%
}{not}

\newtcbtheorem[use counter*=remark]{fremark}{Remark}{%
    fonttitle=\bfseries\upshape,
    colback=Goldenrod!5!white,
    colframe=Goldenrod!75!black}{ex}

\newenvironment{bproof}{\textit{Proof.}}{\hfill$\square$}
\tcolorboxenvironment{bproof}{%
    blanker,
    breakable,
    left=3mm,
    before skip=5pt,
    after skip=10pt,
    borderline west={0.6mm}{0pt}{green!80!black}
}

\AtEndEnvironment{lexample}{$\hfill\textcolor{red}{\blacksquare}$}
\newtcbtheorem[use counter*=example]{lexample}{Example}{%
    empty,
    title={Example~\theexample},
    boxed title style={%
        empty,
        size=minimal,
        toprule=2pt,
        top=0.5\topsep,
    },
    coltitle=red,
    fonttitle=\bfseries,
    parbox=false,
    boxsep=0pt,
    before=\par\vspace{2ex},
    left=0pt,
    right=0pt,
    top=3ex,
    bottom=1ex,
    before=\par\vspace{2ex},
    after=\par\vspace{2ex},
    breakable,
    pad at break*=0mm,
    vfill before first,
    overlay unbroken={%
        \draw[red, line width=2pt]
            ([yshift=-1.2ex]title.south-|frame.west) to
            ([yshift=-1.2ex]title.south-|frame.east);
        },
    overlay first={%
        \draw[red, line width=2pt]
            ([yshift=-1.2ex]title.south-|frame.west) to
            ([yshift=-1.2ex]title.south-|frame.east);
    },
}{ex}

\AtEndEnvironment{ldefinition}{$\hfill\textcolor{Blue}{\blacksquare}$}
\newtcbtheorem[use counter*=definition]{ldefinition}{Definition}{%
    empty,
    title={Definition~\thedefinition:~{#1}},
    boxed title style={%
        empty,
        size=minimal,
        toprule=2pt,
        top=0.5\topsep,
    },
    coltitle=Blue,
    fonttitle=\bfseries,
    parbox=false,
    boxsep=0pt,
    before=\par\vspace{2ex},
    left=0pt,
    right=0pt,
    top=3ex,
    bottom=0pt,
    before=\par\vspace{2ex},
    after=\par\vspace{1ex},
    breakable,
    pad at break*=0mm,
    vfill before first,
    overlay unbroken={%
        \draw[Blue, line width=2pt]
            ([yshift=-1.2ex]title.south-|frame.west) to
            ([yshift=-1.2ex]title.south-|frame.east);
        },
    overlay first={%
        \draw[Blue, line width=2pt]
            ([yshift=-1.2ex]title.south-|frame.west) to
            ([yshift=-1.2ex]title.south-|frame.east);
    },
}{def}

\AtEndEnvironment{ltheorem}{$\hfill\textcolor{Green}{\blacksquare}$}
\newtcbtheorem[use counter*=theorem]{ltheorem}{Theorem}{%
    empty,
    title={Theorem~\thetheorem:~{#1}},
    boxed title style={%
        empty,
        size=minimal,
        toprule=2pt,
        top=0.5\topsep,
    },
    coltitle=Green,
    fonttitle=\bfseries,
    parbox=false,
    boxsep=0pt,
    before=\par\vspace{2ex},
    left=0pt,
    right=0pt,
    top=3ex,
    bottom=-1.5ex,
    breakable,
    pad at break*=0mm,
    vfill before first,
    overlay unbroken={%
        \draw[Green, line width=2pt]
            ([yshift=-1.2ex]title.south-|frame.west) to
            ([yshift=-1.2ex]title.south-|frame.east);},
    overlay first={%
        \draw[Green, line width=2pt]
            ([yshift=-1.2ex]title.south-|frame.west) to
            ([yshift=-1.2ex]title.south-|frame.east);
    }
}{thm}

%--------------------Declared Math Operators--------------------%
\DeclareMathOperator{\adjoint}{adj}         % Adjoint.
\DeclareMathOperator{\Card}{Card}           % Cardinality.
\DeclareMathOperator{\curl}{curl}           % Curl.
\DeclareMathOperator{\diam}{diam}           % Diameter.
\DeclareMathOperator{\dist}{dist}           % Distance.
\DeclareMathOperator{\Div}{div}             % Divergence.
\DeclareMathOperator{\Erf}{Erf}             % Error Function.
\DeclareMathOperator{\Erfc}{Erfc}           % Complementary Error Function.
\DeclareMathOperator{\Ext}{Ext}             % Exterior.
\DeclareMathOperator{\GCD}{GCD}             % Greatest common denominator.
\DeclareMathOperator{\grad}{grad}           % Gradient
\DeclareMathOperator{\Ima}{Im}              % Image.
\DeclareMathOperator{\Int}{Int}             % Interior.
\DeclareMathOperator{\LC}{LC}               % Leading coefficient.
\DeclareMathOperator{\LCM}{LCM}             % Least common multiple.
\DeclareMathOperator{\LM}{LM}               % Leading monomial.
\DeclareMathOperator{\LT}{LT}               % Leading term.
\DeclareMathOperator{\Mod}{mod}             % Modulus.
\DeclareMathOperator{\Mon}{Mon}             % Monomial.
\DeclareMathOperator{\multideg}{mutlideg}   % Multi-Degree (Graphs).
\DeclareMathOperator{\nul}{nul}             % Null space of operator.
\DeclareMathOperator{\Ord}{Ord}             % Ordinal of ordered set.
\DeclareMathOperator{\Prin}{Prin}           % Principal value.
\DeclareMathOperator{\proj}{proj}           % Projection.
\DeclareMathOperator{\Refl}{Refl}           % Reflection operator.
\DeclareMathOperator{\rk}{rk}               % Rank of operator.
\DeclareMathOperator{\sgn}{sgn}             % Sign of a number.
\DeclareMathOperator{\sinc}{sinc}           % Sinc function.
\DeclareMathOperator{\Span}{Span}           % Span of a set.
\DeclareMathOperator{\Spec}{Spec}           % Spectrum.
\DeclareMathOperator{\supp}{supp}           % Support
\DeclareMathOperator{\Tr}{Tr}               % Trace of matrix.
%--------------------Declared Math Symbols--------------------%
\DeclareMathSymbol{\minus}{\mathbin}{AMSa}{"39} % Unary minus sign.
%------------------------New Commands---------------------------%
\DeclarePairedDelimiter\norm{\lVert}{\rVert}
\DeclarePairedDelimiter\ceil{\lceil}{\rceil}
\DeclarePairedDelimiter\floor{\lfloor}{\rfloor}
\newcommand*\diff{\mathop{}\!\mathrm{d}}
\newcommand*\Diff[1]{\mathop{}\!\mathrm{d^#1}}
\renewcommand*{\glstextformat}[1]{\textcolor{RoyalBlue}{#1}}
\renewcommand{\glsnamefont}[1]{\textbf{#1}}
\renewcommand\labelitemii{$\circ$}
\renewcommand\thesubfigure{%
    \arabic{chapter}.\arabic{figure}.\arabic{subfigure}}
\addto\captionsenglish{\renewcommand{\figurename}{Fig.}}
\numberwithin{equation}{section}

\renewcommand{\vector}[1]{\boldsymbol{\mathrm{#1}}}

\newcommand{\uvector}[1]{\boldsymbol{\hat{\mathrm{#1}}}}
\newcommand{\topspace}[2][]{(#2,\tau_{#1})}
\newcommand{\measurespace}[2][]{(#2,\varSigma_{#1},\mu_{#1})}
\newcommand{\measurablespace}[2][]{(#2,\varSigma_{#1})}
\newcommand{\manifold}[2][]{(#2,\tau_{#1},\mathcal{A}_{#1})}
\newcommand{\tanspace}[2]{T_{#1}{#2}}
\newcommand{\cotanspace}[2]{T_{#1}^{*}{#2}}
\newcommand{\Ckspace}[3][\mathbb{R}]{C^{#2}(#3,#1)}
\newcommand{\funcspace}[2][\mathbb{R}]{\mathcal{F}(#2,#1)}
\newcommand{\smoothvecf}[1]{\mathfrak{X}(#1)}
\newcommand{\smoothonef}[1]{\mathfrak{X}^{*}(#1)}
\newcommand{\bracket}[2]{[#1,#2]}

%------------------------Book Command---------------------------%
\makeatletter
\renewcommand\@pnumwidth{1cm}
\newcounter{book}
\renewcommand\thebook{\@Roman\c@book}
\newcommand\book{%
    \if@openright
        \cleardoublepage
    \else
        \clearpage
    \fi
    \thispagestyle{plain}%
    \if@twocolumn
        \onecolumn
        \@tempswatrue
    \else
        \@tempswafalse
    \fi
    \null\vfil
    \secdef\@book\@sbook
}
\def\@book[#1]#2{%
    \refstepcounter{book}
    \addcontentsline{toc}{book}{\bookname\ \thebook:\hspace{1em}#1}
    \markboth{}{}
    {\centering
     \interlinepenalty\@M
     \normalfont
     \huge\bfseries\bookname\nobreakspace\thebook
     \par
     \vskip 20\p@
     \Huge\bfseries#2\par}%
    \@endbook}
\def\@sbook#1{%
    {\centering
     \interlinepenalty \@M
     \normalfont
     \Huge\bfseries#1\par}%
    \@endbook}
\def\@endbook{
    \vfil\newpage
        \if@twoside
            \if@openright
                \null
                \thispagestyle{empty}%
                \newpage
            \fi
        \fi
        \if@tempswa
            \twocolumn
        \fi
}
\newcommand*\l@book[2]{%
    \ifnum\c@tocdepth >-3\relax
        \addpenalty{-\@highpenalty}%
        \addvspace{2.25em\@plus\p@}%
        \setlength\@tempdima{3em}%
        \begingroup
            \parindent\z@\rightskip\@pnumwidth
            \parfillskip -\@pnumwidth
            {
                \leavevmode
                \Large\bfseries#1\hfill\hb@xt@\@pnumwidth{\hss#2}
            }
            \par
            \nobreak
            \global\@nobreaktrue
            \everypar{\global\@nobreakfalse\everypar{}}%
        \endgroup
    \fi}
\newcommand\bookname{Book}
\renewcommand{\thebook}{\texorpdfstring{\Numberstring{book}}{book}}
\providecommand*{\toclevel@book}{-2}
\makeatother
\titleformat{\part}[display]
    {\Large\bfseries}
    {\partname\nobreakspace\thepart}
    {0mm}
    {\Huge\bfseries}
\titlecontents{part}[0pt]
    {\large\bfseries}
    {\partname\ \thecontentslabel: \quad}
    {}
    {\hfill\contentspage}
\titlecontents{chapter}[0pt]
    {\bfseries}
    {\chaptername\ \thecontentslabel:\quad}
    {}
    {\hfill\contentspage}
\newglossarystyle{longpara}{%
    \setglossarystyle{long}%
    \renewenvironment{theglossary}{%
        \begin{longtable}[l]{{p{0.25\hsize}p{0.65\hsize}}}
    }{\end{longtable}}%
    \renewcommand{\glossentry}[2]{%
        \glstarget{##1}{\glossentryname{##1}}%
        &\glossentrydesc{##1}{~##2.}
        \tabularnewline%
        \tabularnewline
    }%
}
\newglossary[not-glg]{notation}{not-gls}{not-glo}{Notation}
\newcommand*{\newnotation}[4][]{%
    \newglossaryentry{#2}{type=notation, name={\textbf{#3}, },
                          text={#4}, description={#4},#1}%
}
%--------------------------LENGTHS------------------------------%
% Spacings for the Table of Contents.
\addtolength{\cftsecnumwidth}{1ex}
\addtolength{\cftsubsecindent}{1ex}
\addtolength{\cftsubsecnumwidth}{1ex}
\addtolength{\cftfignumwidth}{1ex}
\addtolength{\cfttabnumwidth}{1ex}

% Indent and paragraph spacing.
\setlength{\parindent}{0em}
\setlength{\parskip}{0em}
\graphicspath{{../../../images/}}   % Path to Image Folder.
%--------------------------Main Document----------------------------%
\begin{document}
    \ifx\ifmathcourses\undefined
        \title{Chaos Theory}
        \author{Ryan Maguire}
        \date{\vspace{-5ex}}
        \maketitle
        \tableofcontents
        \chapter*{Chaos Theory}
        \markboth{}{CHAOS THEORY}
        \setcounter{chapter}{1}
    \else
        \chapter{Chaos Theory}
    \fi
    \section{Miscellaneous Notes}
        \subsection{Notes from Fall 2016 (UML)}
            \subsubsection{Notes on the Jordan Normal Form}
                Every square matrix $A$ is similar to an upper
                triangular matrix $J$ in
                \textit{Jordan normal form} whose diagonal entries
                are the eigenvalues of $A$. That is, there exists
                an invertible matrix $P$ such that
                $P^{-1}AP=J$. The trace of $A$ is equal to the
                trace of $J$:
                \begin{equation*}
                    \Tr(J)
                    =\Tr(P^{-1}AP)
                    =\Tr(P^{-1}PA)
                    =\Tr(IA)=\Tr(A)
                \end{equation*}
            \subsubsection{Notes on Conjugacy}
                We have:
                \begin{align*}
                    X'(t)&=AX(t),
                    \quad
                    X(0)=X_{0}\\
                    \Rightarrow
                    X(t)&=e^{tA}X_{0}\\
                    \Rightarrow
                    \phi^{A}(t,X_{0})
                    &=e^{tA}X_{0}
                \end{align*}
                Thus, if $B=T^{-1}AT$, for some matrix $T$, then:
                \begin{align*}
                    Y'(t)&=BY(t),
                    \quad
                    Y(0)=Y_{0}\\
                    &=T^{-1}ATY(t)\\
                    \Rightarrow
                    Y(t)
                    &=e^{tT^{-1}AT}Y_{0}\\
                    &=T^{-1}e^{tA}TY_{0}\\
                    \Rightarrow
                    \phi^{B}(t,Y_{0})
                    &=T^{-1}e^{tA}TY_{0}
                \end{align*}
                Thus, the homeomorphism is $h(X)=T^{-1}X$,
                and we have:
                \begin{equation*}
                    \phi^{B}(t,h(X_{0}))
                    =\phi^{B}(t,T^{-1}X_{0})
                    =T^{-1}e^{tA}T(T^{-1}X_{0})
                    =T^{-1}e^{tA}X_{0}
                    =h(\phi^{A}(t,X_{0}))
                \end{equation*}
\end{document}
    %        \addtocontents{toc}{\protect\newpage}

    % \book{Geometry}
    %     \part{Riemannian Geometry}
    %         \renewcommand{\PATH}{\TOPPATH/Geometry}
    %         \chapter{Semi-Riemannian Geometry}
    Here we talk about Semi-Riemannian Geometry.
    %------------------------------------------------------------------------------%

    %         \addtocontents{toc}{\protect\newpage}


    %\book{Physics}
    %    \import{books/Physics/}{Physics_Main.tex}
    %    \addtocontents{toc}{\protect\newpage}

    % Print glossaries and acronyms page.
    \printnoidxglossary[type=\acronymtype]
    \clearpage
    \printnoidxglossary[style=longpara]

    % Print bibliographies from all texts.
    \clearpage
    \bibliography{biblio}

    % Print the index.
    \clearpage
    \printindex
\end{document}