\documentclass[crop=false,class=book,oneside]{standalone}
%----------------------------Preamble-------------------------------%
%---------------------------Packages----------------------------%
\usepackage{geometry}
\geometry{b5paper, margin=1.0in}
\usepackage[T1]{fontenc}
\usepackage{graphicx, float}            % Graphics/Images.
\usepackage{natbib}                     % For bibliographies.
\bibliographystyle{agsm}                % Bibliography style.
\usepackage[french, english]{babel}     % Language typesetting.
\usepackage[dvipsnames]{xcolor}         % Color names.
\usepackage{listings}                   % Verbatim-Like Tools.
\usepackage{mathtools, esint, mathrsfs} % amsmath and integrals.
\usepackage{amsthm, amsfonts, amssymb}  % Fonts and theorems.
\usepackage{tcolorbox}                  % Frames around theorems.
\usepackage{upgreek}                    % Non-Italic Greek.
\usepackage{fmtcount, etoolbox}         % For the \book{} command.
\usepackage[newparttoc]{titlesec}       % Formatting chapter, etc.
\usepackage{titletoc}                   % Allows \book in toc.
\usepackage[nottoc]{tocbibind}          % Bibliography in toc.
\usepackage[titles]{tocloft}            % ToC formatting.
\usepackage{pgfplots, tikz}             % Drawing/graphing tools.
\usepackage{imakeidx}                   % Used for index.
\usetikzlibrary{
    calc,                   % Calculating right angles and more.
    angles,                 % Drawing angles within triangles.
    arrows.meta,            % Latex and Stealth arrows.
    quotes,                 % Adding labels to angles.
    positioning,            % Relative positioning of nodes.
    decorations.markings,   % Adding arrows in the middle of a line.
    patterns,
    arrows
}                                       % Libraries for tikz.
\pgfplotsset{compat=1.9}                % Version of pgfplots.
\usepackage[font=scriptsize,
            labelformat=simple,
            labelsep=colon]{subcaption} % Subfigure captions.
\usepackage[font={scriptsize},
            hypcap=true,
            labelsep=colon]{caption}    % Figure captions.
\usepackage[pdftex,
            pdfauthor={Ryan Maguire},
            pdftitle={Mathematics and Physics},
            pdfsubject={Mathematics, Physics, Science},
            pdfkeywords={Mathematics, Physics, Computer Science, Biology},
            pdfproducer={LaTeX},
            pdfcreator={pdflatex}]{hyperref}
\hypersetup{
    colorlinks=true,
    linkcolor=blue,
    filecolor=magenta,
    urlcolor=Cerulean,
    citecolor=SkyBlue
}                           % Colors for hyperref.
\usepackage[toc,acronym,nogroupskip,nopostdot]{glossaries}
\usepackage{glossary-mcols}
%------------------------Theorem Styles-------------------------%
\theoremstyle{plain}
\newtheorem{theorem}{Theorem}[section]

% Define theorem style for default spacing and normal font.
\newtheoremstyle{normal}
    {\topsep}               % Amount of space above the theorem.
    {\topsep}               % Amount of space below the theorem.
    {}                      % Font used for body of theorem.
    {}                      % Measure of space to indent.
    {\bfseries}             % Font of the header of the theorem.
    {}                      % Punctuation between head and body.
    {.5em}                  % Space after theorem head.
    {}

% Italic header environment.
\newtheoremstyle{thmit}{\topsep}{\topsep}{}{}{\itshape}{}{0.5em}{}

% Define environments with italic headers.
\theoremstyle{thmit}
\newtheorem*{solution}{Solution}

% Define default environments.
\theoremstyle{normal}
\newtheorem{example}{Example}[section]
\newtheorem{definition}{Definition}[section]
\newtheorem{problem}{Problem}[section]

% Define framed environment.
\tcbuselibrary{most}
\newtcbtheorem[use counter*=theorem]{ftheorem}{Theorem}{%
    before=\par\vspace{2ex},
    boxsep=0.5\topsep,
    after=\par\vspace{2ex},
    colback=green!5,
    colframe=green!35!black,
    fonttitle=\bfseries\upshape%
}{thm}

\newtcbtheorem[auto counter, number within=section]{faxiom}{Axiom}{%
    before=\par\vspace{2ex},
    boxsep=0.5\topsep,
    after=\par\vspace{2ex},
    colback=Apricot!5,
    colframe=Apricot!35!black,
    fonttitle=\bfseries\upshape%
}{ax}

\newtcbtheorem[use counter*=definition]{fdefinition}{Definition}{%
    before=\par\vspace{2ex},
    boxsep=0.5\topsep,
    after=\par\vspace{2ex},
    colback=blue!5!white,
    colframe=blue!75!black,
    fonttitle=\bfseries\upshape%
}{def}

\newtcbtheorem[use counter*=example]{fexample}{Example}{%
    before=\par\vspace{2ex},
    boxsep=0.5\topsep,
    after=\par\vspace{2ex},
    colback=red!5!white,
    colframe=red!75!black,
    fonttitle=\bfseries\upshape%
}{ex}

\newtcbtheorem[auto counter, number within=section]{fnotation}{Notation}{%
    before=\par\vspace{2ex},
    boxsep=0.5\topsep,
    after=\par\vspace{2ex},
    colback=SeaGreen!5!white,
    colframe=SeaGreen!75!black,
    fonttitle=\bfseries\upshape%
}{not}

\newtcbtheorem[use counter*=remark]{fremark}{Remark}{%
    fonttitle=\bfseries\upshape,
    colback=Goldenrod!5!white,
    colframe=Goldenrod!75!black}{ex}

\newenvironment{bproof}{\textit{Proof.}}{\hfill$\square$}
\tcolorboxenvironment{bproof}{%
    blanker,
    breakable,
    left=3mm,
    before skip=5pt,
    after skip=10pt,
    borderline west={0.6mm}{0pt}{green!80!black}
}

\AtEndEnvironment{lexample}{$\hfill\textcolor{red}{\blacksquare}$}
\newtcbtheorem[use counter*=example]{lexample}{Example}{%
    empty,
    title={Example~\theexample},
    boxed title style={%
        empty,
        size=minimal,
        toprule=2pt,
        top=0.5\topsep,
    },
    coltitle=red,
    fonttitle=\bfseries,
    parbox=false,
    boxsep=0pt,
    before=\par\vspace{2ex},
    left=0pt,
    right=0pt,
    top=3ex,
    bottom=1ex,
    before=\par\vspace{2ex},
    after=\par\vspace{2ex},
    breakable,
    pad at break*=0mm,
    vfill before first,
    overlay unbroken={%
        \draw[red, line width=2pt]
            ([yshift=-1.2ex]title.south-|frame.west) to
            ([yshift=-1.2ex]title.south-|frame.east);
        },
    overlay first={%
        \draw[red, line width=2pt]
            ([yshift=-1.2ex]title.south-|frame.west) to
            ([yshift=-1.2ex]title.south-|frame.east);
    },
}{ex}

\AtEndEnvironment{ldefinition}{$\hfill\textcolor{Blue}{\blacksquare}$}
\newtcbtheorem[use counter*=definition]{ldefinition}{Definition}{%
    empty,
    title={Definition~\thedefinition:~{#1}},
    boxed title style={%
        empty,
        size=minimal,
        toprule=2pt,
        top=0.5\topsep,
    },
    coltitle=Blue,
    fonttitle=\bfseries,
    parbox=false,
    boxsep=0pt,
    before=\par\vspace{2ex},
    left=0pt,
    right=0pt,
    top=3ex,
    bottom=0pt,
    before=\par\vspace{2ex},
    after=\par\vspace{1ex},
    breakable,
    pad at break*=0mm,
    vfill before first,
    overlay unbroken={%
        \draw[Blue, line width=2pt]
            ([yshift=-1.2ex]title.south-|frame.west) to
            ([yshift=-1.2ex]title.south-|frame.east);
        },
    overlay first={%
        \draw[Blue, line width=2pt]
            ([yshift=-1.2ex]title.south-|frame.west) to
            ([yshift=-1.2ex]title.south-|frame.east);
    },
}{def}

\AtEndEnvironment{ltheorem}{$\hfill\textcolor{Green}{\blacksquare}$}
\newtcbtheorem[use counter*=theorem]{ltheorem}{Theorem}{%
    empty,
    title={Theorem~\thetheorem:~{#1}},
    boxed title style={%
        empty,
        size=minimal,
        toprule=2pt,
        top=0.5\topsep,
    },
    coltitle=Green,
    fonttitle=\bfseries,
    parbox=false,
    boxsep=0pt,
    before=\par\vspace{2ex},
    left=0pt,
    right=0pt,
    top=3ex,
    bottom=-1.5ex,
    breakable,
    pad at break*=0mm,
    vfill before first,
    overlay unbroken={%
        \draw[Green, line width=2pt]
            ([yshift=-1.2ex]title.south-|frame.west) to
            ([yshift=-1.2ex]title.south-|frame.east);},
    overlay first={%
        \draw[Green, line width=2pt]
            ([yshift=-1.2ex]title.south-|frame.west) to
            ([yshift=-1.2ex]title.south-|frame.east);
    }
}{thm}

%--------------------Declared Math Operators--------------------%
\DeclareMathOperator{\adjoint}{adj}         % Adjoint.
\DeclareMathOperator{\Card}{Card}           % Cardinality.
\DeclareMathOperator{\curl}{curl}           % Curl.
\DeclareMathOperator{\diam}{diam}           % Diameter.
\DeclareMathOperator{\dist}{dist}           % Distance.
\DeclareMathOperator{\Div}{div}             % Divergence.
\DeclareMathOperator{\Erf}{Erf}             % Error Function.
\DeclareMathOperator{\Erfc}{Erfc}           % Complementary Error Function.
\DeclareMathOperator{\Ext}{Ext}             % Exterior.
\DeclareMathOperator{\GCD}{GCD}             % Greatest common denominator.
\DeclareMathOperator{\grad}{grad}           % Gradient
\DeclareMathOperator{\Ima}{Im}              % Image.
\DeclareMathOperator{\Int}{Int}             % Interior.
\DeclareMathOperator{\LC}{LC}               % Leading coefficient.
\DeclareMathOperator{\LCM}{LCM}             % Least common multiple.
\DeclareMathOperator{\LM}{LM}               % Leading monomial.
\DeclareMathOperator{\LT}{LT}               % Leading term.
\DeclareMathOperator{\Mod}{mod}             % Modulus.
\DeclareMathOperator{\Mon}{Mon}             % Monomial.
\DeclareMathOperator{\multideg}{mutlideg}   % Multi-Degree (Graphs).
\DeclareMathOperator{\nul}{nul}             % Null space of operator.
\DeclareMathOperator{\Ord}{Ord}             % Ordinal of ordered set.
\DeclareMathOperator{\Prin}{Prin}           % Principal value.
\DeclareMathOperator{\proj}{proj}           % Projection.
\DeclareMathOperator{\Refl}{Refl}           % Reflection operator.
\DeclareMathOperator{\rk}{rk}               % Rank of operator.
\DeclareMathOperator{\sgn}{sgn}             % Sign of a number.
\DeclareMathOperator{\sinc}{sinc}           % Sinc function.
\DeclareMathOperator{\Span}{Span}           % Span of a set.
\DeclareMathOperator{\Spec}{Spec}           % Spectrum.
\DeclareMathOperator{\supp}{supp}           % Support
\DeclareMathOperator{\Tr}{Tr}               % Trace of matrix.
%--------------------Declared Math Symbols--------------------%
\DeclareMathSymbol{\minus}{\mathbin}{AMSa}{"39} % Unary minus sign.
%------------------------New Commands---------------------------%
\DeclarePairedDelimiter\norm{\lVert}{\rVert}
\DeclarePairedDelimiter\ceil{\lceil}{\rceil}
\DeclarePairedDelimiter\floor{\lfloor}{\rfloor}
\newcommand*\diff{\mathop{}\!\mathrm{d}}
\newcommand*\Diff[1]{\mathop{}\!\mathrm{d^#1}}
\renewcommand*{\glstextformat}[1]{\textcolor{RoyalBlue}{#1}}
\renewcommand{\glsnamefont}[1]{\textbf{#1}}
\renewcommand\labelitemii{$\circ$}
\renewcommand\thesubfigure{%
    \arabic{chapter}.\arabic{figure}.\arabic{subfigure}}
\addto\captionsenglish{\renewcommand{\figurename}{Fig.}}
\numberwithin{equation}{section}

\renewcommand{\vector}[1]{\boldsymbol{\mathrm{#1}}}

\newcommand{\uvector}[1]{\boldsymbol{\hat{\mathrm{#1}}}}
\newcommand{\topspace}[2][]{(#2,\tau_{#1})}
\newcommand{\measurespace}[2][]{(#2,\varSigma_{#1},\mu_{#1})}
\newcommand{\measurablespace}[2][]{(#2,\varSigma_{#1})}
\newcommand{\manifold}[2][]{(#2,\tau_{#1},\mathcal{A}_{#1})}
\newcommand{\tanspace}[2]{T_{#1}{#2}}
\newcommand{\cotanspace}[2]{T_{#1}^{*}{#2}}
\newcommand{\Ckspace}[3][\mathbb{R}]{C^{#2}(#3,#1)}
\newcommand{\funcspace}[2][\mathbb{R}]{\mathcal{F}(#2,#1)}
\newcommand{\smoothvecf}[1]{\mathfrak{X}(#1)}
\newcommand{\smoothonef}[1]{\mathfrak{X}^{*}(#1)}
\newcommand{\bracket}[2]{[#1,#2]}

%------------------------Book Command---------------------------%
\makeatletter
\renewcommand\@pnumwidth{1cm}
\newcounter{book}
\renewcommand\thebook{\@Roman\c@book}
\newcommand\book{%
    \if@openright
        \cleardoublepage
    \else
        \clearpage
    \fi
    \thispagestyle{plain}%
    \if@twocolumn
        \onecolumn
        \@tempswatrue
    \else
        \@tempswafalse
    \fi
    \null\vfil
    \secdef\@book\@sbook
}
\def\@book[#1]#2{%
    \refstepcounter{book}
    \addcontentsline{toc}{book}{\bookname\ \thebook:\hspace{1em}#1}
    \markboth{}{}
    {\centering
     \interlinepenalty\@M
     \normalfont
     \huge\bfseries\bookname\nobreakspace\thebook
     \par
     \vskip 20\p@
     \Huge\bfseries#2\par}%
    \@endbook}
\def\@sbook#1{%
    {\centering
     \interlinepenalty \@M
     \normalfont
     \Huge\bfseries#1\par}%
    \@endbook}
\def\@endbook{
    \vfil\newpage
        \if@twoside
            \if@openright
                \null
                \thispagestyle{empty}%
                \newpage
            \fi
        \fi
        \if@tempswa
            \twocolumn
        \fi
}
\newcommand*\l@book[2]{%
    \ifnum\c@tocdepth >-3\relax
        \addpenalty{-\@highpenalty}%
        \addvspace{2.25em\@plus\p@}%
        \setlength\@tempdima{3em}%
        \begingroup
            \parindent\z@\rightskip\@pnumwidth
            \parfillskip -\@pnumwidth
            {
                \leavevmode
                \Large\bfseries#1\hfill\hb@xt@\@pnumwidth{\hss#2}
            }
            \par
            \nobreak
            \global\@nobreaktrue
            \everypar{\global\@nobreakfalse\everypar{}}%
        \endgroup
    \fi}
\newcommand\bookname{Book}
\renewcommand{\thebook}{\texorpdfstring{\Numberstring{book}}{book}}
\providecommand*{\toclevel@book}{-2}
\makeatother
\titleformat{\part}[display]
    {\Large\bfseries}
    {\partname\nobreakspace\thepart}
    {0mm}
    {\Huge\bfseries}
\titlecontents{part}[0pt]
    {\large\bfseries}
    {\partname\ \thecontentslabel: \quad}
    {}
    {\hfill\contentspage}
\titlecontents{chapter}[0pt]
    {\bfseries}
    {\chaptername\ \thecontentslabel:\quad}
    {}
    {\hfill\contentspage}
\newglossarystyle{longpara}{%
    \setglossarystyle{long}%
    \renewenvironment{theglossary}{%
        \begin{longtable}[l]{{p{0.25\hsize}p{0.65\hsize}}}
    }{\end{longtable}}%
    \renewcommand{\glossentry}[2]{%
        \glstarget{##1}{\glossentryname{##1}}%
        &\glossentrydesc{##1}{~##2.}
        \tabularnewline%
        \tabularnewline
    }%
}
\newglossary[not-glg]{notation}{not-gls}{not-glo}{Notation}
\newcommand*{\newnotation}[4][]{%
    \newglossaryentry{#2}{type=notation, name={\textbf{#3}, },
                          text={#4}, description={#4},#1}%
}
%--------------------------LENGTHS------------------------------%
% Spacings for the Table of Contents.
\addtolength{\cftsecnumwidth}{1ex}
\addtolength{\cftsubsecindent}{1ex}
\addtolength{\cftsubsecnumwidth}{1ex}
\addtolength{\cftfignumwidth}{1ex}
\addtolength{\cfttabnumwidth}{1ex}

% Indent and paragraph spacing.
\setlength{\parindent}{0em}
\setlength{\parskip}{0em}
%----------------------------GLOSSARY-------------------------------%
\makeglossaries
\loadglsentries{../../glossary}
\loadglsentries{../../acronym}
%--------------------------Main Document----------------------------%
\begin{document}
\chapter{Analysis Summary}
    \section{Definitions and Theorems}
        \subsection{Definitions}
            \begin{definition}
                \label{Definition:MathEnc:Analysis:Sum:Sets}
                A set is a collection of elements, none of which
                are the set itself.
            \end{definition}
            \begin{definition}
                \label{Definition:MathEnc:Analysis:Sum:EmptySet}
                The empty set is the set $\emptyset$
                such that $\forall_{x}$, $x\notin A$.
            \end{definition}
            \begin{definition}
                \label{Definition:MathEnc:Analysis:Sum:Subsets}
                A subset of a set $A$ is a set $B$, denoted
                $B\subset A$, such that $\forall_{x\in B}$,
                $x\in A$.
            \end{definition}
            \begin{definition}
                \label{Definition:MathEnc:Analysis:Sum:Equality}
                Equal sets are sets $A$ and $B$, denoted $A=B$,
                such that $A\subset B$ and $B\subset A$.
            \end{definition}
            \begin{definition}
                \label{Definition:MathEnc:Analysis:Sum:OrderedPair}
                An ordered pair of an element $a$ with respect
                to an element $b$, denoted $(a,b)$,
                is the set $(a,b)=\{\{a\},\{a,b\}\}$
            \end{definition}
            \begin{definition}
                \label{%
                    Definition:MathEnc:Analysis:%
                    Sum:CartesianProduct%
                }
                The Cartesian Product of a set $A$ with respect
                to a set $B$, denoted $A\times B$, is the set
                $A\times B=\{(a,b):a\in A,b\in B\}$
            \end{definition}
            \begin{definition}
                \label{%
                    Definition:MathEnc:Analysis:%
                    Sum:BinaryRelation%
                }
                A binary relation on a set $X$ is
                a subset $R$ of $X\times X$.
            \end{definition}
            \begin{definition}
                \label{%
                    Definition:MathEnc:Analysis:%
                    Sum:ComparableElements%
                }
                Comparable elements in a set $X$ with respect
                to a binary relation $R$
                are elements $x,y\in X$ such that either
                $(x,y)\in R$, denoted $xRy$,
                or $(y,x)\in R$, denoted $yRx$.
            \end{definition}
            \begin{definition}
                \label{%
                    Definition:MathEnc:Analysis:%
                    Sum:ConnexRelation%
                }
                A connex relation on a set $X$ is a
                binary relation $R$ on $X$ such that
                $\forall_{x,y\in X}$, either $xRy$ or $yRx$.
            \end{definition}
            \begin{definition}
                \label{%
                    Definition:MathEnc:Analysis:%
                    Sum:TransitiveRelation%
                }
                A transitive relation on a set $X$
                is a binary relation $R$ on $X$ such
                that $\forall_{x,y,z\in X}$,
                $xRy\land yRz\Rightarrow xRz$.
            \end{definition}
            \begin{definition}
                \label{%
                    Definition:MathEnc:Analysis:%
                    Sum:AntisymmetricRelation%
                }
                An antisymmetric relation is a binary relation
                $R$ on a set $X$ such that
                $\forall_{x,y\in X}$, $xRy\land yRx\Rightarrow x=y$.
            \end{definition}
            \begin{definition}
                \label{Definition:MathEnc:Analysis:Sum:TotalOrder}
                A total order on a set $X$ is a binary relation $R$
                on $X$ that is a transitive relation, an
                antisymmetric relation, and a connex relation.
            \end{definition}
            \begin{definition}
                \label{%
                    Definition:MathEnc:Analysis:%
                    Sum:TrichotomousRelation%
                }
                A trichotomous relation on a set $X$ is a
                binary relation $R$ on $X$ such that
                $\forall_{x,y\in X}$ precisely one of the
                following are true: $xRy$, $yRx$, or $x=y$.
            \end{definition}
            \begin{definition}
                \label{Definition:MathEnc:Analysis:Sum:Function}
                A function $f$ from a set $A$ to a set $B$,
                denoted $f:A\rightarrow B$,
                is a subset $f\subset A\times B$ such that
                $\forall_{a\in A}$ there is
                a unique $b\in B$ such that $(a,b)\in f$.
            \end{definition}
            \begin{definition}
                \label{Definition:MathEnc:Analysis:Sum:Image}
                The image of an element $x\in A$ with respect to a
                function $f:A\rightarrow B$, denoted $f(x)$, is the
                unique element $b\in B$ such that $(a,b)\in f$.
            \end{definition}
            \begin{definition}
                \label{%
                    Definition:MathEnc:Analysis:Sum:BinaryOperation%
                }
                A binary operation on a set $A$ is a function
                $*:A\times A\rightarrow A$
            \end{definition}
            \begin{definition}
                \label{Definition:MathEnc:Analysis:Sum:Product}
                The product of elements $a,b\in A$ with respect
                to a binary operation
                $*$ on $A$, denoted $a*b$, is the image of $(a,b)$
                with respect to $*$.
            \end{definition}
            \begin{definition}
                \label{%
                    Definition:MathEnc:Analysis:%
                    Sum:CommunativeOperation%
                }
                A commutative operation on a set $A$
                is a binary operation $*$ on $A$
                such that $\forall_{x,y\in A}$, $a*b=b*a$.
            \end{definition}
            \begin{definition}
                \label{%
                    Definition:MathEnc:Analysis:%
                        Sum:AssociativeOperation%
                }
                An associative operation on a set $A$ is a
                binary operation $*$ on $A$
                such that $\forall_{a,b,c\in A}$, $a*(b*c)=(a*b)*c$.
            \end{definition}
            \begin{definition}
                \label{%
                    Definition:MathEnc:Analysis:%
                    Sum:RightDistribute%
                }
                A binary operation that right distributes
                over a binary operation $+$ on
                a set $A$ is a binary operation $\cdot$ on $A$
                such that $\forall_{a,b,c\in A}$,
                $a\cdot (b+c)=(a\cdot b)+(a\cdot c)$.
            \end{definition}
            \begin{definition}
                \label{%
                    Definition:MathEnc:Analysis:%
                    Sum:LeftDistribute%
                }
                A binary operation that left distributes
                over a binary operation $+$ on
                a set $A$ is a binary operation $\cdot$ on $A$
                such that $\forall_{a,b,c\in A}$,
                $(b+c)\cdot a=(b\cdot a)+(c\cdot a)$.
            \end{definition}
            \begin{definition}
                \label{Definition:MathEnc:Analysis:Sum:Distribute}
                A binary operation that distributes over a binary
                operation $+$ on a set $A$ is a binary operation
                $\cdot$ on $A$ such that $\cdot$ both left and right
                distributes over $+$.
            \end{definition}
            \begin{definition}
                \label{Definition:MathEnc:Analysis:Sum:LeftIdentity}
                A left identity in a set $A$ with respect
                to a binary operation $*$ is
                an element $e\in A$ such that $\forall_{a\in A}$,
                $e*a=a$.
            \end{definition}
            \begin{definition}
                \label{Definition:MathEnc:Analysis:Sum:RightIdentity}
                A right identity in a set $A$ with respect to a
                binary operation $*$ is an element $e\in A$
                such that $\forall_{a\in A}$, $a*e=a$.
            \end{definition}
            \begin{definition}
                \label{%
                    Definition:MathEnc:Analysis:%
                    Sum:IdentityElement%
                }
                An identity element in a set $A$ with
                respect to a binary operation $*$
                on $A$ is an element $e\in A$ that is
                both a right and a left identity.
            \end{definition}
            \begin{definition}
                \label{Definition:MathEnc:Analysis:Sum:LeftInverse}
                A left inverse of an element $a\in A$ with
                respect to a binary operation $*$
                on $A$ and an identity $e\in A$ is an element
                $b\in A$ such that $b*a=e$.
            \end{definition}
            \begin{definition}
                \label{Definition:MathEnc:Analysis:Sum:RightInverse}
                A right inverse of an element $a\in A$ with
                respect to a binary operation $*$
                on $A$ and an identity $e\in A$ is an element
                $b\in A$ such that $a*b=e$.
            \end{definition}
            \begin{definition}
                \label{Definition:MathEnc:Analysis:Sum:Inverse}
                An inverse of an element $a\in A$ with respect to a
                binary operation $*$ on $A$ and an identity
                $e\in A$ is an element $b\in A$ that is both a
                right and a left inverse of $a$.
            \end{definition}
            \begin{definition}
                \label{Definition:MathEnc:Analysis:Sum:Semigroup}
                A semigroup is an ordered pair $(A,*)$ such that
                $A$ is a set and $*$ is an associative binary
                operation on $A$.
            \end{definition}
            \begin{definition}
                \label{Definition:MathEnc:Analysis:Sum:Monoid}
                A monoid is a semigroup $(A,*)$ such that
                $\exists e\in A$ such that $e$ is
                and identity element with respect to $*$.
            \end{definition}
            \begin{definition}
                \label{Definition:MathEnc:Analysis:Sum:Group}
                A group is a monoid $(G,*)$, denoted
                $\langle G,*\rangle$, with an identity
                element $e$ such that
                $\forall_{a\in A}\exists_{b\in A}:a*b=e$ 
            \end{definition}
            \begin{definition}
                \label{Definition:MathEnc:Analysis:Sum:AbelianGroup}
                An Abelian Group is a group $\langle G,*\rangle$
                such that $*$ is a
                commutative binary operation on $G$.
            \end{definition}
            \begin{definition}
                \label{Definition:MathEnc:Analysis:Sum:3Tuple}
                A 3-tuple of an element $a$ with respect
                to an ordered pair $(b,c)$ is the
                set $(a,b,c)=\{\{a\},\{a,\{b,c\}\}\}$
            \end{definition}
            \begin{definition}
                \label{Definition:MathEnc:Analysis:Sum:Ring}
                A ring is a 3-tuple $(A,\cdot,+)$ such that
                $\langle A,+\rangle$ is an
                \hyperref[%
                    Definition:MathEnc:Analysis:%
                    Sum:AbelianGroup%
                ]{Abelian Group},
                $(A,\cdot)$ is a
                \hyperref[%
                    Definition:MathEnc:Analysis:%
                    Sum:Semigroup%
                ]{semigroup},
                and $*$
                \hyperref[%
                    Definition:MathEnc:Analysis:%
                    Sum:Distribute%
                ]{distributes}
                over $+$.
            \end{definition}
            \begin{definition}
                \label{Definition:MathEnc:Analysis:Sum:RingUnit}
                A ring with unity is a
                \hyperref[Definition:MathEnc:Analysis:Sum:Ring]{ring}
                $(A,\cdot,+)$ such that $(A,\cdot)$ is a
                \hyperref[%
                    Definition:MathEnc:%
                    Analysis:Sum:Monoid%
                ]{monoid}.
            \end{definition}
            \begin{definition}
                \label{Definition:MathEnc:Analysis:Sum:ComRing}
                A commutative ring is a
                \hyperref[Definition:MathEnc:Analysis:Sum:Ring]{ring}
                $(A,\cdot,+)$ such that $\cdot$ is a
                \hyperref[%
                    Definition:MathEnc:%
                    Analysis:Sum:CommunativeOperation%
                ]{commutative}
                binary operation over $A$.
            \end{definition}
            \begin{definition}
                \label{Definition:MathEnc:Analysis:Sum:ComRingUnit}
                A commutative ring with unity is a 
                \hyperref[%
                    Definition:MathEnc:%
                    Analysis:Sum:RingUnit%
                ]{ring with unity}
                such that $\cdot$ is a
                \hyperref[%
                    Definition:MathEnc:%
                    Analysis:Sum:CommunativeOperation%
                ]{commutative}
                binary operation over $A$.
            \end{definition}
            \begin{definition}
                \label{Definition:MathEnc:Analysis:Sum:Field}
                A field is a 
                \hyperref[%
                    Definition:MathEnc:%
                    Analysis:Sum:ComRingUnit%
                ]{commutative ring with unity}
                $(A,\cdot,+)$ such that $\forall_{a\in A}$
                such that $a$ is not an identity with respect
                to $\cdot$, $\exists_{b\in A}$ such that $b$ is an
                \hyperref[%
                    Definition:MathEnc:Analysis:%
                    Sum:Inverse%
                ]{inverse} of $a$ with respect to $\cdot$.
            \end{definition}
        \subsection{Theorems}
            \begin{theorem}
                \label{%
                    Theorem:MathEnc:Analysis:Sum:SqurPresIneqPosNum%
                }
                If $a,b\in\mathbb{R}^{+}$ and $a<b$,
                then $a^{2}<b^{2}$
            \end{theorem}
            \begin{proof}
                If $a<b$, and $0<a$, then $a\cdot a<a\cdot b$
                \hfill
                (Multiplicative Property of Ordered Fields)\par
                But $a\cdot a = a^{2}$, and thus $a^{2}<a\cdot b$
                \hfill
                (Definition of Exponents)\par
                But if $a<b$ and $0<b$, then $a\cdot b<b\cdot b$
                \hfill
                (Multiplicative Property of Ordered Fields)\par
                Therefore $a\cdot b<b^{2}$
                \hfill
                (Definiiton of Exponents)\par
                But if $a^{2}<a\cdot b$ and $a\cdot b<b^{2}$,
                then $a^{2}<b^{2}$
                \hfill
                (Transitive Property of Inequalities)\par
                Therefore, $a^{2}<b^{2}$
            \end{proof}
            \begin{theorem}
                If $a,b\in\mathbb{R}^{+}$ and $a^{2}=b^{2}$,
                then $a=b$.
            \end{theorem}
            \begin{proof}
                If $a^{2}=b^{2}$, then $b^{2}-a^{2}=0$\hfill
                (Additive Property of Ordered Fields)\par
                If $a<b$, then $0<b-a$\hfill
                (Additive Property of Ordered Fields)\par
                If $a,b\in\mathbb{R}^{+}$, then $0<b+a$
                \hfill
                (Closure of Addition in $\mathbb{R}^{+}$)\par
                If $0<b-a$ and $0<b+a$, then $0<(b-a)\cdot (b+a)$
                \hfill
                (Closure of Multiplication in $\mathbb{R}^{+}$)\par
                But $(b-a)\cdot(b+a)=b^{2}-a^{2}$
                \hfill
                (Elementary Algebra)\par
                But $b^{2}-a^{2}=0$, and thus $0\not<b^{2}-a^{2}$
                \hfill
                (Trichotomous Property of Inequalities)\par
                Therefore, $a\not<b$. Similarly, $b\not<a$
                \hfill
                (Proof by Contradiction)\par
                But if $a\not<b$ and $b\not<a$, then $a=b$
                \hfill
                (Trichotomous Property of Inequalities)\par
                Therefore, $a=b$
            \end{proof}
            \begin{theorem}
                If $y\in(0,1)$, then there is an $x\in(0,1)$
                such that $y=x^{2}$.
            \end{theorem}
            \begin{proof}
                Let $A=\{\xi\in\mathbb{R}^{+}:\xi^{2}\leq y\}$.\par
                But $y\in(0,1)$ and therefore $y<1$\hfill
                (Definition of $(0,1)$)\par
                Therefore $A$ is bounded above by $1$.\par
                Thus there exists a
                least upper bound $x$ of $A$\hfill
                (Completeness of $\mathbb{R}$)\par
                If $x^{2}>y$, then $\frac{x^{2}-y}{2}>0$.\hfill
                (Additive Property of Ordered Fields)\par
                Then
                $(x-\frac{x^{2}-y}{2})^{2}=%
                 a+(\frac{x^{2}-y}{2})^{2}>y$\hfill
                (Additive Property of Ordered Fields)\par
                Thus $x-\frac{x^{2}-y}{2}$ is an
                upper bound of A.\hfill
                (Definition of Upper Bounds)\par
                But $x$ is the least upper bound,
                a contradiction.\par
                Therefore $x^{2}\not>a$.\hfill
                (Proof by Contradiction)\par
                If $x^{2}<y$, then $0<\frac{y-x^{2}}{2x+1}$\hfill
                (Additive Property of Ordered Fields)\par
                Let $\epsilon=\min\{\frac{y-x^{2}}{2x+1},1\}$\par
                Then
                $(x+\epsilon)^{2}=x^{2}+2x\epsilon+\epsilon^{2}%
                 <x^{2}+2x\epsilon+\epsilon$\hfill
                (Thm.~\ref{%
                    Theorem:MathEnc:Analysis:%
                    Sum:SqurPresIneqPosNum%
                })\par
                But
                $x^{2}+2x\epsilon+\epsilon=x^{2}+(2x+1)\epsilon=y$
                \hfill (Elementary Algebra)\par
                Therefore $(x+\epsilon)^{2}<y$\hfill
                (Transitive Property of Total Orders)\par
                But $\epsilon>0$, and thus $x+\epsilon>x$.\hfill
                (Additive Property of Ordered Fields)\par
                Thus $x+\epsilon\in A$, a contradiction as
                $x$ is a least upper bound of $A$.\par
                Therefore $x^{2}\not<y$.\hfill
                (Proof by Contradiction)\par
                Therefore $x^{2}=y$\hfill
                (Trichotomous Property of Inequalities)\par
            \end{proof}
            \clearpage
\section{Notes}
\end{document}