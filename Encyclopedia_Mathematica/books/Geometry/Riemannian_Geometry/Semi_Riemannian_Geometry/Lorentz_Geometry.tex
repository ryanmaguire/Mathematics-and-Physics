\section{Manifolds}
    The inner product
    $\langle{\cdot,\,\cdot\rangle}:\mathbb{R}^{n}\rightarrow\mathbb{R}$
    defined by:
    \begin{equation}
        \langle{\mathbf{x},\,\mathbf{y}}\rangle
        =\sum_{k=1}^{n}x_{i}y_{i}
    \end{equation}
    induces the standard norm and metric on $\mathbb{R}^{n}$:
    \par\hfill\par
    \begin{subequations}
        \begin{minipage}[b]{0.49\textwidth}
            \begin{equation}
                \norm{\mathbf{x}}
                =\sqrt{\langle{\mathbf{x},\,\mathbf{x}}\rangle}
            \end{equation}
        \end{minipage}
        \hfill
        \begin{minipage}[b]{0.49\textwidth}
            \begin{equation}
                d(\mathbf{x},\,\mathbf{y})
                =\norm{\mathbf{x}-\mathbf{y}}
            \end{equation}
        \end{minipage}
    \end{subequations}
    \par\vspace{2.5ex}
    which further generates the standard topology on $\mathbb{R}^{n}$.
    \begin{fdefinition}{Smooth Real-Valued Functions On $\mathbb{R}$}
                       {Smooth_Real_Valued_Functions_on_R}
        A smooth real-valued function on an open subset
        $\mathcal{U}\subseteq\mathbb{R}^{n}$ is a function
        $f:\mathcal{U}\rightarrow\mathbb{R}$ such that all mixed partial
        derivatives of all orders exist and are continuous for all
        $\mathbf{x}\in\mathcal{U}$.
    \end{fdefinition}
    $\mathbb{R}^{n}$ can be defined as the set of all functions
    $\mathbf{x}:\mathbb{Z}_{n}\rightarrow\mathbb{R}$. Given an element
    $\mathbf{x}\in\mathbb{R}^{n}$ and $k\in\mathbb{Z}_{n}$ we denote image
    of $k$ as $x_{k}=\mathbf{x}(k)$. This is called the $k^{th}$ coordinate
    of $\mathbf{x}$. The projection mapping
    $\pi_{k}:\mathbb{R}^{n}\rightarrow\mathbb{R}$ is the functions defined
    by $\pi(\mathbf{x})=x_{k}$.
    \begin{ldefinition}{Smooth Euclidean Functions}{Smooth_Euclidean_Functions}
        A smooth function on a subset $\mathcal{U}\subseteq\mathbb{R}^{n}$
        to $\mathbb{R}^{m}$ is a function
        $f:\mathcal{U}\rightarrow\mathbb{R}^{m}$ such that, for all
        $k\in\mathbb{Z}_{m}$, the function $\pi_{k}\circ{f}$ is a smooth
        real-valued function.
    \end{ldefinition}
    \begin{ldefinition}{Charts}{Charts}
        A chart of dimension $n\in\mathbb{N}$ in a topological space
        $(X,\tau)$, denoted $(\mathcal{U},\phi)$, is an open set
        $\mathcal{U}\in\tau$ and a function
        $\phi:\mathcal{U}\rightarrow\mathbb{R}^{n}$ such that $\phi$ is
        a homeomorphism between $\mathcal{U}$ and its image
        $\phi(\mathcal{U})$.
    \end{ldefinition}
    The coordinate functions of a chart $(\mathcal{U},\phi)$ are the
    compositions $\pi_{k}\circ\phi:\mathcal{U}\rightarrow\mathbb{R}$.
    \begin{fdefinition}{Smoothly Overlapping Charts}
                       {Smoothly_Overlapping_Charts}
        Smoothly overlapping charts of dimension $n\in\mathbb{N}$ are
        charts $(\mathcal{U}_{1},\phi_{1})$ and
        $(\mathcal{U}_{2},\phi_{2})$ of dimension $n$ on a topological
        space $(X,\tau)$ such that
        $\phi_{1}\circ\phi_{2}^{\minus{1}}:%
         \phi_{2}(\mathcal{U}_{1}\cap\mathcal{U}_{2})%
         \rightarrow\mathbb{R}^{n}$ and
        $\phi_{2}\circ\phi_{1}^{\minus{1}}:%
         \phi_{1}(\mathcal{U}_{1}\cap\mathcal{U}_{2})%
         \rightarrow\mathbb{R}^{n}$ are smooth functions.
    \end{fdefinition}
    Given charts $(\mathcal{U}_{1},\phi_{1})$ and
    $(\mathcal{U}_{2},\phi_{2})$ in a topological space $(X,\tau)$ such
    that $\mathcal{U}_{1}$ and $\mathcal{U}_{2}$ are disjoint, we see that
    the two charts are automatically smoothly overlapping in a
    vacuous sense.
    \begin{figure}[H]
        \centering
        \captionsetup{type=figure}
        %--------------------------------Dependencies----------------------------------%
%   tikz                                                                       %
%       arrows.meta                                                            %
%-------------------------------Main Document----------------------------------%
\begin{tikzpicture}[>=Latex, line width=0.2mm]
    % Coordinates for the manifold X.
    \coordinate (X0) at (-5.0,  0.0);
    \coordinate (X1) at (-3.5, -2.5);
    \coordinate (X2) at ( 1.0, -2.0);
    \coordinate (X3) at ( 5.0,  0.0);
    \coordinate (X4) at ( 0.0,  1.0);

    % Coordinates for the subset U.
    \coordinate (U0) at (-4.0, -0.5);
    \coordinate (U1) at (-3.0, -2.0);
    \coordinate (U2) at ( 1.5, -0.5);
    \coordinate (U3) at (-0.6,  0.2);

    % Coordinates for the subset V.
    \coordinate (V0) at ( 4.0,  0.0);
    \coordinate (V1) at ( 3.0, -1.5);
    \coordinate (V2) at (-1.5, -0.5);
    \coordinate (V3) at ( 0.6,  0.2);

    % Draw the manifold X.
    \draw   (X0) to[out=-90, in=120]  (X1)
                 to[out=-60, in=-170] (X2)
                 to[out=10, in=-90]   (X3)
                 to[out=90, in=0]     (X4)
                 to[out=-180, in=90]  cycle;

    % Fill in U and V first and then outline with dashes.
    % This prevents the fill option from drawing over the outline.
    % Setting opacity makes the overlapping part mix colors as well.

    % Fill in the background of U blue.
    \draw[fill=blue, opacity=0.5, draw=none]
        (U0) to[out=-90, in=-180] (U1)
             to[out=0, in=-100]   (U2)
             to[out=80, in=0]     (U3)
             to[out=-180, in=90]  cycle;

    % Fill in the background of V red.
    \draw[fill=red, opacity=0.5, draw=none]
        (V0) to[out=-90, in=0]   (V1)
             to[out=180, in=-80] (V2)
             to[out=100, in=180] (V3)
             to[out=0, in=90]    cycle;

    % Draw dashed lines around U.
    \draw[densely dashed]
        (U0) to[out=-90, in=-180] (U1)
             to[out=0, in=-100]   (U2)
             to[out=80, in=0]     (U3)
             to[out=-180, in=90]  cycle;

    \draw[densely dashed]
        (V0) to[out=-90, in=0]   (V1)
             to[out=180, in=-80] (V2)
             to[out=100, in=180] (V3)
             to[out=0, in=90]    cycle;

    \begin{scope}[xshift=-5cm, yshift=3cm]

        % Coordinates for phi of U.
        \coordinate (P0) at (0.5, 0.5);
        \coordinate (P1) at (1.5, 0.2);
        \coordinate (P2) at (3.3, 0.8);
        \coordinate (P3) at (2.8, 2.1);
        \coordinate (P4) at (2.2, 3.6);
        \coordinate (P5) at (1.2, 2.8);

        % Coordinate for some midpoint inside U.
        \coordinate (PM) at (2.0, 1.5);

        \draw[->] (-0.5,  0.0) to ( 4.0,  0.0);
        \draw[->] ( 0.0, -0.5) to ( 0.0,  4.0);

        \draw[draw=none, fill=blue!20!white]
            (P0)    to[out=-30,  in=180]    (P1)
                    to[out=0,    in=-90]    (P2)
                    to[out=90,   in=-120]   (P3)
                    to[out=60,   in=30]     (P4)
                    to[out=-150, in=60]     (P5)
                    to[out=-120, in=150]    cycle;

        \draw[densely dashed]
            (P0)    to[out=-30,  in=180]    (P1)
                    to[out=0,    in=-90]    (P2)
                    to[out=90,   in=-120]   (P3)
                    to[out=60,   in=30]     (P4)
                    to[out=-150, in=60]     (P5)
                    to[out=-120, in=150]    cycle;

        \draw[densely dashed, fill=cyan]
            (P3)    to[out=180,  in=70]   (PM)
                    to[out=-110, in=180]  (P1)
                    to[out=0,    in=-90]  (P2)
                    to[out=90,   in=-120] cycle;

        \node at (2.00, 3.0) {$\phi(\mathcal{U})$};
        \node at (2.45, 0.8) {$\phi(\mathcal{U}\cap\mathcal{V})$};
        \node at (3.50, 3.5) {\large{$\mathbb{R}^{n}$}};
    \end{scope}

    \begin{scope}[xshift=2cm, yshift=3cm]

        % Coordinates for phi of U.
        \coordinate (Q0) at (3.5, 0.5);
        \coordinate (Q1) at (2.5, 0.2);
        \coordinate (Q2) at (0.5, 0.8);
        \coordinate (Q3) at (1.2, 2.1);
        \coordinate (Q4) at (1.8, 3.6);
        \coordinate (Q5) at (2.8, 2.8);

        % Coordinate for some midpoint inside U.
        \coordinate (QM) at (2.0, 1.5);

        \draw[->] (-0.5,  0.0) to ( 4.0,  0.0);
        \draw[->] ( 0.0, -0.5) to ( 0.0,  4.0);

        \draw[draw=none, fill=red!20!white]
            (Q0)    to[out=-150,    in=0]       (Q1)
                    to[out=-180,    in=-90]     (Q2)
                    to[out=90,      in=-120]    (Q3)
                    to[out=60,      in=150]     (Q4)
                    to[out=-30,     in=60]      (Q5)
                    to[out=-120,    in=30]      cycle;

        \draw[densely dashed]
            (Q0)    to[out=-150,    in=0]       (Q1)
                    to[out=-180,    in=-90]     (Q2)
                    to[out=90,      in=-120]    (Q3)
                    to[out=60,      in=150]     (Q4)
                    to[out=-30,     in=60]      (Q5)
                    to[out=-120,    in=30]      cycle;

        \draw[densely dashed, fill=red!50!white]
            (Q3)    to[out=-60,     in=70]      (QM)
                    to[out=-110,    in=0]       (Q1)
                    to[out=-180,    in=-90]     (Q2)
                    to[out=90,      in=-120]    cycle;

        \node at (2.00, 3.0) {$\xi(\mathcal{V})$};
        \node at (1.25, 0.8) {$\xi(\mathcal{U}\cap\mathcal{V})$};
        \node at (3.50, 3.5) {\large{$\mathbb{R}^{n}$}};
    \end{scope}

    \begin{scope}[line width=0.4mm, ->, font=\large]
        \draw (-2.0, 0.5) to[out=130, in=-100] node[left]  {$\phi$} (-3.0, 3);
        \draw ( 2.5, 0.7) to[out=50,  in=-80]  node[right] {$\xi$}  ( 3.5, 3);
        \draw (-1.5, 4.5) to[out=30, in=150]
            node[above] {$\xi\circ\phi^{\minus{1}}$} ( 1.5, 4.5);
        \draw ( 1.5, 3.5) to[out=-150, in=-30]
            node[below] {$\phi\circ\xi^{\minus{1}}$} (-1.5, 3.5);
    \end{scope}

    \node at (-4.0,  0.5) {$X$};
    \node at (-3.0, -1.5) {$\mathcal{U}$};
    \node at ( 3.0, -1.3) {$\mathcal{V}$};
    \node at ( 0.0, -0.5) {$\mathcal{U}\cap\mathcal{V}$};
\end{tikzpicture}
        \caption{Smoothly Overlapping Charts}
        \label{fig:Smoothly_Overlapping_Charts}
    \end{figure}
    \begin{fdefinition}{Atlas}{Atlas}
        An atlas on a topological space $(X,\tau)$ is a set of charts
        $\mathcal{A}$ on $(X,\tau)$ such that, for all $x\in{X}$, there
        is a $(\mathcal{U},\phi)\in\mathcal{O}$ such that $x\in\mathcal{U}$.
    \end{fdefinition}
    \begin{ldefinition}{$n$ Dimensional Atlas}{n_Dimensional_Atlas}
        An atlas of dimension $n\in\mathbb{N}$ on a topological space
        $(X,\tau)$ is an atlas $\mathcal{A}$ on $(X,\tau)$ such that, for
        all $(\mathcal{U},\phi)\in\mathcal{A}$, $(\mathcal{U},\phi)$ is
        a chart of dimension $n$.
    \end{ldefinition}
    \begin{fdefinition}{Transition Function}{Transition_Function}
        The transition function of a chart $(\mathcal{U},\phi)$ with
        respect a chart $(\mathcal{V},\xi)$ on a topological space
        $(X,\tau)$ is the function
        $f:\phi(\mathcal{U}\cap\mathcal{V})%
         \rightarrow\xi(\mathcal{V}\cap\mathcal{V})$ defined by:
        \begin{equation}
            f(x)=(\xi\circ\phi^{\minus{1}})(x)
        \end{equation}
    \end{fdefinition}
    \begin{fdefinition}{Smooth Atlas}{Smooth Atlas}
        A smooth atlas on a topological space $(X,\tau)$ is an
        atlas $\mathcal{A}$ such that for all charts
        $(\mathcal{U},\phi),(\mathcal{V},\xi)\in\mathcal{A}$, the
        transition function of $(\mathcal{U},\phi)$ with respect to
        $(\mathcal{V},\xi)$ is smooth.
    \end{fdefinition}
    \begin{fdefinition}{Maximal Smooth Atlas}{Maximal_Smooth_Atlas}
        A complete atlas on a topological space $(X,\tau)$ is a smooth
        atlas $\mathcal{A}$ on $(X,\tau)$ such that, for all charts
        $(\mathcal{U},\phi)$ of $(X,\tau)$ such that $(\mathcal{U},\phi)$
        overlaps smoothly with all $(\mathcal{V},\xi)\in\mathcal{A}$, it
        is true that $(\mathcal{U},\phi)\in\mathcal{A}$.
    \end{fdefinition}
    \begin{theorem}
        If $(X,\tau)$ is a topological space and if $\mathcal{A}$ is a
        smooth atlas of dimension $n\in\mathbb{N}$ on a topological space
        $(X,\tau)$, then there is a unique maximal unique atlas
        $\mathcal{C}$ on $(X,\tau)$ such that
        $\mathcal{A}\subseteq\mathcal{C}$.
    \end{theorem}
    \begin{proof}
        For let $\mathcal{C}$ be the set of all charts on $(X,\tau)$ that
        overlap smoothly with the charts in $\mathcal{A}$. Then since
        $\mathcal{A}$ is an atlas, for all
        $(\mathcal{U},\phi)\in\mathcal{A}$ and for all
        $(\mathcal{V},\xi)\in\mathcal{A}$, we have that $(\mathcal{U},\phi)$
        and $(\mathcal{V},\xi)$ overlap smoothly, and thus
        $(\mathcal{U},\phi)\in\mathcal{C}$. Therefore
        $\mathcal{A}\subseteq\mathcal{C}$. But $\mathcal{A}$ is an
        atlas and thus for all $x\in{X}$ there is a chart
        $(\mathcal{U},\phi)\in\mathcal{A}$ such that $x\in\mathcal{U}$.
        But $\mathcal{A}\subseteq\mathcal{C}$ and thus
        $(\mathcal{U},\phi)\in\mathcal{C}$. Thus, for all $x\in{X}$ there
        is a chart $(\mathcal{U},\phi)\in\mathcal{C}$ such that
        $x\in\mathcal{U}$. Suppose
        $(\mathcal{U}_{1},\phi_{1}),%
         (\mathcal{U}_{2},\phi_{2})\in\mathcal{C}$.
        If $\mathcal{U}_{1}$ and $\mathcal{U}_{2}$ are disjoint, then
        these two charts overlap smoothly. Suppose it is non-empty and let
        $f$ be the transition function of $(\mathcal{U}_{1},\phi_{1})$ with
        respect to $(\mathcal{U}_{2},\phi_{2})$. Let
        $p\in\phi_{1}(\mathcal{U}_{1}\cap\mathcal{U}_{2})$. But then there
        is a chart $\xi\in\mathcal{A}$ such that $\phi_{2}^{\minus{1}}(p)$
        is contained in the domain of $\xi$. From the associativity of
        composition, we have:
        \begin{equation}
            \phi_{1}\circ\phi_{2}^{\minus{1}}
            =(\phi_{1}\circ\xi^{\minus{1}})\circ
             (\xi\circ\phi_{2}^{\minus{1}})
        \end{equation}
        But by the definition of $\mathcal{C}$, $\phi_{1}$ and $\phi_{2}$
        overlap smoothly with $\xi$, and thus this is the composition of
        smooth functions, and is therefore smooth. Therefore
        $\phi_{1}\circ\phi_{2}^{\minus{1}}$ is smooth and thus
        $(\mathcal{U}_{1},\phi_{1})$ and $(\mathcal{U}_{2},\phi_{2})$
        overlap smoothly. Thus, $\mathcal{C}$ is a smooth atlas. Moreover,
        it is complete from the construction. Given any other complete
        atlas $\mathcal{C}'$ that contains $\mathcal{A}$ we would have
        $\mathcal{C}\subseteq\mathcal{C}'$ and
        $\mathcal{C}'\subseteq\mathcal{C}$, and therefore
        $\mathcal{C}=\mathcal{C}'$. Thus, this completion is unique.
    \end{proof}
    \begin{fdefinition}{Smooth Manifold}{Smooth_Manifold}
        A smooth manifold of dimension $n\in\mathbb{N}$, denoted
        $(X,\tau,\mathcal{A})$ is a Hausdorff topological space
        $(X,\tau)$ with a maximal smooth atlas $\mathcal{A}$ of
        dimension $n$.
    \end{fdefinition}
    Any smooth atlas $\mathcal{A}$ on a topological space $(X,\tau)$
    defines a a smooth manifold if we let $\mathcal{C}$ be the maximal
    smooth atlas generated by $\mathcal{A}$.
    \begin{example}
        Let $(\mathbb{R}^{n},\tau_{\mathbb{R}^{n}})$ be the standard
        $n$ dimensional Euclidean space. We can define a trivial smooth
        atlas on this space by let
        $\mathcal{A}=\{(\mathbb{R}^{n},\textrm{id})\}$, where $\textrm{id}$
        is the identity function. This defines a smooth atlas. By
        considering the unique maximal smooth atlas generated by this
        we obtain the standard smooth structure on $\mathbb{R}^{n}$.
    \end{example}
    \begin{theorem}
        If $(X,\tau,\mathcal{A})$ is a smooth manifold of dimension
        $n\in\mathbb{N}$, $(\mathcal{U},\phi)\in\mathcal{A}$,
        $\mathcal{V}\in\tau$, and if $\phi_{\mathcal{V}}$ denotes
        the restriction mapping:
        $\phi_{\mathcal{V}}:\mathcal{V}\rightarrow\mathbb{R}^{n}$,
        then $(\mathcal{V},\phi_{\mathcal{V}})\in\mathcal{A}$.
    \end{theorem}
    \begin{proof}
        For since $\phi$ is a homeomorphism from $\mathcal{U}$ to
        $\phi(\mathcal{U})$, and since $\mathcal{V}\in\tau$, we have
        that $\phi_{\mathcal{V}}$ is a homeomorphism between
        $\mathcal{V}$ and $\phi_{\mathcal{V}}(\mathcal{V})$, and
        therefore $(\mathcal{V},\phi_{\mathcal{V}})$ is a chart. But
        this chart meets $(\mathcal{U},\phi)$ smoothly, and $\mathcal{A}$
        is complete. Thus,
        $(\mathcal{V},\phi_{\mathcal{A}})\in\mathcal{A}$.
    \end{proof}
    \begin{theorem}
        If $n\in\mathbb{N}$, then there is a complete atlas
        $\mathcal{A}$ on $(S^{n},\tau)$, where $\tau$ is the inherited
        topology from $\mathbb{R}^{n+1}$.
    \end{theorem}
    \begin{proof}
        For all $k\in\mathbb{Z}_{n+1}$, let $\mathcal{U}_{k}^{+}$ and
        $\mathcal{U}_{k}^{\minus}$ be defined as:
        \par\hfill\par
        \begin{minipage}[b]{0.49\textwidth}
            \begin{equation}
                \mathcal{U}_{k}^{+}
                =\{\,\mathbf{x}\in{S}^{n}\,:\,x_{k}>0\,\}
            \end{equation}
        \end{minipage}
        \hfill
        \begin{minipage}[b]{0.49\textwidth}
            \begin{equation}
                \mathcal{U}_{k}^{\minus}
                =\{\,\mathbf{x}\in{S}^{n}\,:\,x_{k}<0\,\}
            \end{equation}
        \end{minipage}
        \par\vspace{2.5ex}
        Define $\phi_{\mathcal{U}_{k}^{+}}:\mathcal{U}_{k}^{+}%
                \rightarrow\mathbb{R}^{n}$ by:
        \begin{equation}
            \phi_{\mathcal{U}_{k}^{+}}(\mathbf{x})
            =(x_{1},\,\dots,\,x_{k-1},\,x_{k+1},\,\dots,\,x_{n+1})
        \end{equation}
        That is, the mapping that projects the point onto the plane
        defined by $x_{k}=0$. Define $\phi_{\mathcal{U}_{k}^{\minus}}$
        similarly. Then all such $\phi$ are homeomorphisms from their
        domain to their image. Let $\mathcal{A}$ be defined as follows:
        \begin{equation}
            \mathcal{A}
            =\big\{\,(\mathcal{U}_{k}^{+},\,\phi_{\mathcal{U}_{k}^{+}})
                   \,:\,k\in\mathbb{Z}_{n}\big\}\bigcup
            \big\{\,(\mathcal{U}_{k}^{-},\,\phi_{\mathcal{U}_{k}^{-}})
                   \,:\,k\in\mathbb{Z}_{n}\big\}
        \end{equation}
        Then $\mathcal{A}$ is an atlas of $(S^{n},\tau)$. For if
        $\mathbf{x}\in{S}^{n}$, then $\norm{\mathbf{x}}_{2}=1$. But then
        there is a coordinate $x_{k}$ of $\mathbf{x}$ such that
        $x_{k}\ne{0}$. But then either $x_{k}>0$ or $x_{k}<0$, and thus
        either $\mathbf{x}\in\mathcal{U}_{k}^{+}$ or
        $\mathbf{x}\in\mathcal{U}_{k}^{\minus}$. If
        $(\mathcal{V}_{2},\phi_{2})$ and $(\mathcal{V}_{2},\phi_{2})$
        are charts, then either $\phi_{1}(\mathcal{V}_{1})$
        and $\phi_{2}(\mathcal{V}_{2})$ are disjoint or they are not.
        If they are disjoint, then $\phi_{1}$ and $\phi_{2}$ overlap
        smoothly. If they are not disjoint, let $\mathbf{x}$ be contained
        in the intersection. But then, for all $k\in\mathbb{Z}_{n}$,
        $\pi_{k}\circ(\phi_{1}\circ\phi_{2}^{\minus{1}})$ is smooth,
        and thus $\phi_{1}$ and $\phi_{2}$ overlap smoothly.
    \end{proof}
    \begin{fdefinition}{Open Submanifold}{Open_Submanifold}
        An open submanifold on a manifold $(X,\tau,\mathcal{A})$ is a
        an open subset $\mathcal{U}\subseteq{X}$ and the collection
        $\mathcal{A}_{\mathcal{U}}$ defined by:
        \begin{equation}
            \mathcal{A}_{\mathcal{U}}
            =\{\,(\mathcal{V},\phi)\in\mathcal{A}\,:\,
                 \mathcal{V}\subseteq\mathcal{U}\,\}
        \end{equation}
        Together with the inherited topology $\tau_{\mathcal{U}}$.
    \end{fdefinition}
    \begin{theorem}
        If $(X,\tau,\mathcal{A})$ is a smooth manifold and if
        $(\mathcal{U},\tau_{\mathcal{U}},\mathcal{A}_{\mathcal{U}})$
        is an open submanifold, then it is a smooth manifold.
    \end{theorem}
    \begin{proof}
        For by the previous theorem, $\mathcal{A}_{\mathcal{U}}$ is a
        complete atlas. Moreover, a subspace of a Hausdorff topological
        space is also a Hausdorff topological space, and hence
        $(\mathcal{U},\tau_\mathcal{U})$ is a Hausdorff space. Thus,
        $(\mathcal{U},\tau_\mathcal{U},\mathcal{A}_{\mathcal{U}})$ is
        a smooth manifold.
    \end{proof}
    \begin{fdefinition}{Product Chart}{Product_Chart}
        The product chart of an $n$ dimensional chart $(\mathcal{U},\phi)$
        on a topological space $(X,\tau_{X})$ with an $m$ dimensional chart
        $(\mathcal{V},\xi)$ on a topological space $(Y,\tau_{Y})$ is
        the ordered pair $(\mathcal{U}\times\mathcal{V},f)$ where
        $f:\mathcal{U}\times\mathcal{V}\rightarrow\mathbb{R}^{n+m}$
        defined by:
        \begin{equation}
            f(p,q)_{k}=
            \begin{cases}
                \phi(p)_{k},&k<n\\
                \xi(q)_{k},&n\leq{k}<n+m
            \end{cases}
        \end{equation}
        Where $\phi(p)_{k}$ is the $k^{th}$ coordinate of
        $\phi(p)\in\mathbb{R}^{n}$ and $\xi(q)_{k}$ is the $k^{th}$
        coordinate of $\xi(q)\in\mathbb{R}^{m}$. We denote this by
        $(\mathcal{U},\phi)\times(\mathcal{V},\xi)$.
    \end{fdefinition}
    Thinking of the elements of $\mathbb{R}^{n+m}$ as tuples of length
    $n+m$, we can write:
    \begin{equation}
        f(p,q)=\big(x_{1}(p),\dots,x_{n}(p),y_{1}(q),\dots,y_{m}(q)\big)
    \end{equation}
    \begin{theorem}
        If $(X,\tau_{X},\mathcal{A}_{X})$ and $(Y,\tau_{Y},\mathcal{A}_{Y})$
        are smooth manifolds, and if $\mathcal{A}$ is the set of all
        product charts on $X\times{Y}$, then $\mathcal{A}$ is a smooth
        atlas on $(X\times{Y},\tau_{X\times{Y}})$, where
        $\tau_{X\times{Y}}$ is the product topology.
    \end{theorem}
    \begin{proof}
        For if $p\in{X}\times{Y}$ then there is an $x\in{X}$ and a
        $y\in{Y}$ such that $p=(x,y)$. But $\mathcal{A}_{X}$ is a smooth
        atlas on $(X,\tau_{X})$, and thus if $x\in{X}$ then there is a
        $(\mathcal{U},\phi)\in\mathcal{A}_{X}$ such that $x\in\mathcal{U}$.
        Similarly, there is a $(\mathcal{V},\xi)\in\mathcal{A}_{Y}$ such
        that $y\in\mathcal{V}$. But then $p\in\mathcal{U}\times\mathcal{V}$,
        and $\mathcal{U}\times\mathcal{V}\in\tau_{X\times{Y}}$. But if
        $\phi:\mathcal{U}\rightarrow\mathbb{R}^{n}$ is a homeomorphism
        between $\mathcal{U}$ and $\phi(\mathcal{U})$ and
        $\xi:\mathcal{V}\rightarrow\mathbb{R}^{m}$ is a homemorphism
        between $\mathcal{V}$ and $\xi(\mathcal{V})$, then
        $f:\mathcal{U}\times\mathcal{V}\rightarrow\mathbb{R}^{n+m}$ is
        a homeomorphism between $\mathcal{U}\times\mathcal{V}$ and
        $f(\mathcal{U}\times\mathcal{V})$, and thus the product chart
        is a chart in $(X\times{Y},\tau_{X\times{Y}})$. Moreover, all of
        the elements of $\mathcal{A}$ are smoothly overlapping. Thus,
        $\mathcal{A}$ is an atlas on $(X\times{Y},\tau_{X\times{Y}})$.
    \end{proof}
    Using the maximal smooth atlas generated by the product atlas
    $\mathcal{A}$ creates the product manifold.
    \subsection{Smooth Mappings}
        \begin{fdefinition}{Smooth Real-Valued Functions}
                           {Smooth_Real_Valued_Functions}
            A smooth real-valued function on a manifold
            $(X,\tau,\mathcal{A})$ of dimension $n\in\mathbb{N}$ is a
            function $f:X\rightarrow\mathbb{R}$ such that, for every chart
            $(\mathcal{U},\phi)\in\mathcal{A}$, the function
            $f\circ\phi^{\minus{1}}:\phi(\mathcal{U})\rightarrow\mathbb{R}$
            is a smooth Euclidean function.
        \end{fdefinition}
        \begin{theorem}
            If $(X,\tau,\mathcal{A})$ is a manifold and if
            $f,g:X\rightarrow\mathbb{R}$ are smooth real-valued functions,
            then $(f+g):X\rightarrow\mathbb{R}$ defined by:
            \begin{equation}
                (f+g)(x)=f(x)+g(x)
                \quad\quad
                x\in{X}
            \end{equation}
            Is a smooth real-valued function.
        \end{theorem}
        \begin{theorem}
            If $(X,\tau,\mathcal{A})$ is a manifold and if
            $f,g:X\rightarrow\mathbb{R}$ are smooth real-valued functions,
            then $(f\cdot{g}):X\rightarrow\mathbb{R}$ defined by:
            \begin{equation}
                (f\cdot{g})(x)=f(x)\cdot{g}(x)
                \quad\quad
                x\in{X}
            \end{equation}
            Is a smooth real-valued function.
        \end{theorem}
        \begin{fdefinition}{Smooth Functions Between Manifolds}
                           {Smooth Functions Between Manifolds}
            A smooth function from a manifold $(X,\tau_{X},\mathcal{A}_{X})$
            of dimension $n\in\mathbb{N}$ to a manifold
            $(Y,\tau_{Y},\mathcal{A}_{Y})$ of dimension $m\in\mathbb{N}$ is
            a function $f:X\rightarrow{Y}$ such that, for every chart
            $(\mathcal{U},\phi)\in\mathcal{A}_{X}$ and for every chart
            $(\mathcal{V},\xi)\in\mathcal{A}_{Y}$, the function
            $\xi\circ{f}\circ\phi^{\minus{1}}:\xi(\mathcal{V})%
             \rightarrow\mathbb{R}^{m}$ is a smooth Euclidean function.
        \end{fdefinition}
        \begin{theorem}
            If $(X,\tau_{X},\mathcal{A}_{X})$ and
            $(Y,\tau_{Y},\mathcal{A}_{Y})$ are manifolds, if
            $A_{X}\subseteq\mathcal{A}_{X}$ is an atlas on $(X,\tau_{X})$,
            if $A_{Y}\subseteq\mathcal{A}_{Y}$ is an atlas on
            $(Y,\tau_{Y})$, and if $f:X\rightarrow{Y}$ is a function such
            that, for all $(\mathcal{U},\phi)\in{A}_{X}$ and for all
            $(\mathcal{Y},\xi)\in{A}_{y}$ it is true that
            $\xi\circ{f}\circ\phi^{\minus{1}}:\xi(\mathcal{V})%
             \rightarrow\mathbb{R}^{m}$ is a smooth Euclidean function,
            then $f$ is smooth.
        \end{theorem}
        \begin{proof}
            Since charts in $\mathcal{A}_{X}$ and $\mathcal{A}_{Y}$
            overlap smoothly with charts in $A_{X}$ and $A_{Y}$, and since
            the atlases $A_{X}$ and $A_{Y}$ cover $X$ and $Y$, respectively,
            we are done.
        \end{proof}
        \begin{theorem}
            If $(X,\tau_{X},\mathcal{A}_{X})$ is a manifold, then
            $\textrm{id}:X\rightarrow{X}$ is a smooth function.
        \end{theorem}
        \begin{theorem}
            If $(X,\tau_{X},\mathcal{A}_{X})$,
            $(Y,\tau_{Y},\mathcal{A}_{Y})$, and
            $(Z,\tau_{Z},\mathcal{A}_{Z})$ are manifolds, if
            $f:X\rightarrow{Y}$ and $g:Y\rightarrow{Z}$ are smooth, then
            $g\circ{f}:X\rightarrow{Z}$ is smooth.
        \end{theorem}
        Smoothness is a local property. A function $\phi:M\rightarrow{N}$
        is smooth at $p\in{M}$ if there is a neighborhood
        $\mathcal{U}$ of $p$ such that the restriction of $\phi$ to
        $\mathcal{U}$ is smooth. A smooth function is thus a function
        that is smooth at every point.
        \begin{theorem}
            If $(X,\mathcal{A}_{X},\tau_{X})$ and
            $(Y,\mathcal{A}_{Y},\tau_{Y})$ are manifolds and if
            $f:X\rightarrow{Y}$ is smooth, then $f$ is continuous.
        \end{theorem}
        \begin{fdefinition}{Diffeomorphism}{Diffeomorphism}
            A diffeomorphism from a manifold $(X,\tau_{X},\mathcal{A}_{X})$
            to a manifold $(Y,\tau_{Y},\mathcal{A}_{Y})$ is a bijective
            function $f:X\rightarrow{Y}$ such that $f$ and $f^{\minus{1}}$
            are smooth.
        \end{fdefinition}
        \begin{lexample}
            For any $a,b\in\mathbb{R}$ with $a<b$, the interval
            $(a,b)$ is diffeomorphic to the unit interval $(0,1)$. For
            let $\phi:(0,1)\rightarrow(a,b)$ be defined by:
            \begin{equation}
                \phi(t)=(a-b)t+b
            \end{equation}
            Then $\phi$ is a smooth bijection and it's inverse is smooth.
            Moreover, the unit interval is diffeomorphic to $\mathbb{R}$.
            For let $\xi:(0,1)\rightarrow\mathbb{R}$ be defined by:
            \begin{equation}
                \xi(t)=\frac{2t}{t(1-t)}
            \end{equation}
        \end{lexample}
        \begin{theorem}
            If $(X,\tau_{X},\mathcal{A}_{X})$ and
            $(Y,\tau_{Y},\mathcal{A}_{Y})$ are manifolds, and if
            $f:X\rightarrow{Y}$ is a diffeomorphism, then $f$ is a
            homeomorphism from $(X,\tau_{X})$ to $(Y,\tau_{Y})$.
        \end{theorem}
        \begin{proof}
            For if $f$ is a diffeomorphism, then it is a smooth bijection
            such that it's inverse is smooth. But if $f$ is smooth, then
            it is continuous and therefore it is a continuous bijection.
            But if $f^{\minus{1}}$ is smooth, then it is continuous, and
            thus $f$ is a bicontinuous bijective function, and is therefore
            a homeomorphism.
        \end{proof}
        A smooth homeomorphism need not be a diffeomorphism. The inverse
        function may not be smooth. For let
        $f:\mathbb{R}\rightarrow\mathbb{R}$ be defined by $f(x)=x^{3}$.
        Then $f$ is a homeomorphism and it's forward direction is smooth,
        but $f^{\minus{1}}$ is not smooth at the origin.
        \begin{ftheorem}{}{}
            If $A$ is a set, if $(X,\tau,\mathcal{A})$ is a manifold, and
            if $f:A\rightarrow{X}$ is an bijective function, then there
            exists a topology $\tau_{A}$ and an atlas $\mathcal{A}_{A}$
            on $X$ such that $f$ is a diffeomorphism.
        \end{ftheorem}