%------------------------------------------------------------------------------%
\documentclass[crop=false,class=article]{standalone}                           %
%------------------------------Preamble----------------------------------------%
\makeatletter                                                                  %
    \def\input@path{{../../../}}                                               %
\makeatother                                                                   %
%---------------------------Packages----------------------------%
\usepackage{geometry}
\geometry{b5paper, margin=1.0in}
\usepackage[T1]{fontenc}
\usepackage{graphicx, float}            % Graphics/Images.
\usepackage{natbib}                     % For bibliographies.
\bibliographystyle{agsm}                % Bibliography style.
\usepackage[french, english]{babel}     % Language typesetting.
\usepackage[dvipsnames]{xcolor}         % Color names.
\usepackage{listings, lstlinebgrd}      % Verbatim-Like Tools.
\usepackage{mathtools, esint, mathrsfs} % amsmath and integrals.
\usepackage{amsthm, amsfonts}           % Fonts and theorems.
\usepackage{tabularx}
\usepackage{tcolorbox}                  % Frames around theorems.
\usepackage{upgreek}                    % Non-Italic Greek.
\usepackage{paracol}                    % Two-column styling.
\usepackage{wrapfig}                    % Wrap text around figure.
\usepackage{fmtcount, etoolbox}         % For the \book{} command.
\usepackage[newparttoc]{titlesec}       % Formatting chapter, etc.
\usepackage{titletoc}                   % Allows \book in toc.
\usepackage[nottoc]{tocbibind}          % Bibliography in toc.
\usepackage[titles]{tocloft}            % ToC formatting.
\usepackage{multicol, enumitem}         % Multi-column/enumerate.
\usepackage{import}                     % Import external files.
\usepackage{pgfplots, tikz}             % Drawing/graphing tools.
\usetikzlibrary{
    calc,                   % Calculating right angles and more.
    angles,                 % Drawing angles within triangles.
    arrows.meta,            % Latex and Stealth arrows.
    quotes,                 % Adding labels to angles.
    positioning,            % Relative positioning of nodes.
    decorations.markings,   % Adding arrows in the middle of a line.
    patterns,
    arrows,
    shapes,
    shapes.geometric,
    cd,
    hobby,
    babel
}                                       % Libraries for tikz.
\pgfplotsset{compat=1.9}                % Version of pgfplots.
\usepackage[font=scriptsize,
            labelformat=simple,
            labelsep=colon]{subcaption} % Subfigure captions.
\usepackage[font={scriptsize},
            hypcap=true,
            labelsep=colon]{caption}    % Figure captions.
\usepackage{hyperref}                   % Allows for hyperlinks.
\hypersetup{
    colorlinks=true,
    linkcolor=blue,
    filecolor=magenta,
    urlcolor=Cerulean,
    citecolor=SkyBlue
}                           % Colors for hyperref.
\usepackage[toc,acronym,nogroupskip]{glossaries} % Glossaries and acronyms.
\usepackage[subpreambles=false]{standalone}      % Complileable sub files.

% Various font stuff from kiwi.
% Use this for Times text and Computer Modern math
%\usepackage{times}

% Quite nice
%\usepackage[charter, greekfamily=, greekuppercase=italicized]{mathdesign}
%\usepackage[utopia, greekuppercase=italicized]{mathdesign}    % Math is narrower

% Use this for Times text and math
%\usepackage{newtxtext}
%\usepackage[libertine,cmintegrals]{newtxmath}
%\usepackage{fix-cm}

%\usepackage{txfontsb}
% or
%\usepackage{mathptmx}

%\usepackage[scaled=0.92]{helvet}
%\renewcommand{\rmdefault}{ptm}

%\usepackage{mathpazo}    % add possibly `sc` and `osf` options
%\usepackage{eulervm}

%\usepackage{fourier}
%\renewcommand{\rmdefault}{ptm}
%\usepackage{mathptm}

%\usepackage{fontspec}
%\setmainfont{lmodern}

%\usepackage[varg]{txfonts}
%\usepackage{fouriernc}
%\usepackage{mathpazo}

%\usepackage{bookman}
%\usepackage[scaled]{uarial}
%\usepackage[scaled]{helvet}
%\renewcommand*\familydefault{\sfdefault}
%\usepackage[math]{anttor}

%\newcommand\fgeorgia{\fontfamily{jvn}\selectfont}
%\newcommand\ftimes{\fontfamily{ptm}\selectfont}
%\newcommand\fhelvetica{\fontfamily{phv}\selectfont}
%\newcommand\fcourier{\fontfamily{pcr}\selectfont}
%\newcommand\fbookman{\fontfamily{pbk}\selectfont}
%\newcommand\fnewcentury{\fontfamily{pnc}\selectfont}
%\newcommand\fpalatino{\fontfamily{ppl}\selectfont}
%\newcommand\favantgarde{\fontfamily{pag}\selectfont}
%\newcommand\fnormal{\normalfont}
%\newcommand\fsize[1]{\ifnum#1>0\fontsize{#1}{#1}\selectfont\else\normalsize\fi}
%------------------------Theorem Styles-------------------------%
% Define theorem style for default spacing and normal font.
\newtheoremstyle{normal}
    {\topsep}               % Amount of space above the theorem.
    {\topsep}               % Amount of space below the theorem.
    {}                      % Font used for body of theorem.
    {}                      % Measure of space to indent.
    {\bfseries}             % Font of the header of the theorem.
    {}                      % Punctuation between head and body.
    {.5em}                  % Space after theorem head.
    {}

% Define theorem style for default spacing with italicized font.
\newtheoremstyle{normalit}{\topsep}{\topsep}
                {\itshape}{}{\bfseries}{}{.5em}{}

% Italic header environment.
\newtheoremstyle{thmit}{\topsep}{\topsep}{}{}{\itshape}{}{0.5em}{}

% Define italicized environments.
\theoremstyle{normalit}
\newtheorem{theorem}{Theorem}[section]
\newtheorem{lemma}{Lemma}[section]
\newtheorem{corollary}{Corollary}[section]
\newtheorem{proposition}{Proposition}[section]
\newtheorem*{theorem*}{Theorem}

% Define environments with italic headers.
\theoremstyle{thmit}
\newtheorem*{solution}{Solution}
\newtheorem*{fsolution}{Solution}

% Define default environments.
\theoremstyle{normal}
\newtheorem{example}{Example}[section]
\newtheorem{definition}{Definition}[section]
\newtheorem{problem}{Problem}[section]
\newtheorem{question}{Question}[section]
\newtheorem{remark}{Remark}[section]
\newtheorem{properties}{Properties}[section]
\newtheorem{notation}{Notation}[section]
\newtheorem{axiom}{Axiom}[section]
\newtheorem*{properties*}{Properties}
\newtheorem*{remark*}{Remark}
\newtheorem*{definition*}{Definition}
\theoremstyle{plain}

% Define framed environment.
\tcbuselibrary{most}
\newtcbtheorem[use counter*=theorem]{ftheorem}{Theorem}%
    {colback=green!5,colframe=green!35!black,
     fonttitle=\bfseries\upshape}{th}

\newtcbtheorem[use counter*=example]{fdefinition}{Definition}%
    {fonttitle=\bfseries\upshape,
     colback=blue!5!white,colframe=blue!75!black}{def}

\newtcbtheorem[use counter*=example]{fexample}{Example}%
    {fonttitle=\bfseries\upshape,
     colback=red!5!white,colframe=red!75!black}{ex}

\newtcbtheorem[use counter*=notation]{fnotation}{Notation}%
    {fonttitle=\bfseries\upshape,
     colback=SeaGreen!5!white,colframe=SeaGreen!75!black}{ex}

\newtcbtheorem[use counter*=corollary]{fcorollary}{Corollary}%
    {fonttitle=\bfseries\upshape,
     colback=Orchid!5!white,colframe=Orchid!75!black}{ex}

\newenvironment{bproof}{\textit{Proof.}}{\hfill$\square$}
\tcolorboxenvironment{bproof}{blanker,breakable,left=5mm,
                             before skip=10pt,after skip=10pt,
                             borderline west={1mm}{0pt}{red}}
\tcolorboxenvironment{fsolution}
    {enhanced jigsaw,colframe=cyan,interior hidden,breakable}

%--------------------Declared Math Operators--------------------%
\DeclareMathOperator{\Refl}{Refl}           % Reflection operator.
\DeclareMathOperator{\Span}{Span}           % Span of a set of vectors.
\DeclareMathOperator{\Card}{Card}           % Cardinality of set.
\DeclareMathOperator{\Ord}{Ord}             % Ordinal of ordered set.
\DeclareMathOperator{\Tr}{Tr}               % Trace of matrix.
\DeclareMathOperator{\adjoint}{adj}         % Adjoint of matrix.
\DeclareMathOperator{\rk}{rk}               % Rank of operator.
\DeclareMathOperator{\nul}{nul}             % Null space of operator.
\DeclareMathOperator{\sgn}{sgn}             % Sign of a number.
\DeclareMathOperator{\multideg}{mutlideg}   % Multi-Degree (Graphs).
\DeclareMathOperator{\GCD}{GCD}             % Greatest common denominator.
\DeclareMathOperator{\LM}{LM}               % Leading monomial
\DeclareMathOperator{\LC}{LC}               % Leading coefficient.
\DeclareMathOperator{\LT}{LT}               % Leading term.
\DeclareMathOperator{\LCM}{LCM}             % Least common multiple.
\DeclareMathOperator{\Mon}{Mon}             % Monomial.
\DeclareMathOperator{\Spec}{Spec}           % Spectrum.
\DeclareMathOperator{\proj}{proj}           % Projection.
\DeclareMathOperator{\comp}{comp}           % Component.
\DeclareMathOperator{\sinc}{sinc}           % Sinc function.
\DeclareMathOperator{\Ima}{Im}              % Image of operator.
\DeclareMathOperator{\Prin}{Prin}           % Principal value.
\DeclareMathOperator{\Mod}{mod}             % Modulus.
%------------------------New Commands---------------------------%
\DeclarePairedDelimiter\norm{\lVert}{\rVert}
\DeclarePairedDelimiter\ceil{\lceil}{\rceil}
\DeclarePairedDelimiter\floor{\lfloor}{\rfloor}
\newcommand*\diff{\mathop{}\!\mathrm{d}}
\newcommand*\Diff[1]{\mathop{}\!\mathrm{d^#1}}
\renewcommand{\mod}{\ \Mod}
\renewcommand*{\glstextformat}[1]{\textcolor{RoyalBlue}{#1}}
\renewcommand{\glsnamefont}[1]{\textbf{#1}}
\renewcommand\labelitemii{$\circ$}
\renewcommand\thesubfigure{\arabic{chapter}.\arabic{figure}}
\renewcommand\thesubfigure{%
    \arabic{chapter}.\arabic{figure}.\arabic{subfigure}}
\addto\captionsenglish{\renewcommand{\figurename}{Fig.}}
%------------------------Book Command---------------------------%
\makeatletter
\renewcommand\@pnumwidth{1cm}
\newcounter{book}
\renewcommand\thebook{\@Roman\c@book}
\newcommand\book{%
    \if@openright
        \cleardoublepage
    \else
        \clearpage
    \fi
    \thispagestyle{plain}%
    \if@twocolumn
        \onecolumn
        \@tempswatrue
    \else
        \@tempswafalse
    \fi
    \null\vfil
    \secdef\@book\@sbook
}
\def\@book[#1]#2{%
    \ifnum \c@secnumdepth >-3\relax
        \refstepcounter{book}%
        \addcontentsline{toc}{book}{
            \bookname\ \thebook:\hspace{1em}#1
        }
    \else
        \addcontentsline{toc}{book}{#1}%
    \fi
    \markboth{}{}%
    {\centering
     \interlinepenalty \@M
     \normalfont
     \ifnum \c@secnumdepth >-2\relax
       \huge\bfseries \bookname\nobreakspace\thebook
       \par
       \vskip 20\p@
     \fi
     \Huge \bfseries #2\par}%
    \@endbook}
\def\@sbook#1{%
    {\centering
     \interlinepenalty \@M
     \normalfont
     \Huge \bfseries #1\par}%
    \@endbook}
\def\@endbook{
    \vfil\newpage
        \if@twoside
            \if@openright
                \null
                \thispagestyle{empty}%
                \newpage
            \fi
        \fi
        \if@tempswa
            \twocolumn
        \fi
}
\newcommand*\l@book[2]{%
    \ifnum \c@tocdepth >-2\relax
        \addpenalty{-\@highpenalty}%
        \addvspace{2.25em \@plus\p@}%
        \setlength\@tempdima{3em}%
        \begingroup
            \parindent \z@ \rightskip \@pnumwidth
            \parfillskip -\@pnumwidth
            {
                \leavevmode
                \Large \bfseries #1\hfil \hb@xt@\@pnumwidth{
                    \hss #2
                }
            }
            \par
            \nobreak
            \global\@nobreaktrue
            \everypar{\global\@nobreakfalse\everypar{}}%
        \endgroup
    \fi}
\newcommand\bookname{Book}
\renewcommand{\thebook}{\texorpdfstring{\Numberstring{book}}{book}}
\providecommand*{\toclevel@book}{-2}
\makeatother
\titlecontents{chapter}[0pt]
    {\bfseries}
    {\chaptername\ \thecontentslabel:\quad}
    {}
    {\hfill\contentspage}
\titleformat{\part}[display]
    {\Large\bfseries}
    {\partname\nobreakspace\thepart}
    {0mm}
    {\Huge\bfseries}
    \titlecontents{part}[0pt]
    {\large\bfseries}
    {\partname\ \thecontentslabel: \quad}
    {}
    {\hfill\contentspage}
\newcommand{\MarkRightAngle}[4][.3cm]
    {\coordinate (tempa) at ($(#3)!#1!(#2)$);
     \coordinate (tempb) at ($(#3)!#1!(#4)$);
     \coordinate (tempc) at ($(tempa)!0.5!(tempb)$);%midpoint
     \draw (tempa) -- ($(#3)!2!(tempc)$) -- (tempb);}
%--------------------------LENGTHS------------------------------%
% Spacings for the Table of Contents.
\addtolength{\cftsecnumwidth}{1ex}
\addtolength{\cftsubsecindent}{1ex}
\addtolength{\cftsubsecnumwidth}{1ex}
\addtolength{\cftfignumwidth}{1ex}
\addtolength{\cfttabnumwidth}{1ex}

% Spacing for multi-column and enumerate environments.
\setlength{\multicolsep}{6pt}
\setlist[enumerate]{itemsep=0pt,topsep=3pt}

% Indent and paragraph spacing.
\setlength{\parindent}{0em}
\setlength{\parskip}{0em}                                                           %
%----------------------------Main Document-------------------------------------%
\begin{document}
    \title{Topics in Algebra}
    \author{Ryan Maguire}
    \date{\vspace{-5ex}}
    \maketitle
    \section{Stuff}
        If $\langle\cdot|\cdot\rangle$ is a symmetric bilinear form, it is
        represented by its Gram matrix relative to a bsis $\mathscr{B}$ of the
        finite dimensional vector space $\mathscr{B}$. We have:
        \begin{equation}
            \langle{v}|w\rangle
            =[V]_{\mathscr{B}}^{T}G_{\mathscr{B}}[W]\mathscr{B}
        \end{equation}
        Let $\langle\cdot|\cdot\rangle:V\times{V}\rightarrow{k}$ be bilinear,
        let $\mathscr{B}$ and $\mathscr{C}$ be bases of $V$, and let
        $G_{\mathscr{B}}$ and $G_{\mathscr{C}}$ be the corresponding Gram
        matrices. Let $x,y\in{V}$. Then:
        \begin{subequations}
            \begin{align}
                [x]_{\mathscr{B}}^{T}G_{\mathscr{B}}[y]_{\mathscr{B}}
                &=\langle{x}|y\rangle\\
                &=[x]_{\mathscr{C}}^{T}G_{\mathscr{C}}[Y]_{\mathscr{C}}\\
                &=[\textrm{Id}(x)]_{\mathscr{C}}^{T}G_{\mathscr{C}}
                    [\textrm{Id}(y)]_{\mathscr{C}}\\
                &=\Big([\textrm{Id}]_{\mathscr{C}}^{\mathscr{B}}
                    [x]_{\mathscr{B}}\Big)^{T}G_{\mathscr{C}}
                    \Big([\textrm{Id}_{\mathscr{C}}^{\mathscr{B}}]
                        [y]_{\mathscr{B}}\Big)
            \end{align}
        \end{subequations}
        From this, we obtain the formula for the Gram matrix:
        \begin{equation}
            G_{\mathscr{B}}=P^{T}G_{\mathscr{C}}P
        \end{equation}
        \begin{fdefinition}{Congruent Matrices}{Congruent_Matrices}
            Congruent matrices are matrices $A,B\in{M}_{n}(k)$ such that there
            is an invertible matrice $P\in{GL}_{n}(k)$ such that:
            \begin{equation}
                B=P^{T}AP
            \end{equation}
        \end{fdefinition}
    \section{Diagonalizing Real Quadratic Forms}
        \begin{ltheorem}{Sylvester's Theorem}{Sylvesters_Theorem}
            If $V$ is a finite dimensional vector space over $\mathbb{R}$ and if
            $\langle\cdot|\cdot\rangle$ is a symmetric bilinear form, then there
            exists a basis $\mathscr{B}$ of $V$ such that:
            \begin{equation}
                G_{\mathscr{B}}=
                \begin{bmatrix*}[r]
                    1&\dots&0&0&\dots&0&0&\dots&0\\
                    \vdots&\ddots&\vdots&\vdots&\ddots
                        &\vdots&\vdots&\ddots&\vdots\\
                    0&\dots&1&0&\dots&0&0&\dots&0\\
                    0&\dots&0&\minus{1}&\dots&0&0&\dots&0\\
                    \vdots&\ddots&\vdots&\vdots
                        &\ddots&\vdots&\vdots&\ddots&\vdots\\
                    0&\dots&0&0&\dots&\minus{1}&0&\dots&0\\
                    0&\dots&0&0&\dots&0&0&\dots&0\\
                    \vdots&\ddots&\vdots&\vdots&\ddots
                        &\vdots&\vdots&\ddots&\vdots\\
                    0&\dots&0&0&\dots&0&0&\dots&0\\
                \end{bmatrix*}
            \end{equation}
            Where there are $r$ 1's, $s$ negative 1's, and $t$ zeros, and
            $r$, $s$, and $t$ are uniquely determined.
        \end{ltheorem}
    \section{Orthogonal Transformations}
        Let $(V,\langle\cdot|\cdot\rangle)$ be a nondegenerate bilinear space.
        Then a linear map $T:V\rightarrow{V}$ is orthogonal if or a linear
        isometry if:
        \begin{equation}
            \langle{T}(v)|T(w)\rangle=\langle{v}|w\rangle
        \end{equation}
        Let $\mathscr{B}$ be a basis of $V$ and let $G_{\mathscr{B}}$ be the
        Gram matrix of $\langle\cdot|\cdot\rangle$. Let $A$ be the representing
        matrix of $T$ over the basis $\mathscr{B}$. Then:
        \begin{equation}
            \langle{v}|w\rangle
            =\langle{T}(v)|T(w)\rangle
            =[T(v)]_{\mathscr{B}}^{T}G_{\mathscr{B}}[T(w)]_{\mathscr{B}}
            =[v]_{\mathscr{B}}^{T}A^{T}G_{\mathscr{B}}A[w]_{\mathscr{B}}
        \end{equation}
        From this we can compute what the Gram matrix is:
        \begin{equation}
            G_{\mathscr{B}}=A^{T}G_{\mathscr{B}}A
        \end{equation}
        If $A$ represents an orthogonal transformation, then $A$ must satisfy
        this equation. In the special case of when $\mathscr{B}$ is orthonormal,
        then $G_{\mathscr{B}}$ is simply the identity matrix and thus we have
        that $A^{T}A=I$, or $A^{T}=A^{\minus{1}}$.
        \par\hfill\par
        A nondegenerate skew symmetric bilinear form on a $2n$ dimensional real
        vector space is called a symplectic form. The Gram matrix is:
        \begin{equation}
            J=
            \begin{bmatrix*}[r]
                0&I_{n}\\
                \minus{I}_{n}&0
            \end{bmatrix*}
        \end{equation}
        A transformation $A$ such that $A^{T}JA=J$ is called a symplectic or
        a canonical transformation.
        \par\hfill\par
        If $\langle\cdot|\cdot\rangle$ is the Lorentz form on $\mathbb{R}^{4}$,
        then an orthogonal transformation is called a Lorentz transformation.
    \section{Sesquilinear Geometry}
        Let $V$ and $W$ be $\mathbb{R}$ inner product spaces. That is, $V$ and
        $W$ are equipped with a symmetric bilinear form
        $\langle\cdot|\cdot\rangle_{V}$ and $\langle\cdot|\cdot\rangle_{W}$ that
        are positive-definite:
        \begin{equation}
            \langle{v}|v\rangle\geq{0}
        \end{equation}
        With equality if and only if $v=0$. Note that positive definite implies
        nondegenerate since $\langle{v}|v\rangle>0$ for nonzero $v$. Let
        $T:V\rightarrow{W}$ be a linear map. Then $T$ induces
        $T^{*}:W^{*}\rightarrow{V}^{*}$ by something.
        Let $V$ be a finite dimension $\mathbb{R}$ vector space. An inner
        product on $V$ is a bilinear form that is symmetric and
        positive-definite. Euclidean geometry is derived from the inner product.
        This induces a norm and the notion of angle:
        \begin{equation}
            \norm{v}=\sqrt{\langle{v}|v\rangle}
            \quad\quad
            \theta=\cos^{\minus{1}}\Big(
                \frac{\langle{v}|w\rangle}{\norm{v}\norm{w}}\Big)
        \end{equation}
        Cauchy-Schwartz comes out of this. Let $V$ be a vector space over
        $\mathbb{C}$, say $V=\mathbb{C}^{n}$. If we define:
        \begin{equation}
            \langle{v}|w\rangle=\sum_{k=1}^{n}v_{k}w_{k}
        \end{equation}
        We lose positive-definiteness since $(1,2i)\cdot(1,2i)=\minus{3}$,
        in $\mathbb{C}^{2}$, for example. We define the dot product on
        $\mathbb{C}^{n}$ by:
        \begin{equation}
            \langle{z}|w\rangle=\sum_{k=1}^{n}\overline{z}_{k}\overline{w}_{k}
        \end{equation}
        That is, $\langle{v}|w\rangle=\overline{v}\cdot{w}$. From this we have
        that $\langle{z}|\cdot\rangle$ is linear on $\mathbb{C}$. However,
        looking at $\langle\cdot|w\rangle$, we have that it is $\mathbb{R}$
        linear but only conjugate linear over $\mathbb{C}$. This gives rise to
        the notion of a sesquilinear product.
        \begin{fdefinition}{Sesquilinear Product}{Sesquilinear_Product}
            A sesquilinear product on a vector space $V$ over $\mathbb{C}$ is a
            function $\langle\cdot|\cdot\rangle:V\times{V}\rightarrow\mathbb{C}$
            such that $\langle{z}|\cdot\rangle$ is $\mathbb{C}$ linear and
            $\langle\cdot|w\rangle$ is $\mathbb{R}$ linear and $\mathbb{C}$
            conjugate linear.
        \end{fdefinition}
        With this we can define a Hermitian form on a $\mathbb{C}$ vector space
        $V$. Note that a sesquilinear form is conjugate symmetric. That is,
        $\langle{w}|z\rangle=\overline{\langle{z}|w\rangle}$.
        \begin{fdefinition}{Hermitian Form}{Hermitian_Form}
            A Hermitian form on a $\mathbb{C}$ vector space $V$ is a function
            $\langle\cdot|\cdot\rangle:V\times{V}\rightarrow\mathbb{C}$ that is
            sesquilinear and conjugate symmetric.
        \end{fdefinition}
        \begin{fdefinition}{Hermitian Inner Product Space}
                           {Hermitian_Inner_Product_Space}
            A Hermitian inner product space is a vector space $V$ over
            $\mathbb{C}$ with a Hermitian inner product.
        \end{fdefinition}
        Let $\mathcal{B}$ be a basis of a Hermitian inner product space $V$ and
        let $G_{\mathcal{B}}=[g_{ij}]_{ij}$ where
        $g_{ij}=\langle{e}_{i}|e_{j}\rangle$. Then for $v,w\in{V}$, we have:
        \begin{equation}
            v=\sum_{k=1}^{n}a_{k}e_{k}
            \quad\quad
            w=\sum_{k=1}^{n}b_{k}e_{k}
        \end{equation}
        And moreover:
        \begin{equation}
            \langle{v}|w\rangle=
            \langle\sum_{j=1}^{n}a_{j}e_{j}|\sum_{k=1}^{n}b_{k}e_{k}\rangle
            =\sum_{k=1}^{n}\sum_{j=1}^{n}\overline{a}_{j}b_{k}
                \langle{e_{j}}|e_{k}\rangle
            =\sum_{j=1}^{n}\sum_{k=1}^{n}\overline{a}_{j}g_{jk}b_{k}
        \end{equation}
        So we have:
        \begin{equation}
            \langle{v}|w\rangle
            =\overline{[v]_{\mathcal{B}}^{T}}G_{\mathcal{B}}[w]_{\mathcal{B}}
        \end{equation}
        This gives rise to the definition of a Hermitian transpose.
        \begin{fdefinition}{Hermitian Transpose}{Hermitian_Transpose}
            The Hermitian transpose of a matrix $A$ is the matrix:
            \begin{equation*}
                A^{H}=\overline{A^{T}}
            \end{equation*}
        \end{fdefinition}
    \section{Unitary Transformations}
        \begin{fdefinition}{Unitary Transformations}{Unitary_Transformations}
            A unitary transformation on a vector space $V$ over $\mathbb{C}$ is
            a linear function $T:V\rightarrow{V}$ such that:
            \begin{equation*}
                \langle{T}(v)|T(w)\rangle=\langle{v|w}\rangle
            \end{equation*}
        \end{fdefinition}
        Let $V$ and $W$ be Hermitian inner product spaces and let
        $T:V\rightarrow{W}$ be $\mathbb{C}$ linear. $T$ induces
        $T^{*}:W^{*}\rightarrow{V}^{*}$ by $T^{*}(\psi)=\psi\circ{T}$.
        \begin{fdefinition}{Self-Adjoint Hermitian Operator}
                           {Self-Adjoint_Hermitian_Operator}
            A self-adjoint operator on a Hermitian inner product space $V$ is a
            linear function $T:\mathbb{C}\rightarrow\mathbb{C}$ such that
            $T=T^{H}$.
        \end{fdefinition}
        \begin{fdefinition}{Normal Hermitian Operator}
                           {Normal_Hermitian_Operator}
            A function $T:\mathbb{C}\rightarrow\mathbb{C}$ such that
            $TT^{H}=T^{H}T$
        \end{fdefinition}
\end{document} 