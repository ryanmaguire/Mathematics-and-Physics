%------------------------------------------------------------------------------%
\documentclass[crop=false,class=book,oneside]{standalone}                      %
%----------------------------Preamble------------------------------------------%
%---------------------------Packages----------------------------%
\usepackage{geometry}
\geometry{b5paper, margin=1.0in}
\usepackage[T1]{fontenc}
\usepackage{graphicx, float}            % Graphics/Images.
\usepackage{natbib}                     % For bibliographies.
\bibliographystyle{agsm}                % Bibliography style.
\usepackage[french, english]{babel}     % Language typesetting.
\usepackage[dvipsnames]{xcolor}         % Color names.
\usepackage{listings}                   % Verbatim-Like Tools.
\usepackage{mathtools, esint, mathrsfs} % amsmath and integrals.
\usepackage{amsthm, amsfonts, amssymb}  % Fonts and theorems.
\usepackage{tcolorbox}                  % Frames around theorems.
\usepackage{upgreek}                    % Non-Italic Greek.
\usepackage{fmtcount, etoolbox}         % For the \book{} command.
\usepackage[newparttoc]{titlesec}       % Formatting chapter, etc.
\usepackage{titletoc}                   % Allows \book in toc.
\usepackage[nottoc]{tocbibind}          % Bibliography in toc.
\usepackage[titles]{tocloft}            % ToC formatting.
\usepackage{pgfplots, tikz}             % Drawing/graphing tools.
\usepackage{imakeidx}                   % Used for index.
\usetikzlibrary{
    calc,                   % Calculating right angles and more.
    angles,                 % Drawing angles within triangles.
    arrows.meta,            % Latex and Stealth arrows.
    quotes,                 % Adding labels to angles.
    positioning,            % Relative positioning of nodes.
    decorations.markings,   % Adding arrows in the middle of a line.
    patterns,
    arrows
}                                       % Libraries for tikz.
\pgfplotsset{compat=1.9}                % Version of pgfplots.
\usepackage[font=scriptsize,
            labelformat=simple,
            labelsep=colon]{subcaption} % Subfigure captions.
\usepackage[font={scriptsize},
            hypcap=true,
            labelsep=colon]{caption}    % Figure captions.
\usepackage[pdftex,
            pdfauthor={Ryan Maguire},
            pdftitle={Mathematics and Physics},
            pdfsubject={Mathematics, Physics, Science},
            pdfkeywords={Mathematics, Physics, Computer Science, Biology},
            pdfproducer={LaTeX},
            pdfcreator={pdflatex}]{hyperref}
\hypersetup{
    colorlinks=true,
    linkcolor=blue,
    filecolor=magenta,
    urlcolor=Cerulean,
    citecolor=SkyBlue
}                           % Colors for hyperref.
\usepackage[toc,acronym,nogroupskip,nopostdot]{glossaries}
\usepackage{glossary-mcols}
%------------------------Theorem Styles-------------------------%
\theoremstyle{plain}
\newtheorem{theorem}{Theorem}[section]

% Define theorem style for default spacing and normal font.
\newtheoremstyle{normal}
    {\topsep}               % Amount of space above the theorem.
    {\topsep}               % Amount of space below the theorem.
    {}                      % Font used for body of theorem.
    {}                      % Measure of space to indent.
    {\bfseries}             % Font of the header of the theorem.
    {}                      % Punctuation between head and body.
    {.5em}                  % Space after theorem head.
    {}

% Italic header environment.
\newtheoremstyle{thmit}{\topsep}{\topsep}{}{}{\itshape}{}{0.5em}{}

% Define environments with italic headers.
\theoremstyle{thmit}
\newtheorem*{solution}{Solution}

% Define default environments.
\theoremstyle{normal}
\newtheorem{example}{Example}[section]
\newtheorem{definition}{Definition}[section]
\newtheorem{problem}{Problem}[section]

% Define framed environment.
\tcbuselibrary{most}
\newtcbtheorem[use counter*=theorem]{ftheorem}{Theorem}{%
    before=\par\vspace{2ex},
    boxsep=0.5\topsep,
    after=\par\vspace{2ex},
    colback=green!5,
    colframe=green!35!black,
    fonttitle=\bfseries\upshape%
}{thm}

\newtcbtheorem[auto counter, number within=section]{faxiom}{Axiom}{%
    before=\par\vspace{2ex},
    boxsep=0.5\topsep,
    after=\par\vspace{2ex},
    colback=Apricot!5,
    colframe=Apricot!35!black,
    fonttitle=\bfseries\upshape%
}{ax}

\newtcbtheorem[use counter*=definition]{fdefinition}{Definition}{%
    before=\par\vspace{2ex},
    boxsep=0.5\topsep,
    after=\par\vspace{2ex},
    colback=blue!5!white,
    colframe=blue!75!black,
    fonttitle=\bfseries\upshape%
}{def}

\newtcbtheorem[use counter*=example]{fexample}{Example}{%
    before=\par\vspace{2ex},
    boxsep=0.5\topsep,
    after=\par\vspace{2ex},
    colback=red!5!white,
    colframe=red!75!black,
    fonttitle=\bfseries\upshape%
}{ex}

\newtcbtheorem[auto counter, number within=section]{fnotation}{Notation}{%
    before=\par\vspace{2ex},
    boxsep=0.5\topsep,
    after=\par\vspace{2ex},
    colback=SeaGreen!5!white,
    colframe=SeaGreen!75!black,
    fonttitle=\bfseries\upshape%
}{not}

\newtcbtheorem[use counter*=remark]{fremark}{Remark}{%
    fonttitle=\bfseries\upshape,
    colback=Goldenrod!5!white,
    colframe=Goldenrod!75!black}{ex}

\newenvironment{bproof}{\textit{Proof.}}{\hfill$\square$}
\tcolorboxenvironment{bproof}{%
    blanker,
    breakable,
    left=3mm,
    before skip=5pt,
    after skip=10pt,
    borderline west={0.6mm}{0pt}{green!80!black}
}

\AtEndEnvironment{lexample}{$\hfill\textcolor{red}{\blacksquare}$}
\newtcbtheorem[use counter*=example]{lexample}{Example}{%
    empty,
    title={Example~\theexample},
    boxed title style={%
        empty,
        size=minimal,
        toprule=2pt,
        top=0.5\topsep,
    },
    coltitle=red,
    fonttitle=\bfseries,
    parbox=false,
    boxsep=0pt,
    before=\par\vspace{2ex},
    left=0pt,
    right=0pt,
    top=3ex,
    bottom=1ex,
    before=\par\vspace{2ex},
    after=\par\vspace{2ex},
    breakable,
    pad at break*=0mm,
    vfill before first,
    overlay unbroken={%
        \draw[red, line width=2pt]
            ([yshift=-1.2ex]title.south-|frame.west) to
            ([yshift=-1.2ex]title.south-|frame.east);
        },
    overlay first={%
        \draw[red, line width=2pt]
            ([yshift=-1.2ex]title.south-|frame.west) to
            ([yshift=-1.2ex]title.south-|frame.east);
    },
}{ex}

\AtEndEnvironment{ldefinition}{$\hfill\textcolor{Blue}{\blacksquare}$}
\newtcbtheorem[use counter*=definition]{ldefinition}{Definition}{%
    empty,
    title={Definition~\thedefinition:~{#1}},
    boxed title style={%
        empty,
        size=minimal,
        toprule=2pt,
        top=0.5\topsep,
    },
    coltitle=Blue,
    fonttitle=\bfseries,
    parbox=false,
    boxsep=0pt,
    before=\par\vspace{2ex},
    left=0pt,
    right=0pt,
    top=3ex,
    bottom=0pt,
    before=\par\vspace{2ex},
    after=\par\vspace{1ex},
    breakable,
    pad at break*=0mm,
    vfill before first,
    overlay unbroken={%
        \draw[Blue, line width=2pt]
            ([yshift=-1.2ex]title.south-|frame.west) to
            ([yshift=-1.2ex]title.south-|frame.east);
        },
    overlay first={%
        \draw[Blue, line width=2pt]
            ([yshift=-1.2ex]title.south-|frame.west) to
            ([yshift=-1.2ex]title.south-|frame.east);
    },
}{def}

\AtEndEnvironment{ltheorem}{$\hfill\textcolor{Green}{\blacksquare}$}
\newtcbtheorem[use counter*=theorem]{ltheorem}{Theorem}{%
    empty,
    title={Theorem~\thetheorem:~{#1}},
    boxed title style={%
        empty,
        size=minimal,
        toprule=2pt,
        top=0.5\topsep,
    },
    coltitle=Green,
    fonttitle=\bfseries,
    parbox=false,
    boxsep=0pt,
    before=\par\vspace{2ex},
    left=0pt,
    right=0pt,
    top=3ex,
    bottom=-1.5ex,
    breakable,
    pad at break*=0mm,
    vfill before first,
    overlay unbroken={%
        \draw[Green, line width=2pt]
            ([yshift=-1.2ex]title.south-|frame.west) to
            ([yshift=-1.2ex]title.south-|frame.east);},
    overlay first={%
        \draw[Green, line width=2pt]
            ([yshift=-1.2ex]title.south-|frame.west) to
            ([yshift=-1.2ex]title.south-|frame.east);
    }
}{thm}

%--------------------Declared Math Operators--------------------%
\DeclareMathOperator{\adjoint}{adj}         % Adjoint.
\DeclareMathOperator{\Card}{Card}           % Cardinality.
\DeclareMathOperator{\curl}{curl}           % Curl.
\DeclareMathOperator{\diam}{diam}           % Diameter.
\DeclareMathOperator{\dist}{dist}           % Distance.
\DeclareMathOperator{\Div}{div}             % Divergence.
\DeclareMathOperator{\Erf}{Erf}             % Error Function.
\DeclareMathOperator{\Erfc}{Erfc}           % Complementary Error Function.
\DeclareMathOperator{\Ext}{Ext}             % Exterior.
\DeclareMathOperator{\GCD}{GCD}             % Greatest common denominator.
\DeclareMathOperator{\grad}{grad}           % Gradient
\DeclareMathOperator{\Ima}{Im}              % Image.
\DeclareMathOperator{\Int}{Int}             % Interior.
\DeclareMathOperator{\LC}{LC}               % Leading coefficient.
\DeclareMathOperator{\LCM}{LCM}             % Least common multiple.
\DeclareMathOperator{\LM}{LM}               % Leading monomial.
\DeclareMathOperator{\LT}{LT}               % Leading term.
\DeclareMathOperator{\Mod}{mod}             % Modulus.
\DeclareMathOperator{\Mon}{Mon}             % Monomial.
\DeclareMathOperator{\multideg}{mutlideg}   % Multi-Degree (Graphs).
\DeclareMathOperator{\nul}{nul}             % Null space of operator.
\DeclareMathOperator{\Ord}{Ord}             % Ordinal of ordered set.
\DeclareMathOperator{\Prin}{Prin}           % Principal value.
\DeclareMathOperator{\proj}{proj}           % Projection.
\DeclareMathOperator{\Refl}{Refl}           % Reflection operator.
\DeclareMathOperator{\rk}{rk}               % Rank of operator.
\DeclareMathOperator{\sgn}{sgn}             % Sign of a number.
\DeclareMathOperator{\sinc}{sinc}           % Sinc function.
\DeclareMathOperator{\Span}{Span}           % Span of a set.
\DeclareMathOperator{\Spec}{Spec}           % Spectrum.
\DeclareMathOperator{\supp}{supp}           % Support
\DeclareMathOperator{\Tr}{Tr}               % Trace of matrix.
%--------------------Declared Math Symbols--------------------%
\DeclareMathSymbol{\minus}{\mathbin}{AMSa}{"39} % Unary minus sign.
%------------------------New Commands---------------------------%
\DeclarePairedDelimiter\norm{\lVert}{\rVert}
\DeclarePairedDelimiter\ceil{\lceil}{\rceil}
\DeclarePairedDelimiter\floor{\lfloor}{\rfloor}
\newcommand*\diff{\mathop{}\!\mathrm{d}}
\newcommand*\Diff[1]{\mathop{}\!\mathrm{d^#1}}
\renewcommand*{\glstextformat}[1]{\textcolor{RoyalBlue}{#1}}
\renewcommand{\glsnamefont}[1]{\textbf{#1}}
\renewcommand\labelitemii{$\circ$}
\renewcommand\thesubfigure{%
    \arabic{chapter}.\arabic{figure}.\arabic{subfigure}}
\addto\captionsenglish{\renewcommand{\figurename}{Fig.}}
\numberwithin{equation}{section}

\renewcommand{\vector}[1]{\boldsymbol{\mathrm{#1}}}

\newcommand{\uvector}[1]{\boldsymbol{\hat{\mathrm{#1}}}}
\newcommand{\topspace}[2][]{(#2,\tau_{#1})}
\newcommand{\measurespace}[2][]{(#2,\varSigma_{#1},\mu_{#1})}
\newcommand{\measurablespace}[2][]{(#2,\varSigma_{#1})}
\newcommand{\manifold}[2][]{(#2,\tau_{#1},\mathcal{A}_{#1})}
\newcommand{\tanspace}[2]{T_{#1}{#2}}
\newcommand{\cotanspace}[2]{T_{#1}^{*}{#2}}
\newcommand{\Ckspace}[3][\mathbb{R}]{C^{#2}(#3,#1)}
\newcommand{\funcspace}[2][\mathbb{R}]{\mathcal{F}(#2,#1)}
\newcommand{\smoothvecf}[1]{\mathfrak{X}(#1)}
\newcommand{\smoothonef}[1]{\mathfrak{X}^{*}(#1)}
\newcommand{\bracket}[2]{[#1,#2]}

%------------------------Book Command---------------------------%
\makeatletter
\renewcommand\@pnumwidth{1cm}
\newcounter{book}
\renewcommand\thebook{\@Roman\c@book}
\newcommand\book{%
    \if@openright
        \cleardoublepage
    \else
        \clearpage
    \fi
    \thispagestyle{plain}%
    \if@twocolumn
        \onecolumn
        \@tempswatrue
    \else
        \@tempswafalse
    \fi
    \null\vfil
    \secdef\@book\@sbook
}
\def\@book[#1]#2{%
    \refstepcounter{book}
    \addcontentsline{toc}{book}{\bookname\ \thebook:\hspace{1em}#1}
    \markboth{}{}
    {\centering
     \interlinepenalty\@M
     \normalfont
     \huge\bfseries\bookname\nobreakspace\thebook
     \par
     \vskip 20\p@
     \Huge\bfseries#2\par}%
    \@endbook}
\def\@sbook#1{%
    {\centering
     \interlinepenalty \@M
     \normalfont
     \Huge\bfseries#1\par}%
    \@endbook}
\def\@endbook{
    \vfil\newpage
        \if@twoside
            \if@openright
                \null
                \thispagestyle{empty}%
                \newpage
            \fi
        \fi
        \if@tempswa
            \twocolumn
        \fi
}
\newcommand*\l@book[2]{%
    \ifnum\c@tocdepth >-3\relax
        \addpenalty{-\@highpenalty}%
        \addvspace{2.25em\@plus\p@}%
        \setlength\@tempdima{3em}%
        \begingroup
            \parindent\z@\rightskip\@pnumwidth
            \parfillskip -\@pnumwidth
            {
                \leavevmode
                \Large\bfseries#1\hfill\hb@xt@\@pnumwidth{\hss#2}
            }
            \par
            \nobreak
            \global\@nobreaktrue
            \everypar{\global\@nobreakfalse\everypar{}}%
        \endgroup
    \fi}
\newcommand\bookname{Book}
\renewcommand{\thebook}{\texorpdfstring{\Numberstring{book}}{book}}
\providecommand*{\toclevel@book}{-2}
\makeatother
\titleformat{\part}[display]
    {\Large\bfseries}
    {\partname\nobreakspace\thepart}
    {0mm}
    {\Huge\bfseries}
\titlecontents{part}[0pt]
    {\large\bfseries}
    {\partname\ \thecontentslabel: \quad}
    {}
    {\hfill\contentspage}
\titlecontents{chapter}[0pt]
    {\bfseries}
    {\chaptername\ \thecontentslabel:\quad}
    {}
    {\hfill\contentspage}
\newglossarystyle{longpara}{%
    \setglossarystyle{long}%
    \renewenvironment{theglossary}{%
        \begin{longtable}[l]{{p{0.25\hsize}p{0.65\hsize}}}
    }{\end{longtable}}%
    \renewcommand{\glossentry}[2]{%
        \glstarget{##1}{\glossentryname{##1}}%
        &\glossentrydesc{##1}{~##2.}
        \tabularnewline%
        \tabularnewline
    }%
}
\newglossary[not-glg]{notation}{not-gls}{not-glo}{Notation}
\newcommand*{\newnotation}[4][]{%
    \newglossaryentry{#2}{type=notation, name={\textbf{#3}, },
                          text={#4}, description={#4},#1}%
}
%--------------------------LENGTHS------------------------------%
% Spacings for the Table of Contents.
\addtolength{\cftsecnumwidth}{1ex}
\addtolength{\cftsubsecindent}{1ex}
\addtolength{\cftsubsecnumwidth}{1ex}
\addtolength{\cftfignumwidth}{1ex}
\addtolength{\cfttabnumwidth}{1ex}

% Indent and paragraph spacing.
\setlength{\parindent}{0em}
\setlength{\parskip}{0em}                                                           %
%--------------------------Main Document---------------------------------------%
\begin{document}
        \pagenumbering{roman}
        \title{Topics in Algebra}
        \author{Ryan Maguire}
        \date{\vspace{-5ex}}
        \maketitle
        \tableofcontents
        \clearpage
        \chapter*{Topics in Algebra}
        \addcontentsline{toc}{chapter}{Topics in Algebra}
        \markboth{}{SURGERY THEORY}
        \vspace{10ex}
        \setcounter{chapter}{1}
        \pagenumbering{arabic}
    \section{Lie Algebras}
        \begin{ldefinition}{Lie Algebras}{Lie_Algebras}
            A Lie algebra over a field $\mathbb{F}$ is an
            a vector space with a bilinear map $(X,Y)\rightarrow[X,Y]$
            satisfying:
            \begin{enumerate}
                \item Alternating $[X,X]=0$. This implies
                      $0=[X+Y,X+y]=[X,X]+[X,Y]+[Y,X]+[Y,Y]$, and thus
                      $[X,Y]=-[X,Y]$.
                \item $[X,[Y,Z]]+[Y,[Z,X]]+[Z,[X,Y]]=0$. Equivalently:
                      $[X,[Y,Z]]=[[X,Y],Z]+[Y,[X,Z]]$.
            \end{enumerate}
        \end{ldefinition}
        For any algebra $A$, a derivative of $A$ is a linear map
        $D:A\rightarrow{A}$ such that $D(ab)=D(b)b+aD(b)$ (Liebniz Rule).
        This says that $[X,\cdot]$ is a derivative.
        \begin{ldefinition}{Lie Algebra Homomorphism}{Lie_Algebra_Homomorphism}
            A Lie Algebra Homomorphism between two Lie algebras $G$ and $H$
            is a linear map $f:G\rightarrow{H}$ such that
            $f([X,Y])=[f(X),f(Y)]$.
        \end{ldefinition}
        \begin{ldefinition}{Lie Algebra Isomorphism}{Lie_Algebra_Isomorphism}
            A Lie algebra isomorphism between two Lie algebras $G$ and $H$
            is a homomorphism $f:G\rightarrow{H}$ such that $f$ is a bijection
            and $f^{\minus{1}}$ is a homomorphism.
        \end{ldefinition}
        \begin{lexample}{}{}
            A Lie algebra homomorphism is an isomorphism if and only if it has
            an inverse. Let $G$ be a vector space over $\mathbb{F}$. Define
            $[\cdot,\cdot]:G\times{G}\rightarrow{G}$ by $[X,Y]=0$ for all
            $X,Y\in{G}$. This is a Lie group.
        \end{lexample}
        \begin{ldefinition}{Abelian Lie Algebra}{Abelian Lie Algebra}
            An Abelian Lie Algebra is a Lie algebra $G$ such that, for all
            $X,Y\in{G}$, $[X,Y]=0$.
        \end{ldefinition}
        \begin{ldefinition}{Commutative Elements of a Lie Algebra}
                           {Commutative Elements of a Lie Algebra}
            Commutative elements are such that $[X,Y]=0$.
        \end{ldefinition}
        \begin{example}
            Let $G$ be an associative $\mathbb{F}$ algebra. Define a Lie
            algebra $\textrm{Lie}(A)$ as a vector space $\textrm{Lie}(A)=A$.
            Define $[a,b]=ab-ba$. Special case, $A=M_{n}(\mathbb{F})$.
            Then $\textrm{Lie}(M_{n}(\mathbb{F})$ is denoted by
            $\textrm{GL}_{n}(\mathbb{F})$.
        \end{example}
        \begin{ldefinition}{Lie Subalgebra}{Lie_Subalgebra}
            Let $G$ be a Lie algebra. A subspace $H\subseteq{G}$ is called a
            Lie subalgebra if it is closed under $[\cdot,\cdot]$.
        \end{ldefinition}
        Note that for subspaces $V,W\subseteq{G}$,
        $[V,W]=\textrm{Span}\{[X,Y]:X\in{V},Y\in{W}\}$.
        \begin{ldefinition}{Ideal of a Subspace}{Ideal_of_a_Subspace}
            A subspace $H\subseteq{G}$ is an ideal, denoted
            $H(TRIANGLE)G$ if $[G,H]\subseteq{G}$. That is, for all
            $h\in{H}$ and $x\in{G}$, $[x,h]\in{H}$.
        \end{ldefinition}
        \begin{theorem}
            If $f:G\rightarrow{H}$ is a Lie algebra homomorphism, then
            $\ker(f)$ is an ideal of $G$.
        \end{theorem}
        \begin{proof}
            For if $k\in\ker(F)$ and $x\in{G}$, then
            $f([x,k])=[f(x),f(k)]=0$, so $[x,k]\in\ker(f)$.
        \end{proof}
        \begin{theorem}
            If $H(TRIANGLE)G$, then $G/H$ has a Lie algebra structure such that
            $\pi:G\rightarrow{G}/H$, defined by $x\mapsto{x+H}$, is a Lie
            algebra homomorphism.
        \end{theorem}
        \begin{proof}
            For:
            \begin{equation}
                [X+H,Y+H]=[\pi(X),\pi(Y)]=\pi([X,Y])=[X,Y]+H
            \end{equation}
            Let $\overline{X}=X+H$. If $\overline{X}'=\overline{X}$ and
            $\overline{Y}'=\overline{Y}$, show that
            $\overline{[X,Y]}=\overline{[X',Y']}$. Thus:
            \begin{equation}
                X-X'=H_{1}\in{H}\quad\quad
                Y-Y'=H_{2}\in{H}
            \end{equation}
            And:
            \begin{equation}
                [X,Y]=[X'+H_{1},Y'+H_{2}]
                =[X',Y']+[H_{1},Y']+[X',H_{2}]+[H_{1},H_{2}]
            \end{equation}
            And this last sum of three is in $H$.
        \end{proof}
        \begin{theorem}
            If $H$ is an ideal of $G$ then there is a homomorphism $f$
            such that $H=\ker(f)$.
        \end{theorem}
        \begin{theorem}
            Let $f:G\rightarrow{H}$ be a map of Lie algebras, and let
            $K=\ker(f)$. Then there is a unique map
            $\overline{f}:G/K\rightarrow{H}$ such that $\overline{f}$ is
            injective and such that some commutative diagram thing.
        \end{theorem}
        \begin{example}
            \begin{enumerate}
                \item $0,G(TRIANGLE))G$
                \item $Z(G)=\{Z\in{G}:[X,Z]=0\}$ is called the center of $G$.
                \item $[G,G]=\textrm{Span}\{[X,Y]:X,Y\in{G}\}(TRIANGLE)G$
                \item Fuck it.
                \item For ideals $a,b(TRIANGLE)G$, $a+b(TRIANGLE)G$.
                \item Similarly for subtraction.
            \end{enumerate}
        \end{example}
        \begin{example}
            \begin{enumerate}
                \item V finite dimensional vector space over $\mathbb{F}$,
                      $A=\textrm{End}_{\mathbb{F}}(V)$ an associative
                      $\mathbb{F}$ algebra, then
                      $GL(V)=Lie(End_{\mathbb{F}}(V))$
                      with $[X,Y]=XY-YX$.
                \item $Tr:gl(V)\rightarrow\mathbb{F}$ abelian is a Lie algebra
                      homomorphism. $Tr([X,Y])=Tr(X,Y-YX)=0=[Tr(X),Tr(Y)]$.
                      $\ker(Tr)=SL(V)=\{X\in{GL}(V):Tr(X)=0\}$
            \end{enumerate}
        \end{example}
        If $V=\mathbb{F}^{n}$, denote $GL(V)$ by $GL_{n}(\mathbb{F})$ and $SL(V)$
        by $SL_{n}(\mathbb{F})$. Special cases: $SL_{2}(\mathbb{F})$. This has basis
        $(E^{ij})_{k\ell}=\delta^{i}_{k}\delta^{j}_{\ell}$.
        \begin{example}
            Compute $[X,Y]=[E^{12},E^{21}]=E^{11}-E^{22}=H$.
            And $[H,X]=2X$, $[H,Y]=\minus{2}Y$.
        \end{example}
        \begin{ldefinition}{Simple Lie Algebra}{Simple_Lie_Algebra}
            A Lie algebra is simple if it is non-abelian and its only ideals are
            $0$ and itself.
        \end{ldefinition}
        \begin{theorem}
            $SL_{2}(\mathbb{F})$ is simple for $\mathbb{F}=\mathbb{R}$ or
            $\mathbb{F}=\mathbb{C}$.
        \end{theorem}
        \begin{proof}
            Bracket notation says that $X$ is a 2-eigenvector of $[H,\cdot]$
            and $Y$ is a -2-eigenvector of $[H,\cdot]$. Let $A$ be a non-trivial
            ideal. We must show that $A=SP_{2}(\mathbb{F})$. Then
            $[X,U]\in{G}$. But:
            \begin{equation}
                [X,[X,U]]=[X,[X,xX+yY+hH]]
                =[X,gH-2hX]=\minus{2}yX
            \end{equation}
            So thus, either $X\in{A}$ or $Y=0$. Doing the same for $[Y,[Y,U]]$ shows
            that $\minus{2}xY=[Y,[Y,U]]$ and thus either $Y\in{A}$ or $X=0$.
            There are two cases now. If If $x=y=0$ then $h\ne{0}$ since $U$ is
            non-zero. This would imply $H\in{A}$. But
            $2X=[X,H]\in{A}$ so $X\in{A}$. Also, $Y\in{A}$. Similarly if $y\ne{0}$.
        \end{proof}
        This can be generalized for $SL_{n}(\mathbb{F})$ as well.
        \begin{ldefinition}{Normalizer of a Lie Algebra Subspace}
                           {Normalizer_of_Lie_Alg_Subspace}
            The Normalizer of $H\subset{G}$ of a Lie Algebra $G$ is the set:
            \begin{equation}
                N_{G}(H)=\{X\in{G}:[X,h]\in{H}\}
            \end{equation}
        \end{ldefinition}
        By the Jacobian identity, the normalizer of a subspace is also a
        subalgebra. Moreover, by def, $H$ is an ideal of its normalizer.
        \begin{ldefinition}{Centralizer of a Lie Algebra Subspace}
                           {Centralizer_of_a_Lie_Algebra_Subspace}
            The centralizer of $H\subset{G}$ is the set:
            \begin{equation}
                C_{G}(H)=\{X\in{G}:[X,H]=0\}
            \end{equation}
        \end{ldefinition}
        \begin{ldefinition}{Product Lie Algebras}{Product_Lie_Algebra}
            For Lie algebras $G_{1},G_{2}$, their product
            $G_{1}\times{G}_{2}$, also written $G_{1}\oplus{G}_{2}$, is the set
            $G_{1}\oplus{G}_{2}$ as a vector space with the bracket:
            \begin{equation}
                [(X_{1},X_{2}),(Y_{1},Y_{2})]=([X_{1},Y_{1}],[X_{2},Y_{2}]
            \end{equation}
        \end{ldefinition}
        $G_{1}\oplus{G}_{2}$ has ideals $\overline{G}_{1}=G_{1}\oplus{0}$ and
        $\overline{G}_{2}=0\oplus{G}_{2}$, and moreover
        $\overline{G}_{1}+\overline{G}_{2}=G_{1}\oplus{G}_{2}$
        \begin{theorem}
            If $G$ is a lie algebra, $a$, $b$ ideals of $G$ satisfying:
            \begin{enumerate}
                \item $a+b=G$
                \item $a\cap{b}=\emptyset$
                \item $[a,b]=0$
            \end{enumerate}
            Then $G$ is isomorphic to $a\oplus{b}$.
        \end{theorem}
        \begin{proof}
            Define $\varphi:a\oplus{b}\rightarrow{G}$ by
            $\varphi(A,B)=A+B$. By vector space theory, this is a vector
            space isomorphism. We need to check that it preserves brackets.
            But:
            \begin{equation}
                \varphi([(A,B),(A',B')])=\varphi([A,A'],[B,B'])
                =[A,A']+[B,B']
            \end{equation}
            But also:
            \begin{equation}
                [\varphi(A,B),\varphi(A',B')]=[A+B,A'+B']
                =[A,A']+[B,A']+[A,B']+[B,B']
            \end{equation}
            But $[B,A']=[A,B']=0$, completing the proof.
        \end{proof}
    \section{Review of Differentiation}
        If $\mathcal{U}\subseteq\mathbb{R}^{n}$ is open, and if
        $f:\mathcal{U}\rightarrow\mathbb{R}^{n}$ is a function, then
        $Df(p)$ (if it is exists) is a linear map
        $Df(p):\mathbb{R}^{n}\rightarrow\mathbb{R}^{m}$ such that:
        \begin{equation}
            \underset{h\rightarrow{0}}{\lim}
                \frac{f(p+h)-(f(p)+Df(p)(h)}{\norm{h}}=0
        \end{equation}
        That is, $Df(p)$ is the best linear affine approximation to $f$ at
        $p$.
        \begin{theorem}
            If $f\in{C}^{1}$, then $Df(p)$ exists and:
            \begin{equation}
                Df(p)(v)=\underset{t\rightarrow{0}}{\lim}
                \frac{f(p+tr)-f(p)}{t}
            \end{equation}
        \end{theorem}
        Chain rule, if $p\in\mathcal{U}$,
        $f:\mathcal{U}\rightarrow\mathbb{R}$,
        then $D(g\circ{f})(o)=Dg(f(p))Df(p)$
        \begin{theorem}
            If $V_{1}$, $V_{2}$, $W$ are $\mathbb{R}$ vector spaces, and
            if $B:V_{1}\times{V}_{2}\rightarrow{W}$ is bilinear, then
            for $p_{1},v_{1}\in{V}_{1}$, $p_{2},v_{2}\in{V}_{2}$:
            \begin{equation}
                DB(p_{1},p_{2})(v_{1},v_{2})=B(p_{1},v_{2})+B(v_{1},p_{2})
            \end{equation}
        \end{theorem}
        This can be generalized to multilinear maps.
        \begin{theorem}
            if $B:V_{1}\times\cdots\times{V}_{n}\rightarrow{W}$ is
            multilinear, then:
            \begin{equation}
                DB(p_{1},\dots,p_{n})(v_{1},\dots,v_{n})=
                \sum_{j=1}^{n}B(p_{1},\dots,p_{j-1},v_{j},p_{j+1},\dots,p_{n})
            \end{equation}
        \end{theorem}
        \begin{example}
            Let $det:M_{n}(\mathbb{R})\rightarrow\mathbb{R}$ be the
            determinant function. Then $D(det)(x)(H)=Tr(X^{Thing}H)$,
            where $X^{Thing}$ is the classical adjugate matrix:
            \begin{equation}
                (X^{Thing})_{ij}=(\minus{1})^{i+j}det(M_{ij}(X)
            \end{equation}
            Where $M_{ij}(X)$ is the minor of $X$ formed by crossing out the
            $i^{th}$ row and $j^{th}$ column.
        \end{example}
        \begin{example}
            Let $F_{k}:M_{n}(\mathbb{R})\rightarrow{M}_{n}(\mathbb{R})$ be
            defined by $F_{k}(X)=X^{k}$. Then:
            \begin{equation}
                DF_{k}(X)(H)=X^{k-1}H+x^{k-2}HX+\cdots+XHX^{k-1}+HX^{k-1}
            \end{equation}
            If $X$ and $H$ commute, then:
            \begin{equation}
                DF_{k}(X)(H)=kX^{k-1}H
            \end{equation}
        \end{example}
        \begin{theorem}
            If $V:GL_{n}(\mathbb{R})\rightarrow{M}_{n}(\mathbb{R})$ is
            defined by $V(X)=X^{\minus{1}}$, then:
            \begin{equation}
                DV(X)(H)=\minus{X}^{\minus{1}}HX^{\minus{1}}
            \end{equation}
        \end{theorem}
    \section{Lie Groups}
        Let $\mathbb{F}$ denote either $\mathbb{R}$ or $\mathbb{C}$.
        \begin{ldefinition}{Real Lie Groups}{Real_Lie_Group}
            A real Lie group is a $C^{\infty}$ manifold $C$ with a group
            structure such that the operation is smooth. That is,
            $m:G\imes{G}\rightarrow{G}$ which maps $(x,y)\mapsto{x}*y$ is
            smooth, and $v:G\rightarrow{G}$ which maps
            $x\mapsto{x}^{\minus{1}}$ is also smooth.
        \end{ldefinition}
        \begin{ldefinition}{Lie Group Homomorphism}{}
            If $G$ and $H$ are Lie groups, a Lie group homomorphism is a
            smooth group homomorphism $f:G\rightarrow{H}$.
        \end{ldefinition}
        \begin{example}
            A Complex Lie group is a complex manifold with a group structure
            whose operations are holomorphic. Let $\mathbb{R}$ or
            $\mathbb{F}$ be a finite dimensional $\mathbb{F}$ vector space.
            Then $G:(V)\subseteq\textrm{End}(V)$ is an open subset of
            the vector space $\textrm{End}_{\mathbb{F}}(V)$. So this is
            a Lie group (real if $\mathbb{F}=\mathbb{R}$, complex if
            $\mathbb{F}=\mathbb{C}$) and
            $GL_{n}(\mathbb{F})=GL(\mathbb{F}^{n})$. As another example,
            $SL_{n}(\mathbb{F})=\{g\in{GL}_{n}(\mathbb{F}):det(g)=1\}$.
            We now how to show that this is a Lie group. We have that
            $SL_{n}(\mathbb{F})=det^{\minus{1}}(1)$. By the implicit
            function theorem, $det^{\minus{1}}(1)$ is a smooth
            manifold provided that
            $D(det)(X):M_{n}(\mathbb{F})\rightarrow\mathbb{F}$ is a
            surjective map for any $X\in{det}^{\minus{1}}(1)$. But
            $D(det)(X)H)=Tr(X^{Thing}H)$, so:
            \begin{equation}
                D(det)(X)(X)=Tr(X^{Thing}X)=Tr(det(X)I)=n\ne{0}
            \end{equation}
        \end{example}
        \begin{example}
            Let $T_{n}(\mathbb{F})$ be the subgroup of $GL_{n}(\mathbb{F})$
            stabilizing the standard flag
            $0=V_{0}\subseteq{V}_{1}\susbeteq\dots\subseteq{V}_{n}=\mathbb{F}^{n}$. This is often called the Borel subgroup. Let $V_{j}$ be
            the span of $\{e_{1},\dots,e_{j}\}$. There is a special case when
            $\mathbb{F}=\mathbb{R}$ and $n=3$. This is called the
            Heisenberg group, and is the set of all matrices of the following:
            \begin{equation}
                A=\begin{bmatrix}
                    1&x&y\\
                    0&1&z\\
                    0&0&1
                \end{bmatrix}
            \end{equation}
            Topologically this is homeomorphic to $\mathbb{R}^{3}$.
        \end{example}
        Let $<>$ be a bilinear form on $\mathbb{F}$. Pick a basis $B$ of $V$.
        Define $G$ by $G_{ij}=<v_{i},v_{j}>$.
\end{document}