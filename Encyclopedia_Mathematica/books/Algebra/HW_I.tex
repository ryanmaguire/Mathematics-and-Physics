%------------------------------------------------------------------------------%
\documentclass[crop=false,class=article]{standalone}                           %
%------------------------------Preamble----------------------------------------%
\makeatletter                                                                  %
    \def\input@path{{../../../}}                                               %
\makeatother                                                                   %
%---------------------------Packages----------------------------%
\usepackage{geometry}
\geometry{b5paper, margin=1.0in}
\usepackage[T1]{fontenc}
\usepackage{graphicx, float}            % Graphics/Images.
\usepackage{natbib}                     % For bibliographies.
\bibliographystyle{agsm}                % Bibliography style.
\usepackage[french, english]{babel}     % Language typesetting.
\usepackage[dvipsnames]{xcolor}         % Color names.
\usepackage{listings}                   % Verbatim-Like Tools.
\usepackage{mathtools, esint, mathrsfs} % amsmath and integrals.
\usepackage{amsthm, amsfonts, amssymb}  % Fonts and theorems.
\usepackage{tcolorbox}                  % Frames around theorems.
\usepackage{upgreek}                    % Non-Italic Greek.
\usepackage{fmtcount, etoolbox}         % For the \book{} command.
\usepackage[newparttoc]{titlesec}       % Formatting chapter, etc.
\usepackage{titletoc}                   % Allows \book in toc.
\usepackage[nottoc]{tocbibind}          % Bibliography in toc.
\usepackage[titles]{tocloft}            % ToC formatting.
\usepackage{pgfplots, tikz}             % Drawing/graphing tools.
\usepackage{imakeidx}                   % Used for index.
\usetikzlibrary{
    calc,                   % Calculating right angles and more.
    angles,                 % Drawing angles within triangles.
    arrows.meta,            % Latex and Stealth arrows.
    quotes,                 % Adding labels to angles.
    positioning,            % Relative positioning of nodes.
    decorations.markings,   % Adding arrows in the middle of a line.
    patterns,
    arrows
}                                       % Libraries for tikz.
\pgfplotsset{compat=1.9}                % Version of pgfplots.
\usepackage[font=scriptsize,
            labelformat=simple,
            labelsep=colon]{subcaption} % Subfigure captions.
\usepackage[font={scriptsize},
            hypcap=true,
            labelsep=colon]{caption}    % Figure captions.
\usepackage[pdftex,
            pdfauthor={Ryan Maguire},
            pdftitle={Mathematics and Physics},
            pdfsubject={Mathematics, Physics, Science},
            pdfkeywords={Mathematics, Physics, Computer Science, Biology},
            pdfproducer={LaTeX},
            pdfcreator={pdflatex}]{hyperref}
\hypersetup{
    colorlinks=true,
    linkcolor=blue,
    filecolor=magenta,
    urlcolor=Cerulean,
    citecolor=SkyBlue
}                           % Colors for hyperref.
\usepackage[toc,acronym,nogroupskip,nopostdot]{glossaries}
\usepackage{glossary-mcols}
%------------------------Theorem Styles-------------------------%
\theoremstyle{plain}
\newtheorem{theorem}{Theorem}[section]

% Define theorem style for default spacing and normal font.
\newtheoremstyle{normal}
    {\topsep}               % Amount of space above the theorem.
    {\topsep}               % Amount of space below the theorem.
    {}                      % Font used for body of theorem.
    {}                      % Measure of space to indent.
    {\bfseries}             % Font of the header of the theorem.
    {}                      % Punctuation between head and body.
    {.5em}                  % Space after theorem head.
    {}

% Italic header environment.
\newtheoremstyle{thmit}{\topsep}{\topsep}{}{}{\itshape}{}{0.5em}{}

% Define environments with italic headers.
\theoremstyle{thmit}
\newtheorem*{solution}{Solution}

% Define default environments.
\theoremstyle{normal}
\newtheorem{example}{Example}[section]
\newtheorem{definition}{Definition}[section]
\newtheorem{problem}{Problem}[section]

% Define framed environment.
\tcbuselibrary{most}
\newtcbtheorem[use counter*=theorem]{ftheorem}{Theorem}{%
    before=\par\vspace{2ex},
    boxsep=0.5\topsep,
    after=\par\vspace{2ex},
    colback=green!5,
    colframe=green!35!black,
    fonttitle=\bfseries\upshape%
}{thm}

\newtcbtheorem[auto counter, number within=section]{faxiom}{Axiom}{%
    before=\par\vspace{2ex},
    boxsep=0.5\topsep,
    after=\par\vspace{2ex},
    colback=Apricot!5,
    colframe=Apricot!35!black,
    fonttitle=\bfseries\upshape%
}{ax}

\newtcbtheorem[use counter*=definition]{fdefinition}{Definition}{%
    before=\par\vspace{2ex},
    boxsep=0.5\topsep,
    after=\par\vspace{2ex},
    colback=blue!5!white,
    colframe=blue!75!black,
    fonttitle=\bfseries\upshape%
}{def}

\newtcbtheorem[use counter*=example]{fexample}{Example}{%
    before=\par\vspace{2ex},
    boxsep=0.5\topsep,
    after=\par\vspace{2ex},
    colback=red!5!white,
    colframe=red!75!black,
    fonttitle=\bfseries\upshape%
}{ex}

\newtcbtheorem[auto counter, number within=section]{fnotation}{Notation}{%
    before=\par\vspace{2ex},
    boxsep=0.5\topsep,
    after=\par\vspace{2ex},
    colback=SeaGreen!5!white,
    colframe=SeaGreen!75!black,
    fonttitle=\bfseries\upshape%
}{not}

\newtcbtheorem[use counter*=remark]{fremark}{Remark}{%
    fonttitle=\bfseries\upshape,
    colback=Goldenrod!5!white,
    colframe=Goldenrod!75!black}{ex}

\newenvironment{bproof}{\textit{Proof.}}{\hfill$\square$}
\tcolorboxenvironment{bproof}{%
    blanker,
    breakable,
    left=3mm,
    before skip=5pt,
    after skip=10pt,
    borderline west={0.6mm}{0pt}{green!80!black}
}

\AtEndEnvironment{lexample}{$\hfill\textcolor{red}{\blacksquare}$}
\newtcbtheorem[use counter*=example]{lexample}{Example}{%
    empty,
    title={Example~\theexample},
    boxed title style={%
        empty,
        size=minimal,
        toprule=2pt,
        top=0.5\topsep,
    },
    coltitle=red,
    fonttitle=\bfseries,
    parbox=false,
    boxsep=0pt,
    before=\par\vspace{2ex},
    left=0pt,
    right=0pt,
    top=3ex,
    bottom=1ex,
    before=\par\vspace{2ex},
    after=\par\vspace{2ex},
    breakable,
    pad at break*=0mm,
    vfill before first,
    overlay unbroken={%
        \draw[red, line width=2pt]
            ([yshift=-1.2ex]title.south-|frame.west) to
            ([yshift=-1.2ex]title.south-|frame.east);
        },
    overlay first={%
        \draw[red, line width=2pt]
            ([yshift=-1.2ex]title.south-|frame.west) to
            ([yshift=-1.2ex]title.south-|frame.east);
    },
}{ex}

\AtEndEnvironment{ldefinition}{$\hfill\textcolor{Blue}{\blacksquare}$}
\newtcbtheorem[use counter*=definition]{ldefinition}{Definition}{%
    empty,
    title={Definition~\thedefinition:~{#1}},
    boxed title style={%
        empty,
        size=minimal,
        toprule=2pt,
        top=0.5\topsep,
    },
    coltitle=Blue,
    fonttitle=\bfseries,
    parbox=false,
    boxsep=0pt,
    before=\par\vspace{2ex},
    left=0pt,
    right=0pt,
    top=3ex,
    bottom=0pt,
    before=\par\vspace{2ex},
    after=\par\vspace{1ex},
    breakable,
    pad at break*=0mm,
    vfill before first,
    overlay unbroken={%
        \draw[Blue, line width=2pt]
            ([yshift=-1.2ex]title.south-|frame.west) to
            ([yshift=-1.2ex]title.south-|frame.east);
        },
    overlay first={%
        \draw[Blue, line width=2pt]
            ([yshift=-1.2ex]title.south-|frame.west) to
            ([yshift=-1.2ex]title.south-|frame.east);
    },
}{def}

\AtEndEnvironment{ltheorem}{$\hfill\textcolor{Green}{\blacksquare}$}
\newtcbtheorem[use counter*=theorem]{ltheorem}{Theorem}{%
    empty,
    title={Theorem~\thetheorem:~{#1}},
    boxed title style={%
        empty,
        size=minimal,
        toprule=2pt,
        top=0.5\topsep,
    },
    coltitle=Green,
    fonttitle=\bfseries,
    parbox=false,
    boxsep=0pt,
    before=\par\vspace{2ex},
    left=0pt,
    right=0pt,
    top=3ex,
    bottom=-1.5ex,
    breakable,
    pad at break*=0mm,
    vfill before first,
    overlay unbroken={%
        \draw[Green, line width=2pt]
            ([yshift=-1.2ex]title.south-|frame.west) to
            ([yshift=-1.2ex]title.south-|frame.east);},
    overlay first={%
        \draw[Green, line width=2pt]
            ([yshift=-1.2ex]title.south-|frame.west) to
            ([yshift=-1.2ex]title.south-|frame.east);
    }
}{thm}

%--------------------Declared Math Operators--------------------%
\DeclareMathOperator{\adjoint}{adj}         % Adjoint.
\DeclareMathOperator{\Card}{Card}           % Cardinality.
\DeclareMathOperator{\curl}{curl}           % Curl.
\DeclareMathOperator{\diam}{diam}           % Diameter.
\DeclareMathOperator{\dist}{dist}           % Distance.
\DeclareMathOperator{\Div}{div}             % Divergence.
\DeclareMathOperator{\Erf}{Erf}             % Error Function.
\DeclareMathOperator{\Erfc}{Erfc}           % Complementary Error Function.
\DeclareMathOperator{\Ext}{Ext}             % Exterior.
\DeclareMathOperator{\GCD}{GCD}             % Greatest common denominator.
\DeclareMathOperator{\grad}{grad}           % Gradient
\DeclareMathOperator{\Ima}{Im}              % Image.
\DeclareMathOperator{\Int}{Int}             % Interior.
\DeclareMathOperator{\LC}{LC}               % Leading coefficient.
\DeclareMathOperator{\LCM}{LCM}             % Least common multiple.
\DeclareMathOperator{\LM}{LM}               % Leading monomial.
\DeclareMathOperator{\LT}{LT}               % Leading term.
\DeclareMathOperator{\Mod}{mod}             % Modulus.
\DeclareMathOperator{\Mon}{Mon}             % Monomial.
\DeclareMathOperator{\multideg}{mutlideg}   % Multi-Degree (Graphs).
\DeclareMathOperator{\nul}{nul}             % Null space of operator.
\DeclareMathOperator{\Ord}{Ord}             % Ordinal of ordered set.
\DeclareMathOperator{\Prin}{Prin}           % Principal value.
\DeclareMathOperator{\proj}{proj}           % Projection.
\DeclareMathOperator{\Refl}{Refl}           % Reflection operator.
\DeclareMathOperator{\rk}{rk}               % Rank of operator.
\DeclareMathOperator{\sgn}{sgn}             % Sign of a number.
\DeclareMathOperator{\sinc}{sinc}           % Sinc function.
\DeclareMathOperator{\Span}{Span}           % Span of a set.
\DeclareMathOperator{\Spec}{Spec}           % Spectrum.
\DeclareMathOperator{\supp}{supp}           % Support
\DeclareMathOperator{\Tr}{Tr}               % Trace of matrix.
%--------------------Declared Math Symbols--------------------%
\DeclareMathSymbol{\minus}{\mathbin}{AMSa}{"39} % Unary minus sign.
%------------------------New Commands---------------------------%
\DeclarePairedDelimiter\norm{\lVert}{\rVert}
\DeclarePairedDelimiter\ceil{\lceil}{\rceil}
\DeclarePairedDelimiter\floor{\lfloor}{\rfloor}
\newcommand*\diff{\mathop{}\!\mathrm{d}}
\newcommand*\Diff[1]{\mathop{}\!\mathrm{d^#1}}
\renewcommand*{\glstextformat}[1]{\textcolor{RoyalBlue}{#1}}
\renewcommand{\glsnamefont}[1]{\textbf{#1}}
\renewcommand\labelitemii{$\circ$}
\renewcommand\thesubfigure{%
    \arabic{chapter}.\arabic{figure}.\arabic{subfigure}}
\addto\captionsenglish{\renewcommand{\figurename}{Fig.}}
\numberwithin{equation}{section}

\renewcommand{\vector}[1]{\boldsymbol{\mathrm{#1}}}

\newcommand{\uvector}[1]{\boldsymbol{\hat{\mathrm{#1}}}}
\newcommand{\topspace}[2][]{(#2,\tau_{#1})}
\newcommand{\measurespace}[2][]{(#2,\varSigma_{#1},\mu_{#1})}
\newcommand{\measurablespace}[2][]{(#2,\varSigma_{#1})}
\newcommand{\manifold}[2][]{(#2,\tau_{#1},\mathcal{A}_{#1})}
\newcommand{\tanspace}[2]{T_{#1}{#2}}
\newcommand{\cotanspace}[2]{T_{#1}^{*}{#2}}
\newcommand{\Ckspace}[3][\mathbb{R}]{C^{#2}(#3,#1)}
\newcommand{\funcspace}[2][\mathbb{R}]{\mathcal{F}(#2,#1)}
\newcommand{\smoothvecf}[1]{\mathfrak{X}(#1)}
\newcommand{\smoothonef}[1]{\mathfrak{X}^{*}(#1)}
\newcommand{\bracket}[2]{[#1,#2]}

%------------------------Book Command---------------------------%
\makeatletter
\renewcommand\@pnumwidth{1cm}
\newcounter{book}
\renewcommand\thebook{\@Roman\c@book}
\newcommand\book{%
    \if@openright
        \cleardoublepage
    \else
        \clearpage
    \fi
    \thispagestyle{plain}%
    \if@twocolumn
        \onecolumn
        \@tempswatrue
    \else
        \@tempswafalse
    \fi
    \null\vfil
    \secdef\@book\@sbook
}
\def\@book[#1]#2{%
    \refstepcounter{book}
    \addcontentsline{toc}{book}{\bookname\ \thebook:\hspace{1em}#1}
    \markboth{}{}
    {\centering
     \interlinepenalty\@M
     \normalfont
     \huge\bfseries\bookname\nobreakspace\thebook
     \par
     \vskip 20\p@
     \Huge\bfseries#2\par}%
    \@endbook}
\def\@sbook#1{%
    {\centering
     \interlinepenalty \@M
     \normalfont
     \Huge\bfseries#1\par}%
    \@endbook}
\def\@endbook{
    \vfil\newpage
        \if@twoside
            \if@openright
                \null
                \thispagestyle{empty}%
                \newpage
            \fi
        \fi
        \if@tempswa
            \twocolumn
        \fi
}
\newcommand*\l@book[2]{%
    \ifnum\c@tocdepth >-3\relax
        \addpenalty{-\@highpenalty}%
        \addvspace{2.25em\@plus\p@}%
        \setlength\@tempdima{3em}%
        \begingroup
            \parindent\z@\rightskip\@pnumwidth
            \parfillskip -\@pnumwidth
            {
                \leavevmode
                \Large\bfseries#1\hfill\hb@xt@\@pnumwidth{\hss#2}
            }
            \par
            \nobreak
            \global\@nobreaktrue
            \everypar{\global\@nobreakfalse\everypar{}}%
        \endgroup
    \fi}
\newcommand\bookname{Book}
\renewcommand{\thebook}{\texorpdfstring{\Numberstring{book}}{book}}
\providecommand*{\toclevel@book}{-2}
\makeatother
\titleformat{\part}[display]
    {\Large\bfseries}
    {\partname\nobreakspace\thepart}
    {0mm}
    {\Huge\bfseries}
\titlecontents{part}[0pt]
    {\large\bfseries}
    {\partname\ \thecontentslabel: \quad}
    {}
    {\hfill\contentspage}
\titlecontents{chapter}[0pt]
    {\bfseries}
    {\chaptername\ \thecontentslabel:\quad}
    {}
    {\hfill\contentspage}
\newglossarystyle{longpara}{%
    \setglossarystyle{long}%
    \renewenvironment{theglossary}{%
        \begin{longtable}[l]{{p{0.25\hsize}p{0.65\hsize}}}
    }{\end{longtable}}%
    \renewcommand{\glossentry}[2]{%
        \glstarget{##1}{\glossentryname{##1}}%
        &\glossentrydesc{##1}{~##2.}
        \tabularnewline%
        \tabularnewline
    }%
}
\newglossary[not-glg]{notation}{not-gls}{not-glo}{Notation}
\newcommand*{\newnotation}[4][]{%
    \newglossaryentry{#2}{type=notation, name={\textbf{#3}, },
                          text={#4}, description={#4},#1}%
}
%--------------------------LENGTHS------------------------------%
% Spacings for the Table of Contents.
\addtolength{\cftsecnumwidth}{1ex}
\addtolength{\cftsubsecindent}{1ex}
\addtolength{\cftsubsecnumwidth}{1ex}
\addtolength{\cftfignumwidth}{1ex}
\addtolength{\cfttabnumwidth}{1ex}

% Indent and paragraph spacing.
\setlength{\parindent}{0em}
\setlength{\parskip}{0em}                                                           %
%----------------------------Main Document-------------------------------------%
\begin{document}
    \title{Topics in Algebra}
    \author{Ryan Maguire}
    \date{\vspace{-5ex}}
    \maketitle
    \section{Homework I}
        \begin{problem}
            Let $k$ be a field, let $V$ and $W$ be finite dimensional vector
            spaces over $k$, and let $T:V\rightarrow{W}$ be a linear
            transformations. Show that there exists bases $\mathscr{B}$ of $V$
            and $\mathscr{C}$ of $W$ such that the representing matrix
            $[T]_{\mathscr{C}}^{\mathscr{B}}$ has the following form: All
            off-diagonal entries are zero, while all the entries lie in the set
            $\{0,\,1\}$. Give two proofs:
            \begin{enumerate}
                \item   An abstract proof (Rank-Nullity Theorem)
                \item   A computational proof [Recall row/column operations]
            \end{enumerate}
        \end{problem}
        \begin{solution}
            If $V$ and $W$ are finite dimensional vector spaces then there
            exists a basis $\mathcal{V}$ and $\mathcal{W}$ of $V$ and $W$,
            respectively, such that $\textrm{Card}(\mathcal{V})\in\mathbb{N}$
            and $\textrm{Card}(\mathcal{W})\in\mathbb{N}$. Let
            $m=\textrm{Card}(\mathcal{V})$ and $n=\textrm{Card}(\mathcal{W})$.
            We want a matrix $[T]_{\mathscr{C}}^{\mathscr{B}}$ such that:
            \begin{equation}
                T\mathbf{x}=T
                \begin{bmatrix}
                    x_{1}\\
                    \vdots\\
                    x_{m}
                \end{bmatrix}=
                \begin{bmatrix}
                    a_{11}&\cdots&a_{1m}\\
                    \vdots&\ddots&\vdots\\
                    a_{n1}&\cdots&a_{nm}
                \end{bmatrix}
                \begin{bmatrix}
                    x_{1}\\
                    \vdots\\
                    x_{m}
                \end{bmatrix}
            \end{equation}
            Where the $a_{ij}$ are zero off of the diagonal. First, removing
            this requirement, such a matrix does exists if we define
            $a_{ij}$ by:
            \begin{equation}
                \begin{bmatrix}
                    a_{1i}\\
                    \vdots\\
                    a_{ni}
                \end{bmatrix}=Tv_{i}
            \end{equation}
            To obtain bases of the form we desire, write:
            \begin{equation}
                \begin{bmatrix}
                    y_{1}\\
                    \vdots\\
                    y_{n}
                \end{bmatrix}=
                \begin{bmatrix}
                    a_{11}&\cdots&a_{1m}\\
                    \vdots&\ddots&\vdots\\
                    a_{n1}&\cdots&a_{nm}
                \end{bmatrix}
                \begin{bmatrix}
                    x_{1}\\
                    \vdots\\
                    x_{m}
                \end{bmatrix}
            \end{equation}
            We can row reduce this matrix by means of Gaussian elimination,
            keeping track of the row operations as applied to the $x_{i}$
            and $y_{j}$, so that the matrix has only entries $0$ or $1$.
            That is, place the matrix into row-echelon form. To obtain a
            matrix that has only zeroes of ones on the diagonal, further
            reduce this matrix by column operations, placing the matrix
            in \textit{column} echelon form. The resulting matrix will thus
            be both row reduced and column reduced, and is therefore only
            non-zero on the diagonal. The bases $\mathscr{B}$ and
            $\mathscr{C}$ are obtained from these operations.
            \par\hfill\par
            Going the abstract way, by the rank-nullity theorem:
            \begin{equation}
                \textrm{dim}\big(T(V)\big)+
                \textrm{dim}\big(\ker(T)\big)=
                \textrm{dim}(V)=m
            \end{equation}
            Moreover the kernel of $T$ is a subspace of $V$. Let
            $\mathcal{N}$ be a basis of this and let $\mathcal{O}$ be such
            that $\mathcal{O}\cup\mathcal{N}$ is a basis of $V$, and such
            that $\mathcal{O}\cap\mathcal{N}=\emptyset$.
            Define $\mathscr{B}=\mathcal{O}\cup\mathcal{N}$. By the
            definition of the kernel, for all $e\in\mathcal{N}$, $Te=0$.
            Moreover, $T\mathcal{O}$ is a basis for the image of $T$.
            Let $w_{i}=Tv_{i}$ for $v_{i}\in\mathcal{O}$. We can then
            extend $T\mathcal{O}$ with a set $\mathcal{U}$ such that
            $T\mathcal{O}\cup\mathcal{U}$ is a basis of $W$ and such that
            $T\mathcal{O}\cap\mathcal{U}=\emptyset$. None of the
            points in this span (other than zero) are mapped to as they lie
            outside of the image. Let
            $\mathscr{C}=T\mathcal{O}\cup\mathcal{U}$.
            The representing matrix $[T]_{\mathscr{C}}^{\mathscr{B}}$ will
            then have elements only in $\{0,1\}$, and by construction all
            off-diagonal elements will be zero.
        \end{solution}
        \begin{problem}
            Let $V$ be a finite dimensional vector space and
            $T:V\rightarrow{V}$ a linear operator. When does $V$ have a
            basis $\mathscr{B}$ such that $[T]_{\mathscr{B}}^{\mathscr{B}}$
            is a diagonal matrix with all diagonal entires in the set
            $\{0,1\}$? Find a necessary and sufficient condition on $T$.
            Why was no condition needed in the previous problem?
        \end{problem}
        \begin{solution}
            No condition was required in the previous problem sinced we
            allowed the bases of both the domain and the target space to
            vary. Now we require that both bases are identical. If
            $T$ is idempotent, then $T$ will have a basis such that
            $[T]_{\mathscr{B}}^{\mathscr{B}}$ is non-zero only on the
            diagonal, and having values in $\{0,1\}$ on the diagonal.
            Let $\mathcal{N}$ be a basis for $\ker(T)$, and let
            $\mathcal{B}$ be a basis for image of $T$. By the rank-nullity
            theorem, $\mathcal{B}\cup\mathcal{N}$ is a basis for $V$.
            Let $\mathscr{B}=\mathcal{N}\cup{T}\mathcal{B}$, where
            $T\mathcal{B}$ is the set of all $Tv$ such that
            $v\in\mathcal{B}$. Then this is a basis for $V$. Moreover,
            the representative matrix is only non-zero on the diagonal.
            For if $v\in{T}\mathcal{B}$, then:
            \begin{equation}
                Tv_{i}=\sum_{j}a_{ij}v_{j}
                \Longrightarrow
                T^{2}v_{i}=\sum_{j}a_{ij}Tv_{j}
            \end{equation}
            But $T$ is idempotent, so we have:
            \begin{equation}
                Tv_{i}=\sum_{j}a_{ij}Tv_{j}
                \Longrightarrow
                T(v_{i}-\sum_{j}a_{ij}v_{i})=0
                \Longrightarrow
                v_{i}-\sum_{j}a_{ij}v_{i}=0
            \end{equation}
            But then $v_{j}$ are independent, and therefore
            $a_{ij}=0$ for all $i\ne{j}$.
            \par\hfill\par
            Going the other way, if there is a basis $\mathscr{B}$ such
            that $[T]_{\mathscr{B}}^{\mathscr{B}}$ is non-zero only on
            the diagonal, and whose diagonal elements are either $0$ or
            $1$, then $T$ is idempotent. For If $T$ has such a
            representation, then $T^{2}=T$, since all non-diagonal
            components are zero and for all $a_{ii}$ on the diagonal,
            $a_{ii}^{2}=a_{ii}$.
        \end{solution}
        \begin{problem}
            Let $R_{1}$ and $R_{2}$ be rings. Let $M$ be a module over
            the product ring $R=R_{1}\times{R}_{2}$. Show that $M$ is
            the direct sum $M=M_{1}\oplus{M}_{2}$ of two
            $R$ submodules $M_{1}$ and $M_{2}$ where $M_{1}$ is an
            $R_{1}$ submodule (viewed as an $R$ module by pullback along
            the projection $\pi_{1}:R_{1}\times{R}_{2}\rightarrow{R}_{1}$
            defined by $(x,y)\mapsto{x}$) and $M_{2}$ is an
            $R_{2}$ module (viewed similarly as above).
        \end{problem}
        \begin{solution}
            For let $M_{1}$ and $M_{2}$ be defined as follows:
            \par
            \begin{subequations}
                \begin{minipage}[b]{0.49\textwidth}
                    \begin{equation}
                        M_{1}=\{\,(1_{1},\,0_{2})
                                \boldsymbol{\cdot}\mathbf{v}\,:\,
                                \mathbf{v}\in{M}\,\}
                    \end{equation}
                \end{minipage}
                \hfill
                \begin{minipage}[b]{0.49\textwidth}
                    \begin{equation}
                        M_{2}=\{\,(0_{1},\,1_{2})
                                \boldsymbol{\cdot}\mathbf{v}\,:\,
                                \mathbf{v}\in{M}\,\}
                    \end{equation}
                \end{minipage}
            \end{subequations}
            \par\vspace{2.5ex}
            Where $\boldsymbol{\cdot}$ is the multiplicative operation
            $\boldsymbol{\cdot}:R\times{M}\rightarrow{M}$,
            $1_{1}$ is the multiplicative identity of $R_{1}$,
            $1_{2}$ is the multiplicative identity of $R_{2}$,
            $0_{1}$ is the additive identity of $R_{1}$, and
            $0_{2}$ is the additive identity of $R_{2}$. Then
            $M_{1}$ is a submodule of $M$. For if $\mathbf{u}\in{M}_{1}$,
            then there is a $\mathbf{v}\in{M}$ such that
            $\mathbf{u}=(1_{1},0_{2})\boldsymbol{\cdot}\mathbf{v}$.
            But then:
            \begin{subequations}
                \begin{align}
                    (1_{1},\,1_{2})\boldsymbol{\cdot}\mathbf{u}
                    &=(1_{1},\,1_{2})\boldsymbol{\cdot}
                        \big[
                            (1_{1},\,0_{2})\boldsymbol{\cdot}\mathbf{v}
                        \big]\\
                    &=\big[(1_{1},\,1_{2})\cdot(1_{1},\,0_{2})\big]
                        \boldsymbol{\cdot}\mathbf{v}\\
                    &=(1_{1},\,0_{2})\boldsymbol{\cdot}\mathbf{v}\\
                    &=\mathbf{u}
                \end{align}
            \end{subequations}
            Associativity of ring elements with $M_{1}$ comes for free from
            the associativity of $M$, similarly for the distributive laws.
            Moreover, it is closed:
            \begin{equation}
                \mathbf{u}_{1}+\mathbf{u}_{2}=
                (1_{1},\,0_{2})\boldsymbol{\cdot}\mathbf{v}_{1}+
                (1_{1},\,0_{2})\boldsymbol{\cdot}\mathbf{v}_{2}
                =(1_{1},\,0_{2})\boldsymbol{\cdot}
                    \big[\mathbf{v}_{1}+\mathbf{v}_{2}\big]
            \end{equation}
            And this is an element of $M_{1}$. Similarly:
            \begin{equation}
                (a,\,b)\boldsymbol{\cdot}\mathbf{u}
                =(a,\,b)\boldsymbol{\cdot}
                    \big[
                        (1_{1},\,0_{2})\boldsymbol{\cdot}\mathbf{v}
                    \big]
                =\big[
                    (a,\,b)\cdot(1_{1},\,0_{2})
                \big]
                    \boldsymbol{\cdot}\mathbf{v}
                =(a,\,0_{2})\boldsymbol{\cdot}\mathbf{v}
            \end{equation}
            Which is again contained in $M_{1}$. Similarly, $M_{2}$ is an
            $R$ submodule. Moreover, $M=M_{1}\oplus{M}_{2}$ For if
            $\mathbf{v}\in{M}$, then:
            \begin{equation}
                \mathbf{v}=(1_{1},\,0_{2})\boldsymbol{\cdot}\mathbf{v}+
                           (0_{1},\,1_{2})\boldsymbol{\cdot}\mathbf{v}
            \end{equation}
            And the left part of the sum is an element of $M_{1}$ and the
            right part is an element of $M_{2}$. It also follows that
            $M_{1}$ is an $R_{1}$ module, viewing along pullback by the
            projection $\pi_{1}(x,y)=x$. That is, form the operation
            $*_{1}:R_{1}\times{M}_{1}\rightarrow{M}_{1}$ by:
            \begin{equation}
                r*_{1}\mathbf{m}=(r,\,0_{2})\boldsymbol{\cdot}\mathbf{m}
                \quad\quad
                r\in{R}_{1},\,\mathbf{m}\in{M}_{1}
            \end{equation}
            Similarly for $M_{2}$:
            \begin{equation}
                r*_{2}\mathbf{m}=(0_{1},\,r)\boldsymbol{\cdot}\mathbf{m}
                \quad\quad
                r\in{R}_{2},\,\mathbf{m}\in{M}_{2}
            \end{equation}
        \end{solution}
        \begin{problem}
            The equivalence class one.
        \end{problem}
        \begin{problem}
            Let $A$ and $B$ be rings and let $R=A\times{B}$ be the
            direct product ring. Show that $R$ contains elements $e_{1}$
            and $e_{2}$ such that:
            \begin{enumerate}
                \item Both $e_{i}$ are idempotent.
                \item Each $e_{i}$ is central.
                \item They are orthogonal: $e_{1}\cdot{e}_{2}=0_{R}$.
                \item They are complete: $e_{1}+e_{2}=1_{R}$.
            \end{enumerate}
            Conversely, show if $R$ is a ring with two such elements then
            there are rings $A$ and $B$ such that $R$ is isomorphic to
            $A\times{B}$.
        \end{problem}
        \begin{solution}
            For let $e_{1}$ and $e_{2}$ be defined as follows:
            \par\hfill\par
            \begin{subequations}
                \begin{minipage}[b]{0.49\textwidth}
                    \begin{equation}
                        e_{1}=(1_{A},\,0_{B})
                    \end{equation}
                \end{minipage}
                \hfill
                \begin{minipage}[b]{0.49\textwidth}
                    \begin{equation}
                        e_{2}=(0_{A},\,1_{B})
                    \end{equation}
                \end{minipage}
            \end{subequations}
            \par\vspace{2.5ex}
            Where $0_{\alpha}$ and $1_{\alpha}$ are the additive and
            multiplicative identities of their respective rings,
            respectively. Then:
            \begin{subequations}
                \begin{align}
                    e_{1}\cdot{e}_{1}
                    =(1_{A},\,0_{B})\cdot(1_{A},\,0_{B})
                    =(1_{A}*_{A}1_{A},\,0_{B}*_{B}0_{B})
                    =(1_{A},\,0_{B})
                    =e_{1}\\
                    e_{2}\cdot{e}_{2}
                    =(0_{A},\,1_{B})\cdot(0_{A},\,1_{B})
                    =(0_{A}*_{A}0_{A},\,1_{B}*_{B}1_{B})
                    =(0_{A},\,1_{B})
                    =e_{2}
                \end{align}
            \end{subequations}
            Thus satisfying idempotency. They are central, for:
            \begin{subequations}
                \begin{align}
                    e_{1}\cdot{r}
                    &=(1_{A},\,0_{B})\cdot(r_{1},\,r_{2})\\
                    &=(1_{A}*_{A}r_{1},\,0_{B}*_{B}r_{2})\\
                    &=(r_{1}*_{A}1_{A},\,r_{2}*_{B}0_{B})\\
                    &=(r_{1},\,r_{2})\cdot(1_{A},\,0_{B})\\
                    &=r\cdot{e}_{1}
                \end{align}
            \end{subequations}
            Similarly for $e_{2}$. They are indeed orthogonal:
            \begin{equation}
                e_{1}\cdot{e}_{2}
                =(1_{A},\,0_{B})\cdot(0_{A},\,1_{B})
                =(1_{A}*_{A}0_{A},\,0_{B}*_{B}1_{B})
                =(0_{A},\,0_{B})
                =0_{R}
            \end{equation}
            Completeness follows from:
            \begin{equation}
                e_{1}+e_{1}
                =(1_{A},\,0_{B})+(0_{A},\,1_{B})
                =(1_{A}+0_{A},\,0_{B}+1_{B})
                =(1_{A},\,1_{B})
                =1_{R}
            \end{equation}
            Conversely, suppose such elements $e_{1}$ and $e_{2}$ exist.
            Let $A$ and $B$ be defined by:
            \par\hfill\par
            \begin{subequations}
                \begin{minipage}[b]{0.49\textwidth}
                    \begin{equation}
                        A=\{\,e_{1}\cdot{r}\,:\,r\in{R}\}
                    \end{equation}
                \end{minipage}
                \hfill
                \begin{minipage}[b]{0.49\textwidth}
                    \begin{equation}
                        B=\{\,e_{2}\cdot{r}\,:\,r\in{R}\}
                    \end{equation}
                \end{minipage}
            \end{subequations}
            \par\vspace{2.5ex}
            Let $\varphi:R\rightarrow{A}\times{B}$ be defined by:
            \begin{equation}
                \varphi(r)=(r\cdot{e}_{1},\,r\cdot{e}_{2})
            \end{equation}
            Then $\varphi$ is an isomorphism. Firstly, it is injective.
            For if $\varphi(r_{1})=\varphi(r_{2})$, then:
            \begin{equation}
                r_{1}\cdot{e}_{1}=r_{2}\cdot{e}_{1}
                \quad\quad
                \textrm{and}
                \quad\quad
                r_{1}\cdot{e}_{2}=r_{2}\cdot{e_{2}}
            \end{equation}
            From the completeness, summing these we obtain $r_{1}=r_{2}$.
            It is also surjective, for if $(r_{1},r_{2})\in{A}\times{B}$
            let $r=r_{1}+r_{2}$. Since $r_{1}\in{A}$ there is a
            $u_{1}\in{R}$ such that $r_{1}=u_{1}\cdot{e}_{1}$, and similarly
            a $u_{2}$ for $r_{2}$. From orthogonality and idempotency,
            we have:
            \begin{subequations}
                \begin{align}
                    \varphi(r)
                    &=\big((r_{1}+r_{2})\cdot{e}_{1},\,
                           (r_{1}+r_{2})\cdot{e}_{2})\big)\\
                    &=\big(
                        (u_{1}\cdot{e}_{1}+u_{2}\cdot{e}_{2})\cdot{e}_{1},\,
                        (u_{1}\cdot{e}_{1}+u_{2}\cdot{e}_{2})\cdot{e}_{2})
                    \big)\\
                    &=(u_{1}\cdot{e}_{1}^{2}+
                       u_{2}\cdot{e}_{1}\cdot{e}_{2},\,
                       u_{1}\cdot{e}_{1}\cdot{e}_{2}+
                       u_{2}\cdot{e}_{2}^{2})\\
                     &=(u_{1}\cdot{e}_{1},\,u_{2}\cdot{e}_{2})\\
                     &=(r_{1},\,r_{2})
                \end{align}
            \end{subequations}
            And thus $\varphi$ is a bijection. Lastly, it is a homomorphism.
            For, by idempotency and centrality:
            \begin{subequations}
                \begin{align}
                    \varphi(r\cdot{s})
                    &=\big((r\cdot{s})\cdot{e}_{1},\,
                           (r\cdot{s})\cdot{e}_{2}\big)\\
                    &=\big((r\cdot{e}_{1})\cdot(s\cdot{e}_{1}),\,
                           (r\cdot{e}_{1})\cdot(s\cdot{e}_{1})\big)\\
                    &=(r\cdot{e}_{1},\,s\cdot{e}_{2})\boldsymbol{\cdot}
                      (s\cdot{e}_{1},\,s\cdot{e}_{2})\\
                    &=\varphi(r)\boldsymbol{\cdot}\varphi(s)
                \end{align}
            \end{subequations}
            Similarly for addition:
            \begin{subequations}
                \begin{align}
                    \varphi(r+s)
                    &=\big((r+s)\cdot{e}_{1},\,
                           (r+s)\cdot{e}_{2}\big)\\
                    &=(r\cdot{e}_{1}+r\cdot{e}_{1},\,
                       r\cdot{e}_{2}+s\cdot{e}_{2})\\
                    &=(r\cdot{e}_{1},\,r\cdot{e}_{2})\boldsymbol{+}
                      (s\cdot{e}_{1},\,s\cdot{e}_{2})\\
                    &=\varphi(r)\boldsymbol{+}\varphi(s)
                \end{align}
            \end{subequations}
            Thus, $\varphi$ is a bijective homomorphism and is therefore
            an isomorphism. It follows that $R$ is isomorphic to
            $A\times{B}$.
        \end{solution}
        \begin{problem}
            \par\hfill\par
            \begin{enumerate}
                \item   Let $A$ and $B$ be rings, and let
                        $\mathfrak{a}\subseteq{A}$ and
                        $\mathfrak{b}\subseteq{B}$ be left ideals.
                        Show that $\mathfrak{a}\times\mathfrak{b}$ is a
                        left ideal of $A\times{B}$.
                \item   Show that if $\mathfrak{c}$ is a left ideal of
                        $A\times{B}$ then there are left ideals
                        $\mathfrak{a}\subseteq{A}$ and
                        $\mathfrak{b}\subseteq{B}$ such that
                        $\mathfrak{c}=\mathfrak{a}\times\mathfrak{b}$.
                \item   Show that, if $R$ is a ring and $M_{1}$ and $M_{2}$
                        are left $R$ modules, then it may not be true that
                        every submodule of $M_{1}\times{M}_{2}$ is of the
                        form $N_{1}\times{N}_{2}$, where
                        $N_{1}\subseteq{M}_{2}$ and $N_{2}\subseteq{M}_{2}$.
            \end{enumerate}
        \end{problem}
        \begin{solution}
            For if $\mathfrak{a}$ and $\mathfrak{b}$ are left ideals of
            $A$ and $B$, respectively, let
            $r_{1},r_{2}\in\mathfrak{a}\times\mathfrak{b}$. Then:
            \begin{equation}
                r_{1}\boldsymbol{+}r_{2}
                =(a_{1},\,b_{1})\boldsymbol{+}(a_{2},\,b_{2})
                =(a_{1}+a_{2},\,b_{1}+b_{2})
            \end{equation}
            But $\mathfrak{a}$ and $\mathfrak{b}$ are left ideals, and
            thus $a_{1}+a_{2}\in\mathfrak{a}$ and
            $b_{1}+b_{2}\in\mathfrak{b}$, and thus
            $r_{1}+r_{2}\in\mathfrak{a}\times\mathfrak{b}$. Moreover, if
            $r\in{A}\times{B}$ and $s\in\mathfrak{a}\times\mathfrak{b}$,
            then:
            \begin{equation}
                r\boldsymbol{\cdot}{s}
                =(r_{1},\,r_{2})\boldsymbol{\cdot}(s_{1},\,s_{2})
                =(r_{1}\cdot{s}_{2},\,r_{2}\cdot{s}_{2})
            \end{equation}
            But $\mathfrak{a}$ is a left ideal, and thus
            $r_{1}\cdot{s}_{1}\in\mathfrak{a}$. Similarly, $\mathfrak{b}$
            is a left ideal and thus $r_{2}\cdot{s}_{2}\in\mathfrak{b}$.
            Thus, $r\boldsymbol{\cdot}s\in\mathfrak{a}\times\mathfrak{b}$.
            That is, $\mathfrak{a}\times\mathfrak{b}$ is a left ideal of
            $A\times{B}$.
            \par\hfill\par
            If $\mathfrak{c}\subseteq{A}\times{B}$ is a left ideal,
            let $\mathfrak{a}=\pi_{1}(\mathfrak{c})$ and
            $\mathfrak{b}=\pi_{2}(\mathfrak{c})$, where $\pi_{1}$ and
            $\pi_{2}$ are the projection mappings. Then $\mathfrak{a}$ is
            a left ideal of $A$. For if $r,s\in\mathfrak{a}$, then
            $(r,0_{B})$ and $(s,0_{B})$ are elements of $\mathfrak{c}$,
            and thus:
            \begin{equation}
                (r,\,0_{B})\boldsymbol{+}(s,\,0_{B})
                =(r+s,\,0_{B}+0_{B})
                =(r+s,\,0_{B})
            \end{equation}
            Which is an element of $\mathfrak{c}$ since $\mathfrak{c}$ is
            a left ideal, and therefore $r+s\in\mathfrak{a}$. If
            $r\in\mathfrak{a}$ and $s\in{A}$, then:
            \begin{equation}
                (r,\,0_{B})\boldsymbol{\cdot}(s,\,0_{B})
                =(r\cdot{s},\,0_{B})
            \end{equation}
            Which is an element of $\mathfrak{c}$ since $\mathfrak{c}$
            is a left ideal, and therefore $r\cdot{s}\in\mathfrak{a}$. Thus,
            $\mathfrak{a}$ is a left ideal of $A$. Similarly,
            $\mathfrak{b}$ is a left ideal of $B$. Moreover, since
            $\pi_{1}$ and $\pi_{2}$ are simply the projection mappings,
            it follows that:
            \begin{equation}
                \mathfrak{c}=\mathfrak{a}\times\mathfrak{b}
            \end{equation}
            Thus completing the proof.
            \par\hfill\par
            Lastly, let $R=\mathbb{R}$ with it's usual field structure.
            Since all fields are also rings, $\mathbb{R}$ has a ring
            structure. Consider then the vector space of $\mathbb{R}$ over
            itself. Since all vector spaces are also modules, we have that
            $\mathbb{R}$ is an $\mathbb{R}$ module. The product of
            $\mathbb{R}$ with itself will be $\mathbb{R}^{2}$ with its
            usual vector space structure. Any line through the origin will
            then be subspace, however the only subspaces of
            $\mathbb{R}$ are $0$ and the entire real line itself. Thus,
            the only possible product spaces are:
            \begin{equation*}
                \mathbf{0}
                \quad\quad
                \{\,(r,\,0)\,:\,r\in\mathbb{R}\,\}
                \quad\quad
                \{\,(0,\,r)\,:\,r\in\mathbb{R}\,\}
                \quad\quad
                \mathbb{R}^{2}
            \end{equation*}
            Thus the subspace $\{(r,r):r\in\mathbb{R}\}$ cannot be written
            as a product.
        \end{solution}
        \begin{problem}
            \par\hfill\par
            \begin{enumerate}
                \item   Let $R$ be a ring and $M$ be a left $R$ module.
                        Then the set $\textrm{Hom}_{R}(R,M)$ of $R$ module
                        homomorphisms can be given an Abelian group
                        structure by defining $f+g$ as the map
                        $(f+g)(x)=f(x)+g(x)$ for all $x\in{R}$. Show that
                        there is an isomorphism between this group and $M$.
                \item   If $R$ is a commutative ring, then
                        $\textrm{Hom}_{R}(R,M)$ has the structure of an
                        $R$ module give by: if $r\in{R}$ and if
                        $f\in\textrm{Hom}_{R}(R,M)$, define
                        $(rf)(x)=r\cdot{f(x)}$ for all $x\in{R}$. Show that
                        this makes $\textrm{Hom}_{R}(R,M)$ an $R$ module
                        and that $\varepsilon$ from above is an isomorphism
                        of $R$ modules. If $R$ is a field, what simple
                        observation is this?
                \item   If $R$ is a ring and $M=R$, so
                        $\textrm{Hom}_{R}(R,M)=\textrm{End}_{R}(R)$ is
                        now the set of endomorphisms on $R$ as a left
                        $R$ module. Then $\textrm{End}_{R}(R)$ is a ring
                        where the product is given by function
                        composition. Is $\varepsilon$ as defined before
                        a ring isomorphism? If not, how can we modify this
                        so that it is?
            \end{enumerate}
        \end{problem}
        \begin{solution}
            For let $\varepsilon:\textrm{Hom}_{R}(R,M)\rightarrow{M}$ be
            defined by:
            \begin{equation}
                \varepsilon(f)=f(1)
            \end{equation}
            This is injective, for if $\varepsilon(f)=\varepsilon(g)$,
            then $f(1)=g(1)$. But then, for all $x\in{R}$:
            \begin{equation}
                f(x)=f(x\cdot{1})=xf(1)=xg(1)=g(x\cdot{1})=g(x)
            \end{equation}
            And thus $f=g$. Moreover, it is surjective. For let
            $\mathbf{m}\in{M}$ and let $f:R\rightarrow{M}$ be defined by:
            \begin{equation}
                f(r)=r\cdot\mathbf{m}
            \end{equation}
            From the module structure of $M$, $f\in\textrm{Hom}_{R}(R,M)$,
            but also $\varepsilon(f)=f(1)=m$. Therefore, $f$ is a bijection.
            Lastly, $\varepsilon$ is indeed a group homomorphism. For if
            $f,g\in\textrm{Hom}_{R}(R,M)$, then:
            \begin{equation}
                \varepsilon(f+g)=(f+g)(1)=f(1)+g(1)
                                =\varepsilon(f)+\varepsilon(g)
            \end{equation}
            Thus, $\varepsilon$ is an isomorphism.
            \par\hfill\par
            This is indeed an $R$ module. For if
            $f\in\textrm{Hom}_{R}(R,M)$, then:
            \begin{equation}
                (1f)(x)=1\cdot{f}(x)=f(x)
            \end{equation}
            Since $M$ is an $R$ module. If $r,s\in{R}$ and
            $f\in\textrm{Hom}_{R}(R,M)$, then:
            \begin{equation}
                \big((rs)f\big)(x)
                =(rs)f(x)
                =r\big(sf(x)\big)
                =\big(r(sf)\big)(x)
            \end{equation}
            Again from the module structure of $M$. The distributive law
            is satisfied as well:
            \begin{equation}
                \big((r+s)f)(x)=(r+s)f(x)=rf(x)+sf(x)=(rf)(x)+(sf)(x)
            \end{equation}
            And lastly, if $r\in{R}$ and $u,v\in{M}$, then:
            \begin{subequations}
                \begin{align}
                    \big(r(u+v)\big)(x)
                    &=r\big((u+v)(x)\big)\\
                    &=r\big(u(x)+rv(x)\big)\\
                    &=ru(x)+rv(x)\\
                    &=(ru)(x)+(rv)(x)
                \end{align}
            \end{subequations}
            Moreover, $\varepsilon$ is an isomorphism. We have seen that
            it is bijective and preserves addition, we must now check
            that it preserve multiplication from the ring:
            \begin{equation}
                \varepsilon(rf)
                =(rf)(1)
                =r\cdot{f}(1)
                =r\cdot\varepsilon(f)
            \end{equation}
            \par\hfill\par
            For part 3, it is not, in general, an isomorphism, since:
            \begin{equation}
                \varepsilon(f\circ{g})
                =f(g(1))
                =f(g(1)\cdot{1})
                =g(1)f(1)
            \end{equation}
            And this may not, necessarily, be equal to $f(1)g(1)$ if
            $R$ is not commutative. If we define a new operation
            $*$ by:
            \begin{equation}
                r*s=s\cdot{r}
            \end{equation}
            Then $\varepsilon$ will be a ring isomorphism from
            $\textrm{End}_{R}(R)$ to $R$, since:
            \begin{equation}
                \varepsilon(f\circ{g})=
                f(g(1))=g(1)f(1)=f(1)*g(1)
                =\varepsilon(f)*\varepsilon(g)
            \end{equation}
        \end{solution}
        And thus $\varepsilon$ is a ring isomorphism.
    \begin{problem}
        A $k$ algebra is a $k$ module $A$ with a ring structure such
        that ring multiplication and scalar multiplication by elements
        of $k$ satisfy:
        \begin{equation}
            (\lambda\cdot{a})b=\lambda\cdot(ab)=a(\lambda\cdot{b})
        \end{equation}
        Show that $A$ is a $k$ algebra if and only if $A$ is a ring such
        that there exists a ring homomorphism $\varphi:k\rightarrow{Z}(A)$,
        where $Z(A)$ is the center of $A$.
    \end{problem}
    \begin{solution}
        Going one way, suppose $A$ is a ring with homomorphism
        $\varphi:k\rightarrow{Z}(A)$. Define $\cdot$ by:
        \begin{equation}
            \lambda\cdot{a}=\varphi(\lambda)a
        \end{equation}
        Then $A$ is a $k$ algebra. For:
        \begin{equation}
            (\lambda\cdot{a})b
            =(\varphi(\lambda)a)(b)
            =\varphi(\lambda)(ab)
            =\lambda\cdot(ab)
        \end{equation}
        And also:
        \begin{equation}
            (\lambda\cdot{a})b
            =(a\varphi(\lambda))b
            =a(\varphi(\lambda)b)
            =a(\lambda\cdot{b})
        \end{equation}
        From associativity and from the fact that $\varphi(\lambda)$ is
        a commutative element. Going the other way, if $A$ is a $k$ algebra,
        let $\varphi:k\rightarrow{Z}(A)$ be defined by:
        \begin{equation}
            \varphi(\lambda)=\lambda\cdot{1}
        \end{equation}
        Then:
        \begin{equation}
            \varphi(a+b)
            =(a+b)\cdot{1}
            =a\cdot{1}+b\cdot{1}
            =\varphi(a)+\varphi(b)
        \end{equation}
        And also:
        \begin{equation}
            \varphi(ab)
            =(ab)\cdot{1}
            =(a\cdot{1})b
            =(a\cdot{1})(b\cdot{1})
            =\varphi(a)\varphi(b)
        \end{equation}
        And thus $\varphi$ is a homomorphism.
    \end{solution}
\end{document}