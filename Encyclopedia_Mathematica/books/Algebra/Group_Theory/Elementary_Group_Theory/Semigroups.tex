\section{Semigroups}
    The bulk of group theory should discuss groups. A group is a set with a
    binary operation\index{Binary Operation} that satisfies three particular
    properties. Many theorems that can be proved about groups do no need all
    three of these properties and thus it becomes natural to generalize groups
    to slightly weaker structures. The first two objects to discuss are
    semigroups and monoids. Developing monoids is particularly useful for when
    we wish to develop rings, which is a set with two binary operations.
    \begin{fdefinition}{Semigroup}{Semigroup}
        A \gls{semigroup} is a set $G$ and an \gls{associative operation} $*$
        on $G$.
    \end{fdefinition}
    Associativity is usually a crucial operation to have, otherwise we've no
    idea how to combine three elements to get a fourth.
    \begin{fexample}{Example of a Semigroup}{Example_of_a_Semigroup}
        Let $X$ be a set with several distinct elements and let $\mathscr{F}$ be
        the set of all functions $f:X\rightarrow{X}$ to such that $f$ is a
        constant mapping. That is, there is some $c\in{X}$ such that for all
        $x\in{X}$ it is true that $f(x)=c$. In other words, the image of $X$ is
        $\{c\}$: $f(X)=\{c\}$. Let $\circ$ denote function composition. We know
        that function composition is associative. Moreover, $\circ$ takes
        elements of $\mathscr{F}$ to $\mathscr{F}$. For if $f,g\in\mathscr{F}$
        then there are $c_{f},c_{g}\in{X}$ such that $f(X)=\{c_{f}\}$ and
        $g(X)=\{c_{g}\}$. But then, for all $x\in{X}$ we have:
        \begin{equation}
            (g\circ{f})(x)=g(f(x))=g(c_{f})=c_{g}
        \end{equation}
        and thus $g\circ{f}$ is a constant mapping as well. Therefore $\circ$ is
        an associative binary operation on $\mathscr{F}$ and
        $(\mathscr{F},\circ)$ is a semigroup.
    \end{fexample}
    The example shown in Ex.~\ref{ex:Example_of_a_Semigroup} is missing most
    algebraic properties. Most notably, there is no identity element. For since
    we chose $X$ to have at least several distinct elements, for any two
    distinct functions $f,g\in\mathscr{F}$, we have that
    $g\circ{f}\ne{f}\circ{g}$, and thus there can be no unital element. There
    can also be no left or right unital element,
    and as such there can be no invertible elements. Moreover, as this previous
    expression shows, the operation is not commutative. Thus
    $(\mathscr{F},\circ)$ is a semigroup but can't possible be any of the nicer
    objects like monoids or groups. While such examples may be trivial, this
    does show that it may be worth while studying the structure of these weaker
    algebraic systems.