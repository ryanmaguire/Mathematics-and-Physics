\section{Groups}
    While monoids are structurally simpler than groups, the notion of a group is
    one of the most basic and fundamental algebraic structures that one can
    consider.
    \begin{fdefinition}{Group}{Group}
        A group is a \gls{monoid} $(G,*)$ such that for all $g\in{G}$, $g$ is an
        \gls{invertible element} with respect to $*$. That is:
        \begin{itemize}
            \item The binary operation $*$ is associative.
            \item There exists an identity element $e\in{G}$.
            \item For every element $a\in{A}$, there is an inverse element.
        \end{itemize}
    \end{fdefinition}
    Note that it is not necessarily true that $a*b=b*a$. Such groups are called
    Abelian. There are several examples of groups that one is likely familiar
    with.
    \begin{example}
        If $+$ denotes the usual addition on $\mathbb{Z}$, then
        $(\mathbb{Z},+)$ is a group. The unital element is 0, and for all
        $n\in\mathbb{Z}$, $\minus{n}$ is an inverse element of $n$. That is,
        $n+(\minus{n})=0$, which is a unital element. Moreover, addition is
        associative and therefore $(\mathbb{Z},+)$ is a group.
    \end{example}
    \begin{example}
        If $\mathbb{R}^{+}$ denotes the positive real numbers, and if $\cdot$
        denotes the usual multiplication of real numbers, then
        $(\mathbb{R},\cdot)$ is a group. For this group, 1 is the unital element
        for all $r\in\mathbb{R}^{+}$, $1/r$ is the inverse element. Lastly,
        the operation is indeed associative.
    \end{example}
    \begin{example}
        Consider $\mathbb{Z}_{n}$ with modular addition $+$. Then
        $(\mathbb{Z}_{n},+)$ is a group. The unital element is $0$ since for all
        $k\in\mathbb{Z}_{n}$, $k+0=k$. The inverse of $k\in\mathbb{Z}_{n}$ is
        the unique integer $m\in\mathbb{Z}_{n}$ such that $k+m=n$. That is,
        $m=n-k$. For then $k+m=n$, and $n$ is equivalent to zero in
        $\mathbb{Z}_{n}$. Since modular addition is commutative, it follows that
        $0$ is a left and right unital element and that each element is left and
        right invertible, and thus $(\mathbb{Z}_{n},+)$ is a group.
    \end{example}
    \begin{example}
        Consider $\mathbb{Z}_{4}$ with it's modular addition $+$. Then
        $(\mathbb{Z}_{4},+)$ is a group. We can represent the operation $+$ with
        the following table:
        \begin{table}[H]
            \centering
            \captionsetup{type=table}
            \begin{tabular}{c|cccc}
                $+$&0&1&2&3\\
                \hline
                0&0&1&2&3\\
                1&1&2&3&0\\
                2&2&3&0&1\\
                3&3&0&1&2
            \end{tabular}
            \caption{The Group Structure of $\mathbb{Z}_{4}$}
        \end{table}
    \end{example}
    \begin{theorem}
        If $G$ is a set, if $*$ is an associative binary operation on $G$, if
        $e\in{G}$ is a unique right unital element, and if for all $g\in{G}$ it
        is true that $g$ is weakly right invertible, then $G$ is a group.
    \end{theorem}
    \begin{proof}
        For by
        Thm.~\ref{thm:existence_of_weak_left_and_weak_right_implies_unital},
        $e$ is a unital element. And by
        Thm.~\ref{thm:unqiue_right_unit_and_weak_r_inv_implies_inv}, for all
        $g\in{G}$ it is true that $g$ is invertible. Therefore, $(G,*)$ is a
        group (Def.~\ref{def:Group}).
    \end{proof}
    Thus it suffices to check that there is a unique right identity and that all
    elements are weakly right intertible. The unital element of a group is
    unique (Thm.~\ref{thm:Unital_Elements_are_Unique}), as are the inverses of
    elements (Thm.~\ref{thm:Assoc_Op_Inverses_are_Unique}). Moreover, the
    inverse of a unital element is itself
    (Thm.~\ref{thm:Unital_Elements_Are_Invertible}).
    \begin{theorem}
        \label{thm:Group_Inverse_of_Product}%
        If $(G,*)$ is a group and $a,b\in G$, then:
        \begin{equation}
            (a*b)^{\minus{1}}=b^{\minus{1}}*a^{\minus{1}}
        \end{equation}
    \end{theorem}
    \begin{proof}
        For if $(G,*)$ is a group, and if $a,b\in{G}$, then $*$ is associative
        and $a$ and $b$ are invertible (Def.~\ref{def:Group}). But then by
        Thm.~\ref{thm:assoc_op_prod_of_inv_is_inv},
        $a*b=b^{\minus{1}}*a^{\minus{1}}$.
    \end{proof}
    \begin{theorem}
        If $(G,*)$ is a group and $a\in{G}$, then:
        \begin{equation}
            (a^{\minus{1}})^{\minus{1}}=a
        \end{equation}
    \end{theorem}
    \begin{proof}
        For if $e$ is the unital element of $G$, then:
        \begin{align}
            a^{\minus{1}}*(a^{\minus{1}})^{\minus{1}}
            &=(a^{\minus{1}}* a)^{\minus{1}}
            \tag{Thm.~\ref{thm:Group_Inverse_of_Product}}\\
            &=e
            \tag{Inverse Property}
        \end{align}
        From uniqueness, $(a^{\minus{1}})^{\minus{1}}=a$.
    \end{proof}
    \begin{fdefinition}{Abelian Group}{Abelian_Group}
        An \gls{Abelian group} is a \gls{group} $(G,*)$ such that $*$ is
        a \gls{commutative operation}.
    \end{fdefinition}
    \begin{lexample}{The Dihedral Group $D_{6}$}{Dihedral_Group_D_6}
        Not every group is Abelian, and a classic non-Abelian group is the group
        of symmetries on an equilateral triangle. This is the dihedral group
        $D_{6}$. It is formed by considering all of the distinct ways one can
        rearrange the three points on an equilateral triangle by means of
        rotation by $60^{\circ}$ and by reflection across the $y$ axis, as well
        as any combination of these two (see Fig.~\ref{fig:Dihedral_Group_D_6}).
        \begin{figure}[H]
            \centering
            \captionsetup{type=figure}
            \begin{tikzpicture}[>=Latex]
                \coordinate (A) at ( 0.0,     1.0);
                \coordinate (B) at ( 1.1547, -1.0);
                \coordinate (C) at (-1.1547, -1.0);
    
                \draw[fill=Apricot,opacity=0.8] (A) to (B) to (C) to cycle;
    
                \draw[densely dashed, draw=red] (-2, -1.4880) to (2,  0.8213);
                \draw[densely dashed, draw=red] (-2,  0.8213) to (2, -1.4880);
                \draw[densely dashed, draw=red] ( 0, -1.8000) to (0,  1.8000);
    
                \draw[fill=black] (A) circle (0.1);
                \draw[fill=black] (B) circle (0.1);
                \draw[fill=black] (C) circle (0.1);
            \end{tikzpicture}
            \caption{The Dihedral Group $D_{6}$}
            \label{fig:Dihedral_Group_D_6}
        \end{figure}
        It turns out there are 6 distinct such
        moves, but by considering $*$ to be the \textit{successor} operation,
        $(D_{6},*)$ forms a group. The successor operation means that if $r$
        denotes rotation and $a$ denotes reflection, then $r*a$ denotes
        \textit{rotate and then reflect}. By studying the triangle we get the
        following \textit{Cayley table} of this operation. A few things to note,
        the identity of our group is the \textit{do nothing} symmetry. That is,
        we neither rotate nor reflect. Also note that reflecting twice in a row
        or rotating three times in a row is equivalent to doing nothing. The
        last thing to note is that reflection, rotation, then reflecting again
        is the same as rotating \textit{backwards}. In other words, we have the
        following constraints:
        \begin{equation}
            r^{3}=e
            \quad\quad
            a^{2}=e
            \quad\quad
            a*r*a=r^{\minus{1}}
        \end{equation}
        This tells us that rotation and reflection has an inverse notion
        (rotate backwards and reflect again, respectively). Since the group is
        determined by these two operations, we need only check that
        $r*(ar)=(r*a)*r$ and $a*(r*a)=(a*r)*a$ to determine the associative of
        the rest of the group. We can do this by examing the triangle
        (see Fig.~\ref{fig:Assoc_of_Dihedral_Group_D6}). That is, if we
        rotate, and the follow by reflecting and then rotating, it's the same
        thing as rotating and then reflecting, following by rotating again.
        Similarly, if we reflect, and then follow by rotating and then
        reflecting, this is equivalence to reflecting and then rotating, and
        then following with another reflection.
        \begin{figure}[H]
            \centering
            \captionsetup{type=figure}
            \resizebox{\textwidth}{!}{%
                \begin{tikzpicture}[%
    >=Latex,
    p_arc/.style args={#1:#2:#3}{
        insert path={+ (#1:#3) arc (#1:#2:#3)},->
    },
    semithick
]
    \newcommand*{\defcoords}{%
        \coordinate (O) at ( 0.0,    -0.333333);
        \coordinate (A) at ( 0.0,     1.0);
        \coordinate (B) at ( 1.1547, -1.0);
        \coordinate (C) at (-1.1547, -1.0);
    }

    \begin{scope}[xshift=-7.0cm]
        \defcoords;
        \draw (A) to (B) to (C) to cycle;
        \draw (O) [p_arc=80:-20:1.333];
        \draw (O) [p_arc=-40:-140:1.333];
        \draw (O) [p_arc=200:100:1.333];
        \node at (A) [above]       {$A$};
        \node at (B) [below right] {$B$};
        \node at (C) [below left]  {$C$};
    \end{scope}

    \begin{scope}[xshift=-2.333cm]
        \defcoords;
        \draw (A) to (B) to (C) to cycle;
        \draw (O) [p_arc=-140:-40:1.333];
        \draw (0,-1.6) [p_arc=40:140:1.2];
        \draw[densely dashed,thin,red] (0, 1) to (0,-1.0);
        \node at (A) [above]       {$C$};
        \node at (B) [below right] {$A$};
        \node at (C) [below left]  {$B$};
    \end{scope}

    \begin{scope}[xshift=2.333cm]
        \defcoords;
        \draw (A) to (B) to (C) to cycle;
        \draw (O) [p_arc=80:-20:1.333];
        \draw (O) [p_arc=-40:-140:1.333];
        \draw (O) [p_arc=200:100:1.333];
        \node at (A) [above]       {$C$};
        \node at (B) [below right] {$B$};
        \node at (C) [below left]  {$A$};
    \end{scope}
    \begin{scope}[xshift=7.0cm]
        \defcoords;
        \draw (A) to (B) to (C) to cycle;
        \node at (A) [above]       {$C$};
        \node at (B) [below right] {$A$};
        \node at (C) [below left]  {$B$};
    \end{scope}
    \draw[draw=blue,->,thick] (-5.16, 0) to node[above] {$r$} (-4.16, 0);
    \draw[draw=blue,->,thick] (-0.50, 0) to node[above] {$a$} ( 0.50, 0);
    \draw[draw=blue,->,thick] ( 4.16, 0) to node[above] {$r$} ( 5.16, 0);

    \begin{scope}[yshift=-4cm]
        \begin{scope}[xshift=-7.0cm]
            \defcoords;
            \draw (A) to (B) to (C) to cycle;
            \draw (O) [p_arc=-140:-40:1.333];
            \draw (0,-1.6) [p_arc=40:140:1.2];
            \draw[densely dashed,thin,red] (0, 1) to (0,-1.0);
            \node at (A) [above]       {$A$};
            \node at (B) [below right] {$B$};
            \node at (C) [below left]  {$C$};
        \end{scope}
    
        \begin{scope}[xshift=-2.333cm]
            \defcoords;
            \draw (A) to (B) to (C) to cycle;
            \draw (O) [p_arc=80:-20:1.333];
            \draw (O) [p_arc=-40:-140:1.333];
            \draw (O) [p_arc=200:100:1.333];
            \node at (A) [above]       {$A$};
            \node at (B) [below right] {$C$};
            \node at (C) [below left]  {$B$};
        \end{scope}
    
        \begin{scope}[xshift=2.333cm]
            \defcoords;
            \draw (A) to (B) to (C) to cycle;
            \draw (O) [p_arc=-140:-40:1.333];
            \draw (0,-1.6) [p_arc=40:140:1.2];
            \draw[densely dashed,thin,red] (0, 1) to (0,-1.0);
            \node at (A) [above]       {$B$};
            \node at (B) [below right] {$A$};
            \node at (C) [below left]  {$C$};
        \end{scope}

        \begin{scope}[xshift=7.0cm]
            \defcoords;
            \draw (A) to (B) to (C) to cycle;
            \node at (A) [above]       {$B$};
            \node at (B) [below right] {$C$};
            \node at (C) [below left]  {$A$};
        \end{scope}
        \draw[draw=blue,->,thick] (-5.16, 0) to node[above] {$a$} (-4.16, 0);
        \draw[draw=blue,->,thick] (-0.50, 0) to node[above] {$r$} ( 0.50, 0);
        \draw[draw=blue,->,thick] ( 4.16, 0) to node[above] {$a$} ( 5.16, 0);
    \end{scope}
    \let\defcoords\undefined
\end{tikzpicture}
            }
            \caption{Associativity of the Dihedral Group $D_{6}$}
            \label{fig:Assoc_of_Dihedral_Group_D6}
        \end{figure}
        With this we can compute the table.
        \begin{table}[H]
            \centering
            \captionsetup{type=table}
            \begin{tabular}{c|cccccc}
                $*$&$e$&$r$&$r^{2}$&$a$&$a*r$&$a*r^{2}$\\
                \hline
                $e$&$e$&$r$&$r^{2}$&$a$&$a*r$&$a*r^{2}$\\
                $r$&$r$&$r^{2}$&$e$&$a*r^{2}$&$a$&$a*r$\\
                $r^{2}$&$r^{2}$&$e$&$r$&$a*r$&$a*r^{2}$&$a$\\
                $a$&$a$&$a*r$&$a*r^{2}$&$e$&$r$&$r^{2}$\\
                $a*r$&$a*r$&$a*r^{2}$&$a$&$r^{2}$&$e$&$r$\\
                $a*r^{2}$&$a*r^{2}$&$a$&$a*r$&$r$&$r^{2}$&$e$
            \end{tabular}
            \caption{Cayley Table of $D_{6}$}
            \label{tab:Cayley_Table_D_6}
        \end{table}
    \end{lexample}
    \begin{definition}
        If $(G,*)$ and $(G',\circ)$ are groups and
        $f:G\rightarrow G'$ is a bijective function, then $f$ is said to
        be an isomorphism between $(G,*)$ and $(G',\circ)$ if and only if
        for all $a,b\in{G}$, $f(a*b)=f(a)\circ{f}(b)$.
    \end{definition}
    \begin{theorem}
        If $(G,*)$ and $(G',\circ)$ are isomorphic with identities $e_{*}$
        and $e_{\circ}$ are the identities, then $f(e_{*})=e_{\circ}$.
    \end{theorem}
    \begin{proof}
        $\forall a\in G,\ f(a)=f(a* e_*) = f(a)\circ f(e_*)$ as $f$ is
        an isomorphism. As identities are unique, $f(e_*)=e_{\circ}$.
    \end{proof}
    \begin{theorem}
        If $(G,*)$ and $(G',\circ)$ are isomorphic, with isomorphism $f$, and if
        $a\in{G}$, then $f(a^{\minus{1}})=f(a)^{\minus{1}}$.
    \end{theorem}
    \begin{proof}
        For:
        \begin{equation}
            e_{\circ}=f(e_*)
                     =f(a*a^{-1})
                     =f(a^{-1}*a)
                     =f(a)\circ f(a^{-1})
                     =f(a^{-1})\circ f(a)
        \end{equation}
        As inverses are unique, $f(a^{-1})=f(a)^{-1}$.
    \end{proof}
    \begin{example}
        $G=\{1\}$ is an Abelian group under multiplication.
        This is the trivial group.
    \end{example}
    \begin{theorem}
        If $(G,*)$ is a group, then the following are true:
        \begin{enumerate}
            \item If $a*b=a*c$, then $b=c$.
            \item If $b*a=c*a$, then $b=c$.
            \item $\forall_{{a,b}\in{G}}\exists_{{x}\in{G}}:a*x=b$
        \end{enumerate}
    \end{theorem}
    \begin{definition}
        The order of a group is number of elements in the
        group.
    \end{definition}
    \begin{definition}
        The direct product of two groups $(G,*)$ and
        $(H,\circ)$ is the group  $({G}\times{H},\star)$
        where $\star$ is the binary operation defined by
        $(g_{1},h_{1})\star(g_{2},h_{2})%
         =(g_{1}*g_{2},{h_{1}}\circ{h_{2}})$
    \end{definition}
    \begin{definition}
        A permutation group on $n$ elements is a
        group whose elements are permutations of
        $n$ elements.
    \end{definition}
    \begin{definition}
        The symmetric group on $n$ elements,
        denoted $S_{n}$, is the group formed by
        permuting $n$ elements.
    \end{definition}
    \begin{definition}
        A homomorphism from a group $(G,*)$ to
        a group $(H,\circ)$ is a function
        $h:{G}\rightarrow{H}$ such that for all
        ${a,b}\in{G}$, $h(a*b)={h(a)}\circ{h(b)}$
    \end{definition}
    \begin{definition}
        An epimorphism from a group $(G,*)$ to
        a group $(H,\circ)$ is a homomorphism
        $h:{G}\rightarrow{H}$ such that
        $h$ is surjective.
    \end{definition}
    \begin{definition}
        A monomorphism from a group $(G,*)$ to
        a group $(H,\circ)$ is a homomorphism
        $h:{G}\rightarrow{H}$ such that
        $h$ is injective.
    \end{definition}
    \begin{definition}
        An isomorphism from a group $(G,*)$ to
        a group $(H,\circ)$ is a homomorphism
        $h:{G}\rightarrow{H}$ such that
        $h$ is bijective.
    \end{definition}
    \begin{definition}
        A ring is a set $R$ and two binary operations
        on $R$, denoted $(R,\cdot,+)$, such that:
        \begin{enumerate}
            \item $(R,+)$ is an Abelian group.
            \item $a\cdot({b}\cdot{c})=({a}\cdot{b})\cdot{c}$
            \item ${a}\cdot(b+c)={a}\cdot{b}+{a}\cdot{c}$
        \end{enumerate}
    \end{definition}
    \begin{definition}
        A ring with identity is a ring $(R,\cdot,+)$
        such that there is a ${1}\in{R}$ such that for
        all ${a}\in{R}$, ${a}\cdot{1}={1}\cdot{a}=a$.
    \end{definition}
    Left and right identities are elements such that ${e_{L}}\cdot{a}=a$
    and ${e_{R}}\cdot{a}=a$. If inverses $a_{L}^{-1}$ and $a_{R}^{-1}$
    exist for $a$, then $a_{L}^{-1}=a_{R}^{-1}$. That is, the inverse
    is the same for both right and left identities.
    \begin{definition}
        A commutative ring is a ring $(R,\cdot,+)$ such that
        $\cdot$ is commutative.
    \end{definition}
    \begin{definition}
        A commutative ring with identity is a ring with identity such
        that $\cdot$ is commutative.
    \end{definition}