\section{Groups}
    While monoids are structurally simpler than groups, the notion of a group is
    one of the most basic and fundamental algebraic structures that one can
    consider.
    \begin{fdefinition}{Group}{Group}
        A group is a \gls{monoid} $(G,*)$ such that for all $g\in{G}$, $g$ is an
        \gls{invertible element} with respect to $*$. That is:
        \begin{itemize}
            \item The binary operation $*$ is associative.
            \item There exists an identity element $e\in{G}$.
            \item For every element $a\in{A}$, there is an inverse element.
        \end{itemize}
    \end{fdefinition}
    Note that it is not necessarily true that $a*b=b*a$. Such groups are called
    Abelian. There are several examples of groups that one is likely familiar
    with.
    \begin{example}
        If $+$ denotes the usual addition on $\mathbb{Z}$, then
        $(\mathbb{Z},+)$ is a group. The unital element is 0, and for all
        $n\in\mathbb{Z}$, $\minus{n}$ is an inverse element of $n$. That is,
        $n+(\minus{n})=0$, which is a unital element. Moreover, addition is
        associative and therefore $(\mathbb{Z},+)$ is a group.
    \end{example}
    \begin{example}
        If $\mathbb{R}^{+}$ denotes the positive real numbers, and if $\cdot$
        denotes the usual multiplication of real numbers, then
        $(\mathbb{R},\cdot)$ is a group. For this group, 1 is the unital element
        for for all $r\in\mathbb{R}^{+}$, $1/r$ is the inverse element. Lastly,
        the operation is indeed associative.
    \end{example}
    An introductory chapter on groups usually shows that the unital
    element is unique and that so are inverses. However, we've already proven in
    much more generality that a unital element of a binary operation is
    unique (Thm.~\ref{Unital_Elements_are_Unique}) and that for any associative
    operation $*$, invertible element have unique inverses
    (Thm.~\ref{thm:Assoc_Op_Inverses_are_Unique}), and as such we won't prove
    these again. Similarly we've seen that unital elements are invertible and
    that they are their own inverses. One thing we've yet to prove is how to
    compute the inverse of a product. To do such a thing requires an
    associative operation that has inverses, and so we must consider the
    structure of a group to answer such a question.
    \begin{theorem}
        \label{thm:Group_Inverse_of_Product}%
        If $(G,*)$ is a group and $a,b\in G$, then:
        \begin{equation}
            (a*b)^{\minus{1}}=b^{\minus{1}}*a^{\minus{1}}
        \end{equation}
    \end{theorem}
    \begin{proof}
        For:
        \begin{align}
            (a*b)*(b^{\minus{1}}*a^{\minus{1}})
                &=a*(b*b^{\minus{1}})*a^{\minus{1}}
            \tag{Associativity}\\
            &=a*(e)*a^{-1}
            \tag{Inverse}\\
            &=a*a^{-1}
            \tag{Identity}\\
            &=e
            \tag{Inverse}
        \end{align}
        Thus $b^{-1}*a^{-1}$ is a right-inverse of $a*b$. But since $(G,*)$ is
        a group, right-inverses are left-inverses, and therefore
        $b^{-1}*a^{-1}$ is an inverse of $a*b$. But inverses are unique
        (Thm.~\ref{thm:Group_Inverses_Are_Unique}). Thus,
        $(a*b)^{\minus{1}}=b^{-1}*a^{-1}$.
    \end{proof}
    \begin{theorem}
        If $(G,*)$ is a group and $a\in{G}$, then:
        \begin{equation}
            (a^{-1})^{-1}=a
        \end{equation}
    \end{theorem}
    \begin{proof}
        For:
        \begin{align}
            a^{-1}*(a^{-1})^{-1}
            &=(a^{-1}* a)^{-1}
            \tag{Thm.~\ref{thm:Group_Theory_Inverse_of_Product}}\\
            &=e
            \tag{Inverse Property}
        \end{align}
        From uniqueness, $(a^{-1})^{-1}=a$.
    \end{proof}
    \begin{definition}
        If $(G,*)$ and $(G',\circ)$ are groups and
        $f:G\rightarrow G'$ is a bijective function, then $f$ is said to
        be an isomorphism between $(G,*)$ and $(G',\circ)$ if and only if
        for all $a,b\in{G}$, $f(a*b)=f(a)\circ{f}(b)$.
    \end{definition}
    \begin{theorem}
        If $(G,*)$ and $(G',\circ)$ are isomorphic with identities $e_{*}$
        and $e_{\circ}$ are the identities, then $f(e_{*})=e_{\circ}$.
    \end{theorem}
    \begin{proof}
        $\forall a\in G,\ f(a)=f(a* e_*) = f(a)\circ f(e_*)$ as $f$ is
        an isomorphism. As identities are unique, $f(e_*)=e_{\circ}$.
    \end{proof}
    \begin{theorem}
        If $(G,*)$ and $(G',\circ)$ are isomorphic, with isomorphism $f$, and if
        $a\in{G}$, then $f(a^{\minus{1}})=f(a)^{\minus{1}}$.
    \end{theorem}
    \begin{proof}
        For:
        \begin{equation}
            e_{\circ}=f(e_*)
                     =f(a*a^{-1})
                     =f(a^{-1}*a)
                     =f(a)\circ f(a^{-1})
                     =f(a^{-1})\circ f(a)
        \end{equation}
        As inverses are unique, $f(a^{-1})=f(a)^{-1}$.
    \end{proof}
    \begin{definition}
    A binary operation $*$ on a set $A$ is said to be commutative if and only for all $a,b\in A$, $a*b = b*a$.
    \end{definition}
    \begin{fdefinition}{Abelian Group}{Abelian_Group}
        An \gls{Abelian group} is a \gls{group} $(G,*)$ such that $*$ is
        a \gls{commutative operation}.
    \end{fdefinition}
    \begin{example}
        $G=\{1\}$ is an Abelian group under multiplication.
        This is the trivial group.
    \end{example}
    \begin{theorem}
        If $(G,*)$ is a group, then the following are true:
        \begin{enumerate}
            \item If $a*b=a*c$, then $b=c$.
            \item If $b*a=c*a$, then $b=c$.
            \item $\forall_{{a,b}\in{G}}\exists_{{x}\in{G}}:a*x=b$
        \end{enumerate}
    \end{theorem}
    \begin{definition}
        The order of a group is number of elements in the
        group.
    \end{definition}
    \begin{definition}
        The direct product of two groups $(G,*)$ and
        $(H,\circ)$ is the group  $({G}\times{H},\star)$
        where $\star$ is the binary operation defined by
        $(g_{1},h_{1})\star(g_{2},h_{2})%
         =(g_{1}*g_{2},{h_{1}}\circ{h_{2}})$
    \end{definition}
    \begin{definition}
        A permutation group on $n$ elements is a
        group whose elements are permutations of
        $n$ elements.
    \end{definition}
    \begin{definition}
        The symmetric group on $n$ elements,
        denoted $S_{n}$, is the group formed by
        permuting $n$ elements.
    \end{definition}
    \begin{definition}
        A homomorphism from a group $(G,*)$ to
        a group $(H,\circ)$ is a function
        $h:{G}\rightarrow{H}$ such that for all
        ${a,b}\in{G}$, $h(a*b)={h(a)}\circ{h(b)}$
    \end{definition}
    \begin{definition}
        An epimorphism from a group $(G,*)$ to
        a group $(H,\circ)$ is a homomorphism
        $h:{G}\rightarrow{H}$ such that
        $h$ is surjective.
    \end{definition}
    \begin{definition}
        A monomorphism from a group $(G,*)$ to
        a group $(H,\circ)$ is a homomorphism
        $h:{G}\rightarrow{H}$ such that
        $h$ is injective.
    \end{definition}
    \begin{definition}
        An isomorphism from a group $(G,*)$ to
        a group $(H,\circ)$ is a homomorphism
        $h:{G}\rightarrow{H}$ such that
        $h$ is bijective.
    \end{definition}
    \begin{definition}
        A ring is a set $R$ and two binary operations
        on $R$, denoted $(R,\cdot,+)$, such that:
        \begin{enumerate}
            \item $(R,+)$ is an Abelian group.
            \item $a\cdot({b}\cdot{c})=({a}\cdot{b})\cdot{c}$
            \item ${a}\cdot(b+c)={a}\cdot{b}+{a}\cdot{c}$
        \end{enumerate}
    \end{definition}
    \begin{definition}
        A ring with identity is a ring $(R,\cdot,+)$
        such that there is a ${1}\in{R}$ such that for
        all ${a}\in{R}$, ${a}\cdot{1}={1}\cdot{a}=a$.
    \end{definition}
    Left and right identities are elements such that ${e_{L}}\cdot{a}=a$
    and ${e_{R}}\cdot{a}=a$. If inverses $a_{L}^{-1}$ and $a_{R}^{-1}$
    exist for $a$, then $a_{L}^{-1}=a_{R}^{-1}$. That is, the inverse
    is the same for both right and left identities.
    \begin{definition}
        A commutative ring is a ring $(R,\cdot,+)$ such that
        $\cdot$ is commutative.
    \end{definition}
    \begin{definition}
        A commutative ring with identity is a ring with identity such
        that $\cdot$ is commutative.
    \end{definition}