%------------------------------------------------------------------------------%
\section{Definitions}
    \begin{fdefinition}{Modules}{Module}
        A module on a ring $(R,\,+,\,\cdot\,)$ is a set $M$ with a binary
        operation $\boldsymbol{+}$ and a function
        $\star:R\times{M}\rightarrow{M}$ such that $(M,\boldsymbol{+})$ is an
        Abelian group and such that, for all $r_{1},r_{2}\in{R}$ and for all
        $m_{1},m_{2}\in{M}$, the following are true:
        \begin{align}
            r_{1}\star(m_{1}\boldsymbol{+}m_{2}
            &=(r_{1}\star{m}_{1})\boldsymbol{+}(r_{1}\star{m}_{2})
            \tag{Scalar Distributivity}\\
            (r_{1}+r_{2})\star{m}_{1}
            &=(r_{1}\star{m}_{1})\boldsymbol{+}(r_{2}\star{m}_{1})
            \tag{Module Distributivity}\\
            r_{1}\star(r_{2}\star{m}_{1})
            &=(r_{1}\cdot{r}_{2})\star{m}_{1}
            \tag{Associativity}\\
            1\star{m}_{1}&=m_{1}
            \tag{Identity}
        \end{align}
        Where 1 is the unital element of $R$. We denote a module by
        $(M,\,\boldsymbol{+},\,\star)$.
    \end{fdefinition}
    If $(R,\,+,\,\cdot\,)$ is a field, and if $(M,\,\boldsymbol{+},\,\star)$
    is a module over this field, then it will also be a vector space. That
    is, modules are the ring-analog of vector spaces.
    \begin{fdefinition}{Ring Homomorphism}{Ring_Homomorphism}
        A \gls{ring homomorphism} from a \gls{ring} $(R_{1},\,+,\,\cdot\,)$
        to a ring $(R_{2},\,+',\,*\,)$ is a \gls{function}
        $f:R_{1}\rightarrow{R}_{2}$ such that, for all $x,y\in{R}_{1}$, the
        following are true:
        \begin{align}
            f(x+y)&=f(x)+'f(y)
            \tag{Preservation of Addition}\\
            f(x\cdot{y})&=f(x)*f(y)
            \tag{Preservation of Multiplication}\\
            f(1_{R_{1}})&=1_{R_{2}}
            \tag{Preservation of Identities}
        \end{align}
        Where $1_{R_{1}}$ is the unital element of $R_{1}$ and
        $1_{R_{2}}$ is the unital element of $R_{2}$.
    \end{fdefinition}
    There's a special name for a homomorphism from a ring
    $(R,\,+,\,\cdot\,)$ to itself.
    \begin{fdefinition}{Ring Endomorphisms}{Ring_Endomorphisms}
        A \gls{ring endomorphism} on a \gls{ring} $(R,\,+,\,\cdot\,)$ is a
        \gls{ring homomorphism} from $(R,\,+,\,\cdot\,)$ to itself. That is,
        a ring homomorphism $f:R\rightarrow{R}$.
    \end{fdefinition}
    \begin{fnotation}{Set of Ring Endomorphisms}{Ring_Endomorphisms}
        The set of ring endomorphisms on a ring $\mathcal{R}=(R,\,+,\,\cdot)$
        is denoted $\textrm{End}(\mathcal{R})$.
    \end{fnotation}
    \begin{theorem}
        If $(M,\,\boldsymbol{+},\star)$ is a module over a ring
        $(R,\,+,\,\cdot\,)$, if $\circ$ denotes function composition,
        and if $\boldsymbol{+}'$ denotes function addition:
        \begin{equation}
            (f+'g)(x)=f(x)\boldsymbol{+}g(x)
        \end{equation}
        Then $(\textrm{End}(M,\boldsymbol{+},\star),\boldsymbol{+}',\circ)$
        is a ring.
    \end{theorem}
    \begin{theorem}
        \label{thm:Scalar_Mult_in_Module_Defines_Group_Endo}%
        If $\mathcal{M}=(M,\,\boldsymbol{+},\,\star)$ is a module over a ring
        $(R,\,+,\,\cdot\,)$ and if $r\in{R}$, then the function
        $f:M\rightarrow{M}$ defined by:
        \begin{equation}
            f(m)=r\star{m}
        \end{equation}
        is a group endomorphism on the group $(M,\boldsymbol{+})$.
    \end{theorem}
    \begin{proof}
        For let $m_{1},m_{2}\in{M}$. But since $(M,\,\boldsymbol{+},\star)$
        is a module, $\star$ left-distributes over $\boldsymbol{+}$
        (Def~\ref{def:Module}). Therefore:
        \begin{equation}
            f(m_{1}\boldsymbol{+}m_{2})
            =r\star(m_{1}\boldsymbol{+}m_{2})
            =(r\star{m}_{1})\boldsymbol{+}(r\star{m}_{2})
            =f(m_{1})\boldsymbol{+}f(m_{2})
        \end{equation}
        Thus, $f$ is a group endomorphism on $(M,\boldsymbol{+})$.
    \end{proof}
    \begin{theorem}
        If $(M,\,\boldsymbol{+},\,\star)$ is a module over a ring
        $(R,\,+,\,\cdot\,)$, then there exists a ring homomorphism
        $\varphi:R\rightarrow\textrm{End}(M,\,\boldsymbol{+},\star)$.
    \end{theorem}
    \begin{proof}
        For let $\varphi:R\rightarrow\textrm{End}(M,\,\boldsymbol{+},\star)$
        be defined by the function that maps $r\in{R}$ to the function
        $\varphi_{r}$ by:
        \begin{equation}
            \varphi_{r}(m)=r\star{m}
            \quad\quad
            m\in{M}
        \end{equation}
        By Thm.~\ref{thm:Scalar_Mult_in_Module_Defines_Group_Endo},
        for all $r\in{R}$,
        $\varphi_{r}\in\textrm{End}(M,\,\boldsymbol{+},\star)$.
        Moreover, $\varphi$ is a ring homomorphism. For if
        $r_{1},r_{2}\in{R}$, then for all $m\in{M}$:
        \begin{align}
            \varphi_{r_{1}+r_{2}}(m)
            &=(r_{1}+r_{2})\star{m}
            \tag{Definition of $\varphi$}\\
            &=(r_{1}\star{m})\boldsymbol{+}(r_{2}\star{m})
            \tag{Module Distributivity}\\
            &=\varphi_{r_{1}}(m)\boldsymbol{+}\varphi_{r_{2}}(m)
            \tag{Definition of $\varphi$}\\
            &=(\varphi_{r_{1}}\boldsymbol{+}'\varphi_{r_{2}})(m)
            \tag{Definition of $\boldsymbol{+}'$}
        \end{align}
        And thus $\varphi$ preserves addition. Moreover:
        \begin{align}
            \varphi_{r_{1}\cdot{r}_{2}}(m)
            &=(r_{1}\cdot{r}_{2})\star{m}
            \tag{Definition of $\varphi$}\\
            &=r_{1}\star(r_{2}\star{m})
            \tag{Associativity}\\
            &=r_{1}\star\big(\varphi_{r_{2}}(m)\big)
            \tag{Definition of $\varphi$}\\
            &=\varphi_{r_{1}}\big(\varphi_{r_{2}}(m)\big)
            \tag{Definition of $\varphi$}\\
            &=(\varphi_{r_{1}}\circ\varphi_{r_{2}})(m)
            \tag{Definition of $\circ$}
        \end{align}
        And thus $\varphi$ preserves multiplication. Lastly:
        \begin{align}
            \varphi_{1_{R}}(m)&=1_{R}\star{m}
            \tag{Definition of $\varphi$}\\
            &=m
            \tag{Identity}
        \end{align}
        And thus $\varphi_{1}=\textrm{id}_{M}$, and $\textrm{id}_{M}$ is
        the unital element of $M$. Thus $\varphi$ preserves identity,
        and therefore $\varphi$ is a ring homomorphism
        (Def.~\ref{def:Ring_Homomorphism}).
    \end{proof}
    \begin{fdefinition}{Module Homomorphism}{Module_Homomorphism}
        A \gls{module homomorphism} from a \gls{module}
        $(M_{1},\boldsymbol{+}_{1},\star_{1})$ to a module
        $(M_{2},\boldsymbol{+}_{2},\star_{2})$ over a \gls{ring}
        $(R,\,+,\,\cdot\,)$ is a \gls{function} $f:M_{1}\rightarrow{M}_{2}$
        such that, for all $x,y\in{M}_{1}$, and for all $r\in{R}$, the
        following are true:
        \begin{align}
            f(x\boldsymbol{+}_{1}y)&=f(x)\boldsymbol{+}_{2}f(y)
            \tag{Preservation of Addition}\\
            f(r\star_{1}{x})&=r\star_{2}f(x)
            \tag{Preservation of Scalar Multiplication}
        \end{align}
    \end{fdefinition}
    And from this we can define a module isomorphism.