%------------------------------------------------------------------------------%
\section{Definitions}
    The first thing to do is to define a module\index{Module} over a given ring.
    \begin{fdefinition}{Modules}{Module}
        A module\index{Module} on a ring $(R,\,+,\,\cdot\,)$ is an
        \gls{Abelian group} $(M,\boldsymbol{+})$ with a function
        $\star:R\times{M}\rightarrow{M}$ such that for all $r_{1},r_{2}\in{R}$
        and for all $m_{1},m_{2}\in{M}$, the following are true:
        \begin{align}
            r_{1}\star(m_{1}\boldsymbol{+}m_{2})
                &=(r_{1}\star{m}_{1})\boldsymbol{+}(r_{1}\star{m}_{2})
                \tag{Scalar Distributivity}\\
            (r_{1}+r_{2})\star{m}_{1}
                &=(r_{1}\star{m}_{1})\boldsymbol{+}(r_{2}\star{m}_{1})
                \tag{Module Distributivity}\\
            r_{1}\star(r_{2}\star{m}_{1})
                &=(r_{1}\cdot{r}_{2})\star{m}_{1}
                \tag{Associativity}\\
            1\star{m}_{1}&=m_{1}
            \tag{Identity}
        \end{align}
        Where 1 is the unital element of $R$. We denote a module by
        $(M,\,\boldsymbol{+},\,\star)$.
    \end{fdefinition}
    If $(R,\,+,\,\cdot\,)$ is a \textit{field}\index{Field}, and if
    $(M,\,\boldsymbol{+},\,\star)$ is a module over this field, then it will
    also be a \textit{vector space}\index{Vector Space}. That is, modules are
    the ring-analog of vector spaces. Vectors spaces will be discussed in
    detail later.
    \begin{example}
        If $(R,+,\cdot\,)$ is a ring, then it can be thought of as a module over
        itself. That is, define $(R,\boldsymbol{+},\star)$ as follows:
        \par
        \begin{subequations}
            \begin{minipage}[b]{0.49\textwidth}
                \begin{equation}
                    r_{1}\boldsymbol{+}r_{2}=r_{1}+r_{2}
                \end{equation}
            \end{minipage}
            \hfill
            \begin{minipage}[b]{0.49\textwidth}
                \begin{equation}
                    r_{1}\star{r}_{2}=r_{1}\cdot{r}_{2}
                \end{equation}
            \end{minipage}
        \end{subequations}
        \par\vspace{2.5ex}
        From this we have that $R$ is a module over itself. This simply follows
        from the associative property that ring multiplication has.
    \end{example}
    \begin{example}
        Another important example that is often studied is that of the
        polynomial ring $R[x]$. That is, the ring of all polynomials over $R$ in
        one variable. This will act as a module over $R$. That is, define
        multiplication by:
        \begin{equation}
            r\star{m}(x)
            =r\star\Big(\sum_{k=1}^{n}a_{k}x^{k}\Big)
            =\sum_{k=1}^{n}(a\cdot{r}_{k})x^{k}
        \end{equation}
        where $\cdot$ is ring multiplication and the $a_{k}$ are elements of
        $R$. We extend $a_{k}$ or $b_{k}$ to be zero for indices greater than
        $n$ and $m$, respectively. Module addition is simply polynomial
        addition:
        \begin{equation}
            m_{1}\boldsymbol{+}m_{2}
            =\Big(\sum_{k=1}^{n}a_{k}x^{k}\Big)
                +\Big(\sum_{k=1}^{m}b_{k}x^{k}\Big)
            =\sum_{k=1}^{\max\{n,m\}}(a_{k}+b_{k})x^{k}
        \end{equation}
        Then $(R[x],\boldsymbol{+},\star)$ is a module over $(R,+,\cdot\,)$. We
        can also think about modules on $R[x]$ itself, since $R[x]$ is a ring.
        If we consider the case where $R$ is itself a field, then any module
        over $R[x]$ will be equivalent to a vector space over $R$ that is
        equipped with a \textit{linear transformation} from the vector space to
        itself. Again, this will be elaborated on later when vector spaces are
        discussed.
    \end{example}
    \begin{fexample}{Abelian Groups as $\mathbb{Z}$ modules}
                    {Abelian_Groups_as_Z_Modules}
        If $(G,*)$ is an Abelian group, then it may be thought of as a module
        over the ring of integers $(\mathbb{Z},+,\cdot\,)$. Define $\star$
        inductively for all $n\in\mathbb{N}^{+}$ by:
        \begin{equation}
            1\star{g}=g
            \quad\quad
            (n+1)\star{g}=(n\star{g})*g
        \end{equation}
        For negative elements, define the following:
        \begin{equation}
            (\minus{n})\star{g}=(n\star{g})^{\minus{1}}
        \end{equation}
        where $(n\star{g})^{\minus{1}}$ is the group inverse element of
        $n\star{g}$. This is well defined since $n\star{g}$ is an element of $G$
        and $G$ is a group. Trivially, define:
        \begin{equation}
            0\star{g}=e
        \end{equation}
        where $e$ is the unital element of the group $(G,*)$. With this we have
        that $(G,*,\star)$ is a module over $(\mathbb{Z},+,\cdot\,)$.
    \end{fexample}
    It is worthwhile to prove the claim that $(G,*,\star)$ is a module over
    $(\mathbb{Z},+,\cdot\,)$. The distributive laws can be proved by induction.
    Let $n,m\in\mathbb{N}^{+}$.
    \begin{subequations}
        \begin{align}
            (n+m)\star{g}
            &=(n+m-1+1)\star{g}\\
            &=\big((n+m-1)\star{g}\big)*g\\
            &=\Big((n\star{g})*\big((m-1)\star{g}\big)\Big)*g\\
            &=(n\star{g})*\Big(\big((m-1)\star{g})*g\Big)\\
            &=(n\star{g})*\big((m-1+1)\star{g}\big)\\
            &=(n\star{g})*(m\star{g})
        \end{align}
    \end{subequations}
    For negatives we have:
    \begin{subequations}
        \begin{align}
            \big((\minus{n})+(\minus{m})\big)\star{g}
                &=\big((n+m)\star{g}\big)^{\minus{1}}\\
                &=\big((n\star{g})*(m\star{g})\big)^{\minus{1}}\\
                &=(n\star{g})^{\minus{1}}*(m\star{g})^{\minus{1}}
        \end{align}
    \end{subequations}
    This last equality comes from the fact tha $(G,*)$ is an Abelian group, and
    therefore $(a*b)^{\minus{1}}=a^{\minus{1}}*b^{\minus{1}}$. Continuing, we
    obtain:
    \begin{subequations}
        \begin{align}
            \big((\minus{n})+(\minus{m})\big)\star{g}
            &=(n\star{g})^{\minus{1}}*(m\star{g})^{\minus{1}}\\
            &=\big((\minus{n})\star{g}\big)*\big((\minus{m})\star{g}\big)
        \end{align}
    \end{subequations}
    And thus we have the distributive law holds again. Next we need to check for
    when we have one positive and one negative. We get:
    \begin{subequations}
        \begin{align}
            \big((n+1)+(\minus{m})\big)\star{g}
            &=\Big(\big(n+(\minus{m})\big)+1\Big)\star{g}\\
            &=\Big(\big(n+(\minus{m})\big)\star{g}\Big)*g\\
            &=\Big((n\star{g})*\big((\minus{m})\star{g}\big)\Big)*g
        \end{align}
    \end{subequations}
    Since $(G,*)$ is an Abelian group, we can simplify this further to get:
    \begin{subequations}
        \begin{align}
            \big((n+1)+(\minus{m})\big)\star{g}
            &=\Big((n\star{g})*\big((\minus{m})\star{g}\big)\Big)*g\\
            &=\Big((n\star{g})*g\Big)*\big((\minus{m})\star{g}\big)\\
            &=\Big((n+1)\star{g})\Big)*\big((\minus{m})\star{g}\big)
        \end{align}
    \end{subequations}
    If either $n$ or $m$ is zero, the identity holds trivially. Thus we have the
    preservation of module distributivity. For the scalar distributive law, we
    have by induction (and since $(G,*)$ is Abelian):
    \begin{subequations}
        \begin{align}
            (n+1)\star(g*h)
            &=\big(n\star(g*h)\big)*(g*h)\\
            &=(n\star{g})*(n\star{g})*(g*h)\\
            &=\big((n\star{g})*g\big)*\big((n\star{h})*h\big)\\
            &=\big((n+1)\star{g}\big)*\big((n+1)\star{h}\big)
        \end{align}
    \end{subequations}
    And therefore the distributive law over modules holds for positive integers.
    For negative integers we have:
    \begin{subequations}
        \begin{align}
            (\minus{n})\star(g*h)
            &=\big(n\star(g*h)\big)^{\minus{1}}\\
            &=\big((n\star{g})*(n\star{h})\big)^{\minus{1}}\\
            &=(n\star{g})^{\minus{1}}*(n\star{h})^{\minus{1}}\\
            &=\big((\minus{n})\star{g}\big)*\big((\minus{n})\star{g}\big)
        \end{align}
    \end{subequations}
    For $n=0$ this is trivial since the product is simply the identity element
    of $G$. The last thing to check is the compatibility of ring multiplication
    with $\star$. If $n,m\in\mathbb{N}$, we have:
    \begin{subequations}
        \begin{align}
            n\star\big((m+1)\star{g}\big)
            &=n\star\big((m\star{g})*g\big)\\
            &=\big(n\star(m\star{g})\big)*(n\star{g})\\
            &=\big((n\cdot{m})\star{g}\big)*(n\star{g})\\
            &=(n\cdot{m}+n)\star{g}\\
            &=\big(n\cdot(m+1)\big)\star{g}
        \end{align}
    \end{subequations}
    The structure of a $\mathbb{Z}$ module is different from a vector space.
    In $\mathbb{Z}$, $\{2\}$ is a linearly independent set that cannot be
    extended to a basis. This is because for all $n\in\mathbb{Z}$ such that
    $n\ne{2}$ we have that $0=x\cdot{2}+(\minus{2})\cdot{x}$, and thus the set
    $\{2,x\}$ is linearly dependent. This is contrary to the usual notions of
    vector spaces where a linearly independent set can always be extended to a
    basis. Moreover, $\{2,3\}$ is a generated set of $\mathbb{Z}$ but no subset
    of $\{2,3\}$ is a basis. To see that it generates $\mathbb{Z}$, note that
    $1=1\cdot{3}-1\cdot{2}$, and 1 generated $\mathbb{Z}$. We've already seen
    that $\{2\}$ is not a basis and cannot be extended to form one, and neither
    can 3.
    \begin{theorem}
        \label{thm:Scalar_Mult_in_Module_Defines_Group_Endo}%
        If $\mathcal{M}=(M,\,\boldsymbol{+},\,\star)$ is a module over a ring
        $(R,\,+,\,\cdot\,)$ and if $r\in{R}$, then the function
        $f:M\rightarrow{M}$ defined by:
        \begin{equation}
            f(m)=r\star{m}
        \end{equation}
        is a group endomorphism on the group $(M,\boldsymbol{+})$.
    \end{theorem}
    \begin{proof}
        For let $m_{1},m_{2}\in{M}$. But since $(M,\,\boldsymbol{+},\star)$
        is a module, $\star$ left-distributes over $\boldsymbol{+}$
        (Def~\ref{def:Module}). Therefore:
        \begin{equation}
            f(m_{1}\boldsymbol{+}m_{2})
            =r\star(m_{1}\boldsymbol{+}m_{2})
            =(r\star{m}_{1})\boldsymbol{+}(r\star{m}_{2})
            =f(m_{1})\boldsymbol{+}f(m_{2})
        \end{equation}
        Thus, $f$ is a group endomorphism on $(M,\boldsymbol{+})$.
    \end{proof}
    \begin{theorem}
        If $(M,\,\boldsymbol{+},\,\star)$ is a module over a ring
        $(R,\,+,\,\cdot\,)$, then there exists a ring homomorphism
        $\varphi:R\rightarrow(\textrm{End}(M),\boldsymbol{+},\circ)$.
    \end{theorem}
    \begin{proof}
        For let $\varphi:R\rightarrow\textrm{End}(M,\,\boldsymbol{+},\star)$ be
        defined by the function that maps $r\in{R}$ to the function
        $\varphi_{r}$ by:
        \begin{equation}
            \varphi_{r}(m)=r\star{m}
            \quad\quad
            m\in{M}
        \end{equation}
        By Thm.~\ref{thm:Scalar_Mult_in_Module_Defines_Group_Endo}, for all
        $r\in{R}$, $\varphi_{r}\in\textrm{End}(M,\,\boldsymbol{+},\star)$.
        Moreover, $\varphi$ is a ring homomorphism. For if
        $r_{1},r_{2}\in{R}$, then for all $m\in{M}$:
        \begin{align}
            \varphi_{r_{1}+r_{2}}(m)
            &=(r_{1}+r_{2})\star{m}
            \tag{Definition of $\varphi$}\\
            &=(r_{1}\star{m})\boldsymbol{+}(r_{2}\star{m})
            \tag{Module Distributivity}\\
            &=\varphi_{r_{1}}(m)\boldsymbol{+}\varphi_{r_{2}}(m)
            \tag{Definition of $\varphi$}\\
            &=(\varphi_{r_{1}}\boldsymbol{+}'\varphi_{r_{2}})(m)
            \tag{Definition of $\boldsymbol{+}'$}
        \end{align}
        And thus $\varphi$ preserves addition. Moreover:
        \begin{align}
            \varphi_{r_{1}\cdot{r}_{2}}(m)
            &=(r_{1}\cdot{r}_{2})\star{m}
            \tag{Definition of $\varphi$}\\
            &=r_{1}\star(r_{2}\star{m})
            \tag{Associativity}\\
            &=r_{1}\star\big(\varphi_{r_{2}}(m)\big)
            \tag{Definition of $\varphi$}\\
            &=\varphi_{r_{1}}\big(\varphi_{r_{2}}(m)\big)
            \tag{Definition of $\varphi$}\\
            &=(\varphi_{r_{1}}\circ\varphi_{r_{2}})(m)
            \tag{Definition of $\circ$}
        \end{align}
        And thus $\varphi$ preserves multiplication. Lastly:
        \begin{align}
            \varphi_{1_{R}}(m)&=1_{R}\star{m}
            \tag{Definition of $\varphi$}\\
            &=m
            \tag{Identity}
        \end{align}
        And thus $\varphi_{1}=\textrm{id}_{M}$, and $\textrm{id}_{M}$ is
        the unital element of $M$. Thus $\varphi$ preserves identity,
        and therefore $\varphi$ is a ring homomorphism
        (Def.~\ref{def:Ring_Homomorphism}).
    \end{proof}
    The converse of this theorem is true as well. That is, if
    $(M,\boldsymbol{+})$ is an Abelian group and if
    $\varphi:(R,+,\cdot)\rightarrow(\textrm{End}(M),\boldsymbol{+},\circ)$ is
    a ring homomorphism, then there exists an operation
    $\star:R\times{M}\rightarrow{M}$ that makes $(M,\boldsymbol{+},\circ)$ a
    module over $(R,+,\cdot)$.
    \begin{theorem}
        If $(R,+,\cdot\,)$ is a ring, if $(M,\boldsymbol{+})$ is an Abelian
        group, and if
        $\varphi:(R,+,\cdot)\rightarrow(\textrm{End}(M),\boldsymbol{+},\circ)$
        is a ring homomorphism, then there is a function
        $\star:R\times{M}\rightarrow{M}$ such that $(M,\boldsymbol{+},\circ)$ is
        a module over $(R,+,\cdot)$.
    \end{theorem}
    \begin{proof}
        For let $\star:R\times{M}\rightarrow{M}$ be defined by:
        \begin{equation}
            r\star{m}=\varphi_{r}(m)
        \end{equation}
        Where $\varphi_{r}$ is the endomorphism that $r\in{R}$ gets mapped to
        by $\varphi$. Then $(M,\boldsymbol{+},\star)$ is a module over
        $(R,+,\cdot\,)$.
    \end{proof}
    \begin{lexample}{Another Example of a Module}{Another_Example_of_a_Module}
        Let $(\mathbb{F},+,\cdot\,)$ be a field and let $V$ be a finite
        dimensional vector space $(V,\boldsymbol{+},\boldsymbol{\cdot}\,)$ over
        $\mathbb{F}$. Let $T:V\rightarrow{V}$ be a linear operator and let
        $\mathbb{F}[x]$ be the polynomial ring with coefficients in
        $\mathbb{F}$. We can define a module struct over $V$ by letting
        $\star:\mathbb{F}[x]\times{V}\rightarrow{V}$ be defined by:
        \begin{equation}
            f\star{v}=\sum_{k=0}^{n}a_{k}T^{k}(v)
        \end{equation}
        where $a_{k}$ are the coefficients of the polynomial $f\in\mathbb{F}[x]$
        and where $T^{k}$ is the $k^{th}$ composition of $T$ with itself. That
        is, $T^{2}=T\circ{T}$, $T^{n+1}=T^{n}\circ{T}$. Letting
        $\boldsymbol{+}'$ denote polynomial addition, we have that
        $(V,\boldsymbol{+},\star)$ is a module over $\mathbb{F}[x]$ with its
        usual ring structure.
    \end{lexample}
    \begin{fdefinition}{Module Homomorphism}{Module_Homomorphism}
        A \gls{module homomorphism}\index{Module Homomorphism} from a
        \gls{module} $(M_{1},\boldsymbol{+},\star)$ to another module
        $(M_{2},\boldsymbol{+}',\diamond)$ over a \gls{ring}
        $(R,\,+,\,\cdot\,)$ is a \gls{function} $f:M_{1}\rightarrow{M}_{2}$
        such that, for all $x,y\in{M}_{1}$, and for all $r\in{R}$, the
        following are true:
        \begin{align}
            f(x\boldsymbol{+}y)&=f(x)\boldsymbol{+}'f(y)
            \tag{Preservation of Addition}\\
            f(r\star{x})&=r\diamond{f}(x)
            \tag{Preservation of Scalar Multiplication}
        \end{align}
    \end{fdefinition}
    And from this we can define a module isomorphism\index{Module Isomorphism}.
    \begin{fdefinition}{Module Isomorphism}{Module_Isomorphism}
        A \gls{module isomorphism} from a \gls{module}
        $(M_{1},\boldsymbol{+},\star)$ to a module
        $(M_{2},\boldsymbol{+}',\diamond)$ over a \gls{ring}
        $(R,\,+,\,\cdot\,)$ is a \gls{module homomorphism} $f$ such that $f$ is
        a \gls{bijective function}.
    \end{fdefinition}
    \begin{fdefinition}{Submodule}{Submodule}
        A \gls{submodule}\index{Submodule} of a \gls{module}
        $(M,\boldsymbol{+},\star)$ over a ring $(R,+,\cdot)$ is a \gls{subgroup}
        $(N,\boldsymbol{+})$ of $(M,\boldsymbol{+})$ such that, for all
        $n\in{N}$ and for all $r\in{R}$ it is true that $r\star{n}\in{N}$.
    \end{fdefinition}