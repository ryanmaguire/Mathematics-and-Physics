\section{Rngs and Rings}
    \begin{fdefinition}{Rng}{Rng}
        A rng is an \gls{Abelian group} $(R,+)$ and a \gls{semigroup}
        $(R,\cdot\,)$ such that $\cdot$ is a \gls{distributive operation}
        over $+$. A rng is denoted $(R,+,\cdot\,)$.
    \end{fdefinition}
    The unital element of the Abelian group $(R,+)$ is often denoted 0 and is
    called the zero element. Zero has the unique property that multiplication
    by zero returns zero for any element.
    \begin{theorem}
        \label{thm:Mult_By_Zero_in_Rng}%
        If $(R,+,\cdot\,)$ is a rng, if 0 is the unital element of $(R,+)$, and
        if $r\in{R}$, then $r\cdot{0}=0$.
    \end{theorem}
    \begin{proof}
        For:
        \begin{align}
            0&=r\cdot{0}-r\cdot{0}
            \tag{Inverse Property of Groups}\\
            &=r\cdot(0-0)
            \tag{Distributive Property}\\
            &=r\cdot{0}
            \tag{Identity Property}
        \end{align}
        And therefore $r\cdot{0}=0$.
    \end{proof}
    \begin{fdefinition}{Ring}{Ring}
        A \gls{ring} is a \gls{rng} $(R,+,\cdot\,)$ such that $(R,\cdot\,)$ is
        a \gls{monoid}. That is, an \gls{Abelian group} $(R,+)$ and a monoid
        $(R,\cdot\,)$ such that $\cdot$ is a \gls{distributive operation} over
        $+$.
    \end{fdefinition}
    Zero is the only element that does this. The converse of
    Thm.~\ref{thm:Mult_By_Zero_in_Rng} is true.
    \begin{theorem}
        \label{thm:Mult_by_Zero_Always_Zero_Implies_Zero}%
        If $(R,+,\cdot\,)$ is a rng, if $0$ is the unital element of $(R,+)$,
        and if $r$ is such that for all $s\in{R}$ it is true that $r\cdot{s}=0$,
        then $r=0$.
    \end{theorem}
    \begin{proof}
        For if $(R,+,\cdot\,)$ is a ring, then $(R,\cdot\,)$ is a monoid
        (Def.~\ref{def:Ring}) and thus there is a unital element, 1, of
        $(R,\cdot\,)$ (Def.~\ref{def:Monoid}). But then for all $s\in{R}$
        we have:
        \begin{align}
            r+s&=(r+s)\cdot{1}
            \tag{Identity Property of 1}\\
            &=(r\cdot{1})+(s\cdot{1})
            \tag{Distributive Property}\\
            &=0+s\cdot{1}
            \tag{Hypothesis}\\
            &=s\cdot{1}
            \tag{Identity Property of 0}\\
            &=s
            \tag{Identity Property of 1}
        \end{align}
        And thus, for all $s\in{R}$, $r+s=s$. But $(R,+)$ is an Abelian group,
        and thus if $r+s=s$, then $s+r=s$ (Def.~\ref{def:Abelian_Group}) and
        therefore $r$ is a unital element of $(R,+)$
        (Def.~\ref{def:Unital_Element}). But the unital element of a group is
        unique, and therefore $r=0$.
    \end{proof}

