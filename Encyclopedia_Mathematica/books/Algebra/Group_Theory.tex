\chapter{Group Theory}
    \section{Relations}
        \begin{ldefinition}{Relation on a Set}{Relation}
            A relation on a set $A$ is a subset $R$ of $A\times{A}$.
            That is, $R\subseteq{A}\times{A}$. For elements
            $(a,b)\in{R}$ we write $aRb$.
        \end{ldefinition}
        For a relation $R$ it is not necessary true that $aRb$
        implies $bRa$, nor is it necessarily true that $aRa$. These
        are called symmetric and reflexive relations, respectively.
        There are many basic properties that relations have, and we
        prove them now.
        \begin{theorem}
            \label{thm:Cartesian_Product_Is_Relation}%
            If $A$ is a set, then $A\times{A}$ is a relation on $A$.
        \end{theorem}
        \begin{proof}
            For if $A$ is a set, then
            $A\times{A}\subseteq{A}\times{A}$. Therefore, etc.
        \end{proof}
        \begin{theorem}
            \label{thm:Empty_Set_Is_Relation}%
            If $A$ is a set, and then $\emptyset$ is a relation
            on $A$.
        \end{theorem}
        \begin{proof}
            For if $A$ is a set, then
            $\emptyset\subseteq{A}\times{A}$. Therefore, etc.
        \end{proof}
        These provide the two most basic examples of relations on a
        set. The empty set is the relation that says no two elements
        are related. Indeed, even single elements are unrelated to
        themselves. The second, the entire Cartesian product
        $A\times{A}$, says that everything is related. These are the
        two extreme cases, but provide useful examples and
        counterexamples in various contexts. More useful is that the
        union and intersection of relations is also a relation. We
        prove this now.
        \begin{theorem}
            \label{thm:Intersection_of_Relations_Is_Relation}%
            If $A$ is a set and if $R_{1}$ and $R_{2}$ are relations
            on $A$, then $R_{1}\cap{R}_{2}$ is a relation on $A$.
        \end{theorem}
        \begin{proof}
            For let $R=R_{1}\cap{R}_{2}$ and suppose $R$ is not a
            relation on $A$. Then there is an $x\in{R}$ such that
            $x\notin{A}\times{A}$. But if $x\in{R}$ then
            $x\in{R}_{1}$ and $x\in{R}_{2}$. But for all
            $x\in{R}_{1}$, $x\in{A}\times{A}$, since $R_{1}$ is a
            relation on $A$, a contradiction as
            $x\notin{A}\times{A}$. Therefore, $R$ is a relation on
            $A$.
        \end{proof}
        \begin{theorem}
            \label{thm:Set_Theory_Union_of_Relations_Is_Relation}
            If $A$ is a set and if $R_{1}$ and $R_{2}$ are relations
            on $A$, then $R_{1}\cup{R}_{2}$ is a relation on $A$.
        \end{theorem}
        \begin{proof}
            For let $R=R_{1}\cup{R}_{2}$ and suppose $R$ is not a
            relation on $A$. Then there is an $x\in{R}$ such that
            $x\notin{A}\times{A}$. But if $x\in{R}$ then
            $x\in{R}_{1}$ or $x\in{R}_{2}$. But for all $x\in{R}_{1}$
            and for all $x\in{R}_{2}$,
            $x\in{A}\times{A}$, since $R_{1}$ and $R_{2}$ are
            relations on $A$, a contradiction. Therefore, etc.
        \end{proof}
        \begin{theorem}
            If $A$ is a set and $R$ is a relation on $A$, then there
            is a relation $\mathcal{U}$ on $A$ such that
            $R\cap\mathcal{U}=R$.
        \end{theorem}
        \begin{proof}
            For let $\mathcal{U}={A}\times{A}$. Then by
            Thm.~\ref{thm:Set_Theory_Entire_%
                      Cartesian_Product_Is_Relation}, $A\times{A}$ is
            a relation on $A$. But since $R$ is a relation,
            $R\subseteq{A}\times{A}$. But then
            $R\cap\mathcal{U}=R$. Therefore, etc.
        \end{proof}
        \begin{theorem}
            If $A$ is a set and $R$ is a relation on $A$, then there
            is a relation $\mathcal{U}$ on $A$ such that
            $R\cup\mathcal{U}=R$
        \end{theorem}
        \begin{proof}
            For let $\mathcal{U}=\emptyset$. Then by
            Thm.~\ref{thm:Set_Theory_Empty_Set_Is_Relation},
            $\mathcal{U}$ is a relation. But if $R$ is a set, then
            $R\cup\emptyset=R$. Thus, $R\cup\mathcal{U}=R$.
            Therefore, etc.
        \end{proof}
        Since a general relation is simply a subset of $A\times{A}$,
        there's not much structure on them, and thus there's not a lot
        that can be said about them. We can add more constraints to
        certain relations to get the more familiar properties
        we're used to.
        \begin{ldefinition}{Reflexive Relations}
            A reflexive relation on a set $A$ is a
            relation $R$ on $A$ such that for all $a\in{A}$
            it is true that $aRa$.
        \end{ldefinition}
        A reflexive relation on $A$ is simply any subset of
        $A\times{A}$ that contains the entire \textit{diagonal}. That,
        all of the pairs $(a,a)$. A reflexive relation can contain more
        than this, however. The only strict requirement is that
        $aRa$ for all $a\in{A}$.
        \begin{theorem}
            If $A$ is a set, and if $R_{1}$ and $R_{2}$ are reflexive
            relations on $A$, then $R_{1}\cap{R}_{2}$ is a reflexive
            relation on $A$.
        \end{theorem}
        \begin{proof}
            For let $R=R_{1}\cap{R}_{2}$. Then by
            Thm.~\ref{thm:Set_Theory_Intersection_of_%
                      Relations_Is_Relation}, $R$ is a relation.
            Suppose $R$ is not reflexive.
            Then there is an $a\in{A}$ such that $(a,a)\notin{R}$. But
            if $a\in{A}$, then $(a,a)\in{R}_{1}$, since $R_{1}$ is
            reflexive. Similarly, $(a,a)\in{R}_{2}$ since $R_{2}$ is
            reflexive. But if $(a,a)\in{R}_{1}$ and $(a,a)\in{R}_{2}$,
            then $(a,a)\in{R}$ since $R=R_{1}\cap{R}_{2}$, a
            contradiction. Therefore, $R$ is reflexive.
        \end{proof}
        \begin{theorem}
            If $A$ is a set, if $R_{1}$ is a reflexive relation on
            $A$, and if $R_{2}$ is a relation on $A$, then
            $R_{1}\cup{R}_{2}$ is a reflexive relation on $A$.
        \end{theorem}
        \begin{proof}
            For let $R=R_{1}\cup{R}_{2}$. Since $R_{1}$ and $R_{2}$ are
            relations, by
            Thm.~\ref{thm:Set_Theory_Union_of_Relations_Is_Relation},
            $R$ is a relation. Suppose it is not reflexive.
            Then there is an $a\in{A}$ such that
            $(a,a)\notin{R}$. But if $a\in{A}$ then $(a,a)\in{R}_{1}$
            since $R_{1}$ is reflexive. But if $(a,a)\in{R}_{1}$ then
            $(a,a)\in{R}_{1}\cup{R}_{2}$, a contradiction.
            Therefore, etc.
        \end{proof}
        Given an arbitrary relation $R$ on a set $A$, it may not be
        true that $R$ is reflexive. It may often be useful to add in
        only the necessary points of $A$ that will make $R$
        reflexive. This is called the reflexive closure of $R$.
        \begin{ldefinition}{Reflexive Closure of a Relation}
              {Reflexive_Closure_of_Relation}
            The reflexive closure of a relation $R$ on a set $A$
            is the set:
            \begin{equation}
                S=R\cup\{(a,a):a\in{A}\}
            \end{equation}
        \end{ldefinition}
        \begin{theorem}
            If $A$ is a set, $R$ is a relation on $A$, and if $S$ is the
            reflexive closure of $R$, then $S$ is a reflexive relation on $A$.
        \end{theorem}
        \begin{theorem}
            \label{thm:Set_Theory_Refl_Clos_Is_Smallest_Refl_With_R}
            If $A$ is a set, if $R$ is a relation on $A$, if
            $S$ is the reflexive closure of $R$, and if $T$ is a
            reflexive relation on $A$ such that $R\subseteq{T}$, then
            $S\subseteq{T}$.
        \end{theorem}
        \begin{proof}
            For if $x\in{S}$, then either $x\in{R}$ or there is an
            $a\in{A}$ such that $x=(a,a)$. But if $x\in{R}$, then
            $x\in{T}$ since $R\subseteq{T}$. If $x\notin{R}$ then
            there is an $a\in{A}$ such that $x=(a,a)$. But $T$ is
            reflexive, and therefore $(a,a)\in{T}$. But then
            $x\in{T}$. Therefore, $S\subseteq{T}$.
        \end{proof}
        Thm.~\ref{thm:Set_Theory_Refl_Clos_Is_Smallest_Refl_With_R}
        says that the reflexive closure of a relation $R$ is, in a sense,
        the \textit{smallest} relation that is reflexive and contains
        $R$ as a subset.
        \begin{theorem}
            If $A$ is a set, $R_{1}$ and $R_{2}$ are relations on $A$,
            and if $S_{1}$ and $S_{2}$ are the reflexive closures of
            $R_{1}$ and $R_{2}$, respectively, then the reflexive closure
            of $R_{1}\cap{R}_{2}$ is:
            \begin{equation}
                S=S_{1}\cap{S}_{2}
            \end{equation}
        \end{theorem}
        \begin{proof}
            By the definition of reflexive closure, we have:
            \begin{align}
                S_{1}&=R_{1}\cup\{(a,a):a\in{A}\}
                \tag{Def.~\ref{def:Reflexive_Closure_of_Relation}}\\
                S_{1}&=R_{2}\cup\{(a,a):a\in{A}\}
                \tag{Def.~\ref{def:Reflexive_Closure_of_Relation}}\\
                \nonumber
                S_{1}\cap{S}_{2}&=
                (R_{1}\cup\{(a,a):a\in{A}\})
                \cap(R_{2}\cup\{(a,a):a\in{A}\})\\
                &=(R_{1}\cap{R}_{2})
                \cup\{(a,a):a\in{A}\}
                \tag{Distributive Law}
            \end{align}
            But by the definition of the transitive closure of
            $R_{1}\cap{R}_{2}$:
            \begin{equation}
                S=(R_{1}\cap{R}_{2})\cup\{(a,a):a\in{A}\}
                \tag{Def.~\ref{def:Set_Theory_Reflexive_%
                               Closure_of_Relation}}
            \end{equation}
            Therefore, etc.
        \end{proof}
        \begin{ldefinition}{Symmetric Relations}
            A symmetric relation on a set $A$ is a
            relation $R$ on $A$ such that for all $a,b\in{A}$
            such that $aRb$, it is true that $bRa$.
        \end{ldefinition}
        \begin{ldefinition}{Transitive Relation}
            A transitive relation on a set $A$ is a relation $R$ on $A$
            such that for all $a,b,c\in{A}$ such that $aRb$ and $bRc$,
            is it true that $aRc$.
        \end{ldefinition}
        \begin{ldefinition}{Transitive Closure}
            The transitive closure of a relation $R$ on a set
            $A$ is the the set $R^{t}\subseteq{A}\times{A}$ defined by:
            \begin{equation}
                R^{t}
            \end{equation}
        \end{ldefinition}
        \begin{ldefinition}{Asymmetric Relation}
            An asymmetric relation on a set $A$ is a relation $R$
            on $A$ such that for all $a,b\in{A}$ such that $aRb$
            it is true that $(b,a)\notin{R}$.
        \end{ldefinition}
        \begin{ldefinition}{Total Relation}
            A total relation on a set $A$ is a relation $R$ on $A$ such
            that for all $a,b\in{A}$ it is true that either
            $aRb$ or $bRa$, or both.
        \end{ldefinition}
        The notion of equality can be defined as a relation
        with the following properties:
        \begin{enumerate}
            \item Equality is Reflexive: $a=a$ for all $a\in{A}$.
            \item Equality is Symmetric: $a=b$ if and only if $b=a$.
            \item Equality is Transitive: If $a=b$ and $b=c$, then $a=c$.
            \item The relation is uniquely defined by the set
                  $\{(a,a)\in A\times A:a\in A\}$.
        \end{enumerate}
        That is, equality can be seen as the \textit{diagonal} in the
        Cartesian product $A\times{A}$.
        \begin{ldefinition}{Antisymmetric Relation}
            An antisymmetric relation on a set $A$ is a relation $R$ on $A$
            such that for all $a,b\in{A}$ such that $aRb$ and $bRa$, it
            is true that $a=b$.
        \end{ldefinition}
    \section{Functions}
        \begin{ldefinition}{Functions}
            A function from a set $A$ to a set $B$, denoted
            $f:A\rightarrow B$, is a subset of $A\times{B}$
            such that for all $a\in{A}$ there is a unique
            $b\in{B}$, denoted $f(a)$, such that $(a,b)\in{f}$.
        \end{ldefinition}
        Functions can also be called maps or mappings. The unique point
        $b\in{B}$ such that $(a,b)\in{f}$ is often called the image of
        $a$ under $f$. We sometimes write $a\mapsto{b}$, but most often
        will write $f(a)=b$.
        \begin{ldefinition}{Image of a Set}
            The image of a subset $A$ of a set $X$ under a function
            $f:X\rightarrow{Y}$ is the set:
            \begin{equation}
                f(A)=\{f(x)\in{Y}:x\in{A}\}
            \end{equation}
        \end{ldefinition}
        The image of a subset $A\subseteq{X}$ is the set of all points
        that get mapped onto by the function $f$ by the elements in $A$.
        In a similar manner we can define the opposite of this notion,
        called the pre-image.
        \begin{ldefinition}{Pre-Image}
            The pre-image of a subset $B$ of a set $Y$ under a function
            $f:X\rightarrow{Y}$ is the set:
            \begin{equation}
                f^{-1}(B)=\{x\in{X}:f(x)\in{B}\}
            \end{equation}
        \end{ldefinition}
        \begin{axiom}
            If $X$ is a non-empty set such that for all $x\in{X}$,
            $x\ne\emptyset$, then there is a function
            $f:X\rightarrow\bigcup_{x\in{X}}x$ such that,
            for all $x\in{X}$, $f(x)\in{x}$.
        \end{axiom}
        This is called the axiom of choice. It can be made into
        a blatantly obvious statement, by choosing a more careful
        wording, however many of the results it gives are far
        from intuitive. This says that, given a collection of sets,
        each of which is non-empty, one may choose a single element from
        each set. This choosing is the function $f$, and is often called
        a \textit{choice function}. For those interested, the axiom
        of choice is consistent with modern set theory (Called
        Zermelo-Fraenkel set theory, or ZF). It may thus be rejected
        or accepted without logical contradiction.
        \begin{theorem}
            \label{theorem:Set_Theory_Image_of_Empty_Set_Is_Empty}
            If $A$ and $B$ are sets, and if $f:A\rightarrow{B}$
            is a function, then:
            \begin{equation}
                f(\emptyset)=\emptyset
            \end{equation}
        \end{theorem}
        \begin{proof}
            For suppose not. Let $y\in{f}(\emptyset)$.
            But then there is an
            $x\in\emptyset$ such that $f(x)=y$, a contradiction 
            ince for all $x$, $x\notin\emptyset$.
            Thus, $f(\emptyset)=\emptyset$.
        \end{proof}
        \begin{theorem}
            If $A$ and $B$ are sets, and if $f:A\rightarrow{B}$ is a function, then:
            \begin{equation}
                f^{-1}(\emptyset)=\emptyset
            \end{equation}
        \end{theorem}
        \begin{proof}
            For suppose not. Then there is an $x\in{X}$ such that
            $f(x)\in\emptyset$, a contradiction since for all $x$,
            $f(x)\notin\emptyset$. Therefore, etc.
        \end{proof}
        \begin{theorem}
            If $X$ and $Y$ are sets, if $A\subseteq{X}$, and if
            $f:X\rightarrow{Y}$ is a function such that
            $f(A)=\emptyset$, then $A=\emptyset$.
        \end{theorem}
        \begin{proof}
            For suppose not. If $A\ne\emptyset$, then there is an
            $x\in{A}$. But then $f(x)\in{f}(A)$, a contradiction as
            $f(A)=\emptyset$. Therefore, etc.
        \end{proof}
        \begin{theorem}
            If $X$ and $Y$ are sets, if $B$ is a subset of $Y$,
            and if $f:X\rightarrow{Y}$ is a function, then:
            \begin{equation}
                f\big(f^{-1}(B)\big)\subseteq{B}
            \end{equation}
        \end{theorem}
        \begin{proof}
            For if $y\in{f(f^{-1}(B))}$, then there is an
            $x\in{f^{-1}(B)}$ such that $y=f(x)$. But if
            $x\in{f^{-1}(B)}$, then $f(x)\in{B}$. Thus,
            $y\in{B}$. Therefore, etc.
        \end{proof}
        \begin{theorem}
            If $X$ and $Y$ are non-empty sets and if there exists
            $y_{1},y_{2}\in{Y}$ such that $y_{1}\ne{y}_{2}$, then
            there is a function $f:X\rightarrow{Y}$ and a
            $B\subseteq{Y}$ such that:
            \begin{equation}
                f\big(f^{-1}(B)\big)\ne{B}
            \end{equation}
        \end{theorem}
        \begin{proof}
            \begin{subequations}
                For if $X$ and $Y$ are non-empty, let $f:X\rightarrow{Y}$
                be defined by:
                \begin{equation}
                    f=\{(x,y_{1}):x\in{X}\}
                \end{equation}
                Then $f$ is a function, since $f\subseteq{X}\times{Y}$
                as $y_{1}\in{Y}$. Moreover, for all $x\in{X}$ there is a
                unique $y\in{Y}$ such that $(x,y)\in{f}$. Thus, $f$ is a
                function from $X$ to $Y$. However since for all
                $x\in{X}$, $f(x)=y_{1}$, we have that:
                \begin{equation}
                    f^{-1}(\{y_{2}\})=\emptyset
                \end{equation}
                For suppose $x\in{f}^{-1}(\{y_{2}\})$.
                Then $f(x)=y_{2}$, but for all $x\in{X}$, $f(x)=y_{1}$,
                and $y_{1}\ne{y}_{2}$. Thus
                $f^{-1}(\{y_{2}\})=\emptyset$. But by
                Thm.~\ref{theorem:Set_Theory_Image_%
                          of_Empty_Set_Is_Empty},
                $f(\emptyset)=\emptyset$. Therefore:
                \begin{equation}
                    f\big(f^{-1}(\{y_{2}\})\big)=\emptyset
                \end{equation}
                But $\{y_{2}\}\ne\emptyset$ and
                $\{y_{2}\}\subseteq{Y}$. Therefore, etc.
            \end{subequations}
        \end{proof}
        \begin{theorem}
            If $X$ and $Y$ are sets, if $A$ is a subset of $X$,
            and if $f:X\rightarrow{Y}$ is a function, then:
            \begin{equation}
                A\subseteq{f^{-1}}\big(f(A)\big)
            \end{equation}
        \end{theorem}
        \begin{proof}
            For if $x\in{A}$, then there is a
            $y\in{f}(A)$ such that $f(x)=y$. But then
            $x\in{f^{-1}(f(A))}$. Therefore, etc.
        \end{proof}
        \begin{theorem}
            If $X$ and $Y$ are sets, if $A_{1}$ and $A_{2}$ are
            subsets of $X$ such that $A_{1}\subseteq{A}_{2}$,
            and if $f:X\rightarrow{Y}$ is a function, then:
            \begin{equation}
                f(A_{1})\subseteq{f}(A_{2})
            \end{equation}
        \end{theorem}
        \begin{proof}
            For if $y\in{f}(A_{1})$, then there is an $x\in{A}_{1}$
            such that $f(x)=y$. But $A_{1}\subseteq{A}_{2}$, and
            therefore $x\in{A}_{2}$. But if $x\in{A}_{2}$, then
            $f(x)\in{f}(A_{2})$. Thus, $y\in{f}(A_{2})$. Therefore, etc.
        \end{proof}
        \begin{theorem}
            If $X$ and $Y$ are sets, if $B_{1}$ and $B_{2}$ are subsets of
            $Y$ such that $B_{1}\subseteq{B}_{2}$, and if $f:X\rightarrow{Y}$
            is a function, then:
            \begin{equation}
                f^{-1}(B_{1})\subseteq{f^{-1}}(B_{2})
            \end{equation}
        \end{theorem}
        \begin{proof}
            For if $x\in{f}^{-1}(B_{1})$, then there is a
            $y\in{B}_{1}$ such that $f(x)=y$. But
            $B_{1}\subseteq{B}_{2}$, and therefore $y\in{B}_{2}$.
            Thus, $x\in{f}^{-1}(B_{2})$. Therefore, etc.
        \end{proof}
        \begin{theorem}
        If $f:A\rightarrow B$, $A_1,A_2\subset A$, then $f(A_1 \cup A_2) = f(A_1)\cup f(A_2)$.
        \end{theorem}
        \begin{proof}
        $[y\in f(A_1\cup A_2)]\Rightarrow [\exists x\in A_1 \cup A_2:y=f(x)]\Rightarrow [y \in f(A_1)\cup f(A_2)]$. $[y\in f(A_1)\cup f(A_2)]\Rightarrow \big[[\exists x\in A_1] \lor[\exists x\in A_2]: y=f(x)\big]\Rightarrow [x\in A_1\cup A_2]\Rightarrow [f(x)\in f(A_1\cup A_2)]$
        \end{proof}
        \begin{theorem}
        If $f:A\rightarrow B$, $A_1,A_2\subset A$, then $f(A_1\cap A_2)\subset f(A_1)\cap f(A_2)$.
        \end{theorem}
        \begin{proof}
        $[y\in f(A_1 \cap A_2)]\Rightarrow [\exists x\in A_1 \cap A_2:y=f(x)]\Rightarrow [x\in A_1 \land x \in A_2] \Rightarrow[y \in f(A_1)\cap f(A_2)]$.
        \end{proof}
        \begin{theorem}
        If $f:A\rightarrow B$, $B_1,B_2\subset B$, then $f^{-1}(B_1\cup B_2) = f^{-1}(B_1)\cup f^{-1}(B_2)$.
        \end{theorem}
        \begin{proof}
        $[x\in B_1\cup B_2]\Rightarrow [f(x)\in B_1\cup B_2]\Rightarrow [f(x)\in B_1\lor f(x)\in B_2]\Rightarrow [x\in f^{-1}(B_1)\cup f^{-1}(B_2)]$. $[x \in f^{-1}(B_1)\cupf^{-1}(B_2)]\Rightarrow [f(x)\in B_1\lor f(x) \in B_2]\Rightarrow [f(x) \in B_1\cup B_2]\Rightarrow [x\in f^{-1}(B_1\cup B_2)]$.
        \end{proof}
        \begin{theorem}
        If $f:A\rightarrow B$, $B_1,B_2\subset B$, then $f^{-1}(B_1\cap B_2) = f^{-1}(B_1)\cap f^{-1}(B_2)$.
        \end{theorem}
        \begin{proof}
        $[x\in f^{-1}(B_1\cap B_2)]\Rightarrow [f(x) \in B_1 \cap B_2]\Rightarrow [f(x)\in B_1\land f(x) \in B_2 ]\Rightarrow [x\in f^{-1}(B_1)\cap f^{-1}(B_2)]$. $[x\in f^{-1}(B_1)\capf^{-1}(B_2)]\Rightarrow [x\in f^{-1}(B_1)\land x\in f^{-1}(B_2)]\Rightarrow [f(x) \in B_1\land f(x) \in B_2]\Rightarrow [f(x)\in B_1\cap B_2]\Rightarrow [x\in f^{-1}(B_1\cap B_2)]$.
        \end{proof}
        \begin{theorem}
        If $f:A\rightarrow B$, $B_1 \subset B$, then $f^{-1}(B\setminus B_1) = f^{-1}(B)\setminus f^{-1}(B_1)$.
        \end{theorem}
        \begin{proof}
        $[x\in f^{-1}(B\setminus B_1)]\Leftrightarrow [f(x)\notin B_1]\Leftrightarrow [x\in f^{-1}(B)\setminus f^{-1}(B_1)]$
        \end{proof}
        If $f:A\rightarrow B$, the image of $A$ under $f$
        is often called the range (A is often called the domain).
        \begin{ldefinition}{Injective Functions}
            An injective function from a set $A$ to a set $B$ is a function
            $f:A\rightarrow{B}$ such that for all $x,y\in{A}$ such that
            $x\ne{y}$, it is true that $f(x)\ne{f}(y)$.
        \end{ldefinition}
        \begin{ldefinition}{Inverse of Injective Functions}
            The inverse of an injective function $f:A\rightarrow{B}$
            is the function $f^{-1}:f(A)\rightarrow{A}$ defined by
            $f^{-1}(y)=x$, where $x\in{A}$ is the unique element such
            that $f(x)=y$.
        \end{ldefinition}
        \begin{ldefinition}{Surjective Functions}
            A surjective function from a set $A$ to a set $B$ is
            a function $f:A\rightarrow{B}$ such that for all
            $y\in{B}$ there is an $x\in{A}$ such that $y=f(x)$.
        \end{ldefinition}
        \begin{ldefinition}{Bijective Functions}
            A bijective function from a set $A$ to a set $B$
            is a function $f:A\rightarrow{B}$ such that $f$ is
            injective and surjective.
        \end{ldefinition}
        \begin{ldefinition}{Permutations}
            A permutation on a set $A$ is a bijective function
            $f:A\rightarrow{A}$.
        \end{ldefinition}
        \begin{theorem}
        If $f:A\rightarrow B$ is bijective, then $f^{-1}$ is bijective.
        \end{theorem}
        \begin{proof}
        $[f^{-1}(y_1) = f^{-1}(y_2)]\Rightarrow [\exists x\in A:[f(x) = y_1]\land [f(x)=y_2]]\Rightarrow [y_1=y_2]$. By definition, $f^{-1}$ is surjective.
        \end{proof}
        \begin{definition}
        If $f:A\rightarrow B$ and $g:B\rightarrow C$, then $g\circ f:A\rightarrow C$ is defined by the image $g(f(x)), x\in A$. 
        \end{definition}
        \begin{theorem}
        If $f:A\rightarrow B$, $g:B\rightarrow C$, and $\mathcal{V}\subset C$, then $(g\circ g)^{-1}(\mathcal{V}) = f^{-1}(g^{-1}(\mathcal{V}))$.
        \end{theorem}
        \begin{proof}
        $[x\in (g\circ f)^{-1}(\mathcal{V})]\Leftrightarrow [g(f(x))\in \mathcal{V}] \Leftrightarrow [f(x)\in g^{-1}(\mathcal{V})]\Leftrightarrow [x\in f^{-1}(g^{-1}(\mathcal{V}))]$.
        \end{proof}
        \begin{theorem}
        If $f:A\rightarrow B$ is bijective, $g:B\rightarrow C$ is bijective, then $g\circ f$ is bijective.
        \end{theorem}
        \begin{proof}
        $\big[[f(A) = B]\land [g(B) = C]\big]\Rightarrow [g(f(A)) = g(B) = C]$. $[g(f(x_1))=g(f(x_2))]\Leftrightarrow [f(x_1)=f(x_2)]\Leftrightarrow [x_1=x_2]$.
        \end{proof}
        \begin{theorem}
        If $f:A\rightarrow B$ is bijective, $A_1\subset A$, and $f(A_1) = B$, then $A_1=A$.
        \end{theorem}
        \begin{proof}
        $\Big[\big[[A_1^c \ne \emptyset]\Rightarrow [f(A_1^c) \ne \emptyset]\big]\land[f(A_1)\cap f(A_1^c) = \emptyset]\Big]\Rightarrow [\exists y\in B:y\notin f(A_1)]$, a contradiction.
        \end{proof}
        \begin{definition}
        If $A$ is a set, then a binary operation $*$ on the set $A$ is a function from $A\times A$ to $A$.
        \end{definition}
        \begin{definition}
        A binary operation $*$ is said to be associative if and only if $a*(b*c) = (a*b)*c$.
        \end{definition}
        \begin{definition}
        An element $e\in A$ is said to be an identity element if and only if for all $a\in A$, $e*a = a*e = a$.
        \end{definition}
        \begin{definition}
        An element $b\in A$ is said to be an inverse of $a$ if and only if $a*b=b*a = e$. We write $b=a^{-1}$
        \end{definition}
        \begin{ldefinition}{Groups}
            A group is a set $G$ with a binary operation $*$,
            denoted $(G,*)$, such that: 
            \begin{enumerate}
                \item There exists an identity element $e$.
                \item For every element $a\in{A}$, there is an inverse element.
                \item The binary operation $*$ is associative.
            \end{enumerate}
        \end{ldefinition}
        Note that it is not necessarily true that $a*b = b*a$.
        These are special groups that are called Abelian.
        \begin{theorem}
            If $(G,*)$ is a group and $e$ is the identity, then it is unique.
        \end{theorem}
        \begin{proof}
            For if $e$ and $e'$ are identities, then:
            \begin{equation}
                e'=e'*e=e
            \end{equation}
            Therefore, etc.
        \end{proof}
        \begin{theorem}
            \label{thm:Group_Theory_Inverses_Are_Unique}
            If $(G,*)$ is a group, $a\in{G}$, and if $a^{-1}$ and
            $a'^{-1}$ are inverses of $a$, then $a^{-1}=a'^{-1}$.
        \end{theorem}
        \begin{proof}
            For if $e$ is the identity, and
            $a^{-1}$ and $a'^{-1}$ are inverses of $a$, then:
            \begin{align}
                a^{-1}&=a^{-1}*e
                \tag{Identitive Property}\\
                &=a^{-1}*(a*a'^{-1})
                \tag{Inverse Property}\\
                &=(a^{-1}*a)*a'^{-1}
                \tag{Associative Property}\\
                &=e*a'^{-1}
                \tag{Inverse Property}\\
                &=a'^{-1}
                \tag{Identitive Property}
            \end{align}
            Therefore, etc.
        \end{proof}
        \begin{theorem}
            If $(G,*)$ is a group, and if $e\in{G}$ is the identity,
            then $e^{-1}=e$.
        \end{theorem}
        \begin{proof}
            For $e=e*e$, and by Thm.~\ref{thm:Group_Theory_Inverses_Are_Unique}
            inverses are unique. Therefore, etc.
        \end{proof}
        \begin{theorem}
            \label{thm:Group_Theory_Inverse_of_Product}
            If $(G,*)$ is a group and $a,b\in G$, then:
            \begin{equation}
                (a*b)^{-1} = b^{-1}*a^{-1}
            \end{equation}
        \end{theorem}
        \begin{proof}
            For:
            \begin{align}
                (a*b)*(b^{-1}*a^{-1})&=
                a*(b*b^{-1})*a^{-1}
                \tag{Associative Property}\\
                &=a*(e)*a^{-1}
                \tag{Inverse Property}\\
                &=a*a^{-1}
                \tag{Identitive Property}\\
                &=e
                \tag{Inverse Property}
            \end{align}
            Thus $b^{-1}*a^{-1}$ is a right-inverse of $a*b$.
            But since $(G,*)$ is a group, right-inverses are
            left-inverses, and therefore $b^{-1}*a^{-1}$ is
            an inverse of $a*b$. But by
            Thm.~\ref{thm:Group_Theory_Inverses_Are_Unique},
            inverses are unique. Therefore, etc.
        \end{proof}
        \begin{theorem}
            If $(G,*)$ is a group and $a\in{G}$, then:
            \begin{equation}
                (a^{-1})^{-1}=a
            \end{equation}
        \end{theorem}
        \begin{proof}
            For:
            \begin{align}
                a^{-1}*(a^{-1})^{-1}
                &=(a^{-1}* a)^{-1}
                \tag{Thm.~\ref{thm:Group_Theory_Inverse_of_Product}}\\
                &=e
                \tag{Inverse Property}
            \end{align}
            From uniqueness, $(a^{-1})^{-1}=a$.
        \end{proof}
        \begin{definition}
        If $\langle G, * \rangle$ and $\langle G',\circ \rangle$ are groups and $f:G\rightarrow G'$ is a bijective function, then $f$ is said to be an isomorphism between $\langle G, *\rangle$ and $\langle G',\circ \rangle$ if and only if for all $a,b\in G$, $f(a*b) =f(a)\circ f(b)$.
        \end{definition}
        \begin{definition}
        $\langle G, *\rangle$ and $\langle G', \circ \rangle$ are said to be isomorphic if and only if there is an isomorphism between them.
        \end{definition}
        \begin{theorem}
        If $\langle G, * \rangle$ and $\langle G', \circ \rangle$ are isomorphic with identities $e_*$ and $e_{\circ}$ are the identities, then $f(e_*) = e_{\circ}$.
        \end{theorem}
        \begin{proof}
        $\forall a\in G,\ f(a)=f(a* e_*) = f(a)\circ f(e_*)$ as $f$ is an isomorphism. As identities are unique, $f(e_*) = e_{\circ}$.
        \end{proof}
        \begin{theorem}
        If $\langle G, * \rangle$ and $\langle G', \circ \rangle$ are isomorphic, with isomorphism $f$, and if $a\in G$, then $f(a^{-1}) = f(a)^{-1}$.
        \end{theorem}
        \begin{proof}
        For $e_{\circ}=f(e_*) = f(a*a^{-1}) = f(a^{-1}*a) = f(a)\circ f(a^{-1})=f(a^{-1})\circ f(a)$. As inverses are unique, $f(a^{-1})=f(a)^{-1}$.
        \end{proof}
        \begin{definition}
        A binary operation $*$ on a set $A$ is said to be commutative if and only for all $a,b\in A$, $a*b = b*a$.
        \end{definition}
        \begin{definition}
        A field is a set $F$ with two operations $+$ and $\cdot$, denoted $\langle F, +,\cdot \rangle$, with the following properties:
        \begin{enumerate}
        \item $a+b=b+a$ \hfill [Addition is Commutative]
        \item $a+(b+c)=(a+b)+c$ \hfill [Addition is Associative]
        \item $a\cdot b = b\cdot a$ \hfill [Multiplication is Commutative]
        \item $a\cdot (b\cdot c) = (a\cdot b)\cdot c$ \hfill [Multiplication is Associative]
        \item There is a $0\in F$ such that $0+a=a$ for all $a\in F$ \hfill [Existence of Additive Identity]
        \item There is a $1\in F$ such that $1\cdot a = a$ for all $a\in F$ \hfill [Existence of Multiplicative Identity]
        \item For each $a\in F$ there is a $b\in F$ such that $a+b = 0$. $b$ is denoted $-a$ \hfill [Existence of Additive Inverses]
        \item For each $a\in F$, $a\ne 0$ there is a $b\in F$ such that $a\cdot b = 1$. $b$ is denoted $a^{-1}$. \hfill [Existence of Multiplicative Inverses]
        \item $a\cdot(b+c) = a\cdot b + a\cdot c$ \hfill [Distributive Property]
        \end{enumerate}
        \end{definition}
        \begin{definition}
        A subfield of a field $\langle F,+,\cdot \rangle$ is a set $K\subset F$, such that $\langle K, +,\cdot \rangle$ is a field.
        \end{definition}
        \begin{theorem}
        In a field, $0$ and $1$ are unique.
        \end{theorem}
        \begin{proof}
        For suppose not, and let $0'$ and $1'$ be other identities. Then $1'=1'\cdot 1 = 1$ and $0'=0'+0=0$.
        \end{proof}
        \begin{theorem}
        For any field $\langle F,+,\cdot \rangle$, for any $a\in F$, $a\cdot 0 = 0$.
        \end{theorem}
        \begin{proof}
        For $0 = a\cdot 0 + (-a\cdot 0) = a\cdot(0+0) +(-a\cdot 0) = a\cdot 0 + a\cdot 0 + (-a\cdot 0) = a\cdot 0$. Thus, $a\cdot 0 = 0$.
        \end{proof}
        \begin{remark}
        If $1=0$, then $a=a\cdot 1 = a\cdot 0 = 0$, and thus every element is zero. A very boring field.
        \end{remark}
        \begin{corollary}
        In a field $\langle F, +,\cdot \rangle$, if $0\ne 1$, then $0$ has no inverse.
        \end{corollary}
        \begin{proof}
        For let $a$ be such an inverse. Then $a\cdot 0 = 1$. But for any element of $F$, $a \cdot 0 = 0$. But $0\ne 1$, a contradiction.
        \end{proof}
        \begin{theorem}
        If $a+b = 0$, then $b= (-1)\cdot a$ where $(-1)$ is the solution to $1+(-1)=0$.
        \end{theorem}
        \begin{proof}
        $a+(-1)a = a(1+(-1)) = a\cdot 0 = 0$. From uniqueness, $b=(-1)a$. We may thus write additive inverses as $-a$
        \end{proof}
        \begin{definition}
        Given two fields $\langle F,+,\cdot \rangle$ and $\langle F', +',\times \rangle$, a bijection function $f:F\rightarrow F'$ is said to be a field isomorphism if and only if for allelements $a,b\in F$, $f(a+b)=f(a)+'f(b)$, and $f(a\cdot b) = f(a)\times f(b)$
        \end{definition}
        \begin{definition}
        $\langle F,+,\cdot \rangle$ and $\langle F', +',\times \rangle$, are said to be isomorphic if and only if they have an isomorphism.
        \end{definition}
        \begin{theorem}
        Given an ismorphism between two fields $\langle F,+,\cdot \rangle$ and $\langle F', +',\times \rangle$, $f(1) = 1'$ and $f(0) = 0'$.
        \end{theorem}
        \begin{proof}
        For let $x\in F$. Then $f(x)=f(x\cdot 1) = f(x)\times f(1)$, and $f(x)=f(x+0) = f(x)+'f(0)$. Therefore, etc.
        \end{proof}
        \begin{theorem}
        In a field $\langle F,+,\cdot \rangle$, $(a+ b)^2 = a^2 + 2ab + b^2$ ($2$ being the solution to $1+1$).
        \end{theorem}
        \begin{proof}
        For $(a+b)^2 = (a+b)(a+b) = a(a+b)+b(a+b) = a^2 + ab + ba + b^2 = a^2 +ab(1+1)+b^2 = a^2 + 2ab + b^2$.
        \end{proof}
        \begin{definition}
            A group is a set $G$ and a binary relation $*$
            on $G$, denoted $(G,*)$, such that:
            \begin{enumerate}
                \item For all ${a,b,c}\in{G}$, $(a*b)*c=a*(b*c)$
                \item There is an ${e}\in{G}$ such that for all
                    ${a}\in{G}$, $a*e=e*a=a$.
                \item For all ${a}\in{G}$ there is a ${b}\in{G}$
                    such that $a*b=e$
            \end{enumerate}
        \end{definition}
        \begin{definition}
            An Abelian group is a group $(G,*)$ such that
            $*$ is commutative.
        \end{definition}
        \begin{example}
            $G=\{1\}$ is an Abelian group
            under multiplication.
            This is the trivial group.
        \end{example}
        \begin{theorem*}
            If $(G,*)$ is a group, then the following are true:
            \begin{enumerate}
                \begin{multicols}{2}
                    \item The identity ${e}\in{G}$ is unique.
                    \item If $a*b=a*c$, then $b=c$.
                    \item If $b*a=c*a$, then $b=c$.
                    \item Inverses $a^{-1}$ are unique.
                    \item $\forall_{{a,b}\in{G}}%
                        \exists_{{x}\in{G}}:a*x=b$
                    \item $(a*b)^{-1}=b^{-1}*a^{-1}$
                \end{multicols}
            \end{enumerate}
        \end{theorem*}
        \begin{definition}
            The order of a group is number of elements in the
            group.
        \end{definition}
        \begin{definition}
            A group of finite order, or a finite group,
            is a group with finitely many elements.
        \end{definition}
        \begin{definition}
            The direct product of two groups $(G,*)$ and
            $(H,\circ)$ is the group  $({G}\times{H},\star)$
            where $\star$ is the binary operation defined by
            $(g_{1},h_{1})\star(g_{2},h_{2})%
             =(g_{1}*g_{2},{h_{1}}\circ{h_{2}})$
        \end{definition}
        \begin{definition}
            A permutation group on $n$ elements is a
            group whose elements are permutations of
            $n$ elements.
        \end{definition}
        \begin{definition}
            The symmetric group on $n$ elements,
            denoted $S_{n}$, is the group formed by
            permuting $n$ elements.
        \end{definition}
        \begin{definition}
            A homomorphism from a group $(G,*)$ to
            a group $(H,\circ)$ is a function
            $h:{G}\rightarrow{H}$ such that for all
            ${a,b}\in{G}$, $h(a*b)={h(a)}\circ{h(b)}$
        \end{definition}
        \begin{definition}
            An epimorphism from a group $(G,*)$ to
            a group $(H,\circ)$ is a homomorphism
            $h:{G}\rightarrow{H}$ such that
            $h$ is surjective.
        \end{definition}
        \begin{definition}
            A monomorphism from a group $(G,*)$ to
            a group $(H,\circ)$ is a homomorphism
            $h:{G}\rightarrow{H}$ such that
            $h$ is injective.
        \end{definition}
        \begin{definition}
            An isomorphism from a group $(G,*)$ to
            a group $(H,\circ)$ is a homomorphism
            $h:{G}\rightarrow{H}$ such that
            $h$ is bijective.
        \end{definition}
        \begin{definition}
            A ring is a set $R$ and two binary operations
            on $R$, denoted $(R,\cdot,+)$, such that:
            \begin{enumerate}
                \begin{multicols}{3}
                \item $(R,+)$ is an Abelian group.
                \item $a\cdot({b}\cdot{c})%
                       =({a}\cdot{b})\cdot{c}$
                \item ${a}\cdot(b+c)%
                       ={a}\cdot{b}+{a}\cdot{c}$
                \end{multicols}
            \end{enumerate}
        \end{definition}
        \begin{definition}
            A ring with identity is a ring $(R,\cdot,+)$
            such that there is a ${1}\in{R}$ such that for
            all ${a}\in{R}$, ${a}\cdot{1}={1}\cdot{a}=a$.
        \end{definition}
        \begin{remark}
            Left and right identities are elements such
            that ${e_{L}}\cdot{a}=a$ and ${e_{R}}\cdot{a}=a$.
            If inverses $a_{L}^{-1}$ and $a_{R}^{-1}$ exist
            for $a$, then $a_{L}^{-1}=a_{R}^{-1}$. That is,
            the inverse is the same for both right and left
            identities.
        \end{remark}
        \begin{definition}
            A commutative ring is a ring $(R,\cdot,+)$ such that
            $\cdot$ is commutative.
        \end{definition}
        \begin{definition}
            A commutative ring with identity is a
            ring with identity such that $\cdot$
            is commutative.
        \end{definition}
        \begin{definition}
            A Field is a commutative ring with identity
            $(F,\cdot,+)$
            such that for all ${a}\in{F}$ such that
            $a$ is not an identity with respect to $+$,
            there is a $b\in{F}$ such that ${a}\cdot{b}=1$.
        \end{definition}
        \begin{definition}
            Equivalent sets are sets $A$ and $B$ such that
            there exists a bijective function
            $f:{A}\rightarrow{B}$
        \end{definition}
        \begin{definition}
            A finite set is a set $A$ such that there
            is an ${n}\in{\mathbb{N}}$ such that $A$
            is equivalent to $\mathbb{Z}_{n}$.
        \end{definition}
        \begin{definition}
            A countable set (Or a denumerable set) is a
            set $A$ that is equivalent to $\mathbb{N}$.
        \end{definition}
        \begin{definition}
            An uncountable set is a set that is neither finite
            nor countable.
        \end{definition}
        \begin{theorem*}
            Set Equivalence is an equivalence relation.
        \end{theorem*}
        This equivalence allows to classify all sets by the
        number of elements they contain or, more generally,
        by their cardinality. We say that two sets $A$ and
        $B$ have the same cardinality, denoted
        $\Card(A)$, if and only if $A$ and $B$ are equivalent.
        \begin{theorem*}
            The following are true:
            \begin{enumerate}
                \begin{multicols}{2}
                    \item $\Card(A)=0$ if and only if
                          $A=\emptyset$.
                    \item If ${A}\sim{\mathbb{Z}_{n}}$, then
                          $\Card(A)=n$.
                \end{multicols}
            \end{enumerate}
        \end{theorem*}
        \begin{definition}
            A finite cardinal number is a cardinal
            number of a finite set.
        \end{definition}
        \begin{definition}
            The standard ordering on the finite cardinal
            number is $0<1<\hdots<n<n+1<\hdots$
        \end{definition}
        Thus, if $A$ and $B$ are finite sets, then we write
        $\Card(A)<\Card(B)$ if $A$ is equivalent to a
        subset of $B$ but not equivalent to $B$.
        We take this notion and generalize to
        all sets. For $A$ and $B$, we write
        $\Card(A)<\Card(B)$ if $A$ is equivalent to a subset
        of $B$ but is not equivalent to $B$. This is the
        same as saying $A$ is equivalent to a subset of $B$,
        but $B$ is not equivalent to a subset of $A$.
        We write that
        $\Card(A)\leq\Card(B)$ is $A$ is equivalent to a
        subset of $B$.
        \begin{theorem*}[Schr\"{o}der-Bernstein Theorem]
            If $A$ and $B$ are sets such that
            $\Card(A)\leq\Card(B)$ and
            $\Card(B)\leq\Card(A)$, then
            $\Card(A)=\Card(B)$.
        \end{theorem*}
        \begin{theorem*}
            The following are true:
            \begin{enumerate}
                \item If $\Card(A)\leq\Card(B)$ and
                      $\Card(B)\leq\Card(A)$, then
                      $\Card(A)\leq\Card(C)$.
                \item If $\Card(A)\leq\Card(B)$, then
                      $\Card(A)+\Card(C)\leq\Card(B)+\Card(C)$
            \end{enumerate}
        \end{theorem*}
        \begin{theorem*}
            If ${A}\subset{B}\subset{C}$, and
            $\Card(A)=\Card(C)$, then $\Card(B)=\Card(C)$
        \end{theorem*}
        \begin{theorem*}
            If $f:{X}\rightarrow{Y}$ is a function,
            then $\Card(f(X))\leq\Card(X)$.
        \end{theorem*}
        \begin{proof}
            Note that $f^{-1}(\{y\})$ creates a set of
            mutually disjoint subsets of $X$. By the
            axiom of choice there is a function
            $F:{f(X)}\rightarrow{X}$
            such that for all ${y}\in{f(X)}$,
            ${F(y)}\in{f^{-1}(\{y\})}$. But since these
            sets are disjoint, $F$ is injective.
            Thus, $f(X)$ is equivalent to a subset of $X$.
            Therefore, $\Card(f(X))\leq\Card(X)$.
        \end{proof}
        The Schr\"{o}der-Bernstein theorem can be restated
        equivalently as ``If $A$ is equivalent to a subset
        of $B$ and $B$ is equivalent to a subset of $A$,
        then $A$ is equivalent to $B$.''
        Addition and multiplication of finite cardinals
        follows directly from the standard arithmetic
        for the natural numbers. For cardinals of infinite
        sets, the arithmetic becomes a little more complicated.
        \begin{definition}
            The sum of two cardinal numbers is the
            cardinality of the union of two disjoint sets $A$
            and $B$. That is, if ${A}\cap{B}=\emptyset$, then
            $\Card(A)+\Card(B)=\Card({A}\cup{B})$.
        \end{definition}
        \begin{theorem*}
            If $a$ and $b$ are distinct cardinal numbers,
            then there exists sets $A$ and $B$ such that
            ${A}\cap{B}=\emptyset$, $\Card(A)=a$, and
            $\Card(B)=b$.
        \end{theorem*}
        \begin{theorem*}
            If $A,B,C,$ and $D$ are sets such that
            $\Card(A)=\Card(C)$, $\Card(B)=\Card(D)$,
            and if ${A}\cap{B}=\emptyset$ and
            ${C}\cap{D}=\emptyset$, then
            $\Card({A}\cup{B})=\Card({C}\cup{D})$.
        \end{theorem*}
        \begin{theorem*}
            If $x,y,$ and $z$ are cardinal numbers, then
            $x+y=y+x$ and $x+(y+z)=(x+y)+z$.
        \end{theorem*}
        \begin{notation}
            The carinality of the set of natural numbers
            is denoted $\aleph_{0}$. That is,
            $\Card(\mathbb{N})=\aleph_{0}$
        \end{notation}
        \begin{example}
            Find the cardinal sum of $2$ and $5$. Let
            $N_{2}=\{1,2\}$ and $N_{5}=\{3,4,5,6,7\}$.
            Then $N_{2}$ and $N_{5}$ are disjoint,
            $\Card(N_{2})=2$ and $\Card(N_{5})=5$.
            Therefore $2+5=\Card(N_{2}\cup{N_{5}})$.
            But ${N_{2}}\cup{N_{5}}$ is just $\mathbb{Z}_{7}$,
            and $\Card(\mathbb{Z}_{7})=7$. Thus, $2+5=7$.
        \end{example}
        \begin{theorem*}
            If $n$ and $m$ are finite cardinalities,
            then the cardinal sum of $n$ and $m$ is the
            integer $n+m$, where $+$ is the usual
            arithmetic addition.
        \end{theorem*}
        \begin{example}
            Compute the cardinal sum
            $\aleph_{0}+\aleph_{0}$. Let
            $\mathbb{N}_{e}$ be the set of even natural
            numbers, and let $\mathbb{N}_{o}$ be the set
            of odd natural numbers. Then
            $\Card(\mathbb{N}_{e})=\aleph_{0}$,
            $\Card(\mathbb{N}_{o})=\aleph_{0}$, and
            ${\mathbb{N}_{o}}\cap{\mathbb{N}_{e}}=\emptyset$.
            Thus
            $\aleph_{0}+\aleph_{0}%
             =\Card({\mathbb{N}_{o}}\cup{\mathbb{N}_{e}})$.
            But
            ${\mathbb{N}_{o}}\cup{\mathbb{N}_{e}}%
             =\mathbb{N}$ and $\Card(\mathbb{N})=\aleph_{0}$.
            Therefore, $\aleph_{0}+\aleph_{0}=\aleph_{0}$.
        \end{example}
        \begin{example}
            Find $n+\aleph_{0}$, where $n\in\mathbb{N}$.
            We have that
            $\Card(\mathbb{Z}_{n}z)=n$ and
            $\Card(\mathbb{N}\setminus\mathbb{Z}_{n})%
             =\aleph_{0}$
            But then
            $n+\aleph_{0}=%
             \Card(\mathbb{Z}_{n}\cup%
             \mathbb{N}\setminus\mathbb{Z}_{n})%
             =\Card(\mathbb{N})=\aleph_{0}$.
            Therefore, $n+\aleph_{0}=\aleph_{0}$.
        \end{example}
        \begin{definition}
            The cardinality of the continuum,
            denoted $\mathfrak{c}$, is the
            cardinality of the set of real numbers.
            That is, $\mathfrak{c}=\Card(\mathbb{R})$.
        \end{definition}
        \begin{theorem*}
            $\Card([0,1])=\mathfrak{c}$.
        \end{theorem*}
        \begin{theorem*}
            $\Card\big((0,1)\big)=\mathfrak{c}$.
        \end{theorem*}
        \begin{theorem*}
            $\mathbb{R}$ is uncountable. That is,
            $\mathfrak{c}>\aleph_{0}$.
        \end{theorem*}
        \begin{theorem*}
            $\mathfrak{c}+\aleph_{0}=\mathfrak{c}$.
        \end{theorem*}
        \begin{proof}
            We have $\Card((0,1))=\mathfrak{c}$ and
            $\Card(\mathbb{N})=\aleph_{0}$. But
            $(0,1)\cap\mathbb{N}=\emptyset$, and thus
            $\aleph_{0}+\mathfrak{c}%
             =\Card((0,1)\cup\mathbb{N})$.
            But $\mathbb{R}\sim(0,1)$ and
            $\mathbb{N}\cup(0,1)\subset\mathbb{R}$.
            By the Schr\"{o}der-Bernstein theorem,
            $\mathbb{N}\cup(0,1)\sim\mathbb{R}$.
            Therefore, etc.
        \end{proof}
        \begin{definition}
            The product of two cardinal numbers $a$ and $b$
            is the cardinality of the cartesian product
            of two set $A$ and $B$ such that
            $\Card(A)=a$ and $\Card(B)=b$. That is,
            ${a}\times{b}=\Card({A}\times{B})$.
        \end{definition}
        \begin{theorem*}
            The following are true of cardinal numbers:
            \begin{enumerate}
                \begin{multicols}{3}
                    \item $xy=yx$
                    \item $x(yz)=(xy)z$
                    \item $x(y+z)=xy+xz$
                \end{multicols}
            \end{enumerate}
        \end{theorem*}
        \begin{proof}[Proof of Part 3]
            Let $A,B,$ and $C$ be disjoint.
            Then
            ${A}\times{({B}\cup{C})}%
             =({A}\times{B})\cup({A}\times{C})$, and thus
            $\Card({A}\times{({B}\cup{C})})%
             =\Card(({A}\times{B})\cup({A}\times{C}))$.
            But ${A}\times{B}$ and ${A}\times{C}$ are disjoint.
            Thus we have
            $\Card(({A}\times{B})\cup({A}\times{C}))%
             =\Card({A}\times{B})+\Card({A}\times{C})$.
            Therefore, etc.
        \end{proof}
        \begin{theorem*}
            If $\Card(T)=x$ and
            $F:{T}\rightarrow{\mathcal{P}(T)}$
            is a set-valued mapping such that for all
            ${t}\in{T}$ we have that
            $\Card(F(t))=y$ and
            for all ${t}\ne{t}$,
            ${F(t)}\cap{F(t')}=\emptyset$, then
            $\Card(\cup_{t=1}^{N}F(t))=xy$
        \end{theorem*}
        \begin{example}
            Let $f:{\mathbb{N}^{2}}\rightarrow{\mathbb{N}}$
            be defined by $f(n,m)=2^{n}3^{m}$.
            Then $f$ is injective, since $2$ and $3$
            are coprime. Therefore,
            $\aleph_{0}\times\aleph_{0}=\aleph_{0}$.
        \end{example}
        \begin{example}
            Show that $\mathbb{R}^{2}\sim\mathbb{R}$.
            Let $f:\mathbb{R}^{2}\rightarrow\mathbb{R}$
            be the rather bizarre function defined by the image
            $f(x_{0}.x_{1}x_{2}\hdots,y_{0}.y_{1}y_{2}\hdots)%
             =x_{0}y_{0}.x_{0}y_{0}x_{1}y_{1}\hdots$ Then
            $f$ is inective. But the mapping
            $g:\mathbb{R}\rightarrow\mathbb{R}^{2}$
            defined by $g(x)=(x,0)$ is also injective.
            By Schr\"{o}der-Bernstein,
            $\mathbb{R}^{2}\sim\mathbb{R}$.
        \end{example}
        \begin{definition}
           Order isomorphic set are two sets $A$ and $B$
           with well orders $<_{A}$ and $<_{B}$ such that
           there exists a bijection $f:{B}A\rightarrow{B}$
           such that for all $a_{1},a_{2}\in{A}$ such that
           $a_{1}<_{A}a_{2}$, $f(a_{1})<_{B}f(a_{2})$.
        \end{definition}
        \begin{theorem*}
           Order-Isomorphism is an equivalence relation.
        \end{theorem*}
        To every well ordered set, an ordinal number is
        assigned, denoted $\Ord(A,<_{A})$. Conversely,
        for every ordinal number there is a set with a
        well order corresponding to it. Two ordinal numbers
        are equal if and only if the well-ordered sets
        corresponding to them are order isomorphic.
        That is,
        $\Ord(A,<_{A})=\Ord(B,<_{B})$ if and only if
        $(A,<_{A})$ and $(B,<_{B})$ are order isomorphic.
        \begin{theorem*}
           If $(A,<_{A})$ and $(B,<_{B})$ are well ordered
           sets, and if $\Card(A)=\Card(B)$, then
           $(A,<_{A})$ and $(B,<_{B})$ are order
           isomorphic.
        \end{theorem*}
        The ordinal number of the empty set is $0$. The
        ordinal number of a finite set of $n$ elements with
        a well ordering is denoted $n\in\mathbb{N}$.
        The ordinal for the natural numbers $\mathbb{N}$
        with their usual well-ordering is denoted $\omega$.
        A given well-ordered set has only one cardinal number,
        but it is possible for it to have two ordinal numbers.
        \begin{definition}
           An ordinal number $\alpha$ is less than or equal
           to an ordinal number $\beta$ if there are
           well-ordered sets $(A,<_{A})$ and $(B,<_{B})$
           such that $\alpha=\Ord((A,<_{A}))$ and
           $\beta=\Ord(B,<_{B})$, and $(A,<_{B})$ is
           order isomorphic to subset of
           $(B,<_{B})$.
        \end{definition}
        \begin{theorem*}
           The only order isomorphism from a well ordered
           set $(A,<_{A})$ to itself is the identity
           isomorphism.
        \end{theorem*}
        \begin{theorem*}
           If $\alpha$ and $\beta$ are ordinal numbers and
           ${\alpha}\leq{\beta}$ and ${\beta}\leq{\alpha}$,
           then $\alpha=\beta$.
        \end{theorem*}
        \begin{theorem*}
           If $\alpha$ and $\beta$ are ordinal numbers,
           either ${\alpha}\leq{\beta}$, or
           ${\beta}\leq{\alpha}$.
        \end{theorem*}
        \begin{theorem*}
           If $\alpha$ and $\beta$ are ordinal numbers,
           either $\alpha<\beta$, $\beta<\alpha$, or
           $\alpha=\beta$.
        \end{theorem*}
        \begin{definition}
           The total ordering relation of a
           well-ordered set $(A,<_{A})$ with respect
           to a well-ordered set $(B,<_{B})$ is the ordering
           on the set ${A}\cup{B}$ defined as: For all
           $a_{1},a_{2}\in{A}$ such that $a_{1}<_{A}a_{2}$,
           $a_{1}<_{*}a_{2}$, for all $b_{1},b_{2}\in{B}$
           such that $b_{1}<_{B}b_{2}$, $b_{1}<_{*}b_{2}$,
           and for all ${a}\in{A}$ and ${b}\in{B}$,
           ${a}<_{*}{b}$.
        \end{definition}
        \begin{theorem*}
           The total ordering relation $<_{*}$ on the set
           ${A}\cup{B}$ is a well-ordering.
        \end{theorem*}
        \begin{definition}
           The ordinal sum of two ordinal numbers
           $\Ord((A,<_{A}))$ and $\Ord((B,<_{B}))$,
           where $A$ and $B$ are disjoint,
           is the ordinal number
           $\Ord(({A}\cup{B},<_{*}))$.
        \end{definition}
        \begin{theorem*}
           The following are true of ordinal numbers:
           \begin{enumerate}
               \begin{multicols}{3}
                   \item $\alpha<\beta\Rightarrow%
                          \alpha+\gamma<\beta+\gamma$
                   \item $(\alpha+\beta)+\gamma%
                          =\alpha+(\beta+\gamma)$
                   \item $\alpha+\beta=\alpha+\gamma%
                          \Rightarrow\beta=\gamma$.
               \end{multicols}
           \end{enumerate}
        \end{theorem*}
        \begin{definition}
           The lexicographic ordering on the cartesian
           product of well ordered set $(A,<_{A})$ and
           $(B,<_{B})$ is the ordering on
           ${A}\times{B}$ defined by: If ${a}<_{A}{x}$,
           then $(a,b)<_{*}(x,y)$ for all $b,y\in{B}$, and
           if $a=x$ and $b<_{B}y$, then $(a,b)<_{*}(x,y)$.
        \end{definition}
        \begin{theorem*}
           If $(A,<_{A})$ and $(B,<_{B})$ are well ordered
           sets, then the lexicographic ordering
           on ${A}\times{B}$ is a well ordering.
        \end{theorem*}
        \begin{definition}
           The ordinal product of two ordinal numbers
           $\Ord((A,<_{A}))$ and $\Ord((B,<_{B}))$,
           is $\Ord(({A}\times{B},<_{*}))$
        \end{definition}
        \begin{theorem*}
           The following are true of ordinal numbers:
           \begin{enumerate}
               \begin{multicols}{2}
                   \item $\alpha(\beta\gamma)%
                          =(\alpha\beta)\gamma$
                   \item $\alpha(\beta+\gamma)%
                          =\alpha\beta+\alpha\gamma$
               \end{multicols}
           \end{enumerate}
        \end{theorem*}
        \begin{definition}
           Relatively prime integers are integers
           $a,b\in\mathbb{N}$ such that $\gcd(a,b)=1$.
        \end{definition}
        \begin{theorem*}
           If $p$ is prime and $a\in\mathbb{N}$ is
           such that $p$ does not divide $a$, then $a$ and $p$
           are relatively prime.
        \end{theorem*}
        \begin{theorem*}
           There are infinitely many prime numbers.
        \end{theorem*}
        \begin{theorem*}
           If $a\in\mathbb{N}$, $a>1$, then either
           $a$ is a prime number, or $a$ is the product
           of finitely many primes.
        \end{theorem*}
        \begin{theorem*}
           If $a\in\mathbb{N}$, $a>1$, and if $a$ is not
           prime, then the prime expansion of $a$ is
           unique.
        \end{theorem*}
        \begin{definition}
           A diophantine equation is an equation whose
           solutions are required to be integers.
        \end{definition}
        \begin{definition}
           A linear diophantine equation in two variables
           $x$ and $y$ is an equation
           $ax+by=c$, where $a,b,c\in\mathbb{Z}$.
        \end{definition}
        \begin{theorem*}
           If $a,b,c\in\mathbb{Z}$ $d=\gcd(a,b)$,
           and if $d$ does not divide $c$,
           then $ax+by=c$ has no integral solutions.
        \end{theorem*}
        \begin{theorem*}
           If $a,b,c\in\mathbb{Z}$ $d=\gcd(a,b)$,
           and if $d$ divides $c$,
           then $ax+by=c$ has infinitely many solutions.
        \end{theorem*}