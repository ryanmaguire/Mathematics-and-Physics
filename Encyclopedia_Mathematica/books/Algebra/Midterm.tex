%------------------------------------------------------------------------------%
\documentclass[crop=false,class=article]{standalone}                           %
%------------------------------Preamble----------------------------------------%
\makeatletter                                                                  %
    \def\input@path{{../../../}}                                               %
\makeatother                                                                   %
%---------------------------Packages----------------------------%
\usepackage{geometry}
\geometry{b5paper, margin=1.0in}
\usepackage[T1]{fontenc}
\usepackage{graphicx, float}            % Graphics/Images.
\usepackage{natbib}                     % For bibliographies.
\bibliographystyle{agsm}                % Bibliography style.
\usepackage[french, english]{babel}     % Language typesetting.
\usepackage[dvipsnames]{xcolor}         % Color names.
\usepackage{listings}                   % Verbatim-Like Tools.
\usepackage{mathtools, esint, mathrsfs} % amsmath and integrals.
\usepackage{amsthm, amsfonts, amssymb}  % Fonts and theorems.
\usepackage{tcolorbox}                  % Frames around theorems.
\usepackage{upgreek}                    % Non-Italic Greek.
\usepackage{fmtcount, etoolbox}         % For the \book{} command.
\usepackage[newparttoc]{titlesec}       % Formatting chapter, etc.
\usepackage{titletoc}                   % Allows \book in toc.
\usepackage[nottoc]{tocbibind}          % Bibliography in toc.
\usepackage[titles]{tocloft}            % ToC formatting.
\usepackage{pgfplots, tikz}             % Drawing/graphing tools.
\usepackage{imakeidx}                   % Used for index.
\usetikzlibrary{
    calc,                   % Calculating right angles and more.
    angles,                 % Drawing angles within triangles.
    arrows.meta,            % Latex and Stealth arrows.
    quotes,                 % Adding labels to angles.
    positioning,            % Relative positioning of nodes.
    decorations.markings,   % Adding arrows in the middle of a line.
    patterns,
    arrows
}                                       % Libraries for tikz.
\pgfplotsset{compat=1.9}                % Version of pgfplots.
\usepackage[font=scriptsize,
            labelformat=simple,
            labelsep=colon]{subcaption} % Subfigure captions.
\usepackage[font={scriptsize},
            hypcap=true,
            labelsep=colon]{caption}    % Figure captions.
\usepackage[pdftex,
            pdfauthor={Ryan Maguire},
            pdftitle={Mathematics and Physics},
            pdfsubject={Mathematics, Physics, Science},
            pdfkeywords={Mathematics, Physics, Computer Science, Biology},
            pdfproducer={LaTeX},
            pdfcreator={pdflatex}]{hyperref}
\hypersetup{
    colorlinks=true,
    linkcolor=blue,
    filecolor=magenta,
    urlcolor=Cerulean,
    citecolor=SkyBlue
}                           % Colors for hyperref.
\usepackage[toc,acronym,nogroupskip,nopostdot]{glossaries}
\usepackage{glossary-mcols}
%------------------------Theorem Styles-------------------------%
\theoremstyle{plain}
\newtheorem{theorem}{Theorem}[section]

% Define theorem style for default spacing and normal font.
\newtheoremstyle{normal}
    {\topsep}               % Amount of space above the theorem.
    {\topsep}               % Amount of space below the theorem.
    {}                      % Font used for body of theorem.
    {}                      % Measure of space to indent.
    {\bfseries}             % Font of the header of the theorem.
    {}                      % Punctuation between head and body.
    {.5em}                  % Space after theorem head.
    {}

% Italic header environment.
\newtheoremstyle{thmit}{\topsep}{\topsep}{}{}{\itshape}{}{0.5em}{}

% Define environments with italic headers.
\theoremstyle{thmit}
\newtheorem*{solution}{Solution}

% Define default environments.
\theoremstyle{normal}
\newtheorem{example}{Example}[section]
\newtheorem{definition}{Definition}[section]
\newtheorem{problem}{Problem}[section]

% Define framed environment.
\tcbuselibrary{most}
\newtcbtheorem[use counter*=theorem]{ftheorem}{Theorem}{%
    before=\par\vspace{2ex},
    boxsep=0.5\topsep,
    after=\par\vspace{2ex},
    colback=green!5,
    colframe=green!35!black,
    fonttitle=\bfseries\upshape%
}{thm}

\newtcbtheorem[auto counter, number within=section]{faxiom}{Axiom}{%
    before=\par\vspace{2ex},
    boxsep=0.5\topsep,
    after=\par\vspace{2ex},
    colback=Apricot!5,
    colframe=Apricot!35!black,
    fonttitle=\bfseries\upshape%
}{ax}

\newtcbtheorem[use counter*=definition]{fdefinition}{Definition}{%
    before=\par\vspace{2ex},
    boxsep=0.5\topsep,
    after=\par\vspace{2ex},
    colback=blue!5!white,
    colframe=blue!75!black,
    fonttitle=\bfseries\upshape%
}{def}

\newtcbtheorem[use counter*=example]{fexample}{Example}{%
    before=\par\vspace{2ex},
    boxsep=0.5\topsep,
    after=\par\vspace{2ex},
    colback=red!5!white,
    colframe=red!75!black,
    fonttitle=\bfseries\upshape%
}{ex}

\newtcbtheorem[auto counter, number within=section]{fnotation}{Notation}{%
    before=\par\vspace{2ex},
    boxsep=0.5\topsep,
    after=\par\vspace{2ex},
    colback=SeaGreen!5!white,
    colframe=SeaGreen!75!black,
    fonttitle=\bfseries\upshape%
}{not}

\newtcbtheorem[use counter*=remark]{fremark}{Remark}{%
    fonttitle=\bfseries\upshape,
    colback=Goldenrod!5!white,
    colframe=Goldenrod!75!black}{ex}

\newenvironment{bproof}{\textit{Proof.}}{\hfill$\square$}
\tcolorboxenvironment{bproof}{%
    blanker,
    breakable,
    left=3mm,
    before skip=5pt,
    after skip=10pt,
    borderline west={0.6mm}{0pt}{green!80!black}
}

\AtEndEnvironment{lexample}{$\hfill\textcolor{red}{\blacksquare}$}
\newtcbtheorem[use counter*=example]{lexample}{Example}{%
    empty,
    title={Example~\theexample},
    boxed title style={%
        empty,
        size=minimal,
        toprule=2pt,
        top=0.5\topsep,
    },
    coltitle=red,
    fonttitle=\bfseries,
    parbox=false,
    boxsep=0pt,
    before=\par\vspace{2ex},
    left=0pt,
    right=0pt,
    top=3ex,
    bottom=1ex,
    before=\par\vspace{2ex},
    after=\par\vspace{2ex},
    breakable,
    pad at break*=0mm,
    vfill before first,
    overlay unbroken={%
        \draw[red, line width=2pt]
            ([yshift=-1.2ex]title.south-|frame.west) to
            ([yshift=-1.2ex]title.south-|frame.east);
        },
    overlay first={%
        \draw[red, line width=2pt]
            ([yshift=-1.2ex]title.south-|frame.west) to
            ([yshift=-1.2ex]title.south-|frame.east);
    },
}{ex}

\AtEndEnvironment{ldefinition}{$\hfill\textcolor{Blue}{\blacksquare}$}
\newtcbtheorem[use counter*=definition]{ldefinition}{Definition}{%
    empty,
    title={Definition~\thedefinition:~{#1}},
    boxed title style={%
        empty,
        size=minimal,
        toprule=2pt,
        top=0.5\topsep,
    },
    coltitle=Blue,
    fonttitle=\bfseries,
    parbox=false,
    boxsep=0pt,
    before=\par\vspace{2ex},
    left=0pt,
    right=0pt,
    top=3ex,
    bottom=0pt,
    before=\par\vspace{2ex},
    after=\par\vspace{1ex},
    breakable,
    pad at break*=0mm,
    vfill before first,
    overlay unbroken={%
        \draw[Blue, line width=2pt]
            ([yshift=-1.2ex]title.south-|frame.west) to
            ([yshift=-1.2ex]title.south-|frame.east);
        },
    overlay first={%
        \draw[Blue, line width=2pt]
            ([yshift=-1.2ex]title.south-|frame.west) to
            ([yshift=-1.2ex]title.south-|frame.east);
    },
}{def}

\AtEndEnvironment{ltheorem}{$\hfill\textcolor{Green}{\blacksquare}$}
\newtcbtheorem[use counter*=theorem]{ltheorem}{Theorem}{%
    empty,
    title={Theorem~\thetheorem:~{#1}},
    boxed title style={%
        empty,
        size=minimal,
        toprule=2pt,
        top=0.5\topsep,
    },
    coltitle=Green,
    fonttitle=\bfseries,
    parbox=false,
    boxsep=0pt,
    before=\par\vspace{2ex},
    left=0pt,
    right=0pt,
    top=3ex,
    bottom=-1.5ex,
    breakable,
    pad at break*=0mm,
    vfill before first,
    overlay unbroken={%
        \draw[Green, line width=2pt]
            ([yshift=-1.2ex]title.south-|frame.west) to
            ([yshift=-1.2ex]title.south-|frame.east);},
    overlay first={%
        \draw[Green, line width=2pt]
            ([yshift=-1.2ex]title.south-|frame.west) to
            ([yshift=-1.2ex]title.south-|frame.east);
    }
}{thm}

%--------------------Declared Math Operators--------------------%
\DeclareMathOperator{\adjoint}{adj}         % Adjoint.
\DeclareMathOperator{\Card}{Card}           % Cardinality.
\DeclareMathOperator{\curl}{curl}           % Curl.
\DeclareMathOperator{\diam}{diam}           % Diameter.
\DeclareMathOperator{\dist}{dist}           % Distance.
\DeclareMathOperator{\Div}{div}             % Divergence.
\DeclareMathOperator{\Erf}{Erf}             % Error Function.
\DeclareMathOperator{\Erfc}{Erfc}           % Complementary Error Function.
\DeclareMathOperator{\Ext}{Ext}             % Exterior.
\DeclareMathOperator{\GCD}{GCD}             % Greatest common denominator.
\DeclareMathOperator{\grad}{grad}           % Gradient
\DeclareMathOperator{\Ima}{Im}              % Image.
\DeclareMathOperator{\Int}{Int}             % Interior.
\DeclareMathOperator{\LC}{LC}               % Leading coefficient.
\DeclareMathOperator{\LCM}{LCM}             % Least common multiple.
\DeclareMathOperator{\LM}{LM}               % Leading monomial.
\DeclareMathOperator{\LT}{LT}               % Leading term.
\DeclareMathOperator{\Mod}{mod}             % Modulus.
\DeclareMathOperator{\Mon}{Mon}             % Monomial.
\DeclareMathOperator{\multideg}{mutlideg}   % Multi-Degree (Graphs).
\DeclareMathOperator{\nul}{nul}             % Null space of operator.
\DeclareMathOperator{\Ord}{Ord}             % Ordinal of ordered set.
\DeclareMathOperator{\Prin}{Prin}           % Principal value.
\DeclareMathOperator{\proj}{proj}           % Projection.
\DeclareMathOperator{\Refl}{Refl}           % Reflection operator.
\DeclareMathOperator{\rk}{rk}               % Rank of operator.
\DeclareMathOperator{\sgn}{sgn}             % Sign of a number.
\DeclareMathOperator{\sinc}{sinc}           % Sinc function.
\DeclareMathOperator{\Span}{Span}           % Span of a set.
\DeclareMathOperator{\Spec}{Spec}           % Spectrum.
\DeclareMathOperator{\supp}{supp}           % Support
\DeclareMathOperator{\Tr}{Tr}               % Trace of matrix.
%--------------------Declared Math Symbols--------------------%
\DeclareMathSymbol{\minus}{\mathbin}{AMSa}{"39} % Unary minus sign.
%------------------------New Commands---------------------------%
\DeclarePairedDelimiter\norm{\lVert}{\rVert}
\DeclarePairedDelimiter\ceil{\lceil}{\rceil}
\DeclarePairedDelimiter\floor{\lfloor}{\rfloor}
\newcommand*\diff{\mathop{}\!\mathrm{d}}
\newcommand*\Diff[1]{\mathop{}\!\mathrm{d^#1}}
\renewcommand*{\glstextformat}[1]{\textcolor{RoyalBlue}{#1}}
\renewcommand{\glsnamefont}[1]{\textbf{#1}}
\renewcommand\labelitemii{$\circ$}
\renewcommand\thesubfigure{%
    \arabic{chapter}.\arabic{figure}.\arabic{subfigure}}
\addto\captionsenglish{\renewcommand{\figurename}{Fig.}}
\numberwithin{equation}{section}

\renewcommand{\vector}[1]{\boldsymbol{\mathrm{#1}}}

\newcommand{\uvector}[1]{\boldsymbol{\hat{\mathrm{#1}}}}
\newcommand{\topspace}[2][]{(#2,\tau_{#1})}
\newcommand{\measurespace}[2][]{(#2,\varSigma_{#1},\mu_{#1})}
\newcommand{\measurablespace}[2][]{(#2,\varSigma_{#1})}
\newcommand{\manifold}[2][]{(#2,\tau_{#1},\mathcal{A}_{#1})}
\newcommand{\tanspace}[2]{T_{#1}{#2}}
\newcommand{\cotanspace}[2]{T_{#1}^{*}{#2}}
\newcommand{\Ckspace}[3][\mathbb{R}]{C^{#2}(#3,#1)}
\newcommand{\funcspace}[2][\mathbb{R}]{\mathcal{F}(#2,#1)}
\newcommand{\smoothvecf}[1]{\mathfrak{X}(#1)}
\newcommand{\smoothonef}[1]{\mathfrak{X}^{*}(#1)}
\newcommand{\bracket}[2]{[#1,#2]}

%------------------------Book Command---------------------------%
\makeatletter
\renewcommand\@pnumwidth{1cm}
\newcounter{book}
\renewcommand\thebook{\@Roman\c@book}
\newcommand\book{%
    \if@openright
        \cleardoublepage
    \else
        \clearpage
    \fi
    \thispagestyle{plain}%
    \if@twocolumn
        \onecolumn
        \@tempswatrue
    \else
        \@tempswafalse
    \fi
    \null\vfil
    \secdef\@book\@sbook
}
\def\@book[#1]#2{%
    \refstepcounter{book}
    \addcontentsline{toc}{book}{\bookname\ \thebook:\hspace{1em}#1}
    \markboth{}{}
    {\centering
     \interlinepenalty\@M
     \normalfont
     \huge\bfseries\bookname\nobreakspace\thebook
     \par
     \vskip 20\p@
     \Huge\bfseries#2\par}%
    \@endbook}
\def\@sbook#1{%
    {\centering
     \interlinepenalty \@M
     \normalfont
     \Huge\bfseries#1\par}%
    \@endbook}
\def\@endbook{
    \vfil\newpage
        \if@twoside
            \if@openright
                \null
                \thispagestyle{empty}%
                \newpage
            \fi
        \fi
        \if@tempswa
            \twocolumn
        \fi
}
\newcommand*\l@book[2]{%
    \ifnum\c@tocdepth >-3\relax
        \addpenalty{-\@highpenalty}%
        \addvspace{2.25em\@plus\p@}%
        \setlength\@tempdima{3em}%
        \begingroup
            \parindent\z@\rightskip\@pnumwidth
            \parfillskip -\@pnumwidth
            {
                \leavevmode
                \Large\bfseries#1\hfill\hb@xt@\@pnumwidth{\hss#2}
            }
            \par
            \nobreak
            \global\@nobreaktrue
            \everypar{\global\@nobreakfalse\everypar{}}%
        \endgroup
    \fi}
\newcommand\bookname{Book}
\renewcommand{\thebook}{\texorpdfstring{\Numberstring{book}}{book}}
\providecommand*{\toclevel@book}{-2}
\makeatother
\titleformat{\part}[display]
    {\Large\bfseries}
    {\partname\nobreakspace\thepart}
    {0mm}
    {\Huge\bfseries}
\titlecontents{part}[0pt]
    {\large\bfseries}
    {\partname\ \thecontentslabel: \quad}
    {}
    {\hfill\contentspage}
\titlecontents{chapter}[0pt]
    {\bfseries}
    {\chaptername\ \thecontentslabel:\quad}
    {}
    {\hfill\contentspage}
\newglossarystyle{longpara}{%
    \setglossarystyle{long}%
    \renewenvironment{theglossary}{%
        \begin{longtable}[l]{{p{0.25\hsize}p{0.65\hsize}}}
    }{\end{longtable}}%
    \renewcommand{\glossentry}[2]{%
        \glstarget{##1}{\glossentryname{##1}}%
        &\glossentrydesc{##1}{~##2.}
        \tabularnewline%
        \tabularnewline
    }%
}
\newglossary[not-glg]{notation}{not-gls}{not-glo}{Notation}
\newcommand*{\newnotation}[4][]{%
    \newglossaryentry{#2}{type=notation, name={\textbf{#3}, },
                          text={#4}, description={#4},#1}%
}
%--------------------------LENGTHS------------------------------%
% Spacings for the Table of Contents.
\addtolength{\cftsecnumwidth}{1ex}
\addtolength{\cftsubsecindent}{1ex}
\addtolength{\cftsubsecnumwidth}{1ex}
\addtolength{\cftfignumwidth}{1ex}
\addtolength{\cfttabnumwidth}{1ex}

% Indent and paragraph spacing.
\setlength{\parindent}{0em}
\setlength{\parskip}{0em}                                                           %
%----------------------------Main Document-------------------------------------%
\begin{document}
    \title{Midterm}
    \author{Ryan Maguire}
    \date{\vspace{-5ex}}
    \maketitle
    \section{}
        Let $(k,+,\cdot)$ be a commutative ring,
        $(M,\boldsymbol{+},\boldsymbol{\cdot})$ be a module over $k$, and let
        $k[x]$ be the ring of polynomials. That is, $k[x]$ is the space of
        finitely supported sequences of ring elements in $k$ equipped with
        sequence addition and the Cauchy product of two sequences. Let $M[x]$ be
        the same space but for sequences of elements in $M$. Give $M[x]$ the
        same pointwise addition and let $\star:k[x]\times{M}[x]\rightarrow{M}[x]$
        be the following product:
        \begin{equation}
            r(x)\star{m}(x)=
            \sum_{i=0}^{N}\sum_{j=0}^{M}(r_{i}\boldsymbol{\cdot}m_{j})x^{i+j}
        \end{equation}
        This makes $M[x]$ a module over $k[x]$. For we have:
        \begin{equation}
            1(x)\star{m}(x)=\sum_{i=0}^{N}(1\boldsymbol{\cdot}m_{i})x^{i}
            =\sum_{i=0}^{N}m_{i}x^{i}
            =m(x)
        \end{equation}
        For the distributive law let $n$ and $m$ be polynomials in $M[x]$ and
        extend $n$ or $m$ with zeros so that we may write them as a sum over the
        same powers. Then we have:
        \begin{equation}
            r(x)\star(m(x)+n(x))=
            \sum_{i=0}^{N}\sum_{j=0}^{M}r_{i}\boldsymbol{\cdot}(m_{j}+n_{j})x^{i+j}
            =\sum_{i=0}^{N}\sum_{j=0}^{M}
                (r_{i}\boldsymbol{m}_{j}+r_{i}\boldsymbol{\cdot}n_{j})x^{i+j}
        \end{equation}
        Thinking of these as finitely supported sequences we'd be done here, but
        alas, they are polynomials. So:
        \begin{subequations}
            \begin{align}
                r(x)\star(m(x)+n(x))
                &=\sum_{i=0}^{N}\sum_{j=0}^{M}r_{i}\boldsymbol{\cdot}m_{j}x^{i+j}
                    +\sum_{i=0}^{N}\sum_{j=0}^{M}
                    r_{i}\boldsymbol{\cdot}n_{j}x^{i+j}\\
                &=(r(x)\star{m}(x))+(r(x)\star{n}(x))
            \end{align}
        \end{subequations}
        Similarly for the distributive law over modules. Lastly, checking the
        compatibility of multiplication, we have:
        \begin{subequations}
            \begin{align}
                r(x)\star(s(x)\boldsymbol{\cdot}m(x))
                &=r(x)\star\sum_{i=0}^{N}\sum_{j=0}^{M}s_{i}m_{j}x^{i+j}\\
                &=\sum_{\ell=0}^{L}\sum_{i=0}^{N}\sum_{j=0}^{M}
                    r_{\ell}s_{i}m_{j}x^{\ell+i+j}\\
                &=\Big(\sum_{\ell=0}^{L}\sum_{i=0}^{N}
                    r_{\ell}s_{i}x^{\ell+i}\Big)\star{m}(x)\\
                &=(r(x)\cdot{s}(x))\star{m}(x)
            \end{align}
        \end{subequations}
        Thus $M[x]$ is a $k[x]$ module. Let $\iota:M\rightarrow{M}[x]$ be the
        inclusion mapping. That is $m\mapsto{m}(x)$, where $m(x)$ is the
        constant polynomial $m(x)=m$. Suppose $N$ is a $k$ module and
        $\varphi:M\rightarrow{N}$ is a module homomorphism. If $\tilde{\varphi}$
        is a $k[x]$ module homomorphism from $M[x]$ to $N$ such that the diagram
        commutes thatn $(\tilde{\varphi}\circ\iota)(m)=\varphi(m)$. But:
        \begin{equation}
            (\tilde{\varphi}\circ\iota)(m)
            =\tilde{\varphi}(\iota(m))
            =\tilde{\varphi}(m)
            =\varphi(m)
        \end{equation}
        Thus let $\tilde{\varphi}(m(x))=\varphi(m_{0})$, where $m_{0}$ is the
        constant term of the polynomial $m(x)$. Then $\tilde{\varphi}$ makes the
        diagram commute. Moreover, from our chain of equalities, we have that
        such a map must be unique. Thus $\iota$ is universal.
    \section{}
        Put $A$ into rational canonical form. That is, the matrix $R_{A}$ where:
        \begin{equation}
            R_{A}=\textrm{diag}(C(f_{1}),\dots,C(f_{n}))
        \end{equation}
        where the $C(f_{1})$ are the companion matrices to the polynomials
        $f_{i}$ such that $f_{i}|f_{i+1}$ are minimal and whose product is the
        characteristic polynomial. $A$ is similar to $R_{A}$. But the rational
        canonical form for $A^{T}$ is also $R_{A}$ since $A$ and $A^{T}$ have
        the same characteristic polynomial and since:
        \begin{equation}
            \prod_{k=1}^{N}(\lambda_{k}I-A)=0
            \Leftrightarrow(\lambda_{k}I-A^{T})=0
        \end{equation}
        and therefore $A$ and $A^{T}$ will have the same minimial polynomial,
        and thus their rational canonical forms will be made from the same
        block diagonal matrices. But if $A$ is similar to $R_{A}$ and
        $A^{T}$ is similar to $R_{A}$, then $A$ is similar to $A^{T}$ since
        similarity is an equivalence relation. Thus, $A$ is similar to its
        transpose.
    \section{}
    \section{}
        For let $\mathscr{B}=\{e_{1},\dots,e_{n}\}$ be a basis of $V$. Note that
        $J$ is invertible since:
        \begin{equation}
            JJ^{3}=(J^{2})^{2}=(\minus\textrm{Id})^{2}=\textrm{Id}
        \end{equation}
        Since $J$ is a real valued matrix, $\det(A^{2})=\det(A)^{2}\geq{0}$.
        But if $n$ is odd, then $\det(\minus{I})=(\minus{1})^{n}=\minus{1}$,
        a contradiction since $\det(A^{2})=\det(\minus{I})$. Therefore $n$ is
        even. From the fact that the characteristic polynomial of $J^{2}$ is
        $(\xi+1)^{n}$ we have that the characteristic polynomial of $J$ is
        $(\lambda^{2}+1)^{n}$. Over $\mathbb{R}$ the minimial polynomial is then
        $\lambda^{2}+1$. The rational canonical form is therefore:
        \begin{equation}
            R_{J}=
            \begin{bmatrix}
                1&0&\dots&0&1\\
                0&1&\dots&0&1\\
                \vdots&\vdots&\ddots&\vdots&\vdots\\
                0&0&\dots&1&1\\
                0&0&\dots&0&1
            \end{bmatrix}
        \end{equation}
    \section{}
    \section{}
        Let $\textrm{Nil}(A)$ be the set of nilpotent element of $A$. This is an
        ideal. For if $r\in{A}$ and $n\in{N}$, then since $A$ is commutative we
        have that $(rn)^{k}=r^{k}n^{k}=r^{k}0=0$ for some $k\in\mathbb{N}$, and
        therefore $rn$ is nilpotent. Similarly, if $n,m\in\textrm{Nil}(A)$, then
        from the binomial theorem:
        \begin{equation}
            (n+m)^{k}=\sum_{i=1}^{k}\binom{k}{i}n^{k-i}y^{i}
        \end{equation}
        For large enough $k$ every term in this expansion is zero, and thus
        $n+m$ is nilpotent. If $\mathfrak{p}$ is a prime ideal, then
        $\textrm{Nil}(A)\subseteq\mathfrak{p}$. For if $n\in\textrm{Nil}(A)$,
        then $n^{k}=0$, which is contained in $\mathfrak{p}$, and therefore
        $n^{k-1}n=0$ so either $n\in\mathfrak{p}$ or $n^{k-1}\in\mathfrak{p}$.
        Continuing by induction we find that $n\in\mathfrak{p}$. Therefore
        $\textrm{Nil}(\mathfrak{p})=\textrm{Nil}(A)$ for any prime ideal.
        Therefore $A$ is reduced if and only if for every prime ideal, the
        only nilpotent element is zero.
        \par\hfill\par
        Let $R=\mathbb{Z}_{2}^{2}$. The prime ideals of this are the ideals
        generated by $(1,0)$ and $(0,1)$, both of which are domains
        (since they are essentially $\mathbb{Z}_{2}$). However
        $\mathbb{Z}_{2}^{2}$ is not a domain since $(1,0)\cdot(0,1)=(0,0)$,
        yet neither $(1,0)$ nor $(0,1)$ are the zero element.
        \section{}
        The characteristic polynomial is:
        \begin{equation}
            (\lambda-4)^{3}
        \end{equation}
        And from this we have eigenvalues $4$ with triple multiplicity. From
        this we compute the Jordan canonical form by:
        \begin{equation}
            J_{A}=
            \begin{bmatrix}
                4&1&0\\
                0&4&1\\
                0&0&4
            \end{bmatrix}
        \end{equation}
        For the rational canonical form we have again that the characteristic
        polynomial is $(\lambda-4)^{3}$. We have that $A-4I\ne{0}$ and
        $(A-4I)^{2}\ne{0}$, and lastly $(A-4I)^{3}=0$. So the minimal
        polynomial is simply:
        \begin{equation}
            (\lambda-4)^{3}=x^{3}-12x^{2}+48x-64
        \end{equation}
        Reading this off, we have that the rational canonical form is:
        \begin{equation}
            R_{A}=
            \begin{bmatrix*}[r]
                0&0&64\\
                1&0&\minus{48}\\
                0&1&12
            \end{bmatrix*}
        \end{equation}
        To find an invertible matrix $P$ such that $A=P^{\minus{1}}J_{A}P$ we
        set up the following augmented matrix:
        \begin{equation}
            \begin{bmatrix*}[r]
                3&1&0&\vline&4&1&0\\
                1&4&1&\vline&0&4&1\\
                3&\minus{2}&5&\vline&0&0&4
            \end{bmatrix*}
        \end{equation}
        Interchanging the two with elementary row operations gives us:
        \begin{equation}
            P^{\minus{1}}=
            \begin{bmatrix*}[r]
                0&0&1\\
                3&\minus{2}&1\\
                \minus{2}&1&\minus{1}
            \end{bmatrix*}
        \end{equation}
        This is indeed intervtible since it has non-zero determinant (and since
        it was obtain via elementary operations) and the inverse is:
        \begin{equation}
            P=
            \begin{bmatrix*}[r]
                \minus{1}&\minus{1}&\minus{2}\\
                \minus{1}&\minus{2}&\minus{3}\\
                1&0&0
            \end{bmatrix*}
        \end{equation}
\end{document}