%--------------------------------------------------------------------------%
\documentclass[crop=false,class=book,oneside]{standalone}                  %
%--------------------------Preamble----------------------------------------%
%---------------------------Packages----------------------------%
\usepackage{geometry}
\geometry{b5paper, margin=1.0in}
\usepackage[T1]{fontenc}
\usepackage{graphicx, float}            % Graphics/Images.
\usepackage{natbib}                     % For bibliographies.
\bibliographystyle{agsm}                % Bibliography style.
\usepackage[french, english]{babel}     % Language typesetting.
\usepackage[dvipsnames]{xcolor}         % Color names.
\usepackage{listings, lstlinebgrd}      % Verbatim-Like Tools.
\usepackage{mathtools, esint, mathrsfs} % amsmath and integrals.
\usepackage{amsthm, amsfonts}           % Fonts and theorems.
\usepackage{tabularx}
\usepackage{tcolorbox}                  % Frames around theorems.
\usepackage{upgreek}                    % Non-Italic Greek.
\usepackage{paracol}                    % Two-column styling.
\usepackage{wrapfig}                    % Wrap text around figure.
\usepackage{fmtcount, etoolbox}         % For the \book{} command.
\usepackage[newparttoc]{titlesec}       % Formatting chapter, etc.
\usepackage{titletoc}                   % Allows \book in toc.
\usepackage[nottoc]{tocbibind}          % Bibliography in toc.
\usepackage[titles]{tocloft}            % ToC formatting.
\usepackage{multicol, enumitem}         % Multi-column/enumerate.
\usepackage{import}                     % Import external files.
\usepackage{pgfplots, tikz}             % Drawing/graphing tools.
\usetikzlibrary{
    calc,                   % Calculating right angles and more.
    angles,                 % Drawing angles within triangles.
    arrows.meta,            % Latex and Stealth arrows.
    quotes,                 % Adding labels to angles.
    positioning,            % Relative positioning of nodes.
    decorations.markings,   % Adding arrows in the middle of a line.
    patterns,
    arrows,
    shapes,
    shapes.geometric,
    cd,
    hobby,
    babel
}                                       % Libraries for tikz.
\pgfplotsset{compat=1.9}                % Version of pgfplots.
\usepackage[font=scriptsize,
            labelformat=simple,
            labelsep=colon]{subcaption} % Subfigure captions.
\usepackage[font={scriptsize},
            hypcap=true,
            labelsep=colon]{caption}    % Figure captions.
\usepackage{hyperref}                   % Allows for hyperlinks.
\hypersetup{
    colorlinks=true,
    linkcolor=blue,
    filecolor=magenta,
    urlcolor=Cerulean,
    citecolor=SkyBlue
}                           % Colors for hyperref.
\usepackage[toc,acronym,nogroupskip]{glossaries} % Glossaries and acronyms.
\usepackage[subpreambles=false]{standalone}      % Complileable sub files.

% Various font stuff from kiwi.
% Use this for Times text and Computer Modern math
%\usepackage{times}

% Quite nice
%\usepackage[charter, greekfamily=, greekuppercase=italicized]{mathdesign}
%\usepackage[utopia, greekuppercase=italicized]{mathdesign}    % Math is narrower

% Use this for Times text and math
%\usepackage{newtxtext}
%\usepackage[libertine,cmintegrals]{newtxmath}
%\usepackage{fix-cm}

%\usepackage{txfontsb}
% or
%\usepackage{mathptmx}

%\usepackage[scaled=0.92]{helvet}
%\renewcommand{\rmdefault}{ptm}

%\usepackage{mathpazo}    % add possibly `sc` and `osf` options
%\usepackage{eulervm}

%\usepackage{fourier}
%\renewcommand{\rmdefault}{ptm}
%\usepackage{mathptm}

%\usepackage{fontspec}
%\setmainfont{lmodern}

%\usepackage[varg]{txfonts}
%\usepackage{fouriernc}
%\usepackage{mathpazo}

%\usepackage{bookman}
%\usepackage[scaled]{uarial}
%\usepackage[scaled]{helvet}
%\renewcommand*\familydefault{\sfdefault}
%\usepackage[math]{anttor}

%\newcommand\fgeorgia{\fontfamily{jvn}\selectfont}
%\newcommand\ftimes{\fontfamily{ptm}\selectfont}
%\newcommand\fhelvetica{\fontfamily{phv}\selectfont}
%\newcommand\fcourier{\fontfamily{pcr}\selectfont}
%\newcommand\fbookman{\fontfamily{pbk}\selectfont}
%\newcommand\fnewcentury{\fontfamily{pnc}\selectfont}
%\newcommand\fpalatino{\fontfamily{ppl}\selectfont}
%\newcommand\favantgarde{\fontfamily{pag}\selectfont}
%\newcommand\fnormal{\normalfont}
%\newcommand\fsize[1]{\ifnum#1>0\fontsize{#1}{#1}\selectfont\else\normalsize\fi}
%------------------------Theorem Styles-------------------------%
% Define theorem style for default spacing and normal font.
\newtheoremstyle{normal}
    {\topsep}               % Amount of space above the theorem.
    {\topsep}               % Amount of space below the theorem.
    {}                      % Font used for body of theorem.
    {}                      % Measure of space to indent.
    {\bfseries}             % Font of the header of the theorem.
    {}                      % Punctuation between head and body.
    {.5em}                  % Space after theorem head.
    {}

% Define theorem style for default spacing with italicized font.
\newtheoremstyle{normalit}{\topsep}{\topsep}
                {\itshape}{}{\bfseries}{}{.5em}{}

% Italic header environment.
\newtheoremstyle{thmit}{\topsep}{\topsep}{}{}{\itshape}{}{0.5em}{}

% Define italicized environments.
\theoremstyle{normalit}
\newtheorem{theorem}{Theorem}[section]
\newtheorem{lemma}{Lemma}[section]
\newtheorem{corollary}{Corollary}[section]
\newtheorem{proposition}{Proposition}[section]
\newtheorem*{theorem*}{Theorem}

% Define environments with italic headers.
\theoremstyle{thmit}
\newtheorem*{solution}{Solution}
\newtheorem*{fsolution}{Solution}

% Define default environments.
\theoremstyle{normal}
\newtheorem{example}{Example}[section]
\newtheorem{definition}{Definition}[section]
\newtheorem{problem}{Problem}[section]
\newtheorem{question}{Question}[section]
\newtheorem{remark}{Remark}[section]
\newtheorem{properties}{Properties}[section]
\newtheorem{notation}{Notation}[section]
\newtheorem{axiom}{Axiom}[section]
\newtheorem*{properties*}{Properties}
\newtheorem*{remark*}{Remark}
\newtheorem*{definition*}{Definition}
\theoremstyle{plain}

% Define framed environment.
\tcbuselibrary{most}
\newtcbtheorem[use counter*=theorem]{ftheorem}{Theorem}%
    {colback=green!5,colframe=green!35!black,
     fonttitle=\bfseries\upshape}{th}

\newtcbtheorem[use counter*=example]{fdefinition}{Definition}%
    {fonttitle=\bfseries\upshape,
     colback=blue!5!white,colframe=blue!75!black}{def}

\newtcbtheorem[use counter*=example]{fexample}{Example}%
    {fonttitle=\bfseries\upshape,
     colback=red!5!white,colframe=red!75!black}{ex}

\newtcbtheorem[use counter*=notation]{fnotation}{Notation}%
    {fonttitle=\bfseries\upshape,
     colback=SeaGreen!5!white,colframe=SeaGreen!75!black}{ex}

\newtcbtheorem[use counter*=corollary]{fcorollary}{Corollary}%
    {fonttitle=\bfseries\upshape,
     colback=Orchid!5!white,colframe=Orchid!75!black}{ex}

\newenvironment{bproof}{\textit{Proof.}}{\hfill$\square$}
\tcolorboxenvironment{bproof}{blanker,breakable,left=5mm,
                             before skip=10pt,after skip=10pt,
                             borderline west={1mm}{0pt}{red}}
\tcolorboxenvironment{fsolution}
    {enhanced jigsaw,colframe=cyan,interior hidden,breakable}

%--------------------Declared Math Operators--------------------%
\DeclareMathOperator{\Refl}{Refl}           % Reflection operator.
\DeclareMathOperator{\Span}{Span}           % Span of a set of vectors.
\DeclareMathOperator{\Card}{Card}           % Cardinality of set.
\DeclareMathOperator{\Ord}{Ord}             % Ordinal of ordered set.
\DeclareMathOperator{\Tr}{Tr}               % Trace of matrix.
\DeclareMathOperator{\adjoint}{adj}         % Adjoint of matrix.
\DeclareMathOperator{\rk}{rk}               % Rank of operator.
\DeclareMathOperator{\nul}{nul}             % Null space of operator.
\DeclareMathOperator{\sgn}{sgn}             % Sign of a number.
\DeclareMathOperator{\multideg}{mutlideg}   % Multi-Degree (Graphs).
\DeclareMathOperator{\GCD}{GCD}             % Greatest common denominator.
\DeclareMathOperator{\LM}{LM}               % Leading monomial
\DeclareMathOperator{\LC}{LC}               % Leading coefficient.
\DeclareMathOperator{\LT}{LT}               % Leading term.
\DeclareMathOperator{\LCM}{LCM}             % Least common multiple.
\DeclareMathOperator{\Mon}{Mon}             % Monomial.
\DeclareMathOperator{\Spec}{Spec}           % Spectrum.
\DeclareMathOperator{\proj}{proj}           % Projection.
\DeclareMathOperator{\comp}{comp}           % Component.
\DeclareMathOperator{\sinc}{sinc}           % Sinc function.
\DeclareMathOperator{\Ima}{Im}              % Image of operator.
\DeclareMathOperator{\Prin}{Prin}           % Principal value.
\DeclareMathOperator{\Mod}{mod}             % Modulus.
%------------------------New Commands---------------------------%
\DeclarePairedDelimiter\norm{\lVert}{\rVert}
\DeclarePairedDelimiter\ceil{\lceil}{\rceil}
\DeclarePairedDelimiter\floor{\lfloor}{\rfloor}
\newcommand*\diff{\mathop{}\!\mathrm{d}}
\newcommand*\Diff[1]{\mathop{}\!\mathrm{d^#1}}
\renewcommand{\mod}{\ \Mod}
\renewcommand*{\glstextformat}[1]{\textcolor{RoyalBlue}{#1}}
\renewcommand{\glsnamefont}[1]{\textbf{#1}}
\renewcommand\labelitemii{$\circ$}
\renewcommand\thesubfigure{\arabic{chapter}.\arabic{figure}}
\renewcommand\thesubfigure{%
    \arabic{chapter}.\arabic{figure}.\arabic{subfigure}}
\addto\captionsenglish{\renewcommand{\figurename}{Fig.}}
%------------------------Book Command---------------------------%
\makeatletter
\renewcommand\@pnumwidth{1cm}
\newcounter{book}
\renewcommand\thebook{\@Roman\c@book}
\newcommand\book{%
    \if@openright
        \cleardoublepage
    \else
        \clearpage
    \fi
    \thispagestyle{plain}%
    \if@twocolumn
        \onecolumn
        \@tempswatrue
    \else
        \@tempswafalse
    \fi
    \null\vfil
    \secdef\@book\@sbook
}
\def\@book[#1]#2{%
    \ifnum \c@secnumdepth >-3\relax
        \refstepcounter{book}%
        \addcontentsline{toc}{book}{
            \bookname\ \thebook:\hspace{1em}#1
        }
    \else
        \addcontentsline{toc}{book}{#1}%
    \fi
    \markboth{}{}%
    {\centering
     \interlinepenalty \@M
     \normalfont
     \ifnum \c@secnumdepth >-2\relax
       \huge\bfseries \bookname\nobreakspace\thebook
       \par
       \vskip 20\p@
     \fi
     \Huge \bfseries #2\par}%
    \@endbook}
\def\@sbook#1{%
    {\centering
     \interlinepenalty \@M
     \normalfont
     \Huge \bfseries #1\par}%
    \@endbook}
\def\@endbook{
    \vfil\newpage
        \if@twoside
            \if@openright
                \null
                \thispagestyle{empty}%
                \newpage
            \fi
        \fi
        \if@tempswa
            \twocolumn
        \fi
}
\newcommand*\l@book[2]{%
    \ifnum \c@tocdepth >-2\relax
        \addpenalty{-\@highpenalty}%
        \addvspace{2.25em \@plus\p@}%
        \setlength\@tempdima{3em}%
        \begingroup
            \parindent \z@ \rightskip \@pnumwidth
            \parfillskip -\@pnumwidth
            {
                \leavevmode
                \Large \bfseries #1\hfil \hb@xt@\@pnumwidth{
                    \hss #2
                }
            }
            \par
            \nobreak
            \global\@nobreaktrue
            \everypar{\global\@nobreakfalse\everypar{}}%
        \endgroup
    \fi}
\newcommand\bookname{Book}
\renewcommand{\thebook}{\texorpdfstring{\Numberstring{book}}{book}}
\providecommand*{\toclevel@book}{-2}
\makeatother
\titlecontents{chapter}[0pt]
    {\bfseries}
    {\chaptername\ \thecontentslabel:\quad}
    {}
    {\hfill\contentspage}
\titleformat{\part}[display]
    {\Large\bfseries}
    {\partname\nobreakspace\thepart}
    {0mm}
    {\Huge\bfseries}
    \titlecontents{part}[0pt]
    {\large\bfseries}
    {\partname\ \thecontentslabel: \quad}
    {}
    {\hfill\contentspage}
\newcommand{\MarkRightAngle}[4][.3cm]
    {\coordinate (tempa) at ($(#3)!#1!(#2)$);
     \coordinate (tempb) at ($(#3)!#1!(#4)$);
     \coordinate (tempc) at ($(tempa)!0.5!(tempb)$);%midpoint
     \draw (tempa) -- ($(#3)!2!(tempc)$) -- (tempb);}
%--------------------------LENGTHS------------------------------%
% Spacings for the Table of Contents.
\addtolength{\cftsecnumwidth}{1ex}
\addtolength{\cftsubsecindent}{1ex}
\addtolength{\cftsubsecnumwidth}{1ex}
\addtolength{\cftfignumwidth}{1ex}
\addtolength{\cfttabnumwidth}{1ex}

% Spacing for multi-column and enumerate environments.
\setlength{\multicolsep}{6pt}
\setlist[enumerate]{itemsep=0pt,topsep=3pt}

% Indent and paragraph spacing.
\setlength{\parindent}{0em}
\setlength{\parskip}{0em}                                                       %
%------------------------Main Document-------------------------------------%
\begin{document}
        \pagenumbering{roman}
        \title{Topics in Algebra}
        \author{Ryan Maguire}
        \date{\vspace{-5ex}}
        \maketitle
        \tableofcontents
        \clearpage
        \chapter*{Topics in Algebra}
        \addcontentsline{toc}{chapter}{Topics in ALgebra}
        \markboth{}{SURGERY THEORY}
        \vspace{10ex}
        \setcounter{chapter}{1}
        \pagenumbering{arabic}
    \section{Modules}
        Let $R$ be a ring with identity. Ring homomorphisms are required to
        to preserve identities. The identity is idemptotent:
        $1^{2}=1$. In a group, if $x^{2}=x$ then $x=1$, by the cancellation
        laws. This is not necessary in a ring. Let $R=\mathbb{Z}^{2}$. Here,
        there are several idemptotent elements: $(0,1)$, $(1,0)$, and
        $(1,1)$. For a group, a homomorphism that preserves that product
        also preserves the identity and the inverses. For rings we must
        add that to the definition. Any ring without identity can be
        embedded into a ring with identity, so there's not much harm in
        assuming our rings have identity.
        \begin{ldefinition}{$R$ Modules}{R_Module}
            And $R$ module on a ring $R$ is a set $M$ with a function
            $\cdot:R\times{M}\rightarrow{M}$ with the following properties:
            \begin{enumerate}
                \item $r\cdot(m_{1}+m_{2})=r\cdot{m}_{1}+r\cdot{m}_{2}$
                \item $(r_{1}+r_{2})\cdot{m}=r_{1}\cdot{m}+r_{2}\cdot{m}$
                \item $r_{1}\cdot(r_{2}\cdot{m})=(r_{1}r_{2})\cdot{m}$
                \item $1\cdot{m}=m$
            \end{enumerate}
        \end{ldefinition}
        \begin{theorem}
            An $R$ module is an Abelian group $M$ equipped with a ring
            homomorphism $\lambda:R\rightarrow\textrm{End}(M)$, where
            $\textrm{End}(M)$ is the ring of endomorphisms.
        \end{theorem}
        The ring of endomorphisms is the set of all endomorphisms equipped
        with function addition and composition. That is,
        $f+g$ is the function defined by $(f+g)(x)=f(x)+g(x)$. The product
        is composition $(g\circ{f})(x)=g(f(x))$.
        Given an R module in the sense of the definition, we need to define
        a ring homomorphism $\lambda:R\rightarrow\textrm{End}(M)$.
        Let $(\lambda(r))(m)=r\cdot{m}$. Then $\lambda(r)\in\textrm{End}(M)$,
        by the first part of the definition of an $R$ module. By the second
        part, $\lambda(r_{1}+r_{2})=\lambda(r_{1})+\lambda(r_{2})$. By three,
        $\lambda(r_{1}r_{2})=\lambda(r_{1})\circ\lambda(r_{2})$. Finally, by
        the fourth portion of our definition,
        $\lambda(1_{R})=1_{\textrm{End}(M)}$.
        \begin{ldefinition}{Module Homomorphism}{Module_Homomorphism}
            For $R$ modules $M$ and $N$, a homomorphism is an Abelian group
            homomorphism $f:M\rightarrow{N}$ such that, for all $r,m$,
            $f(r\cdot{m})=r\cdot{f}(m)$.
        \end{ldefinition}
        \begin{ldefinition}{Module Isomorphism}{Module_Isomorphism}
            $f:M\rightarrow{N}$ is an isomorphism if there exists a
            homomorphism $g:N\rightarrow{M}$ such that
            $f\circ{g}=\textrm{id}_{N}$ and $g\circ{f}=\textrm{id}_{M}$.
        \end{ldefinition}
        \begin{example}
            Let $R$ be a field. Then any $R$ module is simply a vector space
            over $R$. If we let $R=\mathbb{Z}$, then a $\mathbb{Z}$ module
            is an Abelian group.
        \end{example}
        For any ring $E$ there exists a unique ring homomorphism
        $\mathbb{Z}\rightarrow{E}$,
        $\mathbb{Z}\overset{\lambda}{\rightarrow}\textrm{End}(M)$.
        From appearance, modules seem very similar to vector spaces, but
        structurally they are very difference. Every vector space has a
        basis, but this is not true for modules.
        \begin{lexample}{}{}
            If $k$ is a field and $V$ is a finite dimensional vector space
            over $k$, and if $T:V\rightarrow{V}$ is a linear operator. Make
            $V$ a $k[x]$ module by defining $f\cdot{r}=(f(T))(v)$. For
            example, $(2-x+3x^2)\cdot{v}=2v-T(v)+3T^{2}(v)$.
        \end{lexample}
        A subspace $W\subseteq{V}$ is a $k[X]$ submodule such that
        $T(W)\subseteq{W}$. That is, a $T$ invariant space. If $W$ is invariant
        under $T$, then it is invariant under $T^{2}$ and $T^{3}$, etc., and
        it is invariant under linear combinations. That is, it is invariant
        under polynomials.
        \begin{example}
            Let $\varphi:A\rightarrow{B}$ be a ring homomorphism. Then any
            $B$ module $M$ becomes an $A$ module by pullback along $\varphi$.
        \end{example}
        Special case, if $R$ is a ring and $I$ is a two-sided ideal of $R$,
        then $\pi:R\rightarrow{R}/I$. Any $R/I$ module $M$ becomes an
        $R$ module by pullback. $M$ is annihilated by $I$. An $R/I$ module is
        just an $R$ module annihilated by $I$. Let $M$ be an $R$ module
        annihilated by $I$. Then $\lambda(I)=0$.
        \begin{ldefinition}{Direct Sum of Modules}{Direct_Sum_of_Modules}
            The direct sum of two $R$ modules $M_{1}$ and $M_{2}$ is the abelian
            group $M_{1}\oplus{M}_{2}$ defined by $M_{1}\times{M}_{2}$ where
            $r\cdot(m_{1},m_{2})=(r\cdot{m_{1}},r\cdot{m_{2}})$.
        \end{ldefinition}
        \begin{ltheorem}{Recognition Principle for Direct Sums}
                        {Recognition_Principle_for_Direct_Sums}
            It's a very cool theorem.
        \end{ltheorem}
        \begin{ltheorem}{Primary Decomposition Theorem}
                        {Primary Decomposition_Theorem}
            If $R$ is a commutative ring and if $I$ and $J$ are coprime ideals
            of $R$ ($I+J=R$), and if $M$ is an R module annihilated by the
            product ideal $IJ$, then $M=_{I}M+_{J}M$, where
            $_{K}M$ is the set of all $m\in{M}$ such that $Km=0$.
        \end{ltheorem}
        \begin{proof}
            We need to show that the sum is everything and that the intersection
            is trivial.
        \end{proof}
        Let $V$ be a finite dimensional $k$ vector space and let
        $T:V\rightarrow{V}$ be a linear map. $T$ is diagonalizable if and only
        if it's minimal polynomial $\mu_{T}$ factors as
        $(x-\lambda_{1})\cdots(x-\lambda_{r})$ with distinct linear factors.
\end{document}