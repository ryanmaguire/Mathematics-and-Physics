\section{Topologies}
    When one uses the notation developed for intervals on the real line $[a,b]$
    and $(a,b)$, the former is called the \textit{closed} interval, and the
    latter the \textit{open} interval. The clearest difference is that $[a,b]$
    has it's endpoints, whereas $(a,b)$ does not. We can generalize this
    further. For every point $x\in(a,b)$ there is some $\varepsilon>0$ such that
    we can fit $(x-\varepsilon,x+\varepsilon)$ inside of the interval $(a,b)$.
    Namely, define $\varepsilon$ to be:
    \begin{equation}
        \varepsilon=\frac{\min\{b-x,x-a\}}{2}
    \end{equation}
    This cannot be done for the closed unit interval. For if we let $x=a$, for
    any $\varepsilon>0$ we have that $(a-\varepsilon,a+\varepsilon)$ has points
    that fall outside of $[a,b]$. That is, all of the points between
    $a-\varepsilon$ and $a$. This is the distinction we want to note between a
    closed interval and an open interval. If we take the intersection of two
    intervals $(a,b)$ and $(c,d)$, then either the result is the empty set or it
    is once again an open interval. Lastly, if we take the union over an
    arbitrary collection of open intervals, then end result may not be an open
    interval, but it will still have the property that for any $x$ in the union
    there is a $\varepsilon>0$ such that $(x-\varepsilon,x+\varepsilon)$ fits
    inside the union. It is these two properties that we wish to capture:
    Closure to finite intersections and arbitrary unions. We've already noted
    that the intersection of two open intervals may be empty, and so we want to
    claim that the empty set is open as well. Lastly, we may regard the entire
    real line $\mathbb{R}$ as the \textit{interval} $(\minus\infty,\infty)$, and
    because of this we wish to regard all of $\mathbb{R}$ to be open too. In
    capturing these notions we can define a \textit{topology} on a set $X$.
    \begin{fdefinition}{Topology}{Topology}
        A topology on a set $X$ is a subset
        $\tau\subseteq\mathcal{P}(X)$ such that $\emptyset\in\tau$, $X\in\tau$,
        and for any subset $\mathcal{O}\subseteq\tau$, it is true that:
        \begin{equation}
            \bigcup_{\mathcal{U}\in\mathcal{O}}\mathcal{U}\in\tau
        \end{equation}
        And such that for all $A,B\in\tau$, it is true that $A\cap{B}\in\tau$.
    \end{fdefinition}
    There are many trivial examples of topologies on any set $X$, but the
    trivial examples often provide excellent counterexamples for various
    propositions. The two extreme topologies are the \textit{discrete} topology
    and the \textit{chaotic} topology. The chaotic topology is also called the
    trivial topology or the indiscrete topology.
    \begin{fexample}{Discrete and Chaotic Topologies}{Discrete_and_Chaotic_Top}
        The Discrete Topology\index{Discrete Topology} on a set $X$ is simply
        $\tau=\mathcal{P}(X)$. That is, the \gls{power set} of $X$. This is
        indeed a topology since it contains $\emptyset$, $X$, and is most
        certainly closed to finite intersections. Moreover, from how the union
        of a set is defined (Def.~\ref{def:Union_over_a_Set}) we have that it
        is closed to arbitrary unions as wells. The chaotic topology%
        \index{Chaotic Topology} is $\tau=\{\emptyset,X\}$. This is also a
        topology, and we can check simply by computation. The empty set
        and the whole space are contained within $\tau$, and there are only four
        possible ways to perform unions and intersections. The commutativity of
        union and intersection reduces this to three. We have:
        \par
        \begin{subequations}
            \begin{minipage}[b]{0.49\textwidth}
                \begin{align}
                    \emptyset\cap\emptyset&=\emptyset\\
                    \emptyset\cap{X}&=\emptyset\\
                    X\cap{X}&=x
                \end{align}
            \end{minipage}
            \hfill
            \begin{minipage}[b]{0.49\textwidth}
                \begin{align}
                    \emptyset\cup\emptyset&=\emptyset\\
                    \emptyset\cup{X}&=X\\
                    X\cup{X}&=x
                \end{align}
            \end{minipage}
        \end{subequations}
        \par\vspace{2.5ex}
        The name for the discrete topology will be made clear later when we
        discuss \textit{metric spaces}. The discrete topology on any set $X$
        will always be a topology induced by the \textit{discrete metric} on
        $X$, and thus we have that the discrete topology is always a metric
        topology (these terms will be discussed in great detail later). The
        chaotic topology is so named for one of it's strange properties. As we
        will see once we've defined continuity, the chaotic topology has the
        property that every single function $f:A\rightarrow{X}$ will be
        continuous, regardless of what the topology chosen on $A$ was.
    \end{fexample}
    \begin{ldefinition}{Topological Space}{Topological_Space}
        A topological space, denote $(X,\tau)$ is a subset $X$ and a topology
        $\tau$ on $X$.
    \end{ldefinition}
    \begin{ldefinition}{Open Subsets}{Open_Subsets}
        An open subset of a topological space $(X,\tau)$ is a set
        $\mathcal{U}\subseteq{X}$ such that $\mathcal{U}\in\tau$.
    \end{ldefinition}
    \begin{theorem}
        \label{thm:Emptyset_Is_Open}%
        If $(X,\tau)$ is a topological space, then $\emptyset$ is open.
    \end{theorem}
    \begin{proof}
        For since $(X,\tau)$ is a topological space, $\tau$ is a topology on $X$
        (Def.~\ref{def:Topological_Space}). But then
        $X\in\tau$ (Def.~\ref{def:Topology}) and thus $X$ is an open subset of
        $(X,\tau)$ (Def.~\ref{def:Open_Subsets}). Therefore, etc.
    \end{proof}
    \begin{theorem}
        \label{thm:Whole_Space_Is_Open}%
        If $(X,\tau)$ is a topological space, then $X$ is an open subset.
    \end{theorem}
    \begin{proof}
        For since $(X,\tau)$ is a topological space, $\tau$ is a topology on $X$
        (Def.~\ref{def:Topological_Space}). But then $\emptyset\in\tau$
        (Def.~\ref{def:Topology}) and thus $\emptyset$ is an open subset of
        $(X,\tau)$ (Def.~\ref{def:Open_Subsets}). Therefore, etc.
    \end{proof}
    \begin{ldefinition}{Closed Subsets}{Closed_Subsets}
        A closed subset of a topological space $(X,\tau)$ is a subset
        $\mathcal{C}\subseteq{X}$ such that $X\setminus\mathcal{C}\in\tau$.
        That is, the complement of $\mathcal{C}$ is an open subset of
        $(X,\tau)$.
    \end{ldefinition}
    \begin{theorem}
        \label{thm:Emptyset_Is_Closed}%
        If $(X,\tau)$ is a topological space, then $\emptyset$ is closed.
    \end{theorem}
    \begin{proof}
        For since $(X,\tau)$ is a topological space, $X$ is an open subset
        (Thm.~\ref{thm:Whole_Space_Is_Open}). But $X\setminus\emptyset=X$, and
        thus the complement of $\emptyset$ is open. Thus, $\emptyset$ is closed
        (Def.~\ref{def:Closed_Subsets}). Therefore, etc.
    \end{proof}
    \begin{theorem}
        \label{thm:Whole_Space_Is_Closed}%
        If $(X,\tau)$ is a topological space, then $X$ is closed.
    \end{theorem}
    \begin{proof}
        For since $(X,\tau)$ is a topological space, $\emptyset$ is an open
        subset (Thm.~\ref{thm:Emptyset_Is_Open}). But $X\setminus{X}=\emptyset$,
        and thus the complement of $X$ is open. Thus, $X$ is closed
        (Def.~\ref{def:Closed_Subsets}). Therefore, etc.
    \end{proof}
    \begin{theorem}
        \label{thm:Comp_of_Open_is_Closed}%
        If $(X,\tau)$ is a topological space, and if $\mathcal{U}\subseteq{X}$
        is open, then $X\setminus\mathcal{U}$ is closed.
    \end{theorem}
    \begin{proof}
        For if $\mathcal{U}\subseteq{X}$ is a set, then
        $X\setminus(X\setminus\mathcal{U})=\mathcal{U}$. But $\mathcal{U}$ is
        open, and thus the complement of $X\setminus{U}$ is open. But then
        $X\setminus{U}$ is closed (Def.~\ref{def:Closed_Subsets}).
    \end{proof}
    \begin{ldefinition}{Relative Topology}{Relative_Topology}
        The relative topology of a subset $A\subseteq{X}$ in a topological space
        $(X,\tau)$ is the set $\tau_{A}$ defined by:
        \begin{equation}
            \tau_{A}=\big\{\;A\cap\mathcal{U}\,:\,\mathcal{U}\in\tau\;\big\}
        \end{equation}
    \end{ldefinition}
    \begin{theorem}
        If $(X,\tau)$ is a topological space, if $A\subseteq{X}$, and if
        $\tau_{A}$ is the relative topology on $A$, then $\tau_{A}$ is a
        topology on $A$.
    \end{theorem}
    \begin{proof}
        For since $\emptyset\in\tau$, and $\emptyset\cap{A}=\emptyset$, we have
        that $\emptyset\in\tau_{A}$. Similarly, since $A\subseteq{X}$, and
        since $X\in\tau$, we see that $A=A\cap{X}\in\tau_{A}$. If $\mathcal{O}$
        is a subset of $\tau_{A}$, then there is a subset $\Delta\subseteq\tau$
        such that:
        \begin{equation}
            \mathcal{O}=\big\{\;A\cap\mathcal{U}\,:\,\mathcal{U}\in\Delta\;\}
        \end{equation}
        Define $\mathcal{D}$ by:
        \begin{equation}
            \mathcal{D}=\bigcup_{\mathcal{U}\in\Delta}\mathcal{U}
        \end{equation}
        But then:
        \begin{equation}
            \bigcup_{\mathcal{V}\in\mathcal{O}}\mathcal{V}
            =\bigcup_{\mathcal{U}\in\Delta}(A\cap\mathcal{U})
            =A\cap\Big(\bigcup_{\mathcal{U}\in\Delta}\mathcal{U}\Big)
            =A\cap\mathcal{D}
        \end{equation}
        But $\tau$ is a topology on $X$, and thus $\mathcal{D}\in\tau$
        (Def.~\ref{def:Topology}). But then $A\cap\mathcal{D}\in\tau_{A}$
        (Def.~\ref{def:Relative_Topology}). Thus, $\tau_{A}$
        is closed to arbitrary unions.
    \end{proof}
    \begin{definition}
        If $(X,\tau)$ is a topological space and $S\subset{X}$, then a set
        $\mathcal{U}\subset S$ is said to be open in $S$ if and only if
        $\mathcal{U}\in \mathscr{T}$, where $\mathscr{T}$ is the relative
        topology on $S$.
    \end{definition}
    \begin{fdefinition}{Continuous Functions}{Cont_Func_Top}
        A continuous function\index{Continuous Function} from a
        topological space $(X,\tau_{X})$ to a topological space $(Y,\tau_{Y})$
        is a function $f:X\rightarrow{Y}$ such that for all
        $\mathcal{U}\in\tau_{Y}$ it is true that
        $f^{\minus{1}}(\mathcal{U})\in\tau_{X}$. That is, the pre-image of open
        sets is open.
    \end{fdefinition}
    \begin{theorem}
        If $(X,\tau)$ is a topological space, and if $\textrm{id}_{X}$ is the
        identity function of $X$, then $\textrm{id}_{X}$ is continuous.
    \end{theorem}
    \begin{proof}
        For suppose not. Then there is a $\mathcal{U}\in\tau$ such that
        $\textrm{id}_{X}(\mathcal{U})\notin\tau$. But if $\mathcal{U}\in\tau$
        then $\mathcal{U}\subseteq{X}$, and therefore
        $\textrm{id}_{X}(\mathcal{U})=\mathcal{U}$. But then
        $\textrm{id}_{X}(\mathcal{U})\in\tau$, a contradiction. Therefore,
        $\textrm{id}_{X}$ is continuous.
    \end{proof}
    \begin{theorem}
        If $(X,\tau_{X})$ and $(Y,\tau_{Y})$ are topological spaces, and if
        $f:X\rightarrow{Y}$ is a constant mapping, then $f$ is continuous.
    \end{theorem}
    \begin{proof}
        For suppose not. Then there is a $\mathcal{U}\in\tau_{Y}$ such that
        $f^{\minus{1}}(\mathcal{U})\notin\tau_{X}$. But if $f$ is a constant
        mapping, then there is a $y\in{Y}$ such that, for all $x\in{X}$ it is
        true that $f(x)=y$. By the law of the excluded middle either
        $y\in\mathcal{U}$ or $y\notin\mathcal{U}$. If $y\notin\mathcal{U}$ then
        $f^{\minus{1}}(\mathcal{U})=\emptyset$. But $\tau_{X}$ is a topology and
        therefore $\emptyset\in\tau_{X}$. Thus $y\in\mathcal{U}$. But if
        $y\in\mathcal{U}$ then $f^{\minus{1}}(\mathcal{U})=X$. But $\tau_{X}$ is
        a topology and thus $X\in\tau_{X}$. But then
        $f^{\minus{1}}(\mathcal{U})\in\tau_{X}$, a contradiction. Therefore,
        $f$ is continuous.
    \end{proof}
    \begin{theorem}
        If $(X,\tau_{X})$, $(Y,\tau_{Y})$, and $(Z,\tau_{Z})$ are topological
        spaces, and if the functions $f:X\rightarrow{Y}$ and $g:Y\rightarrow{Z}$
        are continuous, then $g\circ{f}:X\rightarrow{Z}$ is continuous.
    \end{theorem}
    \begin{proof}
        For if $\mathcal{V}\in\tau_{Z}$ is and open set, then
        $g^{\minus{1}}(\mathcal{V})\in\tau_{Y}$, since $g$ is continuous. But
        then since $f$ is continuous,
        $f^{\minus{1}}(g^{\minus{1}}(\mathcal{V}))\in\tau_{X}$
        (Def.~\ref{def:Cont_Func_Top}). Thus $g\circ{f}$ is continuous.
    \end{proof}
    \begin{ldefinition}{Convergent Sequences In Topological Spaces}
                       {Conv_Seq_Top}
        A convergent sequence in a topological space $(X,\tau)$ is a sequence
        $a:\mathbb{N}\rightarrow{X}$ such that there is an $x\in{X}$ such that,
        for all $\mathcal{U}\in\tau$ such that $x\in\mathcal{U}$, there is an
        $N\in\mathbb{N}$ such that, for all $n>N$, it is true that
        $x_{n}\in\mathcal{U}$. We denote this by $a_{n}\rightarrow{x}$.
    \end{ldefinition}
    \begin{ldefinition}{Limits of Sequences in Topological Spaces}{Lim_Seq_Top}
        A limit of a sequence $a:\mathbb{N}\rightarrow{X}$ in a topological
        space $(X,\tau)$ is a point $x\in{X}$ such that $a_{n}\rightarrow{x}$.
    \end{ldefinition}
    \begin{theorem}
        There exist topological spaces with convergent sequences that do not
        have unique limits.
    \end{theorem}
    \begin{proof}
        For let $X=\{1,2,3\}$, and let $\tau=\{\emptyset, \{1,2\},\{1,2,3\}\}$.
        We see that $\emptyset,X\in \tau$, unions and intersections are in
        $\tau,$ and thus $\tau$ is a topology. Let:
        \begin{equation}
            x_{n}=
            \begin{cases}
                1,&n\textrm{ odd}\\
                2,&n\textrm{ even}
            \end{cases}
        \end{equation}
        Then $x_n \rightarrow 1$ and $x_n \rightarrow 2$. To see this, let
        $\mathcal{U}$ be an open set such that $1\in \mathcal{U}$. Our choices
        are $\{1,2\}$ and $\{1,2,3\}$. Then for all $n\in \mathbb{N}$,
        $x_{n}\in\mathcal{U}$, and thus $x_{n}\rightarrow{1}$. Similarly,
        $x_n \rightarrow 2$. Convergence is not necessarily unique in
        topological spaces.
    \end{proof}
    \begin{theorem}
        If $(X,\tau_{X})$ and $(Y,\tau_{Y})$ are topological spaces, if
        $a:\mathbb{N}\rightarrow{X}$ is a convergent sequence in $X$, and if
        $f:X\rightarrow{Y}$ is a continuous function, then the sequence
        $b:\mathbb{N}\rightarrow{Y}$ defined by $b_{n}=f(a_{n})$ is a convergent
        sequence in $Y$.
    \end{theorem}
    \begin{proof}
        For if $a:\mathbb{N}\rightarrow{X}$ is a convergent sequence, then there
        is a point $x\in{X}$ such that, for all $\mathcal{U}\in\tau_{X}$ such
        that $x\in\mathcal{U}$, there is an $N\in\mathbb{N}$ such that for all
        $n\in\mathbb{N}$ with $n>N$, it is true that $a_{n}\in\mathcal{U}$.
        Suppose $b:\mathbb{N}\rightarrow{Y}$ is not a convergent sequence. Then
        for all $y\in{Y}$ there exists $\mathcal{V}\in\tau_{Y}$ such that
        $y\in\mathcal{V}$ and for all $N\in\mathbb{N}$ there exists an
        $n\in\mathbb{N}$ such that $n>N$ and $b_{n}\notin\mathcal{V}$. Let
        $y=f(x)$ and let $\mathcal{V}_{y}$ be such that $y\in\mathcal{V}_{y}$
        and for all $N\in\mathbb{N}$ there is an $n\in\mathbb{N}$ such that
        $n>N$ and $b_{n}\notin\mathcal{V}_{y}$. But $f$ is continuous, and thus
        $f^{\minus{1}}(\mathcal{V}_{y})\in\tau_{X}$. Moreover, since $y=f(x)$,
        it is true that $x\in{f}^{\minus{1}}(\mathcal{V}_{y})$. But then there
        is an $N\in\mathbb{N}$ such that, for all $n\in\mathbb{N}$ with $n>N$,
        it is true that $a_{n}\in{f}^{\minus{1}}(\mathcal{V}_{y})$. But then
        for all such $n$ it is true that $f(a_{n})\in\mathcal{V}_{y}$. But
        $b_{n}=f(a_{n})$ and there exists an $n>N$ such that
        $b_{n}\notin\mathcal{V}_{y}$, a contradiction. Therefore, $b$ is a
        convergent sequence in $Y$.
    \end{proof}
    \subsection{Separation Axioms}
    \begin{ldefinition}{Fr\'{e}chet Topological Space}
                       {Frechet_Topological_Space}
        A Fr\'{e}chet Topological Space is a topological space $(X,d)$ such that
        for all distinct $x,y\in{X}$ there is an open set $\mathcal{U}$ such
        that $x\in\mathcal{U}$ and $y\notin\mathcal{U}$.
    \end{ldefinition}
    \begin{theorem}
        If $(X,\tau)$ is a Fr\'{e}chet Topological space, and if $x\in{X}$,
        $\{x\}$ is closed subset.
    \end{theorem}
    \begin{proof}
        For if $x\in{X}$, then for all $y\in{X}$ such that $y\ne{x}$, there is
        a $\mathcal{U}_{y}\in\tau$ such that $x\notin\mathcal{U}_{y}$ and
        $y\in\mathcal{U}_{y}$. Define $\mathcal{V}$ by:
        \begin{equation}
            \mathcal{V}\;=\bigcup_{y\in{X}\setminus\{x\}}\mathcal{U}_{y}
        \end{equation}
        But then $\mathcal{V}\in\tau$, since $\tau$ is a topology
        (Def.~\ref{def:Topology}). And moreover, for all $y\in{X}$ such that
        $y\ne{x}$, it is true that $y\in\mathcal{V}$. Lastly,
        $x\notin\mathcal{V}$. Therefore $X\setminus\mathcal{V}=\{x\}$. But if
        $\mathcal{V}$ is open, then $X\setminus\mathcal{V}$ is closed
        (Thm.~\ref{thm:Comp_of_Open_is_Closed}). Therefore, etc.
    \end{proof}
    \begin{ldefinition}{Hausdorff Topological Space}
          {Hausdoff_Top}
        A Hausdorff Topological space is a topological space
        $(X,\tau)$ such that, for all distinct points
        $x,y\in{X}$, there are disjoint open subsets
        $\mathcal{U}_{x}$, $\mathcal{U}_{y}\in\tau$ such that
        $x\in\mathcal{U}_{x}$ and $y\in\mathcal{U}_{y}$.
    \end{ldefinition}
    \begin{theorem}
        If $(X,\tau)$ is a Hausdorff topological space, then
        it is a Fr\'{e}chet topological space.
    \end{theorem}
    \begin{proof}
        For if $x$ and $y$ are distinct points in $X$ and if
        $(X,\tau)$ is a Hausdorff topological space, then there
        is are disjoint open subsets $\mathcal{U}_{x}$ and
        $\mathcal{U}_{y}$ such that $x\in\mathcal{U}_{x}$ and
        $y\in\mathcal{U}_{y}$. But then there is a open subset
        such that $x\notin\mathcal{U}_{y}$ and
        $y\in\mathcal{U}_{y}$. Therefore, $(X,\tau)$ is a
        Fr'{e}chet topological space.
    \end{proof}
        \begin{theorem}
        Convergence in a Hausdorff Space $(X,\tau)$ is unique.
        \end{theorem}
        \begin{proof}
        $[x_n \rightarrow x\in X]\land [x_n \rightarrow y\in X]\land[x\ne y]\Rightarrow [\exists \mathcal{U},\mathcal{V}:\mathcal{U}\cap \mathcal{V}=\emptyset\land x\in \mathcal{U}\land y\in \mathcal{V}]$. $[x_n\rightarrow x]\Rightarrow [\exists N_1\in \mathbb{N}:n>N_1\Rightarrow x_n \in \mathbb{N}]$. $[x_n\rightarrow y]\Rightarrow [N_2\in \mathbb{N}:n>N\Rightarrow x_n \in \mathcal{V}]$. $[n>\max\{N_1,N_2\}]\Rightarrow [x_n \in \mathcal{U}\cap \mathcal{V}]$, a contradiction. Therefore, etc.
        \end{proof}
        \begin{definition}
        A topological space $(X,\tau)$ is said to be regular if for each closed subset $E\subset X$ and for each point $x\in E^c$, there exist disjoint open sets $\mathcal{U}$ and $\mathcal{V}$ such that $x\in \mathcal{U}$ and $E\subset \mathcal{V}$.
        \end{definition} 
        \begin{definition}
        In a topological space $(X,\tau)$, a point $p$ is said to have a neighborhood $S\subset X$ if and only if there is a set $\mathcal{U}\subset S$ such that $\mathcal{U}\in \tau$ and $p\in \mathcal{U}$.
        \end{definition}
        \begin{definition}
        A $T_3$ space is a regular $T_1$ space.
        \end{definition}
        \begin{theorem}
        A $T_3$ space $(X,\tau)$ is a $T_2$ space.
        \end{theorem}
        \begin{proof}
        Let $x,y\in X$ be distinct. As a $T_3$ space is $T_1$, $\{x\}$ is closed. Thus $\exists \mathcal{U},\mathcal{V}\in\tau: \mathcal{U}\cap\mathcal{V}=\emptyset, \{x\}\subset \mathcal{U}$, and $y\in \mathcal{V}$.
        \end{proof}
        \begin{definition}
        A topological space $(X,\tau)$ is said to be normal if and only if for all disjoint closed subsets $E,F\subset X$, there are disjoint open sets $\mathcal{U}$ and $\mathcal{V}$ such that $E\subset \mathcal{U}$ and $F\subset \mathcal{V}$.
        \end{definition}
        \begin{definition}
        A $T_4$ space is a normal $T_1$ space.
        \end{definition}
        \begin{theorem}
        A $T_4$ space $(X,\tau)$ is a $T_3$ space.
        \end{theorem}
        \begin{proof}
        A $T_4$ space is $T_1$. If $E\underset{Closed}\subset X$ and $x\in E^c$, then $\{x\}$ is closed. Thus $\exists \mathcal{U},\mathcal{V}\in\tau: \mathcal{U}\cap\mathcal{V}=\emptyset, \{x\}\subset \mathcal{U}$, and $E\subset \mathcal{V}$.
        \end{proof}
        \begin{definition}
        A homeomorphism between two topological spaces $(X,\tau)$ and $(Y,\tau)$ is a continuous bijection $f:X\rightarrow Y$ such that $f^{-1}:Y\rightarrow X$ is continuous.
        \end{definition}
        \begin{definition}
        If $(X,\tau)$ is a topological space, and $S\subset X$, then an open cover $\mathcal{O}$ of $S$ is a set of open sets $\mathcal{U}_{\alpha}$ such that $S\subset \cup_{\alpha\in A} \mathcal{U}_{\alpha}$, where $A$ is some index set.
        \end{definition}
        \begin{definition}
        A subcover of an open cover $\mathcal{O}$ is a subset of $\mathcal{O}$ that is also a cover.
        \end{definition}
        \begin{definition}
        If $(X,\tau)$ is a topological space and $S\subset X$, then $S$ is said to be compact if and only if every open cover of $S$ has a finite subcover.
        \end{definition}
        \begin{theorem}
        If $S$ is a compact subset of a Hausdorff space, then for all $x\in S^c$ there are disjoint open sets $\mathcal{U}$ and $\mathcal{V}$ such that $x\in \mathcal{U}$ and $S\subset \mathcal{V}$.
        \end{theorem}
        \begin{proof}
        For let $x\in S^c$. For all $y\in S$ there are disjoint open sets $\mathcal{U}_y$ and $\mathcal{V}_y$ such that $x\in \mathcal{U}$ and $y\in \mathcal{V}$. But then $\cup_{y\in S} \mathcal{U}_y$ is an open cover of $S$. As $S$ is compact, there is a finite subcover, that is sets $\mathcal{V}_{y_1},\hdots, \mathcal{V}_{y_n}$ that cover $S$. But then $\cap_{k=1}^{n} \mathcal{U}_{y_k}$ is open, contains $x$ and is disjoint from $\cup_{k=1}^{n} \mathcal{V}_{y_k}$. Therefore, etc.
        \end{proof}
        \begin{theorem}
        Every compact subset of a Hausdorff space $(X,\tau)$ is closed.
        \end{theorem}
        \begin{proof}
        Let $S$ be a compact subset of a X. $\forall x\in S^c, \exists \mathcal{U}_x\in \tau:\mathcal{U}_x\cap S = \emptyset:x\in \mathcal{U}_x$. But then $S^c \subset \underset{x\in S^c}\cup\mathcal{U}_x$. But also $S\cap (\cup_{x\in S^c}\mathcal{U}_x) = \emptyset$. Thus $S^c = \cup_{x\in S^c}\mathcal{U}_x$, and therefore $S^c$ is open. Thus $S$ is closed.
        \end{proof}
        \begin{theorem}
        If $S$ is a closed subset of a compact space $(X,\tau)$, $S$ is compact.
        \end{theorem}
        \begin{proof}
        For let $\mathcal{O}$ be an open cover of $S$. As $S$ is closed, $S^c$ is open, and thus $\{S^c\} \cup \mathcal{O}$ is an open cover $X$. As $X$ is compact, there is a finite subcover, call it $\mathscr{O}$. But then $\mathscr{O}\setminus \{S^c\}$ is a finite subcover $\mathcal{O}$ that covers $S$. Thus, etc.
        \end{proof}
        \begin{theorem}
        If $f:X\rightarrow Y$ is continuous and $X$ is compact, then $f(X)\subset Y$ is compact.
        \end{theorem}
        \begin{proof}
        Let $\mathcal{O}$ be an open cover of $f(X)$. As $f$ is continuous, $\mathcal{U}\in\mathcal{O}\Rightarrow f^{-1}(\mathcal{U})$ is open in $X$. Thus $\cup_{\mathcal{U}\in \mathcal{O}} f^{-1}(\mathcal{U})$ is an open cover of $X$. As $X$ is compact, there is a finite subcover, say $\mathscr{O}$. But then $\cup_{\mathcal{V}\in \mathscr{O}} \mathcal{V}$ is a finite subcover of $\mathcal{O}$. Therefore, etc.
        \end{proof}
        \begin{theorem}
        If $f:X\rightarrow Y$ is a continuous bijection, $X$ is compact and $Y$ is Hausdorff, then $f$ is a homeomorphism.
        \end{theorem}
        \begin{proof}
        If suffices to show that if $\mathcal{U}$ is open in $X$, then $f(\mathcal{U})$ is open in $f(X)$. Let $\mathcal{U}$ be open in $X$. As $X$ is compact and $\mathcal{U}$ is open, $\mathcal{U}^c$ is compact. But then $f(\mathcal{U}^c) = f(X)\setminus f(\mathcal{U})$ is compact. Thus $f(X)\setminus f(\mathcal{U})\underset{Closed}\subset f(X)\Rightarrow f(\mathcal{U})\underset{Open}\subset f(X)$.
        \end{proof}
        \begin{definition}
        A topological space $(X,\tau)$ is said to be disconnected if and only if there are two disjoint nonempty open sets $\mathcal{U}$ and $\mathcal{V}$ such that $X = \mathcal{U}\cup \mathcal{V}$.
        \end{definition}
        \begin{theorem}
        A topological space $(X,\tau)$ is disconnected if and only if there are two non-empty disjoint closed set $\mathcal{C}$ and $\mathcal{D}$ such that $X=\mathcal{C}\cup\mathcal{D}$.
        \end{theorem}
        \begin{proof}
        $\big[\exists \mathcal{U},\mathcal{V}\in \tau: [\mathcal{U}\cap \mathcal{V}=\emptyset]\land [X=\mathcal{U}\cup \mathcal{V}]\land [\mathcal{U},\mathcal{V}\ne \emptyset]\big]\Rightarrow [X = \mathcal{U}^c\cup \mathcal{V}^c]$ thus, $X$ is the union of disjoint, non-empty closed set. $[\mathcal{C}^c,\mathcal{D}^c\in \tau]\land[\mathcal{C},\mathcal{V}\ne\emptyset]\land[\mathcal{C}\cap \mathcal{D}=\emptyset]\land[\mathcal{C}\cup\mathcal{D}=X]\Rightarrow [X=\mathcal{C}^c\cup\mathcal{D}^c].$ Thus $X$ is disconnected.
        \end{proof}
        \begin{theorem}
        $(X,\tau)$ is disconnected if and only if there is a proper, nonempty set $A\subset X$ that is both open and closed.
        \end{theorem}
        \begin{proof}
        $\big[\exists \mathcal{U},\mathcal{V}\in \tau:[\mathcal{U}\cap \mathcal{V}=\emptyset]\land [X=\mathcal{U}\cup\mathcal{V}]\land[\mathcal{U},\mathcal{V}\ne \emptyset]\big]\Rightarrow [\mathcal{U}^c = \mathcal{V}]\Rightarrow [\mathcal{U}^c\in \tau]$. Thus, $\mathcal{U}$ is open and closed.
        \end{proof}
        \begin{definition}
        A topological space is called connected if and only if it is not disconnected.
        \end{definition}
        \begin{theorem}
        If $f:X\rightarrow Y$ is a continuous function and $X$ is connected, then $f(X)$ is connected.
        \end{theorem}
        \begin{proof}
        For let $f$ be continuous and $X$ be connected. Suppose $f(X)$ is disconnected. Then there are two nonempty open disjoint sets $\mathcal{U}$ and $\mathcal{V}$ such that $f(X) = \mathcal{U}\cap \mathcal{V}$. But then their preimage is open, and thus $X=f^{-1}(\mathcal{U})\cup f^{-1}(\mathcal{V})$, and thus $X$ is disconnected, a contradiction. Thus $f(X)$ is connected.
        \end{proof}
        \begin{definition}
        If $(X,\tau)$ and $(Y,\tau')$ are topological spaces, then the product topology on the set $X\times Y$ is the set $\mathscr{T} = \{\mathcal{U}\times \mathcal{V}:\mathcal{U}\in\tau,\mathcal{V}\in \tau'\}$.
        \end{definition}
        \begin{theorem}
        The product topology is a topology.
        \end{theorem}
        \begin{proof}
        \
        \begin{enumerate}
        \item As $\emptyset \in \tau$ and $\emptyset\in \tau'$, $\emptyset =\emptyset\times \emptyset \in \mathscr{T}$.
        \item If $\mathscr{U}_{\alpha}\in \mathscr{T}$, then $\cup_{\alpha} \mathscr{U}_{\alpha} = \cup_{\alpha} (\mathcal{U}_{\alpha},\mathcal{V}_{\alpha})$. As $\tau$ and $\tau'$ are topologies, $\cup_{\alpha} \mathcal{U} \in \tau$ and $\cup_{\alpha}\mathcal{V}_{\alpha} \in \tau'$. Thus, $\cup_{\alpha}\mathscr{U}_{\alpha} \in \mathscr{T}$.
        \item $\cap_{k=1}^{n} \mathscr{U}_{k} = \cap_{k=1}^{n} (\mathcal{U}_k,\mathcal{V}_k)$. As $\tau$ and $\tau'$ are topologies, $\cap_{k=1}^{n}\mathcal{U}_k \in \tau$ and $\cap_{k=1}^{n}\mathcal{V}_{k} \in \tau'$. Thus $\cap_{k=1}^{n} \mathscr{U}_k \in \mathscr{T}$
        \end{enumerate}
        \end{proof}
        \begin{definition}
        The projection map $\pi_1$ is defined as $\pi_1:X_1\times X_2\rightarrow X_1$ by $(x_1,x_2)\mapsto x_1$. Similarly for $\pi_2$.
        \end{definition}
        \begin{theorem}
        The projection map is continuous.
        \end{theorem}
        \begin{proof}
        Let $\pi_1:X_1\times X_2\rightarrow X_1$ be the projection map, $X\times Y$ having the project topology. Let $\mathcal{U}\underset{Open}\subset X_1$. Then $f^{-1}(\mathcal{U}) = \{(x_1,x_2):x_1\in \mathcal{U}, x_2\in X_2\}$. But $\mathcal{U}$ and $X_2$ are open, and thus $f^{-1}(\mathcal{U})$ is open (In the product topology).
        \end{proof}
        \begin{definition}
        An open mapping is a function $f:X\rightarrow Y$ such that $\mathcal{U}\underset{Open}\subset X\Rightarrow f(\mathcal{U}) \underset{Open}\subset Y$.
        \end{definition}
        \begin{theorem}
        The projection map is an open mapping.
        \end{theorem}
        \begin{proof}
        For let $\mathscr{U}$ be an open set in $X\times Y$ (With the product topology). That is, there are open sets $\mathcal{U}\subset X$ and $\mathcal{V}\subset Y$ such that $\mathscr{U}= \{(x,y):x\in \mathcal{U},y\in \mathcal{V}\}$, Then $\pi_1(\mathscr{U}) =\mathcal{U}$, which is open. Therefore, etc.
        \end{proof}
        \begin{theorem}
        If $X$ and $Y$ are compact, then $X\times Y$ is compact with the product topology.
        \end{theorem}
        \begin{proof}
        For let $\mathscr{O}$ be an open cover of $X\times Y$. Then $\{\pi_X(\mathscr{U}):\mathscr{U}\in \mathscr{O}\}$ is an open cover of $X$ and $\{\pi_{Y}(\mathscr{V}):\mathscr{V}\in \mathscr{O}\}$ is an open cover of $Y$. As $X$ and $Y$ are compact, there exist finite subcovers of each, say $\mathcal{O}_X$ and $\mathcal{O}_Y$. But then $\{\pi_{X}^{-1}(\mathcal{U}):\mathcal{U}\in \mathcal{O}_X\}\cup \{\pi_{Y}^{-1}(\mathcal{V}):\mathcal{V}\in \mathcal{O}_Y\}$ is a finite subcover of $\mathscr{O}$. Thus, $X\times Y$ is compact.
        \end{proof}
        \begin{theorem}
        If $X,Y\subset Z$ are compact, $X\cup Y$ is compact.
        \end{theorem}
        \begin{proof}
        Let $\mathcal{O}$ be an open cover of $X\cup Y$. Then there is a finite subcover of $X$ and a finite subcover of $Y$, and thus the combination of these subcovers is a cover of $X\cup Y$.
        \end{proof}
    \section{Old Notes}
    \begin{example}
        The Sierpinski topology
        on $\{0,1\}$ is the set
        $\{\emptyset,\{0\},\{0,1\}\}$.
    \end{example}
    \begin{theorem}
        If $T_{\omega}$ is a set of topologies on
        a topological space $X$, then
        $\bigcap{T_{\omega}}$ is a topology on $X$.
    \end{theorem}
    However, the union of topologies may not be a
    topology. A topology $\tau_{1}$ is set to
    be finer than a topology $\tau_{0}$ if
    $\tau_{0}\subset\tau_{1}$. An accumulation point
    of a set $A$ is a point $x$ such that, for
    all open neighborhoods $U$ of $A$,
    $U\cap{A}\ne\emptyset$.
    \begin{theorem}[Bolzano-Weierstrass Theorem]
        If $X$ is a bounded, infinite subset of
        $\mathbb{R}$, then $X$ has at least
        one accumulation point.
    \end{theorem}
    \begin{definition}
        The Euclidean topology on
        $\mathbb{R}$ is the set of
        all open sets in the sense that
        $U$ is open if, for all $x\in{U}$,
        there is an $\varepsilon>0$ such
        that $(x-\varepsilon,x+\varepsilon)\subset{U}$.
    \end{definition}
    \begin{definition}
        A closed subset of a topological space
        $(X,\tau)$ is a set $A$ such that
        $A^{C}\in\tau$.
    \end{definition}
    \begin{theorem}
        The intersection of an arbitrary collection of
        closed sets is closed. The union of finitely
        many closed sets is closed.
    \end{theorem}
    \begin{proof}
        Apply DeMorgan's theorem to the properties
        of a topological space $\tau$.
    \end{proof}
    \begin{definition}
        The closure of a set $A$,
        denoted $\overline{A}$, is the
        intersection of all closed sets
        containing $A$.
    \end{definition}
    \begin{theorem}
        If $A$ is a set in a topological space
        $(X,\tau)$, then $A\subset\overline{A}$.
    \end{theorem}
    There's also something called the derived
    set of a set $A$. The interior of $A$
    is the union of all open subset of $A$.
    The boundary of $A$ is the set difference
    of the closure of $A$ and the interior of
    $A$.
    \begin{theorem}
        If $A$ is a set, then
        the interior of $A$ is equal to
        $(\overline{A^{C}})^{C}$
    \end{theorem}
    \begin{definition}
        A dense subset of a topological
        space $(X,\tau)$ is a set $A$
        such that $\overline{A}=X$.
    \end{definition}
    \begin{definition}
        The neighborhood system of a point
        $x$ in a topological space $(X,\tau)$
        is the set of all neighborhoods of
        $x$.
    \end{definition}
    \begin{definition}
        A sequence in a topological space
        $a_{n}$ converges to a point $a$ if,
        for all open neighborhoods $U$ of $a$,
        there is an $N\in\mathbb{N}$ such that,
        for all $n>N$, $a_{n}\in{U}$.
    \end{definition}
    Limits of sequences in topological spaces are NOT
    necessarily unique. This is different from convergence
    in $\mathbb{R}$, where convergence is always unique.
    \begin{definition}
        The relative topology of a
        topological space $(X,\tau)$ with
        respect to a subset $A\subset{X}$
        is $\tau_{A}=\{A\cap{U}:U\in\tau\}$
    \end{definition}
    \begin{theorem}
        If $(X,\tau)$ is a topological space and
        $A\subset{X}$, then
        $(A,\tau_{A})$ is a topological space.
    \end{theorem}
    $(A,\tau_{A})$ is called a subspace of
    $(X,\tau)$.
    \begin{definition}
        A basis of a topological space
        $(X,\tau)$ is a subset $B$ of
        $\tau$ such that every element
        of $\tau$ is the union of some of the
        elements of $B$.
    \end{definition}
    \begin{theorem}
        A subset $B\subset\tau$ is a basis
        for $\tau$ if and only if for all
        $U\in\tau$ and all $x\in{U}$, there is
        a $V\in{B}$ such that
        $x\in{V}\subset{U}$.
    \end{theorem}
    \begin{theorem}
        If $B$ is a basis of $\tau$, then
        $U$ is open if and only if for all
        $x\in{U}$ there is a $V\in{B}$ such that
        $x\in{V}\subset{U}$.
    \end{theorem}
    \begin{theorem}
        $\mathbb{R}$ has a countable basis.
    \end{theorem}
    \begin{proof}
        For the set of open intervals
        $(p,q)$, where $p$ and $q$ are rational
        numbers, forms a basis for the standard
        topology on $\mathbb{R}$. Moreover, this
        is countable.
    \end{proof}
    \begin{definition}
        If $(X,\tau)$ is a topological space
        and $S\subset\tau$, then $S$ is a subbase
        if a finite intersection of elements of $S$
        forms a base of $\tau$.
    \end{definition}
    \begin{definition}
        A local base for a point
        $x$ in a topological space $(X,\tau)$
        is a set of open neighborhods $B_{x}$ of
        $x$ such that for all open neighborhoods $G$
        of $x$, there is a $G_{x}\in{B_{x}}$ such that
        $x\in{G_{x}}\subset{G}$.
    \end{definition}
    \begin{theorem}
        If $(X,\tau)$ is a topological space, $x\in{X}$,
        and if $B$ is a base for $\tau$, then
        the set of elements $G_{x}$ in $B$ such that
        $x\in{G_{x}}$ is a local base for $x$.
    \end{theorem}
