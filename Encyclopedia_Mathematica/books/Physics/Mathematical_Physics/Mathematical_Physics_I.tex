\documentclass[crop=false,class=book,oneside]{standalone}
%----------------------------Preamble-------------------------------%
%---------------------------Packages----------------------------%
\usepackage{geometry}
\geometry{b5paper, margin=1.0in}
\usepackage[T1]{fontenc}
\usepackage{graphicx, float}            % Graphics/Images.
\usepackage{natbib}                     % For bibliographies.
\bibliographystyle{agsm}                % Bibliography style.
\usepackage[french, english]{babel}     % Language typesetting.
\usepackage[dvipsnames]{xcolor}         % Color names.
\usepackage{listings}                   % Verbatim-Like Tools.
\usepackage{mathtools, esint, mathrsfs} % amsmath and integrals.
\usepackage{amsthm, amsfonts, amssymb}  % Fonts and theorems.
\usepackage{tcolorbox}                  % Frames around theorems.
\usepackage{upgreek}                    % Non-Italic Greek.
\usepackage{fmtcount, etoolbox}         % For the \book{} command.
\usepackage[newparttoc]{titlesec}       % Formatting chapter, etc.
\usepackage{titletoc}                   % Allows \book in toc.
\usepackage[nottoc]{tocbibind}          % Bibliography in toc.
\usepackage[titles]{tocloft}            % ToC formatting.
\usepackage{pgfplots, tikz}             % Drawing/graphing tools.
\usepackage{imakeidx}                   % Used for index.
\usetikzlibrary{
    calc,                   % Calculating right angles and more.
    angles,                 % Drawing angles within triangles.
    arrows.meta,            % Latex and Stealth arrows.
    quotes,                 % Adding labels to angles.
    positioning,            % Relative positioning of nodes.
    decorations.markings,   % Adding arrows in the middle of a line.
    patterns,
    arrows
}                                       % Libraries for tikz.
\pgfplotsset{compat=1.9}                % Version of pgfplots.
\usepackage[font=scriptsize,
            labelformat=simple,
            labelsep=colon]{subcaption} % Subfigure captions.
\usepackage[font={scriptsize},
            hypcap=true,
            labelsep=colon]{caption}    % Figure captions.
\usepackage[pdftex,
            pdfauthor={Ryan Maguire},
            pdftitle={Mathematics and Physics},
            pdfsubject={Mathematics, Physics, Science},
            pdfkeywords={Mathematics, Physics, Computer Science, Biology},
            pdfproducer={LaTeX},
            pdfcreator={pdflatex}]{hyperref}
\hypersetup{
    colorlinks=true,
    linkcolor=blue,
    filecolor=magenta,
    urlcolor=Cerulean,
    citecolor=SkyBlue
}                           % Colors for hyperref.
\usepackage[toc,acronym,nogroupskip,nopostdot]{glossaries}
\usepackage{glossary-mcols}
%------------------------Theorem Styles-------------------------%
\theoremstyle{plain}
\newtheorem{theorem}{Theorem}[section]

% Define theorem style for default spacing and normal font.
\newtheoremstyle{normal}
    {\topsep}               % Amount of space above the theorem.
    {\topsep}               % Amount of space below the theorem.
    {}                      % Font used for body of theorem.
    {}                      % Measure of space to indent.
    {\bfseries}             % Font of the header of the theorem.
    {}                      % Punctuation between head and body.
    {.5em}                  % Space after theorem head.
    {}

% Italic header environment.
\newtheoremstyle{thmit}{\topsep}{\topsep}{}{}{\itshape}{}{0.5em}{}

% Define environments with italic headers.
\theoremstyle{thmit}
\newtheorem*{solution}{Solution}

% Define default environments.
\theoremstyle{normal}
\newtheorem{example}{Example}[section]
\newtheorem{definition}{Definition}[section]
\newtheorem{problem}{Problem}[section]

% Define framed environment.
\tcbuselibrary{most}
\newtcbtheorem[use counter*=theorem]{ftheorem}{Theorem}{%
    before=\par\vspace{2ex},
    boxsep=0.5\topsep,
    after=\par\vspace{2ex},
    colback=green!5,
    colframe=green!35!black,
    fonttitle=\bfseries\upshape%
}{thm}

\newtcbtheorem[auto counter, number within=section]{faxiom}{Axiom}{%
    before=\par\vspace{2ex},
    boxsep=0.5\topsep,
    after=\par\vspace{2ex},
    colback=Apricot!5,
    colframe=Apricot!35!black,
    fonttitle=\bfseries\upshape%
}{ax}

\newtcbtheorem[use counter*=definition]{fdefinition}{Definition}{%
    before=\par\vspace{2ex},
    boxsep=0.5\topsep,
    after=\par\vspace{2ex},
    colback=blue!5!white,
    colframe=blue!75!black,
    fonttitle=\bfseries\upshape%
}{def}

\newtcbtheorem[use counter*=example]{fexample}{Example}{%
    before=\par\vspace{2ex},
    boxsep=0.5\topsep,
    after=\par\vspace{2ex},
    colback=red!5!white,
    colframe=red!75!black,
    fonttitle=\bfseries\upshape%
}{ex}

\newtcbtheorem[auto counter, number within=section]{fnotation}{Notation}{%
    before=\par\vspace{2ex},
    boxsep=0.5\topsep,
    after=\par\vspace{2ex},
    colback=SeaGreen!5!white,
    colframe=SeaGreen!75!black,
    fonttitle=\bfseries\upshape%
}{not}

\newtcbtheorem[use counter*=remark]{fremark}{Remark}{%
    fonttitle=\bfseries\upshape,
    colback=Goldenrod!5!white,
    colframe=Goldenrod!75!black}{ex}

\newenvironment{bproof}{\textit{Proof.}}{\hfill$\square$}
\tcolorboxenvironment{bproof}{%
    blanker,
    breakable,
    left=3mm,
    before skip=5pt,
    after skip=10pt,
    borderline west={0.6mm}{0pt}{green!80!black}
}

\AtEndEnvironment{lexample}{$\hfill\textcolor{red}{\blacksquare}$}
\newtcbtheorem[use counter*=example]{lexample}{Example}{%
    empty,
    title={Example~\theexample},
    boxed title style={%
        empty,
        size=minimal,
        toprule=2pt,
        top=0.5\topsep,
    },
    coltitle=red,
    fonttitle=\bfseries,
    parbox=false,
    boxsep=0pt,
    before=\par\vspace{2ex},
    left=0pt,
    right=0pt,
    top=3ex,
    bottom=1ex,
    before=\par\vspace{2ex},
    after=\par\vspace{2ex},
    breakable,
    pad at break*=0mm,
    vfill before first,
    overlay unbroken={%
        \draw[red, line width=2pt]
            ([yshift=-1.2ex]title.south-|frame.west) to
            ([yshift=-1.2ex]title.south-|frame.east);
        },
    overlay first={%
        \draw[red, line width=2pt]
            ([yshift=-1.2ex]title.south-|frame.west) to
            ([yshift=-1.2ex]title.south-|frame.east);
    },
}{ex}

\AtEndEnvironment{ldefinition}{$\hfill\textcolor{Blue}{\blacksquare}$}
\newtcbtheorem[use counter*=definition]{ldefinition}{Definition}{%
    empty,
    title={Definition~\thedefinition:~{#1}},
    boxed title style={%
        empty,
        size=minimal,
        toprule=2pt,
        top=0.5\topsep,
    },
    coltitle=Blue,
    fonttitle=\bfseries,
    parbox=false,
    boxsep=0pt,
    before=\par\vspace{2ex},
    left=0pt,
    right=0pt,
    top=3ex,
    bottom=0pt,
    before=\par\vspace{2ex},
    after=\par\vspace{1ex},
    breakable,
    pad at break*=0mm,
    vfill before first,
    overlay unbroken={%
        \draw[Blue, line width=2pt]
            ([yshift=-1.2ex]title.south-|frame.west) to
            ([yshift=-1.2ex]title.south-|frame.east);
        },
    overlay first={%
        \draw[Blue, line width=2pt]
            ([yshift=-1.2ex]title.south-|frame.west) to
            ([yshift=-1.2ex]title.south-|frame.east);
    },
}{def}

\AtEndEnvironment{ltheorem}{$\hfill\textcolor{Green}{\blacksquare}$}
\newtcbtheorem[use counter*=theorem]{ltheorem}{Theorem}{%
    empty,
    title={Theorem~\thetheorem:~{#1}},
    boxed title style={%
        empty,
        size=minimal,
        toprule=2pt,
        top=0.5\topsep,
    },
    coltitle=Green,
    fonttitle=\bfseries,
    parbox=false,
    boxsep=0pt,
    before=\par\vspace{2ex},
    left=0pt,
    right=0pt,
    top=3ex,
    bottom=-1.5ex,
    breakable,
    pad at break*=0mm,
    vfill before first,
    overlay unbroken={%
        \draw[Green, line width=2pt]
            ([yshift=-1.2ex]title.south-|frame.west) to
            ([yshift=-1.2ex]title.south-|frame.east);},
    overlay first={%
        \draw[Green, line width=2pt]
            ([yshift=-1.2ex]title.south-|frame.west) to
            ([yshift=-1.2ex]title.south-|frame.east);
    }
}{thm}

%--------------------Declared Math Operators--------------------%
\DeclareMathOperator{\adjoint}{adj}         % Adjoint.
\DeclareMathOperator{\Card}{Card}           % Cardinality.
\DeclareMathOperator{\curl}{curl}           % Curl.
\DeclareMathOperator{\diam}{diam}           % Diameter.
\DeclareMathOperator{\dist}{dist}           % Distance.
\DeclareMathOperator{\Div}{div}             % Divergence.
\DeclareMathOperator{\Erf}{Erf}             % Error Function.
\DeclareMathOperator{\Erfc}{Erfc}           % Complementary Error Function.
\DeclareMathOperator{\Ext}{Ext}             % Exterior.
\DeclareMathOperator{\GCD}{GCD}             % Greatest common denominator.
\DeclareMathOperator{\grad}{grad}           % Gradient
\DeclareMathOperator{\Ima}{Im}              % Image.
\DeclareMathOperator{\Int}{Int}             % Interior.
\DeclareMathOperator{\LC}{LC}               % Leading coefficient.
\DeclareMathOperator{\LCM}{LCM}             % Least common multiple.
\DeclareMathOperator{\LM}{LM}               % Leading monomial.
\DeclareMathOperator{\LT}{LT}               % Leading term.
\DeclareMathOperator{\Mod}{mod}             % Modulus.
\DeclareMathOperator{\Mon}{Mon}             % Monomial.
\DeclareMathOperator{\multideg}{mutlideg}   % Multi-Degree (Graphs).
\DeclareMathOperator{\nul}{nul}             % Null space of operator.
\DeclareMathOperator{\Ord}{Ord}             % Ordinal of ordered set.
\DeclareMathOperator{\Prin}{Prin}           % Principal value.
\DeclareMathOperator{\proj}{proj}           % Projection.
\DeclareMathOperator{\Refl}{Refl}           % Reflection operator.
\DeclareMathOperator{\rk}{rk}               % Rank of operator.
\DeclareMathOperator{\sgn}{sgn}             % Sign of a number.
\DeclareMathOperator{\sinc}{sinc}           % Sinc function.
\DeclareMathOperator{\Span}{Span}           % Span of a set.
\DeclareMathOperator{\Spec}{Spec}           % Spectrum.
\DeclareMathOperator{\supp}{supp}           % Support
\DeclareMathOperator{\Tr}{Tr}               % Trace of matrix.
%--------------------Declared Math Symbols--------------------%
\DeclareMathSymbol{\minus}{\mathbin}{AMSa}{"39} % Unary minus sign.
%------------------------New Commands---------------------------%
\DeclarePairedDelimiter\norm{\lVert}{\rVert}
\DeclarePairedDelimiter\ceil{\lceil}{\rceil}
\DeclarePairedDelimiter\floor{\lfloor}{\rfloor}
\newcommand*\diff{\mathop{}\!\mathrm{d}}
\newcommand*\Diff[1]{\mathop{}\!\mathrm{d^#1}}
\renewcommand*{\glstextformat}[1]{\textcolor{RoyalBlue}{#1}}
\renewcommand{\glsnamefont}[1]{\textbf{#1}}
\renewcommand\labelitemii{$\circ$}
\renewcommand\thesubfigure{%
    \arabic{chapter}.\arabic{figure}.\arabic{subfigure}}
\addto\captionsenglish{\renewcommand{\figurename}{Fig.}}
\numberwithin{equation}{section}

\renewcommand{\vector}[1]{\boldsymbol{\mathrm{#1}}}

\newcommand{\uvector}[1]{\boldsymbol{\hat{\mathrm{#1}}}}
\newcommand{\topspace}[2][]{(#2,\tau_{#1})}
\newcommand{\measurespace}[2][]{(#2,\varSigma_{#1},\mu_{#1})}
\newcommand{\measurablespace}[2][]{(#2,\varSigma_{#1})}
\newcommand{\manifold}[2][]{(#2,\tau_{#1},\mathcal{A}_{#1})}
\newcommand{\tanspace}[2]{T_{#1}{#2}}
\newcommand{\cotanspace}[2]{T_{#1}^{*}{#2}}
\newcommand{\Ckspace}[3][\mathbb{R}]{C^{#2}(#3,#1)}
\newcommand{\funcspace}[2][\mathbb{R}]{\mathcal{F}(#2,#1)}
\newcommand{\smoothvecf}[1]{\mathfrak{X}(#1)}
\newcommand{\smoothonef}[1]{\mathfrak{X}^{*}(#1)}
\newcommand{\bracket}[2]{[#1,#2]}

%------------------------Book Command---------------------------%
\makeatletter
\renewcommand\@pnumwidth{1cm}
\newcounter{book}
\renewcommand\thebook{\@Roman\c@book}
\newcommand\book{%
    \if@openright
        \cleardoublepage
    \else
        \clearpage
    \fi
    \thispagestyle{plain}%
    \if@twocolumn
        \onecolumn
        \@tempswatrue
    \else
        \@tempswafalse
    \fi
    \null\vfil
    \secdef\@book\@sbook
}
\def\@book[#1]#2{%
    \refstepcounter{book}
    \addcontentsline{toc}{book}{\bookname\ \thebook:\hspace{1em}#1}
    \markboth{}{}
    {\centering
     \interlinepenalty\@M
     \normalfont
     \huge\bfseries\bookname\nobreakspace\thebook
     \par
     \vskip 20\p@
     \Huge\bfseries#2\par}%
    \@endbook}
\def\@sbook#1{%
    {\centering
     \interlinepenalty \@M
     \normalfont
     \Huge\bfseries#1\par}%
    \@endbook}
\def\@endbook{
    \vfil\newpage
        \if@twoside
            \if@openright
                \null
                \thispagestyle{empty}%
                \newpage
            \fi
        \fi
        \if@tempswa
            \twocolumn
        \fi
}
\newcommand*\l@book[2]{%
    \ifnum\c@tocdepth >-3\relax
        \addpenalty{-\@highpenalty}%
        \addvspace{2.25em\@plus\p@}%
        \setlength\@tempdima{3em}%
        \begingroup
            \parindent\z@\rightskip\@pnumwidth
            \parfillskip -\@pnumwidth
            {
                \leavevmode
                \Large\bfseries#1\hfill\hb@xt@\@pnumwidth{\hss#2}
            }
            \par
            \nobreak
            \global\@nobreaktrue
            \everypar{\global\@nobreakfalse\everypar{}}%
        \endgroup
    \fi}
\newcommand\bookname{Book}
\renewcommand{\thebook}{\texorpdfstring{\Numberstring{book}}{book}}
\providecommand*{\toclevel@book}{-2}
\makeatother
\titleformat{\part}[display]
    {\Large\bfseries}
    {\partname\nobreakspace\thepart}
    {0mm}
    {\Huge\bfseries}
\titlecontents{part}[0pt]
    {\large\bfseries}
    {\partname\ \thecontentslabel: \quad}
    {}
    {\hfill\contentspage}
\titlecontents{chapter}[0pt]
    {\bfseries}
    {\chaptername\ \thecontentslabel:\quad}
    {}
    {\hfill\contentspage}
\newglossarystyle{longpara}{%
    \setglossarystyle{long}%
    \renewenvironment{theglossary}{%
        \begin{longtable}[l]{{p{0.25\hsize}p{0.65\hsize}}}
    }{\end{longtable}}%
    \renewcommand{\glossentry}[2]{%
        \glstarget{##1}{\glossentryname{##1}}%
        &\glossentrydesc{##1}{~##2.}
        \tabularnewline%
        \tabularnewline
    }%
}
\newglossary[not-glg]{notation}{not-gls}{not-glo}{Notation}
\newcommand*{\newnotation}[4][]{%
    \newglossaryentry{#2}{type=notation, name={\textbf{#3}, },
                          text={#4}, description={#4},#1}%
}
%--------------------------LENGTHS------------------------------%
% Spacings for the Table of Contents.
\addtolength{\cftsecnumwidth}{1ex}
\addtolength{\cftsubsecindent}{1ex}
\addtolength{\cftsubsecnumwidth}{1ex}
\addtolength{\cftfignumwidth}{1ex}
\addtolength{\cfttabnumwidth}{1ex}

% Indent and paragraph spacing.
\setlength{\parindent}{0em}
\setlength{\parskip}{0em}
\graphicspath{{../../../images/}}       % Path to Image Folder.
%----------------------------GLOSSARY-------------------------------%
\makeglossaries
\loadglsentries{../../../glossary}
\loadglsentries{../../../acronym}
%--------------------------Main Document----------------------------%
\begin{document}
\chapter{Mathematical Physics I}
\section{Cheat Sheet}
\subsection{Series}
\begin{theorem*}[The Comparison Test]
If $a_n$ and $b_n$ are sequences of non-negative real numbers, if there is an $N_0\in \mathbb{N}$ such that for all $n>N_0$, $a_n\leq b_n$, and if $\sum_{n=0}^{N}b_n$ converges, then $\sum_{n=0}^{N}a_n$ converges.
\end{theorem*}
\begin{proof}
Let $B_N=\sum_{n=0}^{N}b_n$, $S = \sum_{n=0}^{N_{0}-1}a_n$ and $B=\lim_{n\rightarrow \infty}B_{n}$. As $b_n$ is non-negative, for all $N\in \mathbb{N}$, $B_{N} \leq B$. Then for $N>N_{0}$, $\sum_{n=0}^{N}a_n = S+\sum_{n=N_{0}}^{N}a_n \leq S+\sum_{n=N_{0}}^{N}b_n \leq S+B_{N} \leq S+B$. That is, $\sum_{n=0}^{N}a_n$ bounded by $S+B$. And as $a_n$ is non-negative, $\sum_{n=0}^{N}a_n$ increasing monotonically, and is thus a monotonically increases sequence that is bounded above, and therefore converges.
\end{proof}
\begin{theorem*}[Generalized Geometric Series Theorem]
If $r$ is a real number, then $\sum_{n=0}^{N} r^n = \frac{1-r^{N+1}}{1-r}$
\end{theorem*}
\begin{proof}
By induction. The base case is $1+r = (1+r)\frac{1-r}{1-r} = \frac{1-r^2}{1-r}$. Suppose it is true for $N \in \mathbb{N}$. Then $\sum_{n=0}^{N+1}r^n=\sum_{n=0}^{N}r^n+r^{N+1}= \frac{1-r^{N+1}}{1-r}+r^{N+1}=\frac{1-r^{N+1}+r^{N+1}(1-r)}{1-r}=\frac{1-r^{N+2}}{1-r}$.
\end{proof}
\begin{theorem}[Geometric Series Theorem]
If $|r|<1$, then $\sum_{n=0}^{\infty} r^n = \frac{1}{1-r}$
\end{theorem}
\begin{proof}
For $\sum_{n=0}^{N}r^{n}=\frac{1-r^{N+1}}{1-r}$. As $|r|<1$, $\lim_{N\rightarrow \infty}r^{N}=0$. Therefore $\sum_{n=0}^{\infty}=\frac{1}{1-r}$
\end{proof}
\begin{theorem*}[First Root Test Theorem]
If $a_n$ is positive and $\lim_{n\rightarrow \infty} \sqrt[n]{a_n}<1$, then $\sum_{n=0}^{\infty} a_n$ converges.
\end{theorem*}
\begin{proof}
If $\lim_{n\rightarrow \infty}\sqrt[n]{a_n}<1$, then there is a $c\in (0,1)$ and an $N\in \mathbb{N}$ such that for all $n>N$, $\sqrt[n]{a_n}<c$. But then $a_n<c^n<1$, so $\sum_{n=0}^{\infty}c^n = \frac{1}{1-c}$. Thus, by the comparison test, $\sum_{n=0}^{N}a_n$ converges.
\end{proof}
\begin{theorem*}[Second Root Test Theorem]
If $a_n$ is positive and $\lim_{n\rightarrow \infty} \sqrt[n]{a_n}>1$, then $\sum_{n=0}^{\infty} a_n$ diverges.
\end{theorem*}
\begin{proof}
If $\lim_{n\rightarrow \infty}\sqrt[n]{a_n}>1$, then there is a $c\in (1,\infty)$ and an $N\in \mathbb{N}$ such that for all $n>N$, $\sqrt[n]{a_n}>c$. But then $a_n>c^n>1$, so $\sum_{n=0}^{\infty}c^n$ diverges. By the comparison test, $\sum_{n=0}^{N}a_n$ diverges.
\end{proof}
\begin{theorem*}
There exists sequences of positive real numbers $a_n$ such that $\sqrt[n]{a_n} \rightarrow 1$ and $\sum_{n=1}^{N} a_n$ diverges.
\end{theorem*}
\begin{proof}
Let $a_n = \frac{1}{n}$. Then $\ln(\sqrt[n]{a_n}) = \frac{\ln(\frac{1}{n})}{n} = -\frac{\ln(n)}{n} \rightarrow 0$. So $\sqrt[n]{a_n} \rightarrow 1$. But $\sum_{n=1}^{N} \frac{1}{n}$ diverges.
\end{proof}
\begin{theorem*}
There exists positive sequences $a_{n}$ such that $\sqrt[n]{a_n} \rightarrow 1$ and $\sum_{n=1}^{N}a_n$ converges.
\end{theorem*}
\begin{proof}
Let $a_{n}=\frac{1}{n^2}$. Then $\ln(\sqrt[n]{a_n})=-2\frac{\ln(n)}{n}\rightarrow 0$. Therefore $\sqrt[n]{a_n}\rightarrow 1$. But $\sum_{n=1}^{N}\frac{1}{n^2}$ converges.
\end{proof}
\begin{theorem*}
If $a_n$ is a sequences of positive real numbers, and if $a(x)$ is a monotonically decreasing function such that for all $n\in \mathbb{N}$, $a(n) = a_n$, then $\sum_{n=1}^{\infty} a_n$ converges if and only if $\int_{1}^{\infty} a(x)dx$ converges.
\end{theorem*}
\begin{proof}
As $a(x)$ is decreasing, for all $n\in \mathbb{N}$ and for all $x\in [n,n+1]$, $a_{n+1} \leq a(x) \leq a_{n}$. Therefore $\int_{n}^{n+1}a_{n+1}dx\leq\int_{n}^{n+1}a(x)dx\leq \int_{n}^{n+1} a_{n}dn\Rightarrow a_{n+1}\leq\int_{n}^{n+1}a(x)dx\leq a_{n}$. But for all $N\in \mathbb{N}$, $\sum_{n=1}^{N}\int_{n}^{n+1}a(x)dx=\int_{n=1}^{N}a(x)dx$, and thus $\sum_{n=1}^{N}a_{n+1}\leq\int_{n=1}^{N}a(x)dx\leq\sum_{n=1}^{N}a_n$. Let $S_{N}=\sum_{n=1}^{N}a_{n}$. Then $S_{N+1}-a_{1}\leq\int_{n=1}^{N}a(x)dx\leq S_{N}$. If $\int_{n=1}^{N}a(x)dx$ converges, say to $I$, then $S_{N+1} \leq I+a_{1}$. But $S_{N+1}$ is a monotonically increasing sequence that is bounded, and therefore converges. That is, if $\int_{n=1}^{N}a(x)dx$ converges, then $\sum_{n=1}^{N}a_n$ converges. If $\sum_{n=1}^{N} a_n$ converges, say to $S$, then $\int_{n=1}^{N}a(x)dx \leq S$. But $a(x)$ is monotonically decreasing and positive, and thus $I_{N} = \int_{n=1}^{N}a(x)dx$ is a bounded monotonically increasing sequence, and therefore converges. That is, if $\sum_{n=1}^{N}a_n$ converges, then $\int_{1}^{N}a(x)dx$ converges. 
\end{proof}
\begin{theorem*}
If $a_n$ is a positive, monotonically decreasing, and if $a_n \rightarrow 0$, then $\sum_{n=1}^{\infty} (-1)^{n}a_n$ converges.
\end{theorem*}
\begin{proof}
Let $S_N = \sum_{n=1}^{N}(-1)^n a_n$. If $N$ is even, then $S_N = \sum_{n=1}^{N/2}(a_{2n}-a_{2n-1})$. But $a_n$ is decreasing monotonically, so $a_{N+2} - a_{N+1} \leq 0$. That is, if $N$ is even then $S_{N} \geq S_{N+2}$. So $S_{2k}$ is a monotonically decreasing subsequence of $S_{N}$. Moreover, $S_{2k} \geq a_{2} - a_{1}$. Thus, $S_{2k}$ converges, say to $S_1$. If $N$ is odd, then $S_{N} = -a_{N}+\sum_{n=1}^{(N-1)/2}(a_{2n} - a_{2n-1})$. But then $S_{N+2} = -(a_{N+2}-a_{N+1})+S_{N} \geq S_{N}$. That is, $S_{2k-1}$ is a monotonically increasing subsequence. Moreover, $S_{2k-1} \leq a_{2}$. So, $S_{2k-1}$ converges, say to $S_2$. Let $\epsilon>0$ be given. As $a_n \rightarrow 0$, there is an $N_{0}\in \mathbb{N}$ such that for all $n \geq N_{0}$, $a_{n} < \frac{\epsilon}{2}$. There is also an $N_{1}$ such that for $n>N_{1}$, $|S_1 - S_{2n}|<\frac{\varepsilon}{4}$. Finally there is an $N_2$ such that for $n>N_{2}$, $|S_2 - S_{2n-1}|<\frac{\varepsilon}{4}$. Let $N = \max\{N_0,N_1,N_2\}$. Then for $n>N$, we have $|S_1 - S_2| \leq |S_1 - S_{2m}|+|S_2 - S_{2m-1}| + |S_{2n} - S_{2n-1}|< \frac{\varepsilon}{4}+\frac{\varepsilon}{4} + |a_{2n}|<\epsilon$. But $S_1$ and $S_2$ are real numbers, and $\varepsilon$ is arbitrary, so $S_1 = S_2$. Therefore $S_N \rightarrow S_1$.
\end{proof}
\begin{theorem*}
There exists positive sequences $a_n$ such that $a_n \rightarrow 0$ and $\sum_{n=1}^{N}(-1)^n a_n$ diverges.
\end{theorem*}
\begin{proof}
For let $a_{n}=\frac{1}{n}$ for odd $n$ and $a_{n}=\frac{1}{n^{2}}$ for even $n$ and let $S_N = \sum_{n=1}^{N}(-1)^{n}a_n$. Then $S_{2N} = \sum_{n=1}^{N} (\frac{1}{n^2} - \frac{1}{n}) = \sum_{n=1}^{N}\frac{1}{n^2} - \sum_{n=1}^{N}\frac{1}{n}$. But $\sum_{n=1}^{N} \frac{1}{n^2}$ converges and $\sum_{n=1}^{N}\frac{1}{n}$ diverges, and therefore $S_{2N}$ diverges. But if $S_{2N}$ diverges, then $S_{N}$ diverges.
\end{proof}
\begin{theorem*}
If $a_n$ is a sequence of real numbers and $\sum_{n=1}^{N}|a_n|$ converges, then $\sum_{n=1}^{N}a_n$ converges.
\end{theorem*}
\begin{proof}
So let $T_{N} = \sum_{n=1}^{N}|a_n|$ and let $S_{N} = \sum_{n=1}^{N}a_{n}$. As $T_{N}$ converges, it is a Cauchy Sequence. That is, for all $\varepsilon>0$ there is an $N_0 \in \mathbb{N}$ such that for all $n,m>N_{0}$, $|T_{n} - T_{m}|<\varepsilon$. But then for $n,m>N_{0}$, $|S_{n} - S_{m}| = |\sum_{n=m+1}^{n}a_n| \leq \sum_{n=m+1}^{n}|a_n| = |T_{n} - T_{m}| <\varepsilon$. That is $S_{N}$ forms a Cauchy sequence. But Cauchy sequences converge. Therefore $S_{N}$ converges.
\end{proof}
\subsection{Complex Variables}
\begin{theorem}
If $x\in\mathbb{C}$, then $e^{ix}=\cos(x)+i\sin(x)$
\end{theorem}
\begin{proof}
For $e^{ix}$ is the solution to $y'=iy$, $y(0)=1$. But $\frac{d}{dx}(\cos(x)+i\sin(x))i(\cos(x)+i\sin(x))$. Moreover, $\cos(0)+i\sin(0)=1$. From the uniqueness of solutions, $e^{ix}=\cos(x)+i\sin(x)$. 
\end{proof}
\begin{theorem}
If $x\in \mathbb{C}$, then $\cos(x)=\frac{1}{2}(e^{ix}+e^{-ix})$
\end{theorem}
\begin{proof}
For $\cos(x)=\cos(-x)$ and $\sin(x)=-\sin(-x)$. F rom the previous theorem $e^{ix}+e^{-ix} = 2\cos(x)$.
\end{proof}
\begin{definition}
The hyperbolic cosine of $x\in \mathbb{R}$ is $\cosh(x)=\cos(ix)$.
\end{definition}
\begin{theorem}
If $x\in\mathbb{R}$, then $\cosh(x)=\frac{e^{ix}+e^{-ix}}{2}$.
\end{theorem}
\begin{proof}
Apply the previous theorem to $ix$.
\end{proof}
\begin{theorem}
If $x\in\mathbb{C}$, then $\sin(x)=\frac{1}{2i}(e^{i\theta}-e^{i\theta})$
\end{theorem}
\begin{proof}
For $e^{ix} = \cos(x)+i\sin(x)$, and thus $e^{ix}-e^{-ix}=2i\sin(x)$
\end{proof}
\begin{definition}
The hyperbolic sine of $x\in\mathbb{R}$ is $\sinh(x)=-i\sin(ix)$
\end{definition}
\begin{theorem}
If $x\in\mathbb{R}$, then $\sinh(x)=\frac{e^{x}-e^{-x}}{2}$
\end{theorem}
\begin{proof}
Apply the previous theorem to $ix$ and multiply by $-i$.
\end{proof}
\begin{definition}
A complex function that is differentiable at a point $z_{0}\in\mathbb{C}$ is a function $f:\mathbb{C}\rightarrow\mathbb{C}$ such that $\underset{z\rightarrow z_{0}}{\lim}\frac{f(z)-f(z_{0})}{z-z_{0}}$
\end{definition}
\begin{definition}
A differentiable complex function is a function $f:\mathbb{C}\rightarrow\mathbb{C}$ that is differentiable for all $z\in\mathbb{C}$.
\end{definition}
\begin{theorem}[The Cauchy-Riemann Theorem]
A function $f:\mathbb{C}\rightarrow\mathbb{C}$, $f(z)=u(x,y)+iv(x,y)$, where $u,v:\mathbb{R}^{2}\rightarrow \mathbb{R}$, is differentiable if and only if $\frac{\partial u}{\partial x}=\frac{\partial v}{\partial y}$ and $\frac{\partial u}{\partial y} = 0\frac{\partial v}{\partial x}$.
\end{theorem}
\begin{theorem}
If $z\in\mathbb{R}$, then there is a unique $r>0$ and $\theta\in[0,2\pi)$ such that $z=re^{i\theta}$.
\end{theorem}
\begin{theorem}[Cauchy's Integral Theorem]
If $\mathcal{U}\subset\mathbb{C}$ is simply connected, if $f:\mathcal{U}\rightarrow \mathbb{C}$ is differentiable, and if $\gamma:I\rightarrow \mathcal{U}$ is a closed path of finite measure (Length), then $\oint_{\gamma}f(z)dz = 0$.
\end{theorem}
\begin{theorem}[Cauchy's Integral Formula]
If $\mathcal{U}\subset\mathbb{C}$ is open, $z_{0}\in\mathcal{U}$, and if $B_{r}(z_{0})\subset\mathcal{U}$, then for all $a\in B_{r}(z_{0})$, $\oint_{\partial B_{r}(z_{0})}\frac{f(z)}{z-a}dz = f(a)$
\end{theorem}
\subsection{Matrices}
\begin{definition}
The transpose of a matrix $A = (a_{ij})$ is $A^{T} = (a_{ji})$.
\end{definition}
\begin{definition}
The complex transpose of a matrix $A = (a_{ij})$ is $A^{\dagger} = (\overline{a_{ji}})$.
\end{definition}
\begin{definition}
An orthogonal matrix is a matrix $A$ such that $A^T = A^{-1}$.
\end{definition}
\begin{definition}
A Hermitian Matrix is a matrix $A$ such that $A^{\dagger} = A$.
\end{definition}
\begin{definition}
A unitary matrix is a matrix $A$ such that $A^{\dagger} = A^{-1}$.
\end{definition}
\begin{definition}
A singular matrix is a matrix with no inverse.
\end{definition}
\begin{definition}
The Commutator of two matrices $A$ and $B$ is $[A,B] = AB - BA$.
\end{definition}
\begin{definition}
A normal matrix is a matrix $A$ such that $[A,A^{\dagger}] = 0$.
\end{definition}
\begin{definition}
The Levi-Civita symbol is defined as:
\begin{equation*}
    \varepsilon_{ijk} = \begin{cases} 1, & ijk\textrm{ is an even permutation} \\ 0, & i=j\textrm{ OR }i=k\textrm{ OR }j=k\\ -1, & ijk\textrm{ is an odd permutation}\end{cases}
\end{equation*}
\end{definition}
\begin{definition}
The Matrix Representation of $a+ib$ is the matrix $\begin{bmatrix}a & -b \\ b & a \end{bmatrix}$
\end{definition}
\begin{theorem}
$\mathbb{C}$, with it's usual arithmetic, is homomorphic to $\mathbb{R}^{2\times 2}$, with it's usual arithmetic.
\end{theorem}
\begin{proof}
For let $f:\mathbb{C}\rightarrow\mathbb{R}^{2\times 2}$ be defined by $f(z) = f(a+ib) = \begin{bmatrix}a & -b\\b & a\end{bmatrix}$
\begin{align*}
    f(z+w)&=\begin{bmatrix}a+b & -(b+d)\\b+d& a+c\end{bmatrix}=\begin{bmatrix}a&-b\\b&a\end{bmatrix}+\begin{bmatrix}c&-d\\d&c\end{bmatrix}=f(z)+f(w)\\
    f(z\cdot w)&=\begin{bmatrix}ac-bd&-(ad+bc)\\ad+bc&ac-bd\end{bmatrix}=\begin{bmatrix}a&-b\\b&a\end{bmatrix}\begin{bmatrix}c&-d\\d&c\end{bmatrix}=f(z)\cdot f(w)
\end{align*}
\end{proof}
\begin{definition}
The $ij$ minor of an $n\times m$ matrix $A$ is the matrix $M_{ij}=\{(k,\ell,a_{k\ell}),k\ne i,j\ne\ell\}$. That is, it is the matrix formed by removing the $i^{th}$ row and $j^{th}$ column from $A$.
\end{definition}
\begin{definition}
The cofactor matrix of an $n\times n$ matrix $A$ is the matrix $C=(C_{ij})$, where $C_{ij}=(-1)^{i+j}M_{ij}$.
\end{definition}
\begin{theorem}
If $A$ is an $n\times n$ matrix, and $\det(A) \ne 0$, then $A^{-1}= \frac{1}{\det(A)}C^{T}$.
\end{theorem}
\subsection{Vectors}
\begin{theorem}
If $\mathbf{A},\mathbf{B},\mathbf{C}\in \mathbb{R}^{3}$, then $\langle \mathbf{A},\mathbf{B}\times \mathbf{C}\rangle = \langle \mathbf{B}, \mathbf{C}\times \mathbf{A}\rangle$.
\end{theorem}
\begin{proof}
For $\mathbf{B}\times \mathbf{C} = (B_{y}C_{z} - B_{z}C_{y},B_{z}C_{x}-B_{x}C_{z},B_{x}C_{y} - B_{y}C_{x})$, and thus: \begin{align*}
    \langle \mathbf{A},\mathbf{B}\times \mathbf{C}\rangle &= A_{x}(B_{y}C_{z} - B_{z}C_{y}) + A_{y}(B_{z}C_{x} - B_{x}C_{z}) + A_{z}(B_{x}C_{y}-B_{y}C_{x})\\
    &= A_{x}B_{y}C_{z} - A_{x}B_{z}C_{y} + A_{y}B_{z}C_{x} - A_{y}B_{x}C_{z} + A_{z}B_{x}C_{y} - A_{z}B_{y}C_{x}\\
    &= B_{x}(A_{z}C_{y} - A_{y}C_{z}) + B_{y}(A_{x}C_{z} - A_{z}C_{x}) + B_{z}(A_{y}C_{x} - A_{x}C_{y})\\
    &= B_{x}(C_{y}A_{z} - C_{z}A_{y}) + B_{y}(C_{z}A_{x} - C_{x}A_{z}) + B_{z}(C_{x}A_{y} - C_{y}A_{x})\\
    &= \langle \mathbf{B}, (C_{y}A_{z} - C_{z}A_{y}, C_{z}A_{x} - C_{x}A_{z}, C_{x}A_{y} - C_{y}A_{x})\rangle\\
    &= \langle \mathbf{B},\mathbf{C}\times \mathbf{A}\rangle
\end{align*}
\end{proof}
\begin{theorem}
If $\mathbf{A},\mathbf{B},\mathbf{C}\in \mathbb{R}^3$, then $ \mathbf{A}\times(\mathbf{B}\times \mathbf{C}) = \langle \mathbf{A},\mathbf{C}\rangle \mathbf{B} - \langle \mathbf{A},\mathbf{B}\rangle \mathbf{C}$
\end{theorem}
\begin{proof}
For $\mathbf{B}\times \mathbf{C} = (B_{y}C_{z} - B_{z}C_{y},B_{z}C_{x}-B_{x}C_{z},B_{x}C_{y} - B_{y}C_{x})$, and thus:
\begin{align*}
    \mathbf{A}\times(\mathbf{B}\times\mathbf{C})\phantom{+}&=\phantom{+-}\mathbf{A}\times(B_{y}C_{z}-B_{z}C_{y},B_{z}C_{x}-B_{x}C_{z},B_{x}C_{y}-B_{y}C_{x})\\
    &=\phantom{+-}\big(A_{y}(B_{x}C_{y}-B_{y}C_{x})-A_{z}(B_{z}C_{x}-B_{x}C_{z})\big)\hat{\mathbf{x}}\\
    &\phantom{=-}+\big(A_{z}(B_{y}C_{z}-B_{z}C_{y})-A_{x}(B_{x}C_{y}-B_{y}C_{x})\big)\hat{\mathbf{y}}\\
    &\phantom{=-}+\big(A_{x}(B_{z}C_{x}-B_{x}C_{z})-A_{y}(B_{y}C_{z}-B_{z}C_{y})\big)\hat{\mathbf{z}}\\
    &=\phantom{+-}\big(B_{x}(A_{x}C_{x}+A_{y}C_{y}+A_{z}C_{z})-C_{x}(A_{x}B_{x}+A_{y}B_{y}+A_{z}B_{z})\big)\hat{\mathbf{x}}\\
    &\phantom{=-}+\big(B_{y}(A_{x}C_{x}+A_{y}C_{y}+A_{z}C_{z})-C_{y}(A_{x}B_{x}+A_{y}B_{y}+A_{z}B_{z})\big)\hat{\mathbf{y}}\\
    &\phantom{=-}+\big(B_{z}(A_{x}C_{x}+A_{y}C_{y}+A_{z}C_{z})-C_{z}(A_{x}B_{x}+A_{y}B_{y}+A_{z}B_{z})\big)\hat{\mathbf{z}}\\
    &=\phantom{+-}(A_{x}C_{x}+A_{y}C_{y}+A_{z}C+{z})(B_{x},B_{y},B_{z})\\
    &\phantom{=+}-(A_{x}B_{x}+A_{y}B_{y}+A_{z}B_{z})(C_{x},C_{y},C_{z})\\
    &=\phantom{+-}\langle\mathbf{A},\mathbf{C}\rangle\mathbf{B}-\langle \mathbf{A},\mathbf{C}\rangle\mathbf{C}
\end{align*}
\end{proof}
\begin{theorem}[Divergence Theorem]
If $\Omega$ is a closed bounded subset of $\mathbb{R}^{n}$ with a smooth boundary $\partial \Omega$, and if $\mathbf{A}$ is a smooth vector field, then $\iiint_{\Omega} (\nabla \cdot \mathbf{A})d\tau = \oiint_{\partial \Omega}\mathbf{A}\cdot \boldsymbol{d\sigma}$ 
\end{theorem}
\begin{theorem}[Stokes' Theorem]
If $\Sigma$ is a compact, simply connected subset of $\mathbb{R}^{3}$ bounded by a smooth Jordan Curve $\partial \Sigma$, and if $\mathbf{A}$ is a smooth vector field, then $\iint_{\Sigma}(\nabla\times \mathbf{A})\cdot \boldsymbol{d\sigma} = \oint_{\partial \Sigma}\mathbf{A}\cdot \boldsymbol{d\ell}$
\end{theorem}
\begin{theorem}
If $\phi:\mathbb{R}^{3}\rightarrow \mathbb{R}$ is a smooth function, then $\nabla \times \nabla(\phi) = \boldsymbol{0}$.
\end{theorem}
\begin{proof}
For $\nabla(\phi) = \frac{\partial \phi}{\partial x}\hat{\mathbf{x}}+\frac{\partial \phi}{\partial y}\hat{\mathbf{y}}+\frac{\partial \phi}{\partial z}\hat{\mathbf{z}}$, and therefore:
\begin{equation*}
    \nabla \times \nabla(\phi) = \bigg(\frac{\partial^{2} \phi}{\partial y \partial z} - \frac{\partial^{2}\phi}{\partial z \partial y}\bigg)\hat{\mathbf{x}}+\bigg(\frac{\partial^{2} \phi}{\partial z \partial x} - \frac{\partial^{2}\phi}{\partial x \partial z}\bigg)\hat{\mathbf{y}}+\bigg(\frac{\partial^{2} \phi}{\partial x \partial y} - \frac{\partial^{2}\phi}{\partial y \partial x}\bigg)\hat{\mathbf{z}}
\end{equation*}
But, as $\phi$ is smooth, $\frac{\partial^{2}\phi}{\partial x_{i}\partial x_{j}} = \frac{\partial^{2}\phi}{\partial x_{j}\partial x_{i}}$. Therefore, $\nabla \times \nabla(\phi) = \boldsymbol{0}$.
\end{proof}
\begin{theorem}
If $\mathbf{A}:\mathbb{R}^{3}\rightarrow \mathbb{R}^{3}$ is smooth, then $\nabla\cdot(\nabla \times \mathbf{A}) = 0$.
\end{theorem}
\begin{proof}
For $\nabla \times \mathbf{A} = (\frac{\partial A_{z}}{\partial y} - \frac{\partial A_{y}}{\partial z})\hat{\mathbf{x}}+(\frac{\partial A_{x}}{\partial z} - \frac{\partial A_{z}}{\partial x})\hat{\mathbf{y}}+(\frac{\partial A_{y}}{\partial x} - \frac{\partial A_{x}}{\partial y})\hat{\mathbf{z}}$, and thus:
\begin{align*}
    \nabla \cdot (\nabla \times \mathbf{A}) &=\nabla \cdot \bigg(\big(\frac{\partial A_{z}}{\partial y} - \frac{\partial A_{y}}{\partial z}\big)\hat{\mathbf{x}}+\big(\frac{\partial A_{x}}{\partial z} - \frac{\partial A_{z}}{\partial x}\big)\hat{\mathbf{y}}+\big(\frac{\partial A_{y}}{\partial x} - \frac{\partial A_{x}}{\partial y}\big)\hat{\mathbf{z}}\bigg)\\
    &=\bigg(\frac{\partial^{2}A_{z}}{\partial x \partial y} - \frac{\partial^{2}A_{y}}{\partial x \partial z}\bigg)+\bigg(\frac{\partial^{2}A_{x}}{\partial y \partial z} - \frac{\partial^{2}A_{z}}{\partial y \partial x}\bigg) + \bigg(\frac{\partial^{2} A_{y}}{\partial z \partial x} - \frac{\partial^{2} A_{x}}{\partial z \partial y}\bigg)\\
    &= \bigg(\frac{\partial^{2} A_{z}}{\partial x \partial y} - \frac{\partial^{2} A_{z}}{\partial y \partial x}\bigg) + \bigg(\frac{\partial^{2} A_{y}}{\partial z \partial x} - \frac{\partial^{2} A_{y}}{\partial x \partial z}\bigg) + \bigg(\frac{\partial^{2}A_{x}}{\partial y \partial z} - \frac{\partial^{2} A_{x}}{\partial z \partial y}\bigg)
\end{align*}
But $A_{x},A_{y}$, and $A_{z}$ are smooth, and thus $\frac{\partial^{2} A_{k}}{\partial x_{i} \partial x_{j}} = \frac{\partial^{2} A_{k}}{\partial x_{j} \partial x_{i}}$. Therefore, $\nabla \cdot (\nabla \times \mathbf{A}) = \boldsymbol{0}$.
\end{proof}
\end{document}