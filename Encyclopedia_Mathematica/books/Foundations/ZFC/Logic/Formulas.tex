\section{Formulas and Propositions}
    In the set theory we will be working with, known as
    Zermelo-Fraenkel Set Theory (abbreviated as ZFC where C stands for
    \textit{choice}) there are a few words that are left undefined as well as a
    few symbols. This is certainly unavoidable for defining all words would
    ultimately be circular, and thus we try to leave as few words and as few
    symbols possible undefined, and build up everything else from this
    foundation. When we develop set theory our undefined symbol will be
    the symbol $\in$, and our undefined word will be \textit{set}. The symbol
    $\in$ denotes containment. That is, $x\in{y}$ reads as $x$ is \textit{in}
    $y$, or $x$ is \textit{contained} in $y$. It is a
    \textit{\gls{predicate}}\index{Predicate}.
    \begin{fdefinition}{Predicate}{Predicate}
        A \gls{predicate} is a statement that takes in some parameter or
        variable $x$ and returns either \textit{True} or \textit{True}.
    \end{fdefinition}
    We have relied on the word \textit{statement} being already defined, and
    similarly for the words \textit{parameter} or \textit{variable}. For most
    this is not an issue, but for some it may indicate that mathematics may rest
    on some very shakey foundations.
    \par\hfill\par
    From our undefined symbol $\in$, we build new symbols by expressing them in
    terms of a \textit{formula}\index{Formula}, which is simply a finite
    sequence of symbols. Here the word \textit{sequence} is meant to imply that
    the \textit{order} in which we combine these symbols is important, and that
    rearranging said order may create a different inequivalent formula. We build
    formulas by defining a few symbols that stand as placeholders for standard
    words in English. There are five symbols, called
    \textit{\glspl{connective}}\index{Connective (Logic)}, that we use.
    \begin{align*}
        a\land{b}\quad
        &\textrm{True if and only if }a
        \textrm{ is true and }b\textrm{ is true}
        \tag{Conjunction}\\
        a\lor{b}\quad
        &\textrm{True if and only if }a
        \textrm{ is true or }b\textrm{ is true, or both}
        \tag{Disjunction}\\
        \neg{a}\quad
        &\textrm{True if and only if }a\textrm{ is false}
        \tag{Negation}\\
        a\Leftrightarrow{b}\quad
        &a\textrm{ is true if and only if }b\textrm{ is true}
        \tag{Equivalence}\\
        a\Rightarrow{b}\quad
        &a\textrm{ is true implies that }b\textrm{ is true}
        \tag{Implication}
    \end{align*}
    There are other symbols we adopt, such as \textit{does not imply}:
    \begin{equation*}
        a\not\Rightarrow{b}
    \end{equation*}
    But from how we shall define these notions, this new symbol is equivalent to
    a combination of the previous ones:
    \begin{equation*}
        a\not\Rightarrow{b}\Longleftrightarrow
        \neg(a\Rightarrow{b})
        \Longleftrightarrow
        a\land\neg{b}
    \end{equation*}
    There are two more symbols called
    \textit{\glspl{quantifier}}\index{Quantifier}.
    \begin{equation*}
        \forall_{x}\quad\textrm{For all }x
        \quad\quad\quad\quad
        \exists_{x}\quad\textrm{There exists }x
    \end{equation*}
    Quantifiers, together with connectives, the word \textit{set}, and the
    $\in$ symbol are combined to define new terms and new symbols. The rest of
    mathematics rest on trusting ones intuition behind these notions.