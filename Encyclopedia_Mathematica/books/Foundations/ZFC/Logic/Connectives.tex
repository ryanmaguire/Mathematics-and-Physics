\section{Connectives}
    In the set theory we will be working with, known as
    Zermelo-Fraenkel Set Theory (abbreviated as ZFC where C stands for
    \textit{choice}) there are a few words that are left undefined as well as a
    few symbols. This is certainly unavoidable since defining all words would
    ultimately be circular, and thus we try to leave the fewest amount of words
    and symbols undefined as possible, and build up everything else from this
    foundation. Our first undefined symbol is $\in$
    (known as \textit{containment}). This is a type of
    \textit{\gls{predicate}}\index{Predicate}.
    \begin{fdefinition}{Predicate}{Predicate}
        A \gls{predicate} is a statement that takes in some parameter or
        variable $x$ and returns either \textit{True} or \textit{True}.
    \end{fdefinition}
    We have relied on the word \textit{statement} being already defined, and
    similarly for the words \textit{parameter} or \textit{variable}. For most
    this is not an issue, but for some it may indicate that mathematics may rest
    on some shakey foundations.
    \par\hfill\par
    From our undefined symbol $\in$ we build new symbols by expressing them in
    terms of a \textit{formula}\index{Formula}, which is simply a finite
    sequence of symbols. Here the word \textit{sequence} is meant to imply that
    the \textit{order} in which we combine these symbols is important and that
    rearranging said order may create a different inequivalent formula. We build
    formulas by defining a few symbols that stand as placeholders for standard
    words in English. There are five symbols, called
    \textit{\glspl{connective}}\index{Connective (Logic)}, that we use.
    \begin{align*}
        a\land{b}\quad
        &\textrm{True if and only if }a
        \textrm{ is true and }b\textrm{ is true}
        \tag{Conjunction}\\
        a\lor{b}\quad
        &\textrm{True if and only if }a
        \textrm{ is true or }b\textrm{ is true, or both}
        \tag{Disjunction}\\
        \neg{a}\quad
        &\textrm{True if and only if }a\textrm{ is false}
        \tag{Negation}\\
        a\Leftrightarrow{b}\quad
        &a\textrm{ is true if and only if }b\textrm{ is true}
        \tag{Equivalence}\\
        a\Rightarrow{b}\quad
        &a\textrm{ is true implies that }b\textrm{ is true}
        \tag{Implication}
    \end{align*}
    From this we see that we have introduced 6 new words that are undefined but
    require comment. The words are \textit{and, or, if, then, true}, and
    \textit{false}. There are other symbols we could adopt, such as
    \textit{does not imply}:
    \begin{equation*}
        a\not\Rightarrow{b}
    \end{equation*}
    But from how we shall define these notions, this new symbol is equivalent to
    a combination of the previous ones:
    \begin{equation*}
        a\not\Rightarrow{b}\Longleftrightarrow
        \neg(a\Rightarrow{b})
        \Longleftrightarrow
        a\land\neg{b}
    \end{equation*}
    \subsection{Conjunction}
        The conjunction connective is the $\land$ symbol, which denotes the word
        \textit{and}. Given two propositions $P$ and $Q$, $P\land{Q}$ is a true
        statement if and only if both $P$ and $Q$ are true. That is, we
        associate to $\land$ the following truth table:
        \begin{table}[H]
            \centering
            \captionsetup{type=table}
            \begin{tabular}{ccc}
                $P$&$Q$&$P\land{Q}$\\
                \hline
                0&0&0\\
                0&1&0\\
                1&0&0\\
                1&1&1
            \end{tabular}
            \caption{Truth Table for Conjunction}
            \label{tab:Truth_Table_for_Conjunction}
        \end{table}
        There are several \textit{axioms} of conjunctions that are intuitively
        obvious, but must be stated since their use is wide spread.
        \begin{faxiom}{Axioms of Conjunction}{Axioms_of_Conjunction}
            If $P$ and $Q$ are propositions, then the following are true:
            \begin{align}
                P\land{Q}&\Longleftrightarrow{Q}\land{P}
                \tag{Commutativity of Conjunction}\\
                P\land\textrm{True}&\Longleftrightarrow\textrm{P}
                \tag{Identity of Conjunction}
            \end{align}
        \end{faxiom}
    \subsection{Disjunction}
        The disjunction connected is the $\lor$ symbol which represents
        \textit{or}. Given two propositions $P$ and $Q$, $P\lor{Q}$ is true if
        and only if $P$ is true, or $Q$ is true, or both $P$ and $Q$ is true.
        There is an unfortunate ambiguity in English as to whether $P$ or $Q$
        means $P$ is true or $Q$ is true, but not both, or whether it means
        $P$ is true or $Q$ is true, or \textit{both} are true. The convention is
        to adopt the latter definition. That is, $P\lor{Q}$ has the following
        truth table:
        \begin{table}[H]
            \centering
            \captionsetup{type=table}
            \begin{tabular}{ccc}
                $P$&$Q$&$P\lor{Q}$\\
                \hline
                0&0&0\\
                0&1&1\\
                1&0&1\\
                1&1&1
            \end{tabular}
            \caption{Truth Table for Disjunction}
            \label{tab:Truth_Table_for_Disjunction}
        \end{table}
    \subsection{Implication}
        The implication connective is denoted by $P\Rightarrow{Q}$. This is read
        as \textit{if P, then Q} for two given propositions $P$ and $Q$. Here,
        $P$ is called the \textit{hypothesis} and $Q$ is called the
        \textit{conclusion}. There is some ambiguity as to the mean of if-then
        statements. If $P$ is a proposition that is false, and $Q$ is a
        propositions that is true, should $P\Rightarrow{Q}$ be a true statement
        or a false one? The convention is to take this as true. That is, we
        adopt the following truth table.
        \begin{table}[H]
            \centering
            \captionsetup{type=table}
            \begin{tabular}{ccc}
                $P$&$Q$&$P\Rightarrow{Q}$\\
                \hline
                0&0&1\\
                0&1&1\\
                1&0&0\\
                1&1&1
            \end{tabular}
            \caption{Truth Table for Implication}
            \label{tab:Truth_Table_for_Implication}
        \end{table}
        The converse of the statement $P\Rightarrow{Q}$ is the statement
        $Q\Rightarrow{P}$. The validity of the converse is not implied by the
        truth or falsehood of $P\Rightarrow{Q}$, and we need only look at the
        truth table to see this.
        \begin{table}[H]
            \centering
            \captionsetup{type=table}
            \begin{tabular}{cccc}
                $P$&$Q$&$P\Rightarrow{Q}$&$Q\Rightarrow{P}$\\
                \hline
                0&0&1&1\\
                0&1&1&0\\
                1&0&0&1\\
                1&1&1&1
            \end{tabular}
            \caption{Truth Table for the Converse}
            \label{tab:Truth_Table_for_Converse}
        \end{table}
        Examining this, we see that there are scenarios where $P\Rightarrow{Q}$
        is true and $Q\Rightarrow{P}$ is false, and similarly where
        $P\Rightarrow{Q}$ is false and $Q\Rightarrow{P}$ is true. Propositions
        $P$ and $Q$ such that $P\Rightarrow{Q}$ and $Q\Rightarrow{P}$ are called
        \textit{equivalent}, and great deal of mathematics is devoted to the
        search for equivalencies of statements. This is denoted by the
        connective $P\Leftrightarrow{Q}$. We see that this is a redundant
        connective and can be written in terms of implication and conjunction:
        \begin{equation}
            \big(P\Leftrightarrow{Q}\big)\Longleftrightarrow
            \big(P\Rightarrow{Q}\big)\land\big(Q\Rightarrow{P}\big)
        \end{equation}
        To verify this, we examine the truth tables.
        \begin{table}[H]
            \centering
            \captionsetup{type=table}
            \begin{tabular}{cccccc}
                $P$&$Q$&$P\Rightarrow{Q}$&$Q\Rightarrow{P}$
                   &$P\Leftrightarrow{Q}$
                   &$(P\Rightarrow{Q})\land{Q\Rightarrow{P}}$\\
                \hline
                0&0&1&1&1&1\\
                0&1&1&0&0&1\\
                1&0&0&1&0&1\\
                1&1&1&1&1&1
            \end{tabular}
            \caption{Truth Table for Equivalence}
            \label{tab:Truth_Table_for_Equivalence}
        \end{table}
    \subsection{Negation}
        Negation is the first connective we come across that is a \textit{unary}
        operation. That is, it only takes in one proposition rather than two.
        Given a proposition $P$, $\neg{P}$ is true if and only if $P$ if false.
        That is, we have the following truth table.
        \begin{table}[H]
            \centering
            \captionsetup{type=table}
            \begin{tabular}{cc}
                $P$&$\neg{P}$\\
                \hline
                0&1\\
                1&0
            \end{tabular}
            \caption{Truth Table for Negation}
            \label{tab:Truth_Table_for_Negation}
        \end{table}
        The negation connective always us to define the
        \textit{contrapositive}\index{Contrapositive} of the proposition
        $P\Rightarrow{Q}$, which is the new proposition
        $\neg{Q}\Rightarrow\neg{P}$. As it turns out, this is not a new
        proposition at all and is equivalent to $P\Rightarrow{Q}$. To see this,
        note that $P\Rightarrow{Q}$ is only false when $P$ is true, yet $Q$ is
        false. Similarly, $\neg{Q}\Rightarrow\ne{P}$ is only false when
        $\neg{Q}$ is true and $\neg{P}$ is false. But if $\neg{Q}$ is true, then
        $Q$ is false, and if $\neg{P}$ is false, then $P$ is true. Thus
        $\neg{Q}\Rightarrow\neg{P}$ is only false when $P$ is true and $Q$ is
        false. We can further examine this by the truth tables of these
        statements.
        \begin{table}[H]
            \centering
            \captionsetup{type=table}
            \begin{tabular}{cccccc}
                $P$&$Q$&$\neg{P}$&$\neg{Q}$
                    &$P\Rightarrow{Q}$&$\neg{Q}\Rightarrow\neg{P}$\\
                \hline
                0&0&1&1&1&1\\
                0&1&1&0&1&1\\
                1&0&0&1&0&0\\
                1&1&0&0&1&1
            \end{tabular}
            \caption{Truth Table for the Contrapositive}
            \label{tab:Truth_Table_for_Contrapositive}
        \end{table}
        \begin{example}
            Suppose $a$ and $b$ are real variables and $P$ is the proposition
            $a<1/2$ and $b<1/2$, and let $Q$ be the proposition $a+b<1$.
            What is the contrapositive of $P\Rightarrow{Q}$? This would be
            $\neg{Q}\Rightarrow{P}$, and $\neg{Q}$ is the negation of $Q$, which
            reads $a+b\geq{1}$. Similarly, $\neg{P}$ is the statement
            $a\geq{1}/2$ or $b\geq{1}/2$. Thus, the contrapositive says that if
            $a+b\geq{1}$, then either $a\geq{1}/2$ or $b\geq{1}/2$ (or both).
            While the contrapositive of a statement is always equivalent to the
            original statement, the converse need not be. Indeed, this statement
            is true (once one knows the order structure or real numbers), but
            the converse is not. The converse states that if $a+b<1$, then
            $a<1/2$ and $b<1/2$, but letting $a=2$ and $b=\minus{3}$
            contradicts this claim.
        \end{example}
\subsection{Quantifiers}
    There are two more symbols called
    \textit{\glspl{quantifier}}\index{Quantifier}.
    \begin{equation*}
        \forall_{x}\quad\textrm{For all }x
        \quad\quad\quad\quad
        \exists_{x}\quad\textrm{There exists }x
    \end{equation*}
    Quantifiers, together with connectives, the word \textit{set}, and the
    $\in$ symbol are combined to define new terms and new symbols. The rest
    of mathematics rests on trusting ones intuition behind these notions.
    \begin{example}
        Let $\mathbb{R}$ denote the set of real numbers. The symbols
        $\forall_{R\in\mathbb{R}}(n^{2}\geq{0})$ can then be read in English
        as \textit{For all real numbers x, the square of x is non-negative},
        which is indeed a true statement. We can combine quantifiers to
        create more complicated statements, such as:
        \begin{equation}
            \forall_{x\in\mathbb{R}}(x\ne{0})\exists_{y\in\mathbb{R}}(xy=1)
        \end{equation}
        This reads that for all non-zero real numbers $x$, there exists a
        real numbers $y$ such that the product $xy$ is equal to 1. This is
        also a true statement.
    \end{example}
    \begin{example}
        The order of quantifiers is very important and often can not be
        interchanged. Considering the previous example, if we switch the
        order of the quantifiers we get:
        \begin{equation}
            \exists_{y\in\mathbb{R}}(xy=1)\forall_{x\in\mathbb{R}}(x\ne{0})
        \end{equation}
        This states that there exists a real number $y$ such that, for every
        non-zero real number $x$, it is true that $xy=1$. But this is
        certainly not true because if $x=1$ and $z=\minus{1}$, we obtain
        $(1)y=1$ and $(\minus{1})y=1$, and from this we conclude that
        $\minus{1}=1$, which is false. Hence, the order of the quantifiers
        is important.
    \end{example}
    \begin{example}
        Quantifiers can be combined with connectives to make longer and more
        complicated statements. For example, suppose $P$ is the proposition
        \textit{true if n is an even integer, false otherwise}. Furthermore,
        let $Q$ be the proposition \textit{true if n is a square integer},
        \textit{false otherwise}. Lastly, let $r$ be the proposition
        \textit{true if n is divisible by 4, false otherwise}.
        Consider then the following statement:
        \begin{equation}
            \forall_{n\in\mathbb{Z}}(p(n)\land{q(n)}\Rightarrow{r}(n))
        \end{equation}
        This reads in English as \textit{for all integers n, if n is an}
        \textit{integer, and if n is a square, then n is divisible by 4}.
    \end{example}
    \subsection{Negating Quantifiers}
        The negation of the statement \textit{for all x, P(x) is true} implies
        this is false. Thus there must exist one $x$ such that $P(x)$ is false,
        and from this we see that negating the $\forall$ quantifier produces the
        $\exists$ quantifier.
        \begin{example}
            Let $P$ be the proposition \textit{true if $x^{2}=2$} and consider
            the following statement:
            \begin{equation}
                \exists_{x\in\mathbb{Q}}\big(P(x)\big)
            \end{equation}
            This reads in plain English as the statement \textit{there exists a}
            \textit{rational number x whose square is equal to 2}. This has been
            known to be false since the ancient Greeks, and thus it's negation
            is true. We can write the negation as follows:
            \begin{equation}
                \neg\Big(\exists_{x\in\mathbb{Q}}\big(p(x)\big)\Big)
                \Longleftrightarrow\forall_{x\in\mathbb{Q}}\big(\neg{P}(x)\big)
            \end{equation}
            This now says that for all rational numbers $x$, the square of $x$
            is not equal to 2.
        \end{example}