\section{Boolean Algebras}
    \label{sec:Boolean_Algebra}%
    We now attempt to make set theory more algebraic. We wish to model
    as an object the triple $(\mathcal{P}(X),\cup,\cap)$, where
    $\mathcal{P}(X)$ is the \gls{power set} of some set, and $\cup$ and
    $\cap$ and union and intersection, respectively. These can be seen as
    binary operations on $\mathcal{P}(X)$. We take a few of the properties
    of this structure and state them as the definition for our new object:
    \textit{Boolean Algebras}.
    \begin{fdefinition}{Complement in a Boolean Algebra}
                       {Complement}
        A complement of a \gls{set} $A$ with respect to two
        \glspl{binary operation} $*$ and $\circ$ is an element
        $a^{\minus{1}}\in{A}$ such that:
        \begin{equation*}
            a*a^{\minus{1}}=a^{\minus{1}}*a=e_{\circ}
            \quad\quad
            a\circ{a}^{\minus{1}}=a^{\minus{1}}\circ{a}=e_{*}
        \end{equation*}
        Where $e_{\circ}$ and $e_{*}$ are the \glspl{unital element} of
        $\circ$ and $*$, respectively.
    \end{fdefinition}
    \begin{fdefinition}{Boolean Algebras}{Boolean_Algebra}
        A Boolean algebra is a set $A$ with two
        \glslink{commutative operation}{commutative} \glspl{binary operation}
        $\circ$ and $*$ on $A$ with \glspl{unital element} $e_{*}$ and
        $e_{\circ}$, respectively, such that:
        \begin{itemize}
            \item[1.)]  $\circ$ \glslink{distributive operation}{distributes}
                        over $*$ and $*$ distributes over $\circ$.
            \item[2.)]  For all $a\in{A}$ there is a complement of $a$.
        \end{itemize}
    \end{fdefinition}
    The second property is known as the complement property and it is very
    different from the notion of inverses. An inverse of an element $a$ with
    respective to an operation $\cdot$ is such that $a\cdot{b}$ is a unital
    element with respect to the operation $\cdot$. A complement produces a
    unital element with respect the the \textit{other} operation. That is,
    $a*a^{\minus{1}}$ is a unital element of $\circ$, and not $*$. Similarly,
    $a\circ{a}^{\minus{1}}$ is a unital element of $*$ and not $\circ$.
    \begin{theorem}
        If $(A,\circ,*)$ is a Boolean algebra and if $b\in{X}$ is a
        unital element of $\circ$, then $b=e_{\circ}$.
    \end{theorem}
    \begin{proof}
        For unital elements are unique
        (Thm.~\ref{thm:Unital_Elements_are_Unique}), and therefore
        $b=e_{\circ}$.
    \end{proof}
    \begin{theorem}
        If $(X,\circ,*)$ is a Boolean algebra and if $b\in{X}$ is a
        unital element of $*$, then $b=e_{*}$.
    \end{theorem}
    \begin{proof}
        For unital elements are unique
        (Thm.~\ref{thm:Unital_Elements_are_Unique}), and therefore $b=e_{*}$.
    \end{proof}
    \begin{theorem}
        \label{thm:Bool_Alg_Boundary_of_Circ}%
        If $(X,\circ,*)$ is a Boolean algebra, if $e_{*}$ is the unital
        element of $*$, and if $a\in{X}$, then $a\circ{e}_{*}=e_{*}$.
    \end{theorem}
    \begin{proof}
        For if $a\in{X}$ then there is an $a^{\minus{1}}\in{X}$ such that
        $a\circ{a}^{\minus{1}}=e_{*}$ (Def.~\ref{def:Boolean_Algebra}).
        But then:
        \par\vspace{-2.5ex}
        \begin{minipage}[t]{0.51\textwidth}
            \centering
            \begin{align}
                e_{*}&=a\circ{a}^{\minus{1}}
                \tag{Complement}\\
                &=a\circ(a^{\minus{1}}*e_{*})
                \tag{Identity}\\
                &=(a\circ{a}^{\minus{1}})*(a\circ{e}_{*})
                \tag{Distributivity}
            \end{align}
        \end{minipage}
        \hfill
        \begin{minipage}[t]{0.47\textwidth}
            \centering
            \begin{align}
                &=e_{*}*(a\circ{e}_{*})
                \tag{Complement}\\
                &=a\circ{e}_{*}
                \tag{Identity}
            \end{align}
        \end{minipage}
        \par\vspace{2.5ex}
        And therefore $e_{*}=a\circ{e}_{*}$.
    \end{proof}
    This theorem is equivalent to the notion that a Boolean algebra is a
    bounded lattice\index{Bounded Lattice} and the $e_{*}$ is a boundary. The
    theorem holds for $\circ$ as well.
    \begin{theorem}
        \label{thm:Bool_Alg_Boundary_of_Star}%
        If $(X,\circ,*)$ is a Boolean algebra, if $e_{*}$ is the unital
        element of $*$, and if $a\in{X}$, then $a*{e}_{\circ}=e_{\circ}$.
    \end{theorem}
    \begin{proof}
        For if $a\in{X}$ then there is an $a^{\minus{1}}\in{X}$ such that
        $a*{a}^{\minus{1}}=e_{\circ}$ (Def.~\ref{def:Boolean_Algebra}).
        But then:
        \par\vspace{-2.5ex}
        \begin{minipage}[t]{0.51\textwidth}
            \centering
            \begin{align}
                e_{\circ}&=a*{a}^{\minus{1}}
                \tag{Complement}\\
                &=a*(a^{\minus{1}}\circ{e}_{\circ})
                \tag{Identity}\\
                &=(a*a^{\minus{1}})\circ(a*e_{*})
                \tag{Distributivity}
            \end{align}
        \end{minipage}
        \hfill
        \begin{minipage}[t]{0.47\textwidth}
            \centering
            \begin{align}
                &=e_{\circ}\circ(a*e_{\circ})
                \tag{Complement}\\
                &=a*e_{\circ}
                \tag{Identity}
            \end{align}
        \end{minipage}
        \par\vspace{2.5ex}
        And therefore $e_{\circ}=a*e_{\circ}$.
    \end{proof}
    \begin{theorem}
        If $(A,\circ,*)$ is a Boolean algebra, if $e_{\circ}$ and $e_{*}$
        are the unital elements of $\circ$ and $*$, respectively, then
        $e_{\circ}$ is the complement of $e_{*}$ and $e_{*}$ is the
        complement of $e_{\circ}$.
    \end{theorem}
    \begin{proof}
        From identity:
        \par\vspace{-2.5ex}
        \begin{subequations}
            \begin{minipage}[b]{0.49\textwidth}
                \centering
                \begin{equation}
                    e_{\circ}\circ{e}_{*}=e_{*}\circ{e}_{\circ}=e_{*}
                    \tag{Identity}
                \end{equation}
            \end{minipage}
            \hfill
            \begin{minipage}[b]{0.49\textwidth}
                \centering
                \begin{equation}
                    e_{\circ}*{e}_{*}=e_{*}*e_{\circ}=e_{\circ}
                    \tag{Identity}
                \end{equation}
            \end{minipage}
        \end{subequations}
        \par\vspace{2.5ex}
        Thus, $e_{*}$ is a complement of $e_{\circ}$ and $e_{\circ}$ is a
        complement of $e_{*}$ (Def.~\ref{def:Complement}).
    \end{proof}
    Every element of a Boolean algebra is idempotent with respect to both
    operations.
    \begin{theorem}
        \label{thm:Bool_Alg_Idempotent_of_Star}%
        If $(A,\circ,*)$ is a Boolean algebra and if $a\in{A}$, then $a*a=a$.
    \end{theorem}
    \begin{proof}
        For:
        \par\vspace{-2.5ex}
        \begin{subequations}
            \begin{minipage}[t]{0.49\textwidth}
                \centering
                \begin{align}
                    a&=a*e_{*}
                    \tag{Identity}\\
                    &=a*(a\circ{a}^{\minus{1}})
                    \tag{Complement}\\
                    &=(a*a)\circ(a*a^{\minus{1}})
                    \tag{Distributivity}
                \end{align}
            \end{minipage}
            \hfill
            \begin{minipage}[t]{0.49\textwidth}
                \centering
                \begin{align}
                    &=(a*a)\circ{e}_{\circ}
                    \tag{Complement}\\
                    &=a*a\tag{Identity}
                \end{align}
            \end{minipage}
        \end{subequations}
        \par\vspace{2.5ex}
        And therefore $a=a*a$.
    \end{proof}
    \begin{theorem}
        \label{thm:Bool_Alg_Idempotent_of_Circ}%
        If $(A,\circ,*)$ is a Boolean algebra and if $a\in{A}$, then
        $a\circ{a}=a$.
    \end{theorem}
    \begin{proof}
        For:
        \par\vspace{-2.5ex}
        \begin{subequations}
            \begin{minipage}[t]{0.49\textwidth}
                \centering
                \begin{align}
                    a&=a\circ{e}_{\circ}
                    \tag{Identity}\\
                    &=a\circ(a*a^{\minus{1}})
                    \tag{Complement}\\
                    &=(a\circ{a})*(a\circ{a}^{\minus{1}})
                    \tag{Distributivity}
                \end{align}
            \end{minipage}
            \hfill
            \begin{minipage}[t]{0.49\textwidth}
                \centering
                \begin{align}
                    &=(a\circ{a})*{e}_{*}
                    \tag{Complement}\\
                    &=a\circ{a}\tag{Identity}
                \end{align}
            \end{minipage}
        \end{subequations}
        \par\vspace{2.5ex}
        And therefore $a=a\circ{a}$.
    \end{proof}
    \begin{theorem}
        \label{thm:Bool_Alg_aob_equal_a_acb_equal_a_implies_a_equal_b}%
        If $(A,\circ,*)$ is a Boolean algebra, if $a,b\in{A}$, if $a\circ{b}=a$,
        and if $a*b=a$, then $b=a$.
    \end{theorem}
    \begin{proof}
        For:
        \par
        \begin{minipage}[b]{0.49\textwidth}
            \centering
            \begin{align}
                b&=b*e_{*}\tag{Identity}\\
                 &=b*(a\circ a^{-1})\tag{Complement}\\
                 &=(b*a)\circ(b* a^{-1})\tag{Distributivity}\\
                 &=a\circ (b* a^{-1})\tag{Hypothesis}
            \end{align}
        \end{minipage}
        \hfill
        \begin{minipage}[b]{0.49\textwidth}
            \centering
            \begin{align}
                &=(a\circ b)*(a\circ a^{-1})\tag{Distributivity}\\
                &=(a\circ{b})*e_{*}\tag{Complement}\\
                &=a\circ{b}\tag{Identity}\\
                &=a\tag{Hypothesis}
            \end{align}
        \end{minipage}
        \par\vspace{2.5ex}
        And therefore $a=b$.
    \end{proof}
    \begin{theorem}
        If $(A,\circ,*)$ is a Boolean algebra, if $a\in{A}$ is such that
        $a=a^{\minus{1}}$, then $a=e_{\circ}=e_{*}$.
    \end{theorem}
    \begin{proof}
        For let $a\in{A}$ and let $a=a^{\minus{1}}$. Then by
        Thm.~\ref{thm:Bool_Alg_Idempotent_of_Star}:
        \begin{equation}
            a=a*a=a*a^{-1}=e_{\circ}
        \end{equation}
        Similarly, $a=e_{*}$.
    \end{proof}
    \begin{theorem}
        If $(A,\circ,*)$ is a Boolean algebra, if $a\in{A}$, and if $b,c\in{A}$
        are complements of $a$, then $b=c$.
    \end{theorem}
    \begin{proof}
        For:
        \par
        \begin{minipage}[t]{0.49\textwidth}
            \centering
            \begin{align}
                b&=b*e
                \tag{Identity}\\
                &=b*(a\circ{c})
                \tag{Complement}\\
                &=(b*a)\circ(b*c)
                \tag{Distributivity}\\
                &=e_{\circ}\circ(b*c)
                \tag{Complement}\\
                &=(c*a)\circ(b*c)
                \tag{Complement}
            \end{align}
        \end{minipage}
        \hfill
        \begin{minipage}[t]{0.49\textwidth}
            \centering
            \begin{align}
                &=(c*a)\circ(c*b)
                \tag{Commutativity}\\
                &=c\circ(a*b)
                \tag{Distributivity}\\
                &=c\circ{e}_{\circ}
                \tag{Complement}\\
                &=c
                \tag{Identity}
            \end{align}
        \end{minipage}
        \par\vspace{2.5ex}
        Therefore, $b=c$.
    \end{proof}
    \begin{theorem}
        If $(A,\circ,*)$ is a Boolean algebra and if $a\in{A}$,
        then $(a^{\minus{1}})^{\minus{1}}=a$.
    \end{theorem}
    \begin{proof}
        For:
        \begin{align}
            a&=a*e_{*}
            \tag{Identity}\\
            &=a*\big(a^{\minus{1}}\circ(a^{\minus{1}})^{\minus{1}})
            \tag{Complement}\\
            &=(a\circ{a}^{\minus{1}})*
                \big(a\circ(a^{\minus{1}})^{\minus{1}}\big)
            \tag{Distributivity}\\
            &=e_{*}*\big(a\circ(a^{\minus{1}})^{\minus{1}}\big)
            \tag{Complement}\\
            &=a\circ(a^{\minus{1}})^{\minus{1}}
            \tag{Identity}
        \end{align}
        And similarly $a*(a^{\minus{1}})^{\minus{1}}=a$. But if $a*b=a$ and
        $a\circ{b}=a$, then $a=b$
        (Thm.~\ref{thm:Bool_Alg_aob_equal_a_acb_equal_a_implies_a_equal_b}).
        Therefore, $a=(a^{\minus{1}})^{\minus{1}}$.
    \end{proof}
    \begin{ltheorem}{Absorption Laws}{Absorption_Law}
        If $(A,\circ,*)$ is a Boolean algebra, if $a\in{A}$ and if $b\in{A}$,
        then $a*(a\circ{b})=a$ and $a\circ(a*{b})=a$.
    \end{ltheorem}
    \begin{proof}
        For:
        \par
        \begin{minipage}[t]{0.59\textwidth}
            \centering
            \begin{align}
                a*(a\circ{b})&=(a*e_{*})\circ(a*b)
                \tag{Identity}\\
                &=a*(e_{*}\circ{b})
                \tag{Distributivity}
            \end{align}
        \end{minipage}
        \hfill
        \begin{minipage}[t]{0.39\textwidth}
            \centering
            \begin{align}
                &=a*e_{*}
                \tag{Thm.~\ref{thm:Bool_Alg_Boundary_of_Circ}}\\
                &=a
                \tag{Identity}
            \end{align}
        \end{minipage}
        \par\vspace{2.5ex}
        And therefore $a*(a\circ{b})=a$. Similarly:
        \par
        \begin{minipage}[t]{0.59\textwidth}
            \centering
            \begin{align}
                a\circ(a*b)&=(a\circ{e}_{\circ})*(a\circ{b})
                \tag{Identity}\\
                &=a\circ(e_{\circ}*b)
                \tag{Distributivity}
            \end{align}
        \end{minipage}
        \hfill
        \begin{minipage}[t]{0.39\textwidth}
            \begin{align}
                &=a\circ{e}_{\circ}
                \tag{Thm.~\ref{thm:Bool_Alg_Boundary_of_Star}}\\
                &=a
                \tag{Identity}
            \end{align}
        \end{minipage}
        \par\vspace{2.5ex}
        And therefore $a\circ(a*b)=a$.
    \end{proof}
    We can weaken the hypothesis of
    Thm.~\ref{thm:Bool_Alg_aob_equal_a_acb_equal_a_implies_a_equal_b} to obtain
    a more general result.
    \begin{theorem}
        \label{thm:Bool_Alg_aob_equal_acb_implies_a_equal_b}
        If $(A,\circ,*)$ is a Boolean algebra, if $a,b\in{A}$, and if
        $a*b=a\circ{b}$, then $a=b$.
    \end{theorem}
    \begin{proof}
        For:
        \begin{align}
            a&=a*e_{*}
            \tag{Identity}\\
            &=a*(b\circ{b}^{\minus{1}})
            \tag{Complement}\\
            &=(a*b)\circ(a*b^{\minus{1}})
            \tag{Distributivity}\\
            &=(a\circ{b})\circ(a*b^{\minus{1}})
            \tag{Hypothesis}\\
            &=((a\circ{b})\circ{a})*
                \big((a\circ{b})\circ{b}^{\minus{1}}\big)
            \tag{Distributivity}\\
            &=\big((a\circ{b})\circ{a}\big)*
                \big(a\circ(b\circ{b}^{\minus{1}})\big)
            \tag{Associativity}\\
            &=\big((a\circ{b})\circ{a}\big)*(a\circ{e}_{*})
            \tag{Complement}\\
            &=\big((a\circ{b})\circ{a}\big)*e_{*}
            \tag{Thm.~\ref{thm:Bool_Alg_Boundary_of_Circ}}\\
            &=(a\circ{b})\circ{a}
            \tag{Identity}\\
            &=(a\circ{a})\circ{b}
            \tag{Associativity and Commutativity}\\
            &=a\circ{b}
            \tag{Thm~\ref{thm:Bool_Alg_Idempotent_of_Circ}}
        \end{align}
        Thus $a=a\circ{b}$. But $a\circ{b}=a*{b}$, and so $a=a*{b}$. By
        Thm.~\ref{thm:Bool_Alg_aob_equal_a_acb_equal_a_implies_a_equal_b},
        $a=b$.
    \end{proof}
            \begin{definition} For $a\in S$, an inverse, or normal inverse, of the First Operation is an element $b\in S$ such that $a\circ b=e_{\circ}$. An inverse of the Second Operation is similarly defined. The normal inverses are denoted $a^{*}$ and $a^{\circ}$.
            \end{definition}
            \begin{theorem} If $a\in S$ has a normal inverse for either operation, than it is unique.
            \end{theorem}
            \begin{proof} For suppose not. Let $a\in S$ have a normal inverse for the First Operation. That is, there is an $a^{\circ}\in S$ such that $a\circ a^{\circ}=e_{\circ}$ and let $a'^{\circ}$ be a second normal inverse not equal to the first. But then $a^{\circ}=a^{\circ}\circ e_{\circ}=a^{\circ}\circ (a\circ a'^{\circ})$ and from associativity we have $a^{\circ}=(a^{\circ}\circ a)\circ a'^{\circ}=a'^{\circ}$. Thus, the normal inverse is unique. Similarly if there is an inverse for the Second Operation
            \end{proof}
            \begin{theorem} If $a\in S$ has a normal inverse, say $a'$, for one operation, then $a^{-1}=a'^{-1}$.
            \end{theorem}
            \begin{proof} For let $a\in S$ have a normal inverse $a'$ for the First Operation. That is, $a\circ a' = e_{\circ}$. But $a' \circ a'^{-1}=e_{*}$, and from theorem 1.3 $a\circ e_{*}=e_{*}$. So $a\circ (a' \circ a'^{-1})=e_{*}$. And from theorem 1.4, $a\circ a=a$, so we have $(a\circ a)\circ (a'\circ a'^{-1}=a\circ (a\circ a')\circ a'^{-1}=a\circ a'^{-1}=e_{*}$. But $a\circ a^{-1}=e_{\circ}$. And pseudo-inverses are unique. Thus, $a^{-1}=a'^{-1}$. 
            \end{proof}
            \begin{theorem} The identities have normal inverses for their respective operations.
            \end{theorem}
            \begin{proof} As normal inverses are unique, it suffices to find inverses for both identities. But $e_{\circ}\circ e_{\circ}=e_{\circ}$, so $e_{\circ}$ is its own inverse for the First Operation. Similarly, $e_{*}*e_{*}=e_{*}$.
            \end{proof}
            \begin{theorem} \textbf{(The Not-A-Field Theorem)} Only the identities have normal inverses.
            \end{theorem}
            \begin{proof} For suppose not. Suppose $a\in S,\ a\ne e_{\circ},\ a\ne e_{*}$ and a has an inverse for the First Operation. That is $\exists a^{\circ}\in S|\ a\circ a^{\circ}=e_{\circ}$. But by theorem 1.4, $a\circ a^{\circ}=(a\circ a)\circ a^{\circ}$. By associativity, we have $e_{\circ}=a\circ a^{\circ} = a\circ (a\circ a^{\circ})=a\circ e_{\circ}=a$. Thus, $a=e_{\circ}$. But by hypothesis, $a\ne e_{\circ}$. Thus, there is no inverse for $a$. Similarly, a has no inverse for the Second Operation.
            \end{proof}
            \begin{theorem}
            There exist pseudo-fields with only one element.
            \end{theorem}
            \begin{proof}
            For let $e_{\circ} = e_{*}$, and let no other elements be in the set. 
            \end{proof}
            \begin{theorem}
            A pseud-field has one element if and only if $e_{\circ} = e_{*}$.
            \end{theorem}
            \begin{proof}
            For suppose there is another element $a \ne e_{\circ}$. But then $a \circ e_{\circ} = a$, but also $a \circ e_{\circ} = a \circ e_{*} = e_{*}$. So $a = e_{*}$. If there is only one element, then clearly $e_{\circ} = e_{*}$ as otherwise there would be two elements.
            \end{proof}
            \begin{definition} A generating set on a pseudo-field is a subset $g_S \subset S$ such that every element of $S$ can be written as a finite combination of elements in $g_S$ using $\circ$ or $*$.
            \end{definition}
            \begin{theorem}
            The number of elements in a finite pseudo-field is a power of 2.
            \end{theorem}
            \begin{proof}
            Consider the set of all generators $g_S$ on $S$. Clearly for all such generators, $1\leq |g_S|\leq |S|$. Let $G$ be the smallest generator, such that $|G| \leq |g_S|$ for any other given generator. 
            \end{proof}