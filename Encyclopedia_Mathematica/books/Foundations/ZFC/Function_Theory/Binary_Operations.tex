\section{Binary Operations}
    Binary operations are the standard tools that one uses when they develope
    arithmetic. As such, the most familiar examples of binary operations are
    those of addition, multiplication, and subtraction with real numbers.
    On the other hand, division is \textit{not} a binary operation on the real
    numbers since division by zero is undefined. To make this explicit we need
    to give a rigorous definition to binary operations. We can do this with the
    language of functions\index{Function} and by using the Cartesian product
    \index{Cartesian Product} of a set $A$ with itself.
    \begin{fdefinition}{Binary Operation}{Binary_Operation}
        A \gls{binary operation} on a set $A$ is a function
        $*:A\times{A}\rightarrow{A}$.
    \end{fdefinition}
    \begin{example}
        Let $\mathbb{R}$ be the set of real numbers and $+$ denote the addition
        of two real numbers. Then $+$ is a binary operation on $\mathbb{R}$.
        Similarly, if $\cdot$ denotes the multiplication of two real numbers,
        than it two is a binary operation on $\mathbb{R}$. For division, $\div$,
        we are lacking the requirement that \textit{for all}
        $(a,b)\in\mathbb{R}^{2}$ there is a unique $c\in\mathbb{R}$ such that
        $a\div{b}=c$, since if $b=0$ our expression is undefined. That is, this
        is not a function from $\mathbb{R}^{2}$ to $\mathbb{R}$. If we consider
        all of the non-zero elements, then division is a binary operation. That
        is, division is a binary operation on $\mathbb{R}\setminus\{0\}$.
    \end{example}
    \begin{lexample}{Binary Operation on the Set of Functions}
                    {Binary_Operation_on_the_Set_of_Functions}
        If $A$ is a set, and if $\mathcal{F}(A,A)$ denotes the set of all
        functions $f:A\rightarrow{A}$, and if $\circ$ denotes function
        composition, then $\circ$ is a binary operation on $\mathcal{F}(A,A)$.
        That is, for any two functions $f,g\in\mathcal{F}(A,A)$, the composition
        $g\circ{f}:A\rightarrow{A}$ is again an element of $\mathcal{F}(A,A)$
    \end{lexample}
    Just like functions, there are three important conditions that a binary
    operation must satisfy. Given any ordered pair $(a,b)\in{A}^{2}$, it must
    be true that $*(a,b)$ is defined. This comes from the definition of a
    function on a set (Def.~\ref{def:Function}). Next, the image of $(a,b)$ must
    be unique. That is, if $*(a,b)=c$ and $*(a,b)=d$, then $c=d$. Note that this
    is not the same as requiring that $*(a,b)=*(b,a)$, and in general this is
    not true. Such binary operations are called
    \textit{commutative}\index{Commutative Operation}. Lastly, for any
    $(a,b)\in{A}^{2}$, $*(a,b)$ must be an element of $A$. That is,
    $*(a,b)\in{A}$. All of these requirements come from the definition of a
    function, so in a sense it is redundant to repeat these. In practice one
    defines a binary operation by a formula $\varphi$, and it then becomes
    necessary to show that this formula satisfies these properties before we can
    rightly call it a binary operation.
    \begin{example}
        Let $A=\mathbb{Z}_{2}$ and consider all of the binary operations on
        $\mathbb{Z}_{2}$. We can count these by constructing tables:
        \begin{table}[H]
            \centering
            \begin{tabular}{c|c}
                $(x,y)$&$*(x,y)$\\
                \hline
                $(0,0)$&0\\
                $(0,1)$&0\\
                $(1,0)$&1\\
                $(1,1)$&0
            \end{tabular}
            \label{tab:Binary_Operation_on_Z_2}
            \caption{Simple Binary Operation on $\mathbb{Z}_{2}$}
        \end{table}
        This is one such binary operation, there are 15 others. To see this,
        recall that the number of functions from a set $A$ to a set $B$, where
        both $A$ and $B$ are finite sets with $m$ and $n$ elements,
        respectively, is $n^{m}$. Since $\mathbb{Z}_{2}$ has 2 elements, and
        since a binary operation is a function
        $*\mathbb{Z}_{2}\times\mathbb{Z}_{2}\rightarrow\mathbb{Z}_{2}$, the
        total number of binary operations is $2^{(2^{2})}=2^{4}=16$. In general,
        if $A$ has $n$ elements, and if $B$ is the set of all binary operations
        on $A$, then:
        \begin{equation}
            \textrm{Card}(B)=n^{(n^2)}
        \end{equation}
    \end{example}
    \begin{fnotation}{Binary Operation}{Binary_Operation}
        If $A$ is a set and if $*:A\times{A}\rightarrow{A}$ is a binary
        operation on $A$, for any ordered pair $(a,b)\in{A}^{2}$, the image
        of $*(a,b)$ is denoted $a*b$.
    \end{fnotation}
    It is occasionally useful to think of binary operations purely as functions,
    and so we will use function notation at these times. For the most part we
    will stick with notation defined in Not.~\ref{not:Binary_Operation}. There
    are several types of binary operations worth studying, and several key
    properties that these operations can have. One of the most fundamental is
    the existence of a \textit{unital} element, also known as an identity.
    \begin{fdefinition}{Left Unital Element}{Left_Unital_Element}
        A left unital element in a \gls{set} $A$ under a \gls{binary operation}
        $*$ on $A$ is an element $e_{L}\in{A}$ such that, for all $a\in{A}$ it
        is true that $e_{L}*a=a$.
    \end{fdefinition}
    \begin{example}
        From the definition of a left unital element
        (Def.~\ref{def:Left_Unital_Element}) it would seem natural to define a
        right unital element. The importance is to note that left and right
        unital elements need not be equal. Indeed, if $A$ is a set and $*$ is
        a binary operation, given a left identity $e_{L}$ and a right identity
        $e_{R}$ it will be true that $e_{R}=e_{L}$ and thus all left and right
        unital elements will be the same
        (see Thm.~\ref{thm:left_and_right_identity_implies_identity}). Thus to
        find counterexamples to the claim that the existence of a left unital
        element implies the existence of a right unital element we need to think
        of strange operations. Let $A=\mathbb{R}$ and let $*$ be defined by
        $a*b=b$ for all $a,b\in\mathbb{R}$. Then every element of $\mathbb{R}$
        is a left unital element. Moreover, none of the element of $\mathbb{R}$
        are right unital elements.
    \end{example}
    \begin{fdefinition}{Right Unital Element}{Right_Unital_Element}
        A right unital element of a \gls{set} $A$ under a \gls{binary operation}
        $*$ is an element $e_{R}$ such that for all $a\in{A}$ it is true that
        $a*e_{R}=a$.
    \end{fdefinition}
    \begin{theorem}
        \label{thm:left_and_right_identity_implies_identity}%
        If $A$ is a set, if $*$ is a binary operation on $A$, if $e_{L}$ is a
        left unital element of $A$, and if $e_{R}$ is a right unital element of
        $a$, then $e_{L}=e_{R}$.
    \end{theorem}
    \begin{proof}
        For:
        \begin{equation}
            e_{L}=e_{L}*e_{R}=e_{R}
        \end{equation}
        And thus $e_{L}=e_{R}$.
    \end{proof}
    \begin{example}
        Consider a non-empty set $A$ and the set of all functions from $A$ to
        itself, $\mathcal{F}(A,A)$. Let $\circ$ denote the binary operation of
        function composition. Then $\mathcal{F}(A,A)$ has a right identity under
        $\circ$, and a left identity. For
        the identity function $\textrm{id}_{A}$ acts as a right identity:
        \begin{equation}
            (f\circ\textrm{id}_{A})(x)
            =f\big(\textrm{id}_{A}(x)\big)
            =f(x)
        \end{equation}
        And thus $\textrm{id}_{A}$ is a right identity. By
        Thm.~\ref{thm:left_and_right_identity_implies_identity}, any left
        identity must also be a right identity, and so the likely candidate to
        check is $\textrm{id}_{A}$. And indeed we have:
        \begin{equation}
            (\textrm{id}_{A}\circ{f})(x)
            =\textrm{id}_{A}\big(f(x)\big)
            =f(x)
        \end{equation}
        And thus $\textrm{id}_{A}$ is a left identity as well.
    \end{example}
    \begin{fdefinition}{Unital Element}{Unital_Element}
        A \gls{unital element} of a \gls{set} $A$ under a \gls{binary operation}
        $*$ is an element $e\in{A}$ that is both a right unital element and a
        left unital element.
    \end{fdefinition}
    \begin{example}
        Let $\mathbb{R}$ be the set of real numbers and let $+$ be the usual
        notion of addition. Then 0 is a unital element of $\mathbb{R}$ with
        respect to this operation. That is, for any real number $x$ we have
        $x+0=0+x=x$. For multiplication the unital element is 1. This is because
        $1\cdot{x}=x\cdot{1}=x$. Subtraction has a right unital element, and
        again it is 0 since $x-0=x$, but no left identity. To see this, suppose
        $e-x=x$ for all $x$. Applying some algebra we have that $e=2x$, meaning
        there is no constant $e\in\mathbb{R}$ such that for all $x$, $e-x=x$.
        Since subtraction has no left unital element, it has no unital element
        either.
    \end{example}
    \begin{theorem}
        \label{thm:Unital_Elements_are_Unique}%
        If $A$ is a set, if $*$ is a binary operation on $A$, and if $e$ and
        $e'$ are unital elements of $A$, then $e=e'$
    \end{theorem}
    \begin{proof}
        For:
        \begin{equation}
            e=e*e'=e'
        \end{equation}
        And thus by transitivity, $e=e'$.
    \end{proof}
    The next thing to discuss is that of inverses. There are five types, but in
    practice only one of these is discussed.
    \begin{fdefinition}{Weak Right Inverse}{Weak_Right_Inverse}
        A weak right inverse of an element $a$ in a \gls{set} $A$ under a
        \gls{binary operation} $*$ on $A$ is an element $b\in{A}$ such that
        $a*b$ is a right unital element.
    \end{fdefinition}
    This definition will not recieve much use until we discuss
    groups\index{Group}. A group is a set with a binary operation $*$ that has
    a unital element, inverse elements, and is associatied (to be defined soon).
    As it turns out these conditions are stronger than necessary and it suffices
    to check that there are weak right inverses and a right unital element. The
    next thing to define is right inverses.
    \begin{fdefinition}{Right Inverse}{Right_Inverse}
        A right inverse of an element $a$ in a \gls{set} $A$ under a
        \gls{binary operation} is an element $b\in{A}$ such that $a*b$ is a
        \gls{unital element}.
    \end{fdefinition}
    Here, we've simply strengthened the requirement that $a*b$ not only be a
    right unital element, but also a left unital element as well. A right
    inverse is therefore necessarily a weak right inverse.
    \begin{fdefinition}{Weakly Left Invertible}{Weakly_Left_Invertible}
        A weakly left invertible element of a \gls{set} $A$ under a
        \gls{binary operation} $*$ is an element $a\in{A}$ such that there
        exists a $b\in{A}$ such that $b*a$ is a left unital element.
    \end{fdefinition}
    \begin{fdefinition}{Left Invertible Element}{Left_Inverse}
        A left invertible element of a \gls{set} $A$ under a
        \gls{binary operation} is an element $a\in{A}$ such that there exists a
        $b\in{A}$ such that $b*a$ is a \gls{unital element}.
    \end{fdefinition}
    \begin{fdefinition}{Invertible Element}{Invertible_Element}
        An invertible element of a a \gls{set} $A$ under a
        \gls{binary operation} is an element $a\in{A}$ that is both
        left invertible and right invertible.
    \end{fdefinition}
    \begin{fdefinition}{Commutative Operation}{Commutative_Operation}
        A \gls{commutative operation} on a \gls{set} $A$ is a
        \gls{binary operation} $*$ such that for all $(a,b)\in{A}^{2}$ it is
        true that $a*b=b*a$.
    \end{fdefinition}
    \begin{fdefinition}{Associative Operation}{Associative_Operation}
        A \gls{associative operation} on a \gls{set} $A$ is a
        \gls{binary operation} $*$ such that, for all $a,b,c\in{A}$ it is true
        that $a*(b*c)=(a*b)*c$.
    \end{fdefinition}
