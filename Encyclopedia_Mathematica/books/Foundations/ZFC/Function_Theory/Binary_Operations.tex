\section{Binary Operations}
    Binary operations are the standard tools that one uses when they develope
    arithmetic. As such, the most familiar examples of binary operations are
    those of addition, multiplication, and subtraction with real numbers.
    On the other hand, division is \textit{not} a binary operation on the real
    numbers since division by zero is undefined. To make this explicit we need
    to give a rigorous definition to binary operations. We can do this with the
    language of functions\index{Function} and by using the Cartesian product
    \index{Cartesian Product} of a set $A$ with itself.
    \begin{fdefinition}{Binary Operation}{Binary_Operation}
        A \gls{binary operation} on a set $A$ is a function
        $*:A\times{A}\rightarrow{A}$.
    \end{fdefinition}
    \begin{example}
        Let $\mathbb{R}$ be the set of real numbers and $+$ denote the addition
        of two real numbers. Then $+$ is a binary operation on $\mathbb{R}$.
        Similarly, if $\cdot$ denotes the multiplication of two real numbers,
        than it two is a binary operation on $\mathbb{R}$. For division, $\div$,
        we are lacking the requirement that \textit{for all}
        $(a,b)\in\mathbb{R}^{2}$ there is a unique $c\in\mathbb{R}$ such that
        $a\div{b}=c$, since if $b=0$ our expression is undefined. That is, this
        is not a function from $\mathbb{R}^{2}$ to $\mathbb{R}$. If we consider
        all of the non-zero elements, then division is a binary operation. That
        is, division is a binary operation on $\mathbb{R}\setminus\{0\}$.
    \end{example}
    \begin{lexample}{Binary Operation on the Set of Functions}
                    {Binary_Operation_on_the_Set_of_Functions}
        If $A$ is a set, and if $\mathcal{F}(A,A)$ denotes the set of all
        functions $f:A\rightarrow{A}$, and if $\circ$ denotes function
        composition, then $\circ$ is a binary operation on $\mathcal{F}(A,A)$.
        That is, for any two functions $f,g\in\mathcal{F}(A,A)$, the composition
        $g\circ{f}:A\rightarrow{A}$ is again an element of $\mathcal{F}(A,A)$
    \end{lexample}
    Just like functions, there are three important conditions that a binary
    operation must satisfy. Given any ordered pair $(a,b)\in{A}^{2}$, it must
    be true that $*(a,b)$ is defined. This comes from the definition of a
    function on a set (Def.~\ref{def:Function}). Next, the image of $(a,b)$ must
    be unique. That is, if $*(a,b)=c$ and $*(a,b)=d$, then $c=d$. Note that this
    is not the same as requiring that $*(a,b)=*(b,a)$, and in general this is
    not true. Such binary operations are called
    \textit{commutative}\index{Commutative Operation}. Lastly, for any
    $(a,b)\in{A}^{2}$, $*(a,b)$ must be an element of $A$. That is,
    $*(a,b)\in{A}$. All of these requirements come from the definition of a
    function, so in a sense it is redundant to repeat these. In practice one
    defines a binary operation by a formula $\varphi$, and it then becomes
    necessary to show that this formula satisfies these properties before we can
    rightly call it a binary operation.
    \begin{example}
        Let $A=\mathbb{Z}_{2}$ and consider all of the binary operations on
        $\mathbb{Z}_{2}$. We can count these by constructing tables:
        \begin{table}[H]
            \centering
            \begin{tabular}{c|c}
                $(x,y)$&$*(x,y)$\\
                \hline
                $(0,0)$&0\\
                $(0,1)$&0\\
                $(1,0)$&1\\
                $(1,1)$&0
            \end{tabular}
            \label{tab:Binary_Operation_on_Z_2}
            \caption{Simple Binary Operation on $\mathbb{Z}_{2}$}
        \end{table}
        This is one such binary operation, there are 15 others. To see this,
        recall that the number of functions from a set $A$ to a set $B$, where
        both $A$ and $B$ are finite sets with $m$ and $n$ elements,
        respectively, is $n^{m}$. Since $\mathbb{Z}_{2}$ has 2 elements, and
        since a binary operation is a function
        $*\mathbb{Z}_{2}\times\mathbb{Z}_{2}\rightarrow\mathbb{Z}_{2}$, the
        total number of binary operations is $2^{(2^{2})}=2^{4}=16$. In general,
        if $A$ has $n$ elements, and if $B$ is the set of all binary operations
        on $A$, then:
        \begin{equation}
            \textrm{Card}(B)=n^{(n^2)}
        \end{equation}
    \end{example}
    \begin{fnotation}{Binary Operation}{Binary_Operation}
        If $A$ is a set and if $*:A\times{A}\rightarrow{A}$ is a binary
        operation on $A$, for any ordered pair $(a,b)\in{A}^{2}$, the image
        of $*(a,b)$ is denoted $a*b$.
    \end{fnotation}
    It is occasionally useful to think of binary operations purely as functions,
    and so we will use function notation at these times. For the most part we
    will stick with notation defined in Not.~\ref{not:Binary_Operation}.