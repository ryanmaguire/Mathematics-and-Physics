\section{Basic Definitions and Theorems}
    We start by presenting some basic, but crucial, facts about the image and
    pre-images of subsets under a function.
    \begin{theorem}
        \label{thm:Image_of_Empty_Set_Is_Empty}%
        If $A$ and $B$ are sets, if $f:A\rightarrow{B}$ is a function,
        then $f(\emptyset)=\emptyset$.
    \end{theorem}
    \begin{proof}
        For suppose not and suppose $y\in{f}(\emptyset)$. But then by the
        definition of the image of a subset (Def~\ref{def:Image_of_Subset})
        there is an $x\in\emptyset$ such that $f(x)=y$, a contradiction since
        for all $x$ it is true that $x\notin\emptyset$
        (Def.~\ref{def:Empty_Set}). Therefore, $f(\emptyset)=\emptyset$.
    \end{proof}
    \begin{theorem}
        \label{thm:Pre_Image_of_Empty_Set_Is_Empty}%
        If $A$ and $B$ are sets, if $f:A\rightarrow{B}$ is a function, then
        $f^{\minus{1}}(\emptyset)=\emptyset$.
    \end{theorem}
    \begin{proof}
        Suppose not and suppose $x\in{f}^{\minus{1}}(\emptyset)$. Then by the
        definition of the pre-image (Def.~\ref{def:Pre_Image_of_Subset})
        $f(x)\in\emptyset$, a contradiction since for all $y$ it is true that
        $y\notin\emptyset$ (Def.~\ref{def:Empty_Set}). Therefore,
        $f^{\minus{1}}(\emptyset)=\emptyset$.
    \end{proof}
    \begin{theorem}
        \label{thm:Image_of_PreImage_is_Subset}%
        If $A$ and $B$ are sets, $\mathcal{U}\subseteq{B}$, and
        $f:A\rightarrow{B}$ is a function, then:
        \begin{equation*}
            f\big(f^{\minus{1}}(\mathcal{U})\big)\subseteq\mathcal{U}
        \end{equation*}
    \end{theorem}
    \begin{proof}
        For if $y\in{f(f^{\minus{1}}(\mathcal{U}))}$, then there is an
        $x\in{f^{\minus{1}}(\mathcal{U})}$ such that $y=f(x)$
        (Def.~\ref{def:Image_of_Subset}). But if
        $x\in{f^{\minus{1}}(\mathcal{U})}$, theb $f(x)\in\mathcal{U}$
        (Def.~\ref{def:Pre_Image_of_Subset}). Thus, $y\in\mathcal{U}$.
    \end{proof}
    We may not be able to attain equality.
    \begin{example}
        If $A$ and $B$ are non-empty sets and if there exists
        $y_{1},y_{2}\in{B}$ such that $y_{1}\ne{y}_{2}$, then
        there is a function $f:A\rightarrow{B}$ and a
        $\mathcal{U}\subseteq{B}$ such that:
        \begin{equation}
            f\big(f^{-1}(\mathcal{U})\big)\ne\mathcal{U}
        \end{equation}
        For if $A$ and $B$ are non-empty, let $f:A\rightarrow{B}$ be defined by:
        \begin{equation}
            f=\{\,(x,y_{1})\in{A}\times{B}\;|\;x\in{A}\,\}
        \end{equation}
        That is, $f(x)=y_{1}$ for all $x\in{A}$. Then $f$ is a function since
        $f\subseteq{A}\times{B}$ and for all $x\in{A}$ there is a unique
        $y\in{B}$ such that $(x,y)\in{f}$ (Def.~\ref{def:Function}). However
        since for all $x\in{A}$ it is true that $f(x)=y_{1}$, we have:
        \begin{equation}
            f^{\minus{1}}(\{y_{2}\})=\emptyset
        \end{equation}
        For suppose not. If $x\in{f}^{-1}(\{y_{2}\})$, then $f(x)=y_{2}$
        (Def.~\ref{def:Pre_Image_of_Subset}), but for all $x\in{A}$ it is
        true that $f(x)=y_{1}$, and $y_{1}\ne{y}_{2}$, a contradiction. Thus
        $f^{\minus{1}}(\{y_{2}\})=\emptyset$ (Def.~\ref{def:Empty_Set}). But by
        Thm.~\ref{thm:Image_of_Empty_Set_Is_Empty}, the image of the
        empty set is the empty set and thus:
        \begin{equation}
            f\big(f^{-1}(\{y_{2}\})\big)=\emptyset
        \end{equation}
        But $\{y_{2}\}\ne\emptyset$ and $\{y_{2}\}\subseteq{B}$. This shows
        the converse of Thm.~\ref{thm:Image_of_PreImage_is_Subset} may fail.
    \end{example}
    \begin{theorem}
        \label{thm:PreImage_of_Image_is_Superset}%
        If $A$ and $B$ are sets, $\mathcal{U}\subseteq{A}$, and
        $f:A\rightarrow{B}$ is a function, then:
        \begin{equation*}
            \mathcal{U}\subseteq{f}^{\minus{1}}\big(f(\mathcal{U})\big)
        \end{equation*}
    \end{theorem}
    \begin{proof}
        For if $x\in\mathcal{U}$, then $f(x)\in{f}(\mathcal{U})$
        (Def.~\ref{def:Image_of_Subset}). But if $f(x)\in\mathcal{U}$, then
        $x\in{f}^{\minus{1}}(f(\mathcal{U}))$
        (Def.~\ref{def:Pre_Image_of_Subset}), completing the proof.
    \end{proof}
    Again, the converse need not be true.
    \begin{example}
        Let $A=B=\mathbb{N}$, and define $f(n)=0$. Then
        $f^{\minus{1}}(\{0\})=\mathbb{N}$, but $\{0\}\ne\mathbb{N}$. Thus, the
        converse of Thm.~\ref{thm:PreImage_of_Image_is_Superset} may fail.
    \end{example}
    \begin{theorem}
        \label{thm:Image_of_NonEmpty_Set_Is_NonEmpty}%
        If $A$ and $B$ are sets, if $\mathcal{U}\subseteq{A}$ is a non-empty
        subset of $A$, and if $f:A\rightarrow{B}$ is a function, then
        $f(\mathcal{U})$ is non-empty.
    \end{theorem}
    \begin{proof}
        For suppose not. It $f(\mathcal{U})$ is empty, then
        $f^{\minus{1}}(f(\mathcal{U}))$ is empty
        (Thm.~\ref{thm:Pre_Image_of_Empty_Set_Is_Empty}). But by
        Thm.~\ref{thm:PreImage_of_Image_is_Superset},
        $\mathcal{U}\subseteq{f}^{\minus{1}}(f(\mathcal{U}))$, and thus
        $\mathcal{U}\subseteq\emptyset$. But $\emptyset\subseteq\mathcal{U}$
        (Thm.~\ref{thm:Emptyset_Is_Subset}), and thus $\mathcal{U}=\emptyset$
        (Def.~\ref{def:Equal_Sets}), a contradiction as $\mathcal{U}$ is
        non-empty.
    \end{proof}
    A nice rewording of this theorem comes from the contrapositive.
    \begin{theorem}
        If $A$ and $B$ are sets, if $\mathcal{U}\subseteq{A}$ is a non-empty
        set, if $f:A\rightarrow{B}$ is a function, and if
        $f(\mathcal{U})=\emptyset$, then $\mathcal{U}=\emptyset$.
    \end{theorem}
    \begin{proof}
        Suppose not. Then $f(\mathcal{U})$ is non-empty
        (Thm.~\ref{thm:Image_of_NonEmpty_Set_Is_NonEmpty}), a contradiction.
    \end{proof}
    We cannot reverse this theorem for the pre-image. That is, the inverse image
    of a non-empty set may indeed be empty.
    \begin{example}
        Consider the case when $A=B=\mathbb{R}$ and define the function
        $f:\mathbb{R}\rightarrow\mathbb{R}$ by $f(x)=0$ for all
        $x\in\mathbb{R}$. Formally, this is subset of $\mathbb{R}^{2}$ defined
        by:
        \begin{equation}
            f=\big\{\,(x,0)\in\mathbb{R}^{2}\;|\;x\in\mathbb{R}\,\big\}
        \end{equation}
        The pre-image of any subset $\mathcal{U}\subseteq\mathbb{R}$ the does
        not contain 0 will be empty, even though the set itself need not be
        empty. A surjective function is a function with the property that the
        pre-image of a non-empty set is non-empty
        (see Def.~\ref{def:Surjective_Function}).
    \end{example}
    \begin{theorem}
        If $A$ and $B$ are sets, if $\mathcal{U},\mathcal{V}\subseteq{A}$, if
        $\mathcal{U}\subseteq\mathcal{V}$, and if $f:A\rightarrow{B}$ is a
        function, then:
        \begin{equation*}
            f(\mathcal{U})\subseteq{f}(\mathcal{V})
        \end{equation*}
    \end{theorem}
    \begin{proof}
        For suppose not. Then by the definition of subset, there is a
        $y\in{f}(\mathcal{U})$ such that $y\notin{f}(\mathcal{V})$
        (Def.~\ref{def:Subsets}). But if $y\in{f}(\mathcal{U})$, then there is
        an $x\in\mathcal{U}$ such that $f(x)=y$
        (Def.~\ref{def:Image_of_Subset}). But $\mathcal{U}\subseteq\mathcal{V}$,
        and therefore $x\in\mathcal{V}$ (Def.~\ref{def:Subsets}). But if
        $x\in\mathcal{V}$, then $f(x)\in{f}(\mathcal{V})$
        (Def.~\ref{def:Image_of_Subset}). Thus, $y\in{f}(\mathcal{V})$,
        a contradiction.
    \end{proof}
    \begin{theorem}
        If $A$ and $B$ are sets, if $\mathcal{U},\mathcal{V}\subseteq{B}$, if
        $\mathcal{U}\subseteq\mathcal{V}$, and if $f:A\rightarrow{B}$ is a
        function, then:
        \begin{equation*}
            f^{\minus{1}}(\mathcal{U})\subseteq{f^{\minus{1}}}(\mathcal{V})
        \end{equation*}
    \end{theorem}
    \begin{proof}
        For suppose not. Then by the definition of subsets there is an
        $x\in{f}^{\minus{1}}(\mathcal{U})$ such that
        $x\notin{f}^{\minus{1}}(\mathcal{V})$ (Def.~\ref{def:Subsets}). But by
        the definition of pre-image, if $x\in{f}^{\minus{1}}(\mathcal{U})$, then
        there is a $y\in\mathcal{U}$ such that $f(x)=y$
        (Def.~\ref{def:Pre_Image_of_Subset}). But
        $\mathcal{U}\subseteq\mathcal{V}$, and thus $y\in\mathcal{V}$
        (Def.~\ref{def:Subsets}). But then $x\in{f}^{\minus{1}}(\mathcal{V})$, a
        contradiction.
    \end{proof}
    \begin{ftheorem}{Preservation of Intersection by Pre-Images}
                    {Preservation_of_Intersection_by_Pre_Images}
        If $A$ and $B$ are sets, if $f:A\rightarrow{B}$ is a function, if
        $\mathcal{P}(B)$ is the power set of $B$, and if
        $\mathcal{O}\subseteq\mathcal{P}(B)$, then:
        \begin{equation*}
            f^{\minus{1}}\Big(
                \bigcap_{\mathcal{U}\in\mathcal{O}}\mathcal{U}\Big)
            =\bigcap_{\mathcal{U}\in\mathcal{O}}f^{\minus{1}}(\mathcal{U})
        \end{equation*}
    \end{ftheorem}
    \begin{bproof}
        For if $x\in{f}^{\minus{1}}(\bigcap\mathcal{U})$,
        then $f(x)\in\bigcap\mathcal{U}$ (Def.~\ref{def:Pre_Image_of_Subset}).
        But then for all $\mathcal{U}\in\mathcal{O}$, $f(x)\in\mathcal{U}$
        (Def.~\ref{def:Intersection_Over_a_Collection}). But then for all
        $\mathcal{U}\in\mathcal{O}$, $x\in{f}^{\minus{1}}(\mathcal{U})$
        (Def.~\ref{def:Pre_Image_of_Subset}) and therefore
        $x\in\bigcap{f}^{\minus{1}}(\mathcal{U})$
        (Def.~\ref{def:Intersection_Over_a_Collection}). Thus, by the definition
        of subsets:
        \begin{equation}
            f^{\minus{1}}\Big(
                \bigcap_{\mathcal{U}\in\mathcal{O}}\mathcal{U}\Big)
            \subseteq\bigcap_{\mathcal{U}\in\mathcal{O}}
                f^{\minus{1}}(\mathcal{U})
        \end{equation}
        Moreover, if $x\in\bigcap{f}^{\minus{1}}(\mathcal{U})$ then for all
        $\mathcal{U}\in\mathcal{O}$ it is true that
        $x\in{f}^{\minus{1}}(\mathcal{U})$
        (Def.~\ref{def:Intersection_Over_a_Collection}). But then for all
        $\mathcal{U}\in\mathcal{O}$ it is true that $f(x)\in\mathcal{U}$
        (Def.~\ref{def:Pre_Image_of_Subset}) and therefore
        $f(x)\in\bigcap\mathcal{U}$. But then
        $x\in{f}^{\minus{1}}(\bigcap\mathcal{U})$
        (Def.~\ref{def:Pre_Image_of_Subset}) and therefore:
        \begin{equation}
            \bigcap_{\mathcal{U}\in\mathcal{O}}f^{\minus{1}}(\mathcal{U})
            \subseteq{f}^{\minus{1}}\Big(
                \bigcap_{\mathcal{U}\in\mathcal{O}}\mathcal{U}\Big)
        \end{equation}
        From the definition of equality
        (Def.~\ref{def:Equal_Sets}) we can conclude the proof.
    \end{bproof}
    While the proof of this theorem is very straight forward, we have
    highlighted it to emphasize that it is a very important and useful theorem
    that will be used frequently once we delve into measure theory and topology.
    This theorem has a counterpart for unions that is equally important. The
    forward image of a function lacks this preservation property, and we can
    only show that one side is a subset (perhaps proper) of the other.
    \begin{theorem}
        If $A$ and $B$ are sets, if $f:A\rightarrow{B}$ is a function, if
        $\mathcal{P}(A)$ is the power set of $A$, and if
        $\mathcal{O}\subseteq\mathcal{P}(A)$, then:
        \begin{equation}
            f\Big(\bigcap_{\mathcal{U}\in\mathcal{O}}\mathcal{U}\Big)
            \subseteq\bigcap_{\mathcal{U}\in\mathcal{O}}f(\mathcal{U})
        \end{equation}
    \end{theorem}
    \begin{proof}
        For if not, then by the definition of subset there is a
        $y\in{f}(\bigcap\mathcal{U})$ such that $y\notin\bigcap{f}(\mathcal{U})$
        (Def.~\ref{def:Subsets}). But then by the definition of image, there is
        an $x\in\bigcap\mathcal{U}$ such that $f(x)=y$
        (Def.~\ref{def:Image_of_Subset}). But by the definition of intersection,
        for all $\mathcal{U}\in\mathcal{O}$ it is true that $x\in\mathcal{U}$
        (Def.~\ref{def:Intersection_Over_a_Collection}), and thus for all
        $\mathcal{U}\in\mathcal{O}$, $f(x)\in{f}(\mathcal{U})$
        (Def.~\ref{def:Image_of_Subset}). Thus,
        $y\in\bigcap{f}(\mathcal{U})$
        (Def.~\ref{def:Intersection_Over_a_Collection}), a contradiction.
    \end{proof}
    \begin{example}
        Equality need not be attained for a general function. Indeed, let
        $f:\mathbb{N}\rightarrow\mathbb{N}$ be defined by
        $f(n)=0$. Note the $\mathbb{N}_{e}$ and $\mathbb{N}_{o}$, the set of
        even and odd integers, respectively, are disjoint and thus
        $\mathbb{N}_{e}\cap\mathbb{N}_{o}=\emptyset$. Applying
        Thm.~\ref{thm:Image_of_Empty_Set_Is_Empty}, we have:
        \begin{equation}
            f(\mathbb{N}_{e}\cap\mathbb{N}_{o})=\emptyset
        \end{equation}
        However, $f(\mathbb{N}_{e})=\{0\}$, and similarly for $\mathbb{N}_{o}$.
        Thus:
        \begin{equation}
            f(\mathbb{N}_{e})\cap{f}(\mathbb{N}_{o})=\{0\}
        \end{equation}
        and thus equality is not attained. Functions that attain equality are
        \textit{injective}\index{Function!Injective}.
    \end{example}
    Returning to pre-images, we have preservation once again.
    \begin{ftheorem}{Preservation of Union by Pre-Images}
                    {Preservation_of_Union_by_Pre_Images}
        If $A$ and $B$ are sets, if $f:A\rightarrow{B}$ is a function,
        if $\mathcal{P}(B)$ is the power set of $B$, and if
        $\mathcal{O}\subseteq\mathcal{P}(B)$, then:
        \begin{equation*}
            f^{\minus{1}}\Big(
                \bigcup_{\mathcal{U}\in\mathcal{O}}\mathcal{U}\Big)
            =\bigcup_{\mathcal{U}\in\mathcal{O}}f^{\minus{1}}(\mathcal{U})
        \end{equation*}
    \end{ftheorem}
    \begin{bproof}
        For if $x\in{f}^{\minus{1}}(\bigcup\mathcal{U})$, then by the definition
        of pre-image, $f(x)\in\bigcup\mathcal{U}$
        (Def.~\ref{def:Pre_Image_of_Subset}). But then by the definition of
        union, there is a $\mathcal{U}\in\mathcal{O}$ such that
        $f(x)\in\mathcal{U}$. But then again by the definition of pre-image,
        $x\in{f}^{\minus{1}}(\mathcal{U})$. And since
        $f^{\minus{1}}(\mathcal{U})\subseteq\bigcup{f}^{\minus{1}}(\mathcal{U})$
        we have:
        \begin{equation}
            f^{\minus{1}}\Big(
                \bigcup_{\mathcal{U}\in\mathcal{O}}\mathcal{U}\Big)
            \subseteq
            \bigcup_{\mathcal{U}\in\mathcal{O}}f^{\minus{1}}(\mathcal{U})
        \end{equation}
        Similarly, if $x\in\bigcup{f}^{\minus{1}}(\mathcal{U})$, then there is
        a $\mathcal{U}\in\mathcal{O}$ such that
        $x\in{f}^{\minus{1}}(\mathcal{U})$ (Def.~\ref{def:Union_over_a_Set}).
        But then $f(x)\in\mathcal{U}$, and therefore
        $f(x)\in\bigcup\mathcal{U}$. But then
        $x\in{f}^{\minus{1}}(\bigcup\mathcal{U})$
        (Def.~\ref{def:Pre_Image_of_Subset}), and therefore:
        \begin{equation}
            \bigcup_{\mathcal{U}\in\mathcal{O}}f^{\minus{1}}(\mathcal{U})
            \subseteq
            f^{\minus{1}}\Big(
                \bigcup_{\mathcal{U}\in\mathcal{O}}\mathcal{U}\Big)
        \end{equation}
        By the definition of equality (Def.~\ref{def:Equal_Sets}), we are
        done.
    \end{bproof}
    While the forward image does not preserve intersections, it does preserve
    unions. Since the pre-image preserves both intersection and union, it is
    often used as the basic ingredient of many theories of mathematics. For
    example, in topology one defines continuous functions by looking at the
    pre-image of open sets. In measure theory one defines measurable functions
    by looking at the pre-image of measurable sets. In doing so we can very
    easily prove many useful theorems about such functions by resorting to
    Thm.~\ref{thm:Preservation_of_Intersection_by_Pre_Images} and
    Thm.~\ref{thm:Preservation_of_Union_by_Pre_Images}. We now show that the
    forward image is at least half useful.
    \begin{ftheorem}{Preservation of Union by Images}
                    {Preservation_of_Union_by_Images}
        If $A$ and $B$ are sets, if $f:A\rightarrow{B}$ is a function, if
        $\mathcal{P}(A)$ is the power set of $A$, and if
        $\mathcal{O}\subseteq\mathcal{P}(A)$, then:
        \begin{equation*}
            f\Big(\bigcup_{\mathcal{U}\in\mathcal{O}}\mathcal{U}\Big)
            =\bigcup_{\mathcal{U}\in\mathcal{O}}f(\mathcal{U})
        \end{equation*}
    \end{ftheorem}
    \begin{bproof}
        For if $y\in{f}(\bigcup\mathcal{U})$, then there is an
        $x\in\bigcup\mathcal{U}$ such that $y=f(x)$
        (Def.~\ref{def:Image_of_Subset}). But by the definition of union
        there is a $\mathcal{U}\in\mathcal{O}$ such that $x\in\mathcal{U}$
        (Def.~\ref{def:Union_over_a_Set}). But then
        $f(x)\in{f}(\mathcal{U})$, and therefore $y\in{f}(\mathcal{U})$.
        But $f(\mathcal{U})\subseteq\bigcup{f}(\mathcal{U})$, and thus:
        \begin{equation}
            f\Big(\bigcup_{\mathcal{U}\in\mathcal{O}}\mathcal{U}\Big)
            \subseteq\bigcup_{\mathcal{U}\in\mathcal{O}}f(\mathcal{U})
        \end{equation}
        Similarly, if $y\in\bigcup{f}(\mathcal{U})$, then there is a
        $\mathcal{U}\in\mathcal{O}$ such that $y\in{f}(\mathcal{U})$
        (Def.~\ref{def:Union_over_a_Set}). But then there is an
        $x\in\mathcal{U}$ such that $y=f(x)$ (Def.~\ref{def:Image_of_Subset}).
        But if $x\in\mathcal{U}$, then $x\in\bigcup\mathcal{U}$, and thus
        $f(x)\in\bigcup\mathcal{U}$. That is:
        \begin{equation}
            \bigcup_{\mathcal{U}\in\mathcal{O}}f(\mathcal{U})\subseteq
            f\Big(\bigcup_{\mathcal{U}\in\mathcal{O}}\mathcal{U}\Big)
        \end{equation}
        From the definition of equality (Def.~\ref{def:Equal_Sets}) we obtain
        the result.
    \end{bproof}
    These results are by no means \textit{deep} theorems, but their applications
    are spread across a myriad of subjects. So much so that many authors forget
    to thank them! Equally useful theorems relate the pre-image of set
    differences. Again, the image fails to preserve this.
    \begin{theorem}
        \label{thm:Image_of_Set_Difference}%
        If $A$ and $B$ are sets, if $f:A\rightarrow B$ is a function, and if
        $\mathcal{U},\mathcal{V}\subseteq{A}$, then:
        \begin{equation*}
            f(\mathcal{U})\setminus{f}(\mathcal{V})
            \subseteq{f}(\mathcal{U}\setminus\mathcal{V})
        \end{equation*}
    \end{theorem}
    \begin{proof}
        For suppose not. Then by the definition of subset, there is a
        $y\in{f}(\mathcal{U}\setminus\mathcal{V})$ such that
        $y\notin{f}(\mathcal{U})\setminus{f}(\mathcal{V})$
        (Def.~\ref{def:Subsets}). But by the definition of image, there is an
        $x\in\mathcal{U}\setminus\mathcal{V}$ such that $f(x)=y$
        (Def.~\ref{def:Image_of_Subset}). But if
        $x\in\mathcal{U}\setminus\mathcal{V}$, then it is true that
        $x\in\mathcal{U}$ and $x\notin\mathcal{V}$
        (Def.~\ref{def:Set_Difference}). But if $x\in\mathcal{U}$, then
        $f(x)\in{f}(\mathcal{U})$ (Def.~\ref{def:Image_of_Subset}). Similarly,
        if $x\notin\mathcal{V}$ then $f(x)\notin{f}(\mathcal{V})$. But then
        $f(x)\in{f}(\mathcal{U})$ and $f(x)\notin{f}(\mathcal{V})$, and
        therefore $f(x)\in{f}(\mathcal{U})\setminus{f}(\mathcal{V})$
        (Def.~\ref{def:Set_Difference}), a contradiction.
    \end{proof}
    \begin{example}
        The converse of Thm.~\ref{thm:Image_of_Set_Difference} may not be true
        in general. For let $f:\mathbb{N}\rightarrow\mathbb{N}$ be defined by
        $f(x)=0$. Let $\mathcal{U}=\mathbb{N}_{e}$ and
        $\mathcal{V}=\mathbb{N}_{o}$ be the even and odd non-negative integers,
        respectively. Then $f(\mathcal{U})=f(\mathcal{V})=\{0\}$, and therefore
        $f(\mathcal{U})\setminus{f}(\mathcal{V})=\emptyset$. But
        $\mathcal{U}\setminus\mathcal{V}=\mathcal{U}$ since the even and odd
        numbers are disjoint, and thus
        $f(\mathcal{U}\setminus\mathcal{V})=f(\mathcal{U})=\{0\}$, and so
        equality is not attained.
    \end{example}
    We now prove that pre-images preserve set differences.
    \begin{ftheorem}{Preservation of Set Difference by Pre-Image}
                    {Preservation_of_Set_Difference_by_Pre_Image}
        If $A$ and $B$ are sets, if $f:A\rightarrow B$ is a function, and if
        $\mathcal{U},\mathcal{V}\subseteq{B}$, then:
        \begin{equation*}
            f^{\minus{1}}(\mathcal{U}\setminus\mathcal{V})
            =f^{\minus{1}}(\mathcal{U})\setminus{f}^{\minus{1}}(\mathcal{V})
        \end{equation*}
    \end{ftheorem}
    \begin{bproof}
        For if $x\in{f}^{\minus{1}}(\mathcal{U}\setminus\mathcal{V})$, then
        $f(x)\in\mathcal{U}\setminus\mathcal{V}$
        (Def.~\ref{def:Pre_Image_of_Subset}). But then $f(x)\in\mathcal{U}$
        and $f(x)\notin\mathcal{V}$ (Def.~\ref{def:Set_Difference}). But then
        $x\in{f}^{\minus{1}}(\mathcal{U})$ and
        $x\notin{f}^{\minus{1}}(\mathcal{V})$
        (Def.~\ref{def:Pre_Image_of_Subset}), and therefore
        $x\in{f}^{\minus{1}}(\mathcal{U})\setminus{f}^{\minus{1}}(\mathcal{V})$
        (Def.~\ref{def:Set_Difference}). Thus:
        \begin{equation}
            f^{\minus{1}}(\mathcal{U}\setminus\mathcal{V})\subseteq
            f^{\minus{1}}(\mathcal{U})\setminus{f}^{\minus{1}}(\mathcal{V})
        \end{equation}
        Similarly, if
        $x\in{f}^{\minus{1}}(\mathcal{U})\setminus{f}^{\minus{1}}(\mathcal{V})$,
        then $x\in{f}^{\minus{1}}(\mathcal{U})$ and
        $x\notin{f}^{\minus{1}}(\mathcal{V})$ (Def.~\ref{def:Set_Difference}).
        But then $f(x)\in\mathcal{U}$ and $f(x)\notin\mathcal{V}$
        (Def.~\ref{def:Pre_Image_of_Subset}). Thus,
        $f(x)\in\mathcal{U}\setminus\mathcal{V}$
        (Def.~\ref{def:Set_Difference}). But then
        $x\in{f}^{\minus{1}}(\mathcal{U}\setminus\mathcal{V})$
        (Def.~\ref{def:Set_Difference}). Therefore:
        \begin{equation}
            f^{\minus{1}}(\mathcal{U})\setminus{f}^{\minus{1}}(\mathcal{V})
            \subseteq{f}^{\minus{1}}(\mathcal{U}\setminus\mathcal{V})
        \end{equation}
        Thus, by the definition of equality (Def.~\ref{def:Equal_Sets}), we are
        done.
    \end{bproof}
    The theorems presented can be summarized by saying that pre-images preserve
    the notions of inclusion, intersections, unions, and set differences,
    whereas images only preserve unions and inclusion. When we add more
    structure to function, say by adding \textit{surjectivity} or
    \textit{injectivity}, then we can show that forward images have more
    preservation properties.
    \subsection{Surjections}
        \begin{fdefinition}{Surjective Functions}{Surjective_Function}
            A \gls{surjective function} from a \gls{set} $A$ to a set $B$ is a
            \gls{function} $f:A\rightarrow{B}$ such that $f(A)=B$. That is, for
            all $y\in{B}$ there is an $x\in{A}$ such that $f(x)=y$.
            \index{Function!Surjective}
        \end{fdefinition}
        That is, every point $y\in{B}$ gets mapped to by at least one point in
        $A$. Surjective functions are also called \textit{onto}. It may also be
        true that many points in $A$ map to the same point in $B$. Excluding
        this possibility gives rise to the definition of an \textit{injective}
        function.
        \begin{example}
            Given a set $A$ we can define the \textit{identity function} on $A$
            as the function $\textrm{id}_{A}:A\rightarrow{A}$ defined by
            $\textrm{id}_{A}(x)=x$ for all $x\in{A}$. This function will be
            surjective since for all $x\in{A}$, $x$ is the image of itself under
            $\textrm{id}_{A}$. Thus every point in $A$ is mapped to by some
            point in $A$. We can think of less trivial examples if we ponder
            functions in $\mathbb{R}$. Let $f:\mathbb{R}\rightarrow\mathbb{R}$
            be defined by $f(x)=x^{3}$. That is, the cubing function. Given any
            number $y\in\mathbb{R}$, we choose $x=\sqrt[3]{y}$ (the cubed root
            of $y$), and from this we obtain $f(x)=y$. One might recall that the
            square root of a negative number is not defined on $\mathbb{R}$, and
            thus if we define $g:\mathbb{R}\rightarrow\mathbb{R}$ by
            $g(x)=x^{2}$, then $g$ is not surjective since $\minus{1}$ is not
            mapped to by any point.
        \end{example}
        \begin{example}
            The notion of a surjective function depends on both the domain and
            range of a function. For example, consider the arctan function from
            trigonometry $\tan^{\minus{1}}$. If we define this as a function
            from $\mathbb{R}$ into the interval
            $(\minus\frac{\pi}{2},\frac{\pi}{2})$, then it is indeed a
            surjection. However, if instead we write
            $\tan^{\minus{1}}:\mathbb{R}\rightarrow\mathbb{R}$, then it is no
            longer a surjection since $4\in\mathbb{R}$, but there is no real
            number $x$ such that $\tan^{\minus{1}}(x)=4$.
        \end{example}
        \begin{theorem}
            \label{thm:Comp_of_Surj_is_Surj}%
            If $A,B,C$ are sets, and if $f:A\rightarrow{B}$ and
            $g:B\rightarrow{C}$ are surjective functions, then
            $g\circ{f}:A\rightarrow{C}$ is a surjective function.
        \end{theorem}
        \begin{proof}
            For suppose not. Then there is a $z\in{C}$ such that for all
            $x\in{X}$ it is true that $f(x)\ne{z}$. But since $g$ is a
            surjection, there is a $y\in{B}$ such that $g(y)=z$
            (Def.~\ref{def:Surjective_Function}). But since $f$ is a surjection,
            if $y\in{B}$, then there is an $x\in{A}$ such that $f(x)=y$.
            But then $(g\circ{f})(x)=g(f(x))=g(y)=z$, a contradiction.
            Thus, $g\circ{f}$ is surjective.
        \end{proof}
        The converse of this theorem is partially true.
        \begin{theorem}
            If $A,B$, and $C$ are sets, if $f:A\rightarrow{B}$ and
            $g:B\rightarrow{C}$ are functions, and if
            $g\circ{f}:A\rightarrow{C}$ is surjective, then $g$ is surjective.
        \end{theorem}
        \begin{proof}
            For suppose not. Then there is a $z\in{C}$ such that for all
            $y\in{B}$ it is true that $g(y)\ne{z}$. But $g\circ{f}$ is
            surjective, and thus there is an $x\in{A}$ such that
            $(g\circ{f})(x)=z$. But $f:A\rightarrow{B}$ is a function, and thus
            there is a $y\in{B}$ such that $f(x)=y$. But then
            $(g\circ{f})(x)=g(f(x))=g(y)=z$, a contradiction. Thus, $g$ is
            surjective.
        \end{proof}
        \begin{example}
            The full converse of Thm.~\ref{thm:Comp_of_Surj_is_Surj} is not
            true, in general. For let $A=\mathbb{R}$, $B=\mathbb{R}^{2}$, and
            $C=\mathbb{R}$. Define $f$ and $g$ by:
            \par
            \begin{subequations}
                \begin{minipage}[b]{0.49\textwidth}
                    \centering
                    \begin{equation}
                        f(x)=(x,0)
                    \end{equation}
                \end{minipage}
                \hfill
                \begin{minipage}[b]{0.49\textwidth}
                    \centering
                    \begin{equation}
                        g(x,y)=x
                    \end{equation}
                \end{minipage}
            \end{subequations}
            \par\vspace{2.5ex}
            Then $g\circ{f}$ is simply the identity function. That is:
            \begin{equation}
                (g\circ{f})(x)=g(f(x))=g(x,0)=x
            \end{equation}
            and this is certainly surjective. However, $f$ is not surjective.
            To see this, note that the point $(0,1)$ is never mapped to. Thus,
            the converse of Thm.~\ref{thm:Comp_of_Surj_is_Surj} is not true.
        \end{example}
        With surjectivity we can recover the converse of
        Thm.~\ref{thm:Image_of_PreImage_is_Subset}.
        \begin{theorem}
            If $A$ and $B$ are sets, if $\mathcal{U}\subseteq{B}$, and if
            $f:A\rightarrow{B}$ is a surjective function, then:
            \begin{equation*}
                f\big(f^{\minus{1}}(\mathcal{U})\big)=\mathcal{U}
            \end{equation*}
        \end{theorem}
        \begin{proof}
            For suppose not. By Thm.~\ref{thm:Image_of_PreImage_is_Subset},
            $f\big(f^{\minus{1}}(\mathcal{U})\big)\subseteq\mathcal{U}$, and
            thus by the definition of equality (Def.~\ref{def:Equal_Sets}),
            $\mathcal{U}\nsubseteq{f}\big(f^{\minus{1}}(\mathcal{U})\big)$.
            But then there is a $y\in\mathcal{U}$ such that
            $y\notin{f}\big(f^{\minus{1}}(\mathcal{U})\big)$. But $f$ is
            surjective, and thus there is an $x\in{A}$ such that $f(x)=y$
            (Def.~\ref{def:Surjective_Function}). But if $y\in\mathcal{U}$,
            then $x\in{f}^{\minus{1}}(\mathcal{U})$
            (Def.~\ref{def:Pre_Image_of_Subset}). But if
            $x\in{f}^{\minus{1}}(\mathcal{U})$, then
            $f(x)\in{f}\big(f^{\minus{1}}(\mathcal{U})\big)$
            (Def.~\ref{def:Image_of_Subset}), and thus
            $y\in{f}\big(f^{\minus{1}}(\mathcal{U})\big)$, a contradiction.
        \end{proof}
    \subsection{Injections}
        \begin{fdefinition}{Injective Function}{Injective_Function}
            An \gls{injective function} is a function $f:A\rightarrow{B}$ such
            that for all distinct $x,y\in{A}$ it is true that $f(x)\ne{f}(y)$.
        \end{fdefinition}
        That is, an injective function is a function $f:A\rightarrow{B}$ such
        that $f(x_{1})=f(x_{2})$ if and only if $x_{1}=x_{2}$. Such functions
        are also called \textit{one-to-one}.
        \begin{example}
            Consider the natural logarithm
            $\ln:\mathbb{R}^{+}\rightarrow\mathbb{R}$. This is an injective
            function. For let $x,y\in\mathbb{R}^{+}$ be such that $x\ne{y}$.
            Suppose $\ln(x)=\ln(y)$. But then:
            \begin{equation}
                \ln(x)-\ln(y)=\ln\Big(\frac{x}{y}\Big)=0
            \end{equation}
            Recall the definition of the natural logarithm:
            \begin{equation}
                \ln(t)=\int_{1}^{t}\frac{1}{x}\diff{x}
            \end{equation}
            But then $\ln(t)=0$ if and only if $t=1$. Thus $x=y$, a
            contradiction. Therefore $\ln$ is an injective function. Not every
            function is injective, for define
            $f:\mathbb{R}\rightarrow\mathbb{R}$ by $f(x)=x^{2}$. Then, for all
            $x\in\mathbb{R}^{+}$, $f(\minus{x})=f(x)$, and thus $f$ cannot be an
            injective function.
        \end{example}
        One might think that most functions are not injective, and indeed for
        the \textit{finite} case, this is true. For let $A$ and $B$ be finite
        sets with $n$ and $m$ elements, respectively. If $m<n$, there can't be
        any injective function. Consider the case when $n=m$. Then we are simply
        counting the number of ways to permute the elements of $A$. This is
        $n!$. On the other hand, the total number of functions is $n^{n}$. Thus,
        the ratio of the number of injective functions to the number of
        functions is $n!/n^{n}$, and this decays to zero rapidly as $n$ get's
        large. Finally, if $m>n$, then the total number of injective functions
        is $n!\binom{m}{n}$, where $\binom{m}{n}$ is the binomial coefficient.
        The total number of functions is $n^{m}$. The ratio is thus:
        \begin{equation}
            \frac{n!\binom{m}{n}}{n^{m}}=\frac{n!\frac{m!}{n!(m-n)!}}{n^{m}}
                                        =\frac{m!}{(m-n)!n^{m}}
        \end{equation}
        And again, this decays rapidly to zero and $n$ and $m$
        get large. Later, when we define infinite sets
        and the notion of Cardinality, we'll show that this
        trend continues. That is, in a sense, \textit{most}
        functions from a set $A$ to a sufficiently large set
        $B$ are not injective.
    \subsection{Bijections}
        \begin{fdefinition}{Bijective Functions}{Bijective_Function}
            A \gls{bijective function} is a function that is both injective and
            surjective.
        \end{fdefinition}
        \begin{fdefinition}{Permutation}{Permutation}
            A \gls{permutation} on a \gls{set} $A$ is a \gls{bijective function}
            $f:A\rightarrow{A}$.
        \end{fdefinition}
        \begin{theorem}
            If $A$ and $B$ are set and if $f:A\rightarrow B$ is a bijective
            function, then $f^{\minus{1}}:B\rightarrow{A}$ is bijective.
        \end{theorem}
        \begin{proof}
            For $[f^{-1}(y_1) = f^{-1}(y_2)]\Rightarrow [\exists x\in A:[f(x) = y_1]\land [f(x)=y_2]]\Rightarrow [y_1=y_2]$. By definition, $f^{-1}$ is surjective.
        \end{proof}
        \begin{theorem}
            If $A,B$, and $C$ are sets, if $f:A\rightarrow{B}$ and
            $g:B\rightarrow{C}$ are bijective functions, and if
            $\mathcal{V}\subset{C}$, then
            $(g\circ g)^{-1}(\mathcal{V})=f^{-1}(g^{-1}(\mathcal{V}))$.
        \end{theorem}
        \begin{theorem}
            If $A,B$, and $C$ are sets, and if $f:A\rightarrow{B}$ and
            $g:B\rightarrow{C}$ are bijective functions, then
            $g\circ{f}:A\rightarrow{C}$ is bijective.
        \end{theorem}
        \begin{proof}
            For if $f$ and $g$ are bijective, then they are surjective
            (Def.~\ref{def:Bijective_Function}). But if $f$ and $g$ are
            surjective, then $g\circ{f}$ is surjective
            (Thm.~\ref{thm:Comp_of_Surj_is_Surj}). But if $f$ and $g$ are
            bijective, then they are injective
            (Def.~\ref{def:Bijective_Function}). But if $f$ and $g$ are
            injective, then $g\circ{f}$ is injective. Thus $g\circ{f}$ is
            surjective and injective, and is therefore bijective
            (Def.~\ref{def:Bijective_Function}).
        \end{proof}
        \begin{theorem}
        If $f:A\rightarrow B$ is bijective, $A_1\subset A$, and $f(A_1) = B$, then $A_1=A$.
        \end{theorem}
        \begin{proof}
        $\Big[\big[[A_1^c \ne \emptyset]\Rightarrow [f(A_1^c) \ne \emptyset]\big]\land[f(A_1)\cap f(A_1^c) = \emptyset]\Big]\Rightarrow [\exists y\in B:y\notin f(A_1)]$, a contradiction.
        \end{proof}