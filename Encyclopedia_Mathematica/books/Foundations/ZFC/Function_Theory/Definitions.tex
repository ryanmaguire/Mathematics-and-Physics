\section{Basic Definitions and Theorems}
    We start by presenting some basic, but crucial, facts about the image and
    pre-images of subsets under a function.
    \begin{theorem}
        \label{thm:Image_of_Empty_Set_Is_Empty}%
        If $A$ and $B$ are sets, if $f:A\rightarrow{B}$ is a function,
        then $f(\emptyset)=\emptyset$.
    \end{theorem}
    \begin{proof}
        For suppose not and suppose $y\in{f}(\emptyset)$. But then by the
        definition of the image of a subset (Def~\ref{def:Image_of_Subset})
        there is an $x\in\emptyset$ such that $f(x)=y$, a contradiction since
        for all $x$ it is true that $x\notin\emptyset$
        (Def.~\ref{def:Empty_Set}). Therefore, $f(\emptyset)=\emptyset$.
    \end{proof}
    \begin{theorem}
        \label{thm:Pre_Image_of_Empty_Set_Is_Empty}%
        If $A$ and $B$ are sets, if $f:A\rightarrow{B}$ is a function, then
        $f^{\minus{1}}(\emptyset)=\emptyset$.
    \end{theorem}
    \begin{proof}
        Suppose not and suppose $x\in{f}^{\minus{1}}(\emptyset)$. Then by the
        definition of the pre-image (Def.~\ref{def:Pre_Image_of_Subset})
        $f(x)\in\emptyset$, a contradiction since for all $y$ it is true that
        $y\notin\emptyset$ (Def.~\ref{def:Empty_Set}). Therefore,
        $f^{\minus{1}}(\emptyset)=\emptyset$.
    \end{proof}
    \begin{theorem}
        \label{thm:Image_of_PreImage_is_Subset}%
        If $X$ and $Y$ are sets, if $B\subseteq{Y}$, and if $f:X\rightarrow{Y}$
        is a function, then:
        \begin{equation}
            f\big(f^{\minus{1}}(B)\big)\subseteq{B}
        \end{equation}
    \end{theorem}
    \begin{proof}
        For if $y\in{f(f^{\minus{1}}(B))}$, then there is an
        $x\in{f^{\minus{1}}(B)}$ such that $y=f(x)$
        (Def.~\ref{def:Image_of_Subset}). But if $x\in{f^{\minus{1}}(B)}$, then
        $f(x)\in{B}$ (Def.~\ref{def:Pre_Image_of_Subset}). Thus, $y\in{B}$.
    \end{proof}
    We may not be able to attain equality.
    \begin{theorem}
        If $X$ and $Y$ are non-empty sets and if there exists
        $y_{1},y_{2}\in{Y}$ such that $y_{1}\ne{y}_{2}$, then
        there is a function $f:X\rightarrow{Y}$ and a
        $B\subseteq{Y}$ such that:
        \begin{equation}
            f\big(f^{-1}(B)\big)\ne{B}
        \end{equation}
    \end{theorem}
    \begin{proof}
        \begin{subequations}
            For if $X$ and $Y$ are non-empty, let $f:X\rightarrow{Y}$
            be defined by:
            \begin{equation}
                f=\{\,(x,y_{1})\in{X}\times{Y}\;|\;x\in{X}\,\}
            \end{equation}
            Then $f$ is a function, since $f\subseteq{X}\times{Y}$ and for all
            $x\in{X}$ there is a unique $y\in{Y}$ such that $(x,y)\in{f}$
            (Def.~\ref{def:Function}). However since for all $x\in{X}$,
            $f(x)=y_{1}$, we have:
            \begin{equation}
                f^{-1}(\{y_{2}\})=\emptyset
            \end{equation}
            For suppose not. If $x\in{f}^{-1}(\{y_{2}\})$, then $f(x)=y_{2}$
            (Def.~\ref{def:Pre_Image_of_Subset}), but for all $x\in{X}$ it is
            true that $f(x)=y_{1}$, and $y_{1}\ne{y}_{2}$, a contradiction. Thus
            $f^{-1}(\{y_{2}\})=\emptyset$ (Def.~\ref{def:Empty_Set}). But by
            Thm.~\ref{thm:Image_of_Empty_Set_Is_Empty}, the pre-image of the
            empty set is the empty set. Therefore:
            \begin{equation}
                f\big(f^{-1}(\{y_{2}\})\big)=\emptyset
            \end{equation}
            But $\{y_{2}\}\ne\emptyset$ and $\{y_{2}\}\subseteq{Y}$, completing
            the proof.
        \end{subequations}
    \end{proof}
    \begin{theorem}
        \label{thm:PreImage_of_Image_is_Superset}%
        If $A$ and $B$ are sets, if $f:A\rightarrow{B}$ is a function, and if
        $\mathcal{U}\subseteq{A}$, then:
        \begin{equation}
            \mathcal{U}\subseteq{f}^{\minus{1}}\big(f(\mathcal{U})\big)
        \end{equation}
    \end{theorem}
    \begin{proof}
        For if $x\in\mathcal{U}$, then $f(x)\in{f}(\mathcal{U})$
        (Def.~\ref{def:Image_of_Subset}). But if $f(x)\in\mathcal{U}$, then
        $x\in{f}^{\minus{1}}(f(\mathcal{U}))$
        (Def.~\ref{def:Pre_Image_of_Subset}), completing the proof.
    \end{proof}
    \begin{theorem}
        \label{thm:Image_of_NonEmpty_Set_Is_NonEmpty}%
        If $A$ and $B$ are sets, if $\mathcal{U}\subseteq{A}$ is a non-empty
        subset of $A$, and if $f:A\rightarrow{B}$ is a function, then
        $f(\mathcal{U})$ is non-empty.
    \end{theorem}
    \begin{proof}
        For suppose not. It $f(\mathcal{U})$ is empty, then
        $f^{\minus{1}}(f(\mathcal{U}))$ is empty
        (Thm.~\ref{thm:Pre_Image_of_Empty_Set_Is_Empty}). But by
        Thm.~\ref{thm:PreImage_of_Image_is_Superset},
        $\mathcal{U}\subseteq{f}^{\minus{1}}(f(\mathcal{U}))$, and thus
        $\mathcal{U}\subseteq\emptyset$. But $\emptyset\subseteq\mathcal{U}$
        (Thm.~\ref{thm:Emptyset_Is_Subset}), and thus $\mathcal{U}=\emptyset$
        (Def.~\ref{def:Equal_Sets}), a contradiction as $\mathcal{U}$ is
        non-empty.
    \end{proof}
    A nice rewording of this theorem comes from the contrapositive.
    \begin{theorem}
        If $A$ and $B$ are sets, if $\mathcal{U}\subseteq{A}$ is a non-empty,
        if $f:A\rightarrow{B}$ is a function, and if
        $f(\mathcal{U})=\emptyset$, then $\mathcal{U}=\emptyset$.
    \end{theorem}
    \begin{proof}
        Suppose not. Then $f(\mathcal{U})$ is non-empty
        (Thm.~\ref{thm:Image_of_NonEmpty_Set_Is_NonEmpty}), a contradiction.
    \end{proof}
    \begin{theorem}
        If $X$ and $Y$ are sets, if $A_{1}$ and $A_{2}$ are subsets of $X$ such
        that $A_{1}\subseteq{A}_{2}$, and if $f:X\rightarrow{Y}$ is a function,
        then:
        \begin{equation}
            f(A_{1})\subseteq{f}(A_{2})
        \end{equation}
    \end{theorem}
    \begin{proof}
        For if $y\in{f}(A_{1})$, then there is an $x\in{A}_{1}$
        such that $f(x)=y$. But $A_{1}\subseteq{A}_{2}$, and
        therefore $x\in{A}_{2}$. But if $x\in{A}_{2}$, then
        $f(x)\in{f}(A_{2})$. Thus, $y\in{f}(A_{2})$. Therefore, etc.
    \end{proof}
    \begin{theorem}
        If $X$ and $Y$ are sets, if $B_{1}$ and $B_{2}$ are subsets of
        $Y$ such that $B_{1}\subseteq{B}_{2}$, and if $f:X\rightarrow{Y}$
        is a function, then:
        \begin{equation}
            f^{-1}(B_{1})\subseteq{f^{-1}}(B_{2})
        \end{equation}
    \end{theorem}
    \begin{proof}
        For if $x\in{f}^{-1}(B_{1})$, then there is a
        $y\in{B}_{1}$ such that $f(x)=y$. But
        $B_{1}\subseteq{B}_{2}$, and therefore $y\in{B}_{2}$.
        Thus, $x\in{f}^{-1}(B_{2})$. Therefore, etc.
    \end{proof}
    \begin{ftheorem}{Preservation of Intersection by Pre-Images}
                    {Preservation_of_Intersection_by_Pre_Images}
        If $A$ and $B$ are sets, if $f:A\rightarrow{B}$ is a function, if
        $\mathcal{P}(B)$ is the power set of $B$, and if
        $\mathcal{O}\subseteq\mathcal{P}(B)$, then:
        \begin{equation*}
            f^{\minus{1}}\Big(
                \bigcap_{\mathcal{U}\in\mathcal{O}}\mathcal{U}\Big)
            =\bigcap_{\mathcal{U}\in\mathcal{O}}f^{\minus{1}}(\mathcal{U})
        \end{equation*}
    \end{ftheorem}
    \begin{bproof}
        For if
        $x\in{f}^{\minus{1}}(\bigcap_{\mathcal{U}\in\mathcal{O}}\mathcal{U})$,
        then $f(x)\in\bigcap_{\mathcal{U}\in\mathcal{O}}\mathcal{U}$
        (Def.~\ref{def:Pre_Image_of_Subset}). But then for all
        $\mathcal{U}\in\mathcal{O}$, $f(x)\in\mathcal{U}$
        (Def.~\ref{def:Intersection_Over_a_Collection}). But then for all
        $\mathcal{U}\in\mathcal{O}$, $x\in{f}^{\minus{1}}(\mathcal{U})$
        (Def.~\ref{def:Pre_Image_of_Subset}) and therefore
        $x\in\bigcap_{\mathcal{U}\in\mathcal{O}}f^{\minus{1}}(\mathcal{U})$
        (Def.~\ref{def:Intersection_Over_a_Collection}). Thus, by the definition
        of subsets:
        \begin{equation}
            f^{\minus{1}}\Big(
                \bigcap_{\mathcal{U}\in\mathcal{O}}\mathcal{U}\Big)
            \subseteq\bigcap_{\mathcal{U}\in\mathcal{O}}
                f^{\minus{1}}(\mathcal{U})
        \end{equation}
        Moreover, if
        $x\in\bigcap_{\mathcal{U}\in\mathcal{O}}f^{\minus{1}}(\mathcal{U})$ then
        for all $\mathcal{U}\in\mathcal{O}$ it is true that
        $x\in{f}^{\minus{1}}(\mathcal{U})$
        (Def.~\ref{def:Intersection_Over_a_Collection}). But then for all
        $\mathcal{U}\in\mathcal{O}$ it is true that $f(x)\in\mathcal{U}$
        (Def.~\ref{def:Pre_Image_of_Subset}) and therefore
        $f(x)\in\bigcap_{\mathcal{U}\in\mathcal{O}}\mathcal{U}$. But then
        $x\in{f}^{\minus{1}}(\bigcap_{\mathcal{U}\in\mathcal{O}}\mathcal{U})$
        (Def.~\ref{def:Pre_Image_of_Subset}) and therefore:
        \begin{equation}
            \bigcap_{\mathcal{U}\in\mathcal{O}}f^{\minus{1}}(\mathcal{U})
            \subseteq{f}^{\minus{1}}\Big(
                \bigcap_{\mathcal{U}\in\mathcal{O}}\mathcal{U}\Big)
        \end{equation}
        From the definition of equality
        (Def.~\ref{def:Equal_Sets}) we can conclude the proof.
    \end{bproof}
    While the proof of this theorem is very straight forward, we have
    highlighted it to emphasize that it is a very important and useful theorem
    that will be used frequently once we delve into measure theory and topology.
    This theorem has a counterpart for unions that is equally important.




    We cannot reverse this theorem for the pre-image. That is, the inverse image
    of a non-empty set may indeed be empty. Consider the case when
    $A=B=\mathbb{R}$ and define $f:\mathbb{R}\rightarrow\mathbb{R}$
    $f(x)=0$ for all $x\in\mathbb{R}$. Then the pre-image of any subset
    $\mathcal{U}\subseteq\mathbb{R}$ the does not contain 0 will be empty, even
    though the set itself is not empty. A set with the property that the
    pre-image of a non-empty set is non-empty is called \textit{surjective}.
    \begin{fdefinition}{Surjective Functions}{Surjective_Function}
        A \gls{surjective function} from a \gls{set} $A$ to a set $B$ is a
        \gls{function} $f:A\rightarrow{B}$ such that $f(A)=B$. That is, for all
        $y\in{B}$ there is an $x\in{A}$ such that $f(x)=y$.
        \index{Surjective Function}
    \end{fdefinition}
    That is, every point $y\in{B}$ gets mapped to by at least one point in $A$.
    Surjective functions are also called \textit{onto}. It may also be true that
    many points in $A$ map to the same point in $B$. Excluding this possibility
    gives rise to the definition of an \textit{injective} function.
    \begin{example}
        Given a set $A$ we can define the \textit{identity function} on $A$ as
        the function $\textrm{id}_{A}:A\rightarrow{A}$ defined by
        $\textrm{id}_{A}(x)=x$ for all $x\in{A}$. This function will be
        surjective since for all $x\in{A}$, $x$ is the image of itself under
        $\textrm{id}_{A}$. Thus every point in $A$ is mapped to by some point
        in $A$. We can think of less trivial examples if we ponder functions in
        $\mathbb{R}$. Let $f:\mathbb{R}\rightarrow\mathbb{R}$ be defined by
        $f(x)=x^{3}$. That is, the cubing function. Given any number
        $y\in\mathbb{R}$, we choose $x=\sqrt[3]{y}$ (the cubed root of $y$),
        and from this we obtain $f(x)=y$. One might recall that the square root
        of a negative number is not defined on $\mathbb{R}$, and thus if we
        define $g:\mathbb{R}\rightarrow\mathbb{R}$ by $g(x)=x^{2}$, then $g$ is
        not surjective since $\minus{1}$ is not mapped to by any point.
    \end{example}    
    \begin{fdefinition}{Injective Function}{Injective_Function}
        An \gls{injective function} is a function
        $f:X\rightarrow{Y}$ such that, for all
        $x,y\in{X}$ such that $x\ne{y}$, it is true that
        $f(x)\ne{f}(y)$.
    \end{fdefinition}
    That is, an injective function is a function $f:X\rightarrow{Y}$ such that
    $f(x_{1})=f(x_{2})$ if and only if $x_{1}=x_{2}$. Such functions are also
    called \textit{one-to-one}.
    \begin{example}
        Consider the natural logarithm
        $\ln:\mathbb{R}^{+}\rightarrow\mathbb{R}$. This is an injective
        function. For let $x,y\in\mathbb{R}^{+}$ be such that $x\ne{y}$.
        Suppose $\ln(x)=\ln(y)$. But then:
        \begin{equation}
            \ln(x)-\ln(y)=\ln\Big(\frac{x}{y}\Big)=0
        \end{equation}
        Recall the definition of the natural logarithm:
        \begin{equation}
            \ln(t)=\int_{1}^{t}\frac{1}{x}\diff{x}
        \end{equation}
        But then $\ln(t)=0$ if and only if $t=1$. Thus $x=y$, a contradiction.
        Therefore $\ln$ is an injective function. Not every function is
        injective, for define $f:\mathbb{R}\rightarrow\mathbb{R}$ by
        $f(x)=x^{2}$. Then, for all $x\in\mathbb{R}^{+}$, $f(\minus{x})=f(x)$,
        and thus $f$ cannot be an injective function.
    \end{example}
    One might think that most functions are not injective, and indeed for the
    \textit{finite} case, this is true. For let $A$ and $B$ be finite sets with
    $n$ and $m$ elements, respectively. If $m<n$, there can't be any injective
    function. Consider the case when $n=m$. Then we are simply counting the
    number of ways to permute the elements of $A$. This is $n!$. On the other
    hand, the total number of functions is $n^{n}$. Thus, the ratio of the
    number of injective functions to the number of functions is $n!/n^{n}$, and
    this decays to zero rapidly as $n$ get's large. Finally, if $m>n$, then the
    total number of injective functions is $n!\binom{m}{n}$, where
    $\binom{m}{n}$ is the binomial coefficient. The total number of functions is
    $n^{m}$. The ratio is thus:
    \begin{equation}
        \frac{n!\binom{m}{n}}{n^{m}}=\frac{n!\frac{m!}{n!(m-n)!}}{n^{m}}
                                    =\frac{m!}{(m-n)!n^{m}}
    \end{equation}
    And again, this decays rapidly to zero and $n$ and $m$
    get large. Later, when we define infinite sets
    and the notion of Cardinality, we'll show that this
    trend continues. That is, in a sense, \textit{most}
    functions from a set $A$ to a sufficiently large set
    $B$ are not injective.
    \begin{fdefinition}{Bijective Functions}{Bijective_Function}
        A \gls{bijective function} is a function
        that is both injective and surjective.
    \end{fdefinition}
    \begin{theorem}
    If $f:A\rightarrow B$, $A_1,A_2\subset A$, then $f(A_1 \cup A_2) = f(A_1)\cup f(A_2)$.
    \end{theorem}
    \begin{proof}
    $[y\in f(A_1\cup A_2)]\Rightarrow [\exists x\in A_1 \cup A_2:y=f(x)]\Rightarrow [y \in f(A_1)\cup f(A_2)]$. $[y\in f(A_1)\cup f(A_2)]\Rightarrow \big[[\exists x\in A_1] \lor[\exists x\in A_2]: y=f(x)\big]\Rightarrow [x\in A_1\cup A_2]\Rightarrow [f(x)\in f(A_1\cup A_2)]$
    \end{proof}
    \begin{theorem}
        If $f:A\rightarrow B$, $A_{1},A_{}2\subset A$, then
        $f(A_{1}\cap{A}_{2})\subset{f}(A_{1})\cap{f}(A_{2})$.
    \end{theorem}
    \begin{proof}
        $[y\in f(A_1 \cap A_2)]\Rightarrow [\exists x\in A_1 \cap A_2:y=f(x)]\Rightarrow [x\in A_1 \land x \in A_2] \Rightarrow[y \in f(A_1)\cap f(A_2)]$.
    \end{proof}
    \begin{theorem}
        If $A$ and $B$ are sets, $f:A\rightarrow{B}$ is a function,
        and $B_{1},B_{2}\subseteq{B}$, then:
        \begin{equation}
            f^{-1}(B_{1}\cup{B}_{2})=f^{-1}(B_{1})\cup{f}^{-1}(B_{2})
        \end{equation}
    \end{theorem}
    \begin{proof}
        For if $x\in{B}_{1}\cup{B}_{2}$, then
        $f(x)\in{B}_{1}\cup{B}_{2}$. but then either
        $f(x)\in{B}_{1}$ or $f(x)\in{B}_{2}$, and therefore
        $x\in{f}^{\minus{1}}(B_1)\cup{f}^{\minus{1}}(B_2)$. But if
        $x\in{f}^{\minus{1}}(B_{1})\cup{f}^{\minus{1}}(B_2)$, then
        $f(x)\in{B}_{1}$ or $f(x)\in{B}_{2}$. Therefore
        $f(x)\in{B}_{1}\cup{B}_{2}$. Thus, $x\in{f}^{-1}(B_1\cup{B}_2)$.
    \end{proof}
    \begin{theorem}
    If $f:A\rightarrow B$, $B_1 \subset B$, then $f^{-1}(B\setminus B_1) = f^{-1}(B)\setminus f^{-1}(B_1)$.
    \end{theorem}
    \begin{proof}
    $[x\in f^{-1}(B\setminus B_1)]\Leftrightarrow [f(x)\notin B_1]\Leftrightarrow [x\in f^{-1}(B)\setminus f^{-1}(B_1)]$
    \end{proof}
    If $f:A\rightarrow B$, the image of $A$ under $f$
    is often called the range (A is often called the domain).
    \begin{fdefinition}{Permutation}{Permutation}
        A \gls{permutation} on a \gls{set} $A$ is a \gls{bijective function}
        $f:A\rightarrow{A}$.
    \end{fdefinition}
    \begin{theorem}
    If $f:A\rightarrow B$ is bijective, then $f^{-1}$ is bijective.
    \end{theorem}
    \begin{proof}
    $[f^{-1}(y_1) = f^{-1}(y_2)]\Rightarrow [\exists x\in A:[f(x) = y_1]\land [f(x)=y_2]]\Rightarrow [y_1=y_2]$. By definition, $f^{-1}$ is surjective.
    \end{proof}
    \begin{definition}
    If $f:A\rightarrow B$ and $g:B\rightarrow C$, then $g\circ f:A\rightarrow C$ is defined by the image $g(f(x)), x\in A$. 
    \end{definition}
    \begin{theorem}
    If $f:A\rightarrow B$, $g:B\rightarrow C$, and $\mathcal{V}\subset C$, then $(g\circ g)^{-1}(\mathcal{V}) = f^{-1}(g^{-1}(\mathcal{V}))$.
    \end{theorem}
    \begin{proof}
    $[x\in (g\circ f)^{-1}(\mathcal{V})]\Leftrightarrow [g(f(x))\in \mathcal{V}] \Leftrightarrow [f(x)\in g^{-1}(\mathcal{V})]\Leftrightarrow [x\in f^{-1}(g^{-1}(\mathcal{V}))]$.
    \end{proof}
    \begin{theorem}
    If $f:A\rightarrow B$ is bijective, $g:B\rightarrow C$ is bijective, then $g\circ f$ is bijective.
    \end{theorem}
    \begin{proof}
    $\big[[f(A) = B]\land [g(B) = C]\big]\Rightarrow [g(f(A)) = g(B) = C]$. $[g(f(x_1))=g(f(x_2))]\Leftrightarrow [f(x_1)=f(x_2)]\Leftrightarrow [x_1=x_2]$.
    \end{proof}
    \begin{theorem}
    If $f:A\rightarrow B$ is bijective, $A_1\subset A$, and $f(A_1) = B$, then $A_1=A$.
    \end{theorem}
    \begin{proof}
    $\Big[\big[[A_1^c \ne \emptyset]\Rightarrow [f(A_1^c) \ne \emptyset]\big]\land[f(A_1)\cap f(A_1^c) = \emptyset]\Big]\Rightarrow [\exists y\in B:y\notin f(A_1)]$, a contradiction.
    \end{proof}