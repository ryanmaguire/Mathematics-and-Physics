\section{Definitions}
    Given a function $f:X\rightarrow{Y}$, and any non-empty subset
    $S\subseteq{X}$, the image $f(S)$ is non-empty. This is not true for the
    pre-image of a function. For let $f:\mathbb{R}\rightarrow\mathbb{R}$ be
    defined by $f(x)=1$ for all $x\in\mathbb{R}$. Then, for any subset
    $S\subset\mathbb{R}$
    such that $1\notin{S}$, we have that $f^{\minus{1}}(S)=\emptyset$.
    There are many examples of functions, but certain ones are easier
    to study than others. We give some of these special functions names.
    \begin{ldefinition}{Injective Functions}{Injective_Function}
        An \gls{injective function} is a function
        $f:X\rightarrow{Y}$ such that, for all
        $x,y\in{X}$ such that $x\ne{y}$, it is true that
        $f(x)\ne{f}(y)$.
    \end{ldefinition}
    That is, an injective function is a function
    $f:X\rightarrow{Y}$ such that $f(x_{1})=f(x_{2})$
    if and only if $x_{1}=x_{2}$. Such functions are also
    called \textit{one-to-one}.
    \begin{lexample}{}{Natural_Log_Is_Injective}
        Consider the natural logarithm
        $\ln:\mathbb{R}^{+}\rightarrow\mathbb{R}$. This is an injective
        function. For let $x,y\in\mathbb{R}^{+}$ be such that
        $x\ne{y}$. Suppose $\ln(x)=\ln(y)$. But then:
        \begin{equation}
            \ln(x)-\ln(y)=\ln\Big(\frac{x}{y}\Big)=0
        \end{equation}
        Recall the definition of the natural logarithm:
        \begin{equation}
            \ln(t)=\int_{1}^{t}\frac{1}{x}\diff{x}
        \end{equation}
        But then $\ln(t)=0$ if and only if $t=1$. Thus $x=y$, a
        contradiction. Therefore $\ln$ is an injective function. Not
        every function is injective, for define
        $f:\mathbb{R}\rightarrow\mathbb{R}$ by $f(x)=x^{2}$. Then, for
        all $x\in\mathbb{R}^{+}$, $f(\minus{x})=f(x)$, and thus $f$
        cannot be an injective function.
    \end{lexample}
    One might think that most functions are not injective,
    and indeed for the \textit{finite} case, this is true.
    For let $A$ and $B$ be finite sets with $n$ and $m$
    elements, respectively. If $m<n$, there can't be
    any injective function. Consider the case when $n=m$.
    Then we are simply counting the number of ways to
    permute the elements of $A$. This is $n!$. On the
    other hand, the total number of functions is
    $n^{n}$. Thus, the ratio of the number of injective
    functions to the number of functions is
    $n!/n^{n}$, and this decays to zero rapidly as
    $n$ get's large. Finally, if $m>n$, then the total
    number of injective functions is
    $n!\binom{m}{n}$, where $\binom{m}{n}$ is the
    binomial coefficient. The total number of functions
    is $n^{m}$. The ratio is thus:
    \begin{equation}
        \frac{n!\binom{m}{n}}{n^{m}}=\frac{n!\frac{m!}{n!(m-n)!}}{n^{m}}
                                    =\frac{m!}{(m-n)!n^{m}}
    \end{equation}
    And again, this decays rapidly to zero and $n$ and $m$
    get large. Later, when we define infinite sets
    and the notion of Cardinality, we'll show that this
    trend continues. That is, in a sense, \textit{most}
    functions from a set $A$ to a sufficiently large set
    $B$ are not injective. Next, we define
    \textit{surjective} functions.
    \begin{ldefinition}{Surjective Functions}{Surjective_Function}
        A \gls{surjective function} is a function
        $f:X\rightarrow{Y}$ such that $f(X)=Y$.
        That is, for all $y\in{Y}$, there is an
        $x\in{X}$ such that $f(x)=y$.
    \end{ldefinition}
    That is, every point $y\in{Y}$ gets mapped to by
    at least one point in $X$. It may also be true that
    many points in $X$ map to the same point in $Y$.
    The notions of surjective functions and injective
    functions are distinct, and neither implies the
    other. Surjective functions are also called
    \textit{onto}.
    \begin{ldefinition}{Bijective Functions}{Bijective_Function}
        A \gls{bijective function} is a function
        that is both injective and surjective.
    \end{ldefinition}