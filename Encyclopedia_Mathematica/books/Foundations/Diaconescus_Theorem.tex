%------------------------------------------------------------------------------%
\documentclass[crop=false,class=article]{standalone}                           %
%----------------------------Preamble------------------------------------------%
%---------------------------Packages----------------------------%
\usepackage{geometry}
\geometry{b5paper, margin=1.0in}
\usepackage[T1]{fontenc}
\usepackage{graphicx, float}            % Graphics/Images.
\usepackage{natbib}                     % For bibliographies.
\bibliographystyle{agsm}                % Bibliography style.
\usepackage[french, english]{babel}     % Language typesetting.
\usepackage[dvipsnames]{xcolor}         % Color names.
\usepackage{listings}                   % Verbatim-Like Tools.
\usepackage{mathtools, esint, mathrsfs} % amsmath and integrals.
\usepackage{amsthm, amsfonts, amssymb}  % Fonts and theorems.
\usepackage{tcolorbox}                  % Frames around theorems.
\usepackage{upgreek}                    % Non-Italic Greek.
\usepackage{fmtcount, etoolbox}         % For the \book{} command.
\usepackage[newparttoc]{titlesec}       % Formatting chapter, etc.
\usepackage{titletoc}                   % Allows \book in toc.
\usepackage[nottoc]{tocbibind}          % Bibliography in toc.
\usepackage[titles]{tocloft}            % ToC formatting.
\usepackage{pgfplots, tikz}             % Drawing/graphing tools.
\usepackage{imakeidx}                   % Used for index.
\usetikzlibrary{
    calc,                   % Calculating right angles and more.
    angles,                 % Drawing angles within triangles.
    arrows.meta,            % Latex and Stealth arrows.
    quotes,                 % Adding labels to angles.
    positioning,            % Relative positioning of nodes.
    decorations.markings,   % Adding arrows in the middle of a line.
    patterns,
    arrows
}                                       % Libraries for tikz.
\pgfplotsset{compat=1.9}                % Version of pgfplots.
\usepackage[font=scriptsize,
            labelformat=simple,
            labelsep=colon]{subcaption} % Subfigure captions.
\usepackage[font={scriptsize},
            hypcap=true,
            labelsep=colon]{caption}    % Figure captions.
\usepackage[pdftex,
            pdfauthor={Ryan Maguire},
            pdftitle={Mathematics and Physics},
            pdfsubject={Mathematics, Physics, Science},
            pdfkeywords={Mathematics, Physics, Computer Science, Biology},
            pdfproducer={LaTeX},
            pdfcreator={pdflatex}]{hyperref}
\hypersetup{
    colorlinks=true,
    linkcolor=blue,
    filecolor=magenta,
    urlcolor=Cerulean,
    citecolor=SkyBlue
}                           % Colors for hyperref.
\usepackage[toc,acronym,nogroupskip,nopostdot]{glossaries}
\usepackage{glossary-mcols}
%------------------------Theorem Styles-------------------------%
\theoremstyle{plain}
\newtheorem{theorem}{Theorem}[section]

% Define theorem style for default spacing and normal font.
\newtheoremstyle{normal}
    {\topsep}               % Amount of space above the theorem.
    {\topsep}               % Amount of space below the theorem.
    {}                      % Font used for body of theorem.
    {}                      % Measure of space to indent.
    {\bfseries}             % Font of the header of the theorem.
    {}                      % Punctuation between head and body.
    {.5em}                  % Space after theorem head.
    {}

% Italic header environment.
\newtheoremstyle{thmit}{\topsep}{\topsep}{}{}{\itshape}{}{0.5em}{}

% Define environments with italic headers.
\theoremstyle{thmit}
\newtheorem*{solution}{Solution}

% Define default environments.
\theoremstyle{normal}
\newtheorem{example}{Example}[section]
\newtheorem{definition}{Definition}[section]
\newtheorem{problem}{Problem}[section]

% Define framed environment.
\tcbuselibrary{most}
\newtcbtheorem[use counter*=theorem]{ftheorem}{Theorem}{%
    before=\par\vspace{2ex},
    boxsep=0.5\topsep,
    after=\par\vspace{2ex},
    colback=green!5,
    colframe=green!35!black,
    fonttitle=\bfseries\upshape%
}{thm}

\newtcbtheorem[auto counter, number within=section]{faxiom}{Axiom}{%
    before=\par\vspace{2ex},
    boxsep=0.5\topsep,
    after=\par\vspace{2ex},
    colback=Apricot!5,
    colframe=Apricot!35!black,
    fonttitle=\bfseries\upshape%
}{ax}

\newtcbtheorem[use counter*=definition]{fdefinition}{Definition}{%
    before=\par\vspace{2ex},
    boxsep=0.5\topsep,
    after=\par\vspace{2ex},
    colback=blue!5!white,
    colframe=blue!75!black,
    fonttitle=\bfseries\upshape%
}{def}

\newtcbtheorem[use counter*=example]{fexample}{Example}{%
    before=\par\vspace{2ex},
    boxsep=0.5\topsep,
    after=\par\vspace{2ex},
    colback=red!5!white,
    colframe=red!75!black,
    fonttitle=\bfseries\upshape%
}{ex}

\newtcbtheorem[auto counter, number within=section]{fnotation}{Notation}{%
    before=\par\vspace{2ex},
    boxsep=0.5\topsep,
    after=\par\vspace{2ex},
    colback=SeaGreen!5!white,
    colframe=SeaGreen!75!black,
    fonttitle=\bfseries\upshape%
}{not}

\newtcbtheorem[use counter*=remark]{fremark}{Remark}{%
    fonttitle=\bfseries\upshape,
    colback=Goldenrod!5!white,
    colframe=Goldenrod!75!black}{ex}

\newenvironment{bproof}{\textit{Proof.}}{\hfill$\square$}
\tcolorboxenvironment{bproof}{%
    blanker,
    breakable,
    left=3mm,
    before skip=5pt,
    after skip=10pt,
    borderline west={0.6mm}{0pt}{green!80!black}
}

\AtEndEnvironment{lexample}{$\hfill\textcolor{red}{\blacksquare}$}
\newtcbtheorem[use counter*=example]{lexample}{Example}{%
    empty,
    title={Example~\theexample},
    boxed title style={%
        empty,
        size=minimal,
        toprule=2pt,
        top=0.5\topsep,
    },
    coltitle=red,
    fonttitle=\bfseries,
    parbox=false,
    boxsep=0pt,
    before=\par\vspace{2ex},
    left=0pt,
    right=0pt,
    top=3ex,
    bottom=1ex,
    before=\par\vspace{2ex},
    after=\par\vspace{2ex},
    breakable,
    pad at break*=0mm,
    vfill before first,
    overlay unbroken={%
        \draw[red, line width=2pt]
            ([yshift=-1.2ex]title.south-|frame.west) to
            ([yshift=-1.2ex]title.south-|frame.east);
        },
    overlay first={%
        \draw[red, line width=2pt]
            ([yshift=-1.2ex]title.south-|frame.west) to
            ([yshift=-1.2ex]title.south-|frame.east);
    },
}{ex}

\AtEndEnvironment{ldefinition}{$\hfill\textcolor{Blue}{\blacksquare}$}
\newtcbtheorem[use counter*=definition]{ldefinition}{Definition}{%
    empty,
    title={Definition~\thedefinition:~{#1}},
    boxed title style={%
        empty,
        size=minimal,
        toprule=2pt,
        top=0.5\topsep,
    },
    coltitle=Blue,
    fonttitle=\bfseries,
    parbox=false,
    boxsep=0pt,
    before=\par\vspace{2ex},
    left=0pt,
    right=0pt,
    top=3ex,
    bottom=0pt,
    before=\par\vspace{2ex},
    after=\par\vspace{1ex},
    breakable,
    pad at break*=0mm,
    vfill before first,
    overlay unbroken={%
        \draw[Blue, line width=2pt]
            ([yshift=-1.2ex]title.south-|frame.west) to
            ([yshift=-1.2ex]title.south-|frame.east);
        },
    overlay first={%
        \draw[Blue, line width=2pt]
            ([yshift=-1.2ex]title.south-|frame.west) to
            ([yshift=-1.2ex]title.south-|frame.east);
    },
}{def}

\AtEndEnvironment{ltheorem}{$\hfill\textcolor{Green}{\blacksquare}$}
\newtcbtheorem[use counter*=theorem]{ltheorem}{Theorem}{%
    empty,
    title={Theorem~\thetheorem:~{#1}},
    boxed title style={%
        empty,
        size=minimal,
        toprule=2pt,
        top=0.5\topsep,
    },
    coltitle=Green,
    fonttitle=\bfseries,
    parbox=false,
    boxsep=0pt,
    before=\par\vspace{2ex},
    left=0pt,
    right=0pt,
    top=3ex,
    bottom=-1.5ex,
    breakable,
    pad at break*=0mm,
    vfill before first,
    overlay unbroken={%
        \draw[Green, line width=2pt]
            ([yshift=-1.2ex]title.south-|frame.west) to
            ([yshift=-1.2ex]title.south-|frame.east);},
    overlay first={%
        \draw[Green, line width=2pt]
            ([yshift=-1.2ex]title.south-|frame.west) to
            ([yshift=-1.2ex]title.south-|frame.east);
    }
}{thm}

%--------------------Declared Math Operators--------------------%
\DeclareMathOperator{\adjoint}{adj}         % Adjoint.
\DeclareMathOperator{\Card}{Card}           % Cardinality.
\DeclareMathOperator{\curl}{curl}           % Curl.
\DeclareMathOperator{\diam}{diam}           % Diameter.
\DeclareMathOperator{\dist}{dist}           % Distance.
\DeclareMathOperator{\Div}{div}             % Divergence.
\DeclareMathOperator{\Erf}{Erf}             % Error Function.
\DeclareMathOperator{\Erfc}{Erfc}           % Complementary Error Function.
\DeclareMathOperator{\Ext}{Ext}             % Exterior.
\DeclareMathOperator{\GCD}{GCD}             % Greatest common denominator.
\DeclareMathOperator{\grad}{grad}           % Gradient
\DeclareMathOperator{\Ima}{Im}              % Image.
\DeclareMathOperator{\Int}{Int}             % Interior.
\DeclareMathOperator{\LC}{LC}               % Leading coefficient.
\DeclareMathOperator{\LCM}{LCM}             % Least common multiple.
\DeclareMathOperator{\LM}{LM}               % Leading monomial.
\DeclareMathOperator{\LT}{LT}               % Leading term.
\DeclareMathOperator{\Mod}{mod}             % Modulus.
\DeclareMathOperator{\Mon}{Mon}             % Monomial.
\DeclareMathOperator{\multideg}{mutlideg}   % Multi-Degree (Graphs).
\DeclareMathOperator{\nul}{nul}             % Null space of operator.
\DeclareMathOperator{\Ord}{Ord}             % Ordinal of ordered set.
\DeclareMathOperator{\Prin}{Prin}           % Principal value.
\DeclareMathOperator{\proj}{proj}           % Projection.
\DeclareMathOperator{\Refl}{Refl}           % Reflection operator.
\DeclareMathOperator{\rk}{rk}               % Rank of operator.
\DeclareMathOperator{\sgn}{sgn}             % Sign of a number.
\DeclareMathOperator{\sinc}{sinc}           % Sinc function.
\DeclareMathOperator{\Span}{Span}           % Span of a set.
\DeclareMathOperator{\Spec}{Spec}           % Spectrum.
\DeclareMathOperator{\supp}{supp}           % Support
\DeclareMathOperator{\Tr}{Tr}               % Trace of matrix.
%--------------------Declared Math Symbols--------------------%
\DeclareMathSymbol{\minus}{\mathbin}{AMSa}{"39} % Unary minus sign.
%------------------------New Commands---------------------------%
\DeclarePairedDelimiter\norm{\lVert}{\rVert}
\DeclarePairedDelimiter\ceil{\lceil}{\rceil}
\DeclarePairedDelimiter\floor{\lfloor}{\rfloor}
\newcommand*\diff{\mathop{}\!\mathrm{d}}
\newcommand*\Diff[1]{\mathop{}\!\mathrm{d^#1}}
\renewcommand*{\glstextformat}[1]{\textcolor{RoyalBlue}{#1}}
\renewcommand{\glsnamefont}[1]{\textbf{#1}}
\renewcommand\labelitemii{$\circ$}
\renewcommand\thesubfigure{%
    \arabic{chapter}.\arabic{figure}.\arabic{subfigure}}
\addto\captionsenglish{\renewcommand{\figurename}{Fig.}}
\numberwithin{equation}{section}

\renewcommand{\vector}[1]{\boldsymbol{\mathrm{#1}}}

\newcommand{\uvector}[1]{\boldsymbol{\hat{\mathrm{#1}}}}
\newcommand{\topspace}[2][]{(#2,\tau_{#1})}
\newcommand{\measurespace}[2][]{(#2,\varSigma_{#1},\mu_{#1})}
\newcommand{\measurablespace}[2][]{(#2,\varSigma_{#1})}
\newcommand{\manifold}[2][]{(#2,\tau_{#1},\mathcal{A}_{#1})}
\newcommand{\tanspace}[2]{T_{#1}{#2}}
\newcommand{\cotanspace}[2]{T_{#1}^{*}{#2}}
\newcommand{\Ckspace}[3][\mathbb{R}]{C^{#2}(#3,#1)}
\newcommand{\funcspace}[2][\mathbb{R}]{\mathcal{F}(#2,#1)}
\newcommand{\smoothvecf}[1]{\mathfrak{X}(#1)}
\newcommand{\smoothonef}[1]{\mathfrak{X}^{*}(#1)}
\newcommand{\bracket}[2]{[#1,#2]}

%------------------------Book Command---------------------------%
\makeatletter
\renewcommand\@pnumwidth{1cm}
\newcounter{book}
\renewcommand\thebook{\@Roman\c@book}
\newcommand\book{%
    \if@openright
        \cleardoublepage
    \else
        \clearpage
    \fi
    \thispagestyle{plain}%
    \if@twocolumn
        \onecolumn
        \@tempswatrue
    \else
        \@tempswafalse
    \fi
    \null\vfil
    \secdef\@book\@sbook
}
\def\@book[#1]#2{%
    \refstepcounter{book}
    \addcontentsline{toc}{book}{\bookname\ \thebook:\hspace{1em}#1}
    \markboth{}{}
    {\centering
     \interlinepenalty\@M
     \normalfont
     \huge\bfseries\bookname\nobreakspace\thebook
     \par
     \vskip 20\p@
     \Huge\bfseries#2\par}%
    \@endbook}
\def\@sbook#1{%
    {\centering
     \interlinepenalty \@M
     \normalfont
     \Huge\bfseries#1\par}%
    \@endbook}
\def\@endbook{
    \vfil\newpage
        \if@twoside
            \if@openright
                \null
                \thispagestyle{empty}%
                \newpage
            \fi
        \fi
        \if@tempswa
            \twocolumn
        \fi
}
\newcommand*\l@book[2]{%
    \ifnum\c@tocdepth >-3\relax
        \addpenalty{-\@highpenalty}%
        \addvspace{2.25em\@plus\p@}%
        \setlength\@tempdima{3em}%
        \begingroup
            \parindent\z@\rightskip\@pnumwidth
            \parfillskip -\@pnumwidth
            {
                \leavevmode
                \Large\bfseries#1\hfill\hb@xt@\@pnumwidth{\hss#2}
            }
            \par
            \nobreak
            \global\@nobreaktrue
            \everypar{\global\@nobreakfalse\everypar{}}%
        \endgroup
    \fi}
\newcommand\bookname{Book}
\renewcommand{\thebook}{\texorpdfstring{\Numberstring{book}}{book}}
\providecommand*{\toclevel@book}{-2}
\makeatother
\titleformat{\part}[display]
    {\Large\bfseries}
    {\partname\nobreakspace\thepart}
    {0mm}
    {\Huge\bfseries}
\titlecontents{part}[0pt]
    {\large\bfseries}
    {\partname\ \thecontentslabel: \quad}
    {}
    {\hfill\contentspage}
\titlecontents{chapter}[0pt]
    {\bfseries}
    {\chaptername\ \thecontentslabel:\quad}
    {}
    {\hfill\contentspage}
\newglossarystyle{longpara}{%
    \setglossarystyle{long}%
    \renewenvironment{theglossary}{%
        \begin{longtable}[l]{{p{0.25\hsize}p{0.65\hsize}}}
    }{\end{longtable}}%
    \renewcommand{\glossentry}[2]{%
        \glstarget{##1}{\glossentryname{##1}}%
        &\glossentrydesc{##1}{~##2.}
        \tabularnewline%
        \tabularnewline
    }%
}
\newglossary[not-glg]{notation}{not-gls}{not-glo}{Notation}
\newcommand*{\newnotation}[4][]{%
    \newglossaryentry{#2}{type=notation, name={\textbf{#3}, },
                          text={#4}, description={#4},#1}%
}
%--------------------------LENGTHS------------------------------%
% Spacings for the Table of Contents.
\addtolength{\cftsecnumwidth}{1ex}
\addtolength{\cftsubsecindent}{1ex}
\addtolength{\cftsubsecnumwidth}{1ex}
\addtolength{\cftfignumwidth}{1ex}
\addtolength{\cfttabnumwidth}{1ex}

% Indent and paragraph spacing.
\setlength{\parindent}{0em}
\setlength{\parskip}{0em}                                                           %
%--------------------------Main Document---------------------------------------%
\begin{document}
    \title{Diaconescu's Theorem}
    \author{Ryan Maguire}
    \date{\vspace{-5ex}}
    \maketitle
    \tableofcontents
    \vspace{10ex}
    \section{Russell's Paradox}
        Suppose we allow as an axiom of set theory that, given some
        proposition $P$, there is a set $A$ defined by all of the things
        $x$ such that $P(x)$ is true. In set-builder notation, we have:
        \begin{equation}
            A=\{\,x\,:\,P(x)\,\}
        \end{equation}
        This quickly leads to contradictions, showing that such a naive
        version of set theory is inconsistent. For let $P$ be the
        proposition \textit{true if} $x\notin{x}$. That is, $P(x)$ is true if
        $x$ does not contain itself. These were occasionally called
        \textit{proper} sets. Let $B$ be the set of all sets that
        that $P(x)$ is true. The problem arises when we ask if $B\in{B}$.
        \begin{ftheorem}{Russell's Paradox}{Russells_Paradox}
            Naive Set Theory is inconsistent.
        \end{ftheorem}
        \begin{bproof}
            For let $P$ be the proposition
            \textit{true if x is a proper set, false otherwise}, and let $B$ be
            the set of all $x$ such that $P(x)$ is true. If $B\in{B}$ then
            $B$ is not a proper set, and thus $P(B)$ is false. But $B$ is the
            set of all sets such that $P$ is true, and thus $P(B)$ is true, a
            contradiction. Therefore $B\notin{B}$. But if $B\notin{B}$ then
            $P(B)$ is false. But if $P(B)$ is false, then $B\in{B}$, a
            contradiction. Thus $B\in{B}$ if and only if $B\notin{B}$, a
            contradiction. Therefore, naive set theory is inconsistent.
        \end{bproof}
        Since the proof of this theorem relies on the law of the excluded middle,
        at the very least naive set theory is incompatible with this rule. Since
        this is the basis of \textit{proof by contradiction}, it would be nice
        to frame the axioms of set theory in such a way as to allow one to
        consistently use this law. This is achieved by the system known as
        Zermelo-Fraenkel set theory. We will combine this with the axiom of
        choice and prove that the law of the excluded middle is then a theorem.
    \section{Zermelo-Fraenkel Set Theory}
        Before we start proving things we will need the existence of
        \textit{something}. So, we start by stating the existence of nothing.
        \begin{faxiom}{Axiom of the Empty Set}{Axiom_of_the_Empty_Set}
            There exists a set $\emptyset$ (the empty set) such that for
            all $x$ it is true that $x\notin\emptyset$.
        \end{faxiom}
        To get something from nothing we will need to be able to build new sets
        from the empty set. One way is to simply consider the set
        $\{\,\emptyset\,\}$, and then we can consider
        $\{\,\emptyset,\,\{\,\emptyset\,\}\,\}$, and so on. Such initial
        constructions are justified by the axiom of the power set.
        \begin{faxiom}{Axiom of the Power Set}{Axiom_of_the_Power_Set}
            If $A$ is a set, then there exists a set $\mathcal{P}(A)$ such that,
            for all $x$, $x\in\mathcal{P}(A)$ if and only if $x\subseteq{A}$.
        \end{faxiom}
        Using Ax.~\ref{ax:Axiom_of_the_Empty_Set} and
        Ax.~\ref{ax:Axiom_of_the_Power_Set}, we can create a set that has at
        least one element, namely $\mathcal{P}(\emptyset)$. Let us label this
        set as 0. Similarly, we can create a new set $\mathcal{P}(0)$. Label
        this as 1. Next, we build the sets $\{\,0\,\}$ and $\{\,0,\,1,\}$.
        This requires the axioms of pairing and specification.
        \begin{faxiom}{Axiom of Pairing}{Axiom_of_Pairing}
            If $A$ and $B$ are sets, then there exists a set
            $\mathcal{O}$ such that $A\in\mathcal{O}$ and $B\in\mathcal{O}$.
        \end{faxiom}
        Combining this with specification allows for many powerful constructions
        in set theory, many of which are rightly taken for granted.
        \begin{faxiom}{Axiom of Specification}{Axiom_of_Specification}
            If $B$ is a set and $P$ is a proposition, then there is a set
            $A$ such that $x\in{A}$ if and only if $x\in{B}$ and
            $P(x)$ is true.
        \end{faxiom}
        This looks strikingly familiar to the axiom of naive set theory, which
        we have seen is inconsistent. The crucial difference is that we are
        constructing the new set $A$ as a subset of another given set. Before
        we allowed for our sets to be constructed by any definable rule or
        proposition, and this allowed for \textit{very} large objects and led to
        contradiction. With the axiom of pairing and specification we can
        now construct the set $\{\,0,\,1\,\}$.
        \begin{ltheorem}{}{Union_of_Two_Exists}
            If $A$ and $B$ are sets, then there exists a set $C$ such that
            $x\in{A}$ if and only if $x=A$ or $x=B$. That is,
            $C=\{\,A,\,B\,\}$.
        \end{ltheorem}
        \begin{proof}
            For by the axiom of pairing (Ax.~\ref{ax:Axiom_of_Pairing}), there
            exists a set $\mathcal{O}$ such that $A\in\mathcal{O}$ and
            $B\in\mathcal{O}$. Let $P$ be the proposition
            \textit{true if} $x=A$ \textit{or} $x=B$. By the axiom of
            specification (Ax.~\ref{ax:Axiom_of_Specification}) there exists a
            set $C$ such that $x\in\mathcal{O}$ and $P(x)$ is true. But then
            $x\in{C}$ if and only if $x=A$ or $x=B$. This completes the proof.
        \end{proof}
        By letting $A=B=0$ we obtain the set $\{\,0\,\}$, and by letting
        $A=0$ and $B=1$ we obtain $\{\,0,\,1\,\}$. There are a few more axioms
        worth mentioning.
        \begin{faxiom}{Axiom of Extensionality}{Axiom_of_Extensionality}
            If $A$ and $B$ are sets, and if for all $x$, $x\in{A}$ if and only
            if $x\in{B}$, then $A$ and $B$ are equal. That is, $A=B$.
        \end{faxiom}
        Combining this with the axiom of regularity allows us to discern which
        sets are equal and which are not.
        \begin{faxiom}{Axiom of Regularity}{Axiom_of_Regularity}
            If $A$ is a non-empty set, then there is a $B\in{A}$ such that
            $A\cap{B}=\emptyset$.
        \end{faxiom}
        While the axiom of regularity does seem bizarre, it's wording is to
        explicitly prevent statements like \textit{the set of all sets}. This
        also implies that, given a set $A$, $A\notin{A}$. Thus we've avoided
        Russell's paradox. It also shows that $0$ and $1$ are different objects
        since $0\in{1}$, by the definition of the power set, and thus by
        regularity and extension, $0\ne{1}$. So the sets $\{\,0\,\}$ and
        $\{\,0,\,1\,\}$ are also different. We wish discuss the axiom of choice
        next. To do this requires the notion of unions over arbitrary sets.
        This is achieved by the axiom of union.
        \begin{faxiom}{Axiom of Union}{Axiom_of_Union}
            If $\mathcal{O}$ is a non-empty set, then there exists a set
            $\mathcal{F}$ such that $x\in\mathcal{F}$ if and only if there is
            a set $\mathcal{U}\in\mathcal{O}$ such that $x\in\mathcal{U}$.
        \end{faxiom}
        To frame the axiom of union in clearer language, given a collection of
        sets $\mathcal{O}$, the set $\mathcal{F}$ defined by:
        \begin{equation}
            \mathcal{F}=\bigcup_{\mathcal{U}\in\mathcal{O}}\mathcal{U}
        \end{equation}
        is a valid and well defined set. Finally, we introduce the
        controversial \textit{axiom of choice}.
        \begin{faxiom}{Axiom of Choice}{Axiom_of_Choice}
            If $\mathcal{O}$ is a non-empty set, and if
            $\mathcal{F}=\bigcup_{\mathcal{U}\in\mathcal{O}}\mathcal{U}$,
            then there is a function $f:\mathcal{O}\rightarrow\mathcal{U}$ such
            that, for all $x\in\mathcal{O}$, $f(x)\in{x}$
        \end{faxiom}
        Such a function is called a choice function. The axiom can be made
        obviously true if we word it one way, and obviously false if we word it
        another. To show that it's obvious will require talking about products.
        The Cartesian product can be easily defined using the above axioms
        if we define order pairs by:
        \begin{equation}
            (a,\,b)=\{\,\{\,a\,\},\,\{\,a,\,b\,\}\,\}
        \end{equation}
        Applying the axiom of the power set twice gives us the existence
        of the Cartesian product $A\times{B}$, which is the set of all
        ordered pairs $(a,b)$ such that $a\in{A}$ and $b\in{B}$. If we wanted
        to define ordered $n$ tuples we could construct some means of doing
        this in a similar fashion to the above equation, but instead we will
        frame this in terms of functions. We will need the existence of
        $\mathbb{N}$.
        \begin{faxiom}{The Axiom of Infinity}{Axiom_of_Infinity}
            There exists a set $\mathcal{X}$ such that $\emptyset\in\mathcal{X}$
            and for all $x\in\mathcal{X}$, $x\cup\{\,x\,\}\in\mathcal{X}$.
        \end{faxiom}
        By using the axiom of specification and the axiom of infinity, we take
        $\mathbb{N}$ to be the \textit{smallest} subset of $\mathcal{X}$ such
        that $\emptyset\in\mathbb{N}$ and for all $n\in\mathbb{N}$,
        $n\cup\{\,n\,\}\in\mathbb{N}$. We write $n\cup\{\,n\,\}$ as
        $n+1$, and from there the entirety of number theory is born. We've also
        welled ordered $\mathbb{N}$ if we take $n\leq{m}$ to mean that
        $n\subseteq{m}$. We define $\mathbb{Z}_{n}$ to be:
        \begin{equation}
            \mathbb{Z}_{n}=\{\,m\in\mathbb{N}\,:\,m<n\,\}
        \end{equation}
        This is justified by the axiom of specification. To define ordered
        $n$-tuples, given a collection $\mathcal{O}$ of $n$ sets $A_{k}$,
        we can define the product set as:
        \begin{equation}
            A=\{\,f:\mathbb{Z}_{n}\rightarrow\bigcup_{k=1}^{n}A_{k}\,:\,
                  f(k)\in{A}_{k}\,\}
        \end{equation}
        This is very similar to how Cartesian products were defined: The set of
        all ordered pairs whose first entry is in the first set, and whose
        second entry lies in the second set. Now we have the set of all
        functions on $\mathbb{Z}_{n}$ such that $k$ maps to the $k^{th}$ set
        in the product. There was no restriction on the size of the product,
        and we can allow for products over arbitrarily large collections of
        sets. The axiom of choice is thus equivalent to the statement
        \textit{The infinite product of non-empty sets is non-empty}. These
        functions that identify $k$ with the $k^{th}$ set are precisely
        choice functions.
        \par\hfill\par
        We conclude by presenting Diaconescu's theorem, which states that
        Zermelo-Fraenkel set theory, together with the axiom of choice
        (commonly abbreviated ZFC), implies the law of the excluded middle.
    \section{Diaconescu's Theorem}
        \begin{ftheorem}{Diaconescu's Theorem}{Diaconescus_Theorem}
            If $P$ is a proposition on sets and if $x$ is a set, then either
            $P(x)$ is true or $P(x)$ is false. That is, $P\lor\neg{P}$ is true.
        \end{ftheorem}
        \begin{bproof}
            We have proven that the set $\{\,0,\,1\,\}$ exists. Let $P$ be a
            proposition on sets and let $\mathcal{U}$ be defined by:
            \begin{equation}
                \mathcal{U}=\{\,x\in\{\,0,\,1\,\}\,:\,(x=0)\lor{P}\,\}
            \end{equation}
            Such a set exists by the axiom of specification. Similarly, let
            $\mathcal{V}$ be defined by:
            \begin{equation}
                \mathcal{V}=\{\,x\in\{\,0,\,1\,\}\,:\,(x=1)\lor{P}\,\}
            \end{equation}
            By Thm.~\ref{thm:Union_of_Two_Exists}, we have that the set
            $\{\,\mathcal{U},\,\mathcal{V}\,\}$ exists. By the axiom of
            choice, there exists a choice function such that
            $f(\mathcal{U})\in\mathcal{U}$ and $f(\mathcal{V})\in\mathcal{V}$.
            But then, by the definition of $\mathcal{U}$, either
            $f(\mathcal{U})=0$ or $P$ is true. Similarly, either
            $f(\mathcal{V})=1$ or $P$ is true. But then, by the axiom of
            extensionality, either $f(\mathcal{U})\ne{f}(\mathcal{V})$ or
            $P$ is true. Again by the axiom of extensionality, and by the
            definition of $\mathcal{U}$ and $\mathcal{V}$, if $P$ is true
            then $\mathcal{U}=\mathcal{V}$. But then
            $f(\mathcal{U})=f(\mathcal{V})$. But then, by the contrapositive,
            $\neg{P}$ implies that $f(\mathcal{U})\ne{f}(\mathcal{V})$. But
            by extensionality, either $f(\mathcal{U})=f(\mathcal{V})$ or
            $(\mathcal{U})\ne{f}(\mathcal{V})$, and thus either $P$ or
            $\neg{P}$. That is, $P\lor\neg{P}$ is true.
        \end{bproof}
\end{document}