\section{Boolean Algebra}
    We now attempt to make set theory more algebraic. We wish to model
    as an object the triple $(\mathcal{P}(X),\cup,\cap)$, where
    $\mathcal{P}(X)$ is the \gls{power set} of some set, and $\cup$ and
    $\cap$ and union and intersection, respectively. These can be seen as
    binary operations on $\mathcal{P}(X)$. We take a few of the properties
    of this structure and state them as the definition for our new object:
    \textit{Boolean Algebras}
    \begin{fdefinition}{Boolean Algebras}{Boolean_Algebra}
        A Boolean algebra is a set $X$ with two
        \glslink{commutative operation}{commutative} \glspl{binary operation}
        $\circ$ and $*$ on $X$ with \glspl{unital element} $e_{*}$ and
        $e_{\circ}$, respectively, such that:
        \begin{enumerate}
            \item   $\circ$ \glslink{distributive operation}{distributes}
                    over $*$.
            \item   $*$ distributes over $\circ$.
            \item   For all $a\in{X}$ there is an $a_{*}^{\minus{1}}\in{X}$
                    such that $a*b_{*}=e_{\circ}$.
            \item   For all $a\in{X}$ there is a $a_{\circ}^{\minus{1}}\in{X}$
                    such that $a\circ{b}_{\circ}=e_{*}$.
        \end{enumerate}
    \end{fdefinition}
    \begin{theorem}
        If $(X,\circ,*)$ is a Boolean algebra and if $b\in{X}$ is a
        unital element of $\circ$, then $b=e_{\circ}$.
    \end{theorem}
    \begin{proof}
        For if $b$ is a unital element of $*$, then since $e_{*}$ is a
        unital element we have:
        \begin{equation}
            b=b*e_{*}=e_{*}
        \end{equation}
        And therefore $b=e_{*}$
    \end{proof}
    Similarly, $e_{\circ}$ is unique.
    \begin{theorem}
        If $(X,\circ,*)$ is a Boolean algebra, if $e_{*}$ is the unital
        element of $*$, and if $a\in{X}$, then $a\circ{e}_{*}=e_{*}$.
    \end{theorem}
    \begin{proof}
        For if $a\in{X}$ then there is an $a_{\circ}^{\minus{1}}\in{X}$ such
        that $a\circ{a}_{\circ}^{\minus{1}}=e_{*}$
        (Def.~\ref{def:Boolean_Algebra}). But then:
        \begin{align}
            e_{*}&=a\circ{a}_{\circ}^{\minus{1}}
            \tag{Complement Property}\\
            &=a\circ(a_{\circ}^{\minus{1}}*e_{*})
            \tag{Identity}\\
            &=(a\circ{a}_{\circ}^{\minus{1}})*(a\circ{e}_{*})
            \tag{Distributivity}\\
            &=e_{*}*(a\circ{e}_{*})
            \tag{Complement Property}\\
            &=a\circ{e}_{*}
            \tag{Identity}
        \end{align}
        And therefore $e_{*}=a\circ{e}_{*}$.
    \end{proof}
    \begin{theorem}
        \label{thm:Bool_Comp_of_First_Op_is_Comp_of_Second_Op}
        If $(X,*,\circ)$ is a Boolean algebra, if $a\in{X}$, if
        $e_{*}$ and $e_{\circ}$ are the unital element of $*$ and $\circ$,
        respectively, if $a_{*}^{\minus{1}}\in{X}$ is such that
        $a*a_{*}^{\minus{1}}=e_{*}$, and if $a_{\circ}^{\minus{1}}\in{X}$ is
        such that $a\circ{a}_{\circ}^{\minus{1}}=e_{\circ}$, then
        $a_{\circ}^{\minus{1}}=a_{*}^{\minus{1}}$.
    \end{theorem}
    \begin{proof}
        For:
        \begin{align}
            a_{\circ}^{\minus{1}}
            &=a_{\circ}^{\minus{1}}\circ{e}_{\circ}
            \tag{Identity}\\
            &=a_{\circ}^{\minus{1}}\circ(a*a_{*}^{\minus{1}})
            \tag{Complement Property}\\
            &=(a_{\circ}^{\minus{1}}\circ{a})*
              (a_{\circ}^{\minus{1}}\circ{a}_{*}^{\minus{1}})
            \tag{Distributivity}\\
            &=e_{*}*(a_{\circ}^{\minus{1}}\circ{a}_{*}^{\minus{1}})
            \tag{Complement Property}\\
            &=a_{\circ}^{\minus{1}}\circ{a}_{*}^{\minus{1}}
            \tag{Identity}\\
            &=a_{*}^{\minus{1}}\circ{a}_{\circ}^{\minus{1}}
            \tag{Commutativity}\\
            &=e_{*}*(a_{*}^{\minus{1}}\circ{a}_{\circ}^{\minus{1}})
            \tag{Identity}\\
            &=(a_{*}^{\minus{1}}\circ{a})*
              (a_{*}^{\minus{1}}\circ{a}_{\circ}^{\minus{1}})
            \tag{Complement Property}\\
            &=a_{*}^{\minus{1}}*(a\circ{a}_{\circ}^{\minus{1}})
            \tag{Distributivity}\\
            &=a_{*}^{\minus{1}}*e_{*}
            \tag{Complement Property}\\
            &=a_{*}^{\minus{1}}
            \tag{Identity}
        \end{align}
        And therefore $a_{\circ}^{\minus{1}}=a_{*}^{\minus{1}}$.
    \end{proof}
    \begin{theorem}
        \label{thm:Bool_Uniquness_of_Comp}
        If $(X,\circ,*)$ is a Boolean algebra, if $a\in{X}$ and if
        $b,c\in{X}$ are such that $a*b=e_{\circ}$ and $a*c=e_{\circ}$, then
        $b=c$.
    \end{theorem}
    \begin{proof}
        For:
        \begin{align*}
            b&=b\circ{e}_{\circ}
            \tag{Identity}\\
            &=b\circ(a*c)
            \tag{Hypothesis}\\
            &=(b\circ{a})*(b\circ{c})
            \tag{Distributivity}\\
            &=(a\circ{b})*(b\circ{c})
            \tag{Commutativity}
        \end{align*}
        But by Thm.~\ref{thm:Bool_Comp_of_First_Op_is_Comp_of_Second_Op},
        if $a*b=e_{\circ}$, then $a\circ{b}=e_{\circ}$, and therefore:
        \begin{align*}
            (a\circ{b})*(b\circ{c})
            &=e_{*}*(b\circ{c})
            \tag{Thm.~\ref{thm:Bool_Comp_of_First_Op_is_Comp_of_Second_Op}}\\
            &=b\circ{c}
            \tag{Identity}
        \end{align*}
        In a similarly manner, $c=b\circ{c}$, and thus by the transitivity
        of equality, $b=c$.
    \end{proof}
    Thms.~\ref{thm:Bool_Comp_of_First_Op_is_Comp_of_Second_Op} and
    \ref{thm:Bool_Uniquness_of_Comp} allow us to define the complement
    of an element.
    \begin{fdefinition}{Complement in a Boolean Algebra}
                       {Complement in a Boolean Algebra}
        The complement of an element $x$ in a Boolean algebra
        $(X,\circ,*)$ is the unique element $a^{\minus{1}}$ such that:
        \begin{equation*}
            a*a^{\minus{1}}=e_{\circ}
            \quad\quad
            a\circ{a}^{\minus{1}}=e_{*}
        \end{equation*}
        Where $e_{\circ}$ and $e_{*}$ are the \glspl{unital element} of
        $\circ$ and $*$, respectively.
    \end{fdefinition}
            \begin{theorem}
                $e_{\circ}$ and $e_{*}$ are pseudo-inverses of each other.
            \end{theorem}
            \begin{proof}
                From identity:
                \par\hfill\par
                \begin{subequations}
                    \begin{minipage}[b]{0.49\textwidth}
                        \begin{equation}
                            e_{\circ}\circ{e}_{*}=e_{*}
                        \end{equation}
                    \end{minipage}
                    \hfill
                    \begin{minipage}[b]{0.49\textwidth}
                        \begin{equation}
                            e_{\circ}*{e}_{*}=e_{\circ}
                        \end{equation}
                    \end{minipage}
                \end{subequations}
                \par\vspace{2.5ex}
                Thus, $e_{*}$ and $e_{\circ}$ are pseudo-inverses of each other.
            \end{proof}
            \begin{theorem}
                \label{thm:Pseudo_Field_Elements_are_Idempotent}%
                For any $a\in{S}$, $a*a=a\circ{a}=a$.
            \end{theorem}
            \begin{proof}
                For:
                \par\vspace{-2.5ex}
                \begin{subequations}
                    \begin{minipage}[t]{0.49\textwidth}
                        \begin{align}
                            a&=a*e_{*}
                            \tag{Identity}\\
                            &=a*(a\circ{a}^{\minus{1}})
                            \tag{Complement}\\
                            &=(a*a)\circ(a*a^{\minus{1}})
                            \tag{Distributivity}
                        \end{align}
                    \end{minipage}
                    \hfill
                    \begin{minipage}[t]{0.49\textwidth}
                        \begin{align}
                            &=(a*a)\circ{e}_{\circ}
                            \tag{Complement}\\
                            &=a*a\tag{Identity}
                        \end{align}
                    \end{minipage}
                \end{subequations}
                \par\vspace{2.5ex}
                Similarly, $a=a\circ a$.
            \end{proof}
            \begin{theorem}
                For any $a\in S$, $a*e_{\circ}=e_{\circ}$
                and $a\circ e_{*}=e_{*}$
            \end{theorem}
            \begin{proof}
                For:
                \par\vspace{-2.5ex}
                \begin{subequations}
                    \begin{minipage}[t]{0.49\textwidth}
                        \begin{align}
                            a*e_{\circ}
                            &=a*(a^{\minus{1}}*a)\tag{Pseudo-Inverse}\\
                            &=a*(a*a^{\minus{1}})\tag{Commutativity}\\
                            &=(a*a)*a^{\minus{1}}\tag{Associativity}
                        \end{align}
                    \end{minipage}
                    \hfill
                    \begin{minipage}[t]{0.49\textwidth}
                        \begin{align}
                            &=a*a^{\minus{1}}
                            \tag{Thm.~\ref{thm:Pseudo_Field_Elements_%
                                           are_Idempotent}}\\
                            &=e_{\circ}\tag{Pseudo-Inverse}
                        \end{align}
                    \end{minipage}
                \end{subequations}
                \par\vspace{2.5ex}
                Similarly for $a\circ{e}_{*}=e_{*}$.
            \end{proof}
            \begin{theorem}
                If $a\circ{b}=a$ and $a*b=a$, then $b=a$.
            \end{theorem}
            \begin{proof}
                For:
                \par
                \begin{minipage}[b]{0.49\textwidth}
                    \begin{align}
                        b&=b*e_{*}\tag{Identity}\\
                         &=b*(a\circ a^{-1})\tag{Pseudo-Inverse}\\
                         &=(b*a)\circ(b* a^{-1})\tag{Distributivity}\\
                         &=a\circ (b* a^{-1})\tag{Hypothesis}
                    \end{align}
                \end{minipage}
                \hfill
                \begin{minipage}[b]{0.49\textwidth}
                    \begin{align}
                        &=(a\circ b)*(a\circ a^{-1})\tag{Distributivity}\\
                        &=(a\circ{b})*e_{*}\tag{Pseudo-Inverse}\\
                        &=a\circ{b}\tag{Identity}\\
                        &=a\tag{Hypothesis}
                    \end{align}
                \end{minipage}
                \par\vspace{2.5ex}
                And therefore $a=b$.
            \end{proof}
            \begin{theorem}
                If $a*b=a\circ{b}$, then $a=b$.
            \end{theorem}
            \begin{proof}
                For:
                \begin{align*}
                    a&=a*e_{*}\\
                    &=a*(b\circ{b}^{\minus{1}})\\
                    &=(a*b)\circ(a*b^{\minus{1}})\\
                    &=(a\circ{b})\circ(a*b^{\minus{1}})\\
                    &=((a\circ{b})\circ{a})*((a\circ{b})\circ{b}^{\minus{1}})\\
                    &=(a\circ{b})*(a\circ{e}_{*})\\
                    &=a\circ{b}
                \end{align*}
                Thus $a=a\circ{b}$. But $a\circ{b}=a*{b}$, and thus
                $a=a*{b}$. By Thm. (THE PREVIOUS ONE), $a=b$.
            \end{proof}
            \begin{theorem}
                If $a\in{S}$ and $a=a^{\minus{1}}$, then $a=e_{\circ}=e_{*}$
            \end{theorem}
            \begin{proof}
                For let $a\in{S}$ and let $a=a^{\minus{1}}$. Then by
                Thm.~\ref{thm:Pseudo_Field_Elements_are_Idempotent}:
                \begin{equation}
                    a=a*a=a*a^{-1}=e_{\circ}
                \end{equation}
                Similarly, $a=e_{*}$.
            \end{proof}
            \begin{theorem}
                For $a\in{S}$, $(a^{\minus{1}})^{\minus{1}}=a$.
            \end{theorem}
            \begin{proof}
                For we have:
                \begin{align*}
                    a&=e\circ{e}_{\circ}
                    \tag{Identity}\\
                    &=a\circ(a^{\minus{1}}*(a^{\minus{1}})^{\minus{1}})
                    \tag{Complement}\\
                    &=(a\circ a^{\minus{1}})*
                      (a\circ(a^{\minus{1}})^{\minus{1}})
                    \tag{Distributivity}\\
                     &=a\circ(a^{\minus{1}})^{\minus{1}}\\
                     &=e_{*}*(a\circ(a^{\minus{1}})^{\minus{1}})
                     \tag{Complement}\\
                     &=a\circ(a^{\minus{1}})^{\minus{1}}
                     \tag{Identity}
                \end{align*}
                Similarly, $a = a* (a^{-1})^{-1}$.
                But if $a = a\circ (a^{-1})^{-1} = a*(a^{-1})^{-1}$,
                then $a = (a^{-1})^{-1}$.
            \end{proof}
            \begin{definition} For $a\in S$, an inverse, or normal inverse, of the First Operation is an element $b\in S$ such that $a\circ b=e_{\circ}$. An inverse of the Second Operation is similarly defined. The normal inverses are denoted $a^{*}$ and $a^{\circ}$.
            \end{definition}
            \begin{theorem} If $a\in S$ has a normal inverse for either operation, than it is unique.
            \end{theorem}
            \begin{proof} For suppose not. Let $a\in S$ have a normal inverse for the First Operation. That is, there is an $a^{\circ}\in S$ such that $a\circ a^{\circ}=e_{\circ}$ and let $a'^{\circ}$ be a second normal inverse not equal to the first. But then $a^{\circ}=a^{\circ}\circ e_{\circ}=a^{\circ}\circ (a\circ a'^{\circ})$ and from associativity we have $a^{\circ}=(a^{\circ}\circ a)\circ a'^{\circ}=a'^{\circ}$. Thus, the normal inverse is unique. Similarly if there is an inverse for the Second Operation
            \end{proof}
            \begin{theorem} If $a\in S$ has a normal inverse, say $a'$, for one operation, then $a^{-1}=a'^{-1}$.
            \end{theorem}
            \begin{proof} For let $a\in S$ have a normal inverse $a'$ for the First Operation. That is, $a\circ a' = e_{\circ}$. But $a' \circ a'^{-1}=e_{*}$, and from theorem 1.3 $a\circ e_{*}=e_{*}$. So $a\circ (a' \circ a'^{-1})=e_{*}$. And from theorem 1.4, $a\circ a=a$, so we have $(a\circ a)\circ (a'\circ a'^{-1}=a\circ (a\circ a')\circ a'^{-1}=a\circ a'^{-1}=e_{*}$. But $a\circ a^{-1}=e_{\circ}$. And pseudo-inverses are unique. Thus, $a^{-1}=a'^{-1}$. 
            \end{proof}
            \begin{theorem} The identities have normal inverses for their respective operations.
            \end{theorem}
            \begin{proof} As normal inverses are unique, it suffices to find inverses for both identities. But $e_{\circ}\circ e_{\circ}=e_{\circ}$, so $e_{\circ}$ is its own inverse for the First Operation. Similarly, $e_{*}*e_{*}=e_{*}$.
            \end{proof}
            \begin{theorem} \textbf{(The Not-A-Field Theorem)} Only the identities have normal inverses.
            \end{theorem}
            \begin{proof} For suppose not. Suppose $a\in S,\ a\ne e_{\circ},\ a\ne e_{*}$ and a has an inverse for the First Operation. That is $\exists a^{\circ}\in S|\ a\circ a^{\circ}=e_{\circ}$. But by theorem 1.4, $a\circ a^{\circ}=(a\circ a)\circ a^{\circ}$. By associativity, we have $e_{\circ}=a\circ a^{\circ} = a\circ (a\circ a^{\circ})=a\circ e_{\circ}=a$. Thus, $a=e_{\circ}$. But by hypothesis, $a\ne e_{\circ}$. Thus, there is no inverse for $a$. Similarly, a has no inverse for the Second Operation.
            \end{proof}
            \begin{theorem}
            There exist pseudo-fields with only one element.
            \end{theorem}
            \begin{proof}
            For let $e_{\circ} = e_{*}$, and let no other elements be in the set. 
            \end{proof}
            \begin{theorem}
            A pseud-field has one element if and only if $e_{\circ} = e_{*}$.
            \end{theorem}
            \begin{proof}
            For suppose there is another element $a \ne e_{\circ}$. But then $a \circ e_{\circ} = a$, but also $a \circ e_{\circ} = a \circ e_{*} = e_{*}$. So $a = e_{*}$. If there is only one element, then clearly $e_{\circ} = e_{*}$ as otherwise there would be two elements.
            \end{proof}
            \begin{definition} A generating set on a pseudo-field is a subset $g_S \subset S$ such that every element of $S$ can be written as a finite combination of elements in $g_S$ using $\circ$ or $*$.
            \end{definition}
            \begin{theorem}
            The number of elements in a finite pseudo-field is a power of 2.
            \end{theorem}
            \begin{proof}
            Consider the set of all generators $g_S$ on $S$. Clearly for all such generators, $1\leq |g_S|\leq |S|$. Let $G$ be the smallest generator, such that $|G| \leq |g_S|$ for any other given generator. 
            \end{proof}