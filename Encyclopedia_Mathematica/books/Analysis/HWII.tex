%------------------------------------------------------------------------------%
\documentclass[crop=false,class=article]{standalone}                           %
%------------------------------Preamble----------------------------------------%
\makeatletter                                                                  %
    \def\input@path{{../../../}}                                               %
\makeatother                                                                   %
%---------------------------Packages----------------------------%
\usepackage{geometry}
\geometry{b5paper, margin=1.0in}
\usepackage[T1]{fontenc}
\usepackage{graphicx, float}            % Graphics/Images.
\usepackage{natbib}                     % For bibliographies.
\bibliographystyle{agsm}                % Bibliography style.
\usepackage[french, english]{babel}     % Language typesetting.
\usepackage[dvipsnames]{xcolor}         % Color names.
\usepackage{listings, lstlinebgrd}      % Verbatim-Like Tools.
\usepackage{mathtools, esint, mathrsfs} % amsmath and integrals.
\usepackage{amsthm, amsfonts}           % Fonts and theorems.
\usepackage{tabularx}
\usepackage{tcolorbox}                  % Frames around theorems.
\usepackage{upgreek}                    % Non-Italic Greek.
\usepackage{paracol}                    % Two-column styling.
\usepackage{wrapfig}                    % Wrap text around figure.
\usepackage{fmtcount, etoolbox}         % For the \book{} command.
\usepackage[newparttoc]{titlesec}       % Formatting chapter, etc.
\usepackage{titletoc}                   % Allows \book in toc.
\usepackage[nottoc]{tocbibind}          % Bibliography in toc.
\usepackage[titles]{tocloft}            % ToC formatting.
\usepackage{multicol, enumitem}         % Multi-column/enumerate.
\usepackage{import}                     % Import external files.
\usepackage{pgfplots, tikz}             % Drawing/graphing tools.
\usetikzlibrary{
    calc,                   % Calculating right angles and more.
    angles,                 % Drawing angles within triangles.
    arrows.meta,            % Latex and Stealth arrows.
    quotes,                 % Adding labels to angles.
    positioning,            % Relative positioning of nodes.
    decorations.markings,   % Adding arrows in the middle of a line.
    patterns,
    arrows,
    shapes,
    shapes.geometric,
    cd,
    hobby,
    babel
}                                       % Libraries for tikz.
\pgfplotsset{compat=1.9}                % Version of pgfplots.
\usepackage[font=scriptsize,
            labelformat=simple,
            labelsep=colon]{subcaption} % Subfigure captions.
\usepackage[font={scriptsize},
            hypcap=true,
            labelsep=colon]{caption}    % Figure captions.
\usepackage{hyperref}                   % Allows for hyperlinks.
\hypersetup{
    colorlinks=true,
    linkcolor=blue,
    filecolor=magenta,
    urlcolor=Cerulean,
    citecolor=SkyBlue
}                           % Colors for hyperref.
\usepackage[toc,acronym,nogroupskip]{glossaries} % Glossaries and acronyms.
\usepackage[subpreambles=false]{standalone}      % Complileable sub files.

% Various font stuff from kiwi.
% Use this for Times text and Computer Modern math
%\usepackage{times}

% Quite nice
%\usepackage[charter, greekfamily=, greekuppercase=italicized]{mathdesign}
%\usepackage[utopia, greekuppercase=italicized]{mathdesign}    % Math is narrower

% Use this for Times text and math
%\usepackage{newtxtext}
%\usepackage[libertine,cmintegrals]{newtxmath}
%\usepackage{fix-cm}

%\usepackage{txfontsb}
% or
%\usepackage{mathptmx}

%\usepackage[scaled=0.92]{helvet}
%\renewcommand{\rmdefault}{ptm}

%\usepackage{mathpazo}    % add possibly `sc` and `osf` options
%\usepackage{eulervm}

%\usepackage{fourier}
%\renewcommand{\rmdefault}{ptm}
%\usepackage{mathptm}

%\usepackage{fontspec}
%\setmainfont{lmodern}

%\usepackage[varg]{txfonts}
%\usepackage{fouriernc}
%\usepackage{mathpazo}

%\usepackage{bookman}
%\usepackage[scaled]{uarial}
%\usepackage[scaled]{helvet}
%\renewcommand*\familydefault{\sfdefault}
%\usepackage[math]{anttor}

%\newcommand\fgeorgia{\fontfamily{jvn}\selectfont}
%\newcommand\ftimes{\fontfamily{ptm}\selectfont}
%\newcommand\fhelvetica{\fontfamily{phv}\selectfont}
%\newcommand\fcourier{\fontfamily{pcr}\selectfont}
%\newcommand\fbookman{\fontfamily{pbk}\selectfont}
%\newcommand\fnewcentury{\fontfamily{pnc}\selectfont}
%\newcommand\fpalatino{\fontfamily{ppl}\selectfont}
%\newcommand\favantgarde{\fontfamily{pag}\selectfont}
%\newcommand\fnormal{\normalfont}
%\newcommand\fsize[1]{\ifnum#1>0\fontsize{#1}{#1}\selectfont\else\normalsize\fi}
%------------------------Theorem Styles-------------------------%
% Define theorem style for default spacing and normal font.
\newtheoremstyle{normal}
    {\topsep}               % Amount of space above the theorem.
    {\topsep}               % Amount of space below the theorem.
    {}                      % Font used for body of theorem.
    {}                      % Measure of space to indent.
    {\bfseries}             % Font of the header of the theorem.
    {}                      % Punctuation between head and body.
    {.5em}                  % Space after theorem head.
    {}

% Define theorem style for default spacing with italicized font.
\newtheoremstyle{normalit}{\topsep}{\topsep}
                {\itshape}{}{\bfseries}{}{.5em}{}

% Italic header environment.
\newtheoremstyle{thmit}{\topsep}{\topsep}{}{}{\itshape}{}{0.5em}{}

% Define italicized environments.
\theoremstyle{normalit}
\newtheorem{theorem}{Theorem}[section]
\newtheorem{lemma}{Lemma}[section]
\newtheorem{corollary}{Corollary}[section]
\newtheorem{proposition}{Proposition}[section]
\newtheorem*{theorem*}{Theorem}

% Define environments with italic headers.
\theoremstyle{thmit}
\newtheorem*{solution}{Solution}
\newtheorem*{fsolution}{Solution}

% Define default environments.
\theoremstyle{normal}
\newtheorem{example}{Example}[section]
\newtheorem{definition}{Definition}[section]
\newtheorem{problem}{Problem}[section]
\newtheorem{question}{Question}[section]
\newtheorem{remark}{Remark}[section]
\newtheorem{properties}{Properties}[section]
\newtheorem{notation}{Notation}[section]
\newtheorem{axiom}{Axiom}[section]
\newtheorem*{properties*}{Properties}
\newtheorem*{remark*}{Remark}
\newtheorem*{definition*}{Definition}
\theoremstyle{plain}

% Define framed environment.
\tcbuselibrary{most}
\newtcbtheorem[use counter*=theorem]{ftheorem}{Theorem}%
    {colback=green!5,colframe=green!35!black,
     fonttitle=\bfseries\upshape}{th}

\newtcbtheorem[use counter*=example]{fdefinition}{Definition}%
    {fonttitle=\bfseries\upshape,
     colback=blue!5!white,colframe=blue!75!black}{def}

\newtcbtheorem[use counter*=example]{fexample}{Example}%
    {fonttitle=\bfseries\upshape,
     colback=red!5!white,colframe=red!75!black}{ex}

\newtcbtheorem[use counter*=notation]{fnotation}{Notation}%
    {fonttitle=\bfseries\upshape,
     colback=SeaGreen!5!white,colframe=SeaGreen!75!black}{ex}

\newtcbtheorem[use counter*=corollary]{fcorollary}{Corollary}%
    {fonttitle=\bfseries\upshape,
     colback=Orchid!5!white,colframe=Orchid!75!black}{ex}

\newenvironment{bproof}{\textit{Proof.}}{\hfill$\square$}
\tcolorboxenvironment{bproof}{blanker,breakable,left=5mm,
                             before skip=10pt,after skip=10pt,
                             borderline west={1mm}{0pt}{red}}
\tcolorboxenvironment{fsolution}
    {enhanced jigsaw,colframe=cyan,interior hidden,breakable}

%--------------------Declared Math Operators--------------------%
\DeclareMathOperator{\Refl}{Refl}           % Reflection operator.
\DeclareMathOperator{\Span}{Span}           % Span of a set of vectors.
\DeclareMathOperator{\Card}{Card}           % Cardinality of set.
\DeclareMathOperator{\Ord}{Ord}             % Ordinal of ordered set.
\DeclareMathOperator{\Tr}{Tr}               % Trace of matrix.
\DeclareMathOperator{\adjoint}{adj}         % Adjoint of matrix.
\DeclareMathOperator{\rk}{rk}               % Rank of operator.
\DeclareMathOperator{\nul}{nul}             % Null space of operator.
\DeclareMathOperator{\sgn}{sgn}             % Sign of a number.
\DeclareMathOperator{\multideg}{mutlideg}   % Multi-Degree (Graphs).
\DeclareMathOperator{\GCD}{GCD}             % Greatest common denominator.
\DeclareMathOperator{\LM}{LM}               % Leading monomial
\DeclareMathOperator{\LC}{LC}               % Leading coefficient.
\DeclareMathOperator{\LT}{LT}               % Leading term.
\DeclareMathOperator{\LCM}{LCM}             % Least common multiple.
\DeclareMathOperator{\Mon}{Mon}             % Monomial.
\DeclareMathOperator{\Spec}{Spec}           % Spectrum.
\DeclareMathOperator{\proj}{proj}           % Projection.
\DeclareMathOperator{\comp}{comp}           % Component.
\DeclareMathOperator{\sinc}{sinc}           % Sinc function.
\DeclareMathOperator{\Ima}{Im}              % Image of operator.
\DeclareMathOperator{\Prin}{Prin}           % Principal value.
\DeclareMathOperator{\Mod}{mod}             % Modulus.
%------------------------New Commands---------------------------%
\DeclarePairedDelimiter\norm{\lVert}{\rVert}
\DeclarePairedDelimiter\ceil{\lceil}{\rceil}
\DeclarePairedDelimiter\floor{\lfloor}{\rfloor}
\newcommand*\diff{\mathop{}\!\mathrm{d}}
\newcommand*\Diff[1]{\mathop{}\!\mathrm{d^#1}}
\renewcommand{\mod}{\ \Mod}
\renewcommand*{\glstextformat}[1]{\textcolor{RoyalBlue}{#1}}
\renewcommand{\glsnamefont}[1]{\textbf{#1}}
\renewcommand\labelitemii{$\circ$}
\renewcommand\thesubfigure{\arabic{chapter}.\arabic{figure}}
\renewcommand\thesubfigure{%
    \arabic{chapter}.\arabic{figure}.\arabic{subfigure}}
\addto\captionsenglish{\renewcommand{\figurename}{Fig.}}
%------------------------Book Command---------------------------%
\makeatletter
\renewcommand\@pnumwidth{1cm}
\newcounter{book}
\renewcommand\thebook{\@Roman\c@book}
\newcommand\book{%
    \if@openright
        \cleardoublepage
    \else
        \clearpage
    \fi
    \thispagestyle{plain}%
    \if@twocolumn
        \onecolumn
        \@tempswatrue
    \else
        \@tempswafalse
    \fi
    \null\vfil
    \secdef\@book\@sbook
}
\def\@book[#1]#2{%
    \ifnum \c@secnumdepth >-3\relax
        \refstepcounter{book}%
        \addcontentsline{toc}{book}{
            \bookname\ \thebook:\hspace{1em}#1
        }
    \else
        \addcontentsline{toc}{book}{#1}%
    \fi
    \markboth{}{}%
    {\centering
     \interlinepenalty \@M
     \normalfont
     \ifnum \c@secnumdepth >-2\relax
       \huge\bfseries \bookname\nobreakspace\thebook
       \par
       \vskip 20\p@
     \fi
     \Huge \bfseries #2\par}%
    \@endbook}
\def\@sbook#1{%
    {\centering
     \interlinepenalty \@M
     \normalfont
     \Huge \bfseries #1\par}%
    \@endbook}
\def\@endbook{
    \vfil\newpage
        \if@twoside
            \if@openright
                \null
                \thispagestyle{empty}%
                \newpage
            \fi
        \fi
        \if@tempswa
            \twocolumn
        \fi
}
\newcommand*\l@book[2]{%
    \ifnum \c@tocdepth >-2\relax
        \addpenalty{-\@highpenalty}%
        \addvspace{2.25em \@plus\p@}%
        \setlength\@tempdima{3em}%
        \begingroup
            \parindent \z@ \rightskip \@pnumwidth
            \parfillskip -\@pnumwidth
            {
                \leavevmode
                \Large \bfseries #1\hfil \hb@xt@\@pnumwidth{
                    \hss #2
                }
            }
            \par
            \nobreak
            \global\@nobreaktrue
            \everypar{\global\@nobreakfalse\everypar{}}%
        \endgroup
    \fi}
\newcommand\bookname{Book}
\renewcommand{\thebook}{\texorpdfstring{\Numberstring{book}}{book}}
\providecommand*{\toclevel@book}{-2}
\makeatother
\titlecontents{chapter}[0pt]
    {\bfseries}
    {\chaptername\ \thecontentslabel:\quad}
    {}
    {\hfill\contentspage}
\titleformat{\part}[display]
    {\Large\bfseries}
    {\partname\nobreakspace\thepart}
    {0mm}
    {\Huge\bfseries}
    \titlecontents{part}[0pt]
    {\large\bfseries}
    {\partname\ \thecontentslabel: \quad}
    {}
    {\hfill\contentspage}
\newcommand{\MarkRightAngle}[4][.3cm]
    {\coordinate (tempa) at ($(#3)!#1!(#2)$);
     \coordinate (tempb) at ($(#3)!#1!(#4)$);
     \coordinate (tempc) at ($(tempa)!0.5!(tempb)$);%midpoint
     \draw (tempa) -- ($(#3)!2!(tempc)$) -- (tempb);}
%--------------------------LENGTHS------------------------------%
% Spacings for the Table of Contents.
\addtolength{\cftsecnumwidth}{1ex}
\addtolength{\cftsubsecindent}{1ex}
\addtolength{\cftsubsecnumwidth}{1ex}
\addtolength{\cftfignumwidth}{1ex}
\addtolength{\cfttabnumwidth}{1ex}

% Spacing for multi-column and enumerate environments.
\setlength{\multicolsep}{6pt}
\setlist[enumerate]{itemsep=0pt,topsep=3pt}

% Indent and paragraph spacing.
\setlength{\parindent}{0em}
\setlength{\parskip}{0em}                                                           %
%----------------------------Main Document-------------------------------------%
\begin{document}
    \title{Topics in Analysis}
    \author{Ryan Maguire}
    \date{\vspace{-5ex}}
    \maketitle
    \section{Homework II}
        \begin{problem}
            Let $X$ be an uncountable set and $\mathcal{M}$ be the collection of
            countable (or finite) and cocountable (or cofinite) subsets of $X$.
            Define $\mu:\mathcal{M}\rightarrow\mathbb{R}$ by:
            \begin{equation}
                \mu(E)=
                \begin{cases}
                    0,&E\textit{ is finite or countable}\\
                    1,&E\textit{ is cofinite or cocountable}
                \end{cases}
            \end{equation}
            Show that $\mathcal{M}$ is a $\sigma\textrm{-Algebra}$ on $X$ and
            that $\mu$ is a measure on $\mathcal{M}$. Describe the corresponding
            measurable functions and their integrals.
        \end{problem}
        \begin{solution}
            $\mathcal{M}$ is indeed a $\sigma\textrm{-Algebra}$. It is trivially
            closed under complements, and $X\in\mathcal{M}$ since
            $X^{C}=\emptyset$ and the empty set is finite. Given a countable
            collection of elements of $\mathcal{M}$ either there is an
            uncountable element or there is not. If there is not than the union
            over the collection is the countable union of countable (or finite)
            sets and is therefore at most countable. If $\mathcal{O}$ is the
            collection and there is an uncountable element
            $\mathcal{U}\in\mathcal{O}$, then:
            \begin{equation}
                X\setminus\Big(
                    \bigcup_{\mathcal{V}\in\mathcal{O}}\mathcal{V}
                \Big)
                \subseteq{X}\setminus\mathcal{U}
            \end{equation}
            But $\mathcal{U}\in\mathcal{M}$ is uncountable, and thus it must
            be either cofinite or cocountable. But then the complement of the
            union over $\mathcal{O}$ is the subset of a cofinite or cocountable
            subset, and is therefore itself either cocountable or cofinite. Thus
            $\mathcal{M}$ is closed to countable unions.
            \par\hfill\par
            $\mu$ is also a measure. By definition, for all $E\in\mathcal{M}$
            we have that $\mu(E)\geq{0}$. Moreover since $\emptyset$ is finite,
            we obtain $\mu(\emptyset)=0$. If
            $A:\mathbb{N}\rightarrow\mathcal{M}$ is a sequence of mutually
            disjoint measurable sets then either there is an $n\in\mathbb{N}$
            such that $A_{n}$ is uncountable or there is not. If not then, since
            the countable union of countable sets is countable, we have that:
            \begin{equation}
                \mu\Big(\bigcup_{k\in\mathbb{N}}A_{k}\Big)=0
            \end{equation}
            But also:
            \begin{equation}
                \sum_{k\in\mathbb{N}}\mu(A_{k})=\sum_{k\in\mathbb{N}}0=0
            \end{equation}
            And thus $\mu$ is countably additive in this case. If there is an
            $n\in\mathbb{N}$ such that $A_{n}$ is uncountable, then for all
            $m\in\mathbb{N}$ such that $m\ne{n}$ we have that $a_{m}$ is
            countable or finite. For since $A_{n}\cap{A}_{m}=\emptyset$ we
            conclude that $A_{m}\subseteq{X}\setminus{A}_{n}$. But
            $X\setminus{A}_{n}$ is either finite or countable. Piecing this
            together, we get:
            \begin{equation}
                \sum_{k\in\mathbb{N}}\mu(A_{k})
                =\mu(A_{n})+\sum_{k\in\mathbb{N}\setminus{n}}\mu(A_{k})
                =\mu(A_{n})+\sum_{k\in\mathbb{N}\setminus{n}}0
                =1+0
                =1
            \end{equation}
            But since the union over all of the $A_{k}$ is either cocountable or
            cofinite, we have:
            \begin{equation}
                \mu\Big(\bigcup_{k\in\mathbb{N}}A_{k})=1
            \end{equation}
            Therefore $\mu$ is countably additive, and $\mu$ is thus a measure.
        \end{solution}
        \begin{problem}
            Suppose $F:\mathbb{N}\times{X}\rightarrow[0,\infty]$ is a sequence
            of measurable functions and suppose that for all $x\in{X}$ and for
            all $n\in\mathbb{N}$ we have that $F_{n}(x)\geq{F}_{n+1}(x)$. Also
            suppose that $F_{n}(x)\rightarrow{f}(x)$ pointwise and that
            $F_{1}\in{L}^{1}(\mu)$. Prove that:
            \begin{equation}
                \underset{n\rightarrow\infty}{\lim}\int_{X}F_{n}\diff{\mu}
                =\int_{X}f\diff{\mu}
            \end{equation}
            Show that this need not be true if $F_{1}$ is not integrable.
        \end{problem}
        \begin{solution}
            For let $R$ be the integral of $F_{1}$ over $X$ and define
            $G:\mathbb{N}\times{X}\rightarrow[0,\infty]$ by:
            \begin{equation}
                G_{n}(x)=R-F_{n}(x)
            \end{equation}
            From the monotonicity of $F_{n}$, we have that for all $x\in{X}$ and
            for all $n\in\mathbb{N}$ $G_{n}(x)\leq{G}_{n+1}(x)$. Moreover, the
            $G_{n}$ are measurable and $G_{n}\rightarrow{R}-f$. Thus $G$ is a
            monotonically increasing sequence of measurable functions and by
            Lebesgue's monotone convergence theorem we have:
            \begin{equation}
                \underset{n\rightarrow\infty}{\lim}\int_{X}G_{n}\diff{\mu}
                =R-\int_{X}f\diff{\mu}
            \end{equation}
            And since $F_{n}(x)\geq{F}_{n+1}(x)$, we have that
            $F_{1}(x)\geq{f}(x)$ and thus the right hand side is finite. Using
            this we obtain:
            \begin{align}
                R-\int_{X}f\diff{\mu}
                &=\underset{n\rightarrow\infty}{\lim}\int_{X}G_{n}\diff{\mu}\\
                &=\underset{n\rightarrow\infty}{\lim}
                    \int_{X}(R-F_{n})\diff{\mu}\\
                &=\underset{n\rightarrow\infty}{\lim}
                    \Big(R-\int_{X}F_{n}\diff{\mu}\Big)\\
                &=R-\underset{n\rightarrow\infty}{\lim}\int_{X}F_{n}\diff{\mu}
            \end{align}
            Comparing the first and last part of the chain of equalities
            completes the proof. To see that the result may fail if $F_{1}$ is
            not integrable, define
            $F:\mathbb{N}\times\mathbb{R}^{+}\rightarrow[0,\infty]$ as follows:
            \begin{equation}
                F_{n}(x)=\frac{1}{nx}
            \end{equation}
            Then for all $x\in\mathbb{R}^{+}$ and for all $n\in\mathbb{N}$ we
            have that $F_{n+1}(x)\leq{F}_{n}(x)$, each function is continuous
            and therefore measurable, and $F_{n}(x)\rightarrow{0}$ for all $x$.
            However, we have:
            \begin{equation}
                \int_{\mathbb{R}^{+}}F_{n}(x)\diff{\mu}
                =\infty\ne{0}
                =\int_{\mathbb{R}^{+}}0\diff{\mu}
            \end{equation}
        \end{solution}
        \begin{problem}
            Show that if $\mu(X)<\infty$ and that if
            $F:\mathbb{N}\times{X}\rightarrow\mathbb{C}$ is a sequence of
            bounded measurable functions that converge uniformly to
            $f:X\rightarrow\mathbb{C}$ then:
            \begin{equation}
                \underset{n\rightarrow\infty}{\lim}\int_{X}F_{n}\diff{\mu}
                =\int_{X}f\diff{\mu}
            \end{equation}
            Show that the finiteness of $\mu(X)$ cannot be omitted.
        \end{problem}
        \begin{solution}
            For by the definition of uniform convergence, for all
            $\varepsilon>0$ there exists an $N\in\mathbb{N}$ such that, for all
            $x\in{X}$, $|F_{n}(x)-f(x)|<\varepsilon$. That is:
            \begin{equation}
                \underset{n\rightarrow\infty}{\lim}
                \sup\big\{|F_{n}(x)-f(x)|\,:\,x\in{X}\big\}=0
            \end{equation}
            But then we have:
            \begin{equation}
                \underset{n\rightarrow\infty}{\lim}\int_{X}|F_{n}-f|\diff{\mu}
                \leq\underset{n\rightarrow\infty}{\lim}
                \sup\big\{|F_{n}(x)-f(x)\,:\,x\in{X}\big\}\mu(X)
            \end{equation}
            But $\mu(X)<\infty$ and therefore this limit is zero. To show that
            the finiteness of $\mu(X)$ is required, consider $X=\mathbb{C}$ and
            define $F:\mathbb{N}\times\mathbb{C}\rightarrow\mathbb{C}$ by
            $F_{n}(z)=n^{\minus{1}}$. Thus $F_{n}(z)\rightarrow{0}$ for all
            $z\in\mathbb{C}$, but:
            \begin{equation}
                \int_{\mathbb{C}}F_{n}(z)\diff{\mu}
                =\int_{\mathbb{C}}n^{\minus{1}}\diff{\mu}
                =n^{\minus{1}}\mu(\mathbb{C})
                =\infty
                \ne{0}
                =\int_{\mathbb{C}}0\diff{\mu}
            \end{equation}
        \end{solution}
        \begin{problem}
            Suppose $f\in{L}^{1}(\mu)$. Prove that for all $\varepsilon>0$ there
            is a $\delta>0$ such that for all measurable $E$ such that
            $\mu(E)<\delta$, it is true that:
            \begin{equation}
                \int_{E}|f|\diff{\mu}<\varepsilon
            \end{equation}
        \end{problem}
        \begin{solution}
            For let $\varepsilon>0$ and let $A:\mathbb{N}\rightarrow{X}$ be the
            sequence defined by:
            \begin{equation}
                A_{n}=f^{\minus{1}}\big([n,\infty]\big)
            \end{equation}
            Since $f\in{L}^{1}(\mu)$ we have that $\mu(A_{n})\rightarrow{0}$.
            For suppose there is an $\varepsilon>0$ such that for all
            $N\in\mathbb{N}$ there exists an $n>N$ such that
            $\mu(A_{n})\geq\varepsilon$. Invoke choice and choose a strictly
            monotonically increasing sequence
            $k:\mathbb{N}\rightarrow\mathbb{N}$ such that
            $\mu(A_{k_{n}})\geq\varepsilon$ for all $n\in\mathbb{N}$. But then:
            \begin{equation}
                \underset{n\rightarrow\infty}{\lim}\int_{A_{k_{n}}}|f|\diff{\mu}
                \geq\underset{n\rightarrow\infty}{\lim}n\varepsilon
                =\infty\leq\int_{X}|f|\diff{\mu}<\infty
            \end{equation}
            A contradiction. Thus there is an $N\in\mathbb{N}$ such that, for
            all $n\geq{N}$ we have $\mu(A_{n})<\frac{\varepsilon}{2R}$, where
            $R$ is the integral of $|f|$ over $X$. By the Archimedean property,
            there is an $n\in\mathbb{N}$ such that $n>R$. Define $\delta$ by:
            $\delta=\frac{\varepsilon}{2n}$. Then for all $E$ such that
            $\mu(E)<\delta$:
            \begin{subequations}
                \begin{align}
                    \int_{E}|f|\diff{\mu}
                    &=\int_{E\cap{A}_{n}}|f|\diff{\mu}
                        +\int_{E\cap(X\setminus{A}_{n})}|f|\diff{\mu}\\
                    &\leq{R}\mu(E\cap{A}_{n})
                    +n\mu\big(E\cap(X\setminus{A}_{n})\big)\\
                    &<R\frac{\varepsilon}{2R}+n\frac{\varepsilon}{2n}\\
                    &=\varepsilon
                \end{align}
            \end{subequations}
        \end{solution}
        \begin{problem}
            Suppose that $(Y,\tau)$ is a topological space and that
            $\mathcal{M}$ is a $\sigma\textrm{-Algebra}$ on $Y$ such that
            $\tau\subseteq\mathcal{M}$. Suppose that $\mu$ is a measure on
            $\mathcal{M}$ such that, for all $E\in\mathcal{M}$:
            \begin{equation}
                \mu(E)
                =\inf\{\,\mu(V)\;|\;V\in\tau\textrm{ and }E\subseteq{V}\,\}
            \end{equation}
            Suppose that $\mu$ is a $\sigma\textrm{-finite}$ measure as well.
            That is, there is a sequence $A:\mathbb{N}\rightarrow\mathcal{M}$
            such that, for all $n\in\mathbb{N}$ it is true that
            $\mu(A_{n})<\infty$ and:
            \begin{equation}
                Y=\bigcup_{n\in\mathbb{N}}A_{n}
            \end{equation}
            \begin{enumerate}
                \item   Show that the Lebesgue measure is a
                        $\sigma\textrm{-finite}$ outer regular measure on
                        $(\mathbb{R},\mathcal{M})$.
                \item   Suppose $E$ is a $\mu$ measurable subset of $Y$. Given
                        $\varepsilon>0$ show that there is an open set
                        $V\subseteq{Y}$ and a closed set $F\subseteq{Y}$ such
                        that $F\subseteq{E}\subseteq{V}$ and
                        $\mu(V\setminus{F})<\varepsilon$.
                \item   Argue that $(\mathbb{R},\mathcal{M},\lambda)$ is the
                        completion of the restriction of the Lebesgue measure
                        to the Borel sets in $\mathbb{R}$.
            \end{enumerate}
        \end{problem}
        \begin{solution}
            Since the measure of the interval $(a,b)$, with $a<b$, is $b-a$ and
            since any open subset of $\mathbb{R}$ is the countable union of
            disjoint open intervals, with $(a,\infty)$ and $(\minus\infty,a)$
            couting as \textit{intervals} (proof omitted since it is a
            standard topological result), we have that by the definition of the
            Lebesgue measure:
            \begin{equation}
                \lambda(E)=
                \inf\big\{\,\lambda(\mathcal{U})\;|\;E\subseteq\mathcal{U}
                    \textrm{ and }\mathcal{U}\underset{Open}{\subseteq}
                    \mathbb{R}\big\}
            \end{equation}
            This is because any collection of open intervals that cover $E$
            form an open subset of $\mathbb{R}$, and thus can be written as the
            disjoint union of open intervals, so we may assume the collection
            that covered $E$ was disjoint and countable to begin with.
            \par\hfill\par
            For if $E$ is measurable, then $Y\setminus{E}$ is measurable so
            there is a $\mathcal{U}\in\tau$ such that:
            \begin{equation}
                \mu\big(\mathcal{U}\setminus(Y\setminus{E})\big)
                =\mu(\mathcal{U}\cap{E})<\frac{\varepsilon}{2}
            \end{equation}
            But ${Y}\setminus\mathcal{U}$ is closed and also:
            \begin{equation}
                \mu\big(E\setminus(Y\setminus\mathcal{U})\big)
                =\mu(E\cap\mathcal{U})<\varepsilon/2
            \end{equation}
            But since $E$ is measurable there is an open set $\mathcal{O}$ such
            that $E\subseteq\mathcal{O}$ and
            $\mu(E\setminus\mathcal{U})<\varepsilon/2$. But then:
            \begin{equation}
                \mu(\mathcal{O}\setminus\mathcal{U})
                \leq\mu(\mathcal{O}\setminus{E})
                +\mu\big((Y\setminus{E})\setminus\mathcal{U}\big)
                <\frac{\varepsilon}{2}+\frac{\varepsilon}{2}
                =\varepsilon
            \end{equation}
        \end{solution}
        \begin{problem}
            Let $m$ be the Lebesgue measure on $\mathbb{R}$ and suppose that $E$
            is a set of finite measure. Given $\varepsilon>0$ show that there is
            a finite disjoint union $F$ of open intervals such that
            $m(E\ominus{F})<\varepsilon$.
        \end{problem}
        \begin{proof}
            For let $\varepsilon>0$. Then there is a countable collection of
            intervals $I_{n}$ that cover $E$ and such that:
            \begin{equation}
                \mu\Big(\big(\bigcup_{n\in\mathbb{N}}I_{n}\big)
                    \setminus{E}\big)\Big)
                <\frac{\varepsilon}{2}
            \end{equation}
            But from the finiteness of $\mu(E)$, the following series must then
            converge:
            \begin{equation}
                M=\sum_{k\in\mathbb{N}}\mu(I_{k})
            \end{equation}
            And thus there is an $n\in\mathbb{N}$ such that, for all $m\geq{n}$,
            we have:
            \begin{equation}
                \sum_{k\in\mathbb{N}\setminus\mathbb{Z}_{m}}\mu(I_{k})
                <\frac{\varepsilon}{2}
            \end{equation}
            But then:
            \begin{align}
                \mu\Big(E\ominus\bigcup_{k\in\mathbb{Z}_{n}}I_{k}\Big)
                &=\mu\Big(E\setminus\bigcup_{k\in\mathbb{Z}_{n}}I_{k}\Big)+
                    \mu\Big(\bigcup_{k\in\mathbb{Z}_{n}}I_{k}
                    \setminus{E}\Big)\\
                &\leq\mu\Big(E\setminus\bigcup_{k\in\mathbb{N}}I_{k}\Big)
                    +\mu\Big(\bigcup_{k\in\mathbb{N}
                    \setminus\mathbb{Z}_{n}}I_{k}\Big)
                    +\mu\Big(\bigcup_{k\in\mathbb{Z}_{n}}I_{k}
                    \setminus{E}\Big)\\
                &<0+\frac{\varepsilon}{4}+\frac{\varepsilon}{4}\\
                &=\varepsilon
            \end{align}
            Thus $I_{1},\dots,I_{n}$ is a finite collection of open intervals
            such that the symmetric difference of the union of this collection
            with $E$ has measure less than $\varepsilon$.
        \end{proof}
        \begin{problem}
            Let $(X,\mathcal{M},\mu)$ be a measure space and let
            $(X,\mathcal{M}_{0},\mu_{0})$ be it's completion.
            \begin{enumerate}
                \item   Let $f:X\rightarrow\mathbb{C}$ be a $\mathcal{M}_{0}$
                        measurable function and assume that
                        $g:X\rightarrow\mathcal{C}$ is a $\mathcal{M}$
                        measurable function such that $f=g$ $\mu_{0}$ almost
                        everywhere. Is is necessarily true that $f=g$ $\mu$
                        almost everywhere?
            \end{enumerate}
        \end{problem}
        \begin{solution}
            If $\mathcal{M}$ is a proper subset of $\mathcal{M}_{0}$, then the
            first assertion is not necessarily true. For let $E$ be a
            $\mathcal{M}_{0}$ measurable subset such that $\mu_{0}(E)=0$ but
            $E\notin\mathcal{M}$. Let $f(x)=g(x)$ for all $x\notin{E}$ and let
            $f(x)\ne{g}(x)$ for all $x\in{E}$. Then $f=g$ $\mu_{0}$ almost
            everywhere, but not $\mu$ almost everywhere for the set of
            points $x$ such that $f(x)\ne{g}(x)$ is not $\mathcal{M}$ measurable
            and thus $\mu(E)$ is undefined.
            \par\hfill\par
            Let $\mathcal{O}$ be a countable basis of $\mathcal{C}$ and let
            $\mathcal{U}:\mathbb{N}\rightarrow\mathcal{O}$ be a bijection. Such
            a basis exists since $\mathcal{C}$ is second countable. Then there
            exist sequences $A,B:\mathbb{N}\rightarrow\mathcal{M}$ such that
            for all $n\in\mathbb{N}$,
            $A_{n}\subseteq\mathcal{U}_{n}\subseteq{B}_{n}$ and:
            \begin{equation}
                \mu(B_{n}\setminus{A}_{n})=0
            \end{equation}
            Define $g:X\rightarrow\mathbb{C}$ by:
            \begin{equation}
                g(x)=
                \begin{cases}
                    f(x),&x\in\bigcup_{n\in\mathbb{N}}A_{n}\\
                    0,&x\notin\bigcup_{n\in\mathbb{N}}A_{n}
                \end{cases}
            \end{equation}
            Then $g$ is measurable and $f=g$ $\mu_{0}$ almost everywhere.
        \end{solution}
\end{document}