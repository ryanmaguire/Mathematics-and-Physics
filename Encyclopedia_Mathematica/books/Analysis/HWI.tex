%--------------------------------------------------------------------------%
\documentclass[crop=false,class=article]{standalone}                       %
%--------------------------Preamble----------------------------------------%
%---------------------------Packages----------------------------%
\usepackage{geometry}
\geometry{b5paper, margin=1.0in}
\usepackage[T1]{fontenc}
\usepackage{graphicx, float}            % Graphics/Images.
\usepackage{natbib}                     % For bibliographies.
\bibliographystyle{agsm}                % Bibliography style.
\usepackage[french, english]{babel}     % Language typesetting.
\usepackage[dvipsnames]{xcolor}         % Color names.
\usepackage{listings}                   % Verbatim-Like Tools.
\usepackage{mathtools, esint, mathrsfs} % amsmath and integrals.
\usepackage{amsthm, amsfonts, amssymb}  % Fonts and theorems.
\usepackage{tcolorbox}                  % Frames around theorems.
\usepackage{upgreek}                    % Non-Italic Greek.
\usepackage{fmtcount, etoolbox}         % For the \book{} command.
\usepackage[newparttoc]{titlesec}       % Formatting chapter, etc.
\usepackage{titletoc}                   % Allows \book in toc.
\usepackage[nottoc]{tocbibind}          % Bibliography in toc.
\usepackage[titles]{tocloft}            % ToC formatting.
\usepackage{pgfplots, tikz}             % Drawing/graphing tools.
\usepackage{imakeidx}                   % Used for index.
\usetikzlibrary{
    calc,                   % Calculating right angles and more.
    angles,                 % Drawing angles within triangles.
    arrows.meta,            % Latex and Stealth arrows.
    quotes,                 % Adding labels to angles.
    positioning,            % Relative positioning of nodes.
    decorations.markings,   % Adding arrows in the middle of a line.
    patterns,
    arrows
}                                       % Libraries for tikz.
\pgfplotsset{compat=1.9}                % Version of pgfplots.
\usepackage[font=scriptsize,
            labelformat=simple,
            labelsep=colon]{subcaption} % Subfigure captions.
\usepackage[font={scriptsize},
            hypcap=true,
            labelsep=colon]{caption}    % Figure captions.
\usepackage[pdftex,
            pdfauthor={Ryan Maguire},
            pdftitle={Mathematics and Physics},
            pdfsubject={Mathematics, Physics, Science},
            pdfkeywords={Mathematics, Physics, Computer Science, Biology},
            pdfproducer={LaTeX},
            pdfcreator={pdflatex}]{hyperref}
\hypersetup{
    colorlinks=true,
    linkcolor=blue,
    filecolor=magenta,
    urlcolor=Cerulean,
    citecolor=SkyBlue
}                           % Colors for hyperref.
\usepackage[toc,acronym,nogroupskip,nopostdot]{glossaries}
\usepackage{glossary-mcols}
%------------------------Theorem Styles-------------------------%
\theoremstyle{plain}
\newtheorem{theorem}{Theorem}[section]

% Define theorem style for default spacing and normal font.
\newtheoremstyle{normal}
    {\topsep}               % Amount of space above the theorem.
    {\topsep}               % Amount of space below the theorem.
    {}                      % Font used for body of theorem.
    {}                      % Measure of space to indent.
    {\bfseries}             % Font of the header of the theorem.
    {}                      % Punctuation between head and body.
    {.5em}                  % Space after theorem head.
    {}

% Italic header environment.
\newtheoremstyle{thmit}{\topsep}{\topsep}{}{}{\itshape}{}{0.5em}{}

% Define environments with italic headers.
\theoremstyle{thmit}
\newtheorem*{solution}{Solution}

% Define default environments.
\theoremstyle{normal}
\newtheorem{example}{Example}[section]
\newtheorem{definition}{Definition}[section]
\newtheorem{problem}{Problem}[section]

% Define framed environment.
\tcbuselibrary{most}
\newtcbtheorem[use counter*=theorem]{ftheorem}{Theorem}{%
    before=\par\vspace{2ex},
    boxsep=0.5\topsep,
    after=\par\vspace{2ex},
    colback=green!5,
    colframe=green!35!black,
    fonttitle=\bfseries\upshape%
}{thm}

\newtcbtheorem[auto counter, number within=section]{faxiom}{Axiom}{%
    before=\par\vspace{2ex},
    boxsep=0.5\topsep,
    after=\par\vspace{2ex},
    colback=Apricot!5,
    colframe=Apricot!35!black,
    fonttitle=\bfseries\upshape%
}{ax}

\newtcbtheorem[use counter*=definition]{fdefinition}{Definition}{%
    before=\par\vspace{2ex},
    boxsep=0.5\topsep,
    after=\par\vspace{2ex},
    colback=blue!5!white,
    colframe=blue!75!black,
    fonttitle=\bfseries\upshape%
}{def}

\newtcbtheorem[use counter*=example]{fexample}{Example}{%
    before=\par\vspace{2ex},
    boxsep=0.5\topsep,
    after=\par\vspace{2ex},
    colback=red!5!white,
    colframe=red!75!black,
    fonttitle=\bfseries\upshape%
}{ex}

\newtcbtheorem[auto counter, number within=section]{fnotation}{Notation}{%
    before=\par\vspace{2ex},
    boxsep=0.5\topsep,
    after=\par\vspace{2ex},
    colback=SeaGreen!5!white,
    colframe=SeaGreen!75!black,
    fonttitle=\bfseries\upshape%
}{not}

\newtcbtheorem[use counter*=remark]{fremark}{Remark}{%
    fonttitle=\bfseries\upshape,
    colback=Goldenrod!5!white,
    colframe=Goldenrod!75!black}{ex}

\newenvironment{bproof}{\textit{Proof.}}{\hfill$\square$}
\tcolorboxenvironment{bproof}{%
    blanker,
    breakable,
    left=3mm,
    before skip=5pt,
    after skip=10pt,
    borderline west={0.6mm}{0pt}{green!80!black}
}

\AtEndEnvironment{lexample}{$\hfill\textcolor{red}{\blacksquare}$}
\newtcbtheorem[use counter*=example]{lexample}{Example}{%
    empty,
    title={Example~\theexample},
    boxed title style={%
        empty,
        size=minimal,
        toprule=2pt,
        top=0.5\topsep,
    },
    coltitle=red,
    fonttitle=\bfseries,
    parbox=false,
    boxsep=0pt,
    before=\par\vspace{2ex},
    left=0pt,
    right=0pt,
    top=3ex,
    bottom=1ex,
    before=\par\vspace{2ex},
    after=\par\vspace{2ex},
    breakable,
    pad at break*=0mm,
    vfill before first,
    overlay unbroken={%
        \draw[red, line width=2pt]
            ([yshift=-1.2ex]title.south-|frame.west) to
            ([yshift=-1.2ex]title.south-|frame.east);
        },
    overlay first={%
        \draw[red, line width=2pt]
            ([yshift=-1.2ex]title.south-|frame.west) to
            ([yshift=-1.2ex]title.south-|frame.east);
    },
}{ex}

\AtEndEnvironment{ldefinition}{$\hfill\textcolor{Blue}{\blacksquare}$}
\newtcbtheorem[use counter*=definition]{ldefinition}{Definition}{%
    empty,
    title={Definition~\thedefinition:~{#1}},
    boxed title style={%
        empty,
        size=minimal,
        toprule=2pt,
        top=0.5\topsep,
    },
    coltitle=Blue,
    fonttitle=\bfseries,
    parbox=false,
    boxsep=0pt,
    before=\par\vspace{2ex},
    left=0pt,
    right=0pt,
    top=3ex,
    bottom=0pt,
    before=\par\vspace{2ex},
    after=\par\vspace{1ex},
    breakable,
    pad at break*=0mm,
    vfill before first,
    overlay unbroken={%
        \draw[Blue, line width=2pt]
            ([yshift=-1.2ex]title.south-|frame.west) to
            ([yshift=-1.2ex]title.south-|frame.east);
        },
    overlay first={%
        \draw[Blue, line width=2pt]
            ([yshift=-1.2ex]title.south-|frame.west) to
            ([yshift=-1.2ex]title.south-|frame.east);
    },
}{def}

\AtEndEnvironment{ltheorem}{$\hfill\textcolor{Green}{\blacksquare}$}
\newtcbtheorem[use counter*=theorem]{ltheorem}{Theorem}{%
    empty,
    title={Theorem~\thetheorem:~{#1}},
    boxed title style={%
        empty,
        size=minimal,
        toprule=2pt,
        top=0.5\topsep,
    },
    coltitle=Green,
    fonttitle=\bfseries,
    parbox=false,
    boxsep=0pt,
    before=\par\vspace{2ex},
    left=0pt,
    right=0pt,
    top=3ex,
    bottom=-1.5ex,
    breakable,
    pad at break*=0mm,
    vfill before first,
    overlay unbroken={%
        \draw[Green, line width=2pt]
            ([yshift=-1.2ex]title.south-|frame.west) to
            ([yshift=-1.2ex]title.south-|frame.east);},
    overlay first={%
        \draw[Green, line width=2pt]
            ([yshift=-1.2ex]title.south-|frame.west) to
            ([yshift=-1.2ex]title.south-|frame.east);
    }
}{thm}

%--------------------Declared Math Operators--------------------%
\DeclareMathOperator{\adjoint}{adj}         % Adjoint.
\DeclareMathOperator{\Card}{Card}           % Cardinality.
\DeclareMathOperator{\curl}{curl}           % Curl.
\DeclareMathOperator{\diam}{diam}           % Diameter.
\DeclareMathOperator{\dist}{dist}           % Distance.
\DeclareMathOperator{\Div}{div}             % Divergence.
\DeclareMathOperator{\Erf}{Erf}             % Error Function.
\DeclareMathOperator{\Erfc}{Erfc}           % Complementary Error Function.
\DeclareMathOperator{\Ext}{Ext}             % Exterior.
\DeclareMathOperator{\GCD}{GCD}             % Greatest common denominator.
\DeclareMathOperator{\grad}{grad}           % Gradient
\DeclareMathOperator{\Ima}{Im}              % Image.
\DeclareMathOperator{\Int}{Int}             % Interior.
\DeclareMathOperator{\LC}{LC}               % Leading coefficient.
\DeclareMathOperator{\LCM}{LCM}             % Least common multiple.
\DeclareMathOperator{\LM}{LM}               % Leading monomial.
\DeclareMathOperator{\LT}{LT}               % Leading term.
\DeclareMathOperator{\Mod}{mod}             % Modulus.
\DeclareMathOperator{\Mon}{Mon}             % Monomial.
\DeclareMathOperator{\multideg}{mutlideg}   % Multi-Degree (Graphs).
\DeclareMathOperator{\nul}{nul}             % Null space of operator.
\DeclareMathOperator{\Ord}{Ord}             % Ordinal of ordered set.
\DeclareMathOperator{\Prin}{Prin}           % Principal value.
\DeclareMathOperator{\proj}{proj}           % Projection.
\DeclareMathOperator{\Refl}{Refl}           % Reflection operator.
\DeclareMathOperator{\rk}{rk}               % Rank of operator.
\DeclareMathOperator{\sgn}{sgn}             % Sign of a number.
\DeclareMathOperator{\sinc}{sinc}           % Sinc function.
\DeclareMathOperator{\Span}{Span}           % Span of a set.
\DeclareMathOperator{\Spec}{Spec}           % Spectrum.
\DeclareMathOperator{\supp}{supp}           % Support
\DeclareMathOperator{\Tr}{Tr}               % Trace of matrix.
%--------------------Declared Math Symbols--------------------%
\DeclareMathSymbol{\minus}{\mathbin}{AMSa}{"39} % Unary minus sign.
%------------------------New Commands---------------------------%
\DeclarePairedDelimiter\norm{\lVert}{\rVert}
\DeclarePairedDelimiter\ceil{\lceil}{\rceil}
\DeclarePairedDelimiter\floor{\lfloor}{\rfloor}
\newcommand*\diff{\mathop{}\!\mathrm{d}}
\newcommand*\Diff[1]{\mathop{}\!\mathrm{d^#1}}
\renewcommand*{\glstextformat}[1]{\textcolor{RoyalBlue}{#1}}
\renewcommand{\glsnamefont}[1]{\textbf{#1}}
\renewcommand\labelitemii{$\circ$}
\renewcommand\thesubfigure{%
    \arabic{chapter}.\arabic{figure}.\arabic{subfigure}}
\addto\captionsenglish{\renewcommand{\figurename}{Fig.}}
\numberwithin{equation}{section}

\renewcommand{\vector}[1]{\boldsymbol{\mathrm{#1}}}

\newcommand{\uvector}[1]{\boldsymbol{\hat{\mathrm{#1}}}}
\newcommand{\topspace}[2][]{(#2,\tau_{#1})}
\newcommand{\measurespace}[2][]{(#2,\varSigma_{#1},\mu_{#1})}
\newcommand{\measurablespace}[2][]{(#2,\varSigma_{#1})}
\newcommand{\manifold}[2][]{(#2,\tau_{#1},\mathcal{A}_{#1})}
\newcommand{\tanspace}[2]{T_{#1}{#2}}
\newcommand{\cotanspace}[2]{T_{#1}^{*}{#2}}
\newcommand{\Ckspace}[3][\mathbb{R}]{C^{#2}(#3,#1)}
\newcommand{\funcspace}[2][\mathbb{R}]{\mathcal{F}(#2,#1)}
\newcommand{\smoothvecf}[1]{\mathfrak{X}(#1)}
\newcommand{\smoothonef}[1]{\mathfrak{X}^{*}(#1)}
\newcommand{\bracket}[2]{[#1,#2]}

%------------------------Book Command---------------------------%
\makeatletter
\renewcommand\@pnumwidth{1cm}
\newcounter{book}
\renewcommand\thebook{\@Roman\c@book}
\newcommand\book{%
    \if@openright
        \cleardoublepage
    \else
        \clearpage
    \fi
    \thispagestyle{plain}%
    \if@twocolumn
        \onecolumn
        \@tempswatrue
    \else
        \@tempswafalse
    \fi
    \null\vfil
    \secdef\@book\@sbook
}
\def\@book[#1]#2{%
    \refstepcounter{book}
    \addcontentsline{toc}{book}{\bookname\ \thebook:\hspace{1em}#1}
    \markboth{}{}
    {\centering
     \interlinepenalty\@M
     \normalfont
     \huge\bfseries\bookname\nobreakspace\thebook
     \par
     \vskip 20\p@
     \Huge\bfseries#2\par}%
    \@endbook}
\def\@sbook#1{%
    {\centering
     \interlinepenalty \@M
     \normalfont
     \Huge\bfseries#1\par}%
    \@endbook}
\def\@endbook{
    \vfil\newpage
        \if@twoside
            \if@openright
                \null
                \thispagestyle{empty}%
                \newpage
            \fi
        \fi
        \if@tempswa
            \twocolumn
        \fi
}
\newcommand*\l@book[2]{%
    \ifnum\c@tocdepth >-3\relax
        \addpenalty{-\@highpenalty}%
        \addvspace{2.25em\@plus\p@}%
        \setlength\@tempdima{3em}%
        \begingroup
            \parindent\z@\rightskip\@pnumwidth
            \parfillskip -\@pnumwidth
            {
                \leavevmode
                \Large\bfseries#1\hfill\hb@xt@\@pnumwidth{\hss#2}
            }
            \par
            \nobreak
            \global\@nobreaktrue
            \everypar{\global\@nobreakfalse\everypar{}}%
        \endgroup
    \fi}
\newcommand\bookname{Book}
\renewcommand{\thebook}{\texorpdfstring{\Numberstring{book}}{book}}
\providecommand*{\toclevel@book}{-2}
\makeatother
\titleformat{\part}[display]
    {\Large\bfseries}
    {\partname\nobreakspace\thepart}
    {0mm}
    {\Huge\bfseries}
\titlecontents{part}[0pt]
    {\large\bfseries}
    {\partname\ \thecontentslabel: \quad}
    {}
    {\hfill\contentspage}
\titlecontents{chapter}[0pt]
    {\bfseries}
    {\chaptername\ \thecontentslabel:\quad}
    {}
    {\hfill\contentspage}
\newglossarystyle{longpara}{%
    \setglossarystyle{long}%
    \renewenvironment{theglossary}{%
        \begin{longtable}[l]{{p{0.25\hsize}p{0.65\hsize}}}
    }{\end{longtable}}%
    \renewcommand{\glossentry}[2]{%
        \glstarget{##1}{\glossentryname{##1}}%
        &\glossentrydesc{##1}{~##2.}
        \tabularnewline%
        \tabularnewline
    }%
}
\newglossary[not-glg]{notation}{not-gls}{not-glo}{Notation}
\newcommand*{\newnotation}[4][]{%
    \newglossaryentry{#2}{type=notation, name={\textbf{#3}, },
                          text={#4}, description={#4},#1}%
}
%--------------------------LENGTHS------------------------------%
% Spacings for the Table of Contents.
\addtolength{\cftsecnumwidth}{1ex}
\addtolength{\cftsubsecindent}{1ex}
\addtolength{\cftsubsecnumwidth}{1ex}
\addtolength{\cftfignumwidth}{1ex}
\addtolength{\cfttabnumwidth}{1ex}

% Indent and paragraph spacing.
\setlength{\parindent}{0em}
\setlength{\parskip}{0em}                                                       %
%------------------------Main Document-------------------------------------%
\begin{document}
    \title{Topics in Algebra}
    \author{Ryan Maguire}
    \date{\vspace{-5ex}}
    \maketitle
    \tableofcontents
    \section{Homework I}
    \begin{problem}
        Show that the countable union of sets of measure zero has
        measure zero.
    \end{problem}
    \begin{solution}
        If $\mathcal{O}$ is a countable set of sets of measure zero then
        there is a bijection $A:\mathbb{N}\rightarrow\mathcal{O}$ such
        that, for all $n\in\mathbb{N}$, $\mu(A_{n})=0$. For countable
        additivity, we have countable subadditivity. That is:
        \begin{equation}
            \mu\Big(\bigcup_{n\in\mathbb{N}}A_{n}\Big)
            \leq\sum_{n\in\mathbb{N}}\mu(A_{n})
            =\underset{N\rightarrow\infty}{\lim}
                \sum_{n=1}^{N}\mu(A_{n})
            =\underset{N\rightarrow\infty}{\lim}0
            =0
        \end{equation}
        From the non-negativeness of measures, we have:
        \begin{equation}
            \mu\Big(\bigcup_{n\in\mathbb{N}}A_{n}\Big)=0
        \end{equation}
    \end{solution}
    \begin{problem}
        Suppose $f:[a,b]\rightarrow\mathbb{R}$ is bounded and let
        $\mathcal{P}$ and $\mathcal{Q}$ be subdivisions of $[a,b]$. Prove
        that $L(f,\mathcal{P})\leq{U}(f,\mathcal{P})$.
    \end{problem}
    \begin{solution}
        If $\mathcal{P}=\mathcal{Q}$, we have:
        \begin{equation}
                L(f,\mathcal{P})
                =\sum_{k\in\mathbb{Z}_{n}}m_{k}(t_{k+1}-t_{k})
                \leq\sum_{k\in\mathbb{Z}_{n}}M_{k}(t_{k+1}-t_{k})
                =U(f,\mathcal{P})
        \end{equation}
        Simply from the definition of $m_{k}$ and $M_{k}$. But also, if
        $\mathcal{P}\subseteq\mathcal{Q}$, then:
        \begin{equation}
            L(f,\mathcal{P})\leq{L}(f,\mathcal{Q})
            \quad\quad
            \textrm{and}
            \quad\quad
            U(f,\mathcal{Q})\leq{U}(f,\mathcal{P})
        \end{equation}
        But for any partitions $\mathcal{P}$ and $\mathcal{Q}$, we have
        that $\mathcal{P}\subseteq\mathcal{P}\cup\mathcal{Q}$ and
        $\mathcal{Q}\subseteq\mathcal{P}\cup\mathcal{Q}$, thus:
        \begin{equation}
            L(f,\mathcal{Q})\leq{L}(f,\mathcal{Q}\cup\mathcal{P})
            \leq{U}(f,\mathcal{Q}\cup\mathcal{P})
            \leq{U}(f,\mathcal{P})
        \end{equation}
        Proving the result.
    \end{solution}
    \begin{problem}
        Prove that a bounded function $f:[a,b]\rightarrow\mathbb{R}$ is
        Riemann integrable on $[a,b]$ if and only if for all
        $\varepsilon>0$ there is a partition $\mathcal{P}$ of $[a,b]$ such
        that:
        \begin{equation}
            U(f,\mathcal{P})-L(f,\mathcal{P})<\varepsilon
        \end{equation}
    \end{problem}
    \begin{solution}
        For suppose $f$ is Riemann integrable. Then:
        \begin{subequations}
            \begin{align}
                \underline{\mathcal{R}}\int_{a}^{b}f
                &=\sup\Big\{\,
                    L(f,\mathcal{P})\,:\,
                    \mathcal{P}\textrm{ is a partition of }[a,b]\,\Big\}\\
                &=\inf\Big\{\,
                    U(f,\mathcal{P})\,:\,
                    \mathcal{P}\textrm{ is a partition of }[a,b]\,\Big\}\\
                &=\overline{\mathcal{R}}\int_{a}^{b}f
            \end{align}
        \end{subequations}
        Taking the difference, we have that:
        \begin{equation}
            \sup\{\,U(f,\mathcal{P})\,:\,\mathcal{P}\,\}-
            \inf\{\,L(f,\mathcal{P})\,:\,\mathcal{P}\,\}=0
        \end{equation}
        And thus for all $\varepsilon>0$ there is a partition $\mathcal{P}$
        such that:
        \begin{equation}
            |U(f,\mathcal{P})-L(f,\mathcal{P})|<\varepsilon
        \end{equation}
        But for all partitions $\mathcal{P}$,
        $U(f,\mathcal{P})\geq{L}(f,\mathcal{P})$, and thus:
        \begin{equation}
            U(f,\mathcal{P})-L(f,\mathcal{P})<\varepsilon
        \end{equation}
        Going the other way, if for all $\varepsilon>0$ there exists such
        a partition, let $\mathcal{P}_{n}$ be a partition such that:
        \begin{equation}
            U(f,\mathcal{P}_{n})-L(f,\mathcal{P}_{n})<n^{\minus{1}}
        \end{equation}
        But $U(f,\mathcal{P}_{n})$ is bounded below by
        $L(f,\mathcal{P}_{0})$ and is a monotonically decreasing
        sequence of real numbers, and thus from the completeness of
        $\mathbb{R}$ there exists a real number $R$ such that
        $U(f,\mathcal{P}_{n})\rightarrow{R}$. It then follows that
        $L(f,\mathcal{P}_{n})\rightarrow{R}$. And since, for any partitions
        $\mathcal{P}$ and $\mathcal{Q}$, we have that
        $U(f,\mathcal{P})\geq{L}(f,\mathcal{Q})$, it follows that:
        \begin{equation}
            \underline{\mathcal{R}}\int_{a}^{b}f
            =\overline{\mathcal{R}}\int_{a}^{b}f
        \end{equation}
        And therefore $f$ is Riemann integrable.
    \end{solution}
    \begin{problem}
        Suppose that $(X,\mathcal{M})$ is a measurable space. Show that
        if $\mathcal{M}$ is countable, then $\mathcal{M}$ is finite.
    \end{problem}
    \begin{solution}
        For suppose not. Suppose that $\mathcal{M}$ is a countably
        infinite $\sigma\textrm{-Algebra}$ on $X$. For all $x\in{X}$,
        let $\mathcal{O}_{x}$ be defined as:
        \begin{equation}
            \mathcal{O}_{x}=\big\{\,E\in\mathcal{M}\,:\,x\in{E}\,\}
        \end{equation}
        Let $\omega_{x}$ be defined as:
        \begin{equation}
            \omega_{x}=\bigcap_{E\in\mathcal{O}_{x}}E
        \end{equation}
        If $x\ne{y}$ then either $\omega_{x}=\omega_{y}$ or
        $\omega_{x}\cap\omega_{y}=\emptyset$. For if not, then
        $\omega_{x}\setminus\omega_{y}\in\mathcal{M}$ is a measurable
        set that contains $x$ but not $y$, contradicting our construction
        of $\omega_{x}$. Thus the $\omega_{x}$ partition $X$. Invoking
        choice and the countability of $\mathcal{M}$, there
        is a sequence $a:\mathbb{N}\rightarrow{X}$ such that:
        \begin{equation}
            X=\bigcup_{n\in\mathbb{N}}\omega_{a_{n}}
        \end{equation}
        And for all $n\ne{m}$:
        \begin{equation}
            \omega_{a_{n}}\cap\omega_{a_{m}}=\emptyset
        \end{equation}
        That is, $\{\omega_{a_{n}}:n\in\mathbb{N}\}$ is a mutually disjoint
        collection of measurable sets that cover $X$. But
        $\sigma\textrm{-Algebras}$ are closed to countable unions, and
        thus for all $S\subseteq\mathbb{N}$ the set:
        \begin{equation}
            E_{S}=\bigcup_{n\in{S}}\omega_{a_{n}}
        \end{equation}
        Is a measurable set. Moreover, for all
        $S_{1},S_{2}\subseteq\mathbb{N}$ such that $S_{1}\ne{S}_{2}$,
        we have that $E_{S_{1}}\ne{E}_{{S}_{2}}$. But then:
        \begin{equation}
            \mathcal{A}=\big\{\,E_{S}\,:\,S\subseteq\mathbb{N}\,\}
        \end{equation}
        Is a a subset of $\mathcal{M}$ that can be put into a bijection
        with $\mathcal{P}(\mathbb{N})$, the power set of $\mathbb{N}$.
        But then $\mathcal{M}$ contains an uncountable subset, a
        contradiction as $\mathcal{M}$ is countably infinite. Thus,
        $\mathcal{M}$ is finite.
    \end{solution}
    \begin{problem}
        Let $X$ be an uncountable set and let $\mathcal{M}$ be the
        collection of subsets $E$ of $X$ such that either $E$ or
        $E^{C}$ is countable. Prove that $\mathcal{M}$ is a
        $\sigma\textrm{-Algebra}$.
    \end{problem}
    \begin{solution}
        It is closed to complements by design. Moreover, since
        $X^{C}=\emptyset$, $X\in\mathcal{M}$. It thus suffices to show
        that the set is closed to countable unions. Given a
        countable collection $\mathcal{O}\subseteq\mathcal{M}$, either it
        contains an uncountably infinite set or there does not. If there
        does not, then the union over $\mathcal{O}$ is the countable
        union of countable sets, which is countable, and is thus an element
        of $\mathcal{M}$. If there is an uncountably infinite set,
        denoted $\mathcal{U}$, then:
        \begin{equation}
            \textrm{Card}\Big(\big(\bigcup_{E\in\mathcal{O}}E\big)^{C}\Big)
            \leq\textrm{Card}(\mathcal{U}^{C})
        \end{equation}
        But since $\mathcal{U}\in\mathcal{M}$, it follows that
        $\mathcal{U}^{C}$ is countable. Thus the union over $\mathcal{O}$
        is the subset of a countable set, and is therefore countable.
        Thus, $\mathcal{M}$ is closed to unions.
    \end{solution}
    \begin{problem}
        Recall from calculus that if $a:\mathbb{N}\rightarrow\mathbb{R}$
        is a non-negative sequence of real numbers, then:
        \begin{equation}
            \sum_{n=1}^{\infty}a_{n}
            =\sup\Big\{\,\sum_{n=1}^{N}a_{n}\,:\,N\in\mathbb{N}\,\Big\}
        \end{equation}
        \begin{enumerate}
            \item   Show that:
                    \begin{equation}
                        \sum_{n=1}^{\infty}a_{n}
                        =\sup\Big\{\,
                            \sum_{n\in{F}}a_{n}\,:\,
                            F\subseteq\mathbb{N},\,
                            \textrm{Card}(F)\in\mathbb{N}\,
                        \Big\}
                    \end{equation}
            \item   Now let $X$ be a set and $f:X\rightarrow[0,\infty)$
                    a function. For each $E\subseteq{X}$ define:
                    \begin{equation}
                        \nu(E)=\sum_{x\in{E}}f(x)
                    \end{equation}
                    Show that $\nu$ is a measure on $(X,\mathcal{P}(X))$.
            \item   Show that, if $X$, $f$, and $\nu$ are as defined in the
                    previous part, and if $\nu(E)\infty$, then the set:
                    \begin{equation}
                        \mathcal{O}=\{\,x\in{E}\,:\,f(x)>0\,\}
                    \end{equation}
                    Is countable.
        \end{enumerate}
    \end{problem}
    \begin{solution}
        For we have:
        \begin{equation}
            \sup\Big\{\,
                \sum_{n\in{F}}a_{n}\,:\,F\subseteq\mathbb{N},\,
                \textrm{Card}(F)\in\mathbb{N}\,
            \Big\}
            \leq\sum_{n=1}^{\infty}
        \end{equation}
        For let $F:\mathbb{N}\rightarrow\mathcal{P}(\mathbb{N})$ be
        sequence of finite sets such that:
        \begin{equation}
            \underset{n\rightarrow\infty}{\lim}
                \sum_{k\in{F}_{n}}a_{n}
            =\sup\Big\{\,
                \sum_{n\in{F}}a_{n}\,:\,F\subseteq\mathbb{N},\,
                \textrm{Card}(F)\in\mathbb{N}\,
            \Big\}
        \end{equation}
        Then if we set:
        \begin{equation}
            \mathcal{F}=\bigcup_{n\in\mathbb{N}}F_{n}\subseteq\mathbb{N}
        \end{equation}
        We obtain (from the non-negativity of the sequence $a$):
        \begin{subequations}
            \begin{align}
                \underset{n\rightarrow\infty}{\lim}\sum_{k\in{F}_{n}}a_{k}
                \leq\sum_{k\in\mathcal{F}}a_{k}
                \leq\sum_{k\in\mathbb{N}}a_{k}
            \end{align}
        \end{subequations}
        Thus proving our inequality. Going the other way, if we let
        $F_{n}=\mathbb{Z}_{n}$, then:
        \begin{equation}
            \underset{n\rightarrow\infty}{\lim}\sum_{k\in{F}_{n}}a_{k}
            =\underset{n\rightarrow\infty}{\lim}\sum_{k\in\mathbb{Z}_{n}}
            =\sum_{n\in\mathbb{N}}a_{n}
        \end{equation}
        And therefore:
        \begin{equation}
            \sup\Big\{\,
                \sum_{n\in{F}}a_{n}\,:\,F\subseteq\mathbb{N},\,
                \textrm{Card}(F)\in\mathbb{N}\,
            \Big\}
            \leq\sum_{n\in\mathbb{N}}a_{n}
        \end{equation}
        Comparing inequality completes the proof.
        \par\hfill\par
        To show that $\nu$ is a measure, we must show that it's range
        is non-negative range, the $\nu(\emptyset)=0$, and that it is
        countably additive. It is non-negative since our function $f$ has
        non-negative range. Moreover, we have $\nu(\emptyset)=0$ in a
        vacuous manner. Lastly, countable additivity. If $\mathcal{O}$
        is a countable mutually disjoint collection of sets, let
        $\mathcal{U}:\mathbb{N}\rightarrow\mathcal{O}$ be a bijection.
        Then:
        \begin{subequations}
            \begin{align}
                \nu\Big(\bigcup_{n\in\mathbb{N}}\mathcal{U}_{n}\Big)
                &=\sum_{x\in\bigcup_{n\in\mathbb{N}}\mathcal{U}_{n}}f(x)\\
                &=\sup\Big\{\,\sum_{x\in{F}}f(x)\,:\,
                    F\subseteq\bigcup_{n\in\mathbb{N}}\mathcal{U}_{n},\,
                    \textrm{Card}(F)\in\mathbb{N}\,\Big\}\\
                &=\sup\Big\{\,\sum_{n\in\mathbb{N}}
                    \sum_{x\in{F}_{n}}f(x)\,:\,
                    F_{n}\subseteq\mathcal{U}_{n},\,
                    \textrm{Card}(F_{n})\in\mathbb{N}\,\Big\}\\
                &=\sum_{n\in\mathbb{N}}
                    \sup\Big\{\,\sum_{x\in{F}_{n}}f(x)\,:\,
                        F_{n}\subseteq\mathcal{U}_{n},\,
                        \textrm{Card}(F_{n})\in\mathbb{N}\,\Big\}\\
                &=\sum_{n\in\mathbb{N}}\nu\big(\mathcal{U}_{n}\big)
            \end{align}
        \end{subequations}
        And lastly, support the set is uncountable. For all $n\in\mathbb{N}$
        the set:
        \begin{equation}
            A_{n}=\{\,x\in{E}\,:\,f(x)>n^{\minus{1}}\,\}
        \end{equation}
        Is finite. For if not then there is a sequence of subsets of
        $E$ whose sums of $f(x)$ are unbounded, and thus $\nu(E)=\infty$,
        a contradiction. But then:
        \begin{equation}
            \mathcal{O}=\bigcup_{n\in\mathbb{N}}A_{n}
        \end{equation}
        And thus $\mathcal{O}$ is the countable union of finite sets, and
        is therefore countable, a contradiction. Thus, $\mathcal{O}$ is
        not uncountable.
    \end{solution}
    \begin{problem}
        Prove that if $f$ is a real-valued function on a measurable
        space $(X,\mathcal{M})$ such that $\{\,x\,:\,f(x)\geq{r}\,\}$
        is measurable for all $r\in\mathbb{Q}$, then $f$ is measurable.
    \end{problem}
    \begin{proof}
        This follows since the collection of all $[a,\infty)$ and
        $(\minus\infty,a]$, for $a\in\mathbb{Q}$, generate the same
        $\sigma\textrm{-Algebra}$ as $\mathcal{B}$, the Borel
        $\sigma\textrm{-Algebra}$ on $\mathbb{R}$. It suffices to show
        that open intervals are contained in the $\sigma\textrm{-Algebra}$
        generated by this collection. Let
        $r_{1},r_{2}\in\mathbb{R}$, $r_{1}<r_{2}$, and let
        $a,b:\mathbb{N}\rightarrow\mathbb{Q}$ be sequences of rational
        numbers such that $a_{n}\rightarrow{r}_{1}$, and such that
        $a_{n}$ is monotonically increasing, and similarly
        $b_{n}\rightarrow{r}_{1}$ and $b_{n}$ is monotonically decreasing.
        Then:
        \begin{equation}
            [r_{1},r_{2}]
            =\bigcap_{n\in\mathbb{N}}
                \Big((\minus\infty,b_{n}]\cup[a_{n},\infty)\Big)
        \end{equation}
        And thus this $\sigma\textrm{-Algebra}$ is closed to closed
        intervals. But $\sigma\textrm{-Algebras}$ are closed to
        complements, and thus $(\minus{\infty},a)\cup(b,\infty)$ is
        contained as well. From DeMorgan's Law, $\sigma\textrm{-Algebras}$
        are closed to intersections, and from this we have that the
        $\sigma\textrm{-Algebra}$ contains open intervals. Since
        $f$ is measurable on a generating set of $\mathcal{B}$, it follows
        that $f$ is measurable.
    \end{proof}
    \begin{problem}
        Suppose that $f,g:(X,\mathcal{M})\rightarrow[\minus\infty,\infty]$
        are measurable functions. Prove that the two sets:
        \begin{equation}
            \{\,x\,:\,f(x)<g(x)\,\}
            \quad\quad
            \textrm{and}
            \quad\quad
            \{\,x\,:\,f(x)=g(x)\,\}
        \end{equation}
        Are measurable.
    \end{problem}
\end{document}