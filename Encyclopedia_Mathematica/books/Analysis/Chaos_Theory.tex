\chapter{Chaos Theory}
    \section{A Review of Differential Equations}
        \subsection{First Order Equations}
            The first differential equation that is often studied
            is $\dot{x}(t)=ax(t)$, where $\dot{x}$ denotes the
            derivative of $x$ with respect to $t$. In this equation
            $a$ is some fixed constant parameter, and each real
            value $a$ defines a different differential equation.
            We can solve this by integrating and invoking the
            fundamental theorem of calculus. In general, a
            differential equation is an equation that relates
            a differentiable functions to its derivatives.
            The general solution to a differential equation is
            the set of all functions that satisfy the
            differential equation.
            \begin{ftheorem}{}{thm:CHAOS:SIMPLE_DIFF_EQ}
                If $x(t)$ is a differentiable function such
                that $x(0)=x_{0}$ and there is a $k\in\mathbb{R}$
                such that for all $t\in\mathbb{R}$, $\dot{x}(t)=kx(t)$,
                then $x(t)=x_{0}\exp(kt)$.
            \end{ftheorem}
            \begin{bproof}
                Let $y(t)=\exp(kt)$. Then $\dot{y}(t)=ky(t)$,
                and $y(t)\ne{0}$ for all $t\in\mathbb{R}$.
                Let $F(t)=x(t)/y(t)$. Then:
                \begin{equation*}
                    \dot{F}(t)=
                    \frac{y(t)\dot{x}(t)-\dot{y}(t)x(t)}{y^{2}(t)}
                    =\frac{y(t)\big(\dot{x}(t)-kx(t)\big)}{y^{2}(t)}
                    =\frac{1}{y(t)}\big(\dot{x}(t)-kx(t)\big)
                \end{equation*}
                But $\dot{x}(t)-kx(t)=0$, and thus $\dot{F}(t)=0$ for
                all $t\in\mathbb{R}$. But then $F(t)$ is a constant
                function. Thus there is a $C\in\mathbb{R}$ such
                that $x(t)=Cy(t)$. But $x(0)=x_{0}$, and therefore
                $C=x_{0}$. Thus, $x(t)=x_{0}\exp(kt)$.
            \end{bproof}
            The proof shows that the solution to $\dot{x}(t)=kx(t)$
            is uniquely determined if $x(0)$ is known. This can
            be replaced by knowledge of $x(t_{0})$ for any point
            $t_{0}\in\mathbb{R}$.
            Such restraints are called initial conditions to a
            differential equation. Initial conditions are
            requirements that a solution to the differential
            equation must satisfy. Such requirements can be
            the value of $x(t_{0})$ at a certain point
            $t_{0}$, or a requirement on $\dot{x}(t)$.
            \begin{fexample}{}{}
                Consider the following initial value problem:
                \begin{align}
                    \dot{x}(t)&=ax(t)
                    &
                    x(t_{0})&=x_{0}
                \end{align}
                Using the previous theorem, we can translate the
                problem by $\exp(-kt_{0})$ to create the initial
                value problem $\dot{y}(t)=ky(t)$, $y(0)=x_{0}$.
                We know the solution to this is
                $y(t)=x_{0}\exp(kt)$. Thus, the solution to the
                original initial value problem is:
                \begin{equation}
                    x(t)=x_{0}\exp(k(t-t_{0}))
                \end{equation}
            \end{fexample}
            An important class of solutions to differential
            equations are \textit{equilibrium solutions}.
            \begin{fdefinition}{Equilibrium Solution}{}
                An equilibrium solution to a differential
                equation is a solution $x(t)$ such that
                $\dot{x}(t)=0$ for all $t\in\mathbb{R}$.
                That is, a constant solution $x(t)=c$.
            \end{fdefinition}
            Studying our first problem $\dot{x}(t)=kx(t)$,
            we see that the only equilibrium solution
            occurs when $x_{0}=0$. Another important concept
            is the limiting behavior of the solution to a
            differential equation. The limiting behavior
            is often dependent on the various parameters that
            may be present in a given differential equation.
            We can see this by further studying $\dot{x}(t)=kx(t)$.
            If $k=0$ we see that $x(t)$ is a constant function. That
            is, $x(t)=x_{0}$ for all $t\in\mathbb{R}$. For $k>0$,
            solutions diverges monotonically to $\infty$
            ($x_{0}>0$) or $-\infty$ ($x_{0}<0$).
            Finally, if $k<0$ then $x(t)$ converges to $0$
            for all values of $x_{0}$. This notion can be represented
            by a \textit{phase line}. A phase line is a one
            dimensional plot of the independent variable $t$ where
            equilibrium solutions are marked and arrows indicating
            convergence or divergence to the equilibrium solutions
            are drawn. Consider again the example
            we've been studying:
            $\dot{x}(t)=kx(t)$. Suppose $k<0$. We know that, for any
            $x_{0}$, the solution $x(t)$ tends to zero as $t$ tends
            to infinity. We can represent this with a phase line:
            \begin{figure}
                \captionsetup{type=figure}
                \centering
                \begin{tikzpicture}
    \draw[%
        postaction={decorate},
        decoration={%
            markings,
            mark=at position 0.00 with \arrowreversed{latex},
            mark=at position 0.25 with \arrow{stealth},
            mark=at position 0.30 with \arrow{stealth},
            mark=at position 0.50 with
                {\draw[thin] (0, -1mm) to (0, 1mm) node [below=2mm] {0};},
            mark=at position 0.75 with \arrowreversed{stealth},
            mark=at position 0.70 with \arrowreversed{stealth},
            mark=at position 1.00 with \arrow{latex},
        },
    ]   (-2,0) to (2,0) node[below] {$x_{0}$};
\end{tikzpicture}
                \caption{Phase line for $\dot{x}(t)=kx(t)$
                         when $k<0$.}
                \label{fig:CHAOS:Phase_Line_Example}
            \end{figure}
            The initial condition $x_{0}=0$ thus gives rise to a
            \textit{stable} solution to this differential equation.
            \begin{fdefinition}{Stable Equilibrium Solution}   
                {diffeq:Stable_Equilibrium_Solution}
                A stable equilibrium solution to a differential
                equation $\dot{x}(t)=f(t,x(t))$, is
                an equilibrium solution $x(t)=x_{0}$
                such that there exists a
                $\delta>0$ such that for all
                $x_{1}\in(x_{0}-\delta,x_{0}+\delta)$, the solution
                to the initial value problem
                $\dot{x}(t)=f(t,x(t))$, $x(t_{0})=x_{1}$ converges to
                $x_{0}$ as $t\rightarrow\infty$
            \end{fdefinition}
            Being purely stable can be replaced with stable from above
            or stable from below. An equilibrium solution is called
            semi-stable if it is either stable from above or stable
            from below, but not totally stable. An unstable equilibrium
            solution is an equilibrium solution that is neither stable
            nor semi-stable. The stability of solutions often depends
            on the parameters involved in the differential equation.
            In the case of $\dot{x}(t)=kx(t)$ we saw that as
            $k$ cross zero the solutions drastically change. The
            value $k=0$ is said to be a \textit{bifurcation} value
            of the differential equation. Bifurcations are values that
            a parameter can have that alter the limiting behavior of
            solutions. $k=0$ is a bifurcation since, for all $k>0$
            we have that $x(t)$ diverges monotonically, for all
            $k<0$ we see that $x(t)$ converges to zero monotonically,
            and for $k=0$ all solutions are constants. Because of this,
            we say that there is a bifurcation at $k=0$.
            \subsubsection{The Logistics Population Model}
                First order differential equations have many
                applications in both the physical and the life
                sciences. Population modeling is a common such
                application. The growth rate of a population can
                be modelled based on two simple assumptions. The
                first is that, if the population is small, the
                growth rate is roughly proportional to the population.
                The second is that, if the population is too large,
                the growth rate is negative. For example, if the
                population is too large then there is not enough food
                and thus the population will start to decrease.
                To bridge the gap between these two assumptions
                we could make the rather reasonable assumption
                that the ratio of the population growth to the
                population is linear. Thus:
                \begin{equation}
                    \label{diffeq:Simple_Logistic_Population_Model}
                    \frac{\dot{x}(t)}{x(t)}=ax(t)+b
                \end{equation}
                Since the growth rate is negative for large
                $x_{0}$ and positive for small $x_{0}$, there
                must be a value $N$, called the
                \textit{ideal population}, such that
                $\dot{x}(t)=0$ for all $t$. If we let $\alpha$ be
                the \textit{growth factor} of the population,
                Eqn.~\ref{diffeq:Simple_Logistic_Population_Model}
                becomes:
                \begin{equation}
                    \label{diffeq:Logistics_Population_Model_DE}
                    \dot{x}(t)=\alpha{x}(t)\big(1-\frac{x(t)}{N}\big)
                \end{equation}
                By hypothesis, $\alpha$ must be positive. For
                if $x(t)>N$ and if $\alpha<0$, then
                $\dot{x}(t)>0$. This would be quite strange as
                large populations would increase rapidly to infinity.
                From physical considerations we also require that
                $N$ be positive. Otherwise we'd be speaking of
                \textit{negative population}, which is nonsense.
                For the sake of easing the mathematics, we may
                normalize the problem so that the ideal population
                is $N=1$. Let $k$ be the normalized growth factor.
                This gives us the following normalized
                logistics population model:
                \begin{equation}
                    \label{diffeq:Normalized_Log_Pop_Model_DE}
                    \dot{x}(t)=kx(t)\big(1-x(t)\big)
                \end{equation}
                The logistics population model is another example
                of a first order differential equation. In general,
                a first order differential equation is of the form
                $x(t)=f(t,x(t))$. An $n^{th}$ order differential
                equation is one of the form:
                \begin{equation*}
                    x^{(n)}(t)
                    =f(t,x(t),\dot{x}(t),\ddot{x}(t),
                       x^{(3)}(t),\hdots,x^{(n-1)}(t))
                \end{equation*}
                Here we have used the notation
                $x^{(n)}(t)$ to represent the $n^{th}$ deritative
                of $x$ with respect to $t$. A special subset of the
                general $n^{th}$ order differential equation is that
                of the autonomous $n^{th}$ order differential
                equations. These are differential equations where
                $f$ is solely a function of $x$ and its derivatives.
                \begin{fdefinition}{Autonomous Differential Equation}
                    {diffeq:Autonomous_DE}
                    An autonomous $n^{th}$ order differential
                    equation is a differential equation of the
                    form:
                    \begin{equation}
                        x^{(n)}(t)=f(x(t),\hdots,x^{(n-1)}(t))
                    \end{equation}
                \end{fdefinition}
                \begin{fexample}{First Order Autonomous 
                                 Differential Equations}{}
                    Consider the case of first order autonomous
                    differential equations. Here we'd have:
                    \begin{equation}
                        \dot{x}(t)=f(x(t))
                    \end{equation}
                    If $f$ is continuous then we can integrate this
                    and obtain a solution for $x$:
                    \begin{equation}
                        F(x)\equiv\int\frac{1}{f(x)}\diff{x}
                        =\int\diff{t}
                    \end{equation}
                    If the integral on the left hand side produces
                    an invertible function, then we can solve the
                    differential equation and obtain
                    $x(t)=F^{-1}(t+C)$, where $C$ is a constant
                    of integration. An equation like this is
                    also called \textit{separable}. That is, we
                    can separate the $x$ and $y$ terms to create
                    an expression on the left hand side which is
                    written purely in $x$, and an expression on
                    the right which is written purely in $t$.
                \end{fexample}
                The logistics population model is therefore a
                nonlinear first order autonomous differential
                equation. Nonlinear differential equations are
                usually very difficult to solve. When we have
                autonomy, however, the problem can become much
                easier. In the case of the logistics model,
                we solve for $x(t)$ using standard techniques
                from Calculus.
                \begin{equation}
                    \int\frac{1}{x(1-x)}\diff{x}=k\int\diff{t}
                \end{equation}
                Recalling from elementary algebra
                the method of partial fraction decomposition,
                we have:
                \begin{equation}
                    \frac{1}{x(1-x)}=\frac{1}{x}+\frac{1}{1-x}
                \end{equation}
                This simplifies the integral, and so we obtain:
                \begin{equation}
                    \int\Big(\frac{1}{x}+\frac{1}{1-x}\Big)\diff{x}
                    =\ln(x)-\ln(1-x)
                    =\ln\big(\frac{x}{1-x}\big)
                \end{equation}
                Evaluating the right-hand side and exponentiating,
                we get:
                \begin{equation}
                    \frac{x}{1-x}=A\exp(kt)
                \end{equation}
                Where $A$ is the exponential of the constant of
                integration. The left-hand side is and invertible
                function and its inverse is $x/(1+x)$. From this we
                obtain the solution to the logistics population model:
                \begin{equation}
                    x(t)=\frac{1}{1+C\exp(-kt)}
                \end{equation}
                By studying Eqn.~\ref{diffeq:Normalized_Log_Pop_Model_DE}
                we can find two equilibrium solution: $x(t)=0$ and
                $x(t)=1$. This makes physical sense, for if $x(t)=0$
                then the population is extinct and nothing more can be
                done, and if $x(t)=1$ then the population has reached its
                ideal value. From intuition it would seem that $x(t)=0$
                is either unstable or semistable (But since we ignore
                negative populations, this would simply be unstable) and
                that $x(t)=1$ would be stable. And indeed, if
                $0<x(t)<1$, the from
                Eqn.~\ref{diffeq:Normalized_Log_Pop_Model_DE} we have
                $\dot{x}(t)>0$, so the population is increasing
                (To the ideal value). If $x(t)>1$, then $\dot{x}(t)<0$
                and thus the population is decreasing
                (Again, towards its ideal value). We can summarize this
                more generally for first-order autonomous differential
                equations. First we prove a helpful intermediate step.
                \begin{theorem}
                    If $x(t)$ is a differentiable function such that
                    $x(t)\rightarrow{a}$ as $t\rightarrow\infty$, then
                    there is a strictly increasing monotonic sequence
                    $c_{n}$ such that $\dot{x}(c_{n})\rightarrow{0}$.
                \end{theorem}
                \begin{proof}
                    For let $t_{n}=n$ for all $n\in\mathbb{N}$. Then
                    $x(t_{n})\rightarrow{a}$ as $t_{n}\rightarrow\infty$.
                    But convegent sequences are Cauchy sequences, and
                    therefore $x(t_{n+1})-x(t_{n})\rightarrow{0}$.
                    And by the mean value theorem, for all
                    $n\in\mathbb{N}$ there is a
                    $c_{n}\in(t_{n},t_{n+1})$ such that:
                    \begin{equation*}
                        \dot{x}(c_{n})
                        =\frac{x(t_{n+1})-x(t_{n})}{t_{n+1}-t_{n}}
                    \end{equation*}
                    But $t_{n+1}-t_{n}=1$, and thus
                    $\dot{x}(c_{n})=x(t_{n+1})-x(t_{n})$. But
                    $x(t_{n+1})-x(t_{n})\rightarrow{0}$ and therefore
                    $\dot{x}(c_{n})\rightarrow{0}$.
                \end{proof}
                \begin{theorem}
                    If $f:\mathbb{R}\rightarrow\mathbb{R}$ is differentiable,
                    and given the differential equation
                    $\dot{x}(t)=f(x(t))$, if $x_{0}$ is an equilibrium
                    solution and $f'(x_{0})>0$, then $x_{0}$ is
                    an unstable equilibrium solution.
                \end{theorem}
                \begin{proof}
                    If $f(x_{0})=0$ and $f'(x_{0})>0$, then there is a
                    $\delta>0$ such that, for all
                    $x\in(x_{0}-\delta,x_{0}+\delta)$, we have:
                    \begin{equation*}
                        \frac{f(x)-f(x_{0})}{x-x_{0}}>0
                    \end{equation*}
                    But $f(x_{0})=0$. Therefore, if $x_{1}>x_{0}$, we have:
                    \begin{equation*}
                        \frac{f(x_{1})-f(x_{0})}{x_{1}-x_{0}}>0
                        \Rightarrow
                        f(x_{1})-f(x_{0})>0
                        \Rightarrow
                        f(x_{1})>0
                    \end{equation*}
                    Thus, for all $x_{1}$ such that
                    $0<x_{1}-x_{0}<\delta$, the limit of
                    the solution $x(t)$ to the initial value problem
                    $\dot{x}(t)=f(x(t))$, $x(0)=x_{1}$ can't converge to
                    $x_{0}$. For if it did there would be a point
                    $t_{0}$ such that $\dot{x}(t_{0})<0$ and
                    $0<x(t_{0})-x_{0}<\delta$, a contradiction. Similarly
                    for if $x_{1}<x_{0}$. Therefore, $x_{0}$ is unstable.
                \end{proof}
                \begin{theorem}
                    If $f:\mathbb{R}\rightarrow\mathbb{R}$ is differentiable,
                    and given the differential equation
                    $\dot{x}(t)=f(x(t))$, if $x_{0}$ is an equilibrium
                    solution and $f'(x_{0})<0$, then $x_{0}$ is
                    a stable equilibrium solution.
                \end{theorem}
                \begin{proof}
                    If $f'(x_{0})<0$ then there is a $\delta>0$ such
                    that for all $x_{1}\in(x_{0}-\delta,x_{0}+\delta)$,
                    $f'(x_{1})<0$. Then for all $x_{1}$ such that
                    $0<x_{1}-x_{0}<\delta$, we have:
                    \begin{equation*}
                        \frac{f(x_{1})-f(x_{0})}{x_{1}-x_{0}}<0
                        \Rightarrow
                        f(x_{1})-f(x_{0})<0
                        \Rightarrow
                        f(x_{1})<0
                    \end{equation*}
                    
                \end{proof}
        \subsubsection{Notes on the Jordan Normal Form}
            Every square matrix $A$ is similar to an upper
            triangular matrix $J$ in
            \textit{Jordan normal form} whose diagonal entries
            are the eigenvalues of $A$. That is, there exists
            an invertible matrix $P$ such that $P^{-1}AP=J$.
            The trace of $A$ is equal to the trace of $J$:
            \begin{equation*}
                \Tr(J)
                =\Tr(P^{-1}AP)
                =\Tr(P^{-1}PA)
                =\Tr(IA)=\Tr(A)
            \end{equation*}
        \subsubsection{Notes on Conjugacy}
            \begin{minipage}[t]{0.49\textwidth}
                We have:
                \begin{align*}
                    X'(t)&=AX(t),\quad
                    X(0)=X_{0}\\
                    \Rightarrow
                    X(t)&=e^{tA}X_{0}\\
                    \Rightarrow
                    \phi^{A}(t,X_{0})
                    &=e^{tA}X_{0}
                \end{align*}
            \end{minipage}
            \vline
            \hfill
            \begin{minipage}[t]{0.49\textwidth}
                If $B=T^{-1}AT$, for some matrix $T$, then:
                \begin{align*}
                    Y'(t)&=BY(t),\quad
                    Y(0)=Y_{0}\\
                    &=T^{-1}ATY(t)\\
                    \Rightarrow
                    Y(t)
                    &=e^{tT^{-1}AT}Y_{0}\\
                    &=T^{-1}e^{tA}TY_{0}\\
                    \Rightarrow
                    \phi^{B}(t,Y_{0})
                    &=T^{-1}e^{tA}TY_{0}
                \end{align*}
            \end{minipage}
            Thus, the homeomorphism is $h(X)=T^{-1}X$,
            and we have:
            \begin{equation*}
                \phi^{B}(t,h(X_{0}))
                =\phi^{B}(t,T^{-1}X_{0})
                =T^{-1}e^{tA}T(T^{-1}X_{0})
                =T^{-1}e^{tA}X_{0}
                =h(\phi^{A}(t,X_{0}))
            \end{equation*}