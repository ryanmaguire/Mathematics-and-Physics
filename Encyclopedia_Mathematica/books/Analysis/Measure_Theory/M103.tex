%------------------------------------------------------------------------------%
\documentclass[crop=false,class=article]{standalone}                           %
%------------------------------Preamble----------------------------------------%
\makeatletter                                                                  %
    \def\input@path{{../../../../}}                                            %
\makeatother                                                                   %
%---------------------------Packages----------------------------%
\usepackage{geometry}
\geometry{b5paper, margin=1.0in}
\usepackage[T1]{fontenc}
\usepackage{graphicx, float}            % Graphics/Images.
\usepackage{natbib}                     % For bibliographies.
\bibliographystyle{agsm}                % Bibliography style.
\usepackage[french, english]{babel}     % Language typesetting.
\usepackage[dvipsnames]{xcolor}         % Color names.
\usepackage{listings}                   % Verbatim-Like Tools.
\usepackage{mathtools, esint, mathrsfs} % amsmath and integrals.
\usepackage{amsthm, amsfonts, amssymb}  % Fonts and theorems.
\usepackage{tcolorbox}                  % Frames around theorems.
\usepackage{upgreek}                    % Non-Italic Greek.
\usepackage{fmtcount, etoolbox}         % For the \book{} command.
\usepackage[newparttoc]{titlesec}       % Formatting chapter, etc.
\usepackage{titletoc}                   % Allows \book in toc.
\usepackage[nottoc]{tocbibind}          % Bibliography in toc.
\usepackage[titles]{tocloft}            % ToC formatting.
\usepackage{pgfplots, tikz}             % Drawing/graphing tools.
\usepackage{imakeidx}                   % Used for index.
\usetikzlibrary{
    calc,                   % Calculating right angles and more.
    angles,                 % Drawing angles within triangles.
    arrows.meta,            % Latex and Stealth arrows.
    quotes,                 % Adding labels to angles.
    positioning,            % Relative positioning of nodes.
    decorations.markings,   % Adding arrows in the middle of a line.
    patterns,
    arrows
}                                       % Libraries for tikz.
\pgfplotsset{compat=1.9}                % Version of pgfplots.
\usepackage[font=scriptsize,
            labelformat=simple,
            labelsep=colon]{subcaption} % Subfigure captions.
\usepackage[font={scriptsize},
            hypcap=true,
            labelsep=colon]{caption}    % Figure captions.
\usepackage[pdftex,
            pdfauthor={Ryan Maguire},
            pdftitle={Mathematics and Physics},
            pdfsubject={Mathematics, Physics, Science},
            pdfkeywords={Mathematics, Physics, Computer Science, Biology},
            pdfproducer={LaTeX},
            pdfcreator={pdflatex}]{hyperref}
\hypersetup{
    colorlinks=true,
    linkcolor=blue,
    filecolor=magenta,
    urlcolor=Cerulean,
    citecolor=SkyBlue
}                           % Colors for hyperref.
\usepackage[toc,acronym,nogroupskip,nopostdot]{glossaries}
\usepackage{glossary-mcols}
%------------------------Theorem Styles-------------------------%
\theoremstyle{plain}
\newtheorem{theorem}{Theorem}[section]

% Define theorem style for default spacing and normal font.
\newtheoremstyle{normal}
    {\topsep}               % Amount of space above the theorem.
    {\topsep}               % Amount of space below the theorem.
    {}                      % Font used for body of theorem.
    {}                      % Measure of space to indent.
    {\bfseries}             % Font of the header of the theorem.
    {}                      % Punctuation between head and body.
    {.5em}                  % Space after theorem head.
    {}

% Italic header environment.
\newtheoremstyle{thmit}{\topsep}{\topsep}{}{}{\itshape}{}{0.5em}{}

% Define environments with italic headers.
\theoremstyle{thmit}
\newtheorem*{solution}{Solution}

% Define default environments.
\theoremstyle{normal}
\newtheorem{example}{Example}[section]
\newtheorem{definition}{Definition}[section]
\newtheorem{problem}{Problem}[section]

% Define framed environment.
\tcbuselibrary{most}
\newtcbtheorem[use counter*=theorem]{ftheorem}{Theorem}{%
    before=\par\vspace{2ex},
    boxsep=0.5\topsep,
    after=\par\vspace{2ex},
    colback=green!5,
    colframe=green!35!black,
    fonttitle=\bfseries\upshape%
}{thm}

\newtcbtheorem[auto counter, number within=section]{faxiom}{Axiom}{%
    before=\par\vspace{2ex},
    boxsep=0.5\topsep,
    after=\par\vspace{2ex},
    colback=Apricot!5,
    colframe=Apricot!35!black,
    fonttitle=\bfseries\upshape%
}{ax}

\newtcbtheorem[use counter*=definition]{fdefinition}{Definition}{%
    before=\par\vspace{2ex},
    boxsep=0.5\topsep,
    after=\par\vspace{2ex},
    colback=blue!5!white,
    colframe=blue!75!black,
    fonttitle=\bfseries\upshape%
}{def}

\newtcbtheorem[use counter*=example]{fexample}{Example}{%
    before=\par\vspace{2ex},
    boxsep=0.5\topsep,
    after=\par\vspace{2ex},
    colback=red!5!white,
    colframe=red!75!black,
    fonttitle=\bfseries\upshape%
}{ex}

\newtcbtheorem[auto counter, number within=section]{fnotation}{Notation}{%
    before=\par\vspace{2ex},
    boxsep=0.5\topsep,
    after=\par\vspace{2ex},
    colback=SeaGreen!5!white,
    colframe=SeaGreen!75!black,
    fonttitle=\bfseries\upshape%
}{not}

\newtcbtheorem[use counter*=remark]{fremark}{Remark}{%
    fonttitle=\bfseries\upshape,
    colback=Goldenrod!5!white,
    colframe=Goldenrod!75!black}{ex}

\newenvironment{bproof}{\textit{Proof.}}{\hfill$\square$}
\tcolorboxenvironment{bproof}{%
    blanker,
    breakable,
    left=3mm,
    before skip=5pt,
    after skip=10pt,
    borderline west={0.6mm}{0pt}{green!80!black}
}

\AtEndEnvironment{lexample}{$\hfill\textcolor{red}{\blacksquare}$}
\newtcbtheorem[use counter*=example]{lexample}{Example}{%
    empty,
    title={Example~\theexample},
    boxed title style={%
        empty,
        size=minimal,
        toprule=2pt,
        top=0.5\topsep,
    },
    coltitle=red,
    fonttitle=\bfseries,
    parbox=false,
    boxsep=0pt,
    before=\par\vspace{2ex},
    left=0pt,
    right=0pt,
    top=3ex,
    bottom=1ex,
    before=\par\vspace{2ex},
    after=\par\vspace{2ex},
    breakable,
    pad at break*=0mm,
    vfill before first,
    overlay unbroken={%
        \draw[red, line width=2pt]
            ([yshift=-1.2ex]title.south-|frame.west) to
            ([yshift=-1.2ex]title.south-|frame.east);
        },
    overlay first={%
        \draw[red, line width=2pt]
            ([yshift=-1.2ex]title.south-|frame.west) to
            ([yshift=-1.2ex]title.south-|frame.east);
    },
}{ex}

\AtEndEnvironment{ldefinition}{$\hfill\textcolor{Blue}{\blacksquare}$}
\newtcbtheorem[use counter*=definition]{ldefinition}{Definition}{%
    empty,
    title={Definition~\thedefinition:~{#1}},
    boxed title style={%
        empty,
        size=minimal,
        toprule=2pt,
        top=0.5\topsep,
    },
    coltitle=Blue,
    fonttitle=\bfseries,
    parbox=false,
    boxsep=0pt,
    before=\par\vspace{2ex},
    left=0pt,
    right=0pt,
    top=3ex,
    bottom=0pt,
    before=\par\vspace{2ex},
    after=\par\vspace{1ex},
    breakable,
    pad at break*=0mm,
    vfill before first,
    overlay unbroken={%
        \draw[Blue, line width=2pt]
            ([yshift=-1.2ex]title.south-|frame.west) to
            ([yshift=-1.2ex]title.south-|frame.east);
        },
    overlay first={%
        \draw[Blue, line width=2pt]
            ([yshift=-1.2ex]title.south-|frame.west) to
            ([yshift=-1.2ex]title.south-|frame.east);
    },
}{def}

\AtEndEnvironment{ltheorem}{$\hfill\textcolor{Green}{\blacksquare}$}
\newtcbtheorem[use counter*=theorem]{ltheorem}{Theorem}{%
    empty,
    title={Theorem~\thetheorem:~{#1}},
    boxed title style={%
        empty,
        size=minimal,
        toprule=2pt,
        top=0.5\topsep,
    },
    coltitle=Green,
    fonttitle=\bfseries,
    parbox=false,
    boxsep=0pt,
    before=\par\vspace{2ex},
    left=0pt,
    right=0pt,
    top=3ex,
    bottom=-1.5ex,
    breakable,
    pad at break*=0mm,
    vfill before first,
    overlay unbroken={%
        \draw[Green, line width=2pt]
            ([yshift=-1.2ex]title.south-|frame.west) to
            ([yshift=-1.2ex]title.south-|frame.east);},
    overlay first={%
        \draw[Green, line width=2pt]
            ([yshift=-1.2ex]title.south-|frame.west) to
            ([yshift=-1.2ex]title.south-|frame.east);
    }
}{thm}

%--------------------Declared Math Operators--------------------%
\DeclareMathOperator{\adjoint}{adj}         % Adjoint.
\DeclareMathOperator{\Card}{Card}           % Cardinality.
\DeclareMathOperator{\curl}{curl}           % Curl.
\DeclareMathOperator{\diam}{diam}           % Diameter.
\DeclareMathOperator{\dist}{dist}           % Distance.
\DeclareMathOperator{\Div}{div}             % Divergence.
\DeclareMathOperator{\Erf}{Erf}             % Error Function.
\DeclareMathOperator{\Erfc}{Erfc}           % Complementary Error Function.
\DeclareMathOperator{\Ext}{Ext}             % Exterior.
\DeclareMathOperator{\GCD}{GCD}             % Greatest common denominator.
\DeclareMathOperator{\grad}{grad}           % Gradient
\DeclareMathOperator{\Ima}{Im}              % Image.
\DeclareMathOperator{\Int}{Int}             % Interior.
\DeclareMathOperator{\LC}{LC}               % Leading coefficient.
\DeclareMathOperator{\LCM}{LCM}             % Least common multiple.
\DeclareMathOperator{\LM}{LM}               % Leading monomial.
\DeclareMathOperator{\LT}{LT}               % Leading term.
\DeclareMathOperator{\Mod}{mod}             % Modulus.
\DeclareMathOperator{\Mon}{Mon}             % Monomial.
\DeclareMathOperator{\multideg}{mutlideg}   % Multi-Degree (Graphs).
\DeclareMathOperator{\nul}{nul}             % Null space of operator.
\DeclareMathOperator{\Ord}{Ord}             % Ordinal of ordered set.
\DeclareMathOperator{\Prin}{Prin}           % Principal value.
\DeclareMathOperator{\proj}{proj}           % Projection.
\DeclareMathOperator{\Refl}{Refl}           % Reflection operator.
\DeclareMathOperator{\rk}{rk}               % Rank of operator.
\DeclareMathOperator{\sgn}{sgn}             % Sign of a number.
\DeclareMathOperator{\sinc}{sinc}           % Sinc function.
\DeclareMathOperator{\Span}{Span}           % Span of a set.
\DeclareMathOperator{\Spec}{Spec}           % Spectrum.
\DeclareMathOperator{\supp}{supp}           % Support
\DeclareMathOperator{\Tr}{Tr}               % Trace of matrix.
%--------------------Declared Math Symbols--------------------%
\DeclareMathSymbol{\minus}{\mathbin}{AMSa}{"39} % Unary minus sign.
%------------------------New Commands---------------------------%
\DeclarePairedDelimiter\norm{\lVert}{\rVert}
\DeclarePairedDelimiter\ceil{\lceil}{\rceil}
\DeclarePairedDelimiter\floor{\lfloor}{\rfloor}
\newcommand*\diff{\mathop{}\!\mathrm{d}}
\newcommand*\Diff[1]{\mathop{}\!\mathrm{d^#1}}
\renewcommand*{\glstextformat}[1]{\textcolor{RoyalBlue}{#1}}
\renewcommand{\glsnamefont}[1]{\textbf{#1}}
\renewcommand\labelitemii{$\circ$}
\renewcommand\thesubfigure{%
    \arabic{chapter}.\arabic{figure}.\arabic{subfigure}}
\addto\captionsenglish{\renewcommand{\figurename}{Fig.}}
\numberwithin{equation}{section}

\renewcommand{\vector}[1]{\boldsymbol{\mathrm{#1}}}

\newcommand{\uvector}[1]{\boldsymbol{\hat{\mathrm{#1}}}}
\newcommand{\topspace}[2][]{(#2,\tau_{#1})}
\newcommand{\measurespace}[2][]{(#2,\varSigma_{#1},\mu_{#1})}
\newcommand{\measurablespace}[2][]{(#2,\varSigma_{#1})}
\newcommand{\manifold}[2][]{(#2,\tau_{#1},\mathcal{A}_{#1})}
\newcommand{\tanspace}[2]{T_{#1}{#2}}
\newcommand{\cotanspace}[2]{T_{#1}^{*}{#2}}
\newcommand{\Ckspace}[3][\mathbb{R}]{C^{#2}(#3,#1)}
\newcommand{\funcspace}[2][\mathbb{R}]{\mathcal{F}(#2,#1)}
\newcommand{\smoothvecf}[1]{\mathfrak{X}(#1)}
\newcommand{\smoothonef}[1]{\mathfrak{X}^{*}(#1)}
\newcommand{\bracket}[2]{[#1,#2]}

%------------------------Book Command---------------------------%
\makeatletter
\renewcommand\@pnumwidth{1cm}
\newcounter{book}
\renewcommand\thebook{\@Roman\c@book}
\newcommand\book{%
    \if@openright
        \cleardoublepage
    \else
        \clearpage
    \fi
    \thispagestyle{plain}%
    \if@twocolumn
        \onecolumn
        \@tempswatrue
    \else
        \@tempswafalse
    \fi
    \null\vfil
    \secdef\@book\@sbook
}
\def\@book[#1]#2{%
    \refstepcounter{book}
    \addcontentsline{toc}{book}{\bookname\ \thebook:\hspace{1em}#1}
    \markboth{}{}
    {\centering
     \interlinepenalty\@M
     \normalfont
     \huge\bfseries\bookname\nobreakspace\thebook
     \par
     \vskip 20\p@
     \Huge\bfseries#2\par}%
    \@endbook}
\def\@sbook#1{%
    {\centering
     \interlinepenalty \@M
     \normalfont
     \Huge\bfseries#1\par}%
    \@endbook}
\def\@endbook{
    \vfil\newpage
        \if@twoside
            \if@openright
                \null
                \thispagestyle{empty}%
                \newpage
            \fi
        \fi
        \if@tempswa
            \twocolumn
        \fi
}
\newcommand*\l@book[2]{%
    \ifnum\c@tocdepth >-3\relax
        \addpenalty{-\@highpenalty}%
        \addvspace{2.25em\@plus\p@}%
        \setlength\@tempdima{3em}%
        \begingroup
            \parindent\z@\rightskip\@pnumwidth
            \parfillskip -\@pnumwidth
            {
                \leavevmode
                \Large\bfseries#1\hfill\hb@xt@\@pnumwidth{\hss#2}
            }
            \par
            \nobreak
            \global\@nobreaktrue
            \everypar{\global\@nobreakfalse\everypar{}}%
        \endgroup
    \fi}
\newcommand\bookname{Book}
\renewcommand{\thebook}{\texorpdfstring{\Numberstring{book}}{book}}
\providecommand*{\toclevel@book}{-2}
\makeatother
\titleformat{\part}[display]
    {\Large\bfseries}
    {\partname\nobreakspace\thepart}
    {0mm}
    {\Huge\bfseries}
\titlecontents{part}[0pt]
    {\large\bfseries}
    {\partname\ \thecontentslabel: \quad}
    {}
    {\hfill\contentspage}
\titlecontents{chapter}[0pt]
    {\bfseries}
    {\chaptername\ \thecontentslabel:\quad}
    {}
    {\hfill\contentspage}
\newglossarystyle{longpara}{%
    \setglossarystyle{long}%
    \renewenvironment{theglossary}{%
        \begin{longtable}[l]{{p{0.25\hsize}p{0.65\hsize}}}
    }{\end{longtable}}%
    \renewcommand{\glossentry}[2]{%
        \glstarget{##1}{\glossentryname{##1}}%
        &\glossentrydesc{##1}{~##2.}
        \tabularnewline%
        \tabularnewline
    }%
}
\newglossary[not-glg]{notation}{not-gls}{not-glo}{Notation}
\newcommand*{\newnotation}[4][]{%
    \newglossaryentry{#2}{type=notation, name={\textbf{#3}, },
                          text={#4}, description={#4},#1}%
}
%--------------------------LENGTHS------------------------------%
% Spacings for the Table of Contents.
\addtolength{\cftsecnumwidth}{1ex}
\addtolength{\cftsubsecindent}{1ex}
\addtolength{\cftsubsecnumwidth}{1ex}
\addtolength{\cftfignumwidth}{1ex}
\addtolength{\cfttabnumwidth}{1ex}

% Indent and paragraph spacing.
\setlength{\parindent}{0em}
\setlength{\parskip}{0em}                                                           %
%----------------------------Main Document-------------------------------------%
\begin{document}
    \pagenumbering{roman}
    \title{Measure Theory and Complex Analysis}
    \author{Ryan Maguire}
    \date{\vspace{-5ex}}
    \maketitle
    \vspace{10ex}
    \pagenumbering{arabic}
    \section{Riemannian Integration}
        \begin{fdefinition}{Partition}{Partition}
            A partition of an interval $[a,b]$ is a sequence
            $t:\mathbb{Z}_{n}\rightarrow[a,b]$ such that $t$ is monotonically
            increasing and such that $t_{1}=a$ and $t_{n}=b$.
        \end{fdefinition}
        If $f:[a,b]\rightarrow\mathbb{R}$ is a bounded function, and if
        $t:\mathbb{Z}_{n}\rightarrow[a,b]$ is a partition of $[a,b]$, define
        $m,M:\mathbb{Z}_{n-1}\rightarrow[a,b]$ and as follows:
        \par
        \begin{subequations}
            \begin{minipage}[b]{0.49\textwidth}
                \begin{equation}
                    m_{i}=\textrm{inf}\{\,f(x)\,:\,x\in[t_{i},t_{i+1}]\,\}
                \end{equation}
            \end{minipage}
            \hfill
            \begin{minipage}[b]{0.49\textwidth}
                \begin{equation}
                    M_{i}=\textrm{sup}\{\,f(x)\,:\,x\in[t_{i},t_{i+1}]\,\}
                \end{equation}
            \end{minipage}
        \end{subequations}
        \par\vspace{2.5ex}
        Using this, we define the lower sums and upper sums of $f$ about
        the partition $t$ as:
        \par
        \begin{subequations}
            \begin{minipage}[b]{0.49\textwidth}
                \begin{equation}
                    \mathcal{L}(f,t)=\sum_{i=1}^{n-1}m_{i}(t_{i+1}-t_{i})
                \end{equation}
            \end{minipage}
            \hfill
            \begin{minipage}[b]{0.49\textwidth}
                \begin{equation}
                    \mathcal{U}(f,t)=\sum_{i=1}^{n-1}M_{i}(t_{i+1}-t_{i})
                \end{equation}
            \end{minipage}
        \end{subequations}
        \par\vspace{2.5ex}
        As the partition $t$ gets finer, we see that $\mathcal{U}$ decreases
        and $\mathcal{L}$ increases. That is, if we let $P$ be the image of
        our sequence $t$:
        \begin{equation}
            P=\{\,t_{i}\in[a,\,b]\,:\,i\in\mathbb{Z}_{n}\,\}
        \end{equation}
        If $P'$ is another such set generated by the sequence $t'$, then
        and if $P\subseteq{P}'$, then $\mathcal{U}(f,t)\geq\mathcal{U}(f,t')$.
        Similarly, $\mathcal{L}(f,t)\leq\mathcal{L}(f,t')$.
        \begin{fdefinition}{Upper Riemann Integral}{Upper_Riemann_Integral}
            The upper Riemann integral of a function
            $f:[a,b]\rightarrow\mathbb{R}$ is:
            \begin{equation}
                \overline{\mathcal{R}}\int_{a}^{b}f
                =\textrm{inf}\{\mathcal{U}(f,t)\,:\,t
                    \textrm{ is a partition of }[a,b]\,\}
            \end{equation}
        \end{fdefinition}
        The set of all functions from $\mathbb{Z}_{n}$ to $[a,b]$ is a well
        defined set for all $n\in\mathbb{N}$, as is the union over all $n$.
        Since the set of partitions is a subset of this set, the set of all
        partitions of $[a,b]$ is well defined as well. In a similar vein, we
        define the lower Riemann integral.
        \begin{fdefinition}{Lower Riemann Integral}{Lower_Riemann_Integral}
            The lower Riemann integral of a function
            $f:[a,b]\rightarrow\mathbb{R}$ is:
            \begin{equation}
                \underline{\mathcal{R}}\int_{a}^{b}f
                =\textrm{sup}\{\mathcal{L}(f,t)\,:\,t
                    \textrm{ is a partition of }[a,b]\,\}
            \end{equation}
        \end{fdefinition}
        \begin{fdefinition}{Riemann Integrable Functions}
                           {Riemann_Integrable_Functions}
            A Riemann integrable function on an interval $[a,b]$ is a
            function $f:[a,b]\rightarrow\mathbb{R}$ such that:
            \begin{equation}
                \underline{\mathcal{R}}\int_{a}^{b}f
                =\overline{\mathcal{R}}\int_{a}^{b}f
            \end{equation}
            We denote the Riemann integral of $f$ as this number:
            \begin{equation}
                \mathcal{R}\int_{a}^{b}f
                =\underline{\mathcal{R}}\int_{a}^{b}f
                =\overline{\mathcal{R}}\int_{a}^{b}f
            \end{equation}
        \end{fdefinition}
        If $f:[a,b]\rightarrow\mathbb{R}$ is Riemann integrable, then it is
        bounded. For if not, for any partition $t$ of $[a,b]$ there would be a
        subinterval $[t_{i},t_{i+1}]$ such that $f$ has infinite supremum or
        (negative) infinite infinum on this range, and thus
        $|\mathcal{U}(f,t)-\mathcal{L}(f,t)|$ would be infinite. Thus the lower
        and upper Riemann integrals would disagree, and $f$ can not be Riemann
        integrable.
        \begin{fnotation}{Set of Riemann Integrable Functions}
                         {Set_of_Riemann_Integrable_Functions}
            The set of all Riemann integrable functions
            $f:[a,b]\rightarrow\mathbb{R}$ is denoted
            $\mathcal{R}\big([a,b]\big)$.
        \end{fnotation}
        Again, those who study set theory should be weary of the statement
        \textit{the set of all blah}. However, for any two sets $X$ and $Y$, the
        existence of the set of all functions $f:X\rightarrow{Y}$ is provable
        within the standard axioms of Zermelo-Fraenkel set theory, and thus the
        set of all Riemann integrable functions is a subset of the set of all
        functions $f:[a,b]\rightarrow\mathbb{R}$. From this,
        $\mathcal{R}\big([a,b]\big)$ is a well defined set.
        \par\hfill\par
        We see that there are some limitations to the Riemann integrable,
        firstly that is only applies to bounded functions, and secondly that
        we've only defined it on closed and bounded intervals. Topologically
        this means that any function defined on a \textit{non-compact} and
        \textit{non-connected} subset of $\mathbb{R}$ will not have a defined
        integral. Moreover, if the Riemann integral is a measurement of the
        \textit{area} under a curve, then there are functions whose area we know
        from an intuitive point of view, but whose Riemann integral is
        undefined. Such problems with the Riemann integral are the motivate the
        study of a new type of integral and developing measure theory.
        \begin{fdefinition}{Subsets of Measure Zero}{Subsets_of_Measure_Zero}
            A subset of measure zero (in the sense of Lebesgue, or in the sense
            of the Lebesgue \textit{measure}) is a subset $A\subseteq\mathbb{R}$
            such that, for all $\varepsilon$, there exists a sequence of points
            $a:\mathbb{N}\rightarrow\mathbb{R}$ and
            $b:\mathbb{N}\rightarrow\mathbb{R}$ such that, for all
            $n\in\mathbb{N}$, $a_{n}<b_{n}$, and:
            \begin{equation}
                A\subseteq\bigcup_{n\in\mathbb{N}}(a_{n},\,b_{n})
            \end{equation}
            and also such that:
            \begin{equation}
                \sum_{n=0}^{\infty}|b_{n}-a_{n}|<\varepsilon
            \end{equation}
        \end{fdefinition}
        We say in the sense of Lebesgue, or in the sense of the Lebesgue
        measure, to prevent any retroactive confusion when arbitrary measures
        are developed later. This sense of \textit{measure zero} has a sense of
        \textit{measuring} associated to it. Namely, one covers the set given
        set with a union of intervals and takes the length to be the sum of the
        lengths of the individual intervals. If this sum can be made arbitrarily
        small, we say that the set has measure zero. In a similarly way we can
        define a set of measure $r\in\mathbb{R}^{+}$ to be a set that can be
        covered with intervals whose lengths sum to $r+\varepsilon$ for some
        $\varepsilon>0$, but cannot be made less than $r$.
        \begin{ltheorem}{Countable Implies Lebesgue Measure Zero}
                        {Countable_Implies_Lebesgue_Measure_Zero}
            If $A\subseteq\mathbb{R}$ is a countable set, then it has Lebesgue
            measure zero.
        \end{ltheorem}
        \begin{proof}
            For if $A$ is countable, there exists a bijection
            $r:\mathbb{N}\rightarrow{N}$. Let
            $a,b:\mathbb{N}\rightarrow\mathbb{R}$ be defined as follows:
            \par
            \begin{subequations}
                \begin{minipage}[b]{0.49\textwidth}
                    \begin{equation}
                        a_{n}=r_{n}-\frac{\varepsilon}{2^{n+2}}
                    \end{equation}
                \end{minipage}
                \hfill
                \begin{minipage}[b]{0.49\textwidth}
                    \begin{equation}
                        b_{n}=r_{n}+\frac{\varepsilon}{2^{n+2}}
                    \end{equation}
                \end{minipage}
            \end{subequations}
            \par\vspace{2.5ex}
            Then, for all $n\in\mathbb{N}$, $a_{n}\leq{b}_{n}$, and
            $b_{n}-a_{n}=\varepsilon/2^{n+1}$. But also:
            \begin{equation}
                A\subseteq\bigcup_{n\in\mathbb{N}}(a_{n},\,b_{n})
            \end{equation}
            and:
            \begin{equation}
                \sum_{n=1}^{\infty}(b_{n}-a_{n})
                =\varepsilon\sum_{n=1}^{\infty}\frac{1}{2^{n+1}}
                =\frac{\varepsilon}{2}<\varepsilon
            \end{equation}
            Thus, $A$ has Lebesgue measure zero.
        \end{proof}
        Again, to avoid the notion that we are being redundant, there are
        measures where countable sets do not have measure zero. Those
        familiar with physics may have heard of the Dirac
        $\delta$ \textit{function}, the word function being italicized since it
        is not a true function. But we may think of it as a \textit{measure}
        with the following property: If the set $A$ contains zero, then it has
        measure one, otherwise it has measure zero. This is a measure (which we
        will define shortly) and the previous theorem does not apply to it. Such
        measures are called \textit{atomic}. Most of our study will deal with
        \textit{non-atomic} measures, such as the Lebesgue measure.
        \par\hfill\par
        Thm.~\ref{thm:Countable_Implies_Lebesgue_Measure_Zero} is not an if and
        only if theorem. That is, there are subsets $A\subseteq\mathbb{R}$ that
        are \textit{uncountable} but still have measure zero, the most famous
        example being the Cantor set. The notion of Lebesgue measure zero leads
        us to a crucial theorem.
        \begin{ltheorem}{Lebesgues's Criterion for the Riemann Integral}
                        {Lebesgues_Criterion_for_the_Riemann_Integral}
            A function $f:[a,b]\rightarrow\mathbb{R}$ is Riemann integrable
            if and only if $f$ is bounded and such that the set of all
            discontinuities of $f$ on $[a,b]$ has Lebesgue measure zero.
        \end{ltheorem}
        This theorem gives us examples of functions we would like to be
        able to integrate, but cannot in the sense of Riemann.
        \begin{lexample}{Non-Riemann Integrable Function}
                        {Non_Riemann_Integrable_Function}
            Let $f:[0,1]\rightarrow\mathbb{R}$ be defined as follows:
            \begin{equation}
                f(x)=
                \begin{cases}
                    1,&x\in[0,1]\cap\mathbb{Q}\\
                    0,&x\in[0,1]\setminus\mathbb{Q}
                \end{cases}
            \end{equation}
            The set of discontinuities is the entire domain $[0,1]$ and
            this by Lebesgue's criterion, $f$ is not Riemann integrable.
            However, if we use the notion of area, the integral of this
            function should almost certainly be zero. For the function is
            zero only all but a set of measure zero, and thus the
            \textit{width} on this set is zero, and the \textit{height} on
            this set is one. Thus the area should be $0\cdot{1}=0$, and thus
            the integral of $f$ should be 1. The Riemann integral cannot
            capture such things, even though it is desirable. We must look
            for a new type of integral.
        \end{lexample}
        \begin{lexample}
              {Riemann Integrable with Countable Discontinuities}
              {Riemann_Integrable_with_Countable_Discontinuities}
            For $x\in[0,1]\cap\mathbb{Q}$, let $p(x)$ and $q(x)$ be such
            that $p(x)$ and $q(x)$ are coprime and such that:
            \begin{equation}
                x=\frac{p(x)}{q(x)}
            \end{equation}
            Define $f:[0,1]\rightarrow\mathbb{R}$ by:
            \begin{equation}
                f(x)=
                \begin{cases}
                    \frac{1}{q(x)},&x\in[0,1]\cap\mathbb{Q}\\
                    0,&x\in[0,1]\setminus\mathbb{Q}
                \end{cases}
            \end{equation}
            Then for all $x\in[0,1]\setminus\mathbb{Q}$, $f$ is continuous
            at $x$. Given an irrational $x$ and a real number $\varepsilon>0$,
            choose $\delta$ such that all of the nearby rational numbers have
            reduced denominator greater than $1/\varepsilon$. $f$ is also
            discontinuous at all rational points, since $f(x)$ will be
            positive and for any irrational there is a sequence of
            irrationals $a:\mathbb{N}\rightarrow\mathbb{R}$ such that
            $a_{n}\rightarrow{x}$. But then $f(a_{n})=0$ and thus
            $f(a_{n})\not\rightarrow{f}(x)$
        \end{lexample}
        \begin{figure}[H]
            \centering
            \captionsetup{type=figure}
            \documentclass[crop,class=article]{standalone}
%----------------------------Preamble-------------------------------%
\usepackage{tikz}                       % Drawing/graphing tools.
\usetikzlibrary{arrows.meta}            % Latex arrows.
%--------------------------Main Document----------------------------%
\begin{document}
    \begin{tikzpicture}[scale=8,>=latex]
        \draw[->] (-0.1,0) -- (1.1,0)
            node[above left] {$x$};
        \draw[->] (0,-0.1) -- (0,0.6)
            node[right] {$f(x)$};
        \draw (0.02,1/2) -- (-0.02,1/2)
            node[left]{$\frac{1}{2}$};
        \draw (0.02,1/3) -- (-0.02,1/3)
            node[left]{$\frac{1}{3}$};
        \draw (0.02,1/4) -- (-0.02,1/4)
            node[left]{$\frac{1}{4}$};
        \foreach\X[%
            evaluate=\X as \Ymax using {int(\X-1)}]
            in {25,24,...,2}{%
                \foreach\Y in {1,...,\Ymax}{%
                    \ifnum\X<5
                        \draw
                        (\Y/\X,0.02) -- (\Y/\X,-0.02)
                        node[below,fill=white]
                            {$\frac{\Y}{\X}$};
                    \else
                        \draw[ultra thin]
                        (\Y/\X,0.01) to (\Y/\X,-0.01);
                    \fi
                    \pgfmathtruncatemacro{\TST}
                        {gcd(\X,\Y)}
                    \ifnum\TST=1
                        \fill ({\Y/\X},1/\X) 
                            circle (0.2pt); 
                    \fi
                }
        }
        \foreach\X in {0,1,...,80}
        {\fill (\X/80,0) circle(0.2pt);}
    \end{tikzpicture}
\end{document}
            \caption{Caption}
            \label{fig:my_label}
        \end{figure}
    \section{Tonelli's Theorem}
        Let $(X,\mathcal{M},\mu)$ be a measure space, and let:
        \begin{equation}
            L^{+}(X,\mathcal{M},\mu)=
            \{F:X\rightarrow[0,\infty]:f\textrm{ measurable}\}
        \end{equation}
        Let $L^{1}(,\mathcal{M},\mu)$ be the set of functions
        $f:X\rightarrow\mathcal{C}$ that are Borel measurable.
        \begin{ltheorem}{Fubini-Tonelli Theorem}{}
            If $(X,\mathcal{M},\mu)$ and $(Y,\mathcal{N},\nu)$ are $\sigma$
            finite measure spaces, and if
            $f\in{L}^{+}(X\times{Y},\mathcal{M}\otimes\mathcal{N},\mu\times\nu)$
            for all $(x,y)\in{X}\times{Y}$, $f_{x}$ and $f_{y}$ are elements of
            $L^{+}(Y,\mathcal{N},\nu)$ and $L^{+}(X,\mathcal{M},\mu)$,
            respectively, and if:
            \begin{equation}
                g(x)=\int_{Y}f(x,y)\diff{\nu}
                \quad\quad
                h(y)=\int_{X}f(x,y)\diff{\mu}
            \end{equation}
            Then:
            \begin{equation}
                \int_{X\times{Y}}f(x,y)\diff(\mu\times\nu)
                =\int_{X}\Big(\int_{Y}f(x,y)\diff\nu\Big)\diff\mu
                =\int_{Y}\Big(\int_{X}f(x,y)\diff\mu\Big)\diff\nu
            \end{equation}
        \end{ltheorem}
        \begin{ltheorem}{Fubini's Theorem}{Fubinis_Theorem}
            If $f\in{L}^{1}(X\times{Y})$ where $X$ and $Y$ are $\sigma$ finite
            measure spaces, then for all $(x,y)\in{X}\times{Y}$,
            $f_{x}$ and $f^{y}$ are measurable. For $\mu$ almost everywhere
            $x\in{X}$ and $\nu$ almost everywhere $y\in{Y}$, $f_{x}$ and
            $f^{y}$ are Lebesgue integrable. And:
            \begin{equation}
                g(x)=\int_{Y}f_{x}\diff{\nu}\in{L}^{1}
                \quad\quad
                h(y)=\int_{X}f^{y}\diff{\mu}\in{L}^{1}
            \end{equation}
            And:
            \begin{equation}
                \int_{Y}\Big(\int_{Y}f_{x}\diff\nu\Big)\diff\mu
                =\int_{Y}\Big(\int_{X}f^{y}\diff\mu\Big)\diff\nu
                =\int_{X\times{Y}}f\diff(\mu\times\nu)
            \end{equation}
        \end{ltheorem}
        \begin{proof}
            If $f\in{L}^{1}$ then apply Tonelli to the real and imaginary parts
            of the positive and negative parts. Then by $f_{x}$ and $f^{y}$ are
            measurable, we have that:
            \begin{equation}
                \tilde{g}(x)=\int_{Y}f_{x}\diff\nu
            \end{equation}
            But also:
            \begin{equation}
                \int_{X}g(x)\diff\mu<\infty
            \end{equation}
            Thus $N=\{x:\tilde{g}(x)=\infty\}$ is a null set so $g$ is a
            $\mu$ measurable functions and $\tilde{g}=g$ $\mu$ almost
            everywhere. By symmetry, the result follows.
        \end{proof}
        How we use Tonelli-Fubini in the real world. We often simply need to
        reverse the order of integration. First, show that $f$ is
        $\mathcal{M}\times\mathcal{N}$ measurable and then apply Tonelli to:
        \begin{equation}
            \int_{X}\int_{Y}|f(x,y)|\diff\nu\diff\mu
        \end{equation}
        Thus implies that $f\in{L}^{1}(\mu\times\nu)$, and then Fubini applies.
    \section{$L^{1}$ Spaces}
        Examples of norms. Let $V=\mathbb{C}^{2n}$ with:
        \begin{equation}
            \norm{v}_{2}=\Big(\sum_{k=1}^{n}|v_{k}|^{2}\Big)^{1/2}
            \quad
            \norm{v}_{1}=\sum_{k=1}^{n}|v_{k}|
            \quad
            \norm{v}_{\infty}=\max\{v_{k}\}
            \quad
            \norm{v}_{p}=\Big(\sum_{k=1}^{n}|v|^{p}\Big)^{1/p}
        \end{equation}
        To define a norm on $L^{1}$ functions we need to create the equivalence
        class of functions that are equal $\mu$ almost everywhere. Thus,
        $L^{1}$ is a space of equivalence classes of functions and not a space
        of functions, but we often treat these two as the same. Every norm
        induces a metric: $d(x,y)=\norm{x-y}$. A Banach space is a normed space
        such that the induced metric is complete.
        \begin{theorem}
            A normed space $X$ is a Banach space if and only if every sequence
            that has an absolutely convergent series converges.
        \end{theorem}
        \begin{proof}
            For let $\mathbf{x}:\mathbb{N}$ be a sequence such that:
            \begin{equation}
                \sum_{n\in\mathbb{N}}\norm{\mathbf{x}_{n}}<\infty
            \end{equation}
            Let $\varepsilon>0$ and let $S_{N}$ be the partial sums of
            $\mathbf{x}_{k}$. Let $N_{\varepsilon}\in\mathbb{N}$ be such that:
            \begin{equation}
                \sum_{n\in\mathbb{N}\setminus\mathbb{Z}_{N}}
                    \norm{\mathbf{x}_{k}}<\varepsilon
            \end{equation}
            Then:
            \begin{equation}
                \norm{S_{n}-S_{m}}=\norm{\sum_{k=m+1}^{n}\mathbf{x}_{k}}
                <\sum_{k=m+1}^{n}\norm{\mathbf{x}_{k}}<\varepsilon
            \end{equation}
            Therefore $S_{N}$ is a Cauchy sequence and thus converge. Going the
            other way, suppose absolute convergence implies that
            $\mathbf{x}_{k}$ converges. Pick a subsequence such that
            $\norm{\mathbf{x}_{n_{k+1}}-\mathbf{x}_{n_{k}}}<2^{\minus{k}}$.
        \end{proof}
        \begin{theorem}
            If $(X,\mathcal{M},\mu)$ is a measure space and if $L^{1}(x)$ is a
            Banach space.
        \end{theorem}
        \begin{proof}
            For suppose $f_{n}$ is a sequence of function in $L^{1}(X)$ and
            suppose it forms an absolutely convergent series. Let $g(x)$ be
            defined by:
            \begin{equation}
                g(x)=\sum_{n\in\mathbb{N}}|f_{n}(x)|
            \end{equation}
            Then:
            \begin{equation}
                \int_{X}g\diff{\mu}
                =\sum_{n\in\mathbb{N}}\int_{X}|f_{n}|\diff{\mu}
                =\sum_{n\in\mathbb{N}}\norm{f_{n}}<\infty
            \end{equation}
            So $g^{\minus{1}}(\{\infty\})$ has measure zero and $f_{n}$ is
            absolutely convergent on $X\setminus{N}$. Let $s$ be defined as:
            \begin{equation}
                s(x)=
                \begin{cases}
                    \sum_{n\in\mathbb{N}}f_{n}(x),&x\notin{N}\\
                    0,&x\in{N}
                \end{cases}
            \end{equation}
            And let $s_{n}$ be defined by:
            \begin{equation}
                s_{n}=\sum_{i=1}^{n}\chi_{X\setminus{N}}f_{i}
            \end{equation}
            Then $s_{n}\rightarrow{s}$ and $|s_{n}(x)|\leq{g}(x)$. By the
            dominated convergence theorem we obtain:
            \begin{equation}
                \norm{s-s_{n}}=\int_{X}|s-s_{n}|\diff{\mu}
            \end{equation}
            and this converges to zero. Thus, for every absolutely convergent
            sequence, the sequence converges and therefore $L^{1}(X)$ is a
            Banach space.
        \end{proof}
    \section{Complex and Signed Measures}
        \begin{fdefinition}{Complex Measure}{Complex_Measure}
            A complex measure on a measurable space $(X,\mathcal{A})$ is a
            countably additive function $\nu:\mathcal{A}\rightarrow\mathbb{C}$
            such that $\nu(\emptyset)=0$.
        \end{fdefinition}
        If $\nu(\mathcal{A})\subseteq\mathbb{R}$, we call $\nu$ a signed
        measure. The value of $\infty$ is not allowed for signed measures. Since
        the countable additivity part is invariant under rearrangement, we have
        that the sum of any countable disjoint collection of measurable sets is
        absolutely convergent. The values of $\nu$ can be negative as well.
        A subset $F\in\mathcal{A}$ is positive if for all measurable subsets
        of $F$, it is true that $\nu(E)\geq{0}$. Similarly, $N\in\mathcal{A}$
        is negative if the opposite of that. Finally, we say hat $N$ is a null
        set if $N$ is both positive and negative. Measure zero no longer implies
        that it is a null set. Take two sets that have equal absolute measure,
        but one is negative and one is positive. The union will have measure
        zero but this will not be a null set.
        \begin{theorem}
            Every measurable subset of a positive subset is a positive subset.
        \end{theorem}
        \begin{theorem}
            The countable union of positive sets is positive.
        \end{theorem}
    \section{Homework I}
        \begin{problem}
            Show that the countable union of sets of measure zero has measure
            zero.
        \end{problem}
        \begin{solution}
            If $\mathcal{O}$ is a countable set of sets of measure zero then
            there is a bijection $A:\mathbb{N}\rightarrow\mathcal{O}$ such that,
            for all $n\in\mathbb{N}$, $\mu(A_{n})=0$. For countable additivity,
            we have countable subadditivity. That is:
            \begin{equation}
                0\leq\mu\Big(\bigcup_{n\in\mathbb{N}}A_{n}\Big)
                    \leq\sum_{n\in\mathbb{N}}\mu(A_{n})
                    =\underset{N\rightarrow\infty}{\lim}\sum_{n=1}^{N}\mu(A_{n})
                    =\underset{N\rightarrow\infty}{\lim}0
                    =0
            \end{equation}
        \end{solution}
        \begin{problem}
            Suppose $f:[a,b]\rightarrow\mathbb{R}$ is bounded and let
            $\mathcal{P}$ and $\mathcal{Q}$ be subdivisions of $[a,b]$. Prove
            that $L(f,\mathcal{P})\leq{U}(f,\mathcal{P})$.
        \end{problem}
        \begin{solution}
            If $\mathcal{P}=\mathcal{Q}$, we have:
            \begin{equation}
                    L(f,\mathcal{P})
                    =\sum_{k\in\mathbb{Z}_{n}}m_{k}(t_{k+1}-t_{k})
                        \leq\sum_{k\in\mathbb{Z}_{n}}M_{k}(t_{k+1}-t_{k})
                    =U(f,\mathcal{P})
            \end{equation}
            Simply from the definition of $m_{k}$ and $M_{k}$. But also, if
            $\mathcal{P}\subseteq\mathcal{Q}$, then:
            \begin{equation}
                L(f,\mathcal{P})\leq{L}(f,\mathcal{Q})
                \quad\quad\textrm{and}\quad\quad
                U(f,\mathcal{Q})\leq{U}(f,\mathcal{P})
            \end{equation}
            But for any partitions $\mathcal{P}$ and $\mathcal{Q}$, we have that
            $\mathcal{P}\subseteq\mathcal{P}\cup\mathcal{Q}$ and
            $\mathcal{Q}\subseteq\mathcal{P}\cup\mathcal{Q}$, thus:
            \begin{equation}
                L(f,\mathcal{Q})\leq{L}(f,\mathcal{Q}\cup\mathcal{P})
                    \leq{U}(f,\mathcal{Q}\cup\mathcal{P})\leq{U}(f,\mathcal{P})
            \end{equation}
            Proving the result.
        \end{solution}
        \begin{problem}
            Prove that a bounded function $f:[a,b]\rightarrow\mathbb{R}$ is
            Riemann integrable on $[a,b]$ if and only if for all $\varepsilon>0$
            there is a partition $\mathcal{P}$ of $[a,b]$ such that:
            \begin{equation}
                U(f,\mathcal{P})-L(f,\mathcal{P})<\varepsilon
            \end{equation}
        \end{problem}
        \begin{solution}
            For suppose $f$ is Riemann integrable. Then:
            \begin{subequations}
                \begin{align}
                    \underline{\mathcal{R}}\int_{a}^{b}f
                    &=\sup\Big\{\,L(f,\mathcal{P})\,:\,
                        \mathcal{P}\textrm{ is a partition of }[a,b]\,\Big\}\\
                    &=\inf\Big\{\,U(f,\mathcal{P})\,:\,
                        \mathcal{P}\textrm{ is a partition of }[a,b]\,\Big\}\\
                    &=\overline{\mathcal{R}}\int_{a}^{b}f
                \end{align}
            \end{subequations}
            Taking the difference, we have that:
            \begin{equation}
                \sup\{\,U(f,\mathcal{P})\,:\,\mathcal{P}\,\}-
                \inf\{\,L(f,\mathcal{P})\,:\,\mathcal{P}\,\}=0
            \end{equation}
            And thus for all $\varepsilon>0$ there is a partition $\mathcal{P}$
            such that:
            \begin{equation}
                |U(f,\mathcal{P})-L(f,\mathcal{P})|<\varepsilon
            \end{equation}
            But for all partitions $\mathcal{P}$,
            $U(f,\mathcal{P})\geq{L}(f,\mathcal{P})$, and thus:
            \begin{equation}
                U(f,\mathcal{P})-L(f,\mathcal{P})<\varepsilon
            \end{equation}
            Going the other way, if for all $\varepsilon>0$ there exists such a
            partition, let $\mathcal{P}_{n}$ be a partition such that:
            \begin{equation}
                U(f,\mathcal{P}_{n})-L(f,\mathcal{P}_{n})<n^{\minus{1}}
            \end{equation}
            But $U(f,\mathcal{P}_{n})$ is bounded below by $L(f,\mathcal{P}_{0})$
            and is a monotonically decreasing sequence of real numbers, and thus
            from the completeness of $\mathbb{R}$ there exists a real number $R$
            such that $U(f,\mathcal{P}_{n})\rightarrow{R}$. It then follows that
            $L(f,\mathcal{P}_{n})\rightarrow{R}$. And since, for any partitions
            $\mathcal{P}$ and $\mathcal{Q}$, we have that
            $U(f,\mathcal{P})\geq{L}(f,\mathcal{Q})$, it follows that:
            \begin{equation}
                \underline{\mathcal{R}}\int_{a}^{b}f
                =\overline{\mathcal{R}}\int_{a}^{b}f
            \end{equation}
            And therefore $f$ is Riemann integrable.
        \end{solution}
        \begin{problem}
            Suppose that $(X,\mathcal{M})$ is a measurable space. Show that if
            $\mathcal{M}$ is countable, then $\mathcal{M}$ is finite.
        \end{problem}
        \begin{solution}
            For suppose not. Suppose that $\mathcal{M}$ is a countably infinite
            $\sigma\textrm{-Algebra}$ on $X$. For all $x\in{X}$, let
            $\mathcal{O}_{x}$ be defined as:
            \begin{equation}
                \mathcal{O}_{x}=\big\{\,E\in\mathcal{M}\,:\,x\in{E}\,\}
            \end{equation}
            Let $\omega_{x}$ be defined as:
            \begin{equation}
                \omega_{x}=\bigcap_{E\in\mathcal{O}_{x}}E
            \end{equation}
            If $x\ne{y}$ then either $\omega_{x}=\omega_{y}$ or
            $\omega_{x}\cap\omega_{y}=\emptyset$. For if not, then
            $\omega_{x}\setminus\omega_{y}\in\mathcal{M}$ is a measurable set that
            contains $x$ but not $y$, contradicting our construction of
            $\omega_{x}$. Thus the $\omega_{x}$ partition $X$. Invoking choice and
            the countability of $\mathcal{M}$, there is a sequence
            $a:\mathbb{N}\rightarrow{X}$ such that:
            \begin{equation}
                X=\bigcup_{n\in\mathbb{N}}\omega_{a_{n}}
            \end{equation}
            And for all $n\ne{m}$:
            \begin{equation}
                \omega_{a_{n}}\cap\omega_{a_{m}}=\emptyset
            \end{equation}
            That is, $\{\omega_{a_{n}}:n\in\mathbb{N}\}$ is a mutually disjoint
            collection of measurable sets that cover $X$. But
            $\sigma\textrm{-Algebras}$ are closed to countable unions, and thus for
            all $S\subseteq\mathbb{N}$ the set:
            \begin{equation}
                E_{S}=\bigcup_{n\in{S}}\omega_{a_{n}}
            \end{equation}
            Is a measurable set. Moreover, for all $S_{1},S_{2}\subseteq\mathbb{N}$
            such that $S_{1}\ne{S}_{2}$, we have that $E_{S_{1}}\ne{E}_{{S}_{2}}$.
            But then:
            \begin{equation}
                \mathcal{A}=\big\{\,E_{S}\,:\,S\subseteq\mathbb{N}\,\}
            \end{equation}
            Is a a subset of $\mathcal{M}$ that can be put into a bijection with
            $\mathcal{P}(\mathbb{N})$, the power set of $\mathbb{N}$. But then
            $\mathcal{M}$ contains an uncountable subset, a contradiction as
            $\mathcal{M}$ is countably infinite. Thus, $\mathcal{M}$ is finite.
        \end{solution}
        \begin{problem}
            Let $X$ be an uncountable set and let $\mathcal{M}$ be the collection of
            subsets $E$ of $X$ such that either $E$ or $E^{C}$ is countable. Prove
            that $\mathcal{M}$ is a $\sigma\textrm{-Algebra}$.
        \end{problem}
        \begin{solution}
            It is closed to complements by design. Moreover, since
            $X^{C}=\emptyset$, $X\in\mathcal{M}$. It thus suffices to show that the
            set is closed to countable unions. Given a countable collection
            $\mathcal{O}\subseteq\mathcal{M}$, either it contains an uncountably
            infinite set or there does not. If there does not, then the union over
            $\mathcal{O}$ is the countable union of countable sets, which is
            countable, and is thus an element of $\mathcal{M}$. If there is an
            uncountably infinite set, denoted $\mathcal{U}$, then:
            \begin{equation}
                \textrm{Card}\Big(\big(\bigcup_{E\in\mathcal{O}}E\big)^{C}\Big)
                \leq\textrm{Card}(\mathcal{U}^{C})
            \end{equation}
            But since $\mathcal{U}\in\mathcal{M}$, it follows that $\mathcal{U}^{C}$
            is countable. Thus the union over $\mathcal{O}$ is the subset of a
            countable set, and is therefore countable. Thus, $\mathcal{M}$ is closed
            to unions.
        \end{solution}
        \begin{problem}
            Recall from calculus that if $a:\mathbb{N}\rightarrow\mathbb{R}$ is a
            non-negative sequence of real numbers, then:
            \begin{equation}
                \sum_{n=1}^{\infty}a_{n}
                    =\sup\Big\{\,\sum_{n=1}^{N}a_{n}\,:\,N\in\mathbb{N}\,\Big\}
            \end{equation}
            \begin{enumerate}
                \item   Show that:
                        \begin{equation}
                            \sum_{n=1}^{\infty}a_{n}
                            =\sup\Big\{\,\sum_{n\in{F}}a_{n}\,:\,
                                F\subseteq\mathbb{N},\,
                                \textrm{Card}(F)\in\mathbb{N}\,
                            \Big\}
                        \end{equation}
                \item   Now let $X$ be a set and $f:X\rightarrow[0,\infty)$ a
                        function. For each $E\subseteq{X}$ define:
                        \begin{equation}
                            \nu(E)=\sum_{x\in{E}}f(x)
                        \end{equation}
                        Show that $\nu$ is a measure on $(X,\mathcal{P}(X))$.
                \item   Show that, if $X$, $f$, and $\nu$ are as defined in the
                        previous part, and if $\nu(E)\infty$, then the set:
                        \begin{equation}
                            \mathcal{O}=\{\,x\in{E}\,:\,f(x)>0\,\}
                        \end{equation}
                        is countable.
            \end{enumerate}
        \end{problem}
        \begin{solution}
            For we have:
            \begin{equation}
                \sup\Big\{\,\sum_{n\in{F}}a_{n}\,:\,F\subseteq\mathbb{N},\,
                    \textrm{Card}(F)\in\mathbb{N}\,\Big\}
                \leq\sum_{n=1}^{\infty}a_{n}
            \end{equation}
            For let $F:\mathbb{N}\rightarrow\mathcal{P}(\mathbb{N})$ be a sequence
            of finite sets such that:
            \begin{equation}
                \underset{n\rightarrow\infty}{\lim}
                    \sum_{k\in{F}_{n}}a_{n}
                =\sup\Big\{\,\sum_{n\in{F}}a_{n}\,:\,F\subseteq\mathbb{N},\,
                    \textrm{Card}(F)\in\mathbb{N}\,\Big\}
            \end{equation}
            Then if we set:
            \begin{equation}
                \mathcal{F}=\bigcup_{n\in\mathbb{N}}F_{n}\subseteq\mathbb{N}
            \end{equation}
            We obtain (from the non-negativity of the sequence $a$):
            \begin{equation}
                \underset{n\rightarrow\infty}{\lim}\sum_{k\in{F}_{n}}a_{k}
                    \leq\sum_{k\in\mathcal{F}}a_{k}
                    \leq\sum_{k\in\mathbb{N}}a_{k}
            \end{equation}
            Thus proving our inequality. Going the other way, if we let
            $F_{n}=\mathbb{Z}_{n}$, then:
            \begin{equation}
                \underset{n\rightarrow\infty}{\lim}\sum_{k\in{F}_{n}}a_{k}
                =\underset{n\rightarrow\infty}{\lim}\sum_{k\in\mathbb{Z}_{n}}a_{k}
                =\sum_{k\in\mathbb{N}}a_{k}
            \end{equation}
            And therefore:
            \begin{equation}
                \sum_{n\in\mathbb{N}}a_{n}\leq
                \sup\Big\{\,\sum_{n\in{F}}a_{n}\,:\,F\subseteq\mathbb{N},\,
                    \textrm{Card}(F)\in\mathbb{N}\,\Big\}
            \end{equation}
            Comparing inequalities completes the proof.
            \par\hfill\par
            To show that $\nu$ is a measure, we must show that it's range is
            non-negative range, the $\nu(\emptyset)=0$, and that it is countably
            additive. It is non-negative since our function $f$ has non-negative
            range. Moreover, we have $\nu(\emptyset)=0$ in a vacuous manner. Lastly,
            countable additivity. If $\mathcal{O}$ is a countable mutually disjoint
            collection of sets, let $\mathcal{U}:\mathbb{N}\rightarrow\mathcal{O}$
            be a bijection. Then:
            \begin{subequations}
                \begin{align}
                    \nu\Big(\bigcup_{n\in\mathbb{N}}\mathcal{U}_{n}\Big)
                    &=\sum_{x\in\bigcup_{n\in\mathbb{N}}\mathcal{U}_{n}}f(x)\\
                    &=\sup\Big\{\,\sum_{x\in{F}}f(x)\,:\,
                        F\subseteq\bigcup_{n\in\mathbb{N}}\mathcal{U}_{n},\,
                        \textrm{Card}(F)\in\mathbb{N}\,\Big\}\\
                    &=\sup\Big\{\,\sum_{n\in\mathbb{N}}\sum_{x\in{F}_{n}}f(x)\,:\,
                        F_{n}\subseteq\mathcal{U}_{n},\,
                        \textrm{Card}(F_{n})\in\mathbb{N}\,\Big\}\\
                    &=\sum_{n\in\mathbb{N}}
                        \sup\Big\{\,\sum_{x\in{F}_{n}}f(x)\,:\,
                            F_{n}\subseteq\mathcal{U}_{n},\,
                            \textrm{Card}(F_{n})\in\mathbb{N}\,\Big\}\\
                    &=\sum_{n\in\mathbb{N}}\nu\big(\mathcal{U}_{n}\big)
                \end{align}
            \end{subequations}
            And lastly, support the set is uncountable. For all $n\in\mathbb{N}$
            the set:
            \begin{equation}
                A_{n}=\{\,x\in{E}\,:\,f(x)>n^{\minus{1}}\,\}
            \end{equation}
            Is finite. For if not then there is a sequence of subsets of $E$ whose
            sums of $f(x)$ are unbounded, and thus $\nu(E)=\infty$, a contradiction.
            But then:
            \begin{equation}
                \mathcal{O}=\bigcup_{n\in\mathbb{N}}A_{n}
            \end{equation}
            And thus $\mathcal{O}$ is the countable union of finite sets, and is
            therefore countable, a contradiction. Thus, $\mathcal{O}$ is not
            uncountable.
        \end{solution}
        \begin{problem}
            Prove that if $f$ is a real-valued function on a measurable space
            $(X,\mathcal{M})$ such that $\{\,x\,:\,f(x)\geq{r}\,\}$ is measurable
            for all $r\in\mathbb{Q}$, then $f$ is measurable.
        \end{problem}
        \begin{proof}
            This follows since the collection of all $[a,\infty)$ and
            $(\minus\infty,a]$, for $a\in\mathbb{Q}$, generate the same
            $\sigma\textrm{-Algebra}$ as $\mathcal{B}$, the Borel
            $\sigma\textrm{-Algebra}$ on $\mathbb{R}$. It suffices to show that open
            intervals are contained in the $\sigma\textrm{-Algebra}$ generated by
            this collection. Let $r_{1},r_{2}\in\mathbb{R}$, $r_{1}<r_{2}$, and let
            $a,b:\mathbb{N}\rightarrow\mathbb{Q}$ be sequences of rational numbers
            such that $a_{n}\rightarrow{r}_{1}$, and such that $a_{n}$ is
            monotonically increasing, and similarly $b_{n}\rightarrow{r}_{1}$ and
            $b_{n}$ is monotonically decreasing. Then:
            \begin{equation}
                [r_{1},r_{2}]=\bigcap_{n\in\mathbb{N}}
                    \Big((\minus\infty,b_{n}]\cup[a_{n},\infty)\Big)
            \end{equation}
            And thus this $\sigma\textrm{-Algebra}$ is closed to closed intervals.
            But $\sigma\textrm{-Algebras}$ are closed to complements, and thus
            $(\minus{\infty},a)\cup(b,\infty)$ is contained as well. From DeMorgan's
            Law, $\sigma\textrm{-Algebras}$ are closed to intersections, and from
            this we have that the $\sigma\textrm{-Algebra}$ contains open intervals.
            Since $f$ is measurable on a generating set of $\mathcal{B}$, it follows
            that $f$ is measurable.
        \end{proof}
        \begin{problem}
            Suppose that $f,g:(X,\mathcal{M})\rightarrow[\minus\infty,\infty]$
            are measurable functions. Prove that the two sets:
            \begin{equation}
                \{\,x\,:\,f(x)<g(x)\,\}
                \quad\quad\textrm{and}\quad\quad
                \{\,x\,:\,f(x)=g(x)\,\}
            \end{equation}
            Are measurable.
        \end{problem}
    \section{Homework II}
        \begin{problem}
            Let $X$ be an uncountable set and $\mathcal{M}$ be the collection of
            countable (or finite) and cocountable (or cofinite) subsets of $X$.
            Define $\mu:\mathcal{M}\rightarrow\mathbb{R}$ by:
            \begin{equation}
                \mu(E)=
                \begin{cases}
                    0,&E\textit{ is finite or countable}\\
                    1,&E\textit{ is cofinite or cocountable}
                \end{cases}
            \end{equation}
            Show that $\mathcal{M}$ is a $\sigma\textrm{-Algebra}$ on $X$ and
            that $\mu$ is a measure on $\mathcal{M}$. Describe the corresponding
            measurable functions and their integrals.
        \end{problem}
        \begin{solution}
            $\mathcal{M}$ is indeed a $\sigma\textrm{-Algebra}$. It is trivially
            closed under complements, and $X\in\mathcal{M}$ since
            $X\setminus{X}=\emptyset$ and the empty set is finite. Given a
            countable collection of elements of $\mathcal{M}$ either there is an
            uncountable element or there is not. If there is not than the union
            over the collection is the countable union of countable (or finite)
            sets and is therefore at most countable. If $\mathcal{O}$ is the
            collection and there is an uncountable element
            $\mathcal{U}\in\mathcal{O}$, then:
            \begin{equation}
                X\setminus\Big(
                    \bigcup_{\mathcal{V}\in\mathcal{O}}\mathcal{V}
                \Big)
                \subseteq{X}\setminus\mathcal{U}
            \end{equation}
            But $\mathcal{U}\in\mathcal{M}$ is uncountable, and thus it must
            be either cofinite or cocountable. But then the complement of the
            union over $\mathcal{O}$ is the subset of a cofinite or cocountable
            subset, and is therefore itself either cocountable or cofinite. Thus
            $\mathcal{M}$ is closed to countable unions.
            \par\hfill\par
            $\mu$ is also a measure. By definition, for all $E\in\mathcal{M}$
            we have that $\mu(E)\geq{0}$. Moreover since $\emptyset$ is finite,
            we obtain $\mu(\emptyset)=0$. If
            $A:\mathbb{N}\rightarrow\mathcal{M}$ is a sequence of mutually
            disjoint measurable sets then either there is an $n\in\mathbb{N}$
            such that $A_{n}$ is uncountable or there is not. If not then, since
            the countable union of countable or finite sets is at most
            countable, we have that:
            \begin{equation}
                \mu\Big(\bigcup_{k\in\mathbb{N}}A_{k}\Big)=0
            \end{equation}
            But also:
            \begin{equation}
                \sum_{k\in\mathbb{N}}\mu(A_{k})=\sum_{k\in\mathbb{N}}0=0
            \end{equation}
            And thus $\mu$ is countably additive in this case. If there is an
            $n\in\mathbb{N}$ such that $A_{n}$ is uncountable, then for all
            $m\in\mathbb{N}$ such that $m\ne{n}$ we have that $a_{m}$ is
            countable or finite. For since $A_{n}\cap{A}_{m}=\emptyset$ we
            conclude that $A_{m}\subseteq{X}\setminus{A}_{n}$. But
            $X\setminus{A}_{n}$ is either finite or countable. Piecing this
            together, we get:
            \begin{equation}
                \sum_{k\in\mathbb{N}}\mu(A_{k})
                =\mu(A_{n})+\sum_{k\in\mathbb{N}\setminus\{n\}}\mu(A_{k})
                =\mu(A_{n})+\sum_{k\in\mathbb{N}\setminus\{n\}}0
                =1+0
                =1
            \end{equation}
            But since the union over all of the $A_{k}$ is either cocountable or
            cofinite, we have:
            \begin{equation}
                \mu\Big(\bigcup_{k\in\mathbb{N}}A_{k}\Big)=1
            \end{equation}
            Therefore $\mu$ is countably additive, and $\mu$ is thus a measure.
        \end{solution}
        \begin{problem}
            Suppose $F:\mathbb{N}\times{X}\rightarrow[0,\infty]$ is a sequence
            of measurable functions and suppose that for all $x\in{X}$ and for
            all $n\in\mathbb{N}$ we have that $F_{n+1}(x)\leq{F}_{n}(x)$. Also
            suppose that $F_{n}(x)\rightarrow{f}(x)$ pointwise and that
            $F_{1}\in{L}^{1}(\mu)$. Prove that:
            \begin{equation}
                \underset{n\rightarrow\infty}{\lim}\int_{X}F_{n}\diff{\mu}
                =\int_{X}f\diff{\mu}
            \end{equation}
            Show that this need not be true if $F_{1}$ is not integrable.
        \end{problem}
        \begin{solution}
            For let $R$ be the integral of $F_{1}$ over $X$ and define
            $G:\mathbb{N}\times{X}\rightarrow[0,\infty]$ by:
            \begin{equation}
                G_{n}(x)=R-F_{n}(x)
            \end{equation}
            From the monotonicity of $F_{n}$, we have that for all $x\in{X}$ and
            for all $n\in\mathbb{N}$ $G_{n}(x)\leq{G}_{n+1}(x)$. Moreover, the
            $G_{n}$ are measurable and $G_{n}\rightarrow{R}-f$. Thus $G$ is a
            monotonically increasing sequence of measurable functions and by
            Lebesgue's monotone convergence theorem we have:
            \begin{equation}
                \underset{n\rightarrow\infty}{\lim}\int_{X}G_{n}\diff{\mu}
                =R-\int_{X}f\diff{\mu}
            \end{equation}
            And since $F_{n}(x)\geq{F}_{n+1}(x)$, we have that
            $F_{1}(x)\geq{f}(x)$ and thus the right hand side is finite. Using
            this we obtain:
            \begin{align}
                R-\int_{X}f\diff{\mu}
                &=\underset{n\rightarrow\infty}{\lim}\int_{X}G_{n}\diff{\mu}\\
                &=\underset{n\rightarrow\infty}{\lim}
                    \int_{X}(R-F_{n})\diff{\mu}\\
                &=\underset{n\rightarrow\infty}{\lim}
                    \Big(R-\int_{X}F_{n}\diff{\mu}\Big)\\
                &=R-\underset{n\rightarrow\infty}{\lim}\int_{X}F_{n}\diff{\mu}
            \end{align}
            Comparing the first and last part of this chain of equalities
            completes the proof. To see that the result may fail if $F_{1}$ is
            not integrable, define
            $F:\mathbb{N}\times\mathbb{R}^{+}\rightarrow[0,\infty]$ as follows:
            \begin{equation}
                F_{n}(x)=\frac{1}{nx}
            \end{equation}
            Then for all $x\in\mathbb{R}^{+}$ and for all $n\in\mathbb{N}$ we
            have that $F_{n+1}(x)\leq{F}_{n}(x)$, each function is continuous
            and therefore measurable, and $F_{n}(x)\rightarrow{0}$ for all $x$.
            However, we have:
            \begin{equation}
                \int_{\mathbb{R}^{+}}F_{n}(x)\diff{\mu}
                =\infty\ne{0}
                =\int_{\mathbb{R}^{+}}0\diff{\mu}
            \end{equation}
        \end{solution}
        \begin{problem}
            Show that if $\mu(X)<\infty$ and that if
            $F:\mathbb{N}\times{X}\rightarrow\mathbb{C}$ is a sequence of
            bounded measurable functions that converge uniformly to
            $f:X\rightarrow\mathbb{C}$ then:
            \begin{equation}
                \underset{n\rightarrow\infty}{\lim}\int_{X}F_{n}\diff{\mu}
                =\int_{X}f\diff{\mu}
            \end{equation}
            Show that the finiteness of $\mu(X)$ cannot be omitted.
        \end{problem}
        \begin{solution}
            For by the definition of uniform convergence, for all
            $\varepsilon>0$ there exists an $N\in\mathbb{N}$ such that, for all
            $x\in{X}$ it is true that $|F_{n}(x)-f(x)|<\varepsilon$. That is:
            \begin{equation}
                \underset{n\rightarrow\infty}{\lim}
                \sup\big\{|F_{n}(x)-f(x)|\,:\,x\in{X}\big\}=0
            \end{equation}
            But then we have:
            \begin{equation}
                \underset{n\rightarrow\infty}{\lim}\int_{X}|F_{n}-f|\diff{\mu}
                \leq\underset{n\rightarrow\infty}{\lim}
                \sup\big\{|F_{n}(x)-f(x)\,:\,x\in{X}\big\}\mu(X)
            \end{equation}
            But $\mu(X)<\infty$ and therefore this limit is zero. To show that
            the finiteness of $\mu(X)$ is required, consider $X=\mathbb{C}$ and
            define $F:\mathbb{N}\times\mathbb{C}\rightarrow\mathbb{C}$ by
            $F_{n}(z)=n^{\minus{1}}$. Thus $F_{n}(z)\rightarrow{0}$ for all
            $z\in\mathbb{C}$, but:
            \begin{equation}
                \int_{\mathbb{C}}F_{n}(z)\diff{\mu}
                =\int_{\mathbb{C}}n^{\minus{1}}\diff{\mu}
                =n^{\minus{1}}\mu(\mathbb{C})
                =\infty
                \ne{0}
                =\int_{\mathbb{C}}0\diff{\mu}
            \end{equation}
        \end{solution}
        \begin{problem}
            Suppose $f\in{L}^{1}(\mu)$. Prove that for all $\varepsilon>0$ there
            is a $\delta>0$ such that for all measurable $E$ such that
            $\mu(E)<\delta$, it is true that:
            \begin{equation}
                \int_{E}|f|\diff{\mu}<\varepsilon
            \end{equation}
        \end{problem}
        \begin{solution}
            For let $\varepsilon>0$ and let $A:\mathbb{N}\rightarrow{X}$ be the
            sequence defined by:
            \begin{equation}
                A_{n}=f^{\minus{1}}\big([n,\infty]\big)
            \end{equation}
            Since $f\in{L}^{1}(\mu)$ we have that $\mu(A_{n})\rightarrow{0}$.
            For suppose there is an $\varepsilon>0$ such that for all
            $N\in\mathbb{N}$ there exists an $n>N$ such that
            $\mu(A_{n})\geq\varepsilon$. Invoke choice and choose a strictly
            monotonically increasing sequence
            $k:\mathbb{N}\rightarrow\mathbb{N}$ such that
            $\mu(A_{k_{n}})\geq\varepsilon$ for all $n\in\mathbb{N}$. But then:
            \begin{equation}
                \underset{n\rightarrow\infty}{\lim}\int_{A_{k_{n}}}|f|\diff{\mu}
                \geq\underset{n\rightarrow\infty}{\lim}n\varepsilon
                =\infty\leq\int_{X}|f|\diff{\mu}<\infty
            \end{equation}
            A contradiction. Thus there is an $N\in\mathbb{N}$ such that, for
            all $n\geq{N}$ we have $\mu(A_{n})<\frac{\varepsilon}{2R}$, where
            $R$ is the integral of $|f|$ over $X$. By the Archimedean property,
            there is an $n\in\mathbb{N}$ such that $n>R$. Define $\delta$ by:
            $\delta=\frac{\varepsilon}{2n}$. Then for all $E$ such that
            $\mu(E)<\delta$:
            \begin{subequations}
                \begin{align}
                    \int_{E}|f|\diff{\mu}
                    &=\int_{E\cap{A}_{n}}|f|\diff{\mu}
                        +\int_{E\cap(X\setminus{A}_{n})}|f|\diff{\mu}\\
                    &\leq{R}\mu(E\cap{A}_{n})
                    +n\mu\big(E\cap(X\setminus{A}_{n})\big)\\
                    &<R\frac{\varepsilon}{2R}+n\frac{\varepsilon}{2n}\\
                    &=\varepsilon
                \end{align}
            \end{subequations}
        \end{solution}
        \begin{problem}
            Suppose that $(Y,\tau)$ is a topological space and that
            $\mathcal{M}$ is a $\sigma\textrm{-Algebra}$ on $Y$ such that
            $\tau\subseteq\mathcal{M}$. Suppose that $\mu$ is a measure on
            $\mathcal{M}$ such that, for all $E\in\mathcal{M}$:
            \begin{equation}
                \mu(E)
                =\inf\{\,\mu(V)\;|\;V\in\tau\textrm{ and }E\subseteq{V}\,\}
            \end{equation}
            Further suppose that $\mu$ is a $\sigma\textrm{-finite}$. That is,
            there is a sequence $A:\mathbb{N}\rightarrow\mathcal{M}$ such that,
            for all $n\in\mathbb{N}$ it is true that $\mu(A_{n})<\infty$ and:
            \begin{equation}
                Y=\bigcup_{n\in\mathbb{N}}A_{n}
            \end{equation}
            \begin{enumerate}
                \item   Show that the Lebesgue measure is a
                        $\sigma\textrm{-finite}$ outer regular measure on
                        $(\mathbb{R},\mathcal{M})$.
                \item   Suppose $E$ is a $\mu$ measurable subset of $Y$. Given
                        $\varepsilon>0$ show that there is an open set
                        $V\subseteq{Y}$ and a closed set $F\subseteq{Y}$ such
                        that $F\subseteq{E}\subseteq{V}$ and
                        $\mu(V\setminus{F})<\varepsilon$.
                \item   Argue that $(\mathbb{R},\mathcal{M},\lambda)$ is the
                        completion of the restriction of the Lebesgue measure
                        to the Borel sets in $\mathbb{R}$.
            \end{enumerate}
        \end{problem}
        \begin{solution}
            Since the measure of the interval $(a,b)$, with $a<b$, is $b-a$ and
            since any open subset of $\mathbb{R}$ is the countable union of
            disjoint open intervals, with $(a,\infty)$ and $(\minus\infty,a)$
            couting as \textit{intervals}, we have that by the definition of the
            Lebesgue measure:
            \begin{equation}
                \lambda(E)=
                \inf\big\{\,\lambda(\mathcal{U})\;|\;E\subseteq\mathcal{U}
                    \textrm{ and }\mathcal{U}\underset{Open}{\subseteq}
                    \mathbb{R}\big\}
            \end{equation}
            This is because any collection of open intervals that cover $E$
            form an open subset of $\mathbb{R}$, and thus can be written as the
            disjoint union of open intervals, so we may assume the collection
            that covered $E$ was disjoint and countable to begin with.
            \par\hfill\par
            That any open subset is the countable disjoint union of open
            intervals comes from the fact that $\mathbb{R}$ is a Lindel\"{o}f
            space, a consequence of the Heine-Borel theorem. Given a cover
            $\mathcal{O}$ of $\mathbb{R}$ there is a finite subcover of the
            interval $[\minus{n},n]$ since closed bounded intervals are compact.
            Let $\mathcal{C}_{n}$ be such a finite subcover for each
            $n\in\mathbb{N}$ (we may do this by countable choice). Define as a
            countable subcover the union over all $n$ of $\mathcal{C}_{n}$. From
            this we have that $\mathbb{R}$ is Lindel\"{o}f.
            \par\hfill\par
            For if $E$ is measurable, then $Y\setminus{E}$ is measurable so
            there is a $\mathcal{U}\in\tau$ such that:
            \begin{equation}
                \mu\big(\mathcal{U}\setminus(Y\setminus{E})\big)
                =\mu(\mathcal{U}\cap{E})<\frac{\varepsilon}{2}
            \end{equation}
            But ${Y}\setminus\mathcal{U}$ is closed and also:
            \begin{equation}
                \mu\big(E\setminus(Y\setminus\mathcal{U})\big)
                =\mu(E\cap\mathcal{U})<\varepsilon/2
            \end{equation}
            But since $E$ is measurable there is an $\mathcal{O}\in\tau$ such
            that $E\subseteq\mathcal{O}$ and
            $\mu(E\setminus\mathcal{U})<\varepsilon/2$. But then:
            \begin{equation}
                \mu(\mathcal{O}\setminus\mathcal{U})
                \leq\mu(\mathcal{O}\setminus{E})
                +\mu\big((Y\setminus{E})\setminus\mathcal{U}\big)
                <\frac{\varepsilon}{2}+\frac{\varepsilon}{2}
                =\varepsilon
            \end{equation}
        \end{solution}
        \begin{problem}
            Let $m$ be the Lebesgue measure on $\mathbb{R}$ and suppose that $E$
            is a set of finite measure. Given $\varepsilon>0$ show that there is
            a finite disjoint union $F$ of open intervals such that
            $m(E\ominus{F})<\varepsilon$.
        \end{problem}
        \begin{proof}
            For let $\varepsilon>0$. Then there is a countable collection of
            intervals $I_{n}$ that cover $E$ and such that:
            \begin{equation}
                \mu\Big(\big(\bigcup_{n\in\mathbb{N}}I_{n}\big)
                    \setminus{E}\big)\Big)
                <\frac{\varepsilon}{2}
            \end{equation}
            But from the finiteness of $\mu(E)$, the following series must then
            converge:
            \begin{equation}
                M=\sum_{k\in\mathbb{N}}\mu(I_{k})
            \end{equation}
            And thus there is an $n\in\mathbb{N}$ such that, for all $m\geq{n}$,
            we have:
            \begin{equation}
                \sum_{k\in\mathbb{N}\setminus\mathbb{Z}_{m}}\mu(I_{k})
                <\frac{\varepsilon}{2}
            \end{equation}
            But then:
            \begin{align}
                \mu\Big(E\ominus\bigcup_{k\in\mathbb{Z}_{n}}I_{k}\Big)
                &=\mu\Big(E\setminus\bigcup_{k\in\mathbb{Z}_{n}}I_{k}\Big)+
                    \mu\Big(\bigcup_{k\in\mathbb{Z}_{n}}I_{k}
                    \setminus{E}\Big)\\
                &\leq\mu\Big(E\setminus\bigcup_{k\in\mathbb{N}}I_{k}\Big)
                    +\mu\Big(\bigcup_{k\in\mathbb{N}
                    \setminus\mathbb{Z}_{n}}I_{k}\Big)
                    +\mu\Big(\bigcup_{k\in\mathbb{Z}_{n}}I_{k}
                    \setminus{E}\Big)\\
                &<0+\frac{\varepsilon}{2}
                    +\frac{\varepsilon}{2}\\
                &=\varepsilon
            \end{align}
            Thus $\{I_{k}:k\in\mathbb{Z}_{n}\}$ is a finite set of open
            intervals such that the symmetric difference of the union of this
            collection with $E$ has measure less than $\varepsilon$.
        \end{proof}
        \begin{problem}
            Let $(X,\mathcal{M},\mu)$ be a measure space and let
            $(X,\mathcal{M}_{0},\mu_{0})$ be it's completion.
            \begin{enumerate}
                \item   Let $f:X\rightarrow\mathbb{C}$ be a $\mathcal{M}_{0}$
                        measurable function and assume that
                        $g:X\rightarrow\mathcal{C}$ is a $\mathcal{M}$
                        measurable function such that $f=g$ $\mu_{0}$ almost
                        everywhere. Is is necessarily true that $f=g$ $\mu$
                        almost everywhere?
            \end{enumerate}
        \end{problem}
        \begin{solution}
            If $\mathcal{M}$ is a proper subset of $\mathcal{M}_{0}$, then the
            first assertion is not necessarily true. For let $E$ be a
            $\mathcal{M}_{0}$ measurable subset such that $\mu_{0}(E)=0$ but
            $E\notin\mathcal{M}$. Let $f(x)=g(x)$ for all $x\notin{E}$ and let
            $f(x)\ne{g}(x)$ for all $x\in{E}$. Then $f=g$ $\mu_{0}$ almost
            everywhere, but not $\mu$ almost everywhere for the set of
            points $x$ such that $f(x)\ne{g}(x)$ is not $\mathcal{M}$ measurable
            and thus $\mu(E)$ is undefined.
            \par\hfill\par
            Let $\mathcal{O}$ be a countable basis of $\mathcal{C}$ and let
            $\mathcal{U}:\mathbb{N}\rightarrow\mathcal{O}$ be a bijection. Such
            a basis exists since $\mathcal{C}$ is second countable. Then there
            exist sequences $A,B:\mathbb{N}\rightarrow\mathcal{M}$ such that
            for all $n\in\mathbb{N}$,
            $A_{n}\subseteq\mathcal{U}_{n}\subseteq{B}_{n}$ and:
            \begin{equation}
                \mu(B_{n}\setminus{A}_{n})=0
            \end{equation}
            Define $g:X\rightarrow\mathbb{C}$ by:
            \begin{equation}
                g(x)=
                \begin{cases}
                    f(x),&x\in\bigcup_{n\in\mathbb{N}}A_{n}\\
                    0,&x\notin\bigcup_{n\in\mathbb{N}}A_{n}
                \end{cases}
            \end{equation}
            Then $g$ is measurable and $f=g$ $\mu_{0}$ almost everywhere.
        \end{solution}
    \section{Modes of Convergence}
        There are a few types of convergences. Convergence in norm, given a
        sequence $\mathbf{x}_{n}$ we say this converges to $\mathbf{x}$ if
        $\norm{\mathbf{x}-\mathbf{x}_{n}}\rightarrow{0}$. We could use any norm
        on $\mathbb{R}$ since all norms on $\mathbb{R}^{n}$ are equivalent.
        Let $(X,\mathcal{M},\mu)$ be a measure space and $f_{n}$ a sequence of
        measurable functions and $f$ a measurable function. Pointwise
        convergence $f_{n}(x)\rightarrow{f}(x)$ for all $x\in{X}$. Convergence
        almost everywhere if there is an $E\in\mathcal{M}$ such that $\mu(E)=0$
        and $f_{n}\rightarrow{f}$ pointwise on $X\setminus{E}$. Almost
        uniform convergence if for all $\varepsilon>0$ there is a set
        $E\in\mathcal{M}$ such that $\mu(E)<\varepsilon$ and such that
        $f_{n}\rightarrow{f}$ uniformly on $X\setminus{E}$. $L_{1}$ convergence
        if $\norm{f_{n}-f}_{1}\rightarrow{0}$.
        \begin{theorem}
            If $f_{n}\rightarrow{f}$ almost uniformly, then
            $f_{n}\rightarrow{f}$ almost everywhere.
        \end{theorem}
        \begin{proof}
            Let $E=\cap{E}_{n^{\minus{1}}}$. Then $\mu(E)=0$ and
            $f_{n}\rightarrow{f}$ on $E^{C}$.
        \end{proof}
        \begin{theorem}
            f $f_{n}\rightarrow{f}$ almost everywhere and there is a
            $g\in{L}^{1}(\mu)$ such that for all $n$, $|f_{n}|\leq{g}$ almost
            everywhere.
        \end{theorem}
        Note that almost everywhere convergence does not imply convergence in
        $L^{1}$. For let $f_{n}$ be the characteristic function on $[n,n+1]$.
        Then $f_{n}\rightarrow{0}$ but $\norm{f_{n}-0}_{1}=1$ for all $n$.
        Almost uniform convergence does not imply convergence in $L^{1}$ for let
        $f_{n}(x)=2nxe^{\minus{nx^{2}}}$. Then $f_{n}\rightarrow{0}$ almost
        uniformly but $\norm{f_{n}}_{1}=0$. Uniform convergence does not imply
        $L^{1}$ convergence either, for let $f_{n}=n^{\minus{1}}$ on $[0,n]$
        and zero otherwise. So nothing really implies $L^{1}$ convergences.
        Does $L^{1}$ convergence imply anything? Suppose $f_{n}\rightarrow{f}$
        and suppose there is a set $A$ such that $\mu(A)>0$ and such that for
        all $x\in{A}$, $|f_{n}(x)-f(x)|\geq\alpha>0$ for some fixed real number
        $\alpha\in\mathbb{R}$. Then:
        \begin{equation}
            \norm{f_{n}-f}_{1}=\int_{X}|f(x)-f_{n}(x)|\diff\mu
            =\int_{A}|f_{n}-f|\diff\mu
            +\int_{X\setminus{A}}|f(x)-f_{n}(x)|\diff\mu
            \geq\alpha>0
        \end{equation}
        And thus $f_{n}$ can not converge to $f$ in $L^{1}$. Moreover,
        $f_{n}\rightarrow{f}$ in $L^{1}$ does not imply $f_{n}\rightarrow{f}$
        almost everywhere. Let $f_{n}$ be that weird sliding function thing,
        draw the picture. Yeah. $\norm{f_{n}}_{1}\rightarrow{0}$ but $f_{n}$
        does not converge anywhere.
        \begin{theorem}
            If $f_{n}\rightarrow{f}$ in $L^{1}$ and if:
            \begin{equation}
                \sum_{n\in\mathbb{N}}\norm{f_{n}-f}_{1}<\infty
            \end{equation}
            Then $f_{n}\rightarrow{f}$ almost everywhere.
        \end{theorem}
        Convergence in measure if:
        \begin{equation}
            \underset{n\rightarrow\infty}{\lim}
            \mu\Big(\big\{x\in{X}:|f_{n}(x)-f(x)|>\varepsilon\big\}\Big)=0
            \quad\quad
            \varepsilon>0
        \end{equation}
        \begin{theorem}
            If $f_{n}\rightarrow{f}$ in $L^{1}$ then $f_{n}\rightarrow{f}$ in
            measure.
        \end{theorem}
        The converse is no necessarily true.
        \begin{theorem}
            If $f_{n}\rightarrow{f}$ in measure, then there is a subsequence
            such that $f_{n_{k}}\rightarrow{f}$ almost everywhere.
        \end{theorem}
\end{document}