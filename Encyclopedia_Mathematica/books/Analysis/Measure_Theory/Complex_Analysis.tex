\chapter{Complex Analysis}
    \section{Complex Variables}
        A complex function is a function whose argument is a complex
        variable $z=x+iy$, where $i$ is the imaginary unit. Complex
        functions can have the problem of being multi-valued, which
        is a cause for caution when dealing with them. For example,
        in the complex realm every non-zero complex number $z$
        has two square roots $\sqrt{z}$. So the square root
        function is multi-valued. Any complex function $f(z)$ can
        be written as $f(z)=u(x,y)+iv(x,y)$, where $u$ and $v$ are
        purely real functions. The function $w=f(z)$ can be seen
        as a mapping, or transformation, of the $z$ plane to
        the $w$ plane. That is, $f$ is a transformation of
        its domain onto its range, or image. A compound complex
        function is one of the form $F(z)=g(f(z))$. Since complex
        functions are functions of two variables, in a sense, one
        must be careful when considering limits of complex functions.
        \begin{example}
            What is the limit of $z/\overline{z}$ as $z\rightarrow{0}$?
            This is undefined. For:
            \begin{equation*}
                \frac{z}{\overline{z}}=\frac{x+iy}{x-iy}
            \end{equation*}
            Letting $x=0$ and taking the limit on $y$,
            we get:
            \begin{equation*}
                \frac{0+iy}{0-iy}=-1
            \end{equation*}
            Letting $y=0$ and taking the limit on $x$,
            we get:
            \begin{equation*}
                \frac{x+0i}{x-0i}=1
            \end{equation*}
            So the limit does not exist.
        \end{example}
        Continuity and the various properties of limits
        are defined similarly on $\mathbb{C}$ as for
        $\mathbb{R}$, with distance between points being
        defined by
        $d(z_{1},z_{2})=\sqrt{(x_{2}-x_{1})^{2}+(y_{2}-y_{1})^{2}}$.
        Differentiation is defined as:
        \begin{equation*}
            f'(z_{0})=\lim_{z\rightarrow{z_{0}}}\frac{f(z)-f(z_{0})}{z-z_{0}}
        \end{equation*}
        \begin{theorem}
            A complex function $f(z)=u(x,y)+iv(x,y)$ is
            differentiable if and only if it satisfies
            the Cauchy-Riemann equations:
            \begin{align*}
                \frac{\partial{u}}{\partial{x}}
                &=\frac{\partial{v}}{\partial{y}}
                &
                \frac{\partial{u}}{\partial{y}}
                &=-\frac{\partial{v}}{\partial{x}}
            \end{align*}
        \end{theorem}
        \begin{theorem}
            If $f(z)=u(x,y)+iv(x,y)$ is differentiable,
            then:
            \begin{equation*}
                f'(z)=u_{x}(x,y)+iv_{y}(x,y)
            \end{equation*}
        \end{theorem}
        \begin{definition}
            A complex function $f(z)$ is analytic,
            or holomorphic, at a point $z_{0}$ if
            it is differentiable in some neighborhood of
            $z_{0}$.
        \end{definition}
        \begin{definition}
            An entire function is a complex function
            $f(z)$ such that $f$ is analytic at every
            point $z\in\mathbb{C}$.
        \end{definition}
        \begin{definition}
            A harmonic function is a function
            $A(x,y)$ such that all of its second
            partial derivatives exists, and it
            satisfies the Laplace Equation:
            \begin{equation*}
                \nabla^{2}A
                =A_{xx}(x,y)+A_{yy}(x,y)
                =0
            \end{equation*}
        \end{definition}
        \begin{theorem}
            If $f(z)=u(x,y)+iv(x,y)$ is differentiable
            on a domain $D$, then $u$ and $v$ are
            harmonic on the domain.
        \end{theorem}
        \begin{theorem}
            A function $f(z)$ is analytic if and only if
            its real and complex parts are harmonic
            conjugates of each other.
        \end{theorem}
        \begin{definition}
            A level curve of a function $f(x,y)$ is
            a curve in $\mathbb{R}^{2}$ such that
            $f$ is constant on that curve.
        \end{definition}
        One of the most basic and fundamental results from
        complex variables is Euler's Formula:
        \begin{equation*}
            \exp(i\theta)=\cos(\theta)+i\sin(\theta)
        \end{equation*}