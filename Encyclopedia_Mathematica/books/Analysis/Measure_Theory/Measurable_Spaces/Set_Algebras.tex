\section{Set Algebras}
    \begin{definition}
        A set algebra on a set $\Omega$ is a set ring
        on $\Omega$ such that $\Omega\in\mathcal{A}$.
    \end{definition}
    \begin{example}
        Let $\Omega=\{1,2,3,4\}$ and
        $R=\{\emptyset,\{1\},\{2,3\}\}$. Then $R$
        is a set ring, but it is not a set algebra
        since $\Omega\notin{R}$.
    \end{example}
    \begin{lexample}
        If $\Omega$ is an infinite set, and if
        $\mathcal{E}=\big\{\{x\}:x\in\Omega\big\}$, then
        the smallest set algebra that contains $\mathcal{E}$
        is the set of all finite and co-finite subsets of
        $\Omega$. There are a few cases to check. The finite
        union of finite subsets is finite, the finite union of
        co-finite subsets is co-finite, and the finite union
        of finite and co-finite is again co-finite. For set
        difference, the difference of finite with finite is
        again finite, and the difference of co-finite with
        co-finite is either co-finite or finite. The
        difference of co-finite with finite is co-finite,
        and the difference of finite with co-finite is finite.
        Thus, this set is a set algebra on $\Omega$. Moreoever
        it is the smallest set algebra that will contain $\mathcal{E}$.
    \end{lexample}
    From the definition, we see that a set algebra
    is closed under complements. indeed, this creates
    and equivalent definition for set algebras.
    \begin{theorem}
        If $\Omega$ is a set and
        $\mathcal{A}\subseteq\mathcal{P}(\Omega)$,
        then $\mathcal{A}$ is a set algebra on $\Omega$
        if and only if $\Omega\in\mathcal{A}$, and
        $\mathcal{A}$ is closed under union and
        complement.
    \end{theorem}
    \begin{theorem}
        If $\Omega$ is a set and $R$ is a set ring
        on $\Omega$, and if $\mathcal{A}$ is a set
        algebra on $\Omega$ such that
        $R\subset\mathcal{A}$, then for all $A\in{R}$,
        $A\in\mathcal{A}$ and $A^{C}\in\mathcal{A}$.
    \end{theorem}
    This then defines the \textit{smallest} set algebra
    that contains a set ring.
    \begin{theorem}
        If $\Omega$ is a set and $R$ is a set ring on
        $\Omega$, then:
        \begin{equation}
            \mathcal{A}=\{A,A^{C}:A\in{R}\}
        \end{equation}
        Is a set algebra on $\Omega$.
    \end{theorem}
    \begin{theorem}
        If $\Omega$ is a set and $A$ and $B$ are
        disjoint subset of $A$, then:
        \begin{equation}
            \mathcal{A}=
                \{\emptyset,A,B,A\cup{B},
                  \Omega,A^{C},B^{C},A^{C}\cap{B}^{C}\}
        \end{equation}
        is a set algebra on $\Omega$.
    \end{theorem}
    For non-disjoint $A$ and $B$, the smallest
    set algebra becomes more complicated. We saw that
    the collection of all finite subsets of a set is
    a set ring on the set. The collection of all finite
    subsets, and their complements, is a set algebra.