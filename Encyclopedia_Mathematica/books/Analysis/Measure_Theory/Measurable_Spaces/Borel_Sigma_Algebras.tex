\subsection{Borel \texorpdfstring{$\sigma$}{Sigma}-Algebra}
    One of the most important types of $\sigma$-algebras
    is the Borel $\sigma$-algebra. We first define the
    Borel $\sigma$-algebra on $\mathbb{R}$.
    \begin{definition}
        The Borel $\sigma$-algebra on $\mathbb{R}$, denoted
        $\mathcal{B}$, is the $\sigma$-algebra generated
        by the set $\{[a,b):a,b\in\mathbb{R}\}$.
    \end{definition}
    That is, the Borel $\sigma$-algebra on $\mathbb{R}$ is
    the \textit{smallest} $\sigma$-algebra that contains
    all of the semi-intervals. We can equivalently say that
    $\mathcal{B}$ is the $\sigma$-algebra generated by all
    \textit{open} intervals. If we know that every open
    subset of $\mathbb{R}$ is the countable union of open
    subsets, than we can equivalently say that
    $\mathcal{B}$ is the $\sigma$-algebra generated by all
    \textit{open subsets} of $\mathbb{R}$. Writing $[a,b)$
    as the countable intersection of open intervals, or
    $(a,b)$ as the countable union of semi-intervals comes
    from the Archimedean property of the real numbers.
    Thus, the smallest $\sigma$-algebra that contains all
    semi-intervals is the smallest $\sigma$-algebra that
    contains all open intervals, which
    is the smallest $\sigma$-algebra that contains all open
    subsets of $\mathbb{R}$. Similarly, this will contain all
    of the \textit{closed} intervals, intervals of the form
    $[a,b]$. We say that a set $\mathcal{U}\subset\mathbb{R}$
    is open if, for all $x\in\mathcal{U}$, there is an $r>0$
    such that $(x-r,x+r)\subset\mathcal{U}$. That is, every
    point in $\mathcal{U}$ can be surrounded by an interval
    that is entirely contained in $\mathcal{U}$. Thus, any
    open set can be written as:
    \begin{equation}
        \mathcal{U}=\bigcup_{x\in\mathcal{U}}(\alpha_{x},\beta_{x})
    \end{equation}
    This union is not countable, for any open set must
    contain an interval, an intervals are uncountable large.
    This is simply because $(a,b)$ is equivalent to $(0,1)$.
    By adjusting the size of $\alpha_{x}$ and $\beta_{x}$ to
    be rational numbers, we can written $\mathcal{U}$ as the
    union of intervals with rational endpoints. But there are
    only countably many such intervals, and thus
    $\mathcal{U}$ is the union of countably many open
    intervals. Thus, any open set is the union of countably
    many open intervals. From this, the smallest
    $\sigma$-algebra that contains open intervals will contain
    all open sets, since $\sigma$-algebras are closed under
    countable unions. Borel sets are elements of the
    Borel $\sigma$-algebra $\mathcal{B}$. Since all open
    sets are Borel sets, and as $\sigma$-algebras are closed
    under complenents, all closed sets are also Borel sets.
    This is because the complement of an open set is a closed
    set, and vice versa. Thus, equivalently, $\mathcal{B}$ is
    the smallest $\sigma$-algebra containing all closed sets.
    A $G_{\delta}$ sets is a subset that is the countable
    intersection of open sets. As open sets are not
    necessarily closed under countable intersections, not
    all $G_{\delta}$ sets are open. There is another notion,