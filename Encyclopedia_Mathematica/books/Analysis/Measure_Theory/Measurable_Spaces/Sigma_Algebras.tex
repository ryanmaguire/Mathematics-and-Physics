\section{\texorpdfstring{$\sigma$}{Sigma}-Algebras}
    In an analogous manner to how set rings and set algebras
    were defined, there is something called a $\sigma$-algebra.
    This notion will be one of the central themes of measure
    theory.
    \begin{definition}
        A $\sigma$-algebra on a set $\Omega$ is a
        $\sigma$-ring on $\Omega$ such that
        $\Omega\in\mathcal{A}$
    \end{definition}
    That is, given any countable collection of elements in
    $\mathcal{A}$, the union is also contained in
    $\mathcal{A}$. In addition, $\mathcal{A}$ is closed under
    set differences and $\Omega\in\mathcal{A}$.
    \begin{example}
        The first trivial example is the power set
        $\mathcal{P}(\Omega)$. Also the set
        $\{\emptyset,\Omega\}$ defines a $\sigma$-algebra on
        $\Omega$. The set of all countable subsets defines
        a $\sigma$-ring, and the set of all countable and
        co-countable (Sets whose complement is countable)
        will define a $\sigma$-algebra.
    \end{example}
    We can equivalently define a $\sigma$-algebra to be a
    collection of sets $\mathcal{A}$ such that
    $\Omega\in\mathcal{A}$, for all $A\in\mathcal{A}$,
    $A^{C}\in\mathcal{A}$, and $\mathcal{A}$ is closed under
    countable unions. Being closed under countable unions
    implies that it is closed under finite unions as well.
    For let $A_{1}=A$, and for all $n>1$, let $A_{n}=B$.
    Then $\bigcup_{n=1}^{\infty}A_{n}=A\cup{B}$. By induction,
    a $\sigma$-algebra is closed under any finite union.