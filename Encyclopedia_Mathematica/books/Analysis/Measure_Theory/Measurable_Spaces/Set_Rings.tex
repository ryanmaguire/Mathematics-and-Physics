%------------------------------------------------------------------------------%
\section{Set Rings}
    Given a set $\Omega$, $\mathcal{P}(\Omega)$ is the set of all subsets of
    $\Omega$. Often this is too much, and too difficult to handle. Indeed,
    even $\mathcal{P}(\mathbb{R})$ is quite large and hard to get a grasp on.
    We wish to speak of collections of sets that have some structure on them.
    The first thing we will talk about is a set ring.
    \begin{fdefinition}{Set Ring}
        A set ring of a set $\Omega$ is a nonempty subset
        $\mathcal{R}\subseteq\mathcal{P}(\Omega)$ such that
        for all $A,B\in\mathcal{R}$, $A\cup{B}\in\mathcal{R}$, and
        $A\setminus{B}\in\mathcal{R}$.
    \end{fdefinition}
    \begin{fexample}{Example of Set Rings}{Example_of_Set_Rings}
        If $\Omega$ is a set, then $\mathcal{P}(\Omega)$ is a set ring of
        $\Omega$. So is the set $R=\{\emptyset$. For any $A\subset\Omega$,
        the set $R=\{A\}$ is also a set ring. If $\Omega=\{1,2,3\}$, then
        $R=\{\emptyset,\{1\},\{2,3\},\{1,2,3\}\}$ is
        a set ring on $\Omega$.
    \end{fexample}
    \begin{lexample}
        If $\Omega$ is an infinite set, and if
        $\mathcal{E}=\big\{\{x\}:x\in\Omega\big\}$, then the smallest set
        ring that contains $\mathcal{E}$ is the set of all finite subsets of
        $\Omega$. For the union of two finite sets is finite, as is the set
        difference of two finite sets, and thus this satisfies a set ring.
        Moreover, if $\mathcal{R}$ is a set ring that contains $\mathcal{E}$
        then it contains the union of any finite collection of elements in
        $\mathcal{E}$. But $\mathcal{E}$ is the set of all of the singletons
        of $\Omega$, and any finite subset of $\Omega$ can be written as the
        union of finitely many singletons. Thus, $\mathcal{R}$ is the
        smallest set ring that contains $\mathcal{E}$.
    \end{lexample}
    \begin{theorem}
        If $\Omega$ is a set, if $R$ is a set ring on $\Omega$, and if $A$ is
        a finite subset of $R$, then $\cup_{\alpha\in{A}}\alpha$ is an
        element of $R$.
    \end{theorem}
    \begin{proof}
        Apply induction to the closure of unions.
    \end{proof}
    \begin{theorem}
        If $X$ is a set, if $R$ is a set ring on $X$, and if
        $A,B\in{R}$, then $A\cup{B}\in{R}$.
    \end{theorem}
    \begin{proof}
        For $A\cap{B}=A\setminus(A\setminus{B})$, and from the closure of set
        difference, $A\cap{B}\in{R}$.
    \end{proof}
    \begin{theorem}
        If $X$ is a set, if $R$ is a set ring on $X$, and if $A$ is a finite
        subset of $R$, then $\cap_{\alpha\in{A}}\alpha$ is an element of $R$.
    \end{theorem}
    \begin{proof}
        Apply induction to the closure of intersections.
    \end{proof}
    \begin{theorem}
        If $\Omega$ is a set, if $R$ is a set ring on
        $\Omega$, if $A,B\subset\Omega$, and if
        $A\setminus{B}$, $B\setminus{A}$, and
        $A\cap{B}$ are elements of $R$, then
        $A,B\in{R}$.
    \end{theorem}
    Thus, the set ring generated by the set $\{A,B\}$ and
    the set ring generated by
    $\{A\setminus{B},B\setminus{A},A\cap{B}\}$ are the
    same.
    \begin{theorem}
        If $\Omega$ is a set and $R$ is a set ring
        of $\Omega$, then $\emptyset\in{R}$.
    \end{theorem}
    \begin{proof}
        For as $R$ is non-empty, there is an element
        $A\in{R}$. If $A=\emptyset$, then we are done.
        If not, as $R$ is closed under set difference,
        $A\setminus{A}\in{R}$. But
        $A\setminus{A}=\emptyset$.
    \end{proof}
    From this, if we have a collection $R$ of subsets of
    $\Omega$ and we wish to check if $R$ is a set ring
    of $\Omega$, there are several redundant operations
    we don't need to check. Since, for any set $A$,
    we have:
    \begin{align}
        A\setminus\emptyset&=A\\
        A\setminus{A}&=\emptyset\\
        \emptyset\setminus{A}&=\emptyset\\
        A\cup{A}&=A\\
        A\cup\emptyset&=A\\
        \emptyset\cup\emptyset&=\emptyset
    \end{align}
    Using our previous example $\Omega=\{1,2,3\}$,
    we can check laboriously that
    $R=\{\emptyset,\{1\},\{2,3\},\{1,2,3\}\}$ is a
    set ring on $\Omega$. The set
    $\{\emptyset,\{1\},\{2\},\{1,2,3\}\}$ is not
    a set ring, for $\{1,2\}=\{1\}\cup\{2\}$ is not
    an element.
    \begin{theorem}
        If $\Omega$ is a set, and if $A$ and $B$ are
        disjoint subsets of $\Omega$, then
        $R=\{\emptyset,A,B,A\cup{B}\}$ is a set ring
        on $\Omega$.
    \end{theorem}
    \begin{theorem}
        If $\Omega$ is a set, if $A$ and $B$ are
        disjoint subsets of $\Omega$, and if
        $R$ is a set ring such that $A,B\in{R}$,
        then $\{emptyset,A,B,A\cup{B}\}\subset{R}$.
    \end{theorem}
    As such, the set ring $\{\emptyset,A,B,A\cup{B}\}$
    is called the set ring generated by $A$ and $B$. We
    can continue and consider the case of three mutually
    disjoint subsets.
    \begin{theorem}
        If $\Omega$ is a set, and $A_{1},A_{2},A_{3}$ are
        mutually disjoint subsets of $\Omega$, then:
        \begin{equation}
            R=\{\emptyset,A_{1},A_{2},A_{3},
                A_{1}\cup{A}_{2},A_{1}\cup{A}_{3},
                A_{2}\cup{A}_{3},
                A_{1}\cup{A}_{2}\cup{A}_{3}\}
        \end{equation}
        is a set ring on $\Omega$.
    \end{theorem}
    Indeed, we may generalize further.
    \begin{theorem}
        If $\Omega$ is a set and if
        $A$ is a subset of $\mathcal{P}(\Omega)$ of
        $n$ elements such that, for all
        $a,b\in{A}$, $a\cap{B}=\emptyset$, then:
        \begin{equation}
            R=\{\cup_{i\in{I}}A_{i}:
            I\in\mathcal{P}(\mathbb{Z}_{n})\}
        \end{equation}
        Is a set ring on $\Omega$.
    \end{theorem}
    \begin{theorem}
        If $\Omega$ is a set, then the set of all
        finite subsets of $\Omega$ is a set ring on
        $\Omega$.
    \end{theorem}
    A left semi-interval of $\mathbb{R}$ is an interval
    of the form $[a,b)$ where $a\leq{b}$. If $a=b$, this
    is the empty set. The set of all left semi-intervals
    is not a set ring on $\mathbb{R}$ since the union
    of two semi-intervals need not be a semi-interval.
    We need to add more sets to allow this to be a
    set ring. The collection of all finite unions of
    semi-intervals of $\mathbb{R}$ is a set ring.
    First, note the following:
    \begin{equation}
        \Big(\bigcup_{n=1}^{N}[a_{n},b_{n})\Big)
        \setminus[c,d)=\bigcup_{n=1}^{N}
        \Big([a_{n},b_{n})\setminus[c,d)]
    \end{equation}
    This is again the finite union of intervals. By
    induction we see that this collection is a ring on
    $\mathbb{R}$. We have seen that a set ring is
    closed to unions and set differences, and this
    implies that rings are closed under intersections and
    closed under symmetric differences. As it turns out,
    this is an equivalent definition of a set ring.
    \begin{theorem}
        If $\Omega$ is a set and
        $R\subset\mathcal{P}(\Omega)$, then $R$ is
        a set ring of $\Omega$ if and only if $R$ is
        closed under symmetric differences and
        intersections.
    \end{theorem}
    If $R$ is a set ring on $\Omega$, and if
    $A\in{R}$, let $\chi_{A}:\Omega\rightarrow[0,1]$ be
    the indicator function defined as follows:
    \begin{equation}
        \chi_{A}(\omega)=
        \begin{cases}
            0,&\omega\notin{A}\\
            1,&\omega\in{A}
        \end{cases}
    \end{equation}
    Then we have:
    \begin{align}
        \chi_{A\cap{B}}(\omega)
        &=\chi_{A}(\omega)\chi_{B}(\omega)\\
        \chi_{A\ominus{B}}&=
        \big(\chi_{A}(\omega)+\chi_{B}(\omega)\big)
        \mod{2}
    \end{align}
    These two operations satisfy the axioms of a ring.
    That is, a ring in the algebraic sense of the word:
    A set with two operations that behave certain
    properties. It is because of this that set rings
    have earned their name.