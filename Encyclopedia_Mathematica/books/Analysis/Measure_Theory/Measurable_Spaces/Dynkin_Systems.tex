\subsection{Dynkin System}
    A Dynkin system on a set $\Omega$ is a subset
    $\mathcal{D}\subset\mathcal{P}(\Omega)$ such that
    $\Omega\in\mathcal{D}$, if $A,B\in\Omega$ and if
    $A\subseteq{B}$, then $A\setminus{B}\in\mathcal{D}$,
    and for all countable collections of elements of
    $\mathcal{D}$ such that
    $A_{1}\subset{A}_{2}\subset\hdots$,
    $\cup_{n=1}^{\infty}A_{n}\in\mathcal{D}$. There is
    an equivalent defintion for Dynkin Systems.
    $\Omega\in\mathcal{D}$, $A\in\mathcal{D}$ implies
    $A^{C}\in\mathcal{D}$, and for all countable disjoint
    collections of elements in $\mathcal{D}$, the union
    is also contained in $\mathcal{D}$. These requirements
    are weaker than those of a $\sigma$-algebra. Any
    $\sigma$-algebra is a Dynkin system, but not every
    Dynkin system is a $\sigma$-algebra.
    \begin{ldefinition}{Dynkin System}
        A Dynkin System on a set $\Omega$ is a subset
        $\mathcal{D}\subseteq\mathcal{P}(\Omega)$ such that:
        \begin{enumerate}
            \item $\Omega\in\mathcal{D}$.
            \item For all $A,B\in\mathcal{D}$ such that $A\subseteq{B}$,
                  $B\setminus{A}\in\mathcal{D}$.
            \item For any sequence $A_{n}\in\mathcal{D}$ such that
                  $A_{n}\subseteq{A}_{n+1}$,
                  $\cup_{n=1}^{\infty}A_{n}\in\mathcal{D}$
        \end{enumerate}
    \end{ldefinition}
    \begin{theorem}
        If $\Omega$ is a set and $\mathcal{D}\subseteq\mathcal{P}(\Omega)$
        is such that $\Omega\in\mathcal{D}$, for all $A\in\mathcal{D}$,
        $A^{C}\in\mathcal{D}$, and if for all sequences $A_{n}\in\mathcal{D}$
        such that $A_{n}\cap{A}_{m}=\emptyset$ for all $n\ne{m}$,
        $\cup_{n=1}^{\infty}A_{n}\in\mathcal{D}$, then
        $\mathcal{D}$ is a Dynkin System on $\Omega$.
    \end{theorem}
    \begin{theorem}
        If $\mathcal{D}$ is a Dynkin system on a set
        $\Omega$, and if $\mathcal{D}$ is closed with
        respect to intersections, then $\mathcal{D}$
        is a $\sigma$-algebra on $\Omega$.
    \end{theorem}
    \begin{theorem}
        If $\Omega$ is a set, if
        $\mathcal{E}\subset\mathcal{P}(\Omega)$ is closed
        to intersections, and if $\mathcal{D}$ is the
        Dynkin System generated by $\mathcal{E}$, then
        $\mathcal{D}$ is a $\sigma$-algebra.
    \end{theorem}
    \begin{theorem}[Dynkin's Theorem]
        If $\Omega$ is a set, $\mathcal{C}\subseteq\mathcal{P}(\Omega)$
        is intersection-stable, and if $\mathcal{D}$ is the smallest
        Dynkin system that contains $\mathcal{C}$, then $\mathcal{D}$
        is also intersection-stable.
    \end{theorem}
    \begin{proof}
        For let:
        \begin{equation}
            \mathcal{D}_{1}=
            \{D\in\mathcal{D}:\forall_{C\in\mathcal{C}},D\cap{D}\in\mathcal{}\}
        \end{equation}
        Then $\mathcal{D}_{1}$ is a Dynkin system, and thus
        $\mathcal{D}_{1}=\mathcal{D}$. Now define:
        \begin{equation}
            \mathcal{D}_{2}=\{
                D\in\mathcal{D}:\forall_{A\in\mathcal{D}},D\cap{A}\in\mathca{D}
            \}
        \end{equation}
        Then $\mathcal{C}\subseteq\mathcal{D}_{2}$ and $\mathcal{D}_{2}$ is a
        Dynkin System, and thus $\mathcal{D}_{2}=\mathcal{D}$.
    \end{proof}
    \begin{theorem}
        If $\Omega$ is a set, $\mathcal{C}\subseteq\mathcal{P}(\Omega)$
        is intersection-stable, and if $\mathcal{D}$ is the smallest
        Dynkin system that contains $\mathcal{C}$, then $\mathcal{D}$
        is a $\sigma\textrm{-Algebra}$ on $\Omega$.
    \end{theorem}
    Since semi-intervals are closed to intersections,
    the Borel $\sigma$-algebra is equivalently the
    Dynkin system generated by semi-intervals.