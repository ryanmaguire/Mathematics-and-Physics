%------------------------------------------------------------------------------%
\documentclass[crop=false,class=article]{standalone}                           %
%------------------------------Preamble----------------------------------------%
\makeatletter                                                                  %
    \def\input@path{{../../../../}}                                            %
\makeatother                                                                   %
%---------------------------Packages----------------------------%
\usepackage{geometry}
\geometry{b5paper, margin=1.0in}
\usepackage[T1]{fontenc}
\usepackage{graphicx, float}            % Graphics/Images.
\usepackage{natbib}                     % For bibliographies.
\bibliographystyle{agsm}                % Bibliography style.
\usepackage[french, english]{babel}     % Language typesetting.
\usepackage[dvipsnames]{xcolor}         % Color names.
\usepackage{listings}                   % Verbatim-Like Tools.
\usepackage{mathtools, esint, mathrsfs} % amsmath and integrals.
\usepackage{amsthm, amsfonts, amssymb}  % Fonts and theorems.
\usepackage{tcolorbox}                  % Frames around theorems.
\usepackage{upgreek}                    % Non-Italic Greek.
\usepackage{fmtcount, etoolbox}         % For the \book{} command.
\usepackage[newparttoc]{titlesec}       % Formatting chapter, etc.
\usepackage{titletoc}                   % Allows \book in toc.
\usepackage[nottoc]{tocbibind}          % Bibliography in toc.
\usepackage[titles]{tocloft}            % ToC formatting.
\usepackage{pgfplots, tikz}             % Drawing/graphing tools.
\usepackage{imakeidx}                   % Used for index.
\usetikzlibrary{
    calc,                   % Calculating right angles and more.
    angles,                 % Drawing angles within triangles.
    arrows.meta,            % Latex and Stealth arrows.
    quotes,                 % Adding labels to angles.
    positioning,            % Relative positioning of nodes.
    decorations.markings,   % Adding arrows in the middle of a line.
    patterns,
    arrows
}                                       % Libraries for tikz.
\pgfplotsset{compat=1.9}                % Version of pgfplots.
\usepackage[font=scriptsize,
            labelformat=simple,
            labelsep=colon]{subcaption} % Subfigure captions.
\usepackage[font={scriptsize},
            hypcap=true,
            labelsep=colon]{caption}    % Figure captions.
\usepackage[pdftex,
            pdfauthor={Ryan Maguire},
            pdftitle={Mathematics and Physics},
            pdfsubject={Mathematics, Physics, Science},
            pdfkeywords={Mathematics, Physics, Computer Science, Biology},
            pdfproducer={LaTeX},
            pdfcreator={pdflatex}]{hyperref}
\hypersetup{
    colorlinks=true,
    linkcolor=blue,
    filecolor=magenta,
    urlcolor=Cerulean,
    citecolor=SkyBlue
}                           % Colors for hyperref.
\usepackage[toc,acronym,nogroupskip,nopostdot]{glossaries}
\usepackage{glossary-mcols}
%------------------------Theorem Styles-------------------------%
\theoremstyle{plain}
\newtheorem{theorem}{Theorem}[section]

% Define theorem style for default spacing and normal font.
\newtheoremstyle{normal}
    {\topsep}               % Amount of space above the theorem.
    {\topsep}               % Amount of space below the theorem.
    {}                      % Font used for body of theorem.
    {}                      % Measure of space to indent.
    {\bfseries}             % Font of the header of the theorem.
    {}                      % Punctuation between head and body.
    {.5em}                  % Space after theorem head.
    {}

% Italic header environment.
\newtheoremstyle{thmit}{\topsep}{\topsep}{}{}{\itshape}{}{0.5em}{}

% Define environments with italic headers.
\theoremstyle{thmit}
\newtheorem*{solution}{Solution}

% Define default environments.
\theoremstyle{normal}
\newtheorem{example}{Example}[section]
\newtheorem{definition}{Definition}[section]
\newtheorem{problem}{Problem}[section]

% Define framed environment.
\tcbuselibrary{most}
\newtcbtheorem[use counter*=theorem]{ftheorem}{Theorem}{%
    before=\par\vspace{2ex},
    boxsep=0.5\topsep,
    after=\par\vspace{2ex},
    colback=green!5,
    colframe=green!35!black,
    fonttitle=\bfseries\upshape%
}{thm}

\newtcbtheorem[auto counter, number within=section]{faxiom}{Axiom}{%
    before=\par\vspace{2ex},
    boxsep=0.5\topsep,
    after=\par\vspace{2ex},
    colback=Apricot!5,
    colframe=Apricot!35!black,
    fonttitle=\bfseries\upshape%
}{ax}

\newtcbtheorem[use counter*=definition]{fdefinition}{Definition}{%
    before=\par\vspace{2ex},
    boxsep=0.5\topsep,
    after=\par\vspace{2ex},
    colback=blue!5!white,
    colframe=blue!75!black,
    fonttitle=\bfseries\upshape%
}{def}

\newtcbtheorem[use counter*=example]{fexample}{Example}{%
    before=\par\vspace{2ex},
    boxsep=0.5\topsep,
    after=\par\vspace{2ex},
    colback=red!5!white,
    colframe=red!75!black,
    fonttitle=\bfseries\upshape%
}{ex}

\newtcbtheorem[auto counter, number within=section]{fnotation}{Notation}{%
    before=\par\vspace{2ex},
    boxsep=0.5\topsep,
    after=\par\vspace{2ex},
    colback=SeaGreen!5!white,
    colframe=SeaGreen!75!black,
    fonttitle=\bfseries\upshape%
}{not}

\newtcbtheorem[use counter*=remark]{fremark}{Remark}{%
    fonttitle=\bfseries\upshape,
    colback=Goldenrod!5!white,
    colframe=Goldenrod!75!black}{ex}

\newenvironment{bproof}{\textit{Proof.}}{\hfill$\square$}
\tcolorboxenvironment{bproof}{%
    blanker,
    breakable,
    left=3mm,
    before skip=5pt,
    after skip=10pt,
    borderline west={0.6mm}{0pt}{green!80!black}
}

\AtEndEnvironment{lexample}{$\hfill\textcolor{red}{\blacksquare}$}
\newtcbtheorem[use counter*=example]{lexample}{Example}{%
    empty,
    title={Example~\theexample},
    boxed title style={%
        empty,
        size=minimal,
        toprule=2pt,
        top=0.5\topsep,
    },
    coltitle=red,
    fonttitle=\bfseries,
    parbox=false,
    boxsep=0pt,
    before=\par\vspace{2ex},
    left=0pt,
    right=0pt,
    top=3ex,
    bottom=1ex,
    before=\par\vspace{2ex},
    after=\par\vspace{2ex},
    breakable,
    pad at break*=0mm,
    vfill before first,
    overlay unbroken={%
        \draw[red, line width=2pt]
            ([yshift=-1.2ex]title.south-|frame.west) to
            ([yshift=-1.2ex]title.south-|frame.east);
        },
    overlay first={%
        \draw[red, line width=2pt]
            ([yshift=-1.2ex]title.south-|frame.west) to
            ([yshift=-1.2ex]title.south-|frame.east);
    },
}{ex}

\AtEndEnvironment{ldefinition}{$\hfill\textcolor{Blue}{\blacksquare}$}
\newtcbtheorem[use counter*=definition]{ldefinition}{Definition}{%
    empty,
    title={Definition~\thedefinition:~{#1}},
    boxed title style={%
        empty,
        size=minimal,
        toprule=2pt,
        top=0.5\topsep,
    },
    coltitle=Blue,
    fonttitle=\bfseries,
    parbox=false,
    boxsep=0pt,
    before=\par\vspace{2ex},
    left=0pt,
    right=0pt,
    top=3ex,
    bottom=0pt,
    before=\par\vspace{2ex},
    after=\par\vspace{1ex},
    breakable,
    pad at break*=0mm,
    vfill before first,
    overlay unbroken={%
        \draw[Blue, line width=2pt]
            ([yshift=-1.2ex]title.south-|frame.west) to
            ([yshift=-1.2ex]title.south-|frame.east);
        },
    overlay first={%
        \draw[Blue, line width=2pt]
            ([yshift=-1.2ex]title.south-|frame.west) to
            ([yshift=-1.2ex]title.south-|frame.east);
    },
}{def}

\AtEndEnvironment{ltheorem}{$\hfill\textcolor{Green}{\blacksquare}$}
\newtcbtheorem[use counter*=theorem]{ltheorem}{Theorem}{%
    empty,
    title={Theorem~\thetheorem:~{#1}},
    boxed title style={%
        empty,
        size=minimal,
        toprule=2pt,
        top=0.5\topsep,
    },
    coltitle=Green,
    fonttitle=\bfseries,
    parbox=false,
    boxsep=0pt,
    before=\par\vspace{2ex},
    left=0pt,
    right=0pt,
    top=3ex,
    bottom=-1.5ex,
    breakable,
    pad at break*=0mm,
    vfill before first,
    overlay unbroken={%
        \draw[Green, line width=2pt]
            ([yshift=-1.2ex]title.south-|frame.west) to
            ([yshift=-1.2ex]title.south-|frame.east);},
    overlay first={%
        \draw[Green, line width=2pt]
            ([yshift=-1.2ex]title.south-|frame.west) to
            ([yshift=-1.2ex]title.south-|frame.east);
    }
}{thm}

%--------------------Declared Math Operators--------------------%
\DeclareMathOperator{\adjoint}{adj}         % Adjoint.
\DeclareMathOperator{\Card}{Card}           % Cardinality.
\DeclareMathOperator{\curl}{curl}           % Curl.
\DeclareMathOperator{\diam}{diam}           % Diameter.
\DeclareMathOperator{\dist}{dist}           % Distance.
\DeclareMathOperator{\Div}{div}             % Divergence.
\DeclareMathOperator{\Erf}{Erf}             % Error Function.
\DeclareMathOperator{\Erfc}{Erfc}           % Complementary Error Function.
\DeclareMathOperator{\Ext}{Ext}             % Exterior.
\DeclareMathOperator{\GCD}{GCD}             % Greatest common denominator.
\DeclareMathOperator{\grad}{grad}           % Gradient
\DeclareMathOperator{\Ima}{Im}              % Image.
\DeclareMathOperator{\Int}{Int}             % Interior.
\DeclareMathOperator{\LC}{LC}               % Leading coefficient.
\DeclareMathOperator{\LCM}{LCM}             % Least common multiple.
\DeclareMathOperator{\LM}{LM}               % Leading monomial.
\DeclareMathOperator{\LT}{LT}               % Leading term.
\DeclareMathOperator{\Mod}{mod}             % Modulus.
\DeclareMathOperator{\Mon}{Mon}             % Monomial.
\DeclareMathOperator{\multideg}{mutlideg}   % Multi-Degree (Graphs).
\DeclareMathOperator{\nul}{nul}             % Null space of operator.
\DeclareMathOperator{\Ord}{Ord}             % Ordinal of ordered set.
\DeclareMathOperator{\Prin}{Prin}           % Principal value.
\DeclareMathOperator{\proj}{proj}           % Projection.
\DeclareMathOperator{\Refl}{Refl}           % Reflection operator.
\DeclareMathOperator{\rk}{rk}               % Rank of operator.
\DeclareMathOperator{\sgn}{sgn}             % Sign of a number.
\DeclareMathOperator{\sinc}{sinc}           % Sinc function.
\DeclareMathOperator{\Span}{Span}           % Span of a set.
\DeclareMathOperator{\Spec}{Spec}           % Spectrum.
\DeclareMathOperator{\supp}{supp}           % Support
\DeclareMathOperator{\Tr}{Tr}               % Trace of matrix.
%--------------------Declared Math Symbols--------------------%
\DeclareMathSymbol{\minus}{\mathbin}{AMSa}{"39} % Unary minus sign.
%------------------------New Commands---------------------------%
\DeclarePairedDelimiter\norm{\lVert}{\rVert}
\DeclarePairedDelimiter\ceil{\lceil}{\rceil}
\DeclarePairedDelimiter\floor{\lfloor}{\rfloor}
\newcommand*\diff{\mathop{}\!\mathrm{d}}
\newcommand*\Diff[1]{\mathop{}\!\mathrm{d^#1}}
\renewcommand*{\glstextformat}[1]{\textcolor{RoyalBlue}{#1}}
\renewcommand{\glsnamefont}[1]{\textbf{#1}}
\renewcommand\labelitemii{$\circ$}
\renewcommand\thesubfigure{%
    \arabic{chapter}.\arabic{figure}.\arabic{subfigure}}
\addto\captionsenglish{\renewcommand{\figurename}{Fig.}}
\numberwithin{equation}{section}

\renewcommand{\vector}[1]{\boldsymbol{\mathrm{#1}}}

\newcommand{\uvector}[1]{\boldsymbol{\hat{\mathrm{#1}}}}
\newcommand{\topspace}[2][]{(#2,\tau_{#1})}
\newcommand{\measurespace}[2][]{(#2,\varSigma_{#1},\mu_{#1})}
\newcommand{\measurablespace}[2][]{(#2,\varSigma_{#1})}
\newcommand{\manifold}[2][]{(#2,\tau_{#1},\mathcal{A}_{#1})}
\newcommand{\tanspace}[2]{T_{#1}{#2}}
\newcommand{\cotanspace}[2]{T_{#1}^{*}{#2}}
\newcommand{\Ckspace}[3][\mathbb{R}]{C^{#2}(#3,#1)}
\newcommand{\funcspace}[2][\mathbb{R}]{\mathcal{F}(#2,#1)}
\newcommand{\smoothvecf}[1]{\mathfrak{X}(#1)}
\newcommand{\smoothonef}[1]{\mathfrak{X}^{*}(#1)}
\newcommand{\bracket}[2]{[#1,#2]}

%------------------------Book Command---------------------------%
\makeatletter
\renewcommand\@pnumwidth{1cm}
\newcounter{book}
\renewcommand\thebook{\@Roman\c@book}
\newcommand\book{%
    \if@openright
        \cleardoublepage
    \else
        \clearpage
    \fi
    \thispagestyle{plain}%
    \if@twocolumn
        \onecolumn
        \@tempswatrue
    \else
        \@tempswafalse
    \fi
    \null\vfil
    \secdef\@book\@sbook
}
\def\@book[#1]#2{%
    \refstepcounter{book}
    \addcontentsline{toc}{book}{\bookname\ \thebook:\hspace{1em}#1}
    \markboth{}{}
    {\centering
     \interlinepenalty\@M
     \normalfont
     \huge\bfseries\bookname\nobreakspace\thebook
     \par
     \vskip 20\p@
     \Huge\bfseries#2\par}%
    \@endbook}
\def\@sbook#1{%
    {\centering
     \interlinepenalty \@M
     \normalfont
     \Huge\bfseries#1\par}%
    \@endbook}
\def\@endbook{
    \vfil\newpage
        \if@twoside
            \if@openright
                \null
                \thispagestyle{empty}%
                \newpage
            \fi
        \fi
        \if@tempswa
            \twocolumn
        \fi
}
\newcommand*\l@book[2]{%
    \ifnum\c@tocdepth >-3\relax
        \addpenalty{-\@highpenalty}%
        \addvspace{2.25em\@plus\p@}%
        \setlength\@tempdima{3em}%
        \begingroup
            \parindent\z@\rightskip\@pnumwidth
            \parfillskip -\@pnumwidth
            {
                \leavevmode
                \Large\bfseries#1\hfill\hb@xt@\@pnumwidth{\hss#2}
            }
            \par
            \nobreak
            \global\@nobreaktrue
            \everypar{\global\@nobreakfalse\everypar{}}%
        \endgroup
    \fi}
\newcommand\bookname{Book}
\renewcommand{\thebook}{\texorpdfstring{\Numberstring{book}}{book}}
\providecommand*{\toclevel@book}{-2}
\makeatother
\titleformat{\part}[display]
    {\Large\bfseries}
    {\partname\nobreakspace\thepart}
    {0mm}
    {\Huge\bfseries}
\titlecontents{part}[0pt]
    {\large\bfseries}
    {\partname\ \thecontentslabel: \quad}
    {}
    {\hfill\contentspage}
\titlecontents{chapter}[0pt]
    {\bfseries}
    {\chaptername\ \thecontentslabel:\quad}
    {}
    {\hfill\contentspage}
\newglossarystyle{longpara}{%
    \setglossarystyle{long}%
    \renewenvironment{theglossary}{%
        \begin{longtable}[l]{{p{0.25\hsize}p{0.65\hsize}}}
    }{\end{longtable}}%
    \renewcommand{\glossentry}[2]{%
        \glstarget{##1}{\glossentryname{##1}}%
        &\glossentrydesc{##1}{~##2.}
        \tabularnewline%
        \tabularnewline
    }%
}
\newglossary[not-glg]{notation}{not-gls}{not-glo}{Notation}
\newcommand*{\newnotation}[4][]{%
    \newglossaryentry{#2}{type=notation, name={\textbf{#3}, },
                          text={#4}, description={#4},#1}%
}
%--------------------------LENGTHS------------------------------%
% Spacings for the Table of Contents.
\addtolength{\cftsecnumwidth}{1ex}
\addtolength{\cftsubsecindent}{1ex}
\addtolength{\cftsubsecnumwidth}{1ex}
\addtolength{\cftfignumwidth}{1ex}
\addtolength{\cfttabnumwidth}{1ex}

% Indent and paragraph spacing.
\setlength{\parindent}{0em}
\setlength{\parskip}{0em}                                                           %
%----------------------------Main Document-------------------------------------%
\begin{document}
    \pagenumbering{roman}
    \title{Measure Theory and Complex Analysis}
    \author{Ryan Maguire}
    \date{\vspace{-5ex}}
    \maketitle
    \vspace{10ex}
    \pagenumbering{arabic}
    \setcounter{section}{1}
    \section*{Final}
    \begin{problem}
        \label{Final_Problem_1}%
        Let $z_{0}\in\mathbb{C}$, $r>0$, and let $A_{r}(z_{0})$ be defined by:
        \begin{equation}
            A_{r}(z_{0})=\{\,z\in\mathbb{C}\;|\;r<|z-z_{0}|\,\}
        \end{equation}
        In other words,
        $A_{r}(z_{0})=\mathbb{C}\setminus\textrm{Cl}(B_{r}(z_{0}))$, where
        $B_{r}(z_{0})$ is the open ball of radius $r$ centered about $z_{0}$.
        Suppose $b:\mathbb{N}\rightarrow\mathbb{C}$ is a sequence such that,
        for all $z\in{A}_{r}(z_{0})$, the following sum converges:
        \begin{equation}
            g(z)=\sum_{n=1}^{\infty}\frac{b_{n}}{(z-z_{0})^{n}}
        \end{equation}
        Show that $g$ and $g'$ are holomorphic on $A_{r}(z_{0})$ and that:
        \begin{equation}
            \dot{g}(z)=\sum_{n=1}^{\infty}\frac{\minus{n}b_{n}}{(z-z_{0})^{n+1}}
        \end{equation}
    \end{problem}
    \begin{figure}[H]
        \centering
        \captionsetup{type=figure}
        \begin{tikzpicture}[>=Latex]
            \coordinate (z0) at (2.000, 2.000);
            \coordinate (C)  at (3.300, 3.300);
            \coordinate (r)  at (2.707, 2.707);
            \fill[fill=cyan,opacity=0.5] (-2.2, -2.2) rectangle (4.2, 4.2);
            \draw[densely dashed,fill=white] (z0) circle (1);

            \draw[<->] (-2.2,  0.0) to (4.2, 0.0) node[below left]  {$\Re(z)$};
            \draw[<->] ( 0.0, -2.2) to (0.0, 4.2) node[below right] {$\Im(z)$};
            \draw[densely dashed] (2.0, 0.0) to (z0);
            \draw[densely dashed] (0.0, 2.0) to (z0);
            \draw[fill=black] (z0) circle (0.5mm);
            \draw (z0) to node[below] {$r$} (r);
            \node at (z0) [below left] {$z_{0}$};
            \node at (C) {\Large{$\mathbb{C}$}};
        \end{tikzpicture}
        \caption{Drawing for Problem~\ref{Final_Problem_1}}
        \label{fig:Problem_1}
    \end{figure}
    \begin{solution}
        For the convergence can be proven to be almost uniform. Indeed, if
        $R>r$ and $|z-z_{0}|\geq{R}$, then:
        \begin{equation}
            \Big|\sum_{n=1}^{\infty}\frac{b_{n}}{(z-z_{0})^{n}}\Big|
            \leq\sum_{n=1}^{\infty}\Big|\frac{b_{n}}{(z-z_{0})^{n}}\Big|
            \leq\sum_{n=1}^{\infty}\frac{|b_{n}|}{R^{n}}
        \end{equation}
        But the sum converges, and thus by Cauchy's convergence criterion, for
        all $z\in{A}_{r}(z_{0})$, the following is true:
        \begin{equation}
            \underset{n\rightarrow\infty}{\lim}\frac{b_{n}}{(z-z_{0})^{n}}=0
        \end{equation}
        We can thus take the modulus of this, showing:
        \begin{equation}
            \underset{n\rightarrow\infty}{\lim}
            \Big|\frac{b_{n}}{(z-z_{0})^{n}}\Big|=
            \underset{n\rightarrow\infty}{\lim}
            \frac{|b_{n}|}{|z-z_{0}|^{n}}=
            \underset{n\rightarrow\infty}{\lim}
            \frac{|b_{n}|}{R^{n}}=0
        \end{equation}
        But this is true for all $R>r$, and thus it is true of $(R+r)/2$.
        But then:
        \begin{equation}
            \Big|\sum_{n=1}^{\infty}\frac{b_{n}}{(z-z_{0})^{n}}\Big|
            \leq\sum_{n=1}^{\infty}\frac{(R+r)^{n}}{2^{n}R^{n}}
        \end{equation}
        And by the ratio test, this converges. Thus, by the Weierstrass M test,
        for all $R>r$ the sum converges uniformly on $A_{R}(z_{0})$, and since
        $\mu(A_{R}(z_{0})\setminus{A}_{r}(z_{0}))=\pi(R^{2}-r^{2})$ can be made
        arbitrarily small, the convergence is almost uniform. More precisely,
        for all $R>r$ the convergence is uniform on $A_{R}(z_{0})$ and the
        uniform limit of holomorphic functions is holomorphic, and thus
        $g(z)$ is holomorphic on $A_{R}(z_{0})$. But $R>r$ is arbitrary, and
        thus $g(z)$ is holomorphic on $A_{r}(z_{0})$. Similarly, let:
        \begin{equation}
            F(z)=\sum_{n=1}^{\infty}\frac{\minus{n}b_{n}}{(z-z_{0})^{n+1}}
        \end{equation}
        By a similar argument, this sum converges uniformly and thus $F(z)$
        is holomorphic. But uniform limits of functions may be integrated
        term by term. Applying the fundamental theorem of line integrals thus
        shows that $F(z)=\dot{g}(z)$.
        \par\hfill\par
        We can also apply Theorem 23, since we've shown that the convergence is
        uniform, it will thus be uniform on every compact subset of $\Omega$.
        Thus $g$ is holomorphic, and we may obtain $g'$ be differentiating
        term by term.
    \end{solution}
    \clearpage
    \begin{problem}
        Let $u:\Omega\rightarrow\mathbb{R}$ be a non-constant harmonic function
        on a convex open subset $\Omega\subseteq\mathbb{R}^{2}$. Show that $u$
        cannot have a local maximum.
    \end{problem}
    \begin{solution}
        For if $u$ is harmonic, then $u_{xx}+u_{yy}=0$.
        Let $F(z)=u_{x}(x,y)-iu_{y}(x,y)$. Then $F$ is holomorphic since it
        satisfies the Cauchy-Riemann equations:
        \begin{equation}
            \frac{\partial{u}_{x}}{\partial{x}}-
            \frac{\partial(\minus{u}_{y})}{\partial{y}}
            =\frac{\partial^{2}u}{\partial{x}^{2}}
            +\frac{\partial^{2}u}{\partial{y}^{2}}
            =0
        \end{equation}
        And thus the first Cauchy-Riemann equation is satisfied, and similarly
        for the second. But $\Omega$ is convex, and thus if $F$ is holomorphic
        on $\Omega$ then there is an antiderivative $f$ of $F$. That is, a
        function $f:\Omega\rightarrow\mathbb{C}$ such that $f'=F$. But then
        $\Re(f)=u_{x}+C$ for some constant $C$. We can take this to be zero.
        Let $g(z)=\exp(f(z))$. Then $g$ is holomorphic, and thus by the
        maximum modulus principle, $|g|$ has no local maximum. But
        $|g|=|\exp(f)|=\exp(\Re(f))=\exp(u)$, and thus $\exp(u)$ has no local
        maximum. But then $u$ has no local maximum.
    \end{solution}
    \clearpage
    \begin{problem}
        Suppose $f\in{H}(\Omega\setminus\{a\})$ has a pole of order $m$ at $a$.
        Prove that:
        \begin{itemize}
            \item[(a)]  $\underset{z\rightarrow{a}}{\lim}|(z-a)^{k}f(z)|=\infty$
                        for $0\leq{k}<m$.
            \item[(b)]  $\underset{z\rightarrow{a}}{\lim}|(z-a)^{m}f(z)|=L\ne0$ 
        \end{itemize} 
    \end{problem}
    \begin{solution}
        By the definition of a pole of order $m$, the function:
        \begin{equation}
            g(z)=f(z)-\sum_{k=1}^{m}\frac{b_{k}}{(z-a)^{k}}
        \end{equation}
        has a removable singularity at $a$ and $b_{m}\ne{0}$. That is,
        $\lim_{z\rightarrow{a}}g(z)$ exists. But then:
        \begin{equation}
            |(z-a)^{k}f(z)|=
            \Big|g(z)(z-a)^{k}+\sum_{k=1}^{m}\frac{b_{k}}{(z-a)^{m-k}}\Big|
        \end{equation}
        Thus, if $0\leq{k}<m$ the right side tends to infinity, and thus so
        does the left. If $k=m$ the limit is $|b_{m}|$, which is non-zero.
    \end{solution}
    \clearpage
    \begin{problem}
        Let $\Omega$ be a region and $a\in\Omega$. Suppose that
        $g\in{H}(\Omega\setminus\{a\})$ and that $g$ has a pole of order $m$ at
        $a$. Show that there is an $h\in{H}(\Omega)$ such that $h(a)\ne{0}$ and:
        \begin{equation}
            g(z)=\frac{h(z)}{(z-a)^{m}}
            \quad\quad
            z\in\Omega\setminus\{a\}
        \end{equation}
        Show that $f\in{H}(\Omega)$ has a zero of order $m$ at $a$ if and only
        if $1/f$ has a pole of order $m$ at $a$.
    \end{problem}
    \begin{solution}
        For if $g$ has a pole of order $m$ at $a$, then the function:
        \begin{equation}
            f(z)=g(z)-\sum_{k=1}^{m}\frac{b_{k}}{(z-a)^{k}}
        \end{equation}
        has a removable singularity at $a$, and $b_{m}\ne0$.
        But then for $z\ne{a}$:
        \begin{equation}
            g(z)=\frac{1}{(z-a)^{m}}
                \Big(f(z)(z-a)^{m}+\sum_{k=1}^{m}b_{k}(z-a)^{m-k}\Big)
        \end{equation}
        Let $h(z)$ be the numerator. Then $h(a)=b_{m}\ne{0}$, and moreover
        $h\in{H}(\Omega)$. This is because $f\in{H}(\Omega\setminus\{a\})$, and
        $f$ is continuous at $a$. But if $f$ is continuous at $a$, then
        $f(z)(z-a)^{m}$ is differentiable at $a$, and hence
        $f(z)(z-a)^{m}$ is differentiable for all $z\in\Omega$. Thus $h$ is the
        sum of a holomorphic function and a polynomial, and is therefore
        holomorphic.
        \par\hfill\par
        If $f\in{H}(\Omega)$ has a zero of order $m$ at $a$, then:
        \begin{equation}
            f(z)=\sum_{k=m}^{\infty}a_{k}(z-a)^{k}
            =(z-a)^{m}\sum_{k=0}^{\infty}a_{k+m}(z-a)^{k}
            \equiv(z-a)^{m}g(z)
        \end{equation}
        where $a_{m}\ne{0}$. But $g(a)\ne{0}$, and thus $1/g$ is well defined.
        And the reciprocal of a differentiable function is differentiable, and
        thus $1/g$ is differentiable in a neighborhood of $a$ and is therefore
        holomorphic. But then:
        \begin{equation}
            \frac{1}{f(z)}=\frac{\frac{1}{g(z)}}{(z-a)^{m}}
        \end{equation}
        which is the reciprocal of a holomorphic function and a power of
        $z-a$, and thus by the previous part of the problem, $1/f$ has a pole of
        order $m$ at $z=a$. Going the other way is a simple flip of this
        argument. If $f$ has a pole of order $m$, there is a holomorphic
        function $h$ such that:
        \begin{equation}
            f(z)=\frac{h(z)}{(z-a)^{m}}
        \end{equation}
        such that $h(a)\ne{0}$. But then
        $1/f(z)=(z-a)^{m}/h(z)$, and since $h(a)\ne{0}$, there is a neighborhood
        about $a$ such that $h(z)\ne{0}$ for all $z$ in that neighborhood. Thus,
        $1/f$ has a zero of order $m$ at $z=a$.
    \end{solution}
    \clearpage
    \begin{problem}
        Such that if $f\in{L}^{1}(\mathbb{R})$, and if $y\in\mathbb{R}$, then:
        \begin{equation}
            \int_{\mathbb{R}}f(x)\diff\mu(x)
            =\int_{\mathbb{R}}f(x-y)\diff\mu(y)
        \end{equation}
        Then show that $\phi_{y}(f)(x)=f(x-y)$ is a continuous function from
        $\mathbb{R}$ to $L^{1}(\mathbb{R})$.
    \end{problem}
    \begin{solution}
        For if $f\in{L}^{1}(\mathbb{R})$, then given $\varepsilon>0$,
        there is an $N\in\mathbb{N}$ such that for all $n,m>N$:
        \begin{equation}
            \Big|\int_{\mathbb{R}}f\diff\mu-\int_{[\minus{m},n]}f\diff\mu\Big|
            <\varepsilon
        \end{equation}
        But:
        \begin{equation}
            \Big|\int_{\mathbb{R}}f\diff\mu
                -\int_{[\minus{n},n]}f(x-y)\diff\mu(x)\Big|
            =\Big|\int_{\mathbb{R}}f\diff\mu
            -\int_{[\minus{n-y},n-y]}f\diff\mu\Big|
        \end{equation}
        And by the Archimedean property there is an $M\in\mathbb{N}$ such that
        $M>|y|$. Then for all $n>\max\{N,M\}$, we have:
        \begin{equation}
            \Big|\int_{\mathbb{R}}f\diff\mu
            -\int_{[\minus{n-y},n-y]}f\diff\mu\Big|<\varepsilon
        \end{equation}
        Thus proving the first claim. Suppose $y_{n}\rightarrow{y}$. Then:
        \begin{equation}
            \norm{\phi_{y_{n}}(f)-\phi_{y}(f)}_{1}
            =\int_{\mathbb{R}}\big|f(x-y)-f(x-y_{n})\big|\diff\mu
        \end{equation}
        Since $y_{n}\rightarrow{y}$ there exists an $N_{1}\in\mathbb{N}$ such
        that for all $n>N_{1}$ it is true that $|y-y_{n}|<\varepsilon/2$. But
        there is also an $N_{2}$ such that, for all $n,m>N_{2}$:
        \begin{equation}
            \int_{[\minus{m},n]^{C}}\big|f(x-y)\big|\diff\mu
            <\frac{\varepsilon}{2}
        \end{equation}
        But then for all $n>\textrm{max}\{N_{1},N_{2}\}$ we have:
        \begin{subequations}
            \begin{align}
                \norm{\phi_{y_{n}}-\phi_{y}}_{1}
                &=\int_{\mathbb{R}}|f(x-y_{n})-f(x-y)|\diff\mu\\
                \nonumber&=\int_{[\minus{n},n]}|f(x-y_{n})-f(x-y)|\diff\mu\\
                &\quad\quad+\int_{[\minus{n},n]^{C}}
                    |f(x)-y_{n})-f(x-y)|\diff\mu\\
                &<\frac{\varepsilon}{2}
                +\int_{[\minus{n},n]}|f(x-y_{n})-f(x-y)|\diff\mu\\
                &<\varepsilon
            \end{align}
        \end{subequations}
        So $\phi$ is continuous.
    \end{solution}
    \clearpage
    \begin{problem}
        Show that:
        \begin{equation}
            \mathcal{B}(\mathbb{R}^{2})
            =\mathcal{B}(\mathbb{R})\otimes\mathcal(\mathbb{R})
        \end{equation}
    \end{problem}
    \begin{solution}
        For it suffices to show that both of these contain the same generating
        sets. $\mathcal(\mathbb{R}^{2})$ is generated by the topology on
        $\mathbb{R}^{2}$, which is second countable, and the open balls on
        rational points with rational radii form a basis. The basis of
        $\mathcal{B}(\mathbb{R})\otimes\mathcal{B}(\mathbb{R})$ are the open
        rectangles $(a,b)\times(c,d)$. But open rectangles are open in
        $\mathbb{R}^{2}$, and thus:
        \begin{equation}
            \mathcal{B}(\mathbb{R})\times\mathcal{B}(\mathbb{R})
            \subseteq\mathcal{B}(\mathbb{R}^{2})
        \end{equation}
        Going the other way, since any open ball can be written as the countable
        union of open rectangles, and since any open subset of $\mathbb{R}^{2}$
        can be written as the countable union of open balls with rational
        centers and rational radii, it is thus true that every open set can be
        written as the countable union of open rectangles. But
        $\sigma\textrm{-Algebras}$ are closed to countable unions, and thus
        every open subset of $\mathbb{R}^{2}$ is contained in
        $\mathcal{B}(\mathbb{R})\otimes\mathcal{B}(\mathbb{R})$. Thus:
        \begin{equation}
            \mathcal{B}(\mathbb{R}^{2})\subseteq
            \mathcal{B}(\mathbb{R})\times\mathcal{B}(\mathbb{R})
        \end{equation}
        Thus, the two $\sigma\textrm{-Algebras}$ are equal.
    \end{solution}
\end{document}