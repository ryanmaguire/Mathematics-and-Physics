\section{Integration}
    The integral of a constant function on an interval $[a,b]$ is defined by
    the signed area under the rectangle formed by the function. That is, if
    $f(x)=C$, we define the integral on $[a,b]$ to be $C(b-a)$. Given a
    piece-wise constant function, we can define the integral as the sum over the
    various regions. Given an arbitrary function, the only reasonable way to
    define the integral is to take a limit of approximations using piece-wise
    constant functions. This is how the Riemann integral is defined. The
    Riemann integral makes sense if the given function is continuous. By making
    the partition small enough, approximating a continuous function on a small
    interval by a constant can be reasonable. But consider the function:
    \begin{equation}
        f(x)=
        \begin{cases}
            0,&x\notin\mathbb{Q}\\
            1,&x\in\mathbb{Q}
        \end{cases}
    \end{equation}
    Given any interval $(a,b)$, $f$ takes on the values 0 and 1 and thus it is
    not reasonable to approximate this function by a constant anywhere. However,
    the measure, or length, of $\mathbb{Q}$ is zero, and the height of the
    function on $\mathbb{Q}$ is 1. Thus it may be reasonable to define the area
    under this function as zero. While the Riemann integral cannot handle such
    function, the Lebesgue integral can. Given a measurable space
    $(\Omega,\mathcal{A},\mu)$, and a measurable function
    $f:\Omega\rightarrow\mathbb{R}$, where $f$ is $\mathcal{A}-\mathcal{B}$
    measurable, $\mathcal{B}$ being the Borel $\sigma-\textrm{Algebra}$, it is
    possible to define the integral of $f$.
    \begin{ldefinition}{Support of a Real Valued Function}
        The support of a real-valued function
        $f:\Omega\rightarrow\mathbb{R}$ is the set:
        \begin{equation}
            \supp(f)=\{\omega\in\Omega:f(\omega)\ne{0}\}
        \end{equation}
    \end{ldefinition}
    \begin{ldefinition}{Simple Function}
        A simple function from a measurable space $(\Omega,\mathcal{A},\mu)$ to
        the real line $\mathbb{R}$ is a function $f:\Omega\rightarrow\mathbb{R}$
        such that $f$ is $\mathcal{A}-\mathcal{B}$ measurable, where
        $\mathcal{B}$ is the Borel $\sigma-\textrm{Algebra}$, the range of $f$,
        $f(\Omega)$, is finite, and the measure of the support of $f$ is finite.
    \end{ldefinition}
    \begin{lexample}
        Let $(\Omega,\mathcal{A},\mu)$ be a measurable space, and let
        $E\subseteq\Omega$ be measurable and of finite measure. Define the
        following:
        \begin{equation}
            \chi_{E}(\omega)=
            \begin{cases}
                0,&\omega\notin{E}\\
                1,&\omega\in{E}
            \end{cases}
        \end{equation}
        This is called the indicator function of $E$. It is a simple function
        on the measurable space $(\Omega,\mathcal{A},\mu)$ since
        $\mu(E)<\infty$. To see that it is measurable, note that the pre-image
        is either $\emptyset$, $E$, $E^{C}$, or $\Omega$, and thus $\chi_{E}$
        is measurable. Finally, it takes on only two values and thus it's range
        is finite.
    \end{lexample}
    \begin{lexample}
        Let $(\Omega,\mathcal{A},\mu)$ be a measurable space, and let
        $B_{1},\dots,B_{n}$ be measurable subsets of $\Omega$. Furthermore, let
        $a_{1},\dots,a_{n}$ be real numbers. If we let $\chi_{B_{i}}$ denote
        that indicator function of $B_{i}$, then we see that their sum is also a
        simple function. That is, define the following:
        \begin{equation}
            f(\omega)=\sum_{k=1}^{n}a_{k}\chi_{B_{k}}(\omega)
        \end{equation}
        THen $f:\Omega\rightarrow\mathbb{R}$ is a simple function. Since the
        sum of measurable functions is measurable, we see that $f$ is
        measurable. We also have that the support is finite since:
        \begin{equation}
            \supp(f)\subseteq\bigcup_{k=1}^{n}B_{k}
        \end{equation}
        Finally, there are $2^{n}$ ways, at most, to combine the various real
        numbers $a_{1},\dots,a_{n}$, and thus the range of $f$ has, at most,
        $2^{n}$ elements. Therefore $f$ is simple.
    \end{lexample}
    Now, let $(\Omega,\mathcal{A},\mu)$ be a measurable space, and let
    $f:\Omega\rightarrow\mathbb{R}$ be a simple function. If $f$ is simple than
    it's range is finite. Let $a_{1},\cdots,a_{n}$ be the distinct elements of
    $f(\Omega)$ and define the following:
    \begin{equation}
        E_{k}=f^{-1}\big(\{a_{k}\}\big)
    \end{equation}
    Since $f$ is simple, it is measurable, and thus $E_{k}\in\mathcal{A}$ for
    all $k\in\mathbb{Z}_{n}$. Moreover, for $i\ne{j}$,
    $E_{i}\cap{E}_{j}=\emptyset$. But, since $f$ is simple, the measure of its
    support is finite, and thus for all $k\in\mathbb{Z}_{n}$, the measure of
    $E_{k}$ is also finite. We can thus obtain the following for $f$:
    \begin{equation}
        f(\omega)=\sum_{k=1}^{n}a_{k}\chi_{E_{k}}(\omega)
    \end{equation}
    This is called the \textrm{Canonical Representation} of $f$.
    \begin{ldefinition}{Integral of a Simple Function}
        The integral of a simple function $f:\Omega\rightarrow\mathbb{R}$ on a
        measurable space $(\Omega,\mathcal{A},\mu)$ with canonical form:
        \begin{equation}
            f(\omega)=\sum_{k=1}^{n}a_{k}\chi_{E_{k}}(\omega)
        \end{equation}
        Is the real number $\int_{\Omega}f\diff{\mu}$ define by:
        \begin{equation}
            \int_{\Omega}f\diff{\mu}=\sum_{k=1}^{n}a_{k}\mu(E_{k})
        \end{equation}
    \end{ldefinition}
    The first thing to check to ensure that this is a good definition of
    integration is that the sum of two integrals is the integral of the sum of
    the two functions. We first prove a result that will make this definition
    more flexible.
    \begin{theorem}
        If $(\Omega,\mathcal{A},\mu)$ is a measurable space, and if
        $B_{1},\dots,B_{n}$ are measurable subsets of $\Omega$ that are pairwise
        disjoint and such that, for all $k\in\mathbb{Z}_{n}$,
        $\mu(B_{k})<\infty$, if $\lambda_{1},\dots,\lambda_{n}$ are real
        numbers, and if $f:\Omega\rightarrow\mathbb{R}$ is defined by:
        \begin{equation}
            f(\omega)=\sum_{k=1}^{n}\lambda_{k}\chi_{B_{k}}(\omega)
        \end{equation}
        Then the integral of $f$ is given by:
        \begin{equation}
            \int_{\Omega}f\diff{\mu}=\sum_{k=1}^{n}\lambda_{k}\mu(B_{k})
        \end{equation}
    \end{theorem}
    \begin{proof}
        For every $\omega\in\Omega$ falls in only one of the $B_{k}$. But if
        $\omega\in{B}_{k}$, then $f(\omega)=\lambda_{k}$. Let
        $a_{1},\hdots,a_{n}$ be the distinct values of $f$ and define
        $\mathcal{J}_{k}$ as:
        \begin{equation}
            \mathcal{J}_{k}=\{j:\lambda_{j}=a_{k}\}
        \end{equation}
        Also, let $\mathcal{J}_{0}$ be defined as:
        \begin{equation}
            \mathcal{J}_{0}=\{j:\lambda_{j}=0\}
        \end{equation}
        If $E_{k}=f^{-1}(\{a_{k}\}$, then:
        \begin{equation}
            E_{k}=\bigcup_{j\in\mathcal{J}_{k}}B_{j}
        \end{equation}
        But since the $B_{j}$ are pair-wise disjoint, we have that:
        \begin{equation}
            \mu(E_{k})=\sum_{j\in\mathcal{J}_{k}}\mu(B_{j})
        \end{equation}
        But then:
        \begin{align}
            \int_{\Omega}f\diff{\mu}
            &=\sum_{k=1}^{n}a_{k}\mu(E_{k})\\
            &=\sum_{k=1}^{n}a_{k}
                \Big(\sum_{j\in\mathcal{J}_{k}}\mu(B_{j})\Big)\\
            &=\sum_{k=1}^{n}\sum_{j\in\mathcal{J}_{k}}a_{k}\mu(B_{j})\\
            &=\sum_{k=1}^{n}\sum_{j\in\mathcal{J}_{k}}\lambda_{j}\mu(B_{j})\\
            &=\sum_{k=1}^{m}\lambda_{k}\mu(B_{k})
        \end{align}
    \end{proof}
    This theorem will make it easier to prove the additive property of
    integrals. We are still only talking about the integral of simple functions.
    \begin{theorem}
        If $(\Omega,\mathcal{A},\mu)$ is a measurable space and
        $f:\Omega\rightarrow\mathbb{R}$ and $g:\Omega\rightarrow\mathbb{R}$ are
        simple functions, then:
        \begin{equation}
            \int_{\Omega}(f+g)\diff{\mu}
            =\int_{\Omega}f\diff{\mu}+\int_{\Omega}g\diff{\mu}
        \end{equation}
    \end{theorem}
    \begin{proof}
        For let $f$ and $g$ have the followingcanonical representations:
        \begin{align}
            f(\omega)&=\sum_{k=1}^{n}\alpha_{k}\chi_{A_{k}}(\omega)\\
            g(\omega)&=\sum_{k=1}^{m}\beta_{k}\chi_{B_{k}}(\omega)
        \end{align}
        Define the following:
        \begin{align}
            A_{n+1}=\Big(\bigcup_{k=1}^{m}B_{k}\Big)\setminus
                \Big(\bigcup_{j=1}^{n}A_{j}\Big)\\
            B_{m+1}=\Big(\bigcup_{k=1}^{n}A_{k}\Big)\setminus
                \Big(\bigcup_{j=1}^{m}B_{j}\Big)
        \end{align}
        From this, we have the following:
        \begin{equation}
            \bigcup_{k=1}^{n+1}A_{k}=\bigcup_{j=1}^{m+1}B_{j}
        \end{equation}
        Let $\alpha_{n+1}=\beta_{m+1}=0$. Then we have:
        \begin{align}
            f(\omega)&=\sum_{k=1}^{n+1}\alpha_{k}\chi_{A_{k}}(\omega)\\
            g(\omega)&=\sum_{k=1}^{m+1}\beta_{k}\chi_{B_{k}}(\omega)
        \end{align}
        Then, for all $i$, we have:
        \begin{align}
            A_{i}&=A_{i}\bigcap\Big(\bigcup_{j=1}^{m+1}B_{j}\Big)\\
            &=\bigcup_{j=1}^{m+1}\Big(A_{i}\cap{B}_{j}\Big)\\
            B_{i}&=B_{i}\bigcap\Big(\bigcup_{k=1}^{n+1}A_{k}\Big)\\
            &=\bigcup_{k=1}^{n+1}\Big(A_{k}\cap{B}_{i}\Big)
        \end{align}
        And these are all pair-wise disjoint sets. So, we have:
        \begin{align}
            f(\omega)&=\sum_{i=1}^{n+1}\sum_{j=1}^{m+1}\alpha_{i}
                \chi_{A_{i}\cap{B}_{j}}(\Omega)\\
            g(\omega)&=\sum_{i=1}^{n+1}\sum_{j=1}^{m+1}\beta_{j}
                \chi_{A_{i}\cap{B}_{j}}(\Omega)
        \end{align}
        Summing these two functions, we get:
        \begin{equation}
            f(\omega)+g(\omega)=\sum_{i=1}^{n+1}\sum_{j=1}^{m+1}
                (\alpha_{i}+\beta_{j})\chi_{A_{i}\cap{B}_{j}}(\Omega)
        \end{equation}
        But by the previous theorem:
        \begin{align}
            \int_{\Omega}(f+g)\diff{\mu}
            &=\sum_{i=1}^{n+1}\sum_{j=1}^{m+1}(\alpha_{i}+\beta_{j})
                \mu(A_{i}\cap{B}_{j})\\
            &=\sum_{i=1}^{n+1}\sum_{j=1}^{m+1}\alpha_{i}\mu(A_{i}\cap{B}_{j})+
                \sum_{i=1}^{n+1}\sum_{j=1}^{m+1}\beta_{j}\mu(A_{i}\cap{B}_{j})\\
            &=\sum_{i=1}^{n+1}\alpha_{i}\mu(A_{i})+
                \sum_{j=1}^{m+1}\beta_{j}\mu(B_{j})\\
            &=\int_{\Omega}f\diff{\mu}+\int_{\Omega}g\diff{\mu}
        \end{align}
    \end{proof}
    The integrals of simple functions also have the property of homogeneity.
    \begin{theorem}
        If $(\Omega,\mathcal{A},\mu)$ is a measurable space,
        $f:\Omega\rightarrow\mathbb{R}$ is a simple function, and if
        $c\in\mathbb{R}$, then:
        \begin{equation}
            \int_{\Omega}cf\diff{\mu}=c\int_{\Omega}f\diff{\mu}
        \end{equation}
    \end{theorem}
    \begin{theorem}
        If $(\Omega,\mathcal{A},\mu)$ is a measurable space,
        $f:\omega\rightarrow\mathbb{R}$ is a simple function, if
        $A_{1},\dots,A_{n}$ are measurable subsets of $\Omega$ with
        finite measure, and if $f$ is such that:
        \begin{equation}
            f(\omega)=\sum_{k=1}^{n}a_{k}\chi_(A_{k})(\omega)
        \end{equation}
        Then:
        \begin{equation}
            \int_{\Omega}f\diff{\mu}=\sum_{k=1}^{n}a_{k}\mu(A_{k})
        \end{equation}
    \end{theorem}
    \subsection{Further Properties of the Integral}
        So far we have defined the integral of a simple function over the
        entire of a given space $\Omega$. We often wish to evaluate the integral
        of a function on a subset of $\Omega$, rather than the entire of it. We
        can do this by defining the following:
        \begin{equation}
            f_{E}(\omega)=
            \begin{cases}
                f(\omega),&\omega\in{E}\\
                0,&\omega\notin{E}
            \end{cases}
        \end{equation}
        We need some properties of $f_{E}$. $f_{E}$ is measurable, has finite
        range, and has support of finite measure, and is therefore simple.
        $f_{E}$ can have only one more value (that is, zero) than $f$. Finally,
        $\supp(f_{E})\subseteq{\supp(f)}$. We define the integral on
        $E\in\mathcal{A}$ as follows:
        \begin{equation}
            \int_{E}f\diff{\mu}=\int_{\Omega}f_{E}\diff{\mu}
        \end{equation}
        Since $f_{E}$ is also simple, the right hand side of this equation is
        well defined.
        \begin{theorem}
            If $(\Omega,\mathcal{A},\mu)$ is a measurable space, if
            $E_{1},E_{2}$ are disjoint measurable subsets of $\Omega$, and if
            $f:\Omega\rightarrow\mathbb{R}$ is a simple function, then:
            \begin{equation}
                \int_{E_{1}\cup{E}_{2}}f\diff{\mu}=
                    \int_{E_{1}}f\diff{\mu}+\int_{E_{2}}f\diff{\mu}
            \end{equation}
        \end{theorem}
        This is similar to a notion that is found when studying the Riemann
        integral. That is:
        \begin{equation}
            \int_{a}^{b}f(x)\diff{x}
                =\int_{a}^{c}f(x)\diff(x)+\int_{c}^{b}f(x)\diff{x}
        \end{equation}
        \begin{theorem}
            If if $f:\omega\rightarrow\mathbb{R}$ is a simple function, and if
            $f(\omega)\geq{0}$ for all $\omega\in\Omega$, then:
            \begin{equation}
                \int_{\Omega}f\diff{\mu}\geq{0}
            \end{equation}
        \end{theorem}
        \begin{theorem}
            If $f:\Omega\rightarrow\mathbb{R}$ is simple, then:
            \begin{equation}
                \int_{\Omega}f\diff{\mu}=0
            \end{equation}
            If and only if $f=0$ $\mu$ almost everywhere.
        \end{theorem}
        \begin{ftheorem}{Triangle Inequality for Simple Functions}{}
            If $f$ is a simple function, then:
            \begin{equation}
                \Big|\int_{\Omega}f\diff{\mu}\Big|\leq\int_{\Omega}|f|\diff{\mu}
            \end{equation}
        \end{ftheorem}
    \subsection{Limit Theorems for Simple Functions}
        \begin{theorem}
            If $f_{n}:\Omega\rightarrow\mathbb{R}$ is a sequence of simple
            functions such that $f_{n}\rightarrow{f}$ uniformly, where $f$ is a
            simple function, and if there is a measurable set $E$ such that
            $\supp(f_{n})\subseteq{E}$ and $\supp(f)\subseteq{E}$, and if
            $\mu(E)<\infty$, then:
            \begin{equation}
                \underset{n\rightarrow\infty}{\lim}\int_{\Omega}f_{n}\diff{\mu}
                    =\int_{\Omega}f\diff{\mu}
            \end{equation}
        \end{theorem}
        \begin{proof}
            For:
            \begin{align}
                \Big|\int_{\Omega}f_{n}\diff{\mu}-\int_{\Omega}f\diff{\mu}\Big|
                &=\Big|\int_{\Omega}(f_{n}-f)\diff{\mu}\Big|\\
                &\leq\int_{\Omega}|f_{n}-f|\diff{\mu}\\
                &=\int_{E}|f_{n}-f|\diff{\mu}\\
                &\leq\int_{E}\varepsilon\diff{\mu}\\
                &=\varepsilon\int_{E}\diff{\mu}\\
                &=\varepsilon\mu(E)
            \end{align}
            Since $\mu(E)<\infty$, this can be made arbitrarily small.
        \end{proof}
        \begin{theorem}
            If $f_{n}\rightarrow{f}$ uniformly, and if $f_{n}$ and $f$ are
            simple, then $f_{n}$ is uniformly bounded.
        \end{theorem}
        \begin{theorem}[Bounded Convergence Theorem]
            If $f_{n}\rightarrow{f}$, $f_{n}$ are simple and $f$ is simple, if
            $f_{n}$ are uniformly bounded, and if there is a measurable set $E$
            of finite measure such that $\supp(f_{n})\subseteq{E}$ and
            $\supp(f)\subseteq{E}$, then:
            \begin{equation}
                \underset{n\rightarrow\infty}{\lim}\int_{\Omega}f_{n}\diff{\mu}
                =\int_{\Omega}f\diff{\mu}
            \end{equation}
        \end{theorem}
        \begin{lexample}
            The additional assumptions are indeed necessary, and without them
            these results may not hold. For let $f_{n}$ be defined as:
            \begin{equation}
                f_{n}(\omega)=
                \begin{cases}
                    n,&0\leq\omega\leq\frac{1}{n}\\
                    0,&\textrm{Otherwise}
                \end{cases}
            \end{equation}
            Then $\int_{\mathbb{R}}f_{n}\diff{\mu}=1$ for all $n$, but the limit
            function is $f=0$, and this has integral zero. It is also not
            guarenteed that the results fail, for let:
            \begin{equation}
                f_{n}(\omega)=
                \begin{cases}
                    n,&0\leq\omega\leq\frac{1}{n^{2}}\\
                    0,&\textrm{Otherwise}
                \end{cases}
            \end{equation}
            Then the limit function is zero, and the integral is $\frac{1}{n}$,
            which does indeed converge to zero. For the requirement that
            $\supp(f_{n})$ and $\supp(f)$ be contained in one set, consider the
            following function:
            \begin{equation}
                f_{n}(\omega)=
                \begin{cases}
                    1,&n\leq\omega\leq{n+1}\\
                    0,&\textrm{Otherwise}
                \end{cases}
            \end{equation}
            Then $f_{n}$ is uniformly bounded, converges to $0$, but the
            integral is 1 for all $n$.
        \end{lexample}
        \begin{theorem}[Monotone Convergence Theorem]
            If $f_{n}$, $f$ are simple functions, if
            $f_{n}\rightarrow{f}$, if $f_{n}\leq{f}_{n+1}$,
            then:
            \begin{equation}
                \int_{\Omega}f_{n}\diff{\mu}
                \rightarrow\int_{\Omega}f\diff{\mu}
            \end{equation}
        \end{theorem}
        \begin{theorem}
            \label{thm:MEASURE_THEORY_LIM_INT_MONO_SIMPLE_FUNCS}%
            If $f_{n}$ and $g_{n}$ are simple and monotonically increasing,
            and if:
            \begin{equation}
                \underset{n\rightarrow\infty}{\lim}f_{n}(\omega)
                =\underset{n\rightarrow\infty}{\lim}g_{n}(\omega)
            \end{equation}
            Then:
            \begin{equation}
                \underset{n\rightarrow\infty}{\lim}\int_{\Omega}f_{n}\diff\mu
                =\underset{n\rightarrow\infty}{\lim}\int_{\Omega}g_{n}\diff\mu
            \end{equation}
        \end{theorem}
        This theorem will allow us to extend the definition of the integral to
        a more general class of functions.
    \subsection{Integration of Non-Negative Measurable Functions}
        Consider a measure space $(\Omega,\mathcal{A},\mu)$ and let
        $f:\Omega\rightarrow\mathbb{R}$ be an $\mathcal{A}-\mathcal{B}$
        measurable function, where $\mathcal{B}$ is the Borel
        $\sigma\textrm{-Algebra}$ on $\mathbb{R}$. If there exist a sequence of
        simple functions $f_{n}$ that are monotonically increasing and such
        that $f_{n}\rightarrow{f}$, then we define the integral of $f$ as
        follows:
        \begin{equation}
            \int_{\Omega}f\diff{\mu}=\underset{n\rightarrow\infty}{\lim}
                \int_{\Omega}f_{n}\diff{\mu}
        \end{equation}
        Because of Thm.~\ref{thm:MEASURE_THEORY_LIM_INT_MONO_SIMPLE_FUNCS} this
        is a well defined concept, since for any two sequences of simples
        functions that are monotonically increasing and converge to $f$, the
        limit of the integrals is the same, thus giving a consitent definition
        to the integral of $f$. The first question that then arises is which
        measurable functions can be approximated arbitrarily well by a sequence
        of monotonically increasing simple functions? From the definition we
        will need that $f$ is bounded below. For now we will discuss measurable
        functions that are non-negative. Let $f:\Omega\rightarrow\mathbb{R}$ be
        any non-negative function, it need not be measurable. We wish to
        construct a sequence of simple functions $f_{n}$ such that $f_{n}$ is
        monotonically increasing, and for all $\omega\in\Omega$,
        $f_{n}(\omega)\rightarrow{f}(\omega)$. We construct such a sequence as
        follows, dividing the $[0,n)$ into $n2^{n}-1$ parts
        $[\frac{i}{2^{n}},\frac{i+1}{2^{n}})$ and define the following sets:
        \begin{align}
            E_{n}^{C}&=\{\omega\in\Omega:f(\omega)\geq{n}\}\\
            E_{n,i}&=\{\omega:\frac{i}{2^{n}}\leq{f}(\omega)
                \leq\frac{i+1}{2^{n}}\}\\
            &=f^{-1}\big([\frac{i}{2^{n}},\frac{i+1}{2^{n}})\big)
        \end{align}
        Then, for every fixed $n\in\mathbb{N}$, the sets $E_{n,i}$ are pairwise
        disjoint. Defined $f_{n}$ as follows:
        \begin{equation}
            f_{n}(\omega)=
            \begin{cases}
                n,&f(\omega)\geq{n}\\
                \frac{i}{2^{n}},&\frac{i}{2^{n}}
                    \leq{f}(\omega)\leq\frac{i+1}{2^{n}}
            \end{cases}
        \end{equation}
        Then, using the sets $E_{n,i}$ and $E_{n}^{C}$, we can rewrite $f_{n}$
        as follows:
        \begin{equation}
            f_{n}(\omega)=n\chi_{E_{n}^{C}}(\omega)+
                \sum_{i=0}^{n2^{n}-1}\frac{i}{2^{n}}\chi_{E_{n,i}}(\omega)
        \end{equation}
        We now have that $f_{n}$ is monotonically increasing and tends to $f$.
        For is $f(\omega)=\infty$, then:
        \begin{equation}
            w\in\cap_{n=1}^{\infty}E_{n}^{C}
            \Rightarrow
            f_{n}(\omega)=n\rightarrow\infty
        \end{equation}
        If $f(\omega)\in\mathbb{R}$, then there is an $N\in\mathbb{N}$ such that
        $f(\omega)<N$. But then, for all $n>N$, $f(\omega)<n$ and thus there is
        an $i\in\mathbb{Z}_{n2^{n}-1}$ such that:
        \begin{equation}
            \frac{i}{2^{n}}\leq{f}(\omega)\leq\frac{i+1}{2^{n}}
        \end{equation}
        But $f_{n}(\omega)=\frac{i}{2^{n}}$ and thus:
        \begin{equation}
            0\leq{f}(\omega)-f_{n}(\omega)\leq\frac{1}{2^{n}}
        \end{equation}
        And therefore $f_{n}\rightarrow{f}$. Finally, $f_{n}$ is monotonically
        increasing. Now let's see what we can add to this if we know that $f$
        is measurable. Since $f$ is measurable, the pre-image
        $f^{-1}([n,\infty))\in\mathcal{A}$, since $[n,\infty)$ is a Borel set
        for all $n\in\mathbb{N}$. Moreover, for all $n$ and $i$,
        $E_{n,i}\in\mathcal{A}$. Then all of the indicator functions
        $\chi_{E_{n,i}}$ are measurable, and thus $f_{n}$ is measurable for all
        $n$. However, the support of the $f_{n}$ may not be finite. Simply take
        $f(\omega)=1$ for all $\omega$, and let $\Omega=\mathbb{R}$. If the
        $\mu(\Omega)<\infty$, then $\mu(\supp(f_{n}))<\infty$. This case is
        particularly important when studying probability theory.
        \begin{ldefinition}{$\sigma\textrm{-Finite}$}
            A $\sigma\textrm{-Finite}$ measure on a $\sigma\textrm{-Algebra}$
            $\mathcal{A}$ of a set $\Omega$ is a measure $\mu$ such that there
            is a sequence of sets $\Omega_{n}$ such that
            $\Omega_{n}\subseteq\Omega_{n+1}$,
            $\Omega=\cup_{n=1}^{\infty}\Omega_{n}$, and for all
            $n\in\mathbb{N}$, $\mu(\Omega_{n})<\infty$.
        \end{ldefinition}
        \begin{lexample}
            Let $\Omega=\mathbb{R}$ and consider the Borel
            $\sigma\textrm{-Algebra}$ $\mathcal{B}$ on $\mathbb{R}$. Then the
            standard Lebesgue-Measure is $\sigma\textrm{-finite}$ since we can
            write:
            \begin{equation}
                \mathbb{R}=\cup_{n=1}^{\infty}[-n,n]
            \end{equation}
            And $\mu([-n,n])=2n$, which is finite.
        \end{lexample}
        Define $\tilde{f}_{n}$ as follows:
        \begin{equation}
            f_{n}(\omega)=f_{n}(\omega)\cdot\chi_{\Omega_{n}}(\omega)
        \end{equation}
        Then we have that $\tilde{f}_{n}$ is simple, measurable, takes on
        finitely many values, and the measure of it's support is finite. Thus
        we have that if $(\Omega,\mathcal{A},\mu)$ is a measure space and if
        $\mu$ is $\sigma\textrm{-Finite}$, then for any non-negative measurable
        function $f:\Omega\rightarrow\mathbb{R}$, the integral of $f$ is well
        defined. We write:
        \begin{equation}
            \int_{\Omega}f\diff{\mu}=\underset{n\rightarrow\infty}{\lim}
                \int_{\Omega}f_{n}\diff{\mu}
        \end{equation}
    \subsection{Properties of the Integral of Non-Negative Functions}
    \begin{theorem}[Homogeneity of the Integral]
        If $f$ is a non-negative measurable function, and if $c>0$, then:
        \begin{equation}
            \int_{\Omega}(cf)\diff{\mu}=c\int_{\Omega}f\diff{\mu}
        \end{equation}
    \end{theorem}
    \begin{proof}
        For let $f_{n}$ be a sequence of simple functions such that
        $f_{n}\rightarrow{f}$ and $f_{n}$ is monotonically increasing. Then:
        \begin{equation}
            \int_{\Omega}(cf)\diff{\mu}=\underset{n\rightarrow\infty}{\lim}
            \int_{\Omega}(cf_{n})\diff{\mu}=c\underset{n\rightarrow\infty}{\lim}
            \int_{\Omega}f_{n}\diff{\mu}=c\int_{\Omega}f\diff{\mu}
        \end{equation}
    \end{proof}
    \begin{theorem}[Additivity of the Integral]
        If $f$ and $g$ are non-negative and measurable, then:
        \begin{equation}
            \int_{\Omega}(f+g)\diff{\mu}
            =\int_{\Omega}f\diff{\mu}+\int_{\Omega}g\diff{\mu}
        \end{equation}
    \end{theorem}
    \begin{proof}
        For let $f_{n}$ and $g_{n}$ be simple functions such that
        $f_{n}\rightarrow{f}$, $g_{n}\rightarrow{g}$, and such that $f_{n}$ and
        $g_{n}$ are monotonically increasing. Then:
        \begin{align}
            \int_{\Omega}(f+g)\diff{\mu}
            &=\underset{n\rightarrow\infty}{\lim}
            \int_{\Omega}(f_{n}+g_{n})\diff{\mu}\\
            &=\underset{n\rightarrow\infty}{\lim}
            \int_{\Omega}f_{n}\diff{\mu}+
            \underset{n\rightarrow\infty}{\lim}
            \int_{\Omega}g_{n}\diff{\mu}\\
            &=\int_{\Omega}f\diff{\mu}+\int_{\Omega}g\diff{\mu}
        \end{align}
    \end{proof}
    \begin{theorem}
        If $f$ is non-negative and measurable, and if
        $E_{1},E_{2}$ are disjoint sets, then:
        \begin{equation}
            \int_{E_{1}\cup{E}_{2}}f\diff{\mu}
            =\int_{E_{1}}f\diff{\mu}+\int_{E_{2}}f\diff{\mu}
        \end{equation}
    \end{theorem}
    \begin{theorem}
        If $f$ is a non-negative measurable function, then:
        \begin{equation}
            \int_{\Omega}f\diff{\mu}\geq{0}
        \end{equation}
    \end{theorem}
    \begin{theorem}
        If $f$ is a non-negative measurable function, then:
        \begin{equation}
            \int_{\Omega}f\diff{\mu}=0
            \Longleftrightarrow{f=0}
            \quad\mu\textrm{-almost everywhere}
        \end{equation}
    \end{theorem}
    \begin{ldefinition}{Summable Functions}
        A non-negative summable function is a non-negative
        and measurable function from
        a measure space $(\Omega,\mathcal{A},\mu)$ where
        $\mu$ is $\sigma\textrm{-Finite}$ to $\mathbb{R}$
        such that:
        \begin{equation}
            \int_{\Omega}f\diff{\mu}<\infty
        \end{equation}
    \end{ldefinition}
    \begin{ltheorem}{Chebyshev's Inequality}
        If $f$ is a non-negative measurable function, then for
        all $a\in\mathbb{R}^{+}$:
        \begin{equation}
            \mu\Big(\{\omega:f(\omega)\geq{a}\}\Big)
            <\frac{1}{a}\int_{\Omega}f\diff{\mu}
        \end{equation}
    \end{ltheorem}
    \begin{proof}
        For define the following:
        \begin{align}
            E_{1}&=\{\omega:f(\omega)\geq{a}\}\\
            E_{2}&=\{\omega:f(\omega)<a\}
        \end{align}
        Then $E_{1}$ and $E_{2}$ are disjoint, and therefore:
        \begin{equation}
            \int_{\Omega}f\diff{\mu}=
            \int_{E_{1}}f\diff{\mu}+
            \int_{E_{2}}f\diff{\mu}
            \geq\int_{E_{1}}f\diff{\mu}
            \geq\int_{\Omega}a\diff{\mu}
            =a\mu(E_{1})
        \end{equation}
        Dividing by $a$ completes the proof.
    \end{proof}
    \begin{theorem}
        If $f$ is a non-negative summable function, then
        for all $a\in\mathbb{R}^{+}$:
        \begin{equation}
            \underset{a\rightarrow\infty}{\lim}
            \mu\Big(\{\omega:f(\omega)\geq{a}\}\Big)=0
        \end{equation}
    \end{theorem}
    \begin{ltheorem}{Monotone Convergence Theorem}
        If $f$ is a non-negative measurable function and if
        $f_{n}$ is a sequence of non-negative measurable
        functions that are monotonically increasing and such
        that $f_{n}\rightarrow{f}$, then:
        \begin{equation}
            \underset{n\rightarrow\infty}{\lim}
            \int_{\Omega}f_{n}\diff{\mu}
            =\int_{\Omega}f\diff{\mu}
        \end{equation}
    \end{ltheorem}
    \begin{proof}
        For all $n\in\mathbb{N}$, there is a function
        $g_{n,k}$ such that $g_{n,k}$ is simple and:
        \begin{equation}
            \int_{\Omega}f_{n}\diff{\mu}=
            \underset{k\rightarrow\infty}{\lim}
            \int_{\Omega}g_{n,k}\diff{\mu}
        \end{equation}
        Define $F_{n}$ as follows:
        \begin{equation}
            F_{n}(\omega)=
            \max\{g_{j,n}(\omega):
                \omega\in\Omega,0\leq{j}\leq{n}\}
        \end{equation}
        Then, for all $n\in\mathbb{N}$, $F_{n}$ is simple.
        For it is the maximum of finitely many measurable
        functions, and is therefore measurable. Moreover:
        \begin{equation}
            \supp(F_{n})\subseteq
            \bigcup_{k=1}^{n}\supp(f_{k,n})
        \end{equation}
        And finally, $F_{n}$ is monotonically increasing from
        it's definition. Now we must show that
        $F_{n}\rightarrow{f}$. For:
        \begin{equation}
            g_{k,n}\leq{F}_{k}\leq{f}_{k}
        \end{equation}
        Since $g_{n,k}$ increases monotonically to $f_{n}$.
        Taking the limit as $k\rightarrow\infty$, we obtain:
        \begin{equation}
            f_{n}\leq\underset{k\rightarrow\infty}{\lim}F_{k}
            \leq{f}
        \end{equation}
        Then taking the limit on $n$, we see that
        $F_{n}\rightarrow{f}$. Integrating this inequality, we
        get:
        \begin{equation}
            \int_{\Omega}g_{k,n}\diff{\mu}
            \leq\int_{\Omega}F_{k}\diff{\mu}
            \leq\int_{\Omega}f\diff{\mu}
        \end{equation}
        Taking the limit as $k\rightarrow\infty$, we get:
        \begin{equation}
            \underset{k\rightarrow\infty}{\lim}
            \int_{\Omega}g_{k,n}\diff{\mu}
            \leq\int_{\Omega}f\diff{\mu}
            \leq\underset{k\rightarrow\infty}{\lim}
            \int_{\Omega}f_{k}\diff{\mu}
        \end{equation}
        Finally, taking the limit on $n$, we get:
        \begin{equation}
            \underset{n\rightarrow\infty}{\lim}
            \int_{\Omega}f_{n}\diff{\mu}
            \leq\int_{\Omega}f\diff{\mu}
            \leq\underset{k\rightarrow\infty}{\lim}
            \int_{\Omega}f_{k}\diff{\mu}
        \end{equation}
        This completes the proof.
    \end{proof}
    Let $(\Omega,\mathcal{A},\mu)$ be a measure space and let
    $f\geq{0}$ be measurable. From before we were able to define
    the integral of $f$ is $\mu$ is $\sigma\textrm{-finite}$. We
    approximate $f$ with an increasing sequence of simple functions
    that are also non-negative. The integral of $f$ is defined as
    the limit of the integrals of the approximating
    simple functions. That is, we define the integral to be:
    \begin{equation}
        \int_{\Omega}f\diff{\mu}=
        \underset{n\rightarrow\infty}{\lim}
        \int_{\Omega}f_{n}\diff{\mu}
    \end{equation}
    We have seen from a previous theorem that the value
    of the integral is independent of the approximating
    sequence. That is, for $f_{n}$ and $g_{n}$ are a
    sequence of simple functions that are monotonically
    increasing to $f$, then:
    \begin{equation}
        \underset{n\rightarrow\infty}{\lim}
        \int_{\Omega}f_{n}\diff{\mu}=
        \underset{n\rightarrow\infty}{\lim}
        \int_{\Omega}g_{n}\diff{\mu}
    \end{equation}
    We then proved the monotone convergence theorem.
    \begin{ltheorem}{Monotone Convergence Theorem}
        If $f_{n}$ is a sequence of positive measurable functions,
        not necessarily simple, and if $f_{n}$ is monotonically
        increasing, then:
        \begin{equation}
            \underset{n\rightarrow\infty}{\lim}\int_{\Omega}f_{n}\diff{\mu}
            =\int_{\Omega}\underset{n\rightarrow\infty}{\lim}f_{n}\diff{\mu}
        \end{equation}
    \end{ltheorem}
    Note that we are still only talking about non-negative measurable
    functions. We have yet to discuss functions that are possibly negative.
    \begin{ltheorem}{Fatou's Theorem}
        If $f_{n}$ is a sequence of non-negative measurable functions, then:
        \begin{equation}
            \int_{\Omega}\underset{n\rightarrow\infty}{\underline{\lim}}
                f_{n}\diff{\mu}
            \leq\underset{n\rightarrow\infty}{\underline{\lim}}
                \int_{\Omega}f_{n}\diff{\mu}
        \end{equation}
        Where $\underline{\lim}$ denotes the limit-inferior.
    \end{ltheorem}
    \begin{proof}
        For:
        \begin{equation}
            0\leq\inf_{k\geq{n}}f_{k}(\omega)
            \leq{f}_{k}(\omega)
        \end{equation}
        And therefore:
        \begin{equation}
            \int_{\Omega}\inf_{k\geq{n}}f_{k}\diff{\mu}
            \leq\int_{\Omega}f_{k}\diff{\mu}
        \end{equation}
        And therefore:
        \begin{equation}
            \int_{\Omega}\inf_{k\geq{n}}f_{k}\diff{\mu}
            \leq\inf_{k\geq{n}}\int_{\Omega}f_{k}\diff{\mu}
        \end{equation}
        But:
        \begin{equation}
            \underset{n\rightarrow\infty}{\lim}
            \int_{\Omega}\inf_{k\geq{n}}f_{k}\diff{\mu}
            \leq
            \underset{n\rightarrow\infty}{\lim}
            \inf_{k\geq{n}}\int_{\Omega}f_{k}\diff{\mu}
            =\underset{n\rightarrow\infty}{\underline{\lim}}
            \int_{\Omega}f_{n}\diff{\mu}
        \end{equation}
        Therefore, etc.
    \end{proof}
    \begin{theorem}
        If $f_{n}$ is a sequence of non-negative measurable functions,
        then the function $f$ defined by:
        \begin{equation}
            f=\underset{N\rightarrow\infty}{\lim}
            \sum_{n=0}^{N}f_{n}
        \end{equation}
        Is measurable.
    \end{theorem}
    \begin{theorem}
        If $f_{n}$ is a sequence of non-negative measurable functions
        and if $f$ is defined by:
        \begin{equation}
            f=\underset{N\rightarrow\infty}{\lim}
            \sum_{n=0}^{N}f_{n}
        \end{equation}
        Then:
        \begin{equation}
            \int_{\Omega}f\diff{\mu}=
            \underset{N\rightarrow\infty}{\lim}
            \sum_{n=0}^{N}\int_{\Omega}f_{n}\diff{\mu}
        \end{equation}
    \end{theorem}
    \begin{theorem}
        If $(\Omega,\mathcal{A},\mu)$ is a measure space,
        if $f$ is measurable and non-negative, and if
        $\nu:\mathcal{A}\rightarrow\mathbb{R}$ is defined by:
        \begin{equation}
            \nu(E)=\int_{E}f\diff{\mu}=
            \int_{\Omega}f_{E}\diff{\mu}
        \end{equation}
        Then $\nu$ is a measure on $\mathcal{A}$.
    \end{theorem}
    \begin{proof}
        For $\mu(\emptyset)=0$ by definition. Since $f$ is positive,
        for all $E\in\mathcal{A}$:
        \begin{equation}
            \nu(E)=\int_{\Omega}f_{E}\diff{\mu}\geq{0}
        \end{equation}
        And finally, if $E_{n}$ are pairwise disjoint then:
        \begin{equation}
            \nu\Big(\bigcup_{n=1}^{\infty}E_{n}\Big)=
            \int_{\bigcup_{n=1}^{\infty}E_{n}}f\diff{\mu}
            =\sum_{n=1}^{\infty}\int_{E_{n}}f\diff{\mu}
            =\sum_{n=1}^{\infty}\nu(E_{n})
        \end{equation}
        Therefore, etc.
    \end{proof}
    \begin{theorem}
        If $(\Omega,\mathcal{A},\mu)$ is a measure space,
        if $f$ is measurable and non-negative, if
        $\nu:\mathcal{A}\rightarrow\mathbb{R}$ is defined by:
        \begin{equation}
            \nu(E)=\int_{E}f\diff{\mu}=
            \int_{\Omega}f_{E}\diff{\mu}
        \end{equation}
        And if $E\in\mathcal{A}$ is such that $\mu(E)=0$, then
        $\nu(E)=0$.
    \end{theorem}
    \begin{ldefinition}{Absolute Continuity}
        An absolutely continuous measure $\nu$ with respect
        to a measure space $(\Omega,\mathcal{A},\mu)$ is a meausre
        $\nu$ on $\mathcal{A}$ such that for all $E\in\mathcal{A}$
        such that $\mu(E)=0$, it is true that $\nu(E)=0$. This is
        denoted $\nu<<\mu$.
    \end{ldefinition}
    \begin{ltheorem}{Radon-Nikodym Theorem}
        If $(\Omega,\mathcal{A},\mu)$ is a measure space and if
        $\nu$ is absolutely continuous with respect to
        $(\Omega,\mathcal{A},\nu)$. then there is a measurable
        non-negative function $f$ such that, for all $E\in\mathcal{A}$:
        \begin{equation}
            \nu(E)=\int_{E}f\diff{\mu}
        \end{equation}
    \end{ltheorem}
    The function $f$ in the previous theorem is often called the
    density of $\nu$ against $\mu$, or the
    Radon-Nikodym derivative of $\nu$ with respect to $\mu$. The
    function $f$ is unique $\mu$ almost everywhere.