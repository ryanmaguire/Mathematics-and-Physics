\section{Product Measures}
    Let $(\Omega_{1},\mathcal{A}_{1},\mu_{1})$ and
    $(\Omega_{2},\mathcal{A}_{2},\mu_{2})$ be measure spaces. We
    wish to define a \textit{natural} measure space
    on the Cartesian product $\Omega_{1}\times\Omega_{2}$.
    Let $\mathcal{P}$ be defined by:
    \begin{equation}
        \mathcal{P}=
        \{A_{1}\times{A}_{2}:
            A_{1}\in\mathcal{A}_{1},A_{2}\in\mathcal{A}_{2}\}
    \end{equation}
    Then $\mathcal{P}$ is a semi-ring, but is not a
    $\sigma\textrm{-Algebra}$ on $\Omega_{1}\times\Omega_{2}$
    This is because the union of two rectangles may not be a
    rectangle. Similarly, the difference of two rectangles may not
    be a rectangle. However, the intersection of two rectangles is
    a rectangle, and hence this is a semi-ring.
    We defined the product $\sigma\textrm{-Algebra}$ to be the
    $\sigma\textrm{-Algebra}$ that is generated by $\mathcal{P}$. 
    \begin{ltheorem}{Carath\'{e}odory Extension Theorem}
        If $(\Omega_{1},\mathcal{A},\mu_{1})$ and
        $(\Omega_{2},\mathcal{A}_{2},\mu_{2})$ are measure spaces,
        if $\mathcal{A}$ is the product $\sigma\textrm{-Algebra}$
        on $\Omega_{1}\times\Omega_{2}$, then there is a unique
        measure $\mu$ on $\mathcal{A}$ such that, for all
        $A_{1}\in\mathcal{A}_{1}$ and all
        $A_{2}\in\mathcal{A}_{2}$:
        \begin{equation}
            \mu(A_{1}\times{A}_{2})
            =\mu_{1}(A_{1})\cdot\mu_{2}(A_{2})
        \end{equation}
    \end{ltheorem}
    \begin{ltheorem}{Funini's Theorem}
        If $f:\Omega_{1}\times\Omega_{2}\rightarrow\mathbb{R}$
        is a non-negative function that is
        $\mathcal{A}-\mathcal{B}$ measurable, where
        $\mathcal{A}$ is the product $\sigma\textrm{-Algebra}$,
        then:
        \begin{equation}
            \int_{\Omega_{1}\times\Omega_{2}}f\diff{\mu}=
            \int_{\Omega_{1}}\Big(
                \int_{\Omega_{2}}f\diff{\mu_{2}}
            \Big)\diff{\mu}_{1}
            =\int_{\Omega_{2}}
                \Big(\int_{\Omega_{1}}f\diff{\mu_{1}}\Big)
                \diff{\mu}_{2}
        \end{equation}
    \end{ltheorem}
    As a summary, when is the following true?
    \begin{equation}
        \underset{n\rightarrow\infty}{\lim}
        \int_{\Omega}f_{n}\diff{\mu}
        \overset{?}{=}\int_{\Omega}
        \underset{n\rightarrow\infty}{\lim}f_{n}\diff{\mu}
    \end{equation}
    There are two special cases when equality can be guarenteed.
    The first is monotone convergence. If
    $f_{n}\rightarrow{f}$, where $f_{n+1}(x)\leq{f}_{n}(x)$ for
    all $n$, and if $f_{n}(x)\geq{F}$, where $F$ is a
    summable minorant, or if $f_{n}\rightarrow{f}$,
    $f_{n+1}(x)\leq{f}_{n}(x)$, and if
    $f_{n}(x)\leq{F}$, where $F$ is a summable majorant, then
    equality holds. The next case is by dominated convergence.
    If the limit of $f_{n}$ exists almost everywhere, and if
    $|f_{n}|\leq{F}$, where $F$ is summable, then by Fatou's
    Lemma:
    \begin{equation}
        \underset{n\rightarrow\infty}{\underline{\lim}}
        \int_{\Omega}f_{n}\diff{\mu}
        \geq\int_{\Omega}
        \underset{n\rightarrow\infty}{\underline{\lim}}
        f_{n}\diff{\mu}
    \end{equation}
    And also:
    \begin{equation}
        \underset{n\rightarrow\infty}{\overline{\lim}}
        \int_{\Omega}f_{n}\diff{\mu}
        \leq\int_{\Omega}
        \underset{n\rightarrow\infty}{\overline{\lim}}
        f_{n}\diff{\mu}
    \end{equation}
\section{Probablity Spaces}
    To add later:
    \begin{enumerate}
        \item Probability space
        \item Independent $\sigma\textrm{-Algebras}$.
        \item Independent sets.
        \item Infinite sequence of $\sigma\textrm{-Algebras}$.
        \item Tail $\sigma\textrm{-Algebra}$.
        \item Terminal $\sigma\textrm{-Algebra}$.
        \item Kologorov zero-one law.
        \item If $\mathcal{A}_{j}$ independent, $F$ is self-independent.
        \item $E_{1},E_{2}\in{F}$,
              $\mu(E_{1}\cap{E}_{2})=\mu(E_{1})\mu(E_{2})$, then
              $\mu(E_{1})=0$ or $\mu(E_{1})=1$.
        \item Uniting $\sigma\textrm{-Algebras}$ lemma.
        \item $A_{1},\dots,A_{n}$ independent, then $A_{k},F$ independent,
              where $F$ is the tail $\sigma\textrm{-Algebra}$.
    \end{enumerate}
\section{Random Variables}
    Let $(\Omega,\mathcal{A},\mu)$ be a probability space.
    A probability space is a measure space such that
    $\mu(\Omega)=1$. Let $f:\Omega\rightarrow\mathbb{R}$ be
    $\mathcal{A}-\mathcal{B}$ measurable, where $\mathcal{B}$ is
    the Borel $\sigma\textrm{-Algebra}$. Such functions are called
    random-variables on $\Omega$. While there's nothing random
    about this, we use such functions to model problems in
    probability theory. The probability of an event
    $A\in\mathcal{A}$ is simply $\mu(A)$. The associated
    $\sigma\textrm{-Algebra}$ is defined as:
    \begin{equation}
        \mathcal{A}_{f}=\{f^{-1}(B):B\in\mathcal{B}\}
    \end{equation}
    This is also called the $\sigma\textrm{-Algebra}$ of events
    bearing on $f$. This is a $\sigma\textrm{-Algebra}$ on
    $\Omega$.
    \begin{ldefinition}{Distribution of a Random Variable}
        The distribution of a random variable
        $f:\Omega\rightarrow\mathbb{R}$ on a probability space
        $(\Omega,\mathcal{A},\mu)$ is the image measure
        $\mu_{f}$ of $f$.
    \end{ldefinition}
    The image measure is the measure:
    \begin{equation}
        \mu_{f}(B)=\mu(f^{-1}(B))
        =\mu(\{\omega\in\Omega:f(\omega)\in{B}\})
    \end{equation}
    This is a Lebesgue-Stieljes Measure on the Borel
    $\sigma\textrm{-Algebra}$ on $\mathbb{R}$.
    \begin{equation}
        \mu_{f}(\mathbb{R})=\mu(f^{-1}(\mathbb{R}))
        =\mu(\Omega)=1
    \end{equation}
    \begin{ldefinition}{Cumulative Distribution Function}
        The Cumulative Distribution Function of a random variable
        $f:\Omega\rightarrow\mathbb{R}$ on a probability space
        $(\Omega,\mathcal{A},\mu)$ is the function
        $F:\mathbb{R}\rightarrow\mathbb{R}$ defined by:
        \begin{equation}
            F(x)=\mu_{f}\big((-\infty,a)\big)
        \end{equation}
        Where $\mu_{f}$ is the distribution of $f$.
    \end{ldefinition}
    Some facts about the cumulative distribution function:
    It is non-decreasing on $\mathbb{R}$, left continuous, and
    $F(\minus\infty)-F(\infty)=1$. By the Caratheodory extension
    theorem, and function $F$ that satisfies these three
    conditions is the cumulative distribution function of some
    Lebesgue-Stieljes probability measure on $\mathbb{R}$. From this
    we also have that every Lebesgue-Stieljes probability measure
    on $\mathbb{R}$ is a distribution for a random variable.
    \begin{example}
        Let $\Omega=\mathbb{R}$, let $\mathcal{A}=\mathcal{B}$,
        where $\mathcal{B}$ is the Borel $\sigma\textrm{-Algebra}$,
        and let $\mu$ be a Lebesgue-Stieljes probability measure
        on $\mathbb{R}$. Define the random variable
        $f:\Omega\rightarrow\mathbb{R}$ by
        $f(\omega)=\omega$. The inverse of any Borel set is itself,
        and thus we see that the distribution and the random
        variable coincide.
    \end{example}
    \begin{example}
        Let $\Omega=[0,1]$, $\mathcal{B}$ be the Borel
        $\sigma\textrm{-Algebra}$, and define
        $f_{1},f_{2}:\Omega\rightarrow\mathbb{R}$ by:
        \begin{equation}
            f_{1}(\omega)=\omega
            \quad\quad
            f_{2}(\omega)=1-\omega
        \end{equation}
        These two functions, while different, will have the same
        cumulative distribution function. For we have:
        \begin{equation}
            F_{1}(u)=\mu_{f_{1}}\big((\minus\infty,u)\big)=
            \mu\big(f^{-1}(\minus\infty,u)\big)
        \end{equation}
        We can evaluate this case by case to get:
        \begin{equation}
            F_{1}(u)=
            \begin{cases}
                \mu(\emptyset)=0,&u\leq{0}\\
                \mu\big([0,u)\big)]u,0<u<1\\
                \mu([0,1])=1,1\leq{u}
            \end{cases}
        \end{equation}
        Looking at $F_{2}$, we have:
        \begin{equation}
            F_{2}(u)=\mu_{f_{2}}\big((\minus\infty,u)\big)
            =\mu\big(f_{2}^{\minus{1}}(\minus\infty,u)\big)
        \end{equation}
        Again, evaluating case by case, we get:
        \begin{equation}
            F_{2}(u)=
            \begin{cases}
                \mu(\emptyset)=0,&u\leq{0}\\
                \mu\big((1-u,1]\big)]u,0<u<1\\
                \mu([0,1])=1,1\leq{u}
            \end{cases}
        \end{equation}
        Thus, $F_{1}=F_{2}$.
    \end{example}
    \begin{ldefinition}{Random Vector}
        A random vector on a probability space
        $(\Omega,\mathcal{A},\mu)$ is an
        $\mathcal{A}-\mathcal{B}_{n}$ measurable function
        $\mathbf{f}:\Omega\rightarrow\mathbb{R}^{n}$, where
        $\mathcal{B}_{n}$ is the Borel $\sigma\textrm{-Algebra}$
        on $\mathbb{R}^{n}$.
    \end{ldefinition}
    As a comment, if $f:\Omega\rightarrow\mathbb{R}$ is
    $\mathcal{A}-\mathcal{B}$ measurable, then
    $\mathcal{A}_{f}\subseteq\mathcal{A}$. The associated
    $\sigma\textrm{-Algebra}$ of a random vector
    $\mathbf{f}:\Omega\rightarrow\mathbb{R}^{n}$ is:
    \begin{equation}
        A_{\mathbf{f}}
        =\{\mathbf{f}^{\minus{1}}(B):B\in\mathcal{B}_{n}\}
    \end{equation}
    \begin{theorem}
        If $(\Omega,\mathcal{A},\mu)$ is a probability space,
        $\mathcal{B}_{n}$ is the Borel $\sigma\textrm{-Algebra}$
        on $\mathbb{R}^{n}$, and if
        $\mathbf{f}:\Omega\rightarrow\mathbb{R}^{n}$ is a random
        vector such that:
        \begin{equation}
            \mathbf{f}(\omega)=(f_{1}(\omega),\dots,f_{n}(\omega))
        \end{equation}
        Then:
        \begin{equation}
            \mathcal{A}_{\mathbf{f}}=
            \sigma\big(
                \mathcal{A}_{f_{1}},\dots,\mathcal{A}_{f_{n}}\big)
        \end{equation}
        Where this is the $\sigma\textrm{-Algebra}$ generated by
        these sets.
    \end{theorem}
    \begin{proof}
        For any $f_{j}$,
        $\mathcal{A}_{f_{j}}\subseteq\mathcal{A}_{\mathbf{f}}$,
        and thus the generated $\sigma\textrm{-Algebra}$ is
        contained in $\mathcal{A}_{\mathbf{f}}$. Going the other
        ways, let $\tilde{\mathcal{B}}$ be the set of subsets
        $B\subseteq\mathbb{R}^{n}$ such that:
        \begin{equation}
            \mathbf{f}^{\minus{1}}(B)\in
            \sigma\big(
                \mathcal{A}_{f_{1}},\dots,\mathcal{A}_{f_{n}}\big)
        \end{equation}
        But then for any sequence $B_{1},\dots,B_{n}\in\mathcal{B}$,
        $B_{1}\times\cdots\times{B}_{n}$ is contained in
        $\tilde{\mathcal{B}}$. But $\mathcal{B}_{n}$ is the
        smallest such $\sigma\textrm{-Algebra}$ to contain such
        sets, and thus
        $\mathcal{B}_{n}\subseteq\tilde{\mathcal{B}}$.
    \end{proof}
    \begin{ldefinition}{Distribution of a Random Vector}
        The distribution of a random vector
        $\mathbf{f}:\Omega\rightarrow\mathbb{R}^{n}$ on a
        measure space $(\Omega,\mathcal{A},\mu)$ is the measure:
        \begin{equation}
            \mu_{\mathbf{f}}(B)=
            \mu(\mathbf{f}^{\minus{1}}(B))
        \end{equation}
        Which is the joint distribution of
        $f_{1},\dots,f_{n}$, where:
        \begin{equation}
            \mathbf{f}(\omega)=(f_{1}(\omega),\dots,f_{n}(\omega))
        \end{equation}
    \end{ldefinition}
    The individual distributions can be computed in terms of the
    joint distribution. This is because:
    \begin{equation}
        \mu_{f_{1}}(B)=
        \mu(f_{1}^{\minus{1}}(B))=
        \mu\big(\mathbf{f}^{\minus{1}}(
            B\times\mathbb{R}^{n-1})\big)=
        \mu_{\mathbf{f}}\big(B\times\mathbb{R}^{n-1}\big)
    \end{equation}
    The joint distribution can not, in general, be computed
    in terms of the individual distributions. There is a special
    exception to this rule, and that is when the random variables
    are independent. That is, if the associated
    $\sigma\textrm{-Algebras}$ are independent. So events that
    bear on $f_{1},\dots,f_{n}$ are independent. If
    $E_{j}\in\mathcal{A}_{f_{j}}$, then:
    \begin{equation}
        \mu\Big(\bigcap_{k=1}^{n}E_{k}\Big)=
        \prod_{k=1}^{n}\mu(E_{k})
    \end{equation}
    \begin{theorem}
        A sequence of random variables $f_{1},\dots,f_{n}$ are
        independent if and only if the joint distribution is
        the product measure of the individual distributions.
    \end{theorem}
    \begin{proof}
        For let $B_{k}\in\mathcal{B}$ and let:
        \begin{equation}
            E_{k}=f_{k}^{\minus{1}}(B_{k})
        \end{equation}
        But then:
        \begin{subequations}
            \begin{align}
                \mu\Big(\bigcap_{k=1}^{n}E_{k}\Big)&=
                \mu\Big(\bigcap_{k=1}^{n}
                    f_{k}^{\minus{1}}(B_{k})\Big)\\
                &=\mu\big(\mathbf{f}^{\minus{1}}
                    (B_{1}\times\dots\times{B}_{n})\big)\\
                &=\mu_{\mathbf{f}}(B_{1}\times\dots\times{B}_{n})\\
                &=\prod_{k=1}^{n}\mu(E_{n})\\
                &=\prod_{k=1}^{n}\mu\big(f^{\minus{1}}(B_{k})\big)\\
                &=\prod_{k=1}^{n}\mu_{f_{k}}(B_{k})
            \end{align}
        \end{subequations}
    \end{proof}
    Let $\mu_{1},\dots,\mu_{n}$ be probability Lebesgue-stieljes
    measures on $\mathbb{R}$, and let $\mu$ be the product
    measure. Consider the probability space
    $(\mathbb{R}^{n},\mathcal{B}_{n},\mu)$ and the projection
    mappings $\pi_{k}:\mathbb{R}^{n}\rightarrow\mathbb{R}$:
    \begin{equation}
        \pi_{k}(\omega_{1},\dots,\omega_{n})=\omega_{k}
    \end{equation}
    \begin{theorem}
        Let $f_{n}$ be an infinite sequence of random variables
        on a probability space $(\Omega,\mathcal{A},\mu)$. Let
        $\mathcal{A}_{f_{n}}$ be the associated
        $\sigma\textrm{-Algebras}$. For every $\omega\in\Omega$,
        let:
        \begin{equation}
            F_{\inf}(\Omega)=
            \underset{n\rightarrow\infty}{\underline{\lim}}
            f_{n}(\omega)
            \quad\quad
            F_{\sup}(\Omega)=
            \underset{n\rightarrow\infty}{\overline{\lim}}
            f_{n}(\omega)
        \end{equation}
        Then $F_{\inf}$ and $F_{\sup}$ are measurable with
        respect to the terminal $\sigma\textrm{-Algebra}$.
    \end{theorem}
    \begin{proof}
        For $F_{\inf}$ is measurable if and only if for all
        $u\in\mathbb{R}$, we have
        $F^{\minus{1}}\big((\minus\infty,u)\big)\in\mathcal{F}$.
        But:
        \begin{subequations}
            \begin{align}
                F^{\minus{1}}\big((\minus\infty,u)\big)
                &=\{\omega:F(\omega)\leq{u}\}\\
                &=\{\omega:\underline{\lim}f_{n}(\omega)\leq{u}\}\\
                &=\{\omega:\underset{n}{\sup}
                    \underset{k\geq{n}}{\lim}f_{k}(\omega)\}\\
                &=\bigcap_{n=1}^{\infty}\Big\{\omega:
                    \underset{n\geq{k}}{\inf}f_{k}(\omega)\leq{u}
                \Big\}\\
                &=\bigcap_{n=N}^{\infty}\Big\{\omega:
                    \underset{n\geq{k}}{\inf}f_{k}(\omega)\leq{u}
                \Big\}
            \end{align}
        \end{subequations}
    \end{proof}
    \begin{theorem}
        If $\mathcal{F}$ is a self-independent
        $\sigma\textrm{-Algebra}$, if $F$ is measurable with
        respect to $\mathcal{F}$, then $F$ is constant almost
        everywhere.
    \end{theorem}
    \begin{proof}
        For since $\mathcal{F}$ is self independent:
        \begin{equation}
            \mu(\{\omega:F(\omega)<u\})=0
            \quad\textrm{or}\quad
            \mu(\{\omega:F(\omega)<u\})=1
        \end{equation}
        Define $A$ and $B$ as follows:
        \begin{align}
            A&=\{u\in\mathbb{F}:\mu(\{\omega:F(\omega)<u\})=0\}\\
            B&=\{u\in\mathbb{F}:\mu(\{\omega:F(\omega)<u\})=1\}\\
        \end{align}
        This separates the real line into two parts. By
        Dedekind's Axiom there is a $c\in\mathbb{R}$ such that,
        for all $a\in{A}$, and for all $b\in{B}$,
        $a\leq{c}\leq{b}$. But then:
        \begin{equation}
            \mu(\{u:F(u)<c+\frac{1}{n}\})=1
        \end{equation}
        From continuity from above, we're done.
    \end{proof}
    \begin{theorem}
        If $(\Omega,\mathcal{A},\mu)$ is a probability space,
        $f_{n}$ is a sequence of independent random variables,
        then the limit inferior and the limit superior are
        constants $\mu$ almost everywhere.
    \end{theorem}
    \begin{proof}
        For the limit inferior and limit superior are measurable
        with respect to the terminal $\sigma\textrm{-Algebra}$.
        By the Kolmogorov zero-one law, $\mathcal{F}$ is
        self-independent if $\mathcal{A}_{f_{n}}$ are independent.
        Thus, by the previous theorem, these functions are constants
        almost everywhere.
    \end{proof}
    Thus the limit of random-variables is entirely not random, but
    constant functions.
    \begin{theorem}
        If $f_{n}$ is a sequence of random variables, then the
        limit of $f_{n}$ almost surely exists, or almost never
        exists.
    \end{theorem}
    \begin{proof}
        For since the limit superior and limit inferior are
        constants almost everywhere, then eithere they agree,
        in which there's convergence almost surely, or they do
        not agree, in which there's convergence almost never.
    \end{proof}
    \begin{ldefinition}{Expectation Value}
        The expectation value of a summable random variable
        $f:\Omega\rightarrow\mathbb{R}$ on a measure space
        $(\Omega,\mathcal{A},\mu)$ is the real number
        $E(f)$ defined by:
        \begin{equation}
            E(f)=\int_{\Omega}f\diff{\mu}
        \end{equation}
    \end{ldefinition}
    The expectation can be expressed in terms of the distribution
    by using the measure transformation theorem. If
    $g:\mathbb{R}\rightarrow\mathbb{R}$ is a real valued function,
    then:
    \begin{equation}
        \int_{\Omega}g\diff{\mu}=
        \int_{\mathbb{R}}g\circ{f}\diff{\mu_{f}}
    \end{equation}
    Now we apply this in the simple case when $g(u)=u$. Then:
    \begin{equation}
        E(f)=\int_{\Omega}f\diff{\mu}
        =\int_\mathbb{R}u\diff{\mu_{f}}
    \end{equation}
    Where we assume that $f$ is summable against $\mu$. Thus,
    $u$ is summable against $\mu_{f}$. So, we have that:
    \begin{equation}
        \int_{\mathbb{R}}|u|\diff{\mu_{f}}<\infty
    \end{equation}
    \begin{ldefinition}{Variance}
        The variance of a random variable
        $f:\Omega\rightarrow\mathbb{R}$ on a measure space
        $(\Omega,\mathcal{A},\mu)$, is the real number
        $Var(f)$ defined by:
        \begin{equation}
            Var(f)=E\big(f-E(f)\big)^{2}
            =\int_{\Omega}\big(f-E(f)\big)^{2}\diff{\mu}
        \end{equation}
    \end{ldefinition}
    \begin{theorem}
        \begin{equation}
            Var(f)=E(f^{2})-E(f)^{2}
        \end{equation}
    \end{theorem}
\section{Lecture 8-ish Maybe}
    If $(\Omega,\mathcal{A},\mu)$ is a measure space,
    $f:\Omega\rightarrow\mathbb{R}$ is a Borel measurable
    function, then the expectation is:
    \begin{equation}
        E(f)=\int_{\Omega}f\diff{\mu}
    \end{equation}
    The functions $f_{1},\dots,f_{n}$ are independent if
    the associated $\sigma\textrm{-Algebras}$ are independent,
    $\mathcal{A}_{f_{1}}.\dots,\mathcal{A}_{f_{n}}$, where
    the associated $\sigma\textrm{-Algebra}$ is defined
    as:
    \begin{equation}
        \mathcal{A}_{f}=\{f^{\minus{1}}(B):B\in\mathcal{B}\}
    \end{equation}
    Where $\mathcal{B}$ is the Borel
    $\sigma\textrm{-Algebra}$. A random vector is a function
    $\mathbf{f}:\Omega\rightarrow\mathbb{R}^{n}$. The
    distribution of $\mathbf{f}$ is defined as:
    \begin{equation}
        \mu_{\mathbf{f}}(B)=
            \mu\big(\mathbf{f}^{\minus{1}}(B)\big)
    \end{equation}
    This is also called the joint distribution. We then proved
    that $f_{1},\dots,f_{n}$ are independent if and only
    if the joint distribution is the product of the
    individual distributions.
    \begin{theorem}
        If $(\Omega,\mathcal{A},\mu)$ is a probabilty space,
        and if $f_{1},\dots,f_{n}$ are independent functions,
        then:
        \begin{equation}
            E\Big(\prod_{k}f_{k}\Big)
            =\prod_{k}E(f_{k})
        \end{equation}
    \end{theorem}
    \begin{proof}
        For define $g:\Omega\rightarrow\mathbb{R}$ by:
        \begin{equation}
            g(\omega)=\prod_{k=1}^{n}f_{k}(\omega)
        \end{equation}
        Let $\mathbf{f}:\Omega\rightarrow\mathbb{R}^{n}$ be
        defined by:
        \begin{equation}
            \mathbf{f}(\omega)=
            \big(f_{1}(\omega),\dots,f_{n}(\omega)\big)
        \end{equation}
        Then using the measure transformation, we have:
        \begin{align}
            \int_{\Omega}\prod_{k=1}^{n}f_{k}\diff{\mu}
            &=\int_{\Omega}g\big(\mathbf{f}(\omega)\big)
                \diff{\mu}\\
            &\int_{\mathbb{R}^{n}}g(u_{1},\dots,u_{n})
                \mu_{\mathbf{f}}\\
            &=\int_{\mathbb{R}^{n}}\prod_{k=1}^{n}u_{k}
                \mu_{\mathbf{f}}
        \end{align}
    \end{proof}
    Suppose $n=2$. Then, since $f_{1}$ and $f_{2}$ are
    independent, $\mu_{(f_{1},f_{2})}$ is the product of
    the measures $\mu_{f_{1}}$ and $\mu_{f_{2}}$. Thus
    by Fubini's theorem:
    \begin{equation}
        \int_{\mathbb{R}^{2}}u_{1}u_{2}\mu_{(f_{1},f_{2})}
        =\int_{\mathbb{R}}\Big(
            \int_{\mathbb{R}}u_{1}u_{2}\mu_{f_{2}}\Big)
                \mu_{f_{1}}
        =\int_{\mathbb{R}}u_{1}\Big(
            \int_{\mathbb{R}}u_{1}\mu_{f_{1}}\Big)
        =\int_{\mathbb{R}}u\mu_{f_{1}}
            \int_{\mathbb{R}}u_{2}\mu_{f_{2}}
        =\int_{\Omega}f_{1}\mu\int_{\Omega}f_{2}\mu
    \end{equation}
    From a course in integral calculus, one should be very
    surprised by this result, for it says that if
    $f_{1},\dots,f_{n}$ are independent, then:
    \begin{equation}
        \int_{\Omega}\prod_{k=1}^{n}f_{k}\diff{\mu}=
        \prod_{k=1}^{n}\int_{\Omega}f_{k}\diff{\mu}
    \end{equation}
    This is almost never true for a given set of functions,
    but if they are independent then the result holds.
    \subsection{Covariance}
        The covariance of $f_{1}$ and $f_{2}$ is:
        \begin{equation}
            E\Big(\big(f_{1}-E(f_{1})\big)
                \big(f_{2}-E(f_{2})\big)\Big)
            =\int_{\Omega}\big(f_{1}-E(f_{1})\big)
                \big(f_{2}-E(f_{2})\big)\diff{\mu}
        \end{equation}
        We can simplify this down to:
        \begin{equation}
            E(f_{1}f_{2})-E(f_{1})E(f_{2})
        \end{equation}
        If $f_{1}$ and $f_{2}$ are independent, then:
        \begin{equation}
            Cov(f_{1},f_{2})=0
        \end{equation}
        The converse is not true. It does not imply that
        $f_{1}$ and $f_{2}$ are independent.
        For let:
        \begin{equation}
            \Omega=\{1,2,3\}
        \end{equation}
        Let $\mathcal{A}=\mathcal{P}(\Omega)$ and let
        $\mu$ be the counting measure on $\Omega$. That is:
        \begin{equation}
            \mu(A)=\frac{\Card(A)}{3}
        \end{equation}
        Then $(\Omega,\mathcal{A},\mu)$ is a probability
        measure. Define $f_{1}$ and $f_{2}$ as follows:
        \begin{align}
            f_{1}(\omega)&=
            \begin{cases}
                1,&\omega=1\\
                0,&\omega=0\\
                1&\omega=2
            \end{cases}\\
            f_{1}(\omega)&=
            \begin{cases}
                1,&\omega=1\\
                0,&\omega=0\\
                \minus{1}&\omega=2
            \end{cases}
        \end{align}
        Then we compute and get:
        \begin{equation}
            E(f_{1})=\int_{\Omega}f_{1}\diff{\mu}=0
        \end{equation}
        And also:
        \begin{equation}
            E(f_{2})=\frac{2}{3}
        \end{equation}
        But if we multiply, we see that
        $f_{1}f_{2}=f_{1}$, and therefore:
        \begin{equation}
            E(f_{1}f_{2})=E(f_{1})=0
        \end{equation}
        But then:
        \begin{equation}
            E(f_{1}f_{2})-E(f_{1})E(f_{2})=0
        \end{equation}
        And thus $f_{1}$ and $f_{2}$ are uncorrelated.
        But they are dependent. We may expect this since
        $f_{2}=f_{1}^{2}$. Let's compute the associated
        $\sigma\textrm{-Algebras}$. We have:
        \begin{align}
            f_{1}^{\minus{1}}(\{1\})
            &=\{1\}\\
            f_{2}^{\minus{1}}&=\{1,3\}
        \end{align}
        But then:
        \begin{align}
            \mu(f_{1}^{\minus{1}}(\{1\})
            &=\frac{1}{3}\\
            \mu(f_{2}^{\minus{1}}(\{1\})&=\frac{2}{3}
        \end{align}
        But the product measure is:
        \begin{equation}
            \mu_{(f_{1},f_{2})}(\{1\})=\frac{1}{3}
        \end{equation}
        And this is not the product of the two measure, and
        therefore it they are not independent.
        If $Cov(f_{1},f_{2})=0$, we say that $f_{1}$ and
        $f_{2}$ are uncorrelated.
        \begin{theorem}
            If $f_{1},\dots,f_{n}$ are random variables
            that are pairwise uncorrelated, then:
            \begin{equation}
                Var(\sum_{k=1}^{n}f_{k})=
                \sum_{k=1}^{n}Var(f_{k})
            \end{equation}
        \end{theorem}
        \begin{proof}
            For:
            \begin{equation}
                \int_{\Omega}
                    \Big(\sum_{k=1}^{n}f_{k}-
                        E(\sum_{k=1}^{n}f_{k}\Big)\diff{\mu}
                =\sum_{i,j}\int_{\Omega}
                    (f_{i}-E(f_{i}))(f_{j}-E(f_{j}))\diff{\mu}
            \end{equation}
            But the $f_{i}$ are pairwise uncorrelated, and
            thus this product is zero if $i\ne{j}$. Thus, we
            get:
            \begin{equation}
                \int_{\Omega}
                    \Big(\sum_{k=1}^{n}f_{k}-
                        E(\sum_{k=1}^{n}f_{k}\Big)\diff{\mu}
                =\sum_{k=1}^{n}Var(f_{k})
            \end{equation}
        \end{proof}
\section{Laws of Large Numbers}
    Consider a fair coin and toss it $n$ times. We would
    expect that, as $n$ gets large, the number of times
    heads occurs and the number of times heads occurs is
    roughly the same. That is:
    \begin{equation}
        \frac{|\textrm{Heads}|-|\textrm{Tails}|}{n^{2}}
        \rightarrow{0}
    \end{equation}
    And also:
    \begin{equation}
        \frac{|\textrm{Heads}|\times|\textrm{Tails}|}{n}
        \rightarrow\frac{1}{2}
    \end{equation}
    We want to build a more rigorous notion from this idea
    and create a mathematical model out of this. We use
    probability spaces as this model. Let
    $(\Omega,\mathcal{A},\mu)$ be a probability space and
    let $f_{j}:\Omega\rightarrow\mathbb{R}$ be random variables
    take on the values $\minus{1}$ and $1$, and such that
    they are independent. Then the associated
    $\sigma\textrm{-Algebra}$ are:
    \begin{equation}
        \mathcal{A}_{f_{j}}=
        \{\emptyset,f^{\minus{1}}(\{\minus{1}\}),
            f^{\minus{1}}(\{1\}),\Omega\}
    \end{equation}
    The measure on the space is such that:
    \begin{equation}
        \mu\Big(f^{\minus{1}}\big(\{\minus{1}\}\big)\Big)=
        \mu\Big(f^{\minus{1}}\big(\{1\}\big)\Big)=
        \frac{1}{2}
    \end{equation}
    Define a new function by:
    \begin{equation}
        F_{N}(\omega)=\frac{1}{N}\sum_{k=1}^{N}f_{k}(\omega)
    \end{equation}
    Then $F_{N}(\omega)$ is the number of times 1 occurs
    minus the number of times -1 occurs, divided by $N$.
    It seems likely that this function should converge to
    zero for large $N$. Recall that there are three different
    types of convergence. We say
    $g_{n}\rightarrow{g}$ almost everywhere if there is a
    set of measure 0 such that $g_{n}\rightarrow{g}$ on the
    complement of this set. We say that
    $g_{n}\rightarrow{g}$ almost uniformly if there is a set
    of arbitrarily small measure $\varepsilon$ such that
    $g_{n}\rightarrow{g}$ uniformly on the complement.
    Finally, $g_{n}\rightarrow{g}$ in measure if for all
    $\delta>0$:
    \begin{equation}
        \mu\Big(\{\omega:|g_{n}(\omega)-g(\omega)|\geq\delta\}
        \Big)\rightarrow{0}
    \end{equation}
    We have seen the almost uniform convergence is the
    strongest and implies the other two. By Egorov, since
    $\mu(\Omega)=1$ in a probability space,
    convergence almost everywhere implies convergence almost
    uniformly. Lastly, convergence in measure implies there
    is a subsequence that converges almost uniformly.
    \begin{ldefinition}{Strong Law of Large Numbers}
        A sequence that obeys the Strong Law of Large Numbers
        in a probability space $(\Omega,\mathcal{A},\mu)$
        is a sequence $f_{n}$ such that:
        \begin{equation}
            \frac{1}{N}\sum_{n=1}^{N}
                \Big[f_{n}(\omega)-E(f_{n})\Big]\rightarrow{0}
        \end{equation}
        $\mu$ almost everywhere.
    \end{ldefinition}
    \begin{ldefinition}{Weak Law of Large Numbers}
        A sequence that obeys the Weak Law of Large Numbers
        in a probability space $(\Omega,\mathcal{A},\mu)$
        is a sequence $f_{n}$ such that:
        \begin{equation}
            \frac{1}{N}\sum_{n=1}^{N}
                \Big[f_{n}(\omega)-E(f_{n})\Big]\rightarrow{0}
        \end{equation}
        Where the convergence is in measure.
    \end{ldefinition}
    \begin{ltheorem}{Khinchin's Weak Law of Large Numbers}
        If $(\Omega,\mathcal{A},\mu)$ is a probability space,
        if $f_{j}$ is a sequence of random variables that
        are pair-wise uncorrelated such that:
        \begin{equation}
            \frac{1}{n^{2}}\sum_{j=1}^{n}Var(f_{k})
            \rightarrow{0}
        \end{equation}
        Then $f_{j}$ obeys the Weak Law of Large Numbers.
    \end{ltheorem}
    \begin{proof}
        For:
        \begin{equation}
            \int_{\Omega}\Big(\frac{1}{n}\sum_{k=1}^{n}\big(
                f_{k}(\omega)-E(f_{k})\big)\Big)^{2}\diff{\mu}
            =\frac{1}{n^{2}}\sum_{k=1}^{n}Var(f_{k})
        \end{equation}
        Let:
        \begin{equation}
            \Omega_{\delta,n}=
            \{\omega:|\frac{1}{n}\sum_{k=1}^{n}
                \big(f_{k}(\omega)-E(f_{k})\big)|\geq\delta\}
        \end{equation}
        But by the Chebyshev inequality, we have:
        \begin{equation}
            \int_{\Omega}\Big(\frac{1}{n}\sum_{k=1}^{n}\big(
                f_{k}(\omega)-E(f_{k})\big)\Big)^{2}\diff{\mu}
            \geq\int_{\Omega_{\delta}}
                \Big(\frac{1}{n}\sum_{k=1}^{n}\big(
                f_{k}(\omega)-E(f_{k})\big)\Big)^{2}\diff{\mu}
            \geq\delta^{2}\int_{\Omega_{\delta}}\diff{\mu}
        \end{equation}
        But then:
        \begin{equation}
            \mu(\Omega_{\delta,n})\leq
            \frac{1}{\delta^{2}}\frac{1}{n}^{2}
            \sum_{j=1}^{n}V(f_{k})
        \end{equation}
        But this last part tends to zero. Therefore, etc.
    \end{proof}
    \begin{lexample}
        If all of the $f_{i}$ have the same distribution, or
        if they are uniformly bounded, then the theorem applies.
        This can be used to show that our model for a fair
        coin toss obeys the weak law of large numbers.
    \end{lexample}
    Suppose $g_{n}(\omega)\rightarrow{g}(\omega)$ almost
    everywhere. Then, for all $\delta>0$ there is an
    $N$ such that, for all $n>N$, we have:
    \begin{equation}
        |g_{n}(\omega)-g(\omega)|<k^{\minus{1}}
    \end{equation}
    For some $k$. Consider the negation of this claim. Then
    there exists $k\in\mathbb{N}$ such that, for all
    $N\in\mathbb{N}$ there is an $n>N$ such that:
    \begin{equation}
        |g_{n}(\omega)-g(\omega)|\geq{k}^{\minus{1}}
    \end{equation}
    COnsider the following set:
    \begin{equation}
        B=\bigcup_{n=1}^{\infty}\bigcap_{N=1}^{\infty}
            \bigcup_{k=N}^{\infty}\big\{\omega:
            |g_{n}(\omega)-g(\omega)|\geq{k}^{\minus{1}}\big\}
    \end{equation}
    This is the set of $\omega$ such that
    $g_{n}(\omega)\not\rightarrow{g}(\omega)$. We wish to show
    that $\mu(B)=0$. This will happen if and only if for all
    $k\in\mathbb{N}$:
    \begin{equation}
        \mu\Big(\bigcap_{N=1}^{\infty}\bigcup_{n=N}^{\infty}
            \big\{\omega:|g_{n}(\omega)-g(\omega)|
            \geq{k}^{\minus{1}}\big\}\Big)=0
    \end{equation}
    Consider a collection of set $A_{n}$ and define:
    \begin{equation}
        \overline{A}=\bigcap_{N=1}^{\infty}
            \bigcup_{n=N}^{\infty}A_{n}
    \end{equation}
    If the $A_{n}$ are independent, then $\overline{A}$ is
    a terminal event, and thus by the Kormogorov zero-one law,
    eighet $\mu(\overline{A})=1$ or $\mu(\overline{A})=0$.
    \begin{theorem}
        If:
        \begin{equation}
            \sum_{n=1}^{\infty}\mu)A_{n})<\infty
        \end{equation}
        Then:
        \begin{equation}
            \mu\Big(\bigcap_{N=1}^{\infty}
                \bigcup_{N=n}^{\infty}A_{n}\Big)=0
        \end{equation}
    \end{theorem}
    \begin{proof}
        For:
        \begin{equation}
            \mu\Big(\bigcap_{N=1}^{\infty}
                \bigcup_{N=n}^{\infty}A_{n}\Big)
            \leq\mu\Big(\bigcup_{N=n}^{\infty}A_{n}\Big)
            \leq\sum_{n=N}^{\infty}\mu(A_{n}
        \end{equation}
        But this sum converges, and thus the tail end can
        be made arbitrarily small.
    \end{proof}
    \begin{ltheorem}{Borel-Cantelli Lemma}
        If $A_{n}$ are pair-wise independent and are such
        that:
        \begin{equation}
            \sum_{k=1}^{\infty}\mu(A_{n})=\infty
        \end{equation}
        Then:
        \begin{equation}
            \mu\Big(\bigcap_{N=1}^{\infty}
                \bigcup_{N=n}^{\infty}A_{n}\Big)=1
        \end{equation}
    \end{ltheorem}
    \begin{proof}
        For if:
        \begin{equation}
            \mu\Big(\bigcup_{N=1}^{\infty}\bigcap_{n=N}^{\infty}
                A_{n}^{C}\big)=0
        \end{equation}
        Then, for all $N$:
        \begin{equation}
            \mu\Big(\bigcap_{n=N}^{\infty}A_{n}^{C}\Big)=0
        \end{equation}
        So it suffices to show that this is true. For let
        $N\in\mathbb{N}$, and define:
        \begin{equation}
            B=\bigcap_{n=N}^{\infty}A_{n}^{C}
        \end{equation}
        Also define:
        \begin{equation}
            B_{M}=\bigcap_{n=N}^{M}A_{n}^{C}
        \end{equation}
        It then follows from continuity from below that:
        \begin{equation}
            \mu(B)=\underset{M\rightarrow\infty}{\lim}\mu(B_{M})
            =\underset{M\rightarrow\infty}{\lim}
                \mu\Big(\bigcap_{n=N}^{M}A_{n}^{C}\Big)
        \end{equation}
        But from independence, we obtain:
        \begin{equation}
            \mu(B)=\underset{M\rightarrow\infty}{\lim}
                \prod_{n=N}^{M}\mu\big(A_{n}^{C}\big)
            =\underset{M\rightarrow\infty}{\lim}
                \prod_{n=N}^{M}\mu\big(1-A_{n}\big)
        \end{equation}
        Using the exponential function, we note that
        $1-x\leq\exp(\minus{x})$, and so:
        \begin{equation}
            \mu(B)\leq               
            \underset{M\rightarrow\infty}{\lim}
                \prod_{n=N}^{M}\exp\big(\minus\mu(A_{n})\big)
            =\underset{M\rightarrow\infty}{\lim}
                \exp\Big(\sum_{n=N}^{M}\mu(A_{n})\Big)=0
        \end{equation}
    \end{proof}
    The independence of the $A_{n}$ is indeed necessary for
    this theorem. For let $A_{n}=A_{0}$, and let
    $\mu(A_{0})=\frac{1}{2}$. Then the sum will indeed
    diverge, but the measure of final set is still
    $\frac{1}{2}$.
    The Borel-Cantelli lemma thus complements the
    Kormogrov Zero-One law by giving the precise criterion for
    when the measure is either one or zero. Given a sequence
    of random events, the terminal event has measure one
    if and only if the sum of the individual measures converges,
    and is equal to one otherwise.
    \begin{ltheorem}{Borel's Strong Law of Large Numbers}
        If $f_{n}$ is a sequence of random variables such that:
        \begin{equation}
            \int_{\Omega}|f_{n}|^{4}\diff{\mu}\leq{M}
        \end{equation}
        For all $n\in\mathbb{N}$, then $f_{n}$ obeys the
        strong law of large numbers.
    \end{ltheorem}
    \begin{proof}
        It suffices to show that, for all $\varepsilon>0$:
        \begin{equation}
            \mu\Big(\bigcap_{N=1}^{\infty}
                \bigcup_{N=n}^{\infty}
                \big\{\omega:|\frac{1}{n}\sum_{k=1}^{n}
                    f_{k}(\omega)|\geq\varepsilon\Big)=0
        \end{equation}
        Denote the sequence of centered random variables by:
        \begin{equation}
            \overset{\circ}{f}_{n}(\omega)=
            f_{n}(\omega)=E(f_{n}(\omega))
        \end{equation}
        To show this, we need to show that:
        \begin{equation}
            \sum_{n=1}^{\infty}\mu\Big(
                \big\{\omega:|\frac{1}{n}\sum_{k=1}^{n}
                    \overset{\circ}{f}_{n}(\omega)|
                    \geq\varepsilon\big\}\Big)<\infty
        \end{equation}
        Define:
        \begin{equation}
            \Omega_{n,\varepsilon}=
            \Big\{\omega:\big|\frac{1}{n}\sum_{k=1}^{n}
                \overset{\circ}{f}_{n}(\omega)\big|
                \geq\varepsilon\Big\}
        \end{equation}
        But then:
        \begin{equation}
            \int_{\Omega}\big|\frac{1}{n}\sum_{k=1}^{n}
                \overset{\circ}{f}_{n}\big|^{4}\diff{\mu}
            \geq\int_{\Omega_{n,\varepsilon}}
            \big|\frac{1}{n}\sum_{k=1}^{n}
                \overset{\circ}{f}_{n}\big|^{4}\diff{\mu}
            \geq\varepsilon^{4}
                \mu\big(\Omega_{\varepsilon,n}\big)
        \end{equation}
        Combining this together, we have:
        \begin{equation}
            \mu\big(\Omega_{\varepsilon,n}\big)
            \leq\frac{1}{\varepsilon^{4}}\frac{1}{n^{4}}
            \int_{\Omega}\Big(\sum_{k=1}^{n}
            \overset{\circ}{f}_{k}(\omega)\Big)^{4}\diff{\mu}
            =\frac{1}{\varepsilon^{4}}\frac{1}{n^{4}}
            \sum_{i,j,k,\ell}\int_{\Omega}
            \overset{\circ}{f}_{i}\overset{\circ}{f}_{j}
            \overset{\circ}{f}_{k}\overset{\circ}{f}_{\ell}
            \diff{\mu}
        \end{equation}
        But the $f_{n}$ are independent, and thus the
        $\overset{\circ}{f}_{n}$ are independent. But then
        $\mathcal{A}_{\overset{\circ}{f}_{n}}$ are independent,
        and thus $\overset{\circ}{f_{i}}$ is independent
        from the product
        $\overset{\circ}{f}_{j}\overset{\circ}{f}_{k}\overset{\circ}{f}_{\ell}$. But if they are independent, then:
        \begin{equation}
            \int_{\Omega}
            \overset{\circ}{f}_{i}\overset{\circ}{f}_{j}
            \overset{\circ}{f}_{k}\overset{\circ}{f}_{\ell}
            \diff{\mu}=
            \int_{\Omega}
            \overset{\circ}{f}_{i}\diff{\mu}
            \int_{\Omega}\overset{\circ}{f}_{j}
            \overset{\circ}{f}_{k}\overset{\circ}{f}_{\ell}
            \diff{\mu}=0
        \end{equation}
        There are two cases left, when the indices are equal
        in pairs, and when all of the indices are equal. In
        the cases where all are equal, we have:
        \begin{equation}
            \sum_{i=1}^{n}
            \int_{\Omega}
            |\overset{\circ}{f}_{i}|^{4}\diff{\mu}
            \leq{M}n
        \end{equation}
        For the case of pairs, we have $n^{2}-n$ possibilities,
        and thus:
        \begin{equation}
            \sum_{i,j}\int_{\Omega}
            |\overset{\circ}{f}_{i}^{2}
            \overset{\circ}{f}_{j}^{2}|\diff{\mu}
            \leq{M}(n^{2}-n)
        \end{equation}
        Therefore, we have:
        \begin{equation}
            \frac{1}{\varepsilon^{4}}\frac{1}{n^{4}}
            \sum_{i,j,k,\ell}\int_{\Omega}
            \overset{\circ}{f}_{i}\overset{\circ}{f}_{j}
            \overset{\circ}{f}_{k}\overset{\circ}{f}_{\ell}
            \diff{\mu}
            \leq\frac{M}{\varepsilon^{4}}
            \frac{1}{n^{2}}
        \end{equation}
        But:
        \begin{equation}
            \sum_{n=1}^{\infty}\frac{1}{n^{2}}
            =\frac{\pi^{2}}{6}<\infty
        \end{equation}
        Thus, the measure is zero.
    \end{proof}
    The $|f_{n}|^{4}$ are called the fourth moments of the
    $f_{n}$. There are sequences that obey the weak law but
    not the strong law. Borel's theorem shows that uniformly
    bounded sequences of random variables automatically obey
    the strong law of strong numbers, since a uniformly
    bounded sequence will have uniformly bounded fourth
    moments. To find a sequence that obeys the weak law but
    not the strong law, we will need to consider sequences
    that take on arbitrarily large values.
    \begin{lexample}
        Let $f_{n}$ be a sequence of random variables such that
        the following are true:
        \begin{subequations}
            \begin{align}
                \mu\Big(\{\omega:f_{n}(\omega)=n\}\Big)
                &=\frac{P_{n}}{2}\\
                \mu\Big(\{\omega:f_{n}(\omega)=\minus{n}\}\Big)
                &=\frac{P_{n}}{2}\\
                \mu\Big(\{\omega:f_{n}(\omega)=0\}\Big)
                &=1-P_{n}
            \end{align}
        \end{subequations}
        We need to find a sequence $P_{n}$ such that the
        $f_{n}$ will obey the weak law but not the strong law.
        Choosing the $P_{n}$ to be small will most likely
        result in the sequence obeying the strong law. Indeed,
        if $P_{n}=0$, then the $f_{n}$ will obey the strong
        law. In fact, if:
        \begin{equation}
            P_{n}\leq\frac{1}{n^{2}}
        \end{equation}
        Then $f_{n}$ will obey the strong law. This is a
        consequence of the Borel-Cantelli lemma. If $P_{n}$
        is to large, it may not be true that the $f_{n}$ obeys
        the weak law. For example, suppose:
        \begin{equation}
            P_{n}=\frac{1}{n}
        \end{equation}
        We cannot apply Khinchin's theorem, since:
        \begin{equation}
            \frac{1}{n^{2}}\sum_{j=1}^{n}V(f_{j})=
            \frac{n(n+1)}{2n^{2}}
        \end{equation}
        And this does not converge to zero. Let:
        \begin{equation}
            P_{n}=\frac{1}{n\ln(n+2)}
        \end{equation}
        Let's now show that $f_{n}$ will obey the weak law.
        It does. But it does not obey the strong law. We will
        need to use the Borel-Cantelli lemma. But the sum:
        \begin{equation}
            \sum_{n=1}^{\infty}\frac{1}{n\ln(n+2)}=\infty
        \end{equation}
        Therefore:
        \begin{equation}
            \mu\Big(\bigcap_{N=1}^{\infty}\bigcup_{n=N}^{\infty}
                \{\omega:|f_{n}(\omega)|\}\Big)=1
        \end{equation}
        This contradicts the strong law of large numbers. For
        suppose not. Then, for almost every $\omega$, and for
        all $N$, there is an $N>N$ such that
        $|f_{n}(\omega)|=n$. Then thre exists a sequence
        $n_{k}$ such that $|f_{n_{k}}(\omega)|=n_{k}$.
        But:
        \begin{equation}
            \frac{1}{n_{k}}\sum_{j=0}^{n_{k}}f_{j}\rightarrow{0}
        \end{equation}
        And therefore:
        \begin{equation}
            \frac{1}{n_{k}-1}\sum_{j=0}^{n_{k}}f_{j}
                \rightarrow{0}
        \end{equation}
        Taking the difference, we get:
        \begin{equation}
            \frac{1}{n_{k}}f_{n_{k}}(\omega)\rightarrow{0}
        \end{equation}
        But $|f_{n_{k}}(\omega)|=n_{k}$, a contradiction.
        So the $f_{n}$ do not obey the strong law.
    \end{lexample}
    \subsection{Borel Numbers}
        Let $0\leq{x}\leq{1}$ and suppose $x$ has the
        representation $x=0.x_{1}x_{2}\dots$ and exclude
        numbers with two representations. For example,
        $1=0.999\dots$. The measure of the set of these numbers
        is zero. Let $0\leq{a}\leq{9}$. Let $C_{n}(x)$ be
        the number of $a$ among the first $n$ digits. Then:
        \begin{equation}
            \frac{C_{n}(x)}{n}\rightarrow\frac{1}{10}
        \end{equation}
        For almost every $x$. Let $\Omega=[0,1]$ and
        $\mathcal{B}$ be the Borel $\sigma\textrm{-Algebra}$.
        Also, let $\mu$ be the Lebesgue measure. Consider
        the functions:
        \begin{equation}
            f_{j}(x)=
            \begin{cases}
                1,&x_{j}=a\\
                0,&x_{j}\ne{a}
            \end{cases}
        \end{equation}
        Then:
        \begin{equation}
            \frac{C_{n}(x)}{n}=\frac{1}{n}\sum_{k=1}^{n}
                f_{k}(x)
        \end{equation}
        If the $f_{k}$ obey the strong law of large numbers,
        then:
        \begin{equation}
            \frac{1}{n}\sum_{k=1}^{n}\big(f_{k}-E(f_{k})\big)
            \rightarrow{0}
        \end{equation}
        $\mu$ almost everywhere. We have that $f_{j}$ are
        bounded, and thus it suffices to show that they are
        also independent. Define:
        \begin{equation}
            \mathcal{A}_{f_{i}}=
            \{\emptyset,A_{j},A_{j}^{C},\Omega\}
        \end{equation}
        Where:
        \begin{equation}
            A_{j}=\{x:f_{j}(x)=1\}
        \end{equation}
        THe $A_{j}$ are the set of elements $x$ such that
        $x=0.x_{1}x_{2}\dots{a}x_{j+1}x_{j+2}\dots$ Using this
        there are $10^{j-1}$ options for the first $j-1$
        digits. This set is covered by $10^{j-1}$ intervals,
        each of length $10^{\minus{j}}$. Thus, the Lebesgue
        measure of $A_{j}$ is $\frac{1}{10}$. We now need to
        show that, for distinct $j,k$, that the measure of the
        intersection is $\frac{1}{100}$. Suppose $j<k$. Then
        $A_{j}\cap{A}_{k}$ is the set of numbers with $a$ in
        the $j^{th}$ decimal and $a$ in the $k^{th}$ decimal.
        There are $10^{j-1}$ ways to choose the first
        $j-1$ digits, and $10^{k-j-1}$ ways to choose the
        next $k-j-1$ digits. Total, there are
        $10^{k-2}$ digits to choose. So we can cover this
        set with $10^{k-2}$ intervals, each of lenght
        $10^{\minus{k}}$. Thus, the measure of the intersection
        is $10^{\minus{2}}$. Theefore, the $f_{j}$ are
        independent. By Borel's Strong Law of Large Numbers,
        the $f_{j}$ obey the strong law of large numbers.
        \par\hfill\par
        Let $\Omega=\mathbb{Z}_{n}$, let $\mathcal{A}$ be
        the power set, and let $\mu$ be the counting
        measure on $\Omega$. Taking the product
        $\Omega^{n}$, and considering the product measure,
        we see that every point has measure $10^{\minus{n}}$.
        Thus, if we consider the infinite product, points will
        have measure zero. This isn't too strange since the
        Lebesgue measure is such that points have measure
        zero. Let $\tilde{\Omega}$ be the infinite product
        and let $B_{n}\in\Omega^{n}$. Then
        $B_{n}\times\Omega_{n+1}\times\dots$ is contained in
        $\tilde{\Omega}$. Let $\tilde{\mathcal{B}}$ be the
        smallest $\sigma\textrm{-Algebra}$ on the product
        space that contains all of these types of sets, and let
        $\tilde{\mu}$ be the extension measure. Then
        $f_{j}(\omega)=\omega_{j}$ are independent by
        construction of the product measure, and also:
        \begin{equation}
            \mu(f_{j}=a)=\frac{1}{10}
        \end{equation}
        There is a map $\tilde{\Omega}\mapsto[0,1]$ by sending
        $(\omega_{1},\dots)$ to $0.\omega_{1}\omega_{2}\dots$.
        The image measure of the product measure $\tilde{\mu}$
        is the Lebesgue measure. So we have an equivalent
        model of $[0,1]$ with the Lebesgue measure.
\section{Central Limit Theorem}
    We now wish to discuss convergence of measures and
    distributions. We restrict ourself to
    Lebesgue-Stieljes measures on the Borel
    $\sigma\textrm{-Algebra}$ of $\mathbb{R}$. We does it
    mean for a sequence of measures $\mu_{n}$ to converge
    to a measure $\mu$? For all $B\in\mathcal{B}$:
    \begin{equation}
        \mu_{n}(B)\rightarrow\mu(B)
    \end{equation}
    This is reminiscent of point-wise convergence of functions
    of a real variable, but turns out to be too much. What if
    we restrict ourselves to sets of the form $[a,b)$? Let:
    \begin{equation}
        f_{n}(x)=
            \begin{cases}
                0,&|x|\geq\frac{1}{n}\\
                n(1-|x|),&|x|<\frac{1}{n}
            \end{cases}
    \end{equation}
    And define:
    \begin{equation}
        \mu_{n}([a,b)]=\int_{a}^{b}\rho_{n}(x)\diff{x}
    \end{equation}
    Then by the Caratheodory extension theorem, there is a
    measure $\nu_{n}$ that agrees with $\mu_{n}$ on all
    such intervals. Then $\nu_{n}$ converges to the
    Dirac measure, which is an example of an atomic measure:
    \begin{equation}
        \delta(B)=
            \begin{cases}
                1,&0\in{B}\\
                0,&0\notin{B}
            \end{cases}
    \end{equation}
    However:
    \begin{equation}
        \mu_{n}([0,b)]\rightarrow\frac{1}{2}
    \end{equation}
    And:
    \begin{equation}
        \mu_{n}([a,0)]\rightarrow\frac{1}{2}
    \end{equation}
    However:
    \begin{align}
        \delta([0,b))&=1\\
        \delta([a,0))&=0
    \end{align}
    This leads us to the correct definition of measure:
    \begin{ldefinition}{Convergence of Measure}
        A sequence of measure $\nu_{n}$ converges to a
        measure $\nu$ if, for all measure sets $B$ such
        that $\nu(\partial{B})=0$, it is true that
        $\nu_{n}(B)\rightarrow\nu(B)$.
    \end{ldefinition}
    Given:
    \begin{equation}
        \int_{\mathbb{R}}\chi_{[a,b)}\diff{\nu_{n}}
        \rightarrow\int_{\mathbb{R}}\chi_{[a,b)}\diff{\nu}
    \end{equation}
    We have that $\nu(\{a\})=\nu(\{b\})=0$, and thus
    $\chi_{[a,b)}$ is continuous $\nu$ almost everywhere.
    Suppose $\nu_{n}$ and $\nu$ are probability
    Lebesgue-Stieltjes measure on $\mathbb{R}$. Then we
    get the equivalent form:
    \begin{theorem}
        If for every continuous bounded function $g(\omega)$,
        we have that:
        \begin{equation}
            \int_{\mathbb{R}}g\diff{\mu}_{n}\rightarrow
            \int_{\mathbb{R}}g\diff{\mu}
        \end{equation}
        Then $\mu_{n}\rightarrow\mu$.
    \end{theorem}
    \begin{theorem}
        If for every continuous function with bounded support,
        if:
        \begin{equation}
            \int_{\mathbb{R}}g\diff{\mu}_{n}\rightarrow
            \int_{\mathbb{R}}g\diff{\mu}
        \end{equation}
        Then $\nu_{n}\rightarrow\nu$.
    \end{theorem}
    \begin{theorem}
        If:
        \begin{equation}
            \int_{\mathbb{R}}\exp(itu)\diff{\nu_{n}}\rightarrow
            \int_{\mathbb{R}}\exp(itu)\diff{\nu}
        \end{equation}
        Then $\nu_{n}\rightarrow\nu$.
    \end{theorem}
    Suppose $(\Omega,\mathcal{A},\mu)$ is a probability space,
    and suppose $f_{n}:\Omega\rightarrow\mathbb{R}$ is a
    sequence of random variables that are
    $\mathcal{A}-\mathcal{B}$ measure. Consider the
    distributions $\mu_{g_{n}}$. If $g_{n}\rightarrow{g}$ in
    measure, then the distribuctions converge to $\mu_{g}$.
    \begin{theorem}
        If $(\Omega,\mathcal{A},\mu)$ is a probability space,
        if $h_{n}:\Omega\rightarrow\mathbb{R}$ is a
        sequence of random variables, if $\mu_{h_{n}}$ are the
        distributions of $g_{n}$, and if $h_{n}\rightarrow{h}$
        in measure, then $\mu_{h_{n}}\rightarrow\mu_{h}$.
    \end{theorem}
    \begin{proof}
        For let $g$ be a continuous function with compact
        support. Then, applying the measure transformation
        theorem, we have:
        \begin{equation}
            \Big|\int_{\mathbb{R}}g\diff{\mu_{n}}-
                \int_{\mathbb{R}}\diff{\mu_{h}}\Big|
            =\Big|\int_{\Omega}g(h_{n})\diff{\mu}-
                \int_{\Omega}g(h)\diff{\mu}\Big|
            \leq\int_{\Omega}|g_{n}(h)-g(h)|\diff{\mu}
        \end{equation}
        But $g$ is continuous on a compact set, and is
        therefore uniformly continuous. Thus, for all
        $\varepsilon>0$ there is a $\delta>0$ such that, for
        all $|u'-u''|<\delta$, we have that
        $|g(u')-g(u'')|<\varepsilon$. Define the following:
        \begin{align}
            E_{1,n,\varepsilon}
            &=\{\omega:|h_{n}(\omega)-h(\omega)|\geq\delta\}\\
            E_{2,n,\varepsilon}
            &=\{\omega:|h_{n}(\omega)-h(\omega)|<\delta\}
        \end{align}
        Then:
        \begin{align}
            \int_{\Omega}|g_{n}(h)-g(h)|\diff{\mu}
            &=\int_{E_{1,n,\varepsilon}}|g_{n}(h)-g(h)|\diff{\mu}
            +\int_{E_{2,n,\varepsilon}}
                |g_{n}(h)-g(h)|\diff{\mu}\\
            &\leq{2}M\mu(E_{1,n,\varepsilon})+
                \varepsilon\mu(E_{2,n,\varepsilon})
        \end{align}
        And this converges to $\varepsilon$.
    \end{proof}
    The converse of this theorem is not true in general,
    since vastly different functions can have the same
    distributions. There is a special case, however, where the
    converse holds. Consider a function $h$ such that it's
    distribution is the Dirac distribution. That is:
    \begin{equation}
        \mu(\{\omega:h(\omega)=a\})=
        \mu_{h}(\{a\})=\delta_{a}(\{a\})=1
    \end{equation}
    Then $h(\omega)=a$ $\mu$ almost everywhere, or if we are
    in a probabilty space, almost surely.
    \begin{theorem}
        If $h_{n}$ is a sequence of random variables such that
        $\mu_{h_{n}}\rightarrow\delta_{a}$, where $\delta_{a}$
        is the Diract measure centered at $a$, then
        $h_{n}\rightarrow{a}$ almost surely.
    \end{theorem}
    \begin{proof}
        For:
        \begin{align}
            \mu(\{\omega:|h_{n}(\omega)-a|\geq\delta\})
            &=\mu_{h_{n}}
                (\mathbb{R}\setminus(a-\delta,a+\delta)\})
            &=1-\mu_{h_{n}}((a-\delta,a+\delta))\\
            &\rightarrow{1}-\delta_{a}((a-\delta,a+\delta))\\
            &=0
        \end{align}
    \end{proof}
    Thus, the weak law of large numbers can be restated by
    saying that, if:
    \begin{equation}
        \mu_{\frac{1}{n}\sum_{j=1}^{n}f_{j}}\rightarrow
        \delta_{0}
    \end{equation}
    Then $f_{j}$ obeys the weak law of large numbers.
    \subsection{Convergence of Distributions}
        A distribution is an arbitrary probability
        Lebesgue-Stieljes measure. That is, a Lebesgue-Stieljes
        measure such that the measure of the entire space is
        one. We say that a sequence of distribuctions
        $\nu_{n}$ converges to a measure $\nu$ if any of
        the following equivalent statements holds:
        \begin{enumerate}
            \item $\nu_{n}([a,b))\rightarrow\nu([a,b))$
                  for all $a<b$.
            \item $\nu_{n}((\minus\infty,c))\rightarrow%
                   \nu(\minus\infty,c))$ for all $c$ such
                   that $\nu(\{c\})=0$. This requirement
                   implies that $\nu$ is continuous at $c$.
                   That is, if $F_{\nu}$ is the cumulative
                   distribution function, then $F_{\nu}$ is
                   continuous at $c$.
            \item For every bounded continuous function $h$,
                  $\int_{\mathbb{R}}h\diff\nu_{n}\rightarrow%
                   \int_{\mathbb{R}}h\diff{\nu}$.
            \item For every continuous function with compact
                  support:
                  $\int_{\mathbb{R}}h\diff\nu_{n}\rightarrow%
                   \int_{\mathbb{R}}h\diff{\nu}$.
            \item $\int_{\mathbb{R}}\exp(itu)\diff{\nu_{n}}%
                   =\int_{\mathbb{R}}\exp(itu)\diff{\nu}$
        \end{enumerate}
        \begin{theorem}
            A sequence of random variables $f_{j}$ obeys
            the weak law of large numbers if and only if:
            \begin{equation}
                \mu_{\frac{1}{n}\sum_{j=1}^{n}f_{j}}
                \rightarrow\delta_{0}
            \end{equation}
        \end{theorem}
        \begin{proof}
            For:
            \begin{equation}
                \mu\Big(\big\{\omega:\Big|\frac{1}{n}
                    \sum_{j=1}^{n}\overset{\circ}{f}_{k}(\omega)
                    \Big|\geq\delta\big\}\Big)=
                \mu_{\frac{1}{n}\sum_{k=1}^{n}
                     \overset{\circ}{f}_{j}}
                     \big((\minus\delta,\delta)^{C}\big)
            \end{equation}
        \end{proof}
        \begin{theorem}
            If $f_{j}$ is a sequence of random variables such
            that the second moments are finite, then the
            first moments are finite.
        \end{theorem}
        \begin{proof}
            For:
            \begin{equation}
                \int_{\Omega}|f_{j}|\diff{\mu}\leq
                \int_{\Omega}(1+|f_{j}|^{2})\diff{\mu}
                =\int_{\Omega}\diff{\mu}+
                \int_{\Omega}|f_{j}|^{2}\diff{\mu}=
                1+\int_{\Omega}|f_{j}|^{2}\diff{\mu}
            \end{equation}
            Therefore, etc.
        \end{proof}
        \begin{ftheorem}{Central Limit Theorem}
              {Measure_Theory_Central_Limit_Theorem}
            If $f_{j}$ are independent and identically
            distributed, with standard deviation $\sigma$,
            then:
            \begin{equation}
                \mu_{\frac{1}{\sigma\sqrt{n}}
                    \sum_{j=1}^{n}\overset{\circ}{f}_{j}}
                \rightarrow\nu_{0,1}
            \end{equation}
            Where $\nu_{0,1}$ is the Gaussian distribution:
            \begin{equation}
                \nu_{0,1}(B)=\frac{1}{\sqrt{2\pi}}
                \int_{B}\exp(\minus{u}^{2}/2)\diff{u}
            \end{equation}
        \end{ftheorem}
        \begin{proof}
            We will use the Fourier transform to prove this.
            We have:
            \begin{equation}
                \int_{\mathbb{R}}\exp(iut)
                    \diff{\nu_{0,1}}=
                \int_{\mathbb{R}}\exp(itu)
                    \exp(\minus\frac{u^{2}}{2})\diff{u}
                    =\exp(\minus{t}^{2}/2)
            \end{equation}
            That is, the Fourier transform of a Gaussian
            is itself. We will use this to make the
            computation easier. Using the measure
            transformation theorem, we have:
            \begin{equation}
                \int_{\mathbb{R}}\exp(iut)
                \mu_{\frac{1}{\sigma\sqrt{n}}
                    \sum_{j=1}^{n}\overset{\circ}{f}_{j}}
                    \diff{\mu}
                =\int_{\Omega}\exp\Big(
                    \frac{i}{\sigma\sqrt{n}}\sum_{j=1}^{n}
                    \overset{\circ}{f}_{j}(\omega)t\Big)
                    \diff{\mu}
            \end{equation}
            We invoke independence to get:
            \begin{subequations}
                \begin{align}
                    \int_{\Omega}\exp\Big(
                        \frac{i}{\sigma\sqrt{n}}\sum_{j=1}^{n}
                        \overset{\circ}{f}_{j}(\omega)t\Big)
                        \diff{\mu}
                    &=\int_{\Omega}\prod_{j=1}^{n}
                        \exp\Big(\frac{i}{\sigma\sqrt{n}}
                        \overset{\circ}{f}_{j}\Big)\diff{\mu}\\
                    &=\prod_{j=1}^{n}\int_{\Omega}
                        \exp\Big(\frac{i}{\sigma\sqrt{n}}
                        \overset{\circ}{f}_{j}\Big)\diff{\mu}
                \end{align}
            \end{subequations}
            But the distributions are identically distributed,
            and thus we have:
            \begin{equation}
                \int_{\Omega}\exp\Big(
                    \frac{i}{\sigma\sqrt{n}}\sum_{j=1}^{n}
                    \overset{\circ}{f}_{j}(\omega)t\Big)
                    \diff{\mu}=
                \Big[\int_{\Omega}\exp\Big(
                    iu\frac{t}{\sqrt{n}}\Big)
                    \diff{\mu}\Big]^{n}
            \end{equation}
            We now need to prove that for an arbitrary
            Lebesgue-Stieltjes Measure on the Borel
            $\sigma\textrm{-Algebra}$ of $\mathbb{R}$,
            such that:
            \begin{equation}
                \int_{\mathbb{R}}\diff{\mu}=0\quad\quad
                \int_{\mathbb{R}}u\diff{\mu}=0\quad\quad
                \int_{\mathbb{R}}u^{2}\diff{\mu}=1
            \end{equation}
            Then:
            \begin{equation}
                \Big[\int_{\mathbb{R}}\exp\Big(
                    iu\frac{t}{\sqrt{n}}\Big)\diff{\mu}\Big]^{n}
                \rightarrow\exp\big(\minus{t}^{2}/2\big)
            \end{equation}
            Consider the function:
            \begin{equation}
                \varphi_{\nu}(t)=
                \int_{\mathbb{R}}\exp(iut)\diff{\nu}
            \end{equation}
            In analysis this is the Fourier transform,
            whereas in probability this is called the
            characteristic function of $\nu$. We are
            tasked with showing that:
            \begin{equation}
                \Big[\varphi_{\mu}
                    \big(\frac{t}{\sqrt{n}}\big)\Big]^{n}
                \rightarrow\exp\big(\minus{t}^{2}/2\big)
            \end{equation}
            If $\mu$ is a Lebesgue-Stieltjes measure, and
            if the second moment if finite, and if:
            \begin{equation}
                \varphi_{\nu}(t)=
                \int_{\mathbb{R}}\exp(itu)\diff{\nu}
            \end{equation}
            then the first two derivatives of $\varphi_{\nu}$
            exist and are continuous. Moreover:
            \begin{equation}
                \varphi_{\nu}(t)=
                \varphi_{\nu}(0)+
                \varphi_{\nu}'(0)t+
                \varphi_{\nu}''(0)t^{2}+h(t)
            \end{equation}
            Where $h$ is such that:
            \begin{equation}
                \underset{t\rightarrow{0}}{\lim}
                \frac{h(t)}{t^{2}}=0
            \end{equation}
            First, it is continuous. For let $t_{k}$ be
            sequence such that $t_{k}\rightarrow{t}$ and let
            $g_{k}=\exp(it_{k}u)$. Then $|g_{k}|=1$, and is
            therefore summable. Moreover, $g_{k}$ tends to
            $\exp(itu)$. Thus, by the dominated convergence
            theorem:
            \begin{equation}
                \underset{n\rightarrow\infty}{\lim}
                \varphi_{\nu}(t_{k})
                =\underset{n\rightarrow\infty}{\lim}
                \int_{\mathbb{R}}\exp(it_{k}u)\diff{\mu}
                =\int_{\mathbb{R}}
                    \underset{n\rightarrow\infty}{\lim}
                    \exp(it_{k}u)\diff{\mu}
                =\varphi_{\nu}(t)
            \end{equation}
            And thus we have continuity. For differentiability,
            suppose $\Delta{t}_{k}$ is a sequence that
            tends to zero, and consider:
            \begin{equation}
                \frac{\varphi_{\nu}(t+\Delta{t}_{k})-
                      \varphi_{\nu}(t)}{\Delta{t}_{k}}=
                \int_{\mathbb{R}}
                \frac{\exp(iu\Delta{t}_{k})-1}{\Delta{t}_{k}}
                \exp(iut)\diff{\mu}
            \end{equation}
            Again, we want to apply the dominated convergence
            theorem. Thus we need to find a summable
            majorant. Consider $f(s)=(\exp(s)-1)/s$. On the
            real axis, this function has finite limit at
            zero and has zero limit at infinity, and therefore
            $f(s)$ is bounded on the real axis by some $K$.
            Thus, $K\exp(iut)$ serves as a summable majorant.
            Applying the dominated convergence theorem shows
            that the limit exists, and thus
            $\varphi_{\nu}$ is differentiable. We obtain:
            \begin{equation}
                \varphi_{\nu}'(t)=
                \int_{\mathbb{R}}iu\exp(iut)\diff{\nu}
            \end{equation}
            Moreover, this is differentiable and:
            \begin{equation}
                \varphi_{\nu}''(t)=
                \minus\int_{\mathbb{R}}u^{2}\exp(iut)
                    \diff{\mu}
            \end{equation}
            From Taylor, we have:
            \begin{equation}
                h(t)=\varphi_{\nu}(t)-\varphi_{\nu}(0)
                    -\varphi_{\nu}'(0)t-\varphi_{\nu}''(0)
                        \frac{t^{2}}{2}
            \end{equation}
            Thus $h''(t)$ exists and is continuous,
            $h(0)=0$, $h'(0)=0$, and $h''(0)=0$. By the
            mean value theorem, we have:
            \begin{equation}
                h(t)=h'(t_{1})t
            \end{equation}
            For some $t_{1}\in(0,t)$. Moreover:
            \begin{equation}
                h(t)=h''(t_{2})t^{2}
            \end{equation}
            Where $0<t_{1}<t_{2}<t$. Thus:
            \begin{equation}
                \frac{h(t)}{t^{2}}=h''(t_{2})
            \end{equation}
            And from the continuity of $h''(t)$, this
            converges to zero as $t$ tends to zero. Thus
            we have that $\varphi_{\nu}(0)=1$,
            $\varphi_{\nu}'(0)=0$, and
            $\varphi_{\nu}''(0)=\minus{1}$. Now we need to 
            finally justify the following limit:
            \begin{equation}
                \Big[\varphi_{\mu}\big(\frac{t}{\sqrt{n}}\big)
                    \Big]^{n}\rightarrow\exp(\minus{t}^{2}/2)
            \end{equation}
            We have:
            \begin{equation}
                \varphi_{\nu}(t)=1-\frac{t^{2}}{2}+h(t)
            \end{equation}
            Where $h(t)/t^{2}\rightarrow{0}$ as
            $t\rightarrow{0}$. Thus:
            \begin{equation}
                \Big[\varphi_{\nu}\big(\frac{t}{\sqrt{n}}
                    \big)\Big]^{n}=
                \Big[1-\frac{t^{2}}{2n}+h(\frac{t}{\sqrt{n}})
                    \Big]^{n}
            \end{equation}
            Define:
            \begin{equation}
                w_{n}(t)=h(t/\sqrt{n})-\frac{t^{2}}{2n}
            \end{equation}
            Then we have:
            \begin{equation}
                \Big[\varphi_{\nu}\big(\frac{t}{\sqrt{n}}
                    \big)\Big]^{n}
                =\Big(\Big[1+w_{n}(t)\Big]^{w_{n}(t)}
                    \Big)^{\frac{n}{w_{n}(t)}}
            \end{equation}
            The inner part is the definition of $e$, so we
            now need to show that $n/w_{n}(t)$ converges to
            $\minus{t}^{2}/2$.
        \end{proof}