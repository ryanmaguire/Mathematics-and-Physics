%------------------------------------------------------------------------------%
\documentclass[crop=false,class=article]{standalone}                           %
%----------------------------Preamble------------------------------------------%
%---------------------------Packages----------------------------%
\usepackage{geometry}
\geometry{b5paper, margin=1.0in}
\usepackage[T1]{fontenc}
\usepackage{graphicx, float}            % Graphics/Images.
\usepackage{natbib}                     % For bibliographies.
\bibliographystyle{agsm}                % Bibliography style.
\usepackage[french, english]{babel}     % Language typesetting.
\usepackage[dvipsnames]{xcolor}         % Color names.
\usepackage{listings, lstlinebgrd}      % Verbatim-Like Tools.
\usepackage{mathtools, esint, mathrsfs} % amsmath and integrals.
\usepackage{amsthm, amsfonts}           % Fonts and theorems.
\usepackage{tabularx}
\usepackage{tcolorbox}                  % Frames around theorems.
\usepackage{upgreek}                    % Non-Italic Greek.
\usepackage{paracol}                    % Two-column styling.
\usepackage{wrapfig}                    % Wrap text around figure.
\usepackage{fmtcount, etoolbox}         % For the \book{} command.
\usepackage[newparttoc]{titlesec}       % Formatting chapter, etc.
\usepackage{titletoc}                   % Allows \book in toc.
\usepackage[nottoc]{tocbibind}          % Bibliography in toc.
\usepackage[titles]{tocloft}            % ToC formatting.
\usepackage{multicol, enumitem}         % Multi-column/enumerate.
\usepackage{import}                     % Import external files.
\usepackage{pgfplots, tikz}             % Drawing/graphing tools.
\usetikzlibrary{
    calc,                   % Calculating right angles and more.
    angles,                 % Drawing angles within triangles.
    arrows.meta,            % Latex and Stealth arrows.
    quotes,                 % Adding labels to angles.
    positioning,            % Relative positioning of nodes.
    decorations.markings,   % Adding arrows in the middle of a line.
    patterns,
    arrows,
    shapes,
    shapes.geometric,
    cd,
    hobby,
    babel
}                                       % Libraries for tikz.
\pgfplotsset{compat=1.9}                % Version of pgfplots.
\usepackage[font=scriptsize,
            labelformat=simple,
            labelsep=colon]{subcaption} % Subfigure captions.
\usepackage[font={scriptsize},
            hypcap=true,
            labelsep=colon]{caption}    % Figure captions.
\usepackage{hyperref}                   % Allows for hyperlinks.
\hypersetup{
    colorlinks=true,
    linkcolor=blue,
    filecolor=magenta,
    urlcolor=Cerulean,
    citecolor=SkyBlue
}                           % Colors for hyperref.
\usepackage[toc,acronym,nogroupskip]{glossaries} % Glossaries and acronyms.
\usepackage[subpreambles=false]{standalone}      % Complileable sub files.

% Various font stuff from kiwi.
% Use this for Times text and Computer Modern math
%\usepackage{times}

% Quite nice
%\usepackage[charter, greekfamily=, greekuppercase=italicized]{mathdesign}
%\usepackage[utopia, greekuppercase=italicized]{mathdesign}    % Math is narrower

% Use this for Times text and math
%\usepackage{newtxtext}
%\usepackage[libertine,cmintegrals]{newtxmath}
%\usepackage{fix-cm}

%\usepackage{txfontsb}
% or
%\usepackage{mathptmx}

%\usepackage[scaled=0.92]{helvet}
%\renewcommand{\rmdefault}{ptm}

%\usepackage{mathpazo}    % add possibly `sc` and `osf` options
%\usepackage{eulervm}

%\usepackage{fourier}
%\renewcommand{\rmdefault}{ptm}
%\usepackage{mathptm}

%\usepackage{fontspec}
%\setmainfont{lmodern}

%\usepackage[varg]{txfonts}
%\usepackage{fouriernc}
%\usepackage{mathpazo}

%\usepackage{bookman}
%\usepackage[scaled]{uarial}
%\usepackage[scaled]{helvet}
%\renewcommand*\familydefault{\sfdefault}
%\usepackage[math]{anttor}

%\newcommand\fgeorgia{\fontfamily{jvn}\selectfont}
%\newcommand\ftimes{\fontfamily{ptm}\selectfont}
%\newcommand\fhelvetica{\fontfamily{phv}\selectfont}
%\newcommand\fcourier{\fontfamily{pcr}\selectfont}
%\newcommand\fbookman{\fontfamily{pbk}\selectfont}
%\newcommand\fnewcentury{\fontfamily{pnc}\selectfont}
%\newcommand\fpalatino{\fontfamily{ppl}\selectfont}
%\newcommand\favantgarde{\fontfamily{pag}\selectfont}
%\newcommand\fnormal{\normalfont}
%\newcommand\fsize[1]{\ifnum#1>0\fontsize{#1}{#1}\selectfont\else\normalsize\fi}
%------------------------Theorem Styles-------------------------%
% Define theorem style for default spacing and normal font.
\newtheoremstyle{normal}
    {\topsep}               % Amount of space above the theorem.
    {\topsep}               % Amount of space below the theorem.
    {}                      % Font used for body of theorem.
    {}                      % Measure of space to indent.
    {\bfseries}             % Font of the header of the theorem.
    {}                      % Punctuation between head and body.
    {.5em}                  % Space after theorem head.
    {}

% Define theorem style for default spacing with italicized font.
\newtheoremstyle{normalit}{\topsep}{\topsep}
                {\itshape}{}{\bfseries}{}{.5em}{}

% Italic header environment.
\newtheoremstyle{thmit}{\topsep}{\topsep}{}{}{\itshape}{}{0.5em}{}

% Define italicized environments.
\theoremstyle{normalit}
\newtheorem{theorem}{Theorem}[section]
\newtheorem{lemma}{Lemma}[section]
\newtheorem{corollary}{Corollary}[section]
\newtheorem{proposition}{Proposition}[section]
\newtheorem*{theorem*}{Theorem}

% Define environments with italic headers.
\theoremstyle{thmit}
\newtheorem*{solution}{Solution}
\newtheorem*{fsolution}{Solution}

% Define default environments.
\theoremstyle{normal}
\newtheorem{example}{Example}[section]
\newtheorem{definition}{Definition}[section]
\newtheorem{problem}{Problem}[section]
\newtheorem{question}{Question}[section]
\newtheorem{remark}{Remark}[section]
\newtheorem{properties}{Properties}[section]
\newtheorem{notation}{Notation}[section]
\newtheorem{axiom}{Axiom}[section]
\newtheorem*{properties*}{Properties}
\newtheorem*{remark*}{Remark}
\newtheorem*{definition*}{Definition}
\theoremstyle{plain}

% Define framed environment.
\tcbuselibrary{most}
\newtcbtheorem[use counter*=theorem]{ftheorem}{Theorem}%
    {colback=green!5,colframe=green!35!black,
     fonttitle=\bfseries\upshape}{th}

\newtcbtheorem[use counter*=example]{fdefinition}{Definition}%
    {fonttitle=\bfseries\upshape,
     colback=blue!5!white,colframe=blue!75!black}{def}

\newtcbtheorem[use counter*=example]{fexample}{Example}%
    {fonttitle=\bfseries\upshape,
     colback=red!5!white,colframe=red!75!black}{ex}

\newtcbtheorem[use counter*=notation]{fnotation}{Notation}%
    {fonttitle=\bfseries\upshape,
     colback=SeaGreen!5!white,colframe=SeaGreen!75!black}{ex}

\newtcbtheorem[use counter*=corollary]{fcorollary}{Corollary}%
    {fonttitle=\bfseries\upshape,
     colback=Orchid!5!white,colframe=Orchid!75!black}{ex}

\newenvironment{bproof}{\textit{Proof.}}{\hfill$\square$}
\tcolorboxenvironment{bproof}{blanker,breakable,left=5mm,
                             before skip=10pt,after skip=10pt,
                             borderline west={1mm}{0pt}{red}}
\tcolorboxenvironment{fsolution}
    {enhanced jigsaw,colframe=cyan,interior hidden,breakable}

%--------------------Declared Math Operators--------------------%
\DeclareMathOperator{\Refl}{Refl}           % Reflection operator.
\DeclareMathOperator{\Span}{Span}           % Span of a set of vectors.
\DeclareMathOperator{\Card}{Card}           % Cardinality of set.
\DeclareMathOperator{\Ord}{Ord}             % Ordinal of ordered set.
\DeclareMathOperator{\Tr}{Tr}               % Trace of matrix.
\DeclareMathOperator{\adjoint}{adj}         % Adjoint of matrix.
\DeclareMathOperator{\rk}{rk}               % Rank of operator.
\DeclareMathOperator{\nul}{nul}             % Null space of operator.
\DeclareMathOperator{\sgn}{sgn}             % Sign of a number.
\DeclareMathOperator{\multideg}{mutlideg}   % Multi-Degree (Graphs).
\DeclareMathOperator{\GCD}{GCD}             % Greatest common denominator.
\DeclareMathOperator{\LM}{LM}               % Leading monomial
\DeclareMathOperator{\LC}{LC}               % Leading coefficient.
\DeclareMathOperator{\LT}{LT}               % Leading term.
\DeclareMathOperator{\LCM}{LCM}             % Least common multiple.
\DeclareMathOperator{\Mon}{Mon}             % Monomial.
\DeclareMathOperator{\Spec}{Spec}           % Spectrum.
\DeclareMathOperator{\proj}{proj}           % Projection.
\DeclareMathOperator{\comp}{comp}           % Component.
\DeclareMathOperator{\sinc}{sinc}           % Sinc function.
\DeclareMathOperator{\Ima}{Im}              % Image of operator.
\DeclareMathOperator{\Prin}{Prin}           % Principal value.
\DeclareMathOperator{\Mod}{mod}             % Modulus.
%------------------------New Commands---------------------------%
\DeclarePairedDelimiter\norm{\lVert}{\rVert}
\DeclarePairedDelimiter\ceil{\lceil}{\rceil}
\DeclarePairedDelimiter\floor{\lfloor}{\rfloor}
\newcommand*\diff{\mathop{}\!\mathrm{d}}
\newcommand*\Diff[1]{\mathop{}\!\mathrm{d^#1}}
\renewcommand{\mod}{\ \Mod}
\renewcommand*{\glstextformat}[1]{\textcolor{RoyalBlue}{#1}}
\renewcommand{\glsnamefont}[1]{\textbf{#1}}
\renewcommand\labelitemii{$\circ$}
\renewcommand\thesubfigure{\arabic{chapter}.\arabic{figure}}
\renewcommand\thesubfigure{%
    \arabic{chapter}.\arabic{figure}.\arabic{subfigure}}
\addto\captionsenglish{\renewcommand{\figurename}{Fig.}}
%------------------------Book Command---------------------------%
\makeatletter
\renewcommand\@pnumwidth{1cm}
\newcounter{book}
\renewcommand\thebook{\@Roman\c@book}
\newcommand\book{%
    \if@openright
        \cleardoublepage
    \else
        \clearpage
    \fi
    \thispagestyle{plain}%
    \if@twocolumn
        \onecolumn
        \@tempswatrue
    \else
        \@tempswafalse
    \fi
    \null\vfil
    \secdef\@book\@sbook
}
\def\@book[#1]#2{%
    \ifnum \c@secnumdepth >-3\relax
        \refstepcounter{book}%
        \addcontentsline{toc}{book}{
            \bookname\ \thebook:\hspace{1em}#1
        }
    \else
        \addcontentsline{toc}{book}{#1}%
    \fi
    \markboth{}{}%
    {\centering
     \interlinepenalty \@M
     \normalfont
     \ifnum \c@secnumdepth >-2\relax
       \huge\bfseries \bookname\nobreakspace\thebook
       \par
       \vskip 20\p@
     \fi
     \Huge \bfseries #2\par}%
    \@endbook}
\def\@sbook#1{%
    {\centering
     \interlinepenalty \@M
     \normalfont
     \Huge \bfseries #1\par}%
    \@endbook}
\def\@endbook{
    \vfil\newpage
        \if@twoside
            \if@openright
                \null
                \thispagestyle{empty}%
                \newpage
            \fi
        \fi
        \if@tempswa
            \twocolumn
        \fi
}
\newcommand*\l@book[2]{%
    \ifnum \c@tocdepth >-2\relax
        \addpenalty{-\@highpenalty}%
        \addvspace{2.25em \@plus\p@}%
        \setlength\@tempdima{3em}%
        \begingroup
            \parindent \z@ \rightskip \@pnumwidth
            \parfillskip -\@pnumwidth
            {
                \leavevmode
                \Large \bfseries #1\hfil \hb@xt@\@pnumwidth{
                    \hss #2
                }
            }
            \par
            \nobreak
            \global\@nobreaktrue
            \everypar{\global\@nobreakfalse\everypar{}}%
        \endgroup
    \fi}
\newcommand\bookname{Book}
\renewcommand{\thebook}{\texorpdfstring{\Numberstring{book}}{book}}
\providecommand*{\toclevel@book}{-2}
\makeatother
\titlecontents{chapter}[0pt]
    {\bfseries}
    {\chaptername\ \thecontentslabel:\quad}
    {}
    {\hfill\contentspage}
\titleformat{\part}[display]
    {\Large\bfseries}
    {\partname\nobreakspace\thepart}
    {0mm}
    {\Huge\bfseries}
    \titlecontents{part}[0pt]
    {\large\bfseries}
    {\partname\ \thecontentslabel: \quad}
    {}
    {\hfill\contentspage}
\newcommand{\MarkRightAngle}[4][.3cm]
    {\coordinate (tempa) at ($(#3)!#1!(#2)$);
     \coordinate (tempb) at ($(#3)!#1!(#4)$);
     \coordinate (tempc) at ($(tempa)!0.5!(tempb)$);%midpoint
     \draw (tempa) -- ($(#3)!2!(tempc)$) -- (tempb);}
%--------------------------LENGTHS------------------------------%
% Spacings for the Table of Contents.
\addtolength{\cftsecnumwidth}{1ex}
\addtolength{\cftsubsecindent}{1ex}
\addtolength{\cftsubsecnumwidth}{1ex}
\addtolength{\cftfignumwidth}{1ex}
\addtolength{\cfttabnumwidth}{1ex}

% Spacing for multi-column and enumerate environments.
\setlength{\multicolsep}{6pt}
\setlist[enumerate]{itemsep=0pt,topsep=3pt}

% Indent and paragraph spacing.
\setlength{\parindent}{0em}
\setlength{\parskip}{0em}                                                           %
%--------------------------Main Document---------------------------------------%
\begin{document}
    \pagenumbering{roman}
    \title{Measure Theory and Complex Analysis}
    \author{Ryan Maguire}
    \date{\vspace{-5ex}}
    \maketitle
    \tableofcontents
    \clearpage
    \vspace{10ex}
    \pagenumbering{arabic}
    \section{Riemannian Integration}
        \begin{fdefinition}{Partition}{Partition}
            A partition of an interval $[a,b]$ is a sequence
            $t:\mathbb{Z}_{n}\rightarrow[a,b]$ such that $t$ is
            monotonically increasing and such that $t_{1}=a$ and $t_{n}=b$.
        \end{fdefinition}
        If $f:[a,b]\rightarrow\mathbb{R}$ is a bounded function, and if
        $t:\mathbb{Z}_{n}\rightarrow[a,b]$ is a partition of $[a,b]$, define
        $m,M:\mathbb{Z}_{n-1}\rightarrow[a,b]$ and as follows:
        \par
        \begin{subequations}
            \begin{minipage}[b]{0.49\textwidth}
                \begin{equation}
                    m_{i}=\textrm{inf}\{\,f(x)\,:\,x\in[t_{i},t_{i+1}]\,\}
                \end{equation}
            \end{minipage}
            \hfill
            \begin{minipage}[b]{0.49\textwidth}
                \begin{equation}
                    M_{i}=\textrm{sup}\{\,f(x)\,:\,x\in[t_{i},t_{i+1}]\,\}
                \end{equation}
            \end{minipage}
        \end{subequations}
        \par\vspace{2.5ex}
        Using this, we define the lower sums and upper sums of $f$ about
        the partition $t$ as:
        \par
        \begin{subequations}
            \begin{minipage}[b]{0.49\textwidth}
                \begin{equation}
                    \mathcal{L}(f,t)=\sum_{i=1}^{n-1}m_{i}(t_{i+1}-t_{i})
                \end{equation}
            \end{minipage}
            \hfill
            \begin{minipage}[b]{0.49\textwidth}
                \begin{equation}
                    \mathcal{U}(f,t)=\sum_{i=1}^{n-1}M_{i}(t_{i+1}-t_{i})
                \end{equation}
            \end{minipage}
        \end{subequations}
        \par\vspace{2.5ex}
        As the partition $t$ gets finer, we see that $\mathcal{U}$ decreases
        and $\mathcal{L}$ increases. That is, if we let $P$ be the image of
        our sequence $t$:
        \begin{equation}
            P=\{\,t_{i}\in[a,\,b]\,:\,i\in\mathbb{Z}_{n}\,\}
        \end{equation}
        If $P'$ is another such set generated by the sequence $t'$, then
        and if $P\subseteq{P}'$, then $\mathcal{U}(f,t)\geq\mathcal{U}(f,t')$.
        Similarly, $\mathcal{L}(f,t)\leq\mathcal{L}(f,t')$.
        \begin{fdefinition}{Upper Riemann Integral}{Upper_Riemann_Integral}
            The upper Riemann integral of a function
            $f:[a,b]\rightarrow\mathbb{R}$ is:
            \begin{equation}
                \overline{\mathcal{R}}\int_{a}^{b}f
                =\textrm{inf}\{\mathcal{U}(f,t)\,:\,t
                    \textrm{ is a partition of }[a,b]\,\}
            \end{equation}
        \end{fdefinition}
        The set of all functions from $\mathbb{Z}_{n}$ to $[a,b]$ is a well
        defined set for all $n\in\mathbb{N}$, as is the union over all $n$.
        Since the set of partitions is a subset of this set, the set of all
        partitions of $[a,b]$ is well defined as well. In a similar vein, we
        define the lower Riemann integral.
        \begin{fdefinition}{Lower Riemann Integral}{Lower_Riemann_Integral}
            The lower Riemann integral of a function
            $f:[a,b]\rightarrow\mathbb{R}$ is:
            \begin{equation}
                \underline{\mathcal{R}}\int_{a}^{b}f
                =\textrm{sup}\{\mathcal{L}(f,t)\,:\,t
                    \textrm{ is a partition of }[a,b]\,\}
            \end{equation}
        \end{fdefinition}
        \begin{fdefinition}{Riemann Integrable Functions}
                           {Riemann_Integrable_Functions}
            A Riemann integrable function on an interval $[a,b]$ is a
            function $f:[a,b]\rightarrow\mathbb{R}$ such that:
            \begin{equation}
                \underline{\mathcal{R}}\int_{a}^{b}f
                =\overline{\mathcal{R}}\int_{a}^{b}f
            \end{equation}
            We denote the Riemann integral of $f$ as this number:
            \begin{equation}
                \mathcal{R}\int_{a}^{b}f
                =\underline{\mathcal{R}}\int_{a}^{b}f
                =\overline{\mathcal{R}}\int_{a}^{b}f
            \end{equation}
        \end{fdefinition}
        If $f:[a,b]\rightarrow\mathbb{R}$ is Riemann integrable, then it
        is bounded. For if not, for any partition $t$ of $[a,b]$ there would
        be a subinterval $[t_{i},t_{i+1}]$ such that $f$ has infinite
        supremum or (negative) infinite infinum on this range, and thus
        $|\mathcal{U}(f,t)-\mathcal{L}(f,t)|$ would be infinite. Thus the
        lower and upper Riemann integrals would disagree, and $f$ can not
        be Riemann integrable.
        \begin{fnotation}{Set of Riemann Integrable Functions}
                         {Set_of_Riemann_Integrable_Functions}
            The set of all Riemann integrable functions
            $f:[a,b]\rightarrow\mathbb{R}$ is denoted
            $\mathcal{R}\big([a,b]\big)$.
        \end{fnotation}
        Again, those who study set theory should be weary of the statement
        \textit{the set of all blah}. However, for any two sets $X$ and $Y$,
        the existence of the set of all functions $f:X\rightarrow{Y}$ is
        provable within the standard axioms of Zermelo-Fraenkel set theory,
        and thus the set of all Riemann integrable functions is a subset of
        the set of all functions $f:[a,b]\rightarrow\mathbb{R}$. From this,
        $\mathcal{R}\big([a,b]\big)$ is a well defined set.
        \par\hfill\par
        We see that there are some limitations to the Riemann integrable,
        firstly that is only applies to bounded functions, and secondly
        that we've only defined it on closed and bounded intervals.
        Topologically this means that any function defined on a
        \textit{non-compact} and \textit{non-connected} subset of
        $\mathbb{R}$ will not have a defined integral. Moreover, if
        the Riemann integral is a measurement of the \textit{area} under a
        curve, then there are functions whose area we know from an intuitive
        point of view, but whose Riemann integral is undefined. Such problems
        with the Riemann integral are the motivate the study of a new
        type of integral and developing measure theory.
        \begin{fdefinition}{Subsets of Measure Zero}{Subsets_of_Measure_Zero}
            A subset of measure zero (in the sense of Lebesgue, or in the
            sense of the Lebesgue \textit{measure}) is a subset
            $A\subseteq\mathbb{R}$ such that, for all $\varepsilon$, there
            exists a sequence of points $a:\mathbb{N}\rightarrow\mathbb{R}$
            and $b:\mathbb{N}\rightarrow\mathbb{R}$ such that, for all
            $n\in\mathbb{N}$, $a_{n}<b_{n}$, and:
            \begin{equation}
                A\subseteq\bigcup_{n\in\mathbb{N}}(a_{n},\,b_{n})
            \end{equation}
            and also such that:
            \begin{equation}
                \sum_{n=0}^{\infty}|b_{n}-a_{n}|<\varepsilon
            \end{equation}
        \end{fdefinition}
        We say in the sense of Lebesgue, or in the sense of the Lebesgue
        measure, to prevent any retroactive confusion when arbitrary measures
        are developed later. This sense of \textit{measure zero} has a sense
        of \textit{measuring} associated to it. Namely, one covers the set
        given set with a union of intervals and takes the length to be the
        sum of the lengths of the individual intervals. If this sum can be
        made arbitrarily small, we say that the set has measure zero. In a
        similarly way we can define a set of measure $r\in\mathbb{R}^{+}$ to
        be a set that can be covered with intervals whose lengths sum to
        $r+\varepsilon$ for some $\varepsilon>0$, but cannot be made less
        than $r$.
        \begin{ltheorem}{Countable Implies Lebesgue Measure Zero}
                        {Countable_Implies_Lebesgue_Measure_Zero}
            If $A\subseteq\mathbb{R}$ is a countable set, then it has
            Lebesgue measure zero.
        \end{ltheorem}
        \begin{proof}
            For if $A$ is countable, there exists a bijection
            $r:\mathbb{N}\rightarrow{N}$. Let
            $a,b:\mathbb{N}\rightarrow\mathbb{R}$ be defined as follows:
            \par
            \begin{subequations}
                \begin{minipage}[b]{0.49\textwidth}
                    \begin{equation}
                        a_{n}=r_{n}-\frac{\varepsilon}{2^{n+2}}
                    \end{equation}
                \end{minipage}
                \hfill
                \begin{minipage}[b]{0.49\textwidth}
                    \begin{equation}
                        b_{n}=r_{n}+\frac{\varepsilon}{2^{n+2}}
                    \end{equation}
                \end{minipage}
            \end{subequations}
            \par\vspace{2.5ex}
            Then, for all $n\in\mathbb{N}$, $a_{n}\leq{b}_{n}$, and
            $b_{n}-a_{n}=\varepsilon/2^{n+1}$. But also:
            \begin{equation}
                A\subseteq\bigcup_{n\in\mathbb{N}}(a_{n},\,b_{n})
            \end{equation}
            and:
            \begin{equation}
                \sum_{n=1}^{\infty}(b_{n}-a_{n})
                =\varepsilon\sum_{n=1}^{\infty}\frac{1}{2^{n+1}}
                =\frac{\varepsilon}{2}<\varepsilon
            \end{equation}
            Thus, $A$ has Lebesgue measure zero.
        \end{proof}
        Again, to avoid the notion that we are being redundant, there are
        measures where countable sets do not have measure zero. Those
        familiar with physics may have heard of the Dirac
        $\delta$ \textit{function}, the word function being italicized since
        it is not a true function. But we may think of it as a
        \textit{measure} with the following property: If the set $A$ contains
        zero, then it has measure one, otherwise it has measure zero. This
        is a measure (which we will define shortly) and the previous theorem
        does not apply to it. Such measures are called \textit{atomic}.
        Most of our study will deal with \textit{non-atomic} measures, such
        as the Lebesgue measure.
        \par\hfill\par
        Thm.~\ref{thm:Countable_Implies_Lebesgue_Measure_Zero} is not an
        if and only if theorem. That is, there are subsets
        $A\subseteq\mathbb{R}$ that are \textit{uncountable} but still have
        measure zero, the most famous example being the Cantor set. The
        notion of Lebesgue measure zero leads us to a crucial theorem.
        \begin{ltheorem}{Lebesgues's Criterion for the Riemann Integral}
                        {Lebesgues_Criterion_for_the_Riemann_Integral}
            A function $f:[a,b]\rightarrow\mathbb{R}$ is Riemann integrable
            if and only if $f$ is bounded and such that the set of all
            discontinuities of $f$ on $[a,b]$ has Lebesgue measure zero.
        \end{ltheorem}
        This theorem gives us examples of functions we would like to be
        able to integrate, but cannot in the sense of Riemann.
        \begin{lexample}{Non-Riemann Integrable Function}
                        {Non_Riemann_Integrable_Function}
            Let $f:[0,1]\rightarrow\mathbb{R}$ be defined as follows:
            \begin{equation}
                f(x)=
                \begin{cases}
                    1,&x\in[0,1]\cap\mathbb{Q}\\
                    0,&x\in[0,1]\setminus\mathbb{Q}
                \end{cases}
            \end{equation}
            The set of discontinuities is the entire domain $[0,1]$ and
            this by Lebesgue's criterion, $f$ is not Riemann integrable.
            However, if we use the notion of area, the integral of this
            function should almost certainly be zero. For the function is
            zero only all but a set of measure zero, and thus the
            \textit{width} on this set is zero, and the \textit{height} on
            this set is one. Thus the area should be $0\cdot{1}=0$, and thus
            the integral of $f$ should be 1. The Riemann integral cannot
            capture such things, even though it is desirable. We must look
            for a new type of integral.
        \end{lexample}
        \begin{lexample}
              {Riemann Integrable with Countable Discontinuities}
              {Riemann_Integrable_with_Countable_Discontinuities}
            For $x\in[0,1]\cap\mathbb{Q}$, let $p(x)$ and $q(x)$ be such
            that $p(x)$ and $q(x)$ are coprime and such that:
            \begin{equation}
                x=\frac{p(x)}{q(x)}
            \end{equation}
            Define $f:[0,1]\rightarrow\mathbb{R}$ by:
            \begin{equation}
                f(x)=
                \begin{cases}
                    \frac{1}{q(x)},&x\in[0,1]\cap\mathbb{Q}\\
                    0,&x\in[0,1]\setminus\mathbb{Q}
                \end{cases}
            \end{equation}
            Then for all $x\in[0,1]\setminus\mathbb{Q}$, $f$ is continuous
            at $x$. Given an irrational $x$ and a real number $\varepsilon>0$,
            choose $\delta$ such that all of the nearby rational numbers have
            reduced denominator greater than $1/\varepsilon$. $f$ is also
            discontinuous at all rational points, since $f(x)$ will be
            positive and for any irrational there is a sequence of
            irrationals $a:\mathbb{N}\rightarrow\mathbb{R}$ such that
            $a_{n}\rightarrow{x}$. But then $f(a_{n})=0$ and thus
            $f(a_{n})\not\rightarrow{f}(x)$
        \end{lexample}
        \begin{figure}[H]
            \centering
            \captionsetup{type=figure}
            %--------------------------------Dependencies----------------------------------%
%   tikz                                                                       %
%       arrows.meta                                                            %
%-------------------------------Main Document----------------------------------%
\begin{tikzpicture}[scale=8, >=Latex]
    \begin{scope}[thick]
        \draw[->] (-0.1,0) to (1.1,0) node[above left] {$x$};
        \draw[->] (0,-0.1) to (0,0.6) node[right]      {$f(x)$};
    \end{scope}

    \draw (0.02,1/2) to (-0.02,1/2) node [left] {$\frac{1}{2}$};
    \draw (0.02,1/3) to (-0.02,1/3) node [left] {$\frac{1}{3}$};
    \draw (0.02,1/4) to (-0.02,1/4) node [left] {$\frac{1}{4}$};
    \foreach\X[evaluate=\X as \Ymax using {int(\X-1)}]in {25,24,...,2}{%
        \foreach\Y in {1,...,\Ymax}{%
            \ifnum\X<4
                \draw (\Y/\X, 0.02) to (\Y/\X, -0.02)
                    node[below] {$\frac{\Y}{\X}$};
            \else
                \draw[ultra thin] (\Y/\X,0.01) to (\Y/\X,-0.01);
            \fi
            \pgfmathtruncatemacro{\TST}{gcd(\X, \Y)}
            \ifnum\TST=1
                \fill ({\Y/\X}, 1/\X)  circle (0.2pt);
            \fi
        }
    }
    \foreach\X in {0, 1, ..., 80}{%
        \fill (\X/80,0) circle (0.2pt);
    }
\end{tikzpicture}
            \caption{Caption}
            \label{fig:my_label}
        \end{figure}
\end{document}