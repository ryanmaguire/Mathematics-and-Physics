\documentclass[crop=false,class=book,oneside]{standalone}
%----------------------------Preamble-------------------------------%
%---------------------------Packages----------------------------%
\usepackage{geometry}
\geometry{b5paper, margin=1.0in}
\usepackage[T1]{fontenc}
\usepackage{graphicx, float}            % Graphics/Images.
\usepackage{natbib}                     % For bibliographies.
\bibliographystyle{agsm}                % Bibliography style.
\usepackage[french, english]{babel}     % Language typesetting.
\usepackage[dvipsnames]{xcolor}         % Color names.
\usepackage{listings}                   % Verbatim-Like Tools.
\usepackage{mathtools, esint, mathrsfs} % amsmath and integrals.
\usepackage{amsthm, amsfonts, amssymb}  % Fonts and theorems.
\usepackage{tcolorbox}                  % Frames around theorems.
\usepackage{upgreek}                    % Non-Italic Greek.
\usepackage{fmtcount, etoolbox}         % For the \book{} command.
\usepackage[newparttoc]{titlesec}       % Formatting chapter, etc.
\usepackage{titletoc}                   % Allows \book in toc.
\usepackage[nottoc]{tocbibind}          % Bibliography in toc.
\usepackage[titles]{tocloft}            % ToC formatting.
\usepackage{pgfplots, tikz}             % Drawing/graphing tools.
\usepackage{imakeidx}                   % Used for index.
\usetikzlibrary{
    calc,                   % Calculating right angles and more.
    angles,                 % Drawing angles within triangles.
    arrows.meta,            % Latex and Stealth arrows.
    quotes,                 % Adding labels to angles.
    positioning,            % Relative positioning of nodes.
    decorations.markings,   % Adding arrows in the middle of a line.
    patterns,
    arrows
}                                       % Libraries for tikz.
\pgfplotsset{compat=1.9}                % Version of pgfplots.
\usepackage[font=scriptsize,
            labelformat=simple,
            labelsep=colon]{subcaption} % Subfigure captions.
\usepackage[font={scriptsize},
            hypcap=true,
            labelsep=colon]{caption}    % Figure captions.
\usepackage[pdftex,
            pdfauthor={Ryan Maguire},
            pdftitle={Mathematics and Physics},
            pdfsubject={Mathematics, Physics, Science},
            pdfkeywords={Mathematics, Physics, Computer Science, Biology},
            pdfproducer={LaTeX},
            pdfcreator={pdflatex}]{hyperref}
\hypersetup{
    colorlinks=true,
    linkcolor=blue,
    filecolor=magenta,
    urlcolor=Cerulean,
    citecolor=SkyBlue
}                           % Colors for hyperref.
\usepackage[toc,acronym,nogroupskip,nopostdot]{glossaries}
\usepackage{glossary-mcols}
%------------------------Theorem Styles-------------------------%
\theoremstyle{plain}
\newtheorem{theorem}{Theorem}[section]

% Define theorem style for default spacing and normal font.
\newtheoremstyle{normal}
    {\topsep}               % Amount of space above the theorem.
    {\topsep}               % Amount of space below the theorem.
    {}                      % Font used for body of theorem.
    {}                      % Measure of space to indent.
    {\bfseries}             % Font of the header of the theorem.
    {}                      % Punctuation between head and body.
    {.5em}                  % Space after theorem head.
    {}

% Italic header environment.
\newtheoremstyle{thmit}{\topsep}{\topsep}{}{}{\itshape}{}{0.5em}{}

% Define environments with italic headers.
\theoremstyle{thmit}
\newtheorem*{solution}{Solution}

% Define default environments.
\theoremstyle{normal}
\newtheorem{example}{Example}[section]
\newtheorem{definition}{Definition}[section]
\newtheorem{problem}{Problem}[section]

% Define framed environment.
\tcbuselibrary{most}
\newtcbtheorem[use counter*=theorem]{ftheorem}{Theorem}{%
    before=\par\vspace{2ex},
    boxsep=0.5\topsep,
    after=\par\vspace{2ex},
    colback=green!5,
    colframe=green!35!black,
    fonttitle=\bfseries\upshape%
}{thm}

\newtcbtheorem[auto counter, number within=section]{faxiom}{Axiom}{%
    before=\par\vspace{2ex},
    boxsep=0.5\topsep,
    after=\par\vspace{2ex},
    colback=Apricot!5,
    colframe=Apricot!35!black,
    fonttitle=\bfseries\upshape%
}{ax}

\newtcbtheorem[use counter*=definition]{fdefinition}{Definition}{%
    before=\par\vspace{2ex},
    boxsep=0.5\topsep,
    after=\par\vspace{2ex},
    colback=blue!5!white,
    colframe=blue!75!black,
    fonttitle=\bfseries\upshape%
}{def}

\newtcbtheorem[use counter*=example]{fexample}{Example}{%
    before=\par\vspace{2ex},
    boxsep=0.5\topsep,
    after=\par\vspace{2ex},
    colback=red!5!white,
    colframe=red!75!black,
    fonttitle=\bfseries\upshape%
}{ex}

\newtcbtheorem[auto counter, number within=section]{fnotation}{Notation}{%
    before=\par\vspace{2ex},
    boxsep=0.5\topsep,
    after=\par\vspace{2ex},
    colback=SeaGreen!5!white,
    colframe=SeaGreen!75!black,
    fonttitle=\bfseries\upshape%
}{not}

\newtcbtheorem[use counter*=remark]{fremark}{Remark}{%
    fonttitle=\bfseries\upshape,
    colback=Goldenrod!5!white,
    colframe=Goldenrod!75!black}{ex}

\newenvironment{bproof}{\textit{Proof.}}{\hfill$\square$}
\tcolorboxenvironment{bproof}{%
    blanker,
    breakable,
    left=3mm,
    before skip=5pt,
    after skip=10pt,
    borderline west={0.6mm}{0pt}{green!80!black}
}

\AtEndEnvironment{lexample}{$\hfill\textcolor{red}{\blacksquare}$}
\newtcbtheorem[use counter*=example]{lexample}{Example}{%
    empty,
    title={Example~\theexample},
    boxed title style={%
        empty,
        size=minimal,
        toprule=2pt,
        top=0.5\topsep,
    },
    coltitle=red,
    fonttitle=\bfseries,
    parbox=false,
    boxsep=0pt,
    before=\par\vspace{2ex},
    left=0pt,
    right=0pt,
    top=3ex,
    bottom=1ex,
    before=\par\vspace{2ex},
    after=\par\vspace{2ex},
    breakable,
    pad at break*=0mm,
    vfill before first,
    overlay unbroken={%
        \draw[red, line width=2pt]
            ([yshift=-1.2ex]title.south-|frame.west) to
            ([yshift=-1.2ex]title.south-|frame.east);
        },
    overlay first={%
        \draw[red, line width=2pt]
            ([yshift=-1.2ex]title.south-|frame.west) to
            ([yshift=-1.2ex]title.south-|frame.east);
    },
}{ex}

\AtEndEnvironment{ldefinition}{$\hfill\textcolor{Blue}{\blacksquare}$}
\newtcbtheorem[use counter*=definition]{ldefinition}{Definition}{%
    empty,
    title={Definition~\thedefinition:~{#1}},
    boxed title style={%
        empty,
        size=minimal,
        toprule=2pt,
        top=0.5\topsep,
    },
    coltitle=Blue,
    fonttitle=\bfseries,
    parbox=false,
    boxsep=0pt,
    before=\par\vspace{2ex},
    left=0pt,
    right=0pt,
    top=3ex,
    bottom=0pt,
    before=\par\vspace{2ex},
    after=\par\vspace{1ex},
    breakable,
    pad at break*=0mm,
    vfill before first,
    overlay unbroken={%
        \draw[Blue, line width=2pt]
            ([yshift=-1.2ex]title.south-|frame.west) to
            ([yshift=-1.2ex]title.south-|frame.east);
        },
    overlay first={%
        \draw[Blue, line width=2pt]
            ([yshift=-1.2ex]title.south-|frame.west) to
            ([yshift=-1.2ex]title.south-|frame.east);
    },
}{def}

\AtEndEnvironment{ltheorem}{$\hfill\textcolor{Green}{\blacksquare}$}
\newtcbtheorem[use counter*=theorem]{ltheorem}{Theorem}{%
    empty,
    title={Theorem~\thetheorem:~{#1}},
    boxed title style={%
        empty,
        size=minimal,
        toprule=2pt,
        top=0.5\topsep,
    },
    coltitle=Green,
    fonttitle=\bfseries,
    parbox=false,
    boxsep=0pt,
    before=\par\vspace{2ex},
    left=0pt,
    right=0pt,
    top=3ex,
    bottom=-1.5ex,
    breakable,
    pad at break*=0mm,
    vfill before first,
    overlay unbroken={%
        \draw[Green, line width=2pt]
            ([yshift=-1.2ex]title.south-|frame.west) to
            ([yshift=-1.2ex]title.south-|frame.east);},
    overlay first={%
        \draw[Green, line width=2pt]
            ([yshift=-1.2ex]title.south-|frame.west) to
            ([yshift=-1.2ex]title.south-|frame.east);
    }
}{thm}

%--------------------Declared Math Operators--------------------%
\DeclareMathOperator{\adjoint}{adj}         % Adjoint.
\DeclareMathOperator{\Card}{Card}           % Cardinality.
\DeclareMathOperator{\curl}{curl}           % Curl.
\DeclareMathOperator{\diam}{diam}           % Diameter.
\DeclareMathOperator{\dist}{dist}           % Distance.
\DeclareMathOperator{\Div}{div}             % Divergence.
\DeclareMathOperator{\Erf}{Erf}             % Error Function.
\DeclareMathOperator{\Erfc}{Erfc}           % Complementary Error Function.
\DeclareMathOperator{\Ext}{Ext}             % Exterior.
\DeclareMathOperator{\GCD}{GCD}             % Greatest common denominator.
\DeclareMathOperator{\grad}{grad}           % Gradient
\DeclareMathOperator{\Ima}{Im}              % Image.
\DeclareMathOperator{\Int}{Int}             % Interior.
\DeclareMathOperator{\LC}{LC}               % Leading coefficient.
\DeclareMathOperator{\LCM}{LCM}             % Least common multiple.
\DeclareMathOperator{\LM}{LM}               % Leading monomial.
\DeclareMathOperator{\LT}{LT}               % Leading term.
\DeclareMathOperator{\Mod}{mod}             % Modulus.
\DeclareMathOperator{\Mon}{Mon}             % Monomial.
\DeclareMathOperator{\multideg}{mutlideg}   % Multi-Degree (Graphs).
\DeclareMathOperator{\nul}{nul}             % Null space of operator.
\DeclareMathOperator{\Ord}{Ord}             % Ordinal of ordered set.
\DeclareMathOperator{\Prin}{Prin}           % Principal value.
\DeclareMathOperator{\proj}{proj}           % Projection.
\DeclareMathOperator{\Refl}{Refl}           % Reflection operator.
\DeclareMathOperator{\rk}{rk}               % Rank of operator.
\DeclareMathOperator{\sgn}{sgn}             % Sign of a number.
\DeclareMathOperator{\sinc}{sinc}           % Sinc function.
\DeclareMathOperator{\Span}{Span}           % Span of a set.
\DeclareMathOperator{\Spec}{Spec}           % Spectrum.
\DeclareMathOperator{\supp}{supp}           % Support
\DeclareMathOperator{\Tr}{Tr}               % Trace of matrix.
%--------------------Declared Math Symbols--------------------%
\DeclareMathSymbol{\minus}{\mathbin}{AMSa}{"39} % Unary minus sign.
%------------------------New Commands---------------------------%
\DeclarePairedDelimiter\norm{\lVert}{\rVert}
\DeclarePairedDelimiter\ceil{\lceil}{\rceil}
\DeclarePairedDelimiter\floor{\lfloor}{\rfloor}
\newcommand*\diff{\mathop{}\!\mathrm{d}}
\newcommand*\Diff[1]{\mathop{}\!\mathrm{d^#1}}
\renewcommand*{\glstextformat}[1]{\textcolor{RoyalBlue}{#1}}
\renewcommand{\glsnamefont}[1]{\textbf{#1}}
\renewcommand\labelitemii{$\circ$}
\renewcommand\thesubfigure{%
    \arabic{chapter}.\arabic{figure}.\arabic{subfigure}}
\addto\captionsenglish{\renewcommand{\figurename}{Fig.}}
\numberwithin{equation}{section}

\renewcommand{\vector}[1]{\boldsymbol{\mathrm{#1}}}

\newcommand{\uvector}[1]{\boldsymbol{\hat{\mathrm{#1}}}}
\newcommand{\topspace}[2][]{(#2,\tau_{#1})}
\newcommand{\measurespace}[2][]{(#2,\varSigma_{#1},\mu_{#1})}
\newcommand{\measurablespace}[2][]{(#2,\varSigma_{#1})}
\newcommand{\manifold}[2][]{(#2,\tau_{#1},\mathcal{A}_{#1})}
\newcommand{\tanspace}[2]{T_{#1}{#2}}
\newcommand{\cotanspace}[2]{T_{#1}^{*}{#2}}
\newcommand{\Ckspace}[3][\mathbb{R}]{C^{#2}(#3,#1)}
\newcommand{\funcspace}[2][\mathbb{R}]{\mathcal{F}(#2,#1)}
\newcommand{\smoothvecf}[1]{\mathfrak{X}(#1)}
\newcommand{\smoothonef}[1]{\mathfrak{X}^{*}(#1)}
\newcommand{\bracket}[2]{[#1,#2]}

%------------------------Book Command---------------------------%
\makeatletter
\renewcommand\@pnumwidth{1cm}
\newcounter{book}
\renewcommand\thebook{\@Roman\c@book}
\newcommand\book{%
    \if@openright
        \cleardoublepage
    \else
        \clearpage
    \fi
    \thispagestyle{plain}%
    \if@twocolumn
        \onecolumn
        \@tempswatrue
    \else
        \@tempswafalse
    \fi
    \null\vfil
    \secdef\@book\@sbook
}
\def\@book[#1]#2{%
    \refstepcounter{book}
    \addcontentsline{toc}{book}{\bookname\ \thebook:\hspace{1em}#1}
    \markboth{}{}
    {\centering
     \interlinepenalty\@M
     \normalfont
     \huge\bfseries\bookname\nobreakspace\thebook
     \par
     \vskip 20\p@
     \Huge\bfseries#2\par}%
    \@endbook}
\def\@sbook#1{%
    {\centering
     \interlinepenalty \@M
     \normalfont
     \Huge\bfseries#1\par}%
    \@endbook}
\def\@endbook{
    \vfil\newpage
        \if@twoside
            \if@openright
                \null
                \thispagestyle{empty}%
                \newpage
            \fi
        \fi
        \if@tempswa
            \twocolumn
        \fi
}
\newcommand*\l@book[2]{%
    \ifnum\c@tocdepth >-3\relax
        \addpenalty{-\@highpenalty}%
        \addvspace{2.25em\@plus\p@}%
        \setlength\@tempdima{3em}%
        \begingroup
            \parindent\z@\rightskip\@pnumwidth
            \parfillskip -\@pnumwidth
            {
                \leavevmode
                \Large\bfseries#1\hfill\hb@xt@\@pnumwidth{\hss#2}
            }
            \par
            \nobreak
            \global\@nobreaktrue
            \everypar{\global\@nobreakfalse\everypar{}}%
        \endgroup
    \fi}
\newcommand\bookname{Book}
\renewcommand{\thebook}{\texorpdfstring{\Numberstring{book}}{book}}
\providecommand*{\toclevel@book}{-2}
\makeatother
\titleformat{\part}[display]
    {\Large\bfseries}
    {\partname\nobreakspace\thepart}
    {0mm}
    {\Huge\bfseries}
\titlecontents{part}[0pt]
    {\large\bfseries}
    {\partname\ \thecontentslabel: \quad}
    {}
    {\hfill\contentspage}
\titlecontents{chapter}[0pt]
    {\bfseries}
    {\chaptername\ \thecontentslabel:\quad}
    {}
    {\hfill\contentspage}
\newglossarystyle{longpara}{%
    \setglossarystyle{long}%
    \renewenvironment{theglossary}{%
        \begin{longtable}[l]{{p{0.25\hsize}p{0.65\hsize}}}
    }{\end{longtable}}%
    \renewcommand{\glossentry}[2]{%
        \glstarget{##1}{\glossentryname{##1}}%
        &\glossentrydesc{##1}{~##2.}
        \tabularnewline%
        \tabularnewline
    }%
}
\newglossary[not-glg]{notation}{not-gls}{not-glo}{Notation}
\newcommand*{\newnotation}[4][]{%
    \newglossaryentry{#2}{type=notation, name={\textbf{#3}, },
                          text={#4}, description={#4},#1}%
}
%--------------------------LENGTHS------------------------------%
% Spacings for the Table of Contents.
\addtolength{\cftsecnumwidth}{1ex}
\addtolength{\cftsubsecindent}{1ex}
\addtolength{\cftsubsecnumwidth}{1ex}
\addtolength{\cftfignumwidth}{1ex}
\addtolength{\cfttabnumwidth}{1ex}

% Indent and paragraph spacing.
\setlength{\parindent}{0em}
\setlength{\parskip}{0em}
\graphicspath{{../../../images/}}   % Path to Image Folder.
%--------------------------Main Document----------------------------%
\begin{document}
    \ifx\ifbiocourses\undefined
        \pagenumbering{roman}
        \title{Biology I}
        \author{Ryan Maguire}
        \date{\vspace{-5ex}}
        \maketitle
        \tableofcontents
        \listoffigures
        \clearpage
        \chapter*{Biology I}
        \addcontentsline{toc}{chapter}{Biology I}
        \markboth{}{BIOLOGY I}
        \setcounter{chapter}{1}
        \pagenumbering{arabic}
    \else
        \chapter{Introductory Biology I}
    \fi
    \section{Introduction}
        Living things often have to adapt to their
        environment in order to survive. One such example
        is that of the ghost plant, found in Northeast
        Mexico. Its strange leaves allow to absorb, store,
        and conserve water. Such adaptation are the
        result of \textbf{evolution}. Evolution is the
        process of change that life undergoes, and has been
        transforming living things since the first
        single-cell organisms appeared on Earth. Evolution
        is the back-bone behind all of
        \textbf{biology}, the study of life. The word
        \textit{life} itself needs some definition. While
        it is a very intuitive concept, it can be hard to
        describe in a consistent manner while encompassing
        all of the things we'd like to consider
        \textit{living} and excluding things that are
        \textit{non-living}. We can attempt to define
        what a living organism by imposing the following
        requirements:
        \begin{enumerate}
            \item \textbf{Order}: Organisms are made of
                  \textit{cells} and these cells arrange
                  themselves in complex manners to perform
                  basic tasks. For example, cardiac muscle
                  cells working together to form a heart.
            \item \textbf{Evolution}: All life evolves.
                  This process occurs by means of natural
                  selection. One such is example is the
                  evolution of the \textit{peppered moth}
                  during the industrial revolution in
                  England. The white bodied peppered moth,
                  or \textit{Biston betularia f. typica},
                  once thrived in pre-industrial evolution.
                  But as chimney stacks formed and soot
                  became common amongst the trees, their
                  bright colors made them easy picking
                  for predators. The black bodied
                  peppered moth, or
                  \textit{Biston betularia f. carbonaria},
                  thrived as it easily blended in to the
                  polluted trees and vegetation. There are
                  many examples of evolution occuring,
                  including that of the
                  pygmy sea horse using camouflage to
                  blend into its environment, hiding it
                  from predators.
            \item \textbf{Respond to External Stimuli}:
                  All living organisms respond
                  to the various stimuli
                  that are present in their environment.
                  For humans this may be a car horn
                  honking, or a cell phone ringing.
                  All forms of life
                  react to external stimuli in one way or
                  another.
            \item \textbf{Maintains Homeostasis}:
                  Homeostasis is a fancy word for a
                  regular internal ``environment,'' in
                  a living organism. For mammals this could
                  be the warm internal body temperatures
                  or a steady blood flow to various organs.
                  Mammals maintain homeostasis in
                  many ways.
                  Humans, for example, sweat when too hot
                  and shiver when too cold.
            \item \textbf{Reproduction}: All living things
                  reproduce. Reproduction can be very
                  different from the manner in which
                  mammals reproduce. There is both
                  sexual and asexual reproduction. Single
                  celled organism often reproduce via
                  asexual means, and all mammals
                  reproduce by sexual ones.
            \item \textbf{Energy Consumption}: Energy
                  consumption does not necessarily mean
                  eating other organisms. This is a
                  trait found in animals. Plants obtain
                  energy via
                  \textit{photosynthesis} and can convert
                  sunlight into useable energy.
            \item \textbf{Growth and Development}: Living
                  things grow. Humans start out as a
                  couple of cells called a zygote and
                  eventually grow into 70 trillion+
                  celled organism.
        \end{enumerate}
        The study of life extends from the level of
        microscopic cells ($10^{-6}\textrm{m}$) to the
        macroscopic biosphere ($10^{6}\textrm{m}$).
        The structure of cells is very important for
        life to prosper. As mentioned before plants
        obtain their energy in a process called
        \textit{photosynthesis}. This takes place
        in an \textit{organelle} called the
        \textit{chloroplast}. However,
        photosynthesis will not occur in a random
        mixture of chloroplasts and chlorophyll in
        a test-tube. The structure and organization of
        plant cells is crucial for photosynthesis to
        be successful. The organization of
        constituent components is an important aspect
        outside of biology as well. In chemistry one
        can take carbon and re-arrange in different manners
        to obtain different objects. Graphite and
        diamonds are different objects, but are both
        pure carbon. The lattice structure in these two
        objects is different, however. To better study
        life biologists often take the approach of
        \textit{reductionism}. This is the method of
        reducing complex systems into smaller and simpler
        components to ease the analysis. On the other hand
        one can study biology in a more hollistic approach.
        In biology, a system is a combination of
        components that work together to perform
        some function. A human can be considered as a
        system, with cells, blood, skin, and a plethora
        of complex organ systems working together to
        perform the basic tasks humans undergo.
        \textbf{Systems Biology} is the method of
        modelling and explaining phenomena in biology
        by means of studying a system's parts. There
        are ten main layers we wish to study in biology.
        \begin{enumerate}
            \item \textbf{Biosphere}: This is the Earth.
                  This includes the continents and oceans
                  that it is comprised of, the forests
                  and islands, and all of the organisms
                  that live on it.
            \item \textbf{Ecosystems}: Ecosystems are
                  smaller subsets of the biosphere. This
                  includes the Amazon rain forest, the
                  arctic tundras, and the Sahara desert.
                  There are many other kinds of ecosystems,
                  such as forests, grasslands, coral reefs,
                  and open oceans.
            \item \textbf{Communities}: Communities are the
                  entire collection of organisms that live
                  in a given ecosystems. For the arctic
                  this includes polar bears, reindeer,
                  beluga whales, walruses, and seals.
                  In the amazon it includes all of the
                  plants, bacteria, and animals that
                  one might find.
            \item \textbf{Populations}: The populations
                  of an ecosystems are the various
                  collections of a single species one
                  might find. All of the penguins in
                  antarctica make up one population, and
                  all of the sea lions make another.
            \item \textbf{Organisms}: Individual living
                  things are called organisms. A human
                  is an organism, as is a cat, a sunflower,
                  or a sequoia tree.
            \item \textbf{Organs/Organ Systems}:
                  Examples of organs in mammals include
                  a heart, brain, kidney, and liver. In
                  plants this includes the stem and
                  and roots. For humans, the integumentary
                  system (Or skin) also constitutes an
                  organ system.
            \item \textbf{Tissues}: Tissues are structures
                  of cells working together for a common
                  function. Most examples of tissue require
                  a microscope to see, and thus leave
                  the realm of intuition.
            \item \textbf{Cells}: Cells are the
                  fundamental building blocks of life.
                  Indeed, cells are part of the
                  definition of life. Cells are extremely
                  small, about 10 to 100 micrometers
                  ($\mu\textrm{m}$), and require a
                  microscope to examine. Some organisms,
                  such as bacteria, or amoebas, are
                  single celled. Others, such as humans,
                  have more than 70 trillion cells.
            \item \textbf{Organelles}: Organelles are the
                  pieces that make up a cell. In a plant
                  cell this may include the cell wall or
                  chloroplasts, and in animal cells this
                  may include ribosomes, mitochondria,
                  vacuoles, and more.
            \item \textbf{Molecules}: Molecules are
                  beyond the realm of \textrm{living},
                  and are comprised of chains of atoms.
                  Molecules are the principle concept of
                  study in \textit{chemistry}, and also
                  of substantial interets in
                  \textit{physics}. Chlorophyll is an
                  example of a molecule, a very long and
                  complex chain of different elements.
                  Water is another, simpler example,
                  containing three atoms: Two hydrogen and
                  one oxygen. That is,
                  $\textrm{H}_{2}O$.
        \end{enumerate}
        Ecosystems consist of both organisms and
        physical factors. Plants and animals
        are examples of organisms, and sunlight,
        carbon dioxide, and oxygen represent
        non-living physical factors. Both organisms and
        the environment are affected by the interactions
        between them. Most of the diatomic oxygen that is
        found in the Earth's atmosphere is the result of
        the byproduct of photosynthesizing organisms such
        as plants. Organisms also interact with other
        organisms creating a cycle of nutrients. Humans
        also interact with their environment, and sometimes
        in a negative way. Since the late 1800's the
        $\textrm{CO}_{2}$ output from human's has created
        a layer of gas that traps radiation (or light) from
        the sun. This \textit{greenhouse} effect has led to
        an estimated $1^{\circ}\textrm{C}$ increase in
        average global temperatures since 1900. This
        phenonoma is not only observed on Earth, but on
        Venus as well. One who has studied a little
        astronomy may find it odd that Venus is the hottest
        planet, and yet Mercury is the closest to the sun.
        The reason is that Venus has a far worse greenhouse
        effect than the one that is starting to form on
        Earth. The thick atmosphere of venus allows sunlight
        to penetrate, but traps it inside causing temperatures
        in excess of $850^{\circ}\textrm{F}$.  Compare that
        to the $800^{\circ}\textrm{F}$ temperatures
        found on Mercury, even though Mercury is roughly
        half the distance to the sun as Venus. This
        greenhouse effect can have a very negative impact
        on Earth as well. One of the themes of life was
        the consumption of energy. For plants this often
        occurs via the process of photosynthesis.
        Chlorophyll molecules in leaves can absorb energy
        from sunlight, similar to a solar panel. The
        process of photosynthesis converts carbon dioxide
        ($\textrm{CO}_{2}$) and water
        ($\textrm{H}_{2}\textrm{O}$) into sugars and
        oxygen. One common sugar is \textit{glucose}.
        This has the chemical formula
        $\textrm{C}_{6}\textrm{H}_{12}\textrm{O}_{6}$. An
        important concept in chemistry is that of
        conservation of matter. When dealing with
        processes such as photosynthesis it is import that
        one has balanced equations. The equation for
        photosynthesis looks like the following:
        \begin{equation}
            6\textrm{CO}_{2}+6\textrm{H}_{2}\textrm{O}
            \overset{\textrm{Sunlight}}{\longrightarrow}
            \textrm{C}_{6}\textrm{H}_{12}\textrm{O}_{6}
            +6\textrm{O}_{2}
        \end{equation}
        This is a nice balanced equation with the same number
        of oxygen, hydrogen, and carbon atoms on either side.
        In plain English, this says the following: Given
        6 molecules of carbon dioxide, 6 molecules of water,
        and enough sunlight, you can then produce
        1 molecule of glucose and 6 molecules of
        diatomic oxygen ($\textrm{O}_{2}$). The glucose
        and oxygen produced in this process is important for
        \textit{consumers}. Consumers are organisms that
        eat producers or other consumers.
        For example, animals are consumers and feed on
        plants, fungi, or other animals. Muscle cells
        found in many animals require glucose as energy for
        movement. Many of these processes described involve
        converting chemical or potential energy into
        kinetic energy. To make the physicists happy,
        not all of the energy can be converted perfectly
        from one form to another, and there is always some
        loss to \textit{heat}. The curious one may want to
        research the \textit{First Law of Thermodynamics}.
        Another theme in biology is that of structure and
        function. The thin and flat shape of a leaf allows
        it to maximize the amount of sunlight that is
        captured by its chloroplasts. The shape of a birds
        wings and the ways its bones are structured allow
        for flight. All structure fits some sort of function.
        At the bottom of the structural hierarchy
        is the cell. This is the smallest thing that
        can still be considered living,
        and can perform all of the
        activities required for life. While there are
        certainly smaller things in the universe:
        Molecules, atoms, protons, and so forth, none of
        these can be considered as living things.
        Macroscopic actions, such as moving ones arm,
        require a combined effort of billions or trillions
        of individual cells. As such, understanding how
        cells work is the very foundation of biology
        (Along with Evolution). Certain characteristics
        are shared by all cells. For example, all cells have
        a membrane that encloses it and regulates what may
        pass into and out of the cell. Another shared trait
        is that all cells contain DNA as their genetic
        information. Cells can be broken up into two main
        groups: Prokaryotic and Eukaryotic Cells.
        Prokaryotic cells comprise archea and bacteria,
        whereas eukaryotic cells compose all other forms of
        life, including plants, animals, and fungi.
        \par\hfill\par
        A \textbf{eukaryotic cell} can be subdivided into
        internal membranes into various
        \textit{organelles}. These organelles are also
        enclosed by membranes. The defining characteristic
        of a eukaryotic cell is the existence of a
        \textit{nucleus}. For most eukaryotic cells, this
        is the largest organelle, and is where the
        cell's DNA lies. The region between the
        nucleus and the cell membrane is called the
        \textit{cytoplasm}. The cytoplasm is where all of
        the other organelles lie. We've already come
        across some organelles, such as the chloroplast
        that are found in plant cells.
        \textbf{Prokaryotic} cells are simpler and often
        smaller than eukaryotic cells. The DNA
        in a prokaryotic cell is not separated from the
        rest of the cell by any membrane. That is,
        prokaryotic cells lack a cell nucleus.
        \par\hfill\par
        The ability of cells to divide and form new cells
        is the basis for reproduction and growth. This
        also allows for multi-cellular organisms to
        repair damages tissue. Most of the DNA of a cell
        is contained in \textbf{chromosomes}. DNA itself
        is an acronym for
        DeoxyriboNucleic Acid. This is the substance of
        genes which allows for the transportation of
        information from parent to offspring. For example,
        your eye color, blood type, etc., are the result
        of certain genes inherited from your parents.
        Each chromosome contains a very long DNA molecule,
        which contains hundreds of genes. These in turn
        hold the information need to build other molecules
        in the cell, such as protein. DNA repliactes as
        a cell prepares to divide, and each of the two
        cellular offspring inherit a complete set
        of genes identical to the parent cell. DNA
        serves as a central database and controls the
        development and maintenance of the entire
        organism. Each DNA molecule is made up of
        two long chains, called strands. These strands
        are arranged in a double helix. These chains are
        made up of four kinds of chemicals called
        adenine, cytosine, guanine, and thymine. These
        are often abbreviated as A, T, C, and G. The
        way these four are combined determine what is
        to be built. This is analogous with language.
        The words rat, tar, and art are comprised of the
        same three letters, yet their order changes the
        overall meaning. The sequential arrangement of
        these four can be hundreds or thousands of
        \textit{nucleotides} long. DNA details how to make
        proteins, which are then used for building and
        maintaining cells. Enzymes that speed up
        (or catalyze) various processes are made up
        of proteins and are crucial to all cells. Protein
        production is controlled indirectly by the DNA
        of genes using RNA. A sequence of nucleotides
        along a gene are transcribed with RNA, and then form
        a specific protein with a specific function. This
        process is called
        \textbf{gene expression}. All life uses the same
        genetic code. One sequence says the same thing in
        one organism as another. The differences in the
        organisms arise from differences in their
        nucleotide sequences.
        \par\hfill\par
        The set of genetic instructions that an
        organism inherits is called its \textbf{genome}.
        Most human cells contains two similar sets of
        chromosomes, each set having about
        three billion nucleotide pairs. The Human Genome
        Project was a massive effort in the 1990's
        and early 2000's to find map out the entire
        human genome. In April, 2003 the project was
        completed and now the human genome is indeed
        known. After this effort was completed,
        researchers diverted their attention to
        studying the DNA and proteins of other organisms
        and comparing them. Rather than studying a
        single gene, research study sets of genes and
        compare between species. This is the study
        of \textbf{genomics}. These studies have been made
        possible by advancing technology and
        \textbf{bioinformatics}, which combines biology,
        computers, and mathematics.
        \par\hfill\par
        As stated before, proteins that catalyze chemical
        processes are called enzyme. Each enzyme
        catalyzes a specific process. The ability of a
        biological process to self-regulate is called
        feedback. The output, or product, regulates the
        process. A common form is that of
        \textbf{negative feedback}. In this, the
        accumulation of an end product of a process
        slows down the process. For example in animal
        cells when sugar is broken down there is
        chemical energy produced in the form of ATP. When
        there is too much ATP then the cell can use,
        the excess \textit{feeds back} and inhibits
        an enzyme, slowing or halting production. On the
        other hand, \textbf{positive feedback} speeds up
        a process. Blood clotting in a human is such
        an example. Platelets in blood accumulate at the
        damaged site, mending the cut. The effect is
        \textit{more} platelets are needed, and thus more
        are produced. 
        \par\hfill\par
        The final idea to reintroduce is that of evolution.
        This is the core theme behind all of biology.
        Life itself has been evolving for billions
        of years, starting from single celled organisms
        and transforming, ever so slowly, into
        immensely complex beings like mammals. The theme
        of evolution allows for living organisms to be
        broken down into very specific pieces, such as
        domains, kingdoms, phylums, and so on, all the way
        down to species. While sea horses and rabbits may
        look very different, they have the same basic
        skeletal structure. We thus group them together
        into the same \textit{phylum} called
        chordates (Or \textit{chordata}). Humans too are
        chordates. Meanwhile, a rabbit and a bear seam
        more similar than a rabbit and a sea horse. We
        categorize this further by placing bears and
        rabbits into the \textit{class} mammals
        (Or \textit{mammalia}). All mammals are chordates,
        but not all chordates are mammals. Mammals can
        be classified by their four-chambered hearts,
        kidney and liver structures, and a complex brain.
        Another (almost universal) feature is that of
        hair and live birth, though there are a few
        exceptions to this rule.
        \par\hfill\par
        The classification of species extends further.
        Tigers and bears appear more similar than
        bears and rabbits, and indeed lie in the same
        \textit{order}: Carnivora. A polar bear and a
        panda bear have even more similar characterstics
        and lie in the \textit{Family}: Ursidaye.
        Finally, grizzly bears and polar bears are nearly
        identical and lie in the same \textit{Genus}:
        Ursus. The very bottom of the classification is
        \textit{species}. Extending in the other direction,
        we may talb about \textit{kingdom} and
        \textit{domain}. All of the examples mentioned here
        lie in the kingdom \textit{animalia}, and
        all animals lie in the domain
        \textit{eukarya}. The study of grouping various
        organisms into different groups is called
        \textbf{taxonomy}.
        \par\hfill\par
        All species were once broken up into various
        \textit{kingdoms}. There has been a change to
        include a new classification \textit{above}
        kingdom called \textbf{domain}. This new method
        of grouping has come in the wake of comparisons
        of DNA sequences of different organisms. The
        proposal ranges from six to dozens of kingdoms,
        but in the domain level there is a consensus that
        there are three domains. The prokaryotic domains
        are \textbf{bacteria} and
        \textbf{archea}. The eukaryotic organisms are
        grouped into the domain
        \textbf{eukarya}. The split of bacteria and archea
        into two domains came as evidence that the two
        are equally different from each other as they are
        from eukarya was discovered. Bacteria are the most
        diverse and widespread prokaryotes and are
        sub-divided into various kingdoms.
        Archea live in the most extreme environments of
        Earth and can be found in salty lakes and hot
        springs. Eukarya is the most familiar domain, and
        includes the three most intuitive kingdoms:
        Plantae, Fungi, and Animalia. That is,
        plants, fungus, and animals. Each of these three
        kingdoms consists solely of multi-cellular
        organisms and are differentiated by means of their
        energy sources. Plants produces their own energy
        by means of photosynthesis. Fungi absorb
        dissolved nutrients from their surroundings.
        Some do this by decomposing dead matter and
        waste. Animals obtain their energy via
        digestion; they eat things. This can further be
        broken down by \textit{what} they eat,
        herbivores eat plants, carnivores eat other animals,
        and so on. There is a fourth kingdom of
        eukarya called \textbf{protista}, which
        consist of single-celled eukaryotic organisms
        called protists. Protists are very complicated and
        recent evidence suggests that some are more closely
        related to plants and fungi than they are to
        other protista. As such there has been a trend
        of sub-dividing protista into other kingdoms.
\end{document}