This work contains mathematics and physics. There are no exercises, but rather
an abundance of worked out examples, and an attempt was made to prove
every claim in a logical and consistent order, mimicing (but not replicating,
of course) Euclid's \textit{Elements}. That is, every line in a proof should be
justified by axiom, definition, or a previous theorem. There are no logical
prerequesites to read the theorems and proofs, but the examples often presume
a belief in the existence of real numbers (in particular, the non-negative
integers and rational numbers). Some early examples also use Calculus and the
elementary algebra of a polynomial in one real variable before these concepts
are formally introduced (indeed, calculus is not developed until
Book~\ref{book:Analysis}!). Theorems and proofs do not rely on examples, and in
this sense there are no prerequesites. A reader lacking calculus will simply
find no motivation in many definitions and axioms.