\documentclass[crop=false,class=book,oneside]{standalone}
%----------------------------Preamble-------------------------------%
%---------------------------Packages----------------------------%
\usepackage{geometry}
\geometry{b5paper, margin=1.0in}
\usepackage[T1]{fontenc}
\usepackage{graphicx, float}            % Graphics/Images.
\usepackage{natbib}                     % For bibliographies.
\bibliographystyle{agsm}                % Bibliography style.
\usepackage[french, english]{babel}     % Language typesetting.
\usepackage[dvipsnames]{xcolor}         % Color names.
\usepackage{listings}                   % Verbatim-Like Tools.
\usepackage{mathtools, esint, mathrsfs} % amsmath and integrals.
\usepackage{amsthm, amsfonts, amssymb}  % Fonts and theorems.
\usepackage{tcolorbox}                  % Frames around theorems.
\usepackage{upgreek}                    % Non-Italic Greek.
\usepackage{fmtcount, etoolbox}         % For the \book{} command.
\usepackage[newparttoc]{titlesec}       % Formatting chapter, etc.
\usepackage{titletoc}                   % Allows \book in toc.
\usepackage[nottoc]{tocbibind}          % Bibliography in toc.
\usepackage[titles]{tocloft}            % ToC formatting.
\usepackage{pgfplots, tikz}             % Drawing/graphing tools.
\usepackage{imakeidx}                   % Used for index.
\usetikzlibrary{
    calc,                   % Calculating right angles and more.
    angles,                 % Drawing angles within triangles.
    arrows.meta,            % Latex and Stealth arrows.
    quotes,                 % Adding labels to angles.
    positioning,            % Relative positioning of nodes.
    decorations.markings,   % Adding arrows in the middle of a line.
    patterns,
    arrows
}                                       % Libraries for tikz.
\pgfplotsset{compat=1.9}                % Version of pgfplots.
\usepackage[font=scriptsize,
            labelformat=simple,
            labelsep=colon]{subcaption} % Subfigure captions.
\usepackage[font={scriptsize},
            hypcap=true,
            labelsep=colon]{caption}    % Figure captions.
\usepackage[pdftex,
            pdfauthor={Ryan Maguire},
            pdftitle={Mathematics and Physics},
            pdfsubject={Mathematics, Physics, Science},
            pdfkeywords={Mathematics, Physics, Computer Science, Biology},
            pdfproducer={LaTeX},
            pdfcreator={pdflatex}]{hyperref}
\hypersetup{
    colorlinks=true,
    linkcolor=blue,
    filecolor=magenta,
    urlcolor=Cerulean,
    citecolor=SkyBlue
}                           % Colors for hyperref.
\usepackage[toc,acronym,nogroupskip,nopostdot]{glossaries}
\usepackage{glossary-mcols}
%------------------------Theorem Styles-------------------------%
\theoremstyle{plain}
\newtheorem{theorem}{Theorem}[section]

% Define theorem style for default spacing and normal font.
\newtheoremstyle{normal}
    {\topsep}               % Amount of space above the theorem.
    {\topsep}               % Amount of space below the theorem.
    {}                      % Font used for body of theorem.
    {}                      % Measure of space to indent.
    {\bfseries}             % Font of the header of the theorem.
    {}                      % Punctuation between head and body.
    {.5em}                  % Space after theorem head.
    {}

% Italic header environment.
\newtheoremstyle{thmit}{\topsep}{\topsep}{}{}{\itshape}{}{0.5em}{}

% Define environments with italic headers.
\theoremstyle{thmit}
\newtheorem*{solution}{Solution}

% Define default environments.
\theoremstyle{normal}
\newtheorem{example}{Example}[section]
\newtheorem{definition}{Definition}[section]
\newtheorem{problem}{Problem}[section]

% Define framed environment.
\tcbuselibrary{most}
\newtcbtheorem[use counter*=theorem]{ftheorem}{Theorem}{%
    before=\par\vspace{2ex},
    boxsep=0.5\topsep,
    after=\par\vspace{2ex},
    colback=green!5,
    colframe=green!35!black,
    fonttitle=\bfseries\upshape%
}{thm}

\newtcbtheorem[auto counter, number within=section]{faxiom}{Axiom}{%
    before=\par\vspace{2ex},
    boxsep=0.5\topsep,
    after=\par\vspace{2ex},
    colback=Apricot!5,
    colframe=Apricot!35!black,
    fonttitle=\bfseries\upshape%
}{ax}

\newtcbtheorem[use counter*=definition]{fdefinition}{Definition}{%
    before=\par\vspace{2ex},
    boxsep=0.5\topsep,
    after=\par\vspace{2ex},
    colback=blue!5!white,
    colframe=blue!75!black,
    fonttitle=\bfseries\upshape%
}{def}

\newtcbtheorem[use counter*=example]{fexample}{Example}{%
    before=\par\vspace{2ex},
    boxsep=0.5\topsep,
    after=\par\vspace{2ex},
    colback=red!5!white,
    colframe=red!75!black,
    fonttitle=\bfseries\upshape%
}{ex}

\newtcbtheorem[auto counter, number within=section]{fnotation}{Notation}{%
    before=\par\vspace{2ex},
    boxsep=0.5\topsep,
    after=\par\vspace{2ex},
    colback=SeaGreen!5!white,
    colframe=SeaGreen!75!black,
    fonttitle=\bfseries\upshape%
}{not}

\newtcbtheorem[use counter*=remark]{fremark}{Remark}{%
    fonttitle=\bfseries\upshape,
    colback=Goldenrod!5!white,
    colframe=Goldenrod!75!black}{ex}

\newenvironment{bproof}{\textit{Proof.}}{\hfill$\square$}
\tcolorboxenvironment{bproof}{%
    blanker,
    breakable,
    left=3mm,
    before skip=5pt,
    after skip=10pt,
    borderline west={0.6mm}{0pt}{green!80!black}
}

\AtEndEnvironment{lexample}{$\hfill\textcolor{red}{\blacksquare}$}
\newtcbtheorem[use counter*=example]{lexample}{Example}{%
    empty,
    title={Example~\theexample},
    boxed title style={%
        empty,
        size=minimal,
        toprule=2pt,
        top=0.5\topsep,
    },
    coltitle=red,
    fonttitle=\bfseries,
    parbox=false,
    boxsep=0pt,
    before=\par\vspace{2ex},
    left=0pt,
    right=0pt,
    top=3ex,
    bottom=1ex,
    before=\par\vspace{2ex},
    after=\par\vspace{2ex},
    breakable,
    pad at break*=0mm,
    vfill before first,
    overlay unbroken={%
        \draw[red, line width=2pt]
            ([yshift=-1.2ex]title.south-|frame.west) to
            ([yshift=-1.2ex]title.south-|frame.east);
        },
    overlay first={%
        \draw[red, line width=2pt]
            ([yshift=-1.2ex]title.south-|frame.west) to
            ([yshift=-1.2ex]title.south-|frame.east);
    },
}{ex}

\AtEndEnvironment{ldefinition}{$\hfill\textcolor{Blue}{\blacksquare}$}
\newtcbtheorem[use counter*=definition]{ldefinition}{Definition}{%
    empty,
    title={Definition~\thedefinition:~{#1}},
    boxed title style={%
        empty,
        size=minimal,
        toprule=2pt,
        top=0.5\topsep,
    },
    coltitle=Blue,
    fonttitle=\bfseries,
    parbox=false,
    boxsep=0pt,
    before=\par\vspace{2ex},
    left=0pt,
    right=0pt,
    top=3ex,
    bottom=0pt,
    before=\par\vspace{2ex},
    after=\par\vspace{1ex},
    breakable,
    pad at break*=0mm,
    vfill before first,
    overlay unbroken={%
        \draw[Blue, line width=2pt]
            ([yshift=-1.2ex]title.south-|frame.west) to
            ([yshift=-1.2ex]title.south-|frame.east);
        },
    overlay first={%
        \draw[Blue, line width=2pt]
            ([yshift=-1.2ex]title.south-|frame.west) to
            ([yshift=-1.2ex]title.south-|frame.east);
    },
}{def}

\AtEndEnvironment{ltheorem}{$\hfill\textcolor{Green}{\blacksquare}$}
\newtcbtheorem[use counter*=theorem]{ltheorem}{Theorem}{%
    empty,
    title={Theorem~\thetheorem:~{#1}},
    boxed title style={%
        empty,
        size=minimal,
        toprule=2pt,
        top=0.5\topsep,
    },
    coltitle=Green,
    fonttitle=\bfseries,
    parbox=false,
    boxsep=0pt,
    before=\par\vspace{2ex},
    left=0pt,
    right=0pt,
    top=3ex,
    bottom=-1.5ex,
    breakable,
    pad at break*=0mm,
    vfill before first,
    overlay unbroken={%
        \draw[Green, line width=2pt]
            ([yshift=-1.2ex]title.south-|frame.west) to
            ([yshift=-1.2ex]title.south-|frame.east);},
    overlay first={%
        \draw[Green, line width=2pt]
            ([yshift=-1.2ex]title.south-|frame.west) to
            ([yshift=-1.2ex]title.south-|frame.east);
    }
}{thm}

%--------------------Declared Math Operators--------------------%
\DeclareMathOperator{\adjoint}{adj}         % Adjoint.
\DeclareMathOperator{\Card}{Card}           % Cardinality.
\DeclareMathOperator{\curl}{curl}           % Curl.
\DeclareMathOperator{\diam}{diam}           % Diameter.
\DeclareMathOperator{\dist}{dist}           % Distance.
\DeclareMathOperator{\Div}{div}             % Divergence.
\DeclareMathOperator{\Erf}{Erf}             % Error Function.
\DeclareMathOperator{\Erfc}{Erfc}           % Complementary Error Function.
\DeclareMathOperator{\Ext}{Ext}             % Exterior.
\DeclareMathOperator{\GCD}{GCD}             % Greatest common denominator.
\DeclareMathOperator{\grad}{grad}           % Gradient
\DeclareMathOperator{\Ima}{Im}              % Image.
\DeclareMathOperator{\Int}{Int}             % Interior.
\DeclareMathOperator{\LC}{LC}               % Leading coefficient.
\DeclareMathOperator{\LCM}{LCM}             % Least common multiple.
\DeclareMathOperator{\LM}{LM}               % Leading monomial.
\DeclareMathOperator{\LT}{LT}               % Leading term.
\DeclareMathOperator{\Mod}{mod}             % Modulus.
\DeclareMathOperator{\Mon}{Mon}             % Monomial.
\DeclareMathOperator{\multideg}{mutlideg}   % Multi-Degree (Graphs).
\DeclareMathOperator{\nul}{nul}             % Null space of operator.
\DeclareMathOperator{\Ord}{Ord}             % Ordinal of ordered set.
\DeclareMathOperator{\Prin}{Prin}           % Principal value.
\DeclareMathOperator{\proj}{proj}           % Projection.
\DeclareMathOperator{\Refl}{Refl}           % Reflection operator.
\DeclareMathOperator{\rk}{rk}               % Rank of operator.
\DeclareMathOperator{\sgn}{sgn}             % Sign of a number.
\DeclareMathOperator{\sinc}{sinc}           % Sinc function.
\DeclareMathOperator{\Span}{Span}           % Span of a set.
\DeclareMathOperator{\Spec}{Spec}           % Spectrum.
\DeclareMathOperator{\supp}{supp}           % Support
\DeclareMathOperator{\Tr}{Tr}               % Trace of matrix.
%--------------------Declared Math Symbols--------------------%
\DeclareMathSymbol{\minus}{\mathbin}{AMSa}{"39} % Unary minus sign.
%------------------------New Commands---------------------------%
\DeclarePairedDelimiter\norm{\lVert}{\rVert}
\DeclarePairedDelimiter\ceil{\lceil}{\rceil}
\DeclarePairedDelimiter\floor{\lfloor}{\rfloor}
\newcommand*\diff{\mathop{}\!\mathrm{d}}
\newcommand*\Diff[1]{\mathop{}\!\mathrm{d^#1}}
\renewcommand*{\glstextformat}[1]{\textcolor{RoyalBlue}{#1}}
\renewcommand{\glsnamefont}[1]{\textbf{#1}}
\renewcommand\labelitemii{$\circ$}
\renewcommand\thesubfigure{%
    \arabic{chapter}.\arabic{figure}.\arabic{subfigure}}
\addto\captionsenglish{\renewcommand{\figurename}{Fig.}}
\numberwithin{equation}{section}

\renewcommand{\vector}[1]{\boldsymbol{\mathrm{#1}}}

\newcommand{\uvector}[1]{\boldsymbol{\hat{\mathrm{#1}}}}
\newcommand{\topspace}[2][]{(#2,\tau_{#1})}
\newcommand{\measurespace}[2][]{(#2,\varSigma_{#1},\mu_{#1})}
\newcommand{\measurablespace}[2][]{(#2,\varSigma_{#1})}
\newcommand{\manifold}[2][]{(#2,\tau_{#1},\mathcal{A}_{#1})}
\newcommand{\tanspace}[2]{T_{#1}{#2}}
\newcommand{\cotanspace}[2]{T_{#1}^{*}{#2}}
\newcommand{\Ckspace}[3][\mathbb{R}]{C^{#2}(#3,#1)}
\newcommand{\funcspace}[2][\mathbb{R}]{\mathcal{F}(#2,#1)}
\newcommand{\smoothvecf}[1]{\mathfrak{X}(#1)}
\newcommand{\smoothonef}[1]{\mathfrak{X}^{*}(#1)}
\newcommand{\bracket}[2]{[#1,#2]}

%------------------------Book Command---------------------------%
\makeatletter
\renewcommand\@pnumwidth{1cm}
\newcounter{book}
\renewcommand\thebook{\@Roman\c@book}
\newcommand\book{%
    \if@openright
        \cleardoublepage
    \else
        \clearpage
    \fi
    \thispagestyle{plain}%
    \if@twocolumn
        \onecolumn
        \@tempswatrue
    \else
        \@tempswafalse
    \fi
    \null\vfil
    \secdef\@book\@sbook
}
\def\@book[#1]#2{%
    \refstepcounter{book}
    \addcontentsline{toc}{book}{\bookname\ \thebook:\hspace{1em}#1}
    \markboth{}{}
    {\centering
     \interlinepenalty\@M
     \normalfont
     \huge\bfseries\bookname\nobreakspace\thebook
     \par
     \vskip 20\p@
     \Huge\bfseries#2\par}%
    \@endbook}
\def\@sbook#1{%
    {\centering
     \interlinepenalty \@M
     \normalfont
     \Huge\bfseries#1\par}%
    \@endbook}
\def\@endbook{
    \vfil\newpage
        \if@twoside
            \if@openright
                \null
                \thispagestyle{empty}%
                \newpage
            \fi
        \fi
        \if@tempswa
            \twocolumn
        \fi
}
\newcommand*\l@book[2]{%
    \ifnum\c@tocdepth >-3\relax
        \addpenalty{-\@highpenalty}%
        \addvspace{2.25em\@plus\p@}%
        \setlength\@tempdima{3em}%
        \begingroup
            \parindent\z@\rightskip\@pnumwidth
            \parfillskip -\@pnumwidth
            {
                \leavevmode
                \Large\bfseries#1\hfill\hb@xt@\@pnumwidth{\hss#2}
            }
            \par
            \nobreak
            \global\@nobreaktrue
            \everypar{\global\@nobreakfalse\everypar{}}%
        \endgroup
    \fi}
\newcommand\bookname{Book}
\renewcommand{\thebook}{\texorpdfstring{\Numberstring{book}}{book}}
\providecommand*{\toclevel@book}{-2}
\makeatother
\titleformat{\part}[display]
    {\Large\bfseries}
    {\partname\nobreakspace\thepart}
    {0mm}
    {\Huge\bfseries}
\titlecontents{part}[0pt]
    {\large\bfseries}
    {\partname\ \thecontentslabel: \quad}
    {}
    {\hfill\contentspage}
\titlecontents{chapter}[0pt]
    {\bfseries}
    {\chaptername\ \thecontentslabel:\quad}
    {}
    {\hfill\contentspage}
\newglossarystyle{longpara}{%
    \setglossarystyle{long}%
    \renewenvironment{theglossary}{%
        \begin{longtable}[l]{{p{0.25\hsize}p{0.65\hsize}}}
    }{\end{longtable}}%
    \renewcommand{\glossentry}[2]{%
        \glstarget{##1}{\glossentryname{##1}}%
        &\glossentrydesc{##1}{~##2.}
        \tabularnewline%
        \tabularnewline
    }%
}
\newglossary[not-glg]{notation}{not-gls}{not-glo}{Notation}
\newcommand*{\newnotation}[4][]{%
    \newglossaryentry{#2}{type=notation, name={\textbf{#3}, },
                          text={#4}, description={#4},#1}%
}
%--------------------------LENGTHS------------------------------%
% Spacings for the Table of Contents.
\addtolength{\cftsecnumwidth}{1ex}
\addtolength{\cftsubsecindent}{1ex}
\addtolength{\cftsubsecnumwidth}{1ex}
\addtolength{\cftfignumwidth}{1ex}
\addtolength{\cfttabnumwidth}{1ex}

% Indent and paragraph spacing.
\setlength{\parindent}{0em}
\setlength{\parskip}{0em}
\graphicspath{{../images/}}   % Path to Image Folder.
%----------------------------GLOSSARY-------------------------------%
\makeglossaries
\loadglsentries{../glossary}
\loadglsentries{../acronym}
%--------------------------Main Document----------------------------%
\begin{document}
    \ifx\ifmathcourses\undefined
        \pagenumbering{roman}
        \title{Algebraic Topology}
        \author{Ryan Maguire}
        \date{\vspace{-5ex}}
        \maketitle
        \tableofcontents
        \listoffigures
        \clearpage
        \chapter{Algebraic Topology}
        %\markboth{}{ALGEBRAIC TOPOLOGY}
        %\setcounter{chapter}{1}
        \pagenumbering{arabic}
    \else
        \chapter{Algebraic Topology}
    \fi
    \section{A Review of Algebra}
        A \textbf{Group} is a set $G$ and a
        binary operation $*$ such
        that the following are true:
        \begin{enumerate}
            \item[G1] $\forall_{a,b,c\in{G}},%
                       a*(b*c)=(a*b)*c$
                      \hfill[Associativity]
            \item[G2] $\exists_{e\in{G}}\forall_{a\in{G}}:a*e=a$
                      \hfill[Existence of Right Identity]
            \item[G3] $\forall_{a\in{G}}\exists_{a^{-1}\in{G}}:%
                       a*a^{-1}=e$
                      \hfill[Existence of Right Inverse]
        \end{enumerate}
        Usually one sees $a*e=e*a=a$ and
        $a*a^{-1}=a^{-1}*a=e$, but this can be relaxed to just
        left or right and then one can prove they are both
        equivalent. We often write $(G,*)$ to denote a group.
        An \textit{Abelian Group} is a group $(G,*)$ such that:
        \begin{enumerate}
            \item[G4] $\forall_{a,b\in{G}},a*b=b*a$
                      \hfill[Commutativity]
        \end{enumerate}
        A \textbf{Ring} is a set $R$ with
        two operations $+$ and $\cdot$, usually called
        addition and multiplication, respectively,
        with the following properties:
        \begin{enumerate}
            \item[R1] $(R,+)$ is an Abelian Group.
            \item[R2] $\forall_{a,b,c\in{R}},%
                       a\cdot(b\cdot{c})=(a\cdot{b})\cdot{c}$
                      \hfill[Associativity of Multiplication]
            \item[R3] $\forall_{a,b,c\in{R}},%
                       a\cdot(b+c)=(a\cdot{b})+(a\cdot{c})$
                      \hfill[Left-Distributive Law]
            \item[R4] $\forall_{a,b,c\in{R}}%
                       (b+c)\cdot{a}=(b\cdot{a})+(c\cdot{a})$
                      \hfill[Right-Distributive Law]
        \end{enumerate}
        We write $(R,+,\cdot)$ to denote a ring.
        A \textbf{Ring with Identity}, or a ring with unity, is
        a ring $(R,+,\cdot)$ such that:
        \begin{enumerate}
            \item[R5] $\exists_{1\in{R}}\forall_{a\in{R}}:%
                       1\cdot{a}=a\cdot{1}=a$
                      \hfill[Multiplicative Identity]
        \end{enumerate}
        A \textbf{Commutative Ring} is a ring $(R,+,\cdot)$
        such that:
        \begin{enumerate}
            \item[R6] $\forall_{a,b\in{R}},a\cdot{b}=b\cdot{a}$
                      \hfill[Commutativity of Multiplication]
        \end{enumerate}
        In a commutative ring, one can replace the
        Left-Distributive Law and the Right-Distributive Law with
        simply one Distributive Law. Commutativity implying
        they're equivalent. A $\textbf{Field}$ is a
        commutative ring with identity $(F,+,\cdot)$ such that:
        \begin{enumerate}
            \item[F1] $\forall_{a\in{F}, a\ne{0}}%
                       \exists_{a^{-1}\in{F}}:a\cdot{a^{-1}}=1$
                      \hfill[Multiplicative Inverse]
        \end{enumerate}
        One can prove some very intuitive results about the
        additive identity of a ring $R$.
        \begin{theorem}
            If $(R,+,\cdot)$ is a ring and $0$ is the additive
            identity of $R$, then for all $a\in{R}$,
            $a\cdot{0}=0$.
        \end{theorem}
        \begin{proof}
            For:
            \begin{align*}
                0&=a\cdot{0}-a\cdot{0}
                &
                &=a\cdot{0}+(a\cdot{0}-a\cdot{0})\\
                &=a\cdot(0+0)-a\cdot{0}
                &
                &=a\cdot{0}+0\\
                &=(a\cdot{0}+a\cdot{0})-a\cdot{0}
                &
                &=a\cdot{0}
            \end{align*}
        \end{proof}
        Thus, if one has a field $(F,+,\cdot)$, and if
        $0$ has a multiplicative inverse, then every element of
        $F$ is equal to $0$.
        \begin{theorem}
            If $(F,+,\cdot)$ is a field, $0$ is the additive
            identity, and if $0$ has a multiplicative inverse,
            then $F=\{0\}$.
        \end{theorem}
        \begin{proof}
            For $1=0\cdot{0}^{-1}$, but
            $0\cdot{0}^{-1}=0$, and thus $1=0$. But for all
            $a\in{F}$, $a=a\cdot{1}=a\cdot{0}=0$.
        \end{proof}
        Because of this, some require that $0\ne{1}$ in the
        definition of a field, and others call
        $F=\{0\}$ the trivial field.
        \begin{definition}
            A left module of a ring with identity $(R,+,\cdot)$
            is an abelian group $(M,+_{M})$ and a function
            $*:R\times{M}\rightarrow{M}$ such that:
            \begin{enumerate}
                \item $r*(a+_{M}b)=(r*a)+(r*b)$
                      \hfill[Left Scalar Distribution]
                \item $(r\cdot{s})*a=r*(s*a)$
                      \hfill[Scalar Associativity]
                \item $(r+s)*a=(r*a)+_{M}(s*a)$
                      \hfill[Right Module Distribution]
                \item $1*a=a$
                      \hfill[Identity Element]
            \end{enumerate}
        \end{definition}
        An attempt has been made to preserve the differences
        between the various operations in a left module.
        $+$ and $\cdot$ are binary operations that act on
        elements of $R$. That is, for $a,b\in{R}$, $a+b$
        gives another element of $R$, as does $a\cdot{b}$.
        However, $+_{M}$ is a binary operation over
        $M$. If $a,b\in{R}$, $a+_{M}b$ has no meaning.
        For $a,b\in{M}$, $a+_{M}b$ is well defined, and returns
        another element of $M$. The ``function,'' $*$
        takes an ordered pair $(r,a)$, where $r\in{R}$ and
        $a\in{M}$, and returns another element in $M$. For
        convenience we write $r*a$. If $a,b\in{R}$, then
        $a*b$ has no meaning, and if $a,b\in{M}$ then
        $a*b$ also has no meaning. Usually this is very
        unimportant, and $+$ and $+_{M}$ are given the same
        symbol, as are $\cdot$ and $*$. We can then more loosely
        rewrite the definition as, for all $r,s\in{R}$, and
        all $a,b\in{M}$:
        \begin{enumerate}
            \begin{multicols}{2}
                \item $r(a+b)=ra+rb$
                \item $(rs)(a)=r(sa)$
                \item $(r+s)(a)=(ra)+(rs)$
                \item $1a=a$
            \end{multicols}
        \end{enumerate}
        This is the more natural notation one finds when defining
        vector spaces. A module is analogous to a vector space:
        In a vector space one has a set $V$ and a
        \textit{field} $K$, whereas in a module one has a set
        $M$ and a \textit{ring with identity} $R$.
        \begin{theorem}
            If $(G,+)$ is an Abelian group, then there is a
            function $*:\mathbb{Z}\times{G}\rightarrow{G}$
            such that $(G,+_{G})$ is a left module of
            $(\mathbb{Z},+,\cdot)$, where $+$ and $\cdot$ are
            the standard arithmetic operations over $\mathbb{Z}$.
        \end{theorem}
        \begin{proof}
            For define $0*a=e$ and $1*a=a$ for all $a\in{G}$,
            and for all
            $n\in\mathbb{Z}$, $n>1$ inductively define
            $(n+1)*a=n*a+_{G}a$. For $n<0$ define
            $n*a=((-n)*a)^{-1}$, where the inverse is taken
            with respect to the group $G$. Then $*$ is a function
            $*:\mathbb{Z}\times{G}\rightarrow{G}$. If $n>0$, we
            have the following:
            \begin{align*}
                (n*a)+_{G}(n*b)
                &=\underset{n}{\underbrace{(a+_{G}\cdots+_{G}a)}}
                +_{G}
                \underset{n}{\underbrace{(b+_{G}\cdots+_{G}b)}}\\
                &=\underset{n}
                    {\underbrace{(a+b)+_{G}\cdots+_{G}(a+b)}}\\
                &=n*(a+b)
            \end{align*}
            If $n,m>0$, we have:
            \begin{equation*}
                n*a+_{G}m*a
                =\underset{n+m}{\underbrace{a+_{G}\cdots+_{G}a}}
                =(n+m)*a
            \end{equation*}
            And finally:
            \begin{equation*}
                (n\cdot{m})*a=
                \underset{n\cdot{m}}
                    {\underbrace{a+_{G}\cdots+_{G}a}}
                =n*(\underset{m}{\underbrace{a+_{G}\cdots+_{G}a}})
                =n*(m*a)
            \end{equation*}
            Similarly for when $n,m<0$, $n<0<m$, or $m<0<n$.
        \end{proof}
        Thus, every Abelian group $(G,+_{G})$ can be seen
        as a left module over $(R,+,\cdot)$. Moreover the
        function $*$ is unique, so this correspondence is
        unique as well. As another example, every vector space
        $V$ over a field $K$ is a left module over $K$, since
        any field $K$ is also a ring with identity.
        A \textbf{Left Ideal} of a ring $(R,+,*)$ is a subset
        $I\subseteq{R}$ such that:
        \begin{enumerate}
            \item $\forall_{a,b\in{I}},a+b\in{I}$
            \item $\forall_{r\in{R}}\forall_{a\in{I}},%
                   r\cdot{a}\in{I}$
        \end{enumerate}
        This can be rephrased by saying that $(I,+)$ is a subgroup
        of $(R,+)$, and $I$ absorbs left-multiplication. A
        \textbf{Right Ideal} replaces $r\cdot{a}$ with
        $a\cdot{r}$. An \textbf{Ideal} or \textbf{Two-Sided Ideal}
        is a subset that is both a left and a right ideal.
        \begin{theorem}
            If $(R,+,\cdot)$ is a ring with identity
            and $(I,+)$ is a left ideal
            of $R$, then there is a function
            $*:R\times{I}\rightarrow{I}$ such that
            $(I,+)$ is a left module over $R$.
        \end{theorem}
        \begin{proof}
            For let $*$ be the restriction of $\cdot$ to
            $R\times{I}$. Then, for all $a\in{I}$,
            $1\cdot{a}=a$. If $a,b\in{I}$ and $r\in{R}$, then:
            \begin{equation*}
                r*(a+b)
                =r\cdot(a+b)
                =(r\cdot{a})+(r\cdot{b})
            \end{equation*}
            If $r,s\in{R}$ and $a\in{I}$, then:
            \begin{gather*}
                (r\cdot{s})*a
                =(r\cdot{s})\cdot{a}
                =r\cdot(s\cdot{a})\\
                (r+s)*(a)
                =(r+s)\cdot{a}
                =(r\cdot{a})+(s\cdot{a})
            \end{gather*}
        \end{proof}
        \begin{definition}
            The Annihilator of a Left Module $(M,+_{M})$ over
            a ring with identity $(R,+,\cdot)$ is the set:
            \begin{equation*}
                I=\{r\in{R}:\forall_{m\in{M}},r*m=0\}
            \end{equation*}
        \end{definition}
        \begin{theorem}
            If $(M,+_{M})$ is a Left Module over a ring with
            identity $(R,+,\cdot)$, and if $I$ is the
            annihilator of $M$, then $I$ is a two-sided
            ideal of $R$.
        \end{theorem}
        \begin{proof}
            For if $r,s\in{I}$, then for all $m\in{M}$,
            $r*m=0$ and $s*m=0$. But $(r+s)*m=(r*m)+_{M}(s*m)=0$.
            Therefore $r+s\in{I}$. If $r\in{R}$ and $s\in{I}$,
            then $(r\cdot{s})*m=r*(s*m)=r*0=0$, and therefore
            $r\cdot{s}\in{I}$. Furthermore,
            $(s\cdot{r})*m=s*(r*m)=0$, and thus $s\cdot{r}\in{I}$.
            Therefore $I$ is a two-sided ideal.
        \end{proof}
        
    \section{The Fundamental Group}
    \section{Homology}
    \section{Cohomology}
    \section{Homotopy}
\end{document}