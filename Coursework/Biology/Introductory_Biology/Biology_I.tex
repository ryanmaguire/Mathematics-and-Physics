\documentclass[crop=false,class=book,oneside]{standalone}
%----------------------------Preamble-------------------------------%
%---------------------------Packages----------------------------%
\usepackage{geometry}
\geometry{b5paper, margin=1.0in}
\usepackage[T1]{fontenc}
\usepackage{graphicx, float}            % Graphics/Images.
\usepackage{natbib}                     % For bibliographies.
\bibliographystyle{agsm}                % Bibliography style.
\usepackage[french, english]{babel}     % Language typesetting.
\usepackage[dvipsnames]{xcolor}         % Color names.
\usepackage{listings, lstlinebgrd}      % Verbatim-Like Tools.
\usepackage{mathtools, esint, mathrsfs} % amsmath and integrals.
\usepackage{amsthm, amsfonts}           % Fonts and theorems.
\usepackage{tabularx}
\usepackage{tcolorbox}                  % Frames around theorems.
\usepackage{upgreek}                    % Non-Italic Greek.
\usepackage{paracol}                    % Two-column styling.
\usepackage{wrapfig}                    % Wrap text around figure.
\usepackage{fmtcount, etoolbox}         % For the \book{} command.
\usepackage[newparttoc]{titlesec}       % Formatting chapter, etc.
\usepackage{titletoc}                   % Allows \book in toc.
\usepackage[nottoc]{tocbibind}          % Bibliography in toc.
\usepackage[titles]{tocloft}            % ToC formatting.
\usepackage{multicol, enumitem}         % Multi-column/enumerate.
\usepackage{import}                     % Import external files.
\usepackage{pgfplots, tikz}             % Drawing/graphing tools.
\usetikzlibrary{
    calc,                   % Calculating right angles and more.
    angles,                 % Drawing angles within triangles.
    arrows.meta,            % Latex and Stealth arrows.
    quotes,                 % Adding labels to angles.
    positioning,            % Relative positioning of nodes.
    decorations.markings,   % Adding arrows in the middle of a line.
    patterns,
    arrows,
    shapes,
    shapes.geometric,
    cd,
    hobby,
    babel
}                                       % Libraries for tikz.
\pgfplotsset{compat=1.9}                % Version of pgfplots.
\usepackage[font=scriptsize,
            labelformat=simple,
            labelsep=colon]{subcaption} % Subfigure captions.
\usepackage[font={scriptsize},
            hypcap=true,
            labelsep=colon]{caption}    % Figure captions.
\usepackage{hyperref}                   % Allows for hyperlinks.
\hypersetup{
    colorlinks=true,
    linkcolor=blue,
    filecolor=magenta,
    urlcolor=Cerulean,
    citecolor=SkyBlue
}                           % Colors for hyperref.
\usepackage[toc,acronym,nogroupskip]{glossaries} % Glossaries and acronyms.
\usepackage[subpreambles=false]{standalone}      % Complileable sub files.

% Various font stuff from kiwi.
% Use this for Times text and Computer Modern math
%\usepackage{times}

% Quite nice
%\usepackage[charter, greekfamily=, greekuppercase=italicized]{mathdesign}
%\usepackage[utopia, greekuppercase=italicized]{mathdesign}    % Math is narrower

% Use this for Times text and math
%\usepackage{newtxtext}
%\usepackage[libertine,cmintegrals]{newtxmath}
%\usepackage{fix-cm}

%\usepackage{txfontsb}
% or
%\usepackage{mathptmx}

%\usepackage[scaled=0.92]{helvet}
%\renewcommand{\rmdefault}{ptm}

%\usepackage{mathpazo}    % add possibly `sc` and `osf` options
%\usepackage{eulervm}

%\usepackage{fourier}
%\renewcommand{\rmdefault}{ptm}
%\usepackage{mathptm}

%\usepackage{fontspec}
%\setmainfont{lmodern}

%\usepackage[varg]{txfonts}
%\usepackage{fouriernc}
%\usepackage{mathpazo}

%\usepackage{bookman}
%\usepackage[scaled]{uarial}
%\usepackage[scaled]{helvet}
%\renewcommand*\familydefault{\sfdefault}
%\usepackage[math]{anttor}

%\newcommand\fgeorgia{\fontfamily{jvn}\selectfont}
%\newcommand\ftimes{\fontfamily{ptm}\selectfont}
%\newcommand\fhelvetica{\fontfamily{phv}\selectfont}
%\newcommand\fcourier{\fontfamily{pcr}\selectfont}
%\newcommand\fbookman{\fontfamily{pbk}\selectfont}
%\newcommand\fnewcentury{\fontfamily{pnc}\selectfont}
%\newcommand\fpalatino{\fontfamily{ppl}\selectfont}
%\newcommand\favantgarde{\fontfamily{pag}\selectfont}
%\newcommand\fnormal{\normalfont}
%\newcommand\fsize[1]{\ifnum#1>0\fontsize{#1}{#1}\selectfont\else\normalsize\fi}
%------------------------Theorem Styles-------------------------%
% Define theorem style for default spacing and normal font.
\newtheoremstyle{normal}
    {\topsep}               % Amount of space above the theorem.
    {\topsep}               % Amount of space below the theorem.
    {}                      % Font used for body of theorem.
    {}                      % Measure of space to indent.
    {\bfseries}             % Font of the header of the theorem.
    {}                      % Punctuation between head and body.
    {.5em}                  % Space after theorem head.
    {}

% Define theorem style for default spacing with italicized font.
\newtheoremstyle{normalit}{\topsep}{\topsep}
                {\itshape}{}{\bfseries}{}{.5em}{}

% Italic header environment.
\newtheoremstyle{thmit}{\topsep}{\topsep}{}{}{\itshape}{}{0.5em}{}

% Define italicized environments.
\theoremstyle{normalit}
\newtheorem{theorem}{Theorem}[section]
\newtheorem{lemma}{Lemma}[section]
\newtheorem{corollary}{Corollary}[section]
\newtheorem{proposition}{Proposition}[section]
\newtheorem*{theorem*}{Theorem}

% Define environments with italic headers.
\theoremstyle{thmit}
\newtheorem*{solution}{Solution}
\newtheorem*{fsolution}{Solution}

% Define default environments.
\theoremstyle{normal}
\newtheorem{example}{Example}[section]
\newtheorem{definition}{Definition}[section]
\newtheorem{problem}{Problem}[section]
\newtheorem{question}{Question}[section]
\newtheorem{remark}{Remark}[section]
\newtheorem{properties}{Properties}[section]
\newtheorem{notation}{Notation}[section]
\newtheorem{axiom}{Axiom}[section]
\newtheorem*{properties*}{Properties}
\newtheorem*{remark*}{Remark}
\newtheorem*{definition*}{Definition}
\theoremstyle{plain}

% Define framed environment.
\tcbuselibrary{most}
\newtcbtheorem[use counter*=theorem]{ftheorem}{Theorem}%
    {colback=green!5,colframe=green!35!black,
     fonttitle=\bfseries\upshape}{th}

\newtcbtheorem[use counter*=example]{fdefinition}{Definition}%
    {fonttitle=\bfseries\upshape,
     colback=blue!5!white,colframe=blue!75!black}{def}

\newtcbtheorem[use counter*=example]{fexample}{Example}%
    {fonttitle=\bfseries\upshape,
     colback=red!5!white,colframe=red!75!black}{ex}

\newtcbtheorem[use counter*=notation]{fnotation}{Notation}%
    {fonttitle=\bfseries\upshape,
     colback=SeaGreen!5!white,colframe=SeaGreen!75!black}{ex}

\newtcbtheorem[use counter*=corollary]{fcorollary}{Corollary}%
    {fonttitle=\bfseries\upshape,
     colback=Orchid!5!white,colframe=Orchid!75!black}{ex}

\newenvironment{bproof}{\textit{Proof.}}{\hfill$\square$}
\tcolorboxenvironment{bproof}{blanker,breakable,left=5mm,
                             before skip=10pt,after skip=10pt,
                             borderline west={1mm}{0pt}{red}}
\tcolorboxenvironment{fsolution}
    {enhanced jigsaw,colframe=cyan,interior hidden,breakable}

%--------------------Declared Math Operators--------------------%
\DeclareMathOperator{\Refl}{Refl}           % Reflection operator.
\DeclareMathOperator{\Span}{Span}           % Span of a set of vectors.
\DeclareMathOperator{\Card}{Card}           % Cardinality of set.
\DeclareMathOperator{\Ord}{Ord}             % Ordinal of ordered set.
\DeclareMathOperator{\Tr}{Tr}               % Trace of matrix.
\DeclareMathOperator{\adjoint}{adj}         % Adjoint of matrix.
\DeclareMathOperator{\rk}{rk}               % Rank of operator.
\DeclareMathOperator{\nul}{nul}             % Null space of operator.
\DeclareMathOperator{\sgn}{sgn}             % Sign of a number.
\DeclareMathOperator{\multideg}{mutlideg}   % Multi-Degree (Graphs).
\DeclareMathOperator{\GCD}{GCD}             % Greatest common denominator.
\DeclareMathOperator{\LM}{LM}               % Leading monomial
\DeclareMathOperator{\LC}{LC}               % Leading coefficient.
\DeclareMathOperator{\LT}{LT}               % Leading term.
\DeclareMathOperator{\LCM}{LCM}             % Least common multiple.
\DeclareMathOperator{\Mon}{Mon}             % Monomial.
\DeclareMathOperator{\Spec}{Spec}           % Spectrum.
\DeclareMathOperator{\proj}{proj}           % Projection.
\DeclareMathOperator{\comp}{comp}           % Component.
\DeclareMathOperator{\sinc}{sinc}           % Sinc function.
\DeclareMathOperator{\Ima}{Im}              % Image of operator.
\DeclareMathOperator{\Prin}{Prin}           % Principal value.
\DeclareMathOperator{\Mod}{mod}             % Modulus.
%------------------------New Commands---------------------------%
\DeclarePairedDelimiter\norm{\lVert}{\rVert}
\DeclarePairedDelimiter\ceil{\lceil}{\rceil}
\DeclarePairedDelimiter\floor{\lfloor}{\rfloor}
\newcommand*\diff{\mathop{}\!\mathrm{d}}
\newcommand*\Diff[1]{\mathop{}\!\mathrm{d^#1}}
\renewcommand{\mod}{\ \Mod}
\renewcommand*{\glstextformat}[1]{\textcolor{RoyalBlue}{#1}}
\renewcommand{\glsnamefont}[1]{\textbf{#1}}
\renewcommand\labelitemii{$\circ$}
\renewcommand\thesubfigure{\arabic{chapter}.\arabic{figure}}
\renewcommand\thesubfigure{%
    \arabic{chapter}.\arabic{figure}.\arabic{subfigure}}
\addto\captionsenglish{\renewcommand{\figurename}{Fig.}}
%------------------------Book Command---------------------------%
\makeatletter
\renewcommand\@pnumwidth{1cm}
\newcounter{book}
\renewcommand\thebook{\@Roman\c@book}
\newcommand\book{%
    \if@openright
        \cleardoublepage
    \else
        \clearpage
    \fi
    \thispagestyle{plain}%
    \if@twocolumn
        \onecolumn
        \@tempswatrue
    \else
        \@tempswafalse
    \fi
    \null\vfil
    \secdef\@book\@sbook
}
\def\@book[#1]#2{%
    \ifnum \c@secnumdepth >-3\relax
        \refstepcounter{book}%
        \addcontentsline{toc}{book}{
            \bookname\ \thebook:\hspace{1em}#1
        }
    \else
        \addcontentsline{toc}{book}{#1}%
    \fi
    \markboth{}{}%
    {\centering
     \interlinepenalty \@M
     \normalfont
     \ifnum \c@secnumdepth >-2\relax
       \huge\bfseries \bookname\nobreakspace\thebook
       \par
       \vskip 20\p@
     \fi
     \Huge \bfseries #2\par}%
    \@endbook}
\def\@sbook#1{%
    {\centering
     \interlinepenalty \@M
     \normalfont
     \Huge \bfseries #1\par}%
    \@endbook}
\def\@endbook{
    \vfil\newpage
        \if@twoside
            \if@openright
                \null
                \thispagestyle{empty}%
                \newpage
            \fi
        \fi
        \if@tempswa
            \twocolumn
        \fi
}
\newcommand*\l@book[2]{%
    \ifnum \c@tocdepth >-2\relax
        \addpenalty{-\@highpenalty}%
        \addvspace{2.25em \@plus\p@}%
        \setlength\@tempdima{3em}%
        \begingroup
            \parindent \z@ \rightskip \@pnumwidth
            \parfillskip -\@pnumwidth
            {
                \leavevmode
                \Large \bfseries #1\hfil \hb@xt@\@pnumwidth{
                    \hss #2
                }
            }
            \par
            \nobreak
            \global\@nobreaktrue
            \everypar{\global\@nobreakfalse\everypar{}}%
        \endgroup
    \fi}
\newcommand\bookname{Book}
\renewcommand{\thebook}{\texorpdfstring{\Numberstring{book}}{book}}
\providecommand*{\toclevel@book}{-2}
\makeatother
\titlecontents{chapter}[0pt]
    {\bfseries}
    {\chaptername\ \thecontentslabel:\quad}
    {}
    {\hfill\contentspage}
\titleformat{\part}[display]
    {\Large\bfseries}
    {\partname\nobreakspace\thepart}
    {0mm}
    {\Huge\bfseries}
    \titlecontents{part}[0pt]
    {\large\bfseries}
    {\partname\ \thecontentslabel: \quad}
    {}
    {\hfill\contentspage}
\newcommand{\MarkRightAngle}[4][.3cm]
    {\coordinate (tempa) at ($(#3)!#1!(#2)$);
     \coordinate (tempb) at ($(#3)!#1!(#4)$);
     \coordinate (tempc) at ($(tempa)!0.5!(tempb)$);%midpoint
     \draw (tempa) -- ($(#3)!2!(tempc)$) -- (tempb);}
%--------------------------LENGTHS------------------------------%
% Spacings for the Table of Contents.
\addtolength{\cftsecnumwidth}{1ex}
\addtolength{\cftsubsecindent}{1ex}
\addtolength{\cftsubsecnumwidth}{1ex}
\addtolength{\cftfignumwidth}{1ex}
\addtolength{\cfttabnumwidth}{1ex}

% Spacing for multi-column and enumerate environments.
\setlength{\multicolsep}{6pt}
\setlist[enumerate]{itemsep=0pt,topsep=3pt}

% Indent and paragraph spacing.
\setlength{\parindent}{0em}
\setlength{\parskip}{0em}
\graphicspath{{../../../images/}}   % Path to Image Folder.
%--------------------------Main Document----------------------------%
\begin{document}
    \ifx\ifbiocourses\undefined
        \pagenumbering{roman}
        \title{Biology I}
        \author{Ryan Maguire}
        \date{\vspace{-5ex}}
        \maketitle
        \tableofcontents
        \listoffigures
        \clearpage
        \chapter*{Biology I}
        \addcontentsline{toc}{chapter}{Biology I}
        \markboth{}{BIOLOGY I}
        \setcounter{chapter}{1}
        \pagenumbering{arabic}
    \else
        \chapter{Introductory Biology I}
    \fi
    \section{Introduction}
        Living things often have to adapt to their
        environment in order to survive. One such example
        is that of the ghost plant, found in Northeast
        Mexico. Its strange leaves allow to absorb, store,
        and conserve water. Such adaptation are the
        result of \textbf{evolution}. Evolution is the
        process of change that life undergoes, and has been
        transforming living things since the first
        single-cell organisms appeared on Earth. Evolution
        is the back-bone behind all of
        \textbf{biology}, the study of life. The word
        \textit{life} itself needs some definition. While
        it is a very intuitive concept, it can be hard to
        describe in a consistent manner while encompassing
        all of the things we'd like to consider
        \textit{living} and excluding things that are
        \textit{non-living}. We can attempt to define
        what a living organism by imposing the following
        requirements:
        \begin{enumerate}
            \item \textbf{Order}: Organisms are made of
                  \textit{cells} and these cells arrange
                  themselves in complex manners to perform
                  basic tasks. For example, cardiac muscle
                  cells working together to form a heart.
            \item \textbf{Evolution}: All life evolves.
                  This process occurs by means of natural
                  selection. One such is example is the
                  evolution of the \textit{peppered moth}
                  during the industrial revolution in
                  England. The white bodied peppered moth,
                  or \textit{Biston betularia f. typica},
                  once thrived in pre-industrial evolution.
                  But as chimney stacks formed and soot
                  became common amongst the trees, their
                  bright colors made them easy picking
                  for predators. The black bodied
                  peppered moth, or
                  \textit{Biston betularia f. carbonaria},
                  thrived as it easily blended in to the
                  polluted trees and vegetation. There are
                  many examples of evolution occuring,
                  including that of the
                  pygmy sea horse using camouflage to
                  blend into its environment, hiding it
                  from predators.
            \item \textbf{Respond to External Stimuli}:
                  All living organisms respond
                  to the various stimuli
                  that are present in their environment.
                  For humans this may be a car horn
                  honking, or a cell phone ringing.
                  All forms of life
                  react to external stimuli in one way or
                  another.
            \item \textbf{Maintains Homeostasis}:
                  Homeostasis is a fancy word for a
                  regular internal ``environment,'' in
                  a living organism. For mammals this could
                  be the warm internal body temperatures
                  or a steady blood flow to various organs.
                  Mammals maintain homeostasis in
                  many ways.
                  Humans, for example, sweat when too hot
                  and shiver when too cold.
            \item \textbf{Reproduction}: All living things
                  reproduce. Reproduction can be very
                  different from the manner in which
                  mammals reproduce. There is both
                  sexual and asexual reproduction. Single
                  celled organism often reproduce via
                  asexual means, and all mammals
                  reproduce by sexual ones.
            \item \textbf{Energy Consumption}: Energy
                  consumption does not necessarily mean
                  eating other organisms. This is a
                  trait found in animals. Plants obtain
                  energy via
                  \textit{photosynthesis} and can convert
                  sunlight into useable energy.
            \item \textbf{Growth and Development}: Living
                  things grow. Humans start out as a
                  couple of cells called a zygote and
                  eventually grow into 70 trillion+
                  celled organism.
        \end{enumerate}
        The study of life extends from the level of
        microscopic cells ($10^{-6}\textrm{m}$) to the
        macroscopic biosphere ($10^{6}\textrm{m}$).
        The structure of cells is very important for
        life to prosper. As mentioned before plants
        obtain their energy in a process called
        \textit{photosynthesis}. This takes place
        in an \textit{organelle} called the
        \textit{chloroplast}. However,
        photosynthesis will not occur in a random
        mixture of chloroplasts and chlorophyll in
        a test-tube. The structure and organization of
        plant cells is crucial for photosynthesis to
        be successful. The organization of
        constituent components is an important aspect
        outside of biology as well. In chemistry one
        can take carbon and re-arrange in different manners
        to obtain different objects. Graphite and
        diamonds are different objects, but are both
        pure carbon. The lattice structure in these two
        objects is different, however. To better study
        life biologists often take the approach of
        \textit{reductionism}. This is the method of
        reducing complex systems into smaller and simpler
        components to ease the analysis. On the other hand
        one can study biology in a more hollistic approach.
        In biology, a system is a combination of
        components that work together to perform
        some function. A human can be considered as a
        system, with cells, blood, skin, and a plethora
        of complex organ systems working together to
        perform the basic tasks humans undergo.
        \textbf{Systems Biology} is the method of
        modelling and explaining phenomena in biology
        by means of studying a system's parts. There
        are ten main layers we wish to study in biology.
        \begin{enumerate}
            \item \textbf{Biosphere}: This is the Earth.
                  This includes the continents and oceans
                  that it is comprised of, the forests
                  and islands, and all of the organisms
                  that live on it.
            \item \textbf{Ecosystems}: Ecosystems are
                  smaller subsets of the biosphere. This
                  includes the Amazon rain forest, the
                  arctic tundras, and the Sahara desert.
                  There are many other kinds of ecosystems,
                  such as forests, grasslands, coral reefs,
                  and open oceans.
            \item \textbf{Communities}: Communities are the
                  entire collection of organisms that live
                  in a given ecosystems. For the arctic
                  this includes polar bears, reindeer,
                  beluga whales, walruses, and seals.
                  In the amazon it includes all of the
                  plants, bacteria, and animals that
                  one might find.
            \item \textbf{Populations}: The populations
                  of an ecosystems are the various
                  collections of a single species one
                  might find. All of the penguins in
                  antarctica make up one population, and
                  all of the sea lions make another.
            \item \textbf{Organisms}: Individual living
                  things are called organisms. A human
                  is an organism, as is a cat, a sunflower,
                  or a sequoia tree.
            \item \textbf{Organs/Organ Systems}:
                  Examples of organs in mammals include
                  a heart, brain, kidney, and liver. In
                  plants this includes the stem and
                  and roots. For humans, the integumentary
                  system (Or skin) also constitutes an
                  organ system.
            \item \textbf{Tissues}: Tissues are structures
                  of cells working together for a common
                  function. Most examples of tissue require
                  a microscope to see, and thus leave
                  the realm of intuition.
            \item \textbf{Cells}: Cells are the
                  fundamental building blocks of life.
                  Indeed, cells are part of the
                  definition of life. Cells are extremely
                  small, about 10 to 100 micrometers
                  ($\mu\textrm{m}$), and require a
                  microscope to examine. Some organisms,
                  such as bacteria, or amoebas, are
                  single celled. Others, such as humans,
                  have more than 70 trillion cells.
            \item \textbf{Organelles}: Organelles are the
                  pieces that make up a cell. In a plant
                  cell this may include the cell wall or
                  chloroplasts, and in animal cells this
                  may include ribosomes, mitochondria,
                  vacuoles, and more.
            \item \textbf{Molecules}: Molecules are
                  beyond the realm of \textrm{living},
                  and are comprised of chains of atoms.
                  Molecules are the principle concept of
                  study in \textit{chemistry}, and also
                  of substantial interets in
                  \textit{physics}. Chlorophyll is an
                  example of a molecule, a very long and
                  complex chain of different elements.
                  Water is another, simpler example,
                  containing three atoms: Two hydrogen and
                  one oxygen. That is,
                  $\textrm{H}_{2}O$.
        \end{enumerate}
        Ecosystems consist of both organisms and
        physical factors. Plants and animals
        are examples of organisms, and sunlight,
        carbon dioxide, and oxygen represent
        non-living physical
\end{document}