\documentclass[crop=false,class=article,oneside]{standalone}
%----------------------------Preamble-------------------------------%
%---------------------------Packages----------------------------%
\usepackage{geometry}
\geometry{b5paper, margin=1.0in}
\usepackage[T1]{fontenc}
\usepackage{graphicx, float}            % Graphics/Images.
\usepackage{natbib}                     % For bibliographies.
\bibliographystyle{agsm}                % Bibliography style.
\usepackage[french, english]{babel}     % Language typesetting.
\usepackage[dvipsnames]{xcolor}         % Color names.
\usepackage{listings}                   % Verbatim-Like Tools.
\usepackage{mathtools, esint, mathrsfs} % amsmath and integrals.
\usepackage{amsthm, amsfonts, amssymb}  % Fonts and theorems.
\usepackage{tcolorbox}                  % Frames around theorems.
\usepackage{upgreek}                    % Non-Italic Greek.
\usepackage{fmtcount, etoolbox}         % For the \book{} command.
\usepackage[newparttoc]{titlesec}       % Formatting chapter, etc.
\usepackage{titletoc}                   % Allows \book in toc.
\usepackage[nottoc]{tocbibind}          % Bibliography in toc.
\usepackage[titles]{tocloft}            % ToC formatting.
\usepackage{pgfplots, tikz}             % Drawing/graphing tools.
\usepackage{imakeidx}                   % Used for index.
\usetikzlibrary{
    calc,                   % Calculating right angles and more.
    angles,                 % Drawing angles within triangles.
    arrows.meta,            % Latex and Stealth arrows.
    quotes,                 % Adding labels to angles.
    positioning,            % Relative positioning of nodes.
    decorations.markings,   % Adding arrows in the middle of a line.
    patterns,
    arrows
}                                       % Libraries for tikz.
\pgfplotsset{compat=1.9}                % Version of pgfplots.
\usepackage[font=scriptsize,
            labelformat=simple,
            labelsep=colon]{subcaption} % Subfigure captions.
\usepackage[font={scriptsize},
            hypcap=true,
            labelsep=colon]{caption}    % Figure captions.
\usepackage[pdftex,
            pdfauthor={Ryan Maguire},
            pdftitle={Mathematics and Physics},
            pdfsubject={Mathematics, Physics, Science},
            pdfkeywords={Mathematics, Physics, Computer Science, Biology},
            pdfproducer={LaTeX},
            pdfcreator={pdflatex}]{hyperref}
\hypersetup{
    colorlinks=true,
    linkcolor=blue,
    filecolor=magenta,
    urlcolor=Cerulean,
    citecolor=SkyBlue
}                           % Colors for hyperref.
\usepackage[toc,acronym,nogroupskip,nopostdot]{glossaries}
\usepackage{glossary-mcols}
%------------------------Theorem Styles-------------------------%
\theoremstyle{plain}
\newtheorem{theorem}{Theorem}[section]

% Define theorem style for default spacing and normal font.
\newtheoremstyle{normal}
    {\topsep}               % Amount of space above the theorem.
    {\topsep}               % Amount of space below the theorem.
    {}                      % Font used for body of theorem.
    {}                      % Measure of space to indent.
    {\bfseries}             % Font of the header of the theorem.
    {}                      % Punctuation between head and body.
    {.5em}                  % Space after theorem head.
    {}

% Italic header environment.
\newtheoremstyle{thmit}{\topsep}{\topsep}{}{}{\itshape}{}{0.5em}{}

% Define environments with italic headers.
\theoremstyle{thmit}
\newtheorem*{solution}{Solution}

% Define default environments.
\theoremstyle{normal}
\newtheorem{example}{Example}[section]
\newtheorem{definition}{Definition}[section]
\newtheorem{problem}{Problem}[section]

% Define framed environment.
\tcbuselibrary{most}
\newtcbtheorem[use counter*=theorem]{ftheorem}{Theorem}{%
    before=\par\vspace{2ex},
    boxsep=0.5\topsep,
    after=\par\vspace{2ex},
    colback=green!5,
    colframe=green!35!black,
    fonttitle=\bfseries\upshape%
}{thm}

\newtcbtheorem[auto counter, number within=section]{faxiom}{Axiom}{%
    before=\par\vspace{2ex},
    boxsep=0.5\topsep,
    after=\par\vspace{2ex},
    colback=Apricot!5,
    colframe=Apricot!35!black,
    fonttitle=\bfseries\upshape%
}{ax}

\newtcbtheorem[use counter*=definition]{fdefinition}{Definition}{%
    before=\par\vspace{2ex},
    boxsep=0.5\topsep,
    after=\par\vspace{2ex},
    colback=blue!5!white,
    colframe=blue!75!black,
    fonttitle=\bfseries\upshape%
}{def}

\newtcbtheorem[use counter*=example]{fexample}{Example}{%
    before=\par\vspace{2ex},
    boxsep=0.5\topsep,
    after=\par\vspace{2ex},
    colback=red!5!white,
    colframe=red!75!black,
    fonttitle=\bfseries\upshape%
}{ex}

\newtcbtheorem[auto counter, number within=section]{fnotation}{Notation}{%
    before=\par\vspace{2ex},
    boxsep=0.5\topsep,
    after=\par\vspace{2ex},
    colback=SeaGreen!5!white,
    colframe=SeaGreen!75!black,
    fonttitle=\bfseries\upshape%
}{not}

\newtcbtheorem[use counter*=remark]{fremark}{Remark}{%
    fonttitle=\bfseries\upshape,
    colback=Goldenrod!5!white,
    colframe=Goldenrod!75!black}{ex}

\newenvironment{bproof}{\textit{Proof.}}{\hfill$\square$}
\tcolorboxenvironment{bproof}{%
    blanker,
    breakable,
    left=3mm,
    before skip=5pt,
    after skip=10pt,
    borderline west={0.6mm}{0pt}{green!80!black}
}

\AtEndEnvironment{lexample}{$\hfill\textcolor{red}{\blacksquare}$}
\newtcbtheorem[use counter*=example]{lexample}{Example}{%
    empty,
    title={Example~\theexample},
    boxed title style={%
        empty,
        size=minimal,
        toprule=2pt,
        top=0.5\topsep,
    },
    coltitle=red,
    fonttitle=\bfseries,
    parbox=false,
    boxsep=0pt,
    before=\par\vspace{2ex},
    left=0pt,
    right=0pt,
    top=3ex,
    bottom=1ex,
    before=\par\vspace{2ex},
    after=\par\vspace{2ex},
    breakable,
    pad at break*=0mm,
    vfill before first,
    overlay unbroken={%
        \draw[red, line width=2pt]
            ([yshift=-1.2ex]title.south-|frame.west) to
            ([yshift=-1.2ex]title.south-|frame.east);
        },
    overlay first={%
        \draw[red, line width=2pt]
            ([yshift=-1.2ex]title.south-|frame.west) to
            ([yshift=-1.2ex]title.south-|frame.east);
    },
}{ex}

\AtEndEnvironment{ldefinition}{$\hfill\textcolor{Blue}{\blacksquare}$}
\newtcbtheorem[use counter*=definition]{ldefinition}{Definition}{%
    empty,
    title={Definition~\thedefinition:~{#1}},
    boxed title style={%
        empty,
        size=minimal,
        toprule=2pt,
        top=0.5\topsep,
    },
    coltitle=Blue,
    fonttitle=\bfseries,
    parbox=false,
    boxsep=0pt,
    before=\par\vspace{2ex},
    left=0pt,
    right=0pt,
    top=3ex,
    bottom=0pt,
    before=\par\vspace{2ex},
    after=\par\vspace{1ex},
    breakable,
    pad at break*=0mm,
    vfill before first,
    overlay unbroken={%
        \draw[Blue, line width=2pt]
            ([yshift=-1.2ex]title.south-|frame.west) to
            ([yshift=-1.2ex]title.south-|frame.east);
        },
    overlay first={%
        \draw[Blue, line width=2pt]
            ([yshift=-1.2ex]title.south-|frame.west) to
            ([yshift=-1.2ex]title.south-|frame.east);
    },
}{def}

\AtEndEnvironment{ltheorem}{$\hfill\textcolor{Green}{\blacksquare}$}
\newtcbtheorem[use counter*=theorem]{ltheorem}{Theorem}{%
    empty,
    title={Theorem~\thetheorem:~{#1}},
    boxed title style={%
        empty,
        size=minimal,
        toprule=2pt,
        top=0.5\topsep,
    },
    coltitle=Green,
    fonttitle=\bfseries,
    parbox=false,
    boxsep=0pt,
    before=\par\vspace{2ex},
    left=0pt,
    right=0pt,
    top=3ex,
    bottom=-1.5ex,
    breakable,
    pad at break*=0mm,
    vfill before first,
    overlay unbroken={%
        \draw[Green, line width=2pt]
            ([yshift=-1.2ex]title.south-|frame.west) to
            ([yshift=-1.2ex]title.south-|frame.east);},
    overlay first={%
        \draw[Green, line width=2pt]
            ([yshift=-1.2ex]title.south-|frame.west) to
            ([yshift=-1.2ex]title.south-|frame.east);
    }
}{thm}

%--------------------Declared Math Operators--------------------%
\DeclareMathOperator{\adjoint}{adj}         % Adjoint.
\DeclareMathOperator{\Card}{Card}           % Cardinality.
\DeclareMathOperator{\curl}{curl}           % Curl.
\DeclareMathOperator{\diam}{diam}           % Diameter.
\DeclareMathOperator{\dist}{dist}           % Distance.
\DeclareMathOperator{\Div}{div}             % Divergence.
\DeclareMathOperator{\Erf}{Erf}             % Error Function.
\DeclareMathOperator{\Erfc}{Erfc}           % Complementary Error Function.
\DeclareMathOperator{\Ext}{Ext}             % Exterior.
\DeclareMathOperator{\GCD}{GCD}             % Greatest common denominator.
\DeclareMathOperator{\grad}{grad}           % Gradient
\DeclareMathOperator{\Ima}{Im}              % Image.
\DeclareMathOperator{\Int}{Int}             % Interior.
\DeclareMathOperator{\LC}{LC}               % Leading coefficient.
\DeclareMathOperator{\LCM}{LCM}             % Least common multiple.
\DeclareMathOperator{\LM}{LM}               % Leading monomial.
\DeclareMathOperator{\LT}{LT}               % Leading term.
\DeclareMathOperator{\Mod}{mod}             % Modulus.
\DeclareMathOperator{\Mon}{Mon}             % Monomial.
\DeclareMathOperator{\multideg}{mutlideg}   % Multi-Degree (Graphs).
\DeclareMathOperator{\nul}{nul}             % Null space of operator.
\DeclareMathOperator{\Ord}{Ord}             % Ordinal of ordered set.
\DeclareMathOperator{\Prin}{Prin}           % Principal value.
\DeclareMathOperator{\proj}{proj}           % Projection.
\DeclareMathOperator{\Refl}{Refl}           % Reflection operator.
\DeclareMathOperator{\rk}{rk}               % Rank of operator.
\DeclareMathOperator{\sgn}{sgn}             % Sign of a number.
\DeclareMathOperator{\sinc}{sinc}           % Sinc function.
\DeclareMathOperator{\Span}{Span}           % Span of a set.
\DeclareMathOperator{\Spec}{Spec}           % Spectrum.
\DeclareMathOperator{\supp}{supp}           % Support
\DeclareMathOperator{\Tr}{Tr}               % Trace of matrix.
%--------------------Declared Math Symbols--------------------%
\DeclareMathSymbol{\minus}{\mathbin}{AMSa}{"39} % Unary minus sign.
%------------------------New Commands---------------------------%
\DeclarePairedDelimiter\norm{\lVert}{\rVert}
\DeclarePairedDelimiter\ceil{\lceil}{\rceil}
\DeclarePairedDelimiter\floor{\lfloor}{\rfloor}
\newcommand*\diff{\mathop{}\!\mathrm{d}}
\newcommand*\Diff[1]{\mathop{}\!\mathrm{d^#1}}
\renewcommand*{\glstextformat}[1]{\textcolor{RoyalBlue}{#1}}
\renewcommand{\glsnamefont}[1]{\textbf{#1}}
\renewcommand\labelitemii{$\circ$}
\renewcommand\thesubfigure{%
    \arabic{chapter}.\arabic{figure}.\arabic{subfigure}}
\addto\captionsenglish{\renewcommand{\figurename}{Fig.}}
\numberwithin{equation}{section}

\renewcommand{\vector}[1]{\boldsymbol{\mathrm{#1}}}

\newcommand{\uvector}[1]{\boldsymbol{\hat{\mathrm{#1}}}}
\newcommand{\topspace}[2][]{(#2,\tau_{#1})}
\newcommand{\measurespace}[2][]{(#2,\varSigma_{#1},\mu_{#1})}
\newcommand{\measurablespace}[2][]{(#2,\varSigma_{#1})}
\newcommand{\manifold}[2][]{(#2,\tau_{#1},\mathcal{A}_{#1})}
\newcommand{\tanspace}[2]{T_{#1}{#2}}
\newcommand{\cotanspace}[2]{T_{#1}^{*}{#2}}
\newcommand{\Ckspace}[3][\mathbb{R}]{C^{#2}(#3,#1)}
\newcommand{\funcspace}[2][\mathbb{R}]{\mathcal{F}(#2,#1)}
\newcommand{\smoothvecf}[1]{\mathfrak{X}(#1)}
\newcommand{\smoothonef}[1]{\mathfrak{X}^{*}(#1)}
\newcommand{\bracket}[2]{[#1,#2]}

%------------------------Book Command---------------------------%
\makeatletter
\renewcommand\@pnumwidth{1cm}
\newcounter{book}
\renewcommand\thebook{\@Roman\c@book}
\newcommand\book{%
    \if@openright
        \cleardoublepage
    \else
        \clearpage
    \fi
    \thispagestyle{plain}%
    \if@twocolumn
        \onecolumn
        \@tempswatrue
    \else
        \@tempswafalse
    \fi
    \null\vfil
    \secdef\@book\@sbook
}
\def\@book[#1]#2{%
    \refstepcounter{book}
    \addcontentsline{toc}{book}{\bookname\ \thebook:\hspace{1em}#1}
    \markboth{}{}
    {\centering
     \interlinepenalty\@M
     \normalfont
     \huge\bfseries\bookname\nobreakspace\thebook
     \par
     \vskip 20\p@
     \Huge\bfseries#2\par}%
    \@endbook}
\def\@sbook#1{%
    {\centering
     \interlinepenalty \@M
     \normalfont
     \Huge\bfseries#1\par}%
    \@endbook}
\def\@endbook{
    \vfil\newpage
        \if@twoside
            \if@openright
                \null
                \thispagestyle{empty}%
                \newpage
            \fi
        \fi
        \if@tempswa
            \twocolumn
        \fi
}
\newcommand*\l@book[2]{%
    \ifnum\c@tocdepth >-3\relax
        \addpenalty{-\@highpenalty}%
        \addvspace{2.25em\@plus\p@}%
        \setlength\@tempdima{3em}%
        \begingroup
            \parindent\z@\rightskip\@pnumwidth
            \parfillskip -\@pnumwidth
            {
                \leavevmode
                \Large\bfseries#1\hfill\hb@xt@\@pnumwidth{\hss#2}
            }
            \par
            \nobreak
            \global\@nobreaktrue
            \everypar{\global\@nobreakfalse\everypar{}}%
        \endgroup
    \fi}
\newcommand\bookname{Book}
\renewcommand{\thebook}{\texorpdfstring{\Numberstring{book}}{book}}
\providecommand*{\toclevel@book}{-2}
\makeatother
\titleformat{\part}[display]
    {\Large\bfseries}
    {\partname\nobreakspace\thepart}
    {0mm}
    {\Huge\bfseries}
\titlecontents{part}[0pt]
    {\large\bfseries}
    {\partname\ \thecontentslabel: \quad}
    {}
    {\hfill\contentspage}
\titlecontents{chapter}[0pt]
    {\bfseries}
    {\chaptername\ \thecontentslabel:\quad}
    {}
    {\hfill\contentspage}
\newglossarystyle{longpara}{%
    \setglossarystyle{long}%
    \renewenvironment{theglossary}{%
        \begin{longtable}[l]{{p{0.25\hsize}p{0.65\hsize}}}
    }{\end{longtable}}%
    \renewcommand{\glossentry}[2]{%
        \glstarget{##1}{\glossentryname{##1}}%
        &\glossentrydesc{##1}{~##2.}
        \tabularnewline%
        \tabularnewline
    }%
}
\newglossary[not-glg]{notation}{not-gls}{not-glo}{Notation}
\newcommand*{\newnotation}[4][]{%
    \newglossaryentry{#2}{type=notation, name={\textbf{#3}, },
                          text={#4}, description={#4},#1}%
}
%--------------------------LENGTHS------------------------------%
% Spacings for the Table of Contents.
\addtolength{\cftsecnumwidth}{1ex}
\addtolength{\cftsubsecindent}{1ex}
\addtolength{\cftsubsecnumwidth}{1ex}
\addtolength{\cftfignumwidth}{1ex}
\addtolength{\cfttabnumwidth}{1ex}

% Indent and paragraph spacing.
\setlength{\parindent}{0em}
\setlength{\parskip}{0em}
\graphicspath{{../../../../images/}}    % Path to Image Folder.
%----------------------------GLOSSARY-------------------------------%
\makeglossaries
\loadglsentries{../../../../glossary}
\loadglsentries{../../../../acronym}
%--------------------------Main Document----------------------------%
\begin{document}
    \ifx\ifmathcourseselementaryalgebra\undefined
        \section*{Elementary Algebra}
        \setcounter{section}{1}
        \renewcommand\thefigure{\arabic{section}.\arabic{figure}}
        \renewcommand\thesubfigure{%
            \arabic{section}.\arabic{figure}.\arabic{subfigure}}
    \else
        \section{North Shore Placement Exam}
    \fi
    \subsection{Order of Operations}
        First we start with the fundamental properties of
        addition, multiplication, and exponentiation.
        \begin{properties}[Arithmetic Properties]
            \label{property:North_Shore_Arithmetic_Properties}
            \
            \begin{enumerate}
                \item
                    \label{%
                        property:%
                        North_Shore_Arithmetic_Properties_%
                        Com_Add%
                    }
                    $a+b=b+a$\hfill
                    [Commutativity of Addition]
                \item
                    \label{
                        property:%
                        north_shore_arithmetic_properties_%
                        assoc_add%
                    }
                    $a+(b+c)=(a+b)+c$\hfill
                    [Associativity of Addition]
                \item
                    \label{%
                        property:%
                        north_shore_arithmetic_properties_%
                        comm_mult%
                    }
                    ${a}\cdot{b}={b}\cdot{a}$\hfill
                    [Commutativity of Multiplication]
                \item
                    \label{%
                        property:%
                        north_shore_arithmetic_properties_%
                        assoc_mult%
                    }
                    ${a}\cdot{({b}\cdot{c})}%
                     ={({a}\cdot{b})}\cdot{c}$\hfill
                    [Associativity of Multiplication]
                \item
                    \label{%
                        property:%
                        north_shore_arithmetic_properties_%
                        add_identity
                    }
                    $a+0=a$\hfill%
                    [Identity Property of Addition]
                \item
                    \label{%
                        property:%
                        north_shore_arithmetic_properties_%
                        mult_identity%
                    }
                    ${a}\cdot{1}=a$\hfill
                    [Identity Property of Multiplication]
                \item
                    \label{%
                        property:%
                        north_shore_arithmetic_properties_%
                        add_inverse%
                    }
                    $a+(-a)=0$\hfill
                    [Inverse Property of Addition]
                \item
                    \label{%
                        property:%
                        north_shore_arithmetic_properties_%
                        mult_inverse%
                    }
                    If ${a}\ne{0}$, then $\frac{a}{a}=1$\hfill
                    [Inverse Property of Multiplication]
                \item
                    \label{%
                        property:%
                        north_shore_arithmetic_properties_%
                        distributive_property%
                    }
                    ${a}\cdot{(b+c)}%
                     ={a}\cdot{b}+{a}\cdot{c}$\hfill
                    [Distributive Property of
                     Multiplication over Addition]
            \end{enumerate}
        \end{properties}
        \begin{properties}[Properties of Exponents]
            \
            \label{property:North_Shore_Exponent_Rules}
            \begin{enumerate}
                \item
                    \label{%
                        property:%
                        north_shore_distributive_property_%
                        of_expo%
                    }
                    $({x}\cdot{y})^{n}%
                     ={x^{n}}\cdot{y^{n}}$\hfill
                    [Distributive Property of Exponents]
                \item
                    \label{%
                        property:%
                        north_shore_inverse_property_of_expo%
                    }
                    $x^{-n}=\frac{1}{x^n}$\hfill
                    [Inverse Property of Exponents]
                \item
                    \label{%
                        property:%
                        north_shore_power_property_of_expo%
                    }
                    $(x^n)^{m}=x^{{n}\cdot{m}}$\hfill
                    [Power Property of Exponents]
                \item
                    \label{%
                        property:%
                        north_shore_product_property_of_expo%
                    }
                    $x^{n}x^{m}=x^{n+m}$\hfill
                    [Multiplicative Property of Exponents]
            \end{enumerate}
        \end{properties}
        The order of operations \gls{pemdas} tells one
        how to simplify expressions.
        \begin{enumerate}
            \label{North_Shore_PEMDAS}
            \item \textbf{P}arenthesis.
            \item \textbf{E}xponents.
            \item \textbf{M}ultiplication or \textbf{D}ivision, 
                in the order they appear from left to right.
            \item \textbf{A}ddition or \textbf{S}ubtraction,
                in the order they appear from left to right.
        \end{enumerate}
        \begin{example}
            \
            \begin{enumerate}
                \item ${2}\cdot{3}+{4}\cdot{5}%
                       \overset{\textrm{\tiny{M}}}{=}%
                       6+20\overset{\textrm{\tiny{A}}}{=}%
                       \boxed{26}$
                \item ${2}\cdot{3}%
                       +4\overset{\textrm{\tiny{M}}}{=}%
                       6+4\overset{\textrm{\tiny{A}}}{=}%
                       \boxed{10}$
                \item ${3}\cdot{3+2^{4}}%
                       \overset{\textrm{\tiny{E}}}{=}%
                       {3}\cdot {3}+16%
                       \overset{\textrm{\tiny{M}}}{=}%
                       9+16\overset{\textrm{\tiny{A}}}{=}%
                       \boxed{25}$
                \item $(4+1)^2-17\cdot 3%
                    \overset{\textrm{\tiny{P}}}{=}%
                    5^{2}-17\cdot 3%
                    \overset{\textrm{\tiny{E}}}{=}25-17\cdot 3%
                    \overset{\textrm{\tiny{M}}}{=}25-51%
                    \overset{\textrm{\tiny{S}}}{=}\boxed{-26}$
                \item ${(1+1)^{(2+3)}\cdot}{5}-%
                       {2}\cdot{3}
                       \overset{\textrm{\tiny{P}}}{=}%
                       {2^{5}}\cdot{5}-{2}\cdot{3}%
                       \overset{\textrm{\tiny{E}}}{=}%
                       {32}\cdot{5}-{2}\cdot{3}%
                       \overset{\textrm{\tiny{M}}}{=}%
                       160-6\overset{\textrm{\tiny{S}}}{=}%
                       \boxed{154}$
                \item $(2+2)-(16-11)^{(1+1)}%
                       \overset{\textrm{\tiny{P}}}{=}4-5^{2}%
                       \overset{\textrm{\tiny{E}}}{=}4-25%
                       \overset{\textrm{\tiny{S}}}{=}%
                       \boxed{-21}$
            \end{enumerate}
        \end{example}
        \begin{remark}
            \label{remark:North_Shore_Radicals_Def}
            Many problems involve radicals.
            These are defined by
            $\sqrt{x}=x^{\frac{1}{2}}$ and
            $\sqrt[n]{x}=x^{\frac{1}{n}}$
        \end{remark}
        \newpage
        \begin{example}
            \
            \begin{enumerate}
                \begin{multicols}{4}
                    \item $\sqrt{x}=x^{\frac{1}{2}}$
                    \item $\sqrt[3]{x}=x^{\frac{1}{3}}$
                    \item $\sqrt[5]{x}=x^{\frac{1}{5}}$
                    \item $\sqrt[27]{x}=x^{\frac{1}{27}}$
                \end{multicols}
            \end{enumerate}
        \end{example}
        \begin{remark}
            \label{remark:North_Shore_Exponent_and_Radical_Def}
            Radicals can be combined with exponents. These are
            defined by $\sqrt[n]{x^m}=x^{\frac{m}{n}}$
        \end{remark}
        \begin{example}
            \
            \begin{enumerate}
                \begin{multicols}{4}
                    \item $\sqrt{x^3}=x^{\frac{3}{2}}$
                    \item $\sqrt[3]{x^2}=x^{\frac{2}{3}}$
                    \item $\sqrt[15]{x^3}=x^{\frac{1}{5}}$
                    \item $\sqrt[3]{x^3}=x$
                \end{multicols}
            \end{enumerate}
        \end{example}
        \begin{theorem}
            \label{%
                thm:%
                North_Shore_square_root_of_product_%
                of_positive_reals%
            }
            If $a$ and $b$ are \textbf{positive} real numbers,
            then $\sqrt{{a}\cdot{b}}=\sqrt{a}\cdot\sqrt{b}$
        \end{theorem}
        \begin{example}
            \
            \begin{enumerate}
                \begin{multicols}{4}
                    \item $\sqrt{6}=\sqrt{2}\cdot\sqrt{3}$
                    \item $\sqrt{8}=2\sqrt{2}$
                    \item $\sqrt{1,\!000}=10\sqrt{10}$
                    \item $\sqrt{50}=5\sqrt{2}$
                \end{multicols}
            \end{enumerate}
        \end{example}
        \begin{theorem}
            \label{thm:North_Shore_square_root_of_negative_real}
            If $x$ is \textbf{positive}
            and $i$ is the imaginary unit
            $(i^{2}=-1)$,
            then $\sqrt{-x}=i\sqrt{x}$
        \end{theorem}
        \begin{example}
            \
            \begin{enumerate}
                \begin{multicols}{4}
                    \item $\sqrt{-7}=i\sqrt{7}$
                    \item $\sqrt{-9}=3i$
                    \item $\sqrt{-4}=2i$
                    \item $\sqrt{-2}=i\sqrt{2}$
                \end{multicols}
            \end{enumerate}
        \end{example}
        \begin{theorem}
            \label{thm:North_Shore_odd_nth_root_of_real}
            If $n$ is an \textbf{odd} integer
            and $x$ is a \textbf{real} number,
            then $\sqrt[n]{-x}=-\sqrt[n]{x}$
        \end{theorem}
        \begin{example}
            \
            \begin{enumerate}
                \begin{multicols}{4}
                \item $\sqrt[3]{-7}=-\sqrt[3]{7}$
                \item $\sqrt[17]{-1}=-1$
                \item $\sqrt[5]{-9}=-\sqrt[5]{9}$
                \item $\sqrt[3]{-27}=-3$
                \end{multicols}
            \end{enumerate}
        \end{example}
        \begin{remark}
            \label{remark:North_Shore_non_zero_raised_to_zero}
            For all non-zero  real numbers, $x^{0}=1$.
            $0^0$ is left undefined.
        \end{remark}
        \begin{example}
            \
            \begin{enumerate}
                \begin{multicols}{4}
                    \item $\pi^{0}=1$
                    \item $(-5)^{0}=1$
                    \item $(\frac{1}{2})^{0}=1$
                    \item $0^{0}$ is undefined.
                \end{multicols}
            \end{enumerate}
        \end{example}
        \begin{definition}
            \label{definiiton:North_Shore_Absolute_Value_Def}
            The absolute value of a real number $x$ is
            defined as
            $|x|=\begin{cases}%
                \phantom{-}x, & x\geq 0\\%
                -x, & x<0%
            \end{cases}$
        \end{definition}
        \begin{example}
            \
            \begin{enumerate}
                \begin{multicols}{5}
                    \item[1.] $|3|=3$
                    \item[6.] $|24|=24$
                    \item[2.] $|-3|=3$
                    \item[7.] $|-24|=24$
                    \item[3.] $|\pi|=\pi$
                    \item[8.] $|\frac{1}{2}|=\frac{1}{2}$
                    \item[4.] $|-\pi|=\pi$
                    \item[9.] $|-\frac{1}{2}|=\frac{1}{2}$
                    \item[5.] $|0|=0$
                    \item[10.] $|-1|=1$
                \end{multicols}
            \end{enumerate}
        \end{example}
        \begin{remark}
            \label{remark:North_Shore_Rational_Expressions}
            Expressions like $\frac{a+b}{c+d}$ should be treated
            as equivalent to ${(a+b)}\div{(c+d)}$
        \end{remark}
        \subsubsection{Problems}
            \begin{problem}
                Simplify $3^{2}+5-\sqrt{4}+4^{0}$
            \end{problem}
            \begin{proof}[Solution]
                \begin{flalign*}
                    3^{2}+5-\sqrt{4}+4^{0}
                    &=3^{2}+5-4^{\frac{1}{2}}+4^{0}
                    &\tag{%
                        Remark~\ref{%
                            remark:%
                            North_Shore_Radicals_Def%
                        }
                    }\\[-0.5ex]
                    &=3^{2}+5-4^{\frac{1}{2}}+1
                    &\tag{%
                        Remark~\ref{%
                            remark:%
                            North_Shore_non_zero_raised_to_zero%
                        }
                    }\\
                    &=9+5-2+1&\tag{Exponentiation}\\
                    &=14-2+1&\tag{Addition}\\
                    &=12+1&\tag{Subtraction}\\
                    &=\boxed{13}&\tag{Addition}
                \end{flalign*}
            \end{proof}
            \begin{problem}
                Simplify $(5+1)(4-2)-3$
            \end{problem}
            \begin{proof}[Solution]
                \begin{flalign*}
                    (5+1)(4-2)-3&=6\cdot 2-3&\tag{Parenthesis}\\
                    &=12-3&\tag{Multiplication}\\
                    &=\boxed{9}&\tag{Subtraction}
                \end{flalign*}
            \end{proof}
            \begin{problem}
                Simplify ${3}\cdot{7^{2}}$
            \end{problem}
            \begin{proof}[Solution]
                \begin{flalign*}
                    {3}\cdot{7^{2}}
                    &={3}\cdot{49}&\tag{Exponentiation}\\
                    &=\boxed{147}&\tag{Multiplication}
                \end{flalign*}
            \end{proof}
            \begin{problem}
                Simplify ${2}\cdot{(7+3)^{2}}$
            \end{problem}
            \begin{proof}[Solution]
                \begin{flalign*}
                    {2}\cdot{(7+3)^{2}}
                    &={2}\cdot{10^{2}}&\tag{Parenthesis}\\
                    &={2}\cdot{100}&\tag{Exponentiation}\\
                    &=\boxed{200}&\tag{Multiplication}
                \end{flalign*}
            \end{proof}
            \begin{problem}
                Simplify ${49}\div{7}-{2}\cdot{2}$
            \end{problem}
            \begin{proof}[Solution]
                \begin{flalign*}
                    {49}\div{7}-{2}\cdot{2}
                    &=7-{2}\cdot{2}&\tag{Division}\\
                    &=7-4&\tag{Multiplication}\\
                    &=\boxed{3}&\tag{Subtraction}
                \end{flalign*}
            \end{proof}
            \begin{problem}
                Simplify ${9}\div{3}\cdot{5}-{8}\div{2}+27$
            \end{problem}
            \begin{proof}[Solution]
                \begin{flalign*}
                    {9}\div{3}\cdot{5}-{8}\div{2}+27
                    &={3}\cdot{5}-4+27
                    &\tag{%
                        Multiplication/Division
                        from Left to Right%
                    }\\
                    &=15-4+27&\tag{Multiplication}\\
                    &=11+27&\tag{Subtraction}\\
                    &=\boxed{38}&\tag{Addition}
                \end{flalign*}
            \end{proof}
            \begin{problem}
                Simplify $3+2(5)-|-7|$
            \end{problem}
            \begin{proof}[Solution]
                \begin{flalign*}
                    3+2(5)-|-7|
                    &=3+2(5)-7
                    &\tag{%
                        Def.~\ref{%
                            definiiton:North_Shore_%
                            Absolute_Value_Def%
                        }%
                    }\\
                    &=3+10-7&\tag{Multiplication}\\
                    &=13-7&\tag{Addition}\\
                    &=\boxed{6}&\tag{Addition}
                \end{flalign*}
            \end{proof}
            \begin{problem}
                Simplify $\frac{{5}\cdot{5}-4(4)}{2^{2}-1}$
            \end{problem}
            \begin{proof}[Solution]
                \begin{flalign*}
                    \tfrac{{5}\cdot{5}-4(4)}{2^{2}-1}
                    &=({5}\cdot{5}-4(4))\div{(2^{2}-1)}
                    &\tag{%
                        Remark~\ref{%
                            remark:North_Shore_%
                            Rational_Expressions%
                        }%
                    }\\
                    &={({5}\cdot{5}-4(4))}\div{(4-1)}
                    &\tag{Exponentiation Inside Parenthesis}\\
                    &={(25-16)}\div{(4-1)}
                    &\tag{Multiplication Inside Parenthesis}\\
                    &={9}\div{3}&\tag{Parenthesis}\\
                    &=\boxed{3}&\tag{Division}
                \end{flalign*}
            \end{proof}
            \begin{problem}
                Simplify $\frac{4^{2}-5^{2}}{(4-5)^{2}}$
            \end{problem}
            \begin{proof}[Solution]
                \begin{flalign*}
                    \tfrac{4^{2}-5^2}{(4-5)^{2}}
                    &={(4^{2}-5^{2})}\div{((4-5)^2)}
                    &\tag{%
                        Remark~\ref{%
                            remark:North_Shore_%
                            Rational_Expressions%
                        }%
                    }\\
                    &={(4^{2}-5^{2})}\div{((-1)^2)}
                    &\tag{Parenthesis}\\
                    &={(16-25)}\div{(1)}
                    &\tag{Exponentiation Inside Parenthesis}\\
                    &={-9}\div{1}&\tag{Parenthesis}\\
                    &=\boxed{-9}&\tag{Division}
                \end{flalign*}
            \end{proof}
            \begin{problem}
                Simplify $-5^{2}$
            \end{problem}
            \begin{proof}[Solution]
                \begin{flalign*}
                    -5^{2}&=\boxed{-25}&\tag{Exponentiation}
                \end{flalign*}
            \end{proof}
            \begin{problem}
                Simplify ${48}\div{2(3+9)}$
            \end{problem}
            \begin{proof}[Solution]
                \begin{flalign*}
                    {48}\div{2(3+9)}
                    &={48}\div{2(12)}
                    &\tag{Parenthesis}\\
                    &=24(12)&\tag{Division}\\
                    &=\boxed{288}&\tag{Multiplication}
                \end{flalign*}
            \end{proof}
            \begin{problem}
                Simplify ${6}\cdot{(3+2)^{2}}+14$
            \end{problem}
            \begin{proof}[Solution]
                \begin{flalign*}
                    {6}\cdot{(3+2)^{2}}+14
                    &={6}\cdot{5^{2}+14}
                    &\tag{Parenthesis}\\
                    &={6}\cdot{25}+14&\tag{Exponentiation}\\
                    &=150+14&\tag{Multiplication}\\
                    &=\boxed{164}
                \end{flalign*}
            \end{proof}
            \begin{problem}
                Simplify $(5+1)^{(3-2)}$
            \end{problem}
            \begin{proof}[Solution]
                \begin{flalign*}
                    (5+1)^{(3-2)}&=6^{1}&\tag{Parenthesis}\\
                    &=\boxed{6}&\tag{Exponentiation}
                \end{flalign*}
            \end{proof}
    \subsection{Scientific Notation}
        Non-zero real numbers can be written as
        ${r}\times{10^{n}}$, where ${1}\leq{|r|}<10$
        and $n$ is an integer.
        \begin{example}
            \
            \begin{enumerate}
                \begin{multicols}{3}
                    \item[1.] $12,\!345={1.2345}\times{10^{4}}$
                    \item[4.] $10={1}\times{10^{1}}$
                    \item[2.] $0.01={1}\times{10^{-2}}$
                    \item[5.] $124={1.24}\times{10^{2}}$
                    \item[3.] $36.24={3.624}\times{10^{1}}$
                    \item[6.] $0.000314={3.14}\times{10^{-4}}$
                \end{multicols}
            \end{enumerate}
        \end{example}
        \begin{remark}
            Several constants in chemistry/physics are
            written in scientific notation.
            \begin{enumerate}
                \item Planck's Constant:
                    $h={6.626}\times{10^{-34}}%
                    \textrm{J}\cdot\textrm{s}$
                \item Universal Gravitation Constant:
                    $G={6.67}\times{10^{-11}}%
                    \textrm{Nm}^{2}\textrm{kg}^{-2}$
                \item Avogadro's Number:
                    $N_{A}={6.0221}\times{10^{23}}%
                    \textrm{mol}^{-1}$
                \item Speed of Light:
                    $c={2.998}\times{10^{8}\textrm{ms}^{-1}}$
            \end{enumerate}
        \end{remark}
        \subsubsection{Problems}
            \begin{problem}
                Write the following in scientific notation:
                \begin{enumerate}
                    \begin{multicols}{4}
                        \item $350,\!000,\!000$
                        \item $120,\!500,\!000,\!000$
                        \item $0.0000000523$
                        \item $10.01$
                    \end{multicols}
                \end{enumerate}
            \end{problem}
            \begin{proof}[Solution]
                \
                \begin{enumerate}
                    \begin{multicols}{4}
                        \item ${3.5}\times{10^8}$
                        \item ${1.205}\times{10^{11}}$
                        \item ${5.23}\times{10^{-8}}$
                        \item ${1.001}\times{10^{1}}$
                    \end{multicols}
                \end{enumerate}
            \end{proof}
            \begin{problem}
                Write in expanded form:
                \begin{enumerate}
                    \begin{multicols}{3}
                        \item ${6.02}\times{10^{15}}$
                        \item ${3.0}\times{10^{8}}$
                        \item ${1.819}\times{10^{-9}}$
                    \end{multicols}
                \end{enumerate}
            \end{problem}
            \begin{proof}[Solution]
                \
                \begin{enumerate}
                    \begin{multicols}{3}
                        \item $6,\!020,\!000,\!000,\!000,\!000$
                        \item $300,\!000,\!000$
                        \item $0.000000001819$
                    \end{multicols}
                \end{enumerate}
            \end{proof}
            \begin{problem}
                Simplify:
                \begin{enumerate}
                    \begin{multicols}{4}
                        \item $({3}\times{10^{3}})%
                               ({5}\times{10^{6}})$
                        \item $\frac{{6}\times{10^{9}}}%
                               {{3}\times{10^{4}}}$
                        \item $({3}\times{10^{-4}})^{2}$
                        \item 
                            $\frac{%
                                   ({3.2}\times{10^{5}})%
                                   ({2}\times{10^{-3}})%
                             }{%
                               {2}\times{10^{-5}}%
                            }$
                    \end{multicols}
                \end{enumerate}
            \end{problem}
            \begin{proof}[Solution]
                \
                \begin{enumerate}
                    \item $(3\times 10^{3})(5\times 10^{6})=%
                        3\times 5\times 10^{3}\times 10^{6}=%
                        15\times 10^{6+3}=%
                        15\times 10^{9}=%
                        \boxed{1.5\times 10^{10}}$
                    \item $\frac{6\times 10^{9}}{3\times 10^{4}}=%
                        \frac{6}{3}\times\frac{10^{9}}{10^{4}}=%
                        2\times 10^{9-4}=%
                        \boxed{2\times 10^{5}}$
                    \item $(3\times 10^{-4})^{2}=%
                        3^{2}\times (10^{-4})^{2}=%
                        \boxed{9\times 10^{-8}}$
                    \item 
                        $\frac{%
                            (3.2\times 10^{5})(2\times 10^{-2})%
                        }{%
                            2\times 10^{-5}%
                        }=%
                        3.2\times 10^{5}\times%
                        \frac{%
                            2\times 10^{-3}%
                        }{%
                            2\times 10^{-5}%
                        }=%
                        3.2\times 10^{5}\times 10^{2}=%
                        \boxed{3.2\times 10^{7}}$
                \end{enumerate}
            \end{proof}
    \subsection{Substitution}
        \subsubsection{Problems}
            \begin{problem}
                Solve $xyz-4z$ for $x=3,y=-4,z=2$.
            \end{problem}
            \begin{proof}[Solution]
                \begin{flalign*}
                    xyz-4z\big|_{x=3,y=-4,z=2}
                    &=(3)(-4)(2)-4(2)&\tag{Substitution}\\
                    &=-24-8&\tag{Multiplication}\\
                    &=\boxed{-32}&\tag{Subtraction}
                \end{flalign*}
            \end{proof}
            \begin{problem}
                Solve $2x-y$ for $x=3,y=-4,z=2$.
            \end{problem}
            \begin{proof}[Solution]
                \begin{flalign*}
                    2x-y\big|_{x=3,y=-4,z=2}
                    &=2(3)-(-4)&\tag{Substitution}\\
                    &=6+4&\tag{Multiplication}\\
                    &=\boxed{10}&\tag{Addition}
                \end{flalign*}
            \end{proof}
            \newpage
            \begin{problem}
                Solve $x(y-3z)$ for $x=3,y=-4,z=2$.
            \end{problem}
            \begin{proof}[Solution]
                \begin{flalign*}
                    x(y-3z)\big|_{x=3,y=-4,z=2}
                    &=(3)((-4)-3(2))&\tag{Substitution}\\
                    &=3(-10)&\tag{Parenthesis}\\
                    &=\boxed{-30}&\tag{Multiplication}
                \end{flalign*}
            \end{proof}
            \begin{problem}
                Solve $\frac{5x-z}{xy}$ for $x=3,y=-4,z=2$.
            \end{problem}
            \begin{proof}[Solution]
                \begin{flalign*}
                    \tfrac{5x-z}{xy}\big|_{x=3,y=-4,z=2}
                    &=\tfrac{5(3)-(2)}{(3)(-4)}
                    &\tag{Substitution}\\
                    &=\tfrac{15-2}{-12}&\tag{Multiplication}\\
                    &=\boxed{-\tfrac{13}{12}}&\tag{Subtraction}
                \end{flalign*}
            \end{proof}
            \begin{problem}
                Solve $3y^{2}-2x+4z$ for $x=3,y=-4,z=2$.
            \end{problem}
            \begin{proof}[Solution]
                \begin{flalign*}
                    3y^{2}-2x+4z\big|_{x=3,y=-4,z=2}
                    &=3(-4)^{2}-2(3)+4(2)&\tag{Substitution}\\
                    &=3(16)-2(3)+4(2)&\tag{Exponentiation}\\
                    &=48-6+8&\tag{Multiplication}\\
                    &=\boxed{50}&\tag{Addition/Subtraction}
                \end{flalign*}
            \end{proof}
            \begin{problem}
                Solve $x+y+z$ for $x=1,y=2,z=3$.
            \end{problem}
            \begin{proof}[Solution]
                \begin{flalign*}
                    x+y+z\big|_{x=1,y=2,z=3}
                    &=(1)+(2)+(3)&\tag{Substitution}\\
                    &=\boxed{6}&\tag{Addition}
                \end{flalign*}
            \end{proof}
            \begin{problem}
                Solve $(x+1)(y-2)$ for $x=1,y=2,z=3$
            \end{problem}
            \begin{proof}[Solution]
                \begin{flalign*}
                    (x+1)(y-2)\big|_{x=1,y=2,z=3}
                    &=((1)+1)((2)-2)&\tag{Substitution}\\
                    &=2\cdot 0&\tag{Parenthesis}\\
                    &=\boxed{0}&\tag{Multipliation}
                \end{flalign*}
            \end{proof}
            \begin{problem}
                Solve $x^{2}+y^{2}-z^{2}$ for $x=1,y=2,z=3$.
            \end{problem}
            \begin{proof}[Solution]
                \begin{flalign*}
                    x^{2}+y^{2}-z^{2}\big|_{x=1,y=2,z=3}
                    &=(1)^{2}+(2)^{2}-(3)^{2}&\tag{Substitution}\\
                    &=1+4-9&\tag{Exponentiation}\\
                    &=\boxed{-4}&\tag{Addition/Subtraction}
                \end{flalign*}
            \end{proof}
            \newpage
            \begin{problem}
                Solve $\frac{z+1}{y(x+1)}$ for $x=1,y=2,z=3$.
            \end{problem}
            \begin{proof}[Solution]
                \begin{flalign*}
                    \tfrac{z+1}{y(x+1)}\big|_{x=1,y=2,z=3}
                    &=\tfrac{(3)+1}{(2)((1)+1)}&\tag{Substitution}\\
                    &=\tfrac{4}{2(2)}&\tag{Parenthesis}\\
                    &=\tfrac{4}{4}&\tag{Multiplication}\\
                    &=\boxed{1}&\tag{Division}
                \end{flalign*}
            \end{proof}
            \begin{problem}
                Solve $xy+xz+yz$ for $x=0,y=3,z=-1$.
            \end{problem}
            \begin{proof}[Solution]
                \begin{flalign*}
                    xy+xz+yz\big|_{x=0,y=3,z=-1}
                    &=(0)(3)+(0)(-1)+(3)(-1)&\tag{Substitution}\\
                    &=0+0+(-3)&\tag{Multiplication}\\
                    &=\boxed{-3}&\tag{Addition}
                \end{flalign*}
            \end{proof}
            \begin{problem}
                Solve $y^{x}+z^{y}$ for $x=0,y=3,z=-1$.
            \end{problem}
            \begin{proof}[Solution]
                \begin{flalign*}
                    y^{x}+z^{y}\big|_{x=0,y=3,z=-1}
                    &=(3)^{(0)}+(-1)^{(3)}&\tag{Substitution}\\
                    &=1+(-1)&\tag{Exponentiation}\\
                    &=\boxed{0}&\tag{Addition}
                \end{flalign*}
            \end{proof}
            \begin{problem}
                Solve $\frac{y+z}{x}$ for $x=0,y=3,z=-1$.
            \end{problem}
            \begin{proof}[Solution]
                \begin{flalign*}
                    \tfrac{y+z}{x}\big|_{x=0,y=3,z=-1}
                    &=\tfrac{(3)+(-1)}{(0)}&\tag{Substitution}\\
                    &=\boxed{\textrm{Undefined}}
                    &\tag{Division by Zero}
                \end{flalign*}
            \end{proof}
            \begin{problem}
                Solve $xy^{z^y}+y$ for $x=0,y=3,z=-1$.
            \end{problem}
            \begin{proof}[Solution]
                \begin{flalign*}
                    xy^{z^{y}}+y\big|_{x=0,y=3,z=-1}
                    &=(0)(3)^{(-1)^{(3)}}+3&\tag{Substitution}\\
                    &=0\cdot 3^{-1}+3&\tag{Exponentiation}\\
                    &=0+3&\tag{Multiplication}\\
                    &=\boxed{3}&\tag{Addition}
                \end{flalign*}
            \end{proof}
            \newpage
    \subsection{Linear Equations in One Variable}
        \subsubsection{Problems}
        \begin{problem}
        Solve for $x$: $6x-48=6$
        \end{problem}
        \begin{proof}[Solution]
        \begin{flalign*}
            6x-48&=6\\
            \Rightarrow 6x&=54&\tag{Add $48$ to Both Sides}\\
            \Rightarrow x&=\tfrac{54}{6}&\tag{Divide Both Sides by $6$}\\
            \Rightarrow x&=\boxed{9}&\tag{Division}
        \end{flalign*}
        \end{proof}
        \begin{problem}
        Solve for $x$: $\frac{2}{3}x-5=x-3$
        \end{problem}
        \begin{proof}[Solution]
        \begin{flalign*}
            \tfrac{2}{3}x-5&=x-3\\
            \Rightarrow 2x-15&=3x-9&\tag{Multiply Both Sides by $3$}\\
            \Rightarrow -x-15&=-9&\tag{Subtract $3x$ from Both Sides}\\
            \Rightarrow -x&=6&\tag{Add $15$ to Both Sides}\\
            \Rightarrow x&=\boxed{-6}&\tag{Multiply Both Sides by $-1$}
        \end{flalign*}
        \end{proof}
        \begin{problem}
        Solve for $x$: $50-x-(3x+2)=0$
        \end{problem}
        \begin{proof}[Solution]
        \begin{flalign*}
            50-x-(3x+2)&=0\\
            \Rightarrow 50-x-3x-2&=0&\tag{Distribute the Minus Sign}\\
            \Rightarrow 48-4x&=0&\tag{Simplify the Left-Hand Side}\\
            \Rightarrow 4x&=48&\tag{Add $4x$ to Both Sides}\\
            \Rightarrow x&=\tfrac{48}{4}&\tag{Divide Both Sides by $4$}\\
            \Rightarrow x&=\boxed{12}&\tag{Division}
        \end{flalign*}
        \end{proof}
        \begin{problem}
        Solve for $x$: $8-4(x-1)=2+3(4-x)$
        \end{problem}
        \begin{proof}[Solution]
        \begin{flalign*}
            8-4(x-1)&=2+3(4-x)\\
            \Rightarrow8-4x+4
            &=2+12-3x&\tag{Simplify Both Sides}\\
            \Rightarrow 12-4x&=14-3x&\tag{Simplify Both Sides}\\
            \Rightarrow 12&=14+x&\tag{Add $4x$ to Both Sides}\\
            \Rightarrow x&=\boxed{-2}&\tag{Subtract $14$ from Both Sides}
        \end{flalign*}
        \end{proof}
        \begin{problem}
        Solve $x+1=1$ for $x$.
        \end{problem}
        \begin{proof}[Solution]
        \begin{flalign*}
            x+1&=1\\
            \Rightarrow x&=\boxed{0}&\tag{Subtract $1$ from Both Sides}
        \end{flalign*}
        \end{proof}
        \begin{problem}
        Solve $4(x-1)+x=0$ for $x$.
        \end{problem}
        \begin{proof}[Solution]
        \begin{flalign*}
            4(x-1)+x&=0\\
            \Rightarrow 5x-4&=0&\tag{Simplify the Left-Hand Side}\\
            \Rightarrow 5x&=4&\tag{Add $4$ to Both Sides}\\
            \Rightarrow x&=\boxed{\tfrac{4}{5}}&\tag{Divide Both Sides by $5$}
        \end{flalign*}
        \end{proof}
        \begin{problem}
        Solve $1-x+10=7$ for $x$.
        \end{problem}
        \begin{proof}[Solution]
        \begin{flalign*}
            1-x+10&=7\\
            \Rightarrow 11-x&=7&\tag{Simplify the Left-Hand Side}\\
            \Rightarrow 11&=7+x&\tag{Add $x$ to both sides}\\
            \Rightarrow x&=\boxed{4}&\tag{Subtract $7$ from Both Sides}
        \end{flalign*}
        \end{proof}
    \subsection{Formulas}
        \subsubsection{Problems}
        \begin{problem}
        Solve $PV=nRT$ for $T$.
        \end{problem}
        \begin{proof}[Solution]
        This is the Ideal Gas Law from chemistry: $\boxed{T=\tfrac{PV}{nR}}$
        \end{proof}
        \begin{problem}
        Solve $y=3x+2$ for $x$.
        \end{problem}
        \begin{proof}[Solution]
        $y=3x+2\Rightarrow y-2=3x\Rightarrow\boxed{x=\tfrac{y-2}{3}}$
        \end{proof}
        \begin{problem}
        Solve $C=2\pi r$ for $r$.
        \end{problem}
        \begin{proof}[Solution]
        This is the formula for the circumference $C$ of a circle of radius $r$: $\boxed{r=\tfrac{C}{2\pi}}$
        \end{proof}
        \begin{problem}
        Solve $\tfrac{x}{2}+\tfrac{y}{5}=1$ for $y$.
        \end{problem}
        \begin{proof}[Solution]
        $\tfrac{x}{2}+\tfrac{y}{5}=1\Rightarrow=\tfrac{y}{5}=1-\tfrac{x}{2}\Rightarrow\boxed{y=5-\tfrac{5}{2}x}$.
        \end{proof}
        \begin{problem}
        Solve $y=hx+4x$ for $x$.
        \end{problem}
        \begin{proof}[Solution]
        $y=hx+4x\Rightarrow x(h+4)=y\Rightarrow\boxed{x=\tfrac{y}{h+4}}$
        \end{proof}
    \subsection{Word Problems}
        \subsubsection{Problems}
        \begin{problem}
        $y$ is $5$ more than twice that of $x$, their sum is $35$. Find $x$ and $y$.
        \end{problem}
        \begin{proof}[Solution]
        We have $y=5+2x$ and $x+y=35$. Substituting $y$, we get $x+2x+5=35\Rightarrow 3x+5=35\Rightarrow 3x=30\Rightarrow\boxed{x=10}$. But $y=2+2x=5+2(10)\Rightarrow\boxed{y=25}$
        \end{proof}
        \begin{problem}
        Ms. Jones invested $\$18,\!000$ in two accounts, one pays $6\%$ and the other $8\%$. Her total interest was $\$1,\!290$. How much did she have in each account?
        \end{problem}
        \begin{proof}[Solution]
        Let $x$ and $y$ be the amounts in the $6\%$ and $8\%$ accounts, respectively. Then $x+y=\$18,\!000$ and $\frac{6}{100}x+\frac{8}{100}y=\$1,\!290$. Solving the first equation we get $y=\$18,\!000-x$. Substituting we have: $\frac{6}{100}x+\frac{8}{100}(\$18,\!000-x)=\$1,\!290\Rightarrow \$1440-\frac{2}{100}x=\$1,\!290\Rightarrow\frac{2}{100}x=\$150\Rightarrow\boxed{x=\$7,\!500}$ But $y=\$18,\!000-x\Rightarrow y=\$18,\!000-\$7,\!500\Rightarrow\boxed{y=\$10,\!500}$
        \end{proof}
        \begin{problem}
        How many liters of $40\%$ and $16\%$ solution are needed to obtain $20$ liters of $22\%$ solution?
        \end{problem}
        \begin{proof}[Solution]
        Let $x$ and $y$ be the number of liters of $40\%$ and $16\%$ solution, respectively. Then $x+y=20$, and $\frac{40}{100}x+\frac{16}{100}y=\frac{22}{100}20\Rightarrow 40x+16y=440$. Solving the first equation, we get $y=20-x$. Substituting, we have: $40x+16(20-x)=440\Rightarrow 24x+320=440\Rightarrow 24x=120\Rightarrow x=\frac{120}{24}\Rightarrow\boxed{x=5}$. But $y=20-x=20-5\Rightarrow \boxed{y=15}$ so there are $\boxed{5\textrm{ liters of }40\%}$ and $\boxed{15\textrm{ liters of }16\%}$ solution.
        \end{proof}
        \begin{problem}
        Sheila bought burgers and fries for her children and some friends. The burgers cost $\$2.05$ each and the fries are $\$0.85$ each. She bought a total of $14$ items for a total cost of $\$19.10$. How many of each did she buy?
        \end{problem}
        \begin{proof}[Solution]
        Let $x$ and $y$ be the number of burgers and fries, respectively. Then $2.05x+0.85y=19.10$, and $x+y=14$. Solving the second equation, we get $y=14-x$. Substitute this into the second equation to $2.05x+0.85(14-x)=19.10\Rightarrow 1.20x+11.90=19.10\Rightarrow 1.20x=7.20\Rightarrow\boxed{x=6}$ But $y=14-x\Rightarrow\boxed{y=8}$ Sheila bought $\boxed{6\textrm{ burgers}}$ and $\boxed{8\textrm{ fries}}$
        \end{proof}
    \subsection{Inequalities}
        There are three main rules for dealing with inequalities:
        \begin{properties}\label{property:north_shore_properties_of_inequalities}
        \
        \begin{enumerate}
            \item \label{property:north_shore_additive_property_inequals}If $c$ is real and $a<b$, then $a+c<b+c$\hfill [Additive Property of Inequalities]
            \item \label{property:north_shore_multiplicative_property_inequals}If $c$ is $\mathbf{positive}$ and $a<b$, then $ac < bc$\hfill [Multiplicative Property of Inequalities]
            \item \label{property:north_shore_inverse_property_inequals}If $c$ is $\mathbf{negative}$ and $a<b$, then $bc<ac$\hfill [Inverse Property of Inequalities]
        \end{enumerate}
        \end{properties}
        \subsubsection{Problems}
        \begin{problem}
        Solve for $x$: $2x-7\geq 3$
        \end{problem}
        \begin{proof}[Solution]
            \begin{flalign*}
                2x-7&\geq 3\\
                \Rightarrow 2x&\geq 10&\tag{Add $7$ to Both Sides, property~\ref{property:north_shore_properties_of_inequalities} part~\ref{property:north_shore_additive_property_inequals}}\\
                \Rightarrow x&\geq\tfrac{10}{2}&\tag{Divide Both Sides by $2$, property~\ref{property:north_shore_properties_of_inequalities} part~\ref{property:north_shore_multiplicative_property_inequals}}\\
                \Rightarrow x&\geq 5&\tag{Division}
            \end{flalign*}
        \end{proof}
        \begin{problem}
        Solve for $x$: $-5(2x+3)<2x-3$
        \end{problem}
        \begin{proof}[Solution]
        \begin{flalign*}
        -5(2x+3)&<2x-3\\
        \Rightarrow -10x-15&<2x-3&\tag{Simplify the Left-Hand Side}\\
        \Rightarrow -15+3&<2x+10x&\tag{Add $10x+3$ to Both Sides, property~\ref{property:north_shore_properties_of_inequalities} part~\ref{property:north_shore_additive_property_inequals}}\\
        \Rightarrow -12&<12x&\tag{Simplify Both Sides}\\
        \Rightarrow x&>-1&\tag{Divide Both Sides by $12$, property~\ref{property:north_shore_properties_of_inequalities} part~\ref{property:north_shore_multiplicative_property_inequals}}
        \end{flalign*}
        \end{proof}
        \begin{problem}
        Solve for $x$: $3(x-4)-(x+1)\leq -12$
        \end{problem}
        \begin{proof}[Solution]
            \begin{flalign*}
                3(x-4)-(x+1)&\leq -12\\
                \Rightarrow 3x-12-x-1&\leq -12&\tag{Simplify the Left-Hand Side}\\
                \Rightarrow 2x-13&\leq -12&\tag{Simplify the Left-Hand Side}\\
                \Rightarrow 2x&\leq 1&\tag{Add $13$ to Both Sides, property~\ref{property:north_shore_properties_of_inequalities} part~\ref{property:north_shore_additive_property_inequals}}\\
                \Rightarrow x&\leq\tfrac{1}{2}&\tag{Divide Both Sides by $2$, property~\ref{property:north_shore_properties_of_inequalities} part~\ref{property:north_shore_multiplicative_property_inequals}}
            \end{flalign*}
        \end{proof}
    \subsection{Exponents and Polynomials}
        The two main rules for problem with polynomials and exponents are:
        \begin{align}
            (a_{1}x^{2}+b_{1}x+c_{1})+(a_{2}x^{2}+b_{2}x+c_{2})&=x^{2}(a_{1}+a_{2})+x(b_{1}+b_{2})+(c_{1}+c_{2})\label{equation:north_shore_additive_polynomial_property}\\
            (ax+b)(cx+d)&=acx^{2}+(ad+bc)x+bd\label{equation:north_shore_multiplicative_polynomial_property}
        \end{align}
        \begin{remark}
        Eqn.~\ref{equation:north_shore_additive_polynomial_property} says that the coefficients of like terms can be added together to simplify the expression, and eqn.~\ref{equation:north_shore_multiplicative_polynomial_property} is often called the \gls{foil} rule.
        \end{remark}
        \begin{remark}
        \label{remark:north_shore_square_of_a_sum_foil}
        A special case of \gls{foil}, we can write $(a+b)^{2}=a^{2}+ab+ba+b^{2}=a^{2}+2ab+b^{2}$
        \begin{equation}
        \label{equation:North_Shore_square_of_a_sum}
            (a+b)^{2}=a^{2}+2ab+b^{2}
        \end{equation}
        \end{remark}
        \subsubsection{Problems}
        \begin{problem}
        Simplify using only positive exponents: $(3x^{0}y^{5}z^{6})(-2xy^{3}z^{-2})$
        \end{problem}
        \begin{proof}[Solution]
            \begin{flalign*}
                (3x^{0}y^{5}z^{6})(-2xy^{3}z^{-2})&=(3\cdot (-2))(x^{0}\cdot x^{1})(y^{5}\cdot y^{3})(z^{6}\cdot z^{-2})&\tag{Property~\ref{property:North_Shore_Arithmetic_Properties} part~\ref{property:north_shore_arithmetic_properties_assoc_mult}}\\
                &=-6x^{0+1}y^{5+3}z^{6-2}&\tag{Property~\ref{property:North_Shore_Exponent_Rules} part~\ref{property:north_shore_product_property_of_expo}}\\
                &=-6xy^{8}z^{4}&\tag{Simplify Exponents}
            \end{flalign*}
        \end{proof}
        \begin{problem}
        Simplify using only positive exponents: $(3x^{2}-5x-6)+(5x^{2}+4x+4)$
        \end{problem}
        \begin{proof}[Solution]
        \begin{flalign*}
            (3x^{2}-5x-6)+(5x^{2}+4x+4)&=(3+5)x^{2}+(-5+4)x+(-6+4)&\tag{Eqn.~\ref{equation:north_shore_additive_polynomial_property}}\\
            &=8x^{2}-x-2&\tag{Simplify Parenthesis}
        \end{flalign*}
        \end{proof}
        \begin{problem}
        Simplify using only positive exponents: $\frac{(2a^{-5}b^{4}c^{3})^{-2}}{(3a^{3}b^{-7}c^3)^{2}}$
        \end{problem}
        \begin{proof}[Solution]
        \begin{flalign*}
            \tfrac{(2a^{-5}b^{4}c^{3})^{-2}}{(3a^{3}b^{-7}c^{3})^{2}}&=(2a^{-5}b^{4}c^{3})^{-2}\cdot\tfrac{1}{(3a^{3}b^{-7}c^{3})^{2}}&\tag{Remark~\ref{remark:North_Shore_Rational_Expressions}}\\
            &=\tfrac{1}{(2a^{-5}b^{4}c^{3})^{2}}\cdot\tfrac{1}{(3a^{3}b^{-7}c^{3})^{2}}&\tag{Property~\ref{property:North_Shore_Exponent_Rules} part~\ref{property:north_shore_inverse_property_of_expo}}\\
            &=\tfrac{1}{2^{2}(a^{-5})^{2}(b^{4})^{2}(c^{3})^{2}}\cdot\tfrac{1}{3^{2}(a^{3})^{2}(b^{-7})^{2}(c^{3})^{2}}&\tag{Property~\ref{property:North_Shore_Exponent_Rules} part~\ref{property:north_shore_distributive_property_of_expo}}\\
            &=\tfrac{1}{4a^{-10}b^{8}c^{6}}\cdot\tfrac{1}{9a^{6}b^{-14}c^{6}}&\tag{Property~\ref{property:North_Shore_Exponent_Rules} part~\ref{property:north_shore_power_property_of_expo}}\\
            &=\tfrac{a^{10}}{4b^{8}c^{6}}\cdot\tfrac{b^{14}}{9a^{6}c^{6}}&\tag{Property~\ref{property:North_Shore_Exponent_Rules} part~\ref{property:north_shore_inverse_property_of_expo}}\\
            &=\tfrac{a^{10}b^{14}}{36a^{6}b^{8}c^{6}c^{6}}&\tag{Multiplication}\\
            &=\tfrac{a^{10}b^{14}}{36a^{6}b^{8}c^{12}}&\tag{Property~\ref{property:North_Shore_Exponent_Rules} part~\ref{property:north_shore_product_property_of_expo}}\\
            &=\tfrac{a^{4}b^{6}}{36c^{12}}&\tag{Division}
        \end{flalign*}
        \end{proof}
        \begin{problem}
        Simplify using only positive exponents: $(-a^{5}b^{7}c^{9})^{4}$
        \end{problem}
        \begin{proof}[Solution]
            \begin{flalign*}
                (-a^{5}b^{7}c^{9})^{4}&=((-1)\cdot a^{5}b^{7}c^{9})^{4}&\tag{Rewrite Expression}\\
                &=(-1)^{4}(a^{5})^{4}(b^{7})^{4}(c^{9})^{4}&\tag{Property~\ref{property:North_Shore_Exponent_Rules} part~\ref{property:north_shore_distributive_property_of_expo}}\\
                &=a^{20}b^{28}c^{36}&\tag{Property~\ref{property:North_Shore_Exponent_Rules} part~\ref{property:north_shore_power_property_of_expo}}
            \end{flalign*}
        \end{proof}
        \begin{problem}
        Simplify using only positive exponents: $\frac{24x^{4}-32x^{3}+16x^{2}}{8x^{2}}$
        \end{problem}
        \begin{proof}[Solution]
            \begin{flalign*}
                \tfrac{24x^{4}-32x^{3}+16x^{2}}{8x^{2}}&=\tfrac{8x^{2}(3x^{2}-4x+2)}{8x^{2}}&\tag{Property~\ref{property:North_Shore_Arithmetic_Properties} part~\ref{property:north_shore_arithmetic_properties_distributive_property}}\\
                &=3x^{2}-4x+2&\tag{Property~\ref{property:North_Shore_Arithmetic_Properties} part~\ref{property:north_shore_arithmetic_properties_mult_inverse}}
            \end{flalign*}
        \end{proof}
        \begin{problem}
        Simplify using only positive exponents: $(x^{2}-5x)(2x^{3}-7)$
        \end{problem}
        \begin{proof}[Solution]
            \begin{flalign*}
                (x^{2}-5x)(2x^{3}-7)&=2x^{5}-7x^{2}-10x^{4}+35x&\tag{\gls{foil}}\\
                &=x(2x^{4}-10x^{3}-7x+35)&\tag{Property~\ref{property:North_Shore_Arithmetic_Properties} part~\ref{property:north_shore_arithmetic_properties_distributive_property}}
            \end{flalign*}
        \end{proof}
        \begin{problem}
        Simplify using only positive exponents: $(4x^{2}y^{6}z)^{2}(-2x^{-2}y^{3}z^{4})^{6}$
        \end{problem}
        \begin{proof}[Solution]
            \begin{flalign*}
                (4x^{2}y^{6}z)^{2}(-x^{-2}y^{3}z^{4})^{6}&=(4^{2}(x^{2})^{2}(y^{6})^{2}(z)^{2})((x^{-2})^{6}(y^{3})^{6}(z^{4})^{6})&\tag{Property~\ref{property:North_Shore_Exponent_Rules} part~\ref{property:north_shore_distributive_property_of_expo}}\\
                &=(16x^{4}y^{12}z^{2})(x^{-12}y^{18}z^{24})&\tag{Property~\ref{property:North_Shore_Exponent_Rules} part~\ref{property:north_shore_power_property_of_expo}}\\
                &=16(x^{4}x^{-12})(y^{12}y^{18})(z^{2}z^{24})&\tag{Property~\ref{property:North_Shore_Arithmetic_Properties} part~\ref{property:north_shore_arithmetic_properties_assoc_mult}}\\
                &=16x^{4-12}y^{18+12}z^{2+24}&\tag{Property~\ref{property:North_Shore_Exponent_Rules} part~\ref{property:north_shore_product_property_of_expo}}\\
                &=16x^{-8}y^{30}z^{26}&\tag{Simplify Exponents}\\ 
                &=\tfrac{16y^{30}z^{26}}{x^{8}}&\tag{Property~\ref{property:North_Shore_Exponent_Rules} part~\ref{property:north_shore_inverse_property_of_expo}}
            \end{flalign*}
        \end{proof}
        \begin{problem}
        Simplify using only positive exponents: $(x^{2}-5x)(2x^{3}-7)$
        \end{problem}
        \begin{proof}[Solution]
            \begin{flalign*}
                (x^{2}-5x)(2x^{3}-7)&=2x^{5}-7x^{2}-10x^{4}+35x&\tag{\gls{foil}}\\
                &=x(2x^{4}-10x^{3}-7x+35)&\tag{Property~\ref{property:North_Shore_Arithmetic_Properties} part~\ref{property:north_shore_arithmetic_properties_distributive_property}}
            \end{flalign*}
        \end{proof}
        \begin{problem}
        Simplify using only positive exponents: $\frac{26a^{2}b^{-5}c^{9}}{-4a^{-6}bc^{9}}$
        \end{problem}
        \begin{proof}[Solution]
            \begin{flalign*}
                \tfrac{26a^{2}b^{-5}c^{9}}{-4a^{-6}bc^{9}}&=26a^2b^{-5}c^{9}\cdot\tfrac{1}{-4a^{-6}bc^{9}}&\tag{Remark~\ref{remark:North_Shore_Rational_Expressions}}\\
                &=\tfrac{26a^{2}c^{9}}{b^{5}}\cdot\tfrac{a^{6}}{-4bc^{9}}&\tag{Property~\ref{property:North_Shore_Exponent_Rules} part~\ref{property:north_shore_inverse_property_of_expo}}\\
                &= -\tfrac{26}{4}\cdot\tfrac{a^{2+6}c^{9}}{b^{5+1}c^{9}}&\tag{Property~\ref{property:North_Shore_Exponent_Rules} part~\ref{property:north_shore_product_property_of_expo}}\\
                &= -\tfrac{13a^{8}}{2b^{6}}&\tag{Property~\ref{property:North_Shore_Arithmetic_Properties} part~\ref{property:north_shore_arithmetic_properties_mult_inverse}}
            \end{flalign*}
        \end{proof}
        \begin{problem}
        Simplify using only positive exponents: $(5a+6)^2$
        \end{problem}
        \begin{proof}[Solution]
            \begin{flalign*}
                (5a+6)^{2}&=25a^{2}+60a+36&\tag{Remark~\ref{remark:north_shore_square_of_a_sum_foil} Eqn.~\ref{equation:North_Shore_square_of_a_sum}}
            \end{flalign*}
        \end{proof}
        \begin{problem}
        Simplify using only positive exponents: $(5x+1)(x+3)$
        \end{problem}
        \begin{proof}[Solution]
            \begin{flalign*}
                (5x+1)(x+3)&=5x^{2}+16x+3&\tag{Eqn.~\ref{equation:north_shore_multiplicative_polynomial_property}}
            \end{flalign*}
        \end{proof}
    \subsection{Factoring}
        \begin{definition}
        A quadratic is an equation of the form $y=ax^{2}+bx+c$.
        \end{definition}
        \begin{definition}
        The Factorization of a quadratic $y=ax^{2}+bx+c$ is an equivalent expression of the form $y=\gamma(x+\alpha)(x+\beta)$.
        \end{definition}
        \begin{theorem}
        \label{theorem:north_shore_factorization_of_quadratic_when_a_is_equal_to_zero}
        If $y=x^{2}+bx+c$ has the factorization $y=(x+\alpha)(x+\beta)$, then $b=\alpha+\beta$ and $c=\alpha\cdot\beta$
        \end{theorem}
        \begin{proof}
        Suppose $y=x^{2}+bx+c$ and $y=(x+\alpha)(x+\beta)$. By \gls{foil}, $y=(x+\alpha)(x+\beta)=x^{2}+x(\alpha+\beta)+\alpha\cdot\beta$. But also $y=x^{2}+bx+c$. But the coefficients must be equal, and therefore $\alpha+\beta=b$ and $\alpha\cdot\beta=c$.
        \end{proof}
        \begin{remark}\label{remark:north_shore_example_of_using_factorization_when_a_equals_zero}
        This result helps us factor quadratic equations when the leading term has a coefficient of $1$ (That is, $a=1$). If we see something like $x^{2}-2x+1$ and we want to factor it, all we need to ask is ``What two numbers add to $-2$ and multiply to $1$?" Often times guessing and checking will get the answer after a few tries. For $x^{2}-2x+1$ We see that $(-1)+(-1)=-2$, and $(-1)\cdot (-1)=1$. So, $x^{2}-2x+1=(x-1)(x-1)=(x-1)^{2}$.
        \end{remark}
        \begin{theorem}[The Difference of Squares]
        \label{theorem:north_shore_difference_of_squares}
        If $a$ and $b$ are real numbers, then $a^{2}-b^{2}=(a-b)(a+b)$
        \end{theorem}
        \begin{proof}
        By \gls{foil}, $(a-b)(a+b)=a^{2}-ab+ba-b^{2}=a^{2}-b^{2}$. Therefore, $(a-b)(a+b)=a^{2}-b^{2}$.
        \end{proof}
        \begin{remark}
        This helps factor quadratics quickly if two squares are being subtracted.
        \end{remark}
        \begin{example}
        Factor the expression: $9x^{2}-64$. Note that $9=3^{2}$, so $9x^{2}=(3x)^{2}$. Also note that $64=8^{2}$. So we can write $9x^{2}-64=(3x)^{2}-(8)^{2}$. Using the difference of squares formula, we have $9x^{2}-64=(3x)^{2}-(8)^{2}=(3x-8)(3x+8)$
        \end{example}
        \begin{example}
        Factor the expression: $16x^{4}-81y^{4}$. While this looks like a ``Quartic Equation" (Equations involving $x^4$), it can be rewriten as a quadratic one. Note that $16=4^{2}$ and $81=9^{2}$. But also $x^{4}=(x^{2})^{2}$ and $y^{4}=(y^{2})^{2}$. So we have $16x^{4}-81y^{4}=(4x^{2})^{2}-(9y^{2})^{2}$. This is a difference of squares, and so we can apply the difference of squares formula. $16x^{4}-81y^{4}=(4x^{2})^{2}-(9y^{2})^{2}=(4x^{2}+9y^{2})(4x^{2}-9y^{2})$. We're not quite done yet, for we can simplify $4x^{2}-9y^{2}$ as well. Note that $4x^{2}=(2x)^{2}$, and $9y^{2}=(3y)^{2}$. So we have $4x^{2}-9y^{2}=(2x)^{2}-(3y)^{2}=(2x-3y)(2x+3y)$. Together, we have: $16x^{4}-81y^{4}=(4x^{2}+9y^{2})(2x-3y)(2x+3y)$.
        \end{example}
        \begin{theorem}[The Difference of Cubes]
        \label{theorem:north_shore_difference_of_cubes}
        If $a$ and $b$ are real numbers, then $a^{3}-b^{3}=(a-b)(a^{2}+ab+b^{2})$
        \end{theorem}
        \begin{proof}
        By the distributive property, $(a-b)(a^{2}+ab+b^{2})=a(a^{2}+ab+b^{2})-b(a^{2}+ab+b^{2})$. Distributing again, we get $a^{3}+a^{2}b+ab^{2}-ba^{2}-ab^{2}-b^{3}$. Rearranging, we have $(a^{3}-b^{3})+(a^{2}b-ba^{2})+(ab^{2}-b^{2}a)$. By the commutative property, $a^{2}b=ba^{2}$, and $ab^{2}=b^{2}a$. Therefore, $(a^{2}b-ba^{2})+(ab^{2}-b^{2}a)=0$. Thus, $(a-b)(a^{2}+ab+b^{2})=a^{3}-b^{3}$.
        \end{proof}
        \begin{remark}
        We can use the difference of cubes formula to factor cubic expressions.
        \end{remark}
        \begin{theorem}[The Sum of Squares]
        \label{theorem:north_shore_sum_of_squares}
        If $x$ and $a$ are real numbers, and if $a$ is non-zero, then there is no \textbf{real} factorization of $x^{2}+a^{2}$.
        \end{theorem}
        \begin{proof}
        If $(x+\alpha)(x+\gamma)$ is a real factorization, then by Thm.~\ref{theorem:north_shore_factorization_of_quadratic_when_a_is_equal_to_zero}, $\alpha+\beta=0$ and $\alpha\cdot\beta=a^{2}$. But then $\alpha=-\beta$ and therefore $-\alpha^{2}=a^{2}$. But $a$ is real, and therefore $a^{2}\geq 0$. But as $\alpha$ is real and non-zero, $\alpha^{2}>0$, and thus $-\alpha^{2}<0$. But $a^{2}=-\alpha^{2}$, a contradiction as $a^{2}\geq 0$. There is no real factorization.
        \end{proof}
        \begin{remark}
        If we see a sum of squares we know that there is no real factorization.
        \end{remark}
        \subsubsection{Problems}
        \begin{problem}
        Factor $x^{2}+5x-6$
        \end{problem}
        \begin{proof}[Solution]
        By Thm.~\ref{theorem:north_shore_factorization_of_quadratic_when_a_is_equal_to_zero}, if $x^{2}+5x-6=(x+\alpha)(x+\beta)$, then $\alpha+\beta=5$ and $\alpha\cdot\beta=-6$. By guessing and checking a few common factors of $5$ and $6$, we get $\alpha=6$ and $\beta=-1$. So, $x^{2}+5x-6=(x+6)(x-1)$
        \end{proof}
        \begin{problem}
        $x^{2}-5x-6$
        \end{problem}
        \begin{proof}[Solution]
        By Thm.~\ref{theorem:north_shore_factorization_of_quadratic_when_a_is_equal_to_zero}, if $x^{2}-5x-6=(x+\alpha)(x+\beta)$, then $\alpha+\beta=-5$ and $\alpha\cdot\beta=-6$. After guessing and checking, we have $\alpha=-6$ and $\beta=1$. So $x^{2}-5x-6=(x-6)(x+1)$
        \end{proof}
        \begin{problem}
        $4x^{2}-36$
        \end{problem}
        \begin{proof}[Solution]
        $4x^{2}-36=4(x^{2}-3^{2})$. By Thm.~\ref{theorem:north_shore_difference_of_squares}, $x^{2}-3^{2}=(x+3)(x-3)$. Thus, $4x^{2}-36=4(x-3)(x+3)$
        \end{proof}
        \begin{problem}
        $x^{2}+4$
        \end{problem}
        \begin{proof}[Solution]
        This is a sum of squares. By Thm.~\ref{theorem:north_shore_sum_of_squares}, there is no real factorization.
        \end{proof}
        \begin{problem}
        $64x^{4}-4y^{4}$
        \end{problem}
        \begin{proof}[Solution]
        Note that $64x^{2}-4y^{2}=(8x^{2})^{2}-(2y^{2})^{2}$. By Thm.~\ref{theorem:north_shore_difference_of_squares}, $(8x^{2})^{2}-(2y^{2})^{2}=(8x^{2}+2y^{2})(8x^{2}-2y^{2})$. Note that $8x^{2}$ and $2y^{2}$ share a common factor of $2$ so we can simplify this as $4(4x^{2}+y^{2})(4x^{2}-y^{2})$. Again by the difference of squares, we have $4x^{2}-y^{2}=(2x)^{2}-y^{2}=(2x-y)(2x+y)$. Piecing this all back together, we have $64x^{2}-4y^{2}=4(4x^{2}+y^{2})(2x-y)(2x+y)$.
        \end{proof}
        \begin{problem}
        $8x^{3}-27$
        \end{problem}
        \begin{proof}[Solution]
        By Thm.~\ref{theorem:north_shore_difference_of_cubes}, $8x^{3}-27=(2x)^{3}-(3)^{3}=(2x-3)(4x^{2}+6x+9)$
        \end{proof}
        \begin{problem}
        $49y^{2}+84y+36$
        \end{problem}
        \begin{proof}[Solution]
        First note that $49=7^{2}$, $36=6^{2}$, and $84=2\cdot 6\cdot 7$. So, $49x^{2}+84x+36=(7x)^{2}+2(7)(6)x+(6)^{2}$. But $(ax+b)^{2}=a^{2}x^{2}+2abx+b^{2}$. We have $a=7,b=6$, so $49x^{2}+84x+36=(7x+6)^{2}$.
        \end{proof}
        \begin{problem}
        $12x^{2}+12x+3$
        \end{problem}
        \begin{proof}[Solution]
        $4x^{2}+4x+1=3((2x)^{2}+2(2)(1)(x)+(1)^{2})=3(2x+1)^{2}$
        \end{proof}
    \subsection{Quadratic Expressions}
        \begin{theorem}[Completing the Square]
        \label{theorem:north_shore_completing_the_square}
        If $y=ax^{2}+bx+c$, then $y=a(x+\frac{b}{2a})^{2}-\frac{b^2}{4a}+c$
        \end{theorem}
        \begin{proof}
        For $(x+\frac{b}{2a})^{2}=x^{2}+\frac{b}{a}x+\frac{b^{2}}{4a^{2}}$. But then $a(x+\frac{b}{2a})^{2}-\frac{b^{2}}{4a}+c=ax^{2}+bx+c$
        \end{proof}
        \begin{theorem}[The Quadratic Formula]
        \label{theorem:north_shore_quadratic_formula_theorem}
        If $ax^{2}+bx+c=0$, $a\ne 0$, then $x=\frac{-b\pm\sqrt{b^{2}-4ac}}{2a}$
        \end{theorem}
        \begin{proof}
        By Thm.~\ref{theorem:north_shore_completing_the_square}, $ax^{2}+bx+c=a(x+\frac{b}{2a})^{2}-\frac{b^{2}}{4a}+c$. But $ax^{2}+bx+c=0$, so $a(x+\frac{b}{2a}x)^{2}-\frac{b^{2}}{4a}+c=0$. Therefore $a(x+\frac{b}{2a})^{2}=\frac{b^2}{4a}-c$. But $c=\frac{4ac}{4a}$, so $a(x^2+\frac{b}{2a})^{2}=\frac{b^{2}-4ac}{4a}$. Dividing both sides by $a$, we get $(x+\frac{b}{2a})^{2}=\frac{b^{2}-4ac}{4a^{2}}$. Now we take square roots, but note that there are two possible square roots. So, $x+\frac{b}{2a}=\pm\frac{\sqrt{b^{2}-4ac}}{2a}$. Subtracting $\frac{b}{2a}$ from both sides we get $x=-\frac{b}{2a}\pm\frac{\sqrt{b^{2}-4ac}}{2a}=\frac{-b\pm\sqrt{b^{2}-4ac}}{2a}$
        \end{proof}
        \begin{remark}
        Remember that $ax^{2}+bx+c=0$ has two solutions: $x=\frac{-b+\sqrt{b^{2}-4ac}}{2a}$ and $x=\frac{-b-\sqrt{b^{2}-4ac}}{2a}$
        \end{remark}
        \begin{theorem}
        \label{theorem:north_shore_zeros_of_a_factored_polynomial}
        If $\gamma(x-\alpha)(x-\beta)=0$, then either $x=\alpha$ or $x=\beta$.
        \end{theorem}
        \begin{remark}
        If we see $ax^{2}+bx+c=0$ and can \textit{factor} into $\gamma(x-\alpha)(x-\beta)$, then $x=\alpha$ or $x=\beta$.
        \end{remark}
        \subsubsection{Problems}
        \begin{problem}
        Find all solutions for a: $4a^{2}+9a+2=0$
        \end{problem}
        \begin{proof}[Solution]
        By Thm.~\ref{theorem:north_shore_quadratic_formula_theorem}, $a=\frac{-9\pm\sqrt{9^{2}-4\cdot 4\cdot 2}}{2\cdot 4}=\frac{-9\pm 7}{8}$. $a=-2$ or $a=-\frac{1}{4}$
        \end{proof}
        \begin{problem}
        Find all solutions for $x$: $9x^{2}-81=0$
        \end{problem}
        \begin{proof}[Solution]
        By Thm.~\ref{theorem:north_shore_difference_of_squares}, $9x^{2}-81=9(x-3)(x+3)$. By Thm.~\ref{theorem:north_shore_zeros_of_a_factored_polynomial}, $x=3$ or $x=-3$.
        \end{proof}
        \begin{problem}
        Find all solutions for $x$: $25x^{2}-6=30$.
        \end{problem}
        \begin{proof}[Solution]
        Subtract $30$ to get $25x^{2}-36=0$. By Thm.~\ref{theorem:north_shore_difference_of_squares}, $25x^{2}-36 =(5x-6)(5x+6)$. By Thm.~\ref{theorem:north_shore_zeros_of_a_factored_polynomial}, $5x=6$ or $5x=-6$. $x=\frac{6}{5}$ or $x=-\frac{6}{5}$.
        \end{proof}
        \begin{problem}
        Find all solutions for $x$: $3x^{2}-5x-2=0$
        \end{problem}
        \begin{proof}[Solution]
        By Thm.~\ref{theorem:north_shore_quadratic_formula_theorem}, $x=\frac{5\pm\sqrt{(-5)^{2}-4(3)(-2)}}{2(3)}=\frac{5\pm 7}{6}$. So $x=2$ or $x=-\frac{1}{3}$
        \end{proof}
        \begin{problem}
        Find all solutions for $x$: $(3x+2)^{2}=16$
        \end{problem}
        \begin{proof}[Solution]
        Subtracting $16$ we get $(3x+2)^{2}-16$. By Thm.~\ref{theorem:north_shore_difference_of_squares}, $(3x+2)^{2}-16=(3x+6)(3x-2)$. By Thm.~\ref{theorem:north_shore_zeros_of_a_factored_polynomial}, $3x=-6$ or $3x=2$. Thus, $x=-2$ or $x=\frac{2}{3}$.
        \end{proof}
        \begin{problem}
        Find all solutions for $r$: $r^{2}-2r-4=0$
        \end{problem}
        \begin{proof}[Solution]
        By Thm.~\ref{theorem:north_shore_quadratic_formula_theorem}, $r=\frac{2\pm\sqrt{(-2)^{2}-4(1)(-4)}}{2(1)}=1\pm\sqrt{5}$. $x=1+\sqrt{5}$ or $x=1-\sqrt{5}$
        \end{proof}
    \subsection{Rational Expressions}
        \begin{theorem}[Cross Multiplying]
        \label{theorem:north_shore_cross_multiplying}
        If $a,b,c,d$ are real numbers, $b\ne 0$, $d\ne 0$, then $\frac{a}{b}+\frac{c}{d}=\frac{ad+bc}{bd}$
        \end{theorem}
        \begin{definition}
        \label{definition:north_shore_numerator_of_rational_expression}
        The numerator of a rational expression $\frac{P(x)}{Q(x)}$ is the polynomial $P(x)$.
        \end{definition}
        \begin{definition}
        \label{definition:north_shore_denominator_of_rational_expression}
        The denominator of a rational expression $\frac{P(x)}{Q(x)}$ is the polynomial $Q(x)$.
        \end{definition}
        \subsubsection{Problems}
        \begin{problem}
        Simplify: $\frac{4}{2a-2}+\frac{3a}{a^{2}-a}$
        \end{problem}
        \begin{proof}[Solution]
            \begin{flalign*}
                \tfrac{4}{2a-2}+\tfrac{3a}{a^{2}-a}&=\tfrac{2}{a-1}+\tfrac{3}{a-1}&\tag{Factor and Simplify Terms}\\
                &=\tfrac{5}{a-1}&\tag{Addition}
            \end{flalign*}
        \end{proof}
        \begin{problem}
        Simplify: $\frac{3}{x^{2}-1}-\frac{4}{x^{2}+3x+2}$
        \end{problem}
        \begin{proof}[Solution]
            \begin{flalign*}
                \tfrac{3}{x^{2}-1}-\tfrac{4}{x^{2}+3x+2}&=\tfrac{3}{(x-1)(x+1)}-\tfrac{4}{(x+2)(x+1)}&\tag{Factor Denominators}\\
                &=\tfrac{3(x+2)-4(x-1)}{(x-1)(x+2)(x+1)}&\tag{Thm.~\ref{theorem:north_shore_cross_multiplying}}\\
                &=\tfrac{10-x}{(x+1)(x+2)(x-1)}&\tag{Simplify Numerator}
            \end{flalign*}
        \end{proof}
        \begin{problem}
        Simplify: $\frac{\frac{2}{x}-\frac{1}{y}}{\frac{1}{xy}}$
        \end{problem}
        \begin{proof}[Solution]
            \begin{flalign*}
                \tfrac{\frac{2}{x}-\frac{1}{y}}{\frac{1}{xy}}&=xy(\tfrac{2}{x}-\tfrac{1}{y})&\tag{Property~\ref{property:North_Shore_Exponent_Rules} part~\ref{property:north_shore_inverse_property_of_expo}}\\
                &=2y-x&\tag{Multiplication}
            \end{flalign*}
        \end{proof}
        \begin{problem}
        Simplify: $\frac{1}{1-\sqrt{x}}+\frac{1}{1+\sqrt{x}}$
        \end{problem}
        \begin{proof}[Solution]
            \begin{flalign*}
                \tfrac{1}{1-\sqrt{x}}+\tfrac{1}{1+\sqrt{x}}&=\tfrac{1+\sqrt{x} + 1-\sqrt{x}}{(1-\sqrt{x})(1+\sqrt{x})}&\tag{Thm.~\ref{theorem:north_shore_cross_multiplying}}\\
                &=\tfrac{2}{1-x}&\tag{Simplify and \gls{foil}}
            \end{flalign*}
        \end{proof}
        \newpage
        \begin{problem}
        $\frac{2}{x-1}+\frac{1}{x+1}=\frac{5}{4}$
        \end{problem}
        \begin{proof}[Solution]
            \begin{flalign*}
                \tfrac{2}{x-1}+\tfrac{1}{x+1}&=\tfrac{2(x+1)+(x-1)}{(x+1)(x-1)}&\tag{Thm.~\ref{theorem:north_shore_cross_multiplying}}\\
                &=\tfrac{3x+1}{(x+1)(x-1)}&\tag{Simplify the Numerator}\\
                \Rightarrow \tfrac{3x+1}{(x+1)(x-1)}&=\tfrac{5}{4}&\tag{Substitution}\\
                \Rightarrow 3x+1&=\tfrac{5}{4}(x+1)(x-1)&\tag{Multiply Both Sides by $(x+1)(x-1)$}\\
                \Rightarrow 12x+4&=5x^{2}-5&\tag{Multiply Both Sides by $4$ and Simplify}\\
                \Rightarrow 5x^{2}-12x-9&=0&\tag{Subtract $12x+4$ from Both Sides}\\
                \Rightarrow x&=\tfrac{12\pm\sqrt{(12)^{2}-4\cdot (5)\cdot(-9)}}{2\cdot(5)}&\tag{Thm.~\ref{theorem:north_shore_quadratic_formula_theorem}}\\
                \Rightarrow x&=\tfrac{12\pm\sqrt{324}}{10}&\tag{Simplify}\\
                \Rightarrow x&=\tfrac{12\pm 18}{10}&\tag{Simplify}\\
                \Rightarrow x&=3\textrm{ or }-\tfrac{3}{5}&\tag{Simplify}
            \end{flalign*}
        \end{proof}
    \subsection{Graphing}
        \subsubsection{Problems}
        \begin{problem}
        \label{problem:north_shore_exam_graph_everything}Graph the following:
        \begin{enumerate}
        \begin{multicols}{4}
            \item $3x-2y=6$
            \item $x=-3$
            \item $y=2$
            \item $y=-\frac{2}{3}x+5$
            \item $y=|x-3|$
            \item $y=-x^2+2$
            \item $y=\sqrt{2-x}$
            \item $y=x-10$
        \end{multicols}
        \end{enumerate}
        \end{problem}
        \begin{figure}[H]
            \centering
            \captionsetup{type=figure}
            \subimport{../../../../tikz/}{Elementary_Algebra_Graphing_Problem}
            \caption{The chaos that is the solution to problem
                     \ref{problem:north_shore_exam_graph_everything}.}
            \label{fig:north_shore_graphing_problem}
        \end{figure}
    \subsection{Systems of Equations}
        A system of equations is a set of 2 or more equations involving the same variables. Solving systems of linear equations is one of the main focuses in the study of linear algebra.
        \begin{example}
        \label{example:north_shore_example_of_a_system_of_linear_equations}Consider the following system of linear equations:
        \begin{align*}
            2x+3y&=1\\
            x+6y&=2
        \end{align*}
        Solving this system of equations asks ``Which ordered pairs $(x,y)$ solve both of these equations?" The second equations says that $x=2-6y$. We can substitute this back into the first equation to get $2(2-6y)+3y=1$ which simplifies to $4-9y=1$. The solution to this is $y=\frac{1}{3}$. Solving for $x$, we have $x=2-6y=2-6(\frac{1}{3})=2-2=0$. $(0,\frac{1}{3})$ is a solution.
        \end{example}
        We can picture these types of problems graphically as well. For the example above there are two linear equations, which can be represented graphically as lines. A solution to this system can then be interpreted as a point where the two lines intersect. The graph of the two equations in example \ref{example:north_shore_example_of_a_system_of_linear_equations} are shown in Fig.~\ref{fig:north_shore_systems_of_linear_equations}.
        \begin{figure}[H]
            \centering
            \captionsetup{type=figure}
            \subimport{../../../../tikz/}{System_of_Equations_Geometry}
            \caption{The graphs of the two equations shown in example \ref{example:north_shore_example_of_a_system_of_linear_equations}.}
            \label{fig:north_shore_systems_of_linear_equations}
        \end{figure}
        Any system of linear equations has 3 possible outcomes: No solutions, one solution, or infinitely many solutions. This makes sense if one considers the few possibilities allowed. Given two parallel lines there will be no solutions for parallel lines never intersect. Given two equations that represent the same line there will be infinitely many solutions, for any point on the line will work. Given two equations for two distinct non-parallel lines there will be only one solution, for such lines can only intersect once.
        \subsubsection{Problems}
        \begin{problem}
        Solve the following system of equations:
        \begin{align*}
            2x-3y&=-12\\
            x-2y&=-9
        \end{align*}
        \end{problem}
        \begin{proof}[Solution]
        Subtracting 2 of the second equation from first equation, we get:
        \begin{equation*}
            (2x-3y)-2(x-2y)=-12-2(-9)
        \end{equation*}
        Which simplifies to $y=6$. The second equation yield $x-2(6)=-9\Rightarrow x=3$. The solution is $(3,6)$
        \end{proof}
        \begin{problem}
        Solve the following system of equations:
        \begin{align*}
            4x+6y&=10\\
            2x+3y&=5
        \end{align*}
        \end{problem}
        \begin{proof}[Solution]
        Dividing both sides of the first equation by 2 gives us the second equation. These equations represent the same line, and there are infinitely many solutions: $y=\frac{5}{3}-\frac{2}{3}x$
        \end{proof}
        \begin{problem}
        Solve the following system of equations:
        \begin{align*}
            x+2y&=5\\
            x+2y&=7
        \end{align*}
        \end{problem}
        \begin{proof}[Solution]
        Let $z=x+2y$. Then $z=5$ and $z=7$. But this is impossible, for $5\ne 7$. No solutions.
        \end{proof}
        \begin{problem}
        \begin{align*}
            2x-3y&=-4\\
            2x+y\phantom{3}&=\phantom{-}4
        \end{align*}
        \end{problem}
        \begin{proof}[Solution]
        Subtracting the first equation from the second gives us $4y=8\Rightarrow y=2$. But then from the first equation we obtain $2x=3(2)-4=2\Rightarrow x=1$. The solution is $(1,2)$.
        \end{proof}
    \subsection{Radicals}
        To simplify expressions with radicals we use something called the conjugate of a radical expression. 
        \begin{definition}
        The conjugate of a rational expression $\sqrt{a}-\sqrt{b}$ is $\sqrt{a}+\sqrt{b}$.
        \end{definition}
        Using the \gls{foil} rule from before, we can obtain the following theorem.
        \begin{theorem}
        If $a,b\geq 0$, and if $a\ne b$, then $\frac{1}{\sqrt{a}-\sqrt{b}}=\frac{\sqrt{a}+\sqrt{b}}{a-b}$
        \end{theorem}
        \begin{proof}
        As $a,b\geq 0$ and $a\ne b$, $\frac{1}{\sqrt{a}-\sqrt{b}}$ is well-defined and $\sqrt{a}+\sqrt{b}\ne 0$. But then:
        \begin{align*}
            \tfrac{1}{\sqrt{a}-\sqrt{b}}&=\tfrac{1}{\sqrt{a}-\sqrt{b}}\tfrac{\sqrt{a}+\sqrt{b}}{\sqrt{a}+\sqrt{b}}\\
            &=\tfrac{\sqrt{a}+\sqrt{b}}{(\sqrt{a}-\sqrt{b})(\sqrt{a}+\sqrt{b})}\\
            &=\tfrac{\sqrt{a}+\sqrt{b}}{a-b}&\tag{Thm.~\ref{theorem:north_shore_difference_of_squares}}
        \end{align*}
        \end{proof}
        \subsubsection{Problems}
        \begin{problem}
        Simplify the following so that there are no radicals in the denominator:
        \begin{enumerate}
        \begin{multicols}{3}
            \item $\sqrt{8}\sqrt{10}$
            \item $\sqrt[4]{\frac{81}{x^4}}$
            \item $\sqrt{\frac{4}{3}}$
            \item $\sqrt{\frac{12}{18}}$
            \item $\sqrt[3]{24x^{3}y^{6}}$
            \item $\frac{\sqrt{3}}{5-\sqrt{3}}$
        \end{multicols}
        \end{enumerate}
        \end{problem}
        \begin{proof}[Solution]
        \
        \begin{enumerate}
        \begin{multicols}{3}
            \item $\sqrt{8}\sqrt{10}=\sqrt{16\cdot 5}=4\sqrt{5}$
            \item $\sqrt[4]{\frac{81}{x^{4}}}=\frac{3}{|x|}$
            \item $\sqrt{\frac{4}{3}}=\frac{2}{\sqrt{3}}=\frac{2\sqrt{3}}{3}$
            \item $\sqrt{\frac{12}{18}}=\frac{2\sqrt{3}}{3\sqrt{2}}=\frac{2\sqrt{2}\sqrt{3}}{6}$
            \item $\sqrt[3]{24x^{3}y^{6}}=2\sqrt[3]{3}xy^{2}$
            \item $\frac{\sqrt{3}}{5-\sqrt{3}}=\frac{\sqrt{3}(5+\sqrt{3})}{5-3}=\frac{5\sqrt{3}+3}{2}$
        \end{multicols}
        \end{enumerate}
        \end{proof}
    \ifx\ifmathcourseselementaryalgebra\undefined
        \newpage
        \printglossary[type=\acronymtype]
        \newpage
        \printglossary[style=long]
    \fi
\end{document}