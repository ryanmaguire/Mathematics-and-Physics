\documentclass[crop=false,class=book,oneside]{standalone}
%----------------------------Preamble-------------------------------%
%---------------------------Packages----------------------------%
\usepackage{geometry}
\geometry{b5paper, margin=1.0in}
\usepackage[T1]{fontenc}
\usepackage{graphicx, float}            % Graphics/Images.
\usepackage{natbib}                     % For bibliographies.
\bibliographystyle{agsm}                % Bibliography style.
\usepackage[french, english]{babel}     % Language typesetting.
\usepackage[dvipsnames]{xcolor}         % Color names.
\usepackage{listings}                   % Verbatim-Like Tools.
\usepackage{mathtools, esint, mathrsfs} % amsmath and integrals.
\usepackage{amsthm, amsfonts, amssymb}  % Fonts and theorems.
\usepackage{tcolorbox}                  % Frames around theorems.
\usepackage{upgreek}                    % Non-Italic Greek.
\usepackage{fmtcount, etoolbox}         % For the \book{} command.
\usepackage[newparttoc]{titlesec}       % Formatting chapter, etc.
\usepackage{titletoc}                   % Allows \book in toc.
\usepackage[nottoc]{tocbibind}          % Bibliography in toc.
\usepackage[titles]{tocloft}            % ToC formatting.
\usepackage{pgfplots, tikz}             % Drawing/graphing tools.
\usepackage{imakeidx}                   % Used for index.
\usetikzlibrary{
    calc,                   % Calculating right angles and more.
    angles,                 % Drawing angles within triangles.
    arrows.meta,            % Latex and Stealth arrows.
    quotes,                 % Adding labels to angles.
    positioning,            % Relative positioning of nodes.
    decorations.markings,   % Adding arrows in the middle of a line.
    patterns,
    arrows
}                                       % Libraries for tikz.
\pgfplotsset{compat=1.9}                % Version of pgfplots.
\usepackage[font=scriptsize,
            labelformat=simple,
            labelsep=colon]{subcaption} % Subfigure captions.
\usepackage[font={scriptsize},
            hypcap=true,
            labelsep=colon]{caption}    % Figure captions.
\usepackage[pdftex,
            pdfauthor={Ryan Maguire},
            pdftitle={Mathematics and Physics},
            pdfsubject={Mathematics, Physics, Science},
            pdfkeywords={Mathematics, Physics, Computer Science, Biology},
            pdfproducer={LaTeX},
            pdfcreator={pdflatex}]{hyperref}
\hypersetup{
    colorlinks=true,
    linkcolor=blue,
    filecolor=magenta,
    urlcolor=Cerulean,
    citecolor=SkyBlue
}                           % Colors for hyperref.
\usepackage[toc,acronym,nogroupskip,nopostdot]{glossaries}
\usepackage{glossary-mcols}
%------------------------Theorem Styles-------------------------%
\theoremstyle{plain}
\newtheorem{theorem}{Theorem}[section]

% Define theorem style for default spacing and normal font.
\newtheoremstyle{normal}
    {\topsep}               % Amount of space above the theorem.
    {\topsep}               % Amount of space below the theorem.
    {}                      % Font used for body of theorem.
    {}                      % Measure of space to indent.
    {\bfseries}             % Font of the header of the theorem.
    {}                      % Punctuation between head and body.
    {.5em}                  % Space after theorem head.
    {}

% Italic header environment.
\newtheoremstyle{thmit}{\topsep}{\topsep}{}{}{\itshape}{}{0.5em}{}

% Define environments with italic headers.
\theoremstyle{thmit}
\newtheorem*{solution}{Solution}

% Define default environments.
\theoremstyle{normal}
\newtheorem{example}{Example}[section]
\newtheorem{definition}{Definition}[section]
\newtheorem{problem}{Problem}[section]

% Define framed environment.
\tcbuselibrary{most}
\newtcbtheorem[use counter*=theorem]{ftheorem}{Theorem}{%
    before=\par\vspace{2ex},
    boxsep=0.5\topsep,
    after=\par\vspace{2ex},
    colback=green!5,
    colframe=green!35!black,
    fonttitle=\bfseries\upshape%
}{thm}

\newtcbtheorem[auto counter, number within=section]{faxiom}{Axiom}{%
    before=\par\vspace{2ex},
    boxsep=0.5\topsep,
    after=\par\vspace{2ex},
    colback=Apricot!5,
    colframe=Apricot!35!black,
    fonttitle=\bfseries\upshape%
}{ax}

\newtcbtheorem[use counter*=definition]{fdefinition}{Definition}{%
    before=\par\vspace{2ex},
    boxsep=0.5\topsep,
    after=\par\vspace{2ex},
    colback=blue!5!white,
    colframe=blue!75!black,
    fonttitle=\bfseries\upshape%
}{def}

\newtcbtheorem[use counter*=example]{fexample}{Example}{%
    before=\par\vspace{2ex},
    boxsep=0.5\topsep,
    after=\par\vspace{2ex},
    colback=red!5!white,
    colframe=red!75!black,
    fonttitle=\bfseries\upshape%
}{ex}

\newtcbtheorem[auto counter, number within=section]{fnotation}{Notation}{%
    before=\par\vspace{2ex},
    boxsep=0.5\topsep,
    after=\par\vspace{2ex},
    colback=SeaGreen!5!white,
    colframe=SeaGreen!75!black,
    fonttitle=\bfseries\upshape%
}{not}

\newtcbtheorem[use counter*=remark]{fremark}{Remark}{%
    fonttitle=\bfseries\upshape,
    colback=Goldenrod!5!white,
    colframe=Goldenrod!75!black}{ex}

\newenvironment{bproof}{\textit{Proof.}}{\hfill$\square$}
\tcolorboxenvironment{bproof}{%
    blanker,
    breakable,
    left=3mm,
    before skip=5pt,
    after skip=10pt,
    borderline west={0.6mm}{0pt}{green!80!black}
}

\AtEndEnvironment{lexample}{$\hfill\textcolor{red}{\blacksquare}$}
\newtcbtheorem[use counter*=example]{lexample}{Example}{%
    empty,
    title={Example~\theexample},
    boxed title style={%
        empty,
        size=minimal,
        toprule=2pt,
        top=0.5\topsep,
    },
    coltitle=red,
    fonttitle=\bfseries,
    parbox=false,
    boxsep=0pt,
    before=\par\vspace{2ex},
    left=0pt,
    right=0pt,
    top=3ex,
    bottom=1ex,
    before=\par\vspace{2ex},
    after=\par\vspace{2ex},
    breakable,
    pad at break*=0mm,
    vfill before first,
    overlay unbroken={%
        \draw[red, line width=2pt]
            ([yshift=-1.2ex]title.south-|frame.west) to
            ([yshift=-1.2ex]title.south-|frame.east);
        },
    overlay first={%
        \draw[red, line width=2pt]
            ([yshift=-1.2ex]title.south-|frame.west) to
            ([yshift=-1.2ex]title.south-|frame.east);
    },
}{ex}

\AtEndEnvironment{ldefinition}{$\hfill\textcolor{Blue}{\blacksquare}$}
\newtcbtheorem[use counter*=definition]{ldefinition}{Definition}{%
    empty,
    title={Definition~\thedefinition:~{#1}},
    boxed title style={%
        empty,
        size=minimal,
        toprule=2pt,
        top=0.5\topsep,
    },
    coltitle=Blue,
    fonttitle=\bfseries,
    parbox=false,
    boxsep=0pt,
    before=\par\vspace{2ex},
    left=0pt,
    right=0pt,
    top=3ex,
    bottom=0pt,
    before=\par\vspace{2ex},
    after=\par\vspace{1ex},
    breakable,
    pad at break*=0mm,
    vfill before first,
    overlay unbroken={%
        \draw[Blue, line width=2pt]
            ([yshift=-1.2ex]title.south-|frame.west) to
            ([yshift=-1.2ex]title.south-|frame.east);
        },
    overlay first={%
        \draw[Blue, line width=2pt]
            ([yshift=-1.2ex]title.south-|frame.west) to
            ([yshift=-1.2ex]title.south-|frame.east);
    },
}{def}

\AtEndEnvironment{ltheorem}{$\hfill\textcolor{Green}{\blacksquare}$}
\newtcbtheorem[use counter*=theorem]{ltheorem}{Theorem}{%
    empty,
    title={Theorem~\thetheorem:~{#1}},
    boxed title style={%
        empty,
        size=minimal,
        toprule=2pt,
        top=0.5\topsep,
    },
    coltitle=Green,
    fonttitle=\bfseries,
    parbox=false,
    boxsep=0pt,
    before=\par\vspace{2ex},
    left=0pt,
    right=0pt,
    top=3ex,
    bottom=-1.5ex,
    breakable,
    pad at break*=0mm,
    vfill before first,
    overlay unbroken={%
        \draw[Green, line width=2pt]
            ([yshift=-1.2ex]title.south-|frame.west) to
            ([yshift=-1.2ex]title.south-|frame.east);},
    overlay first={%
        \draw[Green, line width=2pt]
            ([yshift=-1.2ex]title.south-|frame.west) to
            ([yshift=-1.2ex]title.south-|frame.east);
    }
}{thm}

%--------------------Declared Math Operators--------------------%
\DeclareMathOperator{\adjoint}{adj}         % Adjoint.
\DeclareMathOperator{\Card}{Card}           % Cardinality.
\DeclareMathOperator{\curl}{curl}           % Curl.
\DeclareMathOperator{\diam}{diam}           % Diameter.
\DeclareMathOperator{\dist}{dist}           % Distance.
\DeclareMathOperator{\Div}{div}             % Divergence.
\DeclareMathOperator{\Erf}{Erf}             % Error Function.
\DeclareMathOperator{\Erfc}{Erfc}           % Complementary Error Function.
\DeclareMathOperator{\Ext}{Ext}             % Exterior.
\DeclareMathOperator{\GCD}{GCD}             % Greatest common denominator.
\DeclareMathOperator{\grad}{grad}           % Gradient
\DeclareMathOperator{\Ima}{Im}              % Image.
\DeclareMathOperator{\Int}{Int}             % Interior.
\DeclareMathOperator{\LC}{LC}               % Leading coefficient.
\DeclareMathOperator{\LCM}{LCM}             % Least common multiple.
\DeclareMathOperator{\LM}{LM}               % Leading monomial.
\DeclareMathOperator{\LT}{LT}               % Leading term.
\DeclareMathOperator{\Mod}{mod}             % Modulus.
\DeclareMathOperator{\Mon}{Mon}             % Monomial.
\DeclareMathOperator{\multideg}{mutlideg}   % Multi-Degree (Graphs).
\DeclareMathOperator{\nul}{nul}             % Null space of operator.
\DeclareMathOperator{\Ord}{Ord}             % Ordinal of ordered set.
\DeclareMathOperator{\Prin}{Prin}           % Principal value.
\DeclareMathOperator{\proj}{proj}           % Projection.
\DeclareMathOperator{\Refl}{Refl}           % Reflection operator.
\DeclareMathOperator{\rk}{rk}               % Rank of operator.
\DeclareMathOperator{\sgn}{sgn}             % Sign of a number.
\DeclareMathOperator{\sinc}{sinc}           % Sinc function.
\DeclareMathOperator{\Span}{Span}           % Span of a set.
\DeclareMathOperator{\Spec}{Spec}           % Spectrum.
\DeclareMathOperator{\supp}{supp}           % Support
\DeclareMathOperator{\Tr}{Tr}               % Trace of matrix.
%--------------------Declared Math Symbols--------------------%
\DeclareMathSymbol{\minus}{\mathbin}{AMSa}{"39} % Unary minus sign.
%------------------------New Commands---------------------------%
\DeclarePairedDelimiter\norm{\lVert}{\rVert}
\DeclarePairedDelimiter\ceil{\lceil}{\rceil}
\DeclarePairedDelimiter\floor{\lfloor}{\rfloor}
\newcommand*\diff{\mathop{}\!\mathrm{d}}
\newcommand*\Diff[1]{\mathop{}\!\mathrm{d^#1}}
\renewcommand*{\glstextformat}[1]{\textcolor{RoyalBlue}{#1}}
\renewcommand{\glsnamefont}[1]{\textbf{#1}}
\renewcommand\labelitemii{$\circ$}
\renewcommand\thesubfigure{%
    \arabic{chapter}.\arabic{figure}.\arabic{subfigure}}
\addto\captionsenglish{\renewcommand{\figurename}{Fig.}}
\numberwithin{equation}{section}

\renewcommand{\vector}[1]{\boldsymbol{\mathrm{#1}}}

\newcommand{\uvector}[1]{\boldsymbol{\hat{\mathrm{#1}}}}
\newcommand{\topspace}[2][]{(#2,\tau_{#1})}
\newcommand{\measurespace}[2][]{(#2,\varSigma_{#1},\mu_{#1})}
\newcommand{\measurablespace}[2][]{(#2,\varSigma_{#1})}
\newcommand{\manifold}[2][]{(#2,\tau_{#1},\mathcal{A}_{#1})}
\newcommand{\tanspace}[2]{T_{#1}{#2}}
\newcommand{\cotanspace}[2]{T_{#1}^{*}{#2}}
\newcommand{\Ckspace}[3][\mathbb{R}]{C^{#2}(#3,#1)}
\newcommand{\funcspace}[2][\mathbb{R}]{\mathcal{F}(#2,#1)}
\newcommand{\smoothvecf}[1]{\mathfrak{X}(#1)}
\newcommand{\smoothonef}[1]{\mathfrak{X}^{*}(#1)}
\newcommand{\bracket}[2]{[#1,#2]}

%------------------------Book Command---------------------------%
\makeatletter
\renewcommand\@pnumwidth{1cm}
\newcounter{book}
\renewcommand\thebook{\@Roman\c@book}
\newcommand\book{%
    \if@openright
        \cleardoublepage
    \else
        \clearpage
    \fi
    \thispagestyle{plain}%
    \if@twocolumn
        \onecolumn
        \@tempswatrue
    \else
        \@tempswafalse
    \fi
    \null\vfil
    \secdef\@book\@sbook
}
\def\@book[#1]#2{%
    \refstepcounter{book}
    \addcontentsline{toc}{book}{\bookname\ \thebook:\hspace{1em}#1}
    \markboth{}{}
    {\centering
     \interlinepenalty\@M
     \normalfont
     \huge\bfseries\bookname\nobreakspace\thebook
     \par
     \vskip 20\p@
     \Huge\bfseries#2\par}%
    \@endbook}
\def\@sbook#1{%
    {\centering
     \interlinepenalty \@M
     \normalfont
     \Huge\bfseries#1\par}%
    \@endbook}
\def\@endbook{
    \vfil\newpage
        \if@twoside
            \if@openright
                \null
                \thispagestyle{empty}%
                \newpage
            \fi
        \fi
        \if@tempswa
            \twocolumn
        \fi
}
\newcommand*\l@book[2]{%
    \ifnum\c@tocdepth >-3\relax
        \addpenalty{-\@highpenalty}%
        \addvspace{2.25em\@plus\p@}%
        \setlength\@tempdima{3em}%
        \begingroup
            \parindent\z@\rightskip\@pnumwidth
            \parfillskip -\@pnumwidth
            {
                \leavevmode
                \Large\bfseries#1\hfill\hb@xt@\@pnumwidth{\hss#2}
            }
            \par
            \nobreak
            \global\@nobreaktrue
            \everypar{\global\@nobreakfalse\everypar{}}%
        \endgroup
    \fi}
\newcommand\bookname{Book}
\renewcommand{\thebook}{\texorpdfstring{\Numberstring{book}}{book}}
\providecommand*{\toclevel@book}{-2}
\makeatother
\titleformat{\part}[display]
    {\Large\bfseries}
    {\partname\nobreakspace\thepart}
    {0mm}
    {\Huge\bfseries}
\titlecontents{part}[0pt]
    {\large\bfseries}
    {\partname\ \thecontentslabel: \quad}
    {}
    {\hfill\contentspage}
\titlecontents{chapter}[0pt]
    {\bfseries}
    {\chaptername\ \thecontentslabel:\quad}
    {}
    {\hfill\contentspage}
\newglossarystyle{longpara}{%
    \setglossarystyle{long}%
    \renewenvironment{theglossary}{%
        \begin{longtable}[l]{{p{0.25\hsize}p{0.65\hsize}}}
    }{\end{longtable}}%
    \renewcommand{\glossentry}[2]{%
        \glstarget{##1}{\glossentryname{##1}}%
        &\glossentrydesc{##1}{~##2.}
        \tabularnewline%
        \tabularnewline
    }%
}
\newglossary[not-glg]{notation}{not-gls}{not-glo}{Notation}
\newcommand*{\newnotation}[4][]{%
    \newglossaryentry{#2}{type=notation, name={\textbf{#3}, },
                          text={#4}, description={#4},#1}%
}
%--------------------------LENGTHS------------------------------%
% Spacings for the Table of Contents.
\addtolength{\cftsecnumwidth}{1ex}
\addtolength{\cftsubsecindent}{1ex}
\addtolength{\cftsubsecnumwidth}{1ex}
\addtolength{\cftfignumwidth}{1ex}
\addtolength{\cfttabnumwidth}{1ex}

% Indent and paragraph spacing.
\setlength{\parindent}{0em}
\setlength{\parskip}{0em}
\graphicspath{{../../../images/}}   % Path to Image Folder.
%--------------------------Main Document----------------------------%
\begin{document}
    \ifx\ifmathcourses\undefined
        \pagenumbering{roman}
        \title{Measure Theory}
        \author{Ryan Maguire}
        \date{\vspace{-5ex}}
        \maketitle
        \tableofcontents
        \clearpage
        \chapter*{Measure Theory}
        \addcontentsline{toc}{chapter}{Measure Theory}
        \markboth{}{MEASURE THEORY}
        \vspace{10ex}
        \setcounter{chapter}{1}
        \pagenumbering{arabic}
    \else
        \chapter{Measure Theory}
    \fi
    \section{A Review of Set Theory}
        \subsection{Cardinality}
            We begin by talking about cardinality. This is the
            \textit{size} of a set. For an infinite set, it
            doesn't make sense to talk about the \textit{number}
            of elements, but we can specify what it means two sets
            to have the same size. Sets $A$ and $B$ are equivalent
            if there exists a bijection $f:A\rightarrow{B}$.
            We then say that $A$ and $B$ have the same cardinality.
            The notation is written as $|A|=|B|$ or
            $\Card(A)=\Card(B)$. A finite set is a set $A$ such that
            there is a bijection between $A$ and $\mathbb{Z}_{n}$.
            We can then view the elements of $A$ as
            $A=\{a_{1},\hdots,a_{n}\}$. A countable set is a set
            $A$ such that there is a bijection between $A$ and
            $\mathbb{N}$. Here, $\mathbb{N}$ is the set of all
            natural numbers, or positive integers.
            \begin{lexample}
                There are many commonly discussed sets that are
                countably infinite. $\mathbb{N}$ is a trivial
                such example, but also $\mathbb{N}_{e}$ and
                $\mathbb{N}_{o}$, the sets of even and odd positive
                integers, respectively. For consider as bijections
                the following functions:
                \par
                \begin{subequations}
                    \begin{minipage}[b]{0.49\textwidth}
                        \centering
                        \begin{equation}
                            f_{e}(n)=2n
                        \end{equation}
                    \end{minipage}
                    \hfill
                    \begin{minipage}[b]{0.49\textwidth}
                        \centering
                        \begin{equation}
                            f_{0}(n)=2n-1
                        \end{equation}
                    \end{minipage}
                    \par\hfill\par
                    $\mathbb{Z}$ is also countable, as shown in
                    Fig.~\ref{fig:MEASURE_THEORY:BIJECTION_N_AND_Z}.
                    An explicit bijection for $\mathbb{Z}$ is:
                    \begin{equation}
                        f(n)=
                        \begin{cases}
                            \frac{n}{2},&n\mod{2}=0\\
                            \frac{1-n}{2},&n\mod{2}=1
                        \end{cases}
                    \end{equation}
                \end{subequations}
            \end{lexample}
            $\mathbb{Q}$ is also countable. We may intuitively think of
            $\mathbb{N}$ as being smaller than $\mathbb{Q}$, since there
            are simple \textit{surjections} from that can be constructed
            from $\mathbb{Q}$ to $\mathbb{N}$. There is also a
            surjection from $\mathbb{N}$ onto $\mathbb{Q}^{+}$, as is
            shown in
            Fig.~\ref{fig:MEASURE_THEORY:BIJECTION_N_AND_Q_Plus}.
            \newpage
            \begin{figure}[H]
                \centering
                \captionsetup{type=figure}
                \subimport{../../../tikz/}{Surjection_From_N_to_Z}
                \caption{Diagram of a Bijection Between
                         $\mathbb{N}$ and $\mathbb{Z}$.}
                \label{fig:MEASURE_THEORY:BIJECTION_N_AND_Z}
            \end{figure}
            To construct such a surjection, write out all of the
            positive rational numbers in a grid so that $a_{nm}$ is
            the number $n/m$. Then zig-zag along the diagonals
            to construct the function. Thus there is a surjection
            $f:\mathbb{Q}\rightarrow\mathbb{N}$
            and a surjection $g:\mathbb{N}\rightarrow\mathbb{Q}$. The
            Cantor-Schr\"{o}eder-Bernstein theorem says that if there
            is surjection from $A$ to $B$ and a surjection from $B$
            to $A$, then there is a bijection between $A$ and $B$.
            Therefore there is a bijection between $\mathbb{N}$ and
            $\mathbb{Q}^{+}$, and $\mathbb{Q}^{+}$ is countable.
            \begin{figure}[H]
                \centering
                \captionsetup{type=figure}
                \resizebox{0.7\textwidth}{!}{%
                    \subimport{../../../tikz/}
                              {Surjection_From_N_to_Q_Plus.tex}
                }
                \caption{Diagram of a Surjection from
                         $\mathbb{N}$ onto $\mathbb{Q}^{+}$.}
                \label{fig:MEASURE_THEORY:BIJECTION_N_AND_Q_Plus}
            \end{figure}
            We can modify
            Fig.~\ref{fig:MEASURE_THEORY:BIJECTION_N_AND_Q_Plus}
            slightly to create a surjection between $\mathbb{N}$ and
            $\mathbb{Q}$, see
            Fig.~\ref{fig:MEASURE_THEORY:BIJECTION_N_AND_Q}.
            It is important to note that this bijection will not
            preserve the order of the rational numbers. The bijection
            will have to jump around back and forth. Any such
            bijection will be forced to do this, as the rationals are
            everywhere dense on $\mathbb{R}$. Any monotonic sequence of
            $\mathbb{Q}$ cannot possibly be a bijection.
            \begin{figure}[H]
                \centering
                \captionsetup{type=figure}
                \resizebox{\textwidth}{!}{%
                    \subimport{../../../tikz/}
                              {Surjection_From_N_to_Q.tex}
                }
                \caption{Diagram of a Surjection from
                         $\mathbb{N}$ onto $\mathbb{Q}$.}
                \label{fig:MEASURE_THEORY:BIJECTION_N_AND_Q}
            \end{figure}
            \begin{theorem}
                If $A$ is a countable infinite set and
                $B\subseteq{A}$, then either $B$ is finite or
                $B$ is countable.
            \end{theorem}
            \begin{bproof}
                As $A$ is countable, there is a bijection
                $a:\mathbb{N}\rightarrow{A}$. Define the following:
                \begin{equation}
                    K=\{n\in\mathbb{N}:a_{n}\in{B}\}
                \end{equation}
                As $B\subseteq{A}$,
                this set contains a subsequence of points in
                $\mathbb{N}$ that get mapped onto $B$. If $K$ is finite,
                then $B$ is finite, and if not then $K$ is countably
                infinite, and thus $B$ is countably infinite.
            \end{bproof}
            \begin{theorem}
                If $A$ is an infinite set, then there exists a
                countable subset $B\subseteq{A}$.
            \end{theorem}
            \begin{bproof}
                If $A$ is infinite then there is an
                $a_{1}\in{A}$. But, as $A$ is infinite,
                $A\setminus\{a_{1}\}$ is infinite, and there
                is an $a_{2}\in{A}\setminus\{a_{1}\}$. Continuing
                we obtain a sequence of distinct elements of $A$.
                Let $B=\{a_{n}:n\in\mathbb{N}\}$. Then
                $B\subseteq{A}$, and $B$ is countable.
            \end{bproof}
            \begin{lexample}
                Suppose we have a collection of disjoint intervals
                of $\mathbb{R}$. This collection is either finite
                or countable. For in every interval, choose a
                rational number $q_{n}$. Let
                $A=\{q_{1},q_{2},\hdots\}$. Then
                $A\subseteq\mathbb{Q}$, and thus $A$ is either
                finite or countable. But this is also an enumeration
                of the intervals in the collection, and thus the
                collection is either finite or countable.
            \end{lexample}
            Given a countable collection of sets
            $A=\{\mathcal{A}_{1},\mathcal{A}_{2},\hdots\}$ such
            that, for all $n\in\mathbb{N}$, $\mathcal{A}_{n}$ is
            also a countable set, then the union is countable. That is:
            \begin{equation}
                B=\bigcup_{n=1}^{\infty}\mathcal{A}_{n}
            \end{equation}
            is a countable set. The proof of this is a mimicry of
            the proof of the countability of $\mathbb{Q}$. Not
            every set is either finite or countable. The first
            example that is often discussed is the set of all
            real numbers, $\mathbb{R}$.
            \begin{definition}
                The power set of a set $\Omega$ is the set:
                \begin{equation}
                    \mathcal{P}(\Omega)=
                    \{A:A\subseteq\Omega\}
                \end{equation}
            \end{definition}
            The power set of any set is strictly larger than the
            original set. If $\Omega$ is finite with $n$ elements, it
            can be shown that $\mathcal{P}(\Omega)$ has $2^{n}$
            elements. There is a trivial surjection from
            $\mathcal{P}(\Omega)$ onto $\Omega$, for any elements
            $x$, let $f(\{x\})=x$.
            \begin{theorem}
                If $\Omega$ is a set, then there is no bijection
                $f:\Omega\rightarrow\mathcal{P}(\Omega)$
            \end{theorem}
            \begin{bproof}
                For suppose not, and let
                $f:\Omega\rightarrow\mathcal{P}(\Omega)$ be such a
                bijection. Define:
                \begin{equation}
                    A=\{x\in\Omega:x\in{f}(x)\}
                \end{equation}
                Then $A\subseteq\Omega$, and thus
                $A\in\mathcal{P}(\Omega)$. But then the complement of
                $A$ is also an element of $\mathcal{P}(\Omega)$. But
                $f$ is a bijection and thus there is an $x\in\Omega$
                such that $f(x)=A^{C}$. If $x\in{f}(x)$, then
                $x\in{A}$, a contradiction as $f(x)=A^{C}$, and thus
                $x\in{A}^{C}$ as well. Therefore $x\notin{f}(x)$. But
                then $x\in{A}^{C}$. But, from the definition of $A$,
                since $x\in{A}^{C}$ and $f(x)=A^{C}$, $x\in{f}(x)$
                and thus $x\in{A}$, a contradiction. Thus there is no
                $x$ such that $f(x)=A^{C}$. Therefore, $f$ is not a
                bijection.
            \end{bproof}
            \begin{theorem}
                $\mathbb{R}$ is uncountably infinite.
            \end{theorem}
            The proof of this theorem first came from Cantor.
            Given a countable subset of the unit interval, let
            $r_{nm}$ be the $m^{th}$ digit of the $n^{th}$ element.
            Construct the real number $y$ as
            $0.y_{1}y_{2}\hdots$ where $y_{n}\ne{r}_{nn}$, for
            all $n\in\mathbb{N}$. Then $y$ is not in the countable
            set, for it differs in the $n^{th}$ place of the
            $n^{th}$ element. This shows that $\mathbb{R}$ is
            not countable, for the stereographic projection can be
            used to form a bjection between $(0,1)$ and $\mathbb{R}$.
            If a set $A$ has the same cardinality as $\mathbb{R}$,
            we say that $A$ has the cardinality of the
            continuum. This is a bijection between the open unit
            square $(0,1)\times(0,1)$ and the open unit interval
            $(0,1)$. For an element $(x,y)\in(0,1)\times(0,1)$,
            let $z\in(0,1)$ be defined as
            $z=0.x_{1}y_{1}x_{2}y_{2}x_{3}y_{3}\hdots$. This is
            a bijection, for all $(x,y)$ is the square there is
            a corresponding $z\in(0,1)$, and for all
            $z\in(0,1)$ there is a corresponding element of
            $(0,1)\times(0,1)$. We can say that $(x,y)$ can
            be coded into $z$, and the $z$ can be decoded into
            $(x,y)$. Hence, $(0,1)\times(0,1)$ has the cardinality
            of the continuum. But, via the stereographic projection,
            $\mathbb{R}^{2}$ has the same cardinality as
            $(0,1)\times(0,1)$, and thus
            $|\mathbb{R}^{2}|=|\mathbb{R}|$.
            \begin{lexample}
                Consider all sequence of real numbers
                $a:\mathbb{N}\rightarrow\mathbb{R}$. This can be
                represented as a subset of
                $\mathbb{N}\times\mathbb{R}$, which is itself
                a subset of $\mathbb{R}^{2}$. From this we have
                that the set of all sequences of real numbers has
                the same cardinality as $\mathbb{R}$.
            \end{lexample}
            If $\Omega$ is a set, the \textit{power set} of
            $\Omega$ is the set of all subsets of $\Omega$.
            We denote this set as $\mathcal{P}(\Omega)$.
            \begin{example}
                If $\Omega=\{1,2\}$, then the power set of $\Omega$
                is:
                \begin{equation}
                    \mathcal{P}(\Omega)=
                    \{\emptyset,\{1\},\{2\},\{1,2\}\}
                \end{equation}
                We do indeed have to consider the empty set, since
                for any set $A$, $\emptyset\subset{A}$. If
                $\Omega=\{1,2,3\}$, we have:
                \begin{equation}
                    \mathcal{P}(\Omega)=
                    \{\emptyset,\{1\},\{2\},\{3\},\{1,2\},
                      \{1,3\},\{2,3\},\{1,2,3\}\}
                \end{equation}
                We see that, in the first example, a set with
                2 elements has a power set with 4 elements. In the
                second example we see that a set with 3 elements has
                a power set with 8 elements. This pattern continues
                for finite sets. If $A$ has $n$ elements, then
                $\mathcal{P}(A)$ has $2^{n}$ elements. If
                $A$ is an infinite set, then automatically
                $\mathcal{P}(A)$ is uncountably infinite. Indeed:
                \begin{equation}
                    |\mathcal{P}(\mathbb{N})|=|\mathbb{R}|
                \end{equation}
                We can show this by using the binary representation
                of real numbers. We construct a bijection as
                follows: If $A\subset\mathbb{N}$, then
                let $r=0.n_{1}n_{2}\hdots$ where
                $n_{i}=0$ if $i\in{A}$, and $n_{i}=0$ otherwise.
                Then every element of $(0,1)$ gets mapped too in
                a one-to-one manner, and thus
                $\mathcal{P}(\mathbb{N})$ and $(0,1)$ are of the
                same cardinality. But $(0,1)$ and $\mathbb{R}$
                are of the same cardinality, and thus
                $|\mathcal{P}(\mathbb{N})=|\mathbb{R}|$.
            \end{example}
            For any set $\Omega$, $|\Omega|<|\mathcal{P}(\Omega)|$.
            This inequality is strict. For finite numbers we see
            that $n<2^{n}$ for all $n\in\mathbb{N}$. For infinite
            sets, this holds as well.
        \subsection{Set Operations}
            \begin{enumerate}
                \item Set union: $A\cup{B}$
                \item Intersection: $A\cap{B}$
                \item Set Difference: $A\setminus{B}$
                \item Complement: $A^{C}$
                \item $A\setminus{B}=A\cap{B}^{C}$
                \item Symmetric Difference: $A\ominus{B}$
                \item DeMorgan's Laws:
                      \subitem $(A\cup{B})^{C}=A^{C}\cap{B}^{C}$
                      \subitem $(A\cap{B})^{C}=A^{C}\cup{B}^{C}$
            \end{enumerate}
            DeMorgan's Laws hold for arbitrary collections
            of set. If $I$ is some indexing set:
            \begin{align}
                \Big(\bigcup_{\alpha\in{I}}A_{\alpha}\Big)^{C}
                &=\bigcap_{\alpha\in{I}}A_{\alpha}^{C}\\
                \Big(\bigcap_{\alpha\in{I}}A_{\alpha}\Big)^{C}
                &=\bigcup_{\alpha\in{I}}A_{\alpha}^{C}
            \end{align}
            There are also two distributive laws:
            \begin{align}
                (A\cup{B})\cap{C}&=(A\cap{C})\cup(B\cap{C})\\
                (A\cap{B})\cup{C}&=(A\cup{C})\cap(B\cup{C})
            \end{align}
            The set operations thus define binary operations
            on the power set of a set $\Omega$. It's important
            to note the notation. An element of $\Omega$ may
            be anything, while an element of
            $\mathcal{P}(\Omega)$ is a subset of $\Omega$.
            That is, the \textit{points} of $\mathcal{P}(\Omega)$
            are themselves sets. Thus, union, intersection,
            etc., define binary operations on
            $\mathcal{P}(\Omega)$. Given two subsets of
            $\Omega$, $A$ and $B$, $A\cup{B}$ is another
            subset of $\Omega$, as is $A\cap{B}$, and so on.
            The complement can also be seen as a unary operator
            on $\mathcal{P}(\Omega)$. From the various theorems
            presented, we have the following:
            \begin{enumerate}
                \item Union is commutative and associative.
                \item Intersection is commutative and
                      associative.
                \item Union distributes over intersection.
                \item Intersection distributes over union.
                \item DeMorgan's Laws hold.
                \item Set difference is not commutative,
                      nor is it associative.
            \end{enumerate}
    \section{\texorpdfstring{$\sigma$}{Sigma}-Algebras}
        \subsection{Set Rings}
            Given a set $\Omega$, $\mathcal{P}(\Omega)$ is the
            set of all subsets of $\Omega$. Often this is too
            much, and too difficult to handle. Indeed, even
            $\mathcal{P}(\mathbb{R})$ is quite large and hard
            to get a grasp on. We wish to speak of collections
            of sets that have some structure on them.
            The first thing we will talk about is a set ring.
            \begin{ldefinition}{Set Ring}
                A set ring of a set $\Omega$ is a nonempty subset
                $\mathcal{R}\subseteq\mathcal{P}(\Omega)$ such that:
                \begin{enumerate}
                    \item For all $A,B\in\mathcal{R}$,
                          $A\cup{B}\in\mathcal{R}$
                    \item For all $A,B\in\mathcal{R}$,
                          $A\setminus{B}\in\mathcal{R}$
                \end{enumerate}
            \end{ldefinition}
            \begin{example}{Set Ring}
                If $\Omega$ is a set, then
                $\mathcal{P}(\Omega)$ is a set ring of
                $\Omega$. So is the set $R=\{\emptyset$.
                For any $A\subset\Omega$, the set
                $R=\{A\}$ is also a set ring. If
                $\Omega=\{1,2,3\}$, then
                $R=\{\emptyset,\{1\},\{2,3\},\{1,2,3\}\}$ is
                a set ring on $\Omega$.
            \end{example}
            \begin{theorem}
                If $\Omega$ is a set, if $R$ is a set ring
                on $\Omega$, and if $A$ is a finite subset of
                $R$, then $\cup_{\alpha\in{A}}\alpha$ is an
                element of $R$.
            \end{theorem}
            \begin{proof}
                Apply induction to the closure of unions.
            \end{proof}
            \begin{theorem}
                If $\Omega$ is a set, if $R$ is a set ring on
                $\Omega$, and if $A,B\in{R}$, then
                $A\cup{B}\in{R}$.
            \end{theorem}
            \begin{proof}
                For $A\cap{B}=A\setminus(A\setminus{B})$, and
                from the closure of set difference,
                $A\cap{B}\in{R}$.
            \end{proof}
            \begin{theorem}
                If $\Omega$ is a set, if $R$ is a set ring
                on $\Omega$, and if $A$ is a finite subset of
                $R$, then $\cap_{\alpha\in{A}}\alpha$ is an
                element of $R$.
            \end{theorem}
            \begin{proof}
                Apply induction to the closure of intersections.
            \end{proof}
            \begin{theorem}
                If $\Omega$ is a set, if $R$ is a set ring on
                $\Omega$, if $A,B\subset\Omega$, and if
                $A\setminus{B}$, $B\setminus{A}$, and
                $A\cap{B}$ are elements of $R$, then
                $A,B\in{R}$.
            \end{theorem}
            Thus, the set ring generated by the set $\{A,B\}$ and
            the set ring generated by
            $\{A\setminus{B},B\setminus{A},A\cap{B}\}$ are the
            same.
            \begin{theorem}
                If $\Omega$ is a set and $R$ is a set ring
                of $\Omega$, then $\emptyset\in{R}$.
            \end{theorem}
            \begin{proof}
                For as $R$ is non-empty, there is an element
                $A\in{R}$. If $A=\emptyset$, then we are done.
                If not, as $R$ is closed under set difference,
                $A\setminus{A}\in{R}$. But
                $A\setminus{A}=\emptyset$.
            \end{proof}
            From this, if we have a collection $R$ of subsets of
            $\Omega$ and we wish to check if $R$ is a set ring
            of $\Omega$, there are several redundant operations
            we don't need to check. Since, for any set $A$,
            we have:
            \begin{align}
                A\setminus\emptyset&=A\\
                A\setminus{A}&=\emptyset\\
                \emptyset\setminus{A}&=\emptyset\\
                A\cup{A}&=A\\
                A\cup\emptyset&=A\\
                \emptyset\cup\emptyset&=\emptyset
            \end{align}
            Using our previous example $\Omega=\{1,2,3\}$,
            we can check laboriously that
            $R=\{\emptyset,\{1\},\{2,3\},\{1,2,3\}\}$ is a
            set ring on $\Omega$. The set
            $\{\emptyset,\{1\},\{2\},\{1,2,3\}\}$ is not
            a set ring, for $\{1,2\}=\{1\}\cup\{2\}$ is not
            an element.
            \begin{theorem}
                If $\Omega$ is a set, and if $A$ and $B$ are
                disjoint subsets of $\Omega$, then
                $R=\{\emptyset,A,B,A\cup{B}\}$ is a set ring
                on $\Omega$.
            \end{theorem}
            \begin{theorem}
                If $\Omega$ is a set, if $A$ and $B$ are
                disjoint subsets of $\Omega$, and if
                $R$ is a set ring such that $A,B\in{R}$,
                then $\{emptyset,A,B,A\cup{B}\}\subset{R}$.
            \end{theorem}
            As such, the set ring $\{\emptyset,A,B,A\cup{B}\}$
            is called the set ring generated by $A$ and $B$. We
            can continue and consider the case of three mutually
            disjoint subsets.
            \begin{theorem}
                If $\Omega$ is a set, and $A_{1},A_{2},A_{3}$ are
                mutually disjoint subsets of $\Omega$, then:
                \begin{equation}
                    R=\{\emptyset,A_{1},A_{2},A_{3},
                        A_{1}\cup{A}_{2},A_{1}\cup{A}_{3},
                        A_{2}\cup{A}_{3},
                        A_{1}\cup{A}_{2}\cup{A}_{3}\}
                \end{equation}
                is a set ring on $\Omega$.
            \end{theorem}
            Indeed, we may generalize further.
            \begin{theorem}
                If $\Omega$ is a set and if
                $A$ is a subset of $\mathcal{P}(\Omega)$ of
                $n$ elements such that, for all
                $a,b\in{A}$, $a\cap{B}=\emptyset$, then:
                \begin{equation}
                    R=\{\cup_{i\in{I}}A_{i}:
                    I\in\mathcal{P}(\mathbb{Z}_{n})\}
                \end{equation}
                Is a set ring on $\Omega$.
            \end{theorem}
            \begin{theorem}
                If $\Omega$ is a set, then the set of all
                finite subsets of $\Omega$ is a set ring on
                $\Omega$.
            \end{theorem}
            A left semi-interval of $\mathbb{R}$ is an interval
            of the form $[a,b)$ where $a\leq{b}$. If $a=b$, this
            is the empty set. The set of all left semi-intervals
            is not a set ring on $\mathbb{R}$ since the union
            of two semi-intervals need not be a semi-interval.
            We need to add more sets to allow this to be a
            set ring. The collection of all finite unions of
            semi-intervals of $\mathbb{R}$ is a set ring.
            First, note the following:
            \begin{equation}
                \Big(\bigcup_{n=1}^{N}[a_{n},b_{n})\Big)
                \setminus[c,d)=\bigcup_{n=1}^{N}
                \Big([a_{n},b_{n})\setminus[c,d)]
            \end{equation}
            This is again the finite union of intervals. By
            induction we see that this collection is a ring on
            $\mathbb{R}$. We have seen that a set ring is
            closed to unions and set differences, and this
            implies that rings are closed under intersections and
            closed under symmetric differences. As it turns out,
            this is an equivalent definition of a set ring.
            \begin{theorem}
                If $\Omega$ is a set and
                $R\subset\mathcal{P}(\Omega)$, then $R$ is
                a set ring of $\Omega$ if and only if $R$ is
                closed under symmetric differences and
                intersections.
            \end{theorem}
            If $R$ is a set ring on $\Omega$, and if
            $A\in{R}$, let $\chi_{A}:\Omega\rightarrow[0,1]$ be
            the indicator function defined as follows:
            \begin{equation}
                \chi_{A}(\omega)=
                \begin{cases}
                    0,&\omega\notin{A}\\
                    1,&\omega\in{A}
                \end{cases}
            \end{equation}
            Then we have:
            \begin{align}
                \chi_{A\cap{B}}(\omega)
                &=\chi_{A}(\omega)\chi_{B}(\omega)\\
                \chi_{A\ominus{B}}&=
                \big(\chi_{A}(\omega)+\chi_{B}(\omega)\big)
                \mod{2}
            \end{align}
            These two operations satisfy the axioms of a ring.
            That is, a ring in the algebraic sense of the word:
            A set with two operations that behave certain
            properties. It is because of this that set rings
            have earned their name.
        \subsection{Set Algebras}
            \begin{definition}
                A set algebra on a set $\Omega$ is a subset
                $\mathcal{A}$ of $\mathcal{P}(\Omega)$ such that
                $\mathcal{A}$ is a set ring on $\Omega$ and
                $\Omega\in\mathcal{A}$.
            \end{definition}
            \begin{example}
                Let $\Omega=\{1,2,3,4\}$ and
                $R=\{\emptyset,\{1\},\{2,3\}\}$. Then $R$
                is a set ring, but it is not a set algebra
                since $\Omega\notin{R}$.
            \end{example}
            From the definition, we see that a set algebra
            is closed under complements. indeed, this creates
            and equivalent definition for set algebras.
            \begin{theorem}
                If $\Omega$ is a set and
                $\mathcal{A}\subseteq\mathcal{P}(\Omega)$,
                then $\mathcal{A}$ is a set algebra on $\Omega$
                if and only if $\Omega\in\mathcal{A}$, and
                $\mathcal{A}$ is closed under union and
                complement.
            \end{theorem}
            \begin{theorem}
                If $\Omega$ is a set and $R$ is a set ring
                on $\Omega$, and if $\mathcal{A}$ is a set
                algebra on $\Omega$ such that
                $R\subset\mathcal{A}$, then for all $A\in{R}$,
                $A\in\mathcal{A}$ and $A^{C}\in\mathcal{A}$.
            \end{theorem}
            This then defines the \textit{smallest} set algebra
            that contains a set ring.
            \begin{theorem}
                If $\Omega$ is a set and $R$ is a set ring on
                $\Omega$, then:
                \begin{equation}
                    \mathcal{A}=\{A,A^{C}:A\in{R}\}
                \end{equation}
                Is a set algebra on $\Omega$.
            \end{theorem}
            \begin{theorem}
                If $\Omega$ is a set and $A$ and $B$ are
                disjoint subset of $A$, then:
                \begin{equation}
                    \mathcal{A}=
                        \{\emptyset,A,B,A\cup{B},
                          \Omega,A^{C},B^{C},A^{C}\cap{B}^{C}\}
                \end{equation}
                is a set algebra on $\Omega$.
            \end{theorem}
            For non-disjoint $A$ and $B$, the smallest
            set algebra becomes more complicated. We saw that
            the collection of all finite subsets of a set is
            a set ring on the set. The collection of all finite
            subsets, and their complements, is a set algebra.
        \subsection{\texorpdfstring{$\sigma$}{Sigma}-Rings}
            If $\Omega$ is a set, then $R\subset\mathcal{P}(\Omega)$
            is called a set ring on $\Omega$ if, for all
            $A,B\in{R}$, $A\cup{B}\in{R}$ and
            $A\setminus{B}\in{R}$. From this, given a ring $R$ on
            $\Omega$, the empty set is included, that is
            $\emptyset\in{R}$, and if $A,B\in{R}$, then
            $A\cap{B}\in{R}$. By induction, for any finite collection
            of elements in $R$, the union of these subsets is also
            contained in $R$, as well as the intersection. A set
            algebra on $\Omega$ is a ring $\mathcal{A}$
            on $\Omega$ such that $\Omega\in\mathcal{A}$. That is,
            $\mathcal{A}\subset\mathcal{P}(\Omega)$, and
            $\mathcal{A}$ is closed under union, set difference, and
            $\Omega\in\mathcal{A}$. There is an equivalent definition:
            $\Omega\in\mathcal{A}$, for all $A\in\mathcal{A}$,
            $A^{C}\in\mathcal{A}$, and for all $A,B\in\mathcal{A}$,
            $A\cup{B}\in\mathcal{A}$. The complement of $A$,
            $A^{C}$, is defined as $\Omega\setminus{A}$. The
            equivalence of the two definitions comes from DeMorgan's
            laws, since
            $A\setminus{B}=A\cap{B}^{C}=(A^{C}\cup{B})^{C}$. We now
            talk about $\sigma$-Ring.
            \begin{definition}
                A $\sigma$-Ring on a set $\Omega$ is a set
                $\sigma\subset\mathcal{P}(\Omega)$ such that,
                for all countable subsets of $\sigma$, the union
                $\bigcup_{i=1}^{\infty}A_{i}\in\sigma$, and for all
                $A,B\in\sigma$, $A\setminus{B}\in\sigma$.
            \end{definition}
            The requirement that the collections be countable is
            important to note. A \textit{topology} is a subset
            of $\mathcal{P}(\Omega)$ with the property that it is
            closed under arbitrary unions. $\sigma$-Rings need only
            be closed under countable unions.
            \begin{example}
                Every $\sigma$-Ring is a set ring, but not every
                set ring is a $\sigma$-ring. Let $\Omega$ be
                uncountable, and let $R$ be the set of all finite
                subsets of $\Omega$. Then $R$ is a ring, but it is
                not a $\sigma$-ring. For, as $\Omega$ is uncountably
                infinite, it has a countable subset $A$, and we
                may subscript the elements as $a_{n}$. But
                $\bigcup_{n=1}^{\infty}\{a_{n}\}$ is not a finite
                subset of $\Omega$, and is therefore not contained
                in $R$. Thus, $R$ is not closed under countable unions
                and $R$ is not a $\sigma$-ring. However, if we let
                $\sigma$ be the set of all \textit{countable} subsets
                of $\Omega$, the $\sigma$ is indeed a $\sigma$-ring.
            \end{example}
            \begin{example}
                The collection of all semi-intervals and finite
                unions of semi-intervals defines a ring on
                $\mathbb{R}$. It is tempting to think tha the
                collection of all countable unions of semi-intervals
                is a $\sigma$-ring on $\mathbb{R}$, but this is not
                the case. The Cantor set is an example of a subset
                that can be constructed by a countable number of
                steps of removing intervals from a given interval,
                but the resulting set is not the countable union of
                semi-intervals. To construct the Cantor set, consider
                the interval $[0,1]$. From this, remove
                $(\frac{1}{3},\frac{2}{3})$. Continuing removing the
                middle third from each sub-interval obtained. The
                resulting set contains no interval as a subset, and
                thus cannot be the union of countably many intervals,
                or semi-intervals.
            \end{example}
        \subsection{Dynkin System}
            A Dynkin system on a set $\Omega$ is a subset
            $\mathcal{D}\subset\mathcal{P}(\Omega)$ such that
            $\Omega\in\mathcal{D}$, if $A,B\in\Omega$ and if
            $A\subseteq{B}$, then $A\setminus{B}\in\mathcal{D}$,
            and for all countable collections of elements of
            $\mathcal{D}$ such that
            $A_{1}\subset{A}_{2}\subset\hdots$,
            $\cup_{n=1}^{\infty}A_{n}\in\mathcal{D}$. There is
            an equivalent defintion for Dynkin Systems.
            $\Omega\in\mathcal{D}$, $A\in\mathcal{D}$ implies
            $A^{C}\in\mathcal{D}$, and for all countable disjoint
            collections of elements in $\mathcal{D}$, the union
            is also contained in $\mathcal{D}$. These requirements
            are weaker than those of a $\sigma$-algebra. Any
            $\sigma$-algebra is a Dynkin system, but not every
            Dynkin system is a $\sigma$-algebra.
            \begin{theorem}
                If $\mathcal{D}$ is a Dynkin system on a set
                $\Omega$, and if $\mathcal{D}$ is closed with
                respect to intersections, then $\mathcal{D}$
                is a $\sigma$-algebra on $\Omega$.
            \end{theorem}
            \begin{theorem}
                If $\Omega$ is a set, if
                $\mathcal{E}\subset\mathcal{P}(\Omega)$ is closed
                to intersections, and if $\mathcal{D}$ is the
                Dynkin System generated by $\mathcal{E}$, then
                $\mathcal{D}$ is a $\sigma$-algebra.
            \end{theorem}
            Since semi-intervals are closed to intersections,
            the Borel $\sigma$-algebra is equivalently the
            Dynkin system generated by semi-intervals.
        \subsection{\texorpdfstring{$\sigma$}{Sigma}-Algebras}
            In an analogous manner to how set rings and set algebras
            were defined, there is something called a $\sigma$-algebra.
            This notion will be one of the central themes of measure
            theory.
            \begin{definition}
                A $\sigma$-algebra on a set $\Omega$ is a
                $\sigma$-ring on $\Omega$ such that
                $\Omega\in\mathcal{A}$
            \end{definition}
            That is, given any countable collection of elements in
            $\mathcal{A}$, the union is also contained in
            $\mathcal{A}$. In addition, $\mathcal{A}$ is closed under
            set differences and $\Omega\in\mathcal{A}$.
            \begin{example}
                The first trivial example is the power set
                $\mathcal{P}(\Omega)$. Also the set
                $\{\emptyset,\Omega\}$ defines a $\sigma$-algebra on
                $\Omega$. The set of all countable subsets defines
                a $\sigma$-ring, and the set of all countable and
                co-countable (Sets whose complement is countable)
                will define a $\sigma$-algebra.
            \end{example}
            We can equivalently define a $\sigma$-algebra to be a
            collection of sets $\mathcal{A}$ such that
            $\Omega\in\mathcal{A}$, for all $A\in\mathcal{A}$,
            $A^{C}\in\mathcal{A}$, and $\mathcal{A}$ is closed under
            countable unions. Being closed under countable unions
            implies that it is closed under finite unions as well.
            For let $A_{1}=A$, and for all $n>1$, let $A_{n}=B$.
            Then $\bigcup_{n=1}^{\infty}A_{n}=A\cup{B}$. By induction,
            a $\sigma$-algebra is closed under any finite union.
        \subsection{Borel \texorpdfstring{$\sigma$}{Sigma}-Algebra}
            One of the most important types of $\sigma$-algebras
            is the Borel $\sigma$-algebra. We first define the
            Borel $\sigma$-algebra on $\mathbb{R}$.
            \begin{definition}
                The Borel $\sigma$-algebra on $\mathbb{R}$, denoted
                $\mathcal{B}$, is the $\sigma$-algebra generated
                by the set $\{[a,b):a,b\in\mathbb{R}\}$.
            \end{definition}
            That is, the Borel $\sigma$-algebra on $\mathbb{R}$ is
            the \textit{smallest} $\sigma$-algebra that contains
            all of the semi-intervals. We can equivalently say that
            $\mathcal{B}$ is the $\sigma$-algebra generated by all
            \textit{open} intervals. If we know that every open
            subset of $\mathbb{R}$ is the countable union of open
            subsets, than we can equivalently say that
            $\mathcal{B}$ is the $\sigma$-algebra generated by all
            \textit{open subsets} of $\mathbb{R}$. Writing $[a,b)$
            as the countable intersection of open intervals, or
            $(a,b)$ as the countable union of semi-intervals comes
            from the Archimedean property of the real numbers.
            Thus, the smallest $\sigma$-algebra that contains all
            semi-intervals is the smallest $\sigma$-algebra that
            contains all open intervals, which
            is the smallest $\sigma$-algebra that contains all open
            subsets of $\mathbb{R}$. Similarly, this will contain all
            of the \textit{closed} intervals, intervals of the form
            $[a,b]$. We say that a set $\mathcal{U}\subset\mathbb{R}$
            is open if, for all $x\in\mathcal{U}$, there is an $r>0$
            such that $(x-r,x+r)\subset\mathcal{U}$. That is, every
            point in $\mathcal{U}$ can be surrounded by an interval
            that is entirely contained in $\mathcal{U}$. Thus, any
            open set can be written as:
            \begin{equation}
                \mathcal{U}=
                    \bigcup_{x\in\mathcal{U}}(\alpha_{x},\beta_{x})
            \end{equation}
            This union is not countable, for any open set must
            contain an interval, an intervals are uncountable large.
            This is simply because $(a,b)$ is equivalent to $(0,1)$.
            By adjusting the size of $\alpha_{x}$ and $\beta_{x}$ to
            be rational numbers, we can written $\mathcal{U}$ as the
            union of intervals with rational endpoints. But there are
            only countably many such intervals, and thus
            $\mathcal{U}$ is the union of countably many open
            intervals. Thus, any open set is the union of countably
            many open intervals. From this, the smallest
            $\sigma$-algebra that contains open intervals will contain
            all open sets, since $\sigma$-algebras are closed under
            countable unions. Borel sets are elements of the
            Borel $\sigma$-algebra $\mathcal{B}$. Since all open
            sets are Borel sets, and as $\sigma$-algebras are closed
            under complenents, all closed sets are also Borel sets.
            This is because the complement of an open set is a closed
            set, and vice versa. Thus, equivalently, $\mathcal{B}$ is
            the smallest $\sigma$-algebra containing all closed sets.
            A $G_{\delta}$ sets is a subset that is the countable
            intersection of open sets. As open sets are not
            necessarily closed under countable intersections, not
            all $G_{\delta}$ sets are open. There is another notion,
    \section{Measures}
        \subsection{A Review Infinite Series}
            Given a sequence of real numbers,
            $a:\mathbb{N}\rightarrow\mathbb{R}$, the sum of this
            sequence is defined as the limit of
            finite partial sums. That is:
            \begin{equation}
                \sum_{n=1}^{\infty}a_{n}=
                \underset{N\rightarrow\infty}\sum_{n=1}^{N}a_{n}
            \end{equation}
            In general, this limit may not in general exists. If it
            does, we say the series converges. If the limit does
            not exists, we do not define the sum and instead just
            have a meaningless combination of symbols. If the
            sequence is positive, then the sequence of partial sums
            will be increasing. If this sequence is bounded, then
            the limit exists. This comes from the fact that bounded
            monotonic sequences converge, a result that stems from
            the least upper bound property of $\mathbb{R}$.
            Moreover, if $a:\mathbb{N}\rightarrow\mathbb{R}$ is a
            sequence of positive real numbers, and if
            $f:\mathbb{N}\rightarrow\mathbb{N}$ is any bijective
            function, then the following is true:
            \begin{equation}
                \sum_{n=1}^{\infty}a_{n}
                =\sum_{n=1}^{\infty}a_{f(n)}
            \end{equation}
            We can also split the sequence into a grid,
            and take the
            double sum, obtaining the same result. If
            $A_{1},A_{2},\hdots$ are disjoint sets whose union is
            $\mathbb{N}$, and if $b_{nm}$ is the $n^{th}$ element
            of $A_{m}$, then:
            \begin{equation}
                \sum_{i=1}^{\infty}a_{i}=
                \sum_{n=1}^{\infty}\sum_{m=1}^{\infty}b_{nm}
            \end{equation}
            We should be precise in what we mean. The double
            sum is the \textit{limit of a limit}.
            \begin{equation}
                \sum_{n=1}^{\infty}\sum_{m=1}^{\infty}a_{nm}
                =\underset{N\rightarrow\infty}{\lim}\sum_{n=1}^{N}
                \Big(\underset{M\rightarrow\infty}{\lim}
                \sum_{m=1}^{M}a_{nm}\Big)
            \end{equation}
            We use infinite series to define \textit{measures} on
            $\sigma$-algebra.
        \subsection{Measure Functions}
            A set function on a collection of sets $\mathcal{E}$
            is a function $\mu:\mathcal{E}\rightarrow\mathbb{R}$.
            For example, if we consider the set of all
            semi-intervals of the form $[a,b)$, where
            $a,b\in\mathbb{R}$ and $a\leq{b}$, then we can define
            $\mu([a,b))=b-a$. This gives rise to the notion of
            a measure function.
            A measure function on a collection of set
            $\mathcal{E}$ is a function
            $\mu:\mathcal{E}\rightarrow\mathbb{R}$ such that:
            \begin{enumerate}
                \item If $\emptyset\in\mathcal{E}$, then
                      $\mu(\emptyset)=0$
                \item For all $A\in\mathcal{E}$, $\mu(A)\geq{0}$
                \item For any countable collection of pair-wise
                      disjoint sets whose
                      union also lies in $\mathcal{E}$,
                      $\mu(\cup_{n=1}^{\infty}A_{n})=%
                       \sum_{n=1}^{\infty}\mu(A_{n})$
            \end{enumerate}
            It helps if we don't have to consider the case where
            $\mu(\emptyset)$ is undefined, or where we don't have
            closure under countable unions, so we discuss measure
            functions on $\sigma$-algebras.
            \begin{definition}
                A measure on a $\sigma$-algebra
                $\mathcal{A}$ is a function
                $\mu:\mathcal{A}\rightarrow\mathbb{R}$ such that:
                \begin{enumerate}
                    \item $\mu(\emptyset)=0$
                    \item For all $A\in\mathcal{A}$,
                          $\mu(A)\geq{0}$
                    \item For any countable collection of pairwise
                          disjoint elements of $\mathcal{A}$,
                          $\mu(\cup_{n=1}^{\infty}A_{n})=%
                           \sum_{n=1}^{\infty}\mu(A_{n})$
                \end{enumerate}
            \end{definition}
            \begin{example}
                Let $\Omega$ be a set, and let
                $\mathcal{A}=\mathcal{P}(\Omega)$. Then
                $\mathcal{A}$ is a $\sigma$-algebra on $\Omega$.
                If $\omega_{1},\hdots,\omega_{n}\in\Omega$ and if
                $p_{1},\hdots,p_{n}\in\mathbb{R}^{+}$, then:
                \begin{equation}
                    \mu(A)=\sum_{k=1}^{n}p_{k}\chi_{A}(\omega_{k})
                \end{equation}
                Where $\xi_{A}$ is the characteristic function:
                \begin{equation}
                    \chi_{A}(\omega)=
                    \begin{cases}
                        0,&\omega\notin{A}\\
                        1,&\omega\in{A}
                    \end{cases}
                \end{equation}
                This is an example of a \textit{point measure}
                on $\mathcal{A}$. It defines a measure function.
            \end{example}
\end{document}