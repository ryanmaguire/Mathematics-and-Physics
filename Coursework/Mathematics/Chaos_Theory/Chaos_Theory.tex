\documentclass[crop=false,class=book,oneside]{standalone}
%----------------------------Preamble-------------------------------%
%---------------------------Packages----------------------------%
\usepackage{geometry}
\geometry{b5paper, margin=1.0in}
\usepackage[T1]{fontenc}
\usepackage{graphicx, float}            % Graphics/Images.
\usepackage{natbib}                     % For bibliographies.
\bibliographystyle{agsm}                % Bibliography style.
\usepackage[french, english]{babel}     % Language typesetting.
\usepackage[dvipsnames]{xcolor}         % Color names.
\usepackage{listings}                   % Verbatim-Like Tools.
\usepackage{mathtools, esint, mathrsfs} % amsmath and integrals.
\usepackage{amsthm, amsfonts, amssymb}  % Fonts and theorems.
\usepackage{tcolorbox}                  % Frames around theorems.
\usepackage{upgreek}                    % Non-Italic Greek.
\usepackage{fmtcount, etoolbox}         % For the \book{} command.
\usepackage[newparttoc]{titlesec}       % Formatting chapter, etc.
\usepackage{titletoc}                   % Allows \book in toc.
\usepackage[nottoc]{tocbibind}          % Bibliography in toc.
\usepackage[titles]{tocloft}            % ToC formatting.
\usepackage{pgfplots, tikz}             % Drawing/graphing tools.
\usepackage{imakeidx}                   % Used for index.
\usetikzlibrary{
    calc,                   % Calculating right angles and more.
    angles,                 % Drawing angles within triangles.
    arrows.meta,            % Latex and Stealth arrows.
    quotes,                 % Adding labels to angles.
    positioning,            % Relative positioning of nodes.
    decorations.markings,   % Adding arrows in the middle of a line.
    patterns,
    arrows
}                                       % Libraries for tikz.
\pgfplotsset{compat=1.9}                % Version of pgfplots.
\usepackage[font=scriptsize,
            labelformat=simple,
            labelsep=colon]{subcaption} % Subfigure captions.
\usepackage[font={scriptsize},
            hypcap=true,
            labelsep=colon]{caption}    % Figure captions.
\usepackage[pdftex,
            pdfauthor={Ryan Maguire},
            pdftitle={Mathematics and Physics},
            pdfsubject={Mathematics, Physics, Science},
            pdfkeywords={Mathematics, Physics, Computer Science, Biology},
            pdfproducer={LaTeX},
            pdfcreator={pdflatex}]{hyperref}
\hypersetup{
    colorlinks=true,
    linkcolor=blue,
    filecolor=magenta,
    urlcolor=Cerulean,
    citecolor=SkyBlue
}                           % Colors for hyperref.
\usepackage[toc,acronym,nogroupskip,nopostdot]{glossaries}
\usepackage{glossary-mcols}
%------------------------Theorem Styles-------------------------%
\theoremstyle{plain}
\newtheorem{theorem}{Theorem}[section]

% Define theorem style for default spacing and normal font.
\newtheoremstyle{normal}
    {\topsep}               % Amount of space above the theorem.
    {\topsep}               % Amount of space below the theorem.
    {}                      % Font used for body of theorem.
    {}                      % Measure of space to indent.
    {\bfseries}             % Font of the header of the theorem.
    {}                      % Punctuation between head and body.
    {.5em}                  % Space after theorem head.
    {}

% Italic header environment.
\newtheoremstyle{thmit}{\topsep}{\topsep}{}{}{\itshape}{}{0.5em}{}

% Define environments with italic headers.
\theoremstyle{thmit}
\newtheorem*{solution}{Solution}

% Define default environments.
\theoremstyle{normal}
\newtheorem{example}{Example}[section]
\newtheorem{definition}{Definition}[section]
\newtheorem{problem}{Problem}[section]

% Define framed environment.
\tcbuselibrary{most}
\newtcbtheorem[use counter*=theorem]{ftheorem}{Theorem}{%
    before=\par\vspace{2ex},
    boxsep=0.5\topsep,
    after=\par\vspace{2ex},
    colback=green!5,
    colframe=green!35!black,
    fonttitle=\bfseries\upshape%
}{thm}

\newtcbtheorem[auto counter, number within=section]{faxiom}{Axiom}{%
    before=\par\vspace{2ex},
    boxsep=0.5\topsep,
    after=\par\vspace{2ex},
    colback=Apricot!5,
    colframe=Apricot!35!black,
    fonttitle=\bfseries\upshape%
}{ax}

\newtcbtheorem[use counter*=definition]{fdefinition}{Definition}{%
    before=\par\vspace{2ex},
    boxsep=0.5\topsep,
    after=\par\vspace{2ex},
    colback=blue!5!white,
    colframe=blue!75!black,
    fonttitle=\bfseries\upshape%
}{def}

\newtcbtheorem[use counter*=example]{fexample}{Example}{%
    before=\par\vspace{2ex},
    boxsep=0.5\topsep,
    after=\par\vspace{2ex},
    colback=red!5!white,
    colframe=red!75!black,
    fonttitle=\bfseries\upshape%
}{ex}

\newtcbtheorem[auto counter, number within=section]{fnotation}{Notation}{%
    before=\par\vspace{2ex},
    boxsep=0.5\topsep,
    after=\par\vspace{2ex},
    colback=SeaGreen!5!white,
    colframe=SeaGreen!75!black,
    fonttitle=\bfseries\upshape%
}{not}

\newtcbtheorem[use counter*=remark]{fremark}{Remark}{%
    fonttitle=\bfseries\upshape,
    colback=Goldenrod!5!white,
    colframe=Goldenrod!75!black}{ex}

\newenvironment{bproof}{\textit{Proof.}}{\hfill$\square$}
\tcolorboxenvironment{bproof}{%
    blanker,
    breakable,
    left=3mm,
    before skip=5pt,
    after skip=10pt,
    borderline west={0.6mm}{0pt}{green!80!black}
}

\AtEndEnvironment{lexample}{$\hfill\textcolor{red}{\blacksquare}$}
\newtcbtheorem[use counter*=example]{lexample}{Example}{%
    empty,
    title={Example~\theexample},
    boxed title style={%
        empty,
        size=minimal,
        toprule=2pt,
        top=0.5\topsep,
    },
    coltitle=red,
    fonttitle=\bfseries,
    parbox=false,
    boxsep=0pt,
    before=\par\vspace{2ex},
    left=0pt,
    right=0pt,
    top=3ex,
    bottom=1ex,
    before=\par\vspace{2ex},
    after=\par\vspace{2ex},
    breakable,
    pad at break*=0mm,
    vfill before first,
    overlay unbroken={%
        \draw[red, line width=2pt]
            ([yshift=-1.2ex]title.south-|frame.west) to
            ([yshift=-1.2ex]title.south-|frame.east);
        },
    overlay first={%
        \draw[red, line width=2pt]
            ([yshift=-1.2ex]title.south-|frame.west) to
            ([yshift=-1.2ex]title.south-|frame.east);
    },
}{ex}

\AtEndEnvironment{ldefinition}{$\hfill\textcolor{Blue}{\blacksquare}$}
\newtcbtheorem[use counter*=definition]{ldefinition}{Definition}{%
    empty,
    title={Definition~\thedefinition:~{#1}},
    boxed title style={%
        empty,
        size=minimal,
        toprule=2pt,
        top=0.5\topsep,
    },
    coltitle=Blue,
    fonttitle=\bfseries,
    parbox=false,
    boxsep=0pt,
    before=\par\vspace{2ex},
    left=0pt,
    right=0pt,
    top=3ex,
    bottom=0pt,
    before=\par\vspace{2ex},
    after=\par\vspace{1ex},
    breakable,
    pad at break*=0mm,
    vfill before first,
    overlay unbroken={%
        \draw[Blue, line width=2pt]
            ([yshift=-1.2ex]title.south-|frame.west) to
            ([yshift=-1.2ex]title.south-|frame.east);
        },
    overlay first={%
        \draw[Blue, line width=2pt]
            ([yshift=-1.2ex]title.south-|frame.west) to
            ([yshift=-1.2ex]title.south-|frame.east);
    },
}{def}

\AtEndEnvironment{ltheorem}{$\hfill\textcolor{Green}{\blacksquare}$}
\newtcbtheorem[use counter*=theorem]{ltheorem}{Theorem}{%
    empty,
    title={Theorem~\thetheorem:~{#1}},
    boxed title style={%
        empty,
        size=minimal,
        toprule=2pt,
        top=0.5\topsep,
    },
    coltitle=Green,
    fonttitle=\bfseries,
    parbox=false,
    boxsep=0pt,
    before=\par\vspace{2ex},
    left=0pt,
    right=0pt,
    top=3ex,
    bottom=-1.5ex,
    breakable,
    pad at break*=0mm,
    vfill before first,
    overlay unbroken={%
        \draw[Green, line width=2pt]
            ([yshift=-1.2ex]title.south-|frame.west) to
            ([yshift=-1.2ex]title.south-|frame.east);},
    overlay first={%
        \draw[Green, line width=2pt]
            ([yshift=-1.2ex]title.south-|frame.west) to
            ([yshift=-1.2ex]title.south-|frame.east);
    }
}{thm}

%--------------------Declared Math Operators--------------------%
\DeclareMathOperator{\adjoint}{adj}         % Adjoint.
\DeclareMathOperator{\Card}{Card}           % Cardinality.
\DeclareMathOperator{\curl}{curl}           % Curl.
\DeclareMathOperator{\diam}{diam}           % Diameter.
\DeclareMathOperator{\dist}{dist}           % Distance.
\DeclareMathOperator{\Div}{div}             % Divergence.
\DeclareMathOperator{\Erf}{Erf}             % Error Function.
\DeclareMathOperator{\Erfc}{Erfc}           % Complementary Error Function.
\DeclareMathOperator{\Ext}{Ext}             % Exterior.
\DeclareMathOperator{\GCD}{GCD}             % Greatest common denominator.
\DeclareMathOperator{\grad}{grad}           % Gradient
\DeclareMathOperator{\Ima}{Im}              % Image.
\DeclareMathOperator{\Int}{Int}             % Interior.
\DeclareMathOperator{\LC}{LC}               % Leading coefficient.
\DeclareMathOperator{\LCM}{LCM}             % Least common multiple.
\DeclareMathOperator{\LM}{LM}               % Leading monomial.
\DeclareMathOperator{\LT}{LT}               % Leading term.
\DeclareMathOperator{\Mod}{mod}             % Modulus.
\DeclareMathOperator{\Mon}{Mon}             % Monomial.
\DeclareMathOperator{\multideg}{mutlideg}   % Multi-Degree (Graphs).
\DeclareMathOperator{\nul}{nul}             % Null space of operator.
\DeclareMathOperator{\Ord}{Ord}             % Ordinal of ordered set.
\DeclareMathOperator{\Prin}{Prin}           % Principal value.
\DeclareMathOperator{\proj}{proj}           % Projection.
\DeclareMathOperator{\Refl}{Refl}           % Reflection operator.
\DeclareMathOperator{\rk}{rk}               % Rank of operator.
\DeclareMathOperator{\sgn}{sgn}             % Sign of a number.
\DeclareMathOperator{\sinc}{sinc}           % Sinc function.
\DeclareMathOperator{\Span}{Span}           % Span of a set.
\DeclareMathOperator{\Spec}{Spec}           % Spectrum.
\DeclareMathOperator{\supp}{supp}           % Support
\DeclareMathOperator{\Tr}{Tr}               % Trace of matrix.
%--------------------Declared Math Symbols--------------------%
\DeclareMathSymbol{\minus}{\mathbin}{AMSa}{"39} % Unary minus sign.
%------------------------New Commands---------------------------%
\DeclarePairedDelimiter\norm{\lVert}{\rVert}
\DeclarePairedDelimiter\ceil{\lceil}{\rceil}
\DeclarePairedDelimiter\floor{\lfloor}{\rfloor}
\newcommand*\diff{\mathop{}\!\mathrm{d}}
\newcommand*\Diff[1]{\mathop{}\!\mathrm{d^#1}}
\renewcommand*{\glstextformat}[1]{\textcolor{RoyalBlue}{#1}}
\renewcommand{\glsnamefont}[1]{\textbf{#1}}
\renewcommand\labelitemii{$\circ$}
\renewcommand\thesubfigure{%
    \arabic{chapter}.\arabic{figure}.\arabic{subfigure}}
\addto\captionsenglish{\renewcommand{\figurename}{Fig.}}
\numberwithin{equation}{section}

\renewcommand{\vector}[1]{\boldsymbol{\mathrm{#1}}}

\newcommand{\uvector}[1]{\boldsymbol{\hat{\mathrm{#1}}}}
\newcommand{\topspace}[2][]{(#2,\tau_{#1})}
\newcommand{\measurespace}[2][]{(#2,\varSigma_{#1},\mu_{#1})}
\newcommand{\measurablespace}[2][]{(#2,\varSigma_{#1})}
\newcommand{\manifold}[2][]{(#2,\tau_{#1},\mathcal{A}_{#1})}
\newcommand{\tanspace}[2]{T_{#1}{#2}}
\newcommand{\cotanspace}[2]{T_{#1}^{*}{#2}}
\newcommand{\Ckspace}[3][\mathbb{R}]{C^{#2}(#3,#1)}
\newcommand{\funcspace}[2][\mathbb{R}]{\mathcal{F}(#2,#1)}
\newcommand{\smoothvecf}[1]{\mathfrak{X}(#1)}
\newcommand{\smoothonef}[1]{\mathfrak{X}^{*}(#1)}
\newcommand{\bracket}[2]{[#1,#2]}

%------------------------Book Command---------------------------%
\makeatletter
\renewcommand\@pnumwidth{1cm}
\newcounter{book}
\renewcommand\thebook{\@Roman\c@book}
\newcommand\book{%
    \if@openright
        \cleardoublepage
    \else
        \clearpage
    \fi
    \thispagestyle{plain}%
    \if@twocolumn
        \onecolumn
        \@tempswatrue
    \else
        \@tempswafalse
    \fi
    \null\vfil
    \secdef\@book\@sbook
}
\def\@book[#1]#2{%
    \refstepcounter{book}
    \addcontentsline{toc}{book}{\bookname\ \thebook:\hspace{1em}#1}
    \markboth{}{}
    {\centering
     \interlinepenalty\@M
     \normalfont
     \huge\bfseries\bookname\nobreakspace\thebook
     \par
     \vskip 20\p@
     \Huge\bfseries#2\par}%
    \@endbook}
\def\@sbook#1{%
    {\centering
     \interlinepenalty \@M
     \normalfont
     \Huge\bfseries#1\par}%
    \@endbook}
\def\@endbook{
    \vfil\newpage
        \if@twoside
            \if@openright
                \null
                \thispagestyle{empty}%
                \newpage
            \fi
        \fi
        \if@tempswa
            \twocolumn
        \fi
}
\newcommand*\l@book[2]{%
    \ifnum\c@tocdepth >-3\relax
        \addpenalty{-\@highpenalty}%
        \addvspace{2.25em\@plus\p@}%
        \setlength\@tempdima{3em}%
        \begingroup
            \parindent\z@\rightskip\@pnumwidth
            \parfillskip -\@pnumwidth
            {
                \leavevmode
                \Large\bfseries#1\hfill\hb@xt@\@pnumwidth{\hss#2}
            }
            \par
            \nobreak
            \global\@nobreaktrue
            \everypar{\global\@nobreakfalse\everypar{}}%
        \endgroup
    \fi}
\newcommand\bookname{Book}
\renewcommand{\thebook}{\texorpdfstring{\Numberstring{book}}{book}}
\providecommand*{\toclevel@book}{-2}
\makeatother
\titleformat{\part}[display]
    {\Large\bfseries}
    {\partname\nobreakspace\thepart}
    {0mm}
    {\Huge\bfseries}
\titlecontents{part}[0pt]
    {\large\bfseries}
    {\partname\ \thecontentslabel: \quad}
    {}
    {\hfill\contentspage}
\titlecontents{chapter}[0pt]
    {\bfseries}
    {\chaptername\ \thecontentslabel:\quad}
    {}
    {\hfill\contentspage}
\newglossarystyle{longpara}{%
    \setglossarystyle{long}%
    \renewenvironment{theglossary}{%
        \begin{longtable}[l]{{p{0.25\hsize}p{0.65\hsize}}}
    }{\end{longtable}}%
    \renewcommand{\glossentry}[2]{%
        \glstarget{##1}{\glossentryname{##1}}%
        &\glossentrydesc{##1}{~##2.}
        \tabularnewline%
        \tabularnewline
    }%
}
\newglossary[not-glg]{notation}{not-gls}{not-glo}{Notation}
\newcommand*{\newnotation}[4][]{%
    \newglossaryentry{#2}{type=notation, name={\textbf{#3}, },
                          text={#4}, description={#4},#1}%
}
%--------------------------LENGTHS------------------------------%
% Spacings for the Table of Contents.
\addtolength{\cftsecnumwidth}{1ex}
\addtolength{\cftsubsecindent}{1ex}
\addtolength{\cftsubsecnumwidth}{1ex}
\addtolength{\cftfignumwidth}{1ex}
\addtolength{\cfttabnumwidth}{1ex}

% Indent and paragraph spacing.
\setlength{\parindent}{0em}
\setlength{\parskip}{0em}
\graphicspath{{../../../images/}}   % Path to Image Folder.
%--------------------------Main Document----------------------------%
\begin{document}
    \newif\ifmathcourseschaostheory
    \ifx\ifmathcourses\undefined
        \title{Chaos Theory}
        \author{Ryan Maguire}
        \date{\vspace{-5ex}}
        \maketitle
        \tableofcontents
        \chapter*{Chaos Theory}
        \markboth{}{CHAOS THEORY}
        \setcounter{chapter}{1}
    \else
        \chapter{Chaos Theory}
    \fi 
    \section{A Review of Differential Equations}
        \subsection{First Order Equations}
        The first differential equation that is often studied
        is $\dot{x}(t)=ax(t)$, where $\dot{x}$ denotes the
        derivative of $x$ with respect to $t$. In this equation
        $a$ is some fixed constant parameter, and each real
        value $a$ defines a different differential equation.
        We can solve this by integrating and invoking the
        fundamental theorem of calculus. In general, a
        differential equation is an equation that relates
        a differentiable functions to its derivatives.
        The general solution to a differential equation is
        the set of all functions that satisfy the
        differential equation.
        \begin{ftheorem}{}{thm:CHAOS:SIMPLE_DIFF_EQ}
            If $x(t)$ is a differentiable function such
            that $x(0)=x_{0}$ and there is a $k\in\mathbb{R}$
            such that for all $t\in\mathbb{R}$, $\dot{x}(t)=kx(t)$,
            then $x(t)=x_{0}\exp(kt)$.
        \end{ftheorem}
        \begin{bproof}
            Let $y(t)=\exp(kt)$. Then $\dot{y}(t)=ky(t)$,
            and $y(t)\ne{0}$ for all $t\in\mathbb{R}$.
            Let $F(t)=x(t)/y(t)$. Then:
            \begin{equation*}
                \dot{F}(t)=
                \frac{y(t)\dot{x}(t)-\dot{y}(t)x(t)}{y^{2}(t)}
                =\frac{y(t)\big(\dot{x}(t)-kx(t)\big)}{y^{2}(t)}
                =\frac{1}{y(t)}\big(\dot{x}(t)-kx(t)\big)
            \end{equation*}
            But $\dot{x}(t)-kx(t)=0$, and thus $\dot{F}(t)=0$ for
            all $t\in\mathbb{R}$. But then $F(t)$ is a constant
            function. Thus there is a $C\in\mathbb{R}$ such
            that $x(t)=Cy(t)$. But $x(0)=x_{0}$, and therefore
            $C=x_{0}$. Thus, $x(t)=x_{0}\exp(kt)$.
        \end{bproof}
        The proof shows that the solution to $\dot{x}(t)=kx(t)$
        is uniquely determined if $x(0)$ is known. This can
        be replaced by knowledge of $x(t_{0})$ for any point
        $t_{0}\in\mathbb{R}$.
        Such restraints are called initial conditions to a
        differential equation. Initial conditions are
        requirements that a solution to the differential
        equation must satisfy. Such requirements can be
        the value of $x(t_{0})$ at a certain point
        $t_{0}$, or a requirement on $\dot{x}(t)$.
        \begin{fexample}{}{}
            Consider the following initial value problem:
            \begin{align*}
                \dot{x}(t)&=ax(t)
                &
                x(t_{0})&=x_{0}
            \end{align*}
            Using the previous theorem, we can translate the
            problem by $\exp(-kt_{0})$ to create the initial
            value problem $\dot{y}(t)=ky(t)$, $y(0)=x_{0}$.
            We know the solution to this is
            $y(t)=x_{0}\exp(kt)$. Thus, the solution to the
            original initial value problem is:
            \begin{equation*}
                x(t)=x_{0}\exp(k(t-t_{0}))
            \end{equation*}
        \end{fexample}
        An important class of solutions to differential
        equations are \textit{equilibrium solutions}.
        \begin{fdefinition}{Equilibrium Solution}{}
            An equilibrium solution to a differential
            equation is a solution $x(t)$ such that
            $\dot{x}(t)=0$ for all $t\in\mathbb{R}$.
            That is, a constant solution $x(t)=c$.
        \end{fdefinition}
        Studying our first problem $\dot{x}(t)=kx(t)$,
        we see that the only equilibrium solution
        occurs when $x_{0}=0$. Another important concept
        is the limiting behavior of the solution to a
        differential equation. The limiting behavior
        is often dependent on the various parameters that
        may be present in a given differential equation.
        We can see this by further studying $\dot{x}(t)=kx(t)$.
        If $k=0$ we see that $x(t)$ is a constant function. That
        is, $x(t)=x_{0}$ for all $t\in\mathbb{R}$. For $k>0$,
        solutions diverges monotonically to $\infty$
        ($x_{0}>0$) or $-\infty$ ($x_{0}<0$).
        Finally, if $k<0$ then $x(t)$ converges to $0$
        for all values of $x_{0}$.
        \subsubsection{Notes on the Jordan Normal Form}
            Every square matrix $A$ is similar to an upper
            triangular matrix $J$ in
            \textit{Jordan normal form} whose diagonal entries
            are the eigenvalues of $A$. That is, there exists
            an invertible matrix $P$ such that $P^{-1}AP=J$.
            The trace of $A$ is equal to the trace of $J$:
            \begin{equation*}
                \Tr(J)
                =\Tr(P^{-1}AP)
                =\Tr(P^{-1}PA)
                =\Tr(IA)=\Tr(A)
            \end{equation*}
        \subsubsection{Notes on Conjugacy}
            \begin{minipage}[t]{0.49\textwidth}
                We have:
                \begin{align*}
                    X'(t)&=AX(t),\quad
                    X(0)=X_{0}\\
                    \Rightarrow
                    X(t)&=e^{tA}X_{0}\\
                    \Rightarrow
                    \phi^{A}(t,X_{0})
                    &=e^{tA}X_{0}
                \end{align*}
            \end{minipage}
            \vline
            \hfill
            \begin{minipage}[t]{0.49\textwidth}
                If $B=T^{-1}AT$, for some matrix $T$, then:
                \begin{align*}
                    Y'(t)&=BY(t),\quad
                    Y(0)=Y_{0}\\
                    &=T^{-1}ATY(t)\\
                    \Rightarrow
                    Y(t)
                    &=e^{tT^{-1}AT}Y_{0}\\
                    &=T^{-1}e^{tA}TY_{0}\\
                    \Rightarrow
                    \phi^{B}(t,Y_{0})
                    &=T^{-1}e^{tA}TY_{0}
                \end{align*}
            \end{minipage}
            Thus, the homeomorphism is $h(X)=T^{-1}X$,
            and we have:
            \begin{equation*}
                \phi^{B}(t,h(X_{0}))
                =\phi^{B}(t,T^{-1}X_{0})
                =T^{-1}e^{tA}T(T^{-1}X_{0})
                =T^{-1}e^{tA}X_{0}
                =h(\phi^{A}(t,X_{0}))
            \end{equation*}
\end{document}