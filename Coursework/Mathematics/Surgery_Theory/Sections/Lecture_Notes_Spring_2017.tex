\documentclass[crop=false,class=article,oneside]{standalone}
%----------------------------Preamble-------------------------------%
%---------------------------Packages----------------------------%
\usepackage{geometry}
\geometry{b5paper, margin=1.0in}
\usepackage[T1]{fontenc}
\usepackage{graphicx, float}            % Graphics/Images.
\usepackage{natbib}                     % For bibliographies.
\bibliographystyle{agsm}                % Bibliography style.
\usepackage[french, english]{babel}     % Language typesetting.
\usepackage[dvipsnames]{xcolor}         % Color names.
\usepackage{listings}                   % Verbatim-Like Tools.
\usepackage{mathtools, esint, mathrsfs} % amsmath and integrals.
\usepackage{amsthm, amsfonts, amssymb}  % Fonts and theorems.
\usepackage{tcolorbox}                  % Frames around theorems.
\usepackage{upgreek}                    % Non-Italic Greek.
\usepackage{fmtcount, etoolbox}         % For the \book{} command.
\usepackage[newparttoc]{titlesec}       % Formatting chapter, etc.
\usepackage{titletoc}                   % Allows \book in toc.
\usepackage[nottoc]{tocbibind}          % Bibliography in toc.
\usepackage[titles]{tocloft}            % ToC formatting.
\usepackage{pgfplots, tikz}             % Drawing/graphing tools.
\usepackage{imakeidx}                   % Used for index.
\usetikzlibrary{
    calc,                   % Calculating right angles and more.
    angles,                 % Drawing angles within triangles.
    arrows.meta,            % Latex and Stealth arrows.
    quotes,                 % Adding labels to angles.
    positioning,            % Relative positioning of nodes.
    decorations.markings,   % Adding arrows in the middle of a line.
    patterns,
    arrows
}                                       % Libraries for tikz.
\pgfplotsset{compat=1.9}                % Version of pgfplots.
\usepackage[font=scriptsize,
            labelformat=simple,
            labelsep=colon]{subcaption} % Subfigure captions.
\usepackage[font={scriptsize},
            hypcap=true,
            labelsep=colon]{caption}    % Figure captions.
\usepackage[pdftex,
            pdfauthor={Ryan Maguire},
            pdftitle={Mathematics and Physics},
            pdfsubject={Mathematics, Physics, Science},
            pdfkeywords={Mathematics, Physics, Computer Science, Biology},
            pdfproducer={LaTeX},
            pdfcreator={pdflatex}]{hyperref}
\hypersetup{
    colorlinks=true,
    linkcolor=blue,
    filecolor=magenta,
    urlcolor=Cerulean,
    citecolor=SkyBlue
}                           % Colors for hyperref.
\usepackage[toc,acronym,nogroupskip,nopostdot]{glossaries}
\usepackage{glossary-mcols}
%------------------------Theorem Styles-------------------------%
\theoremstyle{plain}
\newtheorem{theorem}{Theorem}[section]

% Define theorem style for default spacing and normal font.
\newtheoremstyle{normal}
    {\topsep}               % Amount of space above the theorem.
    {\topsep}               % Amount of space below the theorem.
    {}                      % Font used for body of theorem.
    {}                      % Measure of space to indent.
    {\bfseries}             % Font of the header of the theorem.
    {}                      % Punctuation between head and body.
    {.5em}                  % Space after theorem head.
    {}

% Italic header environment.
\newtheoremstyle{thmit}{\topsep}{\topsep}{}{}{\itshape}{}{0.5em}{}

% Define environments with italic headers.
\theoremstyle{thmit}
\newtheorem*{solution}{Solution}

% Define default environments.
\theoremstyle{normal}
\newtheorem{example}{Example}[section]
\newtheorem{definition}{Definition}[section]
\newtheorem{problem}{Problem}[section]

% Define framed environment.
\tcbuselibrary{most}
\newtcbtheorem[use counter*=theorem]{ftheorem}{Theorem}{%
    before=\par\vspace{2ex},
    boxsep=0.5\topsep,
    after=\par\vspace{2ex},
    colback=green!5,
    colframe=green!35!black,
    fonttitle=\bfseries\upshape%
}{thm}

\newtcbtheorem[auto counter, number within=section]{faxiom}{Axiom}{%
    before=\par\vspace{2ex},
    boxsep=0.5\topsep,
    after=\par\vspace{2ex},
    colback=Apricot!5,
    colframe=Apricot!35!black,
    fonttitle=\bfseries\upshape%
}{ax}

\newtcbtheorem[use counter*=definition]{fdefinition}{Definition}{%
    before=\par\vspace{2ex},
    boxsep=0.5\topsep,
    after=\par\vspace{2ex},
    colback=blue!5!white,
    colframe=blue!75!black,
    fonttitle=\bfseries\upshape%
}{def}

\newtcbtheorem[use counter*=example]{fexample}{Example}{%
    before=\par\vspace{2ex},
    boxsep=0.5\topsep,
    after=\par\vspace{2ex},
    colback=red!5!white,
    colframe=red!75!black,
    fonttitle=\bfseries\upshape%
}{ex}

\newtcbtheorem[auto counter, number within=section]{fnotation}{Notation}{%
    before=\par\vspace{2ex},
    boxsep=0.5\topsep,
    after=\par\vspace{2ex},
    colback=SeaGreen!5!white,
    colframe=SeaGreen!75!black,
    fonttitle=\bfseries\upshape%
}{not}

\newtcbtheorem[use counter*=remark]{fremark}{Remark}{%
    fonttitle=\bfseries\upshape,
    colback=Goldenrod!5!white,
    colframe=Goldenrod!75!black}{ex}

\newenvironment{bproof}{\textit{Proof.}}{\hfill$\square$}
\tcolorboxenvironment{bproof}{%
    blanker,
    breakable,
    left=3mm,
    before skip=5pt,
    after skip=10pt,
    borderline west={0.6mm}{0pt}{green!80!black}
}

\AtEndEnvironment{lexample}{$\hfill\textcolor{red}{\blacksquare}$}
\newtcbtheorem[use counter*=example]{lexample}{Example}{%
    empty,
    title={Example~\theexample},
    boxed title style={%
        empty,
        size=minimal,
        toprule=2pt,
        top=0.5\topsep,
    },
    coltitle=red,
    fonttitle=\bfseries,
    parbox=false,
    boxsep=0pt,
    before=\par\vspace{2ex},
    left=0pt,
    right=0pt,
    top=3ex,
    bottom=1ex,
    before=\par\vspace{2ex},
    after=\par\vspace{2ex},
    breakable,
    pad at break*=0mm,
    vfill before first,
    overlay unbroken={%
        \draw[red, line width=2pt]
            ([yshift=-1.2ex]title.south-|frame.west) to
            ([yshift=-1.2ex]title.south-|frame.east);
        },
    overlay first={%
        \draw[red, line width=2pt]
            ([yshift=-1.2ex]title.south-|frame.west) to
            ([yshift=-1.2ex]title.south-|frame.east);
    },
}{ex}

\AtEndEnvironment{ldefinition}{$\hfill\textcolor{Blue}{\blacksquare}$}
\newtcbtheorem[use counter*=definition]{ldefinition}{Definition}{%
    empty,
    title={Definition~\thedefinition:~{#1}},
    boxed title style={%
        empty,
        size=minimal,
        toprule=2pt,
        top=0.5\topsep,
    },
    coltitle=Blue,
    fonttitle=\bfseries,
    parbox=false,
    boxsep=0pt,
    before=\par\vspace{2ex},
    left=0pt,
    right=0pt,
    top=3ex,
    bottom=0pt,
    before=\par\vspace{2ex},
    after=\par\vspace{1ex},
    breakable,
    pad at break*=0mm,
    vfill before first,
    overlay unbroken={%
        \draw[Blue, line width=2pt]
            ([yshift=-1.2ex]title.south-|frame.west) to
            ([yshift=-1.2ex]title.south-|frame.east);
        },
    overlay first={%
        \draw[Blue, line width=2pt]
            ([yshift=-1.2ex]title.south-|frame.west) to
            ([yshift=-1.2ex]title.south-|frame.east);
    },
}{def}

\AtEndEnvironment{ltheorem}{$\hfill\textcolor{Green}{\blacksquare}$}
\newtcbtheorem[use counter*=theorem]{ltheorem}{Theorem}{%
    empty,
    title={Theorem~\thetheorem:~{#1}},
    boxed title style={%
        empty,
        size=minimal,
        toprule=2pt,
        top=0.5\topsep,
    },
    coltitle=Green,
    fonttitle=\bfseries,
    parbox=false,
    boxsep=0pt,
    before=\par\vspace{2ex},
    left=0pt,
    right=0pt,
    top=3ex,
    bottom=-1.5ex,
    breakable,
    pad at break*=0mm,
    vfill before first,
    overlay unbroken={%
        \draw[Green, line width=2pt]
            ([yshift=-1.2ex]title.south-|frame.west) to
            ([yshift=-1.2ex]title.south-|frame.east);},
    overlay first={%
        \draw[Green, line width=2pt]
            ([yshift=-1.2ex]title.south-|frame.west) to
            ([yshift=-1.2ex]title.south-|frame.east);
    }
}{thm}

%--------------------Declared Math Operators--------------------%
\DeclareMathOperator{\adjoint}{adj}         % Adjoint.
\DeclareMathOperator{\Card}{Card}           % Cardinality.
\DeclareMathOperator{\curl}{curl}           % Curl.
\DeclareMathOperator{\diam}{diam}           % Diameter.
\DeclareMathOperator{\dist}{dist}           % Distance.
\DeclareMathOperator{\Div}{div}             % Divergence.
\DeclareMathOperator{\Erf}{Erf}             % Error Function.
\DeclareMathOperator{\Erfc}{Erfc}           % Complementary Error Function.
\DeclareMathOperator{\Ext}{Ext}             % Exterior.
\DeclareMathOperator{\GCD}{GCD}             % Greatest common denominator.
\DeclareMathOperator{\grad}{grad}           % Gradient
\DeclareMathOperator{\Ima}{Im}              % Image.
\DeclareMathOperator{\Int}{Int}             % Interior.
\DeclareMathOperator{\LC}{LC}               % Leading coefficient.
\DeclareMathOperator{\LCM}{LCM}             % Least common multiple.
\DeclareMathOperator{\LM}{LM}               % Leading monomial.
\DeclareMathOperator{\LT}{LT}               % Leading term.
\DeclareMathOperator{\Mod}{mod}             % Modulus.
\DeclareMathOperator{\Mon}{Mon}             % Monomial.
\DeclareMathOperator{\multideg}{mutlideg}   % Multi-Degree (Graphs).
\DeclareMathOperator{\nul}{nul}             % Null space of operator.
\DeclareMathOperator{\Ord}{Ord}             % Ordinal of ordered set.
\DeclareMathOperator{\Prin}{Prin}           % Principal value.
\DeclareMathOperator{\proj}{proj}           % Projection.
\DeclareMathOperator{\Refl}{Refl}           % Reflection operator.
\DeclareMathOperator{\rk}{rk}               % Rank of operator.
\DeclareMathOperator{\sgn}{sgn}             % Sign of a number.
\DeclareMathOperator{\sinc}{sinc}           % Sinc function.
\DeclareMathOperator{\Span}{Span}           % Span of a set.
\DeclareMathOperator{\Spec}{Spec}           % Spectrum.
\DeclareMathOperator{\supp}{supp}           % Support
\DeclareMathOperator{\Tr}{Tr}               % Trace of matrix.
%--------------------Declared Math Symbols--------------------%
\DeclareMathSymbol{\minus}{\mathbin}{AMSa}{"39} % Unary minus sign.
%------------------------New Commands---------------------------%
\DeclarePairedDelimiter\norm{\lVert}{\rVert}
\DeclarePairedDelimiter\ceil{\lceil}{\rceil}
\DeclarePairedDelimiter\floor{\lfloor}{\rfloor}
\newcommand*\diff{\mathop{}\!\mathrm{d}}
\newcommand*\Diff[1]{\mathop{}\!\mathrm{d^#1}}
\renewcommand*{\glstextformat}[1]{\textcolor{RoyalBlue}{#1}}
\renewcommand{\glsnamefont}[1]{\textbf{#1}}
\renewcommand\labelitemii{$\circ$}
\renewcommand\thesubfigure{%
    \arabic{chapter}.\arabic{figure}.\arabic{subfigure}}
\addto\captionsenglish{\renewcommand{\figurename}{Fig.}}
\numberwithin{equation}{section}

\renewcommand{\vector}[1]{\boldsymbol{\mathrm{#1}}}

\newcommand{\uvector}[1]{\boldsymbol{\hat{\mathrm{#1}}}}
\newcommand{\topspace}[2][]{(#2,\tau_{#1})}
\newcommand{\measurespace}[2][]{(#2,\varSigma_{#1},\mu_{#1})}
\newcommand{\measurablespace}[2][]{(#2,\varSigma_{#1})}
\newcommand{\manifold}[2][]{(#2,\tau_{#1},\mathcal{A}_{#1})}
\newcommand{\tanspace}[2]{T_{#1}{#2}}
\newcommand{\cotanspace}[2]{T_{#1}^{*}{#2}}
\newcommand{\Ckspace}[3][\mathbb{R}]{C^{#2}(#3,#1)}
\newcommand{\funcspace}[2][\mathbb{R}]{\mathcal{F}(#2,#1)}
\newcommand{\smoothvecf}[1]{\mathfrak{X}(#1)}
\newcommand{\smoothonef}[1]{\mathfrak{X}^{*}(#1)}
\newcommand{\bracket}[2]{[#1,#2]}

%------------------------Book Command---------------------------%
\makeatletter
\renewcommand\@pnumwidth{1cm}
\newcounter{book}
\renewcommand\thebook{\@Roman\c@book}
\newcommand\book{%
    \if@openright
        \cleardoublepage
    \else
        \clearpage
    \fi
    \thispagestyle{plain}%
    \if@twocolumn
        \onecolumn
        \@tempswatrue
    \else
        \@tempswafalse
    \fi
    \null\vfil
    \secdef\@book\@sbook
}
\def\@book[#1]#2{%
    \refstepcounter{book}
    \addcontentsline{toc}{book}{\bookname\ \thebook:\hspace{1em}#1}
    \markboth{}{}
    {\centering
     \interlinepenalty\@M
     \normalfont
     \huge\bfseries\bookname\nobreakspace\thebook
     \par
     \vskip 20\p@
     \Huge\bfseries#2\par}%
    \@endbook}
\def\@sbook#1{%
    {\centering
     \interlinepenalty \@M
     \normalfont
     \Huge\bfseries#1\par}%
    \@endbook}
\def\@endbook{
    \vfil\newpage
        \if@twoside
            \if@openright
                \null
                \thispagestyle{empty}%
                \newpage
            \fi
        \fi
        \if@tempswa
            \twocolumn
        \fi
}
\newcommand*\l@book[2]{%
    \ifnum\c@tocdepth >-3\relax
        \addpenalty{-\@highpenalty}%
        \addvspace{2.25em\@plus\p@}%
        \setlength\@tempdima{3em}%
        \begingroup
            \parindent\z@\rightskip\@pnumwidth
            \parfillskip -\@pnumwidth
            {
                \leavevmode
                \Large\bfseries#1\hfill\hb@xt@\@pnumwidth{\hss#2}
            }
            \par
            \nobreak
            \global\@nobreaktrue
            \everypar{\global\@nobreakfalse\everypar{}}%
        \endgroup
    \fi}
\newcommand\bookname{Book}
\renewcommand{\thebook}{\texorpdfstring{\Numberstring{book}}{book}}
\providecommand*{\toclevel@book}{-2}
\makeatother
\titleformat{\part}[display]
    {\Large\bfseries}
    {\partname\nobreakspace\thepart}
    {0mm}
    {\Huge\bfseries}
\titlecontents{part}[0pt]
    {\large\bfseries}
    {\partname\ \thecontentslabel: \quad}
    {}
    {\hfill\contentspage}
\titlecontents{chapter}[0pt]
    {\bfseries}
    {\chaptername\ \thecontentslabel:\quad}
    {}
    {\hfill\contentspage}
\newglossarystyle{longpara}{%
    \setglossarystyle{long}%
    \renewenvironment{theglossary}{%
        \begin{longtable}[l]{{p{0.25\hsize}p{0.65\hsize}}}
    }{\end{longtable}}%
    \renewcommand{\glossentry}[2]{%
        \glstarget{##1}{\glossentryname{##1}}%
        &\glossentrydesc{##1}{~##2.}
        \tabularnewline%
        \tabularnewline
    }%
}
\newglossary[not-glg]{notation}{not-gls}{not-glo}{Notation}
\newcommand*{\newnotation}[4][]{%
    \newglossaryentry{#2}{type=notation, name={\textbf{#3}, },
                          text={#4}, description={#4},#1}%
}
%--------------------------LENGTHS------------------------------%
% Spacings for the Table of Contents.
\addtolength{\cftsecnumwidth}{1ex}
\addtolength{\cftsubsecindent}{1ex}
\addtolength{\cftsubsecnumwidth}{1ex}
\addtolength{\cftfignumwidth}{1ex}
\addtolength{\cfttabnumwidth}{1ex}

% Indent and paragraph spacing.
\setlength{\parindent}{0em}
\setlength{\parskip}{0em}
\graphicspath{{../../../../images/}}    % Path to Image Folder.
%--------------------------Main Document----------------------------%
\begin{document}
    \ifx\ifmathcoursessurgery\undefined
        \section*{Surgery Theory}
        \setcounter{section}{1}
        \renewcommand\thefigure{\arabic{section}.\arabic{figure}}
        \renewcommand\thesubfigure{%
            \arabic{section}.\arabic{figure}.\arabic{subfigure}}
    \else
        \section{Lecture Notes from Spring 2017 (Wellesley College)}
    \fi
    \subsection{Lecture 1: Homotopy}
        \begin{wrapfigure}[8]{l}{0.35\textwidth}
            \centering
            \captionsetup{type=figure}
            \subimport{../../../../tikz/}{Homotopy_Example}
            \caption[Homotopy Diagram.]
                    {Diagram for a homotopy between two
                    functions $f,g:X\rightarrow Y$.}
            \label{fig:surgery_theory_course_homotopy_diagram_%
                   for_depicting_what_a_homotopy_is}
        \end{wrapfigure}
        Let $X$ and $Y$ be topological spaces, and let
        $f:{X}\rightarrow{Y}$ and $g:{X}\rightarrow{Y}$
        be continuous functions. We now define what it
        means for $f$ and $g$ to be \textit{homotopic}.
        \begin{definition}
            A homotopy between continuous functions
            $f,g:{X}\rightarrow{Y}$ is a continuous
            function $H:{X}\times{I}\rightarrow{Y}$
            such that $H(x,0)=f(x)$ and $H(x,1)=g(x)$.
        \end{definition}
        \begin{definition}
            Homotopic functions are continuous functions
            $f,g:{X}\rightarrow{Y}$, denoted ${f}\simeq{g}$,
            with a homotopy between them.
        \end{definition}
        \begin{notation}
            The set of continuous functions
            $f:{X}\rightarrow{Y}$ is denoted $C(X,Y)$.
        \end{notation}
        Fig.~\ref{%
            fig:surgery_theory_course_homotopy_diagram_%
            for_depicting_what_a_homotopy_is%
        }
        shows two topological spaces and two homotopic
        continuous functions $f,g:{X}\rightarrow{Y}$.
        \begin{example}
            Let $X=\mathbb{R}^{n}$ and $Y=\mathbb{R}^{m}$. Let
            $f,g:\mathbb{R}^{n}\rightarrow\mathbb{R}^{m}$
            be arbitrary continuous functions.
            The `straight line' homotopy is a homotopy
            between any such functions. Let
            $H:\mathbb{R}^{n}\times{I}\rightarrow\mathbb{R}^{m}$
            be defined by $H(x,t)=(1-t)f(x)+tg(x)$. Then
            $H(x,0)=f(x)$, $H(x,1)=g(x)$, and
            $H$ is continuous. Thus, ${f}\simeq{g}$.
            Note that $g(x)=constant$ is possible.
            Any continuous function
            $f:\mathbb{R}^{n}\rightarrow\mathbb{R}^{m}$
            is homotopic to a point.
        \end{example}
        \begin{wrapfigure}[8]{r}{0.32\textwidth}
            \centering
            \captionsetup{type=figure}
            \vspace{-1ex}
            \subimport{../../../../tikz/}{Homotopy_on_Unit_Interval}
            \caption{Straight-Line Homotopy.}
            \label{fig:surgery_theory_course_homotopy_diagram_%
                   for_straight_line_homotopy}
        \end{wrapfigure}
        One can visualize a homotopy by letting $X=[0,1]$,
        and $Y\subset\mathbb{R}^{2}$ be a nice blob,
        like the one shown in
        Fig.~\ref{%
            fig:surgery_theory_course_homotopy_diagram_for_%
            straight_line_homotopy%
        }.
        Let $f:[0,1]\rightarrow Y$ and $g:[0,1]\rightarrow Y$
        be smooth curves within the blob. Then the homotopy
        $H(x,t)=(1-t)f(x)+tg(x)$ is the map that drags $f(x)$
        to $g(x)$ via the straight line connecting the two
        points. This is done for every point $x\in [0,1]$.
        The next thing to show is that the notion of
        homotopy $\simeq$ is an equivalence relation on the
        set $C(X,Y)$.
        \begin{theorem}
            Homotopic is an equivalence relation.
        \end{theorem}
        \begin{proof}
            We must show that $\simeq$ is reflexive,
            symmetric, and transitive.
            \begin{enumerate}
                \item If ${f}\in{C(X,Y)}$,
                    let $H(x,t)=f(x)$. Then
                    $H\in{C({X}\times{I},Y)}$, $H(x,0)=f(x)$,
                    and $H(x,1)=f(x)$.
                    Therefore $H$ is a homotopy between $f$ and
                    itself. That is, $f\simeq f$.
                \item If ${f,g}\in{C(X,Y)}$ and ${f}\simeq{g}$,
                    then there exists a homotopy $H$
                    between $f$ and $g$.
                    Let $G=H(x,1-t)$. Then
                    $G\in{C({X}\times{I},Y)}$,
                    $G(x,0)=g(x)$, and $G(x,1)=f(x)$.
                    Therefore, ${g}\simeq{f}$.
                \item If ${f,g,h}\in{C(X,Y)}$, ${f}\simeq{g}$,
                    and ${g}\simeq{h}$, then there exists
                    homotopy $H_{1}$ between $f$ and $g$ and
                    a homotopy $H_{2}$ between $g$ and $h$.
                    Let $H:{X}\times{I}\rightarrow{Y}$
                    be defined by:
                    \begin{equation*}
                        H_{3}(x,t)=
                        \begin{cases}
                            H_{1}(x,2t),
                            &{0}\leq{t}\leq\frac{1}{2}\\
                            H_{2}(x,2t-1),
                            &\frac{1}{2}<{t}\leq{1}
                        \end{cases}
                    \end{equation*}
                    By the pasting lemma, 
                    $H_{3}\in{C({X}\times{I},Y)}$.
                    But $H_{3}(x,0)=f(x)$ and
                    $H_{3}(x,1)=h(x)$.
                    Thus, $f\simeq h$.
            \end{enumerate}
        \end{proof}
        \clearpage
        \begin{definition}
            Homotopy equivalent spaces are topological
            spaces $X$ and $Y$ such that there exists functions
            ${f}\in{C(X,Y)}$ and ${g}\in{C(Y,X)}$
            such that
            ${f}\circ{g}\simeq{id_{Y}}$
            and ${g}\circ{f}\simeq{id_{X}}$.
        \end{definition}
        \begin{definition}
            A homeomorphism from a topological space $X$ to a
            topological space $Y$ is a continuous bijection
            $f:{X}\rightarrow{Y}$ such that
            $f^{-1}:{Y}\rightarrow{X}$ is continuous.
        \end{definition}
        \begin{definition}
            Homeomorphic topological spaces are topological
            spaces $X$ and $Y$ such that there
            exists a homeomorphism
            $f:{X}\rightarrow{Y}$ between them.
        \end{definition}
        \begin{theorem}
            \label{%
                theorem:surgery_theory_homeomorphic_%
                implies_homotopy_equivalent%
            }
            If $X$ and $Y$ are homeomorphic, then they
            are homotopy equivalent.
        \end{theorem}
        \begin{proof}
            If $X$ and $Y$ are homeomorphic, then there is a
            homeomorphism $f:X\rightarrow Y$. But then $f$ is a
            continuous map from $X$ to $Y$, and $f^{-1}$ is a
            continuous map from $Y$ to $X$. Moreover,
            ${f}\circ{f^{-1}}=id_{Y}$, and
            ${f^{-1}}\circ{f}=id_{X}$,
            for $f$ is a bijection. But
            ${id_{X}}\simeq{id_{X}}$,
            and ${id_{Y}}\simeq{id_{Y}}$. Therefore, etc.
        \end{proof}
        The study of surgery theory ask about the
        converse of theorem
        \ref{%
            theorem:surgery_theory_homeomorphic_%
            implies_homotopy_equivalent%
        }.
        The converse of this theorem is not always true,
        as we will now demonstrate.
        \begin{theorem}
            \label{%
                theorem:surgery_theory_homotopic_does_%
                not_imply_homeomorphic%
            }
            There exist homotopy equivalent spaces
            that are not homeomorphic.
        \end{theorem}
        \begin{proof}
            Let $X=\mathbb{R}^{2}$ and $Y=\{(0,0)\}$.
            Let $f:{X}\rightarrow{Y}$ be defined by
            $f(x,y)=(0,0)$. Let $g=id_{Y}$. Then
            $g\circ{f}=(0,0)$. Let $H(x,y,t)=(1-t)(x,y)$.
            Then $H$ is continuous, $H(x,y,0)=(x,y)$,
            and $H(x,y,1)=(0,0)$. Thus, $H$ is a
            homotopy between ${g}\circ{f}$ and $id_{X}$, and
            therefore ${g}\circ{f}\simeq{id_{X}}$. But also
            ${f}\circ{g}=id_{Y}$, and
            ${id_{Y}}\simeq{id_{Y}}$.
            Therefore $X$ and $Y$ are homotopy equivalent.
            If $h:{X}\rightarrow{Y}$ is a
            homeomorphism, then it is a bijection. But if $h$
            is a bijection, then $|X|=|Y|$. But
            $\mathbb{R}^{2}$ is uncountable, and $|Y|=1$. A
            contradiction. Therefore $X$ and $Y$ are
            not homeomorphic.
        \end{proof}
        Fig.~\subref{%
            fig:surgery_theory_course_homotopy_equivalence_%
            diagram_of_plane_with_point%
        }
        shows the mapping $f$ between $\mathbb{R}^{2}$ and
        $\{(0,0)\}$.
        Theorem~\ref{%
            theorem:surgery_theory_homotopic_does_not_%
            imply_homeomorphic%
        }
        relies on the fact that $\mathbb{R}^{2}$ and $\{(0,0)\}$
        are of different \textit{cardinality}. However, even
        if the topological spaces $X$ and $Y$ are homotopy
        equivalent, and are of the same cardinality, it is
        still possible that they are not homeomorphic.
        We will need to show that homeomorphisms preserve
        the notion of \textit{compactness}.
        \begin{definition}
            An open cover of a subset $A$ of a topological
            space $X$ with topology $\tau$
            is a set of open sets
            $\mathcal{O}\subset\tau$ such that
            $A\subset\cup_{\mathcal{U}\in\mathcal{O}}\mathcal{U}$.
        \end{definition}
        There's is a fundamental theorem from topology and
        the study of Euclidean spaces that we will use frequently.
        There are some building blocks to get to it.
        \begin{definition}
            A compact subset of a topological space $X$ is a
            set $A\subset{X}$ such that for every open cover
            $\mathcal{O}$ of $A$, there is a finite subcover
            $\Delta\subset\mathcal{O}$. That is,
            $\Delta$ is finite and an open cover of $A$.
        \end{definition}
        \begin{theorem}
            If $X$ is compact and
            $S\subset{X}$ is closed, then
            $S$ is compact.
        \end{theorem}
        \begin{proof}
            For let $\mathcal{O}$ be an open cover
            of $S$. Then
            $\mathcal{O}\cup\{S^{C}\}$ is an open
            cover of $X$, as $S$ is closed and therefore
            $S^{C}$ is open. But $X$ is compact and therefore
            there is an open subcover $\Delta$. But then
            $\Delta\setminus\{S^{C}\}$ is an finite subcover
            of $S$. Therefore, etc.
        \end{proof}
        \begin{theorem}
            If $a,b\in\mathbb{R}$ and $a<b$, then
            $[a,b]$ is compact.
        \end{theorem}
        \begin{proof}
            For suppose not. Then there is an open
            cover $\mathcal{O}$ of $[a,b]$ with no finite
            subcover. Let $A$ be the set
            $A=\{r\in\mathbb{R}:[a,r]%
                 \textrm{ has a finite subcover}\}$.
            As $\mathcal{O}$ is an open cover, there is
            an open subset $\mathcal{U}_{1}\in\mathcal{O}$ such
            that $a\in\mathcal{U}_{1}$. Therefore $A$ is
            not empty. Moreoever, as $[a,b]$ is not compact,
            for all $r\in{A}$, $r<b$. Therefore $A$ is bounded
            above. By the least upper bound property there
            is a $\gamma\in\mathbb{R}$ such that for
            all $r\in{A}$, $r\leq\gamma$. But, as
            $\mathcal{U}_{1}$ is open and $a\in\mathcal{U}_{1}$,
            $a<\gamma\leq{b}$. But then $\gamma\in[a,b]$, and
            thus there is a $\mathcal{U}_{2}$ such that
            $\gamma\in\mathcal{U}_{2}$. But as
            $\mathcal{U}_{2}$ is open, there is an $r>0$ such
            that $(\gamma-r,\gamma+r)\subset\mathcal{U}_{2}$.
            But then $[a,\gamma+r/2]$ has a finite subcover,
            a contradiction as $\gamma$ is the least upper bound
            of $A$. Therefore $[a,b]$ is compact.
        \end{proof}
        \begin{theorem}
            \label{%
                theorem:surgery_theory_%
                product_of_compact_is_compact
            }
            If $A$ and $B$ are compact, then
            $A\times{B}$ is compact (With respect to the
            product topology).
        \end{theorem}
        \begin{proof}
            For let $\mathcal{O}$ be an open cover
            of $A\times{B}$.  Then
            $\{\pi_{A}(\mathcal{U}):\mathcal{U}\in\mathcal{O}\}$,
            that is, the set of projections of open sets in
            $\mathcal{O}$ onto $A$, is an open cover of $A$.
            Similarly for $B$. But $A$ and $B$ are compact, and
            therefore there exists finite subcovers. Taking the
            union of these two gives
            a finite subcover of $A\times{B}$.
        \end{proof}
        \begin{theorem}
            \label{%
                theorem:surgery_theory_finite_%
                product_of_compact_is_compact%
            }
            If $A_{1},\hdots,A_{n}$ are compact,
            then $A_{1}\times\hdots\times{A_{n}}$ is compact.
        \end{theorem}
        The finiteness of the product in Thm.~\ref{%
            theorem:surgery_theory_finite_%
            product_of_compact_is_compact%
        } is
        unnecessary (But it makes the proof easier). There
        is a result called Tychonoff's Theorem, which is
        actually equivalent to the axiom of choice, which
        says that given an arbitrary collection of compact sets,
        the space formed by the product of these sets is also
        compact, with respect to the product topology.
        We can now prove our main result.
        \begin{theorem}[Heine-Borel Theorem]
            \label{theorem:surgery_theory_Heine_Borel}
            A subset $S\subset\mathbb{R}^{n}$ is compact if
            and only if it closed and bounded.
        \end{theorem}
        \begin{proof}
            Suppose $S$ is compact and
            suppose it is unbounded. Then the set of open
            balls about the origin
            $B_{n}(0)=\{\mathbf{x}\in\mathbb{R}^{n}%
                        :\norm{\mathbf{x}}<n\}$
            is an open cover of $S$, since it is an open
            cover of $\mathbb{R}^{n}$, and yet no
            finite subcover exists. For if one did, then there
            is a least $N\in\mathbb{N}$ such that
            $S\subset{B_{N}(0)}$, a contradiction as
            $S$ is unbounded. Therefore $S$ is bounded.
            Furthermore, suppose $S$ is
            not closed. Then there exists a point
            $\mathbf{x}\in{S^{C}}$ such that, for all
            $r>0$, $B_{r}(\mathbf{x})\cap{S}\ne\emptyset$,
            where
            $B_{r}(\mathbf{x})=\{\mathbf{y}\in\mathbb{R}^{n}:%
             \norm{\mathbf{x}-\mathbf{y}}<r\}$.
            Let $\overline{B}_{r}(\mathbf{x})$ be the
            closure of these sets (That is, the closed ball
            about $\mathbf{x}$). Then the set of complements
            $\overline{B}_{n}(\mathbf{x})^{C}$ is an open
            cover of of $S$, for it is an open cover of
            $\mathbb{R}^{n}\setminus\{\mathbf{x}\}$, but no
            finite subcover exists. Thus $S$ is closed. Therefore,
            if $S$ is compact then it is closed and bounded.
            If $S$ is bounded, then there is an $r\in\mathbb{R}$
            such that $S\subset[-r,r]^{n}$. But
            $[-r,r]^{n}$ is the product of compact sets, and
            is therefore compact. But $S$ is closed, and closed
            subsets of compact spaces are compact. Therefore
            $S$ is compact.
        \end{proof}
        This will help find examples and counterexamples for
        the converse of Thm.~\ref{%
            theorem:surgery_theory_homeomorphic_%
            implies_homotopy_equivalent%
        }. Homeomorphisms preserve the notion of compactness.
        \begin{theorem}
            If $X$ and $Y$ are homeomorphic, and if
            $X$ is compact, then $Y$ is compact.
        \end{theorem}
        \begin{proof}
            For if $X$ and $Y$ are homeomorphic, then there
            is a continuous function $f:X\rightarrow{Y}$.
            Let $\mathcal{O}$ be an open cover of $Y$.
            Then $\{f^{-1}(\mathcal{U}):\mathcal{U}\in\mathcal{O}\}$
            is an open cover of $X$. But $X$ is compact, and
            therefore there is a finite subcover
            $\Delta$. But then
            $\{\mathcal{U}\in\mathcal{O}:%
             f^{-1}(\mathcal{U})\in\Delta\}$ is a finite
            subcover of $Y$.
        \end{proof}
        \begin{theorem}
            \label{%
                theorem:surgery_theory_Homotopy_%
                equivalance_of_plane_without_point_and_unit_%
                disc_but_not_homeomorphic%
            }
            $\mathbb{R}^{2}\setminus\{(0,0)\}$ is homotopy
            equivalent to $S^{1}$, but not homeomorphic.
        \end{theorem}
        \begin{proof}
            For let $X=\mathbb{R}^{2}\setminus\{(0,0)\}$,
            and let
            $Y=S^{1}=\{(x,y)\in\mathbb{R}^{2}:x^{2}+y^{2}=1\}$.
            Let $f:{X}\rightarrow{Y}$ be defined by
            $f(x,y)=(x,y)/\norm{(x,y)}$. Let
            $g:{Y}\rightarrow{X}$ be defined by $g(x,y)=(x,y)$.
            Define the function $H$ by
            $H(x,y,t)=(1-t)f(x,y)+tg(x,y)$.
            But then $H(x,y,0)=f(x,y)$,
            and $H(x,y,1)=g(x,y)$. Thus $H$ is a
            homotopy between ${g}\circ{f}$ and $id_{X}$. But also
            $({f}\circ{g})(x,y)=(x,y)$, for all $(x,y)\in S^{1}$.
            Therefore ${f}\circ{g}=id_{Y}$, and
            ${id_{Y}}\simeq{id_{Y}}$. Therefore, $X$ and $Y$ are
            homotopy equivalent. But $X$ is unbounded,
            and is therefore not compact,
            and $Y$ is closed and bounded,
            and is thus compact. But homeomorphisms
            preserve compactness. Therefore $X$ and $Y$ are
            not homeomorphic.
        \end{proof}
        \begin{figure}[H]
            \centering
            \captionsetup{type=figure}
            \begin{subfigure}[b]{0.33\textwidth}
                \captionsetup{type=figure}
                \centering
                \subimport{../../../../tikz/}{Retraction_of_Plane_to_Point}
                \subcaption{Retraction of $\mathbb{R}^{2}$
                            to $(0,0)$}
                \label{fig:surgery_theory_course_homotopy_%
                       equivalence_diagram_of_plane_with_point}
            \end{subfigure}
            \begin{subfigure}[b]{0.66\textwidth}
                \captionsetup{type=figure}
                \centering
                \subimport{../../../../tikz/}
                          {HE_Plane_Without_Point_to_Circle}
                \subcaption{Homotopy Equivalence of
                            $\mathbb{R}^{2}\setminus\{(0,0)\}$
                            and $S^{1}$}
                \label{fig:surgery_theory_homotopy_equivalence_%
                       between_the_plane_with_a_point_removed_%
                       and_the_unit_circle}
            \end{subfigure}
            \caption{Examples of Homotopy Equivalences
                     That are not Homeomorphic.}
            \label{%
                fig:surgery_theory_course_various_HE_%
                but_not_homeo_examples%
            }
        \end{figure}
        Theorem \ref{%
            theorem:surgery_theory_Homotopy_%
            equivalance_of_plane_without_point_%
            and_unit_disc_but_not_homeomorphic%
        }
        relies on the fact that $S^{1}$ is compact and
        $\mathbb{R}^{2}\setminus\{(0,0)\}$ isn't.
        However, even if $X$ and $Y$ are both
        compact, and of the same cardinality, it is possible
        that they are homotopy equivalent but not homeomorphic.
        We'll need some results about connectedness to show this.
        \begin{definition}
            A disconnected subset of a topological space $X$
            is a set $S\subset{X}$ such that there exists
            disjoint non-empty open sets $X_{1},X_{2}$ such
            that $A=X_{1}\cup{X_{2}}$.
        \end{definition}
        \begin{definition}
            A connected subset is a set that is not
            disconnected.
        \end{definition}
        \begin{theorem}
            If $X$ and $Y$ are homeomorphic and
            $X$ is connected, then $Y$ is connected.
        \end{theorem}
        \begin{proof}
            Suppose not. If $Y$ is disconnected, then
            there are disjoint non-empty open sets $Y_{1},Y_{2}$
            such that $Y=Y_{1}\cup{Y_{2}}$. But as $X$ and $Y$
            are homeomorphic, there is a continuous function
            $f:X\rightarrow{Y}$. But then
            $f^{-1}(Y_{1})$ and $f^{-1}(Y_{2})$ are
            non-empty, as $f$ is a bijection, and moreoever
            they are disjoint open subsets of $X$, as
            $f$ is continuous. But then $X$ is disconnected,
            a contradiction. Therefore $Y$ is connected.
        \end{proof}
        \begin{theorem}
            $[-1,1]$ and $[-1,1]^{2}$ are homotopy equivalent,
            but not homeomorphic.
        \end{theorem}
        \begin{proof}
            Let $X=[-1,1]$ and $Y=[-1,1]^{2}$.
            Let $f:X\rightarrow{Y}$ be defined by
            $f(x)=(x,0)$ and $g:Y\rightarrow{Y}$ be defined
            by $g(x,y)=x$.
            Then $H(x,t)=f(x)$ is a homotopy between
            $g\circ{f}$ and $id_{X}$, and thus
            $g\circ{f}\simeq{id_{X}}$. But also
            $H(x,y,t)=(1-t)g(x,0)+(x,y)$ is a homotopy
            between $f\circ{g}$ and $id_{Y}$, and thus
            $f\circ{g}\simeq{id_{Y}}$. Therefore $X$ and $Y$
            are homotopy equivalent. Suppose $h$ is a
            homeomorphism $h:X\rightarrow{Y}$ and let
            $h(0)=\mathbf{x}\in{Y}$. If $h$ is a homeomorphism
            between $X$ and $y$, then the restriction of
            $h$ to $X\setminus\{0\}$ is a homeomophism
            between $[-1,0)\cup(0,1]$ and
            $[-1,1]^{2}\setminus\{\mathbf{x}\}$. But
            $[-1,1]^{2}\setminus\{\mathbf{x}\}$ is connected,
            and $[-1,0)\cup(0,1]$ is not. But homeomorphisms
            preserve connectedness. Therefore, $X$ and
            $Y$ are not homeomorphic.
        \end{proof}
        If $X$ and $Y$ are of the same dimension,
        it is still possible that they are homotopy equivalent,
        but not homeomorphic. First, we show that
        $S^{2}\setminus\{(0,0,1)\}$ is homeomorphic to $D^{2}$.
        \begin{theorem}
            \label{%
                theorem:surgery_theory_the_sphere_%
                with_a_point_removed_is_homeomorphic_%
                to_the_plane%
            }
            $S^{2}\setminus\{(0,0,1)\}$ is
            homeomorphic to $\mathbb{R}^{2}$
        \end{theorem}
        \begin{proof}
            For let
            $f:S^{2}\setminus\{(0,0,1)\}%
             \rightarrow \mathbb{R}^{2}$
            be the stereographic projection mapping,
            $f(x,y,z)=(\frac{x}{1-z},\frac{y}{1-z})$,
            for $(x,y,z)\in S^{2}\setminus\{(0,0,1)\}$.
            If $(X,Y)\in\mathbb{R}^{2}$, let:
            \begin{align*}
                x&=\frac{2X}{\norm{(X,Y)}^{2}+1}&
                y&=\frac{2Y}{\norm{(X,Y)}^{2}+1}&
                z&=\frac{\norm{(X,Y)}^{2}-1}{\norm{(X,Y)}^{2}+1} 
            \end{align*}
            Then:
            \begin{equation*}
                \bigg(\frac{x}{1-z},\frac{y}{1-z}\bigg)
                =
                \bigg(
                    \frac{\frac{2X}{\norm{(X,Y)}^{2}+ 1}}
                    {\frac{2}{\norm{(X,Y)}^{2}+1}},
                    \frac{\frac{2Y}{\norm{(X,Y)}^{2}+1}}
                    {\frac{2}{\norm{(X,Y)}^{2}+1}}
                \bigg)
                =(X,Y)    
            \end{equation*}
            and
            \begin{align*}
                \norm{(x,y,z)}
                &=\sqrt{
                    \frac{4X^{2}}
                    {\big(\norm{(X,Y)}^{2}+1\big)^{2}}
                    +
                    \frac{4Y^{2}}
                    {\big(\norm{(X,Y)}^{2}+1\big)^{2}}
                    + \frac{(\norm{(X,Y)}^{2}-1)^{2}}
                    {(\norm{(X,Y)}^{2}+1)^{2}}
                }\\
                &=\sqrt{
                    \frac{
                        4\norm{(X,Y)}^{2}
                        +\norm{(X,Y)}^{4}
                        -2\norm{(X,Y)}^{2}+1}
                    {(\norm{(X,Y)}+1)^{2}}
                }\\
                &=\sqrt{
                    \frac{\norm{(X,Y)}^{4}+2\norm{(X,Y)}^{2}+1}
                    {(\norm{(X,Y)}^{2}+1)^{2}}
                }
                =\sqrt{
                    \frac{(\norm{(X,Y)}^{2}+1)^{2}}
                    {(\norm{(X,Y)}^{2}+1)^{2}}
                }
                =1
            \end{align*}
            Thus, $(x,y,z)\in S^{2}\setminus\{(0,0,1)\}$,
            and $f$ is surjective.
            If $f(x_{1},y_{1},z_{1})=f(x_{2},y_{2},z_{2})$,
            then $z_{1}=z_{2}$.
            For as
            $(x_{1},y_{1},z_{1})\in S^{2}\setminus\{(0,0,1)\}$,
            and therefore
            $x_{1}^{2}+y_{1}^{2}=1-z_{1}^{2}$, we have:
            \begin{equation*}
                \norm{(X,Y)}^{2}
                =\frac{x_{1}^2+y_{1}^2}{(1-z_{1})^{2}}
                =\frac{1-z_{1}^{2}}{(1-z_{1})^{2}}
                =\frac{x_{2}^{2}+y_{2}^{2}}{(1-z_{2})^{2}}
                =\frac{1-z_{2}^{2}}{(1-z_{2})^{2}}
            \end{equation*}
            So we have
            $\frac{1-z_{1}^{2}}{(1-z_{1})^{2}}%
             =\frac{1-z_{2}^{2}}{(1-z_{2})^{2}}$.
            Simplifying, we get
            $\frac{1+z_{1}}{1-z_{1}}=\frac{1+z_{2}}{1-z_{2}}$.
            But $f(x)=\frac{1+x}{1-x}$ is an injective function,
            and therefore $z_{1}=z_{2}$.
            From this $x_{1}=x_{2}$ and $y_{1}=y_{2}$.
            Thus, $f$ is a bijection. Moreoever,
            $f$ is continuous and $f^{-1}(X,Y)%
             =(\frac{2X}{\norm{(X,Y)}^{2}+1},%
               \frac{2Y}{\norm{(X,Y)}^{2}+1},%
               \frac{\norm{(X,Y)}^{2}-1}{\norm{(X,Y)}^{2}+1})$,
            which is continuous. $f$ is a homeomorphism.
        \end{proof}
        \begin{figure}[H]
            \captionsetup{type=figure}
            \centering
            \subimport{../../../../tikz/}{Stereographic_Projection}
            \caption{Stereographic Projection of the
                     Sphere onto the Plane.}
            \label{fig:surgery_theory_stereographic_%
                   projection_of_sphere_to_plane_homeomorphism}
        \end{figure}
        Fig.~\ref{%
            fig:surgery_theory_stereographic_%
            projection_of_sphere_to_plane_homeomorphism%
        }
        depicts the stereographic projection used
        to prove theorem
        \ref{%
            theorem:surgery_theory_the_sphere_with%
            _a_point_removed_is_homeomorphic_to_the_plane%
        }.
        It can be seen that $(0,0,1)$ projects `to infinity'.
        Because of this, it is not uncommon to call this point
        infinity. Next, we prove that $\mathbb{R}^{2}$ is
        homeomorphic to $D^{2}$, almost completing our claim
        that $S^{2}\setminus\{(0,0,1)\}$
        is homeomorphic to $D^{2}$.
        \begin{theorem}
            $\mathbb{R}^{2}$ is homeomorphic to $D^{2}$.
        \end{theorem}
        \begin{proof}
            Let $f:D^{2}\rightarrow\mathbb{R}^{2}$
            be defined by
            $f(\mathbf{x})%
             =\frac{\mathbf{x}}{1-\norm{\mathbf{x}}}$.
            $f$ is surjective.
            For $\mathbf{0}\mapsto\mathbf{0}$.
            If $\mathbf{y}\in\mathbb{R}^2\setminus\mathbf{0}$,
            then let
            $\mathbf{x}=\frac{\mathbf{y}}{1+\norm{\mathbf{y}}}$.
            Then
            $\norm{\mathbf{x}}%
             =\frac{\norm{\mathbf{y}}}{1+\norm{\mathbf{y}}}<1$,
            and thus $\mathbf{x}\in D^{2}$.
            But
            $f(\mathbf{x})%
             =\frac{\mathbf{y}}{1+\norm{\mathbf{y}}}%
              (1-\frac{\norm{\mathbf{y}}}%
                      {1+\norm{\mathbf{y}}})^{-1}%
             =\mathbf{y}$.
            Moreover, $f$ is injective.
            For if
            $f(\mathbf{x}_{1})=f(\mathbf{x}_{2})$,
            then
            $\frac{\norm{\mathbf{x}_{1}}}%
             {1+\norm{\mathbf{x}_{1}}}%
             =\norm{f(\mathbf{x}_{1})}%
             =\norm{f(\mathbf{x}_{2})}%
             =\frac{\norm{\mathbf{x}_{2}}}%
              {1+\norm{\mathbf{x}}_{2}}$,
            and therefore
            $\norm{\mathbf{x}}_{1}=\norm{\mathbf{x}_{2}}$.
            But
            $\frac{\mathbf{x}_{1}}{1+\norm{\mathbf{x}_{1}}}%
             =\frac{\mathbf{x}_{2}}{1+\norm{\mathbf{x}_{2}}}$,
            and therefore $\mathbf{x}_{1}=\mathbf{x}_{2}$.
            $f$ is bijective.
            Moreover, $f$ is continuous. Finally,
            $f^{-1}(\mathbf{y})%
             =\frac{\mathbf{y}}{1+\norm{\mathbf{y}}}$
            is continuous. $f$ is a homeomorphism.
        \end{proof}
        \begin{theorem}
            $S^{2}\setminus\{(0,0,1)\}$
            is homeomorphic to $D^{2}$.
        \end{theorem}
        \begin{proof}
            For $S^{2}\setminus\{(0,0,1)\}$ is
            homeomorphic to $\mathbb{R}^{2}$, and
            $\mathbb{R}^{2}$ is homeomorphic to $D^{2}$.
            But homeomorphism is an equivalence relation, so
            $S^{2}\setminus\{(0,0,1)\}$
            is homeomorphic to $D^{2}$.
        \end{proof}
        \begin{figure}[H]
            \centering
            \captionsetup{type=figure}
            \subimport{../../../../tikz/}{Turning_Sphere_Into_Plane}
            \caption[Homeomorphism Between 
                     $S^{2}\setminus\{(0,0,1)\}$ and $D^{2}$.]
                    {The units phere with a point removed can be
                     continuously deformed into the open unit disc.}
            \label{fig:my_label}
        \end{figure}
        We can use the fact that $S^{2}\setminus \{(0,0,1)\}$
        is homeomorphic to $D^{2}$ to construct examples of
        topological manifolds of the same dimensions that are
        homotopy equivalent, but not homeomorphic. We may
        generalize to $S^{2}$ with $n$ points removed is
        homeomorphic to $D^{2}$ with $n-1$ points removed.
        We now define the notion of
        \textit{manifold} and \textit{dimension}.
        \begin{definition}
            An $n$ dimensional manifold is a Hausdorff
            topological space $X$ such that for all
            $p\in{X}$ there is an open neighborhood
            $\mathcal{U}$ of $p$, such that $\mathcal{U}$
            is homeomorphic to $\mathbb{R}^{n}$.
        \end{definition}
        It can be shown that if $X$ and $Y$ are homeomorphic
        manifolds, then they are of the same dimension.
        This is simply because $\mathbb{R}^{n}$ is homeomorphic
        to $\mathbb{R}^{m}$ if and only if $n=m$. Therefore,
        homeomorphisms preserve dimension. We use the fact that
        a sphere is not homeomorphic to a torus. We also use
        the following visual representation of a torus:
        \begin{figure}[H]
            \centering
            \captionsetup{type=figure}
            \subimport{../../../../tikz/}
                      {Plane_Representation_of_Torus}
            \caption[Plane Representation of a Torus]
                    {The unit square with a particular
                     equivalence relation on it can be
                     used to represent a torus.}
            \label{fig:surgery_theory_plane_%
                   representation_of_a_torus}
        \end{figure}
        \begin{theorem}
            There exist manifolds $X$ and $Y$ such that
            $\dim(X)=\dim(Y)$, ${X}\simeq{Y}$,
            yet $X$ and $Y$ are not homeomorphic.
        \end{theorem}
        \begin{proof}
            For let
            $X=S^{2}\setminus\{(0,0,1),(0,1,0),(1,0,0)\}$,
            and let
            $Y=T^{2}\setminus\{(1,0,0)\}$.
            That is, $X$ is a sphere with three points removed,
            and $Y$ is a torus with one point removed. Then
            $\dim(X)=\dim(Y)=2$.
            Moreover, $X\simeq Y$. For $X$ is homeomorphic
            to the plane with $2$ points removed. This is
            homotopy equivalent to a figure $8$.
            Using the square representation of a torus in
            Fig.~\ref{%
                fig:surgery_theory_plane_%
                representation_of_a_torus%
            }
            we see that the torus with a point removed
            is also homotopy equivalent to a figure $8$.
            But homotopy equivalence is an equivalence
            relation, and thus $X\simeq Y$. But the a sphere
            is not homeomorphic to a torus, and similarly a
            sphere with $3$ points removed is not homeomorphic
            to a torus with $1$ point removed.
        \end{proof}
        \begin{figure}[H]
                \centering
                \captionsetup{type=figure}
                \subimport{../../../../tikz/}{Figure_8_HE}
                \caption{Equivalency of $S^{2}\setminus\{a,b,c\}$,
                         $T^{2}\setminus\{\alpha\}$, and a figure-8.}
                \label{fig:surgery_theory_homotopy_equivalence_%
                       sphere_with_3_holes_torus_with_1_hole}
        \end{figure} 
        Fig.~\ref{%
            fig:surgery_theory_homotopy_equivalence_%
            sphere_with_3_holes_torus_with_1_hole%
        }
        shows how both $S^{2}$ with three points
        removed and $T^{2}$ with one point removed are
        homotopy equivalent. Recall that
        $\mathbb{R}^{2}\setminus \{(0,0)\}$ is homotopy
        equivalent to $S^{1}$. In a similar manner,
        the plane with two points removed is homotopy
        equivalent to two circles whose intersection
        contains a single points (That is, a figure-$8$).
        While the ``Proof,'' given was hand wavy,
        the fact that the sphere is not homeomorphic to the
        torus comes from the fact that these two objects
        have different boundary components, something
        preserved by homeomorphism. Intiutively,
        one can think of removing a great circle
        (Or a ``line'') from the sphere.
        Removing such an object creates two disconnected
        components. However, removing a circle from the
        torus still leaves one connected surface.
        The next question is
        ``What about compact manifolds without boundary?''
        \begin{theorem}[The Generalized Poincare-Conjecture]
            If $X$ is an $n$ dimensional manifold that
            is homotopy equivalent to $S^{n+1}$, then $X$
            is homeomorphic to $S^{n+1}$.
        \end{theorem}
        \begin{definition}
            A rigid manifold is a manifold $X$ such that
            for all homotopy equivalent closed manifolds $Y$,
            $X$ is homeomorphic to $Y$.
        \end{definition}
        The question then becomes
        ``Which manifolds are rigid, and which are not?''
        From the Poincare theorem, $S^{n}$ is rigid for all
        $n\in\mathbb{N}$. The first example of a non-rigid
        manifold came in the 1930's from Franz, Reidemeister,
        and de Rham, and is called a Lens Space.
        Let $p$ and $q$ be coprime positive integers.
        Divide $S^{3}$ into $p$ equal parts, and then divide
        this into its northern and southern hemispheres.
        Take a piece of the northern hemisphere and move
        it over $q$ pieces, and then glue this to the
        southern hemisphere. Take the piece that is already
        there and move it over $q$ pieces, and then glue
        that to the northern hemisphere. Repeat this
        process until all slices are done. The is called
        the Lens Space $L(p,q)$. $L(1,1)$ is simply the
        sphere. $L(2,1)$ is the real projective plane
        $\mathbb{RP}^{2}$.
        See Fig.~\ref{%
            fig:surgery_theory_lens_space_drawing%
        }
        to see how this construction occurs.
        It can be shown that for distinct pairs
        $(p,q)$, $(p',q')$, that $L(p,q)$ is homotopy
        equivalent to $L(p',q')$, but not homeomorphic.
        \begin{figure}[H]
            \centering
            \captionsetup{type=figure}
            \subimport{../../../../tikz/}{Lens_Space}
            \caption{How to construct $L(p,q)$.}
            \label{fig:surgery_theory_lens_space_drawing}
        \end{figure}
        We move on to the structure set of topological spaces,
        in particular closed topological
        manifolds $\mathcal{M}$.
        \begin{definition}
                Equivalent homotopies are homotopy equivalences
                $f_{1}:X_{1}\rightarrow Y$,
                $f_{2}:X_{2}\rightarrow Y$,
                denoted $f_{1}\sim{f_{2}}$, such that there
                exists a continuous function
                $g:X_{1}\rightarrow{X_{2}}$ and
                $f_{2}\circ{g}\simeq{f_{1}}$.
            \end{definition}
        The equivalent classes of $Y$ is called the
        structure set of $Y$,.
        denoted $S(Y)$This set contains maps
        like $f_{1}$, $f_{2}$.
        If $g$ is a homeomorphism, then $f_{1}=f_{2}$.
        \begin{example}
                $S(S^{n})=\{S^{n}\}$
            \end{example}
        If $|S(Y)|>1$, then $Y$ is non-rigid.
        \begin{example}
                $|S(L(p,q))|\ne{1}$
            \end{example}
        A few questions naturally arise from the
        definition of the structure set:
        \begin{enumerate}
                \begin{multicols}{2}
                    \item Is $S(Y)$ a group?
                    \begin{itemize}
                        \item Sometimes.
                    \end{itemize}
                    \item Is $S(Y)$ finite?
                    \begin{itemize}
                        \item Sometimes.
                        \begin{itemize}
                            \item $|S(S^{n})| = 1$
                            \item $|S(T^{n})| = 2^{n}$
                        \end{itemize}
                    \end{itemize}
                    \item Can $S(Y)$ be infinite?
                    \begin{itemize}
                        \item Yes.
                        \begin{itemize}
                            \item $|S(\mathbb{RP}^{5})|$ - Finite.
                            \item $|S(\mathbb{RP}^{6})|$ - Finite.
                            \item $|S(\mathbb{RP}^{7})|$ -
                                  \underline{Infinite}.
                            \item $|S(\mathbb{RP}^{8})|$ - Finite.
                        \end{itemize}
                    \end{itemize}
                \end{multicols}
            \end{enumerate}
        A review of some concepts from algebraic topology.
        \begin{definition}
                A path in a topological space $X$ is a
                continuous function $f:I\rightarrow X$
            \end{definition}
        \begin{definition}
                A loop in a topological space $X$ is a
                path $f$ such that $f(0)=f(1)$.
            \end{definition}
        \begin{definition}
            The fundamental group of a topological space
            $X$ is the set
            $\pi_{1}(X)=\{f\in{C(I,X)}:f(0)=f(1)\}/h$,
            where $h$ is the modulo of homotopy,
            equipped with the concatenation operation:
            \begin{equation*}
                (f*g)(t)=
                \begin{cases}
                    f(2t),&0\leq{t}<\frac{1}{2}\\
                    g(2t-1),&\frac{1}{2}\leq{t}<1
                \end{cases}
            \end{equation*}
        \end{definition}
        \begin{theorem}
            If $X$ and $Y$ are homeomorphic topological
            spaces, then $\pi_{1}(X)$ is isomorphic
            to $\pi_{1}(Y)$.
        \end{theorem}
        \begin{proof}
            If $X$ and $Y$ are homemorphic, then there is
            a continuous bijection
            $f:X\rightarrow{Y}$ such that $f^{-1}$ is
            continuous.
            Let $\phi:\pi_{1}(X)\rightarrow\pi_{1}(Y)$
            be the map defined by the image
            $\phi(x(t))=(f\circ{x})(t)$. As $f$ is
            continuous, $\phi(x(t))\in\pi_{1}(Y)$.
            But if $x_{1},x_{2}\in\pi_{1}(X)$, then
            $\phi(x_{1}(t)*x_{2}(t))%
             =\phi(x_{1}(t))*\phi(x_{2}(t))$. Thus
            $\phi$ is a homomorphism. But as
            $f$ is a bijection, so is $\phi$, and
            therefore $\phi$ is an isomorphism.
            Thus, $\pi_{1}(X)$ and
            $\pi_{1}(Y)$ are isomorphic.
        \end{proof}
        \begin{theorem}
            If $X$ and $Y$ are topological spaces,
            and if $\pi_{1}(X)$ and $\pi_{1}(Y)$
            are not isomorphic, then
            $X$ and $Y$ are not homeomorphic.
        \end{theorem}
        Using this theorem we can tell whether or not
        certain spaces are homeomorphic. That is,
        the fundamental group is a
        \textit{topological invariant}.
        \begin{example}
                \
                \begin{enumerate}
                    \begin{multicols}{2}
                        \item $\pi_{1}(S^{n})=\{e\}$ - No Torsion.
                        \item $\pi_{1}(T^{n})=\mathbb{Z}^{n}$ -
                              No Torsion.
                        \item $\pi_{1}(\mathbb{RP}^{n})%
                               =\mathbb{Z}_{2}$ - Torsion.
                        \item $\pi_{1}(L(p,q))=\mathbb{Z}_{p}$
                              - Torsion.
                    \end{multicols}
                \end{enumerate}
            \end{example}
        \begin{definition}
                The order of an element $g$ of a group $G$
                is $O(g)=\inf\{n\in\mathbb{N}:a=a^{n}\}$.
            \end{definition}
        \begin{definition}
                A torsion group is a group $G$ such that
                there exists $g\in G$ such that $1<O(g)<\infty$.
            \end{definition}
        We now arrive at the first ``Surgery Theory''
        based theorem.
        \begin{theorem}
            If $n\geq 5$, $n\equiv{3}\mod{4}$,
            and $\pi_{1}(X)$ is a torsion group,
            then $|S(X^{n})|=\infty$.
        \end{theorem}
        Some other gems:
        $S(\mathbb{C}\mathbb{P}^{n})=\mathbb{Z}_{2}$.
        Chern Manifolds are a thing.
        \subsubsection{The Unsolvable Word Problem}
            \begin{definition}
                A presentation of a group $G$ is a set
                $H\subset{G}$ of generators and a set $R$
                of relations on $H$.
                This is denoted $G=\langle{H}|S\rangle$.
            \end{definition}
            \begin{example}
                \
                \begin{enumerate}
                    \item $\langle{a}|a^{n}=e\rangle$
                          is a the cyclic group of order $n$
                          generated by $a$.
                    \item $\langle{g},h|hg=gh\rangle%
                           =\mathbb{Z}^{2}$
                    \item $\langle{g},h|g^{2}=e,h^{2}=e\rangle%
                           =\mathbb{Z}_{2}*\mathbb{Z}_{n}$
                    \item $\langle{g},h|f^{2}=e,h^{2}=e,%
                           gh=h^{-1}g\rangle=D_{2n}$
                \end{enumerate}
            \end{example}
            \begin{remark}
                The word problem on unsolvability:
                Given two group presentations,
                there is no algorithm to show that
                they are isomorphic.
            \end{remark}
            \begin{definition}
                A finitely presented group is a group with
                a presentation $\langle{H}|R\rangle$ such that $H$
                and $R$ are finite.
            \end{definition}
            \begin{theorem}
                If $n\geq 5$ and $G$ is finitely presented,
                then there is a closed $n$ dimensional manifold
                $\mathcal{M}$ such that $\pi_{1}(\mathcal{M})=G$.
            \end{theorem}
        \subsubsection{%
            Exact Sequences and Surgery Exact Sequences
        }
            \begin{definition}
                An exact sequence
                $\cdots G_{3}%
                 \overset{f_{3}}{\rightarrow}G_{2}%
                 \overset{f_{2}}{\rightarrow}G_{1}%
                 \overset{f_{1}}{\rightarrow}G_{0}$
                is a sequence $f_{n}$ of homomorphisms
                and a sequence $G_{n}$ of groups such that
                $\Ima(f_{n+1})=\ker(f_{n})$
            \end{definition}
            \begin{remark}
                Note, the definition requires that the $f_{n}$
                are \textit{homomorphisms}, not homeomorphisms.
                Homeomorphism is a topological notion,
                not an algebraic one.
            \end{remark}
            \begin{example}
                $O\overset{f}{\rightarrow}G%
                 \overset{g}{\rightarrow}H$.
                $\Ima(f)=0\Rightarrow\ker(g)=0$.
                So $g$ is injective.
            \end{example}
            \begin{example}
                $G\overset{f}{\rightarrow}%
                 H\overset{g}{\rightarrow}O$,
                $\ker(g)=H\Rightarrow\Ima(f)=H$.
                So $f$ is surjective.
            \end{example}
            \begin{definition}
                A short exact sequence is an exact sequence
                $0\overset{f}{\rightarrow}%
                 G\overset{g}{\rightarrow}%
                 H\overset{h}{\rightarrow}%
                 L\overset{\ell}{\rightarrow}0$
            \end{definition}
            We have, from the previous examples,
            that in a short exact sequence $f$ must be
            injective and $g$ must be surjective. We now
            move onto surgery exact sequences (See Wall et. al).
            Let $n\geq 5$, and $\mathcal{M}$ be a closed
            manifold of dimension $n$. Let
            $\pi=\pi_{1}(\mathcal{M})$.
            Let Cat have the following meaning:
            \begin{itemize}
                \item Top: Category of continuous maps.
                      That is, the topological catagory.
                \item PL: Piece-Wise linear category.
                      Maps are piece-wise linear.
                \item Diff: Differentiable category.
                      Maps are diffeomorphisms.
            \end{itemize}
            \begin{example}
                \
                \begin{enumerate}
                    \begin{multicols}{2}
                        \item $S^{Top}(S^{n})=\{S^{n}\}$
                        \item $S^{PL}(S^{n})=\{S^{n}\}$
                        \item $|S^{Diff}(S^{2})|=28$ (Milnor)
                        \item $S^{PL}(T^{n})=\{S^{n}\}$ - Rigid
                        \item $|S^{PL}(T^{n})|=2^{n}$ - Non-Rigid.
                        \item $S^{Diff}(T^{n})$ - Difficult.
                    \end{multicols}
                \end{enumerate}
            \end{example}
            A surgery exact sequence is a sequence of the form:
            \begin{align*}
                S^{Cat}(M\times S')\rightarrow[M\times S',G/Cat]
                &\rightarrow L_{n+1}(\pi_{1}(\mathcal{M}))
                \rightarrow{S^{Cat}}(\mathcal{M})
                \rightarrow\cdots\\
                \cdots
                &\rightarrow{[M,G/Cat]}
                \rightarrow{L_{n}}(\pi_{1}(\mathcal{M}))
            \end{align*}
            Here, $L_{n}(X)$ is a \textit{Wall Group},
            and $[A,B]$ is a type of classifiying space.
    \subsection{Lecture 2: Surgery Structure Sets}
        \begin{wrapfigure}[8]{r}{0.18\textwidth}
            \centering
            \captionsetup{type=figure}
            \vspace{-4ex}
            \subimport{../../../../tikz/}
                      {Simple_Commutative_Diagram}
            \caption{Example of a Commutative Diagram.}
            \label{fig:wellesley_surgery_theory_commutative_%
                   diagram_for_g_for_two_homotopy_equivalences}
        \end{wrapfigure}
        Let $X$, $M_{1}$, and $M_{2}$ be closed,
        compact $n$ dimensional manifolds without boundary.
        Two homotopy equivalences $f_{i}:M_{i}\rightarrow X$
        are called equivalent if there exists a cobordism
        $(W;M_{1},M_{2})$ and a map
        $(F;f_{1},f_{2}):(W;M_{1},M_{2})%
         \rightarrow(X\times[0,1];X\times\{0\},X\times\{1\})$
        such that $F,f_{1},f_{2}$ are homotopy equivalences.
        The structure set $S(X)$ is the set of equivalence
        classes of homotopy equivalences $f:M\rightarrow X$
        from closed manifolds of dimension $n$ to $X$.
        \hfill
        \begin{definition}
            The surgery structure set of a closed
            (without boundary) compact manifold $M$ is
            $S(M)=\{f:N^{n}\rightarrow{M^{n}}|f%
             \textrm{ is a Homotopy Equivalence}\}$
        \end{definition}
        \begin{definition}
            The base point of a surgery structure
            set is the map $id_{X}:X\rightarrow X$.
        \end{definition}
        Let $N_{1}$ and $N_{2}$ be two manifold structures.
        And let $f_{1}:N_{1}^{n}\rightarrow M^{n}$ and
        $f_{2}:N_{2}^{n}\rightarrow M^{n}$ be two homotopy
        equivalences. We call $g:N_{1}\rightarrow N_{2}$ a
        cat-homeomorphism if $g$, together with $f_{1}$ and
        $f_{2}$, form the commutative diagram in Fig.~%
        \ref{%
            fig:wellesley_surgery_theory_commutative_%
            diagram_for_g_for_two_homotopy_equivalences%
        }.
        That is, $g$ is a cat-homeomorphism if it
        homotopy commutes.
        \begin{remark}
            Cat means category. There are three types:
            Top, PL, and Diff. 
            \begin{itemize}
                \begin{multicols}{3}
                    \item Top: Topological
                    \item PL: Piece-wise Linear
                    \item Diff: Diffeomorphism
                \end{multicols}
            \end{itemize}
        \end{remark}
        \begin{example}
            Some examples of surgery structure sets:
            \begin{enumerate}
                \begin{multicols}{3}
                    \item $S^{Top}(S^{n})=\{S^{n}\}$
                    \item $S^{PL}(S^{n})=\{S^{n}\}$
                    \item $S^{Diff}(S^{7})=\mathbb{Z}_{28}$
                \end{multicols}
            \end{enumerate}
        \end{example}
        \subsubsection{Orientable and Non-Orientable}
            Stiefel-Whitney classes $w_{1},\hdots, w_{n}$ are
            cohomological classes.
            Orientiable means that $w_{1}=0$.
            \begin{example}
                \
                \begin{enumerate}
                    \begin{multicols}{2}
                    \item $\mathbb{RP}^{2}$ - Non-Orientable
                    \item $\mathbb{RP}^{4}$ - Non-Orientable
                    \item $\mathbb{RP}^{6}$ - Non-Orientable
                    \item $\mathbb{RP}^{8}$ - Non-Orientable
                    \item $\mathbb{RP}^{3}$ - Orientable
                    \item $\mathbb{RP}^{5}$ - Orientable
                    \item $\mathbb{RP}^{7}$ - Orientable
                    \item $\mathbb{CP}^{n}$ - Orientable for all
                          $n\in\mathbb{N}$
                    \end{multicols}
                \end{enumerate}
            \end{example}
            Returning to surgery exact sequences, the goal
            is to compute $S^{Cat}(\mathcal{M}^{n})$, where $n$
            is the dimension of $\mathcal{M}^{n}$. The notion
            of a surgery helps solve this question. Let
            $X=\mathbb{S}^{2}\setminus%
             \{(a_{1},b_{1},c_{1}),(a_{2},b_{2},c_{2})\}$.
            That is, the sphere with two points removed.
            Stretch these two points out to create a sphere
            with two holes removed. One could imagine taking
            a hollow cylinder and stretching it to connect
            the two holes in the sphere. The result is a
            spherical coffee cup, see
            Fig.~\ref{%
                fig:surgery_theory_example_of_a_surgery%
            }.
            This can be continuously deformed into a torus.
            \begin{figure}[H]
                \centering
                \captionsetup{type=figure}
                \subimport{../../../../tikz/}
                          {Simple_Surgery_on_Sphere}
                \caption{Simple Surgery Example.}
                \label{fig:surgery_theory_example_of_a_surgery}
            \end{figure}
            Recall that $S^{0}$ is two points, and that
            $D^{2}$ is the open unit disc. Then $S^{0}\times D^{2}$
            is simply two disjoint open unit discs. This is a good
            representation of the idea of the disjoint union,
            denoted $X\coprod Y$. We have:
            \begin{equation*}
                S^{0}\times{D^{2}}=D^{2}\coprod{D^{2}}
            \end{equation*}
            We can also represent a cylinder as the closed
            $S^{1}\times \overline{D}^{1}$. The codimension
            of a surgery is the dimension of the object minus
            the dimension of a surgery. So, for the surgery
            in Fig.~\ref{fig:surgery_theory_example_of_a_surgery},
            the dimension of the entire thing is $2$, the dimension
            of the surgery is $2$, so the codimension is $0$.
            This is called a Zero-Surgery. A zero-surgery takes
            out $2$ holes and connects them with a tube.
            \begin{figure}[H]
                \centering
                \captionsetup{type=figure}
                \subimport{../../../../tikz/}{Zero_Surgery}
                \caption{Example of a Zero Surgery.}
                \label{fig:surgery_theory_a_zero_surgery}
            \end{figure}
            Let $\mathcal{M}^{n}$ be an $n$ dimensional manifold.
            Embed
            $S^{k}\times D^{n-k}$ into $\mathcal{M}^{n}$.
            Let $\partial(X)$ be the boundary of $X$.
            Then we have:
            \begin{align*}
                \partial(S^{k}\times {D^{n-k}})
                &=S^{k}\times{S^{n-k-1}}
                &
                \dim(S^{k+1}\times{D^{n-k-1})}
                &=\dim(S^{k}\times{D^{n-k}})=n
            \end{align*}
            Remove $\partial(S^{k}\times{D^{n-k}})$ and
            glue on $S^{k+1}\times{D}^{n-k-1}$.
            We alse have that
            $\partial(D^{k+1}\times{S^{n-k-1}})%
             =S^{k}\times{S^{n-k-1}}$.
            Glue $\mathcal{M}^{n}\cup(D^{k+1}\times S^{n-k-1})$
            along $\partial(S^{k}\times{S^{n-k-1}})$.
            \begin{figure}[H]
                \centering
                \captionsetup{type=figure}
                \subimport{../../../../tikz/}
                          {Constructing_Manifolds_from_Surgery}
                \caption[More Complicated Surgery Example.]
                        {Gluing $D^{k+1}\times S^{n-k}$ along
                         $\partial(S^{k}\times D^{n-k})$. The new
                         manifold is
                         $\mathcal{M}^{n}\setminus(S^{k}\times%
                          D^{n-k}\coprod(D^{k+1}\times S^{n-k-1})$}
                \label{fig:surgery_theory_glueing_S_k_D_n_k_to_M}
            \end{figure}
            We now consider $k$ surgeries
            $\mathcal{M}%
             \overset{\textrm{k-surgery}}{\longrightarrow}%
             \mathcal{N}$.
            We have seen
            $S^{2}%
             \overset{\textrm{0-surgery}}{\longrightarrow}
             T^{2}$.
            Note: $\pi_{1}(S^{2})$ is trivial,
            and $\pi_{1}(T^{2})=\mathbb{Z}^{2}$.
            This happens because $n<5$. When $n\geq{5}$,
            we have the following result.
            \begin{theorem}
                If $\mathcal{M}$ is an $n$ dimensional manifold,
                $n\geq{5}$, and if $\mathcal{N}$ is the result of
                a $k$ surgery on $\mathcal{M}$, then
                $\pi_{1}(\mathcal{M})=\pi_{1}(\mathcal{N})$.
            \end{theorem}
        \subsubsection{More On Surgery Exact Sequences}
            Recall that a surgery exact sequence
            looks like the following:
            \begin{equation*}
                \underset{\textrm{Group}}
                {\underbrace{L_{n+1}(\mathbb{Z}\pi_{1}\mathcal{M})}}
                \rightarrow\cdots\rightarrow
                \underset{\textrm{Not a Group}}
                {\underbrace{S^{Cat}(\mathcal{M}^{n})}}
                \rightarrow
                \underset{\textrm{Group}}
                {\underbrace{[M,G/0]}}
                \rightarrow \underset{\textrm{Group}}
                {\underbrace{L_{n}(\mathbb{Z}\pi_{1}(\mathcal{M}))}} 
            \end{equation*}
            An exact sequence of groups is of then form
            $G_{n+1}\overset{g_{n}}{\rightarrow}%
             G_{n}\rightarrow \hdots$,
             where $\Ima(g_n)=\ker(g_{n-1})$.
             We refine our notion of a surgery exact sequence:
            \begin{equation*}
                \cdots\rightarrow
                L_{n+1}(\mathbb{Z}\pi_{1}(\mathcal{M}))
                \dashrightarrow{S^{Cat}}(\mathcal{M})
                \overset{g}{\rightarrow}[M,G/o]
                \overset{\sigma}{\rightarrow}
                L_{n}(\mathbb{Z}\pi_{1}(\mathcal{M}))
            \end{equation*}
            The dotted line means
            $L_{n+1}(\mathbb{Z}\pi_{1}(\mathcal{M}))$
            acts on $S^{Cat}(\mathcal{M})$.
            Exact means $\Ima(g)=\ker(\sigma)$.
            Each element $f\in{[M,G/o]}$
            either pulls back to $\emptyset$ or
            something non-empty. If the pullback is non-empty,
            you get a blob in
            $S^{Cat}(\mathcal{M})$: $f^{-1}(\{x\})$.
            But:
            \begin{equation*}
                \underset{f\in[M,G/o]}{\cup}g^{-1}(\{f\})
                =S^{Cat}(\mathcal{M})
            \end{equation*}
            This process creates a partition of
            $S^{Cat}(\mathcal{M})$. Now,
            $L_{n+1}(\mathbb{Z}(\pi_{1}\mathcal{M}))$
            acts on $S^{Cat}(\mathcal{M})$ in some fashion.
            Partition the space into orbits. Exactness
            here means that partitioning by point inverses
            is the same as partitioning by orbits. That is,
            the two partitions are identical. See
            Fig.~\ref{fig:surgery_theory_partition_of_S_Cat}
            for a partioning into orbits.
            \begin{figure}[H]
                \centering
                \captionsetup{type=figure}
                \subimport{../../../../tikz/}
                          {Partitioning_of_Surgery_Structure_Set}
                \caption{Partition of $S^{Cat}(\mathcal{M})$.}
                \label{fig:surgery_theory_partition_of_S_Cat}
            \end{figure}
            The next object to talk about is
            $L_{n}(\mathbb{Z}\pi_{1}(\mathcal{M}))$.
            These are called Wall groups.
            They are difficult to compute,
            but there are some facts that are known about them:
            \begin{itemize}
                \item Wall groups only have 2-torsion.
                \begin{itemize}
                    \item 2-torsion means that elements
                          of finite order have order $2$.
                    \item This implies the groups are Abelian.
                \end{itemize}
                \item They can be orientable or not.
                \begin{itemize}
                    \item $L_{n}(%
                           \mathbb{Z}\pi_{1}(\mathcal{M})^{\pm})$
                           indicates orientable or not.
                \end{itemize}
            \end{itemize}
    \subsection{Lecture 3: Vector Bundles}
        \subsubsection{Group Rings}
            \begin{definition}
                If $G$ is a group and $R$ is a ring, then the
                group ring $RG$ is the collection of all
                finite linear combinations (Formal Sums):
                $r_{1}g_{1}+\hdots+r_{n}g_{n}$, where
                $r_{k}\in{R}$ and $g_{k}\in{G}$.
            \end{definition}
            \begin{example}
                If $G$ is a group, and
                $\mathbb{Z}G=\{\sum_{k=0}^{n}n_{k}g_{k}:%
                 n_{k}\in\mathbb{Z},g_{k}\in{G}\}$, then
                $\mathbb{Z}G$ is a group ring. This is a
                special group ring, denoted
                $\textrm{SP}_{\mathbb{Z}}(G)$.
            \end{example}
            \begin{theorem}
                If $R$ is a ring and $G$ is a group, then
                the group ring $RG$ is a ring.
            \end{theorem}
            From the previous lecture we saw that
            $L_{n+1}(\mathbb{Z}\pi_{1}(\mathcal{M}))$ is
            a group. But from the previous theorem, we have
            that $\mathbb{Z}\pi_{1}(\mathcal{M})$ is a ring.
            So, we may thing of the $L_{n}$ as a \textit{Functor}:
            $L_{n}:\textrm{Rings}\rightarrow\textrm{Groups}$.
            To recap the notation, $S(\mathcal{M})$ is the
            Surgery Structure Set on the manifold
            $\mathcal{M}$, and
            $L_{n}(\mathbb{Z}\pi_{1}(\mathcal{M}))$ is
            a Wall Group.
        \subsubsection{Matrices and Vector Bundles}
            The next monster we need to understand in the
            Surgery Exact Sequence is the
            $[\mathcal{M},G/o]$ that keep appearing.
            First, a quick recap on some notions in
            linear algebra.
            \begin{definition}
                An orthogonal matrix is an invertible
                square matrix $A$ such that
                $A^{T}=A^{-1}$
            \end{definition}
            Let $\mathcal{O}(n)$ be the group of
            $n\times{n}$ orthogonal matrices. There
            is a simple map then from
            $\mathcal{O}(n)$ to $\mathcal{O}(n+1)$,
            $\psi_{n}:%
             \mathcal{O}(n)\rightarrow\mathcal{O}(n+1)$,
            defined by:
            \begin{equation*}
                \psi_{n}(A)=
                \left[%
                    \begin{array}{c|c}
                        A&0\\
                        \hline
                        0&1
                    \end{array}
                \right]
            \end{equation*}
            We can also define a map
            $\varphi_{nm}:%
             \mathcal{O}(n)\times\mathcal{O}(m)%
             \rightarrow\mathcal{O}(n+m)$ defined
            by:
            \begin{equation*}
                \varphi_{nm}(A,B)=
                \left[%
                    \begin{array}{c|c}
                        A&0\\
                        \hline
                        0&B
                    \end{array}
                \right]
            \end{equation*}
            This is, in general, not a bijection.
            From this we can create a sequence:
            \begin{equation*}
                \mathcal{O}(1)
                \overset{\psi_{1}}{\longrightarrow}
                \mathcal{O}(2)
                \overset{\psi_{2}}{\longrightarrow}
                \mathcal{O}(3)
                \overset{\psi_{3}}{\longrightarrow}
                \mathcal{O}(4)
                \overset{\psi_{4}}{\longrightarrow}
                \cdots
                \mathcal{O}(n)
                \overset{\psi_{n}}{\longrightarrow}
                \cdots
            \end{equation*}
            We can then define $\mathcal{O}$ as the
            \textit{Direct Limit} of this sequence:
            \begin{equation*}
                \mathcal{O}
                =\underset{n\rightarrow\infty}{\lim}
                \mathcal{O}(n)
            \end{equation*}
            Now, let $\mathcal{M}$ be a manifold.
            An $n$ dimensional vector bundle is a map
            $P:E\rightarrow\mathcal{M}$ such that,
            for each point $x\in\mathcal{M}$,
            the \textit{fiber} of $x$, the
            pre-image $p^{-1}(x)$, is homeomorphic
            to $\mathbb{R}^{n}$.
            \begin{definition}
                The fiber of a point $y$ in a set $Y$
                under the map $f:X\rightarrow{Y}$ is the
                pre-image $f^{-1}(y)\subset{X}$.
            \end{definition}
            \begin{definition}
                A real $n$ dimensional vector bundle
                on a manifold
                $\mathcal{M}$ is a manifold
                $E$ and a continuous map
                $p:E\rightarrow\mathcal{M}$
                such that, for all
                $x\in\mathcal{M}$, the
                fiber of $x$ is homeomorphic
                to $\mathbb{R}^{n}$ and there
                exists an open set $\mathcal{U}$
                such that $x\in\mathcal{U}$ and
                $p^{-1}(\mathcal{U})$ is homeomorphic
                to $\mathcal{U}\times\mathbb{R}^{n}$
            \end{definition}
            The requirement that there is an open neighborhood
            $\mathcal{U}_{x}$ for all $x$ such that
            $p^{-1}(\mathcal{U})$ is homeomoprhic to
            $\mathcal{U}_{x}\times\mathbb{R}^{n}$ is called
            \textit{local triviality}. There is another notion called
            \textit{global triviality}.
            \begin{example}
                A classic example is a cylinder
                with a disk (Or the boundary of
                a cylinder with the circle).
                Given a point $(x,y,z)$ in the
                cylinder, collapse this (Or project it)
                down onto the $xy$ plane by the map
                $p(x,y,z)=(x,y)$. This is continuous,
                and is an example of a vector bundle:
                $(D^{1},D^{1}\times\mathbb{R},p)$.
                The pre-image, or fiber, of any point
                in $D^{1}$ is a line, which is
                certainly homeomorphic to
                $\mathbb{R}$. Again, taking any point $x$
                and looking at an open ball about that
                point that is entirely contained within
                $D^{1}$, the pre-image
                $p^{-1}(B_{r}(x))$ is another cylinder, which is
                homeomorphic to $\mathbb{R}^{3}$, which is itself
                homeomorphic to
                $B_{r}(x)\times\mathbb{R}^{1}$. The fibers of
                $x$ and $\mathcal{U}$ are shown in
                Fig.~\ref{fig:Surgery_Theory_Simply_%
                          Vector_Bundle_Cylinder_to_Disk}
            \end{example}
            \begin{figure}[H]
                \centering
                \captionsetup{type=figure}
                \subimport{../../../../tikz/}
                          {Vector_Bundle_Over_Circle}
                \caption{Example of a Vector Bundle:
                         $(D^{1},D^{1}\times\mathbb{R},p)$.}
                \label{%
                    fig:Surgery_Theory_Simply_%
                    Vector_Bundle_Cylinder_to_Disk%
                }
            \end{figure}
            \begin{example}
                If $\mathcal{M}$ is a manifold,
                $E=\mathcal{M}\times\mathbb{R}^{n}$,
                and if $p:E\rightarrow\mathcal{M}$ is
                defined by
                $p(x,\mathbf{y})=x$ for all
                $(x,\mathbf{y})\in\mathcal{M}\times\mathbb{R}^{n}$,
                then $(E,\mathcal{M},p)$ is a vector bundle. This
                is called the trivial $n$ dimensional vector
                bundle of $\mathbb{R}^{n}$. The fibers
                of points $x\in\mathcal{M}$ are $\mathbb{R}^{n}$,
                which is homeomorphic to $\mathbb{R}^{n}$. Given
                any open set $\mathcal{U}$ containing $x$, the
                pre-image is $\mathcal{U}\times\mathbb{R}^{n}$.
            \end{example}
            \begin{example}
                The M\"{o}bius strip can be seen as
                a vector bundle $S^{1}\times[0,1]\rightarrow[0,1]$
                where the map $(x,t)\rightarrow{x}$ is by a
                ``twist.'' This is a non-oreientable bundle
                which is non-trivial. It has local triviality,
                but no global triviality.
            \end{example}
            \begin{figure}[H]
                \centering
                \captionsetup{type=figure}
                \subimport{../../../../tikz/}
                          {Mobius_Strip}
                \caption{M\"{o}bius Strip.}
                \label{fig:Surgery_Theory_Mobius_%
                       Strip_Vector_Bundle}
            \end{figure}
            Returning to our discussion of orthogonal matrices,
            $\mathcal{O}(n)$ is a group under matrix multiplication.
            The matrix $I_{n}$ is orthogonal, and if $A$ is orthogonal,
            then $(A^{-1})^{T}=(A^{T})^{T}=A$. But $(A^{-1})^{-1}=A$,
            and thus $A^{-1}$ is orthogonal as well. Thus we have
            an identity, associativity, and closure of inverses.
            Therefore $\mathcal{O}(n)$ is a group. This group can
            act on the set of $n$ dimensional vectors in
            $\mathbb{R}^{n}$ by the map
            $(A,\mathbf{v})\rightarrow{A\mathbf{v}}$,
            for all $A\in\mathcal{O}(n),\mathbf{v}\in\mathbb{R}^{n}$.
            Thus, we have the $\mathcal{O}(n)$ acts over the fibers
            of an $n$ dimensional real vector bundle
            $(E,\mathcal{M},p)$. $\mathcal{O}(1)$ is the identity.
            That is, the ``Do nothing,'' action on a 1 dimensional
            vector bundle. $\mathcal{O}(2)$ can perform
            \textit{reflections} and \textit{rotations} on the
            fibers. Endowed with this action, any real
            vector bundle of dimension $n$ is an example of
            a \textit{principal $\mathcal{O}(n)$ bundle}.
            If $g\in\mathcal{O}(n)$ and $v\in{E}$,
            then $p(gv)=g(p(v)$.
        \subsubsection{Principal G-Bundles}
            If $G$ is a group, and if $X$ is a
            topogolical space, then there is a
            structure/notion of a
            \textit{principal G-Bundle} on $X$.
            That is, $X$ has some bundle over it
            (The space $E$ from our previous discussion),
            and $G$ acts on the fibers of $X$. This is
            denoted $\Prin_{G}(X)$.
            \par\hfill\par
            Construction by John Milnor, Classifying space.
            No idea why I wrote this...
            \par\hfill\par
            If $G$ is a group, there is a complex (space) $BG$
            such that we may form the set:
            \begin{equation*}
                [\mathcal{M},BG]
                =\{f:\mathcal{M}\rightarrow{BG}\}/Homotopy
            \end{equation*}
            That is, the set of continuous maps from $\mathcal{M}$ to
            $BG$ modded out by homotopy. Two maps are equivalent if they
            are homotopic.
            \begin{theorem}
                There is a continuous surjective function
                $f:\Prin_{G}(\mathcal{M})\rightarrow%
                 [\mathcal{M},BG]$.
            \end{theorem}
            Elaborating more on the $BG$,
            $B$ is a \textit{functor}
            $B:\textrm{Groups}\rightarrow\textrm{Spaces}$.
            If $G$ is a finitely presented group,
            then $\pi_{1}(BG)=G$, and more over,
            for all $n\geq{2}$,
            $\pi_{n}(BG)=0$. That is, $\pi_{n}(BG)$
            is the trivial group for all $n\geq{2}$.
            $\pi_{1}(X)$ ca be seen as the
            \textit{homotopy class} of $[S^{1},X]$.
            $\pi_{n}$ the homotopy class for
            $[S^{n},X]$.
            \begin{example}
                $\pi_{1}(B\mathbb{Z})=\mathbb{Z}$.
                For all $n\geq{2}$,
                $\pi_{n}(B\mathbb{Z})=0$.
                Thus $B\mathbb{Z}$ is homotopy
                equivalent to $S^{1}$.
            \end{example}
        \subsubsection{Covering Spaces}
            \begin{definition}
                A covering space of a
                topological space $X$
                is a space
                $E$ such that there exists
                a continuous surjection
                $p:E\rightarrow{X}$ such that
                for all $x\in{X}$, there is an
                open set $\mathcal{U}$ such that
                $x\in\mathcal{U}$ such that there
                exists a set of disjoint open sets
                $E_{r}\subset{E}$ where
                $p^{-1}(\mathcal{U})=\cup_{r}E_{r}$
                and for all $r$,
                $p$ is a homeomorphism between
                $E_{r}$ and $\mathcal{U}$.
            \end{definition}
            \begin{example}
                The first example is
                $S^{1}$ and $\mathbb{R}$.
                Define the map
                $p:\mathbb{R}\rightarrow{S^{1}}$
                by $p(x)=\exp(2\pi{i}x)$. This ``wraps,''
                the real line around the circle over and over again.
                Given a point $y\in{S^{1}}$, the pre-image, or fiber,
                of $y$ with respect to $p$ is
                $\{x+n:n\in\mathbb{Z}\}$ for some $x\in[0,1)$.
                Given a small enough neighborhood
                around $y$, the pre-image is of the form
                $\{x+n-\varepsilon,x+n+\varepsilon:n\in\mathbb{Z}\}$,
                which is a bunch of copies of $(0,1)$, or a bunch
                of copies of the neighborhood around $y$.
            \end{example}
            \begin{figure}[H]
                \centering
                \captionsetup{type=figure}
                \subimport{../../../../tikz/}
                          {Covering_Space_Real_Line_and_Circle}
                \caption{$\mathbb{R}$ is a Universal Covering of $S^{1}$.}
                \label{fig:Surgery_Theory_Reals_Cover_Circle}
            \end{figure}
            \begin{definition}
                A universal covering space of a topological space $X$
                is a covering space $E$ of $X$ such that
                $E$ is simply connected.
            \end{definition}
            That is, if $E$ is a covering space of $X$, then we say
            that $E$ is a universal covering space if
            $\pi_{1}(E)=0$. In the previous example we saw that
            $\mathbb{R}$ is a covering space of $S^{1}$. But
            $\mathbb{R}$ is simply connected. That is,
            $\pi_{1}(\mathbb{R})=0$. Therefore $\mathbb{R}$ is a
            universal covering space of $S^{1}$. Up to homotopy equivalence,
            $B\mathbb{Z}^{n}=T^{n}$, the $n$ torus. This is because
            $\pi_{1}(B\mathbb{Z}^{n})=\mathbb{Z}^{n}$, and
            $\pi_{n}(B\mathbb{Z}^{n})=0$ for $n\geq{2}$.
            For $S^{n}$, if $n\geq{2}$ then $S^{n}$ is simply connected,
            $\pi_{1}(S^{n})=0$. But then the identity map makes
            $S^{n}$ a covering space for itself. That is,
            $id_{S^{n}}$ is a covering map. But since $S^{n}$ is
            simply connected ($n\geq{2}$), we have that
            $S^{n}$ is a universal covering of itself.
            Moreover, it can be shown that
            $S^{n}$ is a universal covering space of
            $\mathbb{R}P^{n}$ for $n\geq{2}$.
            All universal covering spaces are homotopy
            equivalent to each other.
        \subsubsection{Eilenburg-MacLane Spaces}
            \begin{definition}
                An Eilenberg-MacLane space is a topological space
                $X$ such that there exists a non-trivial group
                $G$ and an $n\in\mathbb{N}$ such that
                $\pi_{n}(X)=G$ and, for all $m\ne{n}$,
                $\pi_{m}(X)=0$.
            \end{definition}
            This $B$ functor takes a group $G$ and spits out
            an Eilenberg-MacLane space. That is,
            $\pi_{1}(BG)=G$, and $\pi_{n}(BG)=0$ for all
            $n\geq{2}$. Eilenberg-MacLane spaces are analogous to
            prime numbers in Number Theory, but for the
            study of topological spaces. These have a
            special notation:
            \begin{notation}
                An Eilenberg-MacLane space $X$ is of the type
                $K(G,n)$ if $\pi_{n}(X)=G$ and, for all
                $m\ne{n}$, $\pi_{m}(X)=0$. We write
                $X\in{K(G,n)}$.
            \end{notation}
            Every principal $G$ bundle over $\mathcal{M}$
            can be imagined as $[\mathcal{M},BG]$.
            A principal $\mathcal{O}(n)$ bundle can be
            identitifed with a map
            $f:\mathcal{M}\rightarrow{B}\mathcal{O}(n)$.
            For example, $f$ be the constant map.
            Constant maps are homotopic to each other. This
            is the easiest bundle.
    \subsection{Lecture 4: Principal G-Bundles}
        A brief discussion on complexes. A simplex
        is a generalization of the notation of a triangle.
        A triangle can be considered as the
        convex-hull of $3$ non-coplanar points.
        This is called a $2$-simplex. A $0$-simplex
        is thus a point, and a $1$-simplex is a line.
        This can be generalized to higher dimensions.
        A $3$-simplex is a tetrahedron,
        and an $n$-simplex is an $n$ dimensional triangle,
        defined on $n+1$ non-hyper-coplanar points.
        \begin{figure}[H]
            \centering
            \captionsetup{type=figure}
            \subimport{../../../../tikz/}{Simplices}
            \caption{Examples of Simplices.}
            \label{fig:surgery_theory_simplexes}
        \end{figure}
        A simplicial complex is a set of simplices
        $\mathcal{H}$ such that the face of any element
        of $\mathcal{H}$ is also contained in $\mathcal{H}$,
        and the intersection of two simplices
        $\sigma_{1},\sigma_{2}\in \mathcal{K}$ is a
        face of both $\sigma_{1}$ and $\sigma_{2}$.
        We return to studying surgery exact sequences
        for $n\geq 5$. Let $\mathcal{M}$ be an $n$
        dimensional manifold, and let $G = \pi_{1}(\mathcal{M})$.
        In our surgery exact sequence we still have this
        mysterious object $[\mathcal{M},G/Cat]$. Let Cat
        be either PL or Top. The generalized Poincare
        Conjecture says that, for $n\geq 5$,
        $S^{PL}(S^{n})=S^{Top}(S^{n})=\{S^{n}\}$.
        Let $\mathcal{M}=S^{n}$. Then
        $G=\pi_{1}(\mathcal{M})=\{e\}$.
        Then we have the following:
        \begin{figure}[H]
            \centering
            \captionsetup{type=figure}
            \subimport{../../../../tikz/}{Surgery_Exact_Sequence}
            \caption{Diagram for the Surgery
                     Exact Sequence of $S^{5}$.}
            \label{fig:surgery_theory_example_diagram_%
                   for_surgery_exact_sequence}
        \end{figure}
        So, $\pi_{5}(G/Cat)=\{e\}$. This gives us:
        \begin{align*}
            \cdots\rightarrow
            S^{Cat}(S^{6})\rightarrow
            [S^{6},G/Cat]
            &\rightarrow
            L_{6}(\mathbb{Z})\rightarrow\cdots\\
            \cdots
            &\rightarrow{0}\rightarrow\pi_{6}(G/Cat)\rightarrow
            \mathbb{Z}_{2}\rightarrow{0}
        \end{align*}
        So, we have $\pi_{6}(G/Cat)\cong\mathbb{Z}_{2}$.
        In general, $\pi_{n}(G/o)\cong L_{n}(\mathbb{Z})$.
        \begin{theorem}[Wall's Theorem]
            \begin{equation*}
                L_{n}(\mathbb{Z})=
                \begin{cases}
                    \mathbb{Z},&n\equiv{0}\mod{4}\\
                    0,&n\equiv{1}\mod{4}\\
                    \mathbb{Z}_{2},&n\equiv{2}\mod{4}\\
                    0,&n\equiv{3}\mod{4}
                \end{cases}
            \end{equation*}
        \end{theorem}
        All $L$ groups are periodic, and never have odd
        torsion. That is, there is never
        $\mathbb{Z}_{3},\mathbb{Z}_{5}$, etc. Wall groups
        are hard to compute. Whatever $G/Cat$ is, its
        homotopy groups for $n\geq 5$ are known.
        \subsubsection{Principle G-Bundles}
            A few things are needed:
            \begin{itemize}
                \item Map $p:E\rightarrow X$,
                      where $E$ is a total space
                      and $X$ is a base space.
                \item The inverse-image $E_{x}=p^{-1}(\{x\})$
                      is called the fiber over $x$.
                \item $G$ (Group) acts on each $E_{x}$
                      freely and transitively.
                \item $G$ has to act `continuously.'
                      Nearby points are taken to nearby points.
            \end{itemize}
            Then $p:E\rightarrow X$ is a $G-$principle bundle.
            \begin{remark}
                Freely means the only element that
                fixes everything is the identity.
            \end{remark}
            \begin{example}
                Take a sphere $S^{n}$ and a projection
                $p:S^{n} \rightarrow \mathbb{RP}^{n}$.
                $\mathbb{RP}^{n}$ is created by glueing
                antipotal points together.
                If $x\in \mathbb{RP}^{n}$, then $p^{-1}(\{x\})$
                consists of $2$ antipotal points in $S^{n}$.
                Now $\mathbb{Z}_{2}=\{0,1\}$
                can act on a sphere.
                $0$ maps $x\mapsto{x}$ and $1$ maps
                $x\mapsto{-x}$.
                Note that $1+1 = 0$, as in $\mathbb{Z}_{2}$.
                Given any point, you can get to another point
                in the fiber. This is trivial in this example
                as there are only two points in the fiber.
                Also only the identity maps a point back
                to itself. This action is free and transitive,
                so $p:S^{n}\rightarrow\mathbb{RP}^{n}$
                is a $\mathbb{Z}_{2}$-principle bundle.
            \end{example}
            \begin{remark}
                Let $M$ be a manifold (Or a space)
                with dimension $n$ and fundamental group $G$.
                A universal cover $\tilde{M}$ of $M$ includes
                a map $p:\tilde{M}\rightarrow M$ such that
                $\tilde{M}$ is simply connected of dimension
                $n$, i.e. $\pi_{1}(\tilde{M})=e$,
                and $\forall_{x\in M}$,
                $p^{-1}(x)$ is a collection of discrete points.
            \end{remark}
            \begin{remark}
                $\pi_{1}(\mathbb{RP}^{n})=\mathbb{Z}_{2}$
                and $S^{n}$ is a universal cover of
                $\mathbb{RP}^{n}$. This might come
                from a general theory.
            \end{remark}
            \begin{example}
                Take the circle $S^{1}$.
                $\pi_{1}(S^{1})=\mathbb{Z}$.
                There's a map $p(x)=e^{2\pi{ix}}$ of modulus 1.
                Note that $p^{-1}(0) = \mathbb{Z}$.
                So $p^{-1}(x)$ is just a shift of
                $\mathbb{Z}$ to $\mathbb{Z}+r$.
                Note that $\pi_{1}(\mathbb{R})=e$.
                So $\mathbb{R}$ is a universal cover of $S^{1}$.
            \end{example}
            \begin{example}
                We may think $\mathbb{R}^2$ is a universal
                cover of $S^{2}$, but $S^{2}$ is already
                simply connected. So $p$ is the
                identity map, and the universal cover of $S^{2}$
                is $S^{2}$. All universal covers are
                homotopy equivalent.
            \end{example}
            \begin{remark}
                Let $x\in\mathcal{M}$. Then, for all
                $z\in p^{-1}(\{x\})$, and for all
                $g\in \pi_{1}(M)$, there is an action
                $gx\in p^{-1}(x)$. This uses the homotopy
                lifting property.
            \end{remark}
                There is an action $G$ on $\tilde{\mathcal{M}}$
                which preserves the fiber
                (Takes every element of a fiber to the
                same fiber. It does not mix fibers),
                is transitive, and is free.
                The map
                $p:\tilde{\mathcal{M}}\rightarrow\mathcal{M}$
                is a $\pi_{1}(M)$ Principal Bundle.
            \subsubsection{Functors}
                Let $F$ be a functor
                $F:\textit{Space}\rightarrow\textit{Groups}$.
                So for all spaces $X$, we have a group $F(X)$.
                There are many such examples:
                \begin{itemize}
                    \begin{multicols}{4}
                        \item Cohomology
                        \item Homology
                        \item K-Theory
                        \item Other Stuff
                    \end{multicols}
                \end{itemize}
                \begin{remark}
                    Homology: Take $M$ and triangulate.
                    Take maps from the simplicial complex
                    of $M$ to $G$ (Group) (Certain conditions).
                    There's an equivalence relation on
                    these maps. That set after taking the
                    equivalence relations is the homology:
                    $H_{n}(M,G)$. $n$ describes the type
                    of simplicies. If $n>\dim(M)$,
                    then $H_{n}(G,M)=0$.
                    $H_{n}(M,G)=\{f:\Delta^{n}\rightarrow G\}$.
                \end{remark}
                \begin{remark}
                    Cohomology is the set
                    $H^{n}(M,G)=[H_{n}(M,G),G]$,
                    that is, the \textit{dual}.
                \end{remark}
                We want to talk about cohomology.
                Under special conditions there is
                something called the Brown Representation
                Theorem. Consider Cohomology $H^{n}(M,G)$,
                with coefficients in $G$. Cohomology
                is Homotopy invariant, that is if
                $M\cong{N}$, then $H^{n}(M,G)\cong H^{n}(N,G)$.
                The Brown-Representation Theorem says
                that there is a classifying space $BG$
                such that, for all spaces $M$, there is
                a one-to-one correspondence between
                $H^{n}(X,G)\leftrightarrow[X,BG]$.
                In general, if $F$ is a functor,
                then the Brown-Representation Theorem
                says that there is a classifying space
                $Y$ such that $F(X)$ has a one-to-one
                correspondence with the homotopy classes
                of maps, $[X,Y]$.
                $F(x)\leftrightarrow[X,Y]$.
                \begin{example}
                    The Eilenberg-MacLane Space $K(G,n)$
                    has the property that
                    $\forall_{j\ne{n}}$,
                    $\pi_{n}(K(G,n))=G$,
                    and $\pi_{j}(K(G,n))=0$.
                    $K(G,n)$ is the classifying
                    space for cohomology.
                \end{example}
                \begin{theorem}
                    $K(G,n)$ is the classifying
                    space for cohomology. That is,
                    up to homotopy,
                    $H^{n}(X,G)\leftrightarrow[X,K(G,n)]$.
                \end{theorem}
                Let $\textrm{Prin}_{G}(X)$ be the collection
                of $G-$principal bundles on $X$. With a
                certain equivalence relation, it turns out
                the $\textrm{Prin}_{G}(X)$ is a group. So
                $\textrm{Prin}_{G}:\textit{Spaces}\rightarrow%
                 \textit{Groups}$
                is a functor. The Brown-Representation
                Theorem implies that there is a classifying
                space $BG$
                $\textrm{Prin}_{G}\leftrightarrow[X,BG]$.
                \begin{theorem}
                    If $p:E\rightarrow X$ and $p':E'\rightarrow X$
                    are both bundles over $X$, then there exists
                    $p\oplus p':E\oplus E' \rightarrow X$
                    COME BACK TO LATER
                \end{theorem}
            \subsubsection{Grothendique Groupification of Semigroup}
                \begin{definition}
                A semi-group is a group without the
                requirement for invereses.
                \end{definition}
                \begin{example}
                    $\{0,1,2,\hdots\}$ is a semi-group
                    under addition.
                \end{example}
                Let $G$ be a semi-group. Constraint
                $G\times{G}/\sim$.
                $(a,b)\sim(c,d)$ if $a+d=b+c$.
                So, $(2,3)\sim(4,5)\sim(7,8)\sim(-1,0)\equiv -1$.
                The equivalence class of all of these things is
                called $-1$. We still have all of the positive
                integers, $(4,2)\sim(5,3) \sim(6,4) \equiv 2$.
                This process adds all of the negatives. This
                process, called Grothendique Construction on a
                Semi-group creates a group out of a semi-group.
                It is, in a way, the 'smallest' group containing
                the semi-group. The groupification of
                $\{0,1,2,3,\hdots\}$ will be $\mathbb{Z}$.
                \begin{example}
                    What are the vector bundles over a dot?
                    There is $\mathbb{R}^{0}$ (A dot),
                    $\mathbb{R}^{n},\hdots,\mathbb{R}^n,\hdots$
                    There is an operation on this set
                    $\{\mathbb{R}^{n}:n \geq 0\}$. This makes a
                    semi-group, and there is a Grothendique
                    Groupification
                    $G_{r}(\mathbb{Z}_{\geq 0},+)=\mathbb{Z}$
                \end{example}
                Suppose M is a monoid/semigroup.
                Not required to have an inverse but should
                have an identity. For example,
                $(\mathbb{Z}_{\geq 0},+)$ is a monoid.
                Has identity, but no inverse.
                Construct $M\times M=\{(a,b):a,b\in M\}$
                with the operation $(a,b)+(c,d) = (a+c,b+d)$.
                Think of $(a,b)$ as $a-b$. Note that in
                regular math $'3-1'='4-2'$, so we want $(3,1)$
                to equal $(4,2)$. We do this with the
                equivalence rlation $(a,b)\sim (c,d)$ if
                and only if $a+d=b+c$. Let $M\times M/\sim$
                be called $M_{G}$. 
                \begin{theorem}
                    $M_{G}$ is a group.
                \end{theorem}
                \begin{theorem}
                    There is an injection $i:M\rightarrow M_{G}$
                    with the following property:
                \begin{enumerate}
                    \item $i(a)\sim (a,0)\sim(a+1,1)\sim(a+2,2)\sim\hdots$
                \end{enumerate}
                \end{theorem}
            This construction is functorial, so if there are monoids $M,N$ with a semi-homomorphism $\phi:M\rightarrow N$ $(\phi(a*b)=\phi(a)*\phi(b))$ (Homomorphism for a semi-group), then there is a HM $\phi_{G}:M_{G}\rightarrow N_{G}$ so $G(Monoids,Semihomomorphism)\rightarrow (Groups,Homomorphism)$ is a functor.
            \subsubsection{Suspension}
            Let $X$ and $Y$ be disjoint topological spaces. The wedge product $X\vee Y$ is the one-point union of $X$ and $Y$. Take $X$, take $Y$, and glue one point together. 
            \begin{theorem}
            If $X$ and $Y$ are disjoint topological spaces, then $\pi_{1}(X\vee Y)=\pi_{1}(X)\oplus\pi_{1}(Y)$.
            \end{theorem}
            In the same context, the smash product of $X$ and $Y$ is $X\wedge Y=X\times Y/(X\vee Y)$. Picture $X=(0,1]$ in the $x$ axis and $Y=(0,1]$ in the $y$ axis. Then $X\times Y$ is a square in the $xy$ plane, and $X\vee Y$ is the $x$ and $y$ axes from $0$ to $1$. So $X\wedge Y$ takes all of the points on the two axes between $0$ and $1$ and smashes them down to the origin.
            \begin{example}
            $S^{1}\times [0,1]$ is the hollow cylinder, and $S^{1}\vee [0,1]$ is the boundary of the edge of the cylinder (The lid) and the line going down the cylinder parallel with the z-axis (the spine). So $X\wedge Y$ smashes down to a cone. This is then homeomorphic to $D^{2}$. 
            \end{example}
            \begin{example}
            The torus can be visualized by the diagram shown in \ref{fig:surgery_theory_plane_representation_of_a_torus}. So $T^{2}=S^{1}\times S^{1}$. Using the diagram, we can see that the smash product $S^{1}\wedge S^{1}$ is homotopy equivalent to a sphere.
            \end{example}
            \begin{definition}
            Let $X$ be a topological space. Then the suspension of $X$, denoted $\Sigma X$, is $S^{1}\wedge X$.
            \end{definition}
            So $\Sigma S^{n}=S^{n+1}$. The usefulness of smash has to do with $\pi_{k}(\Sigma X)=\pi_{k-1}(X)$. There's another thing called the Freudenthal suspension theorem.
            \subsubsection{Higher Homotopy}
            The fundamental group, which is the first homotopy group, is $\pi_{1}(X)$. Ingredients needed:
            \begin{enumerate}
                \item Topological Space $X$
                \item A basepoint $x_{0}$
            \end{enumerate}
            \begin{definition}
            The fundamental group of a topological space $X$ about a base point $x_{0}$ is the set $\pi_{1}(X)=[(S^{1},\star),(X,\star)]=Hom\big((S^{1},\star),(X,\star)\big)=\{f:S^{1}\rightarrow X:f(x)=\star\}/\textrm{Homotopy}$.
            \end{definition}
            $\pi_{1}(X)$ is a group using concatenation. Higher homotopy groups:
            \begin{definition}
            $\pi_{n}(X)=[(S^{n},\star),(X,\star)]$
            \end{definition}
            It turns out that $\pi_{n}(X)$ has a certain operation, for $n\geq 2$, such that it is an Abelian group. However, $\pi_{1}(X)$ need not be an Abelian group.
            \begin{example}
            The Klein bottle is an example of a space such that $\pi_{1}(X)$ is not an Abelian group.
            \end{example}
            The question becomes 'What are the Homotopy groups of sphere?' That is, what is $\pi_{m}(S^{n})$? Recall stereographic projection from before. Take $S^{n}$ and remove the north pole (The point $(0,0,1)$). This can be projected down to $\mathbb{R}^{2}$. This can be generalized to $n$ dimensions, and in general $S^{n}\setminus\{\textrm{North Pole}\}$ is homeomorphic to $\mathbb{R}^{n}$. Now $\mathbb{R}^{n}$ has $0$ homotopy groups because it is contractible (Can be smushed down to a point). If $m<n$, then we are mapping a small 'sphere' into a big 'sphere.
            \begin{theorem}
            If $n\ne m$, then there is no continuous function $f$ such that $f:S^{n}\rightarrow S^{m}$ is surjective.
            \end{theorem}
            Now, if $m<n$, then we map $S^{m}$ into $S^{n}$. But since there is no surjective continuous function we can remove a point from $S^{n}$, map it down to $\mathbb{R}^{n}$ and then contract. So, $\pi_{m}(S^{n})=0$ for all $m<n$. The next case is when $m=n$. There are three obvious maps: The constant map, the identity map, and the antipotal map. It can be shown that there are, for all $n$, countably many maps. So $\pi_{n}(S^{n})=\mathbb{Z}$. Another neat little fun fact is that $\pi_{3}(S^{2})=\mathbb{Z}$. (Related to Hopf fibration). Now, the suspension theorem says that $\pi_{n+1}(\Sigma X)=\pi_{n}(X)$. So $\mathbb{Z}=\pi_{3}(S^{2})=\pi_{4}(\Sigma S^4)=\pi_{4}(S^{3})=\pi_{5}(\Sigma S^{3})=\pi_{5}(S^{4})=\hdots$ so, if $m-n=1$, then $\pi_{m}(S^{n})=\mathbb{Z}$.
            \begin{theorem}
            $\pi_{3}(S^{2})=\mathbb{Z}$
            \end{theorem}
            \begin{theorem}
            If $m-n=1$, and $n\geq 2$, then $\pi_{m}(S^{n})=\mathbb{Z}$
            \end{theorem}
            These are examples of stability theorems, or stability results.
            \subsubsection{Fibrations}
            \begin{definition}
            A fibration is a map between topological spaces that has the homotopy lifting property for every space $X$.
            \end{definition}
            A fibration gives rise to a long exect sequence of homotopy groups
            \begin{align*}
                \pi_{3}(S^{1})\rightarrow\pi_{3}(S^{3})\rightarrow\pi_{3}(S^{2})\rightarrow\pi_{2}(S^{1})&\rightarrow\pi_{2}(S^{3})\rightarrow\pi_{2}(S^{1})\rightarrow\hdots\\
                &\hdots\rightarrow\pi_{2}(S^{3})\rightarrow\pi_{2}(S^{2})\rightarrow\pi_{1}(S^{1})\rightarrow\pi_{1}(S^{3})\rightarrow\pi_{1}(S^{2})
            \end{align*}
            We need to know that $\pi_{n}(S^{1})=0$ if $n\geq 2$. An element of $\pi_{n}(S^{1})$ is $f:S^{n}\rightarrow S^{1}$. Stanley owe's me an explanation.
            This becomes:
            \begin{align*}
                0\rightarrow\mathbb{Z}\rightarrow A\rightarrow 0\rightarrow 0\rightarrow\mathbb{Z}\rightarrow\mathbb{Z}\rightarrow 0\rightarrow 0
            \end{align*}
            \subsubsection{Classifying Space}
            If $G$ is a group (discrete or not), then there is a classifying space (Topological space) $BG$ (unique up to homotopy) such that :
            \begin{enumerate}
                \item $\pi_{1}(BG)=G$ and $\pi_{n}(BG)=0$ for all $n\geq 2$.
                \item There is a contractible space $EG$ that is a principle $G$ bundle with a $G$ action such that $BG\simeq EG/G$. $EG\rightarrow BG$.
                \item For all spaces $X$ with a continuous map $f:X\rightarrow BG$, there is a pullback diagram
                \item The correspondence $(X\rightarrow BG)\rightarrow (Y_{f}\downarrow X)$ has the property that, if $f\simeq g$, then $(Y_{f}\downarrow X)\overset{\textrm{Principle G-Bundle}}{=}(Y_{g}\downarrow X)$. This gives a map $[X,BG]\rightarrow Prin_{G}(X)$ which is a bijection.
            \end{enumerate}
            \begin{definition}
            Let $V$ be a finite dimensional vector space over $\mathbb{F}$. We say that $B:V\times V\rightarrow\mathbb{F}$ is symmetric bilinear if:
            \begin{enumerate}
                \item $B(v,v')=B(v',v)$
                \item $B(v+w,v')=B(v,v')+B(w,v')$
                \item $B(\lambda v,w)=\lambda B(v,w)$
            \end{enumerate}
            \end{definition}
            \begin{example}
            Let $A$ be a symmetric real $n\times n$ matrix. Then $B:\mathbb{R}^{n}\rightarrow\mathbb{R}^{n}\rightarrow \mathbb{R}$ given by $B(X,Y) = X^{T}AY$. This $B$ gives rise to a map $Q:V\rightarrow \mathbb{F}$ given by $Q(x)=B(x,x)$.
            \begin{equation*}
                Q(X)=
                \begin{bmatrix}
                    x,y
                \end{bmatrix}
                \begin{bmatrix}
                    2&0\\
                    0&1
                \end{bmatrix}
                \begin{bmatrix}
                    x&y
                \end{bmatrix}=2x^{2}+y^{2}
            \end{equation*}
            Which is a quadratic form.
            \end{example}
    \subsection{Lecture 5: The Wall L-Groups}
        The Wall L-Groups are defined on all commutative rings.
        In fact, there is a functor $L_{n}$ which takes
        commutative rings to groups. Some facts about this:
        \begin{enumerate}
            \item L-Groups are $4$ periodic.
                  For all commutative rings $R$, we have
                  $L_{n}(R)\simeq L_{n+4}(R)$.
                  This is hard to prove.
            \item Surgery theory requires for
                  $R=\mathbb{Z}G$, where $G$ is the
                  fundamental group of the manifold in
                  question, and $\mathbb{Z}G$ is all
                  finite linear combinations of the elements
                  in $G$.
            \item There is a whole theory of computing
                  $L_{n}(R)$ when $R$ is a field,
                  usually denoted $\mathbb{F}$.
            \item Potentially true statement: L-groups
                  only have 2 or 4 torsion, if they have
                  torsion at all. This is hard to prove,
                  as well.
            \item For $G$ equal to the trivial group,
                  $L_{n}(\mathbb{Z}[e])$ we have
                  $L_{n}(\mathbb{Z}[e])%
                   =\begin{cases}%
                        \mathbb{Z},&n\cong{0}(4)\\%
                        0,&n\cong{1}(4)\\%
                        \mathbb{Z}_{2},&n\cong{2}(4)\\%
                        0,&n\cong{3}(4)%
                    \end{cases}$
        \end{enumerate}
        Suppose $\mathcal{M}^{n}$ is a closed
        manifold and $n\geq 5$, and $\pi_{1}(M)=e$.
        Suppose $n=5$. Then:
        \begin{align*}
            L_{6}(\mathbb{Z}[e])
            &\longrightarrow[\mathcal{M},G/Cat]
            \longrightarrow{S^{Cat}}(\mathcal{M})
            \longrightarrow{L_{5}}(\mathbb{Z}[e]\\
            \mathbb{Z}_{2}&\longrightarrow
            [\mathcal{M},G/Cat]
            \overset{f}{\longrightarrow}S^{Cat}(\mathcal{M})
            \longrightarrow{0}
        \end{align*}
        If $n=6$, we have:
        \begin{align*}
            L_{7}(\mathbb{Z}[e])
            &\rightarrow[M,G/Cat]\rightarrow
            S^{cat}(\mathcal{M})\rightarrow L_{6}(\mathbb{Z}[e])\\
            0&\rightarrow [M,G/Cat]
            \overset{g}{\rightarrow}S^{Cat}(\mathcal{M})
            \overset{?}{\rightarrow}\mathbb{Z}_{2}
        \end{align*}
        In the case of $n=4$, there are these
        things called 'Good' groups in which some
        of these results still hold. The
        dimensions can be broken up like this:
        \begin{itemize}
            \item $2$ Completely solved.
            \item $3$ This is Knot Theory.
            \item $4$ Very hard.
            \item $\geq 5$ Surgery Theory.
        \end{itemize}
        How do you classify manifolds?
        \begin{itemize}
            \item In two dimensions the genus
                  (number of wholes) and the
                  orientation (Euler characteristic)
                  gives you everything.
            \item In three dimensions, Thurnston and
                  Perelman did the classification of this.
            \item Four is a big vacuum of unsolved
                  stuff. 'Good' groups come up here.
            \item For every group $G$ there is a
                  manifold $\mathcal{M}$ of dimensions
                  $5$ or greater such that
                  $\pi_{1}(\mathcal{M})=G$
        \end{itemize}
        Let's study $L_{n}(\mathbb{Z}[e])$.
        This is surprisingly hard enough to study.
        The computation of this was known by Brouder,
        but the use of the surgery exact sequences
        wasn't done until Wall (Hence, Wall groups).
        \subsubsection{The Witt Group}
            $L_{0}(\mathbb{Z}[e])$ is equal to something
            called the Witt group $Witt(\mathbb{Z})$.
            First let's talk about the Witt group of fields.
            The Witt group of a field $\mathbb{F}$ is
            the set of symmetric bilinear forms
            $B:V\times{V}\rightarrow\mathbb{F}$
            of finite dimensional vector spaces
            $V$ over $\mathbb{F}$, modulo some
            equivalence relation. So $B$ can be
            represented by some symmetric matrix
            in $M_{n}(\mathbb{F})$ with respect to
            some basis $\{\beta\}$. So if
            $B:V\times{V}\rightarrow\mathbb{F}$
            has matrix $A_{\beta}$ and if
            $D:V\times{V}\rightarrow\mathbb{F}$
            has matrix $A'_{\delta}$,
            then construct the matrix:
            \begin{equation*}
                \tilde{A}=
                \begin{pmatrix}
                    A&0\\
                    0&A'
                \end{pmatrix}_{\beta,\delta}
            \end{equation*}
            Then one can get a Bilinear form
            $B\perp{D}:V\oplus{V}\rightarrow{V}\oplus{V}$
            using this matrix. This is called the
            orthogonal sum. The equivalence relation
            on these Bilinear forms is a bit complicated.
            Consider the matrix:
            \begin{equation*}
                \begin{bmatrix}
                    x&y
                \end{bmatrix}
                \begin{bmatrix}
                    1&0\\
                    0&-1
                \end{bmatrix}
                \begin{bmatrix}
                    x&y
                \end{bmatrix}
                =x^{2}-y^{2}
            \end{equation*}
            This is called a Hyperbolic form $H_{2}(\mathbb{F})$
            \begin{enumerate}
                \item Two forms 
                      $B_{1}:V\times V\rightarrow \mathbb{F}$
                      and $B_{2}:W\times W\rightarrow \mathbb{F}$,
                      with $\dim(V)=\dim(W)$.
                      If $A^{T}[B_{1}]A=[B_{2}]$,
                      then $B_{1}\sim B_{2}$.
                \item We can also write
                      $B_{1}\sim B_{2}$ is
                      $B_{1}%
                       =B_{2}\underset{k}{\perp}H_{2}(\mathbb{F})$.
                      So $H_{2}(\mathbb{F})$ is the $0$
                      element in $Witt(\mathbb{F})$.
                      Note that, since $H_{2}(\mathbb{F})$
                      has dimension $2$,
                      $\dim(B_{1})=\dim(B_{2})\mod{2}$.
                \item What is the inverse of $B_{1}$?
                      It is a form $B_{2}$ for which
                      $B_{1}\perp B_{2}\simeq H_{2}(\mathbb{F})$
            \end{enumerate}
            There is a map, called the signature map,
            of a matrix
            $W(\mathbb{F})\rightarrow%
             L_{0}(\mathbb{Z}[e])\simeq \mathbb{Z}$.
            It is defined for matrices with real eigenvalues.
            It is the number of positive eigenvalues minus
            the number of negative eignvalues.
        \subsubsection{Manifold Structures}
            Let $X$ be a closed manifold. Then a homotopy
            equivalence $f:\mathcal{N}\rightarrow{X}$ is called a
            manifold structure on $X$. Two manifold structures,
            $f_{1}:\mathcal{N}_{1}\rightarrow{\mathcal{M}}$ and
            $f_{2}:\mathcal{N}_{2}\rightarrow{\mathcal{M}}$, are called
            equivalent on $S(X)$ if there is a homeomorphism
            $g:\mathcal{M}\rightarrow\mathcal{N}$ that
            Fig.~\subref{fig:Surgery_Theory_Equivalent_Manifold_Structure_Diagram}
            is a commutative diagram. Since the composition of homeomorphisms
            is a homeomorphism, if $f_{1}:\mathcal{L}\rightarrow{X}$ and
            $f_{2}:\mathcal{M}\rightarrow{X}$ are equivalent manifold structures
            on $X$, and if $f_{2}:\mathcal{M}\rightarrow{X}$
            and $f_{3}:\mathcal{N}\rightarrow{X}$ are equivalent
            manifold structures, then $f_{1}:\mathcal{L}\rightarrow{X}$
            and $f_{3}:\mathcal{N}\rightarrow{X}$ are equivalent manifold
            structures. That is, the diagram shown in
            Fig.~\subref{fig:Surgery_Theory_Equivalent_Manifold_%
                         Structure_Diagram_Equivalence_Relation}
            is a commutative diagram.
            \begin{figure}[H]
                \captionsetup{type=figure}
                \begin{subfigure}[b]{0.49\textwidth}
                    \centering
                    \captionsetup{type=figure}
                    \subimport{../../../../tikz/}
                              {Equivalent_Manifold_Structure_Diagram}
                    \subcaption{Commutative Diagram for Manifold Structures}
                    \label{fig:Surgery_Theory_Equivalent_Manifold_Structure_Diagram}
                \end{subfigure}
                \begin{subfigure}[b]{0.49\textwidth}
                    \centering
                    \captionsetup{type=figure}
                    \subimport{../../../../tikz/}
                              {Equivalent_Manifold_Structure_%
                               Diagram_Equivalence_Relation}
                    \subcaption{Equivalent Manifolds form an Equivalence Relation.}
                    \label{fig:Surgery_Theory_Equivalent_%
                           Manifold_Structure_Diagram_Equivalence_Relation}
                \end{subfigure}
                \label{Commutative Diagrams for Manifold Structures.}
                \label{fig:Commutative_Diagrams_for_Manifold_Structures}
            \end{figure}
    \subsection{Lecture 6: The Brown Representation Theorem}
        A functor $f:\textrm{Spaces}\rightarrow\textrm{Groups}$
        takes a topological space and returns a group. There are
        many examples, such as homology, cohomoloy, and K-Theory.
        Under certain circumstances there is a space
        $B_{\circ{f}}$ such that there is a one-to-one functor
        $f(\mathcal{M})\leftrightarrow[\mathcal{M},B_{\circ{f}}]$,
        where $\mathcal{M}$ is a manifold.
        \begin{example}
            Let $G$ be a group, and $X\in{K}(G,n)$ an
            Eilenberg-MacLane space. This is not usually a
            manifold, but may be a complex, for example.
            As $X\in{K}(G,n)$, we have that
            $\pi_{n}(X)=G$ and, for all $m\ne{n}$,
            $\pi_{m}(X)=0$. THen $K(G,n)$ is the
            $B_{\circ{f}}$, where the $\circ{f}$ is cohomology
            with coefficients in $G$. That is, we have
            $H^{n}(\mathcal{M};G)%
             \leftrightarrow[\mathcal{M},K(G,n)]$
            is a one-to-one mapping.
        \end{example}
        Conside a manifold $\mathcal{M}$ and the semi-group of
        vector bundles $V(\mathcal{M})$ on $\mathcal{M}$
        with $\oplus$ give by the \textit{Whitney Sum}
        (More on that later). The Grothendique construction gives
        us a group $E(\mathcal{M})$ where the elements are vector
        bundles and ``negative,'' vector bundles (Virtual bundels).
        $E$ can then be thought of as a functor from spaces to
        groups: $E:\textrm{Spaces}\rightarrow\textrm{Groups}$.
        There is some sloppiness ahead that will be clarified later.
        There os a space $BO$ such that
        $E(\mathcal{M})\leftrightarrow[\mathcal{M},BO]$
        is a one-to-one mapping. Note that the
        $B_{\circ{f}}$ are classifying spaces. $BO$ is also
        a classifying space. Let $\mathcal{O}(n)$ be the
        set of orthogonal matrices, as defined in a previous
        lecture. $n\times{n}$ matrices such that $A^{T}=A^{-1}$.
        We saw before that there is a natual mapping $\psi_{n}$
        of $\mathcal{O}(n)$ into $\mathcal{O}(n+1)$. We can
        then form the sequence:
        \begin{equation*}
            \mathcal{O}(1)
            \overset{\psi_{1}}{\longrightarrow}
            \mathcal{O}(2)
            \overset{\psi_{2}}{\longrightarrow}
            \mathcal{O}(3)
            \overset{\psi_{3}}{\longrightarrow}
            \mathcal{O}(4)
            \overset{\psi_{4}}{\longrightarrow}
            \cdots
            \mathcal{O}(n)
            \overset{\psi_{n}}{\longrightarrow}
            \cdots
        \end{equation*}
        And define $\mathcal{O}$ to be the
        \textit{direct limit} of this sequence.
        This is a subset of ``infinite'' dimensional
        matrices. Orthogonal matrices act on vector bundles,
        there are ``rotations,'' of the fibers in
        various dimensions. Let $H$ be a vector bundle over
        $\mathcal{M}$. ``Compactify,'' the fibers,
        which are homeomorphic to $\mathbb{R}^{n}$, making
        them now homeomorphic to $S^{n}$. This is, in a way,
        adding a point ``at infinity,'' or performing the
        one point compactification of $\mathbb{R}^{n}$.
        An example is the stereographic projection of the
        sphere onto the plane. The compactification of
        this is adding the ``North Pole,'' which was
        previously ignored as it is projected
        ``to infinity.'' The stereographic projection
        gives a bijection
        $f:S^{n}\setminus\{\textrm{North Pole}\}%
         \rightarrow\mathbb{R}^{n}$.
        We now have the mapping
        $f:\mathbb{R}^{n}\cup\{\infty\}%
         \rightarrow(S^{n}\setminus\{\textrm{North Pole}\})%
         \cup\{\textrm{North Pole}\}=S^{n}$.
        How to we make $\mathbb{R}^{n}\cup\{\infty\}$ into
        a topological space? If
        $\mathcal{U}$ is open in $\mathbb{R}^{n}$, we say
        that it is still open. Moreover, we say that
        $[-\infty,a)=(-\infty,a)\cup\{\infty\}$ and
        $(a,\infty]=(a,\infty)\cup\{\infty\}$ are also
        open sets. The topology on $\mathbb{R}^{n}$ is
        then the topology generated by the three types
        of sets. Compactifying the fibers of a vector bundle
        creates something called a sphere bundle. An
        example is shown in
        Fig.~\ref{fig:Surgery_Theory_%
                  Compactification_of_Vector_Bundle}.
        \begin{figure}
            \centering
            \captionsetup{type=figure}
            \subimport{../../../../tikz/}
                      {Compactification_of_Vector_Bundle}
            \caption{Turning a Vector Bundle into a Sphere Bundle.}
            \label{fig:Surgery_Theory_%
                   Compactification_of_Vector_Bundle}
        \end{figure}
        A spherical fibration is a space $\mathcal{M}$
        where at each point $x$, the fiber of $x$ is
        equivalent to a sphere $S^{n}$ and all
        \textit{transition maps} are homotopy equivalent,
        rather than homeomorphic. The collection of all
        spherical fibrations on a manifold $\mathcal{M}$
        is a semigroup. The Grothendique group associated
        with this is denoted $S(\mathcal{M})$. This group
        is a classifying space. So we have that a vector
        bundle on $\mathcal{M}$ gives rise to a spherical
        fibration on $\mathcal{M}$.
        \begin{equation*}
            \underset{\textrm{Vector Bundle}}{[\mathcal{M},BO]}
            \longrightarrow
            \underset{\textrm{Spherical Fibration}}{[\mathcal{M},BG]}
        \end{equation*}
        We have the following diagrams to consider:
        \begin{figure}
            \centering
            \captionsetup{type=figure}
            \subimport{../../../../tikz/}
                      {Lifting_Property_Commutative_Diagram}
            \caption{Diagrams for the Lifting Propery.}
            \label{fig:Surgery_Theory_Lifting_Property_Diagram}
        \end{figure}
        Given $\varphi$ and $f$, we can form
        $\hat{\varphi}:\mathcal{M}\rightarrow{BO}$ by taking
        the composition, $\hat{\varphi}(x)=(f\circ\varphi)(x)$.
        The lifting problem is, given the central diagram,
        can we find a continuous map
        $\hat{\varphi}:\mathcal{M}\rightarrow{BG}$ such that
        the diagram becomes commutative? The answer is not always.
        The final diagram gives rise to all ``kernels,'' like
        in exactness: $BG/BO\dashrightarrow{BO}\rightarrow{BG}$.
        We thus define $G/O\equiv{BG/BO}$. The homotopy collection
        $[\mathcal{M},G/O]$ ``computes,'' the extent of which a map
        $\psi:\mathcal{M}\rightarrow{B}$ can be ``lifted,'' to
        $\hat{\varphi}:\mathcal{M}\rightarrow{BO}$.
        Loops on $\mathbb{R}^{n}$, that is, continuous functions
        $f:S^{1}\rightarrow\mathbb{R}^{n}$, can be
        contracted to a point. To put another way, no loops
        can wrap around ``holes'' in $\mathbb{R}^{n}$.
        Therefore, $[\mathbb{R}^{n},G/O]$ is a point. For
        more examples, we have
        $[S^{1},G/O]=\pi_{1}(G/O)$, and
        $[S^{n},G/P]=\pi_{n}(G/O)$. In our discussions
        $O$ gives rise to a differentiable manifold $BO$.
        We can replace this with piece-wise linear or
        topological and obtain spaces like
        $BPL$ or $BTop$, respectively.
        From the Poincare Conjecture we have that
        $S^{Top}(S^{n})=S^{PL}(S^{n})=\{e\}$. That is,
        the trivial group. Our Surgery Exact Sequence now
        becomes:
        \begin{equation*}
            L_{n+1}(\mathbb{Z}\pi)\rightarrow
            0\rightarrow
            [\Sigma\mathcal{M},G/Top]\rightarrow
            L_{n}(\mathbb{Z}\pi)\rightarrow
            0\rightarrow
            [\mathcal{M},G/Top]\rightarrow
            L_{n-1}(\mathbb{Z}\pi)\rightarrow\cdots
        \end{equation*}
        So $L_{n+1}(\mathbb{Ze}\simeq[S^{n+1},G/Top]%
            =\pi_{n}(G/Top)$,
        therefore $L_{n+1}(G/Top)\simeq{L_{n}(\mathbb{Z}e)}$.
        \par\hfill\par
        \textbf{Learn about Suspending Space, Wall Groups,
                and Grothendique complete of semigroup}
\end{document}