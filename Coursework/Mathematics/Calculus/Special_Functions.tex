\documentclass[crop=false,class=book,oneside]{standalone}
%----------------------------Preamble-------------------------------%
%---------------------------Packages----------------------------%
\usepackage{geometry}
\geometry{b5paper, margin=1.0in}
\usepackage[T1]{fontenc}
\usepackage{graphicx, float}            % Graphics/Images.
\usepackage{natbib}                     % For bibliographies.
\bibliographystyle{agsm}                % Bibliography style.
\usepackage[french, english]{babel}     % Language typesetting.
\usepackage[dvipsnames]{xcolor}         % Color names.
\usepackage{listings, lstlinebgrd}      % Verbatim-Like Tools.
\usepackage{mathtools, esint, mathrsfs} % amsmath and integrals.
\usepackage{amsthm, amsfonts}           % Fonts and theorems.
\usepackage{tabularx}
\usepackage{tcolorbox}                  % Frames around theorems.
\usepackage{upgreek}                    % Non-Italic Greek.
\usepackage{paracol}                    % Two-column styling.
\usepackage{wrapfig}                    % Wrap text around figure.
\usepackage{fmtcount, etoolbox}         % For the \book{} command.
\usepackage[newparttoc]{titlesec}       % Formatting chapter, etc.
\usepackage{titletoc}                   % Allows \book in toc.
\usepackage[nottoc]{tocbibind}          % Bibliography in toc.
\usepackage[titles]{tocloft}            % ToC formatting.
\usepackage{multicol, enumitem}         % Multi-column/enumerate.
\usepackage{import}                     % Import external files.
\usepackage{pgfplots, tikz}             % Drawing/graphing tools.
\usetikzlibrary{
    calc,                   % Calculating right angles and more.
    angles,                 % Drawing angles within triangles.
    arrows.meta,            % Latex and Stealth arrows.
    quotes,                 % Adding labels to angles.
    positioning,            % Relative positioning of nodes.
    decorations.markings,   % Adding arrows in the middle of a line.
    patterns,
    arrows,
    shapes,
    shapes.geometric,
    cd,
    hobby,
    babel
}                                       % Libraries for tikz.
\pgfplotsset{compat=1.9}                % Version of pgfplots.
\usepackage[font=scriptsize,
            labelformat=simple,
            labelsep=colon]{subcaption} % Subfigure captions.
\usepackage[font={scriptsize},
            hypcap=true,
            labelsep=colon]{caption}    % Figure captions.
\usepackage{hyperref}                   % Allows for hyperlinks.
\hypersetup{
    colorlinks=true,
    linkcolor=blue,
    filecolor=magenta,
    urlcolor=Cerulean,
    citecolor=SkyBlue
}                           % Colors for hyperref.
\usepackage[toc,acronym,nogroupskip]{glossaries} % Glossaries and acronyms.
\usepackage[subpreambles=false]{standalone}      % Complileable sub files.

% Various font stuff from kiwi.
% Use this for Times text and Computer Modern math
%\usepackage{times}

% Quite nice
%\usepackage[charter, greekfamily=, greekuppercase=italicized]{mathdesign}
%\usepackage[utopia, greekuppercase=italicized]{mathdesign}    % Math is narrower

% Use this for Times text and math
%\usepackage{newtxtext}
%\usepackage[libertine,cmintegrals]{newtxmath}
%\usepackage{fix-cm}

%\usepackage{txfontsb}
% or
%\usepackage{mathptmx}

%\usepackage[scaled=0.92]{helvet}
%\renewcommand{\rmdefault}{ptm}

%\usepackage{mathpazo}    % add possibly `sc` and `osf` options
%\usepackage{eulervm}

%\usepackage{fourier}
%\renewcommand{\rmdefault}{ptm}
%\usepackage{mathptm}

%\usepackage{fontspec}
%\setmainfont{lmodern}

%\usepackage[varg]{txfonts}
%\usepackage{fouriernc}
%\usepackage{mathpazo}

%\usepackage{bookman}
%\usepackage[scaled]{uarial}
%\usepackage[scaled]{helvet}
%\renewcommand*\familydefault{\sfdefault}
%\usepackage[math]{anttor}

%\newcommand\fgeorgia{\fontfamily{jvn}\selectfont}
%\newcommand\ftimes{\fontfamily{ptm}\selectfont}
%\newcommand\fhelvetica{\fontfamily{phv}\selectfont}
%\newcommand\fcourier{\fontfamily{pcr}\selectfont}
%\newcommand\fbookman{\fontfamily{pbk}\selectfont}
%\newcommand\fnewcentury{\fontfamily{pnc}\selectfont}
%\newcommand\fpalatino{\fontfamily{ppl}\selectfont}
%\newcommand\favantgarde{\fontfamily{pag}\selectfont}
%\newcommand\fnormal{\normalfont}
%\newcommand\fsize[1]{\ifnum#1>0\fontsize{#1}{#1}\selectfont\else\normalsize\fi}
%------------------------Theorem Styles-------------------------%
% Define theorem style for default spacing and normal font.
\newtheoremstyle{normal}
    {\topsep}               % Amount of space above the theorem.
    {\topsep}               % Amount of space below the theorem.
    {}                      % Font used for body of theorem.
    {}                      % Measure of space to indent.
    {\bfseries}             % Font of the header of the theorem.
    {}                      % Punctuation between head and body.
    {.5em}                  % Space after theorem head.
    {}

% Define theorem style for default spacing with italicized font.
\newtheoremstyle{normalit}{\topsep}{\topsep}
                {\itshape}{}{\bfseries}{}{.5em}{}

% Italic header environment.
\newtheoremstyle{thmit}{\topsep}{\topsep}{}{}{\itshape}{}{0.5em}{}

% Define italicized environments.
\theoremstyle{normalit}
\newtheorem{theorem}{Theorem}[section]
\newtheorem{lemma}{Lemma}[section]
\newtheorem{corollary}{Corollary}[section]
\newtheorem{proposition}{Proposition}[section]
\newtheorem*{theorem*}{Theorem}

% Define environments with italic headers.
\theoremstyle{thmit}
\newtheorem*{solution}{Solution}
\newtheorem*{fsolution}{Solution}

% Define default environments.
\theoremstyle{normal}
\newtheorem{example}{Example}[section]
\newtheorem{definition}{Definition}[section]
\newtheorem{problem}{Problem}[section]
\newtheorem{question}{Question}[section]
\newtheorem{remark}{Remark}[section]
\newtheorem{properties}{Properties}[section]
\newtheorem{notation}{Notation}[section]
\newtheorem{axiom}{Axiom}[section]
\newtheorem*{properties*}{Properties}
\newtheorem*{remark*}{Remark}
\newtheorem*{definition*}{Definition}
\theoremstyle{plain}

% Define framed environment.
\tcbuselibrary{most}
\newtcbtheorem[use counter*=theorem]{ftheorem}{Theorem}%
    {colback=green!5,colframe=green!35!black,
     fonttitle=\bfseries\upshape}{th}

\newtcbtheorem[use counter*=example]{fdefinition}{Definition}%
    {fonttitle=\bfseries\upshape,
     colback=blue!5!white,colframe=blue!75!black}{def}

\newtcbtheorem[use counter*=example]{fexample}{Example}%
    {fonttitle=\bfseries\upshape,
     colback=red!5!white,colframe=red!75!black}{ex}

\newtcbtheorem[use counter*=notation]{fnotation}{Notation}%
    {fonttitle=\bfseries\upshape,
     colback=SeaGreen!5!white,colframe=SeaGreen!75!black}{ex}

\newtcbtheorem[use counter*=corollary]{fcorollary}{Corollary}%
    {fonttitle=\bfseries\upshape,
     colback=Orchid!5!white,colframe=Orchid!75!black}{ex}

\newenvironment{bproof}{\textit{Proof.}}{\hfill$\square$}
\tcolorboxenvironment{bproof}{blanker,breakable,left=5mm,
                             before skip=10pt,after skip=10pt,
                             borderline west={1mm}{0pt}{red}}
\tcolorboxenvironment{fsolution}
    {enhanced jigsaw,colframe=cyan,interior hidden,breakable}

%--------------------Declared Math Operators--------------------%
\DeclareMathOperator{\Refl}{Refl}           % Reflection operator.
\DeclareMathOperator{\Span}{Span}           % Span of a set of vectors.
\DeclareMathOperator{\Card}{Card}           % Cardinality of set.
\DeclareMathOperator{\Ord}{Ord}             % Ordinal of ordered set.
\DeclareMathOperator{\Tr}{Tr}               % Trace of matrix.
\DeclareMathOperator{\adjoint}{adj}         % Adjoint of matrix.
\DeclareMathOperator{\rk}{rk}               % Rank of operator.
\DeclareMathOperator{\nul}{nul}             % Null space of operator.
\DeclareMathOperator{\sgn}{sgn}             % Sign of a number.
\DeclareMathOperator{\multideg}{mutlideg}   % Multi-Degree (Graphs).
\DeclareMathOperator{\GCD}{GCD}             % Greatest common denominator.
\DeclareMathOperator{\LM}{LM}               % Leading monomial
\DeclareMathOperator{\LC}{LC}               % Leading coefficient.
\DeclareMathOperator{\LT}{LT}               % Leading term.
\DeclareMathOperator{\LCM}{LCM}             % Least common multiple.
\DeclareMathOperator{\Mon}{Mon}             % Monomial.
\DeclareMathOperator{\Spec}{Spec}           % Spectrum.
\DeclareMathOperator{\proj}{proj}           % Projection.
\DeclareMathOperator{\comp}{comp}           % Component.
\DeclareMathOperator{\sinc}{sinc}           % Sinc function.
\DeclareMathOperator{\Ima}{Im}              % Image of operator.
\DeclareMathOperator{\Prin}{Prin}           % Principal value.
\DeclareMathOperator{\Mod}{mod}             % Modulus.
%------------------------New Commands---------------------------%
\DeclarePairedDelimiter\norm{\lVert}{\rVert}
\DeclarePairedDelimiter\ceil{\lceil}{\rceil}
\DeclarePairedDelimiter\floor{\lfloor}{\rfloor}
\newcommand*\diff{\mathop{}\!\mathrm{d}}
\newcommand*\Diff[1]{\mathop{}\!\mathrm{d^#1}}
\renewcommand{\mod}{\ \Mod}
\renewcommand*{\glstextformat}[1]{\textcolor{RoyalBlue}{#1}}
\renewcommand{\glsnamefont}[1]{\textbf{#1}}
\renewcommand\labelitemii{$\circ$}
\renewcommand\thesubfigure{\arabic{chapter}.\arabic{figure}}
\renewcommand\thesubfigure{%
    \arabic{chapter}.\arabic{figure}.\arabic{subfigure}}
\addto\captionsenglish{\renewcommand{\figurename}{Fig.}}
%------------------------Book Command---------------------------%
\makeatletter
\renewcommand\@pnumwidth{1cm}
\newcounter{book}
\renewcommand\thebook{\@Roman\c@book}
\newcommand\book{%
    \if@openright
        \cleardoublepage
    \else
        \clearpage
    \fi
    \thispagestyle{plain}%
    \if@twocolumn
        \onecolumn
        \@tempswatrue
    \else
        \@tempswafalse
    \fi
    \null\vfil
    \secdef\@book\@sbook
}
\def\@book[#1]#2{%
    \ifnum \c@secnumdepth >-3\relax
        \refstepcounter{book}%
        \addcontentsline{toc}{book}{
            \bookname\ \thebook:\hspace{1em}#1
        }
    \else
        \addcontentsline{toc}{book}{#1}%
    \fi
    \markboth{}{}%
    {\centering
     \interlinepenalty \@M
     \normalfont
     \ifnum \c@secnumdepth >-2\relax
       \huge\bfseries \bookname\nobreakspace\thebook
       \par
       \vskip 20\p@
     \fi
     \Huge \bfseries #2\par}%
    \@endbook}
\def\@sbook#1{%
    {\centering
     \interlinepenalty \@M
     \normalfont
     \Huge \bfseries #1\par}%
    \@endbook}
\def\@endbook{
    \vfil\newpage
        \if@twoside
            \if@openright
                \null
                \thispagestyle{empty}%
                \newpage
            \fi
        \fi
        \if@tempswa
            \twocolumn
        \fi
}
\newcommand*\l@book[2]{%
    \ifnum \c@tocdepth >-2\relax
        \addpenalty{-\@highpenalty}%
        \addvspace{2.25em \@plus\p@}%
        \setlength\@tempdima{3em}%
        \begingroup
            \parindent \z@ \rightskip \@pnumwidth
            \parfillskip -\@pnumwidth
            {
                \leavevmode
                \Large \bfseries #1\hfil \hb@xt@\@pnumwidth{
                    \hss #2
                }
            }
            \par
            \nobreak
            \global\@nobreaktrue
            \everypar{\global\@nobreakfalse\everypar{}}%
        \endgroup
    \fi}
\newcommand\bookname{Book}
\renewcommand{\thebook}{\texorpdfstring{\Numberstring{book}}{book}}
\providecommand*{\toclevel@book}{-2}
\makeatother
\titlecontents{chapter}[0pt]
    {\bfseries}
    {\chaptername\ \thecontentslabel:\quad}
    {}
    {\hfill\contentspage}
\titleformat{\part}[display]
    {\Large\bfseries}
    {\partname\nobreakspace\thepart}
    {0mm}
    {\Huge\bfseries}
    \titlecontents{part}[0pt]
    {\large\bfseries}
    {\partname\ \thecontentslabel: \quad}
    {}
    {\hfill\contentspage}
\newcommand{\MarkRightAngle}[4][.3cm]
    {\coordinate (tempa) at ($(#3)!#1!(#2)$);
     \coordinate (tempb) at ($(#3)!#1!(#4)$);
     \coordinate (tempc) at ($(tempa)!0.5!(tempb)$);%midpoint
     \draw (tempa) -- ($(#3)!2!(tempc)$) -- (tempb);}
%--------------------------LENGTHS------------------------------%
% Spacings for the Table of Contents.
\addtolength{\cftsecnumwidth}{1ex}
\addtolength{\cftsubsecindent}{1ex}
\addtolength{\cftsubsecnumwidth}{1ex}
\addtolength{\cftfignumwidth}{1ex}
\addtolength{\cfttabnumwidth}{1ex}

% Spacing for multi-column and enumerate environments.
\setlength{\multicolsep}{6pt}
\setlist[enumerate]{itemsep=0pt,topsep=3pt}

% Indent and paragraph spacing.
\setlength{\parindent}{0em}
\setlength{\parskip}{0em}
\graphicspath{{../../../images/}}   % Path to Image Folder.
%--------------------------Main Document----------------------------%
\begin{document}
    \ifx\ifmathcourses\undefined
        \pagenumbering{roman}
        \title{Special Functions}
        \author{Ryan Maguire}
        \date{\vspace{-5ex}}
        \maketitle
        \tableofcontents
        \setcounter{chapter}{6}
        \chapter{Special Functions}
        \pagenumbering{arabic}
    \else
        \chapter{Special Functions}
    \fi
    \section{The Error Function}
        There are algebraic functions, square roots, polynomials,
        logarithms, exponential, and trigonometric functions.
        The latter three are elementary transcendental functions,
        and special functions covers the rest. These are often
        defined by integrals, or as solutions to strange
        differential equations that arise from modelling. For
        example, the following arises when one models heat:
        \begin{align}
            \ddot{y}(x)+2x\dot{y}(x)&=0\\
            x\ddot{y}(x)+\dot{y}(x)&=0
        \end{align}
        The solutions to these are, respectively:
        \begin{align}
            y(x)&=C_{1}+C_{2}\Erf(x)\\
            y(x)&=C_{1}+C_{2}\ln(x)
        \end{align}
        The logarithmic function is usually studied in
        calculus in detail, and thus knowing this gives us a
        good idea as to what the graph of the solution looks like.
        However, much about logarithms is also glossed over.
        We may first see this when studying the power rule for
        integral powers:
        \begin{equation}
            \int{x}^{n}\diff{x}=\frac{1}{n+1}x^{n+1}+C
        \end{equation}
        And this fails for $n=-1$. However, we \textit{know}
        from Calculus that when $n=-1$ we obtain the natural
        logarithm. This is somewhat circular, since we actually
        \textit{define} logarithm as follows:
        \begin{equation}
            \ln(x)=\int_{1}^{x}t^{-1}\diff{t}
        \end{equation}
        This is somewhat cheating, as we really haven't obtained
        any knew information, we just have something to write
        when $n=-1$. Similarly, we can define $\Erf$ as:
        \begin{equation}
            \Erf(x)=\int_{0}^{x}\exp(-t^{2})\diff{t}
        \end{equation}
        In calculus, we are very happy with writing the integral
        of $x^{-1}$ is $\ln(x)$, but we are probably uncomfortable
        with writing $\Erf$ as the integral of $\exp(-x^{2})$,
        since one may very reasonably ask
        \textit{What the hell if Erf?}. But we can ask the same
        question for $ln(x)$. Thus, to know more about this function
        we can apply some of the basics from calculus and use the
        integral definition to learn the properties of $\ln$.
        We can come up with a few properties almost immediately:
        \begin{equation}
            \ln(1)=\int_{1}^{1}t^{-1}\diff{t}=0
        \end{equation}
        Using the fundamental theorem of calculus, we obtain:
        \begin{equation}
            \frac{\diff}{\diff{x}}\big(\ln(x)\big)
            =x^{-1}
        \end{equation}
        We can also study the integral, using integration by parts:
        \begin{equation}
            \int\ln(x)\diff{x}
            =x\ln(x)-\int\diff{x}
            =x\ln(x)-x+C
        \end{equation}
        It may be the case that we are unable to solve for the
        integral in terms of other elementary functions, in which
        case we may have had to define the integral of $\ln(x)$
        as another special function. Fortunately, we did not have
        to do this here. What about the most important properties
        of logarithms, such as the product rule?
        \begin{subequations}
            \begin{align}
                \ln(xy)&=\int_{1}^{xy}t^{-1}\diff{t}\\
                &=\int_{1}^{x}t^{-1}\diff{t}
                +\int_{x}^{xy}t^{-1}\diff{t}\\
                &=\ln(x)+\int_{1}^{y}(xu)^{-1}x\diff{u}\\
                &=\ln(x)+\int_{1}^{y}u^{-1}\diff{u}\\
                &=\ln(x)+\ln(y)
            \end{align}
        \end{subequations}
        This now gives us the power rule:
        \begin{equation}
            \ln(x^{2})=\ln(x\cdot{x})=\ln(x)+\ln(x)=2\ln(x)
        \end{equation}
        By induction:
        \begin{subequations}
            \begin{align}
                \ln(x^{n+1})&=\ln(x^{n}\cdot{x})\\
                &=\ln(x^{n})+\ln(x)\\
                &=n\ln(x^{n})+\ln(x)\\
                &=(n+1)\ln(x)
            \end{align}
        \end{subequations}
        By studying the definition of $\ln(x)$, we know that
        $\ln(2)>0$. We can thus determine the limiting behavior:
        \begin{equation}
            \ln(2^{n})=n\ln(2)\rightarrow{+\infty}
        \end{equation}
        The derivative is $x^{-1}$, which is always positive
        for $x>0$, and thus the function is monotonic. Moreover,
        the second derivative is $-x^{-2}$, which is always
        negative for $x>0$, and thus $\ln(x)$ is concave down.
        We can also determine the behavior as $x$ tends to
        zero from the right. We have the following:
        \begin{equation}
            \ln(1)=\ln(x\cdot{x}^{-1})=\ln(x)+\ln(x^{-1})
        \end{equation}
        But $\ln(1)=0$, and thus we have:
        \begin{equation}
            \ln(x)+\ln(x^{-1})=0
        \end{equation}
        Therefore:
        \begin{equation}
            \ln(2^{-n})=-n\ln(2)\rightarrow{-\infty}
        \end{equation}
        Monotonicity gives us that $\ln(x)\rightarrow{-\infty}$
        as $x\rightarrow{0^{+}}$. Finally, we can compute the
        Taylor series about $x=1$. Equivalently, let's compute
        the MacLaurin series for $\ln(x+1)$:
        \begin{align}
            \ln(1+x)&=\int_{1}^{1+x}t^{-1}\diff{t}\\
            &=\int_{0}^{u}\frac{1}{1+u}\diff{u}\\
            &=\int_{0}^{x}\sum_{n=0}^{\infty}(-u)^{n}\diff{u}\\
            &=\sum_{n=0}^{\infty}(-1)^{n}\int_{0}^{x}u^{n}\diff{u}\\
            &=\sum_{n=0}^{\infty}\frac{(-1)^{n}}{n+1}x^{n+1}
        \end{align}
        The radius of convergence of the integral of a power
        series does not change. The \textit{interval} may.
        Indeed, for the original series evaluating at
        $u=1$ is invalid, however for the integral, the series at
        $x=1$ is well define and evaluates to $\ln(2)$. We can
        use this series expansion for the purpose of computation
        for when $|x|<1$.
        \par\hfill\par
        We now have that the natural logarithm is well defined on
        $(0,\infty)$, the range of the function is
        $\mathbb{R}$, and that $\ln(x)$ is strictly monotonically
        increasing. Therefore there is an inverse. We call the
        inverse the \textit{exponential} function. Let's see
        if we can determine some properties of it:
        \begin{align}
            y&=\exp(x)\\
            x&=\ln(y)\\
            \Rightarrow\frac{\diff{x}}{\diff{x}}
            &=\frac{\diff}{\diff{x}}\ln(y)\\
            \Rightarrow\frac{1}{y}\frac{\diff{y}}{\diff{x}}&=1\\
            \Rightarrow\frac{\diff{y}}{\diff{x}}&=y
        \end{align}
        From this, we obtain almost all of the familiar rules for
        $\exp(x)$. Since $\ln(1)=0$, we have that $\exp(0)=1$.
        Using the derivative obtained, have have that, for all
        $n$, $y^{(n)}(0)=1$, and thus the Taylor series is:
        \begin{equation}
            \exp(x)=\sum_{n=0}^{\infty}\frac{x^{n}}{n!}
        \end{equation}
        Now let's return to the original differential equation,
        for which we wrote the solution as $C_{1}+C_{2}\Erf(x)$.
        \begin{equation}
            \ddot{y}(x)+2x\dot{y}(x)=0
        \end{equation}
        Let $u(x)=\dot{y}(x)$, so $\dot{u}(x)=\ddot{y}(x)$.
        Then we obtain:
        \begin{equation}
            \dot{u}(x)+2xu(x)=0
        \end{equation}
        This is separable, so we can solve for $u$ as follows:
        \begin{align}
            \int{u}^{-1}\diff{u}
            &=-2\int{2x}\diff{x}\\
            \Rightarrow\ln(u)&=-x^{2}+\ln(C_{1})\\
            \Rightarrow{u}&=C_{1}\exp(-x)^{2}
        \end{align}
        But $u(x)=\dot{y}(x)$, so we can solve for $y$ now:
        \begin{equation}
            y(x)=C_{1}\int\exp(-x^{2})\diff{x}+C_{2}
        \end{equation}
        What can we learn from this? Let's define the following:
        \begin{equation}
            \Erf(x)=\int_{0}^{x}\exp(-t^{2})\diff{t}
        \end{equation}
        From this, we obtain $\Erf(0)=0$, and for all
        $x$, $\Erf(-x)=-\Erf(x)$, since $\exp(-t^{2})$ is an
        even function. Looking at the derivative, we see that
        $\Erf'(x)=\exp(-x^{2})$, which is always positive, and
        thus $\Erf(x)$ is strictly monotonically increasing.
        The second derivative is $-2x\exp(-x^{2})$, which is
        negative for $x>0$ and positive for $x<0$. Thus $\Erf(x)$
        is concave up when $x<0$ and concave down when $x>0$.
        What happens as $x\rightarrow\infty$? We can use a
        familiar trick from Gauss to evaluate this. Consider
        the square of $\Erf(z)$:
        \begin{align}
            \Erf(z)^{2}
            &=\Big(\int_{0}^{z}\exp(-x^{2})\diff{x}\Big)
            \Big(\int_{0}^{z}\exp(-y^{-2})\diff{y}\Big)\\
            &=\int_{0}^{z}
                \Big(\int_{0}^{z}\exp(-x^{2})\diff{x}\Big)
                \exp(-y^{2})\diff{y}\\
            &=\int_{0}^{z}\int_{0}^{z}\exp(-(x^{2}+y^{2})
                \diff{x}\diff{y}
        \end{align}
        This is a square $A$ in the plane. From the positivity of
        $\exp(-(x^{2}+y^{2})$, the integral over $A$ is greater
        than or equal to integral over the circle that fits
        inside the square $A$, and less than or equal to the
        integral over the circle that contains the square $A$.
        Using this, we swap to polar coordinates. The radius
        of the inner circle is $z$, and the radius of the
        outer circle is $\sqrt{2}z$. So we have:
        \begin{align}
            \int_{0}^{\pi/2}\int_{0}^{z}r\exp(-r^{2})
                \diff{r}\diff{\theta}
            &\leq\int_{0}^{z}\int_{0}^{z}\exp(-(x^{2}+y^{2}))
                \diff{x}\diff{y}\\
            &\leq\int_{0}^{\pi/2}\int_{0}^{\sqrt{2}z}r\exp(-r^{2})
                \diff{r}\diff{\theta}
        \end{align}
        The left and right integrals can be evaluated, and we
        have:
        \begin{equation}
            \frac{\pi}{4}\Big[1-\exp(-z^{2})\Big]
            \leq\Erf(z)^{2}
            \leq\frac{\pi}{4}\Big[1-\exp(-2z^{2})\Big]
        \end{equation}
        From the squeeze theorem, we obtain:
        \begin{equation}
            \underset{x\rightarrow\infty}{\lim}
            \Erf(x)=\frac{\sqrt{\pi}}{2}
        \end{equation}
        We now redefine $\Erf$, and give it a name.
        \begin{definition}
            The Error Function is the function
            $\Erf:\mathbb{R}\rightarrow\mathbb{R}$ defined by:
            \begin{equation}
                \Erf(x)=\frac{2}{\sqrt{\pi}}
                    \int_{0}^{x}\exp(-t^{2})\diff{t}
            \end{equation}
        \end{definition}
        The coefficient out in front of the integral is to make
        $\Erf(x)$ asymptotic to $\pm{1}$ as $x\rightarrow\pm\infty$.
        The integral can also be evaluated in similar manner to
        how we evaluated the integral of $\ln(x)$:
        \begin{align}
            \int\Erf(x)\diff{x}
            &=x\Erf(x)-\frac{2}{\sqrt{\pi}}
                \int{x}\exp(-x^{2})\diff{x}\\
            &=x\Erf(x)+\frac{1}{\sqrt{\pi}}\exp(-x^{2}+C
        \end{align}
        Talking about the weirdness of Taylor series, look at
        $\exp(-1000)$. Taylor says that:
        \begin{equation}
            \exp(-1000)=1-1000+\frac{1000^{2}}{2!}-\hdots
        \end{equation}
        The first term is 1, second term is 1000, third term
        is 500,000, and this goes on. What the largest term?
        We see that the terms get bigger and bigger up until
        the $1000^{th}$ term. Using Sterling's approximation,
        we can see that this term is monumental in size,
        over 400 digits long. After this the terms get smaller
        and smaller, cancelling, until we have have
        something on the order of $10^{-400}$, meaning over
        800 digits cancelled. Woah. Ignoring that, we can
        continue and obtain the power series for $\Erf$:
        \begin{align}
            \Erf(x)&=\frac{2}{\sqrt{\pi}}
                \int_{0}^{x}\exp(-t^{2})\diff{t}\\
            &=\frac{2}{\sqrt{\pi}}\int_{0}^{\infty}
                \sum_{n=0}^{\infty}\frac{(-t^{2})^{n}}{n!}
                \diff{t}\\
            &=\frac{2}{\sqrt{\pi}}
                \sum_{n=0}^{\infty}\frac{(-1)^{n}}{n!}
                \int_{0}^{x}t^{2n}\diff{t}\\
            &=\frac{2}{\sqrt{\pi}}\sum_{n=0}^{\infty}
                \frac{(-1)^{n}}{n!}\frac{x^{2n+1}}{2n+1}
        \end{align}
        This confirms our suspicion that $\Erf(x)$ is an
        odd function. The alternating series test tells us how
        accurate any partial sum is, since the error in the
        approximation will be less than the magnitude of the
        last term. For example, if we want to know $\Erf(1)$,
        we have:
        \begin{equation}
            \Erf(1)\approx
                \sum_{n=0}^{N}\frac{(-1)^{n}}{n!(2n+1)}
        \end{equation}
        The error $E$ will be:
        \begin{equation}
            E<\Big|\frac{1}{(N+1)!(2N+3)}\Big|
        \end{equation}
        To obtain $\Erf(1)$ to high precision does not require
        using a substantial amount of terms. For large $x$,
        we also may use the inequality we obtained before:
        \begin{equation}
            1-\exp(-x^{2})\leq\Erf(x)\leq1-\exp(-2x^{2})
        \end{equation}
        So $\Erf(x)\approx{1}$, with error $\exp(-x^{2})$.
        More often in probability and in statistics one defines
        the \textit{complementary error function}
        $\Erfc(x)=1-\Erf(x)$. This is used when studying the
        tail of the normal distribution. We can obtain another
        series as well by using integration by parts:
        \begin{align}
            \frac{\sqrt{\pi}}{2}\Erf(x)
            &=\int_{0}^{x}\exp(-t^{2})\diff{t}\\
            &=x\exp(-x^{2})+\int_{0}^{x}2t\exp(-t^{2})\diff{t}\\
            &=\exp(-x^{2})\sum_{n=0}^{N}
                \frac{2^{n}}{(2n+1)!!}x^{2n+1}
            +\frac{2^{N+1}}{(2N+1)!!}
            \int_{0}^{x}t^{2n+2}\exp(-t^{2})\diff{t}
        \end{align}
        The double factorial sign means skip every other point.
        So $5!!=5\cdot{3}\cdot{1}$. For even numbers, we have:
        \begin{equation}
            (2n)!!=(2n)\cdot(2n-2)\cdots{2}
            =2^{n}n!
        \end{equation}
        Thus, for odd numbers, we can write:
        \begin{equation}
            (2n-1)!!=\frac{(2n)!}{(2n)!!}
            =\frac{(2n)!}{2^{n}n!}
        \end{equation}
        And thus the notation is redundant. We can also obtain a
        series expansion for $\Erfc(x)$.
        \begin{subequations}
            \begin{align}
                \Erfc(x)&=1-\Erf(x)\\
                &=1-\frac{2}{\sqrt{\pi}}
                    \int_{0}^{x}\exp(-t^{2})\diff{t}\\
                &=\frac{2}{\sqrt{\pi}}
                    \int_{0}^{\infty}\exp(-t^{2})\diff{t}
                    +\frac{2}{\sqrt{\pi}}\int_{0}^{x}
                    \exp(-t^{2})\diff{t}\\
                &=\frac{2}{\sqrt{\pi}}\int_{x}^{\infty}
                    \exp(-t^{2})\diff{t}
            \end{align}
        \end{subequations}
        Using this, we obtain a series:
        \begin{subequations}
            \begin{align}
                \frac{\sqrt{\pi}}{2}\Erfc(x)&=
                \int_{x}^{\infty}\exp(-t^{2})\diff{t}\\
                &=\frac{1}{2x}\exp(-x^{2})+
                \int_{x}^{\infty}\frac{1}{2t^{2}}\frac{1}{2t}
                \frac{\diff}{\diff{t}}
                    \Big(\exp(-t^{2})\Big)\diff{t}\\
                &=\frac{\exp(-t^{2})}{2}
                \sum_{n=0}^{N}(-1)^{n}\frac{(2n-1)!!}{(2x^{2})^{n}}
                +R_{N}
            \end{align}
            Where $R_{N}$ is the remainder term:
            \begin{equation}
                R_{N}=(-1)^{N}\frac{(2N-1)!!}{2^{N+3}}
                \int_{x}^{\infty}\frac{1}{t^{2N+2}}\exp(-t^{2})
                \diff{t}
            \end{equation}
        \end{subequations}
        $R_{N}$ diverges for any $x$. This series is useful
        for large $x$, however, since the factorial term that
        cause $R_{N}$ to diverge take a while to get large.
        Thus, $R_{N}$ decreasing for a while, and then eventually,
        once the factorial term is larger than the exponential
        terms, $R_{N}$ starts to decrease. Choosing the $N$ that
        minimizes $R_{N}$, we obtain a series that can very
        accurately approximate the nature of $\Erfc(x)$ for
        large $x$. This is similar to $\exp(-1000)$ where the
        terms got bigger and bigger, until the eventually cancel.
        We now have the opposite, where the terms get smaller
        and smaller, until $R_{N}$ starts to diverse of to
        infinity. If terms past the ``good'' $N$ are added,
        the series will fail to accurate represent $\Erfc(x)$.
        \par\hfill\par
        Make graph of $R_{N}$ vs $N$ for $\Erf$ and $\Erfc$.
        \par\hfill\par
        This type of series is an example of an asymptotic series.
        We can't write $f(x)=\sum{a}_{n}x^{n}$ since the
        series diverges, so we write
        $f(x)\sim\sum{a}_{n}(x)$ as $x\rightarrow\infty$.
        Taylor series say that the series approximation works
        well every as $n$ gets large. Asymptoptic series say
        that the approximation works well as the argument of
        $f$ gets large, rather than the number of terms.
        More precisely:
        \begin{equation}
            \underset{x\rightarrow\infty}{\lim}
            \frac{f(x)-\sum_{n=0}^{N}a_{n}(x)}{a_{N}(x)}=0
        \end{equation}
        As such, there are divergent as convergent asymptotic
        series. For example, consider the following:
        \begin{align}
            \frac{1}{1+x^{2}}
            &=\frac{1}{x^{2}}\frac{1}{1+\frac{1}{x^{2}}}\\
            &=\sum_{n=0}^{\infty}\frac{(-1)^{n}}{x^{n+2}}\\
            &=\sum_{n=0}^{\infty}(-1)^{n}x^{2n}
        \end{align}
        The third equation is an asymptotic series and the
        fourth equation is the Taylor series. The Taylor
        series has a radius of convergence of 1, and thus for
        all $x>1$, the Taylor series will fail. For
        $x>1$, however, the asymptotic expansion does a very
        good job
    \section{Lecture 2}
        We can approximate the error in the power series and
        asymptotic series expansions by looking at the largest
        and smallest terms, respectively. For the power series:
        \begin{align}
            \Big|\frac{a_{n+1}}{a_{n}}\Big|
            &=\frac{x^{2}}{n+1}\frac{2n+3}{2n+1}\\
            &\approx\frac{x^{2}}{n+1}
        \end{align}
        And this is largest when $n\sim{x}^{2}$.
        Using Stirling's Formula, the term is then:
        \begin{align}
            \frac{x^{2n+1}}{n!(2n+1)}
            &\approx\frac{x^{2n+1}}{(x^{2})!(2x^{2}+1)}\\
            &\approx\frac{x^{2x^{2}-1}}{2}
            \frac{1}{2\pi{x}^{2}(x^{2})^{(x^{2})}\exp(-x^{2})}\\
            &=\frac{1}{2}\frac{\exp(x^{2})}{x^{2}\sqrt{2\pi}}
        \end{align}
        Similarly, the lowest term in the asymptotic series
        can be obtained.
        \begin{align}
            \Big|\frac{a_{n+1}}{a_{n}}\Big|
            &=\frac{(2n+2)(2n+1)}{4(n+1)x^{2n+1}}\\
            &=\frac{2n+1}{2x^{2n+1}}\\
            &\approx\frac{n}{x^{2}}
        \end{align}
        So again, the smallest term occurs when
        $n\sim{x}^{2}$. The smallest term is approximately:
        \begin{align}
            \frac{(2x^{2})!}{2^{2x^{2}}(x^{2})!x^{2x^{2}+1}}
            &\approx
            \frac{\sqrt{2\pi{x}^{2}}(2x^{2})^{2x^{2}}\exp(-2x^{2}}
                 {2^{2x^{2}}\sqrt{2\pi{x}^{2}}(x^{2})^{x^{2}}
                  \exp(-x^{2})x^{2x^{2}+1}}\\
            &=\frac{\sqrt{2}\exp(-x^{2})}{x}
        \end{align}
        \subsection{Watson's Lemma}
            The Laplace transform of an integrable function $f$
            is defined as:
            \begin{equation}
                \mathscr{L}(f)_{s}=\int_{0}^{\infty}
                    f(t)\exp(-st)\diff{t}
            \end{equation}
            Watson's lemma says to substitute the MacLaurin
            series for $f$ and integrate term by term. For example,
            letting $f(x)=\Erfc(x)$, we have:
            \begin{align}
                \Erfc(x)&=\int_{x}^{\infty}\exp(-t^{2})\diff{t}\\
                &=\int_{0}^{\infty}\exp\big(-(s+x)^{2}\big)\diff{s}\\
                &=\exp(-x^{2})\int_{0}^{\infty}
                    \exp(-2sx)\exp(-s^{2})\diff{s}\\
                &\sim
                \exp(-x^{2})\sum_{n=0}^{\infty}\frac{(-1)^{n}}{n!}
                    \int_{0}^{\infty}s^{2n}\exp(-2sx)\diff{s}\\
                &=\exp(-x^{2})\sum_{n=0}^{\infty}
                    \frac{(-1)^{n}(2n)!}{2^{2n+1}n!x^{2n+1}}
            \end{align}
        \subsection{The Exponential Integral}
            The indefinite integral of the function
            $\exp(-x)/x$ can not be expressed in terms of
            elementary functions that one sees in a calculus course.
            Similar to how $\Erf$ was define, the exponential
            integral is defined as the \textit{solution} to this
            problem.
            \begin{equation}
                E_{1}(x)=\int_{x}^{\infty}\frac{\exp(-t)}{t}\diff{t}
            \end{equation}
            $E_{1}(x)$ is defined by an improper integral, and
            as such we must show that this is well defined for
            various values. That is, we must show the following
            limit is well defined:
            \begin{equation}
                E_{1}(x)=\underset{b\rightarrow\infty}{\lim}
                \int_{x}^{b}\frac{\exp(-t)}{t}\diff{t}
            \end{equation}
            Since $\exp(-t)/t$ is positive for all
            $t>0$, the area under the curve increases as $b$
            increases. That is, we have monotonicity. Also, for
            all $t\geq{x}$, $\exp(-t)/t\leq\exp(-t)/x$, and thus
            we obtain:
            \begin{equation}
                \int_{x}^{b}\frac{\exp(-t)}{t}\diff{t}
                \leq\int_{x}^{b}\frac{\exp(-t)}{x}\diff{t}
                =\frac{1}{x}\Big[\exp(-x)-\exp(-b)\Big]
            \end{equation}
            Taking the limit as $b\rightarrow\infty$, we see
            that the integral converges and thus $E_{1}(x)$ is
            well defined for all positive $x$. This is a combination
            of the comparison test and the fact that, if
            $f(b)$ is a monotonic bounded function, then it
            converges. Taking $f(b)$ to be the integral of
            $\exp(-t)/t$ over the interval $(x,b)$ gives us the
            result. From the integral definition, we can obtain
            many properties of $E_{1}(x)$. As $x$ increases, there
            is less area, and thus $E_{1}(x)$ is decreasing.
            We can also show this by taking the derivative, and
            observing that $-\exp(-x)/x$ is negative for all
            $x>0$. Looking at the second derivative, we get:
            \begin{equation}
                E_{1}''(x)=\exp(-x)\frac{x+1}{x^{2}}
            \end{equation}
            And this is positive for all $x>0$, and thus
            $E_{1}(x)$ is concave up. We can obtain a series for
            $E_{1}$ as follows:
            \begin{align}
                E_{1}'(x)&=-\frac{\exp(-x)}{x}\\
                &=-\frac{1}{x}\sum_{n=0}^{\infty}
                    \frac{(-1)^{n}x^{n}}{n!}\\
                \Rightarrow
                E_{1}(x)&=-\ln(x)-\sum_{n=1}^{\infty}
                    \frac{(-1)^{n}x^{n}}{nn!}+C
            \end{align}
            Where the $C$ comes out as a constant of integration.
            Evaluating $E_{1}(1)$ gives us this constant.
            We can't evaluate at zero since the log function is
            undefined there, and similarly for infinity.
            \begin{equation}
                C=\int_{1}^{\infty}\frac{\exp(-t)}{t}\diff{t}
                +\sum_{n=1}^{\infty}\frac{(-1)^{n}}{nn!}
            \end{equation}
            Using integration by parts, we have:
            \begin{equation}
                \int_{1}^{\infty}\frac{\exp(-t)}{t}\diff{t}
                =\int_{1}^{\infty}\ln(t)\exp(-t)\diff{t}
            \end{equation}
            Note that this does not say that $t^{-1}$ and
            $\ln(t)$ are the same function, but rather the
            area under $\exp(-t)t^{-1}$ and $\ln(t)\exp(-t)$
            are the same on the interval $(1,\infty)$.
            Let's look at the following function on $(0,1]$:
            \begin{equation}
                f(b)=\int_{b}^{1}\ln(x)\exp(-x)\diff{x}
            \end{equation}
            We can show this is well defined when
            $b\rightarrow{0}^{+}$ since:
            \begin{equation}
                \int_{b}^{1}\ln(t)\diff{t}
                \leq\int_{b}^{1}\ln(t)\exp(-t)\diff{t}
                \leq{0}
            \end{equation}
            And thus, we have:
            \begin{equation}
                -1+b(1-\ln(b))\leq{f}(b)\leq{0}
            \end{equation}
            Since $f$ is monotonic, the limit exists as
            $b\rightarrow{0}^{+}$. This again uses the comparison
            test for integrals. Getting back to our computation
            of $C$, we obtain a series for integrand of the function
            used in the definition of $f(b)$:
            \begin{align}
                \int_{0}^{1}\ln(t)\exp(-t)\diff{t}
                &=\int_{0}^{1}\ln(t)\sum_{n=0}^{\infty}
                    \frac{(-1)^{n}}{n!}t^{n}\diff{t}\\
                &=\sum_{n=0}^{\infty}\frac{(-1)^{n}}{n!}
                \int_{0}^{\infty}t^{n}\ln(t)\diff{t}
            \end{align}
            Examining that last integral, we get:
            \begin{equation}
                \int_{0}^{1}\ln(t)t^{n}\diff{t}
                =\int_{0}^{1}\ln(t)\frac{\diff}{\diff{t}}
                    \Big(\frac{t^{n+1}}{n+1}\Big)\diff{t}
            \end{equation}
            Applying integration by parts and L'H\"{o}pital's rule,
            we see that this simplifies to:
            \begin{equation}
                -\int_{0}^{1}\frac{1}{n+1}t^{n}\diff{t}
                =-\frac{1}{(n+1)^{2}}
            \end{equation}
            Thus, this original integral is:
            \begin{equation}
                \int_{0}^{1}\exp(-t)\ln(t)\diff{t}
                =\sum_{n=1}^{\infty}\frac{(-1)^{n}}{nn!}
            \end{equation}
            But from how $C$ was defined, we obtain:
            \begin{align}
                C&=\int_{1}^{\infty}\frac{\exp(-t)}{t}\diff{t}
                +\sum_{n=1}^{\infty}\frac{(-1)^{n}}{nn!}\\
                &=\int_{0}^{1}\ln(t)\exp(-t)\diff{t}
                +\sum_{n=1}^{\infty}\frac{(-1)^{n}}{nn!}\\
                &=\int_{0}^{\infty}\exp(-t)\ln(t)\diff{t}
            \end{align}
            This constant appears in many different areas of
            mathematics, and is given a symbol and a name.
            It's called the
            \textit{Euler-Mascheroni Constant}, and is
            denoted $\gamma$. It's also called the Euler Gamma
            constant. It's value is approximately $0.57721$. It is
            unknown whether this is an irrational number or not
            (As of January, 2019). With this we obtain our series
            for $E_{i}(x)$:
            \begin{equation}
                E_{i}(x)=-\gamma-\ln(x)
                -\sum_{n=1}^{\infty}\frac{(-1)^{n}}{nn!}x^{n}
            \end{equation}
            We can also examine the indefinite integral by using
            integration by parts:
            \begin{equation}
                \int{E}_{1}(x)\diff{x}=
                xE_{1}(x)+\exp(-x)+C
            \end{equation}
            Where $C$ is a (different) constant of integration.
            What happens as $x\rightarrow\infty$? We saw earlier
            that $E_{1}(x)$ is bounded by
            $\exp(-x)/x$, and thus the indefinite integral is well
            behaved in this limit. We could, however, continue
            using integration by parts and obtain an asymptotic
            series.
\end{document}