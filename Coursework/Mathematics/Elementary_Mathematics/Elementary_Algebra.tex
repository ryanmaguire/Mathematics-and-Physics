\documentclass[crop=false,class=book,oneside]{standalone}
%----------------------------Preamble-------------------------------%
%---------------------------Packages----------------------------%
\usepackage{geometry}
\geometry{b5paper, margin=1.0in}
\usepackage[T1]{fontenc}
\usepackage{graphicx, float}            % Graphics/Images.
\usepackage{natbib}                     % For bibliographies.
\bibliographystyle{agsm}                % Bibliography style.
\usepackage[french, english]{babel}     % Language typesetting.
\usepackage[dvipsnames]{xcolor}         % Color names.
\usepackage{listings}                   % Verbatim-Like Tools.
\usepackage{mathtools, esint, mathrsfs} % amsmath and integrals.
\usepackage{amsthm, amsfonts, amssymb}  % Fonts and theorems.
\usepackage{tcolorbox}                  % Frames around theorems.
\usepackage{upgreek}                    % Non-Italic Greek.
\usepackage{fmtcount, etoolbox}         % For the \book{} command.
\usepackage[newparttoc]{titlesec}       % Formatting chapter, etc.
\usepackage{titletoc}                   % Allows \book in toc.
\usepackage[nottoc]{tocbibind}          % Bibliography in toc.
\usepackage[titles]{tocloft}            % ToC formatting.
\usepackage{pgfplots, tikz}             % Drawing/graphing tools.
\usepackage{imakeidx}                   % Used for index.
\usetikzlibrary{
    calc,                   % Calculating right angles and more.
    angles,                 % Drawing angles within triangles.
    arrows.meta,            % Latex and Stealth arrows.
    quotes,                 % Adding labels to angles.
    positioning,            % Relative positioning of nodes.
    decorations.markings,   % Adding arrows in the middle of a line.
    patterns,
    arrows
}                                       % Libraries for tikz.
\pgfplotsset{compat=1.9}                % Version of pgfplots.
\usepackage[font=scriptsize,
            labelformat=simple,
            labelsep=colon]{subcaption} % Subfigure captions.
\usepackage[font={scriptsize},
            hypcap=true,
            labelsep=colon]{caption}    % Figure captions.
\usepackage[pdftex,
            pdfauthor={Ryan Maguire},
            pdftitle={Mathematics and Physics},
            pdfsubject={Mathematics, Physics, Science},
            pdfkeywords={Mathematics, Physics, Computer Science, Biology},
            pdfproducer={LaTeX},
            pdfcreator={pdflatex}]{hyperref}
\hypersetup{
    colorlinks=true,
    linkcolor=blue,
    filecolor=magenta,
    urlcolor=Cerulean,
    citecolor=SkyBlue
}                           % Colors for hyperref.
\usepackage[toc,acronym,nogroupskip,nopostdot]{glossaries}
\usepackage{glossary-mcols}
%------------------------Theorem Styles-------------------------%
\theoremstyle{plain}
\newtheorem{theorem}{Theorem}[section]

% Define theorem style for default spacing and normal font.
\newtheoremstyle{normal}
    {\topsep}               % Amount of space above the theorem.
    {\topsep}               % Amount of space below the theorem.
    {}                      % Font used for body of theorem.
    {}                      % Measure of space to indent.
    {\bfseries}             % Font of the header of the theorem.
    {}                      % Punctuation between head and body.
    {.5em}                  % Space after theorem head.
    {}

% Italic header environment.
\newtheoremstyle{thmit}{\topsep}{\topsep}{}{}{\itshape}{}{0.5em}{}

% Define environments with italic headers.
\theoremstyle{thmit}
\newtheorem*{solution}{Solution}

% Define default environments.
\theoremstyle{normal}
\newtheorem{example}{Example}[section]
\newtheorem{definition}{Definition}[section]
\newtheorem{problem}{Problem}[section]

% Define framed environment.
\tcbuselibrary{most}
\newtcbtheorem[use counter*=theorem]{ftheorem}{Theorem}{%
    before=\par\vspace{2ex},
    boxsep=0.5\topsep,
    after=\par\vspace{2ex},
    colback=green!5,
    colframe=green!35!black,
    fonttitle=\bfseries\upshape%
}{thm}

\newtcbtheorem[auto counter, number within=section]{faxiom}{Axiom}{%
    before=\par\vspace{2ex},
    boxsep=0.5\topsep,
    after=\par\vspace{2ex},
    colback=Apricot!5,
    colframe=Apricot!35!black,
    fonttitle=\bfseries\upshape%
}{ax}

\newtcbtheorem[use counter*=definition]{fdefinition}{Definition}{%
    before=\par\vspace{2ex},
    boxsep=0.5\topsep,
    after=\par\vspace{2ex},
    colback=blue!5!white,
    colframe=blue!75!black,
    fonttitle=\bfseries\upshape%
}{def}

\newtcbtheorem[use counter*=example]{fexample}{Example}{%
    before=\par\vspace{2ex},
    boxsep=0.5\topsep,
    after=\par\vspace{2ex},
    colback=red!5!white,
    colframe=red!75!black,
    fonttitle=\bfseries\upshape%
}{ex}

\newtcbtheorem[auto counter, number within=section]{fnotation}{Notation}{%
    before=\par\vspace{2ex},
    boxsep=0.5\topsep,
    after=\par\vspace{2ex},
    colback=SeaGreen!5!white,
    colframe=SeaGreen!75!black,
    fonttitle=\bfseries\upshape%
}{not}

\newtcbtheorem[use counter*=remark]{fremark}{Remark}{%
    fonttitle=\bfseries\upshape,
    colback=Goldenrod!5!white,
    colframe=Goldenrod!75!black}{ex}

\newenvironment{bproof}{\textit{Proof.}}{\hfill$\square$}
\tcolorboxenvironment{bproof}{%
    blanker,
    breakable,
    left=3mm,
    before skip=5pt,
    after skip=10pt,
    borderline west={0.6mm}{0pt}{green!80!black}
}

\AtEndEnvironment{lexample}{$\hfill\textcolor{red}{\blacksquare}$}
\newtcbtheorem[use counter*=example]{lexample}{Example}{%
    empty,
    title={Example~\theexample},
    boxed title style={%
        empty,
        size=minimal,
        toprule=2pt,
        top=0.5\topsep,
    },
    coltitle=red,
    fonttitle=\bfseries,
    parbox=false,
    boxsep=0pt,
    before=\par\vspace{2ex},
    left=0pt,
    right=0pt,
    top=3ex,
    bottom=1ex,
    before=\par\vspace{2ex},
    after=\par\vspace{2ex},
    breakable,
    pad at break*=0mm,
    vfill before first,
    overlay unbroken={%
        \draw[red, line width=2pt]
            ([yshift=-1.2ex]title.south-|frame.west) to
            ([yshift=-1.2ex]title.south-|frame.east);
        },
    overlay first={%
        \draw[red, line width=2pt]
            ([yshift=-1.2ex]title.south-|frame.west) to
            ([yshift=-1.2ex]title.south-|frame.east);
    },
}{ex}

\AtEndEnvironment{ldefinition}{$\hfill\textcolor{Blue}{\blacksquare}$}
\newtcbtheorem[use counter*=definition]{ldefinition}{Definition}{%
    empty,
    title={Definition~\thedefinition:~{#1}},
    boxed title style={%
        empty,
        size=minimal,
        toprule=2pt,
        top=0.5\topsep,
    },
    coltitle=Blue,
    fonttitle=\bfseries,
    parbox=false,
    boxsep=0pt,
    before=\par\vspace{2ex},
    left=0pt,
    right=0pt,
    top=3ex,
    bottom=0pt,
    before=\par\vspace{2ex},
    after=\par\vspace{1ex},
    breakable,
    pad at break*=0mm,
    vfill before first,
    overlay unbroken={%
        \draw[Blue, line width=2pt]
            ([yshift=-1.2ex]title.south-|frame.west) to
            ([yshift=-1.2ex]title.south-|frame.east);
        },
    overlay first={%
        \draw[Blue, line width=2pt]
            ([yshift=-1.2ex]title.south-|frame.west) to
            ([yshift=-1.2ex]title.south-|frame.east);
    },
}{def}

\AtEndEnvironment{ltheorem}{$\hfill\textcolor{Green}{\blacksquare}$}
\newtcbtheorem[use counter*=theorem]{ltheorem}{Theorem}{%
    empty,
    title={Theorem~\thetheorem:~{#1}},
    boxed title style={%
        empty,
        size=minimal,
        toprule=2pt,
        top=0.5\topsep,
    },
    coltitle=Green,
    fonttitle=\bfseries,
    parbox=false,
    boxsep=0pt,
    before=\par\vspace{2ex},
    left=0pt,
    right=0pt,
    top=3ex,
    bottom=-1.5ex,
    breakable,
    pad at break*=0mm,
    vfill before first,
    overlay unbroken={%
        \draw[Green, line width=2pt]
            ([yshift=-1.2ex]title.south-|frame.west) to
            ([yshift=-1.2ex]title.south-|frame.east);},
    overlay first={%
        \draw[Green, line width=2pt]
            ([yshift=-1.2ex]title.south-|frame.west) to
            ([yshift=-1.2ex]title.south-|frame.east);
    }
}{thm}

%--------------------Declared Math Operators--------------------%
\DeclareMathOperator{\adjoint}{adj}         % Adjoint.
\DeclareMathOperator{\Card}{Card}           % Cardinality.
\DeclareMathOperator{\curl}{curl}           % Curl.
\DeclareMathOperator{\diam}{diam}           % Diameter.
\DeclareMathOperator{\dist}{dist}           % Distance.
\DeclareMathOperator{\Div}{div}             % Divergence.
\DeclareMathOperator{\Erf}{Erf}             % Error Function.
\DeclareMathOperator{\Erfc}{Erfc}           % Complementary Error Function.
\DeclareMathOperator{\Ext}{Ext}             % Exterior.
\DeclareMathOperator{\GCD}{GCD}             % Greatest common denominator.
\DeclareMathOperator{\grad}{grad}           % Gradient
\DeclareMathOperator{\Ima}{Im}              % Image.
\DeclareMathOperator{\Int}{Int}             % Interior.
\DeclareMathOperator{\LC}{LC}               % Leading coefficient.
\DeclareMathOperator{\LCM}{LCM}             % Least common multiple.
\DeclareMathOperator{\LM}{LM}               % Leading monomial.
\DeclareMathOperator{\LT}{LT}               % Leading term.
\DeclareMathOperator{\Mod}{mod}             % Modulus.
\DeclareMathOperator{\Mon}{Mon}             % Monomial.
\DeclareMathOperator{\multideg}{mutlideg}   % Multi-Degree (Graphs).
\DeclareMathOperator{\nul}{nul}             % Null space of operator.
\DeclareMathOperator{\Ord}{Ord}             % Ordinal of ordered set.
\DeclareMathOperator{\Prin}{Prin}           % Principal value.
\DeclareMathOperator{\proj}{proj}           % Projection.
\DeclareMathOperator{\Refl}{Refl}           % Reflection operator.
\DeclareMathOperator{\rk}{rk}               % Rank of operator.
\DeclareMathOperator{\sgn}{sgn}             % Sign of a number.
\DeclareMathOperator{\sinc}{sinc}           % Sinc function.
\DeclareMathOperator{\Span}{Span}           % Span of a set.
\DeclareMathOperator{\Spec}{Spec}           % Spectrum.
\DeclareMathOperator{\supp}{supp}           % Support
\DeclareMathOperator{\Tr}{Tr}               % Trace of matrix.
%--------------------Declared Math Symbols--------------------%
\DeclareMathSymbol{\minus}{\mathbin}{AMSa}{"39} % Unary minus sign.
%------------------------New Commands---------------------------%
\DeclarePairedDelimiter\norm{\lVert}{\rVert}
\DeclarePairedDelimiter\ceil{\lceil}{\rceil}
\DeclarePairedDelimiter\floor{\lfloor}{\rfloor}
\newcommand*\diff{\mathop{}\!\mathrm{d}}
\newcommand*\Diff[1]{\mathop{}\!\mathrm{d^#1}}
\renewcommand*{\glstextformat}[1]{\textcolor{RoyalBlue}{#1}}
\renewcommand{\glsnamefont}[1]{\textbf{#1}}
\renewcommand\labelitemii{$\circ$}
\renewcommand\thesubfigure{%
    \arabic{chapter}.\arabic{figure}.\arabic{subfigure}}
\addto\captionsenglish{\renewcommand{\figurename}{Fig.}}
\numberwithin{equation}{section}

\renewcommand{\vector}[1]{\boldsymbol{\mathrm{#1}}}

\newcommand{\uvector}[1]{\boldsymbol{\hat{\mathrm{#1}}}}
\newcommand{\topspace}[2][]{(#2,\tau_{#1})}
\newcommand{\measurespace}[2][]{(#2,\varSigma_{#1},\mu_{#1})}
\newcommand{\measurablespace}[2][]{(#2,\varSigma_{#1})}
\newcommand{\manifold}[2][]{(#2,\tau_{#1},\mathcal{A}_{#1})}
\newcommand{\tanspace}[2]{T_{#1}{#2}}
\newcommand{\cotanspace}[2]{T_{#1}^{*}{#2}}
\newcommand{\Ckspace}[3][\mathbb{R}]{C^{#2}(#3,#1)}
\newcommand{\funcspace}[2][\mathbb{R}]{\mathcal{F}(#2,#1)}
\newcommand{\smoothvecf}[1]{\mathfrak{X}(#1)}
\newcommand{\smoothonef}[1]{\mathfrak{X}^{*}(#1)}
\newcommand{\bracket}[2]{[#1,#2]}

%------------------------Book Command---------------------------%
\makeatletter
\renewcommand\@pnumwidth{1cm}
\newcounter{book}
\renewcommand\thebook{\@Roman\c@book}
\newcommand\book{%
    \if@openright
        \cleardoublepage
    \else
        \clearpage
    \fi
    \thispagestyle{plain}%
    \if@twocolumn
        \onecolumn
        \@tempswatrue
    \else
        \@tempswafalse
    \fi
    \null\vfil
    \secdef\@book\@sbook
}
\def\@book[#1]#2{%
    \refstepcounter{book}
    \addcontentsline{toc}{book}{\bookname\ \thebook:\hspace{1em}#1}
    \markboth{}{}
    {\centering
     \interlinepenalty\@M
     \normalfont
     \huge\bfseries\bookname\nobreakspace\thebook
     \par
     \vskip 20\p@
     \Huge\bfseries#2\par}%
    \@endbook}
\def\@sbook#1{%
    {\centering
     \interlinepenalty \@M
     \normalfont
     \Huge\bfseries#1\par}%
    \@endbook}
\def\@endbook{
    \vfil\newpage
        \if@twoside
            \if@openright
                \null
                \thispagestyle{empty}%
                \newpage
            \fi
        \fi
        \if@tempswa
            \twocolumn
        \fi
}
\newcommand*\l@book[2]{%
    \ifnum\c@tocdepth >-3\relax
        \addpenalty{-\@highpenalty}%
        \addvspace{2.25em\@plus\p@}%
        \setlength\@tempdima{3em}%
        \begingroup
            \parindent\z@\rightskip\@pnumwidth
            \parfillskip -\@pnumwidth
            {
                \leavevmode
                \Large\bfseries#1\hfill\hb@xt@\@pnumwidth{\hss#2}
            }
            \par
            \nobreak
            \global\@nobreaktrue
            \everypar{\global\@nobreakfalse\everypar{}}%
        \endgroup
    \fi}
\newcommand\bookname{Book}
\renewcommand{\thebook}{\texorpdfstring{\Numberstring{book}}{book}}
\providecommand*{\toclevel@book}{-2}
\makeatother
\titleformat{\part}[display]
    {\Large\bfseries}
    {\partname\nobreakspace\thepart}
    {0mm}
    {\Huge\bfseries}
\titlecontents{part}[0pt]
    {\large\bfseries}
    {\partname\ \thecontentslabel: \quad}
    {}
    {\hfill\contentspage}
\titlecontents{chapter}[0pt]
    {\bfseries}
    {\chaptername\ \thecontentslabel:\quad}
    {}
    {\hfill\contentspage}
\newglossarystyle{longpara}{%
    \setglossarystyle{long}%
    \renewenvironment{theglossary}{%
        \begin{longtable}[l]{{p{0.25\hsize}p{0.65\hsize}}}
    }{\end{longtable}}%
    \renewcommand{\glossentry}[2]{%
        \glstarget{##1}{\glossentryname{##1}}%
        &\glossentrydesc{##1}{~##2.}
        \tabularnewline%
        \tabularnewline
    }%
}
\newglossary[not-glg]{notation}{not-gls}{not-glo}{Notation}
\newcommand*{\newnotation}[4][]{%
    \newglossaryentry{#2}{type=notation, name={\textbf{#3}, },
                          text={#4}, description={#4},#1}%
}
%--------------------------LENGTHS------------------------------%
% Spacings for the Table of Contents.
\addtolength{\cftsecnumwidth}{1ex}
\addtolength{\cftsubsecindent}{1ex}
\addtolength{\cftsubsecnumwidth}{1ex}
\addtolength{\cftfignumwidth}{1ex}
\addtolength{\cfttabnumwidth}{1ex}

% Indent and paragraph spacing.
\setlength{\parindent}{0em}
\setlength{\parskip}{0em}
\graphicspath{{../../../images/}}   % Path to Image Folder.
%----------------------------GLOSSARY-------------------------------%
\makeglossaries
\loadglsentries{../../../glossary}
\loadglsentries{../../../acronym}
%--------------------------Main Document----------------------------%
\begin{document}
    \ifx\ifmathcourses\undefined
        \pagenumbering{roman}
        \title{Elementary Algebra}
        \author{Ryan Maguire}
        \date{\vspace{-5ex}}
        \maketitle
        \tableofcontents
        \setcounter{chapter}{1}
        \chapter{Elementary Algebra}
        \pagenumbering{arabic}
    \else
        \chapter{Elementary Algebra}
    \fi
    \section{Basic Notions}
        \subsection{Sets}
            We begin by discussing sets of numbers,
            with a primary focus on the real numbers.
            \begin{fdefinition}{Sets}
                {Elementary_Algebra_Set}
                A set is a collection of objects,
                none of which is the set itself.
            \end{fdefinition}
            The requirement that a set cannot
            contain itself
            is to avoid logical paradoxes,
            such as the one discovered by the
            famous mathematician
            Bertrand Russell. The objects contained
            in a set are called the
            elements of that set. We use the
            following notation to denote
            which elements belong to a set:
            \begin{fnotation}{Set Notation}
                {Elem_Alg_Element_Notation}
                If $A$ is a set and $x$ is an element
                of $A$, we write $x\in{A}$.
                If $x$ is not an element of
                $A$, we write $x\notin{A}$.
            \end{fnotation}
            Sets are usually described by one of
            two methods. The first way is to list out
            all of the elements, separated by commas,
            and enclosing them in braces.
            \begin{fexample}{Example of Sets}
                {Elem_Alg_Ex_of_Sets}
                When a set isn't too big it is
                often easiest to describe it by
                listing out all of the elements, enclosing
                them in braces. For example:
                \par\hfill\par
                \begin{subequations}
                    \vspace{-1ex}
                    \begin{minipage}{0.49\textwidth}
                        \begin{equation}
                            A=\{1,2,3\}
                        \end{equation}
                    \end{minipage}
                    \hfill
                    \begin{minipage}{0.49\textwidth}
                        \begin{equation}
                            B=\{a,b,c\}
                        \end{equation}
                    \end{minipage}
                \end{subequations}
            \end{fexample}
            Using our set notation
            (Notation~\ref{not:Elem_Alg_Element_Notation})
            we have that $1\in{A}$,
            since 1 is an element of the set $A$,
            but $4\notin{A}$ since 4 is not contained
            in $A$. When a set has infinitely many
            elements, but the elements can be listed
            in a certain pattern, we use an ellipses
            to indicate the pattern goes on.
            For example, we can write the set of all
            \textit{natural numbers}, or counting
            numbers, as:
            \begin{equation}
                \mathbb{N}=\{1,2,3,\hdots\}
            \end{equation}
            The symbol $\mathbb{N}$ stands for
            \textit{natural}. It is very common in
            to write the set of all
            natural numbers using this symbol.
            Ellipses can be used to describe other sets:
            \par\hfill\par
            \begin{subequations}
                \vspace{-1ex}
                \begin{minipage}{0.49\textwidth}
                    \begin{equation}
                        \mathbb{N}_{e}
                        =\{2,4,6,8,\hdots\}
                    \end{equation}
                \end{minipage}
                \hfill
                \begin{minipage}{0.49\textwidth}
                    \begin{equation}
                        \mathbb{N}_{o}
                        =\{1,3,5,7,\hdots\}
                    \end{equation}
                \end{minipage}
            \end{subequations}
            \par\hfill\par
            Which denote the set of all even and odd
            natural numbers, respectively. As a reminder,
            an even integer is an integer that is
            divisible by 2, and an odd integer is an
            integer that is not divisible by 2. All even
            integers can be written as $n=2k$, where $k$
            is also some integer. Similarly, every odd
            integer can be written as $n=2k+1$, where
            $k$ is some integer. This brings us to the
            second way to describe a set:
            Set-Builder notation. We can describe
            $\mathbb{N}_{e}$ and $\mathbb{N}_{o}$ by
            writing:
            \par\hfill\par
            \begin{subequations}
                \vspace{-1ex}
                \begin{minipage}{0.49\textwidth}
                    \begin{equation}
                        \label{EQN:Elem_Alg_Even_Nat_Def}
                        \mathbb{N}_{e}
                        =\{2k:k\in\mathbb{N}\}
                    \end{equation}
                \end{minipage}
                \hfill
                \begin{minipage}{0.49\textwidth}
                    \begin{equation}
                        \label{EQN:Elem_Alg_Odd_Nat_Def}
                        \mathbb{N}_{o}
                        =\{2k+1:k\in\mathbb{N}\}
                    \end{equation}
                \end{minipage}
            \end{subequations}
            \par\hfill\par
            Eqn.~\ref{EQN:Elem_Alg_Even_Nat_Def} says that
            $\mathbb{N}_{e}$ is the set of all positive
            multiples of 2, which is precisely the
            definition of an even natural number.
            Similarly, Eqn.~\ref{EQN:Elem_Alg_Odd_Nat_Def}
            says that $\mathbb{N}_{o}$ is the set of all
            positive multiples of 2, plus 1. Again, this is
            the definition of an odd number. Set-Builder
            notation often makes it possible to describe
            very large sets in a very compact manner.
            We can even use plain English:
            \begin{equation}
                \mathbb{N}=
                \{n:n\textrm{ is a positive integer}\}
            \end{equation}
            This is the set of all $n$ such that $n$ is a
            positive integer. As another example,
            we could write
            the set of all days of the week as:
            \begin{equation}
                \{x:x\textrm{ is a day of the week}\}
                =\{\textrm{Sunday, Monday, ...,
                           Friday, Saturday}\}
            \end{equation}
            In mathematics, sets can be very
            abstract entities that
            have nothing to do with numbers. The
            elements themselves need not be related,
            nor ``Mathematical,'' in nature.
            For the purpose of elementary algebra,
            we study sets of numbers.
            \begin{example}
                \label{ex:Elem_Alg_More_Examples_of_Sets}
                Sets don't have to contain real numbers, and
                can be very abstract.
                Here are some examples of sets:
                \begin{enumerate}
                    \begin{multicols}{3}
                        \item $\{1,2,3\}$
                        \item $\{a,b,c\}$
                        \item $\{\textrm{Boston, %
                                         New York}\}$
                        \item $\{\textrm{Atlantic, %
                                 Pacific}\}$
                        \item $\{n:n%
                                 \textrm{ is an integer}\}$
                        \item $\{n^2:n%
                                 \textrm{ is an integer}\}$
                    \end{multicols}
                \end{enumerate}
            \end{example}
            Let's write out some of the elements of the
            last example in
            Example~\ref{ex:Elem_Alg_More_Examples_of_Sets}:
            \begin{equation}
                \{n^{2}:n\textrm{ is an integer}\}
                =\{1,4,9,16,25,36,\hdots\}
            \end{equation}
            Here, $n^{2}$ means the \textit{square} of
            $n$. This is defined as $n^{2}=n\cdot{n}$. This
            will be discussed in great detail later on.
            It is possible for two different
            sets to contain the
            same elements, and for two different
            sets to contain
            completely different elements.
            An important concept in the study of
            sets is that of \textit{subsets}. If
            $A$ and $B$ are sets and if every
            element of $A$ is also an element of $B$,
            we call $A$ a subset of $B$.
            \clearpage
            \begin{fdefinition}{Subsets}
                {Elementary_Algebra_Subset}
                A subset of a set $B$ is a set $A$
                such that for all $x\in{A}$,
                it is true that $x\in{B}$. If $A$ is a
                subset of a set $B$, we write
                $A\subseteq{B}$. If not we write
                $A\not\subseteq{B}$.
            \end{fdefinition}
            \begin{figure}[H]
                \centering
                \captionsetup{type=figure}
                \begin{tikzpicture}
                    \draw[fill=red, opacity=0.4]
                        (0,-2) to [out=0,in=-120] (1,-1)
                               to [out=60,in=-45] (1.5,1)
                               to [out=135,in=0] (0,2)
                               to [out=-180,in=70] (-3,0)
                               to [out=-110,in=-180] cycle;
                    \draw[fill=cyan, opacity=0.8]
                        (0,-1) to [out=0,in=-120] (1,0)
                               to [out=60,in=20] (0.5,0.5)
                               to [out=-160,in=45] (-0.5,0.5)
                               to [out=-135,in=90] (-1,0)
                               to [out=-90,in=180] cycle;
                    \node at (-0.1,-0.2) {$A$};
                    \node at (-1.5,0.5) {$B$};
                \end{tikzpicture}
                \caption{Sets can often be visualized as blobs.
                         The blob $A$ is entirely contained
                         inside the blob $B$, and thus we say
                         that $A$ is a
                         \textit{subset} of $B$. Mathematically,
                         we write this
                         as $A\subseteq{B}$.}
                \label{fig:Elem_Alg_Subsets_Example}
            \end{figure}
            If we want to be really rigorous, we can use the
            notion of subset to define what equality is.
            Two things are equal if they are
            the exact same thing. For higher mathematics this
            is too vague, and a clearer definition is needed.
            Let's use the notion of subset to give a rigorous
            definition to equality for sets:
            \begin{definition}
                Equal sets are sets $A$ and $B$ such that
                $A\subseteq{B}$ and $B\subseteq{A}$. If $A$ and $B$
                are equal, we write $A=B$.
            \end{definition}
            This says that $A$ and $B$ are equal if and only if
            they have the exact same elements. Every element of
            $A$ is and element of $B$, and every element of
            $B$ is an element of $A$. Thus, this definition gives
            us our intuitive notion of equality, and is
            rigorous enough to do higher mathematics.
            From the definition of a subset, we can see that for
            any set $A$, $A$ is a subset of itself. For any element
            of $A$ is an element of $A$ (Rather obvious), and thus
            $A$ is a subset of $A$.
            \begin{theorem}
                If $A$ is a set, then $A\subseteq{A}$.
            \end{theorem}
            \begin{proof}
                If $A$ is a set, then for all
                $x\in{A}$ it is true that $x\in{A}$.
                Thus, $A\subseteq{A}$.
            \end{proof}
            This seemingly useless theorem gives rise
            to another concept, a \textit{proper subset}.
            It will be useful to distinguished subsets
            of a set $A$ that aren't also equal to $A$.
            \begin{definition}
                A proper subset of a set $B$ is a set
                $A$ such that for all $x\in{A}$ it is
                true that $x\in{B}$, and $A\ne{B}$.
                That is, there is an element $b\in{B}$
                such that $b\notin{A}$. If $A$
                is a proper subset of $B$,
                we write $A\subset{B}$.
            \end{definition}
            By examining Fig.~\ref{fig:Elem_Alg_Subsets_Example}
            we see that blob $A$ is entirely
            contained within blob $B$, and thus
            $A\subseteq{B}$. However, there are parts of $B$
            that are \textbf{not} contained in $A$, and
            therefore $A\ne{B}$. We can summarize this by
            saying that $A$ is a \textit{proper} subset of
            $B$, and write $A\subset{B}$.
            We should be careful with notation here, since the
            two look very similar. $A\subseteq{B}$ says that
            $A$ is a subset of $B$, and it may be possible that
            $A=B$. $A\subset{B}$ says that $A$ is a subset of $B$,
            and $A\ne{B}$.
            There's a similar notation for inequalities. We
            write $a<b$ to denote $a$ is \textit{less than} $b$,
            but we can also write $a\leq{b}$ to denote
            that $a$ is \textit{less than or equal} to $b$.
            The symbols $\subset$ and $\subseteq$
            behave similarly.
            \clearpage
            \begin{fexample}{Subsets}{Elem_Alg_Subsets_Example}
                Let $A=\{1,2,3,4,5\}$ and
                $B=\{1,2,3,4,5,6,7,8\}$. Then
                $A\subseteq{B}$ for every number
                contained in $A$ is contained in also
                $B$. However, $B\not\subseteq{A}$, for
                there are elements of $B$ that are not
                in $A$. For example, $7\in{B}$ but
                $7\notin{A}$. Therefore $A$ is a
                proper subset of $B$, and we write
                $A\subset{B}$. Sets do not need to
                contain numbers, and we can think of
                abstract sets. For example:
                \begin{equation*}
                    \{\textrm{Atlantic, Pacific}\}\subset
                    \{\textrm{Atlantic, Pacific, Arctic, Indian}\}
                \end{equation*}
            \end{fexample}
            \begin{fremark}{Some Warnings About Sets}
                {Elem_Alg_Sets_Are_Not_Ordered}
                Sets do not have an order on them. That means that
                $\{1,2,3\}$ and $\{3,2,1\}$ are the exact same
                set. That is, they are \textit{equal}. Just from
                the definition, a set is completely determined by
                the elements it contains. We often write:
                \begin{equation*}
                    \mathbb{N}=\{1,2,3,\hdots\}
                \end{equation*}
                But we could equally write:
                \begin{equation*}
                    \mathbb{N}=\{2,4,1,478,18,17,9,\hdots\}
                \end{equation*}
                $\mathbb{N}$ is simply the set of all positive
                integers. There is no order on the set. Sets that
                have order are called \textit{ordered sets}.
                These will be discussed later when we deal with
                inequalities. Since sets do not have order, we
                can rearrange the elements and still have equality.
                For example:
                \begin{align*}
                    \{\textrm{Bob, Carl, George}\}&=
                    \{\textrm{Carl, George, Bob}\}\\
                    \{1,2,3,4\}&=\{4,2,1,3\}
                \end{align*}
                The number of times an element
                repeats doesn't matter either:
                \begin{align*}
                    \{1,1,2,3\}&=\{1,2,3\}\\
                    \{\textrm{Bob, Bob, Carl, George}\}&=
                    \{\textrm{Bob, Carl, George}\}
                \end{align*}
                All that matters is whether or not a certain
                element is in the set. How many times it occurs
                is unimportant. There are things called multi-sets
                in higher mathematics, but we'll never deal with
                them in elementary algebra. As a final comment,
                be wary of the difference between the
                element $x$ and the and set $\{x\}$. These are
                different things. The set $\{1,2\}$ and and set
                $\{1,\{2\}\}$ are \textbf{NOT} equal. The first
                set contains the numbers 1 and 2, and the set
                contains the number 1 and the
                \textit{set that contains the number 2}. In
                elementary algebra we will never come across
                something like $\{1,\{2\}\}$, but in
                higher mathematics this is a fundamental concept.
            \end{fremark}
            There are four common operations that one can perform
            on sets. These are: Union, intersection, difference,
            and symmetric difference.
            \begin{fdefinition*}{Set Union and Intersection}{}
                \begin{definition}
                    The union of two sets $A$ and $B$, denoted
                    $A\cup{B}$, is:
                    \begin{equation*}
                        A\cup{B}=\{x:x\in{A}\textrm{ or }x\in{B}\}
                    \end{equation*}
                \end{definition}
                \begin{definition}
                    The intersection of two set $A$ and $B$,
                    denoted $A\cap{B}$, is:
                    \begin{equation*}
                        A\cap{B}=
                        \{x:x\in{A}\textrm{ and }x\in{B}\}
                    \end{equation*}
                \end{definition}
            \end{fdefinition*}
            The \textit{union} of two sets is the collection of
            all elements that are contained in either of them.
            For example:
            \begin{equation}
                \{1,2,3\}\cup\{2,3,4,5,6\}=\{1,2,3,4,5,6\}
            \end{equation}
            Recall that the number of times an element appears
            does not matter. So even though 2 and 3 appears in
            both sets, in the union we only write each of them
            once. Working with more abstract sets, we can write:
            \begin{equation}
                \{\textrm{Bob, Carl}\}\cup\{\textrm{Apple, Joe}\}
                =\{\textrm{Bob, Carl, Apple, Joe}\}
            \end{equation}
            The \textit{intersection} of two sets is the set of all
            elements that are contained in both, simultaneously.
            The definitions of intersection and union are a cause
            of great confusion for many students. Here the word
            \textit{or} is used inclusively and the word
            $\textit{and}$ is used exclusively. Hopefully this will
            become more clear with examples:
            \begin{equation}
                \{1,2,3\}\cap\{1,2,3,4,5,6\}
                =\{2,3\}
            \end{equation}
            The intersection of two sets is the set of
            all elements common to both. Here we see that only
            2 and 3 belong to both sets, so their intersection is
            $\{2,3\}$. What happens if there are no common elements?
            When this occurs we say their intersection is
            \textit{empty}, and we use the \textit{empty set}
            to denote this.
            \begin{definition}
                The empty set, denoted $\emptyset$, is the set
                that contains no elements.
            \end{definition}
            We often write $\emptyset=\{\}$ to indicate that
            it contains no elements.
            \begin{equation}
                \{\textrm{Bob, Carl}\}\cap\{\textrm{Apple, Joe}\}
                =\emptyset
            \end{equation}
            The empty set is a very bizarre creature, but
            fortunately we won't have to deal with it
            often in elementary algebra. It is important not
            to confuse $\emptyset$ and $\{\emptyset\}$, for these
            are very different things. The empty set
            $\emptyset$ is the set that contains no elements,
            while $\{\emptyset\}$ is a set that contains one
            element (That is, it contains the empty set). Again, 
            this is all very strange and bizarre and for the most
            part will not affect us. For those who wish to go on
            to higher mathematics, the difference between
            $\emptyset$ and $\{\emptyset\}$ is crucial.
            There are a few theorems worth noting.
            \begin{theorem}
                \label{thm:Elem_Alg_Subsets_of_Unions}
                If $A$ and $B$ are sets, then
                $A\subseteq{A}\cup{B}$ and
                $B\subseteq{A}\cup{B}$.
            \end{theorem}
            \begin{proof}
                For by definition, if $x\in{A}$ then
                $x\in{A}\cup{B}$. Therefore,
                $A\subseteq{A}\cup{B}$.
                Similarly, if $x\in{B}$, then
                $x\in{A}\cup{B}$ and therefore
                $B\subseteq{A}\cup{B}$
            \end{proof}
            \begin{theorem}
                \label{thm:Elem_Alg_Subsets_of_Intersections}
                If $A$ and $B$ are sets, then
                $A\cap{B}\subseteq{A}$ and
                $A\cap{B}\subseteq{B}$.
            \end{theorem}
            \begin{proof}
                For if $x\in{A}\cap{B}$, then by definition
                $x\in{A}$ \textbf{and} $x\in{B}$.
                Therefore, by the definition of subsets, ${A}\cap{B}\subseteq{A}$ and
                ${A}\cap{B}\subseteq{B}$.
            \end{proof}
            Theorems \ref{thm:Elem_Alg_Subsets_of_Unions}
            and \ref{thm:Elem_Alg_Subsets_of_Intersections}
            can best be summarized by pictures. We often use
            \textit{Venn Diagrams} to depict sets as circles
            and denote intersections and unions by shading
            in the appropriate regions. We draw two circles,
            one labeled $A$ and the other $B$. The combination
            of the two represents their \textit{union},
            and the region contained in the center
            represents their \textit{intersection}.
            This is shown in Fig.~\ref{fig:Elem_Alg_Venn_Diagram}.
            \begin{figure}[H]
                \captionsetup{type=figure}
                \centering
                \begin{subfigure}[b]{0.49\textwidth}
                    \captionsetup{type=figure}
                    \centering
                    \begin{tikzpicture}
                        \draw (-2.5,-2) rectangle (2.5,2);
                        \fill[cyan] (-0.8cm,0) circle (1.5cm);
                        \fill[cyan] (0.8cm,0) circle (1.5cm);
                        \draw (-0.8cm,0) circle (1.5cm);
                        \draw (0.8cm,0) circle (1.5cm);
                        \node at (-1,1.1) {$A$};
                        \node at (1,1.1) {$B$};
                        \node at (-1,1.8) {$A\cup{B}$};
                    \end{tikzpicture}
                    \subcaption{The Union of Sets $A$ and $B$.}
                    \label{fig:Elem_Alg_Union_Venn_Diagram}
                \end{subfigure}
                \begin{subfigure}[b]{0.49\textwidth}
                    \captionsetup{type=figure}
                    \centering
                    \begin{tikzpicture}
                        \draw (-2.5,-2) rectangle (2.5,2);
                        \draw (-0.8cm,0) circle (1.5cm);
                        \draw (0.8cm,0) circle (1.5cm);
                        \draw[fill=cyan]
                            (0,-1.26886) arc(-57.77:57.77:1.5)
                                         arc(122.231:237.7690:1.5);
                        \node at (-1,1.1) {$A$};
                        \node at (1,1.1) {$B$};
                        \node at (-1,1.8) {$A\cap{B}$};
                    \end{tikzpicture}
                    \subcaption{The Intersection of
                                Sets $A$ and $B$.}
                    \label{fig:Elem_Alg_Intersection_Venn_diagram}
                \end{subfigure}
                \caption{Venn Diagrams can be used to represent
                         various set operations such as union
                         and intersection.}
                \label{fig:Elem_Alg_Venn_Diagram}
            \end{figure}
            \begin{theorem}
                \label{thm:Elem_Alg_Union_of_Subset}
                If $A$ and $B$ are sets, and $A\subseteq{B}$,
                then $A\cup{B}=B$.
            \end{theorem}
            \begin{proof}
                For if $x\in{A\cup{B}}$, then by definition
                either $x\in{A}$ or $x\in{B}$, or both. But
                if $A\subseteq{B}$, then for all $x\in{A}$ it is
                true that $x\in{B}$. Thus, if $x\in{A}\cup{B}$
                then $x\in{B}$. Therefore
                $A\cup{B}\subseteq{B}$. But if $x\in{B}$, then
                $x\in{A}\cup{B}$ by definition. Therefore
                $B\subseteq{A}\cup{B}$. But from the definition
                of equality, if $A\cup{B}\subseteq{B}$ and
                $B\subseteq{A}\cup{B}$, then
                $A\cup{B}=B$. This completes the proof.
            \end{proof}
            \begin{theorem}
                \label{thm:Elem_Alg_Intersection_of_Subset}
                If $A$ and $B$ are sets, and $A\subseteq{B}$,
                then $A\cap{B}=A$.
            \end{theorem}
            \begin{proof}
                For if $x\in{A}\cap{B}$, then by
                definition $x\in{A}$ and $x\in{B}$,
                and therefore ${A}\cap{B}\subseteq{A}$.
                But since $A\subseteq{B}$, if
                $x\in{A}$, then $x\in{B}$. That is, if
                $x\in{A}$ then $x\in{A}$ \textbf{and} $x\in{B}$.
                Thus, by definition, $x\in{A}\cap{B}$.
                Therefore $A\subseteq{A}\cap{B}$. Thus, by
                the definition of equality,
                $A\cap{B}=A$.
            \end{proof}
            Theorems \ref{thm:Elem_Alg_Union_of_Subset} and
            \ref{thm:Elem_Alg_Intersection_of_Subset} can best
            be shown using Venn Diagrams. This is shown
            in Fig.~\ref{fig:Elem_Alg_Venn_Diagram_for_Subset}.
            Combining some of these theorem, we obtain the
            following:
            \begin{align}
                A&\cap{B}\subseteq{A}\subseteq{A}\cup{B}\\
                A&\cap{B}\subseteq{B}\subseteq{A}\cup{B}
            \end{align}
            \begin{figure}[H]
                \captionsetup{type=figure}
                \centering
                \begin{subfigure}[b]{0.49\textwidth}
                    \captionsetup{type=figure}
                    \centering
                    \begin{tikzpicture}
                        \draw (-3,-2.3) rectangle (2.5,2.3);
                        \draw[fill=cyan] (0,0) circle (2);
                        \draw (0.5,-0.3) circle (0.85);
                        \node at (0.5,-0.1) {$A$};
                        \node at (0,1.3) {$B$};
                        \node at (-2.3,1.5) {$A\cup{B}$};
                    \end{tikzpicture}
                    \subcaption{The Union of Sets $A$ and $B$.}
                    \label{fig:Elem_Alg_Union_of_Subset}
                \end{subfigure}
                \begin{subfigure}[b]{0.49\textwidth}
                    \captionsetup{type=figure}
                    \centering
                    \begin{tikzpicture}
                        \draw (-3,-2.3) rectangle (2.5,2.3);
                        \draw (0,0) circle (2);
                        \draw[fill=cyan] (0.5,-0.3) circle (0.85);
                        \node at (0.5,-0.1) {$A$};
                        \node at (0,1.3) {$B$};
                        \node at (-2.3,1.5) {$A\cap{B}$};
                    \end{tikzpicture}
                    \subcaption{The Intersections of Sets
                                $A$ and $B$.}
                    \label{fig:Elem_Alg_Intersection_of_Subset}
                \end{subfigure}
                \caption{Venn Diagrams for the
                         Union and Intersection of Two
                         Sets $A$ and $B$ When
                         $A\subseteq{B}$.}
                \label{fig:Elem_Alg_Venn_Diagram_for_Subset}
            \end{figure}
            \begin{fexample}{Unions and Intersections}{}
                And now a plethora of examples to really
                get this planted into our brains.
                \begin{align*}
                    \{1,2,3,4,5\}\cup\{3,4,5,6,7\}
                    &=\{1,2,3,4,5,6,7\}\\
                    \{1,2,3,4,5\}\cap\{3,4,5,6,7\}
                    &=\{3,4,5\}\\
                    \{3,4,5\}\cup\{1,2,3,4,5\}
                    &=\{1,2,3,4,5\}\\
                    \{3,4,5\}\cap\{1,2,3,4,5\}
                    &=\{3,4,5\}\\
                    \{1,2,3,4,5\}\cup\{6,7,8,9,10\}
                    &=\{1,2,3,4,5,6,7,8,9,10\}\\
                    \{1,2,3,4,5\}\cap\{6,7,8,9,10\}
                    &=\emptyset\\
                    \{1,2,3\}\cup\{3,4,5\}&=\{1,2,3,4,5\}\\
                    \{1,2,3\}\cap\{3,4,5\}&=\{3\}\\
                    \{a,b,c\}\cup\{c,d,e\}&=\{a,b,c,d,e\}\\
                    \{a,b,c\}\cap\{c,d,e\}&=\{c\}\\
                    \{\textrm{Boston, Seattle}\}\cup
                    \{\textrm{Seattle, Chicago}\}
                    &=\{\textrm{Boston, Seattle, Chicago}\}\\
                    \{\textrm{Boston, Seattle}\}\cap
                    \{\textrm{Seattle, Chicago}\}
                    &=\{\textrm{Seattle}\}\\
                    \{\textrm{Polar Bear, Spongebob}\}
                    \cup\{1,a,\mathcal{L}\}&=
                    \{\textrm{Polar Bear, Spongebob},
                    1,a,\mathcal{L}\}\\
                    \{\textrm{Polar Bear, Spongebob}\}
                    \cap\{1,a,\mathcal{L}\}&=\emptyset
                \end{align*}
            \end{fexample}
            The next topic to discuss
            is that of set difference. This is similar to
            subtraction in basic arithmetic, with
            some subtle differences.
            \begin{fdefinition}{Set Difference}{}
                The set difference of a set $A$ with respect
                to a set $B$, denoted $A\setminus{B}$, is:
                \begin{equation*}
                    A\setminus{B}=
                    \{x\in{A}:x\notin{B}\}
                \end{equation*}
            \end{fdefinition}
            This is the set of all elements in $A$ that are
            not in $B$. In a way, the elements of $B$ are
            \textit{subtracted} from $A$. If $A$ and $B$
            have no elements in common, then
            $A\setminus{B}=A$. We can write this as follows:
            \begin{theorem}
                If $A$ and $B$ are sets and $A\cap{B}=\emptyset$,
                then $A\setminus{B}=A$.
            \end{theorem}
            Similarly, if $A$ is a subset of $B$ then
            $A\setminus{B}$ is empty.
            \begin{theorem}
                If $A$ and $B$ are sets and $A\subseteq{B}$,
                then $A\setminus{B}=\emptyset$.
            \end{theorem}
            \begin{fexample}{Set Difference Examples}{}
                Let's solidify our understanding
                with some examples:
                \begin{align*}
                    \{1,2,3\}\setminus\{1,2\}&=\{3\}\\
                    \{\textrm{Apple, Bob}\}\setminus
                    \{\textrm{Carl, Joe}\}
                    &=\{\textrm{Apple, Bob}\}\\
                    \{1,2,3,4\}\setminus\{1,2,3,4,5,6\}
                    &=\emptyset\\
                    \mathbb{N}_{0}\setminus\mathbb{N}&=\{0\}\\
                    \mathbb{Z}\setminus\mathbb{N}&=
                    \{\hdots,-3,-2,-1,0\}
                \end{align*}
                Recall that $\mathbb{N}_{0}=\{0,1,2,3,\hdots\}$
                and $\mathbb{N}=\{1,2,3,\hdots\}$.
                Taking the difference, we see that that
                only element of $\mathbb{N}_{0}$ that
                is not also contained in $\mathbb{N}$ is 0.
                Therefore
                $\mathbb{N}_{0}\setminus\mathbb{N}=\{0\}$.
                When computing
                $\mathbb{Z}\setminus\mathbb{N}$
                we see that only the positive integers
                are subtracted out, leaving 0 and all of
                the negative integers.
            \end{fexample}
            The symmetric difference of two sets is defined in
            terms of the union, intersection, and set difference
            all at once. It does, however, have a very intuitive
            meaning.
            \begin{fdefinition}{Symmetric Difference}{}
                The symmetric difference of two sets $A$ and
                $B$, denoted $A\ominus{B}$, is:
                \begin{equation}
                    A\ominus{B}=
                    (A\cup{B})\setminus(A\cap{B})
                \end{equation}
            \end{fdefinition}
            This is the set of all elements in $A$ or $B$,
            but not in both. Examples:
            \begin{align*}
                \{1,2,3\}\ominus\{1,2,3,4,5,6\}
                &=\{1,4,5,6\}\\
                \mathbb{N}_{0}\ominus\mathbb{N}&=\{0\}\\
                \{\textrm{Apple, Bob}\}\ominus
                \{\textrm{Carl, Joe}\}
                &=\{\textrm{Apple, Bob, Carl, Joe}\}\\
                \{a,b,c\}\ominus\{a,b,c\}&=\emptyset
            \end{align*}
            A few theorems for dealing with the
            symmetric difference:
            \begin{theorem}
                If $A$ and $B$ are sets and
                $A\cap{B}=\emptyset$, then
                $A\ominus{B}=A\cup{B}$.
            \end{theorem}
            \begin{proof}
                From the definition:
                \begin{equation*}
                    A\ominus{B}=
                    (A\cup{B})\setminus(A\cap{B})
                \end{equation*}
                But $A\cap{B}=\emptyset$, and therefore:
                \begin{equation*}
                    A\ominus{B}=
                    (A\cup{B})\setminus\emptyset
                \end{equation*}
                But the empty set contains no elements, and
                therefore $(A\cup{B})\setminus\emptyset=A\cup{B}$.
                Thus:
                \begin{equation*}
                    A\ominus{B}=A\cup{B}
                \end{equation*}
            \end{proof}
            \begin{theorem}
                If $A$ and $B$ are sets, and if
                $A\subseteq{B}$, then
                $A\ominus{B}=B\setminus{A}$.
            \end{theorem}
            \begin{proof}
                From the definition:
                \begin{equation*}
                    A\ominus{B}=(A\cup{B})\setminus(A\cap{B})
                \end{equation*}
                But from
                Thm.~\ref{thm:Elem_Alg_Union_of_Subset}, if
                $A\subseteq{B}$, then $A\cup{B}=B$. And from
                Thm.~\ref{thm:Elem_Alg_Intersection_of_Subset},
                if $A\subseteq{B}$, then $A\cap{B}=A$. Thus:
                \begin{equation*}
                    (A\cup{B})\setminus(A\cap{B})
                    =B\setminus{A}
                \end{equation*}
            \end{proof}
            We saw before that Venn Diagrams can be used to
            visualize unions and intersections, but they can
            also be used to visualize set difference and
            symmetric difference.
            \begin{figure}[H]
                \captionsetup{type=figure}
                \centering
                \begin{subfigure}[b]{0.49\textwidth}
                    \captionsetup{type=figure}
                    \centering
                    \begin{tikzpicture}
                        \draw (-2.5,-2) rectangle (2.5,2);
                        \fill[cyan] (-0.8cm,0) circle (1.5cm);
                        \fill[white] (0.8cm,0) circle (1.5cm);
                        \draw (-0.8cm,0) circle (1.5cm);
                        \draw (0.8cm,0) circle (1.5cm);
                        \node at (-1,1.1) {$A$};
                        \node at (1,1.1) {$B$};
                        \node at (-1,1.75) {$A\setminus{B}$};
                    \end{tikzpicture}
                    \subcaption{The Set Difference of
                                $A$ and $B$.}
                    \label{fig:Elem_Alg_Set_Difference}
                \end{subfigure}
                \begin{subfigure}[b]{0.49\textwidth}
                    \captionsetup{type=figure}
                    \centering
                    \begin{tikzpicture}
                        \draw (-2.5,-2) rectangle (2.5,2);
                        \draw[fill=cyan]
                            (-0.8cm,0) circle (1.5cm);
                        \draw[fill=cyan]
                            (0.8cm,0) circle (1.5cm);
                        \draw[fill=white]
                            (0,-1.26886)
                                arc(-57.77:57.77:1.5)
                                arc(122.231:237.7690:1.5);
                        \node at (-1,1.1) {$A$};
                        \node at (1,1.1) {$B$};
                        \node at (-1,1.8) {$A\ominus{B}$};
                    \end{tikzpicture}
                    \subcaption{Symmetric Difference of
                                $A$ and $B$.}
                    \label{fig:Elem_Alg_Symmetric_Differece}
                \end{subfigure}
                \caption{Venn Diagrams for the
                         Set Difference and Symmetric Difference
                         of $A$ and $B$.}
                \label{fig:Elem_Alg_Venn_Diagram_Differences}
            \end{figure}
        \subsection{Numbers}
            There are many kinds of numbers,
            but certain ones are particularly interesting.
            These sets have a standard notation
            that is agreed upon by most mathematicians.
            \begin{fnotation}{Sets of Numbers}{}
                The Natural numbers.
                \begin{equation*}
                    \mathbb{N}=
                    \{1,2,3,4,\hdots\}
                \end{equation*}
                The Whole Numbers.
                \begin{equation*}
                    \mathbb{N}_{0}=
                    \{0,1,2,3,4,\hdots\}
                \end{equation*}
                The Integers.
                \begin{equation*}
                    \mathbb{Z}=
                    \{\hdots,-2,-1,0,1,
                      2,\hdots\}
                \end{equation*}
                The Rational Numbers.
                \begin{equation*}
                    \mathbb{Q}=
                    \{p/q:
                      p,q\in\mathbb{Z},
                      q\ne{0}\}
                \end{equation*}
                The Real Numbers.
                \begin{equation*}
                    \mathbb{R}=
                    \{x:x\textrm{ is real.}\}
                \end{equation*}
                The irrational numbers.
                \begin{equation*}
                    \mathbb{R}\setminus\mathbb{Q}
                    =\{x\in\mathbb{R}:
                       x\notin\mathbb{Q}\}
                \end{equation*}
            \end{fnotation}
            The only serious disagreement that one may find
            is with $\mathbb{N}_{0}$. Computer scientists
            often define $\mathbb{N}$ to include $0$,
            but they're crazy, and some mathematicians write
            $\mathbb{W}=\{0,1,2,3,\hdots\}$
            ($\mathbb{W}$ for \textit{whole} numbers) but they're
            also crazy. While no mathematician would disagree that
            $\mathbb{R}\setminus\mathbb{Q}$ is certainly the set
            of irrational numbers, some (Also crazy) individuals
            choose to write $\mathbb{I}$ to represent this.
            $\mathbb{Z}$, $\mathbb{Q}$, and $\mathbb{R}$ are
            universally accepted. $\mathbb{Z}$ stand for
            \textit{Zahl}, which is German for number.
            $\mathbb{Q}$ stands for \textit{quotient}, since
            rational numbers are quotients of integers. Finally,
            $\mathbb{R}$ simply means \textit{real}.
            From the definitions given, we have the
            following:
            \begin{subequations}
                \begin{equation}
                    \mathbb{N}
                    \subset\mathbb{N}_{0}
                    \subset\mathbb{Z}
                    \subset\mathbb{Q}
                    \subset\mathbb{R}
                \end{equation}
                \begin{equation}
                    \mathbb{R}\setminus\mathbb{Q}
                    \subset\mathbb{R}
                \end{equation}
            \end{subequations}
        \begin{remark}
        One might note that, simply by definition, none of the
        rational numbers are irrational and none of the
        irrational numbers are rational.
            $0.111\hdots$ is just a fancy way to write $\frac{1}{9}$.
            Strange beings like $\pi$ are included in the real numbers,
            but cannot be expressed as fractions. It can be shown that
            any rational number has a repeating decimal expansion.
            Therefore a number like $0.1234567891011121314151617181920\hdots$
            cannot possibly be rational, for the decimal never repeats.
            Similarly $0.123412341234\hdots$ must be a rational number,
            for its decimal expansion repeats.
            Indeed this number is equal to $\frac{1234}{9999}$.
        \end{remark}
        We compare the size of numbers using the inequality
        symbols $<$ (Less than) and $>$ (Greater than).
        \begin{properties}[Order Property of Real Numbers]
        For two real numbers $a,b\in \mathbb{R}$:
        \begin{enumerate}
        \begin{multicols}{2}
        \item $a<b$ if $a$ is to the left of $b$ on the number line.
        \item $a>b$ if $a$ is to the right of $b$ on the number line.
        \end{multicols}
        \end{enumerate}
        \end{properties}
        \begin{definition}
        A variable is a symbol used to represent an unknown quantity.
        \end{definition}
        \begin{example}
        $x$ and $y$ are commonly used to represent real numbers. $n$ and $m$ are commonly used to represent integers. $z$ is often used to represent complex numbers, but we won't get into that until later.
        \end{example}
        \begin{example}
        Let's use a variable to represent the sentence "To hit a baseball out of the park, the ball must travel more than 315 feet." Let $d$ be the distance the ball must travel to be a home run. Then $d>315$
        \end{example}
        \begin{notation}
        If we wish to say $a$ is less than or equal to $b$, we write $a\leq b$. This means that either $a<b$ or $a=b$. Similarly, if we wish to write that $a$ is greater than or equal to $b$, we write $a\geq b$. This means that either $a>b$ or $a=b$.
        \end{notation}
        The absolute value of a real number is the distance from that number to the origin (The number $0$).
        \begin{definition}
        The absolute value of a real number $x\in \mathbb{R}$ is $|x| = \begin{cases} x, & x \geq 0 \\ -x, & x<0 \end{cases}$
        \end{definition}
        \begin{theorem}
        If $a$ and $b$ are real numbers, then the following are true:
        \begin{enumerate}
        \begin{multicols}{4}
        \item $|a+b| = |-a-b|$
        \item $|a-b| = |b-a|$
        \item $|-a| = |a|$
        \item $|a\cdot b| = |a|\cdot |b|$
        \end{multicols}
        \end{enumerate}
        \end{theorem}
        
        \begin{example}[Examples of Absolute Value]
        \
        \begin{enumerate}
        \begin{multicols}{4}
        \item $|1| = 1$
        \item $|-1| = 1$
        \item $|0| = 0$
        \item $|\pi| = \pi$
        \item $|-\pi| = \pi$
        \item $|\sqrt{2}| = \sqrt{2}$
        \item $|-\sqrt{2}| = \sqrt{2}$
        \item $|\frac{3}{4}| = \frac{3}{4}$
        \item $|-\frac{3}{4}| = \frac{3}{4}$
        \item $|\frac{3}{-4}| = \frac{3}{4}$.
        \item $|(-3)\cdot4| = 12$
        \item $|(-3)\cdot(-4)| = 12$
        \end{multicols}
        \end{enumerate}
        \end{example}
        \begin{definition}
        The exponentiation of a real number $a\in \mathbb{R}$ by a natural number $n\in \mathbb{N}$ is the number $a^n = \underset{n\ times}{\underbrace{a\cdots a}}$.
        \end{definition}
        \begin{example}[Examples of Exponentiation]
        \
        \begin{enumerate}
        \begin{multicols}{3}
        \item $10^2 = 10\cdot 10 = 100$.
        \item $10^4 = 10\cdot 10 \cdot 10 \cdot 10 = 10000$
        \item $10^1 = 10$
        \item $2^3 = 2\cdot 2 \cdot 2 = 8$
        \item $\pi^2 = \pi\cdot \pi = 9.869\hdots$
        \item $(\sqrt{2})^2 = \sqrt{2}\cdot \sqrt{2} = 2$.
        \end{multicols}
        \end{enumerate}
        \end{example}
        \begin{remark}
        The fact that $(\sqrt{2})^2 = 2$ is really the definition of the number $\sqrt{2}$. We'll see this in a bit.
        \end{remark}
        \begin{theorem}
        The following are true:
        \begin{enumerate}
        \begin{multicols}{3}
        \item If $n\in \mathbb{N}$, then $0^n = 0$.
        \item $(-1)^2 = 1$
        \item $(-x)^2 = x^2$
        \item $(-1)^3 = -1$
        \item $(-x)^3=-x^3$
        \item If $n$ is even, then $(-1)^n = 1$
        \item if $n$ is even, then $(-x)^n = x^n$
        \item If $n$ is odd, then $(-1)^n = -1$
        \item If $n$ is odd, then $(-x)^n = -x^n$. 
        \end{multicols}
        \end{enumerate}
        \end{theorem}
        \begin{theorem}
        If $y$ is a positive real number, then there is a unique positive real number $x$ such that $y=x^2$.
        \end{theorem}
        \begin{remark}
        The theorem says there is a unique $positive$ real number. There are actually two real numbers satisfying this property. For if $y=x^2$, then $y=(-x)^2$, and thus $x$ and $-x$ are solutions. One of these will be negative, though.
        \end{remark}
        \begin{definition}
        The principal square root of a positive real number $x$, denoted $\sqrt{x}$, is the unique positive real number such that $(\sqrt{x})^2 = x$. The symbol $\sqrt{\ \ }$ is called a radical, and $x$ is called the radicand.
        \end{definition}
        \begin{example}[Examples of Square Roots]
        \
        \begin{enumerate}
        \begin{multicols}{4}
        \item $\sqrt{1} = 1$
        \item $\sqrt{4} = 2$
        \item $\sqrt{9} = 3$
        \item $\sqrt{16} = 4$
        \item $\sqrt{25} = 5$
        \item $\sqrt{36} = 6$
        \item $\sqrt{49} = 7$
        \item $\sqrt{64} = 8$
        \end{multicols}
        \end{enumerate}
        \end{example}
        \begin{theorem}
        If $y\in \mathbb{R}$ is a real number, then there is a unique real number $x$ such that $y=x^3$.
        \end{theorem}
        \begin{definition}
        The cube root of a real number $x$, denoted $\sqrt[3]{x}$, is the unique real number such that $(\sqrt[3]{x})^3 = x$.
        \end{definition}
        \begin{example}[Examples of Cube Roots]
        \
        \begin{enumerate}
        \begin{multicols}{4}
        \item $\sqrt[3]{-1} = -1$
        \item $\sqrt[3]{8} = 2$
        \item $\sqrt[3]{-8} = -2$
        \item $\sqrt[3]{27} = 3$
        \item $\sqrt[3]{125} = 5$
        \item $\sqrt[3]{-125} = -5$
        \item $\sqrt[3]{-64} = -4$
        \item $\sqrt[3]{1000} = 10$
        \end{multicols}
        \end{enumerate}
        \end{example}
        \begin{remark}
        The cube root theorem is more relaxed than the square root theorem. The square root theorem requires that $y$ is positive. Negative real numbers do not have square roots. However, all real number have cube roots.
        \end{remark}
        \begin{remark}
        Note that for any positive real number $a$, $\big(\sqrt{a}\big)^2 = a$, and for any real number $b$, $(\sqrt[3]{b})^3 = b$.
        \end{remark}
        \begin{theorem}
        If $r\in \mathbb{R}$ is positive and $n\in \mathbb{N}$, then there is a unique positive real number $\sqrt[n]{r}$, such that $(\sqrt[n]{r})^n = r$.
        \end{theorem}
        \begin{definition}
        The principal $n^{th}$ root of a positive $r\in \mathbb{R}$ is the unique positive real number such that $(\sqrt[n]{r})^n = r$.
        \end{definition}
        \begin{example}[Examples of $n^{th}$ Roots]
        \
        \begin{enumerate}
        \begin{multicols}{4}
        \item $\sqrt{4} = 2$
        \item $\sqrt[3]{27} = 3$
        \item $\sqrt[4]{16} = 2$
        \item $\sqrt[3]{125} = 5$
        \item $\sqrt[3]{8} = 2$
        \item $\sqrt[4]{81} = 3$
        \item $\sqrt[5]{32} = 2$
        \item $\sqrt[6]{729} = 3$
        \end{multicols}
        \end{enumerate}
        \end{example}
        \begin{properties}[The Order of Operations]
        When performing arithmetic to simplify expressions, use the following order:
        \begin{enumerate}
        \item Perform operations inside parenthesis, brackets, braces, etc., first.
        \item Next, perform exponentiation.
        \item Then perform multiplication and division from left to right in the order they appear in the expression.
        \item Finally perform addition and subtraction from left to right in the order they appear in the expression.
        \end{enumerate}
        \end{properties}
        \begin{remark}
        For some reason, the internet was once obsessed with the expression $48\div 2(9+3)$. Depending on what you do, you either got $288$ or $2$. According to the order of operations, the $correct$ answer is $288$. However the $real$ correct answer is: \textbf{If you write ambiguous expressions like this instead of using parantheses, then you are a bad person}. Parenthesis rid of ambiguity. $48\div\big(2(9+3)\big) = 2$, unambiguously. $\big(48\div 2\big)(9+3) = 288$, again unambiguously. Furthermore, stop using the $\div$ symbol. It's archaic and ambiguous. Writing $\frac{48}{2(9+3)}$ or $\frac{48}{2}(9+3)$ leaves no ambiguity.
        \end{remark}
        \subsubsection{Solved Problems}
        \begin{enumerate}
        \begin{multicols}{4}
        \item $|-2.75| = 2.75$
        \item $|-7.24| = 7.24$
        \item $-|-4| = -4$
        \item $-|-6| = -6$
        \item $|3-(-6)| = 9$
        \item $|-4-7| = 11$
        \item $|-7.5-2.5| = 10$
        \item $|13.4 - (-2.6)| = 16$
        \item $|5-2| = 3$
        \item $|-1-2| = 3$
        \item $7^2 = 49$
        \item $(-7)^2 = 49$
        \item $-7^2 = -49$
        \item $-(-7)^2 = -49$.
        \item $3^3 = 27$
        \item $(-3)^3 = -27$
        \item $-(-3)^3 = 27$
        \item $(-1)^2 = 1$
        \item $(-1)^3 = -1$
        \item $(-1)^4 = 1$
        \item $(-1)^5 = -1$
        \item $-24 - (-31) = 7$
        \item $\frac{-\frac{3}{4}}{\frac{7}{8}} = -\frac{6}{7}$
        \item $\frac{-20}{\frac{1}{2}} = -40$.
        \end{multicols}
        \end{enumerate}
        \subsubsection{Algebraic Expressions and the Properties of Real Numbers}
        \begin{definition}
        An algebraic term is a collection of factors such as numbers, variables, or expressions within parentheses.
        \end{definition}
        \begin{example}[Examples of Algebraic Terms]
        \
        \begin{enumerate}
        \begin{multicols}{6}
        \item $3$
        \item $-x$
        \item $5xy$
        \item $-3n^3$
        \item $4y$
        \item $2(x+3)$.
        \end{multicols}
        \end{enumerate}
        \end{example}
        \begin{definition}
        The numerical value in a term is called its coefficient.
        \end{definition}
        \begin{example}[Examples of Coefficients]
        \
        \begin{enumerate}
        \begin{multicols}{3}
        \item $4$ is the coefficient of $4xy$
        \item $3$ is the coefficient of $3z^2$
        \item $1$ is the coefficient of $xyz$
        \item $\frac{1}{\pi}$ is the coefficient of $\frac{1}{p}x^3$.
        \item $1$ is the coefficient of $x$
        \item $10$ is the coefficient of $10t^2$
        \end{multicols}
        \end{enumerate}
        \end{example}
        \begin{definition}
        A constant is an algebraic term with only a numerical factor in it.
        \end{definition}
        \begin{definition}
        A variable term is an algebraic term that contains a variable.
        \end{definition}
        \begin{example}[Examples of Variable Terms]
        \
        \begin{enumerate}
        \begin{multicols}{4}
        \item $x^2$
        \item $2xyz^2$
        \item $5t^4$
        \item $3x(y+1)$
        \item $t(x+y)(x-y)$
        \item $3t^2y$
        \item $x^4y^3z^2w$
        \item $2\pi r$
        \end{multicols}
        \end{enumerate}
        \end{example}
        \begin{definition}
        An algebraic expression is a single term or the sum of finitely many terms.
        \end{definition}
        \begin{example}
        Let's translate the following sentences into mathematical expressions:
        \begin{enumerate}
        \begin{multicols}{2}
        \item Twice a number, increased by $5$: $2n+5$
        \item Six less than three times a number: $3n-6$
        \end{multicols}
        \end{enumerate}
        \end{example}
        \begin{properties}[Evaluating a Mathematical Expression]
        To evaluate an expression, do the following:
        \begin{enumerate}
        \begin{multicols}{2}
        \item Substitute the given values for each variable.
        \item Simplify.
        \end{multicols}
        \end{enumerate}
        \end{properties}
        \begin{example}
        Solve $x^3-2x^2+5$ for $x=-3$: $(-3)^3+2(-3)^2+5 = -27-18+5 = -40$.
        \end{example}
        \begin{properties}[The Commutative Properties]
        If $a$ and $b$ are real numbers, then the following is true:
        \begin{enumerate}
        \item $a+b = b+a$ \hfill [The Commutative Property of Addition]
        \item $a\cdot b = b\cdot a$ \hfill [The Commutative Property of Multiplication]
        \end{enumerate}
        \end{properties}
        \begin{example}[Examples of the Commutative Properties]
        \
        \begin{enumerate}
        \begin{multicols}{3}
        \item $2+3 = 5,\ 3+2 = 5$
        \item $2+(-8) = -6,\ (-8)+2 = - 6$
        \item $5\cdot 6 = 30,\ 6\cdot 5 = 30$
        \end{multicols}
        \end{enumerate}
        \end{example}
        \begin{properties}[The Associative Properties]
        If $a,b,c\in \mathbb{R}$, then the following is true:
        \begin{enumerate}
        \item $a+(b+c) = (a+b)+c$ \hfill [The Associative Property of Addition]
        \item $a\cdot(b\cdot c) = (a\cdot b)\cdot c$ \hfill [The Associative Property of Multiplication]
        \end{enumerate}
        \end{properties}
        \begin{example}[Examples of the Associative Properties]
        \
        \begin{enumerate}
        \begin{multicols}{2}
        \item $1+(2+3) = 1+5 = 6,\ (1+2)+3 = 3+3 = 6$
        \item $5+(2+8) = 5+10 = 15,\ (5+2)+8 = 7+8 = 15$
        \item $2\cdot(3\cdot 4) = 2\cdot 12 = 24,\ (2\cdot 3)\cdot 4 = 6\cdot 4 = 24$
        \item $\frac{1}{2}\cdot(2\cdot 3) = \frac{1}{2}\cdot 6 = 3,\ (\frac{1}{2}\cdot 2)\cdot 3 = 1\cdot 3 = 3$.
        \end{multicols}
        \end{enumerate}
        \end{example}
        \begin{properties}[The Distributive Property of Multiplication over Addition]
        If $a,b,c\in \mathbb{R}$, then the following is true:
        \begin{enumerate}
        \item $a\cdot(b+c) = (a\cdot b) + (a\cdot c)$\hfill [The Distributive Property of Multiplication over Addition]
        \end{enumerate}
        \end{properties}
        \begin{example}[Examples of the Distributive Property]
        \
        \begin{enumerate}
        \item $2\cdot(1+1) = 2\cdot 2 = 4,\ 2\cdot(1+1) = (2\cdot 1)+(2\cdot 1) = 2+2 = 4$
        \item $2\cdot(2+3) = 2\cdot 5 = 10,\ 2\cdot(2+3) = (2\cdot 2)+(2\cdot 3) = 4+6 = 10$
        \end{enumerate}
        \end{example}
        \begin{definition}
        Like terms are two algebraic terms that have the same variable factors.
        \end{definition}
        \begin{example}
        In $xy+x+2xy$, $xy$ and $2x$ are like terms so we can combine them to get $3xy+x$.
        \end{example}
        When trying to simplify an expression, we use the distributive property, the associative properties, and the commutative properties to combine like terms to get and form simplified expressions.
        \begin{enumerate}
        \begin{multicols}{2}
        \item Seven fewer then a number: $n-7$
        \item $x$ decreased by $6$: $x-6$
        \item The number of a number and four: $n+4$
        \item A number increased by $9$: $n+9$
        \item The difference between a number and five is squared: $(n-5)^2$
        \item The sum of a number and two is cubed: $(n+2)^3$
        \item Thirteen less than twice a number: $2n-13$
        \item Five less than double a number: $2n-13$
        \end{multicols}
        \item[] Let $x=2$ and $y=-3$. Evaluate:
        \begin{multicols}{4}
        \item $4x-2y: 14$
        \item $5x-3y: 19$
        \item $-2x^2+3y^2: 19$
        \item $-5x^2+4y^2: 16$
        \item $2y^2+5y-3: 0$
        \item $3x^2+2x-5: 11$
        \item $(2x-2y)^2:144$
        \item $(2x-3y)^2: 169$
        \end{multicols}
        \end{enumerate}
        \subsubsection{Exponents, Scientific Notation, and a Review of Polynomials}
        \begin{notation}
        If $a\in \mathbb{R}$, $a\ne 0$, and $n\in \mathbb{Z}$, $n<0$, then $a^n = \frac{1}{a^{|n|}}$.
        \end{notation}
        \begin{example}[Examples of Negative Exponents]
        \
        \begin{enumerate}
        \begin{multicols}{4}
        \item $2^{-1} = \frac{1}{2}$
        \item $2^{-3} = \frac{1}{2^3} = \frac{1}{8}$
        \item $10^{-1} =\frac{1}{10}$
        \item $10^{-3} = \frac{1}{10^3} = \frac{1}{1000}$
        \item $4^{-2} = \frac{1}{4^2} = \frac{1}{16}$
        \item $\pi^{-1} = \frac{1}{\pi}$
        \item $\pi^{-2} = \frac{1}{\pi^2}$
        \item $\big(\frac{1}{2}\big)^{-1} = 2$
        \item $\big(\frac{3}{2}\big)^{-1} = \frac{2}{3}$
        \item $\big(\frac{1}{10}\big)^{-2} = 10^2 = 100$
        \item $\big(\frac{1}{2}\big)^{-3} = 2^3 = 8$
        \item $\big(\frac{3}{2}\big)^{-4} = \big(\frac{2}{3}\big)^4 = \frac{16}{81}$
        \end{multicols}
        \end{enumerate}
        \end{example}
        \begin{notation}
        If $a \in \mathbb{R}$, $a>0$, and $p,q \in \mathbb{Z}$, then $a^{\frac{p}{q}}$ is the unique positive number such that $\big(a^{\frac{p}{q}}\big)^q = a^p$.
        \end{notation}
        \begin{example}[Examples of Fractional Exponents]
        \
        \begin{enumerate}
        \begin{multicols}{4}
        \item $2^{\frac{1}{2}} = \sqrt{2}$
        \item $2^{\frac{1}{3}} = \sqrt[3]{2}$
        \item $10^{\frac{1}{n}} = \sqrt[n]{10}$
        \item $2^{\frac{3}{2}} = \sqrt{2^3} = \sqrt{8}$
        \item $10^{\frac{4}{3}} = \sqrt[3]{10^4} = \sqrt[3]{10000}$
        \item $3^{\frac{4}{5}} = \sqrt[5]{3^4} = \sqrt[5]{81}$
        \item $2^{\frac{5}{6}} = \sqrt[6]{2^5} = \sqrt[6]{32}$
        \item $6^{\frac{5}{5}} = \sqrt[5]{6^5} = 6$. 
        \end{multicols}
        \end{enumerate}
        \end{example}
        Thus, we have defined exponentiation for negative integers and fractions as well.
        \begin{properties}[The Properties of Exponents]
        If $a,b\in \mathbb{R}$ and $n,m,p\in \mathbb{N}$, then the following are true:
        \begin{enumerate}
        \item $\big(a^n\big)^m = a^{n\cdot m}$ \hfill [Power Property]
        \item $a^n \cdot a^m = a^{n+m}$ \hfill [Product Property]
        \item $\big(a^m\cdot b^n\big)^p = a^{m \cdot p} \cdot b^{n\cdot p}$ \hfill [Product to a Power Property]
        \item If $b\ne 0$, $\big(\frac{a^n}{b^n}\big)^p = \frac{a^{n\cdot p}}{b^{n\cdot p}}$ \hfill [Quotient to a Power Property]
        \item If $a\ne 0$, $\frac{a^n}{a^m} = a^{n-m}$ \hfill [Quotient Property]
        \item If $a\ne 0$, $a^0 = 1$\hfill [The Zero Property]
        \end{enumerate}
        \end{properties}
        \begin{remark}
        WARNING: Some notation from calculus ahead. We leave $0^0$ undefined. In various scenarios in calculus, we use the convention that $0^0 = 1$, but this is no more than a convention. An example is in infinite series. Say we want to add $1+x+x^2+x^3+x^4+\hdots$. We use the notation $\sum_{n=0}^{\infty} x^n=x^0+x^1+x^2+x^3+\hdots$ to represent this. We plug in a value for $x$ and get a number. For every value of $x$ other than $x=0$, we have $x^0 = 1$ and so we get back the original sum. It would be really annoying to continuously talk about the special case when $x=0$, and so we adopt the convention that $0^0 =1$ as well. This is just a convention for this area of mathematics, and $0^0$ is, in general, left undefined.
        \end{remark}
        \begin{definition}
        The scientific notation of a real number $x$ is $x = r\times 10^n$, where $0\leq |r| < 10$, and $n\in \mathbb{Z}$.
        \end{definition}
        Every real number has a scientific representation.
        \begin{example}[Examples of Scientific Notation]
        \
        \begin{enumerate}
        \begin{multicols}{3}
        \item $101 = 1.01\times 10^2$
        \item $10,000 = 1\times 10^4$
        \item $314.15926\hdots = \pi \times 10^2$
        \item $-123.456 = -1.23456\times 10^2$
        \item $-0.031415926\hdots = -\pi \times 10^{-2}$
        \item $0.00001 = 1\times 10^{-5}$
        \end{multicols}
        \item $0.\underset{33\ times}{\underbrace{0\hdots 0}662607004} = 6.62607004\times 10^{-34} = h$ (Physicist's like this number)
        \end{enumerate}
        \end{example}
        \begin{definition}
        A monomial is a term with only whole number variable exponents and no variables in the denominator.
        \end{definition}
        \begin{example}[Examples of Monomials]
        \
        \begin{enumerate}
        \begin{multicols}{4}
        \item $4x^2$
        \item $3xyz$
        \item $3y^2z$
        \item $3z^2$
        \end{multicols}
        \end{enumerate}
        \end{example}
        \begin{definition}
        A polynomial is a sum of monomials.
        \end{definition}
        \begin{example}[Examples of Polynomials]
        \
        \begin{enumerate}
        \begin{multicols}{4}
        \item $x^2+x+1$
        \item $3xy+6z^2+w$
        \item $x^2+2xy+y^2$
        \item $1+xyz+x^2y^2z^2$
        \end{multicols}
        \end{enumerate}
        \end{example}
        \begin{definition}
        The degree of a polynomial in one variable is the largest exponent of any of the terms.
        \end{definition}
        \begin{definition}
        The degree of a polynomial in many variables is the largest sum of exponents of any of the terms.
        \end{definition}
        \begin{definition}
        A binomial is a polynomial with two monomial terms.
        \end{definition}
        
        \begin{definition}
        A trinomial is a polynomial with three monomial terms.
        \end{definition}
        
        \begin{example}
        \
        \begin{enumerate}
        \begin{multicols}{2}
        \item $5x^2y-2xy$ is a binomial of degree $3$.
        \item $3x^2 - 1$ is a binomial of degree $2$.
        \item $z^3-3z^2+9z-27$ is a polynomial of degree $3$.
        \item $x+5$ is a binomial of degree $1$.
        \item $2x^2+x+3$ is a trinomial of degree $2$.
        \item $xyz+1$ is a binomial of degree $3$
        \item $x^2yz +x+1$ is a trinomial of degree $4$
        \item $x^2y^2z + 1$ is a binomial of degree $5$
        \item $x^2y^2z^2 + 2$ is a binomial of degree $6$
        \item $x^{10} y^{26} z^3 + 2x^2+1$ is a trinomial of degree $39$
        \end{multicols}
        \end{enumerate}
        \end{example}
        To add polynomials, we combine like terms and simplify using the commutative and associative properties.
        \begin{example}[Adding Polynomials]
        \
        \begin{enumerate}
        \item $(3x^2y+x+y)+(2x^2y-y) = 5x^2y+x$
        \item $(1+x+x^2)+(x+x^2+x^3) = 1+2x+2x^2+x^3$
        \item $(1+x)+(x+x^2) = 1+2x+x^2$
        \item $(xyz+xy+x+z) + (y+xz+yz-xyz) = xy+xz+yz+x+y+z$
        \end{enumerate}
        \end{example}
        To multiply two polynomials, we use the distributive property and then combine like terms.
        \begin{example}[Multiplying Polynomials]
        \
        \begin{enumerate}
        \begin{multicols}{2}
        \item $xy(x+z) = x^2y+xyz$
        \item $x(y+z) = xy+xz$
        \item $(x+1)(y^2+z) = xy^2+xz+y^2+z$
        \item $(x+y)(x-y) = x^2-y^2$.
        \end{multicols}
        \end{enumerate}
        \end{example}
        \begin{theorem}
        If $A$ and $B$ are real numbers, the following is true:
        \begin{enumerate}
        \item $(A+B)(A-B) = A^2-B^2$ \hfill [Binomial Conjugates]
        \item $(A+B)^2 = A^2 + 2AB + B^2$ \hfill [Square of a Sum]
        \item $(A-B)^2 = A^2-2AB + B^2$ \hfill [Square of a Difference]
        \end{enumerate}
        \end{theorem}
        \begin{enumerate}
        \begin{multicols}{3}
        \item $n^2\cdot 21n^5= 21n^7$
        \item $5x^2 \cdot 7x^2= 35 x^4$
        \item $(-6p^2q)(2p^3q^3)= -12p^5q^4$
        \item $(a^2)^4\cdot (a^2)^3\cdot b^2\cdot b^5= a^{14}b^7$
        \item $(6pq^2)^3= 216p^3q^6$
        \item $\frac{-6 w^5}{-2 w^2}= 3w^3$
        \item $\frac{8 z^7}{16 z^5}= \frac{1}{2}z^2$
        \item $\frac{-12 a^3b^5}{4a^2b^4}= -3ab$
        \item $\big(\frac{2}{3}\big)^{-3}= \frac{27}{8}$
        \item $\frac{5m^3n^5}{10mn^2}= \frac{1}{2}m^2n^3$
        \item $\frac{3}{m^{-2}}= 3m^2$
        \item $\big(\frac{2p^4}{q^3}\big)^2= 4\frac{p^8}{q^6}$
        \item $\big(\frac{-5 v^4}{7w^3}\big)^2= \frac{25 v^8}{49 w^6}$
        \item $\frac{9p^6 q^4}{-12p^4q^6}= -\frac{3}{4} \frac{p^2}{q^2}$
        \item $\frac{5m^2 n^2}{10 m^2 n}= \frac{1}{2} n$
        \item $\frac{5k^3}{20 k^{-2}}= \frac{1}{4} k^5$
        \item $\frac{7x^3}{x} = 7x^2$
        \item $\frac{x^2y^3 z^4}{xyz} = xy^2z^3$
        \end{multicols}
        \item $(x^3+2x^2+x+1)+(3x^3-4x) = 4x^3+2x^2-3x+1$
        \item $(xy+1)+(3x^2-2xy+4) = -xy+3x^2+5$
        \begin{multicols}{2}
        \item $2x(x^2+y^2) = 2x^3+2xy^2$
        \item $(1+x)(1-x) = 1+x^2$
        \item $(1+x+x^2)(1-x) = 1 - x^3$
        \item $(1+x+x^2+x^3)(1-x) = 1 - x^4$
        \item $(2x^2+3y)(x+y) = 2x^3+2x^2y+3xy+3y^2$
        \item $xy(x+y)^2 = x^3y+2x^2y^2+xy^3$
        \end{multicols}
        \end{enumerate}
        \subsubsection{Factoring Polynomials}
        \begin{definition}
        To factor an expression is to rewrite the expression as an equivalent product.
        \end{definition}
        The distributive property of multiplication over addition is an example of factoring.
        \begin{example}
        Let's factor $x^2+2xy+y^2$. Using the distributive property we obtain $x^2+2xy+y^2=x^2+xy+xy+y^2 = x(x+y)+y(x+y) = (x+y)(x+y) = (x+y)^2$.
        \end{example}
        \begin{example}
        Let's factor $12x^2+18xy-30y$. Using the distributive property, $12x^2+12xy-30y = 6(2x^2+3xy-5y)$.
        \end{example}
        \begin{example}
        $x^5+x^2 = x^2(x^3+1)$.
        \end{example}
        \begin{example}
        Let's factor $3t^3+15t^2-6t-30$. Using the distributive property, we have $3\big(t^3+5t^2-2t-10\big) = 3\big(t^2(2t+5)-2(t+5)\big) = 3\big((t^2-2)(t+5)\big) = 3(t^2-2)(t+5)$.
        \end{example}
        \begin{example}
        $(x+3)x^2+5(x+3) = (x+3)(x^2+5)$.
        \end{example}
        \begin{definition}
        A quadratic polynomial is one of the form $ax^2+bx+c$, where $a,b,c\in \mathbb{R}$.
        \end{definition}
        There is a special case of quadratic polynomials where $a=1$. That is, quadratics of the form $x^2+bx+c$ for real numbers $b,c\in \mathbb{R}$. 
        \begin{theorem}
        If $x^2+bx+c$ is a quadratic, where $b,c\in \mathbb{R}$, and if $\alpha,\beta \in \mathbb{R}$ are real numbers such that $\alpha \cdot \beta = c$ and $\alpha+\beta = b$, then $x^2+bx+c = (x+\alpha)(x+\beta)$.
        \end{theorem}
        \begin{proof}
        For $(x+\alpha)(x+\beta) = x^2+\alpha x + \beta x + \alpha\cdot \beta = x^2+x(\alpha + \beta) + \alpha \cdot \beta=x^2+bx+c$.
        \end{proof}
        \begin{example}
        $x^2-11x+24 = (x-3)(x-8)$.
        \end{example}
        \begin{example}
        $x^2-3x-10 = (x-5)(x+2)$
        \end{example}
        \begin{definition}
        A prime polynomial is a polynomial that cannot be factored further.
        \end{definition}
        \begin{example}
        $x^2+9x+15$ is a prime polynomial. We need $\alpha\cdot \beta = 15$ and $\alpha+\beta = 9$. Factoring $15$ into prime numbers, we get $15 = 5\cdot 3$ or $15 = 15\cdot 1$. In either case, $5+3=8\ne 9$ and $15+1 = 16 \ne 9$. So $x^2+9x+15$ cannot be factored further using integer coefficients.
        \end{example}
        For the general case of $a\ne 0$, we let $d = \frac{b}{a}$ and $e = \frac{c}{a}$. Then we find $\alpha$ and $\beta$ such that $(x+\alpha)(x+\beta) = x^2+dx+e$. Multiplying both sides by $a$ gives us $a(x+\alpha)(x+\beta) = ax^2+adx+ae = ax^2+bx+c$. So, $(ax+a\cdot \alpha)(x+\beta) = ax^2+bx+c$. So the general case reduces to the specific case of $a=1$.
        We've seen the following identities before, and they can be used to simplify expressions:
        \begin{enumerate}
        \item $A^2-B^2 = (A+B)(A-B)$ \hfill [Difference of Squares]
        \item $A^2+2AB+B^2 = (A+B)^2$ \hfill [Square of a Sum]
        \item $A^2-2AB+B^2 = (A-B)^2$ \hfill [Square of a Difference]
        \end{enumerate}
        \begin{example}[Examples of Factoring]
        \
        \begin{enumerate}
        \begin{multicols}{2}
        \item $4x^2-81 = (2x+9)(2x-9)$
        \item $x^2+49$ is prime.
        \item $x^2 - 16 = (x+4)(x-4)$.
        \item $x^2-9 = (x+3)(x-3)$
        \item $x^4 - 81 = (x^2+9)(x^2-9) = (x^2+9)(x+3)(x-3)$.
        \item $4x^2+8xy+4y^2 = (2x)^2+2(2x)(2y)+(2y)^2 = (2x+2y)^2$
        \end{multicols}
        \end{enumerate}
        \end{example}
        There are two identities that help us factor cubes very easily.
        \begin{enumerate}
        \item $x^3+y^3 = (x+y)(x^2-xy+y^2)$ \hfill [Sum of Cubes]
        \item $x^3-y^3 = (x-y)(x^2+xy+y^2)$ \hfill [Difference of Cubes]
        \end{enumerate}
        \begin{example}[Examples of Factoring with Cubes]
        \
        \begin{enumerate}
        \begin{multicols}{2}
        \item $x^3+125 = (x+5)(x^2-5x+25)$
        \item $5x^3y - 40y^4 = 5y\big(x^3-8y^3) = 5y(x-2y)(x^2+2xy+4y^2)$
        \end{multicols}
        \end{enumerate}
        \end{example}
        \subsubsection{Solved Problems}
        \begin{enumerate}
        \begin{multicols}{2}
        \item $17x^2-51 = 17(x^2-3)$
        \item $21x^3-13x^2+56x = 7x(3x^2-2x+8)$
        \item $-3x^4+9x^2-6x^3 =-3x^2(x^2+2x-3)$
        \item $-13x^2-52 = -13(x^2+4)$
        \item $2x(x+2)+3(x+2) = (2x+3)(x+2)$
        \item $(x^2+3)3x+(x^2+3)2 = (3x+2)(x^2+3)$
        \item $5x(x-3)-2(x-3) = (5x-3)(x-3)$
        \item $3x(x^2+5) - 3(x^2+5) = 3(x-1)(x^2+5)$
        \item $-x^2 + 5x +15 = -(x-7)(x+2)$
        \item $x^2-4x-45 = (x-9)(x+5)$
        \item $x^2-9x+20 = (x-4)(x-5)$
        \item $3x^2-13x-10 = (3x+2)(x-5)$
        \item $6x^2+x-35 = (2x+5)(3x-7)$
        \item $15x^2-22x-48 = (3x-8)(5x+6)$
        \item $4x^2-25 = (2x+5)(2x-5)$
        \item $50x^2-72 = 2(25x^2-36) = 2(5x+6)(5x-6)$
        \item $8x^3-27 = (2x-3)(4x^2+6x+9)$
        \item $x^3+8 = (x+2)(x^2-2x+4)$
        \item $27x^3-64 = (3x-4)(9x^2+12x +16)$
        \item $x^2-1 = (x+1)(x-1)$
        \end{multicols}
        \end{enumerate}
        \subsubsection{Rational Expressions}
        \begin{definition}
        A rational expression is an expression that can be written as the quotient of two polynomials.
        \end{definition}
        \begin{example}[Examples of Rational Expressions]
        \
        \begin{enumerate}
        \begin{multicols}{4}
        \item $\frac{x^2+1}{x-1}$
        \item $\frac{xy+4}{x^2+x+1}$
        \item $\frac{x}{y}$
        \item $\frac{x+y}{x-y}$
        \end{multicols}
        \end{enumerate}
        \end{example}
        \begin{definition}
        The simplest form of a rational expression is an equivalent expression such that the numerator and denominator have no common factors.
        \end{definition}
        \begin{properties}[Fundamental Property of Rational Expressions]
        If $P,Q,$ and $R$ are polynomials, $Q,R \ne 0$, then:
        \begin{enumerate}
        \item $\frac{P\cdot R}{Q\cdot R} = \frac{P}{Q}$\hfill [Simplification of Rational Expressions]
        \end{enumerate}
        \end{properties}
        \begin{example}[Examples of Simplest Forms]
        \
        \begin{enumerate}
        \begin{multicols}{2}
        \item $\frac{x^2-1}{x+1} = \frac{(x+1)(x-1)}{x-1} = x+1$
        \item $\frac{x^3-y^3}{x^2+xy+y^2} = \frac{(x-y)(x^2+xy+y^2)}{x^2+xy+y^2} = x-y$
        \end{multicols}
        \end{enumerate}
        \end{example}
        \begin{example}
        $\frac{x^2-1}{x^2-3x+2} = \frac{(x+1)(x-1)}{(x-2)(x-1)} = \frac{x+1}{x-2}$
        \end{example}
        \begin{properties}[Multiplication of Rational Expressions]
        If $P,Q,R,$ and $S$ are polynomials, $Q,S\ne 0$, then:
        \begin{enumerate}
        \item $\frac{P}{Q}\cdot \frac{R}{S} = \frac{PR}{QS}$
        \end{enumerate}
        \end{properties}
        \begin{example}[Example of Multiplying Rational Expressions]
        \
        \begin{enumerate}
        \begin{multicols}{2}
        \item $\frac{2x+2}{3x-3x^2} \cdot \frac{3x^2-x-2}{9x^2-4} =\frac{-2(x+1)}{3x(3x-2}$
        \item $\frac{x+1}{x^2-y^2}\frac{x+y}{x+1} = \frac{1}{x-y}$
        \end{multicols}
        \end{enumerate}
        \end{example}
        \begin{properties}[Dividing Rational Expressions]
        If $P,Q,R,$ and $S$ are polynomials and $Q,R,S\ne 0$, then:
        \begin{enumerate}
        \item $\frac{P}{Q}\div \frac{R}{S} = \frac{\frac{P}{Q}}{\frac{R}{S}} = \frac{PS}{QR}$
        \end{enumerate}
        \end{properties}
        \begin{example}[Examples of Dividing Rational Expressions]
        \
        \begin{enumerate}
        \begin{multicols}{2}
        \item $\frac{\frac{x+y}{y^2+1}}{\frac{x-y}{y^2+1}} = \frac{x+y}{y^2+1}\cdot \frac{y^2+1}{x-y}= \frac{x+y}{x-y}$
        \item $\frac{\frac{xyz+y^2}{xy}}{\frac{y}{xy}} = \frac{xyz+y^2}{xy}\cdot \frac{xy}{y}= xz+y$
        \end{multicols}
        \end{enumerate}
        \end{example}
        \begin{properties}
        If $P,Q,R,$ and $S$ are polynomials and $Q,S\ne 0$, then:
        \begin{enumerate}
        \item $\frac{P}{Q} + \frac{R}{S} = \frac{PS+QR}{QS}$ \hfill [Sum of Rational Expressions]
        \item $\frac{P}{Q}-\frac{R}{S} = \frac{PS-QR}{QS}$\hfill [Difference of Rational Expressions]
        \end{enumerate}
        \end{properties}
        \begin{example}[Addition and Subtraction of Rational Expressions]
        \
        \begin{enumerate}
        \begin{multicols}{4}
        \item $\frac{x}{y} + \frac{z}{w} = \frac{xw+yz}{yw}$
        \item $\frac{x+1}{x^2} + \frac{x^2}{x-1} = \frac{x^2-1+x^2}{x^2(x-1)}$
        \item $\frac{2x+1}{x^2} + 1 = \frac{x^2+2x+1}{x^2}$
        \item $\frac{x}{y} - 1 = \frac{x-y}{y}$
        \end{multicols}
        \end{enumerate}
        \end{example}
        \begin{definition}
        A compound fraction is a fraction whose numerator and denominator are also fractions.
        \end{definition}
        \begin{properties}[Simplifying Compound Fractions]
        If $A,B,C,D,E,F,G,H$ are fractions, $B,D,F,H\ne 0$ and $\frac{E}{F}+\frac{G}{H} \ne 0$, then:
        \begin{enumerate}
        \item $\frac{\frac{A}{B}+\frac{C}{D}}{\frac{E}{F}+\frac{G}{H}} = \frac{FH(AD+BC)}{BD(EH+FG)}$ \hfill [Simplified Compound Fraction]
        \end{enumerate}
        \end{properties}
        \begin{remark}
        From this we see that all compound fractions are just normal fractions in disguise.
        \end{remark}
        \begin{example}
        Simplify $\frac{\frac{2}{3x}-\frac{3}{2}}{\frac{3}{4x}-\frac{3}{x^2}}$. We have $\frac{\frac{2}{3x}-\frac{3}{2}}{\frac{3}{4x}-\frac{3}{x^2}} = \frac{8x-18x^2}{9x-4} = -2x\frac{9x-4}{9x-4} = -2x$, so long as $9x-4 \ne 0$.
        \end{example}
        \begin{example}
        An electrical circuit with two resistors in parallel, with resistances $R_1$ and $R_2$, respectively, will have a total resistance $R$ which has the equation $\frac{1}{R} = \frac{1}{R_1}+\frac{1}{R_2}$. Thus $R = \frac{1}{\frac{1}{R_1}+ \frac{1}{R_2}} = \frac{R_1R_2}{R_1+R_2}$
        \end{example}
        \begin{enumerate}
        \begin{multicols}{4}
        \item $\frac{x-7}{-3x+21} = -\frac{1}{3}$
        \item $\frac{2x+6}{4x^2-8x} = \frac{x+3}{2x(x-2)}$
        \item $\frac{x-4}{7x-28} = \frac{1}{7}$
        \item $\frac{x^2-5x-14}{x^2+6x-7}$ is simplified.
        \item $\frac{x^2+3x-10}{x^2+x-6} = \frac{x+5}{x+3}$
        \item $\frac{x-7}{7-x} = -1$
        \item $\frac{x^2-3x-28}{49-x^2} = -\frac{x+4}{x+7}$
        \item $\frac{12x^3y^5}{4x^2y^{-4}} = 3xy^9$
        \item $\frac{7x+21}{63} = \frac{x+3}{9}$
        \item $\frac{x^2-4}{2-x} = -(x+2)$
        \item $\frac{x^3+8}{x^2-2x+4} = x+2$
        \item $\frac{12x^2-13x+3}{27x^3-1} = \frac{4x-3}{9x^2+3x+1}$
        \end{multicols}
        \begin{multicols}{2}
        \item $\frac{x^2-4x+4}{x^2-9}\cdot \frac{x^2-2x-3}{x^2-4} = \frac{(x-2)(x+1)}{(x+3)(x+2)}$
        \item $\frac{x^2+5x-24}{x^2-6x+9}\cdot \frac{x}{x^2-64} = \frac{x}{(x-3)(x-8)}$
        \end{multicols}
        \end{enumerate}
        \subsubsection{Radicals and Radical Expressions}
        We have already seen the definition of $a^{\frac{p}{q}}$ for positive real numbers $a$, and $p,q\in \mathbb{Z}, q\ne 0$. Now for some results on radicals.
        \begin{theorem}
        For all $x\in \mathbb{R}$, $\sqrt{x^2} = |x|$
        \end{theorem}
        \begin{example}
        $\sqrt{169x^2} = 13|x|$
        \end{example}
        \begin{theorem}
        For all real numbers $x\in \mathbb{R}$, $\sqrt[3]{x^3} = x$
        \end{theorem}
        \begin{example}
        $\sqrt[3]{-8} = -2$
        \end{example}
        \begin{remark}
        The following is true:
        \begin{equation}
        \nonumber (x+y)^2 = x^2+2xy+y^2
        \end{equation}
        The following is \textbf{NOT TRUE}:
        \begin{equation}
        \nonumber (x+y)^2 = x^2 + y^2
        \end{equation}
        \textbf{DO NOT WRITE THIS, YOU WILL BE WRONG}.
        Similarly, the following is true:
        \begin{equation}
        \nonumber \sqrt{(x+y)^2} = |x+y|
        \end{equation}
        The following is \textbf{NOT TRUE}:
        \begin{equation}
        \nonumber \sqrt{x^2+y^2} = |x|+|y|
        \end{equation}
        Again, \textbf{DO NO WRITE THIS, YOU WILL BE WRONG}.
        Finally, the following is \textbf{NOT TRUE}:
        \begin{equation}
        \nonumber \sqrt{x+y} = \sqrt{x}+\sqrt{y}
        \end{equation}
        \textbf{DO NOT WRITE THIS, YOU WILL BE WRONG}.
        \end{remark}
        Recall that, for positive real numbers $x$, and $p,q\in \mathbb{Z},q\ne 0$, $x^{\frac{p}{q}}$ is the unique positive real number such that $(x^{\frac{p}{q}})^q = x^p$. In other words, $x^{\frac{p}{q}}=\sqrt[q]{x^p}$. Or equivalently, $x^{\frac{p}{q}} = \big(\sqrt[q]{x}\big)^p$
        \begin{properties}
        If $a,b$ are positive real numbers, $b\ne0$, and $n,m\in \mathbb{Z}, m\ne 0$, then:
        \begin{enumerate}
        \item $\sqrt[n]{a\cdot b} = \sqrt[n]{a}\cdot \sqrt[n]{b}$
        \item $\sqrt[n]{\frac{a}{b}} = \frac{\sqrt[n]{a}}{\sqrt[n]{b}}$
        \end{enumerate}
        \end{properties}
        \begin{remark}
        Again, $\sqrt[n]{a+b} \ne \sqrt[n]{a}+\sqrt[n]{b}$. Do not make the mistake that equality holds here.
        \end{remark}
        To add, subtract, multiply, and divide with radicals, it is often useful to simplify the radical and then treat the remaining part as a variable. For example, consider $\sqrt{8} - \sqrt{2}$. We know that $8 = 4\cdot 2$, so $\sqrt{8} = \sqrt{4\cdot 2}$. But $\sqrt{4\cdot 2} = \sqrt{4}\cdot \sqrt{2}$. And we know $\sqrt{4} = 2$. So we have that $\sqrt{8} = 2\sqrt{2}$. Returning to the original expression, $\sqrt{8} - \sqrt{2} = 2\sqrt{2} - \sqrt{2}$. Factoring out the $\sqrt{2}$ (Like a variable) gives use $\sqrt{2}(2-1) = \sqrt{2}$. So $\sqrt{8}- \sqrt{2} = \sqrt{2}$/
        \begin{example}[Examples of Arithmetic with Radicals]
        \
        \begin{enumerate}
        \begin{multicols}{2}
        \item $\sqrt{50} - \sqrt{8} = 5\sqrt{2}- 2\sqrt{2} = 3\sqrt{2}$
        \item $\sqrt{54} - \sqrt{18} = 3\sqrt{2\cdot 3} - 3\sqrt{2}= 3\sqrt{2}(\sqrt{3}-1)$
        \end{multicols}
        \end{enumerate}
        \end{example}
        One application is that of Pythagoras' Theorem, which may be the most important theorem in all of mathematics. A right triangle is one where the largest angle in the triangle is $90^{\circ}$, or $\frac{\pi}{2}$ radians. The longest side of such a triangle is called the hypotenuse, and the other two sides are called the legs. If a right triangle has legs of lengths $a$ and $b$, and a hypotenuse of length $c$, then Pythagoras' Theorem says that $a^2+b^2 = c^2$. This can be used to find the hypotenuse if we only know the lengths of the legs. Since the length of the hypotenuse is a positive real number, if the lengths of the legs are $a$ and $b$, then $c = \sqrt{a^2+b^2}$. 
        \begin{example}
        A right angle triangle has one leg with length $3$ meters, and another with length $4$ meters. What is the length of the hypotenuse? Well, $c = \sqrt{(3)^2+(4)^2} = \sqrt{9+16} = \sqrt{25} = 5$.
        \end{example}
        \begin{remark}
        If the denominator of an expression contains radicals, it is possible to create a rationalize equivalent expression (One without a radical in the denominator). Given an expression $A+\sqrt{B}$, where $A$ and $B$ are algebraic expressions, we note that $(\sqrt{A}+\sqrt{B})(\sqrt{A}-\sqrt{B}) = A-B$. 
        \end{remark}
        \begin{example}
        Simplify:
        \begin{enumerate}
        \begin{multicols}{2}
        \item $\sqrt{2}{2-\sqrt{6}} = \frac{2(2-\sqrt{6}}{(2-\sqrt{6})(2+\sqrt{6})} = \frac{1}{5}(2-\sqrt{6})$
        \item $\frac{1}{\sqrt{2}-\sqrt{6}} = \frac{\sqrt{2}+\sqrt{6}}{(\sqrt{2}+\sqrt{6})(\sqrt{2}-\sqrt{6})} = \frac{\sqrt{2}+\sqrt{6}}{10}$
        \end{multicols}
        \end{enumerate}
        \end{example}
        \subsubsection{Solved Problems}
        \begin{enumerate}
        \begin{multicols}{4}
        \item $12\sqrt{72} - 9\sqrt{98} = 9\sqrt{2}$
        \item $8\sqrt{48} - 3\sqrt{108} = 14\sqrt{3}$
        \item $7\sqrt{18x} - \sqrt{50x} = 16\sqrt{2x}$
        \item $2\sqrt{28x} - 3\sqrt{63x} = -5\sqrt{7x}$
        \end{multicols}
        \end{enumerate}
    \section{Equations and Inequalities}
        \subsubsection{Linear Equations}
        \begin{definition}
        A family of equations is a set of equations that share a common characteristic.
        \end{definition}
        \begin{example}
        The set of all quadratic polynomials in one variable is a family of equations. So is the set of all polynomials in three variables.
        \end{example}
        We wish to study the family of linear equations of one variable. These have the form $y=mx+b$.
        \begin{definition}
        An equations is a statement that two expressions are equal.
        \end{definition}
        We've already seen plenty of equations in the preliminaries section. There are many arithmetic properties that equations have.
        \begin{properties}[Properties of Equality]
        If $A,B,$ and $C$ are algebraic expressions, then:
        \begin{enumerate}
        \item If $A=B$, then $A+C=B+C$ for all $C$. \hfill [Additive Property of Equality]
        \item If $A=B$, then $A\cdot C = B\cdot C$ \hfill [Multiplicative Property of Equality]
        \item If $C \ne 0$, and $A=B$, then $\frac{A}{C} = \frac{B}{C}$ \hfill [Division Property of Equality]
        \item If $C\ne 0$, and $A\cdot C = B\cdot C$, then $A=B$. \hfill [Cancellation Law of Equality]
        \end{enumerate}
        \end{properties}
        \begin{remark}
        Note that $A\cdot C = B\cdot C$ is not enough to say that $A=B$. If $C = 0$, then $2\cdot 0 = 0$ and $5\cdot 0 = 0$, so $2\cdot 0 = 5\cdot 0$, but $2\ne 5$. If $C = 0$, then we can say that $A=B$.
        \end{remark}
        \begin{example}
        Solve for $x$: $3(x-1) +x = -x+7$. We have $3x-3+x = 4x-3 = -x+7$. Adding $x$ to both sides, $\big(4x-3)+x = \big(-x+7)+x$. So $5x-3 = 7$. Adding $3$ to both sides, $\big(5x-3\big)+3 = 7+3$, so $5x = 10$. Dividing by $5$, we get $x = \frac{10}{5} = 2$.
        \end{example}
        \begin{example}
        Solve for for $n$: $\frac{1}{4}\big(n+8\big) = \frac{1}{2}\big(n-6\big)$: Multiplying both sides by $4$, we have $(n+8)-8 = 2(n-6)$, so $n = 2n-12$. Subtracting $2n$ from both sides, we get $-n = -12$. Multiplying both sides by $-1$, we get $n=12$.
        \end{example}
        The two previous examples are examples of conditional equations.
        \begin{definition}
        A conditional equation is an equation that is only true for certain values of the variables involved in the equation.
        \end{definition}
        \begin{definition}
        An identity is an equation that is true of all values of the variables in the equation.
        \end{definition}
        \begin{example}
        $2x + 4 = 2(x+2)$ is an identity. It is true of all values of $x$.
        \end{example}
        \begin{example}
        $\sqrt{x^2}=|x|$ is an identity. It is true of all values of $x$.
        \end{example}
        \begin{definition}
        A contradiction is an equation that is never true, regardless of the value the variables may take.
        \end{definition}
        \begin{example}
        $x=x+1$ is a contradiction. For $x = x+1$ implies $0=1$, which is false.
        \end{example}
        \begin{example}
        $2x - 5 = 2x$ is a contradiction. This would imply $0=5$, which is false.
        \end{example}
        \begin{definition}
        A literal equation is an equation involving more than one variable.
        \end{definition}
        \begin{example}
        The ideal gas law states that $PV=nRT$, where $P$ is the pressure, $V$ is the volume, $n$ is the number of moles in the system, and $T$ is the temperature. $R$ is a constant known as the universal gas constant. This is a literal equation in $4$ variables ($R$ is not a variable, it is a set constant). Solving for $P$, we get $P = \frac{nRT}{V}$.
        \end{example}
        \begin{example}
        The amount of money $A$ earned from a simple interest model is $A = A_0+A_0Rt$, where $A_0$ is the initial deposit, $R$ is the
        interest rate, and $t$ is the time elapsed since the deposit. Solve for $A_0$ in terms of $A,R,$ and $t$.
        $A_{0}(1+Rt)=A$, so $A_{0}=\frac{A}{1+Rt}$
        \end{example}
        \begin{theorem}[General Solution to a Linear Equation]
        If $a,b,c,d\in \mathbb{R}$, $a,c, \ne 0$, then:
        \begin{enumerate}
        \item If $ax+b=0$, then $x=-\frac{b}{a}$
        \item If $ax+b=d$, then $x=\frac{c-b}{a}$
        \item If $ax+b=cx+d$, and $a\ne c$, then $x =\frac{d-b}{a-c}$
        \end{enumerate}
        \end{theorem}
        \begin{remark}
        $ax+b=c$ is a linear equation in $x$. That is, it does not involve higher powers of $x$ ($x^{2},x^{3}$, etc.). It is a literal
        equation, as $a$, $b$ and $c$ can be any real number (So long as $a\ne 0$), but we only care about solutions for $x$.
        \end{remark}
        \begin{example}
            Give a formula for the sum of three consecutive integers. Let $n$ be the smallest integer. The next integer is $n+1$, and the one
            thereafter if $n+2$. We are summing them, so we have $n+(n+1)+(n+2)$. From the associative and commutative properties of addition,
            we have $(n+n+n)+(1+2)=3n+3=3(n+1)$. So the sum of three consecutive integers, starting at $n$, is $3(n+1)$.
        \end{example}
        \subsubsection{Solved Problems}
        \begin{enumerate}
            \begin{multicols}{2}
                \item $4x+3(x-2)=18-x:x=3$
                \item $15-2x=9-4(x+1):x=-5$
                \item $21-(2x+17)=-(7-x):x=-11$
                \item $12+5x=9+(6x+7):x=-4$
                \item $-3(4x+5)=-15x-20+3x:$ Contradiction.
                \item $5x-9-2=-5(2-x)-1:$ Identity
                \item $8-8(3x+5)=-5+6(x+1):x=-\frac{11}{10}$
                \item $-4(4x+5)=-6-2(8x+7):$ Identity
            \end{multicols}
        \end{enumerate}
        \subsubsection{Linear Inequalities in One Variable}
        \begin{definition}
        A linear inequality in one variable is an inequality involving two linear expressions in one variable.
        \end{definition}
        \begin{definition}
        The solution set to an inequality is the set of real numbers that satisfy the inequality.
        \end{definition}
        \begin{example}
        $x +1 > 5$ has the solution set $\{x:x>4\}$. That is, all real numbers greater than $4$.
        \end{example}
        \begin{notation}
        The following notations are used to represents intervals of the real line. Let $a,b\in\mathbb{R}$, $a<b$:
            \begin{enumerate}
                \item $(a,\infty)=\{x:a<x<\infty\}$\hfill[Open Right-Half Line
                \item $[a,\infty)=\{x:a\leq x<\infty\}$\hfill[Closed Right-Half Line]
                \item $(a,b)=\{x:a<x<b\}$\hfill[Open Interval]
                \item $[a,b)=\{x:a\leq x<b\}$\hfill[Right Semi-Open Interval]
                \item $(a,b]=\{x:a<x\leq b\}$\hfill [Left Semi-Open Interval]
                \item $[a,b]=\{x:a\leq x\leq b\}$\hfill[Closed Interval]
                \item $(-\infty,b]=\{x:x\leq b\}$\hfill[Closed Left-Half Line]
                \item $(-\infty,b)=\{x:x<b\}$\hfill[Open Left-Half Line]
            \end{enumerate}
        \end{notation}
        \begin{properties}[Properties of Inequalities]
        If $a<b$, $c\in\mathbb{R}$, then the following are true:
        \begin{enumerate}
            \item $a+b<b+c$ \hfill [Additive Property of Inequalities]
            \item If $c>0$, then $a\cdot c<b\cdot c$ \hfill [Multiplicative Property of Inequalities]
            \item $-b<-a$ \hfill [Negation Property of Inequalities]
        \end{enumerate}
        \end{properties}
        \begin{example}
        $1<2$, and $1-5=-4<2-5=-3$. Also $\pi>0$, and thus $\pi\cdot 1<\pi\cdot 2$. We know that $3<4$, but multiplying
        by $-1$ flips the inequality, and we get $-4<-3$ (Or $-3>-4)$.
        \end{example}
        \begin{definition}
        A compound inequality is an inequality where the solution set is multiple intervals.
        \end{definition}
        \begin{example}
        $-3x-1<-4$ or $4x+3<-6$. This has the requirement that $x>1$ or $x< -\frac{9}{4}$. To write this in interval notation,
        we do $(-\infty,-\frac{9}{4})\cup (1,\infty)$
        \end{example}
        \begin{example}
        $3x+5>-13$ and $3x+5<-1$. This has the requirement that $x>-6$ and $x<-2$. In interval notation,
        we have $(-6,\infty)\cap (-\infty,-2)=(-6,-2)$.
        \end{example}
        \begin{example}
        Solve $-6\leq\frac{2x+5}{-3}<1$. Multiplying by $-3$, we must flip the inequality to get $-3<2x+5\leq 18$.
        Subtracting $5$, we have $-8<2x\leq 13$. Dividing by $2$, we have $-4 <x \leq\frac{13}{2}$. In interval notation,
        that $(-4,\frac{13}{2}]$.
        \end{example}
        \begin{definition}
        The domain of an expression is the set of values for which the expression is well defined.
        \end{definition}
        \begin{example}
        $\frac{1}{x}$ is undefined at $0$, so its domain is $(-\infty, 0)\cup (0,\infty)$.
        \end{example}
        \begin{remark}
        Remember from interval notation that $(-\infty,0)\cup (0,\infty)$ does not include the number $0$. It includes every
        real number except for $0$.
        \end{remark}
        \subsubsection{Solved Problems}
        Determine the domain of the following expressions in interval notation:
        \begin{enumerate}
            \begin{multicols}{2}
                \item $\frac{12}{x}:(-\infty,0)\cup(0,\infty)$
                \item $\frac{5}{x+7}:(-\infty,-7)\cup(-7,\infty)$
                \item $\frac{1}{x-7}:(-\infty,7)\cup(7,\infty)$
                \item $\frac{4}{x-3}:(-\infty,3)\cup(3,\infty)$
            \end{multicols}
        \end{enumerate}
        \subsubsection{Absolute Value Equations and Inequalities}
        Inequalities can involve the absolute value of a variable. When this happens, we must be careful when solving for the solution set.
        \begin{properties}[Properties of Absolute Value Inequalities]
        If $A$ and $B$ are algebraic expressions and $a>0$, then:
            \begin{enumerate}
                \item $|A|=a$ if and only if either $A=a$, or $A=-a$.
                \item $|A|>a$ if and only if either $A>a$ or $-A>a$
                \item $|A|<a$ if and only if $-a<A<a$
                \item $|A\cdot B|=|A|\cdot|B|$
            \end{enumerate}
        \end{properties}
        \begin{example}[Examples of Absolute Values in Inequalities]
        \
            \begin{enumerate}
                \item $-5|x-7|+2=-13$ is equivalent to $|x-7|=-15$, and so $x-7=3$ or $x-7=-3$. The solution set is $\{4,10\}$.
                \item $|5-2x|=7$ implies either $5-2x=7$, or $5-2x=-7$. The solution set is $\{-1,6\}$.
                \item $|x|<7$ implies $-7<x<7$. The solution set is the interval $(-7,7)$.
                \item $|x-2|<7$ implies $-5<x<9$. The solution set is $(-5,9)$.
                \item $|x+1|>4$ implies $x+1>4$ or $x+1<-4$. The solution set is $(-\infty, -5)\cup (3,\infty)$
            \end{enumerate}
        \end{example}
        \subsubsection{Solved Problems}
        Find the solution sets. Write the answer in set notation or using interval notation.
        \begin{enumerate}
            \begin{multicols}{3}
                \item $2|x-1|-7=3:\{-4,6\}$
                \item $3|x-5|-14=-2:\{1,9\}$
                \item $-3|x+5|+6=-15:\{2,-12\}$
                \item $|x|=1:\{-1,1\}$
                \item $|x-2|\leq 7:[-5,9]$
                \item $|x-2|<7:(-5,9)$
                \item $5|x-2|-7\leq 8:[-1,5]$
                \item $-|x| > 2:\emptyset$
                \item $-|x|<1:\mathbb{R}$
            \end{multicols}
        \end{enumerate}
        \subsubsection{Complex Numbers}
        There is no real number $x\in \mathbb{R}$ such that $x^2 = -1$. That's because the square of a real number is either zero, or
        positive. In fact, $0$ is the only solution to $x^2 = 0$. Thus, every other real number has that property that its square is
        positive. Solving the equation $x^2+1 = 0$ doesn't make sense in the realm of real numbers, and up until now we'd simply say
        there is no solution. That is, there is no such thing is $\sqrt{-1}$. To solve this problem, we introduce imaginary and
        complex numbers.
        \begin{definition}
        The imaginary unit $i$ is a number such that $i^{2}=-1$.
        \end{definition}
        \begin{remark}
        Note that the imaginary unit is not a real number (Hence the name). It is a part of the larger complex numbers.
        \end{remark}
        \begin{notation}
        If $r$ is a positive real number, we write $\sqrt{-r}=i\sqrt{r}$.
        \end{notation}
        \begin{remark}
        Note that if $r$ is a negative real number, then $\sqrt{-r}=\sqrt{|r|}$, which is a real number.
        \end{remark}
        \begin{definition}
        The principle square root of a negative real number $r$, is the complex number $i\sqrt{|r|}$.
        \end{definition}
        \begin{definition}
        A complex number is a sum $a+ib$, where $a$ and $b$ are real numbers, and $i$ is the imaginary unit.
        \end{definition}
        \begin{definition}
        The set of all complex number, denoted $\mathbb{C}$, is the set $\{a+ib:a,b\in \mathbb{R}\}$.
        \end{definition}
        \begin{example}[Examples of Complex Numbers]
        \
        \begin{enumerate}
            \begin{multicols}{4}
                \item $4+\sqrt{-49}=4+7i$
                \item $1-\sqrt{-1}=1-i$
                \item $25+\sqrt{-16}=25+4i$
                \item $1-\sqrt{-100}=1-10i$
            \end{multicols}
        \end{enumerate}
        \end{example}
        \begin{remark}
        Note that from the definition of complex numbers, $\mathbb{R}\subset \mathbb{C}$.
        \end{remark}
        The sum and difference of complex numbers is computed using the distributive property to group purely imaginary and purely
        real components together. We multiply using the distributive property and the fact that $i^{2}=-1$.
        \begin{example}[Examples of Complex Arithmetic]
        \
        \begin{enumerate}
            \begin{multicols}{4}
                \item $(2+3i)+(1+i)=3+4i$
                \item $(1+i)+(1-i)=2$
                \item $(1+i)(1-i)=1-i^{2}=1-(-1)=2$
                \item $i(2+3i)=-3+2i$
            \end{multicols}
        \end{enumerate}
        \end{example}
        \begin{remark}
        The powers of $i$ go in a cycle:
        \begin{enumerate}
            \begin{multicols}{4}
                \item $i^{1}=i$
                \item $i^{2}=-1$ 
                \item $i^{3}=i\cdot i^{2}=-i$
                \item $i^{4}=i^{2}\cdot i^{2}=1$
                \item $i^{5}=i^{4}\cdot i=i$
                \item $i^{6}=i^{4}\cdot i^{2}=-1$
                \item $i^{7}=i^{4}\cdot i^{3}=-i$
                \item $i^{8}=i^{4}\cdot i^{4}=1$
            \end{multicols}
        \end{enumerate}
        \end{remark}
        \begin{example}[Simplifying Powers of $i$]
        \
        \begin{enumerate}
            \begin{multicols}{3}
                \item $i^{22}=\big(i^{4}\big)^{5}\cdot i^{2}=1\cdot i^{2}=-1$
                \item $i^{57}=\big(i^{4}\big)^{14}\cdot i=1\cdot i=i$
                \item $i^{75}=\big(i^{4}\big)^{18}\cdot i^{3}=1\cdot i^{3}=i^{3}$
            \end{multicols}
        \end{enumerate}
        \end{example}
        \begin{definition}
        The complex conjugate of a complex number $z=a+ib$, denoted $\overline{z}$, is the complex number $\overline{z} = a-ib$.
        \end{definition}
        \begin{remark}
        For a real number $z$, $\overline{z}=z$. This is because if $z=a+ib$ is real, then $b=0$.
        \end{remark}
        \begin{theorem}[The Complex Conjugate Theorem]
        If $z=a+ib$ is a complex number, then $z\cdot\overline{z}=a^{2}+b^{2}$
        \end{theorem}
        \begin{proof}
        For $z\cdot\overline{z}=(a+ib)(a-ib)=a^{2}+iab-iab-(ib)^{2}=a^{2}-(i)^{2}b^{2}=a^{2}+b^{2}$
        \end{proof}
        While $x^{2}+y^{2}$ cannot be simplified, in general, over the real numbers, it can by factored over the complex numbers.
        If $x$ and $y$ are real numbers, then $x^{2}+y^{2}=(x+iy)(x-iy)$. This is just an application of the complex conjugate theorem.
        We define division the following way: If $z=a+ib$, $w=c+id$, where $c,d\ne 0$, then
        $\frac{z}{w}=\frac{z\cdot\overline{w}}{w\cdot\overline{w}}=\frac{(ac-bd)+i(ad+cb)}{c^{2}+d^{2}}$
        \begin{example}[Examples of Division by Complex Numbers]
        \
        \begin{enumerate}
            \begin{multicols}{3}
                \item $\frac{2}{5-i}=\frac{5+i}{13}$
                \item $\frac{3-i}{2+i}=1-i$
                \item $\frac{6+6i}{3+3i}=2$
            \end{multicols}
        \end{enumerate}
        \end{example}
        \subsubsection{Solved Problems}
        \begin{enumerate}
            \begin{multicols}{3}
                \item $\sqrt{-16}=4i$
                \item $\sqrt{27}=3\sqrt{3}$
                \item $\sqrt{-81}=9i$
                \item $-\sqrt{-64}=-8i$
                \item $\sqrt{-49}=7i$
                \item $\sqrt{|-25|}=5$
                \item $\sqrt{-17}=i\sqrt{17}$
                \item $\sqrt{-\frac{9}{16}}=\frac{3}{4}i$
                \item $(12-2i)+(7+3i)=19+i$
                \item $(14+i)-(7+3i)=7-2i$
                \item $5+(1-i)=6-i$
                \item $(2+2i)+(-5-i)=-3+i$
                \item $(1+i)(2+i)=1+3i$
                \item $(4+i)(4-i)=17$
                \item $i(6-17i)=17+6i$
                \item $\frac{1+i}{1-i}=i$
                \item $i^{2}7=i^{3}$
                \item $i^{81}=i$
            \end{multicols}
        \end{enumerate}
        \subsubsection{Solving Quadratic Equations}
        \begin{definition}
        A quadratic equation is an equation of the form $ax^{2}+bx+c=0$, where $a,b,c\in\mathbb{R}$, and $a\ne 0$.
        \end{definition}
        \begin{remark}
        All quadratic equations have degree $2$. The highest power that occurs is $x^{2}$, hence they are degree $2$.
        \end{remark}
        Quadratic equations often have two solutions, whereas linear equations have only one. There is a property of real numbers
        that allows us to solve quadratic equations, called the Euclidean Property of Real Numbers.
        \begin{properties}[The Euclidean Property of Real Numbers]
        \
        \begin{enumerate}
        \item If $a,b\in \mathbb{R}$ and $a\cdot b = 0$, then either $a=0$ or $b=0$, or both.
        \end{enumerate}
        \end{properties}
        \begin{remark}
        This says that if the product of two numbers is zero, one of these numbers must be zero. It is named after Euclid of Alexandria,
        one of the greatest mathematicians to ever live, who proved this in the context of geometry around $300$ B.C.
        \end{remark}
        \begin{example}
        Solve $x^{2}=x$. We have that $x^{2}-x=0$, and thus $x(x-1)=0$. From the Euclidean property, either $x=0$ or $x-1=0$.
        Thus, the solutions are $x=0$ and $x=1$.
        \end{example}
        \subsubsection{Completing the Square}
        Recall that $(x+y)^2 = x^2+2xy+y^2$. Given $ax^2+bx+c$, we can reverse this process to obtain a simplified version. This process is called completing the square. Note that $ax^2+bx = a\big(x^2+\frac{b}{a}x\big) = a\big[\big(x+\frac{b}{2a}\big)^2-\frac{b^2}{4a^2}\big]$. So, $ax^2+bx+c = a\big[\big(x+\frac{b}{2a}\big)^2 - \frac{b^2}{4a^2}\big] + c$. Solving for $ax^2+bx+c=0$ is thus equivalent to solving $a\big[\big(x+\frac{b}{2a}\big)^2-\frac{b^2}{4a^2}\big]+c=0$. We get $\big(x+\frac{b}{2a}\big)^2=\frac{b^2}{4a^2} -\frac{c}{a} = \frac{b^2-4ac}{4a^2}$. Taking square roots, we have $x+\frac{b}{2a} = \pm\frac{\sqrt{b^2 - 4ac}}{2a}$, where $\pm$ means 'Plus or Minus.' Meaning both the $+$ symbol in front or the $-$ symbol in front yield correct answers. Finally, we get $x = \frac{-b\pm \sqrt{b^2-4ac}}{2a}$. With this, we have solved every quadratic equation possible (Remember for it to be quadratic $a$ cannot equal $0$).
        \begin{theorem}[The Quadratic Formula]
            If $a,b,c\in\mathbb{R}$, $a\ne 0$, and $ax^{2}+bx+c=0$, then the solution set is
            $\{\frac{-b-\sqrt{b^{2}-4ac}}{2a},\frac{-b+\sqrt{b^{2}-4ac}}{2a}\}$
        \end{theorem}
        \begin{remark}
        It is possible that the solution to a quadratic is complex or purely imaginary. This occurs when $b^{2}-4ac<0$ 
        \end{remark}
        \begin{definition}
        The discriminant of a quadratic $ax^{2}+bx+c$ is the number $b^{2}-4ac$
        \end{definition}
        The discriminant has a useful property.
        \begin{theorem}
            If $a,b,c\in\mathbb{R}$, $a\ne 0$, and $ax^{2}+bx+c=0$, then:
            \begin{enumerate}
                \item There are no real solutions is $b^{2}-4ac<0$
                \item There is one real solution if $b^{2}-4ac=0$
                \item There are two real solutions if $b^{2}-4ac>0$
            \end{enumerate}
        \end{theorem}
        By considering the discriminant, we can determine how many real solutions there are without having to solve the quadratic.
        \subsubsection{Order of Operations}
            We start with the fundamental properties of
            addition, multiplication, and exponentiation.
            \begin{properties}[Arithmetic Properties]
                \label{property:North_Shore_Arithmetic_Properties}
                \
                \begin{enumerate}
                    \item
                        \label{%
                            property:%
                            North_Shore_Arithmetic_Properties_%
                            Com_Add%
                        }
                        $a+b=b+a$\hfill
                        [Commutativity of Addition]
                    \item
                        \label{
                            property:%
                            north_shore_arithmetic_properties_%
                            assoc_add%
                        }
                        $a+(b+c)=(a+b)+c$\hfill
                        [Associativity of Addition]
                    \item
                        \label{%
                            property:%
                            north_shore_arithmetic_properties_%
                            comm_mult%
                        }
                        ${a}\cdot{b}={b}\cdot{a}$\hfill
                        [Commutativity of Multiplication]
                    \item
                        \label{%
                            property:%
                            north_shore_arithmetic_properties_%
                            assoc_mult%
                        }
                        ${a}\cdot{({b}\cdot{c})}%
                         ={({a}\cdot{b})}\cdot{c}$\hfill
                        [Associativity of Multiplication]
                    \item
                        \label{%
                            property:%
                            north_shore_arithmetic_properties_%
                            add_identity
                        }
                        $a+0=a$\hfill%
                        [Identity Property of Addition]
                    \item
                        \label{%
                            property:%
                            north_shore_arithmetic_properties_%
                            mult_identity%
                        }
                        ${a}\cdot{1}=a$\hfill
                        [Identity Property of Multiplication]
                    \item
                        \label{%
                            property:%
                            north_shore_arithmetic_properties_%
                            add_inverse%
                        }
                        $a+(-a)=0$\hfill
                        [Inverse Property of Addition]
                    \item
                        \label{%
                            property:%
                            north_shore_arithmetic_properties_%
                            mult_inverse%
                        }
                        If ${a}\ne{0}$, then $\frac{a}{a}=1$\hfill
                        [Inverse Property of Multiplication]
                    \item
                        \label{%
                            property:%
                            north_shore_arithmetic_properties_%
                            distributive_property%
                        }
                        ${a}\cdot{(b+c)}%
                         ={a}\cdot{b}+{a}\cdot{c}$\hfill
                        [Distributive Property of
                         Multiplication over Addition]
                \end{enumerate}
            \end{properties}
            \begin{properties}[Properties of Exponents]
                \
                \label{property:North_Shore_Exponent_Rules}
                \begin{enumerate}
                    \item
                        \label{%
                            property:%
                            north_shore_distributive_property_%
                            of_expo%
                        }
                        $({x}\cdot{y})^{n}%
                         ={x^{n}}\cdot{y^{n}}$\hfill
                        [Distributive Property of Exponents]
                    \item
                        \label{%
                            property:%
                            north_shore_inverse_property_of_expo%
                        }
                        $x^{-n}=\frac{1}{x^n}$\hfill
                        [Inverse Property of Exponents]
                    \item
                        \label{%
                            property:%
                            north_shore_power_property_of_expo%
                        }
                        $(x^n)^{m}=x^{{n}\cdot{m}}$\hfill
                        [Power Property of Exponents]
                    \item
                        \label{%
                            property:%
                            north_shore_product_property_of_expo%
                        }
                        $x^{n}x^{m}=x^{n+m}$\hfill
                        [Multiplicative Property of Exponents]
                \end{enumerate}
            \end{properties}
            The order of operations \gls{pemdas} tells one
            how to simplify expressions.
            \begin{enumerate}
                \label{North_Shore_PEMDAS}
                \item \textbf{P}arenthesis.
                \item \textbf{E}xponents.
                \item \textbf{M}ultiplication or \textbf{D}ivision, 
                    in the order they appear from left to right.
                \item \textbf{A}ddition or \textbf{S}ubtraction,
                    in the order they appear from left to right.
            \end{enumerate}
            \begin{fexample}{}{}
                \begin{enumerate}
                    \item ${2}\cdot{3}+{4}\cdot{5}%
                           \overset{\textrm{\tiny{M}}}{=}%
                           6+20\overset{\textrm{\tiny{A}}}{=}%
                           \boxed{26}$
                    \item ${2}\cdot{3}%
                           +4\overset{\textrm{\tiny{M}}}{=}%
                           6+4\overset{\textrm{\tiny{A}}}{=}%
                           \boxed{10}$
                    \item ${3}\cdot{3+2^{4}}%
                           \overset{\textrm{\tiny{E}}}{=}%
                           {3}\cdot {3}+16%
                           \overset{\textrm{\tiny{M}}}{=}%
                           9+16\overset{\textrm{\tiny{A}}}{=}%
                           \boxed{25}$
                    \item $(4+1)^2-17\cdot 3%
                        \overset{\textrm{\tiny{P}}}{=}%
                        5^{2}-17\cdot 3%
                        \overset{\textrm{\tiny{E}}}{=}25-17\cdot 3%
                        \overset{\textrm{\tiny{M}}}{=}25-51%
                        \overset{\textrm{\tiny{S}}}{=}\boxed{-26}$
                    \item ${(1+1)^{(2+3)}\cdot}{5}-%
                           {2}\cdot{3}
                           \overset{\textrm{\tiny{P}}}{=}%
                           {2^{5}}\cdot{5}-{2}\cdot{3}%
                           \overset{\textrm{\tiny{E}}}{=}%
                           {32}\cdot{5}-{2}\cdot{3}%
                           \overset{\textrm{\tiny{M}}}{=}%
                           160-6\overset{\textrm{\tiny{S}}}{=}%
                           \boxed{154}$
                    \item $(2+2)-(16-11)^{(1+1)}%
                           \overset{\textrm{\tiny{P}}}{=}4-5^{2}%
                           \overset{\textrm{\tiny{E}}}{=}4-25%
                           \overset{\textrm{\tiny{S}}}{=}%
                           \boxed{-21}$
                \end{enumerate}
            \end{fexample}
            Many problems involve radicals and exponents.
            These are defined by:
            \begin{align*}
                \sqrt{x}&=x^{\frac{1}{2}}
                &
                \sqrt[n]{x}&=x^{\frac{1}{n}}
                &
                \sqrt[n]{x^m}&=x^{\frac{m}{n}}
            \end{align*}
            \begin{fexample}{}{}
                \begin{enumerate}
                    \begin{multicols}{4}
                        \item $\sqrt{x}=x^{\frac{1}{2}}$
                        \item $\sqrt[3]{x}=x^{\frac{1}{3}}$
                        \item $\sqrt[5]{x}=x^{\frac{1}{5}}$
                        \item $\sqrt[27]{x}=x^{\frac{1}{27}}$
                        \item $\sqrt{x^3}=x^{\frac{3}{2}}$
                        \item $\sqrt[3]{x^2}=x^{\frac{2}{3}}$
                        \item $\sqrt[15]{x^3}=x^{\frac{1}{5}}$
                        \item $\sqrt[3]{x^3}=x$
                    \end{multicols}
                \end{enumerate}
            \end{fexample}
            There are several theorems that help one simplify
            equations involving radicals and exponents.
            \begin{ftheorem}{Radical Formulas}{North_Shore_Radical_Formulas}
                \begin{enumerate}
                    \item If $a$ and $b$ are \textbf{positive} real numbers,
                          then $\sqrt{{a}\cdot{b}}=\sqrt{a}\cdot\sqrt{b}$
                    \item If $x$ is \textbf{positive} and $i$ is the
                          imaginary unit $(i^{2}=-1)$, then $\sqrt{-x}=i\sqrt{x}$
                    \item If $n$ is an \textbf{odd} integer
                          and $x$ is a \textbf{real} number,
                          then $\sqrt[n]{-x}=-\sqrt[n]{x}$
                    \item If $x\ne{0}$, then $x^{0}=1$. $0^{0}$ is undefined.
                \end{enumerate}
            \end{ftheorem}
            \begin{fexample}{}{}
                \begin{enumerate}
                    \begin{multicols}{4}
                        \item[1.] $\sqrt{6}=\sqrt{2}\cdot\sqrt{3}$
                        \item[5.] $\sqrt{-7}=i\sqrt{7}$
                        \item[9.] $\sqrt[3]{-7}=-\sqrt[3]{7}$
                        \item[13.] $\pi^{0}=1$
                        \item[2.] $\sqrt{8}=2\sqrt{2}$
                        \item[6.] $\sqrt{-9}=3i$
                        \item[10.] $\sqrt[17]{-1}=-1$
                        \item[14.] $(-5)^{0}=1$
                        \item[3.] $\sqrt{1,\!000}=10\sqrt{10}$
                        \item[7.] $\sqrt{-4}=2i$
                        \item[11.] $\sqrt[5]{-9}=-\sqrt[5]{9}$
                        \item[15.] $(\frac{1}{2})^{0}=1$
                        \item[4.] $\sqrt{50}=5\sqrt{2}$
                        \item[8.] $\sqrt{-2}=i\sqrt{2}$
                        \item[12.] $\sqrt[3]{-27}=-3$
                        \item[16.] $0^{0}$ is undefined.
                    \end{multicols}
                \end{enumerate}
            \end{fexample}
            Often times the real numbers are thought to line on
            a line called the \textit{number line}, or the
            \textit{real line}. It is often convenient to be able
            to talk about the \textit{size} of a number. That is,
            the distance between that number and zero on the number line.
            We define this using the \textit{absolute value} of a number.
            \begin{fdefinition}{Absolute Value}{North_Shore_Absolute_Value_Def}
                The absolute value of a real number $x$ is
                defined as:
                \begin{equation*}
                    |x|=\begin{cases}%
                        \phantom{-}x, & x\geq 0\\%
                        -x, & x<0%
                    \end{cases}
                \end{equation*}
            \end{fdefinition}
            \begin{fexample}{}{}
                \begin{enumerate}
                    \begin{multicols}{5}
                        \item[1.] $|3|=3$
                        \item[6.] $|24|=24$
                        \item[2.] $|-3|=3$
                        \item[7.] $|-24|=24$
                        \item[3.] $|\pi|=\pi$
                        \item[8.] $|\frac{1}{2}|=\frac{1}{2}$
                        \item[4.] $|-\pi|=\pi$
                        \item[9.] $|-\frac{1}{2}|=\frac{1}{2}$
                        \item[5.] $|0|=0$
                        \item[10.] $|-1|=1$
                    \end{multicols}
                \end{enumerate}
            \end{fexample}
            \begin{remark}
                \label{remark:North_Shore_Rational_Expressions}
                Expressions like $\frac{a+b}{c+d}$ should be treated
                as equivalent to ${(a+b)}\div{(c+d)}$
            \end{remark}
        \subsubsection{Problems}
            \begin{minipage}[t]{0.49\textwidth}
                \begin{problem}
                    Simplify $3^{2}+5-\sqrt{4}+4^{0}$
                \end{problem}
                \begin{fsolution}
                    \begin{align*}
                        3^{2}+5-\sqrt{4}+4^{0}
                        &=3^{2}+5-4^{\frac{1}{2}}+4^{0}\\[-0.5ex]
                        &=3^{2}+5-4^{\frac{1}{2}}+1\\
                        &=9+5-2+1\\
                        &=14-2+1\\
                        &=12+1\\
                        &=\boxed{13}
                    \end{align*}
                \end{fsolution}
            \end{minipage}
            \hfill
            \begin{minipage}[t]{0.49\textwidth}
                \begin{problem}
                    Simplify $(5+1)(4-2)-3$
                \end{problem}
                \begin{fsolution}
                    \begin{align*}
                        (5+1)(4-2)-3&=6\cdot 2-3\\
                        &=12-3\\
                        &=\boxed{9}\\
                        &\\
                        &\\
                        &
                    \end{align*}
                \end{fsolution}
            \end{minipage}
            \par\hfill\par\hfill\par
            \begin{minipage}[t]{0.49\textwidth}
                \begin{problem}
                    Simplify ${3}\cdot{7^{2}}$
                \end{problem}
                \begin{fsolution}
                    \begin{align*}
                        {3}\cdot{7^{2}}
                        &={3}\cdot{49}\\
                        &=\boxed{147}\\
                        &
                    \end{align*}
                \end{fsolution}
            \end{minipage}
            \hfill
            \begin{minipage}[t]{0.49\textwidth}
                \begin{problem}
                    Simplify ${2}\cdot{(7+3)^{2}}$
                \end{problem}
                \begin{fsolution}
                    \begin{align*}
                        {2}\cdot{(7+3)^{2}}
                        &={2}\cdot{10^{2}}\\
                        &={2}\cdot{100}\\
                        &=\boxed{200}
                    \end{align*}
                \end{fsolution}
            \end{minipage}
            \par\hfill\par\hfill\par
            \begin{minipage}[t]{0.49\textwidth}
                \begin{problem}
                    Simplify ${49}\div{7}-{2}\cdot{2}$
                \end{problem}
                \begin{fsolution}
                    \begin{align*}
                        {49}\div{7}-{2}\cdot{2}
                        &=7-{2}\cdot{2}\\
                        &=7-4\\
                        &=\boxed{3}\\
                        &
                    \end{align*}
                \end{fsolution}
            \end{minipage}
            \hfill
            \begin{minipage}[t]{0.49\textwidth}
                \begin{problem}
                    Simplify ${9}\div{3}\cdot{5}-{8}\div{2}+27$
                \end{problem}
                \begin{fsolution}
                    \begin{align*}
                        {9}\div{3}\cdot{5}-{8}\div{2}+27
                        &={3}\cdot{5}-4+27\\
                        &=15-4+27\\
                        &=11+27\\
                        &=\boxed{38}
                    \end{align*}
                \end{fsolution}
            \end{minipage}
            \par\hfill\par\hfill\par
            \begin{minipage}[t]{0.49\textwidth}
                \begin{problem}
                    Simplify $3+2(5)-|-7|$
                \end{problem}
                \begin{fsolution}
                    \begin{align*}
                        3+2(5)-|-7|
                        &=3+2(5)-7\\
                        &=3+10-7\\
                        &=13-7\\
                        &=\boxed{6}
                    \end{align*}
                \end{fsolution}
            \end{minipage}
            \hfill
            \begin{minipage}[t]{0.49\textwidth}
                \begin{problem}
                    Simplify $({5}\cdot{5}-4)\div(2^{2}-1)$
                \end{problem}
                \begin{fsolution}
                    \begin{align*}
                        (5\cdot{5}-4)\div(2^{2}-1)
                        &={({5}\cdot{5}-4)}\div{(4-1)}\\
                        &={(25-5)}\div{(4-1)}\\
                        &=21\div{3}\\
                        &=\boxed{7}
                    \end{align*}
                \end{fsolution}
            \end{minipage}
            \par\hfill\par\hfill\par
            \begin{minipage}[t]{0.49\textwidth}
                \begin{problem}
                    Simplify $(4^{2}-5^{2})\div{(4-5)^{2}}$
                \end{problem}
                \begin{fsolution}
                    \begin{align*}
                        (4^{2}-5^{2})\div{(4-5)^{2}}
                        &={(4^{2}-5^{2})}\div{((-1)^2)}\\
                        &={(16-25)}\div{(1)}\\
                        &={-9}\div{1}\\
                        &=\boxed{-9}
                    \end{align*}
                \end{fsolution}
            \end{minipage}
            \hfill
            \begin{minipage}[t]{0.49\textwidth}
                \begin{problem}
                    Simplify ${48}\div{2(3+9)}$
                \end{problem}
                \begin{fsolution}
                    \begin{align*}
                        {48}\div{2(3+9)}
                        &={48}\div{2(12)}\\
                        &=24(12)\\
                        &=\boxed{288}\\
                        &
                    \end{align*}
                \end{fsolution}
            \end{minipage}
            \par\hfill\par\hfill\par
            \begin{minipage}[t]{0.49\textwidth}
                \begin{problem}
                    Simplify ${6}\cdot{(3+2)^{2}}+14$
                \end{problem}
                \begin{fsolution}
                    \begin{align*}
                        {6}\cdot{(3+2)^{2}}+14
                        &={6}\cdot{5^{2}+14}\\
                        &={6}\cdot{25}+14\\
                        &=150+14&\\
                        &=\boxed{164}
                    \end{align*}
                \end{fsolution}
            \end{minipage}
            \hfill
            \begin{minipage}[t]{0.49\textwidth}
                \begin{problem}
                    Simplify $(5+1)^{(3-2)}$
                \end{problem}
                \begin{fsolution}
                    \begin{align*}
                        (5+1)^{(3-2)}&=6^{1}\\
                        &=\boxed{6}\\
                        &\\
                        &
                    \end{align*}
                \end{fsolution}
            \end{minipage}
    \subsection{Scientific Notation}
        Non-zero real numbers can be written as
        ${r}\times{10^{n}}$, where ${1}\leq{|r|}<10$
        and $n$ is an integer.
        \begin{fexample}{}{}
            \begin{enumerate}
                \begin{multicols}{3}
                    \item[1.] $12,\!345={1.2345}\times{10^{4}}$
                    \item[4.] $10={1}\times{10^{1}}$
                    \item[2.] $0.01={1}\times{10^{-2}}$
                    \item[5.] $124={1.24}\times{10^{2}}$
                    \item[3.] $36.24={3.624}\times{10^{1}}$
                    \item[6.] $0.000314={3.14}\times{10^{-4}}$
                \end{multicols}
            \end{enumerate}
            Several constants in chemistry/physics are
            written in scientific notation.
            \begin{enumerate}
                \item Planck's Constant:
                    $h={6.626}\times{10^{-34}}%
                    \textrm{J}\cdot\textrm{s}$
                \item Universal Gravitation Constant:
                    $G={6.67}\times{10^{-11}}%
                    \textrm{Nm}^{2}\textrm{kg}^{-2}$
                \item Avogadro's Number:
                    $N_{A}={6.0221}\times{10^{23}}%
                    \textrm{mol}^{-1}$
                \item Speed of Light:
                    $c={2.998}\times{10^{8}\textrm{ms}^{-1}}$
            \end{enumerate}
        \end{fexample}
        \subsubsection{Problems}
            \begin{problem}
                Write the following in scientific notation:
                \begin{enumerate}
                    \begin{multicols}{4}
                        \item $350,\!000,\!000$
                        \item $120,\!500,\!000,\!000$
                        \item $0.0000000523$
                        \item $10.01$
                    \end{multicols}
                \end{enumerate}
            \end{problem}
            \begin{solution}
                \
                \begin{enumerate}
                    \begin{multicols}{4}
                        \item ${3.5}\times{10^8}$
                        \item ${1.205}\times{10^{11}}$
                        \item ${5.23}\times{10^{-8}}$
                        \item ${1.001}\times{10^{1}}$
                    \end{multicols}
                \end{enumerate}
            \end{solution}
            \begin{problem}
                Write in expanded form:
                \begin{enumerate}
                    \begin{multicols}{3}
                        \item ${6.02}\times{10^{15}}$
                        \item ${3.0}\times{10^{8}}$
                        \item ${1.819}\times{10^{-9}}$
                    \end{multicols}
                \end{enumerate}
            \end{problem}
            \begin{solution}
                \
                \begin{enumerate}
                    \begin{multicols}{3}
                        \item $6,\!020,\!000,\!000,\!000,\!000$
                        \item $300,\!000,\!000$
                        \item $0.000000001819$
                    \end{multicols}
                \end{enumerate}
            \end{solution}
            \begin{problem}
                Simplify:
                \begin{enumerate}
                    \begin{multicols}{4}
                        \item $({3}\times{10^{3}})%
                               ({5}\times{10^{6}})$
                        \item $\frac{{6}\times{10^{9}}}%
                               {{3}\times{10^{4}}}$
                        \item $({3}\times{10^{-4}})^{2}$
                        \item 
                            $\frac{%
                                   ({3.2}\times{10^{5}})%
                                   ({2}\times{10^{-3}})%
                             }{%
                               {2}\times{10^{-5}}%
                            }$
                    \end{multicols}
                \end{enumerate}
            \end{problem}
            \begin{solution}
                \
                \begin{enumerate}
                    \item $(3\times 10^{3})(5\times 10^{6})=%
                        3\times 5\times 10^{3}\times 10^{6}=%
                        15\times 10^{6+3}=%
                        15\times 10^{9}=%
                        \boxed{1.5\times 10^{10}}$
                    \item $\frac{6\times 10^{9}}{3\times 10^{4}}=%
                        \frac{6}{3}\times\frac{10^{9}}{10^{4}}=%
                        2\times 10^{9-4}=%
                        \boxed{2\times 10^{5}}$
                    \item $(3\times 10^{-4})^{2}=%
                        3^{2}\times (10^{-4})^{2}=%
                        \boxed{9\times 10^{-8}}$
                    \item 
                        $\frac{(3.2\times 10^{5})(2\times 10^{-2})}%
                              {2\times 10^{-5}}%
                         =3.2\times 10^{5}\times%
                         \frac{2\times 10^{-3}}{2\times 10^{-5}}%
                         =3.2\times 10^{5}\times 10^{2}%
                         =\boxed{3.2\times 10^{7}}$
                \end{enumerate}
            \end{solution}
    \subsection{Substitution}
        \subsubsection{Problems}
            \begin{minipage}[t]{0.49\textwidth}
                \begin{problem}
                    Evaluate $xyz-4z$ at $(3,-4,2)$
                \end{problem}
                \begin{solution}
                    \begin{align*}
                        xyz-4z\Big|_{(3,-4,2)}
                        &=(3)(-4)(2)-4(2)\\
                        &=-24-8\\
                        &=\boxed{-32}
                    \end{align*}
                \end{solution}
            \end{minipage}
            \hfill
        \begin{minipage}[t]{0.49\textwidth}
            \begin{problem}
                Evaluate $2x-y$ at $(3,-4,2)$
            \end{problem}
            \begin{solution}
                \begin{align*}
                    2x-y\Big|_{(3,-4,2)}
                    &=2(3)-(-4)\\
                    &=6+4\\
                    &=\boxed{10}
                \end{align*}
            \end{solution}
        \end{minipage}
        \par\hfill\par\hfill\par
        \begin{minipage}[t]{0.49\textwidth}
            \begin{problem}
                \par\hfill\par
                Evaluate $x(y-3z)$ at $(x,y,z)=(3,-4,2)$
            \end{problem}
            \begin{solution}
                \begin{align*}
                    x(y-3z)\big|_{x=3,y=-4,z=2}
                    &=(3)((-4)-3(2))\\
                    &=3(-10)\\
                    &=\boxed{-30}
                \end{align*}
            \end{solution}
        \end{minipage}
        \hfill
        \begin{minipage}[t]{0.49\textwidth}
            \begin{problem}
                \par\hfill\par
                Evaluate $\frac{5x-z}{xy}$ at
                $(x,y,z)=(3,-4,2)$
            \end{problem}
            \begin{solution}
                \begin{align*}
                    \tfrac{5x-z}{xy}\big|_{x=3,y=-4,z=2}
                    &=\tfrac{5(3)-(2)}{(3)(-4)}\\
                    &=\tfrac{15-2}{-12}\\
                    &=\boxed{-\tfrac{13}{12}}
                \end{align*}
            \end{solution}
        \end{minipage}
        \par\hfill\par
            \begin{problem}
                Solve $3y^{2}-2x+4z$ for $x=3,y=-4,z=2$.
            \end{problem}
            \begin{proof}[Solution]
                \begin{flalign*}
                    3y^{2}-2x+4z\big|_{x=3,y=-4,z=2}
                    &=3(-4)^{2}-2(3)+4(2)&\tag{Substitution}\\
                    &=3(16)-2(3)+4(2)&\tag{Exponentiation}\\
                    &=48-6+8&\tag{Multiplication}\\
                    &=\boxed{50}&\tag{Addition/Subtraction}
                \end{flalign*}
            \end{proof}
            \begin{problem}
                Solve $x+y+z$ for $x=1,y=2,z=3$.
            \end{problem}
            \begin{proof}[Solution]
                \begin{flalign*}
                    x+y+z\big|_{x=1,y=2,z=3}
                    &=(1)+(2)+(3)&\tag{Substitution}\\
                    &=\boxed{6}&\tag{Addition}
                \end{flalign*}
            \end{proof}
            \begin{problem}
                Solve $(x+1)(y-2)$ for $x=1,y=2,z=3$
            \end{problem}
            \begin{proof}[Solution]
                \begin{flalign*}
                    (x+1)(y-2)\big|_{x=1,y=2,z=3}
                    &=((1)+1)((2)-2)&\tag{Substitution}\\
                    &=2\cdot 0&\tag{Parenthesis}\\
                    &=\boxed{0}&\tag{Multipliation}
                \end{flalign*}
            \end{proof}
            \begin{problem}
                Solve $x^{2}+y^{2}-z^{2}$ for $x=1,y=2,z=3$.
            \end{problem}
            \begin{proof}[Solution]
                \begin{flalign*}
                    x^{2}+y^{2}-z^{2}\big|_{x=1,y=2,z=3}
                    &=(1)^{2}+(2)^{2}-(3)^{2}&\tag{Substitution}\\
                    &=1+4-9&\tag{Exponentiation}\\
                    &=\boxed{-4}&\tag{Addition/Subtraction}
                \end{flalign*}
            \end{proof}
            \begin{problem}
                Solve $\frac{z+1}{y(x+1)}$ for $x=1,y=2,z=3$.
            \end{problem}
            \begin{proof}[Solution]
                \begin{flalign*}
                    \tfrac{z+1}{y(x+1)}\big|_{x=1,y=2,z=3}
                    &=\tfrac{(3)+1}{(2)((1)+1)}&\tag{Substitution}\\
                    &=\tfrac{4}{2(2)}&\tag{Parenthesis}\\
                    &=\tfrac{4}{4}&\tag{Multiplication}\\
                    &=\boxed{1}&\tag{Division}
                \end{flalign*}
            \end{proof}
            \begin{problem}
                Solve $xy+xz+yz$ for $x=0,y=3,z=-1$.
            \end{problem}
            \begin{proof}[Solution]
                \begin{flalign*}
                    xy+xz+yz\big|_{x=0,y=3,z=-1}
                    &=(0)(3)+(0)(-1)+(3)(-1)&\tag{Substitution}\\
                    &=0+0+(-3)&\tag{Multiplication}\\
                    &=\boxed{-3}&\tag{Addition}
                \end{flalign*}
            \end{proof}
            \begin{problem}
                Solve $y^{x}+z^{y}$ for $x=0,y=3,z=-1$.
            \end{problem}
            \begin{proof}[Solution]
                \begin{flalign*}
                    y^{x}+z^{y}\big|_{x=0,y=3,z=-1}
                    &=(3)^{(0)}+(-1)^{(3)}&\tag{Substitution}\\
                    &=1+(-1)&\tag{Exponentiation}\\
                    &=\boxed{0}&\tag{Addition}
                \end{flalign*}
            \end{proof}
            \begin{problem}
                Solve $\frac{y+z}{x}$ for $x=0,y=3,z=-1$.
            \end{problem}
            \begin{proof}[Solution]
                \begin{flalign*}
                    \tfrac{y+z}{x}\big|_{x=0,y=3,z=-1}
                    &=\tfrac{(3)+(-1)}{(0)}&\tag{Substitution}\\
                    &=\boxed{\textrm{Undefined}}
                    &\tag{Division by Zero}
                \end{flalign*}
            \end{proof}
            \begin{problem}
                Solve $xy^{z^y}+y$ for $x=0,y=3,z=-1$.
            \end{problem}
            \begin{proof}[Solution]
                \begin{flalign*}
                    xy^{z^{y}}+y\big|_{x=0,y=3,z=-1}
                    &=(0)(3)^{(-1)^{(3)}}+3&\tag{Substitution}\\
                    &=0\cdot 3^{-1}+3&\tag{Exponentiation}\\
                    &=0+3&\tag{Multiplication}\\
                    &=\boxed{3}&\tag{Addition}
                \end{flalign*}
            \end{proof}
    \subsection{Linear Equations in One Variable}
        \begin{definition}
        An equation is a statement that two algebraic expressions are equal, represented by $``="$.
        \end{definition}
        \begin{example}
        $2x+4=8$ is an equation. If we replace $x$ with $-4$ we get $2\cdot(-4)+8 = 0$, which is true. We call $-4$ a root or a solution to this equation. If we replace $x$ with $3$, we get $2\cdot 3 + 8 = 0$, which is false. $3$ is not a solution.
        \end{example}
        \begin{definition}
        The set of all solutions to an equation is called the solution set for that equation.
        \end{definition}
        \begin{example}
        The solution set for $2x+8=0$ is $\{-4\}$.
        \end{example}
        \begin{definition}
        A linear equation in one variable is an equation of the form $ax+b=0$, where $a,b\in \mathbb{R}$ and $a\ne 0$.
        \end{definition}
        \begin{remark}
        Other letters can be used in place of $x$. We could have $3t+5 = 0$, $9u-4 = 0$, $-18y + 4 = 0$.
        \end{remark}
        \begin{definition}
        Equivalent equations are equations with the same solution set.
        \end{definition}
        \begin{example}
        $2x+8 = 0$ and $2x = -8$ both have the solution set $\{-4\}$, and are therefore equivalent.
        \end{example}
        \begin{remark}
        Adding or subtracting the same real numbers to both sides of an equation results in an equivalent equations. We can also multiply or divide by non-zero numbers to create equivalent equations.
        \end{remark}
        \begin{properties}[Properties of Equality]
        If $A$ and $B$ are algebraic expressions and $C$ is a non-zero number, then the following are equivalent to $A=B$:
        \begin{enumerate}
        \item $A+C = B+C$
        \item $A-C = B-C$
        \item $C\cdot A = C\cdot B$
        \item $\frac{A}{C} = \frac{B}{C}$.
        \end{enumerate}
        \end{properties}
        \begin{remark}
        We can also add, subtract, multiply, and divide algebraic expressions to both sides to obtain equivalent expressions, but we must be careful that the result is well defined. For example, if we have the expression $x=0$ and we add $\frac{1}{x}$ to both sides, we obtain $x+\frac{1}{x} = 0+\frac{1}{x}$. This is not true because $\frac{1}{x}$ is not defined for $x=0$.
        \end{remark}
        \begin{theorem}[Solution to Linear Equations in One Variable]
        If $ax+b=0$ $a,b\in \mathbb{R}$, $a\ne 0$, then the solution $-\frac{b}{a}$.
        \end{theorem}
        \begin{proof}
        We have that $ax+b = 0$. Subtract $b$ to get $ax = -b$. But $a$ is non-zero so we can divide to obtain $x = - \frac{b}{a}$.
        \end{proof}
        \begin{example}
        Let's solve the following:
        \begin{enumerate}
        \begin{multicols}{4}
        \item $3x-4 = 8$
        \item $\frac{1}{2}x - 6 = \frac{3}{4}x - 9$
        \item $3(4x-1) = 4-6(x-3)$
        \item $5(3x - 2) = 5-7(x-1)$
        \end{multicols}
        \end{enumerate}
        \begin{enumerate}
        \item $3x-4 = 8 \Leftrightarrow 3x = 12 \Leftrightarrow x = 4$
        \item $\frac{1}{2}x-6 = \frac{3}{4}x-9 \Leftrightarrow 2x - 24 = 3x - 36 \Leftrightarrow 2x = 3x-12 \Leftrightarrow x-12 = 0 \Leftrightarrow x = 0$
        \item $3(4x-1) = 4-6(x-3) \Leftrightarrow 12x - 3=4 - 6x+18 = 22 - 6x \Leftrightarrow 18x - 3 = 22 \Leftrightarrow 18x = 25 \Leftrightarrow x = \frac{25}{18}$.
        \item $5(3x-2) = 5-7(x-1) \Leftrightarrow 15x - 10 = 5-7x + 7 \Leftrightarrow 22x - 10 = 12 \Leftrightarrow 22x = 22\Leftrightarrow x=1$
        \end{enumerate}
        \end{example}
        
        \begin{definition}
        An identity is an equation that is satisfied by every real number.
        \end{definition}
        
        \begin{example}
        The following are identities:
        \begin{enumerate}
        \begin{multicols}{3}
        \item $3x-1 = 3x-1$
        \item $2x+5x = 7x$
        \item $\frac{x}{x} = 1$
        \end{multicols}
        \end{enumerate}
        \end{example}
        
        \begin{remark}
        Note that $\frac{x}{x} = 1$ is an identity, however $\frac{x}{x}$ is undefined for $x=0$. We do not say that $\frac{0}{0} = 1$ or any other number, and we leave such an expression undefined. The solution set of $\frac{x}{x}$ is thus the set of all non-zero numbers. 
        \end{remark}
        
        \begin{definition}
        An inconsistent equation is an equation that has no solutions.
        \end{definition}
        
        \begin{example}
        The following are inconsistent equations.
        \begin{enumerate}
        \begin{multicols}{3}
        \item $0\cdot x +1 = 0$
        \item $x+3 = x+5$
        \item $9x - 9x = 8$
        \end{multicols}
        \end{enumerate}
        \end{example}
        
        \begin{definition}
        A conditional equation is an equation that is neither an identity nor an inconsistent equation.
        \end{definition}
        
        \begin{example}
        Every linear equation in one variable is a conditional equation with only one solution. The solution to $ax+b = 0$ is $-\frac{b}{a}$. Since a solution exists, the equation is not inconsistent. However, $-\frac{b}{a}$ is the only solution, and therefore the equation is not an identity. Thus it is a conditional equation.
        \end{example}
        
        \begin{example}
        Determine what type of equation $3(x-1)-2x(4-x) = (2x+1)(x-3)$ is. We have $3(x-1) - 2x(4-x) = 3x-3 -8x+2x^2$. Also $(2x+1)(x-3) = 2x^2 - 6x + x -3$. So we have $2x^2 - 5x -3 = 2x^2 - 5x - 3$. Identity.
        \end{example}
        
        \begin{example}
        Let's solve the following:
        \begin{enumerate}
        \begin{multicols}{3}
        \item $\frac{y}{y-3} + 3 = \frac{3}{y-3}$
        \item $\frac{1}{x-1} - \frac{1}{x+1} = \frac{2}{x^2-1}$
        \item $\frac{1}{2} + \frac{1}{x-1} = 1$
        \end{multicols}
        \end{enumerate}
        \begin{enumerate}
        \item Note that for $y=3$, $\frac{3}{y-3}$ and $\frac{y}{y-3}$ are undefined, so must exclude $y=3$. Multiplying both sides by $y-3$, we have $y + 3(y-3) = 3 \Leftrightarrow y+3y- 9 = 3 \Leftrightarrow 4y = 12 \Leftrightarrow y = 3$. But we must exclude $3$ from the solution set as the original equations are undefined for $y=3$. Thus this is an inconsistent equation and has no solution.
        \item We must exclude both $x=1$ and $x=-1$. First note that $(x-1)(x+1) = x^2-1$. Multiplying both sides by this, we get $(x+1) - (x-1) = 2$, which is equivalent to $2=2$. This is always true, and $2$ is valid in the original equation, so we have that the equation is an identity. The solution set is all real number except for $1$ and $-1$.
        \item We must exclude $x=1$ as the left hand side of the equation is undefined for this value. Multiplying by $x-1$, we have $\frac{x-1}{2} + 1 = x-1$, or $\frac{x-1}{2} = 1$, so $x-1 = 2$, and thus $x=3$. The original equation is well defined for $x=3$, so we have that the solution set is $\{3\}$.
        \end{enumerate}
        \end{example}
        
        \subsubsection{Equations Involving an Absolute Value}
        
        \begin{definition}
        The absolute value of a real number $x$ is $|x| = \begin{cases} x, & x>0 \\ 0, & x=0 \\, -x, & x<0\end{cases}$
        \end{definition}
        
        \begin{remark}
        The absolute value of a real number is greater than or equal to zero. Thus the equation $|x| = -6$ has no solution and is inconsistent. The only solution to $|x| = 0$ is $x=0$. Finally, $|x| = 4$ has two solutions. $x=4$ satisfies this equation, but as per the definition of the absolute value, $x=-4$ is also as solution. That is $|4| = |-4| = 4$. So positive numbers have two solutions.
        \end{remark}
        
        \begin{theorem}[Solution Sets of Absolute Values]
        If $|x| = k$, then the following are true:
        \begin{enumerate}
        \item If $k<0$, there are no solutions. The solution set is the empty set $\emptyset$.
        \item If $k = 0$, then $x=0$ is the only solution. The solution set is $\{0\}$.
        \item If $k>0$, then $-k$ and $k$ are solutions. The solution set is $\{-k,k\}$.
        \end{enumerate}
        \end{theorem}
        
        \begin{example}
        Let's solve the following:
        \begin{enumerate}
        \begin{multicols}{2}
        \item $|x-5| = 4$
        \item$2|x+8|-6 = 0$
        \end{multicols}
        \end{enumerate}
        \begin{enumerate}
        \item We have $x-5 = 4$ or $x-5 = -4$. The solution set is $\{1,9\}$.
        \item We have that $2|x+8| = 6\Leftrightarrow |x+8| = 3$. Thus $x+8 = 3$ or $x+8 = -3$. The solution set is $\{-11,-5\}$.
        \end{enumerate}
        \end{example}
        \begin{example}
        If $x$ is the number of years after $1990$ and $y$ is the median income in dollars for working women in the United States, then the equation $355.9x + 11,075.3$ models the real data. What years is the median income $\$20,000?$ We need to solve $20,000 = 355.9x + 11,075.3$. Solving for $x$, we get $x \approx 25.08$. This corresponds to the year $2015$.
        \end{example}
        \subsubsection{Problems}
            \begin{enumerate}
                \item What is the solution set
                    to $5(4-x)=2x-1$
                \item Are $3x-1 = 8$ and $3x-2 = 7$ equivalent?
                \item Is $x+\sqrt{x} = -2+\sqrt{x}$
                    equivalent to $x=-2$?
                \item What is the solution set to $x-x = 7$?
                \item Is $12x = 0$ inconsistent?
                \item Is $|x|=-8$ equivalent to
                    $x=-8$ or $x=8$?
                \item Is $\frac{x}{x-5}=\frac{5}{x-5}$
                    and $x=5$ equivalent?
            \end{enumerate}
        \subsubsection{Problems}
        \begin{problem}
        Solve for $x$: $6x-48=6$
        \end{problem}
        \begin{proof}[Solution]
        \begin{flalign*}
            6x-48&=6\\
            \Rightarrow 6x&=54&\tag{Add $48$ to Both Sides}\\
            \Rightarrow x&=\tfrac{54}{6}&\tag{Divide Both Sides by $6$}\\
            \Rightarrow x&=\boxed{9}&\tag{Division}
        \end{flalign*}
        \end{proof}
        \begin{problem}
        Solve for $x$: $\frac{2}{3}x-5=x-3$
        \end{problem}
        \begin{proof}[Solution]
        \begin{flalign*}
            \tfrac{2}{3}x-5&=x-3\\
            \Rightarrow 2x-15&=3x-9&\tag{Multiply Both Sides by $3$}\\
            \Rightarrow -x-15&=-9&\tag{Subtract $3x$ from Both Sides}\\
            \Rightarrow -x&=6&\tag{Add $15$ to Both Sides}\\
            \Rightarrow x&=\boxed{-6}&\tag{Multiply Both Sides by $-1$}
        \end{flalign*}
        \end{proof}
        \begin{problem}
        Solve for $x$: $50-x-(3x+2)=0$
        \end{problem}
        \begin{proof}[Solution]
        \begin{flalign*}
            50-x-(3x+2)&=0\\
            \Rightarrow 50-x-3x-2&=0&\tag{Distribute the Minus Sign}\\
            \Rightarrow 48-4x&=0&\tag{Simplify the Left-Hand Side}\\
            \Rightarrow 4x&=48&\tag{Add $4x$ to Both Sides}\\
            \Rightarrow x&=\tfrac{48}{4}&\tag{Divide Both Sides by $4$}\\
            \Rightarrow x&=\boxed{12}&\tag{Division}
        \end{flalign*}
        \end{proof}
        \begin{problem}
        Solve for $x$: $8-4(x-1)=2+3(4-x)$
        \end{problem}
        \begin{proof}[Solution]
        \begin{flalign*}
            8-4(x-1)&=2+3(4-x)\\
            \Rightarrow8-4x+4
            &=2+12-3x&\tag{Simplify Both Sides}\\
            \Rightarrow 12-4x&=14-3x&\tag{Simplify Both Sides}\\
            \Rightarrow 12&=14+x&\tag{Add $4x$ to Both Sides}\\
            \Rightarrow x&=\boxed{-2}&\tag{Subtract $14$ from Both Sides}
        \end{flalign*}
        \end{proof}
        \begin{problem}
        Solve $x+1=1$ for $x$.
        \end{problem}
        \begin{proof}[Solution]
        \begin{flalign*}
            x+1&=1\\
            \Rightarrow x&=\boxed{0}&\tag{Subtract $1$ from Both Sides}
        \end{flalign*}
        \end{proof}
        \begin{problem}
        Solve $4(x-1)+x=0$ for $x$.
        \end{problem}
        \begin{proof}[Solution]
        \begin{flalign*}
            4(x-1)+x&=0\\
            \Rightarrow 5x-4&=0&\tag{Simplify the Left-Hand Side}\\
            \Rightarrow 5x&=4&\tag{Add $4$ to Both Sides}\\
            \Rightarrow x&=\boxed{\tfrac{4}{5}}&\tag{Divide Both Sides by $5$}
        \end{flalign*}
        \end{proof}
        \begin{problem}
        Solve $1-x+10=7$ for $x$.
        \end{problem}
        \begin{proof}[Solution]
        \begin{flalign*}
            1-x+10&=7\\
            \Rightarrow 11-x&=7&\tag{Simplify the Left-Hand Side}\\
            \Rightarrow 11&=7+x&\tag{Add $x$ to both sides}\\
            \Rightarrow x&=\boxed{4}&\tag{Subtract $7$ from Both Sides}
        \end{flalign*}
        \end{proof}
    \subsection{Formulas}
        \subsubsection{Problems}
        \begin{problem}
        Solve $PV=nRT$ for $T$.
        \end{problem}
        \begin{proof}[Solution]
        This is the Ideal Gas Law from chemistry: $\boxed{T=\tfrac{PV}{nR}}$
        \end{proof}
        \begin{problem}
        Solve $y=3x+2$ for $x$.
        \end{problem}
        \begin{proof}[Solution]
        $y=3x+2\Rightarrow y-2=3x\Rightarrow\boxed{x=\tfrac{y-2}{3}}$
        \end{proof}
        \begin{problem}
        Solve $C=2\pi r$ for $r$.
        \end{problem}
        \begin{proof}[Solution]
        This is the formula for the circumference $C$ of a circle of radius $r$: $\boxed{r=\tfrac{C}{2\pi}}$
        \end{proof}
        \begin{problem}
        Solve $\tfrac{x}{2}+\tfrac{y}{5}=1$ for $y$.
        \end{problem}
        \begin{proof}[Solution]
        $\tfrac{x}{2}+\tfrac{y}{5}=1\Rightarrow=\tfrac{y}{5}=1-\tfrac{x}{2}\Rightarrow\boxed{y=5-\tfrac{5}{2}x}$.
        \end{proof}
        \begin{problem}
        Solve $y=hx+4x$ for $x$.
        \end{problem}
        \begin{proof}[Solution]
        $y=hx+4x\Rightarrow x(h+4)=y\Rightarrow\boxed{x=\tfrac{y}{h+4}}$
        \end{proof}
    \subsection{Word Problems}
        \subsubsection{Problems}
        \begin{problem}
        $y$ is $5$ more than twice that of $x$, their sum is $35$. Find $x$ and $y$.
        \end{problem}
        \begin{proof}[Solution]
        We have $y=5+2x$ and $x+y=35$. Substituting $y$, we get $x+2x+5=35\Rightarrow 3x+5=35\Rightarrow 3x=30\Rightarrow\boxed{x=10}$. But $y=2+2x=5+2(10)\Rightarrow\boxed{y=25}$
        \end{proof}
        \begin{problem}
        Ms. Jones invested $\$18,\!000$ in two accounts, one pays $6\%$ and the other $8\%$. Her total interest was $\$1,\!290$. How much did she have in each account?
        \end{problem}
        \begin{proof}[Solution]
        Let $x$ and $y$ be the amounts in the $6\%$ and $8\%$ accounts, respectively. Then $x+y=\$18,\!000$ and $\frac{6}{100}x+\frac{8}{100}y=\$1,\!290$. Solving the first equation we get $y=\$18,\!000-x$. Substituting we have: $\frac{6}{100}x+\frac{8}{100}(\$18,\!000-x)=\$1,\!290\Rightarrow \$1440-\frac{2}{100}x=\$1,\!290\Rightarrow\frac{2}{100}x=\$150\Rightarrow\boxed{x=\$7,\!500}$ But $y=\$18,\!000-x\Rightarrow y=\$18,\!000-\$7,\!500\Rightarrow\boxed{y=\$10,\!500}$
        \end{proof}
        \begin{problem}
        How many liters of $40\%$ and $16\%$ solution are needed to obtain $20$ liters of $22\%$ solution?
        \end{problem}
        \begin{proof}[Solution]
        Let $x$ and $y$ be the number of liters of $40\%$ and $16\%$ solution, respectively. Then $x+y=20$, and $\frac{40}{100}x+\frac{16}{100}y=\frac{22}{100}20\Rightarrow 40x+16y=440$. Solving the first equation, we get $y=20-x$. Substituting, we have: $40x+16(20-x)=440\Rightarrow 24x+320=440\Rightarrow 24x=120\Rightarrow x=\frac{120}{24}\Rightarrow\boxed{x=5}$. But $y=20-x=20-5\Rightarrow \boxed{y=15}$ so there are $\boxed{5\textrm{ liters of }40\%}$ and $\boxed{15\textrm{ liters of }16\%}$ solution.
        \end{proof}
        \begin{problem}
        Sheila bought burgers and fries for her children and some friends. The burgers cost $\$2.05$ each and the fries are $\$0.85$ each. She bought a total of $14$ items for a total cost of $\$19.10$. How many of each did she buy?
        \end{problem}
        \begin{proof}[Solution]
        Let $x$ and $y$ be the number of burgers and fries, respectively. Then $2.05x+0.85y=19.10$, and $x+y=14$. Solving the second equation, we get $y=14-x$. Substitute this into the second equation to $2.05x+0.85(14-x)=19.10\Rightarrow 1.20x+11.90=19.10\Rightarrow 1.20x=7.20\Rightarrow\boxed{x=6}$ But $y=14-x\Rightarrow\boxed{y=8}$ Sheila bought $\boxed{6\textrm{ burgers}}$ and $\boxed{8\textrm{ fries}}$
        \end{proof}
    \subsection{Inequalities}
        There are three main rules for dealing with inequalities:
        \begin{properties}\label{property:north_shore_properties_of_inequalities}
        \
        \begin{enumerate}
            \item \label{property:north_shore_additive_property_inequals}If $c$ is real and $a<b$, then $a+c<b+c$\hfill [Additive Property of Inequalities]
            \item \label{property:north_shore_multiplicative_property_inequals}If $c$ is $\mathbf{positive}$ and $a<b$, then $ac < bc$\hfill [Multiplicative Property of Inequalities]
            \item \label{property:north_shore_inverse_property_inequals}If $c$ is $\mathbf{negative}$ and $a<b$, then $bc<ac$\hfill [Inverse Property of Inequalities]
        \end{enumerate}
        \end{properties}
        \subsubsection{Problems}
        \begin{problem}
        Solve for $x$: $2x-7\geq 3$
        \end{problem}
        \begin{proof}[Solution]
            \begin{flalign*}
                2x-7&\geq 3\\
                \Rightarrow 2x&\geq 10&\tag{Add $7$ to Both Sides, property~\ref{property:north_shore_properties_of_inequalities} part~\ref{property:north_shore_additive_property_inequals}}\\
                \Rightarrow x&\geq\tfrac{10}{2}&\tag{Divide Both Sides by $2$, property~\ref{property:north_shore_properties_of_inequalities} part~\ref{property:north_shore_multiplicative_property_inequals}}\\
                \Rightarrow x&\geq 5&\tag{Division}
            \end{flalign*}
        \end{proof}
        \begin{problem}
        Solve for $x$: $-5(2x+3)<2x-3$
        \end{problem}
        \begin{proof}[Solution]
        \begin{flalign*}
        -5(2x+3)&<2x-3\\
        \Rightarrow -10x-15&<2x-3&\tag{Simplify the Left-Hand Side}\\
        \Rightarrow -15+3&<2x+10x&\tag{Add $10x+3$ to Both Sides, property~\ref{property:north_shore_properties_of_inequalities} part~\ref{property:north_shore_additive_property_inequals}}\\
        \Rightarrow -12&<12x&\tag{Simplify Both Sides}\\
        \Rightarrow x&>-1&\tag{Divide Both Sides by $12$, property~\ref{property:north_shore_properties_of_inequalities} part~\ref{property:north_shore_multiplicative_property_inequals}}
        \end{flalign*}
        \end{proof}
        \begin{problem}
        Solve for $x$: $3(x-4)-(x+1)\leq -12$
        \end{problem}
        \begin{proof}[Solution]
            \begin{flalign*}
                3(x-4)-(x+1)&\leq -12\\
                \Rightarrow 3x-12-x-1&\leq -12&\tag{Simplify the Left-Hand Side}\\
                \Rightarrow 2x-13&\leq -12&\tag{Simplify the Left-Hand Side}\\
                \Rightarrow 2x&\leq 1&\tag{Add $13$ to Both Sides, property~\ref{property:north_shore_properties_of_inequalities} part~\ref{property:north_shore_additive_property_inequals}}\\
                \Rightarrow x&\leq\tfrac{1}{2}&\tag{Divide Both Sides by $2$, property~\ref{property:north_shore_properties_of_inequalities} part~\ref{property:north_shore_multiplicative_property_inequals}}
            \end{flalign*}
        \end{proof}
    \subsection{Exponents and Polynomials}
        The two main rules for problem with polynomials and exponents are:
        \begin{align}
            (a_{1}x^{2}+b_{1}x+c_{1})+(a_{2}x^{2}+b_{2}x+c_{2})&=x^{2}(a_{1}+a_{2})+x(b_{1}+b_{2})+(c_{1}+c_{2})\label{equation:north_shore_additive_polynomial_property}\\
            (ax+b)(cx+d)&=acx^{2}+(ad+bc)x+bd\label{equation:north_shore_multiplicative_polynomial_property}
        \end{align}
        \begin{remark}
        Eqn.~\ref{equation:north_shore_additive_polynomial_property} says that the coefficients of like terms can be added together to simplify the expression, and eqn.~\ref{equation:north_shore_multiplicative_polynomial_property} is often called the \gls{foil} rule.
        \end{remark}
        \begin{remark}
        \label{remark:north_shore_square_of_a_sum_foil}
        A special case of \gls{foil}, we can write $(a+b)^{2}=a^{2}+ab+ba+b^{2}=a^{2}+2ab+b^{2}$
        \begin{equation}
        \label{equation:North_Shore_square_of_a_sum}
            (a+b)^{2}=a^{2}+2ab+b^{2}
        \end{equation}
        \end{remark}
        \subsubsection{Problems}
        \begin{problem}
        Simplify using only positive exponents: $(3x^{0}y^{5}z^{6})(-2xy^{3}z^{-2})$
        \end{problem}
        \begin{proof}[Solution]
            \begin{flalign*}
                (3x^{0}y^{5}z^{6})(-2xy^{3}z^{-2})&=(3\cdot (-2))(x^{0}\cdot x^{1})(y^{5}\cdot y^{3})(z^{6}\cdot z^{-2})&\tag{Property~\ref{property:North_Shore_Arithmetic_Properties} part~\ref{property:north_shore_arithmetic_properties_assoc_mult}}\\
                &=-6x^{0+1}y^{5+3}z^{6-2}&\tag{Property~\ref{property:North_Shore_Exponent_Rules} part~\ref{property:north_shore_product_property_of_expo}}\\
                &=-6xy^{8}z^{4}&\tag{Simplify Exponents}
            \end{flalign*}
        \end{proof}
        \begin{problem}
        Simplify using only positive exponents: $(3x^{2}-5x-6)+(5x^{2}+4x+4)$
        \end{problem}
        \begin{proof}[Solution]
        \begin{flalign*}
            (3x^{2}-5x-6)+(5x^{2}+4x+4)&=(3+5)x^{2}+(-5+4)x+(-6+4)&\tag{Eqn.~\ref{equation:north_shore_additive_polynomial_property}}\\
            &=8x^{2}-x-2&\tag{Simplify Parenthesis}
        \end{flalign*}
        \end{proof}
        \begin{problem}
        Simplify using only positive exponents: $\frac{(2a^{-5}b^{4}c^{3})^{-2}}{(3a^{3}b^{-7}c^3)^{2}}$
        \end{problem}
        \begin{proof}[Solution]
        \begin{flalign*}
            \tfrac{(2a^{-5}b^{4}c^{3})^{-2}}{(3a^{3}b^{-7}c^{3})^{2}}&=(2a^{-5}b^{4}c^{3})^{-2}\cdot\tfrac{1}{(3a^{3}b^{-7}c^{3})^{2}}&\tag{Remark~\ref{remark:North_Shore_Rational_Expressions}}\\
            &=\tfrac{1}{(2a^{-5}b^{4}c^{3})^{2}}\cdot\tfrac{1}{(3a^{3}b^{-7}c^{3})^{2}}&\tag{Property~\ref{property:North_Shore_Exponent_Rules} part~\ref{property:north_shore_inverse_property_of_expo}}\\
            &=\tfrac{1}{2^{2}(a^{-5})^{2}(b^{4})^{2}(c^{3})^{2}}\cdot\tfrac{1}{3^{2}(a^{3})^{2}(b^{-7})^{2}(c^{3})^{2}}&\tag{Property~\ref{property:North_Shore_Exponent_Rules} part~\ref{property:north_shore_distributive_property_of_expo}}\\
            &=\tfrac{1}{4a^{-10}b^{8}c^{6}}\cdot\tfrac{1}{9a^{6}b^{-14}c^{6}}&\tag{Property~\ref{property:North_Shore_Exponent_Rules} part~\ref{property:north_shore_power_property_of_expo}}\\
            &=\tfrac{a^{10}}{4b^{8}c^{6}}\cdot\tfrac{b^{14}}{9a^{6}c^{6}}&\tag{Property~\ref{property:North_Shore_Exponent_Rules} part~\ref{property:north_shore_inverse_property_of_expo}}\\
            &=\tfrac{a^{10}b^{14}}{36a^{6}b^{8}c^{6}c^{6}}&\tag{Multiplication}\\
            &=\tfrac{a^{10}b^{14}}{36a^{6}b^{8}c^{12}}&\tag{Property~\ref{property:North_Shore_Exponent_Rules} part~\ref{property:north_shore_product_property_of_expo}}\\
            &=\tfrac{a^{4}b^{6}}{36c^{12}}&\tag{Division}
        \end{flalign*}
        \end{proof}
        \begin{problem}
        Simplify using only positive exponents: $(-a^{5}b^{7}c^{9})^{4}$
        \end{problem}
        \begin{proof}[Solution]
            \begin{flalign*}
                (-a^{5}b^{7}c^{9})^{4}&=((-1)\cdot a^{5}b^{7}c^{9})^{4}&\tag{Rewrite Expression}\\
                &=(-1)^{4}(a^{5})^{4}(b^{7})^{4}(c^{9})^{4}&\tag{Property~\ref{property:North_Shore_Exponent_Rules} part~\ref{property:north_shore_distributive_property_of_expo}}\\
                &=a^{20}b^{28}c^{36}&\tag{Property~\ref{property:North_Shore_Exponent_Rules} part~\ref{property:north_shore_power_property_of_expo}}
            \end{flalign*}
        \end{proof}
        \begin{problem}
        Simplify using only positive exponents: $\frac{24x^{4}-32x^{3}+16x^{2}}{8x^{2}}$
        \end{problem}
        \begin{proof}[Solution]
            \begin{flalign*}
                \tfrac{24x^{4}-32x^{3}+16x^{2}}{8x^{2}}&=\tfrac{8x^{2}(3x^{2}-4x+2)}{8x^{2}}&\tag{Property~\ref{property:North_Shore_Arithmetic_Properties} part~\ref{property:north_shore_arithmetic_properties_distributive_property}}\\
                &=3x^{2}-4x+2&\tag{Property~\ref{property:North_Shore_Arithmetic_Properties} part~\ref{property:north_shore_arithmetic_properties_mult_inverse}}
            \end{flalign*}
        \end{proof}
        \begin{problem}
        Simplify using only positive exponents: $(x^{2}-5x)(2x^{3}-7)$
        \end{problem}
        \begin{proof}[Solution]
            \begin{flalign*}
                (x^{2}-5x)(2x^{3}-7)&=2x^{5}-7x^{2}-10x^{4}+35x&\tag{\gls{foil}}\\
                &=x(2x^{4}-10x^{3}-7x+35)&\tag{Property~\ref{property:North_Shore_Arithmetic_Properties} part~\ref{property:north_shore_arithmetic_properties_distributive_property}}
            \end{flalign*}
        \end{proof}
        \begin{problem}
        Simplify using only positive exponents: $(4x^{2}y^{6}z)^{2}(-2x^{-2}y^{3}z^{4})^{6}$
        \end{problem}
        \begin{proof}[Solution]
            \begin{flalign*}
                (4x^{2}y^{6}z)^{2}(-x^{-2}y^{3}z^{4})^{6}&=(4^{2}(x^{2})^{2}(y^{6})^{2}(z)^{2})((x^{-2})^{6}(y^{3})^{6}(z^{4})^{6})&\tag{Property~\ref{property:North_Shore_Exponent_Rules} part~\ref{property:north_shore_distributive_property_of_expo}}\\
                &=(16x^{4}y^{12}z^{2})(x^{-12}y^{18}z^{24})&\tag{Property~\ref{property:North_Shore_Exponent_Rules} part~\ref{property:north_shore_power_property_of_expo}}\\
                &=16(x^{4}x^{-12})(y^{12}y^{18})(z^{2}z^{24})&\tag{Property~\ref{property:North_Shore_Arithmetic_Properties} part~\ref{property:north_shore_arithmetic_properties_assoc_mult}}\\
                &=16x^{4-12}y^{18+12}z^{2+24}&\tag{Property~\ref{property:North_Shore_Exponent_Rules} part~\ref{property:north_shore_product_property_of_expo}}\\
                &=16x^{-8}y^{30}z^{26}&\tag{Simplify Exponents}\\ 
                &=\tfrac{16y^{30}z^{26}}{x^{8}}&\tag{Property~\ref{property:North_Shore_Exponent_Rules} part~\ref{property:north_shore_inverse_property_of_expo}}
            \end{flalign*}
        \end{proof}
        \begin{problem}
        Simplify using only positive exponents: $(x^{2}-5x)(2x^{3}-7)$
        \end{problem}
        \begin{proof}[Solution]
            \begin{flalign*}
                (x^{2}-5x)(2x^{3}-7)&=2x^{5}-7x^{2}-10x^{4}+35x&\tag{\gls{foil}}\\
                &=x(2x^{4}-10x^{3}-7x+35)&\tag{Property~\ref{property:North_Shore_Arithmetic_Properties} part~\ref{property:north_shore_arithmetic_properties_distributive_property}}
            \end{flalign*}
        \end{proof}
        \begin{problem}
        Simplify using only positive exponents: $\frac{26a^{2}b^{-5}c^{9}}{-4a^{-6}bc^{9}}$
        \end{problem}
        \begin{proof}[Solution]
            \begin{flalign*}
                \tfrac{26a^{2}b^{-5}c^{9}}{-4a^{-6}bc^{9}}&=26a^2b^{-5}c^{9}\cdot\tfrac{1}{-4a^{-6}bc^{9}}&\tag{Remark~\ref{remark:North_Shore_Rational_Expressions}}\\
                &=\tfrac{26a^{2}c^{9}}{b^{5}}\cdot\tfrac{a^{6}}{-4bc^{9}}&\tag{Property~\ref{property:North_Shore_Exponent_Rules} part~\ref{property:north_shore_inverse_property_of_expo}}\\
                &= -\tfrac{26}{4}\cdot\tfrac{a^{2+6}c^{9}}{b^{5+1}c^{9}}&\tag{Property~\ref{property:North_Shore_Exponent_Rules} part~\ref{property:north_shore_product_property_of_expo}}\\
                &= -\tfrac{13a^{8}}{2b^{6}}&\tag{Property~\ref{property:North_Shore_Arithmetic_Properties} part~\ref{property:north_shore_arithmetic_properties_mult_inverse}}
            \end{flalign*}
        \end{proof}
        \begin{problem}
        Simplify using only positive exponents: $(5a+6)^2$
        \end{problem}
        \begin{proof}[Solution]
            \begin{flalign*}
                (5a+6)^{2}&=25a^{2}+60a+36&\tag{Remark~\ref{remark:north_shore_square_of_a_sum_foil} Eqn.~\ref{equation:North_Shore_square_of_a_sum}}
            \end{flalign*}
        \end{proof}
        \begin{problem}
        Simplify using only positive exponents: $(5x+1)(x+3)$
        \end{problem}
        \begin{proof}[Solution]
            \begin{flalign*}
                (5x+1)(x+3)&=5x^{2}+16x+3&\tag{Eqn.~\ref{equation:north_shore_multiplicative_polynomial_property}}
            \end{flalign*}
        \end{proof}
    \subsection{Factoring}
        \begin{fdefinition}{Quadratic Equation}{}
            A quadratic is an equation of the form $y=ax^{2}+bx+c$.
        \end{fdefinition}
        \begin{fdefinition}{Quadratic Factorization}{}
            The factorization of a quadratic equation $y=ax^{2}+bx+c$
            is an equivalent expression of the form
            $y=\gamma(x+\alpha)(x+\beta)$.
        \end{fdefinition}
        \begin{ftheorem}{}{north_shore_factorization_%
                          of_quadratic_when_a_is_equal_to_zero}
            If $y=x^{2}+bx+c$ has the factorization
            $y=(x+\alpha)(x+\beta)$, then
            $b=\alpha+\beta$ and $c=\alpha\cdot\beta$
        \end{ftheorem}
        \begin{proof}
        Suppose $y=x^{2}+bx+c$ and $y=(x+\alpha)(x+\beta)$. By \gls{foil}, $y=(x+\alpha)(x+\beta)=x^{2}+x(\alpha+\beta)+\alpha\cdot\beta$. But also $y=x^{2}+bx+c$. But the coefficients must be equal, and therefore $\alpha+\beta=b$ and $\alpha\cdot\beta=c$.
        \end{proof}
        \begin{remark}\label{remark:north_shore_example_of_using_factorization_when_a_equals_zero}
        This result helps us factor quadratic equations when the leading term has a coefficient of $1$ (That is, $a=1$). If we see something like $x^{2}-2x+1$ and we want to factor it, all we need to ask is ``What two numbers add to $-2$ and multiply to $1$?" Often times guessing and checking will get the answer after a few tries. For $x^{2}-2x+1$ We see that $(-1)+(-1)=-2$, and $(-1)\cdot (-1)=1$. So, $x^{2}-2x+1=(x-1)(x-1)=(x-1)^{2}$.
        \end{remark}
        \begin{ftheorem}{The Difference of Squares}
                       {north_shore_difference_of_squares}
            If $a$ and $b$ are real numbers, then $a^{2}-b^{2}=(a-b)(a+b)$
        \end{ftheorem}
        \begin{proof}
            By \gls{foil}, $(a-b)(a+b)=a^{2}-ab+ba-b^{2}=a^{2}-b^{2}$.
            Therefore, $(a-b)(a+b)=a^{2}-b^{2}$.
        \end{proof}
        \begin{remark}
            This helps factor quadratics quickly if two
            squares are being subtracted.
        \end{remark}
        \begin{fexample}{}{}
            Factor the expression: $9x^{2}-64$.
            Note that $9=3^{2}$, so $9x^{2}=(3x)^{2}$.
            Also note that $64=8^{2}$. So we can write
            $9x^{2}-64=(3x)^{2}-(8)^{2}$.
            Using the difference of squares formula,
            we have $9x^{2}-64=(3x)^{2}-(8)^{2}=(3x-8)(3x+8)$
        \end{fexample}
        \begin{fexample}{}{}
        Factor the expression: $16x^{4}-81y^{4}$. While this looks like a ``Quartic Equation" (Equations involving $x^4$), it can be rewriten as a quadratic one. Note that $16=4^{2}$ and $81=9^{2}$. But also $x^{4}=(x^{2})^{2}$ and $y^{4}=(y^{2})^{2}$. So we have $16x^{4}-81y^{4}=(4x^{2})^{2}-(9y^{2})^{2}$. This is a difference of squares, and so we can apply the difference of squares formula. $16x^{4}-81y^{4}=(4x^{2})^{2}-(9y^{2})^{2}=(4x^{2}+9y^{2})(4x^{2}-9y^{2})$. We're not quite done yet, for we can simplify $4x^{2}-9y^{2}$ as well. Note that $4x^{2}=(2x)^{2}$, and $9y^{2}=(3y)^{2}$. So we have $4x^{2}-9y^{2}=(2x)^{2}-(3y)^{2}=(2x-3y)(2x+3y)$. Together, we have: $16x^{4}-81y^{4}=(4x^{2}+9y^{2})(2x-3y)(2x+3y)$.
        \end{fexample}
        \begin{ftheorem}{The Difference of Cubes}
                       {north_shore_difference_of_cubes}
            If $a$ and $b$ are real numbers,
            then $a^{3}-b^{3}=(a-b)(a^{2}+ab+b^{2})$
        \end{ftheorem}
        \begin{proof}
        By the distributive property, $(a-b)(a^{2}+ab+b^{2})=a(a^{2}+ab+b^{2})-b(a^{2}+ab+b^{2})$.
        Distributing again, we get $a^{3}+a^{2}b+ab^{2}-ba^{2}-ab^{2}-b^{3}$.
        Rearranging, we have $(a^{3}-b^{3})+(a^{2}b-ba^{2})+(ab^{2}-b^{2}a)$.
        By the commutative property, $a^{2}b=ba^{2}$, and $ab^{2}=b^{2}a$.
        Therefore, $(a^{2}b-ba^{2})+(ab^{2}-b^{2}a)=0$.
        Thus, $(a-b)(a^{2}+ab+b^{2})=a^{3}-b^{3}$.
        \end{proof}
        \begin{remark}
        We can use the difference of cubes formula to factor cubic expressions.
        \end{remark}
        \begin{ftheorem}{The Sum of Squares}
                       {north_shore_sum_of_squares}
            If $x$ and $a$ are real numbers, and if
            $a$ is non-zero, then there is no \textbf{real}
            factorization of $x^{2}+a^{2}$.
        \end{ftheorem}
        \begin{proof}
            If $(x+\alpha)(x+\gamma)$ is a real factorization,
            then by Thm.~\ref{th:north_shore_factorization_of_%
                              quadratic_when_a_is_equal_to_zero},
            $\alpha+\beta=0$ and $\alpha\cdot\beta=a^{2}$. But then
            $\alpha=-\beta$ and therefore $-\alpha^{2}=a^{2}$. But
            $a$ is real, and therefore $a^{2}\geq 0$. But as
            $\alpha$ is real and non-zero, $\alpha^{2}>0$, and
            thus $-\alpha^{2}<0$. But $a^{2}=-\alpha^{2}$,
            a contradiction as $a^{2}\geq 0$. There is no real factorization.
        \end{proof}
        \begin{remark}
        If we see a sum of squares we know that there is no real factorization.
        \end{remark}
        \subsubsection{Problems}
        \begin{problem}
        Factor $x^{2}+5x-6$
        \end{problem}
        \begin{proof}[Solution]
        By Thm.~\ref{th:north_shore_factorization_of_quadratic_when_a_is_equal_to_zero}, if $x^{2}+5x-6=(x+\alpha)(x+\beta)$, then $\alpha+\beta=5$ and $\alpha\cdot\beta=-6$. By guessing and checking a few common factors of $5$ and $6$, we get $\alpha=6$ and $\beta=-1$. So, $x^{2}+5x-6=(x+6)(x-1)$
        \end{proof}
        \begin{problem}
        $x^{2}-5x-6$
        \end{problem}
        \begin{proof}[Solution]
        By Thm.~\ref{th:north_shore_factorization_of_quadratic_when_a_is_equal_to_zero}, if $x^{2}-5x-6=(x+\alpha)(x+\beta)$, then $\alpha+\beta=-5$ and $\alpha\cdot\beta=-6$. After guessing and checking, we have $\alpha=-6$ and $\beta=1$. So $x^{2}-5x-6=(x-6)(x+1)$
        \end{proof}
        \begin{problem}
        $4x^{2}-36$
        \end{problem}
        \begin{proof}[Solution]
        $4x^{2}-36=4(x^{2}-3^{2})$. By Thm.~\ref{th:north_shore_difference_of_squares}, $x^{2}-3^{2}=(x+3)(x-3)$. Thus, $4x^{2}-36=4(x-3)(x+3)$
        \end{proof}
        \begin{problem}
        $x^{2}+4$
        \end{problem}
        \begin{proof}[Solution]
        This is a sum of squares. By Thm.~\ref{th:north_shore_sum_of_squares}, there is no real factorization.
        \end{proof}
        \begin{problem}
        $64x^{4}-4y^{4}$
        \end{problem}
        \begin{proof}[Solution]
        Note that $64x^{2}-4y^{2}=(8x^{2})^{2}-(2y^{2})^{2}$. By Thm.~\ref{th:north_shore_difference_of_squares}, $(8x^{2})^{2}-(2y^{2})^{2}=(8x^{2}+2y^{2})(8x^{2}-2y^{2})$. Note that $8x^{2}$ and $2y^{2}$ share a common factor of $2$ so we can simplify this as $4(4x^{2}+y^{2})(4x^{2}-y^{2})$. Again by the difference of squares, we have $4x^{2}-y^{2}=(2x)^{2}-y^{2}=(2x-y)(2x+y)$. Piecing this all back together, we have $64x^{2}-4y^{2}=4(4x^{2}+y^{2})(2x-y)(2x+y)$.
        \end{proof}
        \begin{problem}
        $8x^{3}-27$
        \end{problem}
        \begin{proof}[Solution]
        By Thm.~\ref{th:north_shore_difference_of_cubes}, $8x^{3}-27=(2x)^{3}-(3)^{3}=(2x-3)(4x^{2}+6x+9)$
        \end{proof}
        \begin{problem}
        $49y^{2}+84y+36$
        \end{problem}
        \begin{proof}[Solution]
        First note that $49=7^{2}$, $36=6^{2}$, and $84=2\cdot 6\cdot 7$. So, $49x^{2}+84x+36=(7x)^{2}+2(7)(6)x+(6)^{2}$. But $(ax+b)^{2}=a^{2}x^{2}+2abx+b^{2}$. We have $a=7,b=6$, so $49x^{2}+84x+36=(7x+6)^{2}$.
        \end{proof}
        \begin{problem}
        $12x^{2}+12x+3$
        \end{problem}
        \begin{proof}[Solution]
        $4x^{2}+4x+1=3((2x)^{2}+2(2)(1)(x)+(1)^{2})=3(2x+1)^{2}$
        \end{proof}
    \subsection{Quadratic Expressions}
        \begin{ftheorem}{Completing the Square}{north_shore_completing_the_square}
            If $y=ax^{2}+bx+c$,
            then $y=a(x+\frac{b}{2a})^{2}-\frac{b^2}{4a}+c$
        \end{ftheorem}
        \begin{proof}
            For:
            \begin{align*}
                \Big(x+\frac{b}{2a}\Big)^{2}
                &=x^{2}+\frac{b}{a}x+\frac{b^{2}}{4a^{2}}\\
                \Rightarrow
                a\Big(x+\frac{b}{2a}\Big)^{2}
                &=ax^{2}+bx+\frac{b^{2}}{4a}\\
                \Rightarrow
                a\Big(x+\frac{b}{2a}\Big)^{2}-
                \frac{b^{2}}{4a}
                &=ax^{2}+bx\\
                \Rightarrow
                a\Big(x+\frac{b}{2a}\Big)^{2}-
                \frac{b^{2}}{4a}+c
                &=ax^{2}+bx+c
            \end{align*}
        \end{proof}
        \begin{ftheorem}{The Quadratic Formula}{north_shore_quadratic_formula_theorem}
            If $ax^{2}+bx+c=0$ and $a\ne 0$,
            then:
            \begin{equation*}
                x=\frac{-b\pm\sqrt{b^{2}-4ac}}{2a}
            \end{equation*}
        \end{ftheorem}
        \begin{proof}
        By Thm.~\ref{th:north_shore_completing_the_square}, $ax^{2}+bx+c=a(x+\frac{b}{2a})^{2}-\frac{b^{2}}{4a}+c$. But $ax^{2}+bx+c=0$, so $a(x+\frac{b}{2a}x)^{2}-\frac{b^{2}}{4a}+c=0$. Therefore $a(x+\frac{b}{2a})^{2}=\frac{b^2}{4a}-c$. But $c=\frac{4ac}{4a}$, so $a(x^2+\frac{b}{2a})^{2}=\frac{b^{2}-4ac}{4a}$. Dividing both sides by $a$, we get $(x+\frac{b}{2a})^{2}=\frac{b^{2}-4ac}{4a^{2}}$. Now we take square roots, but note that there are two possible square roots. So, $x+\frac{b}{2a}=\pm\frac{\sqrt{b^{2}-4ac}}{2a}$. Subtracting $\frac{b}{2a}$ from both sides we get $x=-\frac{b}{2a}\pm\frac{\sqrt{b^{2}-4ac}}{2a}=\frac{-b\pm\sqrt{b^{2}-4ac}}{2a}$
        \end{proof}
        \begin{remark}
        Remember that $ax^{2}+bx+c=0$ has two solutions: $x=\frac{-b+\sqrt{b^{2}-4ac}}{2a}$ and $x=\frac{-b-\sqrt{b^{2}-4ac}}{2a}$
        \end{remark}
        \begin{ftheorem}{}
                       {north_shore_zeros_of_a_factored_polynomial}
        If $\gamma(x-\alpha)(x-\beta)=0$, then either $x=\alpha$ or $x=\beta$.
        \end{ftheorem}
        \begin{remark}
        If we see $ax^{2}+bx+c=0$ and can \textit{factor} into $\gamma(x-\alpha)(x-\beta)$, then $x=\alpha$ or $x=\beta$.
        \end{remark}
        \subsubsection{Problems}
        \begin{problem}
        Find all solutions for a: $4a^{2}+9a+2=0$
        \end{problem}
        \begin{proof}[Solution]
        By Thm.~\ref{th:north_shore_quadratic_formula_theorem}, $a=\frac{-9\pm\sqrt{9^{2}-4\cdot 4\cdot 2}}{2\cdot 4}=\frac{-9\pm 7}{8}$. $a=-2$ or $a=-\frac{1}{4}$
        \end{proof}
        \begin{problem}
        Find all solutions for $x$: $9x^{2}-81=0$
        \end{problem}
        \begin{proof}[Solution]
        By Thm.~\ref{th:north_shore_difference_of_squares}, $9x^{2}-81=9(x-3)(x+3)$. By Thm.~\ref{th:north_shore_zeros_of_a_factored_polynomial}, $x=3$ or $x=-3$.
        \end{proof}
        \begin{problem}
        Find all solutions for $x$: $25x^{2}-6=30$.
        \end{problem}
        \begin{proof}[Solution]
        Subtract $30$ to get $25x^{2}-36=0$. By Thm.~\ref{th:north_shore_difference_of_squares}, $25x^{2}-36 =(5x-6)(5x+6)$. By Thm.~\ref{th:north_shore_zeros_of_a_factored_polynomial}, $5x=6$ or $5x=-6$. $x=\frac{6}{5}$ or $x=-\frac{6}{5}$.
        \end{proof}
        \begin{problem}
        Find all solutions for $x$: $3x^{2}-5x-2=0$
        \end{problem}
        \begin{proof}[Solution]
        By Thm.~\ref{th:north_shore_quadratic_formula_theorem}, $x=\frac{5\pm\sqrt{(-5)^{2}-4(3)(-2)}}{2(3)}=\frac{5\pm 7}{6}$. So $x=2$ or $x=-\frac{1}{3}$
        \end{proof}
        \begin{problem}
        Find all solutions for $x$: $(3x+2)^{2}=16$
        \end{problem}
        \begin{proof}[Solution]
        Subtracting $16$ we get $(3x+2)^{2}-16$. By Thm.~\ref{th:north_shore_difference_of_squares}, $(3x+2)^{2}-16=(3x+6)(3x-2)$. By Thm.~\ref{th:north_shore_zeros_of_a_factored_polynomial}, $3x=-6$ or $3x=2$. Thus, $x=-2$ or $x=\frac{2}{3}$.
        \end{proof}
        \begin{problem}
        Find all solutions for $r$: $r^{2}-2r-4=0$
        \end{problem}
        \begin{proof}[Solution]
        By Thm.~\ref{th:north_shore_quadratic_formula_theorem}, $r=\frac{2\pm\sqrt{(-2)^{2}-4(1)(-4)}}{2(1)}=1\pm\sqrt{5}$. $x=1+\sqrt{5}$ or $x=1-\sqrt{5}$
        \end{proof}
    \subsection{Rational Expressions}
        \begin{ftheorem}{Cross Multiplication}{north_shore_cross_multiplying}
        If $a,b,c,d$ are real numbers, $b\ne 0$, $d\ne 0$, then $\frac{a}{b}+\frac{c}{d}=\frac{ad+bc}{bd}$
        \end{ftheorem}
        \begin{fdefinition}{Numerator of a Rational Expression}
                          {definition:north_shore_numerator_%
                           of_rational_expression}
            The numerator of a rational expression $\frac{P(x)}{Q(x)}$
            is the polynomial $P(x)$.
        \end{fdefinition}
        \begin{fdefinition}{Denominator of a Rational Expression}
                          {definition:north_shore_denominator_%
                           of_rational_expression}
            The denominator of a rational expression $\frac{P(x)}{Q(x)}$
            is the polynomial $Q(x)$.
        \end{fdefinition}
        \subsubsection{Problems}
        \begin{problem}
        Simplify: $\frac{4}{2a-2}+\frac{3a}{a^{2}-a}$
        \end{problem}
        \begin{proof}[Solution]
            \begin{flalign*}
                \tfrac{4}{2a-2}+\tfrac{3a}{a^{2}-a}&=\tfrac{2}{a-1}+\tfrac{3}{a-1}&\tag{Factor and Simplify Terms}\\
                &=\tfrac{5}{a-1}&\tag{Addition}
            \end{flalign*}
        \end{proof}
        \begin{problem}
        Simplify: $\frac{3}{x^{2}-1}-\frac{4}{x^{2}+3x+2}$
        \end{problem}
        \begin{proof}[Solution]
            \begin{flalign*}
                \tfrac{3}{x^{2}-1}-\tfrac{4}{x^{2}+3x+2}&=\tfrac{3}{(x-1)(x+1)}-\tfrac{4}{(x+2)(x+1)}&\tag{Factor Denominators}\\
                &=\tfrac{3(x+2)-4(x-1)}{(x-1)(x+2)(x+1)}&\tag{Thm.~\ref{th:north_shore_cross_multiplying}}\\
                &=\tfrac{10-x}{(x+1)(x+2)(x-1)}&\tag{Simplify Numerator}
            \end{flalign*}
        \end{proof}
        \begin{problem}
        Simplify: $\frac{\frac{2}{x}-\frac{1}{y}}{\frac{1}{xy}}$
        \end{problem}
        \begin{proof}[Solution]
            \begin{flalign*}
                \tfrac{\frac{2}{x}-\frac{1}{y}}{\frac{1}{xy}}&=xy(\tfrac{2}{x}-\tfrac{1}{y})&\tag{Property~\ref{property:North_Shore_Exponent_Rules} part~\ref{property:north_shore_inverse_property_of_expo}}\\
                &=2y-x&\tag{Multiplication}
            \end{flalign*}
        \end{proof}
        \begin{problem}
        Simplify: $\frac{1}{1-\sqrt{x}}+\frac{1}{1+\sqrt{x}}$
        \end{problem}
        \begin{proof}[Solution]
            \begin{flalign*}
                \tfrac{1}{1-\sqrt{x}}+\tfrac{1}{1+\sqrt{x}}&=\tfrac{1+\sqrt{x} + 1-\sqrt{x}}{(1-\sqrt{x})(1+\sqrt{x})}&\tag{Thm.~\ref{th:north_shore_cross_multiplying}}\\
                &=\tfrac{2}{1-x}&\tag{Simplify and \gls{foil}}
            \end{flalign*}
        \end{proof}
        \begin{problem}
        $\frac{2}{x-1}+\frac{1}{x+1}=\frac{5}{4}$
        \end{problem}
        \begin{proof}[Solution]
            \begin{flalign*}
                \tfrac{2}{x-1}+\tfrac{1}{x+1}&=\tfrac{2(x+1)+(x-1)}{(x+1)(x-1)}&\tag{Thm.~\ref{th:north_shore_cross_multiplying}}\\
                &=\tfrac{3x+1}{(x+1)(x-1)}&\tag{Simplify the Numerator}\\
                \Rightarrow \tfrac{3x+1}{(x+1)(x-1)}&=\tfrac{5}{4}&\tag{Substitution}\\
                \Rightarrow 3x+1&=\tfrac{5}{4}(x+1)(x-1)&\tag{Multiply Both Sides by $(x+1)(x-1)$}\\
                \Rightarrow 12x+4&=5x^{2}-5&\tag{Multiply Both Sides by $4$ and Simplify}\\
                \Rightarrow 5x^{2}-12x-9&=0&\tag{Subtract $12x+4$ from Both Sides}\\
                \Rightarrow x&=\tfrac{12\pm\sqrt{(12)^{2}-4\cdot (5)\cdot(-9)}}{2\cdot(5)}&\tag{Thm.~\ref{th:north_shore_quadratic_formula_theorem}}\\
                \Rightarrow x&=\tfrac{12\pm\sqrt{324}}{10}&\tag{Simplify}\\
                \Rightarrow x&=\tfrac{12\pm 18}{10}&\tag{Simplify}\\
                \Rightarrow x&=3\textrm{ or }-\tfrac{3}{5}&\tag{Simplify}
            \end{flalign*}
        \end{proof}
    \subsection{Graphing}
        \subsubsection{Problems}
        \begin{problem}
        \label{problem:north_shore_exam_graph_everything}Graph the following:
        \begin{enumerate}
        \begin{multicols}{4}
            \item $3x-2y=6$
            \item $x=-3$
            \item $y=2$
            \item $y=-\frac{2}{3}x+5$
            \item $y=|x-3|$
            \item $y=-x^2+2$
            \item $y=\sqrt{2-x}$
            \item $y=x-10$
        \end{multicols}
        \end{enumerate}
        \end{problem}
        \begin{figure}[H]
            \centering
            \captionsetup{type=figure}
            \subimport{../../../tikz/}{Elementary_Algebra_Graphing_Problem}
            \caption{The chaos that is the solution to problem
                     \ref{problem:north_shore_exam_graph_everything}.}
            \label{fig:north_shore_graphing_problem}
        \end{figure}
    \subsection{Systems of Equations}
        A system of equations is a set of 2 or more equations involving the same variables. Solving systems of linear equations is one of the main focuses in the study of linear algebra.
        \begin{fexample}{}{}
        \label{example:north_shore_example_of_a_system_of_linear_equations}Consider the following system of linear equations:
        \begin{align*}
            2x+3y&=1\\
            x+6y&=2
        \end{align*}
        Solving this system of equations asks ``Which ordered pairs $(x,y)$ solve both of these equations?" The second equations says that $x=2-6y$. We can substitute this back into the first equation to get $2(2-6y)+3y=1$ which simplifies to $4-9y=1$. The solution to this is $y=\frac{1}{3}$. Solving for $x$, we have $x=2-6y=2-6(\frac{1}{3})=2-2=0$. $(0,\frac{1}{3})$ is a solution.
        \end{fexample}
        We can picture these types of problems graphically as well. For the example above there are two linear equations, which can be represented graphically as lines. A solution to this system can then be interpreted as a point where the two lines intersect. The graph of the two equations in example \ref{example:north_shore_example_of_a_system_of_linear_equations} are shown in Fig.~\ref{fig:north_shore_systems_of_linear_equations}.
        \begin{figure}[H]
            \centering
            \captionsetup{type=figure}
            \subimport{../../../tikz/}{System_of_Equations_Geometry}
            \caption{The graphs of the two equations shown in example \ref{example:north_shore_example_of_a_system_of_linear_equations}.}
            \label{fig:north_shore_systems_of_linear_equations}
        \end{figure}
        Any system of linear equations has 3 possible outcomes: No solutions, one solution, or infinitely many solutions. This makes sense if one considers the few possibilities allowed. Given two parallel lines there will be no solutions for parallel lines never intersect. Given two equations that represent the same line there will be infinitely many solutions, for any point on the line will work. Given two equations for two distinct non-parallel lines there will be only one solution, for such lines can only intersect once.
        \subsubsection{Problems}
        \begin{problem}
        Solve the following system of equations:
        \begin{align*}
            2x-3y&=-12\\
            x-2y&=-9
        \end{align*}
        \end{problem}
        \begin{proof}[Solution]
        Subtracting 2 of the second equation from first equation, we get:
        \begin{equation*}
            (2x-3y)-2(x-2y)=-12-2(-9)
        \end{equation*}
        Which simplifies to $y=6$. The second equation yield $x-2(6)=-9\Rightarrow x=3$. The solution is $(3,6)$
        \end{proof}
        \begin{problem}
        Solve the following system of equations:
        \begin{align*}
            4x+6y&=10\\
            2x+3y&=5
        \end{align*}
        \end{problem}
        \begin{proof}[Solution]
        Dividing both sides of the first equation by 2 gives us the second equation. These equations represent the same line, and there are infinitely many solutions: $y=\frac{5}{3}-\frac{2}{3}x$
        \end{proof}
        \begin{problem}
        Solve the following system of equations:
        \begin{align*}
            x+2y&=5\\
            x+2y&=7
        \end{align*}
        \end{problem}
        \begin{proof}[Solution]
        Let $z=x+2y$. Then $z=5$ and $z=7$. But this is impossible, for $5\ne 7$. No solutions.
        \end{proof}
        \begin{problem}
        \begin{align*}
            2x-3y&=-4\\
            2x+y\phantom{3}&=\phantom{-}4
        \end{align*}
        \end{problem}
        \begin{proof}[Solution]
        Subtracting the first equation from the second gives us $4y=8\Rightarrow y=2$. But then from the first equation we obtain $2x=3(2)-4=2\Rightarrow x=1$. The solution is $(1,2)$.
        \end{proof}
    \subsection{Radicals}
        To simplify expressions with radicals we use something called the conjugate of a radical expression. 
        \begin{fdefinition}
        The conjugate of a rational expression $\sqrt{a}-\sqrt{b}$ is $\sqrt{a}+\sqrt{b}$.
        \end{fdefinition}
        Using the \gls{foil} rule from before, we can obtain the following theorem.
        \begin{ftheorem}{}
            If $a,b\geq 0$, and if $a\ne b$, then
            $\frac{1}{\sqrt{a}-\sqrt{b}}=\frac{\sqrt{a}+\sqrt{b}}{a-b}$
        \end{ftheorem}
        \begin{proof}
        As $a,b\geq 0$ and $a\ne b$, $\frac{1}{\sqrt{a}-\sqrt{b}}$ is well-defined and $\sqrt{a}+\sqrt{b}\ne 0$. But then:
        \begin{align*}
            \tfrac{1}{\sqrt{a}-\sqrt{b}}&=\tfrac{1}{\sqrt{a}-\sqrt{b}}\tfrac{\sqrt{a}+\sqrt{b}}{\sqrt{a}+\sqrt{b}}\\
            &=\tfrac{\sqrt{a}+\sqrt{b}}{(\sqrt{a}-\sqrt{b})(\sqrt{a}+\sqrt{b})}\\
            &=\tfrac{\sqrt{a}+\sqrt{b}}{a-b}&\tag{Thm.~\ref{th:north_shore_difference_of_squares}}
        \end{align*}
        \end{proof}
        \subsubsection{Problems}
        \begin{problem}
        Simplify the following so that there are no radicals in the denominator:
        \begin{enumerate}
        \begin{multicols}{3}
            \item $\sqrt{8}\sqrt{10}$
            \item $\sqrt[4]{\frac{81}{x^4}}$
            \item $\sqrt{\frac{4}{3}}$
            \item $\sqrt{\frac{12}{18}}$
            \item $\sqrt[3]{24x^{3}y^{6}}$
            \item $\frac{\sqrt{3}}{5-\sqrt{3}}$
        \end{multicols}
        \end{enumerate}
        \end{problem}
        \begin{proof}[Solution]
        \
        \begin{enumerate}
        \begin{multicols}{3}
            \item $\sqrt{8}\sqrt{10}=\sqrt{16\cdot 5}=4\sqrt{5}$
            \item $\sqrt[4]{\frac{81}{x^{4}}}=\frac{3}{|x|}$
            \item $\sqrt{\frac{4}{3}}=\frac{2}{\sqrt{3}}=\frac{2\sqrt{3}}{3}$
            \item $\sqrt{\frac{12}{18}}=\frac{2\sqrt{3}}{3\sqrt{2}}=\frac{2\sqrt{2}\sqrt{3}}{6}$
            \item $\sqrt[3]{24x^{3}y^{6}}=2\sqrt[3]{3}xy^{2}$
            \item $\frac{\sqrt{3}}{5-\sqrt{3}}=\frac{\sqrt{3}(5+\sqrt{3})}{5-3}=\frac{5\sqrt{3}+3}{2}$
        \end{multicols}
        \end{enumerate}
        \end{proof}
    \ifx\ifmathcourseselementaryalgebra\undefined
        \newpage
        \printglossary[type=\acronymtype]
        \newpage
        \printglossary[style=long]
    \fi

    % Bibliographies and Glossaries
    \ifx\ifmathcourses\undefined
        \clearpage
        \printglossary[type=\acronymtype]
        \clearpage
        \printglossary[style=long]
    \fi
\end{document}