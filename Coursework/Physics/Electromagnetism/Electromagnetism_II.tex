\documentclass[crop=false,class=book,oneside]{standalone}
%----------------------------Preamble-------------------------------%
%---------------------------Packages----------------------------%
\usepackage{geometry}
\geometry{b5paper, margin=1.0in}
\usepackage[T1]{fontenc}
\usepackage{graphicx, float}            % Graphics/Images.
\usepackage{natbib}                     % For bibliographies.
\bibliographystyle{agsm}                % Bibliography style.
\usepackage[french, english]{babel}     % Language typesetting.
\usepackage[dvipsnames]{xcolor}         % Color names.
\usepackage{listings}                   % Verbatim-Like Tools.
\usepackage{mathtools, esint, mathrsfs} % amsmath and integrals.
\usepackage{amsthm, amsfonts, amssymb}  % Fonts and theorems.
\usepackage{tcolorbox}                  % Frames around theorems.
\usepackage{upgreek}                    % Non-Italic Greek.
\usepackage{fmtcount, etoolbox}         % For the \book{} command.
\usepackage[newparttoc]{titlesec}       % Formatting chapter, etc.
\usepackage{titletoc}                   % Allows \book in toc.
\usepackage[nottoc]{tocbibind}          % Bibliography in toc.
\usepackage[titles]{tocloft}            % ToC formatting.
\usepackage{pgfplots, tikz}             % Drawing/graphing tools.
\usepackage{imakeidx}                   % Used for index.
\usetikzlibrary{
    calc,                   % Calculating right angles and more.
    angles,                 % Drawing angles within triangles.
    arrows.meta,            % Latex and Stealth arrows.
    quotes,                 % Adding labels to angles.
    positioning,            % Relative positioning of nodes.
    decorations.markings,   % Adding arrows in the middle of a line.
    patterns,
    arrows
}                                       % Libraries for tikz.
\pgfplotsset{compat=1.9}                % Version of pgfplots.
\usepackage[font=scriptsize,
            labelformat=simple,
            labelsep=colon]{subcaption} % Subfigure captions.
\usepackage[font={scriptsize},
            hypcap=true,
            labelsep=colon]{caption}    % Figure captions.
\usepackage[pdftex,
            pdfauthor={Ryan Maguire},
            pdftitle={Mathematics and Physics},
            pdfsubject={Mathematics, Physics, Science},
            pdfkeywords={Mathematics, Physics, Computer Science, Biology},
            pdfproducer={LaTeX},
            pdfcreator={pdflatex}]{hyperref}
\hypersetup{
    colorlinks=true,
    linkcolor=blue,
    filecolor=magenta,
    urlcolor=Cerulean,
    citecolor=SkyBlue
}                           % Colors for hyperref.
\usepackage[toc,acronym,nogroupskip,nopostdot]{glossaries}
\usepackage{glossary-mcols}
%------------------------Theorem Styles-------------------------%
\theoremstyle{plain}
\newtheorem{theorem}{Theorem}[section]

% Define theorem style for default spacing and normal font.
\newtheoremstyle{normal}
    {\topsep}               % Amount of space above the theorem.
    {\topsep}               % Amount of space below the theorem.
    {}                      % Font used for body of theorem.
    {}                      % Measure of space to indent.
    {\bfseries}             % Font of the header of the theorem.
    {}                      % Punctuation between head and body.
    {.5em}                  % Space after theorem head.
    {}

% Italic header environment.
\newtheoremstyle{thmit}{\topsep}{\topsep}{}{}{\itshape}{}{0.5em}{}

% Define environments with italic headers.
\theoremstyle{thmit}
\newtheorem*{solution}{Solution}

% Define default environments.
\theoremstyle{normal}
\newtheorem{example}{Example}[section]
\newtheorem{definition}{Definition}[section]
\newtheorem{problem}{Problem}[section]

% Define framed environment.
\tcbuselibrary{most}
\newtcbtheorem[use counter*=theorem]{ftheorem}{Theorem}{%
    before=\par\vspace{2ex},
    boxsep=0.5\topsep,
    after=\par\vspace{2ex},
    colback=green!5,
    colframe=green!35!black,
    fonttitle=\bfseries\upshape%
}{thm}

\newtcbtheorem[auto counter, number within=section]{faxiom}{Axiom}{%
    before=\par\vspace{2ex},
    boxsep=0.5\topsep,
    after=\par\vspace{2ex},
    colback=Apricot!5,
    colframe=Apricot!35!black,
    fonttitle=\bfseries\upshape%
}{ax}

\newtcbtheorem[use counter*=definition]{fdefinition}{Definition}{%
    before=\par\vspace{2ex},
    boxsep=0.5\topsep,
    after=\par\vspace{2ex},
    colback=blue!5!white,
    colframe=blue!75!black,
    fonttitle=\bfseries\upshape%
}{def}

\newtcbtheorem[use counter*=example]{fexample}{Example}{%
    before=\par\vspace{2ex},
    boxsep=0.5\topsep,
    after=\par\vspace{2ex},
    colback=red!5!white,
    colframe=red!75!black,
    fonttitle=\bfseries\upshape%
}{ex}

\newtcbtheorem[auto counter, number within=section]{fnotation}{Notation}{%
    before=\par\vspace{2ex},
    boxsep=0.5\topsep,
    after=\par\vspace{2ex},
    colback=SeaGreen!5!white,
    colframe=SeaGreen!75!black,
    fonttitle=\bfseries\upshape%
}{not}

\newtcbtheorem[use counter*=remark]{fremark}{Remark}{%
    fonttitle=\bfseries\upshape,
    colback=Goldenrod!5!white,
    colframe=Goldenrod!75!black}{ex}

\newenvironment{bproof}{\textit{Proof.}}{\hfill$\square$}
\tcolorboxenvironment{bproof}{%
    blanker,
    breakable,
    left=3mm,
    before skip=5pt,
    after skip=10pt,
    borderline west={0.6mm}{0pt}{green!80!black}
}

\AtEndEnvironment{lexample}{$\hfill\textcolor{red}{\blacksquare}$}
\newtcbtheorem[use counter*=example]{lexample}{Example}{%
    empty,
    title={Example~\theexample},
    boxed title style={%
        empty,
        size=minimal,
        toprule=2pt,
        top=0.5\topsep,
    },
    coltitle=red,
    fonttitle=\bfseries,
    parbox=false,
    boxsep=0pt,
    before=\par\vspace{2ex},
    left=0pt,
    right=0pt,
    top=3ex,
    bottom=1ex,
    before=\par\vspace{2ex},
    after=\par\vspace{2ex},
    breakable,
    pad at break*=0mm,
    vfill before first,
    overlay unbroken={%
        \draw[red, line width=2pt]
            ([yshift=-1.2ex]title.south-|frame.west) to
            ([yshift=-1.2ex]title.south-|frame.east);
        },
    overlay first={%
        \draw[red, line width=2pt]
            ([yshift=-1.2ex]title.south-|frame.west) to
            ([yshift=-1.2ex]title.south-|frame.east);
    },
}{ex}

\AtEndEnvironment{ldefinition}{$\hfill\textcolor{Blue}{\blacksquare}$}
\newtcbtheorem[use counter*=definition]{ldefinition}{Definition}{%
    empty,
    title={Definition~\thedefinition:~{#1}},
    boxed title style={%
        empty,
        size=minimal,
        toprule=2pt,
        top=0.5\topsep,
    },
    coltitle=Blue,
    fonttitle=\bfseries,
    parbox=false,
    boxsep=0pt,
    before=\par\vspace{2ex},
    left=0pt,
    right=0pt,
    top=3ex,
    bottom=0pt,
    before=\par\vspace{2ex},
    after=\par\vspace{1ex},
    breakable,
    pad at break*=0mm,
    vfill before first,
    overlay unbroken={%
        \draw[Blue, line width=2pt]
            ([yshift=-1.2ex]title.south-|frame.west) to
            ([yshift=-1.2ex]title.south-|frame.east);
        },
    overlay first={%
        \draw[Blue, line width=2pt]
            ([yshift=-1.2ex]title.south-|frame.west) to
            ([yshift=-1.2ex]title.south-|frame.east);
    },
}{def}

\AtEndEnvironment{ltheorem}{$\hfill\textcolor{Green}{\blacksquare}$}
\newtcbtheorem[use counter*=theorem]{ltheorem}{Theorem}{%
    empty,
    title={Theorem~\thetheorem:~{#1}},
    boxed title style={%
        empty,
        size=minimal,
        toprule=2pt,
        top=0.5\topsep,
    },
    coltitle=Green,
    fonttitle=\bfseries,
    parbox=false,
    boxsep=0pt,
    before=\par\vspace{2ex},
    left=0pt,
    right=0pt,
    top=3ex,
    bottom=-1.5ex,
    breakable,
    pad at break*=0mm,
    vfill before first,
    overlay unbroken={%
        \draw[Green, line width=2pt]
            ([yshift=-1.2ex]title.south-|frame.west) to
            ([yshift=-1.2ex]title.south-|frame.east);},
    overlay first={%
        \draw[Green, line width=2pt]
            ([yshift=-1.2ex]title.south-|frame.west) to
            ([yshift=-1.2ex]title.south-|frame.east);
    }
}{thm}

%--------------------Declared Math Operators--------------------%
\DeclareMathOperator{\adjoint}{adj}         % Adjoint.
\DeclareMathOperator{\Card}{Card}           % Cardinality.
\DeclareMathOperator{\curl}{curl}           % Curl.
\DeclareMathOperator{\diam}{diam}           % Diameter.
\DeclareMathOperator{\dist}{dist}           % Distance.
\DeclareMathOperator{\Div}{div}             % Divergence.
\DeclareMathOperator{\Erf}{Erf}             % Error Function.
\DeclareMathOperator{\Erfc}{Erfc}           % Complementary Error Function.
\DeclareMathOperator{\Ext}{Ext}             % Exterior.
\DeclareMathOperator{\GCD}{GCD}             % Greatest common denominator.
\DeclareMathOperator{\grad}{grad}           % Gradient
\DeclareMathOperator{\Ima}{Im}              % Image.
\DeclareMathOperator{\Int}{Int}             % Interior.
\DeclareMathOperator{\LC}{LC}               % Leading coefficient.
\DeclareMathOperator{\LCM}{LCM}             % Least common multiple.
\DeclareMathOperator{\LM}{LM}               % Leading monomial.
\DeclareMathOperator{\LT}{LT}               % Leading term.
\DeclareMathOperator{\Mod}{mod}             % Modulus.
\DeclareMathOperator{\Mon}{Mon}             % Monomial.
\DeclareMathOperator{\multideg}{mutlideg}   % Multi-Degree (Graphs).
\DeclareMathOperator{\nul}{nul}             % Null space of operator.
\DeclareMathOperator{\Ord}{Ord}             % Ordinal of ordered set.
\DeclareMathOperator{\Prin}{Prin}           % Principal value.
\DeclareMathOperator{\proj}{proj}           % Projection.
\DeclareMathOperator{\Refl}{Refl}           % Reflection operator.
\DeclareMathOperator{\rk}{rk}               % Rank of operator.
\DeclareMathOperator{\sgn}{sgn}             % Sign of a number.
\DeclareMathOperator{\sinc}{sinc}           % Sinc function.
\DeclareMathOperator{\Span}{Span}           % Span of a set.
\DeclareMathOperator{\Spec}{Spec}           % Spectrum.
\DeclareMathOperator{\supp}{supp}           % Support
\DeclareMathOperator{\Tr}{Tr}               % Trace of matrix.
%--------------------Declared Math Symbols--------------------%
\DeclareMathSymbol{\minus}{\mathbin}{AMSa}{"39} % Unary minus sign.
%------------------------New Commands---------------------------%
\DeclarePairedDelimiter\norm{\lVert}{\rVert}
\DeclarePairedDelimiter\ceil{\lceil}{\rceil}
\DeclarePairedDelimiter\floor{\lfloor}{\rfloor}
\newcommand*\diff{\mathop{}\!\mathrm{d}}
\newcommand*\Diff[1]{\mathop{}\!\mathrm{d^#1}}
\renewcommand*{\glstextformat}[1]{\textcolor{RoyalBlue}{#1}}
\renewcommand{\glsnamefont}[1]{\textbf{#1}}
\renewcommand\labelitemii{$\circ$}
\renewcommand\thesubfigure{%
    \arabic{chapter}.\arabic{figure}.\arabic{subfigure}}
\addto\captionsenglish{\renewcommand{\figurename}{Fig.}}
\numberwithin{equation}{section}

\renewcommand{\vector}[1]{\boldsymbol{\mathrm{#1}}}

\newcommand{\uvector}[1]{\boldsymbol{\hat{\mathrm{#1}}}}
\newcommand{\topspace}[2][]{(#2,\tau_{#1})}
\newcommand{\measurespace}[2][]{(#2,\varSigma_{#1},\mu_{#1})}
\newcommand{\measurablespace}[2][]{(#2,\varSigma_{#1})}
\newcommand{\manifold}[2][]{(#2,\tau_{#1},\mathcal{A}_{#1})}
\newcommand{\tanspace}[2]{T_{#1}{#2}}
\newcommand{\cotanspace}[2]{T_{#1}^{*}{#2}}
\newcommand{\Ckspace}[3][\mathbb{R}]{C^{#2}(#3,#1)}
\newcommand{\funcspace}[2][\mathbb{R}]{\mathcal{F}(#2,#1)}
\newcommand{\smoothvecf}[1]{\mathfrak{X}(#1)}
\newcommand{\smoothonef}[1]{\mathfrak{X}^{*}(#1)}
\newcommand{\bracket}[2]{[#1,#2]}

%------------------------Book Command---------------------------%
\makeatletter
\renewcommand\@pnumwidth{1cm}
\newcounter{book}
\renewcommand\thebook{\@Roman\c@book}
\newcommand\book{%
    \if@openright
        \cleardoublepage
    \else
        \clearpage
    \fi
    \thispagestyle{plain}%
    \if@twocolumn
        \onecolumn
        \@tempswatrue
    \else
        \@tempswafalse
    \fi
    \null\vfil
    \secdef\@book\@sbook
}
\def\@book[#1]#2{%
    \refstepcounter{book}
    \addcontentsline{toc}{book}{\bookname\ \thebook:\hspace{1em}#1}
    \markboth{}{}
    {\centering
     \interlinepenalty\@M
     \normalfont
     \huge\bfseries\bookname\nobreakspace\thebook
     \par
     \vskip 20\p@
     \Huge\bfseries#2\par}%
    \@endbook}
\def\@sbook#1{%
    {\centering
     \interlinepenalty \@M
     \normalfont
     \Huge\bfseries#1\par}%
    \@endbook}
\def\@endbook{
    \vfil\newpage
        \if@twoside
            \if@openright
                \null
                \thispagestyle{empty}%
                \newpage
            \fi
        \fi
        \if@tempswa
            \twocolumn
        \fi
}
\newcommand*\l@book[2]{%
    \ifnum\c@tocdepth >-3\relax
        \addpenalty{-\@highpenalty}%
        \addvspace{2.25em\@plus\p@}%
        \setlength\@tempdima{3em}%
        \begingroup
            \parindent\z@\rightskip\@pnumwidth
            \parfillskip -\@pnumwidth
            {
                \leavevmode
                \Large\bfseries#1\hfill\hb@xt@\@pnumwidth{\hss#2}
            }
            \par
            \nobreak
            \global\@nobreaktrue
            \everypar{\global\@nobreakfalse\everypar{}}%
        \endgroup
    \fi}
\newcommand\bookname{Book}
\renewcommand{\thebook}{\texorpdfstring{\Numberstring{book}}{book}}
\providecommand*{\toclevel@book}{-2}
\makeatother
\titleformat{\part}[display]
    {\Large\bfseries}
    {\partname\nobreakspace\thepart}
    {0mm}
    {\Huge\bfseries}
\titlecontents{part}[0pt]
    {\large\bfseries}
    {\partname\ \thecontentslabel: \quad}
    {}
    {\hfill\contentspage}
\titlecontents{chapter}[0pt]
    {\bfseries}
    {\chaptername\ \thecontentslabel:\quad}
    {}
    {\hfill\contentspage}
\newglossarystyle{longpara}{%
    \setglossarystyle{long}%
    \renewenvironment{theglossary}{%
        \begin{longtable}[l]{{p{0.25\hsize}p{0.65\hsize}}}
    }{\end{longtable}}%
    \renewcommand{\glossentry}[2]{%
        \glstarget{##1}{\glossentryname{##1}}%
        &\glossentrydesc{##1}{~##2.}
        \tabularnewline%
        \tabularnewline
    }%
}
\newglossary[not-glg]{notation}{not-gls}{not-glo}{Notation}
\newcommand*{\newnotation}[4][]{%
    \newglossaryentry{#2}{type=notation, name={\textbf{#3}, },
                          text={#4}, description={#4},#1}%
}
%--------------------------LENGTHS------------------------------%
% Spacings for the Table of Contents.
\addtolength{\cftsecnumwidth}{1ex}
\addtolength{\cftsubsecindent}{1ex}
\addtolength{\cftsubsecnumwidth}{1ex}
\addtolength{\cftfignumwidth}{1ex}
\addtolength{\cfttabnumwidth}{1ex}

% Indent and paragraph spacing.
\setlength{\parindent}{0em}
\setlength{\parskip}{0em}
\graphicspath{{../../../images/}}       % Path to Image Folder.
%----------------------------GLOSSARY-------------------------------%
\makeglossaries
\loadglsentries{../../../glossary}
\loadglsentries{../../../acronym}
%--------------------------Main Document----------------------------%
\begin{document}
    \ifx\ifphysicscourses\undefined
        \pagenumbering{roman}
        \title{Electromagnetism II}
        \author{Ryan Maguire}
        \date{\vspace{-5ex}}
        \maketitle
        \tableofcontents
        \listoffigures
        \clearpage
        \setcounter{chapter}{28}
        \chapter{Electromagnetism II}
        \pagenumbering{arabic}
    \else
        \chapter{Electromagnetism II}
    \fi
    \section{Homework Sets}
        \subsection{Homework I}
            Wangsness Chapter 8 - Problems: 2, 4, 7, 8
            \begin{problem}
                \label{problem:EMAG_II_Wangsnes_8_2}
                Given a single point charge $q$ located at the
                point $(a,b,c)$, Find $Q$, $\mathbf{p}$, and
                all components of $Q_{jk}$ for this system. Which
                coordinates are changed there is a charge $\minus{q}$
                at the origin?
            \end{problem}
            \begin{solution}
                The total charge is:
                \begin{equation}
                    Q=\sum_{n=1}^{N}q_{n}
                \end{equation}
                The only charge in this system is $q$, so we have
                $Q=q$. Next, $\mathbf{p}$ is defined as:
                \begin{equation}
                    \mathbf{p}=
                    \sum_{n=1}^{N}q_{n}\mathbf{r}_{n}
                \end{equation}
                Using this, we have:
                \begin{equation}
                    \mathbf{p}=
                    q(a\hat{\mathbf{x}}+b\hat{\mathbf{y}}+c\hat{\mathbf{z}})
                \end{equation}
                Finally, the quadrupole moments are defined as:
                \begin{equation}
                    Q_{jk}=
                    \sum_{i=1}^{N}q_{i}(3x_{i}y_{i}-r^{2}_{i}\delta_{jk})
                \end{equation}
                Using this, we obtain the following table:
                \begin{table}[H]
                    \centering
                    \captionsetup{type=table}
                    \begin{tabular}{|c|c|c|c|}
                        \hline
                        $Q_{ij}$&x&y&z\\
                        \hline
                        x&$q\big(3a^{2}-(a^{2}+b^{2}+c^{2})\big)$&
                        $3qab$&$3qac$\\
                        \hline
                        y&$3qab$&$q\big(3b^{2}-(a^{2}+b^{2}+c^{2})\big)$
                        &$3qbc$\\
                        \hline
                        z&$3qac$&$3qbc$
                        &$q\big(3c^{2}-(a^{2}+b^{2}+c^{2})\big)$\\
                        \hline
                    \end{tabular}
                    \caption{Quadrapole Moments for Problem
                             \ref{problem:EMAG_II_Wangsnes_8_2}.}
                    \label{tab:EMAG_2_Problem_8_2_Wangsness_Quadrupole}
                \end{table}
                Since both the dipole and quadrupole moments are weighted
                by the coordinates of the charges, adding a charge to the
                original does not affect these. However, the total charge will
                now be zero.
            \end{solution}
            \begin{problem}
                \label{problem:EMAG_II_Wangsnes_8_4}
                Compute $\phi$ using the multi-pole expansion for a
                charge $2q$ at $(0,\ell,0)$ and charges $-q$ located at
                $(-a,0,0)$ and $(a,0,0)$.
            \end{problem}
            \begin{solution}
                The multi-pole expansion of $\phi$ is:
                \begin{equation}
                    \phi(\mathbf{r})=\frac{1}{4\pi\epsilon_{0}}\Big(
                    \frac{Q}{r}+\frac{\mathbf{p}\cdot\hat{\mathbf{r}}}{r^{2}}
                    +\frac{1}{2r^{3}}\sum_{j,k}\ell_{j}\ell_{k}Q_{kj}+\cdots\Big)
                \end{equation}
                For this system, we have the following:
                \begin{subequations}
                    \begin{align}
                        Q&=0\\
                        \mathbf{p}&=2q\ell\hat{\mathbf{y}}
                    \end{align}
                \end{subequations}
                Computing the quadrupole moments, we obtain the table below:
                \begin{table}[H]
                    \centering
                    \captionsetup{type=table}
                    \begin{tabular}{|c|c|c|c|}
                        \hline
                        $Q_{ij}$&$x$&$y$&$z$\\
                        \hline
                        $x$&$-2q(\ell^{2}+2a^{2})$&0&0\\
                        \hline
                        $y$&0&$2q(2\ell^{2}+a^{2})$&0\\
                        \hline
                        $z$&0&0&$2q(-\ell^{2}+a^{2})$\\
                        \hline
                    \end{tabular}
                    \caption{Quadrupole Moment for Problem
                             \ref{tab:EMAG_2_Problem_8_2_Wangsness_Quadrupole}.}
                    \label{tab:EMAG_2_Problem_8_4_Wangsness_Quadrupole}
                \end{table}
                Recalling that $r^{2}=x^{2}+y^{2}+z^{2}$, and using
                everything up to the quadrupole moment, we obtain the following
                approximation for $\phi$:
                \begin{subequations}
                    \begin{align}
                        \phi(r)&\approx
                        \frac{1}{4\pi\epsilon_{0}}\Big(
                            0+\frac{2q\ell{y}}{r^{3}}
                             +\frac{1}{2r^{3}}\sum_{i=1}^{3}Q_{ii}\ell_{i}^{2}\Big)\\
                        \begin{split}
                            &=\frac{2q}{4\pi\epsilon_{0}}
                                \Big(\frac{\ell{y}}{r^{3}}+
                                     \frac{1}{2r^{5}}
                                     \big[-2x^{2}(\ell^{2}+2a^{2})
                                          +2y^{2}(2\ell^{2}+a^{2})\\
                                          &\hspace{2.3in}+2z^{2}(-\ell^{2}+a^{2})
                                     \big]
                                \Big)
                        \end{split}\\
                        &=\frac{q}{2\pi\epsilon_{0}r^{3}}
                            \Big(\ell{y}+\frac{1}{2r^{2}}
                                 a\big[\ell^{2}(3y^{2}-r^{2})-a^{2}(3x^{2}-r^{2}\big]
                            \Big)
                    \end{align}
                \end{subequations}
            \end{solution}
            \begin{problem}
                Given a line of change of length $L$ with constant charge
                density $\lambda$ that lies in the first quadrant of the
                $xy$ plane with one end at the origin and making an angle
                $\alpha$ with the $x$ axis, find $Q$, $\mathbf{p}$, and
                all components of $Q_{jk}$. 
            \end{problem}
            \begin{solution}
                The total charge $Q$ is:
                \begin{equation}
                    Q=\int_{C}\lambda\diff{\ell}
                \end{equation}
                Since the path $C$ is a line, and the density $\lambda$
                is a constant, we can integrate this to obtain:
                \begin{equation}
                    Q=\int_{0}^{L}\lambda\diff{r'}=\lambda{L}
                \end{equation}
                \begin{equation}
                    \mathbf{p}=\int_{C}\lambda\mathbf{r}\diff{\ell}
                \end{equation}
                For this distribution of charge, we have:
                \begin{subequations}
                    \begin{align}
                        \mathbf{p}&=
                            \lambda\int_{0}^{L}r'\hat{\mathbf{r}}'\diff{r'}\\
                        &=\lambda\int_{0}^{L}r'
                            \big(\cos(\alpha)\hat{\mathbf{x}}+
                                 \sin(\alpha)\hat{\mathbf{y}}\big)\diff{r'}\\
                        &=\frac{\lambda{L}^{2}}{2}
                            \big(\cos(\alpha)\hat{\mathbf{x}}+
                                 \sin(\alpha)\hat{\mathbf{y}}\big)
                    \end{align}
                \end{subequations}
                The quadrupole components can be obtained by:
                \begin{equation}
                    Q_{jk}=\int_{C}\lambda(\mathbf{r})
                        (3x_{j}x_{k}-r^{2}\delta_{jk})\diff{\ell}
                \end{equation}
                Using this, we obtain the following table:
                \begin{table}[H]
                    \centering
                    \captionsetup{type=table}
                    \begin{tabular}{|c|c|c|c|}
                        \hline
                        $Q_{jk}$&$x$&$y$&$z$\\
                        \hline
                        $x\phantom{\bigg(}$&
                        $\frac{\lambda{L}^3}{3}\big(3\cos^{2}(\alpha)-1\big)$&
                        $\lambda{L}^{3}\cos(\alpha)\sin(\alpha)$&0\\
                        \hline
                        $y\phantom{\bigg(}$&
                        $\lambda{L}^{3}\cos(\alpha)\sin(\alpha)$&
                        $\frac{\lambda{L}^3}{3}\big(3\sin^{2}(\alpha)-1\big)$&0\\
                        \hline
                        $z\phantom{\bigg(}$&
                        0&0&$\minus\frac{\lambda{L}^{3}}{3}$\\
                        \hline
                    \end{tabular}
                    \caption{Caption}
                    \label{tab:my_label}
                \end{table}
                Writing $\mathbf{r}$ using Cartesian unit vectors,
                we may then approximate $\phi$ as:
                \begin{equation}
                    \begin{split}
                        \phi(r)=\frac{\lambda{L}}{4\pi\epsilon_{0}r}
                            \Big[1+&\frac{L}{2r}
                                \big(\cos(\alpha)\cos(\theta)+
                                     \sin(\alpha)\sin(\theta)\big)\\
                                &+\frac{L^{2}}{6r^{4}}
                                \Big(3\big(x\cos(\alpha)+
                                     y\sin(\alpha)\big)^{2}-r^{2}\Big)
                            \Big]
                    \end{split}
                \end{equation}
            \end{solution}
\end{document}