\documentclass[crop=false,class=article,oneside]{standalone}
%----------------------------Preamble-------------------------------%
%---------------------------Packages----------------------------%
\usepackage{geometry}
\geometry{b5paper, margin=1.0in}
\usepackage[T1]{fontenc}
\usepackage{graphicx, float}            % Graphics/Images.
\usepackage{natbib}                     % For bibliographies.
\bibliographystyle{agsm}                % Bibliography style.
\usepackage[french, english]{babel}     % Language typesetting.
\usepackage[dvipsnames]{xcolor}         % Color names.
\usepackage{listings}                   % Verbatim-Like Tools.
\usepackage{mathtools, esint, mathrsfs} % amsmath and integrals.
\usepackage{amsthm, amsfonts, amssymb}  % Fonts and theorems.
\usepackage{tcolorbox}                  % Frames around theorems.
\usepackage{upgreek}                    % Non-Italic Greek.
\usepackage{fmtcount, etoolbox}         % For the \book{} command.
\usepackage[newparttoc]{titlesec}       % Formatting chapter, etc.
\usepackage{titletoc}                   % Allows \book in toc.
\usepackage[nottoc]{tocbibind}          % Bibliography in toc.
\usepackage[titles]{tocloft}            % ToC formatting.
\usepackage{pgfplots, tikz}             % Drawing/graphing tools.
\usepackage{imakeidx}                   % Used for index.
\usetikzlibrary{
    calc,                   % Calculating right angles and more.
    angles,                 % Drawing angles within triangles.
    arrows.meta,            % Latex and Stealth arrows.
    quotes,                 % Adding labels to angles.
    positioning,            % Relative positioning of nodes.
    decorations.markings,   % Adding arrows in the middle of a line.
    patterns,
    arrows
}                                       % Libraries for tikz.
\pgfplotsset{compat=1.9}                % Version of pgfplots.
\usepackage[font=scriptsize,
            labelformat=simple,
            labelsep=colon]{subcaption} % Subfigure captions.
\usepackage[font={scriptsize},
            hypcap=true,
            labelsep=colon]{caption}    % Figure captions.
\usepackage[pdftex,
            pdfauthor={Ryan Maguire},
            pdftitle={Mathematics and Physics},
            pdfsubject={Mathematics, Physics, Science},
            pdfkeywords={Mathematics, Physics, Computer Science, Biology},
            pdfproducer={LaTeX},
            pdfcreator={pdflatex}]{hyperref}
\hypersetup{
    colorlinks=true,
    linkcolor=blue,
    filecolor=magenta,
    urlcolor=Cerulean,
    citecolor=SkyBlue
}                           % Colors for hyperref.
\usepackage[toc,acronym,nogroupskip,nopostdot]{glossaries}
\usepackage{glossary-mcols}
%------------------------Theorem Styles-------------------------%
\theoremstyle{plain}
\newtheorem{theorem}{Theorem}[section]

% Define theorem style for default spacing and normal font.
\newtheoremstyle{normal}
    {\topsep}               % Amount of space above the theorem.
    {\topsep}               % Amount of space below the theorem.
    {}                      % Font used for body of theorem.
    {}                      % Measure of space to indent.
    {\bfseries}             % Font of the header of the theorem.
    {}                      % Punctuation between head and body.
    {.5em}                  % Space after theorem head.
    {}

% Italic header environment.
\newtheoremstyle{thmit}{\topsep}{\topsep}{}{}{\itshape}{}{0.5em}{}

% Define environments with italic headers.
\theoremstyle{thmit}
\newtheorem*{solution}{Solution}

% Define default environments.
\theoremstyle{normal}
\newtheorem{example}{Example}[section]
\newtheorem{definition}{Definition}[section]
\newtheorem{problem}{Problem}[section]

% Define framed environment.
\tcbuselibrary{most}
\newtcbtheorem[use counter*=theorem]{ftheorem}{Theorem}{%
    before=\par\vspace{2ex},
    boxsep=0.5\topsep,
    after=\par\vspace{2ex},
    colback=green!5,
    colframe=green!35!black,
    fonttitle=\bfseries\upshape%
}{thm}

\newtcbtheorem[auto counter, number within=section]{faxiom}{Axiom}{%
    before=\par\vspace{2ex},
    boxsep=0.5\topsep,
    after=\par\vspace{2ex},
    colback=Apricot!5,
    colframe=Apricot!35!black,
    fonttitle=\bfseries\upshape%
}{ax}

\newtcbtheorem[use counter*=definition]{fdefinition}{Definition}{%
    before=\par\vspace{2ex},
    boxsep=0.5\topsep,
    after=\par\vspace{2ex},
    colback=blue!5!white,
    colframe=blue!75!black,
    fonttitle=\bfseries\upshape%
}{def}

\newtcbtheorem[use counter*=example]{fexample}{Example}{%
    before=\par\vspace{2ex},
    boxsep=0.5\topsep,
    after=\par\vspace{2ex},
    colback=red!5!white,
    colframe=red!75!black,
    fonttitle=\bfseries\upshape%
}{ex}

\newtcbtheorem[auto counter, number within=section]{fnotation}{Notation}{%
    before=\par\vspace{2ex},
    boxsep=0.5\topsep,
    after=\par\vspace{2ex},
    colback=SeaGreen!5!white,
    colframe=SeaGreen!75!black,
    fonttitle=\bfseries\upshape%
}{not}

\newtcbtheorem[use counter*=remark]{fremark}{Remark}{%
    fonttitle=\bfseries\upshape,
    colback=Goldenrod!5!white,
    colframe=Goldenrod!75!black}{ex}

\newenvironment{bproof}{\textit{Proof.}}{\hfill$\square$}
\tcolorboxenvironment{bproof}{%
    blanker,
    breakable,
    left=3mm,
    before skip=5pt,
    after skip=10pt,
    borderline west={0.6mm}{0pt}{green!80!black}
}

\AtEndEnvironment{lexample}{$\hfill\textcolor{red}{\blacksquare}$}
\newtcbtheorem[use counter*=example]{lexample}{Example}{%
    empty,
    title={Example~\theexample},
    boxed title style={%
        empty,
        size=minimal,
        toprule=2pt,
        top=0.5\topsep,
    },
    coltitle=red,
    fonttitle=\bfseries,
    parbox=false,
    boxsep=0pt,
    before=\par\vspace{2ex},
    left=0pt,
    right=0pt,
    top=3ex,
    bottom=1ex,
    before=\par\vspace{2ex},
    after=\par\vspace{2ex},
    breakable,
    pad at break*=0mm,
    vfill before first,
    overlay unbroken={%
        \draw[red, line width=2pt]
            ([yshift=-1.2ex]title.south-|frame.west) to
            ([yshift=-1.2ex]title.south-|frame.east);
        },
    overlay first={%
        \draw[red, line width=2pt]
            ([yshift=-1.2ex]title.south-|frame.west) to
            ([yshift=-1.2ex]title.south-|frame.east);
    },
}{ex}

\AtEndEnvironment{ldefinition}{$\hfill\textcolor{Blue}{\blacksquare}$}
\newtcbtheorem[use counter*=definition]{ldefinition}{Definition}{%
    empty,
    title={Definition~\thedefinition:~{#1}},
    boxed title style={%
        empty,
        size=minimal,
        toprule=2pt,
        top=0.5\topsep,
    },
    coltitle=Blue,
    fonttitle=\bfseries,
    parbox=false,
    boxsep=0pt,
    before=\par\vspace{2ex},
    left=0pt,
    right=0pt,
    top=3ex,
    bottom=0pt,
    before=\par\vspace{2ex},
    after=\par\vspace{1ex},
    breakable,
    pad at break*=0mm,
    vfill before first,
    overlay unbroken={%
        \draw[Blue, line width=2pt]
            ([yshift=-1.2ex]title.south-|frame.west) to
            ([yshift=-1.2ex]title.south-|frame.east);
        },
    overlay first={%
        \draw[Blue, line width=2pt]
            ([yshift=-1.2ex]title.south-|frame.west) to
            ([yshift=-1.2ex]title.south-|frame.east);
    },
}{def}

\AtEndEnvironment{ltheorem}{$\hfill\textcolor{Green}{\blacksquare}$}
\newtcbtheorem[use counter*=theorem]{ltheorem}{Theorem}{%
    empty,
    title={Theorem~\thetheorem:~{#1}},
    boxed title style={%
        empty,
        size=minimal,
        toprule=2pt,
        top=0.5\topsep,
    },
    coltitle=Green,
    fonttitle=\bfseries,
    parbox=false,
    boxsep=0pt,
    before=\par\vspace{2ex},
    left=0pt,
    right=0pt,
    top=3ex,
    bottom=-1.5ex,
    breakable,
    pad at break*=0mm,
    vfill before first,
    overlay unbroken={%
        \draw[Green, line width=2pt]
            ([yshift=-1.2ex]title.south-|frame.west) to
            ([yshift=-1.2ex]title.south-|frame.east);},
    overlay first={%
        \draw[Green, line width=2pt]
            ([yshift=-1.2ex]title.south-|frame.west) to
            ([yshift=-1.2ex]title.south-|frame.east);
    }
}{thm}

%--------------------Declared Math Operators--------------------%
\DeclareMathOperator{\adjoint}{adj}         % Adjoint.
\DeclareMathOperator{\Card}{Card}           % Cardinality.
\DeclareMathOperator{\curl}{curl}           % Curl.
\DeclareMathOperator{\diam}{diam}           % Diameter.
\DeclareMathOperator{\dist}{dist}           % Distance.
\DeclareMathOperator{\Div}{div}             % Divergence.
\DeclareMathOperator{\Erf}{Erf}             % Error Function.
\DeclareMathOperator{\Erfc}{Erfc}           % Complementary Error Function.
\DeclareMathOperator{\Ext}{Ext}             % Exterior.
\DeclareMathOperator{\GCD}{GCD}             % Greatest common denominator.
\DeclareMathOperator{\grad}{grad}           % Gradient
\DeclareMathOperator{\Ima}{Im}              % Image.
\DeclareMathOperator{\Int}{Int}             % Interior.
\DeclareMathOperator{\LC}{LC}               % Leading coefficient.
\DeclareMathOperator{\LCM}{LCM}             % Least common multiple.
\DeclareMathOperator{\LM}{LM}               % Leading monomial.
\DeclareMathOperator{\LT}{LT}               % Leading term.
\DeclareMathOperator{\Mod}{mod}             % Modulus.
\DeclareMathOperator{\Mon}{Mon}             % Monomial.
\DeclareMathOperator{\multideg}{mutlideg}   % Multi-Degree (Graphs).
\DeclareMathOperator{\nul}{nul}             % Null space of operator.
\DeclareMathOperator{\Ord}{Ord}             % Ordinal of ordered set.
\DeclareMathOperator{\Prin}{Prin}           % Principal value.
\DeclareMathOperator{\proj}{proj}           % Projection.
\DeclareMathOperator{\Refl}{Refl}           % Reflection operator.
\DeclareMathOperator{\rk}{rk}               % Rank of operator.
\DeclareMathOperator{\sgn}{sgn}             % Sign of a number.
\DeclareMathOperator{\sinc}{sinc}           % Sinc function.
\DeclareMathOperator{\Span}{Span}           % Span of a set.
\DeclareMathOperator{\Spec}{Spec}           % Spectrum.
\DeclareMathOperator{\supp}{supp}           % Support
\DeclareMathOperator{\Tr}{Tr}               % Trace of matrix.
%--------------------Declared Math Symbols--------------------%
\DeclareMathSymbol{\minus}{\mathbin}{AMSa}{"39} % Unary minus sign.
%------------------------New Commands---------------------------%
\DeclarePairedDelimiter\norm{\lVert}{\rVert}
\DeclarePairedDelimiter\ceil{\lceil}{\rceil}
\DeclarePairedDelimiter\floor{\lfloor}{\rfloor}
\newcommand*\diff{\mathop{}\!\mathrm{d}}
\newcommand*\Diff[1]{\mathop{}\!\mathrm{d^#1}}
\renewcommand*{\glstextformat}[1]{\textcolor{RoyalBlue}{#1}}
\renewcommand{\glsnamefont}[1]{\textbf{#1}}
\renewcommand\labelitemii{$\circ$}
\renewcommand\thesubfigure{%
    \arabic{chapter}.\arabic{figure}.\arabic{subfigure}}
\addto\captionsenglish{\renewcommand{\figurename}{Fig.}}
\numberwithin{equation}{section}

\renewcommand{\vector}[1]{\boldsymbol{\mathrm{#1}}}

\newcommand{\uvector}[1]{\boldsymbol{\hat{\mathrm{#1}}}}
\newcommand{\topspace}[2][]{(#2,\tau_{#1})}
\newcommand{\measurespace}[2][]{(#2,\varSigma_{#1},\mu_{#1})}
\newcommand{\measurablespace}[2][]{(#2,\varSigma_{#1})}
\newcommand{\manifold}[2][]{(#2,\tau_{#1},\mathcal{A}_{#1})}
\newcommand{\tanspace}[2]{T_{#1}{#2}}
\newcommand{\cotanspace}[2]{T_{#1}^{*}{#2}}
\newcommand{\Ckspace}[3][\mathbb{R}]{C^{#2}(#3,#1)}
\newcommand{\funcspace}[2][\mathbb{R}]{\mathcal{F}(#2,#1)}
\newcommand{\smoothvecf}[1]{\mathfrak{X}(#1)}
\newcommand{\smoothonef}[1]{\mathfrak{X}^{*}(#1)}
\newcommand{\bracket}[2]{[#1,#2]}

%------------------------Book Command---------------------------%
\makeatletter
\renewcommand\@pnumwidth{1cm}
\newcounter{book}
\renewcommand\thebook{\@Roman\c@book}
\newcommand\book{%
    \if@openright
        \cleardoublepage
    \else
        \clearpage
    \fi
    \thispagestyle{plain}%
    \if@twocolumn
        \onecolumn
        \@tempswatrue
    \else
        \@tempswafalse
    \fi
    \null\vfil
    \secdef\@book\@sbook
}
\def\@book[#1]#2{%
    \refstepcounter{book}
    \addcontentsline{toc}{book}{\bookname\ \thebook:\hspace{1em}#1}
    \markboth{}{}
    {\centering
     \interlinepenalty\@M
     \normalfont
     \huge\bfseries\bookname\nobreakspace\thebook
     \par
     \vskip 20\p@
     \Huge\bfseries#2\par}%
    \@endbook}
\def\@sbook#1{%
    {\centering
     \interlinepenalty \@M
     \normalfont
     \Huge\bfseries#1\par}%
    \@endbook}
\def\@endbook{
    \vfil\newpage
        \if@twoside
            \if@openright
                \null
                \thispagestyle{empty}%
                \newpage
            \fi
        \fi
        \if@tempswa
            \twocolumn
        \fi
}
\newcommand*\l@book[2]{%
    \ifnum\c@tocdepth >-3\relax
        \addpenalty{-\@highpenalty}%
        \addvspace{2.25em\@plus\p@}%
        \setlength\@tempdima{3em}%
        \begingroup
            \parindent\z@\rightskip\@pnumwidth
            \parfillskip -\@pnumwidth
            {
                \leavevmode
                \Large\bfseries#1\hfill\hb@xt@\@pnumwidth{\hss#2}
            }
            \par
            \nobreak
            \global\@nobreaktrue
            \everypar{\global\@nobreakfalse\everypar{}}%
        \endgroup
    \fi}
\newcommand\bookname{Book}
\renewcommand{\thebook}{\texorpdfstring{\Numberstring{book}}{book}}
\providecommand*{\toclevel@book}{-2}
\makeatother
\titleformat{\part}[display]
    {\Large\bfseries}
    {\partname\nobreakspace\thepart}
    {0mm}
    {\Huge\bfseries}
\titlecontents{part}[0pt]
    {\large\bfseries}
    {\partname\ \thecontentslabel: \quad}
    {}
    {\hfill\contentspage}
\titlecontents{chapter}[0pt]
    {\bfseries}
    {\chaptername\ \thecontentslabel:\quad}
    {}
    {\hfill\contentspage}
\newglossarystyle{longpara}{%
    \setglossarystyle{long}%
    \renewenvironment{theglossary}{%
        \begin{longtable}[l]{{p{0.25\hsize}p{0.65\hsize}}}
    }{\end{longtable}}%
    \renewcommand{\glossentry}[2]{%
        \glstarget{##1}{\glossentryname{##1}}%
        &\glossentrydesc{##1}{~##2.}
        \tabularnewline%
        \tabularnewline
    }%
}
\newglossary[not-glg]{notation}{not-gls}{not-glo}{Notation}
\newcommand*{\newnotation}[4][]{%
    \newglossaryentry{#2}{type=notation, name={\textbf{#3}, },
                          text={#4}, description={#4},#1}%
}
%--------------------------LENGTHS------------------------------%
% Spacings for the Table of Contents.
\addtolength{\cftsecnumwidth}{1ex}
\addtolength{\cftsubsecindent}{1ex}
\addtolength{\cftsubsecnumwidth}{1ex}
\addtolength{\cftfignumwidth}{1ex}
\addtolength{\cfttabnumwidth}{1ex}

% Indent and paragraph spacing.
\setlength{\parindent}{0em}
\setlength{\parskip}{0em}
%----------------------------GLOSSARY-------------------------------%
\makeglossaries
\loadglsentries{../../../../glossary}
\loadglsentries{../../../../acronym}
%--------------------------Main Document----------------------------%
\begin{document}
    \ifx\ifcoursesprogrammingtutorials\undefined
        \section*{Python}
        \setcounter{section}{4}
    \else
        \section{Python}
    \fi
    \subsection{Tutorials from LearnPython.org}
        \subsubsection{Introduction}
            Python uses indents for blocks, rather than braces.
            Tabs and spaces are supported, by standard indentation
            requires Python code to use four spaces.
            To print a string in Python 3, do:
            \newline
            \begin{minipage}[t]{.48\textwidth}
                \centering
                \begin{lstlisting}[%
                    language=bash,
                    basicstyle=\small\ttfamily,
                    frame=single,
                    caption=input,
                    gobble=20
                ]
                    print("This line will be printed.")
                    x = 1
                    if x == 1:
                        print("x is 1.")
                \end{lstlisting}
            \end{minipage}
            \hfill
            \begin{minipage}[t]{.48\textwidth}
                \centering
                \begin{lstlisting}[%
                    language=bash,
                    basicstyle=\small\ttfamily,
                    frame=single,
                    caption=output,
                    gobble=20
                ]
                    This line will be printed.
                    x is 1.
                \end{lstlisting}
            \end{minipage}
            And now, for the quintessential coding problem:
            Print "Hello, World!"\newline
            \begin{minipage}[t]{.48\textwidth}
                \centering
                \begin{lstlisting}[%
                    language=bash,
                    basicstyle=\small\ttfamily,
                    frame=single,
                    caption=input,
                    gobble=20
                ]
                    print("Hello, World!")
                \end{lstlisting}
            \end{minipage}
            \hfill
            \begin{minipage}[t]{.48\textwidth}
                \centering
                \begin{lstlisting}[%
                    language=bash,
                    basicstyle=\small\ttfamily,
                    frame=single,
                    caption=output,
                    gobble=20
                ]
                    Hello, World!
            \end{lstlisting}
            \end{minipage}
        \subsubsection{Variables and Types}
            Python is an object oriented language.
            Variables do not need to be declared before using them,
            nor do their types need to be declared.
            Every variable in Python is an object.
            Python supports three types of numbers: Integers,
            floating point numbers, and complex numbers.
            To define an integer, do the following:\newline
            \begin{minipage}[t]{.48\textwidth}
                \centering
                \begin{lstlisting}[%
                    language=bash,
                    basicstyle=\small\ttfamily,
                    frame=single,
                    caption=input,
                    gobble=20
                ]
                    myint = 7
                    print(myint)
                \end{lstlisting}
            \end{minipage}
            \hfill
            \begin{minipage}[t]{.48\textwidth}
                \centering
                \begin{lstlisting}[%
                    language=bash,
                    basicstyle=\small\ttfamily,
                    frame=single,
                    caption=output,
                    gobble=20
                ]
                    7
                \end{lstlisting}
            \end{minipage}\newline
            To define a floating point number, use one of the
            following notations:
            \newline
            \begin{minipage}[t]{.48\textwidth}
                \centering
                \begin{lstlisting}[%
                    language=bash,
                    basicstyle=\small\ttfamily,
                    frame=single,
                    caption=input,
                    gobble=20
                ]
                    myfloat = 7.0
                    print(myfloat)
                    myfloat = float(7)
                    print(myfloat)
                \end{lstlisting}
            \end{minipage}
            \hfill
            \begin{minipage}[t]{.48\textwidth}
                \centering
                \begin{lstlisting}[%
                    language=bash,
                    basicstyle=\small\ttfamily,
                    frame=single,
                    caption=output,
                    gobble=20
                ]
                    7.0
                    7.0
                \end{lstlisting}
            \end{minipage}
            Strings are defined with either " or ' quotations.
            Using " allows apostrophe's to be included
            in a string, whereas using ' means an apostrophe
            would terminate the string.\newline
            \begin{minipage}[t]{.48\textwidth}
                \centering
                \begin{lstlisting}[%
                    language=bash,
                    basicstyle=\small\ttfamily,
                    frame=single,
                    caption=input,
                    gobble=20
                ]
                    mystring = 'hello'
                    print(mystring)
                    mystring = "hello"
                    print(mystring)
                    mystring = "Don't worry"
                    print(mystring)
                \end{lstlisting}
            \end{minipage}\hfill
            \begin{minipage}[t]{.48\textwidth}
                \centering
                \begin{lstlisting}[%
                    language=bash,
                    basicstyle=\small\ttfamily,
                    frame=single,
                    caption=output,
                    gobble=20
                ]
                    hello
                    hello
                    Don't worry
                \end{lstlisting}
            \end{minipage}
            \begin{minipage}[t]{.48\textwidth}
                \centering
                \begin{lstlisting}[%
                    language=bash,
                    basicstyle=\small\ttfamily,
                    frame=single,
                    caption=input,
                    gobble=20
                ]
                    mystring = 'Don't worry'
                    print(mystring)
                \end{lstlisting}
            \end{minipage}\hfill
            \begin{minipage}[t]{.48\textwidth}
                \centering
                \begin{lstlisting}[%
                    language=bash,
                    basicstyle=\small\ttfamily,
                    frame=single,
                    caption=output,
                    gobble=20
                ]
                    SyntaxError: invalid syntax
                \end{lstlisting}
            \end{minipage}
            Strings can be defined to include things like
            carriage returns, backslashes and Unicode characters.
            Simple operations like addition can be performed on
            numbers and strings. Also, multiple variables can be
            assigned simultaneously on the same line.\newline
            \begin{minipage}[t]{.48\textwidth}
                \centering
                \begin{lstlisting}[%
                    language=bash,
                    basicstyle=\small\ttfamily,
                    frame=single,
                    caption=input,
                    gobble=20
                ]
                    one = 1
                    two = 2
                    three = one + two
                    print(three)
                    hello = "hello"
                    world = "world"
                    helloworld = hello + " " + world
                    print(helloworld)
                    a, b = 3, 4
                    print(a,b)
                \end{lstlisting}
            \end{minipage}
            \hfill
            \begin{minipage}[t]{.48\textwidth}
                \centering
                \begin{lstlisting}[%
                    language=bash,
                    basicstyle=\small\ttfamily,
                    frame=single,
                    caption=output,
                    gobble=20
                ]
                    3
                    hello world
                    3, 4
                \end{lstlisting}
            \end{minipage}
            Mixing strings and numbers results in
            an error.\newline
            \begin{minipage}[t]{.48\textwidth}
                \centering
                \begin{lstlisting}[%
                    language=bash,
                    basicstyle=\small\ttfamily,
                    frame=single,
                    caption=input,
                    gobble=20
                ]
                    # This will not work!
                    one = 1
                    two = 2
                    hello = "hello"
                    print(one + two + hello)
                \end{lstlisting}
            \end{minipage}\hfill
            \begin{minipage}[t]{.48\textwidth}
                \centering
                \begin{lstlisting}[%
                    language=bash,
                    basicstyle=\small\ttfamily,
                    frame=single,
                    caption=output,
                    gobble=20
                ]
                    Traceback (most recent call last):
                      File "<stdin>", line 6, in <module>
                        print(one + two + hello)
                    TypeError: unsupported operand type(s)
                        for +: 'int' and 'str'
                \end{lstlisting}
            \end{minipage}
        \subsubsection{Lists}
            Lists can contain any type of variable and as many
            variables as desired. They can be iterated over as
            well.\newline
            \begin{minipage}[t]{.48\textwidth}
                \centering
                \begin{lstlisting}[%
                    language=bash,
                    basicstyle=\small\ttfamily,
                    frame=single,
                    caption=input,
                    gobble=20
                ]
                    mylist = []
                    mylist.append(1)
                    mylist.append(2)
                    mylist.append(3)
                    print(mylist[0]) # prints 1
                    print(mylist[1]) # prints 2
                    print(mylist[2]) # prints 3
                    # prints out 1,2,3
                    for x in mylist:
                        print(x)
                \end{lstlisting}
            \end{minipage}\hfill
            \begin{minipage}[t]{.48\textwidth}
                \centering
                \begin{lstlisting}[%
                    language=bash,
                    basicstyle=\small\ttfamily,
                    frame=single,
                    caption=output,
                    gobble=20
                ]
                    1
                    2
                    3
                    1
                    2
                    3
                \end{lstlisting}
            \end{minipage}
            Accessing an index which does not exist in a
            list produces an error.\newline
            \begin{minipage}[t]{.48\textwidth}
                \centering
                \begin{lstlisting}[%
                    language=bash,
                    basicstyle=\small\ttfamily,
                    frame=single,
                    caption=input,
                    gobble=20
                ]
                    mylist = [1,2,3]
                    print(mylist[10])
                \end{lstlisting}
            \end{minipage}\hfill
            \begin{minipage}[t]{.48\textwidth}
                \centering
                \begin{lstlisting}[%
                    language=bash,
                    basicstyle=\small\ttfamily,
                    frame=single,
                    caption=output,
                    gobble=20
                ]
                    Traceback (most recent call last):
                      File "<stdin>", line 2, in <module>
                        print(mylist[10])
                    IndexError: list index out of range
                \end{lstlisting}
            \end{minipage}
            Both \%s and \% are borrowed from C. It acts as a
            placeholder in a string, so that other values can be
            added to replace this later. Verbatim from the Python
            Documentation:\newline
            \textcolor{blue}{%
                Python supports formatting values into strings.
                Although this can include very complicated
                expressions, the most basic usage is to insert
                values into a string with the \%s placeholder.
            }.\newline
            For example:\newline
            \begin{minipage}[t]{.48\textwidth}
                \centering
                \begin{lstlisting}[%
                    language=bash,
                    basicstyle=\small\ttfamily,
                    frame=single,
                    caption=input,
                    gobble=20
                ]
                    print("Hello %s, my name is\
                          %s" % ('john', 'mike'))
                \end{lstlisting}
            \end{minipage}\hfill
            \begin{minipage}[t]{.48\textwidth}
                \centering
                \begin{lstlisting}[%
                    language=bash,
                    basicstyle=\small\ttfamily,
                    frame=single,
                    caption=output,
                    gobble=20
                ]
                Hello john, my name is mike
                \end{lstlisting}
            \end{minipage}
            The \textbackslash\ character is used to continue a line
            of code onto the next line. That is, the computer sees
            the two lines in the previous example as one long line.
            Here is a brief tutorial on how to append variables to
            different lists.\newline
            \begin{minipage}[t]{.48\textwidth}
                \centering
                \begin{lstlisting}[%
                    language=bash,
                    basicstyle=\small\ttfamily,
                    frame=single,
                    caption=input,
                    gobble=20
                ]
                    numbers = []
                    strings = []
                    names = ["John", "Eric", "Jessica"]
                    numbers.append(1)
                    numbers.append(2)
                    numbers.append(3)
                    strings.append("hello")
                    strings.append("world")
                    second_name = names[1]
                    print(numbers)
                    print(strings)
                    print("Second name in name list is %s"
                          % second_name)
                \end{lstlisting}
            \end{minipage}\hfill
            \begin{minipage}[t]{.48\textwidth}
                \centering
                \begin{lstlisting}[%
                    language=python,
                    frame=single,
                    basicstyle=\footnotesize,
                    frame=single,
                    caption=output,
                    gobble=20
                ]
                    [1, 2, 3]
                    ['hello', 'world']
                    Second name in name list is Eric
                \end{lstlisting}
            \end{minipage}
            \newpage
        \subsubsection{Basic Operators}
            \textbf{Arithmetic Operators}
                The four basic operations can be used with numbers
                inside of Python. Addition $(+)$, subtraction $(-)$,
                multiplication $(*)$ and division $(/)$ are all
                supported. Python also follows the standard
                \gls{pemdas} order of operations. Another useful
                operator is the modulo operator $(\%)$. This returns
                the integer remainder of division. Exponents $(**)$
                are also allowed.\newline
                \begin{minipage}[t]{.48\textwidth}
                    \centering
                    \begin{lstlisting}[%
                        language=python,
                        frame=single,
                        basicstyle=\footnotesize,
                        frame=single,
                        caption=input,
                        gobble=20
                    ]
                        number = 1 + 2 * 3 / 4.0
                        print(number)
                        remainder = 11 % 3
                        print(remainder)
                        squared = 7 ** 2
                        print(squared)
                        cubed = 2 ** 3
                        print(cubed)
                    \end{lstlisting}
                \end{minipage}\hfill
                \begin{minipage}[t]{.48\textwidth}
                    \centering
                    \begin{lstlisting}[%
                    language=python,
                    frame=single,
                    basicstyle=\footnotesize,
                    frame=single,
                    caption=output,
                    ]
                        2.5
                        2
                        49
                        8
                    \end{lstlisting}
                \end{minipage}
            \textbf{String Operators}
                String can be concatenated by using $+$ with two
                strings. Multiplying a string by an integer $n$
                produces a string that is $n$ multiples of the
                original.\newline
                \begin{minipage}[t]{.48\textwidth}
                    \centering
                    \begin{lstlisting}[
                        language=python,
                        frame=single,
                        basicstyle=\footnotesize,
                        frame=single,
                        caption=input,
                        gobble=24
                    ]
                        greetings = "hello" + " " + "world"
                        print(greetings)
                        lotsofhellos = "hello" * 7
                        print(lotsofhellos)
                    \end{lstlisting}
                \end{minipage}\hfill
                \begin{minipage}[t]{.48\textwidth}
                    \centering
                    \begin{lstlisting}[%
                        language=python,
                        frame=single,
                        basicstyle=\footnotesize,
                        frame=single,
                        caption=output,
                        gobble=24
                    ]
                        hello world
                        hellohellohellohellohellohellohello
                    \end{lstlisting}
                \end{minipage}
            \textbf{Using Operators with Lists}
                List can be concatenated with the $+$ operator.
                Similar to how $\%s$ is used as a placeholder for
                strings, $\%d$ is used as a placeholder for integer
                numbers.\newline
                \begin{minipage}[t]{.48\textwidth}
                    \centering
                    \begin{lstlisting}[%
                        language=python,
                        frame=single,
                        basicstyle=\footnotesize,
                        frame=single,
                        caption=input,
                        gobble=24
                    ]
                        even_nums = [2,4,6,8]
                        odd_nums = [1,3,5,7]
                        all_nums = odd_nums + even_nums
                        print(all_nums)
                        print([1,2,3] * 3)
                    \end{lstlisting}
                \end{minipage}\hfill
                \begin{minipage}[t]{.48\textwidth}
                    \centering
                    \begin{lstlisting}[%
                        language=bash,
                        basicstyle=\small\ttfamily,
                        frame=single,
                        caption=output,
                        gobble=24
                    ]
                        [1, 3, 5, 7, 2, 4, 6, 8]
                        [1, 2, 3, 1, 2, 3, 1, 2, 3]
                    \end{lstlisting}
                \end{minipage}
                \begin{minipage}[t]{.48\textwidth}
                    \centering
                    \begin{lstlisting}[%
                        language=python,
                        frame=single,
                        basicstyle=\footnotesize,
                        frame=single,
                        caption=input,
                        gobble=24
                    ]
                        name, number = 'Bob Guy', 42
                        print('%s %d' % (name, number))
                        x, y     = object(),object()
                        x_list   = [x] * 10
                        y_list   = [y] * 10
                        big_list = x_list + y_list
                        nx, ny   = len(x_list), len(y_list)
                        nz       = len(big_list)
                        print("x_list: %d objects" % nx)
                        print("y_list: %d objects" % ny)
                        print("big_list: %d objects" % z)
                        if x_list.count(x) == 10:
                            print("Almost there...")
                        if y_list.count(y) == 10:
                            print("Almost there...")
                         if big_list.count(x) == 10:
                             print("Nearly done...")
                         if big_list.count(y) == 10:
                             print("Great!")
                    \end{lstlisting}
                \end{minipage}\hfill
                \begin{minipage}[t]{0.48\textwidth}
                    \centering
                    \begin{lstlisting}[%
                        language=python,
                        frame=single,
                        basicstyle=\footnotesize,
                        frame=single,
                        caption=output,
                        gobble=24
                    ]
                        Bob Guy 42
                        x_list contains 10 objects
                        y_list contains 10 objects
                        big_list contains 20 objects
                        Almost there...
                        Great!
                    \end{lstlisting}
                \end{minipage}
                \newpage
        \subsubsection{String Formatting}
            Python uses C-style string formatting to create new
            formatted strings. The $\%$ operator (Which is also the
            modulo operator) is used to fomrat a set of variables
            enclosed in an n-tupyle, which is a fixed size list of
            $n$ items, together with a format string, which contains
            normal text and argument specifiers. Argument specifiers
            are the characters like $\%s$ and $\%d$ that we've
            already seen. Objectes that are not strings can be
            formatted as strings using the $\%s$ operator as well.
            Here are the basic argument specifiers:
            \begin{itemize}
                \begin{multicols}{2}
                    \item $\%s$ - String
                    \item $\%d$ - Integers
                    \item $\%f$ - Floating point numbers
                    \item $\%.nf$ - Float with n decimals
                    \item $\%x$ - Int in hex representation
                        (lowercase)
                    \item $\%X$ - Int in hex representation
                        (uppercase)
                \end{multicols}
            \end{itemize}
            \begin{minipage}[t]{.48\textwidth}
                \centering
                \begin{lstlisting}[%
                    language=python,
                    frame=single,
                    basicstyle=\footnotesize,
                    caption=input,
                    gobble=20
                ]
                    # This prints out "Hello, John!"
                    name = "John"
                    print("Hello, %s!" % name)
                    # This prints out "John is 23 years old."
                    name = "John"
                    age = 23
                    print("%s is %d years old." % (name, age))
                    # This prints out: A list: [1, 2, 3]
                    mylist = [1,2,3]
                    print("A list: %s" % mylist)
                    # Example exercise
                    data = ("John", "Doe", 53.44)
                    format_string = "Hello %s %s.\ 
                        Your balance is $%s."
                    print(format_string % data)

                \end{lstlisting}
            \end{minipage}\hfill
            \begin{minipage}[t]{.48\textwidth}
                \centering
                \begin{lstlisting}[
                    language=python,
                    frame=single,
                    basicstyle=\footnotesize,
                    frame=single,
                    caption=output,
                    gobble=20
                ]
                    Hello, John!
                    John is 23 years old.
                    A list: [1, 2, 3]
                    Hello John Doe. Your balance is $53.44.

                \end{lstlisting}
            \end{minipage}
        \subsubsection{Conditions}
            Python uses Boolean variables to evaluate conditions.
            True or False are returned when an expression is
            compared or evaluated. Variable assignment is done
            using the equals sign $(=)$, but comparisons between
            two variables are done using double equal signs
            $(==)$. The 'not equals' operators is an exclamation
            point an an equals sign $(!=)$.
            \textbf{Boolean Operators}
                The ``and" and ``or" Boolean operators allowing
                building Boolean expressions. $A$ and $B$ only
                returns True if both $A$ and $B$ are true. $A$ or
                $B$ returns True if either of $A$ or $B$ are
                true. The ``in" operator allows one to check if a
                specified object exists within an iterable object
                container, such as a list. This is very similar
                to the symbol $\in$ that is used in
                mathematics.\newline
                \begin{minipage}[t]{0.48\textwidth}
                    \begin{lstlisting}[%
                        language=python,
                        frame=single,
                        basicstyle=\footnotesize,
                        frame=single,
                        caption=input,
                        gobble=24
                    ]
                        x = 2
                        print(x == 2) # prints out True
                        print(x == 3) # prints out False
                        print(x < 3)  # prints out True
                        name = "John"
                        age = 23
                        if name == "John" and age == 23:
                            print("You're John and you are 23")
                        if name == "John" or name == "Joe":
                            print("You're either John or Joe")
                        if name in ["John", "Bob"]:
                            print("You're either John or Bob")
                \end{lstlisting}
                \end{minipage}\hfill
                \begin{minipage}[t]{0.48\textwidth}
                    \begin{lstlisting}[%
                        language=python,
                        frame=single,
                        basicstyle=\footnotesize,
                        frame=single,
                        caption=output,
                        gobble=24
                    ]
                        True
                        False
                        True
                        You're John and you are 23
                        You're either John or Joe
                        You're either John or Bob
                    \end{lstlisting}
                \end{minipage}
                Python uses indendation to define code blocks.
                Standard indent is 4 spaces, although tabs and other
                spacings work as long as its consistent. Code blocks
                do not need any termination either.
                \newpage
                \begin{minipage}[t]{.48\textwidth}
                    \centering
                    \begin{lstlisting}[%
                        language=python,
                        frame=single,
                        basicstyle=\footnotesize,
                        frame=single,
                        caption=input,
                        gobble=24
                    ]
                        x = 2
                        if x == 2:
                            print("x equals two!")
                        else:
                           print("x does not equal two.")
                    \end{lstlisting}
                \end{minipage}\hfill
                \begin{minipage}[t]{.48\textwidth}
                    \centering
                    \begin{lstlisting}[%
                        language=python,
                        frame=single,
                        basicstyle=\footnotesize,
                        frame=single,
                        caption=output,
                        gobble=24
                    ]
                        x equals two!
                    \end{lstlisting}
                \end{minipage}\newline
                A statement is considered true if one of the
                following holds:
                \begin{enumerate}
                    \item The True Boolean variable is given or
                        calculated using an expression.
                    \item An object which is not considered empty is
                        passed. Empty objects are those such as the
                        following:
                    \begin{enumerate}
                        \begin{multicols}{2}
                            \item An empty string: ` ' or `` ''
                            \item An empty list: [ ]
                            \item The number zero: 0
                            \item The false Boolean variable: False
                        \end{multicols}
                    \end{enumerate}
                \end{enumerate}
            \textbf{More Operators}
                The ``is" operators is different from the equals
                operator $(==)$. Instead of comparing the values of
                the variables, the ``is" operator compares the
                instances themselves. Using the ``not" operator on a
                Boolean expression simply inverts it.\newline
                \begin{minipage}[t]{.48\textwidth}
                    \centering
                    \begin{lstlisting}[%
                        language=python,
                        frame=single,
                        basicstyle=\footnotesize,
                        frame=single,
                        caption=input,
                        gobble=24
                    ]
                        x = [1,2,3]
                        y = [1,2,3]
                        z = x
                        print(x == y) # Prints out True
                        print(x is y) # Prints out False
                        print(x is z) # Prints out True
                        print(not False)
                        print((not False) == (False))
                    \end{lstlisting}
                \end{minipage}\hfill
                \begin{minipage}[t]{.48\textwidth}
                    \centering
                    \begin{lstlisting}[%
                        language=bash,
                        basicstyle=\small\ttfamily,
                        frame=single,
                        caption=output,
                        gobble=24
                    ]
                        True
                        False
                        True
                        True
                        False
                    \end{lstlisting}
                \end{minipage}
        \subsubsection{Loops}
            \textbf{For Loops}
                For loops iterator over a sequence of numbers using
                the ``range'' and ``xrange'' functions. In Python 3
                there is only ``range,'' which is a direct descendant
                of the Python 2 ``xrange.'' While loops repeat so
                long as a certain Boolean condition is True. \newline
                \begin{minipage}[t]{.48\textwidth}
                    \centering
                    \begin{lstlisting}[%
                        language=python,
                        frame=single,
                        basicstyle=\footnotesize,
                        frame=single,
                        caption=input,
                        gobble=24
                    ]
                        # Prints out the numbers 0,1,2,3,4
                        for x in range(5):
                            print(x)
                        # Prints out 3,4,5
                        for x in range(3, 6):
                            print(x)
                        # Prints out 3,5,7
                        for x in range(3, 8, 2):
                            print(x)
                        # Prints out 0,1,2,3,4
                        count = 0
                        while count < 5:
                            print(count)
                            count += 1 # count = count + 1
                    \end{lstlisting}
                \end{minipage}\hfill
                \begin{minipage}[t]{.48\textwidth}
                    \centering
                    \begin{lstlisting}[%
                        language=python,
                        frame=single,
                        basicstyle=\footnotesize,
                        frame=single,
                        caption=output,
                        gobble=24
                    ]
                        0
                        1
                        2
                        3
                        4
                        3
                        4
                        5
                        3
                        5
                        7
                        0
                        1
                        2
                        3
                        4
                        0
                        1
                        2
                        3
                        4
                    \end{lstlisting}
                \end{minipage}
                The ``break" command is used to exit a for loop or a
                while loop. The ``continue" command is used to skip
                the current block and to return to the for loop or
                while loop.
                \newpage
                \begin{minipage}[t]{.48\textwidth}
                    \centering
                    \begin{lstlisting}[%
                        language=python,
                        frame=single,
                        basicstyle=\footnotesize,
                        frame=single,
                        caption=input,
                        gobble=24
                    ]
                        count = 0
                        while True:
                            print(count)
                            count += 1
                            if count >= 5:
                                break
                        for x in range(10):
                           # Check if x is even
                            if x % 2 == 0:
                                continue # Odd numbers only
                            print(x)
                    \end{lstlisting}
                \end{minipage}\hfill
                \begin{minipage}[t]{.48\textwidth}
                    \centering
                    \begin{lstlisting}[%
                        language=python,
                        frame=single,
                        basicstyle=\footnotesize,
                        frame=single,
                        caption=output,
                        gobble=24
                    ]
                        0
                        1
                        2
                        3
                        4
                        1
                        3
                        5
                        7
                        9
                    \end{lstlisting}
                \end{minipage}
                Unlike in C and C++, else can be used in for
                loops.\newline
                \begin{minipage}[t]{.48\textwidth}
                    \centering
                    \begin{lstlisting}[%
                        language=python,
                        frame=single,
                        basicstyle=\footnotesize,
                        frame=single,
                        caption=input,
                        gobble=24
                    ]
                        count=0
                        while(count<5):
                            print(count)
                            count +=1
                        else:
                            print("value: %d" %(count))
                        # Prints out 1,2,3,4
                        for i in range(1, 10):
                            if(i%5==0):
                                break
                            print(i)
                        else:
                            print("this is not printed")
                    \end{lstlisting}
                \end{minipage}\hfill
                \begin{minipage}[t]{.48\textwidth}
                    \centering
                    \begin{lstlisting}[%
                        language=python,
                        frame=single,
                        basicstyle=\footnotesize,
                        frame=single,
                        caption=output,
                        gobble=24
                    ]
                        0
                        1
                        2
                        3
                        4
                        value: 5
                        1
                        2
                        3
                        4
                    \end{lstlisting}
                \end{minipage}
        \subsubsection{Functions}
            Functions allow one to divide code into blocks, order
            code, and make it more readable and easier to reuse. They
            also provide a means of sharing code. Python makes use of
            blocks. Functions can recieve arguments, which are
            variables passed from the caller to the function. In
            addition, functions can return values to the caller using
            the ``return" keyword.
            \begin{lstlisting}[%
                language=python,
                frame=single,
                basicstyle=\footnotesize,
                frame=single,
                caption=Block Example,
                gobble=16
            ]
                block_head:
                    1st block line
                    2nd block line
                    ...
            \end{lstlisting}
            A block line is more Python code, including another
            block, and the block head contains keywords like ``if",
            ``or", ``for", and ``while," and also something like
            block\_name(argument1,...,argumentn). Functions in Python
            are defined using the block keyword ``def" followed with
            the function's name as the block's name.\newline
            \begin{minipage}[t]{.48\textwidth}
                \centering
                \begin{lstlisting}[%
                    language=python,
                    frame=single,
                    basicstyle=\footnotesize,
                    frame=single,
                    caption=input,
                    gobble=20
                ]
                    def my_function():
                        print("Hello From My Function!")
                    def my_function_with_args(name, greeting):
                        print("Hello, %s! I wish you %s"\
                              % (name, greeting))
                    def sum_two_numbers(a, b):
                        return a + b
                    # print(a simple greeting)
                    my_function()
                    my_function_with_args("John Doe",
                                          "a great year!")
                    x = sum_two_numbers(1,2)
                    print(x)
                \end{lstlisting}
            \end{minipage}\hfill
            \begin{minipage}[t]{.48\textwidth}
                \centering
                \begin{lstlisting}[%
                    language=python,
                    frame=single,
                    basicstyle=\footnotesize,
                    caption=output,
                    gobble=20
                ]
                Hello From My Function!
                Hello, John Doe! I wish you a great year!
                3
                \end{lstlisting}
            \end{minipage}
            \newpage
            An example of many functions being used together:
            \newline
            \begin{minipage}[t]{.48\textwidth}
                \centering
                \begin{lstlisting}[%
                    language=python,
                    frame=single,
                    basicstyle=\footnotesize,
                    caption=input,
                    gobble=20
                ]
                    def list_benefits():
                        return "Organized code",\
                        "Readable code",\ 
                        "Code reuse",\
                        "Sharing code"
                    def build_sentence(benefit):
                        return "%s is a benefit!" % benefit
                    def name_the_benefits_of_functions():
                        list_of_benefits = list_benefits()
                        for benefit in list_of_benefits:
                            print(build_sentence(benefit))
                        name_the_benefits_of_functions()
                \end{lstlisting}
            \end{minipage}\hfill
            \begin{minipage}[t]{.48\textwidth}
                \centering
                \begin{lstlisting}[%
                    language=python,
                    frame=single,
                    basicstyle=\footnotesize,
                    caption=output,
                    gobble=20
                ]
                    Organized code is a benefit!
                    Readable code is a benefit!
                    Code reuse is a benefit!
                    Sharing code is a benefit!
                \end{lstlisting}
            \end{minipage}
        \subsubsection{Classes and Objects}
            Objects are a combination of variables and functions into
            a single entity. Objects get these things from classes,
            which are essentially templates for creating objects.
            When a variable holds a class, it holds all of the
            objects within it. You can create multiple different
            objects that are of the same class and have the same
            variables and functions defined, but each object will
            contain independent copies of the variables in the class.
            Here's an example of a class:\newline
            \begin{minipage}[t]{.48\textwidth}
                \centering
                \begin{lstlisting}[%
                    language=python,
                    frame=single,
                    basicstyle=\footnotesize,
                    caption=input,
                    gobble=20
                ]
                    class MyClass:
                        variable = "blah"
                        def function(self):
                            print("Message Inside Class.")
                    myobjectx = MyClass()
                    myobjecty = MyClass()
                    myobjecty.variable = "yackity"
                    # Then print out both values
                    print(myobjectx.variable)
                    print(myobjecty.variable)
                    myobjectx.function()
                \end{lstlisting}
            \end{minipage}\hfill
            \begin{minipage}[t]{.48\textwidth}
                \centering
                \begin{lstlisting}[%
                    language=python,
                    frame=single,
                    basicstyle=\footnotesize,
                    frame=single,
                    caption=output,
                    gobble=20
                ]
                    blah
                    yackity
                    Message Inside Class.
                \end{lstlisting}
            \end{minipage}
        \subsubsection{Dictionaries}
            A dictionary is a data type that works with keys and
            values, rather than indexes. A key is any type of object,
            a string, a number, a list, etc.\newline
            \begin{minipage}[t]{.48\textwidth}
                \centering
                \begin{lstlisting}[%
                    language=python,
                    frame=single,
                    basicstyle=\footnotesize,
                    frame=single,
                    caption=input,
                    gobble=24
                ]
                        phonebook = {}
                        phonebook["John"] = 7566
                        phonebook["Jack"] = 7264
                        phonebook["Jill"] = 2781
                        print(phonebook)
                        phonebook1 = {
                            "John" : 7566,
                            "Jack" : 7264,
                            "Jill" : 2781
                        }
                        print(phonebook1)
                \end{lstlisting}
            \end{minipage}\hfill
            \begin{minipage}[t]{.48\textwidth}
                \centering
                \begin{lstlisting}[%
                    language=python,
                    frame=single,
                    basicstyle=\footnotesize,
                    frame=single,
                    caption=output,
                    gobble=24
                ]
                        {'Jill': 2781, 'Jack': 7264, 'John': 7566}
                        {'Jill': 2781, 'Jack': 7264, 'John': 7566}
                \end{lstlisting}
            \end{minipage}
            Dictionaries can be iterated over as well. Dictionaries
            do not keep the order of the values stored in it, so
            iterating is a little different.
            \newpage
            \begin{minipage}[t]{.48\textwidth}
                \centering
                \begin{lstlisting}[%
                    language=python,
                    frame=single,
                    basicstyle=\footnotesize,
                    caption=output,
                    gobble=20
                ]
                    phonebook = {
                        "John": 938477566,
                        "Jack": 938377264,
                        "Jill": 947662781
                    }
                    for name, number in phonebook.items():
                        print("Phone number of %s is %d"
                              % (name, number))
                \end{lstlisting}
            \end{minipage}\hfill
            \begin{minipage}[t]{.48\textwidth}
                \centering
                \begin{lstlisting}[%
                    language=python,
                    frame=single,
                    basicstyle=\footnotesize,
                    caption=output,
                    gobble=20
                ]
                    Phone number of Jill is 947662781
                    Phone number of Jack is 938377264
                    Phone number of John is 938477566
                \end{lstlisting}
            \end{minipage}
            There are two ways to remove items from a
            dictionary.\newline
            \begin{minipage}[t]{.48\textwidth}
                \centering
                \begin{lstlisting}[%
                    language=python,
                    frame=single,
                    basicstyle=\footnotesize,
                    caption=output,
                    gobble=20
                ]
                    phonebook = {
                       "John" : 938477566,
                       "Jack" : 938377264,
                       "Jill" : 947662781
                    }
                    del phonebook["John"]
                    print(phonebook)
                    phonebook.pop("Jack")
                    print(phonebook)
                \end{lstlisting}
            \end{minipage}\hfill
            \begin{minipage}[t]{.48\textwidth}
                \centering
                \begin{lstlisting}[%
                    language=python,
                    frame=single,
                    basicstyle=\footnotesize,
                    caption=output,
                    gobble=20
                ]
                    {'Jill': 947662781, 'Jack': 938377264}
                    {'Jill': 947662781}
                \end{lstlisting}
            \end{minipage}
        \subsubsection{Modules and Packages}
        A module is a piece of software that has a specific
        functionality. Each module is a different file which can be
        edited separately. Modules in Python are Python files with a
        .py extension. The name of the module is the name of the
        file. A Python module can have functions, classes, or
        variables defined and implemented. To import from other
        modules use the ``import" command.
    \subsection{Miscellaneous Tests}
        \subsubsection{Time Test for Squaring an Array}
            Running MacOSX 10.13.4 High Sierra on an iMac with a 3.4
            GHz intel quad-core i5 processor, it has been shown that,
            while running Python 3.6.3 and Numpy 1.14.1, the squaring
            is much slower than multiplying a variable by
            itself.\newline
            \begin{minipage}[t]{.48\textwidth}
                \centering
                \begin{lstlisting}[%
                    language=python,
                    frame=single,
                    basicstyle=\footnotesize,
                    caption=Python Code Located in `test.py',
                    gobble=20
                ]
                    import time
                    def square(x):
                        t1       = time.time()
                        for i in range(10000000): y = x**2
                        t2       = time.time()
                        t        = t2-t1
                        return t
                    def square2(x):
                        t1       = time.time()
                        for i in range(10000000): y = x*x
                        t2       = time.time()
                        t        = t2-t1
                        return t
                \end{lstlisting}
            \end{minipage}\hfill
            \begin{minipage}[t]{.48\textwidth}
                \centering
                \begin{lstlisting}[%
                    language=python,
                    frame=single,
                    basicstyle=\footnotesize,
                    caption=Inside iPython,
                    gobble=20
                ]
                    In [1]: import test
                    In [2]: test.square(10)
                    Out[2]: 2.5573959350585938
                    In [3]: test.square2(10)
                    Out[3]: 0.3849520683288574
                    In [4]: x = np.arange(100)
                    In [5]: test.square(x)
                    Out[5]: 6.9455132484436035
                    In [6]: test.square2(x)
                    Out[6]: 4.55057692527771
                \end{lstlisting}
            \end{minipage}\newline
            A similar test for multiplication versus adding.
            \newpage
            \begin{minipage}[t]{.48\textwidth}
                \centering
                \begin{lstlisting}[%
                    language=python,
                    frame=single,
                    basicstyle=\footnotesize,
                    caption=Contents of test.py,
                    gobble=20
                ]
                    import time
            
                    def mult(x):
                        t1       = time.time()
                        for i in range(10000000): y = 2*x
                        t2       = time.time()
                        t        = t2-t1
                        return t
                    
                    def add(x):
                        t1       = time.time()
                        for i in range(10000000): y = x+x
                        t2       = time.time()
                        t        = t2-t1
                        return t
                \end{lstlisting}
            \end{minipage}\hfill
            \begin{minipage}[t]{.48\textwidth}
                \centering
                \begin{lstlisting}[%
                    language=python,
                    frame=single,
                    basicstyle=\footnotesize,
                    caption=Inside iPython,
                    gobble=20
                ]
                    In [1]: from test import *
                    In [2]: mult(100)
                    Out[2]: 0.44980788230895996
                    In [3]: add(100)
                    Out[3]: 0.3794879913330078
                    In [4]: mult(10000000)
                    Out[4]: 0.5141329765319824
                    In [5]: add(10000000)
                    Out[5]: 0.4880061149597168
                \end{lstlisting}
            \end{minipage}\newline
            A test for how long it takes to pass arguments into
            functions.
            \newline
            \begin{lstlisting}[%
                language=python,
                frame=single,
                basicstyle=\footnotesize,
                caption=Inside iPython,
                gobble=16
            ]
                import time
                
                def func1(x):
                	t1 = time.time()
                	for i in range(1000000):
                		y = x
                	t2 = time.time()
                	print(t2-t1)
                
                def func2(x):
                	t1 = time.time()
                	for i in range(1000000):
                		y = x
                		y *= 1
                	t2 = time.time()
                	print(t2-t1)
                
                def func3(x):
                    t1 = time.time()
                    for i in range(1000000):
                        y  = x*1
                    t2 = time.time()
                    print(t2-t1)
                
                def func4(i,n_used,nw,loop):
                    mes = "Pt: %d  Tot: %d  Width: %d  Psi Iters: %d  Fast Inversion"
                    print(mes % (i,n_used,nw,loop),end="\r")
                
                def func5(i,n_used,nw,loop):
                	pass
                
                def func6(x,y):
                	t1 = time.time()
                	if x:
                		for i in range(y): func4(i,y,3,4)
                	else:
                		for i in range(y): func5(i,y,3,4)
                	t2 = time.time()
                	print(t2-t1)
            \end{lstlisting}
            Demonstrating the superiority of C over Python.
            The following is written in C. This time test was
            performed on a 13 inch Mid-2012 Macbook Pro running macOS
            High Sierra 10.13.6 with a 2.9GHz duo-core intel i7
            processor and 8GB of 1600 MHz DDR3 memory.
            \newpage
            \begin{minipage}[t]{.48\textwidth}
                \centering
                \begin{lstlisting}[%
                    language=C,
                    frame=single,
                    basicstyle=\footnotesize,
                    caption=test.c,
                    gobble=20
                ]
                    #include <stdio.h>
                    
                    int add(int n){
                            int x, i;
                            x = 0;
                            for (i=0; i<n; ++i){
                                    x += 1;
                            }
                            return x;
                    }
                \end{lstlisting}
            \end{minipage}\hfill
            \begin{minipage}[t]{.48\textwidth}
                \centering
                \begin{lstlisting}[%
                    language=Python,
                    frame=single,
                    basicstyle=\footnotesize,
                    caption=ctest.py,
                    gobble=20
                ]
                    import ctypes
                    import time
                    lib = ctypes.cdll.LoadLibrary("./test.so")
                    fun = lib.add
                    n = 1000000000
                    t1 = time.time()
                    y = fun(n)
                    t2 = time.time()
                    t3 = time.time()
                    x = 0
                    for i in range(n):
                            x += 1
                    t4 = time.time()
                    tc = t2-t1
                    tp = t4-t2
                    print("C Computation Time: %f" % tc)
                    print("Python Computation Time: %f" % tp)
                    print("C For-Loop Size: %d" % y)
                    print("Python For-Loop Size: %d" % x)
                \end{lstlisting}
            \end{minipage}
            \begin{lstlisting}[%
                language=Python,
                frame=single,
                basicstyle=\footnotesize,
                caption=Compile and Compare,
                gobble=16
            ]
                Ryans-MacBookPro:~ ryan$ gcc -fPIC -shared -o test.so test.c
                Ryans-MacBookPro:~ ryan$ ipython
                
                In [1]: %run ctest.py
                C Computation Time: 1.849072
                Python Computation Time: 123.264628
                C For-Loop Size: 1000000000
                Python For-Loop Size: 1000000000
                
                In [2]: %run ctest.py  # I closed out of all background processes and tried again.
                C Computation Time: 1.771411
                Python Computation Time: 110.674400
                C For-Loop Size: 1000000000
                Python For-Loop Size: 1000000000
                
                In [3]: 110.674400/1.771411  # Time ratio of second test.
                Out[3]: 62.47810361344713
                
                In [4]: 123.264628/1.849072  # Time ratio of first test.
                Out[4]: 66.66296823487674
            \end{lstlisting}
            So we see that C is over 60 times faster at computing even a simple program, such as this ``Add 1" routine.
            Running this purely in C (Using the gcc compiler) has no
            noticeable affect in computation time.
            \begin{lstlisting}[%
                language=C,
                frame=single,
                basicstyle=\footnotesize,
                caption=test.c,
                gobble=12
            ]
                #include <stdio.h>
                #include <time.h>
                
                int add(int n){
                        int x, i;
                        x = 0;
                        for (i=0; i<n; ++i){
                                x += 1;
                        }
                        return x;
                }
                
                int main(){
                        clock_t begin = clock();
                        add(1000000000);
                        clock_t end = clock();
                        double time_spent = (double)(end - begin)/\
                        CLOCKS_PER_SEC;
                        printf("Computation Time: %f\n", time_spent);
                        return 0;
                }
            \end{lstlisting}
            \newpage
            \begin{lstlisting}[%
                language=Bash,
                frame=single,
                basicstyle=\footnotesize,
                caption=Compile and Run,
                gobble=12
            ]
                Ryans-MacBookPro:Desktop ryan$ gcc test.c -o test
                Ryans-MacBookPro:Desktop ryan$ ./test
                Computation Time: 1.869245
                Ryans-MacBookPro:Desktop ryan$ ./test
                Computation Time: 1.778661
                Ryans-MacBookPro:Desktop ryan$ ./test
                Computation Time: 1.781933
            \end{lstlisting}
\end{document}