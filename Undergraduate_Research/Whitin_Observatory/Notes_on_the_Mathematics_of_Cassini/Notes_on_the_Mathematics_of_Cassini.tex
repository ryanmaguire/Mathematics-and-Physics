\documentclass[crop=false,class=book,oneside]{standalone}
%----------------------------Preamble-------------------------------%
%---------------------------Packages----------------------------%
\usepackage{geometry}
\geometry{b5paper, margin=1.0in}
\usepackage[T1]{fontenc}
\usepackage{graphicx, float}            % Graphics/Images.
\usepackage{natbib}                     % For bibliographies.
\bibliographystyle{agsm}                % Bibliography style.
\usepackage[french, english]{babel}     % Language typesetting.
\usepackage[dvipsnames]{xcolor}         % Color names.
\usepackage{listings, lstlinebgrd}      % Verbatim-Like Tools.
\usepackage{mathtools, esint, mathrsfs} % amsmath and integrals.
\usepackage{amsthm, amsfonts}           % Fonts and theorems.
\usepackage{tabularx}
\usepackage{tcolorbox}                  % Frames around theorems.
\usepackage{upgreek}                    % Non-Italic Greek.
\usepackage{paracol}                    % Two-column styling.
\usepackage{wrapfig}                    % Wrap text around figure.
\usepackage{fmtcount, etoolbox}         % For the \book{} command.
\usepackage[newparttoc]{titlesec}       % Formatting chapter, etc.
\usepackage{titletoc}                   % Allows \book in toc.
\usepackage[nottoc]{tocbibind}          % Bibliography in toc.
\usepackage[titles]{tocloft}            % ToC formatting.
\usepackage{multicol, enumitem}         % Multi-column/enumerate.
\usepackage{import}                     % Import external files.
\usepackage{pgfplots, tikz}             % Drawing/graphing tools.
\usetikzlibrary{
    calc,                   % Calculating right angles and more.
    angles,                 % Drawing angles within triangles.
    arrows.meta,            % Latex and Stealth arrows.
    quotes,                 % Adding labels to angles.
    positioning,            % Relative positioning of nodes.
    decorations.markings,   % Adding arrows in the middle of a line.
    patterns,
    arrows,
    shapes,
    shapes.geometric,
    cd,
    hobby,
    babel
}                                       % Libraries for tikz.
\pgfplotsset{compat=1.9}                % Version of pgfplots.
\usepackage[font=scriptsize,
            labelformat=simple,
            labelsep=colon]{subcaption} % Subfigure captions.
\usepackage[font={scriptsize},
            hypcap=true,
            labelsep=colon]{caption}    % Figure captions.
\usepackage{hyperref}                   % Allows for hyperlinks.
\hypersetup{
    colorlinks=true,
    linkcolor=blue,
    filecolor=magenta,
    urlcolor=Cerulean,
    citecolor=SkyBlue
}                           % Colors for hyperref.
\usepackage[toc,acronym,nogroupskip]{glossaries} % Glossaries and acronyms.
\usepackage[subpreambles=false]{standalone}      % Complileable sub files.

% Various font stuff from kiwi.
% Use this for Times text and Computer Modern math
%\usepackage{times}

% Quite nice
%\usepackage[charter, greekfamily=, greekuppercase=italicized]{mathdesign}
%\usepackage[utopia, greekuppercase=italicized]{mathdesign}    % Math is narrower

% Use this for Times text and math
%\usepackage{newtxtext}
%\usepackage[libertine,cmintegrals]{newtxmath}
%\usepackage{fix-cm}

%\usepackage{txfontsb}
% or
%\usepackage{mathptmx}

%\usepackage[scaled=0.92]{helvet}
%\renewcommand{\rmdefault}{ptm}

%\usepackage{mathpazo}    % add possibly `sc` and `osf` options
%\usepackage{eulervm}

%\usepackage{fourier}
%\renewcommand{\rmdefault}{ptm}
%\usepackage{mathptm}

%\usepackage{fontspec}
%\setmainfont{lmodern}

%\usepackage[varg]{txfonts}
%\usepackage{fouriernc}
%\usepackage{mathpazo}

%\usepackage{bookman}
%\usepackage[scaled]{uarial}
%\usepackage[scaled]{helvet}
%\renewcommand*\familydefault{\sfdefault}
%\usepackage[math]{anttor}

%\newcommand\fgeorgia{\fontfamily{jvn}\selectfont}
%\newcommand\ftimes{\fontfamily{ptm}\selectfont}
%\newcommand\fhelvetica{\fontfamily{phv}\selectfont}
%\newcommand\fcourier{\fontfamily{pcr}\selectfont}
%\newcommand\fbookman{\fontfamily{pbk}\selectfont}
%\newcommand\fnewcentury{\fontfamily{pnc}\selectfont}
%\newcommand\fpalatino{\fontfamily{ppl}\selectfont}
%\newcommand\favantgarde{\fontfamily{pag}\selectfont}
%\newcommand\fnormal{\normalfont}
%\newcommand\fsize[1]{\ifnum#1>0\fontsize{#1}{#1}\selectfont\else\normalsize\fi}
%------------------------Theorem Styles-------------------------%
% Define theorem style for default spacing and normal font.
\newtheoremstyle{normal}
    {\topsep}               % Amount of space above the theorem.
    {\topsep}               % Amount of space below the theorem.
    {}                      % Font used for body of theorem.
    {}                      % Measure of space to indent.
    {\bfseries}             % Font of the header of the theorem.
    {}                      % Punctuation between head and body.
    {.5em}                  % Space after theorem head.
    {}

% Define theorem style for default spacing with italicized font.
\newtheoremstyle{normalit}{\topsep}{\topsep}
                {\itshape}{}{\bfseries}{}{.5em}{}

% Italic header environment.
\newtheoremstyle{thmit}{\topsep}{\topsep}{}{}{\itshape}{}{0.5em}{}

% Define italicized environments.
\theoremstyle{normalit}
\newtheorem{theorem}{Theorem}[section]
\newtheorem{lemma}{Lemma}[section]
\newtheorem{corollary}{Corollary}[section]
\newtheorem{proposition}{Proposition}[section]
\newtheorem*{theorem*}{Theorem}

% Define environments with italic headers.
\theoremstyle{thmit}
\newtheorem*{solution}{Solution}
\newtheorem*{fsolution}{Solution}

% Define default environments.
\theoremstyle{normal}
\newtheorem{example}{Example}[section]
\newtheorem{definition}{Definition}[section]
\newtheorem{problem}{Problem}[section]
\newtheorem{question}{Question}[section]
\newtheorem{remark}{Remark}[section]
\newtheorem{properties}{Properties}[section]
\newtheorem{notation}{Notation}[section]
\newtheorem{axiom}{Axiom}[section]
\newtheorem*{properties*}{Properties}
\newtheorem*{remark*}{Remark}
\newtheorem*{definition*}{Definition}
\theoremstyle{plain}

% Define framed environment.
\tcbuselibrary{most}
\newtcbtheorem[use counter*=theorem]{ftheorem}{Theorem}%
    {colback=green!5,colframe=green!35!black,
     fonttitle=\bfseries\upshape}{th}

\newtcbtheorem[use counter*=example]{fdefinition}{Definition}%
    {fonttitle=\bfseries\upshape,
     colback=blue!5!white,colframe=blue!75!black}{def}

\newtcbtheorem[use counter*=example]{fexample}{Example}%
    {fonttitle=\bfseries\upshape,
     colback=red!5!white,colframe=red!75!black}{ex}

\newtcbtheorem[use counter*=notation]{fnotation}{Notation}%
    {fonttitle=\bfseries\upshape,
     colback=SeaGreen!5!white,colframe=SeaGreen!75!black}{ex}

\newtcbtheorem[use counter*=corollary]{fcorollary}{Corollary}%
    {fonttitle=\bfseries\upshape,
     colback=Orchid!5!white,colframe=Orchid!75!black}{ex}

\newenvironment{bproof}{\textit{Proof.}}{\hfill$\square$}
\tcolorboxenvironment{bproof}{blanker,breakable,left=5mm,
                             before skip=10pt,after skip=10pt,
                             borderline west={1mm}{0pt}{red}}
\tcolorboxenvironment{fsolution}
    {enhanced jigsaw,colframe=cyan,interior hidden,breakable}

%--------------------Declared Math Operators--------------------%
\DeclareMathOperator{\Refl}{Refl}           % Reflection operator.
\DeclareMathOperator{\Span}{Span}           % Span of a set of vectors.
\DeclareMathOperator{\Card}{Card}           % Cardinality of set.
\DeclareMathOperator{\Ord}{Ord}             % Ordinal of ordered set.
\DeclareMathOperator{\Tr}{Tr}               % Trace of matrix.
\DeclareMathOperator{\adjoint}{adj}         % Adjoint of matrix.
\DeclareMathOperator{\rk}{rk}               % Rank of operator.
\DeclareMathOperator{\nul}{nul}             % Null space of operator.
\DeclareMathOperator{\sgn}{sgn}             % Sign of a number.
\DeclareMathOperator{\multideg}{mutlideg}   % Multi-Degree (Graphs).
\DeclareMathOperator{\GCD}{GCD}             % Greatest common denominator.
\DeclareMathOperator{\LM}{LM}               % Leading monomial
\DeclareMathOperator{\LC}{LC}               % Leading coefficient.
\DeclareMathOperator{\LT}{LT}               % Leading term.
\DeclareMathOperator{\LCM}{LCM}             % Least common multiple.
\DeclareMathOperator{\Mon}{Mon}             % Monomial.
\DeclareMathOperator{\Spec}{Spec}           % Spectrum.
\DeclareMathOperator{\proj}{proj}           % Projection.
\DeclareMathOperator{\comp}{comp}           % Component.
\DeclareMathOperator{\sinc}{sinc}           % Sinc function.
\DeclareMathOperator{\Ima}{Im}              % Image of operator.
\DeclareMathOperator{\Prin}{Prin}           % Principal value.
\DeclareMathOperator{\Mod}{mod}             % Modulus.
%------------------------New Commands---------------------------%
\DeclarePairedDelimiter\norm{\lVert}{\rVert}
\DeclarePairedDelimiter\ceil{\lceil}{\rceil}
\DeclarePairedDelimiter\floor{\lfloor}{\rfloor}
\newcommand*\diff{\mathop{}\!\mathrm{d}}
\newcommand*\Diff[1]{\mathop{}\!\mathrm{d^#1}}
\renewcommand{\mod}{\ \Mod}
\renewcommand*{\glstextformat}[1]{\textcolor{RoyalBlue}{#1}}
\renewcommand{\glsnamefont}[1]{\textbf{#1}}
\renewcommand\labelitemii{$\circ$}
\renewcommand\thesubfigure{\arabic{chapter}.\arabic{figure}}
\renewcommand\thesubfigure{%
    \arabic{chapter}.\arabic{figure}.\arabic{subfigure}}
\addto\captionsenglish{\renewcommand{\figurename}{Fig.}}
%------------------------Book Command---------------------------%
\makeatletter
\renewcommand\@pnumwidth{1cm}
\newcounter{book}
\renewcommand\thebook{\@Roman\c@book}
\newcommand\book{%
    \if@openright
        \cleardoublepage
    \else
        \clearpage
    \fi
    \thispagestyle{plain}%
    \if@twocolumn
        \onecolumn
        \@tempswatrue
    \else
        \@tempswafalse
    \fi
    \null\vfil
    \secdef\@book\@sbook
}
\def\@book[#1]#2{%
    \ifnum \c@secnumdepth >-3\relax
        \refstepcounter{book}%
        \addcontentsline{toc}{book}{
            \bookname\ \thebook:\hspace{1em}#1
        }
    \else
        \addcontentsline{toc}{book}{#1}%
    \fi
    \markboth{}{}%
    {\centering
     \interlinepenalty \@M
     \normalfont
     \ifnum \c@secnumdepth >-2\relax
       \huge\bfseries \bookname\nobreakspace\thebook
       \par
       \vskip 20\p@
     \fi
     \Huge \bfseries #2\par}%
    \@endbook}
\def\@sbook#1{%
    {\centering
     \interlinepenalty \@M
     \normalfont
     \Huge \bfseries #1\par}%
    \@endbook}
\def\@endbook{
    \vfil\newpage
        \if@twoside
            \if@openright
                \null
                \thispagestyle{empty}%
                \newpage
            \fi
        \fi
        \if@tempswa
            \twocolumn
        \fi
}
\newcommand*\l@book[2]{%
    \ifnum \c@tocdepth >-2\relax
        \addpenalty{-\@highpenalty}%
        \addvspace{2.25em \@plus\p@}%
        \setlength\@tempdima{3em}%
        \begingroup
            \parindent \z@ \rightskip \@pnumwidth
            \parfillskip -\@pnumwidth
            {
                \leavevmode
                \Large \bfseries #1\hfil \hb@xt@\@pnumwidth{
                    \hss #2
                }
            }
            \par
            \nobreak
            \global\@nobreaktrue
            \everypar{\global\@nobreakfalse\everypar{}}%
        \endgroup
    \fi}
\newcommand\bookname{Book}
\renewcommand{\thebook}{\texorpdfstring{\Numberstring{book}}{book}}
\providecommand*{\toclevel@book}{-2}
\makeatother
\titlecontents{chapter}[0pt]
    {\bfseries}
    {\chaptername\ \thecontentslabel:\quad}
    {}
    {\hfill\contentspage}
\titleformat{\part}[display]
    {\Large\bfseries}
    {\partname\nobreakspace\thepart}
    {0mm}
    {\Huge\bfseries}
    \titlecontents{part}[0pt]
    {\large\bfseries}
    {\partname\ \thecontentslabel: \quad}
    {}
    {\hfill\contentspage}
\newcommand{\MarkRightAngle}[4][.3cm]
    {\coordinate (tempa) at ($(#3)!#1!(#2)$);
     \coordinate (tempb) at ($(#3)!#1!(#4)$);
     \coordinate (tempc) at ($(tempa)!0.5!(tempb)$);%midpoint
     \draw (tempa) -- ($(#3)!2!(tempc)$) -- (tempb);}
%--------------------------LENGTHS------------------------------%
% Spacings for the Table of Contents.
\addtolength{\cftsecnumwidth}{1ex}
\addtolength{\cftsubsecindent}{1ex}
\addtolength{\cftsubsecnumwidth}{1ex}
\addtolength{\cftfignumwidth}{1ex}
\addtolength{\cfttabnumwidth}{1ex}

% Spacing for multi-column and enumerate environments.
\setlength{\multicolsep}{6pt}
\setlist[enumerate]{itemsep=0pt,topsep=3pt}

% Indent and paragraph spacing.
\setlength{\parindent}{0em}
\setlength{\parskip}{0em}
%--------------------------Main Document----------------------------%
\begin{document}
    \newif\ifresearchnotesosthemathematicsofcassini
    \ifx\ifresearch\undefined
        \title{Notes on the Mathematics of Cassini}
        \author{Ryan Maguire}
        \date{\vspace{-5ex}}
        \maketitle
        \tableofcontents
        \chapter*{Notes on Mathematics}
        \addcontentsline{toc}{chapter}{Notes on Mathematics}
        \markboth{}{CASSINI MATHEMATICS}
        \setcounter{chapter}{1}
    \else
        \chapter{Notes on the Mathematics of Cassini}
    \fi
    \section{Atmospheric Occultations}
        \subimport{./Atmospheric_Occultations/}{Titan_Geometry}
    \section{Ring Occultation}
        \subimport{./Ring_Occultations/}{Ring_Geometry}
        \subimport{./Ring_Occultations/}
                  {Derivations_of_Equations_in_MTR86}
        \subimport{./Ring_Occultations/}
                  {Shannons_Sampling_Theorem}
        \subsection{Diffraction Through a Square Well}
            We wish to solve for $\hat{T}(\rho_{0})$. We have that:
            \begin{equation*}
            \hat{T}(\rho_0) = \frac{1-i}{2F}\int_{-\infty}^{\infty}T(\rho)e^{i\frac{\pi}{2}\big(\frac{\rho-\rho_0}{F}\big)^2}d\rho
            \end{equation*}
            For the square well:
            \begin{equation*}
            T(\rho) = \begin{cases} 0, & \rho<a \\ M, & a \leq \rho \leq b \\ 0, & b < \rho\end{cases}
            \end{equation*}
            where $a$ and $b$ are the start and end points of the well, respectively. So, we have:
            \begin{equation*}
            \hat{T}(\rho_0) = \frac{1-i}{2F}\int_{a}^{b}Me^{i\frac{\pi}{2}\big(\frac{\rho-\rho_0}{F}\big)^2}d\rho = \frac{M}{F}\frac{1-i}{2}\int_{a}^{b}e^{i\frac{\pi}{2}\big(\frac{\rho-\rho_0}{F}\big)^2}d\rho
            \end{equation*}
            Now from Euler's formula, $e^{i\theta} = \cos(\theta)+i\sin(\theta).$ Letting $\theta = \frac{\pi}{2}\big(\frac{\rho-\rho_0}{F}\big)^2$, we have:
            \begin{equation*}
            e^{i\frac{\pi}{2}\big(\frac{\rho-\rho_0}{F}\big)^2} = \cos\bigg(\frac{\pi}{2}\big(\frac{\rho-\rho_0}{F}\big)^2\bigg)+i\sin\bigg(\frac{\pi}{2}\big(\frac{\rho-\rho_0}{F}\big)^2\bigg)
            \end{equation*}
            The definition of the Fresnel Cosine Integral and Sine Integrals $\big(C(x),S(x)\big)$ are:
            \begin{equation*}
            C(x) = \int_{0}^{x} \cos\big(\frac{\pi}{2}\tau^2\big)d\tau \quad\quad\quad\quad S(x) = \int_{0}^{x} \sin\big(\frac{\pi}{2}\tau^2\big)d\tau
            \end{equation*}
            So:
            \begin{align*}
            \int_{a}^{b} \sin\bigg(\frac{\pi}{2}x^2\bigg)dx &= \int_{0}^{b}\sin\bigg(\frac{\pi}{2}x^2\bigg)dx - \int_{0}^{a}\sin\bigg(\frac{\pi}{2}x^2\bigg)dx = S(b) - S(a) \\
            \int_{a}^{b} \cos\bigg(\frac{\pi}{2}x^2\bigg)dx &= \int_{0}^{b} \cos\bigg(\frac{\pi}{2}x^2\bigg)dx - \int_{0}^{a}\cos\bigg(\frac{\pi}{2}x^2\bigg)dx = C(b) - C(a)
            \end{align*}
            We may use this to compute $\int_{a}^{b} e^{i\frac{\pi}{2}\big(\frac{\rho-\rho_0}{F}\big)}d\rho$. Let $x=\frac{\rho-\rho_0}{F}$. Then $dx = \frac{d\rho}{F}$. We have:
            \begin{align*}
            \nonumber \frac{M}{F}\frac{1-i}{2}\int_{a}^{b}e^{i\frac{\pi}{2}\big(\frac{\rho-\rho_0}{F}\big)^2}d\rho &= \frac{M}{F}\frac{1-i}{2}\Bigg(\int_{a}^{b} \cos\bigg(\frac{\pi}{2}\big(\frac{\rho-\rho_0}{F}\big)^2\bigg)d\rho + i\int_{a}^{b} \sin\bigg(\frac{\pi}{2}\big(\frac{\rho-\rho_0}{F}\big)\bigg)d\rho\Bigg) \\
             &= M\frac{1-i}{2}\Bigg(\int_{\frac{a-\rho_0}{F}}^{\frac{b-\rho_0}{F}}\cos\bigg(\frac{\pi}{2}x^2\bigg)dx + i\int_{\frac{a-\rho_0}{F}}^{\frac{b-\rho_0}{F}}\sin\bigg(\frac{\pi}{2}x^2\bigg)dx\Bigg)\\
            \nonumber &= M\frac{1-i}{2}\Bigg(\bigg(C\big(\frac{b-\rho_0}{F}\big)-C\big(\frac{a-\rho_0}{F}\big)\bigg)+i\bigg(S\big(\frac{b-\rho_0}{F}\big)-S\big(\frac{a-\rho_0}{F}\big)\bigg)\Bigg)
            \end{align*}
            \begin{notation}
            Given a square well $T(\rho)$ of height $M$, starting at $a$ and $b$, we write the solution as:
            \begin{equation*}
            H_{M}(\rho_0,F;a,b) = M\frac{1-i}{2}\Bigg(\bigg(C\big(\frac{b-\rho_0}{F}\big)-C\big(\frac{a-\rho_0}{F}\big)\bigg)+i\bigg(S\big(\frac{b-\rho_0}{F}\big)-S\big(\frac{a-\rho_0}{F}\big)\bigg)\Bigg)
            \end{equation*}
            \end{notation}
        \subsection{Diffraction Through an Inverted Square Well}
            This time we want:
            \begin{equation*}
                T'(\rho) = \begin{cases} M, & \rho<a \\ 0, & a\leq \rho \leq b \\ M, & b<\rho\end{cases}    
            \end{equation*}
            So $T'(\rho) = M - T(\rho)$, where $T(\rho)$ is the impulse from the previous derivation. So we wish to solve:
            \begin{equation*}
            \frac{1-i}{2F}\int_{-\infty}^{\infty}\big(M-T(\rho)\big)e^{i\frac{\pi}{2}\big(\frac{\rho-\rho_0}{F}\big)^2}d\rho
            \end{equation*}
            Using the result from the previous derivation, we have:
            \begin{equation*}
            \frac{1-i}{2F}\int_{-\infty}^{\infty}\big(M-T(\rho)\big)e^{i\frac{\pi}{2}\big(\frac{\rho-\rho_0}{F}\big)^2}d\rho = \frac{M}{F}\frac{1-i}{2}\Bigg(\int_{-\infty}^{\infty}e^{i\frac{\pi}{2}\big(\frac{\rho-\rho_0}{F}\big)^2}d\rho - \int_{a}^{b}e^{i\frac{\pi}{2}\big(\frac{\rho-\rho_0}{F}\big)^2}d\rho\Bigg)\\
            \end{equation*}
            But from the previous derivation:
            \begin{equation*}
                \int_{a}^{b}e^{i\frac{\pi}{2}\big(\frac{\rho-\rho_0}{F}\big)^{2}}d\rho = F\bigg(C\big(\frac{b-\rho_0}{F}\big)-C\big(\frac{a-\rho_0}{F}\big)\bigg)+i\bigg(S\big(\frac{b-\rho_0}{F}\big)-S\big(\frac{a-\rho_0}{F}\big)\bigg)
            \end{equation*}
            And:
            \begin{equation*}
                \int_{-\infty}^{\infty}e^{i\frac{\pi}{2}\big(\frac{\rho-\rho_0}{F}\big)^{2}}d\rho = \underset{a\rightarrow \infty}{\lim} \int_{-a}^{a}e^{i\frac{\pi}{2}\big(\frac{\rho-\rho_0}{F}\big)^{2}}d\rho = F(1+i)
            \end{equation*}
            Piecing this together, and using the notation from before, we have:
            \begin{equation*}
            \hat{T}(\rho_0) = M-H_{M}(\rho_0;F;a,b)
            \end{equation*}
        \subsection{Radial Shift from a Linear Phase Offset}
            \begin{theorem}
            If $\hat{T}_0(\rho_0) = \hat{T}(\rho_0)e^{i(a\rho+b)}$, and if $\hat{f}(\xi) = \mathcal{F}_{\xi}(\hat{T}(\rho_0))$, then $\mathcal{F}_{\xi}(\hat{T}_{0}(\rho_0)) = e^{ib}\hat{f}(\frac{a}{2\pi}+\xi)$.
            \end{theorem}
            \begin{proof}
            Let $\hat{T}(\rho_0)$ be the complex amplitude, $\hat{T} = \norm{\hat{T}}e^{i\theta}$, where $\theta = \theta(\rho_0)$ is the phase. Let $\hat{f}(\xi)$ denote the Fourier Transform of $\hat{T}(\rho_0)$ onto $\xi$. If there is a linear offset in the phase $a\rho_0+b$, we have $\hat{T}_{0} = \hat{T}e^{i(a\rho+ib)}$. Taking the Fourier Transform of this, we have the following:
            \begin{align*}
            \nonumber \mathcal{F}_{\xi} (\hat{T}_{0})&= \int_{-\infty}^{\infty}\hat{T}(\rho_0)e^{i(a\rho+b)}e^{i2\pi \rho_0 \xi}d\rho_0\\
            &=e^{ib} \int_{-\infty}^{\infty} \hat{T}(\rho_0)e^{i\rho_0(a+2\pi \xi)}d\rho_0
            \end{align*}
            
            Letting $\eta = \frac{a}{2\pi}+\xi$, we have:
            
            \begin{align*}
            \nonumber \mathcal{F}_{\xi} (\hat{T}_{0}) &= e^{ib}\int_{-\infty}^{\infty}\hat{T}(\rho_0)e^{i2\pi\rho_{0}\eta}d\rho_0 \\
            \nonumber &= e^{ib}\hat{f}(\eta)\\
            		 &= e^{ib}\hat{f}(\frac{a}{2\pi}+\xi)
            \end{align*}
            
            Thus, we have that:
            
            \begin{equation*}
            \mathcal{F}_{\xi} (\hat{T}_{0}) = e^{ib}\hat{f}(\frac{a}{2\pi}+\xi)
            \end{equation*}
            So we see that a linear offset in phase creates a horizontal shift in the Fourier transform.
            \end{proof}
            
            We want the affects on $T(\rho)$.
            
            \begin{theorem}
            If $\hat{T}_0(\rho_0) = \hat{T}(\rho_0)e^{i(a\rho+b)}$, $F\ne 0$, $T(\rho) = \int_{-\infty}^{\infty}\hat{T}(\rho_0)e^{i\frac{\pi}{2}\big(\frac{\rho-\rho_0}{F}\big)^2} d\rho_0$, and if $T_{0}(\rho) = \int_{-\infty}^{\infty}\hat{T}_{0}(\rho_0)e^{i\frac{\pi}{2}\big(\frac{\rho-\rho_0}{F}\big)^2} d\rho_0$, then $\norm{T_{0}(\rho)} = \norm{T(\rho - \frac{aF^2}{\pi})}$
            \end{theorem}
            \begin{proof}
            \begin{align*}
            T_0(\rho) &= \int_{-\infty}^{\infty} \hat{T}_{0}(\rho_0)e^{i\frac{\pi}{2}\big(\frac{\rho-\rho_0}{F}\big)^2}d\rho_0\\
            &=\int_{-\infty}^{\infty} \hat{T}(\rho_0)e^{i(a\rho_0+b)}e^{i\frac{\pi}{2}\big(\frac{\rho - \rho_0}{F}\big)^2}d\rho_0 \\
            	&= e^{ib}\int_{-\infty}^{\infty}\hat{T}(\rho_0)e^{i\frac{\pi}{2}\bigg[\big(\frac{\rho-\rho_0}{F}\big)^2 + \frac{2a}{\pi}\rho_0\bigg]}d\rho_0
            \end{align*}
            Expanding the terms in the exponential and simplifying, we have:
            \begin{align*}
                \big(\frac{\rho-\rho_0}{F}\big)^2 + \frac{2a}{\pi}\rho &= \frac{\rho_0^2 - 2\rho\rho_0 + \rho^2 + \frac{2aF^2}{\pi}\rho_0}{F^2}\\
                &= \frac{\rho_0^2 - 2\rho_0(\rho - \frac{aF^2}{\pi}) + \rho^2}{F^2}\\
                &= \frac{\big(\rho_0 - (\rho - \frac{aF^2}{\pi})\big)^2 - (\rho - \frac{aF^2}{\pi})^2 + \rho^2}{F^2}\\
                &= \frac{\big(\rho_0 - (\rho-\frac{aF^2}{\pi})\big)^2 +\frac{2aF^2}{\pi}\rho - \frac{a^2F^4}{\pi^2}}{F^2}\\
                &= \frac{\big(\rho_0 - (\rho-\frac{aF^2}{\pi})\big)^2}{F^2} + \frac{2a}{\pi}\rho - \frac{a^2F^2}{\pi^2}
            \end{align*}
            The integral is over $\rho_0$, so we may write:
            \begin{equation*}
            T_0(\rho) = e^{ib}e^{i\frac{\pi}{2}\big(\frac{2a}{\pi}\rho - \frac{a^2F^2}{\pi^2}\big)}\int_{-\infty}^{\infty} \hat{T}(\rho_0)e^{i\frac{\pi}{2}\big(\frac{\rho_0 - (\rho - \frac{aF^2}{\pi})}{F}\big)^2}d\rho_0
            \end{equation*}
            Let $u = \rho - \frac{aF^2}{\pi}$. Then we have:
            \begin{equation*}
            \int_{-\infty}^{\infty} \hat{T}(\rho_0)e^{i\frac{\pi}{2}\big(\frac{\rho_0 - u}{F}\big)^2}d\rho_0 = T(u)
            \end{equation*}
            Therefore:
            \begin{equation*}
            T_0(\rho) = e^{ib}e^{i\frac{\pi}{2}\big(\frac{2a}{\pi}\rho - \frac{a^2F^2}{\pi^2}\big)}T(\rho - \frac{aF^2}{\pi})
            \end{equation*}
            Computing reconstructed power takes the norm $\norm{T_{0}(\rho)}$, and $\norm{e^{ib}e^{i\frac{\pi}{2}\big(\frac{2a}{\pi}\rho - \frac{a^2F^2}{\pi^2}\big)}} = 1$, for all values of $a,F, \rho$ (This is from Euler's theorem). Thus:
            \begin{equation*}
                \norm{T_{0}(\rho)} = \norm{T\big(\rho - \frac{aF^2}{\pi}\big)}    
            \end{equation*}
            So a linear offset $a\rho_0+b$ in the phase creates a radial offset in the reconstructed power of $-\frac{aF^2}{\pi}$.
            \end{proof}
        \subsection{Some Useful Results}
            \begin{theorem}
            $\sqrt{i} = \frac{1+i}{\sqrt{2}}$.
            \end{theorem}
            \begin{proof}
            For $i = e^{i\frac{\pi}{2}}$, and thus $\sqrt{i} = e^{i\frac{\pi}{4}} = \cos(\frac{\pi}{4})+i\sin(\frac{\pi}{4}) = \frac{1+i}{\sqrt{2}}$
            \end{proof}
            \begin{theorem}
            If $a+ib$ is a complex number, and $(a,b) \ne (0,0)$, then $\frac{1}{a+ib} = \frac{a-ib}{a^2+b^2}$.
            \end{theorem}
            \begin{proof}
            If $(a,b)\ne (0,0)$, then $a^2+b^2>0$, so $\frac{a-ib}{a^2+b^2}$ is a complex number. But:
            \begin{equation*}
                (a+ib)\cdot \frac{a-ib}{a^2+b^2} = \frac{(a+ib)(a-ib)}{a^2+b^2} = \frac{a^2+b^2}{a^2+b^2} = 1
            \end{equation*}
            From the uniqueness of inverses, $\frac{1}{a+ib} = \frac{a-ib}{a^2+b^2}$.
            \end{proof}
            \begin{theorem}
            $\int_{-\infty}^{\infty}e^{-x^2}dx = \sqrt{\pi}$.
            \end{theorem}
            \begin{proof}
            Convergence can be shown, for $0 < e^{-x^2} \leq \frac{1}{1+x^2}$ for all $x\in \mathbb{R}$. Therefore:
            \begin{equation*}
            0<\int_{-\infty}^{\infty} e^{-x^2}dx \leq \int_{-\infty}^{\infty} \frac{1}{1+x^2}dx = \tan^{-1}(x)\big|_{-\infty}^{\infty} = \pi
            \end{equation*}
            Let $\mathcal{I} = \int_{-\infty}^{\infty} e^{-x^2}dx$. Then:
            \begin{equation*}
                \mathcal{I}^2 = \bigg(\int_{-\infty}^{\infty}e^{-x^2}dx\bigg)\bigg(\int_{-\infty}^{\infty}e^{-y^2}dy\bigg) = \int_{-\infty}^{\infty}\int_{-\infty}^{\infty}e^{-(x^2+y^2)}dxdy
            \end{equation*}
            Switching from Cartesian to Polar coordinates, we have:
            \begin{equation*}
                \int_{0}^{2\pi}\int_{0}^{\infty} re^{-r^2}dr = 2\pi \int_{0}^{\infty}re^{-r^2}dr
            \end{equation*}
            Let $u = r^2$. Then we have $du = 2rdr$, so the integral becomes $\pi \int_{0}^{\infty} e^{-u}du = \pi$. Hence, $\mathcal{I}^2 = \pi$, and therefore $\mathcal{I} = \pm \sqrt{\pi}$. But $\mathcal{I} > 0$, so $\mathcal{I} = \sqrt{\pi}$. Therefore, $\int_{-\infty}^{\infty}e^{-x^2}dx = \sqrt{\pi}$. 
            \end{proof}
            \begin{theorem}
            For all $x\in \mathbb{R}$, $e^{ix} = \cos(x)+i\sin(x)$.
            \end{theorem}
            \begin{proof}
            For $e^{ix}$ is the solution to $y' = iy, y(0) = 1$. But:
            \begin{equation*}
                \frac{d}{dx}\big(\cos(x)+i\sin(x)\big) = -\sin(x)+i\cos(x) = i\big(\cos(x)+i\sin(x)\big)
            \end{equation*}
            Moreover, $\cos(0)+i\sin(0) = 1$. Therefore $\cos(x)+i\sin(x)$ satisfies the same differential equation. From uniqueness of solutions, $e^{ix} = \cos(x)+i\sin(x)$.
            \end{proof}
            \begin{theorem}
            $\int_{-\infty}^{\infty} e^{i\frac{\pi}{2}x^2}dx = 1+i$
            \end{theorem}
            \begin{proof}
            Let $x = i\sqrt{\frac{2}{i\pi}}z$. Then $i\frac{\pi}{2}x^2 = -z^2$, and $dx = i\sqrt{\frac{2}{i\pi}}dz$. So we have:
            \begin{equation*}
                \int_{-\infty}^{\infty}e^{i\frac{\pi}{2}x^2}dx = i\sqrt{\frac{2}{i\pi}}\int_{-\infty}^{\infty} e^{-z^2}dz    
            \end{equation*}
            But $\sqrt{i} = \frac{1+i}{\sqrt{2}}$, and $\frac{1}{\sqrt{i}} = \frac{\sqrt{2}}{1+i} = \frac{1-i}{\sqrt{2}}$, from the previous theorems. Therefore:
            \begin{equation*}
                i\sqrt{\frac{2}{\pi}}\frac{1-i}{\sqrt{2}}\int_{-\infty}^{\infty}e^{-z^2}dz = (1+i)\frac{1}{\sqrt{\pi}}\int_{-\infty}^{\infty}e^{-z^2}dz
            \end{equation*}
            But $\int_{-\infty}^{\infty}e^{-z^2}dz = \sqrt{\pi}$. Thus, we have $1+i$.
            \end{proof}
            \begin{theorem}
            $\mathcal{F}\big(e^{i\frac{\pi}{2}(\frac{\rho}{F})^2}\big) = (1+i)Fe^{-i2\pi F^2 \xi^2}$
            \end{theorem}
            \begin{proof}
            For:
            \begin{equation*}
                \mathcal{F}\big(e^{i\frac{\pi}{2} \big(\frac{\rho}{F}\big)^2}\big) = \int_{-\infty}^{\infty} e^{i\frac{\pi}{2}\big(\frac{\rho}{F}\big)^2}e^{-2\pi i \rho \xi}d\rho = \int_{-\infty}^{\infty} e^{i\frac{\pi}{2}\big(\frac{\rho}{F}\big)^2-2\pi i \rho \xi}d\rho = \int_{-\infty}^{\infty} e^{i\frac{\pi}{2F^2}\big[\rho^2-4F^2\rho \xi\big]}d\rho    
            \end{equation*}
            Completing the square, we get $(\rho - 2F^2 \xi)^2 - 4F^4\xi^2$. So, the integral becomes:
            \begin{equation*}
                \int_{-\infty}^{\infty} e^{i\frac{\pi}{2F^2}(\rho - 2F^2\xi)^2}e^{-2\pi i F^2 \xi^2}d\rho = e^{-2\pi i F^2 \xi^2}\int_{-\infty}^{\infty} e^{i\frac{\pi}{2F^2}(\rho - 2F^2\xi)^2}d\rho
            \end{equation*}
            Let $u = \rho - 2F^2\xi$, so then $du = d\rho$. We obtain:
            \begin{equation*}
                e^{-2\pi i F^2 \xi^2}\int_{-\infty}^{\infty} e^{i\frac{\pi}{2}\big(\frac{u}{F}\big)^2}du
            \end{equation*}
            Letting $s = \frac{u}{F}$, get:
            \begin{equation*}
            Fe^{-2\pi i F^2 \xi^2} \int_{-\infty}^{\infty} e^{i\frac{\pi}{2}s^2}ds
            \end{equation*}
            But this integral is $1+i$. So, we have $(1+i)Fe^{-2\pi i F^2 \xi^2}$.
            \end{proof}
            \begin{theorem}
            $\mathcal{F}(e^{-i\frac{\pi}{2}\big(\frac{\rho_0}{F}\big)^2}\big) = (1-i)Fe^{2\pi i F^2 \xi^2}$.
            \end{theorem}
            \begin{proof}
            For:
            \begin{equation*}
                \int_{-\infty}^{\infty} e^{-i\frac{\pi}{2}\big(\frac{\rho_0}{F}\big)^2}e^{-2\pi i \rho_0 \xi}d\rho_0 = \int_{-\infty}^{\infty} e^{-i\frac{\pi}{2F^2}\big({\rho_0}^2 + 4F^2 \rho_0 \xi\big)}d\rho_0 = \int_{-\infty}^{\infty} e^{-i\frac{\pi}{2F^2}\big((\rho_0+2F^2\xi)^2 - 4F^4\xi^2\big)}d\rho_0
            \end{equation*}
            But $e^{2\pi i F^2 \xi^2}$ is constant with respect to $\rho_0$, so we have:
            \begin{equation*}
                e^{2\pi i F^2 \xi^2} \int_{-\infty}^{\infty} e^{-i\frac{\pi}{2F^2}(\rho_0+2F^2\xi)}d\rho_0    
            \end{equation*}
            Let $u = \frac{\rho_0 + 2F^2 \xi}{F}$, then $Fdu = d\rho_0$, so we have $Fe^{2\pi i F^2 \xi^2} \int_{-\infty}^{\infty} e^{-i\frac{\pi}{2}u^2}du$. Let $u = -is$, then $du = -ids$, and $u^2 = -s^2$. So we have $-i e^{2\pi i F^2 \xi^2}\int_{-\infty}^{\infty} e^{i\frac{\pi}{2}s^2}ds$. But this integral is $1+i$. So, we have $-iFe^{2\pi i F^2 \xi^2}(1+i) = (1-i)Fe^{2\pi i F^2 \xi^2}$.
            \end{proof} 
            
            \begin{theorem}
            If $\int_{-\infty}^{\infty} |f(t)|dt < \infty$, $\int_{-\infty}^{\infty} |g(t)|dt < \infty$, and $f* g = \int_{-\infty}^{\infty} f(\tau)g(\tau-t)d\tau$, then $\mathcal{F}\big(f * g\big) = \mathcal{F}(f)\cdot \mathcal{F}(g)$.
            \end{theorem}
            \begin{proof}
            Let $\int_{-\infty}^{\infty} |f(t)|dt = \norm{f}_{1}$ and $\int_{-\infty}^{\infty} |g(t)|dt = \norm{g}_{1}$. Then:
            \begin{align*}
            \int_{-\infty}^{\infty}\int_{-\infty}^{\infty}|f(\tau)g(\tau-t)|d\tau dt&\leq\int_{-\infty}^{\infty}|f(\tau)|\int_{-\infty}^{\infty}|g(\tau-t)|d\tau dt\\
            &=\int_{\infty}^{\infty}|f(x)|\norm{g}_{1}dx=\norm{f}_{1}\norm{g}_{1}    
            \end{align*}
            Thus, $h(t) = f* g$ is such that $\int_{-\infty}^{\infty} |h(t)|dt < \infty$. Let $H(\xi) = \mathcal{F}(h)$. Then:
            \begin{align*}
            H(\xi)&=\int_{-\infty}^{\infty}h(t)e^{-2\pi it\xi}dt\\
            &=\int_{-\infty}^{\infty}\int_{-\infty}^{\infty}f(\tau)g(t-\tau)d\tau e^{-2\pi i t\xi}dt
            \end{align*}
            But $|e^{-2\pi i t \xi}f(\tau)g(t-\tau)| = |f(\tau)g(t-\tau)|$, and $\int_{-\infty}^{\infty}|f(\tau)g(t-\tau)|d\tau < \infty$. Thus, by Fubini's Theorem we may swap the integral. Let $y = t-\tau$. Then we have:
            \begin{equation*}
            H(\xi)=\int_{-\infty}^{\infty}f(\tau)\int_{-\infty}^{\infty}g(t-\tau)e^{-2\pi it\xi}dtd\tau=\int_{-\infty}^{\infty}f(\tau)e^{-2\pi i\tau\xi}d\tau\int_{-\infty}^{\infty}g(y)e^{-2\pi iy\xi}dy=\mathcal{F}(f)\cdot\mathcal{F}(g)    
            \end{equation*}
            Therefore, etc.
            \end{proof}
            
            \begin{theorem}
            If $\hat{T}(\rho_0) = \frac{1-i}{2F}\int_{-\infty}^{\infty}T(\rho) e^{i\frac{\pi}{2}\big(\frac{\rho-\rho_0}{F}\big)^2}d\rho$, then $T(\rho) = \frac{1+i}{2F}\int_{-\infty}^{\infty}\hat{T}(\rho_0)e^{-i\frac{\pi}{2}\big(\frac{\rho-\rho_0}{F}\big)^2}d\rho_0$.
            \end{theorem}
            \begin{proof}
            For $\hat{T}(\rho) = \frac{1-i}{2F}(T* e^{i \frac{\pi}{2}\big(\frac{\rho}{F}\big)^2}\big)$. Therefore:
            \begin{align*}
            \mathcal{F}(\hat{T})&=\frac{1-i}{2F}\mathcal{F}(T)\cdot \mathcal{F}(e^{i\frac{\pi}{2}\big(\frac{\rho_0}{F}\big)^2})\\
            &=\frac{1-i}{2F}\mathcal{F}(T)\cdot(1+i)Fe^{-2\pi i F^2 \xi^2}\\
            &=\mathcal{F}(T)e^{-2\pi i F^2 \xi^2}
            \end{align*}
            But $e^{2\pi iF^{2}\xi^{2}}=\frac{1}{(1-i)F}\mathcal{F}\big(e^{-i\frac{\pi}{2}\big(\frac{\rho_0}{F}\big)^2}\big) = \frac{1+i}{2F} \mathcal{F}\big(e^{-i\frac{\pi}{2}\big(\frac{\rho_0}{F}\big)^2}$. Therefore:
            \begin{align*}
            \mathcal{F}(T)&=\frac{1+i}{2F}\mathcal{F}(\hat{T})\cdot\mathcal{F}\big(e^{-i\frac{\pi}{2}\big(\frac{\rho_0}{F}\big)^2}\big)\\
            &=\frac{1+i}{2F}\mathcal{F}(\hat{T}*e^{-i\frac{\pi}{2}\big(\frac{\rho_0}{F}\big)^2})\\
            &=\mathcal{F}\bigg(\frac{1+i}{2F}\int_{-\infty}^{\infty} \hat{T}(\rho_0)e^{-i\frac{\pi}{2}\big(\frac{\rho - \rho_0}{F}\big)^2} d\rho_0\bigg)
            \end{align*}
            Therefore, $T(\rho) = \frac{1+i}{2F}\int_{-\infty}^{\infty}\hat{T}(\rho_0)e^{-i\frac{\pi}{2}\big(\frac{\rho - \rho_0}{F}\big)^2}d\rho_0$.
            \end{proof}
            
            \begin{theorem}
            If $T,\psi \in L^2(\mathbb{R})$, and if $\hat{T}(\rho_0) = \int_{-\infty}^{\infty} T(\rho)e^{i\psi(\rho_0-\rho)}d\rho_0$, then $T(\rho) = \mathcal{F}^{-1}_{\rho}\big(\frac{\mathcal{F}(\hat{T})}{\mathcal{F}(e^{i\psi})}\big)$.
            \end{theorem}
            \begin{proof}
            For $\hat{T}(\rho_0) = T* e^{i\psi}$. But then $\mathcal{F}_{\xi}(\hat{T}) = \mathcal{F}_{\xi}\big(T* e^{i\psi}\big) = \mathcal{F}_{\xi}\big(T\big)\cdot \mathcal{F}_{\xi}\big(e^{i\psi}\big)$. So then $\mathcal{F}_{\xi}(T) = \frac{\mathcal{F}_{\xi}(\hat{T})}{\mathcal{F}_{\xi}(e^{i\psi})}$. Therefore $T(\rho) = \mathcal{F}^{-1}_{\rho}\big(\frac{\mathcal{F}_{\xi}(\hat{T})}{\mathcal{F}_{\xi}(e^{i\psi})}\big)$
            \end{proof}
        \subsection{Some Notes on Fresnel Inversion}
            The main equation we wish to study is:
            \begin{equation*}
            \hat{T}(\rho_0) = \frac{1-i}{2F}\int_{-\infty}^{\infty} T(\rho)e^{i\psi(\rho_0,\phi_0,\rho,\phi_s)}d\rho
            \end{equation*}
            We wish to solve for $T(\rho)$. In general, this is not possible. Indeed, uniqueness is not always necessary. Let us suppose that $\psi = \sum_{n=0}^{\infty}a_n(\rho_0 - \rho)^n$. It is still not the case that we may solve this. What we wish to do is use the Convolution theorem, which states the following:
            \begin{theorem}
            If $f,g\in L^{1}(\mathbb{R})$, then $f*g = \mathcal{F}^{-1}\big(\mathcal{F}_{\xi}(f)\cdot \mathcal{F}_{\xi}(g)\big)$
            \end{theorem}
            The requirement that $f,g\in L^{1}(\mathbb{R})$ is necessary. Suppose we have $\int_{-\infty}^{\infty} e^{-\rho^2}e^{2\pi i(\rho_0 - \rho)}d\rho = T*e^{i2\pi \rho}$. $\mathcal{F}_{\rho_0}(e^{-\rho^2})\cdot \mathcal{F}_{\rho_0}(e^{2\pi i \rho}) = \sqrt{\pi}e^{-\pi^2 \rho_0^2}\int_{-\infty}^{\infty}e^{2\pi i(\rho-\rho_0)}d\rho$, and this integral on the right does not converge. We may speak in terms of distributions, or generalized functions (Most commonly, the delta function), but this makes numerical application difficult. Going back to our original problem, we have:
            \begin{align*}
            \nonumber \psi(\rho_0,\phi_0,\rho,\phi_s) &= kD\sqrt{1+\big(\rho^2+\rho_0^2 - 2\rho \rho_0 \cos(\phi - \phi_s)\big)/D^2 + \cos(B)\big(\rho_0\cos(\phi_0) - \rho\cos(\phi)\big)/D} - \\ &\bigg(1+\cos(B)(\rho_0\cos(\phi_0) - \rho\cos(\phi_s)\big)/D\bigg)
            \end{align*}
            Unfortunately, for all values of $\psi$, $\int_{-\infty}^{\infty} |e^{i\psi}|d\rho = \int_{-\infty}^{\infty} 1d\rho = +\infty$. So, we never have the case that $e^{i\psi} \in L^{1}(\mathbb{R})$. However, there are many examples of $\psi$'s such that this theorem still seems to work. Let $\psi = \frac{\pi}{2}\rho^2$, and let $T(\rho) = e^{-\psi}$. Then $T*e^{i\psi} = \int_{-\infty}^{\infty} e^{-\frac{\pi}{2}\rho^2}e^{i\frac{\pi}{2}(\rho_0-\rho)^2}d\rho = \sqrt{1+i}e^{-\frac{1+i}{4}\rho_0^2}$. And $\mathcal{F}_{\xi}\big(\sqrt{1+i}e^{-\frac{1+i}{4}\rho_0^2}\big) = \sqrt{2}(1+i)e^{-(1+i)2\pi \xi^2}$. Now, $\mathcal{F}_{\xi}(e^{-\frac{\pi}{2}\rho^2})\cdot \mathcal{F}_{\xi}(e^{i\frac{\pi}{2}\rho^2}) = \sqrt{2}e^{-2\pi \xi^2}(1+i)e^{-2\pi i \xi^2} = \sqrt{2}(1+i)e^{-(1+i)2\pi \xi^2}$. So we see that, even though $\psi \notin L^{1}(\mathbb{R})$, we still that the result still holds here. 
        \subsection{Window Width as a Function of Resolution}
            We start with the following definition for resolution.
            \begin{definition}
            The resolution of a reconstruction is:
            \begin{equation*}
                \Delta R=\frac{2F^{2}}{W_{eff}}\frac{\frac{b^2}{2}}{e^{-b}+b-1}=\frac{F^{2}}{W_{eff}}\frac{b^2}{e^{-b}+b-1}
            \end{equation*}
            Where $W_{eff}$ and $b$ have the following definitions:
            \begin{align*}
                W_{eff}&=\frac{W}{N_{eq}}&b&=\frac{\sigma^2\omega^2}{2\dot{\rho}}W_{eff}
            \end{align*}
            Here $N_{eq}$ is the normalized equivalent width, $\sigma$ is the Allen Deviation, $\omega$ is the angular frequency, and $\dot{\rho}=\frac{d\rho}{dt}$, where $\rho(t)$ is the ring radius of the ring intercept point. Let $\alpha = \frac{\sigma^2 \omega^2}{2\dot{\rho}}$. Then $b = \alpha W_{eff}$. So:
            \begin{equation*}
            R = \frac{F^2}{W_{eff}}\frac{\alpha^2 W_{eff}^2}{e^{-\alpha W_{eff}}+\alpha W_{eff}-1} = \alpha F^2 \frac{\alpha W_{eff}}{e^{-\alpha W_{eff}}+\alpha W_{eff} - 1}
            \end{equation*}
            Let $f(x) = xe^{x}$. Then for $x\in \mathbb{R}^{+}$, $f^{-1}(x)$ exists and $f^{-1}(x) = L_{W}(x)$, where $L_{W}(x)$ is the Lambert W Function. Using $b$ again, we have:
            \begin{equation*}
                \Delta R = \alpha F^2 \frac{b}{e^{-b}+b-1}\Rightarrow \frac{R}{\alpha F^2} = \frac{b}{e^{-b}+b-1}
            \end{equation*}
            Letting $y = \frac{\Delta R}{\alpha F^2}$, we have:
            \begin{equation*}
                y = \frac{b}{e^{-b}+b-1}
            \end{equation*}
            This is invertable in terms of $L_{W}$:
            \begin{equation*}
                b = L_{W}\bigg(\frac{y}{1-y}e^{\frac{y}{1-y}}\bigg) - \frac{y}{1-y}
            \end{equation*}
            \end{definition}
        \subsection{The Forward Model}
            Given a plane wave of wavelength $\lambda$
            travelling in the $\mathbf{\hat{u}}_{i}$ direction
            incident on a thin plane screen, with plane angle $B$,
            the complex transmittance at the point
            $\boldsymbol{\uprho_{0}}=(\rho_{0},\phi_{0})$ measured
            by an observer at the point $\mathbf{R_{c}}$ can be
            modelled by the following equation:
            \begin{equation}
                \hat{T}(\boldsymbol{\uprho_{0}})
                =\frac{\sin(B)}
                      {i\lambda}
                \int_{0}^{2\pi}\int_{0}^{\infty}
                    T(\boldsymbol{\rho})
                    e^{%
                        ik\mathbf{\hat{u}}_{i}\cdot
                        (\boldsymbol{\uprho}-\mathbf{R_{c}})
                    }
                    \frac{%
                        e^{%
                            ik\norm{%
                                \mathbf{R_{c}}
                                -\boldsymbol{\uprho}
                            }
                        }
                    }{%
                        \norm{%
                            \mathbf{R_{c}}
                            -\boldsymbol{\uprho}
                        }
                    }
                    \rho{d}\rho{d}\phi
            \end{equation}
            Where $\boldsymbol{\uprho}=(\rho,\phi)$ and $k$
            is the wavenumber $k=\frac{2\pi}{\lambda}$. If
            $\mathbf{R_{c}}$ does not lie in the plane of the
            screen, and if the transmittance
            $T(\boldsymbol{\uprho})$ is bounded, then this
            integral converges. If we let
            $\mathbf{D}=\mathbf{R_{c}}-\boldsymbol{\uprho_{0}}$,
            then we can collect all of the exponential terms
            together as:
            \begin{equation}
                \begin{split}
                    \psi(\rho_{0},\phi_{0};\rho,\phi)
                    =kD\Bigg[%
                        \sqrt{1+2\cos(B)
                        \frac{%
                            \rho_{0}\cos(\phi_{0})
                            -\rho\cos(\phi)
                        }{D}
                        +\frac{%
                            \rho^{2}+\rho_{0}^{2}
                            -2\rho\rho_{0}\cos(\phi-\phi_{0})
                        }{D^2}}\\
                        -\bigg(
                            1+\cos(B)
                            \frac{%
                                \rho_{0}\cos(\phi_{0})
                                -\rho\cos(\phi)
                            }{D}
                        \bigg)
                    \Bigg]
                \end{split}
            \end{equation}
            The integral becomes:
            \begin{equation}
                \hat{T}(\boldsymbol{\uprho_{0}})
                =\frac{\sin(B)}{i\lambda}
                 \int_{0}^{2\pi}\int_{0}^{\infty}\rho
                 T(\boldsymbol{\uprho})
                 \frac{%
                    e^{i\psi(\rho_{0},\phi_{0};\rho,\phi)}
                }{%
                    \norm{\mathbf{R_{c}}-\boldsymbol{\uprho}}
                }d\rho{d}\phi    
            \end{equation}
            We impose that $T$ is non-negative
            (There's no such thing as 'negative' transmittance).
            From this we may use a complex version of Funini's
            theorem to obtain:
            \begin{equation}
                \hat{T}(\boldsymbol{\uprho_{0}})
                =\frac{\sin(B)}{i\lambda}\int_{0}^{\infty}\rho T(\boldsymbol{\uprho})\int_{0}^{2\pi}\frac{e^{i\psi(\rho_{0},\phi_{0};\rho,\phi)}}{\norm{\mathbf{R_{c}}-\boldsymbol{\uprho}}}d\phi d\rho
            \end{equation}
                In general, this is as far as one can simplify the problem. For ring systems, which we are studying, we can suppose that $T(\boldsymbol{\uprho})=T(\uprho)$. That is, the screen is radially symmetric. The task is, given the data $\hat{T}(\boldsymbol{\uprho})$, can we determine $T(\uprho)$? Unfortunately, we don't have an entire planar set of data, but rather some curve. The need then arises to try to collapse this problem down to a single integral by some means of approximation. The Stationary Phase Approximation works here.
        \subsection{Stationary Phase Approximation}
            Suppose $g$ is an analytical function about the origin (i.e. it has a convergent MacLaurin series), and consider the integral:
                \begin{equation}
                    I(k) = \int_{a}^{b}e^{ikg(x)}dx
                \end{equation}
                Suppose that there is a $c\in(a,b)$ such that $g'(c)=0$ and $g''(c)\ne 0$. Then:
                \begin{align}
                    \nonumber I(k)&=e^{ikg(c)}\int_{a}^{b}e^{ik[g(x)-g(c)]}dx\\
                    &=e^{ikg(c)}\int_{a}^{b}e^{ik[\frac{g''(c)}{2}(x-c)^{2}+\hdots]}dx
                \end{align}
                Higher terms are extremely oscillatory, and so we neglect them.
                \begin{remark}
                Note that higher terms can indeed cancel each other out, meaning these neglected terms may not be negligible. For example, if $g(x)=-sin(\pi x)$, then $\exp(ig(x))$ is never too oscillatory. However, so long as the interval $[a,b]$ is small enough, the approximation is still valid. The previously mentioned $g(x)$ is how one approximates the $J_{0}(x)$ Bessel function.
                \end{remark}
                Out integral then becomes:
                \begin{align}
                    I(k)&\approx e^{ikg(c)}\int_{a}^{b}e^{ik\frac{g''(c)}{2}(x-c)^{2}}dx\\
                    &\approx e^{ikg(c)}\int_{\infty}^{\infty}e^{ik\frac{g''(c)}{2}(x-c)^{2}}dx\\
                    &=e^{ikg(c)}\sqrt{\frac{2\pi i}{kg''(c)}}
                \end{align}
                We can use this for our double integral, and make it a single integral. The first and second integrals of $\psi$ are nasty, however.
                \begin{equation}
                    \frac{\partial \psi}{\partial \phi}=kD\Big[\frac{2D\rho\cos(B)\sin(\phi)+2\rho\rho_{0}\sin(\phi-\phi_{0})}{2D^2\sqrt{1+2\cos(B)\frac{\rho_{0}\cos(\phi_{0})-\rho\cos(\phi)}{D}+\frac{\rho^{2}+\rho_{0}^{2}-2\rho\rho_{0}\cos(\phi-\phi_{0})}{D^2}}}-\frac{\rho\cos(B)\sin(\phi)}{D}\Big]
                \end{equation}
                The second derivative is equally bad. Solving for $\frac{\partial \psi}{\partial \phi}=0$ must be done iteratively by successive approximations. A further approximation can be made as $\psi$ is analytic in $\phi$. Let $\phi_{s}$ be the solution to $\frac{\partial\psi}{\partial\phi}$ and let $\phi_{s_{n}}$ be a sequence such that $\phi_{s_{n}}\rightarrow \phi_{s}$.
        \subsection{Notes on the Fresnel Approximation}
            \begin{align*}
                \psi
                &=\bigg(
                      1+\frac{%
                          \rho^{2}+\rho_{0}^{2}
                          -2\rho\rho_{0}\cos(\phi-\phi_{0})
                      }{D^{2}}
                      -2\cos(B)\big(
                          \frac{%
                              \rho\cos(\phi)
                              -\rho_{0}\cos(\phi_{0})
                          }{D}
                      \big)
                  \bigg)^{1/2}\\
                &-(1-\cos(B)\big(\frac{\rho\cos(\phi)-\rho_{0}
                  \cos(\phi_{0})}{D})\\
                &=\sqrt{1+\eta-2\xi}-(1-\xi)\\
                  \Rightarrow\frac{\partial\psi}{\partial\phi}
                &=\frac{1}{2\sqrt{1+\eta-2\xi}}
                  \big(
                      \frac{\partial\eta}{\partial\phi}
                      -2\frac{\partial\xi}{\partial\phi}
                  \big)+\frac{\partial\xi}{\partial\phi}\\
                \Rightarrow
                \frac{\partial^{2}\psi}{\partial\phi^{2}}
                &=\frac{-1}{4(1+\eta-2\xi)^{3/2}}
                  \big(
                      \frac{\partial\eta}{\partial\phi}
                      -2\frac{\partial\xi}{\partial\phi}
                  \big)^{2}
                      +\frac{1}{2\sqrt{1+\eta-2\xi}}
                  \big(
                      \frac{\partial^{2}\eta}{\partial\phi^{2}}
                      -2\frac{\partial^{2}\xi}{\partial\phi^{2}}
                  \big)
                +\frac{\partial^{2}\xi}{\partial\phi^{2}}
            \end{align*}
            Now, from the definitions of $\eta$ and $\xi$,
            we have:
            \begin{align*}
                \eta_{\phi=\phi_{0}}
                &=\big(\frac{\rho-\rho_{0}}{D}\big)^{2}
                &
                \xi_{\phi=\phi_{0}}
                &=\cos(B)\cos(\phi_{0})
                  \big(\frac{\rho-\rho_{0}}{D}\big)\\
                \frac{\partial\eta}
                     {\partial\phi}_{\phi=\phi_{0}}
                &=0
                &
                \frac{\partial\xi}
                     {\partial\phi}_{\phi=\phi_{0}}
                &=-\cos(B)\frac{\rho\sin(\phi_{0})}{D}\\
                \frac{\partial^{2}\eta}
                     {\partial\phi^{2}}_{\phi=\phi_{0}}
                &=\frac{2\rho\rho_{0}}{D^{2}}
                &
                \frac{\partial^{2}\xi}
                     {\partial\phi^{2}}_{\phi=\phi_{0}}
                &=-\cos(B)\frac{\rho\cos(\phi_{0})}{D}
            \end{align*}
            Let $\alpha=\cos(B)\cos(\phi_{0})$ and
            $x=(\frac{\rho-\rho_{0}}{D})^{2}$.
            From this, we obtain:
            \begin{align*}
                \psi_{\phi=\phi_{0}}
                &=\sqrt{1+x^{2}-2\alpha{x}}+\alpha{x}-1\\
                \frac{\partial\psi}
                     {\partial\phi}_{\phi=\phi_{0}}
                &=\cos(B)\sin(\phi_{0})\frac{\rho}{D}
                \bigg(
                    \frac{1}{\sqrt{1+x^{2}-2\alpha{x}}}-1
                \bigg)\\
                \frac{\partial^{2}\psi}
                     {\partial\phi^{2}}_{\phi=\phi_{0}}
                &=\frac{%
                      -\rho^{2}\cos^{2}(B)\sin^{2}(\phi_{0})
                  }{%
                    D^{2}(1+x^{2}-2\alpha{x})^{3/2}
                  }
                  +\frac{\rho\rho_{0}}
                        {D^{2}(1+x^{2}-2\alpha{x})^{1/2}}
                  +\frac{\rho\cos(B)\cos(\phi_{0})}
                        {D}
                  \bigg(
                      \frac{1}{\sqrt{1+x^{2}-2\alpha{x}}}-1
                  \bigg)
            \end{align*}
            A nice property emerges here related to
            Legendre polynomials. The following is true:
            \begin{equation*}
                \frac{1}{\sqrt{1-2\alpha{x}+x^{2}}}
                =\sum_{n=0}^{\infty}P_{n}(\alpha)x^{n}
            \end{equation*}
            Where $P_{n}(\alpha)$ is the $n^{th}$
            Legendre polynomial. That is, this is
            the \textit{generating function} for the
            Legendre polynomials. Using this we can
            easily compute the Taylor series expansions
            for $\psi$ and $\psi_{\phi}$ about $x=0$.
            $\psi_{\phi\phi}$ is a nastier type of monster.
            Using this, we obtain the following equations:
            \begin{align*}
                \psi_{\phi=\phi_{0}}
                &=\frac{1-\alpha^{2}}{2}x^{2}
                 +\frac{\alpha(1-\alpha^{2})}{3}x^{3}
                 +\frac{-5\alpha^{4}+6\alpha^{2}-1}{8}x^{4}+
                 \hdots\\
                  \frac{\partial\psi}
                       {\partial\phi}_{\phi=\phi_{0}}
                &=\cos(B)\sin(\phi_{0})\frac{\rho}{D}
                  \bigg(
                      \alpha{x}+\frac{3\alpha^{2}-1}{2}x^{2}
                      +\frac{35\alpha^{4}-30\alpha^{2}+3}{8}x^{4}
                      +\hdots
                  \bigg)
            \end{align*}
            We wish to find the point $\phi$ for which
            $\psi_{\phi}=0$. We can use Newton-Raphson
            with initial guess
            $\phi=\phi_{0}$. Note that, from analyticity:
            \begin{equation*}
                \psi=\sum_{k=0}^{\infty}
                     \psi^{(k)}(\phi=\phi_{s})
                     \frac{(\phi-\phi_{s})^{k}}{k!}
                    =\psi_{s}+\psi'_{s}(\phi-\phi_{s})
                    +\frac{1}{2}\psi''_{s}(\phi-\phi_{s})^{2}
                    +\hdots
            \end{equation*}
            Where all derivatives are taken with respect to
            $\phi$, and $\psi_{s}$ denotes the derivatives
            evaluated at $\phi=\phi_{s}$.
            We may form a sequence of points and
            functions defined by
            the Newton-Raphson method as follows:
            \begin{align*}
                \phi_{n+1}&=\phi_{n}-\psi'_{n}/\psi''_{n}\\
                \psi&\approx
                \psi_{n}+\psi'_{n}(\phi_{n+1}-\phi_{n})
                +\frac{1}{2}\psi''_{n}(\phi_{n+1}-\phi_{n})^{2}\\
                &=\psi_{n}+\psi'(-\psi'_{n}/\psi''_{n})
                 +\frac{1}{2}
                  \psi''_{n}(-\psi'_{n}/\psi''_{n})^{2}\\
                &=\psi_{n}-\frac{1}{2}\psi'^{2}_{n}/\psi''_{n}
            \end{align*}
            Now, let's use the equations we've found before
            to evaulate the first iteration of this
            approximation. First we only consider the
            leading terms of each expansion. The leading term
            of $\partial^{2}\psi/\partial\phi^{2}$ occurs
            when we evaluate at $x=0$. This occurs only
            when $\rho=\rho_{0}$. From this, we have the
            following:
            \begin{align*}
                \psi_{\phi=\phi_{0}}
                &\approx
                    \frac{1}{2}\big(
                        1-\cos^{2}(B)\cos^{2}(\phi_{0})
                    \big)
                    \Big(\frac{\rho-\rho_{0}}{D}\Big)^{2}\\
                \frac{\partial\psi}
                     {\partial\phi}_{\phi=\phi_{0}}
                &\approx
                    \cos^{2}(B)\sin(\phi_{0})\cos(\phi_{0})
                    \frac{\rho}{D}
                    \Big(\frac{\rho-\rho_{0}}{D}\Big)\\
                \frac{\partial^{2}\psi}
                     {\partial\phi^{2}}_{\phi=\phi_{0}}
                &\approx
                    \big(1-\cos^{2}(B)\sin^{2}(\phi_{0})\big)
                    \Big(\frac{\rho_{0}}{D}\Big)^{2}
            \end{align*}
            From this, the first approximation becomes:
            \begin{align*}
                \phi_{1}
                &=\phi_{0}-
                \frac{\cos^{2}(B)\sin(\phi_{0})\cos(\phi_{0})}
                     {1-\cos^{2}(B)\sin^{2}(\phi_{0})}
                \frac{\rho}{\rho_{0}}
                \Big(\frac{\rho-\rho_{0}}{\rho_{0}}\Big)\\
                \psi
                &\approx
                    \frac{1}{2}\big(
                        1-\cos^{2}(B)\cos^{2}(\phi_{0})
                    \big)
                    \Big(\frac{\rho-\rho_{0}}{D}\Big)^{2}
                    -
                    \frac{1}{2}
                    \frac{%
                        \cos^{4}(B)\sin^{2}(\phi_{0})
                        \cos^{2}(\phi_{0})
                    }{
                    1-\cos^{2}(B)\sin^{2}(\phi_{0})}
                    \frac{\rho^{2}}{\rho_{0}^{2}}
                \Big(\frac{\rho-\rho_{0}}{D}\Big)^{2}\\
                &\approx
                \frac{\sin^{2}(B)}
                     {1-\cos^{2}(B)\sin^{2}(\phi_{0}}
                \Big(\frac{\rho-\rho_{0}}{D}\Big)^{2}
            \end{align*}
            The second approximation comes from the fact
            that we have assumed that
            $\rho^{2}/\rho_{0}^{2}\approx{1}$
            in the algebra for this final expression. When
            we apply the weight of $kD$, where $k$ is the
            wavenumber, we get the classic Fresnel
            approximation:
            \begin{align*}
                \psi_{Q}&=\frac{\pi}{2}
                \Big(\frac{\rho-\rho_{0}}{F}\Big)^{2}
                &
                F^{2}&=
                \frac{\lambda{D}}{2}
                    \frac{1-\cos^{2}(B)\sin^{2}(\phi_{0})}
                         {\sin^{2}(B)}
            \end{align*}
            This gives us a quadratic approximation that is
            both very easy to compute and gives decent
            diffraction reconstructions for many 
            occultation observations. In particular,
            this applies very well to the
            Rev007E Cassini occultation. During more
            pathalogical geometries, there is a need to use
            a better approximation and to use
            higher order terms. For ease of analyis, and
            to better take advantage of these Legendre
            polynomials, we will continue to use
            the approximation that
            $\rho^{2}/\rho_{0}^{2}\approx{1}$.
            Since $\rho_{0}$ is usually greater
            than 65,000 (Radius of Saturn) and
            $\rho-\rho_{0}$ is bounded by the window width,
            the worst case scenario one could reasonably
            expect is 4,000 kilometer windows, in which
            case the ratio is bounded by 1.06. For
            10,000 kilometer windows the ratio is bounded
            by 1.14. The appoximation can thus be
            generalized to:
            \begin{equation*}
                \psi\approx
                \sum_{k=0}^{N-1}b_{k}x^{k+2}
                -\frac{1}{2}
                \frac{\cos^{2}(B)\sin^{2}(\phi_{0})}
                     {1-\cos^{2}(B)\sin^{2}(\phi_{0})}
                \Big(\sum_{k=1}^{N}P_{k}(\alpha)x^{n}\Big)^{2}
            \end{equation*}
            Where $b_{k}$ is the $k^{th}$ term of the expansion
            for $\psi$, and $P_{n}$ is the $n^{th}$
            Legendre Polynomial. When $N=1$, we obtain
            the Fresnel approximation we previously
            derived. It would be convenient if we could
            write the $b_{k}$ in terms of the
            $P_{k}(\alpha)$. We can do this by solving the
            following simple differential equation.
            \begin{align*}
                \frac{d}{dx}
                \Big(\sqrt{1+x^{2}-\alpha{x}}\Big)
                &=\frac{x-\alpha}{\sqrt{1+x^{2}-\alpha{x}}}\\
                &=(x-\alpha)
                  \sum_{k=0}^{\infty}P_{k}(\alpha)x^{k}\\
                &=\sum_{k=0}^{\infty}P_{k}(\alpha)x^{k+1}
                 -\alpha\sum_{k=0}^{\infty}P_{k}(\alpha)x^{k}\\
                \Rightarrow
                \sqrt{1+x^{2}-2\alpha{x}}
                &=1+\sum_{k=0}^{\infty}P_{k}(\alpha)
                    \frac{x^{k+2}}{k+2}
                 -\alpha\sum_{k=0}^{\infty}P_{k}(\alpha)
                  \frac{x^{k+1}}{k+1}\\
                \Rightarrow
                \sqrt{1+x^{2}-2\alpha{x}}+\alpha{x}-1
                &=\sum_{k=0}^{\infty}
                    \Big(
                        P_{k}(\alpha)-\alpha{P_{k+1}}(\alpha)
                    \Big)
                    \frac{x^{k+2}}{k+2}
            \end{align*}
            Where we used a shift of index to obtain the last
            equation. This gives us our formula:
            \begin{equation*}
                b_{k}=\frac{P_{k}(\alpha)-\alpha{P}_{k+1}}{k+2}
            \end{equation*}
            We have the following first order approximation
            for $\psi$. Higher order approximations can be
            computed by evaluating
            $\psi_{n}-\psi_{n}'^{2}/\psi_{n}''$ at
            at better approximations obtained by the
            Newton-Raphson method for $\phi$.
            \begin{equation*}
                \psi\approx
                \sum_{k=0}^{N-1}
                \frac{P_{k}(\alpha)-\alpha{P}_{k+1}}{k+2}x^{k+2}
                -\frac{1}{2}
                \frac{\cos^{2}(B)\sin^{2}(\phi_{0})}
                     {1-\cos^{2}(B)\sin^{2}(\phi_{0})}
                \Big(\sum_{k=1}^{N}P_{k}(\alpha)x^{n}\Big)^{2}
            \end{equation*}
\end{document}