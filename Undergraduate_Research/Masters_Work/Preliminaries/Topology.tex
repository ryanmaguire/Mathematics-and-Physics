\documentclass[crop=false,class=article,oneside]{standalone}
%----------------------------Preamble-------------------------------%
%---------------------------Packages----------------------------%
\usepackage{geometry}
\geometry{b5paper, margin=1.0in}
\usepackage[T1]{fontenc}
\usepackage{graphicx, float}            % Graphics/Images.
\usepackage{natbib}                     % For bibliographies.
\bibliographystyle{agsm}                % Bibliography style.
\usepackage[french, english]{babel}     % Language typesetting.
\usepackage[dvipsnames]{xcolor}         % Color names.
\usepackage{listings}                   % Verbatim-Like Tools.
\usepackage{mathtools, esint, mathrsfs} % amsmath and integrals.
\usepackage{amsthm, amsfonts, amssymb}  % Fonts and theorems.
\usepackage{tcolorbox}                  % Frames around theorems.
\usepackage{upgreek}                    % Non-Italic Greek.
\usepackage{fmtcount, etoolbox}         % For the \book{} command.
\usepackage[newparttoc]{titlesec}       % Formatting chapter, etc.
\usepackage{titletoc}                   % Allows \book in toc.
\usepackage[nottoc]{tocbibind}          % Bibliography in toc.
\usepackage[titles]{tocloft}            % ToC formatting.
\usepackage{pgfplots, tikz}             % Drawing/graphing tools.
\usepackage{imakeidx}                   % Used for index.
\usetikzlibrary{
    calc,                   % Calculating right angles and more.
    angles,                 % Drawing angles within triangles.
    arrows.meta,            % Latex and Stealth arrows.
    quotes,                 % Adding labels to angles.
    positioning,            % Relative positioning of nodes.
    decorations.markings,   % Adding arrows in the middle of a line.
    patterns,
    arrows
}                                       % Libraries for tikz.
\pgfplotsset{compat=1.9}                % Version of pgfplots.
\usepackage[font=scriptsize,
            labelformat=simple,
            labelsep=colon]{subcaption} % Subfigure captions.
\usepackage[font={scriptsize},
            hypcap=true,
            labelsep=colon]{caption}    % Figure captions.
\usepackage[pdftex,
            pdfauthor={Ryan Maguire},
            pdftitle={Mathematics and Physics},
            pdfsubject={Mathematics, Physics, Science},
            pdfkeywords={Mathematics, Physics, Computer Science, Biology},
            pdfproducer={LaTeX},
            pdfcreator={pdflatex}]{hyperref}
\hypersetup{
    colorlinks=true,
    linkcolor=blue,
    filecolor=magenta,
    urlcolor=Cerulean,
    citecolor=SkyBlue
}                           % Colors for hyperref.
\usepackage[toc,acronym,nogroupskip,nopostdot]{glossaries}
\usepackage{glossary-mcols}
%------------------------Theorem Styles-------------------------%
\theoremstyle{plain}
\newtheorem{theorem}{Theorem}[section]

% Define theorem style for default spacing and normal font.
\newtheoremstyle{normal}
    {\topsep}               % Amount of space above the theorem.
    {\topsep}               % Amount of space below the theorem.
    {}                      % Font used for body of theorem.
    {}                      % Measure of space to indent.
    {\bfseries}             % Font of the header of the theorem.
    {}                      % Punctuation between head and body.
    {.5em}                  % Space after theorem head.
    {}

% Italic header environment.
\newtheoremstyle{thmit}{\topsep}{\topsep}{}{}{\itshape}{}{0.5em}{}

% Define environments with italic headers.
\theoremstyle{thmit}
\newtheorem*{solution}{Solution}

% Define default environments.
\theoremstyle{normal}
\newtheorem{example}{Example}[section]
\newtheorem{definition}{Definition}[section]
\newtheorem{problem}{Problem}[section]

% Define framed environment.
\tcbuselibrary{most}
\newtcbtheorem[use counter*=theorem]{ftheorem}{Theorem}{%
    before=\par\vspace{2ex},
    boxsep=0.5\topsep,
    after=\par\vspace{2ex},
    colback=green!5,
    colframe=green!35!black,
    fonttitle=\bfseries\upshape%
}{thm}

\newtcbtheorem[auto counter, number within=section]{faxiom}{Axiom}{%
    before=\par\vspace{2ex},
    boxsep=0.5\topsep,
    after=\par\vspace{2ex},
    colback=Apricot!5,
    colframe=Apricot!35!black,
    fonttitle=\bfseries\upshape%
}{ax}

\newtcbtheorem[use counter*=definition]{fdefinition}{Definition}{%
    before=\par\vspace{2ex},
    boxsep=0.5\topsep,
    after=\par\vspace{2ex},
    colback=blue!5!white,
    colframe=blue!75!black,
    fonttitle=\bfseries\upshape%
}{def}

\newtcbtheorem[use counter*=example]{fexample}{Example}{%
    before=\par\vspace{2ex},
    boxsep=0.5\topsep,
    after=\par\vspace{2ex},
    colback=red!5!white,
    colframe=red!75!black,
    fonttitle=\bfseries\upshape%
}{ex}

\newtcbtheorem[auto counter, number within=section]{fnotation}{Notation}{%
    before=\par\vspace{2ex},
    boxsep=0.5\topsep,
    after=\par\vspace{2ex},
    colback=SeaGreen!5!white,
    colframe=SeaGreen!75!black,
    fonttitle=\bfseries\upshape%
}{not}

\newtcbtheorem[use counter*=remark]{fremark}{Remark}{%
    fonttitle=\bfseries\upshape,
    colback=Goldenrod!5!white,
    colframe=Goldenrod!75!black}{ex}

\newenvironment{bproof}{\textit{Proof.}}{\hfill$\square$}
\tcolorboxenvironment{bproof}{%
    blanker,
    breakable,
    left=3mm,
    before skip=5pt,
    after skip=10pt,
    borderline west={0.6mm}{0pt}{green!80!black}
}

\AtEndEnvironment{lexample}{$\hfill\textcolor{red}{\blacksquare}$}
\newtcbtheorem[use counter*=example]{lexample}{Example}{%
    empty,
    title={Example~\theexample},
    boxed title style={%
        empty,
        size=minimal,
        toprule=2pt,
        top=0.5\topsep,
    },
    coltitle=red,
    fonttitle=\bfseries,
    parbox=false,
    boxsep=0pt,
    before=\par\vspace{2ex},
    left=0pt,
    right=0pt,
    top=3ex,
    bottom=1ex,
    before=\par\vspace{2ex},
    after=\par\vspace{2ex},
    breakable,
    pad at break*=0mm,
    vfill before first,
    overlay unbroken={%
        \draw[red, line width=2pt]
            ([yshift=-1.2ex]title.south-|frame.west) to
            ([yshift=-1.2ex]title.south-|frame.east);
        },
    overlay first={%
        \draw[red, line width=2pt]
            ([yshift=-1.2ex]title.south-|frame.west) to
            ([yshift=-1.2ex]title.south-|frame.east);
    },
}{ex}

\AtEndEnvironment{ldefinition}{$\hfill\textcolor{Blue}{\blacksquare}$}
\newtcbtheorem[use counter*=definition]{ldefinition}{Definition}{%
    empty,
    title={Definition~\thedefinition:~{#1}},
    boxed title style={%
        empty,
        size=minimal,
        toprule=2pt,
        top=0.5\topsep,
    },
    coltitle=Blue,
    fonttitle=\bfseries,
    parbox=false,
    boxsep=0pt,
    before=\par\vspace{2ex},
    left=0pt,
    right=0pt,
    top=3ex,
    bottom=0pt,
    before=\par\vspace{2ex},
    after=\par\vspace{1ex},
    breakable,
    pad at break*=0mm,
    vfill before first,
    overlay unbroken={%
        \draw[Blue, line width=2pt]
            ([yshift=-1.2ex]title.south-|frame.west) to
            ([yshift=-1.2ex]title.south-|frame.east);
        },
    overlay first={%
        \draw[Blue, line width=2pt]
            ([yshift=-1.2ex]title.south-|frame.west) to
            ([yshift=-1.2ex]title.south-|frame.east);
    },
}{def}

\AtEndEnvironment{ltheorem}{$\hfill\textcolor{Green}{\blacksquare}$}
\newtcbtheorem[use counter*=theorem]{ltheorem}{Theorem}{%
    empty,
    title={Theorem~\thetheorem:~{#1}},
    boxed title style={%
        empty,
        size=minimal,
        toprule=2pt,
        top=0.5\topsep,
    },
    coltitle=Green,
    fonttitle=\bfseries,
    parbox=false,
    boxsep=0pt,
    before=\par\vspace{2ex},
    left=0pt,
    right=0pt,
    top=3ex,
    bottom=-1.5ex,
    breakable,
    pad at break*=0mm,
    vfill before first,
    overlay unbroken={%
        \draw[Green, line width=2pt]
            ([yshift=-1.2ex]title.south-|frame.west) to
            ([yshift=-1.2ex]title.south-|frame.east);},
    overlay first={%
        \draw[Green, line width=2pt]
            ([yshift=-1.2ex]title.south-|frame.west) to
            ([yshift=-1.2ex]title.south-|frame.east);
    }
}{thm}

%--------------------Declared Math Operators--------------------%
\DeclareMathOperator{\adjoint}{adj}         % Adjoint.
\DeclareMathOperator{\Card}{Card}           % Cardinality.
\DeclareMathOperator{\curl}{curl}           % Curl.
\DeclareMathOperator{\diam}{diam}           % Diameter.
\DeclareMathOperator{\dist}{dist}           % Distance.
\DeclareMathOperator{\Div}{div}             % Divergence.
\DeclareMathOperator{\Erf}{Erf}             % Error Function.
\DeclareMathOperator{\Erfc}{Erfc}           % Complementary Error Function.
\DeclareMathOperator{\Ext}{Ext}             % Exterior.
\DeclareMathOperator{\GCD}{GCD}             % Greatest common denominator.
\DeclareMathOperator{\grad}{grad}           % Gradient
\DeclareMathOperator{\Ima}{Im}              % Image.
\DeclareMathOperator{\Int}{Int}             % Interior.
\DeclareMathOperator{\LC}{LC}               % Leading coefficient.
\DeclareMathOperator{\LCM}{LCM}             % Least common multiple.
\DeclareMathOperator{\LM}{LM}               % Leading monomial.
\DeclareMathOperator{\LT}{LT}               % Leading term.
\DeclareMathOperator{\Mod}{mod}             % Modulus.
\DeclareMathOperator{\Mon}{Mon}             % Monomial.
\DeclareMathOperator{\multideg}{mutlideg}   % Multi-Degree (Graphs).
\DeclareMathOperator{\nul}{nul}             % Null space of operator.
\DeclareMathOperator{\Ord}{Ord}             % Ordinal of ordered set.
\DeclareMathOperator{\Prin}{Prin}           % Principal value.
\DeclareMathOperator{\proj}{proj}           % Projection.
\DeclareMathOperator{\Refl}{Refl}           % Reflection operator.
\DeclareMathOperator{\rk}{rk}               % Rank of operator.
\DeclareMathOperator{\sgn}{sgn}             % Sign of a number.
\DeclareMathOperator{\sinc}{sinc}           % Sinc function.
\DeclareMathOperator{\Span}{Span}           % Span of a set.
\DeclareMathOperator{\Spec}{Spec}           % Spectrum.
\DeclareMathOperator{\supp}{supp}           % Support
\DeclareMathOperator{\Tr}{Tr}               % Trace of matrix.
%--------------------Declared Math Symbols--------------------%
\DeclareMathSymbol{\minus}{\mathbin}{AMSa}{"39} % Unary minus sign.
%------------------------New Commands---------------------------%
\DeclarePairedDelimiter\norm{\lVert}{\rVert}
\DeclarePairedDelimiter\ceil{\lceil}{\rceil}
\DeclarePairedDelimiter\floor{\lfloor}{\rfloor}
\newcommand*\diff{\mathop{}\!\mathrm{d}}
\newcommand*\Diff[1]{\mathop{}\!\mathrm{d^#1}}
\renewcommand*{\glstextformat}[1]{\textcolor{RoyalBlue}{#1}}
\renewcommand{\glsnamefont}[1]{\textbf{#1}}
\renewcommand\labelitemii{$\circ$}
\renewcommand\thesubfigure{%
    \arabic{chapter}.\arabic{figure}.\arabic{subfigure}}
\addto\captionsenglish{\renewcommand{\figurename}{Fig.}}
\numberwithin{equation}{section}

\renewcommand{\vector}[1]{\boldsymbol{\mathrm{#1}}}

\newcommand{\uvector}[1]{\boldsymbol{\hat{\mathrm{#1}}}}
\newcommand{\topspace}[2][]{(#2,\tau_{#1})}
\newcommand{\measurespace}[2][]{(#2,\varSigma_{#1},\mu_{#1})}
\newcommand{\measurablespace}[2][]{(#2,\varSigma_{#1})}
\newcommand{\manifold}[2][]{(#2,\tau_{#1},\mathcal{A}_{#1})}
\newcommand{\tanspace}[2]{T_{#1}{#2}}
\newcommand{\cotanspace}[2]{T_{#1}^{*}{#2}}
\newcommand{\Ckspace}[3][\mathbb{R}]{C^{#2}(#3,#1)}
\newcommand{\funcspace}[2][\mathbb{R}]{\mathcal{F}(#2,#1)}
\newcommand{\smoothvecf}[1]{\mathfrak{X}(#1)}
\newcommand{\smoothonef}[1]{\mathfrak{X}^{*}(#1)}
\newcommand{\bracket}[2]{[#1,#2]}

%------------------------Book Command---------------------------%
\makeatletter
\renewcommand\@pnumwidth{1cm}
\newcounter{book}
\renewcommand\thebook{\@Roman\c@book}
\newcommand\book{%
    \if@openright
        \cleardoublepage
    \else
        \clearpage
    \fi
    \thispagestyle{plain}%
    \if@twocolumn
        \onecolumn
        \@tempswatrue
    \else
        \@tempswafalse
    \fi
    \null\vfil
    \secdef\@book\@sbook
}
\def\@book[#1]#2{%
    \refstepcounter{book}
    \addcontentsline{toc}{book}{\bookname\ \thebook:\hspace{1em}#1}
    \markboth{}{}
    {\centering
     \interlinepenalty\@M
     \normalfont
     \huge\bfseries\bookname\nobreakspace\thebook
     \par
     \vskip 20\p@
     \Huge\bfseries#2\par}%
    \@endbook}
\def\@sbook#1{%
    {\centering
     \interlinepenalty \@M
     \normalfont
     \Huge\bfseries#1\par}%
    \@endbook}
\def\@endbook{
    \vfil\newpage
        \if@twoside
            \if@openright
                \null
                \thispagestyle{empty}%
                \newpage
            \fi
        \fi
        \if@tempswa
            \twocolumn
        \fi
}
\newcommand*\l@book[2]{%
    \ifnum\c@tocdepth >-3\relax
        \addpenalty{-\@highpenalty}%
        \addvspace{2.25em\@plus\p@}%
        \setlength\@tempdima{3em}%
        \begingroup
            \parindent\z@\rightskip\@pnumwidth
            \parfillskip -\@pnumwidth
            {
                \leavevmode
                \Large\bfseries#1\hfill\hb@xt@\@pnumwidth{\hss#2}
            }
            \par
            \nobreak
            \global\@nobreaktrue
            \everypar{\global\@nobreakfalse\everypar{}}%
        \endgroup
    \fi}
\newcommand\bookname{Book}
\renewcommand{\thebook}{\texorpdfstring{\Numberstring{book}}{book}}
\providecommand*{\toclevel@book}{-2}
\makeatother
\titleformat{\part}[display]
    {\Large\bfseries}
    {\partname\nobreakspace\thepart}
    {0mm}
    {\Huge\bfseries}
\titlecontents{part}[0pt]
    {\large\bfseries}
    {\partname\ \thecontentslabel: \quad}
    {}
    {\hfill\contentspage}
\titlecontents{chapter}[0pt]
    {\bfseries}
    {\chaptername\ \thecontentslabel:\quad}
    {}
    {\hfill\contentspage}
\newglossarystyle{longpara}{%
    \setglossarystyle{long}%
    \renewenvironment{theglossary}{%
        \begin{longtable}[l]{{p{0.25\hsize}p{0.65\hsize}}}
    }{\end{longtable}}%
    \renewcommand{\glossentry}[2]{%
        \glstarget{##1}{\glossentryname{##1}}%
        &\glossentrydesc{##1}{~##2.}
        \tabularnewline%
        \tabularnewline
    }%
}
\newglossary[not-glg]{notation}{not-gls}{not-glo}{Notation}
\newcommand*{\newnotation}[4][]{%
    \newglossaryentry{#2}{type=notation, name={\textbf{#3}, },
                          text={#4}, description={#4},#1}%
}
%--------------------------LENGTHS------------------------------%
% Spacings for the Table of Contents.
\addtolength{\cftsecnumwidth}{1ex}
\addtolength{\cftsubsecindent}{1ex}
\addtolength{\cftsubsecnumwidth}{1ex}
\addtolength{\cftfignumwidth}{1ex}
\addtolength{\cfttabnumwidth}{1ex}

% Indent and paragraph spacing.
\setlength{\parindent}{0em}
\setlength{\parskip}{0em}
%--------------------------Main Document----------------------------%
\begin{document}
    \ifx\ifworkmasterswork\undefined
        \section*{Preliminaries}
        \setcounter{section}{1}
    \fi
    \subsection{Topology}
        \begin{definition}
        If $X$ is a set, then a topology $\tau$ on $X$ is a subset of $\mathcal{P}(X)$ with the following properties:
        \begin{enumerate}
        \item $\emptyset, X\in \tau$.
        \item If $A$ is some indexing set (Finite, countable, or uncountable) and $\mathcal{U}_\alpha \in \tau$ for all $\alpha \in A$, then $\cup_{\alpha \in A} \mathcal{U}_{\alpha} \in \tau$.
        \item If $\mathcal{U}_k\in \tau$, $1\leq k \leq n\in \mathbb{N}$, then $\cap_{k=1}^{n}\mathcal{U}_k \in \tau$.
        \end{enumerate}
        \end{definition}
        \begin{definition}
        A set $X$ with topology $\tau$ is called a topological space and is denoted $(X,\tau)$.
        \end{definition}
        \begin{definition}
        If $(X,\tau)$ is a topological space and $\mathcal{U}\in \tau$, then $\mathcal{U}$ is said to be an open subset of $X$.
        \end{definition}
        \begin{definition}
        If $(X,\tau)$ is a topological space, then a subset $S\subset X$ is said to be closed if and only if $S^c$ is open.
        \end{definition}
        \begin{theorem}
        In a topological space $(X,\tau)$, $X$ and $\emptyset$ are closed.
        \end{theorem}
        \begin{proof}
        For $X^c = \emptyset$, and thus $X$ is closed. $\emptyset^c=X$, and thus $\emptyset$ is closed.
        \end{proof}
        \begin{theorem}
        If $(X,\tau)$ is a topological space, $S\subset X$ is open if and only if $S^c$ is closed.
        \end{theorem}
        \begin{proof}
        If $S$ is open, then $(S^c)^c$ is open, and thus $S^c$ is closed. If $S^c$ is closed, then $(S^c)^c$ is open, and thus $S$ is open.
        \end{proof}
        \begin{definition}
        If $(X,\tau)$ is a topological space and $S\subset X$, then relative topology on $S$ is $\mathscr{T}=\{\mathcal{U}\cap S:\mathcal{U}\in \tau\}$.
        \end{definition}
        \begin{theorem}
        The relative topology is a topology.
        \end{theorem}
        \begin{proof}
        In order,
        \begin{enumerate}
        \item As $X\in \tau$, $S=S\cap X \in \mathscr{T}$. As $\emptyset\in \tau$, $\emptyset\cap S = \emptyset \in \mathscr{T}$. 
        \item For $\mathscr{U}_\alpha\in \mathscr{\tau}$, there is a $\mathcal{U}_\alpha$ such that $\mathscr{U}_\alpha = S\cap \mathcal{U}_\alpha$. Thus, $\cup_{\alpha \in A} \mathscr{U}_\alpha = \cup_{\alpha \in A}(S\cap \mathcal{U}_\alpha) = S\cap (\cup_{\alpha \in A}\mathscr{U}_\alpha)\in \mathscr{T}$
        \item Again, $\cap_{k=1}^{n} \mathscr{U}_k = \cap_{k=1}^{n}(S\cap \mathcal{U}_k) = S\cap (\cap_{k=1}^{n} \mathcal{U}_k)\in \mathscr{T}$.
        \end{enumerate}
        \end{proof}
        \begin{definition}
        If $(X,\tau)$ is a topological space and $S\subset X$, then a set $\mathcal{U}\subset S$ is said to be open in $S$ if and only if $\mathcal{U}\in \mathscr{T}$, where $\mathscr{T}$ is the relative topology on $S$.
        \end{definition}
        \begin{definition}
        If $(X,\tau)$ and $(Y,\tau')$ are topological spaces, then a function $f:X\rightarrow Y$ is said to be continuous if and only if for every open set $V\in Y$, $f^{-1}(V)$ is open in $X$ (With respect to the topologies).
        \end{definition}
        \begin{theorem}
        If $(X,\tau)$, $(Y,\tau')$, and $(Z,\tau'')$ are topological spaces, and if $f:X\rightarrow Y$, $g:Y\rightarrow Z$ are continuous, then $g\circ f:X\rightarrow Z$ is continuous.
        \end{theorem}
        \begin{proof}
        For let $V\in Z$ be open. Then $g^{-1}(V)$ is open in $Y$, as $g$ is continuous. But then $f^{-1}(g^{-1}(V))$ is open in $X$, as $f$ is continuous. Thus $g\circ f$ is continuous.
        \end{proof}
        \begin{definition}
        A sequence $x_n$ in a topological space $(X,\tau)$ is said to converge to a point $x\in X$ if and only if for all $\mathcal{U}\in \tau$ such that $x\in \mathcal{U}$, there is an $N\in \mathbb{N}$ such that for all $n>N$, $x_n \in \mathcal{U}$.
        \end{definition}
        \begin{theorem}
        There exist topological spaces with convergent sequences that do not have unique limits.
        \end{theorem}
        \begin{proof}
        For let $X = \{1,2,3\}$, and let $\tau = \{\emptyset, \{1,2\},\{1,2,3\}\}$. We see that $\emptyset,X\in \tau$, unions and intersections are in $\tau,$ and thus $\tau$ is a topology. Let $x_n = \begin{cases} 1, & n\ odd \\ 2, & n\ even\end{cases}$. Then $x_n \rightarrow 1$ and $x_n \rightarrow 2$. To see this, let $\mathcal{U}$ be an open set such that $1\in \mathcal{U}$. Our choices are $\{1,2\}$ and $\{1,2,3\}$. Then for all $n\in \mathbb{N}$, $x_n \in \mathcal{U}$, and thus $x_n \rightarrow 1$. Similarly, $x_n \rightarrow 2$. Convergence is not necessarily unique in topological spaces.
        \end{proof}
        \begin{definition}
        A topological space $(X,d)$ is said to be a Fr\'{e}chet Space, or $T_1$, if and only if for all distinct $x,y\in X$ there is an open set $\mathcal{U}$ such that $x\in \mathcal{U}$ and $y\notin \mathcal{U}$.
        \end{definition}
        \begin{theorem}
        If $(X,\tau)$ is $T_1$, then for all $x\in X$, $\{x\}$ is closed.
        \end{theorem}
        \begin{proof}
        $[x\in X]\Rightarrow [\forall y\ne x, \exists \mathcal{U}_y\in \tau:y\in \mathcal{U},x\notin\mathcal{U}]\Rightarrow [\cup_{y\ne x}\mathcal{U}_y\in \tau]\Rightarrow [\{x\}=(\cup_{y\ne x}\mathcal{U}_y)^c\ is\ closed]$.
        \end{proof}
        \begin{definition}
        A topological space $(X,d)$ is said to be a Hausdorff Space, or $T_2$, if and only if for all distinct $x,y\in X$, there are disjoint open sets $\mathcal{U}$ and $\mathcal{V}$ such that $x\in \mathcal{U}$ and $y\in \mathcal{V}$.
        \end{definition}
        \begin{theorem}
        A $T_2$ space is a $T_1$ space.
        \end{theorem}
        \begin{proof}
        $[x,y\in X]\land [x\ne y]\Rightarrow [\exists \mathcal{U},\mathcal{V}\in \tau:\mathcal{U}\cap \mathcal{V}=\emptyset\land x\in \mathcal{U},y\in \mathcal{V}]\Rightarrow [\exists \mathcal{U}\in \tau:x\in \mathcal{U}\land y\notin \mathcal{U}]$
        \end{proof}
        \begin{theorem}
        Convergence in a Hausdorff Space $(X,\tau)$ is unique.
        \end{theorem}
        \begin{proof}
        $[x_n \rightarrow x\in X]\land [x_n \rightarrow y\in X]\land[x\ne y]\Rightarrow [\exists \mathcal{U},\mathcal{V}:\mathcal{U}\cap \mathcal{V}=\emptyset\land x\in \mathcal{U}\land y\in \mathcal{V}]$. $[x_n\rightarrow x]\Rightarrow [\exists N_1\in \mathbb{N}:n>N_1\Rightarrow x_n \in \mathbb{N}]$. $[x_n\rightarrow y]\Rightarrow [N_2\in \mathbb{N}:n>N\Rightarrow x_n \in \mathcal{V}]$. $[n>\max\{N_1,N_2\}]\Rightarrow [x_n \in \mathcal{U}\cap \mathcal{V}]$, a contradiction. Therefore, etc.
        \end{proof}
        \begin{definition}
        A topological space $(X,\tau)$ is said to be regular if for each closed subset $E\subset X$ and for each point $x\in E^c$, there exist disjoint open sets $\mathcal{U}$ and $\mathcal{V}$ such that $x\in \mathcal{U}$ and $E\subset \mathcal{V}$.
        \end{definition} 
        \begin{definition}
        In a topological space $(X,\tau)$, a point $p$ is said to have a neighborhood $S\subset X$ if and only if there is a set $\mathcal{U}\subset S$ such that $\mathcal{U}\in \tau$ and $p\in \mathcal{U}$.
        \end{definition}
        \begin{definition}
        A $T_3$ space is a regular $T_1$ space.
        \end{definition}
        \begin{theorem}
        A $T_3$ space $(X,\tau)$ is a $T_2$ space.
        \end{theorem}
        \begin{proof}
        Let $x,y\in X$ be distinct. As a $T_3$ space is $T_1$, $\{x\}$ is closed. Thus $\exists \mathcal{U},\mathcal{V}\in\tau: \mathcal{U}\cap\mathcal{V}=\emptyset, \{x\}\subset \mathcal{U}$, and $y\in \mathcal{V}$.
        \end{proof}
        \begin{definition}
        A topological space $(X,\tau)$ is said to be normal if and only if for all disjoint closed subsets $E,F\subset X$, there are disjoint open sets $\mathcal{U}$ and $\mathcal{V}$ such that $E\subset \mathcal{U}$ and $F\subset \mathcal{V}$.
        \end{definition}
        \begin{definition}
        A $T_4$ space is a normal $T_1$ space.
        \end{definition}
        \begin{theorem}
        A $T_4$ space $(X,\tau)$ is a $T_3$ space.
        \end{theorem}
        \begin{proof}
        A $T_4$ space is $T_1$. If $E\underset{Closed}\subset X$ and $x\in E^c$, then $\{x\}$ is closed. Thus $\exists \mathcal{U},\mathcal{V}\in\tau: \mathcal{U}\cap\mathcal{V}=\emptyset, \{x\}\subset \mathcal{U}$, and $E\subset \mathcal{V}$.
        \end{proof}
        \begin{definition}
        A homeomorphism between two topological spaces $(X,\tau)$ and $(Y,\tau)$ is a continuous bijection $f:X\rightarrow Y$ such that $f^{-1}:Y\rightarrow X$ is continuous.
        \end{definition}
        \begin{definition}
        If $(X,\tau)$ is a topological space, and $S\subset X$, then an open cover $\mathcal{O}$ of $S$ is a set of open sets $\mathcal{U}_{\alpha}$ such that $S\subset \cup_{\alpha\in A} \mathcal{U}_{\alpha}$, where $A$ is some index set.
        \end{definition}
        \begin{definition}
        A subcover of an open cover $\mathcal{O}$ is a subset of $\mathcal{O}$ that is also a cover.
        \end{definition}
        \begin{definition}
        If $(X,\tau)$ is a topological space and $S\subset X$, then $S$ is said to be compact if and only if every open cover of $S$ has a finite subcover.
        \end{definition}
        \begin{theorem}
        If $S$ is a compact subset of a Hausdorff space, then for all $x\in S^c$ there are disjoint open sets $\mathcal{U}$ and $\mathcal{V}$ such that $x\in \mathcal{U}$ and $S\subset \mathcal{V}$.
        \end{theorem}
        \begin{proof}
        For let $x\in S^c$. For all $y\in S$ there are disjoint open sets $\mathcal{U}_y$ and $\mathcal{V}_y$ such that $x\in \mathcal{U}$ and $y\in \mathcal{V}$. But then $\cup_{y\in S} \mathcal{U}_y$ is an open cover of $S$. As $S$ is compact, there is a finite subcover, that is sets $\mathcal{V}_{y_1},\hdots, \mathcal{V}_{y_n}$ that cover $S$. But then $\cap_{k=1}^{n} \mathcal{U}_{y_k}$ is open, contains $x$ and is disjoint from $\cup_{k=1}^{n} \mathcal{V}_{y_k}$. Therefore, etc.
        \end{proof}
        \begin{theorem}
        Every compact subset of a Hausdorff space $(X,\tau)$ is closed.
        \end{theorem}
        \begin{proof}
        Let $S$ be a compact subset of a X. $\forall x\in S^c, \exists \mathcal{U}_x\in \tau:\mathcal{U}_x\cap S = \emptyset:x\in \mathcal{U}_x$. But then $S^c \subset \underset{x\in S^c}\cup\mathcal{U}_x$. But also $S\cap (\cup_{x\in S^c}\mathcal{U}_x) = \emptyset$. Thus $S^c = \cup_{x\in S^c}\mathcal{U}_x$, and therefore $S^c$ is open. Thus $S$ is closed.
        \end{proof}
        \begin{theorem}
        If $S$ is a closed subset of a compact space $(X,\tau)$, $S$ is compact.
        \end{theorem}
        \begin{proof}
        For let $\mathcal{O}$ be an open cover of $S$. As $S$ is closed, $S^c$ is open, and thus $\{S^c\} \cup \mathcal{O}$ is an open cover $X$. As $X$ is compact, there is a finite subcover, call it $\mathscr{O}$. But then $\mathscr{O}\setminus \{S^c\}$ is a finite subcover $\mathcal{O}$ that covers $S$. Thus, etc.
        \end{proof}
        \begin{theorem}
        If $f:X\rightarrow Y$ is continuous and $X$ is compact, then $f(X)\subset Y$ is compact.
        \end{theorem}
        \begin{proof}
        Let $\mathcal{O}$ be an open cover of $f(X)$. As $f$ is continuous, $\mathcal{U}\in\mathcal{O}\Rightarrow f^{-1}(\mathcal{U})$ is open in $X$. Thus $\cup_{\mathcal{U}\in \mathcal{O}} f^{-1}(\mathcal{U})$ is an open cover of $X$. As $X$ is compact, there is a finite subcover, say $\mathscr{O}$. But then $\cup_{\mathcal{V}\in \mathscr{O}} \mathcal{V}$ is a finite subcover of $\mathcal{O}$. Therefore, etc.
        \end{proof}
        \begin{theorem}
        If $f:X\rightarrow Y$ is a continuous bijection, $X$ is compact and $Y$ is Hausdorff, then $f$ is a homeomorphism.
        \end{theorem}
        \begin{proof}
        If suffices to show that if $\mathcal{U}$ is open in $X$, then $f(\mathcal{U})$ is open in $f(X)$. Let $\mathcal{U}$ be open in $X$. As $X$ is compact and $\mathcal{U}$ is open, $\mathcal{U}^c$ is compact. But then $f(\mathcal{U}^c) = f(X)\setminus f(\mathcal{U})$ is compact. Thus $f(X)\setminus f(\mathcal{U})\underset{Closed}\subset f(X)\Rightarrow f(\mathcal{U})\underset{Open}\subset f(X)$.
        \end{proof}
        \begin{definition}
        A topological space $(X,\tau)$ is said to be disconnected if and only if there are two disjoint nonempty open sets $\mathcal{U}$ and $\mathcal{V}$ such that $X = \mathcal{U}\cup \mathcal{V}$.
        \end{definition}
        \begin{theorem}
        A topological space $(X,\tau)$ is disconnected if and only if there are two non-empty disjoint closed set $\mathcal{C}$ and $\mathcal{D}$ such that $X=\mathcal{C}\cup\mathcal{D}$.
        \end{theorem}
        \begin{proof}
        $\big[\exists \mathcal{U},\mathcal{V}\in \tau: [\mathcal{U}\cap \mathcal{V}=\emptyset]\land [X=\mathcal{U}\cup \mathcal{V}]\land [\mathcal{U},\mathcal{V}\ne \emptyset]\big]\Rightarrow [X = \mathcal{U}^c\cup \mathcal{V}^c]$ thus, $X$ is the union of disjoint, non-empty closed set. $[\mathcal{C}^c,\mathcal{D}^c\in \tau]\land[\mathcal{C},\mathcal{V}\ne\emptyset]\land[\mathcal{C}\cap \mathcal{D}=\emptyset]\land[\mathcal{C}\cup\mathcal{D}=X]\Rightarrow [X=\mathcal{C}^c\cup\mathcal{D}^c].$ Thus $X$ is disconnected.
        \end{proof}
        \begin{theorem}
        $(X,\tau)$ is disconnected if and only if there is a proper, nonempty set $A\subset X$ that is both open and closed.
        \end{theorem}
        \begin{proof}
        $\big[\exists \mathcal{U},\mathcal{V}\in \tau:[\mathcal{U}\cap \mathcal{V}=\emptyset]\land [X=\mathcal{U}\cup\mathcal{V}]\land[\mathcal{U},\mathcal{V}\ne \emptyset]\big]\Rightarrow [\mathcal{U}^c = \mathcal{V}]\Rightarrow [\mathcal{U}^c\in \tau]$. Thus, $\mathcal{U}$ is open and closed.
        \end{proof}
        \begin{definition}
        A topological space is called connected if and only if it is not disconnected.
        \end{definition}
        \begin{theorem}
        If $f:X\rightarrow Y$ is a continuous function and $X$ is connected, then $f(X)$ is connected.
        \end{theorem}
        \begin{proof}
        For let $f$ be continuous and $X$ be connected. Suppose $f(X)$ is disconnected. Then there are two nonempty open disjoint sets $\mathcal{U}$ and $\mathcal{V}$ such that $f(X) = \mathcal{U}\cap \mathcal{V}$. But then their preimage is open, and thus $X=f^{-1}(\mathcal{U})\cup f^{-1}(\mathcal{V})$, and thus $X$ is disconnected, a contradiction. Thus $f(X)$ is connected.
        \end{proof}
        \begin{definition}
        If $(X,\tau)$ and $(Y,\tau')$ are topological spaces, then the product topology on the set $X\times Y$ is the set $\mathscr{T} = \{\mathcal{U}\times \mathcal{V}:\mathcal{U}\in\tau,\mathcal{V}\in \tau'\}$.
        \end{definition}
        \begin{theorem}
        The product topology is a topology.
        \end{theorem}
        \begin{proof}
        \
        \begin{enumerate}
        \item As $\emptyset \in \tau$ and $\emptyset\in \tau'$, $\emptyset =\emptyset\times \emptyset \in \mathscr{T}$.
        \item If $\mathscr{U}_{\alpha}\in \mathscr{T}$, then $\cup_{\alpha} \mathscr{U}_{\alpha} = \cup_{\alpha} (\mathcal{U}_{\alpha},\mathcal{V}_{\alpha})$. As $\tau$ and $\tau'$ are topologies, $\cup_{\alpha} \mathcal{U} \in \tau$ and $\cup_{\alpha}\mathcal{V}_{\alpha} \in \tau'$. Thus, $\cup_{\alpha}\mathscr{U}_{\alpha} \in \mathscr{T}$.
        \item $\cap_{k=1}^{n} \mathscr{U}_{k} = \cap_{k=1}^{n} (\mathcal{U}_k,\mathcal{V}_k)$. As $\tau$ and $\tau'$ are topologies, $\cap_{k=1}^{n}\mathcal{U}_k \in \tau$ and $\cap_{k=1}^{n}\mathcal{V}_{k} \in \tau'$. Thus $\cap_{k=1}^{n} \mathscr{U}_k \in \mathscr{T}$
        \end{enumerate}
        \end{proof}
        \begin{definition}
        The projection map $\pi_1$ is defined as $\pi_1:X_1\times X_2\rightarrow X_1$ by $(x_1,x_2)\mapsto x_1$. Similarly for $\pi_2$.
        \end{definition}
        \begin{theorem}
        The projection map is continuous.
        \end{theorem}
        \begin{proof}
        Let $\pi_1:X_1\times X_2\rightarrow X_1$ be the projection map, $X\times Y$ having the project topology. Let $\mathcal{U}\underset{Open}\subset X_1$. Then $f^{-1}(\mathcal{U}) = \{(x_1,x_2):x_1\in \mathcal{U}, x_2\in X_2\}$. But $\mathcal{U}$ and $X_2$ are open, and thus $f^{-1}(\mathcal{U})$ is open (In the product topology).
        \end{proof}
        \begin{definition}
        An open mapping is a function $f:X\rightarrow Y$ such that $\mathcal{U}\underset{Open}\subset X\Rightarrow f(\mathcal{U}) \underset{Open}\subset Y$.
        \end{definition}
        \begin{theorem}
        The projection map is an open mapping.
        \end{theorem}
        \begin{proof}
        For let $\mathscr{U}$ be an open set in $X\times Y$ (With the product topology). That is, there are open sets $\mathcal{U}\subset X$ and $\mathcal{V}\subset Y$ such that $\mathscr{U}= \{(x,y):x\in \mathcal{U},y\in \mathcal{V}\}$, Then $\pi_1(\mathscr{U}) =\mathcal{U}$, which is open. Therefore, etc.
        \end{proof}
        \begin{theorem}
        If $X$ and $Y$ are compact, then $X\times Y$ is compact with the product topology.
        \end{theorem}
        \begin{proof}
        For let $\mathscr{O}$ be an open cover of $X\times Y$. Then $\{\pi_X(\mathscr{U}):\mathscr{U}\in \mathscr{O}\}$ is an open cover of $X$ and $\{\pi_{Y}(\mathscr{V}):\mathscr{V}\in \mathscr{O}\}$ is an open cover of $Y$. As $X$ and $Y$ are compact, there exist finite subcovers of each, say $\mathcal{O}_X$ and $\mathcal{O}_Y$. But then $\{\pi_{X}^{-1}(\mathcal{U}):\mathcal{U}\in \mathcal{O}_X\}\cup \{\pi_{Y}^{-1}(\mathcal{V}):\mathcal{V}\in \mathcal{O}_Y\}$ is a finite subcover of $\mathscr{O}$. Thus, $X\times Y$ is compact.
        \end{proof}
        \begin{theorem}
        If $X,Y\subset Z$ are compact, $X\cup Y$ is compact.
        \end{theorem}
        \begin{proof}
        Let $\mathcal{O}$ be an open cover of $X\cup Y$. Then there is a finite subcover of $X$ and a finite subcover of $Y$, and thus the combination of these subcovers is a cover of $X\cup Y$.
        \end{proof}
\end{document}