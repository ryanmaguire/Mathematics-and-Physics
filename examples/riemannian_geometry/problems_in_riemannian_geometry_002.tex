%Questions for the Discussion:
\documentclass{beamer}
\usepackage{amsmath}

\title{Problems in Riemannian Geometry}
\author{Ryan Maguire}
\date{June 2022}
\usenavigationsymbolstemplate{}
\setbeamertemplate{footline}[frame number]
\begin{document}
    \maketitle
    \begin{frame}
        What is the definition of the exponential map and of the geodesic flow?
        \begin{itemize}
            \item Given a Riemannian manifold $(M,\mathcal{A},g)$ and an affine
                connection $\nabla$, a geodesic is a curve
                $\gamma:(a,b)\rightarrow{M}$ such that
                $\nabla_{\dot{\gamma}}\dot{\gamma}=0$.
            \item Given a point $p\in{M}$ and a tangent vector
                $v\in{T}_{p}M$ there is a unique geodesic $\gamma$ such that
                $\gamma(0)=p$ and $\dot{\gamma}(0)=v$.
            \item The exponential of $M$ at the point $p$ is the function
                $\exp_{p}:T_{p}M\rightarrow{M}$ defined by
                $\exp_{p}(v)=\gamma_{v}(1)$, where $\gamma_{v}$ is the unique
                geodesic with $\gamma_{v}(0)=p$ and $\dot{\gamma}_{v}(0)=v$.
            \item Geodesic flow is defined as a function
                $f:TM\times\mathbb{R}\rightarrow{TM}$. Given $(p,v)\in{TM}$
                and $t\in\mathbb{R}$, $f((p,v),t)=\gamma_{(p,v)}(t)$ where
                $\gamma_{(p,v)}$ is the unique curve with
                $\gamma_{(p,v)}(0)=p$ and $\dot{\gamma}_{(p,v)}(0)=v$.
        \end{itemize}
    \end{frame}
    \begin{frame}
        What is a convex neighborhood on a Riemann manifold? Give one example.
        \begin{itemize}
            \item A convex neighborhood in a Riemann manifold is a subset
                $A\subseteq{M}$ such that for all $p,q\in{A}$ there is a
                unique minimizing geodesic $\gamma$ between $a$ and $b$.
            \item The open unit ball in $\mathbb{R}^{n}$ centered about the
                origin.
            \item The open northern hemisphere on the sphere $\mathbb{S}^{2}$.
        \end{itemize}
    \end{frame}
    \begin{frame}
        Prove the exponential map is an immersion close to the origin.
        \begin{itemize}
            \item The differential pushforward of the exponential map is given
                by:
                \begin{align}
                    \big(\textrm{d}\,\exp_{p}(\mathbf{0})\big)(v)
                        &=\frac{\textrm{d}}{\textrm{d}\,t}
                            \Big(\exp_{p}(tv)\Big)\Big|_{t=0}\\
                        &=\frac{\textrm{d}}{\textrm{d}\,t}
                            \Big(\gamma_{(p,tv)}(1)\Big)\Big|_{t=0}\\
                        &=v
                \end{align}
                Hence $\textrm{d}\,\exp_{p}(\mathbf{0})$ is the identity map,
                which is invertible, and hence by the inverse function theorem
                there is a neighborhood about $\mathbf{0}$ such that
                $\exp_{p}$ is a diffeomorphism onto it's image, which is an
                immersion.
        \end{itemize}
    \end{frame}
    \begin{frame}
        State the Gauss lemma.
        \begin{itemize}
            \item The exponential map preserves the dot product. That is, if
                $p\in{M}$, $v,w\in{T}_{p}M$, and if
                $\exp_{p}(v)$ and $\exp_{p}(w)$ are both defined, then:
                \begin{equation}
                    v\cdot{w}=
                        g_{\exp_{p}(v)}\Big(
                            \big(\textrm{d}\,\exp_{p}(v)\big)(v)\,|\,
                            \big(\textrm{d}\,\exp_{p}(v)\big)(w)\Big)
                \end{equation}
        \end{itemize}
    \end{frame}
    \begin{frame}
        Define the Riemann curvature tensor.
        \begin{itemize}
            \item Given a Riemannian manifold $(M,\mathcal{A},g)$ with an
                affine connection $\nabla$, the Riemann curvature tensor is
                defined as $R:\mathfrak{X}(M)^{3}\rightarrow\mathfrak{X}(M)$ by:
                \begin{equation}
                    R(X,Y)Z=\nabla_{X}\nabla_{Y}Z-
                        \nabla_{Y}\nabla_{X}Z-\nabla_{[X,Y]}Z
                \end{equation}
                where $[X,Y]$ is the Lie bracket of $X$ and $Y$.
        \end{itemize}
    \end{frame}
    \begin{frame}
        What are the identities of the Riemann curvature tensor field and the
        quadruple product?
        \begin{itemize}
            \item $R$ is bilinear in the first two components over
                $C^{\infty}(M,\mathbb{R})$.
            \item $R$ is $C^{\infty}(M,\mathbb{R})$ linear in the third component.
            \item The Bianchi identity holds:
                \begin{equation}
                    R(X,Y)Z+R(Y,Z)X+R(Z,X)Y=0
                \end{equation}
            \item A similar identity holds for the quadruple product:
            \begin{equation}
                (X,Y,Z,T)+(Y,Z,X,T)+(Z,X,Y,T)=0
            \end{equation}
            \item The quadruple product is anti-symmetric in the first two
                components:
                \begin{equation}
                    (X,Y,Z,T)=-(Y,X,Z,T)
                \end{equation}
            \item The quadruple product is anti-symmetric in the last two
                components:
                \begin{equation}
                    (X,Y,Z,T)=-(X,Y,T,Z)
                \end{equation}
            \item The components may be switched:
                \begin{equation}
                    (X,Y,Z,T)=(Z,T,X,Y)
                \end{equation}
        \end{itemize}
    \end{frame}
    \begin{frame}
        What is the definition of sectional curvature and why it does not
        depend on the choice of the basis used to define it?
        \begin{itemize}
            \item Given a 2 dimensional subspace $\delta\subseteq{T}_{p}M$ that
                is spanned by $v$ and $w$, the section curvature of $\delta$ is:
                \begin{equation}
                    K_{\delta}=\frac{(v,w,v,w)}{A(v,w)}
                        =\frac{g_{p}\big(R(v,w)v,w\big)}{\sqrt{||v||^{2}\,||w||^{2}-g_{p}(v,w)^{2}}}
                \end{equation}
            \item The result is independent of the choice of basis of $\delta$
                since any basis can be transformed to another basis by a
                combination of $(x,y)\mapsto(y,x)$,
                $(x,y)\mapsto(\lambda{x},y)$ $(\lambda\ne{0})$, and
                $(x,y)\mapsto(x+\lambda{y},y)$. The formula is invariant under
                these three operations, so it is invariant under change of
                basis.
        \end{itemize}
    \end{frame}
    \begin{frame}
        What is the definition of the Ricci and scalar curvatures?
        \begin{itemize}
            \item Pick a unit vector $x=z_{n}\in{T}_{p}M$. Append to it an
                orthonormal basis $\{z_{1},\dots,z_{n}\}$ of $T_{p}M$.
                The Ricci curvature is then:
                \begin{equation}
                    \textrm{Ric}_{p}(x)=
                        \frac{1}{n-1}\sum_{k=1}^{n-1}
                            g_{p}\Big(R(x,z_{k})x,z_{k}\Big)
                \end{equation}
            \item The scalar curvature is defined in terms of the Ricci
                curvature as follows:
                \begin{equation}
                    K(p)=\frac{1}{n}
                        \sum_{k=1}^{n}\textrm{Ric}_{p}(z_{k})
                \end{equation}
        \end{itemize}
    \end{frame}
    \begin{frame}
        What is the definition of a Jacobi vector field? Give one example.
        \begin{itemize}
            \item A Jacobi field along a geodesic $\gamma$ is a smooth vector
                field $J$ satisfying the \textit{Jacobi equation}:
                \begin{equation}
                    \frac{\textrm{D}^{2}J}{\textrm{d}\,t^{2}}+
                        R(\dot{\gamma},J)\dot{\gamma}=0
                \end{equation}
            \item Trivial Jacobi vector fields are given by $\dot{\gamma}$
                (extension of this to all of $M$) since
                $\textrm{D}^{2}/\textrm{d}\,t^{2}\;\dot{\gamma}=0$, since
                $\gamma$ is a geodesic, and $R(\dot{\gamma},\dot{\gamma})X=0$
                for all $X$ by the antisymmetry of the Riemann curvature
                tensor field. Hence the Jacobi equation is satisfied.
            \item Given a geodesic defined on $[0,a]$,
                $t\dot{\gamma}(t)$ also defines a Jacobi vector field on
                $(0,a)$.
        \end{itemize}
    \end{frame}
    \begin{frame}
        How do the zeroes of Jacobi vector fields determine how
        singular the exponential map is?
        \begin{itemize}
            \item Given a geodesic $\gamma$ with $\gamma(0)=p$, a point
                $q=\gamma(t_{0})$ is conjugate to $\gamma$ if and only if
                $t_{0}\dot{\gamma}(0)$ is a critical point of
                $\exp_{p}$. That is, $t_{0}\dot{\gamma}(t_{0})$ is a critical
                point if and only if there is a Jacobi vector field $J$ that is
                non-zero with $J(0)=J(t_{0})=0$. The multiplicity of
                $\gamma(t_{0})$ as a conjugate point to $p$ is the dimension
                of the kernel of $\textrm{d}\,\exp_{p}(v_{0})$.
        \end{itemize}
    \end{frame}
    \begin{frame}
        What can you say about the length of Jacobi vector fields on a
        negative sectional curvature manifold?
        \begin{itemize}
            \item Given a geodesic $\gamma$ and a Jacobi vector field
                $J$, the length $||J(t)||$ is an increasing function. That is
                if $t_{1}<t_{2}$, then $||J(t_{1})||<||J(t_{2})||$.
                In particular, $||J(0)||<||J(t_{0})||$ for all positive $t_{0}$.
        \end{itemize}
    \end{frame}
    \begin{frame}
        Formulate the Hadamard theorem. What does it say about the higher order
        homotopy groups of the negative sectional curvature manifolds?
        \begin{itemize}
            \item Theorem 1: If $(M,g)$ is a complete simply connected
                $n$ dimensional Riemannian manifold with non-positive section
                curvature, then $M$ is diffeomorphic to $\mathbb{R}^{n}$ and for
                all $p\in{M}$, $\exp_{p}:T_{p}M\rightarrow{M}$ is a covering
                map.
            \item Theorem 2: If $(M,g)$ is a connected complete $n$
                dimensional Riemannian manifold with non-positive sectional
                curvature, then the universal cover of $M$ is $\mathbb{R}^{n}$
                and for all $p\in{M}$, $\exp_{p}:T_{p}M\rightarrow{M}$ is a
                covering map.
            \item The higher order homotopy groups of a path connected space
                are determined by the homotopy groups of the universal cover of
                the space. Since the universal cover of these manifolds is
                $\mathbb{R}^{n}$, which is contractible, all of the higher order
                homotopy groups are trivial.
        \end{itemize}
    \end{frame}
    \begin{frame}
        Formulate the Preissman Theorem. What does it say about the centralizer
        of an element of $\pi_{1}$?
        \begin{itemize}
            \item If $M$ is a compact Riemannian manifold with negative
                curvature, then the non-trivial Abelian subgroup of $\pi_{1}(M)$
                is isomorphic to $\mathbb{Z}$.
            \item The centralizer of an element of $\pi_{1}(M)$ is isomorphic
                to $\mathbb{Z}$.
        \end{itemize}
    \end{frame}
    \begin{frame}
        What can you conclude from the Preissman and Hadamard theorems about
        the fundamental group of the space of immersed closed curves on a
        hyperbolic surface? Explain what the generators of this group are
        geometrically.
    \end{frame}
\end{document}
