%-----------------------------------LICENSE------------------------------------%
%   This file is part of Mathematics-and-Physics.                              %
%                                                                              %
%   Mathematics-and-Physics is free software: you can redistribute it and/or   %
%   modify it under the terms of the GNU General Public License as             %
%   published by the Free Software Foundation, either version 3 of the         %
%   License, or (at your option) any later version.                            %
%                                                                              %
%   Mathematics-and-Physics is distributed in the hope that it will be useful, %
%   but WITHOUT ANY WARRANTY; without even the implied warranty of             %
%   MERCHANTABILITY or FITNESS FOR A PARTICULAR PURPOSE.  See the              %
%   GNU General Public License for more details.                               %
%                                                                              %
%   You should have received a copy of the GNU General Public License along    %
%   with Mathematics-and-Physics.  If not, see <https://www.gnu.org/licenses/>.%
%------------------------------------------------------------------------------%
\documentclass{article}
\usepackage{graphicx}                           % Needed for figures.
\usepackage{amsmath}                            % Needed for align.
\usepackage{amssymb}                            % Needed for mathbb.
\usepackage{amsthm}                             % For the theorem environment.

\newtheoremstyle{normal}
    {\topsep}               % Amount of space above the theorem.
    {\topsep}               % Amount of space below the theorem.
    {}                      % Font used for body of theorem.
    {}                      % Measure of space to indent.
    {\bfseries}             % Font of the header of the theorem.
    {}                      % Punctuation between head and body.
    {.5em}                  % Space after theorem head.
    {}

\theoremstyle{normal}
\newtheorem{definition}{Definition}

\title{Differentiation}
\author{Ryan Maguire}
\date{Fall 2021}

% No indent and no paragraph skip.
\setlength{\parindent}{0em}
\setlength{\parskip}{0em}

\begin{document}
    \maketitle
    Let's compute the tangent line of $f(x)=x^{2}$ at $x_{0}=0.2$. The
    difference quotient for any real number $x\in\mathbb{R}$ is:
    \begin{equation}
        \frac{f(x+h)-f(x)}{h}
    \end{equation}
    For our function $f(x)=x^{2}$ we get:
    \begin{equation}
        \frac{(x+h)^{2}-x^{2}}{h}
    \end{equation}
    This gives us the slope of the \textit{secant} line between the points
    $(x,f(x))$ and $(x+h,f(x+h))$. This is shown in Fig.~\ref{fig:sec_line}.
    For small values of $h$ the secant line
    approximates the tangent line, and the \textit{limit} as $h$ tends to zero
    is precisely the tangent line. The limit of the difference quotient is also
    the definition of the derivative of $f$ at $x$:
    \begin{equation}
        \frac{\textrm{d}f}{\textrm{d}x}(x)
            =f'(x)
            =\lim_{h\rightarrow{0}}\frac{f(x+h)-f(x)}{h}
    \end{equation}
    Let's explicitly evaluate the derivative of our function $f(x)=x^{2}$ for
    any real number $x\in\mathbb{R}$. We'll then use this to calculate the
    equation of the tangent line. We have:
    \begin{align}
        f'(x)&=\lim_{h\rightarrow{0}}\frac{f(x+h)-f(x)}{h}\\
            &=\lim_{h\rightarrow{0}}\frac{(x+h)^{2}-x^{2}}{h}\\
            &=\lim_{h\rightarrow{0}}\frac{x^{2}+2xh+h^{2}-x^{2}}{h}\\
            &=\lim_{h\rightarrow{0}}\frac{2xh+h^{2}}{h}\\
            &=\lim_{h\rightarrow{0}}(2x+h)\\
            &=2x
    \end{align}
    So, we have $f'(x)=2x$. Let's use this. The slope of the tangent line
    at the point $x_{0}$ is given by $f'(x_{0})$. We've chosen $x_{0}=0.2$, so
    we have $f'(x_{0})=f'(0.2)=2(0.2)=0.4$. That is, the slope of the tangent
    line is $0.4$. The tangent line has the formula:
    \begin{equation}
        y_{T}=f'(x_{0})(x-x_{0})+y_{0}
    \end{equation}
    Where $y_{0}=f(x_{0})$. For $x_{0}=0.2$, we have
    $y_{0}=0.04$. So the tangent line is:
    \begin{equation}
        y_{T}=0.2(x-x_{0})+0.04
    \end{equation}
    This is plotted in Fig.~\ref{fig:tan_line}.
    \begin{figure}
        \centering
        \resizebox{\textwidth}{!}{%
            \includegraphics{../../images/secant_line_002.pdf}
        }
        \caption{Secant Line for $f$}
        \label{fig:sec_line}
    \end{figure}
    \begin{figure}
        \centering
        \resizebox{\textwidth}{!}{%
            \includegraphics{../../images/tangent_line_002.pdf}
        }
        \caption{Tangent Line for $f$}
        \label{fig:tan_line}
    \end{figure}
    \newpage
    I, the copyright holder of this work, release it into the public domain.
    This applies worldwide. In some countries this may not be legally possible;
    if so: I grant anyone the right to use this work for any purpose, without
    any conditions, unless such conditions are required by law.
    \par\hfill\par
    The source code used to generate this document is free software and released
    under version 3 of the GNU General Public License.
\end{document}
