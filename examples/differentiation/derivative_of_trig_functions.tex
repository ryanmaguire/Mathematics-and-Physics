%-----------------------------------LICENSE------------------------------------%
%   This file is part of Mathematics-and-Physics.                              %
%                                                                              %
%   Mathematics-and-Physics is free software: you can redistribute it and/or   %
%   modify it under the terms of the GNU General Public License as             %
%   published by the Free Software Foundation, either version 3 of the         %
%   License, or (at your option) any later version.                            %
%                                                                              %
%   Mathematics-and-Physics is distributed in the hope that it will be useful, %
%   but WITHOUT ANY WARRANTY; without even the implied warranty of             %
%   MERCHANTABILITY or FITNESS FOR A PARTICULAR PURPOSE.  See the              %
%   GNU General Public License for more details.                               %
%                                                                              %
%   You should have received a copy of the GNU General Public License along    %
%   with Mathematics-and-Physics.  If not, see <https://www.gnu.org/licenses/>.%
%------------------------------------------------------------------------------%
\documentclass{article}
\usepackage{graphicx}                           % Needed for figures.
\usepackage{amsmath}                            % Needed for align.
\usepackage{amssymb}                            % Needed for mathbb.
\usepackage{amsthm}                             % For the theorem environment.

\newtheoremstyle{normal}
    {\topsep}               % Amount of space above the theorem.
    {\topsep}               % Amount of space below the theorem.
    {}                      % Font used for body of theorem.
    {}                      % Measure of space to indent.
    {\bfseries}             % Font of the header of the theorem.
    {}                      % Punctuation between head and body.
    {.5em}                  % Space after theorem head.
    {}

\theoremstyle{normal}
\newtheorem{definition}{Definition}

\theoremstyle{plain}
\newtheorem{theorem}{Theorem}

\title{Differentiation}
\author{Ryan Maguire}
\date{Fall 2021}

% No indent and no paragraph skip.
\setlength{\parindent}{0em}
\setlength{\parskip}{0em}

\begin{document}
    \maketitle
    Differentiating trigonometric functions requires two bits of information.
    Firstly, the trigonometric identities. Second, the small angle
    approximations for sine and cosine. Memorizing trigonometric formulas is,
    in my opinion, a bad idea. It is easy to forget if some part of the formula
    needs $+$ or $-$ sign, and this could ruin the rest of your work. There are
    two other approaches. First, you could simply consult a textbook or online
    reference for the required formula. This is fine. Memorizing formulas does
    not demonstrate mastery of mathematics, and it's silly to pretend that
    quick references to these equations don't exist. Most working
    mathematicians do not memorize these formulas.
    \par\hfill\par
    The second way of obtaining trigonometric formulas is by \textit{deriving}
    them using a quick and easy trick. This is the way many mathematicians
    actually incorporate these trigonometric identities into their work. Either
    they look them up, or they quickly derive them using the trick we're about
    to learn. The trick requires \textit{complex numbers}. Complex numbers are
    usually taught in high school in the United States, but this may not be a
    global phenomenon, so here's a quick crash course in complex numbers.
    \par\hfill\par
    A complex number is a number $z=x+iy$ where $x$ and $y$ are real numbers,
    and $i$ is the \textit{imaginary unit}. Because of this, the value $x$ is
    called the \textit{real part}, and the value $y$ is called the
    \textit{imaginary part}. The only algebraic property we
    give to $i$ is that $i^{2}=-1$. Just like real numbers, we can do
    arithmetic with complex numbers. First we ask how do we add them? We simply
    use some factoring.
    \begin{align}
        (a+ib)+(c+id)
            &=a+ib+c+id\\
            &=a+c+ib+id\\
            &=(a+c)+i(b+d)
    \end{align}
    Since $a$, $b$, $c$, and $d$ are real numbers, $a+c$ is a real number, and
    $b+d$ is a real number. So $(a+c)+i(b+d)$ is in the form $x+iy$ where
    $x$ and $y$ are real numbers, which is precisely how we defined complex
    numbers. Multiplication is almost as easier. We just need to use the
    distributive property of multiplication. We have:
    \begin{align}
        (a+ib)(c+id)
            &=ac+i^{2}bd+iad+ibc\\
            &=ac+i^{2}bd+i(ad+bc)\\
            &=(ac-bc)+i(ad+bc)
    \end{align}
    Here we used the fact that $i^{2}=-1$. Again we have the expression in the
    form $x+iy$ for real numbers $x$ and $y$. This is just about all we need to
    know about complex numbers in order to derive every trigonometric identity.
    The last bit of information needed is called \textit{Euler's Formula}.
    The proof of this is not hard, but requires \textit{Taylor Series}, which
    is something we don't see until the end of a calculus course. At the end of
    this document I'll present the proof, but if you don't care to read it
    that's fine too. The formula says:
    \begin{equation}
        e^{i\theta}=\cos(\theta)+i\sin(\theta)
    \end{equation}
    Where $\theta$ is any real number. This formula is really bizarre! It says
    the exponential function and the trigonometric functions are somehow
    related when we use complex numbers. Now, let's use it. Suppose we want to
    simplify the expressions $\cos(a+b)$ and $\sin(a+b)$. How can we do this?
    We look at Euler's formula. Remember from exponential rules that
    $e^{a+b}=e^{a}e^{b}$. We can use this to simplify $\sin(a+b)$ and
    $\cos(a+b)$.
    \begin{align}
        e^{i(a+b)}&=\cos(a+b)+i\sin(a+b)\tag{Euler's Formula}\\
        e^{i(a+b)}&=e^{ia+ib}\tag{Distribute $i$}\\
            &=e^{ia}e^{ib}\tag{Exponential Property}\\
            &=\big(\cos(a)+i\sin(a)\big)\big(\cos(b)+i\sin(b)\big)
                \tag{Euler's Formula}
    \end{align}
    To simplify this last expression, we simply use the rule for multiplying
    complex numbers. We treat the expression like we normally would using the
    distributive property, but whenever we see $i^{2}$ we can replace it with
    $-1$. This gives us:
    \begin{align}
        &\big(\cos(a)+i\sin(a)\big)\big(\cos(b)+i\sin(b)\big)\nonumber\\
        &\quad=\big(\cos(a)\cos(b)-\sin(a)\sin(b)\big)+
            i\big(\cos(a)\sin(b)+\cos(b)\sin(a)\big)
    \end{align}
    But wait! From the very beginning, this whole thing is equal to
    $\cos(a+b)+i\sin(a+b)$. So we have:
    \begin{align}
        &\cos(a+b)+i\sin(a+b)\nonumber\\
        &\quad=\big(\cos(a)\cos(b)-\sin(a)\sin(b)\big)+
            i\big(\cos(a)\sin(b)+\cos(b)\sin(a)\big)
    \end{align}
    The real part of the left hand side is $\cos(a+b)$ and the real part of
    the right hand side is $\cos(a)\cos(b)-\sin(a)\sin(b)$. Since we have
    equality, the real parts must be equal. That is:
    \begin{equation}
        \cos(a+b)=\cos(a)\cos(b)-\sin(a)\sin(b)
    \end{equation}
    Similarly, the imaginary part of the left hand side is $\sin(a+b)$ and the
    imaginary part of the right hand side is
    $\cos(a)\sin(b)+\cos(b)\sin(a)$. Since we have equality, the imaginary
    parts must be equal. That is:
    \begin{equation}
        \sin(a+b)=\cos(a)\sin(b)+\cos(b)\sin(a)
    \end{equation}
    If this last step is confusing, the fact that equality means the real and
    imaginary parts must be the same, consider the following. A complex number
    $z=x+iy$ can be thought of as a point in the plane with coordinates
    $(x,y)$. If we have two points in the plane $(a,b)$ and $(c,d)$, what would
    it mean for these points to be \textit{equal}? The should have the same
    $x$ coordinate and the same $y$ coordinate, otherwise they'd be difference
    points! That is, $(a,b)=(c,d)$ is true precisely when $a=c$ and $b=d$.
    Let's translate this back to complex number. $a+ib=c+id$ is true precisely
    when $a=c$ and $b=d$. That is, two complex numbers are equal precisely when
    their real parts are equal and their imaginary parts are equal.
    \par\hfill\par
    A common problem one finds in a calculus textbook when learning about
    integration involves the functions $\cos^{2}(x)$ and $\sin^{2}(x)$
    (particularly, the problem asks to solve $\int\cos^{2}(x)\textrm{d}x$. We
    haven't seen integration yet, so don't worry about this). The problem
    becomes easier if we know the square formula for sine and cosine. Let's
    use Euler's method to derive it. We want to simplify $\cos^{2}(x)$ and
    $\sin^{2}(x)$. We have:
    \begin{align}
        e^{i2x}&=\cos(2x)+i\sin(2x)\tag{Euler's Formula}\\
        e^{i2x}&=\big(e^{ix}\big)^{2}\tag{Exponential Property}\\
        \big(e^{ix}\big)^{2}&=\big(\cos(x)+i\sin(x)\big)^{2}
            \tag{Euler's Formula}\\
            &=\big(\cos^{2}(x)-\sin^{2}(x)\big)+i2\cos(x)\sin(x)
    \end{align}
    Comparining the real and imaginary parts, we obtain the following formulas:
    \begin{align}
        \cos(2x)&=\cos^{2}(x)-\sin^{2}(x)\\
        \sin(2x)&=2\cos(x)\sin(x)
    \end{align}
    Now you say, hold on! We wanted a formula for $\cos^{2}(x)$ and
    $\sin^{2}(x)$. We get that by applying the trigonometric identity
    $\cos^{2}(x)+\sin^{2}(x)=1$. From this, $\sin^{2}(x)=1-\cos^{2}(x)$. We get:
    \begin{align}
        \cos(2x)&=\cos^{2}(x)-\sin^{2}(x)\\
            &=\cos^{2}(x)-\big(1-\cos^{2}(x)\big)\\
            &=\cos^{2}(x)-1+\cos^{2}(x)\\
            &=2\cos^{2}(x)-1
    \end{align}
    Solving for $\cos^{2}(x)$, we get:
    \begin{equation}
        \cos^{2}(x)=\frac{\cos(2x)+1}{2}
    \end{equation}
    For $\sin^{2}(x)$ we again use the fact that $\sin^{2}(x)=1-\cos^{2}(x)$.
    \begin{align}
        \sin^{2}(x)&=1-\cos^{2}(x)\\
            &=1-\frac{\cos(2x)+1}{2}\\
            &=\frac{1-\cos(2x)}{2}
    \end{align}
    So, the formula for $\sin^{2}(x)$ is:
    \begin{equation}
        \sin^{2}(x)=\frac{1-\cos(2x)}{2}
    \end{equation}
    The moral of the story is that you should not feel like mathematics is a
    game of memorization. Memorizing tables of trigonometric functions is
    tedious and not very helpful, and working mathematicians do not do this!
    They do one of two things: Look up the desired formula, or quickly use
    Euler's formula to derive the result they need. With practice, using
    Euler's formula to calculate a certain trigonometric formula can be done
    very quickly in your head. If complex numbers seem too scary right now,
    looking up the formula in a textbook is fine.
    \par\hfill\par
    As stated, to compute the derivative of $\cos(x)$ and $\sin(x)$ requires
    a few trigonometric identities, and the
    \textit{small angle approximations}. These approximations are widespread in
    engineering and physics since they can be very accurate when used correctly,
    and make problems much easier to solve. If $|x|$ is small (say, less than
    $0.1$ radians), then:
    \begin{align}
        \sin(x)&\approx{x}\\
        \cos(x)&\approx{1}-\frac{x^{2}}{2}
    \end{align}
    \newpage
    I, the copyright holder of this work, release it into the public domain.
    This applies worldwide. In some countries this may not be legally possible;
    if so: I grant anyone the right to use this work for any purpose, without
    any conditions, unless such conditions are required by law.
    \par\hfill\par
    The source code used to generate this document is free software and released
    under version 3 of the GNU General Public License.
\end{document}
