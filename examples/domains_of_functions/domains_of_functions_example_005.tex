%-----------------------------------LICENSE------------------------------------%
%   This file is part of Mathematics-and-Physics.                              %
%                                                                              %
%   Mathematics-and-Physics is free software: you can redistribute it and/or   %
%   modify it it under the terms of the GNU General Public License as          %
%   published by the Free Software Foundation, either version 3 of the         %
%   License, or (at your option) any later version.                            %
%                                                                              %
%   Mathematics-and-Physics is distributed in the hope that it will be useful, %
%   but WITHOUT ANY WARRANTY; without even the implied warranty of             %
%   MERCHANTABILITY or FITNESS FOR A PARTICULAR PURPOSE.  See the              %
%   GNU General Public License for more details.                               %
%                                                                              %
%   You should have received a copy of the GNU General Public License along    %
%   with Mathematics-and-Physics.  If not, see <https://www.gnu.org/licenses/>.%
%------------------------------------------------------------------------------%
\documentclass{article}
\usepackage{mathtools, esint, mathrsfs} % amsmath and integrals.
\usepackage{amsthm, amsfonts, amssymb}  % Fonts and theorems.
\usepackage{hyperref}                   % Hyperlinks.

% Colors for hyperref.
\hypersetup{
    colorlinks=true,
    linkcolor=blue,
    filecolor=magenta,
    urlcolor=Cerulean,
    citecolor=SkyBlue
}

\title{Domains of Functions}
\author{Ryan Maguire}
\date{Fall 2021}
\setlength{\parindent}{0em}
\setlength{\parskip}{0em}
\begin{document}
    \maketitle
    Consider the following function:
    \begin{equation}
        f(x)=\frac{\sin(x)}{x}
    \end{equation}
    This function is well defined everywhere except for $x=0$. However, the
    \textit{small angle approximation}, often used by physicists, states that
    if $x$ is a small real number, then $\sin(x)\approx{x}$. The symbol
    $\approx$ means \textit{is approximately equal to}. We can verify this
    from the graph of the two functions close to the origin. By examining
    Fig.~\ref{fig:small_angle_approx} we see that for small values the graphs
    of $\sin(x)$ and $x$ are nearly identical. Using this we have, for
    small $x$, the following:
    \begin{equation}
        \frac{\sin(x)}{x}\approx\frac{x}{x}=1
    \end{equation}
    And indeed the \textit{limit} of $f(x)$ as $x$ approaches zero is 1, even
    though $f(0)$ is undefined. With this we can define a new function by
    \textit{filling in} where $f(x)$ is undefined. This is the
    \textit{sinc} function, and it's use is widespread in physics, engineering,
    and signal processing.
    \begin{equation}
        \textrm{sinc}(x)=
        \begin{cases}
            \frac{\sin(x)}{x},&x\ne{0}\\
            1,&x=0
        \end{cases}
    \end{equation}
    Since the limit of $\textrm{sinc}(x)$ as $x$ approaches zero is 1, and
    since $\textrm{sinc}(0)=1$, we have from the
    \textit{limit definition of continuity} that $\textrm{sinc}(x)$ is
    continuous at 0. This function is shown in
    Fig.~\ref{fig:sinc_x}.
    \begin{figure}
        \centering
        \includegraphics{../../images/small_angle_approximation.pdf}
        \caption{Small Angle Approximation}
        \label{fig:small_angle_approx}
    \end{figure}
    \begin{figure}
        \centering
        \includegraphics{../../images/sinc_x.pdf}
        \caption{The sinc function}
        \label{fig:sinc_x}
    \end{figure}
    \newpage
    I, the copyright holder of this work, release it into the public domain.
    This applies worldwide. In some countries this may not be legally possible;
    if so: I grant anyone the right to use this work for any purpose, without
    any conditions, unless such conditions are required by law.
    \par\hfill\par
    The source code used to generate this document is free software and released
    under version 3 of the GNU General Public License.
\end{document}
