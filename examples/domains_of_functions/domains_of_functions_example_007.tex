%-----------------------------------LICENSE------------------------------------%
%   This file is part of Mathematics-and-Physics.                              %
%                                                                              %
%   Mathematics-and-Physics is free software: you can redistribute it and/or   %
%   modify it under the terms of the GNU General Public License as             %
%   published by the Free Software Foundation, either version 3 of the         %
%   License, or (at your option) any later version.                            %
%                                                                              %
%   Mathematics-and-Physics is distributed in the hope that it will be useful, %
%   but WITHOUT ANY WARRANTY; without even the implied warranty of             %
%   MERCHANTABILITY or FITNESS FOR A PARTICULAR PURPOSE.  See the              %
%   GNU General Public License for more details.                               %
%                                                                              %
%   You should have received a copy of the GNU General Public License along    %
%   with Mathematics-and-Physics.  If not, see <https://www.gnu.org/licenses/>.%
%------------------------------------------------------------------------------%
\documentclass{article}
\usepackage{amsmath}                            % align environment here.
\usepackage{graphicx}                           % Needed for figures.
\usepackage{hyperref}                           % Hyperlinks for figures.
\hypersetup{colorlinks=true, linkcolor=blue}    % Colors for hyperref.

\title{Domains of Functions}
\author{Ryan Maguire}
\date{Fall 2021}

% No indent and no paragraph skip.
\setlength{\parindent}{0em}
\setlength{\parskip}{0em}

\begin{document}
    \maketitle
    Consider the expression:
    \begin{equation}
        f(x)=\frac{1}{\frac{1}{x}-1}
    \end{equation}
    We can simplify this by the following:
    \begin{align}
        f(x)&=\frac{1}{\frac{1}{x}-1}\\
            &=\frac{x}{x}\cdot\frac{1}{\frac{1}{x}-1}\\
            &=\frac{x}{1-x}
    \end{align}
    If we examine this new expression we might conclude that the only value
    where $f(x)$ may be undefined is $x=1$, since then $1-x$ would evaluate to
    zero and our expression would have a division-by-zero. However, in the
    simplification we multiplied our expression by $\frac{x}{x}$. For all
    values other than zero this simplifies to 1, and is therefore a valid
    step, but for 0 this expression is undefined. Looking back to our original
    expression we see that $\frac{1}{x}$ occurs in the denominator. For this
    to be well defined, we need to avoid $x=0$ as well. The domain can be
    written as:
    \begin{equation}
        D=(-\infty, 0)\cup(0,1)\cup(1,\infty)
    \end{equation}
    This function is plotted in Fig.~\ref{fig:graph_of_func}.
    \begin{figure}
        \centering
        \includegraphics{../../images/x_by_one_minus_x.pdf}
        \caption{Graph of the function $f$}
        \label{fig:graph_of_func}
    \end{figure}
    \newpage
    I, the copyright holder of this work, release it into the public domain.
    This applies worldwide. In some countries this may not be legally possible;
    if so: I grant anyone the right to use this work for any purpose, without
    any conditions, unless such conditions are required by law.
    \par\hfill\par
    The source code used to generate this document is free software and released
    under version 3 of the GNU General Public License.
\end{document}
