%-----------------------------------LICENSE------------------------------------%
%   This file is part of Mathematics-and-Physics.                              %
%                                                                              %
%   Mathematics-and-Physics is free software: you can redistribute it and/or   %
%   modify it under the terms of the GNU General Public License as             %
%   published by the Free Software Foundation, either version 3 of the         %
%   License, or (at your option) any later version.                            %
%                                                                              %
%   Mathematics-and-Physics is distributed in the hope that it will be useful, %
%   but WITHOUT ANY WARRANTY; without even the implied warranty of             %
%   MERCHANTABILITY or FITNESS FOR A PARTICULAR PURPOSE.  See the              %
%   GNU General Public License for more details.                               %
%                                                                              %
%   You should have received a copy of the GNU General Public License along    %
%   with Mathematics-and-Physics.  If not, see <https://www.gnu.org/licenses/>.%
%------------------------------------------------------------------------------%
\documentclass{article}
\usepackage{amsmath}                            % Needed for align.
\usepackage{amssymb}                            % Needed for mathbb.
\usepackage{amsthm}                             % For the theorem environment.

\newtheoremstyle{normal}
    {\topsep}               % Amount of space above the theorem.
    {\topsep}               % Amount of space below the theorem.
    {}                      % Font used for body of theorem.
    {}                      % Measure of space to indent.
    {\bfseries}             % Font of the header of the theorem.
    {}                      % Punctuation between head and body.
    {.5em}                  % Space after theorem head.
    {}

\theoremstyle{normal}
\newtheorem{definition}{Definition}

\title{Continuity}
\author{Ryan Maguire}
\date{Fall 2021}

% No indent and no paragraph skip.
\setlength{\parindent}{0em}
\setlength{\parskip}{0em}

\begin{document}
    \maketitle
    The definition of continuity is as follows:
    \begin{definition}
        A real-valued function that is continuous at a point
        $x_{0}\in\mathbb{R}$ is a function $f:\mathbb{R}\rightarrow\mathbb{R}$
        such that for all $\varepsilon>0$ there exists a $\delta>0$ such that
        for all $x\in\mathbb{R}$ with $|x-x_{0}|<\delta$ it is true that
        $|f(x)-f(x_{0})|<\varepsilon$.
    \end{definition}
    Let's prove $f(x)=1/x$ is continuous on $(0,\infty)$.
    \begin{equation}
        \textbf{Want:}\quad
        |x-x_{0}|<\delta
        \Rightarrow
        |f(x)-f(x_{0})|<\varepsilon
    \end{equation}
    Substituting $f$ with $f(x)=1/x$:
    \begin{equation}
        \textbf{Want:}\quad
        |x-x_{0}|<\delta
        \Rightarrow
        \big|\frac{1}{x}-\frac{1}{x_{0}}\big|<\varepsilon
    \end{equation}
    Simplifying the expression, this is equivalent to:
    \begin{equation}
        \textbf{Want:}\quad
        |x-x_{0}|<\delta
        \Rightarrow
        \big|\frac{x-x_{0}}{xx_{0}}\big|<\varepsilon
    \end{equation}
    Which can be further simplified to:
    \begin{equation}
        \textbf{Want:}\quad
        |x-x_{0}|<\delta
        \Rightarrow
        \frac{1}{|xx_{0}|}|x-x_{0}|<\varepsilon
    \end{equation}
    Since we only care about $|x-x_{0}|<\delta$, we have:
    \begin{equation}
        \frac{1}{|xx_{0}|}|x-x_{0}|<\frac{\delta}{|xx_{0}|}
    \end{equation}
    So now we update our wish-list:
    \begin{equation}
        \textbf{Want:}\quad
        |x-x_{0}|<\delta
        \Rightarrow
        \frac{\delta}{|xx_{0}|}\leq\varepsilon
    \end{equation}
    For if this were true, we would have:
    \begin{align}
        |x-x_{0}|<\delta&\Rightarrow\frac{\delta}{|xx_{0}|}\leq\varepsilon\\
        &\Rightarrow\frac{1}{|xx_{0}|}|x-x_{0}|<\varepsilon\\
        &\Rightarrow\big|\frac{x-x_{0}}{xx_{0}}\big|<\varepsilon\\
        &\Rightarrow\big|\frac{1}{x}-\frac{1}{x_{0}}\big|<\varepsilon
    \end{align}
    So, we direct our attention to making $\delta/|xx_{0}|\leq\varepsilon$ a
    true statement. It will \textbf{not} be true for all $x\in(0,\infty)$ since
    as $x$ tends to 0, the expression $1/|xx_{0}|$ gets arbitrarily large. But
    we don't need to care about values $x$ that are really close to zero, we
    only care about values $x$ that are close to $x_{0}$. So let's restrict
    our attention to $x\in(x_{0}/2,3x_{0}/2)$. In other words,
    $|x-x_{0}|<x_{0}/2$. This is equivalent to requiring that
    $\delta\leq{x}_{0}/2$. If we do this, the expression
    $1/|xx_{0}|$ is now bounded. The largest this can be is when $x$ is at
    it's smallest, which is $x_{0}/2$. Using this, we have:
    \begin{equation}
        \frac{\delta}{|xx_{0}|}<\frac{\delta}{|x_{0}^{2}/2|}
        =\frac{2\delta}{x_{0}^{2}}
    \end{equation}
    We update our wish-list one last time:
    \begin{equation}
        \textbf{Want:}\quad
        |x-x_{0}|<\delta
        \Rightarrow
        \frac{2\delta}{x_{0}^{2}}\leq\varepsilon
    \end{equation}
    The likely candidate is $\delta=\varepsilon{x}_{0}^{2}/2$, but we are
    already requiring that $\delta\leq{x}_{0}/2$. We need to choose the smaller
    of these two expression.
    \par\hfill\par
    We have found our candidate $\delta$. Let's show that it works.
    Let $\varepsilon>0$. Choose $\delta=\min(x_{0}/2,\varepsilon{x}_{0}^{2}/2)$.
    If $|x-x_{0}|<\delta$, then
    $|x-x_{0}|<\min(x_{0}/2,\varepsilon{x}_{0}^{2}/2)$. But if
    $|x-x_{0}|<\min(x_{0}/2,\varepsilon{x}_{0}^{2}/2)$, then
    $|x-x_{0}|<\varepsilon{x}_{0}^{2}/2$, by the definition of $\min$. And if
    $|x-x_{0}|<\varepsilon{x}_{0}^{2}/2$, then
    $\frac{2}{x_{0}^{2}}|x-x_{0}|<\varepsilon$. But since
    $|x-x_{0}|<\min(x_{0}/2,\varepsilon{x}_{0}^{2}/2)$, we have
    $|x-x_{0}|<x_{0}/2$, again by the definition of $\min$. And if
    $|x-x_{0}|<x_{0}/2$, then $x_{0}/2<x<3x_{0}/2$ and therefore
    $1/|xx_{0}|\leq2/x_{0}^{2}$. But then:
    \begin{equation}
        \frac{1}{|xx_{0}|}|x-x_{0}|
        \leq\frac{2}{x_{0}^{2}}|x-x_{0}|<\varepsilon
    \end{equation}
    And therefore $|x-x_{0}|/|xx_{0}|<\varepsilon$. But:
    \begin{equation}
        \frac{1}{|xx_{0}|}|x-x_{0}|
        =\big|\frac{x-x_{0}}{xx_{0}}\big|
        =\big|\frac{1}{x}-\frac{1}{x_{0}}\big|
    \end{equation}
    And therefore $|1/x-1/x_{0}|<\varepsilon$.
    \newpage
    I, the copyright holder of this work, release it into the public domain.
    This applies worldwide. In some countries this may not be legally possible;
    if so: I grant anyone the right to use this work for any purpose, without
    any conditions, unless such conditions are required by law.
    \par\hfill\par
    The source code used to generate this document is free software and released
    under version 3 of the GNU General Public License.
\end{document}
