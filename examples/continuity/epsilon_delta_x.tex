%-----------------------------------LICENSE------------------------------------%
%   This file is part of Mathematics-and-Physics.                              %
%                                                                              %
%   Mathematics-and-Physics is free software: you can redistribute it and/or   %
%   modify it under the terms of the GNU General Public License as             %
%   published by the Free Software Foundation, either version 3 of the         %
%   License, or (at your option) any later version.                            %
%                                                                              %
%   Mathematics-and-Physics is distributed in the hope that it will be useful, %
%   but WITHOUT ANY WARRANTY; without even the implied warranty of             %
%   MERCHANTABILITY or FITNESS FOR A PARTICULAR PURPOSE.  See the              %
%   GNU General Public License for more details.                               %
%                                                                              %
%   You should have received a copy of the GNU General Public License along    %
%   with Mathematics-and-Physics.  If not, see <https://www.gnu.org/licenses/>.%
%------------------------------------------------------------------------------%
\documentclass{article}
\usepackage{amsmath}                            % Needed for align.
\usepackage{amssymb}                            % Needed for mathbb.
\usepackage{amsthm}                             % For the theorem environment.
\usepackage{graphicx}                           % Needed for figures.
\usepackage{hyperref}                           % Hyperlinks for figures.
\hypersetup{colorlinks=true, linkcolor=blue}    % Colors for hyperref.

\newtheoremstyle{normal}
    {\topsep}               % Amount of space above the theorem.
    {\topsep}               % Amount of space below the theorem.
    {}                      % Font used for body of theorem.
    {}                      % Measure of space to indent.
    {\bfseries}             % Font of the header of the theorem.
    {}                      % Punctuation between head and body.
    {.5em}                  % Space after theorem head.
    {}

\theoremstyle{normal}
\newtheorem{definition}{Definition}

\title{Continuity}
\author{Ryan Maguire}
\date{Fall 2021}

% No indent and no paragraph skip.
\setlength{\parindent}{0em}
\setlength{\parskip}{0em}

\begin{document}
    \maketitle
    Let's show that $f(x)=x$ is continuous for all $x\in\mathbb{R}$. We'll
    start with the definition of continuity.
    \begin{definition}
        A real-valued function that is continuous at a point
        $x_{0}\in\mathbb{R}$ is a function $f:\mathbb{R}\rightarrow\mathbb{R}$
        such that for all $\varepsilon>0$ there exists a $\delta>0$ such that
        for all $x\in\mathbb{R}$ with $|x-x_{0}|<\delta$ it is true that
        $|f(x)-f(x_{0})|<\varepsilon$.
    \end{definition}
    This is very wordy, but precise. Intuitively if we vary $x$ by no more than
    $\delta$ from the point $x_{0}$, then $f(x)$ varies no more than
    $\varepsilon$ from $f(x_{0})$. That is:
    \begin{center}
        \textbf{Slogan}
        \par
        \textbf{Small perturbations in $x$ result in small perturbations}
        \textbf{in $f(x)$}. 
    \end{center}
    The other slogan to live by is that
    \textit{nearby points go to nearby points}. Both of these phrases give
    intuitive meanings to continuity, but we need the formal definition to
    actually apply it to problems. The crucial thing to note is that
    continuity is defined \textit{point-wise}. That is, a function can be
    continuous at one point and not the other. A function can even be
    continuous at only \textit{one} point on the entire real line
    and discontinuous everywhere else.
    \par\hfill\par
    Let's now show that $f(x)=x$ is continuous. This is different then problems
    we are used to. In elementary algebra we \textit{solve} for variables. Now,
    we need to prove that no matter what $\varepsilon>0$ is given to us, we can
    find a $\delta>0$ such that $|x-x_{0}|<\delta$ implies
    $|f(x)-f(x_{0})|<\varepsilon$. Do not get confused by the modes of thinking
    that apply in other areas of mathematics. We are not trying to
    \textit{solve} for $\varepsilon$, it is given to us. Our job is to find
    the $\delta$. The trick is to work backwards. Suppose we
    found such a $\delta>0$. What would this say? Well, we'd have:
    \begin{equation}
        |x-x_{0}|<\delta\Rightarrow|f(x)-f(x_{0})|<\varepsilon
    \end{equation}
    Let's now substitute $f(x)=x$, obtaining:
    \begin{equation}
        |x-x_{0}|<\delta\Rightarrow|x-x_{0}|<\varepsilon
    \end{equation}
    Now we ask ourselves \textit{what value $\delta>0$ would make this true?}
    The clear candidate is $\delta=\varepsilon$. This would then read:
    \begin{equation}
        |x-x_{0}|<\varepsilon\Rightarrow|x-x_{0}|<\varepsilon
    \end{equation}
    Read aloud,
    \textit{if $|x-x_{0}|<\varepsilon$, then $|x-x_{0}|<\varepsilon$}. This is
    a \textit{tautology}. So $\delta=\varepsilon$ works. It's not the only
    $\delta$ we could have chosen. Indeed, any positive value less than
    $\varepsilon$ would work. Suppose we chose $\delta=\varepsilon/2$. What
    would this say then?
    \begin{equation}
        |x-x_{0}|<\delta\Rightarrow|x-x_{0}|<\frac{\varepsilon}{2}
    \end{equation}
    Well, if $|x-x_{0}|<\varepsilon/2$, then
    $|x-x_{0}|<\varepsilon$ since $\varepsilon/2<\varepsilon$. That is, we
    would have the following chain of inequalities:
    \begin{align}
        |x-x_{0}|<\delta
        &\Rightarrow|x-x_{0}|<\frac{\varepsilon}{2}<\varepsilon\\
        &\Rightarrow|x-x_{0}|<\varepsilon
    \end{align}
    So $\delta=\varepsilon/2$ is a valid choice. Do not get trapped in the
    mindset of finding the \textit{best} $\delta$. The choice
    $\delta=\varepsilon$ is, in a sense, the \textit{best} choice since any
    larger value wouldn't work. But who cares? Just find a $\delta$ that works!
    The freedom to choose smaller values than necessary often makes the
    problem significantly easier.
    \newpage
    I, the copyright holder of this work, release it into the public domain.
    This applies worldwide. In some countries this may not be legally possible;
    if so: I grant anyone the right to use this work for any purpose, without
    any conditions, unless such conditions are required by law.
    \par\hfill\par
    The source code used to generate this document is free software and released
    under version 3 of the GNU General Public License.
\end{document}
