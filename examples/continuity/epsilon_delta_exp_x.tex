%-----------------------------------LICENSE------------------------------------%
%   This file is part of Mathematics-and-Physics.                              %
%                                                                              %
%   Mathematics-and-Physics is free software: you can redistribute it and/or   %
%   modify it under the terms of the GNU General Public License as             %
%   published by the Free Software Foundation, either version 3 of the         %
%   License, or (at your option) any later version.                            %
%                                                                              %
%   Mathematics-and-Physics is distributed in the hope that it will be useful, %
%   but WITHOUT ANY WARRANTY; without even the implied warranty of             %
%   MERCHANTABILITY or FITNESS FOR A PARTICULAR PURPOSE.  See the              %
%   GNU General Public License for more details.                               %
%                                                                              %
%   You should have received a copy of the GNU General Public License along    %
%   with Mathematics-and-Physics.  If not, see <https://www.gnu.org/licenses/>.%
%------------------------------------------------------------------------------%
\documentclass{article}
\usepackage{amsmath}                            % Needed for align.
\usepackage{amssymb}                            % Needed for mathbb.
\usepackage{amsthm}                             % For the theorem environment.

\newtheoremstyle{normal}
    {\topsep}               % Amount of space above the theorem.
    {\topsep}               % Amount of space below the theorem.
    {}                      % Font used for body of theorem.
    {}                      % Measure of space to indent.
    {\bfseries}             % Font of the header of the theorem.
    {}                      % Punctuation between head and body.
    {.5em}                  % Space after theorem head.
    {}

\theoremstyle{normal}
\newtheorem{definition}{Definition}
\theoremstyle{plain}
\newtheorem{theorem}{Theorem}

\title{Continuity (NOT FINISHED)}
\author{Ryan Maguire}
\date{Fall 2021}

% No indent and no paragraph skip.
\setlength{\parindent}{0em}
\setlength{\parskip}{0em}

\begin{document}
    \maketitle
    The definition of continuity is as follows:
    \begin{definition}
        A real-valued function that is continuous at a point
        $x_{0}\in\mathbb{R}$ is a function $f:\mathbb{R}\rightarrow\mathbb{R}$
        such that for all $\varepsilon>0$ there exists a $\delta>0$ such that
        for all $x\in\mathbb{R}$ with $|x-x_{0}|<\delta$ it is true that
        $|f(x)-f(x_{0})|<\varepsilon$.
    \end{definition}
    Let's prove $f(x)=\exp(x)$ is continuous. Alternatively, $f(x)=e^{x}$.
    \begin{equation}
        \textbf{Want:}\quad
        |x-x_{0}|<\delta
        \Rightarrow|f(x)-f(x)|<\varepsilon
    \end{equation}
    Substituting $f$:
    \begin{equation}
        \textbf{Want:}\quad
        |x-x_{0}|<\delta
        \Rightarrow
        |\exp(x)-\exp(x_{0})|<\varepsilon
    \end{equation}
    One of the fundamental identities of the exponential function is:
    \begin{equation}
        \exp(a)\exp(b)=\exp(a+b)
    \end{equation}
    Alternatively, $e^{a}e^{b}=e^{a+b}$. Let's use this:
    \begin{align}
        \exp(x)-\exp(x_{0})
            &=\big(\exp(x)-\exp(x_{0})\big)\frac{\exp(-x_{0})}{\exp(-x_{0})}\\
            &=\frac{\exp(x)\exp(-x_{0})-\exp(x_{0})\exp(-x_{0})}{\exp(-x_{0})}\\
            &=\frac{\exp(x-x_{0})-\exp(x_{0}-x_{0})}{\exp(x_{0})}\\
            &=\frac{\exp(x-x_{0})-1}{\exp(-x_{0})}
    \end{align}
    Using the fact that $\exp(-x_{0})=1/\exp(x_{0})$, we get:
    \begin{equation}
        \exp(x)-\exp(x_{0})
        =\exp(x_{0})\big(\exp(x-x_{0})-1\big)
    \end{equation}
    We update our wish-list:
    \begin{equation}
        \textbf{Want:}\quad
        |x-x_{0}|<\delta
        \Rightarrow
        \exp(x_{0})\big|\exp(x-x_{0})-1\big|<\varepsilon
    \end{equation}
    Since we only care about $|x-x_{0}|<\delta$, we update our
    wish-list:
    \begin{equation}
        \textbf{Want:}\quad
        |x-x_{0}|<\delta
        \Rightarrow
        \exp(x_{0})\big|\exp(\delta)-1\big|\leq\varepsilon
    \end{equation}
    This says we want:
    \begin{equation}
        \frac{-\varepsilon}{\exp(x_{0})}\leq
        \exp(\delta)-1\leq\frac{\varepsilon}{\exp(x_{0})}
    \end{equation}
    Or equivalently:
    \begin{equation}
        \frac{-\varepsilon}{\exp(x_{0})}+1\leq
        \exp(\delta)\leq\frac{\varepsilon}{\exp(x_{0})}+1
    \end{equation}
    Since we seek a $\delta>0$, the expression
    $1-\varepsilon/\exp(x_{0})\leq\exp(\delta)$ will be true since
    $1-\varepsilon/\exp(x_{0})<1$ and $\exp(\delta)>1$ for $\delta>0$. So we
    only need to worry about $\exp(\delta)\leq\varepsilon/\exp(x_{0})+1$.
    Choosing $\delta=\ln\big(1+\varepsilon/\exp(x_{0})\big)$ does the trick.
    \par\hfill\par
    This effort was to find a candidate $\delta$. Now that we've
    found one, let's show that it works. Let $\varepsilon>0$. Choose
    $\delta=\ln\big(1+\varepsilon/\exp(x_{0})\big)$. If
    $|x-x_{0}|<\delta$, then
    $|x-x_{0}|<\ln\big(1+\varepsilon/\exp(x_{0})\big)$. Since $a<b$ implies
    $\exp(a)<\exp(b)$, if $|x-x_{0}|<\ln\big(1+\varepsilon/\exp(x_{0})\big)$,
    then $\exp(|x-x_{0}|)<1+\varepsilon/\exp(x_{0})$. But then
    $\exp(x_{0})(\exp(|x-x_{0}|)-1)<\varepsilon$. But:
    \begin{equation}
        \exp(x_{0})\big(\exp(x-x_{0})-1\big)
        =\exp(x_{0})-\exp(x_{0})
    \end{equation}
    \newpage
    I, the copyright holder of this work, release it into the public domain.
    This applies worldwide. In some countries this may not be legally possible;
    if so: I grant anyone the right to use this work for any purpose, without
    any conditions, unless such conditions are required by law.
    \par\hfill\par
    The source code used to generate this document is free software and released
    under version 3 of the GNU General Public License.
\end{document}
