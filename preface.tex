This work contains mathematics and physics. There are no exercises, but rather
an abundance of worked out examples and an attempt was made to prove
every claim in a logical and consistent order. The goal was to mimic (but not
reproduce, of course) the style of Euclid's \textit{Elements} where every line
in a proof is justified by axiom, definition, or a previous theorem. As such,
there are no logical prerequesites to read the theorems and proofs, but the
examples that are used to build intuition often presume a belief in the
existence of real numbers (in particular, the non-negative integers and rational
numbers), and some of the motivating examples also use calculus and the
elementary algebra of a polynomial in one real variable. Both of these concepts
are, eventually, formally developed, but for pedagogical reasons many
examples use these notions beforehand (indeed, calculus is not developed until
Book~\ref{book:Analysis}!). Theorems and proofs do not rely on examples, and in
this sense there are no prerequesites. A reader lacking calculus may simply
find no motivation in many definitions and axioms, but should be able to follow
along the proofs in the order presented.
\par\hfill\par
This is not a textbook (or collection thereof) in the usual sense in that, as
mentioned previously, all claims are worked out in full. There are no steps that
are \textit{left as an exercise to the reader}. This can still be used as a
textbook if the reader simply reads the claim of a theorem and tries to prove
it first before reading onwards. Since it is all too tempting to allow ones eyes
to wander all of 2 inches to the solution, many excellent textbooks for various
topics are cited in the bibliography. Thus this work can be seen as a supplment
to these, or as a standalone.
\par\hfill\par
This work is very much a work in progress and will remain so for many years, do
in part to the sheer scope of the project. Any and all suggestions, corrections,
and improvements are welcome and the source code is hosted on GitHub%
\footnote{\url{https://github.com/ryanmaguire/Mathematics-and-Physics}} under
the GNU GPL 3 license. My only wish is that this material is not
\textit{stolen} in the sense that one claims the work there own, but all of the
code is freely available and may be used by anyone. This includes all of the
tikz code for reproducing figures.%
\footnote{\url{https://github.com/ryanmaguire/%
               Mathematics-and-Physics/tree/master/tikz}}
\begin{flushright}
    Ryan Maguire,\\
    Lowell, MA\\
\end{flushright}