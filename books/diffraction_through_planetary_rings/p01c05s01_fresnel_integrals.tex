\section{The Fresnel Integrals}
    The Fresnel Sine and Cosine Integrals, which are
    usually denoted $S(x)$ and $C(x)$, respectively,
    occur naturally in the study of diffraction theory.
    By examining the \textit{Fresnel Kernel},
    and using a Taylor series approximation, one
    comes across the following integral:
    \begin{equation}
        F(x)=\int_{0}^{x}\exp(it^{2})\diff{t}
    \end{equation}
    Using Euler's Theorem, we can rewrite this as:
    \begin{equation}
        F(x)=\int_{0}^{x}\cos(t^{2})\diff{t}
            +i\int_{0}^{x}\sin(t^{2})\diff{t}
    \end{equation}
    The Fresnel Cosine and Sine Integrals are defined
    as the real and
    imaginary parts of this equation, respectively.
    \begin{ldefinition}{Fresnel Integrals}
        The Fresnel Sine and Fresnel Cosine, denoted
        $S(x)$ and $C(x)$, respectively, are real valued
        functions defined by:
        \par\hfill\par
        \vspace{-1ex}
        \begin{subequations}
            \begin{minipage}{0.49\textwidth}
                \begin{equation}
                    S(x)=\int_{0}^{x}
                    \sin(t^{2})\diff{t}
                \end{equation}
            \end{minipage}
            \hfill
            \begin{minipage}{0.49\textwidth}
                \begin{equation}
                    C(x)=\int_{0}^{x}
                    \cos(t^{2})\diff{t}
                \end{equation}
            \end{minipage}
        \end{subequations}
        \par
    \end{ldefinition}
    Graph of the Fresnel Sine and Fresnel Cosine
    functions are shown in
    Fig.~\ref{fig:Diff_Theory_Graphs_of_Sinx2_and_Cosx2}.
    \begin{figure}[H]
        \captionsetup{type=figure}
        \centering
        \begin{subfigure}[b]{0.49\textwidth}
            \centering
            \resizebox{\textwidth}{!}{%
                \includegraphics{Fresnel_Cos.pdf}
            }
            \subcaption{Graphs of $\cos(x^{2})$ and $C(x)$.}
        \end{subfigure}
        %\par\hfill\par
        \begin{subfigure}[b]{0.49\textwidth}
            \centering
            \resizebox{\textwidth}{!}{%
                \includegraphics{Fresnel_Sin.pdf}
            }
            \subcaption{Graphs of $\sin(x^{2})$ and $S(x)$.}
        \end{subfigure}
        \caption[Fresnel Integrals]
            {Graphs of the Fresnel Sine and Cosine
             integrals, along with their respective derivatives.}
        \label{fig:Diff_Theory_Graphs_of_Sinx2_and_Cosx2}
    \end{figure}
    We will be interested in functions
    of the form $\exp(i\psi)$ later on. The Fresnel
    Approximation uses the Taylor expansion of $\psi$
    up to the quadratic term, and hence we will see
    something of the form $\exp(i(a+bx+cx^2))$. From
    elementary algebra we can complete the square, and
    do a change of variables to obtain
    $\exp(i(u^{2}-d^{2}))$, where $d$ is some constant.
    We are interested in the integral of this across the
    entire real line. From Euler's Formula
    (Thm.~\ref{thm:Euler_Expo_Formula}) we
    see that $\exp(ix^{2})=\cos(x^{2})+i\sin(x^{2})$.
    But $x^{2}$ grows rapidly,
    and thus $\sin(x^{2})$ and $\cos(x^{2})$ are two
    rapidly oscillating functions. The oscillation are
    so rapid that the areas cancel out, and hence
    $S(x)$ and $C(x)$ are well defined as
    $x\rightarrow\infty$. We will use Cauchy's Integral
    Theorem to evaluate the limits of these two functions.
    First, a result from Gauss.
    \begin{theorem}
        \begin{equation}
            \int_{\minus\infty}^{\infty}
                \exp(\minus{x}^{2})\diff{x}
            =\sqrt{\pi}
        \end{equation}
    \end{theorem}
    \begin{proof}
        Convergence can be shown, since for
        all $x\in\mathbb{R}$:
        \begin{equation}
            0<\exp(\minus{x}^{2})
            \leq\frac{1}{1+x^2}
        \end{equation}
        And therefore:
        \begin{equation}
            0\leq\int_{\minus\infty}^{\infty}
            \exp(\minus{x}^{2})\diff{x}\leq
            \int_{\minus\infty}^{\infty}
            \frac{1}{1+x^2}\diff{x}
            =\tan^{\minus{1}}(x)
            \Big|_{\minus\infty}^{\infty}=\pi
        \end{equation}
        Define the following:
        \begin{equation}
            \mathcal{I}=\int_{\minus\infty}^{\infty}
            \exp(\minus{x}^{2})\diff{x}
        \end{equation}
        Squaring $\mathcal{I}$, we obtain:
        \begin{subequations}
            \begin{align}
                \mathcal{I}^{2}&=
                \bigg(\int_{\minus\infty}^{\infty}
                \exp(\minus{x}^{2})\diff{x}\bigg)
                \bigg(\int_{\minus\infty}^{\infty}
                \exp(\minus{y}^{2})\diff{y}\bigg)\\
                &=\int_{\minus\infty}^{\infty}
                \int_{\minus\infty}^{\infty}
                \exp\big(\!\minus\!(x^{2}+y^{2})\big)
                    \diff{x}\diff{y}
            \end{align}
        \end{subequations}
        Switching from Cartesian to
        Polar coordinates, we have:
        \begin{equation}
            \mathcal{I}^{2}=
            \int_{0}^{2\pi}\int_{0}^{\infty}
            r\exp(\minus{r}^{2})\diff{r}\diff{\phi}
            =2\pi\int_{0}^{\infty}
            r\exp(\minus{r}^{2})\diff{r}
        \end{equation}
        This final integral can be computed from basic
        methods one would find in a Calculus textbook.
        Letting $u=r^{2}$, we have
        $\diff{u}=2r\diff{r}$,
        so the integral becomes:
        \begin{equation}
            \mathcal{I}^{2}=\pi\int_{0}^{\infty}
                \exp(\minus{u})\diff{u}
            =\pi
        \end{equation}
        Therefore $\mathcal{I}=\pm\sqrt{\pi}$.
        But $\mathcal{I}>0$, and thus
        $\mathcal{I}=\sqrt{\pi}$.
    \end{proof}
    This result has many fundamental applications in
    probability theory and in statistics, where
    it is used to define the normal distribution.
    For us, we can use this to evaluate the limits of
    $S(x)$ and $C(x)$ as $x\rightarrow\infty$.
    First note, that since $\exp(-x^{2})$ is an even
    function, the integral on $[0,\infty)$ is half of
    that of the integral on the entire real line. That is:
    \begin{equation}
        \int_{0}^{\infty}\exp(\minus{t}^{2})\diff{t}
        =\frac{\sqrt{\pi}}{2}
    \end{equation}
    We now evaluate the complex version of this.
    \begin{theorem}
        \begin{equation}
            \int_{0}^{\infty}\exp(ix^{2})\diff{x}
            =\sqrt{\frac{\pi}{8}}(1+i)
        \end{equation}
    \end{theorem}
    \begin{proof}
        For let $C_{R}$ be the closed path in the complex plane
        defined by:
        \begin{equation}
            C_{R}(t)=
            \begin{cases}
                3Rt,&0\leq{t}\leq\frac{1}{3}\\
                R\exp\big(i\frac{3\pi}{4}(t-\frac{1}{3})\big),
                &\frac{1}{3}<t<\frac{2}{3}\\
                \frac{3R}{\sqrt{2}}(1+i)(1-t),
                &\frac{2}{3}\leq{t}\leq{1}
            \end{cases}
        \end{equation}
        Then, for all $R>0$, $C_{R}$ is a Jordan Curve in the
        complex differentiable at all but three points. Thus,
        by Cauchy's Theorem, as $\exp(iz^{2})$ is an entire
        function:
        \begin{equation}
            \oint_{C_{R}}\exp(iz^{2})\diff{z}
            =0
        \end{equation}
        But then:
        \begin{equation}
            \begin{split}
                \int_{0}^{R}\exp(ix^{2})\diff{x}+
                \int_{\frac{1}{3}}^{\frac{2}{3}}&
                    \exp(iz(t)^{2})C_{R}'(t)\diff{t}\\
                &+\frac{1+i}{\sqrt{2}}\int_{R}^{0}\exp(-x^{2})\diff{x}
                =0
            \end{split}
        \end{equation}
        But by Jordan's Lemma, this second integral tends to zero as
        $R\rightarrow\infty$. Therefore:
        \begin{align}
            \int_{0}^{\infty}\exp(-ix^{2})\diff{x}
            &=-\frac{1+i}{\sqrt{2}}\int_{\infty}^{0}\exp(-x^{2})\diff{x}\\
            &=\frac{1+i}{\sqrt{2}}\int_{0}^{\infty}\exp(-x^{2})\diff{x}\\
            &=\frac{1+i}{\sqrt{2}}\frac{\sqrt{\pi}}{2}\\
            &=(1+i)\sqrt{\frac{\pi}{8}}
        \end{align}
    \end{proof}
    \begin{theorem}
        If $S$ and $C$ are the Fresnel Sine and Cosine integrals,
        respectively, then:
        \begin{align}
            \underset{x\rightarrow\infty}{\lim}S(x)
            &=\sqrt{\frac{\pi}{8}}\\
            \underset{x\rightarrow\infty}{\lim}C(x)
            &=\sqrt{\frac{\pi}{8}}
        \end{align}
    \end{theorem}
    \begin{proof}
        For:
        \begin{align}
            \underset{x\rightarrow\infty}{\lim}
                \big(C(x)+iS(x)\big)
            &=\underset{x\rightarrow\infty}{\lim}
                \int_{0}^{x}\exp(ix^{2})\diff{x}\\
            &=\sqrt{\frac{\pi}{8}}(1+i)
        \end{align}
        Comparing real and imaginary parts complex the proof.
    \end{proof}
    \begin{figure}[H]
        \centering
        \captionsetup{type=figure}
        \includegraphics{complex_plane_fresnel_integral_path.pdf}
        \caption{Jordan Curve Used to Evaluate the Fresnel Integrals.}
        \label{fig:Jordan_Curve_Fresnel_Integrals}
    \end{figure}
    \par\hfill\par
    \begin{theorem}
        If $F$ is a positive real number and
        $f:\mathbb{R}\rightarrow\mathbb{C}$ is defined by:
        \begin{equation}
            f(\rho)=
                \exp\bigg(
                    i\frac{\pi}{2}
                    \Big(\frac{\rho}{F}\Big)^{2}
                \bigg)
        \end{equation}
        Then:
        \begin{equation}
            \mathcal{F}_{\xi}(f)
            =(1+i)F\exp(\minus{2}\pi{i}F^{2}\xi^{2})
        \end{equation}
    \end{theorem}
    \begin{proof}
    For:
    \begin{align}
        \mathcal{F}_{\xi}(f)
        &=\int_{-\infty}^{\infty}
            \exp\bigg(
                i\frac{\pi}{2}
                \Big(\frac{\rho}{F}\Big)^{2}
            \bigg)
            \exp(-2\pi{i}\rho\xi)\diff{\rho}\\
        &=\int_{-\infty}^{\infty}
            \exp\bigg(
                i\frac{\pi}{2}\Big(
                    \frac{\rho}{F}
                \Big)^{2}-2\pi{i}\rho\xi
            \bigg)
            \diff{\rho}\\
        &=\int_{-\infty}^{\infty}
            \exp\Big(
                \frac{i\pi}{2F^2}
                \big[\rho^2-4F^2\rho \xi\big]
            \Big)\diff{\rho}
    \end{align}
    Completing the square, we get
    $(\rho-2F^{2}\xi)^{2}-4F^{4}\xi^{2}$.
    So, the integral becomes:
    \begin{equation}
        \begin{split}
            \int_{-\infty}^{\infty}
            \exp&\Big(
                i\frac{\pi}{2F^2}
                \big[\rho-2F^{2}\xi\big]^{2}
            \Big)
            \exp(-2\pi{i}F^{2}\xi^{2})\diff{\rho}\\
            &=\exp(-2\pi{i}F^{2}\xi^{2})
            \int_{-\infty}^{\infty}
            \exp\Big(
                i\frac{\pi}{2F^2}
                \big[\rho-2F^{2}\xi\big]^{2}
            \Big)\diff{\rho}
        \end{split}
    \end{equation}
    Let $u=\frac{\rho-2F^{2}\xi}{F}$, so then
    $F\diff{u}=\diff{\rho}$. We obtain:
    \begin{equation}
        \mathcal{F}_{\xi}(f)
        =F\exp(\minus{2}\pi{i}F^{2}\xi^{2})
            \int_{-\infty}^{\infty}
            \exp\Big(i\frac{\pi}{2}s^{2}\Big)\diff{s}
    \end{equation}
    But this integral is $1+i$, completing the proof.
    \end{proof}
    \begin{theorem}
    $\mathcal{F}(e^{-i\frac{\pi}{2}\big(\frac{\rho_0}{F}\big)^2}\big) = (1-i)Fe^{2\pi i F^2 \xi^2}$.
    \end{theorem}
    \begin{proof}
    For:
    \begin{align}
        \mathcal{F}_{\xi}(f)
        &=\int_{-\infty}^{\infty}
            \exp\bigg[
                \minus{i}\frac{\pi}{2}
                \Big(\frac{\rho}{F}\Big)^2
            \bigg]
            \exp(\minus{2}\pi{i}\rho\xi)\diff{\rho}\\
        &=\int_{-\infty}^{\infty}
            \exp\Big(
                {\!}\minus{\!}\frac{i\pi}{2F^{2}}
                \big[
                    {\rho}^{2}+4F^{2}\rho\xi
                \big]\Big)\diff{\rho}\\
        &=\int_{-\infty}^{\infty}
            \exp\Big(
                {\!}\minus{\!}\frac{i\pi}{2F^{2}}
                \big[
                    (\rho+2F^{2}\xi)^2-4F^{4}\xi^{2}
                \big]
            \Big)\diff{\rho}\\
        &=\exp(2\pi{i}F^{2}\xi^{2})\int_{-\infty}^{\infty}
            \exp\Big(
                {\!}\minus{\!}\frac{i\pi}{2F^2}
                (\rho_0+2F^2\xi)
            \Big)\diff{\rho}
    \end{align}
    Let $u = \frac{\rho_0 + 2F^2 \xi}{F}$, then
    $Fdu = d\rho_0$, so we have$Fe^{2\pi i F^2 \xi^2} \int_{-\infty}^{\infty} e^{-i\frac{\pi}{2}u^2}du$.
    Let $u = -is$, then $du = -ids$, and $u^2 = -s^2$. So
    we have $-i e^{2\pi i F^2 \xi^2}\int_{-\infty}^{\infty} e^{i\frac{\pi}{2}s^2}ds$.
    But this integral is $1+i$. So, we have
    $-iFe^{2\pi iF^2\xi^2}(1+i)=(1-i)Fe^{2\pi iF^2 \xi^2}$.
    \end{proof}
    \begin{theorem}
        If $f:\mathbb{R}\rightarrow\mathbb{R}$ and
        $g:\mathbb{R}\rightarrow{R}$ are integrable,
        and if $f*g$ is the convolution of $f$ with
        respect to $g$:
        \begin{equation}
            f*g=\int_{-\infty}^{\infty}
                f(\tau)g(\tau-t)\diff{\tau}
        \end{equation}
        Then:
        \begin{equation}
            \mathcal{F}_{\xi}\big(f*g\big)
            =\mathcal{F}_{\xi}(f)\cdot\mathcal{F}_{\xi}(g)
        \end{equation}
    \end{theorem}
    \begin{proof}
    Let $\int_{-\infty}^{\infty} |f(t)|dt = \norm{f}_{1}$ and $\int_{-\infty}^{\infty} |g(t)|dt = \norm{g}_{1}$. Then:
    \begin{align*}
        \int_{-\infty}^{\infty}\int_{-\infty}^{\infty}
            |f(\tau)g(\tau-t)|\diff{\tau}\diff{t}
        &\leq\int_{-\infty}^{\infty}
            |f(\tau)|\int_{-\infty}^{\infty}
            |g(\tau-t)|\diff{\tau}\diff{t}\\
        &=\int_{\infty}^{\infty}
            |f(x)|\norm{g}_{1}\diff{x}\\
        &=\norm{f}_{1}\norm{g}_{1}
    \end{align*}
    Thus, $h(t)=f*g$ is such that
    $\int_{-\infty}^{\infty} |h(t)|dt < \infty$.
    Let $H(\xi) = \mathcal{F}(h)$. Then:
    \begin{align}
        H(\xi)&=
            \int_{-\infty}^{\infty}
            h(t)\exp(\minus{2}\pi{i}t\xi)\diff{t}\\
        &=\int_{-\infty}^{\infty}
            \Bigg(
                \int_{-\infty}^{\infty}
                f(\tau)g(t-\tau)\diff{\tau}
            \Bigg)\exp(\minus{2}\pi{i}t\xi)\diff{t}
    \end{align}
    But:
    \begin{equation}
        |e^{-2\pi{i}t\xi}f(\tau)g(t-\tau)|
        =|f(\tau)g(t-\tau)|
    \end{equation}
    But this is simply the integrand of $h$, and $h$
    is integrable. Thus, by Fubini's Theorem we may
    swap the integrals. Let $y=t-\tau$. Then:
    \begin{subequations}
        \begin{align}
            H(\xi)&=
                \int_{-\infty}^{\infty}f(\tau)
                \int_{-\infty}^{\infty}g(t-\tau)
                e^{-2\pi{i}t\xi}\diff{t}\diff{\tau}\\
            &=\int_{-\infty}^{\infty}f(\tau)
                e^{-2\pi{i}\tau\xi}\diff{\tau}
                \int_{-\infty}^{\infty}g(y)
                e^{-2\pi{i}y\xi}dy\\
            &=\mathcal{F}(f)\cdot\mathcal{F}(g)
                \vphantom{\int_{-\infty}^{\infty}}
        \end{align}
    \end{subequations}
    Therefore, etc.
    \end{proof}
    \begin{theorem}
        If $T:\mathbb{R}\rightarrow\mathbb{C}$ is
        a Lebesgue integrable function, and if
        $\hat{T}:\mathbb{R}\rightarrow\mathbb{C}$
        is defined by:
        \begin{equation}
            \hat{T}(\rho_0)=
            \frac{1-i}{2F}\int_{\minus\infty}^{\infty}T(\rho)\exp\Big(
                i\frac{\pi}{2}\big(\frac{\rho-\rho_0}{F}\big)^{2}\Big)
            \diff{\rho}
        \end{equation}
        then:
        \begin{equation}
            T(\rho)=\frac{1+i}{2F}
                \int_{\minus\infty}^{\infty}\hat{T}(\rho_{0})\exp\Big(
                \minus{i}\frac{\pi}{2}\big(\frac{\rho-\rho_0}{F}\big)^{2}
                \Big)\diff{\rho_{0}}
        \end{equation}
    \end{theorem}
    \begin{proof}
        For by the definition of $\hat{T}$, and by the
        definition of convolution, we have:
        \begin{equation}
            \hat{T}(\rho)
            =\frac{1-i}{2F}\Big[
                T*\exp\Big(i\frac{\pi}{2}\big(\frac{\rho}{F}\big)^{2}
            \Big)\Big]
        \end{equation}
        But $T$ is a Lebesgue integrable function, and thus by
        the convolution theorem:
        \begin{subequations}
            \begin{align}
                \mathcal{F}(\hat{T})
                &=\frac{1-i}{2F}\mathcal{F}(T)\cdot
                    \mathcal{F}\Big(
                        \exp\Big[i\frac{\pi}{2}
                        \big(\frac{\rho_{0}}{F}\big)^{2}
                    \Big]\Big)\\
                &=\frac{1-i}{2F}\mathcal{F}(T)\cdot
                    (1+i)F\big(
                        \exp(\minus{2}\pi{i}F^{2}\xi^{2}
                    \big)\\
                &=\mathcal{F}(T)\exp(\minus{2}\pi{i}F^{2}\xi^{2})\\
                    \Rightarrow
                    \mathcal{F}(\hat{T})
                    \exp\big(2\pi{i}F^{2}\xi^{2}\big)
                &=\mathcal{F}(T)
            \end{align}
        \end{subequations}
        But the Fourier transform of a Gaussian is
        another Gaussian. That is:
        \begin{subequations}
            \begin{align}
                \exp\big(2\pi{i}F^{2}\xi^{2}\big)
                &=\frac{1}{(1-i)F}\mathcal{F}\Big(
                    \exp\Big[\minus{i}\frac{\pi}{2}
                    \big(\frac{\rho_{0}}{F}\big)^{2}\Big]\Big)\\
                &=\frac{1+i}{2F}\mathcal{F}\Big(
                    \exp\Big[\minus{i}\frac{\pi}{2}
                        \big(\frac{\rho_0}{F}\big)^{2}
                    \Big]\Big)
            \end{align}
        \end{subequations}
        Therefore:
        \begin{subequations}
            \begin{align}
                \mathcal{F}(T)
                &=\frac{1+i}{2F}\mathcal{F}(\hat{T})
                    \cdot\mathcal{F}
                    \big(e^{-i\frac{\pi}{2}
                    \big(\frac{\rho_0}{F}\big)^2}\big)\\
                &=\frac{1+i}{2F}\mathcal{F}
                    (\hat{T}*e^{-i\frac{\pi}{2}
                    \big(\frac{\rho_0}{F}\big)^2})\\
            &=\mathcal{F}\bigg(\frac{1+i}{2F}
                \int_{\minus\infty}^{\infty}\hat{T}(\rho_{0})
                \exp\Big[\minus{i}\frac{\pi}{2}\big(
                    \frac{\rho-\rho_{0}}{F}\big)^{2}\Big]
                \diff{\rho_{0}}\bigg)
            \end{align}
        \end{subequations}
        Therefore, by the uniqueness of the Fourier Transform:
        \begin{equation}
            T(\rho)=\frac{1+i}{2F}\int_{-\infty}^{\infty}
                \hat{T}(\rho_{0})\exp\Big[
                    \!\minus\!\frac{i\pi}{2}
                    \big(\frac{\rho-\rho_{0}}{F}\big)^{2}
                \Big]\diff{\rho_{0}}
        \end{equation}
        Therefore, etc.
    \end{proof}
    \begin{theorem}
            If $T,\psi\in{L}^{2}(\mathbb{R})$, and if
            $\hat{T}:\mathbb{R}\rightarrow\mathbb{C}$
            is defined by:
            \begin{equation}
            T(\rho_{0})=\int_{\minus\infty}^{\infty}
                T(\rho)\exp\big(i\psi(\rho_{0}-\rho)\big)
                \diff{\rho_{0}}
            \end{equation}
            then:
            \begin{equation}
                T(\rho)=\mathcal{F}^{\minus{1}}_{\rho}\Big(
                    \frac{\mathcal{F}(\hat{T})}
                        {\mathcal{F}\big(\exp(i\psi)\big)}
                    \Big)
            \end{equation}
    \end{theorem}
    \begin{proof}
        For $\hat{T}(\rho_{0})=T*\exp(i\psi)$. But then:
        \begin{equation}
            \mathcal{F}_{\xi}(\hat{T})
            =\mathcal{F}_{\xi}\big(T*\exp(i\psi)\big)
            =\mathcal{F}_{\xi}\big(T\big)\cdot
            \mathcal{F}\big(\exp(i\psi)\big)
        \end{equation}
        So then:
        \begin{equation}
            \mathcal{F}_{\xi}(T)
            =\frac{\mathcal{F}(\hat{T})}
                {\mathcal{F}\big(\exp(i\psi)\big)}
        \end{equation}
        From the uniqueness of Fourier transforms, we
        obtain the result.
    \end{proof}
