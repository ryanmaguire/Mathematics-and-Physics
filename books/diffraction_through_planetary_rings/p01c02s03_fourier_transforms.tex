\section{Fourier Transforms}
    Suppose you are asked to compute $z=x/y$ for
    two non-zero real numbers $x$ and $y$. We could perform
    long-hand division, or transform the problem into
    subtraction by using the natural logarithm.
    \begin{equation}
        z=x/y\Rightarrow
        \ln(z)=\ln(x/y)\Rightarrow
        \ln(z)=\ln(x)-\ln(y)
    \end{equation}
    Provided that $\ln(x)$ and $\ln(y)$ are somehow known,
    one can compute the difference and then compute $z$
    by exponentiating the result. In a similar manner, the
    Fourier transform is often introduced as a tool for
    converting one problem into another.
    \begin{ldefinition}{Fourier Transform}{Fourier_Transform}
        The Fourier transform of a complex valued integrable function
        $f:\mathbb{R}\rightarrow\mathbb{C}$ is the function
        $\mathcal{F}_{\xi}(f):\mathbb{R}\rightarrow\mathbb{C}$ defined by:
        \begin{equation}
            F_{\xi}(f)=\int_{\minus\infty}^{\infty}f(t)
                \exp(\minus{2}\pi{i}\xi{t})\diff{t}
        \end{equation}
        This is also called the \textit{spectrum} of $f$.
    \end{ldefinition}
    The requirement that $f$ be integrable is to avoid
    strange issues in mathematics. For the sake of
    physical application, one may assume every function
    is integrable. Mathematically this is far from true,
    but oh well. For the sake of Fourier Analysis, when
    we say integrable we mean Lebesgue integrable. This
    simply means that:
    \begin{equation}
        \int_{\minus\infty}^{\infty}|f(x)|\diff{x}<\infty
    \end{equation}
    \begin{lexample}{}{FT_Hat_Func}
        Consider the hat function:
        \begin{equation}
            f(t)=
            \begin{cases}
                0,&|t|\leq{1}\\
                1,&|t|>1
            \end{cases}
        \end{equation}
        We can compute the Fourier transform of the this
        explicitly:
        \begin{equation}
            \mathcal{F}_{\xi}(f)
            =\int_{\minus\infty}^{\infty}f(t)
                \exp(\minus{2}\pi{i}\xi{t})\diff{t}
            =\int_{\minus{1}}^{1}
                \exp(\minus{2}\pi{i}\xi{t})\diff{t}
        \end{equation}
        Here we invoke Euler's Theorem,
        Thm.~\ref{thm:Euler_Expo_Formula},
        and note that the integral has symmetric limits.
        But $\sin$ is an \textit{odd} function, and thus
        it's integral is zero, and $\cos$ is an even
        function. Thus, we are left with:
        \begin{equation}
            \mathcal{F}_{\xi}(f)=
            2\int_{0}^{1}\cos(2\pi{i}\xi{t})\diff{t}
            =\frac{\sin(2\pi\xi)}{\pi\xi}
        \end{equation}
        We can be even more general, defining:
        \begin{equation}
            f(x)=
            \begin{cases}
                1,&a\leq{x}\leq{b}\\
                0,&\textrm{Otherwise}
            \end{cases}
        \end{equation}
        The Fourier transform is then:
        \begin{equation}
            \mathcal{F}_{\xi}(f)=
            \frac{i\big(\exp(\minus{2}\pi{i}\xi{b})-
                    \exp(\minus{2}\pi{i}\xi{a}\big)}{2\pi\xi}
        \end{equation}
        Thus we see that the range of the Fourier transform
        generally lies in the complex plane. Only with
        sufficient symmetry does the problem collapse down
        to $\mathbb{R}$.
    \end{lexample}
    Recall that a complex number has a polar representation
    $z=r\exp(i\theta)$. Similarly, for a complex valued
    function we can write:
    \begin{equation}
        f(z)=R(r)\exp\big(i\Theta(\theta)\big)
    \end{equation}
    For the Fourier transform of a function, the function
    $R(r)$ is called the principle amplitude, and
    $\Theta(\theta)$ is the \textit{phase offset} from
    this amplitude. The Fourier transform of the hat
    function defined from $\minus{1}$ to $1$ is plotted
    in Fig.~\ref{fig:Diff_Theory_FT_of_Hat_Func}.
    \begin{figure}[H]
        \centering
        \captionsetup{type=figure}
        \begin{subfigure}[b]{0.49\textwidth}
            \centering
            \resizebox{\textwidth}{!}{%
                \includegraphics{fourier_transform_square_well.pdf}
            }
            \subcaption{Time Domain.}
        \end{subfigure}
        \begin{subfigure}[b]{0.49\textwidth}
            \centering
            \resizebox{\textwidth}{!}{%
                \includegraphics{fourier_transform_sinc_function.pdf}
            }
            \subcaption{Frequency Domain.}
        \end{subfigure}
        \caption{Fourier Transform of the Hat Function.}
        \label{fig:Diff_Theory_FT_of_Hat_Func}
    \end{figure}
    \begin{lexample}{}{FT_Decaying_Expo}
        Let $f:\mathbb{R}\rightarrow\mathbb{R}$ be defined by:
        \begin{equation}
            f(t)=
            \begin{cases}
                \beta\exp(\minus\alpha{t}),&t\geq{0}\\
                0,&t<0
            \end{cases}
        \end{equation}
        Where $\alpha$ and $\beta$ are positive real numbers.
        Since $f(t)$ decays to zero rapidly as
        $t\rightarrow\infty$, we see that $f$ is a
        Lebesgue integrable function and has a Fourier
        transform. We can compute this using the standard
        methods obtained from a course on integral calculus.
        \begin{equation}
            \mathcal{F}_{\xi}(f)=\int_{\minus\infty}^{\infty}f(t)
                                    \exp(\minus{2}\pi{i}\xi{t})\diff{t}
                                =\beta\int_{0}^{\infty}
                                    \exp(\minus\alpha{t})
                                    \exp(\minus{2}\pi{i}\xi{t})\diff{t}
        \end{equation}
        Using the product rule for exponents, we can
        reduce this to the following line integral:
        \begin{equation}
            \mathcal{F}_{\xi}(f)=
            \beta\int_{0}^{\infty}\exp\big(
                \minus(\alpha+2\pi{i}\xi)t\big)\diff{t}
        \end{equation}
        Thus, we are integrating the exponential function
        along the line $z=\theta{t}$, where
        $\theta=\tan^{\minus{1}}(2\pi\xi/\alpha)$.
        Since $\alpha$ is positive, we can use the result
        from Jordan's Lemma to obtain the solution:
        \begin{equation}
            \mathcal{F}_{\xi}(f)
            =\frac{\beta}{\alpha+2\pi{i}\xi}
        \end{equation}
        There are two ways to view the Fourier
        transform: The Cartesian form and the
        polar form. As was stated before, the
        polar form represents the \textit{amplitude}
        and \textit{phase offset} of the Fourier
        transform, whereas the Cartesian form simply
        represents the real and imaginary parts. To
        compute the Cartesian form, we simply invoke
        Eqn.~\ref{eqn:Mult_Inv_of_Complex} for the inverse
        of a complex number, and obtain:
        \begin{equation}
            \mathcal{F}_{\xi}(f)
            =\beta\frac{\alpha-2\pi{i}\xi}
                {\alpha^{2}+4\pi^{2}\xi^{2}}
        \end{equation}
        For the polar form, we invoke
        Thm.~\ref{thm:Polar_Form_Comp_Num},
        take the modulus of $\mathcal{F}_{\xi}(f)$
        and compute inverse tangents:
        \begin{equation}
            \mathcal{F}_{\xi}(f)=
            \frac{\beta}{\sqrt{\alpha^{2}+4\pi^{2}\xi^{2}}}
            \exp\Big[i\tan^{\minus{1}}
                \Big(\frac{\minus{2}\pi\xi}{\alpha}\Big)
            \Big]
        \end{equation}
        The two are plotted below for the case of
        $\alpha=\beta=1$.
    \end{lexample}
    \begin{ldefinition}{Inverse Fourier Transform}
          {Diff_Theory_Inverse_Fourier_Transform}
        The inverse Fourier transform of a complex valued
        integrable function
        $f:\mathbb{R}\rightarrow\mathbb{C}$ is the function
        $\mathcal{F}_{t}^{\minus{1}}(f):%
            \mathbb{R}\rightarrow\mathbb{C}$ defined by:
        \begin{equation}
            \mathcal{F}_{t}^{\minus{1}}(f)=
            \int_{\minus{\infty}}^{\infty}
                f(\xi)\exp(2\pi{i}\xi{t})\diff{\xi}
        \end{equation}
    \end{ldefinition}
    We now prove what is probably the most useful theorem
    in Fourier Analysis.
    \begin{theorem}
        If $f:\mathbb{R}\rightarrow\mathbb{R}$ is a
        continuous Lebesgue integrable function,
        and if its spectrum $F$ is also continuous
        and Lebesgue integrable, then:
        \begin{equation}
            f(t)=\int_{-\infty}^{\infty}F(\omega)
            \exp(2\pi{i}\omega{t})\diff{\omega}
        \end{equation}
    \end{theorem}
    A powerful application of this is
    Shannon's Sampling Theorem.
    \begin{theorem}[Shannon's Sampling Theorem]
        If $f(t)$ is a continuous
        Lebesgue integrable function such that
        its spectrum $F(\omega)$ is differentiable and zero
        outside the interval $[-W,W]$,
        then $f(t)$ is uniquely determined
        by the points $f(\frac{n}{2W})$, $n\in\mathbb{N}$.
    \end{theorem}
    \begin{proof}
        For let $F$ be the spectrum of $f$. That is:
        \begin{equation}
            f(t)=\int_{-\infty}^{\infty}F(\omega)
            \exp(-2\pi{i}\omega{t})\diff{\omega}
        \end{equation}
        But $F(\omega)=0$ for $|\omega|>W$. Thus we have:
        \begin{equation}
            f(t)=\int_{-W}^{W}F(\omega)
            \exp(-2\pi{i}\omega{t})\diff{\omega}
        \end{equation}
        Then for $n\in\mathbb{N}$, we have:
        \begin{equation}
            f\big(\frac{n}{2W}\big)=\int_{-W}^{W}F(\omega)
            \exp(-2\pi{i}\frac{n}{2W}\omega)\diff{\omega}
        \end{equation}
        But $F$ is differentiable, and thus it's Fourier
        series converges. That is:
        \begin{subequations}
            \begin{align}
                F(\omega)
                &=\sum_{n=\minus\infty}^{\infty}
                    \exp(2\pi{i}n\omega)
                \int_{\minus{W}}^{W}F(\tau)
                \exp(\minus{2}\pi{i}\frac{n}{2W}\tau)
                    \diff{\tau}\\
                &=\sum_{n=-\infty}^{\infty}f
                  \big(\frac{n}{2W}\big)e^{2\pi{i}n\omega}
            \end{align}
        \end{subequations}
        Therefore $f(\frac{n}{2W})$, $n\in \mathbb{N}$
        uniquely determines $F(\omega)$. But the
        spectrum $F(\omega)$ uniquely determines
        $f(t)$. Therefore $f(t)$ is
        uniquely determined and:
        \begin{equation}
            f(t)=\sum_{n=-\infty}^{\infty}\int_{-W}^{W}
            f\big(\frac{n}{2W}\big)
            \exp(2\pi{i}\omega(n+t))\diff{\omega}
        \end{equation}
    \end{proof}
