%-----------------------------------LICENSE------------------------------------%
%   This file is part of Mathematics-and-Physics.                              %
%                                                                              %
%   Mathematics-and-Physics is free software: you can redistribute it and/or   %
%   modify it under the terms of the GNU General Public License as             %
%   published by the Free Software Foundation, either version 3 of the         %
%   License, or (at your option) any later version.                            %
%                                                                              %
%   Mathematics-and-Physics is distributed in the hope that it will be useful, %
%   but WITHOUT ANY WARRANTY; without even the implied warranty of             %
%   MERCHANTABILITY or FITNESS FOR A PARTICULAR PURPOSE.  See the              %
%   GNU General Public License for more details.                               %
%                                                                              %
%   You should have received a copy of the GNU General Public License along    %
%   with Mathematics-and-Physics.  If not, see <https://www.gnu.org/licenses/>.%
%------------------------------------------------------------------------------%
\newglossaryentry{complex conjugate}
{
    name={Complex Conjugate},
    text={complex conjugate},
    description={A complex number formed by reflecting a
                 given complex number $z$ across the $x$
                 axis. That is, if $z=x+iy$, the complex
                 conjugate is given by:
                 \begin{equation*}
                     \overline{z}=x-iy
                 \end{equation*}
                },
    see=[See:]{complex number}
}

\newglossaryentry{complex number}
{
    name={Complex Number},
    text={complex number},
    description={An element of the Euclidean plane (Also called the
                 Cartesian plane, or just written as $\mathbb{R}^{2}$)
                 with the following arithmetic:
                 \begin{eqnarray*}
                     (a,b)+(c,d)
                     \!&\!=\!&\!(a+c,b+d)\\
                     (a,b)\cdot(c,d)
                     \!&\!=\!&\!(ac-bd,ad+bc)
                 \end{eqnarray*}
                 We often write a complex number $z=(x,y)$ as:
                 \begin{equation*}
                     z=x+iy
                 \end{equation*}
                 where $i$ is called the \textit{imaginary unit}.
                }
}

\newglossaryentry{modulus}
{
    name={Modulus},
    text={modulus},
    description={The size, or absolute value, of a complex number. This
                 is the Euclidean distance in the complex plane from the
                 point $z=x+iy$ to the origin. By Euclidean distance,
                 we mean that it is the distance that satisfies the
                 Pythagorean formula:
                 \begin{equation*}
                     |z|=\sqrt{x^{2}+y^{2}}
                 \end{equation*}
                 If $\overline{z}$ denotes the complex conjugate,
                 we can compute the modulus by
                 $|z|=\sqrt{z\overline{z}}$.
                },
    see=[See:]{complex number,complex conjugate}
}

\newglossaryentry{fresnel transform}
{
    name={Fresnel Transform},
    text={Fresnel transform},
    description={A mathematical operation that is used in
                 diffraction theory. It is defined by:
                 \begin{equation*}
                     E(\mathbf{r})=\frac{1}{i\lambda}
                     \iint_{\Sigma}\hat{E}\frac{\exp(ir)}{r}
                     \cos(\theta)\diff{A}
                 \end{equation*}
                }
}
