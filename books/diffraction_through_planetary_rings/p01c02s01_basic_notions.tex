\section{Basic Notions}
    We wish to define what it means for some function
    $f:\Omega\rightarrow\mathbb{R}$ to be \textit{integrable}, where
    $\Omega$ is some space. Any attempt at defining an integral will
    require one to start with approximations such as the following:
    \begin{equation}
        \int_{\Omega}f(x)\diff{x}\approx\sum_{n}f(x_{n})\mu(X_{n})
    \end{equation}
    Where $X_{n}$ is a bunch of sets that partition $\Omega$,
    $x_{n}\in{X}_{n}$ for all $n$, and $\mu(X_{n})$ is the \textit{size} or
    the \textit{width} of $X_{n}$. This is precisely what is done in a
    calculus course where the Riemann integral is defined. To make this
    concrete, we'll need to decide what sets $X_{n}$ are allowed to
    partition $\Omega$, and what are the properties of our
    \textit{measure} $\mu$. Throughout, $\emptyset$ is used to denote
    the empty set. This is the set with nothing in it. Our inclusion of
    this set in various definitions is for technical reasons that we won't
    often be concerned with.
    \begin{ldefinition}{$\sigma\textrm{-Algebras}$}{Sigma_Algebra}
        A $\sigma$-Algebra on a set $\Omega$ is a collection of subsets
        $\mathcal{A}$ of $\Omega$ such that:
        \begin{enumerate}
            \item It is true that $\emptyset\in\mathcal{A}$ and that
                  $\Omega\in\mathcal{A}$.
            \item For all $A\in\mathcal{A}$, it is true that the complement
                  of $A$ is in $\mathcal{A}$. That is,
                  $A^{C}\in\mathcal{A}$.
            \item For any sequence $A:\mathbb{N}\rightarrow\mathcal{A}$ of
                  sets in $\mathcal{A}$, so is their intersection:
                  \begin{equation}
                      \bigcap_{n=1}^{\infty}A_{n}\in\mathcal{A}
                  \end{equation}
        \end{enumerate}
        The elements of $\mathcal{A}$ are called the
        \textit{measurable subsets} of $\Omega$. Given a set $\Omega$, and
        a $\sigma\textrm{-Algebra}$ $\mathcal{A}$ on $\Omega$, we call
        the pair $(\Omega,\,\mathcal{A})$ a \textit{measure space}.
    \end{ldefinition}
    \begin{lexample}{}{Power_Set_and_Trivial_Sigma_Algebras}
        Given a set $\Omega$, there are two simple
        $\sigma\textrm{-Algebras}$ that one can define. Let
        $\mathcal{A}=\{\emptyset,\,\Omega\}$. This satisfies all three
        properties and is called the trivial $\sigma\textrm{-Algebra}$.
        Going in the other direction, if we let $\mathcal{A}$ be the set of
        \textit{all} subsets of $\Omega$ (Also known as the
        \textit{power set} of $\Omega$, denoted $\mathcal{P}(\Omega)$),
        then this also a $\sigma\textrm{-Algebra}$.
    \end{lexample}
    The motivation for defining measurable sets in such a way is to allow
    one to easily describe \textit{measures} and \textit{probabilities}
    later.
    \begin{ldefinition}{Borel $\sigma\textrm{-Algebra}$}{Borel_Sig_Alg}
        The Borel $\sigma\textrm{-Algebra}$ is the \textit{smallest}
        $\sigma\textrm{-Algebra}$ on $\mathbb{R}$, denoted $\mathcal{B}$,
        such that for all $a<b$, the interval $(a,b)$ is a measurable
        set. That is, $(a,b)\in\mathcal{B}$.
    \end{ldefinition}
    \begin{ldefinition}{Measures}{Measures}
        A measure on a measure space $(\Omega,\,\mathcal{A})$ is a
        function $\mu:\mathcal{A}\rightarrow\mathbb{R}$ such that:
        \begin{enumerate}
            \item For all $A\in\mathcal{A}$, $\mu(A)\geq{0}$.
            \item $\mu(\emptyset)=0$.
            \item Given a \textit{mutually disjoint} list of sets
                  $A_{1},\,A_{2},\,\dots$ that are contained in
                  $\mathcal{A}$, the following is true:
                  \begin{equation}
                      \mu\Big(\bigcup_{n=1}^{\infty}A_{n}\Big)=
                      \sum_{n=1}^{\infty}\mu(A_{n})
                  \end{equation}
        \end{enumerate}
        The triple $(\Omega,\,\mathcal{A},\,\mu)$ is called a
        \textit{measurable space}.
    \end{ldefinition}
    It is important to remember that we are trying to model \textit{size},
    and this is what motivates our definition of measure. The first rule
    says that the width, or length, or size of a set is non-negative,
    and the second rule states that the size of \textit{nothing} is simply
    zero. The last rule, which is called countable additivity, is
    very important but also intuitive. We may define the length of
    the set interval $(0,\,1)$ as $1$, and the length of
    $(a,\,b)$ to be $b-a$ (For $a<b$). What about the length of the
    \textit{union} of the intervals $(0,\,1)$ and $(3,\,4)$? Since they
    have no overlap, we may as well add the lengths of the two
    individual intervals and claim that the length of the whole is 2.
    This is precisely what countable additivity tells us.
