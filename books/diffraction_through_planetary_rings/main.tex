%-----------------------------------LICENSE------------------------------------%
%   This file is part of Mathematics-and-Physics.                              %
%                                                                              %
%   Mathematics-and-Physics is free software: you can redistribute it and/or   %
%   modify it under the terms of the GNU General Public License as             %
%   published by the Free Software Foundation, either version 3 of the         %
%   License, or (at your option) any later version.                            %
%                                                                              %
%   Mathematics-and-Physics is distributed in the hope that it will be useful, %
%   but WITHOUT ANY WARRANTY; without even the implied warranty of             %
%   MERCHANTABILITY or FITNESS FOR A PARTICULAR PURPOSE.  See the              %
%   GNU General Public License for more details.                               %
%                                                                              %
%   You should have received a copy of the GNU General Public License along    %
%   with Mathematics-and-Physics.  If not, see <https://www.gnu.org/licenses/>.%
%------------------------------------------------------------------------------%
\documentclass[oneside]{book}
%---------------------------Packages----------------------------%
\usepackage{geometry}
\geometry{b5paper, margin=1.0in}
\usepackage[T1]{fontenc}
\usepackage{graphicx, float}            % Graphics/Images.
\usepackage{natbib}                     % For bibliographies.
\bibliographystyle{agsm}                % Bibliography style.
\usepackage[french, english]{babel}     % Language typesetting.
\usepackage[dvipsnames]{xcolor}         % Color names.
\usepackage{listings, lstlinebgrd}      % Verbatim-Like Tools.
\usepackage{mathtools, esint, mathrsfs} % amsmath and integrals.
\usepackage{amsthm, amsfonts}           % Fonts and theorems.
\usepackage{tabularx}
\usepackage{tcolorbox}                  % Frames around theorems.
\usepackage{upgreek}                    % Non-Italic Greek.
\usepackage{paracol}                    % Two-column styling.
\usepackage{wrapfig}                    % Wrap text around figure.
\usepackage{fmtcount, etoolbox}         % For the \book{} command.
\usepackage[newparttoc]{titlesec}       % Formatting chapter, etc.
\usepackage{titletoc}                   % Allows \book in toc.
\usepackage[nottoc]{tocbibind}          % Bibliography in toc.
\usepackage[titles]{tocloft}            % ToC formatting.
\usepackage{multicol, enumitem}         % Multi-column/enumerate.
\usepackage{import}                     % Import external files.
\usepackage{pgfplots, tikz}             % Drawing/graphing tools.
\usetikzlibrary{
    calc,                   % Calculating right angles and more.
    angles,                 % Drawing angles within triangles.
    arrows.meta,            % Latex and Stealth arrows.
    quotes,                 % Adding labels to angles.
    positioning,            % Relative positioning of nodes.
    decorations.markings,   % Adding arrows in the middle of a line.
    patterns,
    arrows,
    shapes,
    shapes.geometric,
    cd,
    hobby,
    babel
}                                       % Libraries for tikz.
\pgfplotsset{compat=1.9}                % Version of pgfplots.
\usepackage[font=scriptsize,
            labelformat=simple,
            labelsep=colon]{subcaption} % Subfigure captions.
\usepackage[font={scriptsize},
            hypcap=true,
            labelsep=colon]{caption}    % Figure captions.
\usepackage{hyperref}                   % Allows for hyperlinks.
\hypersetup{
    colorlinks=true,
    linkcolor=blue,
    filecolor=magenta,
    urlcolor=Cerulean,
    citecolor=SkyBlue
}                           % Colors for hyperref.
\usepackage[toc,acronym,nogroupskip]{glossaries} % Glossaries and acronyms.
\usepackage[subpreambles=false]{standalone}      % Complileable sub files.

% Various font stuff from kiwi.
% Use this for Times text and Computer Modern math
%\usepackage{times}

% Quite nice
%\usepackage[charter, greekfamily=, greekuppercase=italicized]{mathdesign}
%\usepackage[utopia, greekuppercase=italicized]{mathdesign}    % Math is narrower

% Use this for Times text and math
%\usepackage{newtxtext}
%\usepackage[libertine,cmintegrals]{newtxmath}
%\usepackage{fix-cm}

%\usepackage{txfontsb}
% or
%\usepackage{mathptmx}

%\usepackage[scaled=0.92]{helvet}
%\renewcommand{\rmdefault}{ptm}

%\usepackage{mathpazo}    % add possibly `sc` and `osf` options
%\usepackage{eulervm}

%\usepackage{fourier}
%\renewcommand{\rmdefault}{ptm}
%\usepackage{mathptm}

%\usepackage{fontspec}
%\setmainfont{lmodern}

%\usepackage[varg]{txfonts}
%\usepackage{fouriernc}
%\usepackage{mathpazo}

%\usepackage{bookman}
%\usepackage[scaled]{uarial}
%\usepackage[scaled]{helvet}
%\renewcommand*\familydefault{\sfdefault}
%\usepackage[math]{anttor}

%\newcommand\fgeorgia{\fontfamily{jvn}\selectfont}
%\newcommand\ftimes{\fontfamily{ptm}\selectfont}
%\newcommand\fhelvetica{\fontfamily{phv}\selectfont}
%\newcommand\fcourier{\fontfamily{pcr}\selectfont}
%\newcommand\fbookman{\fontfamily{pbk}\selectfont}
%\newcommand\fnewcentury{\fontfamily{pnc}\selectfont}
%\newcommand\fpalatino{\fontfamily{ppl}\selectfont}
%\newcommand\favantgarde{\fontfamily{pag}\selectfont}
%\newcommand\fnormal{\normalfont}
%\newcommand\fsize[1]{\ifnum#1>0\fontsize{#1}{#1}\selectfont\else\normalsize\fi}
%------------------------Theorem Styles-------------------------%
% Define theorem style for default spacing and normal font.
\newtheoremstyle{normal}
    {\topsep}               % Amount of space above the theorem.
    {\topsep}               % Amount of space below the theorem.
    {}                      % Font used for body of theorem.
    {}                      % Measure of space to indent.
    {\bfseries}             % Font of the header of the theorem.
    {}                      % Punctuation between head and body.
    {.5em}                  % Space after theorem head.
    {}

% Define theorem style for default spacing with italicized font.
\newtheoremstyle{normalit}{\topsep}{\topsep}
                {\itshape}{}{\bfseries}{}{.5em}{}

% Italic header environment.
\newtheoremstyle{thmit}{\topsep}{\topsep}{}{}{\itshape}{}{0.5em}{}

% Define italicized environments.
\theoremstyle{normalit}
\newtheorem{theorem}{Theorem}[section]
\newtheorem{lemma}{Lemma}[section]
\newtheorem{corollary}{Corollary}[section]
\newtheorem{proposition}{Proposition}[section]
\newtheorem*{theorem*}{Theorem}

% Define environments with italic headers.
\theoremstyle{thmit}
\newtheorem*{solution}{Solution}
\newtheorem*{fsolution}{Solution}

% Define default environments.
\theoremstyle{normal}
\newtheorem{example}{Example}[section]
\newtheorem{definition}{Definition}[section]
\newtheorem{problem}{Problem}[section]
\newtheorem{question}{Question}[section]
\newtheorem{remark}{Remark}[section]
\newtheorem{properties}{Properties}[section]
\newtheorem{notation}{Notation}[section]
\newtheorem{axiom}{Axiom}[section]
\newtheorem*{properties*}{Properties}
\newtheorem*{remark*}{Remark}
\newtheorem*{definition*}{Definition}
\theoremstyle{plain}

% Define framed environment.
\tcbuselibrary{most}
\newtcbtheorem[use counter*=theorem]{ftheorem}{Theorem}%
    {colback=green!5,colframe=green!35!black,
     fonttitle=\bfseries\upshape}{th}

\newtcbtheorem[use counter*=example]{fdefinition}{Definition}%
    {fonttitle=\bfseries\upshape,
     colback=blue!5!white,colframe=blue!75!black}{def}

\newtcbtheorem[use counter*=example]{fexample}{Example}%
    {fonttitle=\bfseries\upshape,
     colback=red!5!white,colframe=red!75!black}{ex}

\newtcbtheorem[use counter*=notation]{fnotation}{Notation}%
    {fonttitle=\bfseries\upshape,
     colback=SeaGreen!5!white,colframe=SeaGreen!75!black}{ex}

\newtcbtheorem[use counter*=corollary]{fcorollary}{Corollary}%
    {fonttitle=\bfseries\upshape,
     colback=Orchid!5!white,colframe=Orchid!75!black}{ex}

\newenvironment{bproof}{\textit{Proof.}}{\hfill$\square$}
\tcolorboxenvironment{bproof}{blanker,breakable,left=5mm,
                             before skip=10pt,after skip=10pt,
                             borderline west={1mm}{0pt}{red}}
\tcolorboxenvironment{fsolution}
    {enhanced jigsaw,colframe=cyan,interior hidden,breakable}

%--------------------Declared Math Operators--------------------%
\DeclareMathOperator{\Refl}{Refl}           % Reflection operator.
\DeclareMathOperator{\Span}{Span}           % Span of a set of vectors.
\DeclareMathOperator{\Card}{Card}           % Cardinality of set.
\DeclareMathOperator{\Ord}{Ord}             % Ordinal of ordered set.
\DeclareMathOperator{\Tr}{Tr}               % Trace of matrix.
\DeclareMathOperator{\adjoint}{adj}         % Adjoint of matrix.
\DeclareMathOperator{\rk}{rk}               % Rank of operator.
\DeclareMathOperator{\nul}{nul}             % Null space of operator.
\DeclareMathOperator{\sgn}{sgn}             % Sign of a number.
\DeclareMathOperator{\multideg}{mutlideg}   % Multi-Degree (Graphs).
\DeclareMathOperator{\GCD}{GCD}             % Greatest common denominator.
\DeclareMathOperator{\LM}{LM}               % Leading monomial
\DeclareMathOperator{\LC}{LC}               % Leading coefficient.
\DeclareMathOperator{\LT}{LT}               % Leading term.
\DeclareMathOperator{\LCM}{LCM}             % Least common multiple.
\DeclareMathOperator{\Mon}{Mon}             % Monomial.
\DeclareMathOperator{\Spec}{Spec}           % Spectrum.
\DeclareMathOperator{\proj}{proj}           % Projection.
\DeclareMathOperator{\comp}{comp}           % Component.
\DeclareMathOperator{\sinc}{sinc}           % Sinc function.
\DeclareMathOperator{\Ima}{Im}              % Image of operator.
\DeclareMathOperator{\Prin}{Prin}           % Principal value.
\DeclareMathOperator{\Mod}{mod}             % Modulus.
%------------------------New Commands---------------------------%
\DeclarePairedDelimiter\norm{\lVert}{\rVert}
\DeclarePairedDelimiter\ceil{\lceil}{\rceil}
\DeclarePairedDelimiter\floor{\lfloor}{\rfloor}
\newcommand*\diff{\mathop{}\!\mathrm{d}}
\newcommand*\Diff[1]{\mathop{}\!\mathrm{d^#1}}
\renewcommand{\mod}{\ \Mod}
\renewcommand*{\glstextformat}[1]{\textcolor{RoyalBlue}{#1}}
\renewcommand{\glsnamefont}[1]{\textbf{#1}}
\renewcommand\labelitemii{$\circ$}
\renewcommand\thesubfigure{\arabic{chapter}.\arabic{figure}}
\renewcommand\thesubfigure{%
    \arabic{chapter}.\arabic{figure}.\arabic{subfigure}}
\addto\captionsenglish{\renewcommand{\figurename}{Fig.}}
%------------------------Book Command---------------------------%
\makeatletter
\renewcommand\@pnumwidth{1cm}
\newcounter{book}
\renewcommand\thebook{\@Roman\c@book}
\newcommand\book{%
    \if@openright
        \cleardoublepage
    \else
        \clearpage
    \fi
    \thispagestyle{plain}%
    \if@twocolumn
        \onecolumn
        \@tempswatrue
    \else
        \@tempswafalse
    \fi
    \null\vfil
    \secdef\@book\@sbook
}
\def\@book[#1]#2{%
    \ifnum \c@secnumdepth >-3\relax
        \refstepcounter{book}%
        \addcontentsline{toc}{book}{
            \bookname\ \thebook:\hspace{1em}#1
        }
    \else
        \addcontentsline{toc}{book}{#1}%
    \fi
    \markboth{}{}%
    {\centering
     \interlinepenalty \@M
     \normalfont
     \ifnum \c@secnumdepth >-2\relax
       \huge\bfseries \bookname\nobreakspace\thebook
       \par
       \vskip 20\p@
     \fi
     \Huge \bfseries #2\par}%
    \@endbook}
\def\@sbook#1{%
    {\centering
     \interlinepenalty \@M
     \normalfont
     \Huge \bfseries #1\par}%
    \@endbook}
\def\@endbook{
    \vfil\newpage
        \if@twoside
            \if@openright
                \null
                \thispagestyle{empty}%
                \newpage
            \fi
        \fi
        \if@tempswa
            \twocolumn
        \fi
}
\newcommand*\l@book[2]{%
    \ifnum \c@tocdepth >-2\relax
        \addpenalty{-\@highpenalty}%
        \addvspace{2.25em \@plus\p@}%
        \setlength\@tempdima{3em}%
        \begingroup
            \parindent \z@ \rightskip \@pnumwidth
            \parfillskip -\@pnumwidth
            {
                \leavevmode
                \Large \bfseries #1\hfil \hb@xt@\@pnumwidth{
                    \hss #2
                }
            }
            \par
            \nobreak
            \global\@nobreaktrue
            \everypar{\global\@nobreakfalse\everypar{}}%
        \endgroup
    \fi}
\newcommand\bookname{Book}
\renewcommand{\thebook}{\texorpdfstring{\Numberstring{book}}{book}}
\providecommand*{\toclevel@book}{-2}
\makeatother
\titlecontents{chapter}[0pt]
    {\bfseries}
    {\chaptername\ \thecontentslabel:\quad}
    {}
    {\hfill\contentspage}
\titleformat{\part}[display]
    {\Large\bfseries}
    {\partname\nobreakspace\thepart}
    {0mm}
    {\Huge\bfseries}
    \titlecontents{part}[0pt]
    {\large\bfseries}
    {\partname\ \thecontentslabel: \quad}
    {}
    {\hfill\contentspage}
\newcommand{\MarkRightAngle}[4][.3cm]
    {\coordinate (tempa) at ($(#3)!#1!(#2)$);
     \coordinate (tempb) at ($(#3)!#1!(#4)$);
     \coordinate (tempc) at ($(tempa)!0.5!(tempb)$);%midpoint
     \draw (tempa) -- ($(#3)!2!(tempc)$) -- (tempb);}
%--------------------------LENGTHS------------------------------%
% Spacings for the Table of Contents.
\addtolength{\cftsecnumwidth}{1ex}
\addtolength{\cftsubsecindent}{1ex}
\addtolength{\cftsubsecnumwidth}{1ex}
\addtolength{\cftfignumwidth}{1ex}
\addtolength{\cfttabnumwidth}{1ex}

% Spacing for multi-column and enumerate environments.
\setlength{\multicolsep}{6pt}
\setlist[enumerate]{itemsep=0pt,topsep=3pt}

% Indent and paragraph spacing.
\setlength{\parindent}{0em}
\setlength{\parskip}{0em}
\makeglossaries
\loadglsentries{glossary}
\loadglsentries{acronym}
\begin{document}
    \title{Diffraction Through Planetary Rings}
    \author{Ryan Maguire}
    \date{\vspace{-5ex}}
    \pagenumbering{roman}
    \maketitle
    \tableofcontents
    \listoffigures
    \listoftables
    \clearpage
    \input{p00c00s00_preface.tex}
    \clearpage
    \pagenumbering{arabic}
    \part{Preliminaries}
        We begin with a few important topics in mathematics that one comes
across in the study of electromagnetism and diffraction theory.
This is particularly useful for studying occultation observations
of planetary rings. We will develop complex analysis, Fourier analysis,
approximation theory, and multivariate calculus so that we may
transition from the core of electromagnetism, \textit{Maxwell's Equations},
and derive the \textit{Fresnel-Huygens Principle}. This is the fundamental
equation on which diffraction theory is based.
\par\hfill\par
There is a standard set of notations in mathematics, and this is
presented below in Table~\ref{tab:Common_Notations}.
\begin{table}[H]
    \centering
    \captionsetup{type=table}
    \begin{tabular}{|l|l|}
        \hline
        Symbol&Definition\\
        \hline
        $\mathbb{N}$&Positive Integers\\
        \hline
        $\mathbb{Z}$&Integers\\
        \hline
        $\mathbb{Z}_{n}$&Positive Integers Between 1 and $n$.\\
        \hline
        $\mathbb{Q}$&Rational numbers\\
        \hline
        $\mathbb{R}$&Real Numbers\\
        \hline
        $\mathbb{C}$&Complex Numbers\\
        \hline
    \end{tabular}
    \caption{Common Notations}
    \label{tab:Common_Notations}
\end{table}
Given some number $x$, we denote that $x$ is a real number by writing
$x\in\mathbb{R}$, and similarly if $x$ is rational we write
$x\in\mathbb{Q}$. Conversely, we write $x\notin\mathbb{R}$ to denote that
$x$ is \textit{not} a real number. The $\in$ symbol should be read
\textit{is an element of}, so $x\in\mathbb{R}$ reads as $x$
\textit{is an element of} $\mathbb{R}$. Recall that a
\textit{sequence} of points in some set $A$ is an ordered list
$a_{1}$, $a_{2}$, $\dots$, $a_{n}$, $\dots$ such that $a_{k}\in{A}$
for all $k\in\mathbb{N}$. This definition lacks rigor in many
respects, and to make the notion useful we define a sequence as
a \textit{function} from $\mathbb{N}$ to $A$. That is, we write
$a:\mathbb{N}\rightarrow{A}$. Rather than writing $a(n)$, we write
$a_{n}$. Subscript notation is reserved for sequences. For a
general function $f:X\rightarrow{Y}$, we
write $f(x)$ to denote the point in $Y$ that corresponds to $x$.

        \chapter{Complex Analysis}
    The theory of complex analysis extends the study of calculus of a
    single real variable to that of a \textit{complex} variable. The
    complex numbers have many interesting and counter-intuitive properties,
    many of which are used regularly in physics.

        \input{p01c01s01_complex_numbers.tex}
        \chapter{Fourier Analysis}
    Fourier analysis is deeply tied to the theory of integration, and as such
    it is worth while to develop a few of the basic elements of this field.
    Many of theorems we wish to use, such as the \textit{convolution theorem},
    \textit{Fubini's theorem}, and the \textit{Fourier inversion theorem},
    are often presented with hand-wavy ``proofs,'' that obscure some of the
    problems that arise when applying these results to real-world problems. We
    will not dive into the entirety of measure theory, but rather present the
    elementary definitions, provide examples, and move on to the
    more important theorems.

        \section{Basic Notions}
    We wish to define what it means for some function
    $f:\Omega\rightarrow\mathbb{R}$ to be \textit{integrable}, where
    $\Omega$ is some space. Any attempt at defining an integral will
    require one to start with approximations such as the following:
    \begin{equation}
        \int_{\Omega}f(x)\diff{x}\approx\sum_{n}f(x_{n})\mu(X_{n})
    \end{equation}
    Where $X_{n}$ is a bunch of sets that partition $\Omega$,
    $x_{n}\in{X}_{n}$ for all $n$, and $\mu(X_{n})$ is the \textit{size} or
    the \textit{width} of $X_{n}$. This is precisely what is done in a
    calculus course where the Riemann integral is defined. To make this
    concrete, we'll need to decide what sets $X_{n}$ are allowed to
    partition $\Omega$, and what are the properties of our
    \textit{measure} $\mu$. Throughout, $\emptyset$ is used to denote
    the empty set. This is the set with nothing in it. Our inclusion of
    this set in various definitions is for technical reasons that we won't
    often be concerned with.
    \begin{ldefinition}{$\sigma\textrm{-Algebras}$}{Sigma_Algebra}
        A $\sigma$-Algebra on a set $\Omega$ is a collection of subsets
        $\mathcal{A}$ of $\Omega$ such that:
        \begin{enumerate}
            \item It is true that $\emptyset\in\mathcal{A}$ and that
                  $\Omega\in\mathcal{A}$.
            \item For all $A\in\mathcal{A}$, it is true that the complement
                  of $A$ is in $\mathcal{A}$. That is,
                  $A^{C}\in\mathcal{A}$.
            \item For any sequence $A:\mathbb{N}\rightarrow\mathcal{A}$ of
                  sets in $\mathcal{A}$, so is their intersection:
                  \begin{equation}
                      \bigcap_{n=1}^{\infty}A_{n}\in\mathcal{A}
                  \end{equation}
        \end{enumerate}
        The elements of $\mathcal{A}$ are called the
        \textit{measurable subsets} of $\Omega$. Given a set $\Omega$, and
        a $\sigma\textrm{-Algebra}$ $\mathcal{A}$ on $\Omega$, we call
        the pair $(\Omega,\,\mathcal{A})$ a \textit{measure space}.
    \end{ldefinition}
    \begin{lexample}{}{Power_Set_and_Trivial_Sigma_Algebras}
        Given a set $\Omega$, there are two simple
        $\sigma\textrm{-Algebras}$ that one can define. Let
        $\mathcal{A}=\{\emptyset,\,\Omega\}$. This satisfies all three
        properties and is called the trivial $\sigma\textrm{-Algebra}$.
        Going in the other direction, if we let $\mathcal{A}$ be the set of
        \textit{all} subsets of $\Omega$ (Also known as the
        \textit{power set} of $\Omega$, denoted $\mathcal{P}(\Omega)$),
        then this also a $\sigma\textrm{-Algebra}$.
    \end{lexample}
    The motivation for defining measurable sets in such a way is to allow
    one to easily describe \textit{measures} and \textit{probabilities}
    later.
    \begin{ldefinition}{Borel $\sigma\textrm{-Algebra}$}{Borel_Sig_Alg}
        The Borel $\sigma\textrm{-Algebra}$ is the \textit{smallest}
        $\sigma\textrm{-Algebra}$ on $\mathbb{R}$, denoted $\mathcal{B}$,
        such that for all $a<b$, the interval $(a,b)$ is a measurable
        set. That is, $(a,b)\in\mathcal{B}$.
    \end{ldefinition}
    \begin{ldefinition}{Measures}{Measures}
        A measure on a measure space $(\Omega,\,\mathcal{A})$ is a
        function $\mu:\mathcal{A}\rightarrow\mathbb{R}$ such that:
        \begin{enumerate}
            \item For all $A\in\mathcal{A}$, $\mu(A)\geq{0}$.
            \item $\mu(\emptyset)=0$.
            \item Given a \textit{mutually disjoint} list of sets
                  $A_{1},\,A_{2},\,\dots$ that are contained in
                  $\mathcal{A}$, the following is true:
                  \begin{equation}
                      \mu\Big(\bigcup_{n=1}^{\infty}A_{n}\Big)=
                      \sum_{n=1}^{\infty}\mu(A_{n})
                  \end{equation}
        \end{enumerate}
        The triple $(\Omega,\,\mathcal{A},\,\mu)$ is called a
        \textit{measurable space}.
    \end{ldefinition}
    It is important to remember that we are trying to model \textit{size},
    and this is what motivates our definition of measure. The first rule
    says that the width, or length, or size of a set is non-negative,
    and the second rule states that the size of \textit{nothing} is simply
    zero. The last rule, which is called countable additivity, is
    very important but also intuitive. We may define the length of
    the set interval $(0,\,1)$ as $1$, and the length of
    $(a,\,b)$ to be $b-a$ (For $a<b$). What about the length of the
    \textit{union} of the intervals $(0,\,1)$ and $(3,\,4)$? Since they
    have no overlap, we may as well add the lengths of the two
    individual intervals and claim that the length of the whole is 2.
    This is precisely what countable additivity tells us.

        \input{p01c02s02_fourier_series.tex}
        \section{Fourier Transforms}
    Suppose you are asked to compute $z=x/y$ for
    two non-zero real numbers $x$ and $y$. We could perform
    long-hand division, or transform the problem into
    subtraction by using the natural logarithm.
    \begin{equation}
        z=x/y\Rightarrow
        \ln(z)=\ln(x/y)\Rightarrow
        \ln(z)=\ln(x)-\ln(y)
    \end{equation}
    Provided that $\ln(x)$ and $\ln(y)$ are somehow known,
    one can compute the difference and then compute $z$
    by exponentiating the result. In a similar manner, the
    Fourier transform is often introduced as a tool for
    converting one problem into another.
    \begin{ldefinition}{Fourier Transform}{Fourier_Transform}
        The Fourier transform of a complex valued integrable function
        $f:\mathbb{R}\rightarrow\mathbb{C}$ is the function
        $\mathcal{F}_{\xi}(f):\mathbb{R}\rightarrow\mathbb{C}$ defined by:
        \begin{equation}
            F_{\xi}(f)=\int_{\minus\infty}^{\infty}f(t)
                \exp(\minus{2}\pi{i}\xi{t})\diff{t}
        \end{equation}
        This is also called the \textit{spectrum} of $f$.
    \end{ldefinition}
    The requirement that $f$ be integrable is to avoid
    strange issues in mathematics. For the sake of
    physical application, one may assume every function
    is integrable. Mathematically this is far from true,
    but oh well. For the sake of Fourier Analysis, when
    we say integrable we mean Lebesgue integrable. This
    simply means that:
    \begin{equation}
        \int_{\minus\infty}^{\infty}|f(x)|\diff{x}<\infty
    \end{equation}
    \begin{lexample}{}{FT_Hat_Func}
        Consider the hat function:
        \begin{equation}
            f(t)=
            \begin{cases}
                0,&|t|\leq{1}\\
                1,&|t|>1
            \end{cases}
        \end{equation}
        We can compute the Fourier transform of the this
        explicitly:
        \begin{equation}
            \mathcal{F}_{\xi}(f)
            =\int_{\minus\infty}^{\infty}f(t)
                \exp(\minus{2}\pi{i}\xi{t})\diff{t}
            =\int_{\minus{1}}^{1}
                \exp(\minus{2}\pi{i}\xi{t})\diff{t}
        \end{equation}
        Here we invoke Euler's Theorem,
        Thm.~\ref{thm:Euler_Expo_Formula},
        and note that the integral has symmetric limits.
        But $\sin$ is an \textit{odd} function, and thus
        it's integral is zero, and $\cos$ is an even
        function. Thus, we are left with:
        \begin{equation}
            \mathcal{F}_{\xi}(f)=
            2\int_{0}^{1}\cos(2\pi{i}\xi{t})\diff{t}
            =\frac{\sin(2\pi\xi)}{\pi\xi}
        \end{equation}
        We can be even more general, defining:
        \begin{equation}
            f(x)=
            \begin{cases}
                1,&a\leq{x}\leq{b}\\
                0,&\textrm{Otherwise}
            \end{cases}
        \end{equation}
        The Fourier transform is then:
        \begin{equation}
            \mathcal{F}_{\xi}(f)=
            \frac{i\big(\exp(\minus{2}\pi{i}\xi{b})-
                    \exp(\minus{2}\pi{i}\xi{a}\big)}{2\pi\xi}
        \end{equation}
        Thus we see that the range of the Fourier transform
        generally lies in the complex plane. Only with
        sufficient symmetry does the problem collapse down
        to $\mathbb{R}$.
    \end{lexample}
    Recall that a complex number has a polar representation
    $z=r\exp(i\theta)$. Similarly, for a complex valued
    function we can write:
    \begin{equation}
        f(z)=R(r)\exp\big(i\Theta(\theta)\big)
    \end{equation}
    For the Fourier transform of a function, the function
    $R(r)$ is called the principle amplitude, and
    $\Theta(\theta)$ is the \textit{phase offset} from
    this amplitude. The Fourier transform of the hat
    function defined from $\minus{1}$ to $1$ is plotted
    in Fig.~\ref{fig:Diff_Theory_FT_of_Hat_Func}.
    \begin{figure}[H]
        \centering
        \captionsetup{type=figure}
        \begin{subfigure}[b]{0.49\textwidth}
            \centering
            \resizebox{\textwidth}{!}{%
                \includegraphics{fourier_transform_square_well.pdf}
            }
            \subcaption{Time Domain.}
        \end{subfigure}
        \begin{subfigure}[b]{0.49\textwidth}
            \centering
            \resizebox{\textwidth}{!}{%
                \includegraphics{fourier_transform_sinc_function.pdf}
            }
            \subcaption{Frequency Domain.}
        \end{subfigure}
        \caption{Fourier Transform of the Hat Function.}
        \label{fig:Diff_Theory_FT_of_Hat_Func}
    \end{figure}
    \begin{lexample}{}{FT_Decaying_Expo}
        Let $f:\mathbb{R}\rightarrow\mathbb{R}$ be defined by:
        \begin{equation}
            f(t)=
            \begin{cases}
                \beta\exp(\minus\alpha{t}),&t\geq{0}\\
                0,&t<0
            \end{cases}
        \end{equation}
        Where $\alpha$ and $\beta$ are positive real numbers.
        Since $f(t)$ decays to zero rapidly as
        $t\rightarrow\infty$, we see that $f$ is a
        Lebesgue integrable function and has a Fourier
        transform. We can compute this using the standard
        methods obtained from a course on integral calculus.
        \begin{equation}
            \mathcal{F}_{\xi}(f)=\int_{\minus\infty}^{\infty}f(t)
                                    \exp(\minus{2}\pi{i}\xi{t})\diff{t}
                                =\beta\int_{0}^{\infty}
                                    \exp(\minus\alpha{t})
                                    \exp(\minus{2}\pi{i}\xi{t})\diff{t}
        \end{equation}
        Using the product rule for exponents, we can
        reduce this to the following line integral:
        \begin{equation}
            \mathcal{F}_{\xi}(f)=
            \beta\int_{0}^{\infty}\exp\big(
                \minus(\alpha+2\pi{i}\xi)t\big)\diff{t}
        \end{equation}
        Thus, we are integrating the exponential function
        along the line $z=\theta{t}$, where
        $\theta=\tan^{\minus{1}}(2\pi\xi/\alpha)$.
        Since $\alpha$ is positive, we can use the result
        from Jordan's Lemma to obtain the solution:
        \begin{equation}
            \mathcal{F}_{\xi}(f)
            =\frac{\beta}{\alpha+2\pi{i}\xi}
        \end{equation}
        There are two ways to view the Fourier
        transform: The Cartesian form and the
        polar form. As was stated before, the
        polar form represents the \textit{amplitude}
        and \textit{phase offset} of the Fourier
        transform, whereas the Cartesian form simply
        represents the real and imaginary parts. To
        compute the Cartesian form, we simply invoke
        Eqn.~\ref{eqn:Mult_Inv_of_Complex} for the inverse
        of a complex number, and obtain:
        \begin{equation}
            \mathcal{F}_{\xi}(f)
            =\beta\frac{\alpha-2\pi{i}\xi}
                {\alpha^{2}+4\pi^{2}\xi^{2}}
        \end{equation}
        For the polar form, we invoke
        Thm.~\ref{thm:Polar_Form_Comp_Num},
        take the modulus of $\mathcal{F}_{\xi}(f)$
        and compute inverse tangents:
        \begin{equation}
            \mathcal{F}_{\xi}(f)=
            \frac{\beta}{\sqrt{\alpha^{2}+4\pi^{2}\xi^{2}}}
            \exp\Big[i\tan^{\minus{1}}
                \Big(\frac{\minus{2}\pi\xi}{\alpha}\Big)
            \Big]
        \end{equation}
        The two are plotted below for the case of
        $\alpha=\beta=1$.
    \end{lexample}
    \begin{ldefinition}{Inverse Fourier Transform}
          {Diff_Theory_Inverse_Fourier_Transform}
        The inverse Fourier transform of a complex valued
        integrable function
        $f:\mathbb{R}\rightarrow\mathbb{C}$ is the function
        $\mathcal{F}_{t}^{\minus{1}}(f):%
            \mathbb{R}\rightarrow\mathbb{C}$ defined by:
        \begin{equation}
            \mathcal{F}_{t}^{\minus{1}}(f)=
            \int_{\minus{\infty}}^{\infty}
                f(\xi)\exp(2\pi{i}\xi{t})\diff{\xi}
        \end{equation}
    \end{ldefinition}
    We now prove what is probably the most useful theorem
    in Fourier Analysis.
    \begin{theorem}
        If $f:\mathbb{R}\rightarrow\mathbb{R}$ is a
        continuous Lebesgue integrable function,
        and if its spectrum $F$ is also continuous
        and Lebesgue integrable, then:
        \begin{equation}
            f(t)=\int_{-\infty}^{\infty}F(\omega)
            \exp(2\pi{i}\omega{t})\diff{\omega}
        \end{equation}
    \end{theorem}
    A powerful application of this is
    Shannon's Sampling Theorem.
    \begin{theorem}[Shannon's Sampling Theorem]
        If $f(t)$ is a continuous
        Lebesgue integrable function such that
        its spectrum $F(\omega)$ is differentiable and zero
        outside the interval $[-W,W]$,
        then $f(t)$ is uniquely determined
        by the points $f(\frac{n}{2W})$, $n\in\mathbb{N}$.
    \end{theorem}
    \begin{proof}
        For let $F$ be the spectrum of $f$. That is:
        \begin{equation}
            f(t)=\int_{-\infty}^{\infty}F(\omega)
            \exp(-2\pi{i}\omega{t})\diff{\omega}
        \end{equation}
        But $F(\omega)=0$ for $|\omega|>W$. Thus we have:
        \begin{equation}
            f(t)=\int_{-W}^{W}F(\omega)
            \exp(-2\pi{i}\omega{t})\diff{\omega}
        \end{equation}
        Then for $n\in\mathbb{N}$, we have:
        \begin{equation}
            f\big(\frac{n}{2W}\big)=\int_{-W}^{W}F(\omega)
            \exp(-2\pi{i}\frac{n}{2W}\omega)\diff{\omega}
        \end{equation}
        But $F$ is differentiable, and thus it's Fourier
        series converges. That is:
        \begin{subequations}
            \begin{align}
                F(\omega)
                &=\sum_{n=\minus\infty}^{\infty}
                    \exp(2\pi{i}n\omega)
                \int_{\minus{W}}^{W}F(\tau)
                \exp(\minus{2}\pi{i}\frac{n}{2W}\tau)
                    \diff{\tau}\\
                &=\sum_{n=-\infty}^{\infty}f
                  \big(\frac{n}{2W}\big)e^{2\pi{i}n\omega}
            \end{align}
        \end{subequations}
        Therefore $f(\frac{n}{2W})$, $n\in \mathbb{N}$
        uniquely determines $F(\omega)$. But the
        spectrum $F(\omega)$ uniquely determines
        $f(t)$. Therefore $f(t)$ is
        uniquely determined and:
        \begin{equation}
            f(t)=\sum_{n=-\infty}^{\infty}\int_{-W}^{W}
            f\big(\frac{n}{2W}\big)
            \exp(2\pi{i}\omega(n+t))\diff{\omega}
        \end{equation}
    \end{proof}

        \input{p01c02s04_convolutions.tex}
        \input{p01c02s05_sampling.tex}
        \input{p01c03s00_numerical_analysis.tex}
        \input{p01c03s01_power_series.tex}
        \input{p01c03s02_asymptotic_expansions.tex}
        \input{p01c03s03_polynomial_approximations.tex}
        \input{p01c03s04_rational_approximations.tex}
        \input{p01c03s05_interpolating.tex}
        \section{Stationary Phase Approximation}
    Suppose $g$ is an analytical function about
    the origin (i.e. it has a convergent MacLaurin series),
    and consider the integral:
    \begin{equation}
        I(k) = \int_{a}^{b}e^{ikg(x)}dx
    \end{equation}
    Suppose that there is a $c\in(a,b)$ such
    that $g'(c)=0$ and $g''(c)\ne 0$. Then:
    \begin{align}
        \nonumber I(k)&=e^{ikg(c)}\int_{a}^{b}e^{ik[g(x)-g(c)]}dx\\
        &=e^{ikg(c)}\int_{a}^{b}e^{ik[\frac{g''(c)}{2}(x-c)^{2}+\hdots]}dx
    \end{align}
    Higher terms are extremely oscillatory, and so we neglect them.
    Note that higher terms can indeed cancel each
    other out, meaning these neglected terms may
    not be negligible. For example, if
    $g(x)=-sin(\pi x)$, then $\exp(ig(x))$ is never
    too oscillatory. However, so long as the interval
    $[a,b]$ is small enough, the approximation is still
    valid. The previously mentioned $g(x)$ is how
    one approximates the $J_{0}(x)$ Bessel function.
    Out integral then becomes:
    \begin{align}
        I(k)&\approx e^{ikg(c)}
                \int_{a}^{b}e^{ik\frac{g''(c)}{2}(x-c)^{2}}dx\\
            &\approx e^{ikg(c)}
                \int_{\infty}^{\infty}
                \exp(ik\frac{g''(c)}{2}(x-c)^{2})\diff{x}\\
            &=e^{ikg(c)}\sqrt{\frac{2\pi i}{kg''(c)}}
    \end{align}
    We can use this for our double integral,
    and make it a single integral. The first and
    second integrals of $\psi$ are nasty, however.
    \begin{equation}
        \begin{split}
            \frac{\partial\psi}{\partial\phi}
            &=kD\Big[\frac{2D\rho\cos(B)
                \sin(\phi)+2\rho\rho_{0}
            \sin(\phi-\phi_{0})}{2D^2\sqrt{1+2\cos(B)
            \frac{\rho_{0}\cos(\phi_{0})-\rho\cos(\phi)}{D}+
            \frac{\rho^{2}+\rho_{0}^{2}-
            2\rho\rho_{0}\cos(\phi-\phi_{0})}{D^2}}}\\
            &\quad\quad\quad\quad\quad
            -\frac{\rho\cos(B)\sin(\phi)}{D}\Big]
        \end{split}
    \end{equation}
    The second derivative is equally bad.
    Solving for $\frac{\partial\psi}{\partial\phi}=0$
    must be done iteratively by successive approximations.
    A further approximation can be made as $\psi$
    is analytic in $\phi$. Let $\phi_{s}$ be
    the solution to $\frac{\partial\psi}{\partial\phi}$
    and let $\phi_{s_{n}}$ be a sequence such that
    $\phi_{s_{n}}\rightarrow\phi_{s}$.

        \section{Root Finding}
    \begin{figure}[H]
        \centering
        \captionsetup{type=figure}
        \resizebox{\textwidth}{!}{%
            \includegraphics{newton_fractal_z_cubed_minus_one.png}
        }
        \caption{Newton Fractal for $z^{3}-1=0$}
        \label{fig:Diff_Theory_Newton_Fractal}
    \end{figure}
    The Python code used to generated this is given below.
    It can be altered to try various functions, and this
    is encouraged.
    \newpage
    \begin{lstlisting}[%
        language=python,
        basicstyle=\footnotesize\ttfamily,
        frame=single,
        gobble=16
    ]
        from PIL import Image
        import numpy as np

        # Set range for x and y axes.
        x_min, x_max = -1.0, 1.0
        y_min, y_max = -1.0, 1.0

        # Colors for the roots (Red, Green, Blue).
        colors = [(255, 0, 30), (0, 255, 30), (0, 30, 255)]

        size = 1024
        img = Image.new("RGB", (size, size), (255, 255, 255))

        # List the roots of z^3-1
        roots = [1.0+0.0j, -0.5+0.8660254037844386j, -0.5-0.8660254037844386j]
        roots = np.array(roots)
        for y in range(size):
            z_y = y * (y_max - y_min)/(size - 1) + y_min
            for x in range(size):
                z_x = x * (x_max - x_min)/(size - 1) + x_min
                z = complex(z_x, z_y)

                # Allow 40 iterations for Newton-Raphson.
                for iters in range(40):
                    # Perfrom Newton-Raphson on z^3 - 1 (Simplifying as well).
                    root = (2.0*z*z*z + 1)/(3.0*z*z)

                    # Checks for convergence
                    if abs(root - z) < 10e-10:
                        break
                    z = root

                ind = np.min((
                    np.abs(root-roots) == np.min(np.abs(root-roots))
                ).nonzero())
                col = colors[ind]
                col = [k for k in col]

                # Create a gradient in color to emphasize rate of convergence.
                col[ind] -= 10*iters
                col = tuple([k for k in col])
                img.putpixel((x, y), col)
        img.save("NewtonRoots.png", "PNG")
    \end{lstlisting}

        \input{p01c04s00_pdes.tex}
        \input{p01c04s01_greens_functions.tex}
        \section{Kirchoff's Integral Formula}

        \input{p01c04s03_the_wave_equation.tex}
        \input{p01c04s04_the_helmholtz_equation.tex}
        \chapter{Special Functions}
    There are many special functions that arise in diffraction
    theory. These are functions that can not be written in
    closed form via a combination of rational functions,
    trigonometric functions, logarithms, or exponentials.
    Usually such functions are defined as the solution to
    a particular differential equation, such as Bessel
    functions, or as the result of integrating a non-trivial
    function, such as Fresnel Integrals. Other times functions
    are defined as the inverse of a tricky algebraic equation,
    such that the Lambert $W$ function. We'll discuss these
    three functions, numerical calculations, and their
    applications.

        \section{The Fresnel Integrals}
    The Fresnel Sine and Cosine Integrals, which are
    usually denoted $S(x)$ and $C(x)$, respectively,
    occur naturally in the study of diffraction theory.
    By examining the \textit{Fresnel Kernel},
    and using a Taylor series approximation, one
    comes across the following integral:
    \begin{equation}
        F(x)=\int_{0}^{x}\exp(it^{2})\diff{t}
    \end{equation}
    Using Euler's Theorem, we can rewrite this as:
    \begin{equation}
        F(x)=\int_{0}^{x}\cos(t^{2})\diff{t}
            +i\int_{0}^{x}\sin(t^{2})\diff{t}
    \end{equation}
    The Fresnel Cosine and Sine Integrals are defined
    as the real and
    imaginary parts of this equation, respectively.
    \begin{ldefinition}{Fresnel Integrals}
        The Fresnel Sine and Fresnel Cosine, denoted
        $S(x)$ and $C(x)$, respectively, are real valued
        functions defined by:
        \par\hfill\par
        \vspace{-1ex}
        \begin{subequations}
            \begin{minipage}{0.49\textwidth}
                \begin{equation}
                    S(x)=\int_{0}^{x}
                    \sin(t^{2})\diff{t}
                \end{equation}
            \end{minipage}
            \hfill
            \begin{minipage}{0.49\textwidth}
                \begin{equation}
                    C(x)=\int_{0}^{x}
                    \cos(t^{2})\diff{t}
                \end{equation}
            \end{minipage}
        \end{subequations}
        \par
    \end{ldefinition}
    Graph of the Fresnel Sine and Fresnel Cosine
    functions are shown in
    Fig.~\ref{fig:Diff_Theory_Graphs_of_Sinx2_and_Cosx2}.
    \begin{figure}[H]
        \captionsetup{type=figure}
        \centering
        \begin{subfigure}[b]{0.49\textwidth}
            \centering
            \resizebox{\textwidth}{!}{%
                \includegraphics{Fresnel_Cos.pdf}
            }
            \subcaption{Graphs of $\cos(x^{2})$ and $C(x)$.}
        \end{subfigure}
        %\par\hfill\par
        \begin{subfigure}[b]{0.49\textwidth}
            \centering
            \resizebox{\textwidth}{!}{%
                \includegraphics{Fresnel_Sin.pdf}
            }
            \subcaption{Graphs of $\sin(x^{2})$ and $S(x)$.}
        \end{subfigure}
        \caption[Fresnel Integrals]
            {Graphs of the Fresnel Sine and Cosine
             integrals, along with their respective derivatives.}
        \label{fig:Diff_Theory_Graphs_of_Sinx2_and_Cosx2}
    \end{figure}
    We will be interested in functions
    of the form $\exp(i\psi)$ later on. The Fresnel
    Approximation uses the Taylor expansion of $\psi$
    up to the quadratic term, and hence we will see
    something of the form $\exp(i(a+bx+cx^2))$. From
    elementary algebra we can complete the square, and
    do a change of variables to obtain
    $\exp(i(u^{2}-d^{2}))$, where $d$ is some constant.
    We are interested in the integral of this across the
    entire real line. From Euler's Formula
    (Thm.~\ref{thm:Euler_Expo_Formula}) we
    see that $\exp(ix^{2})=\cos(x^{2})+i\sin(x^{2})$.
    But $x^{2}$ grows rapidly,
    and thus $\sin(x^{2})$ and $\cos(x^{2})$ are two
    rapidly oscillating functions. The oscillation are
    so rapid that the areas cancel out, and hence
    $S(x)$ and $C(x)$ are well defined as
    $x\rightarrow\infty$. We will use Cauchy's Integral
    Theorem to evaluate the limits of these two functions.
    First, a result from Gauss.
    \begin{theorem}
        \begin{equation}
            \int_{\minus\infty}^{\infty}
                \exp(\minus{x}^{2})\diff{x}
            =\sqrt{\pi}
        \end{equation}
    \end{theorem}
    \begin{proof}
        Convergence can be shown, since for
        all $x\in\mathbb{R}$:
        \begin{equation}
            0<\exp(\minus{x}^{2})
            \leq\frac{1}{1+x^2}
        \end{equation}
        And therefore:
        \begin{equation}
            0\leq\int_{\minus\infty}^{\infty}
            \exp(\minus{x}^{2})\diff{x}\leq
            \int_{\minus\infty}^{\infty}
            \frac{1}{1+x^2}\diff{x}
            =\tan^{\minus{1}}(x)
            \Big|_{\minus\infty}^{\infty}=\pi
        \end{equation}
        Define the following:
        \begin{equation}
            \mathcal{I}=\int_{\minus\infty}^{\infty}
            \exp(\minus{x}^{2})\diff{x}
        \end{equation}
        Squaring $\mathcal{I}$, we obtain:
        \begin{subequations}
            \begin{align}
                \mathcal{I}^{2}&=
                \bigg(\int_{\minus\infty}^{\infty}
                \exp(\minus{x}^{2})\diff{x}\bigg)
                \bigg(\int_{\minus\infty}^{\infty}
                \exp(\minus{y}^{2})\diff{y}\bigg)\\
                &=\int_{\minus\infty}^{\infty}
                \int_{\minus\infty}^{\infty}
                \exp\big(\!\minus\!(x^{2}+y^{2})\big)
                    \diff{x}\diff{y}
            \end{align}
        \end{subequations}
        Switching from Cartesian to
        Polar coordinates, we have:
        \begin{equation}
            \mathcal{I}^{2}=
            \int_{0}^{2\pi}\int_{0}^{\infty}
            r\exp(\minus{r}^{2})\diff{r}\diff{\phi}
            =2\pi\int_{0}^{\infty}
            r\exp(\minus{r}^{2})\diff{r}
        \end{equation}
        This final integral can be computed from basic
        methods one would find in a Calculus textbook.
        Letting $u=r^{2}$, we have
        $\diff{u}=2r\diff{r}$,
        so the integral becomes:
        \begin{equation}
            \mathcal{I}^{2}=\pi\int_{0}^{\infty}
                \exp(\minus{u})\diff{u}
            =\pi
        \end{equation}
        Therefore $\mathcal{I}=\pm\sqrt{\pi}$.
        But $\mathcal{I}>0$, and thus
        $\mathcal{I}=\sqrt{\pi}$.
    \end{proof}
    This result has many fundamental applications in
    probability theory and in statistics, where
    it is used to define the normal distribution.
    For us, we can use this to evaluate the limits of
    $S(x)$ and $C(x)$ as $x\rightarrow\infty$.
    First note, that since $\exp(-x^{2})$ is an even
    function, the integral on $[0,\infty)$ is half of
    that of the integral on the entire real line. That is:
    \begin{equation}
        \int_{0}^{\infty}\exp(\minus{t}^{2})\diff{t}
        =\frac{\sqrt{\pi}}{2}
    \end{equation}
    We now evaluate the complex version of this.
    \begin{theorem}
        \begin{equation}
            \int_{0}^{\infty}\exp(ix^{2})\diff{x}
            =\sqrt{\frac{\pi}{8}}(1+i)
        \end{equation}
    \end{theorem}
    \begin{proof}
        For let $C_{R}$ be the closed path in the complex plane
        defined by:
        \begin{equation}
            C_{R}(t)=
            \begin{cases}
                3Rt,&0\leq{t}\leq\frac{1}{3}\\
                R\exp\big(i\frac{3\pi}{4}(t-\frac{1}{3})\big),
                &\frac{1}{3}<t<\frac{2}{3}\\
                \frac{3R}{\sqrt{2}}(1+i)(1-t),
                &\frac{2}{3}\leq{t}\leq{1}
            \end{cases}
        \end{equation}
        Then, for all $R>0$, $C_{R}$ is a Jordan Curve in the
        complex differentiable at all but three points. Thus,
        by Cauchy's Theorem, as $\exp(iz^{2})$ is an entire
        function:
        \begin{equation}
            \oint_{C_{R}}\exp(iz^{2})\diff{z}
            =0
        \end{equation}
        But then:
        \begin{equation}
            \begin{split}
                \int_{0}^{R}\exp(ix^{2})\diff{x}+
                \int_{\frac{1}{3}}^{\frac{2}{3}}&
                    \exp(iz(t)^{2})C_{R}'(t)\diff{t}\\
                &+\frac{1+i}{\sqrt{2}}\int_{R}^{0}\exp(-x^{2})\diff{x}
                =0
            \end{split}
        \end{equation}
        But by Jordan's Lemma, this second integral tends to zero as
        $R\rightarrow\infty$. Therefore:
        \begin{align}
            \int_{0}^{\infty}\exp(-ix^{2})\diff{x}
            &=-\frac{1+i}{\sqrt{2}}\int_{\infty}^{0}\exp(-x^{2})\diff{x}\\
            &=\frac{1+i}{\sqrt{2}}\int_{0}^{\infty}\exp(-x^{2})\diff{x}\\
            &=\frac{1+i}{\sqrt{2}}\frac{\sqrt{\pi}}{2}\\
            &=(1+i)\sqrt{\frac{\pi}{8}}
        \end{align}
    \end{proof}
    \begin{theorem}
        If $S$ and $C$ are the Fresnel Sine and Cosine integrals,
        respectively, then:
        \begin{align}
            \underset{x\rightarrow\infty}{\lim}S(x)
            &=\sqrt{\frac{\pi}{8}}\\
            \underset{x\rightarrow\infty}{\lim}C(x)
            &=\sqrt{\frac{\pi}{8}}
        \end{align}
    \end{theorem}
    \begin{proof}
        For:
        \begin{align}
            \underset{x\rightarrow\infty}{\lim}
                \big(C(x)+iS(x)\big)
            &=\underset{x\rightarrow\infty}{\lim}
                \int_{0}^{x}\exp(ix^{2})\diff{x}\\
            &=\sqrt{\frac{\pi}{8}}(1+i)
        \end{align}
        Comparing real and imaginary parts complex the proof.
    \end{proof}
    \begin{figure}[H]
        \centering
        \captionsetup{type=figure}
        \includegraphics{complex_plane_fresnel_integral_path.pdf}
        \caption{Jordan Curve Used to Evaluate the Fresnel Integrals.}
        \label{fig:Jordan_Curve_Fresnel_Integrals}
    \end{figure}
    \par\hfill\par
    \begin{theorem}
        If $F$ is a positive real number and
        $f:\mathbb{R}\rightarrow\mathbb{C}$ is defined by:
        \begin{equation}
            f(\rho)=
                \exp\bigg(
                    i\frac{\pi}{2}
                    \Big(\frac{\rho}{F}\Big)^{2}
                \bigg)
        \end{equation}
        Then:
        \begin{equation}
            \mathcal{F}_{\xi}(f)
            =(1+i)F\exp(\minus{2}\pi{i}F^{2}\xi^{2})
        \end{equation}
    \end{theorem}
    \begin{proof}
    For:
    \begin{align}
        \mathcal{F}_{\xi}(f)
        &=\int_{-\infty}^{\infty}
            \exp\bigg(
                i\frac{\pi}{2}
                \Big(\frac{\rho}{F}\Big)^{2}
            \bigg)
            \exp(-2\pi{i}\rho\xi)\diff{\rho}\\
        &=\int_{-\infty}^{\infty}
            \exp\bigg(
                i\frac{\pi}{2}\Big(
                    \frac{\rho}{F}
                \Big)^{2}-2\pi{i}\rho\xi
            \bigg)
            \diff{\rho}\\
        &=\int_{-\infty}^{\infty}
            \exp\Big(
                \frac{i\pi}{2F^2}
                \big[\rho^2-4F^2\rho \xi\big]
            \Big)\diff{\rho}
    \end{align}
    Completing the square, we get
    $(\rho-2F^{2}\xi)^{2}-4F^{4}\xi^{2}$.
    So, the integral becomes:
    \begin{equation}
        \begin{split}
            \int_{-\infty}^{\infty}
            \exp&\Big(
                i\frac{\pi}{2F^2}
                \big[\rho-2F^{2}\xi\big]^{2}
            \Big)
            \exp(-2\pi{i}F^{2}\xi^{2})\diff{\rho}\\
            &=\exp(-2\pi{i}F^{2}\xi^{2})
            \int_{-\infty}^{\infty}
            \exp\Big(
                i\frac{\pi}{2F^2}
                \big[\rho-2F^{2}\xi\big]^{2}
            \Big)\diff{\rho}
        \end{split}
    \end{equation}
    Let $u=\frac{\rho-2F^{2}\xi}{F}$, so then
    $F\diff{u}=\diff{\rho}$. We obtain:
    \begin{equation}
        \mathcal{F}_{\xi}(f)
        =F\exp(\minus{2}\pi{i}F^{2}\xi^{2})
            \int_{-\infty}^{\infty}
            \exp\Big(i\frac{\pi}{2}s^{2}\Big)\diff{s}
    \end{equation}
    But this integral is $1+i$, completing the proof.
    \end{proof}
    \begin{theorem}
    $\mathcal{F}(e^{-i\frac{\pi}{2}\big(\frac{\rho_0}{F}\big)^2}\big) = (1-i)Fe^{2\pi i F^2 \xi^2}$.
    \end{theorem}
    \begin{proof}
    For:
    \begin{align}
        \mathcal{F}_{\xi}(f)
        &=\int_{-\infty}^{\infty}
            \exp\bigg[
                \minus{i}\frac{\pi}{2}
                \Big(\frac{\rho}{F}\Big)^2
            \bigg]
            \exp(\minus{2}\pi{i}\rho\xi)\diff{\rho}\\
        &=\int_{-\infty}^{\infty}
            \exp\Big(
                {\!}\minus{\!}\frac{i\pi}{2F^{2}}
                \big[
                    {\rho}^{2}+4F^{2}\rho\xi
                \big]\Big)\diff{\rho}\\
        &=\int_{-\infty}^{\infty}
            \exp\Big(
                {\!}\minus{\!}\frac{i\pi}{2F^{2}}
                \big[
                    (\rho+2F^{2}\xi)^2-4F^{4}\xi^{2}
                \big]
            \Big)\diff{\rho}\\
        &=\exp(2\pi{i}F^{2}\xi^{2})\int_{-\infty}^{\infty}
            \exp\Big(
                {\!}\minus{\!}\frac{i\pi}{2F^2}
                (\rho_0+2F^2\xi)
            \Big)\diff{\rho}
    \end{align}
    Let $u = \frac{\rho_0 + 2F^2 \xi}{F}$, then
    $Fdu = d\rho_0$, so we have$Fe^{2\pi i F^2 \xi^2} \int_{-\infty}^{\infty} e^{-i\frac{\pi}{2}u^2}du$.
    Let $u = -is$, then $du = -ids$, and $u^2 = -s^2$. So
    we have $-i e^{2\pi i F^2 \xi^2}\int_{-\infty}^{\infty} e^{i\frac{\pi}{2}s^2}ds$.
    But this integral is $1+i$. So, we have
    $-iFe^{2\pi iF^2\xi^2}(1+i)=(1-i)Fe^{2\pi iF^2 \xi^2}$.
    \end{proof}
    \begin{theorem}
        If $f:\mathbb{R}\rightarrow\mathbb{R}$ and
        $g:\mathbb{R}\rightarrow{R}$ are integrable,
        and if $f*g$ is the convolution of $f$ with
        respect to $g$:
        \begin{equation}
            f*g=\int_{-\infty}^{\infty}
                f(\tau)g(\tau-t)\diff{\tau}
        \end{equation}
        Then:
        \begin{equation}
            \mathcal{F}_{\xi}\big(f*g\big)
            =\mathcal{F}_{\xi}(f)\cdot\mathcal{F}_{\xi}(g)
        \end{equation}
    \end{theorem}
    \begin{proof}
    Let $\int_{-\infty}^{\infty} |f(t)|dt = \norm{f}_{1}$ and $\int_{-\infty}^{\infty} |g(t)|dt = \norm{g}_{1}$. Then:
    \begin{align*}
        \int_{-\infty}^{\infty}\int_{-\infty}^{\infty}
            |f(\tau)g(\tau-t)|\diff{\tau}\diff{t}
        &\leq\int_{-\infty}^{\infty}
            |f(\tau)|\int_{-\infty}^{\infty}
            |g(\tau-t)|\diff{\tau}\diff{t}\\
        &=\int_{\infty}^{\infty}
            |f(x)|\norm{g}_{1}\diff{x}\\
        &=\norm{f}_{1}\norm{g}_{1}
    \end{align*}
    Thus, $h(t)=f*g$ is such that
    $\int_{-\infty}^{\infty} |h(t)|dt < \infty$.
    Let $H(\xi) = \mathcal{F}(h)$. Then:
    \begin{align}
        H(\xi)&=
            \int_{-\infty}^{\infty}
            h(t)\exp(\minus{2}\pi{i}t\xi)\diff{t}\\
        &=\int_{-\infty}^{\infty}
            \Bigg(
                \int_{-\infty}^{\infty}
                f(\tau)g(t-\tau)\diff{\tau}
            \Bigg)\exp(\minus{2}\pi{i}t\xi)\diff{t}
    \end{align}
    But:
    \begin{equation}
        |e^{-2\pi{i}t\xi}f(\tau)g(t-\tau)|
        =|f(\tau)g(t-\tau)|
    \end{equation}
    But this is simply the integrand of $h$, and $h$
    is integrable. Thus, by Fubini's Theorem we may
    swap the integrals. Let $y=t-\tau$. Then:
    \begin{subequations}
        \begin{align}
            H(\xi)&=
                \int_{-\infty}^{\infty}f(\tau)
                \int_{-\infty}^{\infty}g(t-\tau)
                e^{-2\pi{i}t\xi}\diff{t}\diff{\tau}\\
            &=\int_{-\infty}^{\infty}f(\tau)
                e^{-2\pi{i}\tau\xi}\diff{\tau}
                \int_{-\infty}^{\infty}g(y)
                e^{-2\pi{i}y\xi}dy\\
            &=\mathcal{F}(f)\cdot\mathcal{F}(g)
                \vphantom{\int_{-\infty}^{\infty}}
        \end{align}
    \end{subequations}
    Therefore, etc.
    \end{proof}
    \begin{theorem}
        If $T:\mathbb{R}\rightarrow\mathbb{C}$ is
        a Lebesgue integrable function, and if
        $\hat{T}:\mathbb{R}\rightarrow\mathbb{C}$
        is defined by:
        \begin{equation}
            \hat{T}(\rho_0)=
            \frac{1-i}{2F}\int_{\minus\infty}^{\infty}T(\rho)\exp\Big(
                i\frac{\pi}{2}\big(\frac{\rho-\rho_0}{F}\big)^{2}\Big)
            \diff{\rho}
        \end{equation}
        then:
        \begin{equation}
            T(\rho)=\frac{1+i}{2F}
                \int_{\minus\infty}^{\infty}\hat{T}(\rho_{0})\exp\Big(
                \minus{i}\frac{\pi}{2}\big(\frac{\rho-\rho_0}{F}\big)^{2}
                \Big)\diff{\rho_{0}}
        \end{equation}
    \end{theorem}
    \begin{proof}
        For by the definition of $\hat{T}$, and by the
        definition of convolution, we have:
        \begin{equation}
            \hat{T}(\rho)
            =\frac{1-i}{2F}\Big[
                T*\exp\Big(i\frac{\pi}{2}\big(\frac{\rho}{F}\big)^{2}
            \Big)\Big]
        \end{equation}
        But $T$ is a Lebesgue integrable function, and thus by
        the convolution theorem:
        \begin{subequations}
            \begin{align}
                \mathcal{F}(\hat{T})
                &=\frac{1-i}{2F}\mathcal{F}(T)\cdot
                    \mathcal{F}\Big(
                        \exp\Big[i\frac{\pi}{2}
                        \big(\frac{\rho_{0}}{F}\big)^{2}
                    \Big]\Big)\\
                &=\frac{1-i}{2F}\mathcal{F}(T)\cdot
                    (1+i)F\big(
                        \exp(\minus{2}\pi{i}F^{2}\xi^{2}
                    \big)\\
                &=\mathcal{F}(T)\exp(\minus{2}\pi{i}F^{2}\xi^{2})\\
                    \Rightarrow
                    \mathcal{F}(\hat{T})
                    \exp\big(2\pi{i}F^{2}\xi^{2}\big)
                &=\mathcal{F}(T)
            \end{align}
        \end{subequations}
        But the Fourier transform of a Gaussian is
        another Gaussian. That is:
        \begin{subequations}
            \begin{align}
                \exp\big(2\pi{i}F^{2}\xi^{2}\big)
                &=\frac{1}{(1-i)F}\mathcal{F}\Big(
                    \exp\Big[\minus{i}\frac{\pi}{2}
                    \big(\frac{\rho_{0}}{F}\big)^{2}\Big]\Big)\\
                &=\frac{1+i}{2F}\mathcal{F}\Big(
                    \exp\Big[\minus{i}\frac{\pi}{2}
                        \big(\frac{\rho_0}{F}\big)^{2}
                    \Big]\Big)
            \end{align}
        \end{subequations}
        Therefore:
        \begin{subequations}
            \begin{align}
                \mathcal{F}(T)
                &=\frac{1+i}{2F}\mathcal{F}(\hat{T})
                    \cdot\mathcal{F}
                    \big(e^{-i\frac{\pi}{2}
                    \big(\frac{\rho_0}{F}\big)^2}\big)\\
                &=\frac{1+i}{2F}\mathcal{F}
                    (\hat{T}*e^{-i\frac{\pi}{2}
                    \big(\frac{\rho_0}{F}\big)^2})\\
            &=\mathcal{F}\bigg(\frac{1+i}{2F}
                \int_{\minus\infty}^{\infty}\hat{T}(\rho_{0})
                \exp\Big[\minus{i}\frac{\pi}{2}\big(
                    \frac{\rho-\rho_{0}}{F}\big)^{2}\Big]
                \diff{\rho_{0}}\bigg)
            \end{align}
        \end{subequations}
        Therefore, by the uniqueness of the Fourier Transform:
        \begin{equation}
            T(\rho)=\frac{1+i}{2F}\int_{-\infty}^{\infty}
                \hat{T}(\rho_{0})\exp\Big[
                    \!\minus\!\frac{i\pi}{2}
                    \big(\frac{\rho-\rho_{0}}{F}\big)^{2}
                \Big]\diff{\rho_{0}}
        \end{equation}
        Therefore, etc.
    \end{proof}
    \begin{theorem}
            If $T,\psi\in{L}^{2}(\mathbb{R})$, and if
            $\hat{T}:\mathbb{R}\rightarrow\mathbb{C}$
            is defined by:
            \begin{equation}
            T(\rho_{0})=\int_{\minus\infty}^{\infty}
                T(\rho)\exp\big(i\psi(\rho_{0}-\rho)\big)
                \diff{\rho_{0}}
            \end{equation}
            then:
            \begin{equation}
                T(\rho)=\mathcal{F}^{\minus{1}}_{\rho}\Big(
                    \frac{\mathcal{F}(\hat{T})}
                        {\mathcal{F}\big(\exp(i\psi)\big)}
                    \Big)
            \end{equation}
    \end{theorem}
    \begin{proof}
        For $\hat{T}(\rho_{0})=T*\exp(i\psi)$. But then:
        \begin{equation}
            \mathcal{F}_{\xi}(\hat{T})
            =\mathcal{F}_{\xi}\big(T*\exp(i\psi)\big)
            =\mathcal{F}_{\xi}\big(T\big)\cdot
            \mathcal{F}\big(\exp(i\psi)\big)
        \end{equation}
        So then:
        \begin{equation}
            \mathcal{F}_{\xi}(T)
            =\frac{\mathcal{F}(\hat{T})}
                {\mathcal{F}\big(\exp(i\psi)\big)}
        \end{equation}
        From the uniqueness of Fourier transforms, we
        obtain the result.
    \end{proof}

        \input{p01c05s02_bessel_functions.tex}
        \input{p01c05s04_lambertw.tex}
        \input{p01c05s05_legendre_polynomials.tex}
    \part{Theory}
        \input{p02c06s00_diffraction_theory.tex}
        \section{Maxwell's Equations}
    At the heart of electromagnetism are Maxwell's equations.
    They are:
    \par
    \begin{subequations}
        \begin{minipage}[b]{0.49\textwidth}
            \centering
            \begin{align}
                \mathrm{curl}(\mathbf{E})
                    &=\minus\frac{\partial\mathbf{B}}{\partial{t}}\\
                \mathrm{div}(\mathbf{E})
                    &=\frac{\rho}{\epsilon_{0}}
            \end{align}
        \end{minipage}
        \hfill
        \begin{minipage}[b]{0.49\textwidth}
            \centering
            \begin{align}
                \mathrm{curl}(\mathbf{B})
                    &=\mu_{0}\mathbf{J}+\mu_{0}\epsilon_{0}
                      \frac{\partial\mathbf{E}}{\partial{t}}\\
                \mathrm{div}(\mathbf{B})&=0
            \end{align}
        \end{minipage}
    \end{subequations}
    \par
    With this, we will derive the Fresnel-Huygens principle.
        \input{p02c06s02_fresnel_fraunhofer_theory.tex}
        \input{p02c06s03_fresnel_approximation.tex}
        \input{p02c06s04_fresnel_inversion.tex}
        \chapter{Geometry}
    \section{Titan Geometry}
        \begin{figure}[H]
        	\centering
        	\captionsetup{type=figure}
        	\begin{subfigure}[b]{0.49\textwidth}
        	    \centering
        	    \captionsetup{type=figure}
        	    \resizebox{\textwidth}{!}{%
                  \includegraphics{titan_occultation_geometry.pdf}
              }
            	\subcaption{Geometry of an Occultation of Titan}
        	    \label{fig:math_titan_geom_vec}
            \end{subfigure}
            \begin{subfigure}[b]{0.49\textwidth}
                \centering
                \captionsetup{type=figure}
                \resizebox{\textwidth}{!}{%
                    \includegraphics{titan_bending_angle_geometry.pdf}
                }
                \subcaption{Geometry of the Bending Angle}
                \label{fig:math_geo_bending_angle}
            \end{subfigure}
            \caption{Various Geometries for Titan}
        \end{figure}
        The following definitions are used:
        \begin{enumerate}
            \item $O$ is the center of Titan.
            \item $E$ is the Earth.
            \item $C$ is the Cassini spacecraft.
            \item $\mathbf{r}_{E}=\overrightarrow{OE}$
            \item $\mathbf{v}_{E}=\dot{\mathbf{r}}_{E}$
            \item $\mathbf{r}_{S}=\overrightarrow{OC}$
            \item $\mathbf{v}_{S}=\dot{\mathbf{r}}_{S}$
            \item $\mathbf{p}_{in}$ is the projection of
                  $O$ onto $\overline{AC}$
            \item $\mathbf{p}_{out}$ is the projection
                  of $O$ onto $\overline{EA}$
            \item $\alpha$ is the bending angle
                  ($\pi-\angle EAC$)
            \item $\hat{\mathbf{n}}_{in}$ is the
                  direction of the emission.
            \item $\hat{\mathbf{n}}_{out}$ is the
                  direction of the reception.
            \item The ray plane lies in the plane $OEC$
            \item $\phi=\angle{AOC}$
            \item $\theta=\angle{ACO}$
            \item $\beta=\angle{OAC}$
            \item $A$ is the intersection of the lines
                  starting at $C$ and $E$, parallel to
                  $\hat{\mathbf{n}}_{in}$ and
                  $\hat{\mathbf{n}}_{out}$, respectively.
        \end{enumerate}

        % Replace the 6pt vspace removed to below the list.
        \vspace{6pt}
        Where $\dot{\mathbf{r}}$ denotes the time derivative
        of $\mathbf{r}$. The following assumptions are made:
        \begin{enumerate}
            \item $\angle{OAE}=\angle{OAC}$
            \item $A$ lies in the plane $OEC$
        \end{enumerate}
        \begin{theorem}
            \label{theorem:ray_plane_perp_to_r_e_cross_r_s}
            The ray plane is perpendicular to
            $\hat{\mathbf{z}}%
             =\frac{\mathbf{r}_{S}\times
             \mathbf{r}_{E}}{\norm{\mathbf{r}_{S}\times
             \mathbf{r}_{E}}}$
        \end{theorem}
        \begin{proof}
            As the ray plane is the plane $OEC$,
            $\mathbf{r}_{S}$ and $\mathbf{r}_{E}$ lie parallel
            to this plane. Moreover, during an occultation,
            $\mathbf{r}_{S}$ and $\mathbf{r}_{E}$ are not
            parallel and therefore $OEC$ is uniquely determined
            by $\mathbf{r}_{E}$, $\mathbf{r}_{S}$, and the point
            $O$. But
            $\hat{\mathbf{z}}%
             =\frac{\mathbf{r}_{S}\times
             \mathbf{r}_{E}}{\norm{\mathbf{r}_{S}\times
             \mathbf{r}_{E}}}$
            is perpendicular to both $\mathbf{r}_{E}$ and
            $\mathbf{r}_{S}$. Therefore $\hat{\mathbf{z}}$ is
            perpendicular to the ray plane.
        \end{proof}
        \begin{theorem}
            \label{theorem:r_e_dot_p_out_equal_p_out_square}
            $\mathbf{r}_{E}\cdot\mathbf{p}_{out}%
             =\norm{\mathbf{p}_{out}}^{2}$
        \end{theorem}
        \begin{proof}
            $\mathbf{p}_{out}$ is the projection of the $O$
            onto $\overline{EA}$. But $\overline{EA}$ lies
            parallel to $\hat{\mathbf{n}}_{out}$, and
            therefore $\mathbf{p}_{out}$ and
            $\hat{\mathbf{n}}_{out}$ are orthogonal,
            and thus
            $\mathbf{p}_{out}\cdot\hat{\mathbf{n}}_{out}=0$.
            Moreoever,
            $\mathbf{r}_{E}%
             =\mathbf{p}_{out}+(\mathbf{r}_{E}\cdot
             \hat{\mathbf{n}}_{out}) \hat{\mathbf{n}}_{out}$.
            But then:
            \begin{align*}
                \mathbf{p}_{out}\cdot \mathbf{r}_{E}
                &=\mathbf{p}_{out}\cdot\big(
                    \mathbf{p}_{out}
                    +(\mathbf{r}_{E}\cdot
                    \hat{\mathbf{n}}_{out})
                    \hat{\mathbf{n}}_{out}
                \big)\\
                \Rightarrow\mathbf{p}_{out}\cdot \mathbf{r}_{E}
                &=\mathbf{p}_{out}\cdot\mathbf{p}_{out}
                 +(\mathbf{r}_{E}\cdot\hat{\mathbf{n}}_{out})
                  \mathbf{p}_{out}\cdot \hat{\mathbf{n}}_{out}\\
                \Rightarrow\mathbf{p}_{out}\cdot\mathbf{r}_{E}
                &=\mathbf{p}_{out}\cdot\mathbf{p}_{out}
            \end{align*}
            Therefore
            $\mathbf{p}_{out}\cdot\mathbf{r}_{E}%
             =\norm{\mathbf{p}_{out}}^{2}$
        \end{proof}
        \begin{theorem}
            $\alpha%
             =\cos^{-1}(\hat{\mathbf{n}}_{in}
              \cdot\hat{\mathbf{n}}_{out})$
        \end{theorem}
        \begin{proof}
            By definition,
            $\alpha=\pi-\angle{EAC}$.
            But $\hat{\mathbf{n}}_{out}$ lies parallel to
            $\overrightarrow{AE}$, and $-\hat{\mathbf{n}}_{in}$
            lies parallel to $\overrightarrow{AC}$. Therefore:
            \begin{equation*}
                -\hat{\mathbf{n}}_{out}\cdot
                 \hat{\mathbf{n}}_{in}
                =\hat{\mathbf{n}}_{out}\cdot
                 (-\hat{\mathbf{n}}_{in})
                =\norm{\hat{\mathbf{n}}_{out}}
                 \norm{-\hat{\mathbf{n}}_{in}}\cos(\angle{EAC})
            \end{equation*}
            But $\hat{\mathbf{n}}_{in}$ and
            $\hat{\mathbf{n}}_{out}$ are unit vectors,
            and therefore
            $\norm{\hat{\mathbf{n}}_{out}}%
             =\norm{-\hat{\mathbf{n}}_{in}}=1$.
            Therefore:
            \begin{equation*}
                \angle EAC
                =\cos^{-1}(
                    -\hat{\mathbf{n}}_{out}\cdot
                    \hat{\mathbf{n}}_{in}
                )
            \end{equation*}
            But $\alpha=\pi-\angle{EAC}$,
            and $\cos^{-1}(-x)=\pi-\cos^{-1}(x)$.
            Therefore:
            \begin{equation*}
                \alpha=\pi-\angle{EAC}
                =\pi-\big(
                    \pi-\cos^{-1}(\hat{\mathbf{n}}_{out}\cdot
                    \hat{\mathbf{n}}_{in})
                \big)
                =\cos^{-1}(\hat{\mathbf{n}}_{out}\cdot
                 \hat{\mathbf{n}}_{in})
            \end{equation*}
        \end{proof}
        \begin{theorem}
            $\theta%
             =\cos^{-1}\big(%
                  \frac{(-\mathbf{r}_{S})\cdot%
                  \hat{\mathbf{n}}_{in}}{\norm{\mathbf{r}_{S}}}%
              \big)$
        \end{theorem}
        \begin{proof}
            For $\theta=\angle OCA$.
            But $\hat{\mathbf{n}}_{in}$ is parallel with
            $\overrightarrow{CA}$, and $(-\mathbf{r}_{S})$
            is parallel with $\overrightarrow{CO}$.
            Therefore:
            \begin{align*}
                (-\mathbf{r}_{S})\cdot\hat{\mathbf{n}}_{in}
                &=\norm{(-\mathbf{r}_{S})}
                  \norm{\hat{\mathbf{n}}_{in}}\cos(\theta)\\
                \Rightarrow\theta
                &=\cos^{-1}\bigg(
                    \frac{%
                        (-\mathbf{r}_{S})\cdot
                        \hat{\mathbf{n}}_{in}
                    }{\norm{\mathbf{r}_{S}}}
                \bigg)
            \end{align*}
        \end{proof}
        \begin{theorem}
            \begin{equation*}
                \beta=\pi-\frac{1}{2}\cos^{-1}
                \Big(\frac{\mathbf{r}_{s}
                     \cdot\mathbf{r}_{E}}
                     {\norm{\mathbf{r}_{s}}
                     \norm{\mathbf{r}_{E}}}\Big)
                -\frac{1}{2}\cos^{-1}
                \Big(\frac{\mathbf{r}_{E}\cdot
                           \hat{\mathbf{n}}_{out}}
                          {\norm{\mathbf{r}_{E}}}\Big)
                -\frac{1}{2}\cos^{-1}
                \Big(\frac{(-\mathbf{r}_{s})\cdot
                           \hat{\mathbf{n}}_{in}}
                          {\norm{\mathbf{r}_{s}}}\Big)
            \end{equation*}
        \end{theorem}
        \begin{proof}
            The sum of the angles in $OEAC$ is $2\pi$.
            But
            $\angle{OAE}=\angle{OAC}=\phi$, and therefore:
            \begin{align*}
                2\beta&=\angle{EAC}\\
                \Rightarrow
                2\pi
                &=2\beta+\angle{AEO}
                 +\angle{EOC}+\angle{OCA}\\
                \Rightarrow\beta
                &=\pi-\frac{\angle{AEO}}{2}
                 -\frac{\angle EOC}{2}-\frac{\angle OCA}{2}
            \end{align*}
            But:
            \begin{align*}
                (-\hat{\mathbf{n}}_{out})\cdot
                (-\hat{\mathbf{r}}_{E})
                &=\norm{\mathbf{r}_{E}}\cos(\angle AEO)\\
                \Rightarrow\angle{AEO}
                &=\cos^{-1}\bigg(
                    \frac{
                        \hat{\mathbf{n}}_{out}\cdot
                        \mathbf{r}_{E}
                    }{\norm{\mathbf{r}_{E}}}
                \bigg)
            \end{align*}
            Also:
            \begin{align*}
                \mathbf{r}_{E}\cdot\mathbf{r}_{S}
                &=\norm{\mathbf{r}_{E}}
                  \norm{\mathbf{r}_{S}}\cos(\angle EOC)\\
                \Rightarrow\angle{EOC}
                &=\cos^{-1}
                  \bigg(
                      \frac{
                          \mathbf{r}_{E}\cdot
                          \mathbf{r}_{S}
                      }{
                          \norm{\mathbf{r}_{E}}
                          \norm{\mathbf{r}_{S}}
                      }
                  \bigg)
            \end{align*}
            But
            $\angle{OCA}%
             =\theta%
             =\cos^{-1}%
              \big(\frac{%
                       (-\mathbf{r}_{S})\cdot%
                       \hat{\mathbf{n}}_{in}%
                   }{\norm{\mathbf{r}_{S}}}%
              \big)$.
            Therefore:
            \begin{equation*}
                \beta=\pi-\frac{1}{2}\cos^{-1}
                    \bigg(
                        \frac{
                            \mathbf{r}_{s}\cdot
                            \mathbf{r}_{E}
                        }{
                            \norm{\mathbf{r}_{s}}
                            \norm{\mathbf{r}_{E}}
                        }
                    \bigg)
                    -\frac{1}{2}\cos^{-1}
                    \bigg(
                        \frac{
                            \mathbf{r}_{E}\cdot
                            \hat{\mathbf{n}}_{out}
                        }{
                            \norm{\mathbf{r}_{E}}
                        }
                    \bigg)
                    -\frac{1}{2}\cos^{-1}
                    \bigg(
                        \frac{
                            (-\mathbf{r}_{s})\cdot
                            \hat{\mathbf{n}}_{in}
                        }{
                            \norm{\mathbf{r}_{s}}
                        }
                    \bigg)
            \end{equation*}
        \end{proof}
        \begin{theorem}
            $\alpha=\pi-2\beta$
        \end{theorem}
        \begin{proof}
            $\alpha$ and $\angle EAC$ are supplementary to
            the ray $\overrightarrow{CA}$, and therefore
            $\alpha+\angle EAC=\pi$.
            But $\angle{EAC}=\angle{EAC}+\angle{OAC}=2\beta$.
            Therefore $\alpha+2\beta=\pi$.
            Thus, $\alpha=\pi-2\beta$.
        \end{proof}
        \begin{theorem}
            $\theta=\frac{\pi}{2}+\frac{\alpha}{2}-\phi$
        \end{theorem}
        \begin{proof}
            As the angles of a triangle sum to $\pi$,
            $\theta+\beta+\phi=\pi$. But
            $\alpha=\pi-2\beta\Rightarrow\beta%
             =\frac{\pi}{2}-\frac{\alpha}{2}$.
            So we have:
            \begin{align*}
                \theta+\phi+\beta
                &=\pi\\
                \Rightarrow
                \theta+\phi+\frac{\pi}{2}-\frac{\alpha}{2}
                &=\pi\\
                \Rightarrow\theta
                &=\frac{\pi}{2}+\frac{\alpha}{2}-\phi
            \end{align*}
        \end{proof}
        \begin{theorem}
            \begin{equation*}
                \phi=\frac{1}{2}\cos^{-1}
                    \Big(\frac{\mathbf{r}_{s}\cdot
                               \mathbf{r}_{E}}
                              {\norm{\mathbf{r}_{s}}
                               \norm{\mathbf{r}_{E}}}\Big)
                +\frac{1}{2}\cos^{-1}
                \Big(\frac{\mathbf{r}_{E}\cdot
                           \hat{\mathbf{n}}_{out}}
                          {\norm{\mathbf{r}_{E}}}\Big)
                -\frac{1}{2}\cos^{-1}
                \Big(\frac{(-\mathbf{r}_{s})\cdot
                           \hat{\mathbf{n}}_{in}}
                          {\norm{\mathbf{r}_{s}}}\Big)
            \end{equation*}
        \end{theorem}
        \begin{proof}
            For:
            \begin{align*}
                \pi&=\beta+\theta+\phi\\
                \theta
                &=\cos^{-1}
                    \bigg(
                        \frac{
                            (-\mathbf{r}_{S})\cdot
                            \hat{\mathbf{n}}_{in}
                        }{\norm{\mathbf{r}_{S}}}
                    \bigg)\\
                \beta&= \pi-\frac{1}{2}\cos^{-1}
                    \bigg(
                        \frac{
                            \mathbf{r}_{s}\cdot
                            \mathbf{r}_{E}
                        }{
                            \norm{\mathbf{r}_{s}}
                            \norm{\mathbf{r}_{E}}
                        }
                    \bigg)
                    -\frac{1}{2}\cos^{-1}
                        \bigg(
                            \frac{
                                \mathbf{r}_{E}\cdot
                                \hat{\mathbf{n}}_{out}
                            }{\norm{\mathbf{r}_{E}}}
                        \bigg)
                        -\frac{1}{2}\cos^{-1}
                            \bigg(
                                \frac{
                                    (-\mathbf{r}_{s})\cdot
                                    \hat{\mathbf{n}}_{in}
                                }{\norm{\mathbf{r}_{s}}}
                            \bigg)\\
                \Rightarrow\phi
                &=\frac{1}{2}\cos^{-1}
                    \bigg(
                        \frac{
                            \mathbf{r}_{s}\cdot
                            \mathbf{r}_{E}
                        }{
                            \norm{\mathbf{r}_{s}}
                            \norm{\mathbf{r}_{E}}
                        }
                    \bigg)
                 +\frac{1}{2}\cos^{-1}
                     \bigg(
                         \frac{
                             \mathbf{r}_{E}\cdot
                             \hat{\mathbf{n}}_{out}
                         }{
                             \norm{\mathbf{r}_{E}}
                         }
                     \bigg)
                 -\frac{1}{2}\cos^{-1}
                     \bigg(
                         \frac{
                             (-\mathbf{r}_{s})\cdot
                             \hat{\mathbf{n}}_{in}
                         }{
                             \norm{\mathbf{r}_{s}}
                         }
                     \bigg)
            \end{align*}
        \end{proof}
        \begin{theorem}
            \label{%
                theorem:impact_parameter_p_%
                closed_form_solution
            }
            $p=\norm{\mathbf{p}_{in}}%
              =\norm{\mathbf{r}_{S}}%
               \cos(\phi-\frac{\alpha}{2})$
        \end{theorem}
        \begin{proof}
            As $P$ is the orthogonal projection of $O$
            onto $\overline{CA}$,
            $\angle{OPC}=\frac{\pi}{2}$. But then:
            \begin{equation*}
                |\overline{OP}|
                =|\overline{OC}|\sin(\angle{OCP})
            \end{equation*}
            But $|\overline{OP}|=\norm{\mathbf{p}_{in}}$,
            $|\overline{OC}|=\norm{\mathbf{r}_{S}}$,
            and $\angle{OCP}=\theta$. Therefore:
            \begin{equation*}
                \norm{\mathbf{p}_{in}}
                =\norm{\mathbf{r}_{S}}\sin(\theta)
            \end{equation*}
            But $\theta=\frac{\pi}{2}+\frac{\alpha}{2}-\phi$,
            and $\sin(\frac{\pi}{2}+x)=\cos(x)$. Therefore:
            \begin{equation*}
                \norm{\mathbf{p}_{in}}
                =\norm{\mathbf{r}_{S}}\cos
                    \big(\frac{\alpha}{2}-\phi\big)
            \end{equation*}
        \end{proof}
        \begin{theorem}
            $\norm{\mathbf{p}_{in}}%
             =|\overline{OA}|\sin(\beta)$
        \end{theorem}
        \begin{proof}
            For $\overline{OP}$ is perpendicular to
            $\overline{CA}$, and therefore $\Delta OPA$
            is a right-angled triangle, and $\overline{OA}$
            is the hypotenuse. Moreoever
            $\angle{PAO}=\beta$. But then:
            \begin{align*}
                |\overline{OP}|
                &=|\overline{OA}|\sin(\angle PAO)\\
                \Rightarrow|OP|
                &=|\overline{OA}|\sin(\beta)
            \end{align*}
            But $\norm{\mathbf{p}_{in}}=|\overline{OP}|$, and
            thus $\norm{\mathbf{p}_{in}}=|OA|\sin(\beta)$
        \end{proof}
        \begin{theorem}
            \label{theorem:p_out_equals_p_in}
            $\norm{\mathbf{p}_{in}}=\norm{\mathbf{p}_{out}}$
        \end{theorem}
        \begin{proof}
            For $\angle OAE=\angle{OAC}=\beta$, and thus:
            \begin{equation*}
                \norm{\mathbf{p}}_{out}
                =|\overline{OA}|\sin(\angle OAE)
                =|\overline{OA}|\sin(\beta)
                =\norm{\mathbf{p}}_{in}
            \end{equation*}
        \end{proof}
    \section{Ring Geometry}
        \begin{theorem}
            \label{theorem:ring_occ_geom_x_y_z_orthonormal_basis}
            If $\hat{\mathbf{u}}$ and $\hat{\mathbf{z}}$ are unit
            vectors and
            $\hat{\mathbf{u}}\times%
             \hat{\mathbf{z}}\ne\mathbf{0}$,
            then:
            \begin{equation}
                \label{eqn:Cassini_Math_Saturn_Basis}
                \{\hat{\mathbf{x}},\hat{\mathbf{y}},
                \hat{\mathbf{z}}\}
                =\Big\{\big(
                    \frac{\hat{\mathbf{u}}\times\hat{\mathbf{z}}}
                         {\norm{\hat{\mathbf{u}}\times
                          \hat{\mathbf{z}}}}
                \big)\times\hat{\mathbf{z}},
                \frac{\hat{\mathbf{u}}\times\hat{\mathbf{z}}}
                     {\norm{\hat{\mathbf{u}}\times
                      \hat{\mathbf{z}}}},
                \hat{\mathbf{z}}\Big\}
            \end{equation}
            is an orthonormal basis of $\mathbb{R}^{3}$.
        \end{theorem}
        \begin{proof}
            Since $\hat{\mathbf{u}}\times\hat{\mathbf{z}}$ is a
            non-zero vector,
            $\norm{\mathbf{u}\times\mathbf{z}}\ne{0}$. Thus, let
            $\hat{\mathbf{y}}%
             =\frac{\hat{\mathbf{u}}\times\hat{\mathbf{z}}}%
                   {\norm{\hat{\mathbf{u}}\times\hat{\mathbf{z}}}}$
            and let
            $\hat{\mathbf{x}}%
             =\hat{\mathbf{y}}\times\hat{\mathbf{z}}$.
            Then $\hat{\mathbf{y}}\cdot\hat{\mathbf{z}}=0$,
            $\hat{\mathbf{y}}\cdot\hat{\mathbf{x}}=0$,
            and $\hat{\mathbf{x}}\cdot\hat{\mathbf{z}}=0$.
            Both $\hat{\mathbf{z}}$ and $\hat{\mathbf{y}}$ are unit
            vectors by definition, and $\hat{\mathbf{x}}$ is the
            cross product of two orthogonal unit vectors, and is
            therefore itself a unit vector. But then
            $\{\hat{\mathbf{x}},\hat{\mathbf{y}},\hat{\mathbf{z}}\}$
            is a set of 3 mutually orthogonal unit vectors.
            By the Vector Space Dimension Theorem,
            $\{\hat{\mathbf{x}},%
               \hat{\mathbf{y}},%
               \hat{\mathbf{z}}\}$
            is an orthonormal basis of $\mathbb{R}^3$.
        \end{proof}
        We define our Saturnian Coordinate System to be the
        Cartesian Coordinate System where $\mathbf{u}$
        is the vector from Earth to the Spacecraft,
        $\hat{\mathbf{z}}$ is Saturn's Pole vector, and let
        $\hat{\mathbf{x}}$ and $\hat{\mathbf{y}}$
        be as defined in
        Eqn.~\ref{eqn:Cassini_Math_Saturn_Basis}.
        The origin is taken to be Saturn's Center.
        The ring plane of Saturn is the plane perpendicular
        to $\hat{\mathbf{z}}$ which contains the origin.
        \begin{theorem}
            Saturn's ring plane is the $xy$ plane.
        \end{theorem}
        \begin{proof}
            This is a restatement of the fact that
            $\{\hat{\mathbf{x}},%
               \hat{\mathbf{y}},%
               \hat{\mathbf{z}}\}$
            is an orthonormal system
            (Thm.~\ref{%
                theorem:ring_occ_geom_%
                x_y_z_orthonormal_basis%
            })
            and from the definition of Saturn's ring plane.
        \end{proof}
        \begin{theorem}
            \label{thm:Cassini_Math_u_parallel_xy}
            The Earth-Spacecraft line, $\mathbf{u}$,
            lies parallel to the $xz$ plane.
        \end{theorem}
        \begin{proof}
            It suffices to show that $\hat{\mathbf{u}}$
            is orthogonal to $\hat{\mathbf{y}}$.
            But:
            \begin{equation}
                \hat{\mathbf{u}}\cdot\hat{\mathbf{y}}
                =\hat{\mathbf{u}}\cdot
                \frac{\hat{\mathbf{u}}\times \hat{\mathbf{z}}}
                     {\norm{\hat{\mathbf{u}}
                      \times \hat{\mathbf{z}}}}
            \end{equation}
            And for any two vectors
            $\mathbf{a}$ and $\mathbf{b}$,
            $\mathbf{a}\cdot(\mathbf{a}\times\mathbf{b})%
             =\mathbf{0}$
            and therefore $\hat{\mathbf{u}}$ is orthogonal
            to $\hat{\mathbf{y}}$. Thus
            $\hat{\mathbf{u}}$ is parallel to the $xz$ plane.
        \end{proof}
        \begin{theorem}
            \label{thm:Cassini_Math_Earth_Line_Parallel_xz}
            In the Saturn Reference frame,
            Earth lies on the $xz$ plane if and
            only if the line from
            Earth to Cassini lies in it.
        \end{theorem}
        \begin{proof}
            If $\hat{\mathbf{u}}$ lies in the $xz$ plane,
            then Earth must also lie in it from the
            definition of $\mathbf{u}$. And from
            Thm.~\ref{thm:Cassini_Math_Earth_Line_Parallel_xz}
            $\hat{\mathbf{u}}$ lies parallel
            to the $xz$ plane. Thus, if Earth lies
            in the $xz$ plane, so must the line from
            Earth to Cassini.
        \end{proof}
        \begin{theorem}
            If $\hat{\mathbf{z}}$ and $\hat{\mathbf{u}}$
            are defined by:
            \begin{subequations}
                \begin{align}
                    \label{eqn:Cassini_Math_Sat_Pole_Coord}
                    \hat{\mathbf{z}}
                    &=z_{1}\hat{\mathbf{x}}_{E}+
                    z_{2}\hat{\mathbf{y}}_{E}+
                    z_3\hat{\mathbf{z}}_{E}\\
                    \label{eqn:Cassini_Math_RIP_Coord}
                    \hat{\mathbf{u}}
                    &=u_{E_{x}}\hat{\mathbf{x}}_{E}+
                    u_{E_{y}}\hat{\mathbf{y}}_{E}+
                    u_{E_{z}}\hat{\mathbf{z}}_{E}
                \end{align}
                then:
                \begin{equation}
                    \hat{\mathbf{y}}
                    =y_{1}\hat{\mathbf{x}}_{E}+
                     y_{2}\hat{\mathbf{y}}_{E}+
                     y_{3}\hat{\mathbf{z}}_{E}
                \end{equation}
                Where:
                \begin{align}
                    y_1
                    &=\frac{z_2u_{E_{z}}-z_{3}u_{E_{y}}}
                           {\sqrt{(z_2u_{E_{z}}-z_3u_{E_{y}})^2+
                            (z_3u_{E_{x}}-z_1u_{E_{z}})^2+
                            (z_1u_{E_{y}}-z_2u_{E_{x}})^2}}\\
                    y_{2}&=
                        \frac{z_3u_{E_{x}}- z_{1}u_{E_{z}}}
                             {\sqrt{(z_2u_{E_{z}}-z_3u_{E_{y}})^2+
                              (z_3u_{E_{x}}-z_1u_{E_{z}})^2+
                              (z_1u_{E_{y}}-z_2u_{E_{x}})^2}}\\
                    y_{3}&=
                        \frac{z_1u_{E_{y}}-z_2u_{E_{x}}}
                             {\sqrt{(z_2u_{E_{z}}-z_3u_{E_{y}})^2+
                              (z_3u_{E_{x}}-z_1u_{E_{z}})^2+
                              (z_1u_{E_{y}}-z_2u_{E_{x}})^2}}
                \end{align}
            \end{subequations}
        \end{theorem}
        \begin{proof}
            From the definition given in
            Thm.~\ref{theorem:ring_occ_geom_%
                      x_y_z_orthonormal_basis},
            $\hat{\mathbf{y}}$ is defined as
            $\frac{\hat{\mathbf{z}}\times\mathbf{u}_{0}}%
                  {\norm{\hat{\mathbf{z}}\times\mathbf{u}_{0}}}$.
            This equation is the
            cross-product divided by the norm.
        \end{proof}
        \begin{theorem}
            If $\hat{\mathbf{z}}$ and
            $\mathbf{u}_{0}$ are as defined in
            Eqn.~\ref{eqn:Cassini_Math_Sat_Pole_Coord}
            and \ref{eqn:Cassini_Math_RIP_Coord}, then:
            \begin{subequations}
                \begin{equation}
                    \hat{\mathbf{x}}=
                        x_{1}\hat{\mathbf{x}}_{E}+
                        x_{2}\hat{\mathbf{y}}_{E}+
                        x_{3}\hat{\mathbf{z}}_{E}
                \end{equation}
                Where:
                \begin{align}
                    x_1
                    &=\frac{z_{3}(z_{3}u_{E_{x}}-z_{1}u_{E_{z}})-
                            z_{2}(z_{1}u_{E_{y}}-z_{2}u_{E_{x}})}
                           {\sqrt{(z_{2}u_{E_{z}}-z_{3}u_{E_{y}})^{2}+
                            (z_{3}u_{E_{x}}-z_{1}u_{E_{z}})^{2}+
                            (z_{1}u_{E_{y}}-z_{2}u_{E_{x}})^{2}}}\\
                    x_2
                    &=\frac{z_{1}(z_{1}u_{E_{y}}-
                            z_{2}u_{E_{x}})-z_{3}(z_{2}u_{E_{z}}-
                            z_{3}u_{E_{y}})}
                           {\sqrt{(z_{2}u_{E_{z}}-z_{3}u_{E_{y}})^{2}+
                            (z_{3}u_{E_{x}}-z_{1}u_{E_{z}})^{2}+
                            (z_{1}u_{E_{y}}-z_{2}u_{E_{x}})^{2}}}\\
                    x_3
                    &=\frac{z_{2}(z_{2}u_{E_{z}}-
                            z_{3}u_{E_{y}})-z_{1}(z_{3}u_{E_{x}}-
                            z_{1}u_{E_{z}})}
                           {\sqrt{(z_{2}u_{E_{z}}-z_{3}u_{E_{y}})^{2}+
                            (z_{3}u_{E_{x}}-z_{1}u_{E_{z}})^{2}+
                            (z_{1}u_{E_{y}}-z_{2}u_{E_{x}})^{2}}}
                \end{align}
            \end{subequations}
            \end{theorem}
        \begin{proof}
            $\hat{\mathbf{x}}$ is defined as
            $\hat{\mathbf{y}}\times\hat{\mathbf{z}}$.
            This equation is merely that product.
        \end{proof}
        Thus if we have $\hat{\mathbf{u}}$ and
        $\hat{\mathbf{z}}$ in an Earth based system,
        we can easily compute the geometry in our
        Saturnian coordinate system. At the very least,
        a computer can easily compute this.
        \begin{theorem}
            If $(S_{x},S_{y},S_{z})$ is location of the
            center of Saturn with respect to the
            center of the Earth and $(x_{E},y_{E},z_{E})$
            is a point in $\mathbb{R}^{3}$ with respect
            to the center of the Earth, then the
            change of coordinates to the
            Saturn-based system is:
            \begin{equation*}
                    \begin{pmatrix}
                        x\\
                        y\\
                        z
                    \end{pmatrix}
                    =
                    \begin{pmatrix}
                        x_{1}&x_{2}&x_{3}\\
                        y_{1}&y_{2}&y_{3}\\
                        z_{1}&z_{2}&z_{3}
                    \end{pmatrix}
                    \begin{pmatrix}
                        x_{E}-S_{x}\\
                        y_{E}-S_{y}\\
                        z_{E}-S_{z}
                    \end{pmatrix}
                \end{equation*}
        \end{theorem}
        \begin{proof}
            The point
            $(x_{E}-S_{x},y_{E}-S_{y},z_{E}-S_{z})$
            translates the point $(x_{E},y_{E},z_{E})$
            to the center of Saturn.
            The rotation matrix then aligns the
            Earth-based coordinates to the
            Saturn-based coordinates.
        \end{proof}
    \section{Derivations of the Fresnel Kernel}
            Let $\hat{\mathbf{u}}$ be the unit
            vector pointing from Earth to the spacecraft.
            Let $\hat{\mathbf{z}}$ be the pole direction
            Saturn. To make the arguments easier,
            we assume the line from Earth to Saturn
            and the line from Earth to Voyager are
            parallel. That is, we assume that
            Saturn is infinitely far away.
            Define the following:
            \par
            \begin{equation}
                \label{eqn:Cassini_Math_Def_B}
                B=\sin^{\minus{1}}
                    (\hat{\mathbf{z}}\cdot\hat{\mathbf{u}})
            \end{equation}
            \begin{subequations}
                \begin{minipage}{0.49\textwidth}
                    \begin{equation}
                        \hat{\mathbf{y}}=
                        \frac{\hat{\mathbf{u}}\times
                              \hat{\mathbf{z}}}
                             {\norm{\hat{\mathbf{u}}\times
                              \hat{\mathbf{z}}}}
                    \end{equation}
                \end{minipage}
                \hfill
                \begin{minipage}{0.49\textwidth}
                    \begin{equation}
                        \hat{\mathbf{x}}
                        =\hat{\mathbf{y}}\times
                        \hat{\mathbf{z}}
                    \end{equation}
                \end{minipage}
            \end{subequations}
            \par\hfill\par
            We take the origin as Saturn's center. From
            Thm.~\ref{thm:Cassini_Math_u_parallel_xy},
            $\hat{\mathbf{u}}$ lies parallel to the
            $xz$ plane, and thus there are numbers
            $a_{1},a_{2}$ such that:
            \begin{equation}
                \hat{\mathbf{u}}=a_{1}\hat{\mathbf{x}}+
                a_{2}\hat{\mathbf{z}}
            \end{equation}
            We can compute for $a_{1}$ and $a_{2}$ by using
            the definition of $B$ in
            Eqn.~\ref{eqn:Cassini_Math_Def_B}.
            \begin{equation}
                \hat{\mathbf{u}}
                =\cos(B)\hat{\mathbf{x}}+
                \sin(B)\hat{\mathbf{z}}
            \end{equation}
            Let $\boldsymbol{\uprho}_{0}$ be the vector pointing
            from Saturn to the ring intercept point,
            and let $\boldsymbol{\uprho}$ be a vector in
            the ring plane. Let $\phi_{0}$ and
            $\phi$ be the angles made with
            $\boldsymbol{\uprho}_{0}$ and $\boldsymbol{\uprho}$
            to the $x$ axis, respectively. Then:
            \par
            \vspace{-1ex}
            \begin{subequations}
                \begin{minipage}{0.49\textwidth}
                    \begin{equation}
                        \boldsymbol{\uprho}_{0}
                        =\rho_{0}\big(\cos(\phi_0)
                        \hat{\mathbf{x}}+
                        \sin(\phi_{0})\hat{\mathbf{y}}\big)
                    \end{equation}
                \end{minipage}
                \hfill
                \begin{minipage}{0.49\textwidth}
                    \begin{equation}
                        \boldsymbol{\uprho}
                        =\rho\big(\cos(\phi)
                        \hat{\mathbf{x}}+
                        \sin(\phi)\hat{\mathbf{y}}\big)
                    \end{equation}
                \end{minipage}
            \end{subequations}
            \par\hfill\par\hfill\par
            \vspace{-1.5ex}
            Let $\mathbf{R}_{c}$ be the vector pointing
            from Saturn to Voyager. Let $D$ be the
            distance from the ring intercept point
            to Voyager.
            We thus have the following:
            \begin{subequations}
                \begin{align}
                    \mathbf{R}_{c}&=
                    \boldsymbol{\uprho}_{0}+
                    D\hat{\mathbf{u}}\\
                    &=\big(
                    \rho_{0}\cos(\phi_0)+D\cos(B)
                    \big)\hat{\mathbf{x}}+
                    \rho_{0}\sin(\phi_{0})\hat{\mathbf{y}}+
                    D\sin(B)\hat{\mathbf{z}}
                \end{align}
            \end{subequations}
            We wish to compute
            $\hat{\mathbf{u}}\cdot%
             \boldsymbol{\uprho}+%
             \norm{\mathbf{R}_{c}-\boldsymbol{\uprho}}$.
            We have:
            \begin{subequations}
                \begin{align}
                    \hat{\mathbf{u}}\cdot\boldsymbol{\uprho}
                    &=\big(\cos(B)\hat{\mathbf{x}}+
                           \sin(B)\hat{\mathbf{z}}\big)
                        \cdot\big(\rho(\cos(\phi)\hat{\mathbf{x}}+
                                  \sin(\phi)\hat{\mathbf{y}})\big)\\
                    &=\rho\cos(B)\cos(\phi)
                \end{align}
            \end{subequations}
            And also:
            \begin{subequations}
                \begin{align}
                    \norm{\mathbf{R}_{c}-\boldsymbol{\uprho}}
                    &=\sqrt{(\mathbf{R}_{c}-
                    \boldsymbol{\uprho})\cdot(\mathbf{R}_{c}-
                    \boldsymbol{\uprho})}\\
                    &=\sqrt{\norm{\mathbf{R}_{c}}^2+
                    \norm{\boldsymbol{\uprho}}^2-
                    2\mathbf{R}_{c}\cdot\boldsymbol{\uprho}}
                \end{align}
            \end{subequations}
            But, since $\hat{\mathbf{u}}$ is a unit vector and
            $\boldsymbol{\rho}_{0}\cdot\boldsymbol{\rho}_{0}%
             =\rho_{0}^{2}$,
            we have:
            \begin{subequations}
                \begin{align}
                    \norm{\mathbf{R}_{c}}^{2}
                    &=\rho_{0}^2+D^{2}+2\rho_{0}D
                    \boldsymbol{\uprho}_{0}\cdot\hat{\mathbf{u}}\\
                    &=\rho_{0}^2+D^{2}+
                    2D\rho_{0}\cos(\phi_{0})\cos(B)
                \end{align}
            \end{subequations}
            Furthering the computation we have:
            \begin{subequations}
                \begin{align}
                    \mathbf{R}_{c}\cdot\boldsymbol{\uprho}
                    &=\rho\cos(\phi)
                    \big(\rho_{0}\cos(\phi_{0})+D\cos(B)\big)+
                    \rho\rho_{0}\sin(\phi)\sin(\phi_{0})\\
                    &=\rho\rho_{0}
                    \big(\cos(\phi)\cos(\phi_{0})+
                         \sin(\phi)\sin(\phi_{0})\big)+
                    \rho{D}\cos(\phi)\cos(B)\\
                    &=\rho\rho_{0}\cos(\phi-\phi_{0})+
                    \rho{D}\cos(\phi)\cos(B)
                \end{align}
            \end{subequations}
            So we have:
            \begin{equation}
                \begin{split}
                    \norm{\mathbf{R}_{c}-\boldsymbol{\uprho}}^{2}
                    =\rho^{2}+\rho_{0}^{2}+D^{2}&-
                    2\rho\rho_{0}\cos(\phi-\phi_{0})
                    \\&+2D\cos(B)\big(\rho_{0}\cos(\phi_0)-
                    \rho\cos(\phi)\big)
                \end{split}
            \end{equation}
            Now the definition of $\hat{T}$ is:
            \begin{equation}
                \hat{T}=\frac{E_{c}}{E_{0}}
                e^{-ik\hat{\mathbf{u}}\cdot\mathbf{R}_{c}}
            \end{equation}
            So we define $\psi$ as:
            \begin{equation}
                \psi=
                k\big(\norm{\mathbf{R}_{c}-\rho}^{2}+
                      \hat{\mathbf{u}}\cdot\boldsymbol{\rho}-
                      \hat{\mathbf{u}}\cdot\mathbf{R}_{c}\big)
            \end{equation}
            Trudging along, we have:
            \begin{subequations}
                \begin{align}
                    \hat{\mathbf{u}}\cdot\mathbf{R}_{c}
                    &=\rho_{0}\cos(\phi_0)\cos(B)+
                    D\cos^{2}(B)+D\sin^2(B)\\
                    &=\rho_{0}\cos(\phi_{0})\cos(B)+D
                \end{align}
            \end{subequations}
            Let's define the following:
            \begin{align}
                \xi&=\frac{\cos(B)\big(\rho\cos(\phi)-
                           \rho_{0}\cos(\phi_{0})\big)}
                          {D}\\
                \eta&=\frac{\rho_{0}^2+\rho^2-
                            2\rho\rho_{0}\cos(\phi-\phi_{0})}
                           {D^{2}}
            \end{align}
            Please note that $\xi$ differs in sign from
            the definition found in MTR86. This is done
            intentionally in order for problem to lend
            itself more naturally to the use of
            Legendre polynomials.
            Using this, we finally obtain:
            \begin{equation}
                \psi=kD\big[\sqrt{1+\eta-2\xi}-(1-\xi)\big]
            \end{equation}
            An important configuration to consider is when
            $\phi=\phi_{0}$. Evaluating $\psi$, we obtain:
            \begin{equation}
                \begin{split}
                    \psi_{\phi=\phi_{0}}
                    =kD\Big[&\cos(\phi_{0})\cos(B)
                    \big(\frac{\rho-\rho_{0}}{D}\big)-1\\
                    &+\sqrt{1+
                    \big(\frac{\rho-\rho_{0}}{D}\big)^{2}
                    -2\cos(\phi_{0})\cos(B)
                    \big(\frac{\rho-\rho_{0}}{D}\big)}\Big]
                \end{split}
            \end{equation}
            Define the following:
            \par\hfill\par
            \vspace{-1ex}
            \begin{subequations}
                \begin{minipage}{0.49\textwidth}
                    \begin{equation}
                        x=\frac{\rho-\rho_{0}}{D}
                    \end{equation}
                \end{minipage}
                \hfill
                \begin{minipage}{0.49\textwidth}
                    \begin{equation}
                        \alpha=\cos(\phi_{0})\cos(B)
                    \end{equation}
                \end{minipage}
            \end{subequations}
            Then we may rewrite $\psi_{\phi=\phi_{0}}$ as:
            \begin{equation}
                \psi_{\phi=\phi_{0}}=
                kD\Big[\alpha{x}-1+\sqrt{1+x^{2}-2x}\Big]
            \end{equation}
            This is where the Legendre polynomials
            come into play. Letting $P_{n}(\alpha)$
            denote the $n^{th}$ Legendre polynomial,
            the generating function is:
            \begin{equation}
                \label{eqn:CASSINI:MATH:Legendre_Gen_Func}
                \sum_{n=0}^{\infty}P_{n}(\alpha)x^{n}
                =\frac{1}{\sqrt{1+x^{2}-2\alpha{x}}}
            \end{equation}
            We don't quite have this, but we can
            produce a differential equation that
            will lead us to this. First note
            the following:
            \begin{equation}
                \frac{\diff}{\diff{x}}
                \Big(\sqrt{1+x^{2}-2\alpha{x}}\Big)
                =\frac{x-\alpha}{\sqrt{1+x^{2}-2\alpha{x}}}
            \end{equation}
            But from
            Eqn.~\ref{eqn:CASSINI:MATH:Legendre_Gen_Func}, we
            have:
            \begin{equation}
                \frac{\diff}{\diff{x}}
                \Big(\sqrt{1+x^{2}-2\alpha{x}}\Big)
                =(x-\alpha)
                \sum_{n=0}^{\infty}P_{n}(\alpha)x^{n}
            \end{equation}
            Integrating across, we then obtain:
            \begin{equation}
                \sqrt{1+x^{2}-2\alpha{x}}
                =1+\sum_{k=0}^{\infty}P_{k}(\alpha)
                \frac{x^{k+2}}{k+2}
                -\alpha\sum_{k=0}^{\infty}P_{k}(\alpha)
                \frac{x^{k+1}}{k+1}
            \end{equation}
            And so finally:
            \begin{equation}
                \sqrt{1+x^{2}-2\alpha{x}}+\alpha{x}-1
                =\sum_{k=0}^{\infty}
                \Big(P_{k}(\alpha)-
                     \alpha{P_{k+1}}(\alpha)\Big)
                \frac{x^{k+2}}{k+2}
            \end{equation}
            We can use this to evaluate and
            approximate $\psi$.
            Define the following sequence:
            \begin{equation}
                b_{k}=
                \frac{P_{k}(\alpha)-\alpha{P_{k+1}}(\alpha)}
                     {k+2}
            \end{equation}
            Then $\psi_{\phi=\phi_{0}}$ may be expressed
            as follows:
            \begin{equation}
                \psi_{\phi=\phi_{0}}=
                kD\sum_{k=0}^{\infty}b_{k}x^{k+2}
            \end{equation}

        \input{p02c08s00_occultations.tex}
    \part{Application}
        \input{p03c09s00_cassini.tex}
        \input{p03c10s00_catalog.tex}
    \clearpage
    \printglossary[style=longpara]
\end{document}
