We begin with a few important topics in mathematics that one comes
across in the study of electromagnetism and diffraction theory.
This is particularly useful for studying occultation observations
of planetary rings. We will develop complex analysis, Fourier analysis,
approximation theory, and multivariate calculus so that we may
transition from the core of electromagnetism, \textit{Maxwell's Equations},
and derive the \textit{Fresnel-Huygens Principle}. This is the fundamental
equation on which diffraction theory is based.
\par\hfill\par
There is a standard set of notations in mathematics, and this is
presented below in Table~\ref{tab:Common_Notations}.
\begin{table}[H]
    \centering
    \captionsetup{type=table}
    \begin{tabular}{|l|l|}
        \hline
        Symbol&Definition\\
        \hline
        $\mathbb{N}$&Positive Integers\\
        \hline
        $\mathbb{Z}$&Integers\\
        \hline
        $\mathbb{Z}_{n}$&Positive Integers Between 1 and $n$.\\
        \hline
        $\mathbb{Q}$&Rational numbers\\
        \hline
        $\mathbb{R}$&Real Numbers\\
        \hline
        $\mathbb{C}$&Complex Numbers\\
        \hline
    \end{tabular}
    \caption{Common Notations}
    \label{tab:Common_Notations}
\end{table}
Given some number $x$, we denote that $x$ is a real number by writing
$x\in\mathbb{R}$, and similarly if $x$ is rational we write
$x\in\mathbb{Q}$. Conversely, we write $x\notin\mathbb{R}$ to denote that
$x$ is \textit{not} a real number. The $\in$ symbol should be read
\textit{is an element of}, so $x\in\mathbb{R}$ reads as $x$
\textit{is an element of} $\mathbb{R}$. Recall that a
\textit{sequence} of points in some set $A$ is an ordered list
$a_{1}$, $a_{2}$, $\dots$, $a_{n}$, $\dots$ such that $a_{k}\in{A}$
for all $k\in\mathbb{N}$. This definition lacks rigor in many
respects, and to make the notion useful we define a sequence as
a \textit{function} from $\mathbb{N}$ to $A$. That is, we write
$a:\mathbb{N}\rightarrow{A}$. Rather than writing $a(n)$, we write
$a_{n}$. Subscript notation is reserved for sequences. For a
general function $f:X\rightarrow{Y}$, we
write $f(x)$ to denote the point in $Y$ that corresponds to $x$.
