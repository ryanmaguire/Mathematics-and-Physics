\chapter{Knot Theory}
    \begin{figure}[H]
        \centering
        \includegraphics{images/Trefoil_Knot.pdf}
        \caption{A Trefoil Knot}
        \label{fig:Trefoil_Knot}
    \end{figure}
    % \begin{figure}[H]
    %     \centering
    %     \includegraphics{images/Trefoil_Gradient.pdf}
    %     \caption{A Colorful Trefoil Knot}
    %     \label{fig:Colorful_Trefoil_Knot}
    % \end{figure}
    \begin{figure}[H]
        \centering
        \includegraphics{images/Trefoil_Tricolor.pdf}
        \caption{Tricoloring of the Trefoil}
        \label{fig:Trefoil_Tricoloring}
    \end{figure}
    % \begin{figure}[H]
    %     \centering
    %     \includegraphics{images/Mobius_Strip.pdf}
    %     \caption{Mobius Strip}
    %     \label{fig:Mobius_Strip}
    % \end{figure}
    % \begin{figure}[H]
    %     \centering
    %     \resizebox{0.5\textwidth}{!}{%
    %         \includegraphics{images/Sea_Shell.pdf}
    %     }
    %     \caption{A Sea Shell}
    %     \label{fig:Sea_Shell}
    % \end{figure}
    \begin{figure}[H]
        \centering
        \includegraphics{images/Seifert_Trefoil.pdf}
        \caption{Seifert Surface for a Trefoil Knot}
        \label{fig:Seifert_Surface_Trefoil}
    \end{figure}
    \begin{figure}[H]
        \centering
        \resizebox{0.5\textwidth}{!}{%
            \includegraphics{images/Seifert_Hopf_Link.pdf}
        }
        \caption{Seifert Surface for a Hopf Link}
        \label{fig:Seifert_Surface_Hopf_Link}
    \end{figure}
    \begin{figure}[H]
        \centering
        \includegraphics{images/Trefoil_Non_Oriented_Surface.pdf}
        \caption{Non-Oriented Surface With Trefoil Boundary}
        \label{fig:Trefoil_Non_Oriented_Surface}
    \end{figure}
\section{Ina Petkova's Summer 2020 Lectures}
    \subsection{Knots and Surfaces}
        \begin{definition}
            A knot $K$ is a smooth embedding
            $K:\nsphere[1]\rightarrow\nspace[2]$, or just it's image in
            $\nspace[3]$. We draw a knot as a diagram by projecting to
            $\nspace[2]$ while making sure we don't have triple crossing points.
            Points that are projected onto twice are called crossings.
        \end{definition}
        \begin{example}
            Unknot, trefoil, trefoil's mirror, polygonal unknot, figure eight.
        \end{example}
        How to measure how complicated a knot is? How many moves are needed
        to deform it into the unknot, if possible? How many crossings need
        to be changed to create the unknot? How to classify knots?
        \begin{example}
            The trefoil can be unknotted with one change of crossings.
        \end{example}
        Construction of Seifert surfaces. First, orient the knot. Resolve all
        crossings.