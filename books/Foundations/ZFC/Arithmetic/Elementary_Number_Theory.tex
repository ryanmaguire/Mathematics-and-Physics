\section{Elementary Number Theory}
    \begin{theorem}
        \label{thm:Equiv_Classes_Form_Partition}%
        If $A$ is a set, if $R$ is an equivalence relation on $A$, and
        if $A/R$ is the quotient set, then $A/R$ is a partition of $A$.
    \end{theorem}
    \begin{proof}
        For all $x\in{A}$ it is true that $[x]\in{A}/R$ and since $R$ is
        an equivalence relation we have $x\in[x]$, hence $A/R$ covers
        $A$. Moreover, if $\mathcal{U},\mathcal{V}\in{A}/R$ and if
        $\mathcal{U}\cap\mathcal{V}$ is non-empty, then there is an
        $x\in{R}$ such that $[x]\in\mathcal{U}$ and $[x]\in\mathcal{V}$.
        But if $[y]\in\mathcal{U}$ and $[z]\in\mathcal{V}$, then
        $yRx$ and $xRz$. But then $yRz$ since $R$ is an equivalence
        relation and therefore $[y]\in\mathcal{V}$. Similarly,
        $[z]\in\mathcal{U}$. Hence, either $\mathcal{U}=\mathcal{V}$ or
        they are disjoint. Therefore, $A/R$ is a partition of $A$.
    \end{proof}
    \begin{theorem}
        If $A$ is a set, and if $\mathcal{O}\subseteq\powset{X}$ is a
        partition of $A$, then there is an equivalence relation $R$ on
        $A$ such that $\mathcal{O}=A/R$.
    \end{theorem}
    \begin{proof}
        For let $R\subseteq{A}\times{A}$ be defined by:
        \begin{equation}
            R=\{\,(x,y)\in{A}\times{A}\;|\;
                \exists_{\mathcal{U}\in\mathcal{O}}
                (x,y\in\mathcal{U})\,\}
        \end{equation}
        then $R$ is an equivalence relation. Since $\mathcal{O}$ is a
        partition, for all $x\in{A}$ there is a
        $\mathcal{U}\in\mathcal{O}$ such that $x\in\mathcal{U}$. But
        then $x\in\mathcal{U}$ and $x\in\mathcal{U}$, and therefore
        $(x,x)\in{R}$. That is, $xRx$. Moreover, if $xRy$ then there is
        a $\mathcal{U}\in\mathcal{O}$ such that $x\in\mathcal{U}$ and
        $y\in\mathcal{U}$. But then $y\in\mathcal{U}$ and
        $x\in\mathcal{U}$ and hence $yRx$. Lastly, if $xRy$ and $yRz$,
        then there is a set $\mathcal{U}\in\mathcal{O}$ and a set
        $\mathcal{V}\in\mathcal{O}$ such that $x,y\in\mathcal{U}$ and
        $y,z\in\mathcal{V}$. But $\mathcal{O}$ is a partition, and hence
        either $\mathcal{U}\cap\mathcal{V}=\emptyset$ or
        $\mathcal{U}=\mathcal{V}$. But
        $\mathcal{U}\cap\mathcal{V}\ne\emptyset$ since $y\in\mathcal{U}$
        and $y\in\mathcal{V}$. Therefore, $\mathcal{U}=\mathcal{V}$ and
        thus, since $z\in\mathcal{V}$, it is true that
        $z\in\mathcal{U}$. That is, $xRz$. Hence, $R$ is an equivalence
        relation. Moreover, by definition, $A/R=\mathcal{O}$.
    \end{proof}
    \begin{theorem}
        \label{thm:Fibers_of_Func_Form_Equiv_Relation}%
        If $A$ and $B$ are sets, if $f:A\rightarrow{B}$ is a function,
        and if $R\subseteq{A}\times{A}$ is the relation defined by:
        \begin{equation}
            R=\{\,(x,y)\in{A}\times{A}\;|\;f(x)=f(y)\,\}
        \end{equation}
        then $R$ is an equivalence relation on $A$.
    \end{theorem}
    \begin{proof}
        For all $x\in{A}$ it is true that $f(x)=f(x)$, and hence $xRx$.
        Moreover, if $x,y\in{A}$ and $xRy$, then $f(x)=f(y)$. But
        equality is reflexive, and hence $f(y)=f(x)$. But then $yRx$.
        Lastly, by the transitivity of equality, if $xRy$ and $yRz$,
        then $f(x)=f(y)$ and $f(y)=f(z)$, hence $f(x)=f(z)$. But then
        $xRz$. Thus, $R$ is and equivalence relation.
    \end{proof}
    \begin{theorem}
        \label{thm:Weak_Euc_Division_Alg}%
        If $n,m\in\mathbb{N}^{+}$, then there exists $q,r\in\mathbb{N}$
        such that $n=q\cdot{m}+r$ with $r<m$.
    \end{theorem}
    \begin{proof}
        For let $G\subseteq\mathbb{N}$ be defined by:
        \begin{equation}
            G=\{\,k\in\mathbb{N}\;|\;n-k\cdot{m}\geq{0}\,\}
        \end{equation}
        if $n<m$, then $G=\{0\}$ and hence choosing $r=n$ and $q=0$
        works. If $n=m$, $q=1$ and $r=0$ does the trick. Otherwise, $G$
        is non-empty and bounded since it is bounded by $n$, and hence
        there is a greatest element. Let $q\in{G}$ be the greatest
        element, and let $r=n-q\cdot{m}$. Since $q\in{G}$, $r>0$.
        Moreover, $r<m$. For if not, then
        $r-m=n-q\cdot{m}-m=n-(q+1)\cdot{m}\geq{0}$, which contradicts
        the maximality of $q$. But then $n=kq+r$ and $r<m$.
    \end{proof}
    There's a strengthening of this theorem, which dates back to
    antiquity.
    \begin{ftheorem}{Euclid's Division Algorithm}
                    {Euclid_Division_Algorithm}
        If $m,n\in\mathbb{Z}$, and if $m\ne{0}$, then there exists
        unique $q\in\mathbb{Z}$ and $r\in\mathbb{N}$ such that $r<|m|$
        and:
        \begin{equation*}
            n=mq+r
        \end{equation*}
    \end{ftheorem}
    \begin{bproof}
        For if $n=0$, then let $r=0$ and $q=0$. If $n>0$, and if $m>0$,
        then by Thm.~\ref{thm:Weak_Euc_Division_Alg} there exists
        $q,r$ such that $n=q\cdot{m}+r$ with $r<m$. If $n>0$ and $m<0$,
        then $\minus{m}>0$. But then by
        Thm.~\ref{thm:Weak_Euc_Division_Alg} there exists $q,r$ such
        that $n=q\cdot(\minus{m})+r$ with $r<\minus{m}$. But then
        $n=(\minus{q})\cdot{m}+r$, and $r<|m|$. If $n<0$ and $m>0$,
        then $\minus{n}>0$ and hence by
        Thm.~\ref{thm:Weak_Euc_Division_Alg} there exists
        $q,r\in\mathbb{N}$ such that $r<m$ and $\minus{n}=q\cdot{m}+r$.
        But then:
        \begin{equation}
            n=(\minus{q})\cdot{m}-r=(\minus{q}-1)\cdot{m}+(m-r)
        \end{equation}
        but since $0\leq{r}<m$ we have $0\leq{m}-r<m$. Lastly, if $n<0$
        and $m<0$, then $\minus{n}>0$ and $\minus{m}>0$. But then by
        Thm.~\ref{thm:Weak_Euc_Division_Alg} there exists $q,r$ with
        $\minus{n}=q\cdot(\minus{m})+r$ and $r<\minus{m}$. But then
        $n=q\cdot{m}-r$, and hence $n=(q+1)\cdot{m}+(\minus{r}-m)$. But 
        $r<\minus{m}$, and therefore $0<\minus{r}-m$ and
        $\minus{r}-m<|m|$. Moreover, $q$ and $r$ are unique. For suppose
        $n=q_{0}\cdot{m}+r_{0}$ and $n=q_{1}\cdot{m}+r_{1}$. If
        $q_{0}\ne{q}_{1}$, then either $q_{0}<q_{1}$ or $q_{1}<q_{0}$.
        Suppose $q_{0}<q_{1}$ and let $k=q_{1}-q_{0}$. But then
        \begin{equation}
            n=q_{0}\cdot{m}+r_{0}=(q_{1}-k)\cdot{m}+r_{0}
                =q_{1}\cdot{m}+r_{0}-k\cdot{m}
                =q_{1}\cdot{m}+r_{1}
        \end{equation}
        and therefore $r_{1}=r_{0}-k\cdot{m}$, so
        $r_{0}=r_{1}+k\cdot{m}$, a contradiction since $r_{0}<m$. Hence,
        $q_{0}=q_{1}$. But then taking the difference we obtain
        $r_{0}=r_{1}$. Thus, they are unique.
    \end{bproof}
    \begin{fdefinition}{Divisor of an Integer}{Divisor_of_Integer}
        A divisor of an integer $n\in\mathbb{Z}$ is an integer
        $m\in\mathbb{Z}$ such that there exists and integer
        $k\in\mathbb{Z}$ where $k\cdot{m}=n$. We denote this by $m|n$.
    \end{fdefinition}
    \begin{theorem}
        \label{thm:One_is_Divisor}%
        If $n\in\mathbb{Z}$, then $1|n$.
    \end{theorem}
    \begin{proof}
        For $n=1\cdot{n}$, and hence $1$ is a divisor of $n$
        (Def.~\ref{def:Divisor_of_Integer}).
    \end{proof}
    \begin{theorem}
        \label{thm:Greater_is_not_divisor_of_lesser}%
        If $n,m\in\mathbb{N}^{+}$ and $m<n$, then $n$ does not divide
        $m$.
    \end{theorem}
    \begin{proof}
        For suppose not. If $n|m$, then there exists $k\in\mathbb{Z}$
        such that $m=k\cdot{n}$ (Def.~\ref{def:Divisor_of_Integer}). But
        $n,m\in\mathbb{N}^{+}$ and therefore $n>0$ and $m>0$. But
        $m=k\cdot{n}$, and hence $k>0$. But then
        $m=k\cdot{n}\leq{1}\cdot{n}=n$, a contradiction since $m<n$.
    \end{proof}
    Well ordering of $\mathbb{N}$. Well ordering of countable set.
    \begin{fdefinition}{Greatest Common Divisor}{GCD}
        A greatest common divisor of two integers
        $n,m\in\mathbb{Z}\setminus\{0\}$ is an integer
        $k\in\mathbb{N}^{+}$ such that $k|n$, $k|m$, and for all
        $j\in\mathbb{N}^{+}$ where $j|n$ and $j|m$ it is true that
        $j\leq{k}$.
    \end{fdefinition}
    We say \textit{a} greatest divisor since we don't know
    \textit{a priori} if there are many, or if there is one at all.
    \begin{theorem}
        \label{thm:GCD_Existence_Theorem}%
        If $n,m\in\mathbb{Z}\setminus\{0\}$, then there exists a
        greatest common divisor.
    \end{theorem}
    \begin{proof}
        For let $G\subseteq\mathbb{N}^{+}$ be defined by:
        \begin{equation}
            G=\{\,k\in\mathbb{N}^{+}\;|\;
                k\textrm{ divides }n\textrm{ and }
                k\textrm{ divides }m\,\}
        \end{equation}
        Then $G$ is non-empty since $1\in{G}$. That is, $1|n$ and $1|m$
        (Thm.~\ref{thm:One_is_Divisor}). If $k\in{G}$, then
        $k$ divides $|n|$ and $|m|$. But since
        $n,m\in\mathbb{Z}\setminus\{0\}$ it is true that
        $|n|,|m|\in\mathbb{N}^{+}$. But then if $k$ divides $|n|$ and $|m|$,
        then $k<|n|$ and $k<|m|$
        (Thm.~\ref{thm:Greater_is_not_divisor_of_lesser}). Hence, $G$ is
        bounded above. But then there is a greatest element $k\in{G}$.
        But then $k|m$ and $k|n$, and for all $j\in\mathbb{N}^{+}$ such
        that $j|m$ and $j|n$ it is true that $j\leq{k}$. Hence, $k$ is
        a greatest common divisor (Def.~\ref{def:GCD}).
    \end{proof}
    \begin{theorem}
        \label{thm:GCD_Unique}%
        If $n,m\in\mathbb{Z}\setminus\{0\}$, if $k$ is a greatest common
        divisor of $n$ and $m$, and if $j$ is a greatest common
        divisor of $n$ and $m$, then $k=j$.
    \end{theorem}
    \begin{proof}
        For if not, then either $j<k$ or $k<j$. But if $j<k$ then $j$ is
        not the greatest common divisor since there exists an element
        $k\in\mathbb{N}^{+}$ such that $k|m$, $k|n$, and such that
        $j<k$, a contradiction (Def.~\ref{def:GCD}). Similarly
        $k\not<j$, and hence $j=k$.
    \end{proof}
    Now that we know there's a unique greatest common divisor, we
    use the following notation:
    \begin{fnotation}{Greatest Common Divisor}{GCD}
        The greatest common divisor of $n,m\in\mathbb{Z}\setminus\{0\}$
        is denoted $\GCD(n,m)$.
    \end{fnotation}
    The algorithmic way to go about showing there exists a unique
    greatest common divisor is to apply Euclid's algorithm. We perform
    division with remainder repeatedly until we're left with no
    remaining term. Suppose we're given $n,m\in\mathbb{N}^{+}$ and want
    to compute $\GCD(n,m)$. We do:
    \par
    \begin{subequations}
        \begin{minipage}[b]{0.49\textwidth}
            \centering
            \begin{align}
                n&=q_{0}m+r_{0}\\
                m&=q_{1}r_{0}+r_{2}\\
                r_{0}&=q_{2}r_{1}+r_{2}
            \end{align}
        \end{minipage}
        \hfill
        \begin{minipage}[b]{0.49\textwidth}
            \centering
            \begin{align}
                r_{n-3}&=q_{n-1}r_{n-2}+r_{n-1}\\
                r_{n-2}&=q_{n}r_{n-1}+r_{n}\\
                r_{n-1}&=q_{n+1}r_{n}
            \end{align}
        \end{minipage}
    \end{subequations}
    \par\vspace{2.5ex}
    The $\GCD$ is the last non-zero remainder term.
    \begin{example}
        Let's compute the $\GCD$ of $34$ and $51$. We have:
        \twocolumneq{34=1\cdot{34}+17}{34=2\cdot{17}}
        and so $\GCD(34,51)=17$. Perhaps bigger numbers will better
        demonstrate the algorithm. Let's compute $\GCD(57970,10353)$. We
        obtain:
        \par
        \begin{subequations}
            \begin{minipage}[b]{0.49\textwidth}
                \centering
                \begin{align}
                    57970&=5\cdot{10353}+6205\\
                    10353&=1\cdot{6205}+4148\\
                    6205&=1\cdot{4148}+2057
                \end{align}
            \end{minipage}
            \hfill
            \begin{minipage}[b]{0.49\textwidth}
                \centering
                \begin{align}
                    4148&=2\cdot{2057}+34\\
                    2057&=60\cdot{34}+17\\
                    34&=2\cdot{17}
                \end{align}
            \end{minipage}
        \end{subequations}
        \par\vspace{2.5ex}
        after running the gauntlet, the last equation has zero
        remainder: $34=2\cdot{17}+0$ and hence $\GCD(57970,10353)=17$.
    \end{example}
    \begin{fdefinition}{Least Common Multiple}{LCM}
        A least common multiple of integers
        $n,m\in\mathbb{Z}\setminus\{0\}$ is an integer
        $k\in\mathbb{N}^{+}$ where $n|k$, $m|k$, and for all
        $j\in\mathbb{N}^{+}$ with $n|k$ and $m|k$ it is true that
        $k\leq{j}$.
    \end{fdefinition}
    Again, we use the phrasing \textit{a} least common multiple since
    the existence and uniqueness of such a concept has not yet been
    proved. We now perform this task.
    \begin{theorem}
        \label{thm:Integer_Divides_Multiple_of_Self}%
        If $n,m\in\mathbb{Z}$, if $k=n\cdot{m}$, then $n$ divides $k$.
    \end{theorem}
    \begin{proof}
        For by hypothesis there exists an $m\in\mathbb{Z}$ such that
        $n\cdot{m}=k$ and hence $n$ divides $k$
        (Def.~\ref{def:Divisor_of_Integer}).
    \end{proof}
    \begin{theorem}
        \label{thm:LCM_Existence_Theorem}%
        If $n,m\in\mathbb{Z}\setminus\{0\}$, then there is a least
        common multiple of $n$ and $m$.
    \end{theorem}
    \begin{proof}
        For let $G\subseteq\mathbb{N}^{+}$ be defined by:
        \begin{equation}
            G=\{\,k\in\mathbb{N}^{+}\;|\;
                n\textit{ divides }k\textrm{ and }
                m\textit{ divides }k\,\}
        \end{equation}
        Then $G$ is non-empty since $|n|\cdot|m|\in{G}$
        (Thm.~\ref{thm:Integer_Divides_Multiple_of_Self}). But if $G$ is
        a non-empty subset of $\mathbb{N}^{+}$, then there is a least
        element $k\in{G}$. But then $n$ divides $k$, $m$ divides $k$,
        and for all $j$ with $n|j$ and $m|j$ it is true that
        $k\leq{j}$. Hence, $k$ is a least common multiple of $n$ and $m$
        (Def.~\ref{def:LCM}).
    \end{proof}
    Much like the greatest common divisor, we now prove that least
    common multiples are unique and then assign a notation to the
    concept.
    \begin{theorem}
        \label{thm:LCM_Unique}%
        If $n,m\in\mathbb{Z}^{+}$, if $k$ is a least common multiple of
        $n$ and $m$, and if $j$ is a least common multiple of $n$ and
        $m$, then $k=j$.
    \end{theorem}
    \begin{proof}
        For if not, then either $j<k$ or $k<j$, violating minimality.
    \end{proof}
    \begin{fnotation}{Least Common Multiple}{LCM}
        The least common multiple of $n,m\in\mathbb{Z}\setminus\{0\}$ is
        denoted $\LCM(n,m)$.
    \end{fnotation}
    One of the most useful theorems in number theory is from the French
    mathematician \'{E}tienne B\'{e}zout. The theorem, known as
    B\'{e}zout's identity, was proved in the $18^{th}$ century in the
    setting of polynomials, but when applied to integers it allows to us
    make many theorems of great historical importance into short
    corollaries. For example, as we will show, the effort in proving
    Euclid's prime number lemma and the Chinese remainder theorem are
    greatly diminshed by applying the identity. Later, in the context of
    \textit{rings}, a generalization will be given.
    \begin{ftheorem}{B\'{e}zout's Identity}{Bezout_Identity}
        If $a,b\in\mathbb{N}\setminus\{0\}$, and if $d=\GCD(a,b)$, then
        there exist $n,m\in\mathbb{Z}$ such that:
        \begin{equation*}
            a\cdot{n}+b\cdot{m}=d
        \end{equation*}
    \end{ftheorem}
    \begin{bproof}
        For let $G\subseteq\mathbb{Z}\setminus\{0\}$ be defined by:
        \begin{equation}
            G=\{\,k\in\mathbb{N}\setminus\{0\}\;|\;
                \textrm{ There exists }n,m\in\mathbb{Z}
                \textrm{ such that }k=an+bm\big)\,\}
        \end{equation}
        then $G$ is non-empty since $a^{2}+b^{2}\in{G}$. But then there
        is a least element $d\in{G}$. By Euclid's division algorithm,
        there exists $q\in\mathbb{Z}$ and $r\in\mathbb{N}$ such that
        $r<d$ and $a=q\cdot{d}+r$
        (Thm.~\ref{thm:Euclid_Division_Algorithm}). But $d\in{G}$, and
        hence there are $n,m\in\mathbb{Z}$ such that
        $d=a\cdot{n}+b\cdot{m}$. But then:
        \begin{equation}
            r=a-qd=a-q(na+mb)=a-aqn-bqm=a(1-qn)+b\cdot(\minus{q}m)
        \end{equation}
        But if $r>0$, then $r\in{G}$ and $r<d$, a contradiction since
        $d$ is the least element of $G$. Hence, $r=0$ and $d$ divides
        $a$ (Def.~\ref{def:Divisor_of_Integer}). Similarly, $d$ divides
        $b$. If $c\in\mathbb{N}^{+}$ is such that $c|a$ and $c|b$, then
        there exists $s,t\in\mathbb{Z}$ such that $a=c\cdot{s}$ and
        $b=c\cdot{t}$. But then:
        \begin{equation}
            d=a\cdot{n}+b\cdot{m}=(c\cdot{s})n+(c\cdot{t})m=c(sn+tm)
        \end{equation}
        and therefore $d$ divides $c$. But it $d$ divides $c$, then
        $d$ is not greater than $c$
        (Thm.~\ref{thm:Greater_is_not_divisor_of_lesser}), and therefore
        $c\leq{d}$. Thus, $d$ is the greatest common divisor
        (Def.~\ref{def:GCD}).
    \end{bproof}
    \begin{theorem}
        \label{thm:Divisor_of_AB_Divides_GCD}%
        If $a,b,n\in\mathbb{Z}\setminus\{0\}$, and if $n|a$ and $n|b$,
        then $n|\GCD(a,b)$.
    \end{theorem}
    \begin{proof}
        By B\'{e}zout's identity there exists $s,t\in\mathbb{Z}$ such
        that $as+bt=\GCD(a,b)$ (Thm.~\ref{thm:Bezout_Identity}). But
        $n|a$ and $n|b$, and thus there exists $u,v\in\mathbb{Z}$ such
        that $nu=a$ and $nv=b$ (Def.~\ref{def:Divisor_of_Integer}). But
        then $n(us+vt)=\GCD(a,b)$. Therefore, $n$ divides $\GCD(a,b)$
        (Def.~\ref{def:Divisor_of_Integer}).
    \end{proof}
    \begin{theorem}
        \label{thm:n_Div_AB_then_A_Div_AS_BT}%
        If $a,b,s,t,n\in\mathbb{Z}$, if $n|a$, and if $n|b$, then $a$
        divides $as+bt$.
    \end{theorem}
    \begin{proof}
        For if $n$ divides $a$ and $b$, then there exists
        $u,v\in\mathbb{Z}$ such that $nu=a$ and $nv=b$
        (Def.~\ref{def:Divisor_of_Integer}). But then
        $as+bt=nus+nvt=n(us+bt)$, and thus $n$ divides $as+bt$
        (Def.~\ref{def:Divisor_of_Integer}).
    \end{proof}
    As claimed, many great theorems from antiquity become corollaries of
    B\'{e}zout's identity. We now demonstrate this.
    \begin{ftheorem}{Euclid's Prime Number Lemma}
                    {Euclid_Prime_Number_Lemma}
        If $a,b,p\in\mathbb{Z}\setminus\{0\}$, if $\GCD(a,p)=1$, and if
        $p$ divides $a\cdot{b}$, then $p$ divides $b$.
    \end{ftheorem}
    \begin{bproof}
        Since $\GCD(a,p)=1$, by B\'{e}zout's identity there exist
        integers $n,m\in\mathbb{Z}$ such that $an+pm=1$
        (Thm.~\ref{thm:Bezout_Identity}). But then $ban+bpm=b$. But $p$
        divides $ab$ and hence there is a $k\in\mathbb{Z}$ such that
        $kp=ab$ (Def.~\ref{def:Divisor_of_Integer}) and therefore
        $kpn+bpm=b$. But then $p(kn+bm)=b$, and hence $p$ divides $b$
        (Def.~\ref{def:Divisor_of_Integer}).
    \end{bproof}
    Euclid's phrasing of this theorem, as presented in his
    \textit{elements}, goes as follows:
    \begin{theorem}
        \label{thm:Prime_Div_AB_then_PdivA_or_PdivB}%
        If $a,b\in\mathbb{N}^{+}$, if $p\in\mathbb{N}^{+}$ is prime, and
        if $p$ divides $a\cdot{b}$, then either $p$ divides $a$ or $p$
        divides $b$.
    \end{theorem}
    \begin{proof}
        For if $p$ does not divide $a$, then $\GCD(a,p)=1$, and thus
        $p$ divides $b$ (Thm.~\ref{thm:Euclid_Prime_Number_Lemma}).
        Similary, if $p$ does not divide $b$ then it divides $a$.
    \end{proof}
    \begin{theorem}
        If $m,n\in\mathbb{N}^{+}$, if $l$ is the least common multiple of
        $n$ and $m$, and if $d$ is the greatest common divisor of $n$ and
        $m$, then $l\cdot{d}=n\cdot{m}$.
    \end{theorem}
    \begin{ftheorem}{Fundamental Theorem of Arithmetic}
                    {Fundamental_Theorem_of_Arithmetic}
        If $n\in\mathbb{N}^{+}$, then there exists a unique
        $n\in\mathbb{N}^{+}$, a unique strictly increasing sequence
        $P:\mathbb{Z}_{n}\rightarrow\mathbb{N}^{+}$ and a unique
        sequence $N:\mathbb{Z}_{n}\rightarrow\mathbb{N}^{+}$ such that
        for all $k\in\mathbb{Z}_{n}$ it is true that $P_{k}$ is prime
        and:
        \begin{equation*}
            n=\prod_{k\in\mathbb{Z}_{n}}P_{k}^{N_{k}}
        \end{equation*}
    \end{ftheorem}
    That is, every integer has a unique prime factorization. If $n$ is a
    prime, the factorization is simply
    $P:\mathbb{Z}_{1}\rightarrow\mathbb{N}^{+}$ defined by $P_{0}=n$ and
    $N:\mathbb{Z}_{1}\rightarrow\mathbb{N}^{+}$ with $N_{0}=1$.
    \begin{theorem}
        \label{thm:Composite_N_Exists_AB_N_Div_AB_and_N_NDiv_A_or_B}%
        If $n\in\mathbb{N}^{+}$ is not prime, then there exists
        integers $a,b\in\mathbb{N}^{+}$ such that $n$ divides
        $ab$, but $n$ does not divide $a$ and $n$ does not divide $b$.
    \end{theorem}
    \begin{proof}
        For of $n$ is composite, then there is a prime $p$ that divides
        $n$. Let $Q=n/p$. By the fundamental theorem of arithmetic,
        there exists a subset
        $S\subseteq\mathbb{N}^{+}\times\mathbb{N}^{+}$ such that for all
        $(q,n)\in{S}$ it is true that $q$ is prime and:
        \begin{equation}
            Q=\prod_{(p,n)\in{S}}q^{n}
        \end{equation}
        Since there are infinitely many primes, there is a prime $P$ not
        contained in the projection of $S$ into the first variable. Let
        $a=p$ and $b=Q\cdot{P}$. Then $n$ does not divide $a$ or $b$,
        but it does divide $a\cdot{b}$.
    \end{proof}
    \begin{example}
        The smallest example we have to work with is 4, and so we direct
        our attention there. 4 divides 12, and $12=4\cdot{3}$. Following
        the proof of
        Thm.~\ref{thm:Composite_N_Exists_AB_N_Div_AB_and_N_NDiv_A_or_B},
        we remove a prime from 4 and are left with 2. We then pick a
        prime that is not in the prime factorization of 4 and multiply
        by this. That is, we have $p=2$, $Q=2$, and $P=3$. We set
        $a=p=2$ and $b=Q\cdot{P}=2\cdot{3}=6$. Then 4 does not divide 2
        since $2<4$ and moreover 4 does not divide 6, but it does divide
        $2\cdot{6}=12$.
    \end{example}
    \begin{ftheorem}{Chinese Remainder Theorem}
                    {Chinese_Remainder_Theorem}
        If $n\in\mathbb{N}^{+}$ is an integer, if
        $N:\mathbb{Z}_{n}\rightarrow\mathbb{N}$ is such that for all
        $i,j\in\mathbb{Z}_{n}$ with $i\ne{j}$ it is true that
        $\GCD(N_{i},N_{j})=1$, and if
        $A:\mathbb{Z}_{n}\rightarrow\mathbb{N}$ is a sequence of
        integers such that $A_{i}<N_{i}$ for all $i\in\mathbb{Z}_{n}$,
        then there is a unique $x\in\mathbb{N}$ such that for all
        $i\in\mathbb{Z}_{n}$ it is true that $x\equiv{A}_{i}\mod{N}_{i}$
        and $x<\prod_{k\in\mathbb{Z}_{n}}N_{k}$
    \end{ftheorem}
    \begin{bproof}
        By induction. The base case, since $N_{1}$ and $N_{2}$ are
        relatively prime, by B\'{e}zout's identity there exist integers
        $m_{1},m_{2}\in\mathbb{Z}$ such that $m_{1}N_{1}+m_{2}N_{2}=1$.
        Let $x=m_{2}N_{2}A_{1}+m_{1}N_{1}A_{2}$. In the induction case,
        there is an $x$ such that $x\equiv{A}_{i}\mod{N}_{i}$ for
        $i\in\mathbb{Z}_{n-1}$ and $x\equiv{A}_{n}A_{n+1}\mod{N}_{n}N_{n+1}$
        since $N_{n}$ and $N_{n+1}$ are relatively prime.
    \end{bproof}
    We can get some use out of this by studying the Euler totient
    function.
    \begin{theorem}
        \label{thm:Rel_Prime_to_Prod_of_Rel_Primes}%
        If $n,m\in\mathbb{N}$, if $\GCD(n,m)=1$, if $k\in\mathbb{N}$,
        and if $k<mn$, then $\GCD(k,n\cdot{m})=1$ if and only if
        $\GCD(k,n)=1$ and $\GCD(k.m)=1$
    \end{theorem}
    \begin{proof}
        Somehow the Chinese remainder theorem does this. Moving on for
        now.
    \end{proof}
    $S_{1}\subseteq{S}_{2}$, we see that
    $\varphi(a)$ divides $\varphi(b)$.  \begin{fdefinition}{Euler Totient Function}{Euler_Totient_Func}
        The Euler totient function is the function
        $\varphi:\mathbb{N}^{+}\rightarrow\mathbb{N}^{+}$ defined by:
        \begin{equation*}
            \varphi(n)=\cardinality{\{\,k\in\mathbb{N}\;|\;
                k\leq{n}\textrm{ and }\GCD(k,n)=1\,\}}
        \end{equation*}
    \end{fdefinition}
    Since there are so many different uses of the symbol $\varphi$, when
    a theorem pertains to the Euler totient function we will explicitly
    say so.
    \begin{theorem}
        \label{thm:Euler_Totient_of_Prime}%
        If $p\in\mathbb{N}$ is a prime number and if $\varphi$ is the
        Euler totient function, then $\varphi(p)=p-1$.
    \end{theorem}
    \begin{proof}
        For all $k\in\mathbb{N}^{+}$ such that $k<p$ it is true that
        $\GCD(k,p)=1$ since $p$ is prime. Hence, $\varphi(p)$ is equal
        to $\cardinality{\mathbb{Z}_{p}\setminus\{0\}}$ which  is $p-1$.
    \end{proof}
    \begin{theorem}
        \label{thm:Euler_Totient_Multiplicative}%
        If $a,b\in\mathbb{N}$, if $\varphi$ is the Euler totient
        function, if $\GCD(a,b)=1$, then:
        \begin{equation}
            \varphi(a\cdot{b})=\varphi(a)\cdot\varphi(b)
        \end{equation}
    \end{theorem}
    \begin{proof}
        For since $\GCD(n,m)=1$, for all $k<mn$ it is true that
        $\GCD(k,mn)=1$ if and only if $\GCD(k,m)=1$ and $\GCD(k,n)=1$
        (Thm.~\ref{thm:Rel_Prime_to_Prod_of_Rel_Primes}). Hence, there
        are $\varphi(a)\cdot\varphi(b)$ such elements.
    \end{proof}
    \begin{theorem}
        \label{thm:Euler_Totient_Powers_of_Primes}%
        If $p\in\mathbb{N}$ is prime, if $\varphi$ is the Euler totient
        function, and if $n\in\mathbb{N}^{+}$, then
        $\varphi(p^{n})=p^{n-1}(p-1)$.
    \end{theorem}
    \begin{proof}
        For if $p$ is prime, and if $m\in\mathbb{Z}_{p}\setminus\{0\}$,
        then since the only factors of $p^{n}$ are powers of $p$,
        $\GCD(p^{n},m)=p^{k}$ for some $k\in\mathbb{Z}_{n}$. There are
        $p\cdot{p}^{n-1}$ elements that are multiples of $p$, and hence
        $p^{n}-p^{n-1}$ elements that are coprime. Hence,
        $\varphi(p^{n})=p^{n}-p^{n-1}=p^{n-1}(p-1)$.
    \end{proof}
    We can combine the fundamental theorem of arithmetic together with
    these theorems to quickly compute the Euler totient function of a
    given value.
    \begin{example}
        The prime factorization of 12 is $2^{2}\cdot{3}$. Thus, we can
        compute $\varphi$ as follows:
        \begin{equation}
            \varphi(12)=\varphi(2^{2}\cdot{3})
                =\varphi(2^{2})\cdot\varphi(3)
                =2^{2-1}(2-1)\cdot(3-1)=2\cdot{1}\cdot{2}=4
        \end{equation}
        we can also just count out the relatively prime elements of
        $\mathbb{N}$ that are less than 12, and we obtain 1, 5, 7, and
        11. Powers of primes are particularly easy:
        \begin{equation}
            \varphi(16)=\varphi(2^{4})=2^{4-1}(2-1)=2^{3}=8
        \end{equation}
        the relatively prime elements are 1, 3, 5, 7, 9, 11, 13, and 15.
        That is, all of the odd numbers less than 16. Lastly, let's try
        $\varphi(75)$. We obtain:
        \begin{equation}
            \varphi(75)=\varphi(5^{2}*3)=5^{2-1}(5-1)\cdot(3-1)
                =5\cdot{4}\cdot{2}=40
        \end{equation}
    \end{example}
    \begin{theorem}
        \label{thm:SQRT_Primes_are_Irrational}%
        If $p\in\mathbb{N}^{+}$ is a prime number, then $\sqrt{p}$ is
        irrational.
    \end{theorem}
    \begin{proof}
        For suppose not. If $\sqrt{p}$ is rational, then there exists
        $a,b\in\mathbb{Z}$ such that $b\ne{0}$, $\GCD(a,b)=1$, and
        $\sqrt{p}=a/b$. But then $a^{2}=pb^{2}$ and thus $p$ divides
        $a\cdot{a}$ (Def.~\ref{def:Divisor_of_Integer}). Since $p$ is
        prime, by Euclid's prime number lemma $p$ divides $a$
        (Thm.~\ref{thm:Euclid_Prime_Number_Lemma}). But then there is a
        $k\in\mathbb{Z}$ such that $a=p\cdot{k}$. But then
        $a^{2}=p^{2}k^{2}=pb^{2}$, and hence $pk^{2}=b^{2}$. But then
        $p$ divides $b^{2}$ (Def.~\ref{def:Divisor_of_Integer}) and thus
        since $p$ is prime, $p$ divides $b$
        (Thm.~\ref{thm:Euclid_Prime_Number_Lemma}). But then $p$ divides
        $a$ and $b$, a contradiction since $\GCD(a,b)=1$ and $1<p$.
        Hence, $\sqrt{p}$ is irrational.
    \end{proof}
    Only finitely many $n$ have $\varphi(n)=N$ for a fixed $N$, where
    $\varphi$ is the Euler totient function.
    \begin{theorem}
        \label{thm:A_DIV_B_then_EulerTotA_Div_EulerTotB}%
        If $a,b\in\mathbb{N}^{+}$, if $a|b$, and if $\varphi$ is the
        Euler totient function, then $\varphi(a)$ divides $\varphi(b)$.
    \end{theorem}
    \begin{proof}
        For by the fundamental theorem of arithmetic
        (Thm.~\ref{thm:Fundamental_Theorem_of_Arithmetic}) there exist
        integers $m,n\in\mathbb{N}^{+}$, sequences
        $P,M:\mathbb{Z}_{m}\rightarrow\mathbb{N}^{+}$ and
        $Q,N:\mathbb{Z}_{n}\rightarrow\mathbb{N}^{+}$ such that for all
        $j\in\mathbb{Z}_{m}$ and $k\in\mathbb{Z}_{n}$, $P_{j}$ and
        $Q_{k}$ are prime, and:
        \twocolumneq{a=\prod_{j\in\mathbb{Z}_{m}}P_{j}^{M_{j}}}
                    {b=\prod_{k\in\mathbb{Z}_{n}}Q_{k}^{N_{k}}}
        But since $a$ divides $b$ there exists a $k\in\mathbb{Z}$ such
        that $b=a\cdot{k}$ (Def.~\ref{def:Divisor_of_Integer}). But
        since $P$ is a strictly increasing sequence and hence the
        $P_{j}$ are distinct, and since all of the $P_{j}$ are prime,
        all of the $P_{j}^{M_{j}}$ are relatively prime. Hence:
        \begin{align*}
            \varphi(a)&=\varphi\Big(
                \prod_{j\in\mathbb{Z}_{m}}P_{j}^{M_{j}}
            \Big)\\
            &=\prod_{j\in\mathbb{Z}_{m}}\varphi(P_{j}^{M_{j}})
                \tag{Thm.~\ref{thm:Euler_Totient_Multiplicative}}\\
            &=\prod_{j\in\mathbb{Z}_{m}}P_{j}^{N_{j}-1}(P_{j}-1)
                \tag{Thm.~\ref{thm:Euler_Totient_Powers_of_Primes}}
        \end{align*}
        and
        \begin{align*}
            \varphi(b)&=\varphi\Big(
                \prod_{k\in\mathbb{Z}_{n}}Q_{k}^{M_{k}}
            \Big)\\
            &=\prod_{k\in\mathbb{Z}_{n}}\varphi(Q_{k}^{M_{k}})
                \tag{Thm.~\ref{thm:Euler_Totient_Multiplicative}}\\
            &=\prod_{k\in\mathbb{Z}_{n}}Q_{k}^{N_{k}-1}(Q_{k}-1)
                \tag{Thm.~\ref{thm:Euler_Totient_Powers_of_Primes}}
        \end{align*}
        since $a$ divides $b$, and from the uniqueness of the sequences
        $P$ and $Q$, we see that $\varphi(a)$ divides $\varphi(b)$.
    \end{proof}