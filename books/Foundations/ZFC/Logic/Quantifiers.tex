\section{Quantifiers}
    There are two more symbols called
    \textit{\glspl{quantifier}}\index{Quantifier}.
    \begin{equation*}
        \forall_{x}\quad\textrm{For all }x
        \quad\quad\quad\quad
        \exists_{x}\quad\textrm{There exists }x
    \end{equation*}
    Quantifiers, together with connectives, the word \textit{set}, and the
    $\in$ symbol are combined to define new terms and new symbols. The rest
    of mathematics rests on trusting ones intuition behind these notions.
    \begin{example}
        Let $\mathbb{R}$ denote the set of real numbers. The symbols
        $\forall_{R\in\mathbb{R}}(n^{2}\geq{0})$ can then be read in English
        as \textit{For all real numbers x, the square of x is non-negative},
        which is indeed a true statement. We can combine quantifiers to
        create more complicated statements, such as:
        \begin{equation}
            \forall_{x\in\mathbb{R}}(x\ne{0})\exists_{y\in\mathbb{R}}(xy=1)
        \end{equation}
        This reads that for all non-zero real numbers $x$, there exists a
        real numbers $y$ such that the product $xy$ is equal to 1. This is
        also a true statement.
    \end{example}
    \begin{example}
        The order of quantifiers is very important and often can not be
        interchanged. Considering the previous example, if we switch the
        order of the quantifiers we get:
        \begin{equation}
            \exists_{y\in\mathbb{R}}(xy=1)\forall_{x\in\mathbb{R}}(x\ne{0})
        \end{equation}
        This states that there exists a real number $y$ such that, for every
        non-zero real number $x$, it is true that $xy=1$. But this is
        certainly not true because if $x=1$ and $z=\minus{1}$, we obtain
        $(1)y=1$ and $(\minus{1})y=1$, and from this we conclude that
        $\minus{1}=1$, which is false. Hence, the order of the quantifiers
        is important.
    \end{example}
    \begin{example}
        Quantifiers can be combined with connectives to make longer and more
        complicated statements. For example, suppose $P$ is the proposition
        \textit{true if n is an even integer, false otherwise}. Furthermore,
        let $Q$ be the proposition \textit{true if n is a square integer},
        \textit{false otherwise}. Lastly, let $r$ be the proposition
        \textit{true if n is divisible by 4, false otherwise}.
        Consider then the following statement:
        \begin{equation}
            \forall_{n\in\mathbb{Z}}(p(n)\land{q(n)}\Rightarrow{r}(n))
        \end{equation}
        This reads in English as \textit{for all integers n, if n is an}
        \textit{integer, and if n is a square, then n is divisible by 4}.
    \end{example}
    \subsection{Negating Quantifiers}
        The negation of the statement \textit{for all x, P(x) is true} implies
        this is false. Thus there must exist one $x$ such that $P(x)$ is false,
        and from this we see that negating the $\forall$ quantifier produces the
        $\exists$ quantifier.
        \begin{example}
            Let $P$ be the proposition \textit{true if $x^{2}=2$} and consider
            the following statement:
            \begin{equation}
                \exists_{x\in\mathbb{Q}}\big(P(x)\big)
            \end{equation}
            This reads in plain English as the statement \textit{there exists a}
            \textit{rational number x whose square is equal to 2}. This has been
            known to be false since the ancient Greeks, and thus it's negation
            is true. We can write the negation as follows:
            \begin{equation}
                \neg\Big(\exists_{x\in\mathbb{Q}}\big(p(x)\big)\Big)
                \Longleftrightarrow\forall_{x\in\mathbb{Q}}\big(\neg{P}(x)\big)
            \end{equation}
            This now says that for all rational numbers $x$, the square of $x$
            is not equal to 2.
        \end{example}