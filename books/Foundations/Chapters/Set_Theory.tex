\documentclass[crop=false,class=book,oneside]{standalone}
%----------------------------Preamble-------------------------------%
%---------------------------Packages----------------------------%
\usepackage{geometry}
\geometry{b5paper, margin=1.0in}
\usepackage[T1]{fontenc}
\usepackage{graphicx, float}            % Graphics/Images.
\usepackage{natbib}                     % For bibliographies.
\bibliographystyle{agsm}                % Bibliography style.
\usepackage[french, english]{babel}     % Language typesetting.
\usepackage[dvipsnames]{xcolor}         % Color names.
\usepackage{listings, lstlinebgrd}      % Verbatim-Like Tools.
\usepackage{mathtools, esint, mathrsfs} % amsmath and integrals.
\usepackage{amsthm, amsfonts}           % Fonts and theorems.
\usepackage{tabularx}
\usepackage{tcolorbox}                  % Frames around theorems.
\usepackage{upgreek}                    % Non-Italic Greek.
\usepackage{paracol}                    % Two-column styling.
\usepackage{wrapfig}                    % Wrap text around figure.
\usepackage{fmtcount, etoolbox}         % For the \book{} command.
\usepackage[newparttoc]{titlesec}       % Formatting chapter, etc.
\usepackage{titletoc}                   % Allows \book in toc.
\usepackage[nottoc]{tocbibind}          % Bibliography in toc.
\usepackage[titles]{tocloft}            % ToC formatting.
\usepackage{multicol, enumitem}         % Multi-column/enumerate.
\usepackage{import}                     % Import external files.
\usepackage{pgfplots, tikz}             % Drawing/graphing tools.
\usetikzlibrary{
    calc,                   % Calculating right angles and more.
    angles,                 % Drawing angles within triangles.
    arrows.meta,            % Latex and Stealth arrows.
    quotes,                 % Adding labels to angles.
    positioning,            % Relative positioning of nodes.
    decorations.markings,   % Adding arrows in the middle of a line.
    patterns,
    arrows,
    shapes,
    shapes.geometric,
    cd,
    hobby,
    babel
}                                       % Libraries for tikz.
\pgfplotsset{compat=1.9}                % Version of pgfplots.
\usepackage[font=scriptsize,
            labelformat=simple,
            labelsep=colon]{subcaption} % Subfigure captions.
\usepackage[font={scriptsize},
            hypcap=true,
            labelsep=colon]{caption}    % Figure captions.
\usepackage{hyperref}                   % Allows for hyperlinks.
\hypersetup{
    colorlinks=true,
    linkcolor=blue,
    filecolor=magenta,
    urlcolor=Cerulean,
    citecolor=SkyBlue
}                           % Colors for hyperref.
\usepackage[toc,acronym,nogroupskip]{glossaries} % Glossaries and acronyms.
\usepackage[subpreambles=false]{standalone}      % Complileable sub files.

% Various font stuff from kiwi.
% Use this for Times text and Computer Modern math
%\usepackage{times}

% Quite nice
%\usepackage[charter, greekfamily=, greekuppercase=italicized]{mathdesign}
%\usepackage[utopia, greekuppercase=italicized]{mathdesign}    % Math is narrower

% Use this for Times text and math
%\usepackage{newtxtext}
%\usepackage[libertine,cmintegrals]{newtxmath}
%\usepackage{fix-cm}

%\usepackage{txfontsb}
% or
%\usepackage{mathptmx}

%\usepackage[scaled=0.92]{helvet}
%\renewcommand{\rmdefault}{ptm}

%\usepackage{mathpazo}    % add possibly `sc` and `osf` options
%\usepackage{eulervm}

%\usepackage{fourier}
%\renewcommand{\rmdefault}{ptm}
%\usepackage{mathptm}

%\usepackage{fontspec}
%\setmainfont{lmodern}

%\usepackage[varg]{txfonts}
%\usepackage{fouriernc}
%\usepackage{mathpazo}

%\usepackage{bookman}
%\usepackage[scaled]{uarial}
%\usepackage[scaled]{helvet}
%\renewcommand*\familydefault{\sfdefault}
%\usepackage[math]{anttor}

%\newcommand\fgeorgia{\fontfamily{jvn}\selectfont}
%\newcommand\ftimes{\fontfamily{ptm}\selectfont}
%\newcommand\fhelvetica{\fontfamily{phv}\selectfont}
%\newcommand\fcourier{\fontfamily{pcr}\selectfont}
%\newcommand\fbookman{\fontfamily{pbk}\selectfont}
%\newcommand\fnewcentury{\fontfamily{pnc}\selectfont}
%\newcommand\fpalatino{\fontfamily{ppl}\selectfont}
%\newcommand\favantgarde{\fontfamily{pag}\selectfont}
%\newcommand\fnormal{\normalfont}
%\newcommand\fsize[1]{\ifnum#1>0\fontsize{#1}{#1}\selectfont\else\normalsize\fi}
%------------------------Theorem Styles-------------------------%
% Define theorem style for default spacing and normal font.
\newtheoremstyle{normal}
    {\topsep}               % Amount of space above the theorem.
    {\topsep}               % Amount of space below the theorem.
    {}                      % Font used for body of theorem.
    {}                      % Measure of space to indent.
    {\bfseries}             % Font of the header of the theorem.
    {}                      % Punctuation between head and body.
    {.5em}                  % Space after theorem head.
    {}

% Define theorem style for default spacing with italicized font.
\newtheoremstyle{normalit}{\topsep}{\topsep}
                {\itshape}{}{\bfseries}{}{.5em}{}

% Italic header environment.
\newtheoremstyle{thmit}{\topsep}{\topsep}{}{}{\itshape}{}{0.5em}{}

% Define italicized environments.
\theoremstyle{normalit}
\newtheorem{theorem}{Theorem}[section]
\newtheorem{lemma}{Lemma}[section]
\newtheorem{corollary}{Corollary}[section]
\newtheorem{proposition}{Proposition}[section]
\newtheorem*{theorem*}{Theorem}

% Define environments with italic headers.
\theoremstyle{thmit}
\newtheorem*{solution}{Solution}
\newtheorem*{fsolution}{Solution}

% Define default environments.
\theoremstyle{normal}
\newtheorem{example}{Example}[section]
\newtheorem{definition}{Definition}[section]
\newtheorem{problem}{Problem}[section]
\newtheorem{question}{Question}[section]
\newtheorem{remark}{Remark}[section]
\newtheorem{properties}{Properties}[section]
\newtheorem{notation}{Notation}[section]
\newtheorem{axiom}{Axiom}[section]
\newtheorem*{properties*}{Properties}
\newtheorem*{remark*}{Remark}
\newtheorem*{definition*}{Definition}
\theoremstyle{plain}

% Define framed environment.
\tcbuselibrary{most}
\newtcbtheorem[use counter*=theorem]{ftheorem}{Theorem}%
    {colback=green!5,colframe=green!35!black,
     fonttitle=\bfseries\upshape}{th}

\newtcbtheorem[use counter*=example]{fdefinition}{Definition}%
    {fonttitle=\bfseries\upshape,
     colback=blue!5!white,colframe=blue!75!black}{def}

\newtcbtheorem[use counter*=example]{fexample}{Example}%
    {fonttitle=\bfseries\upshape,
     colback=red!5!white,colframe=red!75!black}{ex}

\newtcbtheorem[use counter*=notation]{fnotation}{Notation}%
    {fonttitle=\bfseries\upshape,
     colback=SeaGreen!5!white,colframe=SeaGreen!75!black}{ex}

\newtcbtheorem[use counter*=corollary]{fcorollary}{Corollary}%
    {fonttitle=\bfseries\upshape,
     colback=Orchid!5!white,colframe=Orchid!75!black}{ex}

\newenvironment{bproof}{\textit{Proof.}}{\hfill$\square$}
\tcolorboxenvironment{bproof}{blanker,breakable,left=5mm,
                             before skip=10pt,after skip=10pt,
                             borderline west={1mm}{0pt}{red}}
\tcolorboxenvironment{fsolution}
    {enhanced jigsaw,colframe=cyan,interior hidden,breakable}

%--------------------Declared Math Operators--------------------%
\DeclareMathOperator{\Refl}{Refl}           % Reflection operator.
\DeclareMathOperator{\Span}{Span}           % Span of a set of vectors.
\DeclareMathOperator{\Card}{Card}           % Cardinality of set.
\DeclareMathOperator{\Ord}{Ord}             % Ordinal of ordered set.
\DeclareMathOperator{\Tr}{Tr}               % Trace of matrix.
\DeclareMathOperator{\adjoint}{adj}         % Adjoint of matrix.
\DeclareMathOperator{\rk}{rk}               % Rank of operator.
\DeclareMathOperator{\nul}{nul}             % Null space of operator.
\DeclareMathOperator{\sgn}{sgn}             % Sign of a number.
\DeclareMathOperator{\multideg}{mutlideg}   % Multi-Degree (Graphs).
\DeclareMathOperator{\GCD}{GCD}             % Greatest common denominator.
\DeclareMathOperator{\LM}{LM}               % Leading monomial
\DeclareMathOperator{\LC}{LC}               % Leading coefficient.
\DeclareMathOperator{\LT}{LT}               % Leading term.
\DeclareMathOperator{\LCM}{LCM}             % Least common multiple.
\DeclareMathOperator{\Mon}{Mon}             % Monomial.
\DeclareMathOperator{\Spec}{Spec}           % Spectrum.
\DeclareMathOperator{\proj}{proj}           % Projection.
\DeclareMathOperator{\comp}{comp}           % Component.
\DeclareMathOperator{\sinc}{sinc}           % Sinc function.
\DeclareMathOperator{\Ima}{Im}              % Image of operator.
\DeclareMathOperator{\Prin}{Prin}           % Principal value.
\DeclareMathOperator{\Mod}{mod}             % Modulus.
%------------------------New Commands---------------------------%
\DeclarePairedDelimiter\norm{\lVert}{\rVert}
\DeclarePairedDelimiter\ceil{\lceil}{\rceil}
\DeclarePairedDelimiter\floor{\lfloor}{\rfloor}
\newcommand*\diff{\mathop{}\!\mathrm{d}}
\newcommand*\Diff[1]{\mathop{}\!\mathrm{d^#1}}
\renewcommand{\mod}{\ \Mod}
\renewcommand*{\glstextformat}[1]{\textcolor{RoyalBlue}{#1}}
\renewcommand{\glsnamefont}[1]{\textbf{#1}}
\renewcommand\labelitemii{$\circ$}
\renewcommand\thesubfigure{\arabic{chapter}.\arabic{figure}}
\renewcommand\thesubfigure{%
    \arabic{chapter}.\arabic{figure}.\arabic{subfigure}}
\addto\captionsenglish{\renewcommand{\figurename}{Fig.}}
%------------------------Book Command---------------------------%
\makeatletter
\renewcommand\@pnumwidth{1cm}
\newcounter{book}
\renewcommand\thebook{\@Roman\c@book}
\newcommand\book{%
    \if@openright
        \cleardoublepage
    \else
        \clearpage
    \fi
    \thispagestyle{plain}%
    \if@twocolumn
        \onecolumn
        \@tempswatrue
    \else
        \@tempswafalse
    \fi
    \null\vfil
    \secdef\@book\@sbook
}
\def\@book[#1]#2{%
    \ifnum \c@secnumdepth >-3\relax
        \refstepcounter{book}%
        \addcontentsline{toc}{book}{
            \bookname\ \thebook:\hspace{1em}#1
        }
    \else
        \addcontentsline{toc}{book}{#1}%
    \fi
    \markboth{}{}%
    {\centering
     \interlinepenalty \@M
     \normalfont
     \ifnum \c@secnumdepth >-2\relax
       \huge\bfseries \bookname\nobreakspace\thebook
       \par
       \vskip 20\p@
     \fi
     \Huge \bfseries #2\par}%
    \@endbook}
\def\@sbook#1{%
    {\centering
     \interlinepenalty \@M
     \normalfont
     \Huge \bfseries #1\par}%
    \@endbook}
\def\@endbook{
    \vfil\newpage
        \if@twoside
            \if@openright
                \null
                \thispagestyle{empty}%
                \newpage
            \fi
        \fi
        \if@tempswa
            \twocolumn
        \fi
}
\newcommand*\l@book[2]{%
    \ifnum \c@tocdepth >-2\relax
        \addpenalty{-\@highpenalty}%
        \addvspace{2.25em \@plus\p@}%
        \setlength\@tempdima{3em}%
        \begingroup
            \parindent \z@ \rightskip \@pnumwidth
            \parfillskip -\@pnumwidth
            {
                \leavevmode
                \Large \bfseries #1\hfil \hb@xt@\@pnumwidth{
                    \hss #2
                }
            }
            \par
            \nobreak
            \global\@nobreaktrue
            \everypar{\global\@nobreakfalse\everypar{}}%
        \endgroup
    \fi}
\newcommand\bookname{Book}
\renewcommand{\thebook}{\texorpdfstring{\Numberstring{book}}{book}}
\providecommand*{\toclevel@book}{-2}
\makeatother
\titlecontents{chapter}[0pt]
    {\bfseries}
    {\chaptername\ \thecontentslabel:\quad}
    {}
    {\hfill\contentspage}
\titleformat{\part}[display]
    {\Large\bfseries}
    {\partname\nobreakspace\thepart}
    {0mm}
    {\Huge\bfseries}
    \titlecontents{part}[0pt]
    {\large\bfseries}
    {\partname\ \thecontentslabel: \quad}
    {}
    {\hfill\contentspage}
\newcommand{\MarkRightAngle}[4][.3cm]
    {\coordinate (tempa) at ($(#3)!#1!(#2)$);
     \coordinate (tempb) at ($(#3)!#1!(#4)$);
     \coordinate (tempc) at ($(tempa)!0.5!(tempb)$);%midpoint
     \draw (tempa) -- ($(#3)!2!(tempc)$) -- (tempb);}
%--------------------------LENGTHS------------------------------%
% Spacings for the Table of Contents.
\addtolength{\cftsecnumwidth}{1ex}
\addtolength{\cftsubsecindent}{1ex}
\addtolength{\cftsubsecnumwidth}{1ex}
\addtolength{\cftfignumwidth}{1ex}
\addtolength{\cfttabnumwidth}{1ex}

% Spacing for multi-column and enumerate environments.
\setlength{\multicolsep}{6pt}
\setlist[enumerate]{itemsep=0pt,topsep=3pt}

% Indent and paragraph spacing.
\setlength{\parindent}{0em}
\setlength{\parskip}{0em}
%----------------------------GLOSSARY-------------------------------%
\makeglossaries
\loadglsentries{../../../glossary}
\loadglsentries{../../../acronym}
%--------------------------Main Document----------------------------%
\begin{document}
    \ifx\ifmain\undefined
        \pagenumbering{roman}
        \title{Set Theory}
        \author{Ryan Maguire}
        \date{\vspace{-5ex}}
        \maketitle
        \tableofcontents
        \clearpage
        \chapter*{Set Theory}
        \addcontentsline{toc}{chapter}{Set Theory}
        \markboth{}{SET THEORY}
        \vspace{10ex}
        \setcounter{chapter}{1}
        \pagenumbering{arabic}
    \else
        \chapter{Set Theory}
    \fi
    \section{Basic Concepts}
        
        \begin{ldefinition}{Sets}{Sets}
            A set is a collection of objects, called elements,
            none of which is the set itself.
        \end{ldefinition}
        The notion of a \textrm{set} is often left undefined,
        and we really haven't defined it very well here, either.
        The terms \textit{collection} and \textit{object} have
        not been defined, and thus this definition is somewhat
        meaningless. Intuitively, a set is a bunch of things that
        we can discuss, either by the properties these things have,
        or by listing out all of the things. The requirement that
        a set cannot contain itself is to avoid various
        paradoxes, such as the one discovered by Bertrand Russell
        in 1901. Given a set $A$ and an element $x$, we denote that
        $x$ is contained in $A$ by writing $x\in{A}$. If $x$ is
        not an element of $A$, we write $x\notin{A}$. There is a
        special set that contains no elements.
        \begin{ldefinition}{The Empty Set}{Empty_Set}
            The \gls{empty set} is the set $\emptyset$ such that,
            for all $x$, it is true that $x\notin\emptyset$.
        \end{ldefinition}
        The empty set contains no elements, and often we
        write $\emptyset=\{\}$. It is unique. Note that
        the empty set is different from the set $\{\emptyset\}$.
        Indeed, this would violate our requirement that sets do
        not contain themselves. The empty set contains no elements,
        whereas $\{\emptyset\}$ contains one element
        (It contains the empty set).
        \subsection{Sets}
            Sets are determined entirely by their elements and do
            not have a sense of \textit{order} on them. We can
            attempt to place order on sets by defining ordered
            pairs and functions. Before doing so, we should first
            discuss some of the properties of sets.
            \subsubsection{Basic Notions}
                A set is determined entirely by it's elements.
                Thus repetition of elements cannot be accounted for,
                and the sets $\{a,b\}$ and $\{a,a,b\}$ must be
                considered the same, since they have exactly
                the same elements. We define a subset of a
                set as follows:
                \begin{ldefinition}{Subsets}{Subsets}
                    A \gls{subset} of a set $B$ is a set $A$, denoted
                    $A\subseteq{B}$, such that for all $x\in{A}$,
                    it is true that $x\in{B}$. We write
                    $A\nsubseteq{B}$ to denote that $A$ is not
                    a subset of $B$.
                \end{ldefinition}
                This definition leads to the following, somewhat
                redundant, theorems.
                \begin{theorem}
                    \label{thm:Emptyset_Is_Subset}%
                    If $A$ is a set, then $\emptyset\subseteq{A}$.
                \end{theorem}
                \begin{proof}
                    For suppose not. Then there is an $x\in\emptyset$
                    such that $x\notin{A}$, a contradiction as
                    for all $x$, it is true that $x\notin\emptyset$.
                    Therefore, etc.
                \end{proof}
                \begin{theorem}
                    \label{thm:Subset_is_Transitive}%
                    If $A$, $B$, and $C$ are sets, if
                    $A\subseteq{B}$, and if $B\subseteq{C}$, then
                    $A\subseteq{C}$.
                \end{theorem}
                \begin{proof}
                    For suppose not. Then there is an $x\in{A}$ such
                    that $x\notin{C}$. But $A$ is a subset of $B$
                    and thus, by
                    Def.~\ref{def:Subsets}, $x\in{B}$.
                    But $B$ is a subset of $C$ and therefore,
                    again by Def.~\ref{def:Subsets},
                    $x\in{C}$. But $x\notin{C}$, a contradiction.
                    Therefore, etc.
                \end{proof}
                \begin{theorem}
                    \label{thm:Set_Is_Subset_Of_Self}%
                    If $A$ is a set, then $A\subseteq{A}$.
                \end{theorem}
                \begin{proof}
                    Suppose not. Then there is an $x\in{A}$
                    such that $x\notin{A}$, a contradiction.
                \end{proof}
                It would be useful to distinguish between subsets
                that aren't the entire set. Such beings are called
                proper subsets. First, we define the notion of
                equality.
                \begin{ldefinition}{Equal Sets}{Equal_Sets}
                    \Glspl{equal set} are sets $A$ and $B$, denoted
                    $A=B$, such that $A\subseteq{B}$ and
                    $B\subseteq{A}$.
                \end{ldefinition}
                We write $A\ne{B}$ to denote that $A$ and $B$ are
                not equal sets. From the definition, we have
                inequality when either $A\nsubseteq{B}$ or
                $B\nsubseteq{A}$. 
                We can now rigorously restate our claim that the
                empty set is unique.
                \begin{theorem}
                    If $\emptyset$ and $\emptyset'$ are sets with
                    no elements, then $\emptyset=\emptyset'$.
                \end{theorem}
                \begin{proof}
                    For suppose not. Then either
                    $\emptyset\nsubseteq\emptyset'$ or
                    $\emptyset'\nsubseteq\emptyset$. If
                    $\emptyset\nsubseteq\emptyset'$ then there is
                    an $x\in\emptyset$ such that
                    $x\notin\emptyset'$. But for all $x$,
                    $x\notin\emptyset$, a contradiction. Therefore
                    $\emptyset\subseteq\emptyset'$. Similarly,
                    $\emptyset'\subseteq\emptyset$. By the
                    definition of equality,
                    Def.~\ref{def:Equal_Sets},
                    $\emptyset=\emptyset'$.
                \end{proof}
                There are several trivial theorems that come from
                the definition of equality.
                \begin{theorem}
                    \label{thm:Equality_Reflexive}%
                    If $A$ and $B$ are equal sets, then $B$ and
                    $A$ are equal sets.
                \end{theorem}
                \begin{proof}
                    For suppose not. If $B\ne{A}$, then by
                    Def.~\ref{def:Equal_Sets}, either
                    $B\nsubseteq{A}$ or $A\nsubseteq{B}$.
                    But $A=B$, and thus $A\subseteq{B}$ and
                    $B\subseteq{A}$, a contradiction. Therefore, etc.
                \end{proof}
                \begin{theorem}
                    \label{thm:Equality_Symmetric}%
                    If $A$ is a set, then $A=A$.
                \end{theorem}
                \begin{proof}
                    For by
                    Thm.~\ref{thm:Set_Is_Subset_Of_Self}, if $A$ is
                    a set then $A\subseteq{A}$. Therefore, etc.
                \end{proof}
                \begin{theorem}
                    \label{thm:Equality_Transitive}%
                    If $A$, $B$, and $C$ are sets, if $A=B$, and if
                    $B=C$, then $A=C$.
                \end{theorem}
                \begin{proof}
                    By the definition of equality
                    (Def.~\ref{def:Equal_Sets})
                    if $A=B$, then $A\subseteq{B}$. But also if
                    $B=C$, then $B\subseteq{C}$. But by
                    Thm.~\ref{thm:Subset_is_Transitive}, if
                    $A\subseteq{B}$ and $B\subseteq{C}$, then
                    $A\subseteq{C}$. Therefore,
                    $A\subseteq{C}$. But if $B=C$, then
                    $C\subseteq{B}$
                    (Def.~\ref{def:Equal_Sets}).
                    And if $B=A$, then $B\subseteq{A}$. Therefore, by
                    Thm.~\ref{thm:Subset_is_Transitive},
                    $C\subseteq{A}$. But it was just proved that
                    $A\subseteq{C}$, and therefore, by the
                    definition of equality, $A=C$.
                \end{proof}
                These three properties,
                Thms.~\ref{thm:Equality_Reflexive}-%
                \ref{thm:Equality_Transitive}, are the key
                ingredients to define
                \textit{equivalence relations}. We'll develop this
                later once we've discussed the notion of
                \textit{Cartesian Products}. But first, we now
                define what a proper subset is.
                \begin{ldefinition}{Proper Subset}{Proper_Subset}
                    A \gls{proper subset} of a set $B$ is a set $A$,
                    denoted $A\subsetneq{B}$, such that
                    $A\subseteq{B}$ and $A\ne{B}$. We write
                    $A\not\subset{B}$ to denote $A$ is not a proper
                    subset of $B$.
                \end{ldefinition}
                \begin{theorem}
                    \label{thm:Prop_Subset_Not_Equal}%
                    If $A$ and $B$ are sets, and if $A\subsetneq{B}$,
                    then there is an $x\in{B}$ such that
                    $x\notin{A}$.
                \end{theorem}
                \begin{proof}
                    For suppose not. Then for all $x\in{B}$,
                    it is true that $x\in{A}$. But then
                    $B$ is a subset of $A$
                    (Def.~\ref{def:Subsets}).
                    But $A$ is a subset of $B$, and thus
                    by the definition of equality
                    (Def.~\ref{def:Equal_Sets})
                    $A=B$, a contradiction. Therefore, etc.
                \end{proof}
                Theorem \ref{thm:Prop_Subset_Not_Equal} can
                be used as an equivalent definition of a proper
                subset. That is, a proper subset is a subset that
                is missing at least one element. The notation
                $\subseteq$ and $\subset$ is analogous to the
                notation of inequalities that one finds in calculus,
                $\leq$ and $<$. In many texts, the two symbols
                $\subseteq$ and $\subset$ are taken to be
                identical, which can be a cause for confusion.
                We will attempt to be consistent in writing
                $\subseteq$ to mean any subset, and
                $\subset$ to mean a proper subset.
            \subsubsection{Operations on Sets}
                Similar to the arithmetic of real numbers, there
                are standard operations that can be performed on
                sets to obtain new sets. The four most common
                operations are union, intersection, set difference,
                and symmetric difference. Often the
                \textit{complement} of a set is discussed, but as
                we will see, this is just a specific case of set
                difference.
                \begin{ldefinition}{Union of Sets}{Union_of_Two_Sets}
                    The union of two sets, $A$ and $B$, is the set:
                    \begin{equation}
                        A\cup{B}=
                        \{x:x\in{A}\textrm{ or }x\in{B}\}
                    \end{equation}
                    That is, the set of all elements that are
                    contained in either $A$ or $B$, or both.
                \end{ldefinition}
                \begin{theorem}
                    \label{thm:Union_is_Bigger}%
                    If $A$ and $B$ are sets, then
                    $A\subseteq{A}\cup{B}$.
                \end{theorem}
                \begin{proof}
                    For suppose not. Then there is an
                    $x\in{A}$ such that $x\notin{A}\cup{B}$.
                    But by the definition of the union of two sets,
                    Def.~\ref{def:Union_of_Two_Sets},
                    if $x\in{A}$, then $x\in{A}\cup{B}$, a
                    contradiction. Therefore, etc.
                \end{proof}
                \begin{theorem}
                    If $A$ and $B$ are sets, and if
                    $A\subseteq{B}$, then $A\cup{B}=B$.
                \end{theorem}
                \begin{proof}
                    For by Thm.~\ref{thm:Union_is_Bigger},
                    for any two sets $A$ and $B$, it is true that
                    $B\subseteq{A}\cup{B}$. But by the definition of
                    subsets, Def.~\ref{def:Subsets},
                    if $A\subseteq{B}$, then for all $x\in{A}$, it
                    is true that $x\in{B}$.
                    Thus if $x\in{A}$ or if $x\in{B}$, then
                    $x\in{B}$. But then, for all $x\in{A}\cup{B}$,
                    it is true that $x\in{B}$, and therefore
                    $A\cup{B}\subseteq{B}$. But then $A\cup{B}=B$
                    (Def.~\ref{def:Equal_Sets}). Therefore, etc.
                \end{proof}
            \subsubsection{Cartesian Products}
                \begin{ldefinition}{Ordered Pair}{Ordered_Pair}
                    The \gls{ordered pair} of $x$ and $y$ is the set:
                    \begin{equation}
                        (x,y)=\big\{\{x\},\{x,y\}\big\}
                    \end{equation}
                \end{ldefinition}
                Note that from the definition, given two distinct
                elements $x$ and $y$, $(x,y)\ne(y,x)$. This
                definition of an ordered pair is due to Kazimierz
                Kuratowski and was first put forward in 1921. It
                does precisely what we want for an ordered pair,
                and distinguishes the order of the elements. There
                is a caveat to this definition, for we have the
                following reduction:
                \begin{equation}
                    (x,x)=\big\{\{x\},\{x,x\}\big\}
                    =\big\{\{x\},\{x\}\big\}
                    =\big\{\{x\}\big\}
                \end{equation}
                This does not create too much of an issue. An
                alternative definition due to Norbert Wiener was put
                forward in 1914. He defines an ordered pair as:
                \begin{equation}
                    (x,y)_{W}=\Big\{\big\{\{x\},\emptyset\big\},
                        \big\{\{y\}\big\}\Big\}
                \end{equation}
                To order sets that have more than two elements it is
                useful to define functions. First we define the
                \textit{Cartesian Product} of two sets.
                \begin{ldefinition}{Cartesian Product}
                      {Cartesian_Product}
                    The \gls{Cartesian product} of two sets $A$
                    and $B$ is the set:
                    \begin{equation}
                        A\times{B}=\{(a,b):a\in{A},b\in{B}\}
                    \end{equation}
                    Where $(a,b)$ denotes the ordered pair of
                    $a$ with $b$.
                \end{ldefinition}
                \begin{lexample}
                    One example of a Cartesian product
                    is the Euclidean plane, $\mathbb{R}^{2}$. This is
                    the set:
                    \begin{equation}
                        \mathbb{R}^{2}=\mathbb{R}\times\mathbb{R}
                    \end{equation}
                    Where $\mathbb{R}$ denotes the set of real numbers.
                    $\mathbb{R}^{2}$ is thus the set of all ordered
                    pairs of real numbers. This is the same thing as
                    the \textit{Cartesian} plane. Points
                    in the Cartesian plane are identified by their
                    \textit{coordinates}, $P=(x,y)$.
                \end{lexample}
                \begin{lexample}
                    \label{ex:Cartesian_Product}%
                    For the sake of computation, consider the
                    set $A=\{1,2,3\}$ and $B=\{a,b\}$. We can then
                    compute $A\times{B}$ directly:
                    \begin{equation}
                        A\times{B}=\big\{(1,a),(1,b),(2,a),(2,b),
                            (3,a),(3,b)\big\}
                    \end{equation}
                    We can also compute $A\times{A}$ and $B\times{B}$:
                    \begin{subequations}
                        \begin{align}
                            A\times{A}&=\big\{
                                (1,1),(1,2),(1,3),
                                (2,1),(2,2),(2,3),
                                (3,1),(3,2),(3,3)\big\}\\
                            B\times{B}&=\big\{
                                (a,a),(a,b),(b,a),(b,b)\big\}
                        \end{align}
                    \end{subequations}
                    Lastly, compute $B\times{A}$:
                    \begin{equation}
                        B\times{A}=\big\{
                            (a,1),(a,2),(a,3),
                            (b,1),(b,2),(b,3)\big\}
                    \end{equation}
                    Note that, since for distinct elements $\alpha$
                    and $\beta$, we have that
                    $(\alpha,\beta)\ne(\beta,\alpha)$, and thus
                    the Cartesian products $A\times{B}$ and $B\times{A}$
                    are not equal, in general. Equality is obtained
                    if and only if $A=B$.
                \end{lexample}
                Note that in Ex.~\ref{ex:Cartesian_Product}, the
                \textit{size} of the Cartesian product of two sets
                was simply the product of the number of elements of
                the constituent sets. That is, we see that $A$ has
                three elements and $B$ has two elements, but also
                that $A\times{B}$ has six elements. Moreover,
                $A\times{A}$ has nine elements, and $B\times{B}$ has
                four. This pattern holds for \textit{finite} sets.
                To be precise, we need to define what finite sets
                are, and what are \textit{infinite} sets. To do
                this we need the notion of a function.
        \subsection{Functions}
            \subsubsection{Definitions}
                \begin{ldefinition}{Functions}{Functions}
                    A \gls{function} from a set $A$ to a set $B$
                    is a subset $f$, denoted $f:X\rightarrow{Y}$
                    of the Cartesian product $A\times{B}$
                    such that, for all $x\in{A}$, there is a unique
                    $y\in{B}$ such that $(x,y)\in{f}$.
                \end{ldefinition}
                Given two sets $A$ and $B$, and a function
                $f:A\rightarrow{B}$, we often call $A$ the
                domain of $f$ and $B$ the co-domain.
                The \textit{image} of $x\in{X}$ is often written
                as $y=f(x)$. Requiring that $y$ be unique for each
                $x$ is equivalent to the \textit{vertical line test}
                one might find in a calculus course. We can also
                define the image of a subset of $Y$.
                \begin{ldefinition}{Image of a Set}{Image_of_Set}
                    The \gls{set image} of a subset
                    $S\subseteq{X}$ by a function
                    $f:X\rightarrow{Y}$ is the set:
                    \begin{equation}
                        f(S)=\{f(x)\in{Y}:x\in{X}\}
                    \end{equation}
                    That is, the set of all points in $Y$ that $S$
                    gets mapped to by $f$.
                \end{ldefinition}
                This is also called the \textit{range} of a function.
                In a similar manner, we can define the pre-image, or
                inverse image, of a set.
                \begin{ldefinition}{Pre-Image of a Set}
                      {Funct_Analysis_Pre_Image_of_Set}
                    The \gls{pre-image} of $S\subseteq{Y}$
                    by a function $f:X\rightarrow{Y}$ is the set:
                    \begin{equation}
                        f^{\minus{1}}(S)=\{x\in{X}:f(x)\in{S}\}
                    \end{equation}
                    That is, the set of all points in $X$ that map
                    into $S$.
                \end{ldefinition}
                \begin{lexample}
                    Given a function $f:X\rightarrow{Y}$, and any
                    non-empty subset $S\subseteq{X}$, the image
                    $f(S)$ is non-empty. This is not true for the
                    pre-image of a function. For let
                    $f:\mathbb{R}\rightarrow\mathbb{R}$ be defined by
                    $f(x)=1$ for all $x\in\mathbb{R}$. Then, for any
                    subset $S\subset\mathbb{R}$ such that
                    $1\notin{S}$, we have that
                    $f^{\minus{1}}(S)=\emptyset$.
                \end{lexample}
                There are many examples of functions, but certain
                ones are easier to study than others. We give some
                of these special functions names.
                \begin{ldefinition}{Injective Functions}
                      {Injective_Function}
                    An \gls{injective function} is a function
                    $f:X\rightarrow{Y}$ such that, for all
                    $x,y\in{X}$ such that $x\ne{y}$, it is true that
                    $f(x)\ne{f}(y)$.
                \end{ldefinition}
                That is, an injective function is a function
                $f:X\rightarrow{Y}$ such that $f(x_{1})=f(x_{2})$
                if and only if $x_{1}=x_{2}$. Such functions are also
                called \textit{one-to-one}.
                \begin{lexample}
                    Consider the natural logarithm
                    $\ln:\mathbb{R}^{+}\rightarrow\mathbb{R}$. This
                    is an injective function. For let
                    $x,y\in\mathbb{R}^{+}$ be such that $x\ne{y}$.
                    Suppose $\ln(x)=\ln(y)$. But then:
                    \begin{equation}
                        \ln(x)-\ln(y)=\ln\Big(\frac{x}{y}\Big)=0
                    \end{equation}
                    Recall the definition of the natural logarithm:
                    \begin{equation}
                        \ln(t)=\int_{1}^{t}\frac{1}{x}\diff{x}
                    \end{equation}
                    But then $\ln(t)=0$ if and only if $t=1$. Thus
                    $x=y$, a contradiction. Therefore $\ln$ is an
                    injective function. Not every function is
                    injective, for define
                    $f:\mathbb{R}\rightarrow\mathbb{R}$ by
                    $f(x)=x^{2}$. Then, for all $x\in\mathbb{R}^{+}$,
                    $f(\minus{x})=f(x)$, and thus $f$ cannot be an
                    injective function.
                \end{lexample}
                One might think that most functions are not injective,
                and indeed for the \textit{finite} case, this is true.
                For let $A$ and $B$ be finite sets with $n$ and $m$
                elements, respectively. If $m<n$, there can't be
                any injective function. Consider the case when $n=m$.
                Then we are simply counting the number of ways to
                permute the elements of $A$. This is $n!$. On the
                other hand, the total number of functions is
                $n^{n}$. Thus, the ratio of the number of injective
                functions to the number of functions is
                $n!/n^{n}$, and this decays to zero rapidly as
                $n$ get's large. Finally, if $m>n$, then the total
                number of injective functions is
                $n!\binom{m}{n}$, where $\binom{m}{n}$ is the
                binomial coefficient. The total number of functions
                is $n^{m}$. The ratio is thus:
                \begin{equation}
                    \frac{n!\binom{m}{n}}{n^{m}}=
                    \frac{n!\frac{m!}{n!(m-n)!}}{n^{m}}
                    =\frac{m!}{(m-n)!n^{m}}
                \end{equation}
                And again, this decays rapidly to zero and $n$ and $m$
                get large. Later, when we define infinite sets
                and the notion of Cardinality, we'll show that this
                trend continues. That is, in a sense, \textit{most}
                functions from a set $A$ to a sufficiently large set
                $B$ are not injective. Next, we define
                \textit{surjective} functions.
                \begin{ldefinition}{Surjective Functions}
                      {Surjective_Function}
                    A \gls{surjective function} is a function
                    $f:X\rightarrow{Y}$ such that $f(X)=Y$.
                    That is, for all $y\in{Y}$, there is an
                    $x\in{X}$ such that $f(x)=y$.
                \end{ldefinition}
                That is, every point $y\in{Y}$ gets mapped to by
                at least one point in $X$. It may also be true that
                many points in $X$ map to the same point in $Y$.
                The notions of surjective functions and injective
                functions are distinct, and neither implies the
                other. Surjective functions are also called
                \textit{onto}.
                \begin{ldefinition}{Bijective Functions}
                      {Bijective_Function}
                    A \gls{bijective function} is a function
                    that is both injective and surjective.
                \end{ldefinition}
                Sets $X$ and $Y$ such that there
                exists a bijective function $f:X\rightarrow{Y}$ are
                called \textit{equivalent}. Such sets can be said
                to have the same size. We say that $X$ is strictly
                smaller than $Y$ if there is an injective function
                $f:X\rightarrow{Y}$, but no bijective function.
            \begin{definition}
                A finite set is a set $X$ such that there
                exists an $n\in\mathbb{N}$ such that there is
                a bijective function
                $f:\mathbb{Z}_{n}\rightarrow{S}$
            \end{definition}
            \begin{definition}
                A countable set is a set
                $X$ such that there exists a bijective
                function $f:\mathbb{N}\rightarrow{X}$.
            \end{definition}
            Being countable means you can write
            the elements out in a list, or a
            one-to-one correspondence with all of
            the positive integers. Many sets are countable,
            including the whole numbers, integers, rational
            numbers, and \textit{algebraic} numbers. The
            union of finitely many countable sets is also
            countable, as is the union of countably many
            countable sets.
            \begin{example}
                The set of all positive even integers is
                countable. For let $\mathbb{N}_{e}$ be the
                set of all even integers and define
                $f:\mathbb{N}\rightarrow\mathbb{N}_{e}$ be
                $f(n)=2n$ for all $n\in\mathbb{N}$. This is
                a bijection, and thus $\mathbb{N}_{e}$ is
                countable. The set of all odd positive integers
                is countable, as shown by letting
                $f(n)=2n-1$. Even though the set of even
                integers may seem ``smaller,'' than the set of
                all integers, they are equivalent. The set of
                all integers $\mathbb{Z}$ is also countable.
                For let $f:\mathbb{N}\rightarrow\mathbb{Z}$
                be defined as:
                \begin{equation}
                    f(n)=
                    \begin{cases}
                        \frac{1}{2}(n-1),&n\textrm{ odd}\\
                        -\frac{n}{2},&n\textrm{ even}
                    \end{cases}
                \end{equation}
            \end{example}
            Any set that is infinite (Not finite) contains a
            countable subset. Thus, $\mathbb{N}$ can be
            considered as the \textit{smallest} infinite set.
            \begin{theorem}
                If $A$ is an infinite set, then there exists
                $S\subseteq{A}$ such that $S$ is countabl e.
            \end{theorem}
            \begin{proof}
                For as $A$ is infinite, for all $n\in\mathbb{N}$
                there exists a set $B\subseteq{A}$ such that
                $|B|=n$. For all $n\in\mathbb{N}$,
                define the following:
                \begin{equation}
                    \mathcal{S}_{n}=\{B\subseteq{A}:|B|=n\}
                \end{equation}
                Let $\mathcal{S}$ be defined as:
                \begin{equation}
                    \mathcal{S}=\{\mathcal{S}_{n}:n\in\mathbb{N}\}
                \end{equation}
                Then $\mathcal{S}$ is countable, for
                $a:\mathbb{N}\rightarrow\mathcal{S}$ defined
                by $a_{n}=\mathcal{S}_{n}$ is a bijection.
                By the axiom of choice, there is a function:
                \begin{equation}
                    \alpha:\mathcal{S}\rightarrow
                    \bigcup_{n=1}^{\infty}\mathcal{S}_{n}
                \end{equation}
                Such that, for all $x\in\mathcal{S}$,
                $\alpha(x)\in{x}$. But then, for all
                $x\in\mathcal{S}$, $\alpha(x)$ is a subset
                of $A$. But for all $x\in\mathcal{S}$, there
                is an $n\in\mathbb{N}$ such that
                $a_{n}=x$. Thus, let $S$ be the following:
                \begin{equation}
                    S=\bigcup_{n=1}^{\infty}\alpha(a_{n})
                \end{equation}
            \end{proof}
            \begin{theorem}
                \label{thm:Countable_Union_of_Countable}%
                If $A$ is a countable set such that for all
                $\mathcal{U}\in{A}$, $\mathcal{U}$ is a
                countable set, and if for all $a,b\in{A}$,
                $a\cap{b}=\emptyset$, then
                $\bigcup_{\mathcal{U}\in{A}}\mathcal{U}$
                is countable set.
            \end{theorem}
            \textit{Sketch of Proof.} The proof of
            Thm.~\ref{thm:Countable_Union_of_Countable}
            follows in the same manner
            as proving that the rationals are countable. Since
            there are countably many sets, write them out in
            a list $\mathcal{U}_{1}$, $\mathcal{U}_{2}$, and
            so on. Then write out the elements in a table as
            follows:
            \begin{table}[H]
                \captionsetup{type=table}
                \centering
                \begin{tabular}{ccccc}
                    $u_{11}$&$u_{12}$&$u_{13}$
                    &$u_{14}$&$\hdots$\\
                    $u_{21}$&$u_{22}$&$u_{23}$
                    &$u_{24}$&$\hdots$\\
                    $u_{31}$&$u_{32}$&$u_{33}$
                    &$u_{34}$&$\hdots$\\
                    $u_{41}$&$u_{42}$&$u_{43}$
                    &$u_{44}$&$\hdots$\\
                    $\vdots$&$\vdots$&$\vdots$
                    &$\vdots$&$\ddots$
                \end{tabular}
                \caption{Construction of a Bijection on the
                         Countable Union of Countably Infinite
                         Sets.}
                \label{table:Func_Countable_Union_of_Countable}
            \end{table}
            Where $u_{nm}$ is the $m^{th}$ element of
            $\mathcal{U}_{n}$.
            Using the \textit{diagonal argument},
            we obtain:
            \begin{table}[H]
                \captionsetup{type=table}
                \centering
                \begin{tabular}{|c|c|c|c|c|c|c|c|c|c|c|}
                    \hline
                    $\mathbb{N}$&1&2&3&4&5&6&7&8&9&$\hdots$\\
                    \hline
                    $\bigcup_{\mathcal{U}\in{A}}\mathcal{U}$&
                    $u_{11}$&$u_{12}$&$u_{21}$&$u_{13}$&
                    $u_{22}$&$u_{31}$&$u_{14}$&$u_{23}$&
                    $u_{32}$&$\hdots$\\
                    \hline
                \end{tabular}
                \caption{The Bijection Between $\mathbb{N}$ and
                         $\bigcup_{\mathcal{U}\in{A}}\mathcal{U}$}
                \label{table:Bijection_on_Countable_Union}
            \end{table}
            In the absence of the requirement that
            $a\cap{b}=\emptyset$ for all pairs in $\mathcal{U}$,
            we still have that the union is, at most, countable.
            The mapping we found would be a
            \textit{surjection}, rather than a bijection.
            The union is then either finite or countable. The
            Cantor-Schr\"{o}der-Bernstein Theorem can often be
            used to help identify the size of a set. This says
            that if $A$ and $B$ are sets such that there exists
            a surjective function $f:A\rightarrow{B}$ and a
            surjective function $g:B\rightarrow{A}$, then there
            is a bijective function $h:A\rightarrow{B}$. The
            requirement that $f$ and $g$ both be surjective
            can be replaced with the requirement that they both
            be injective. This is similar to saying that if
            $\Card(A)\leq\Card(B)$ and $\Card(B)\leq\Card(A)$,
            then $\Card(A)=\Card(B)$. Here, $\Card(A)$ denotes
            the \textit{cardinality} of the set $A$.
            \begin{definition}
                An uncountable set is a set that is
                neither finite nor countable.
            \end{definition}
            \begin{theorem}
                $\mathbb{Q}$ is countable.
            \end{theorem}
            \begin{proof}
                For we have that:
                \begin{equation}
                    \mathbb{Q}=
                    \bigcup_{n=-\infty}^{\infty}
                    \Big\{\frac{n}{m}:m\in\mathbb{N}\Big\}
                \end{equation}
                And this is the union of countably
                many countable sets, and is thus countable.
            \end{proof}
            \begin{theorem}
                $\mathbb{R}$ is uncountable.
            \end{theorem}
            \textit{Sketch of Proof.} We'll show that the unit
            interval $(0,1)$ is uncountable. Suppose not.
            Let $r_{ij}$ be the $j^{th}$ decimal of the $i^{th}$
            element in the list. We construct the real number
            $d$ as follows: If $d_{j}$ denotes the $j^{th}$
            decimal in $d$, let $d_{j}=r_{jj}+1$ if
            $r_{jj}\ne{9}$, and $d_{j}=0$ otherwise. Then
            $d\in(0,1)$, but $d$ is not on the list. For it's not
            the $n^{th}$ element, for it differs in the
            $n^{th}$ decimal place. Thus there is no bijection.
            Therefore, $(0,1)$ is uncountable. By extension,
            $\mathbb{R}$ is uncountable.
            \par\hfill\par
            \vspace{-2ex}
            For a set $X$, we often write
            $\mathcal{P}(X)$ to denote the
            \textit{power set} of $X$. This is the
            set of all subsets of $X$.
            For any set $X$ you can show that $X$ is
            strictly smaller than $\mathcal{P}(X)$.
            For example, $\mathcal{P}(\mathbb{N})$
            can be shown to be equivalent to $\mathbb{R}$.
            Since $\mathbb{N}$ is stricly smaller than
            $\mathbb{R}$, one might ask if there exists
            a set $X$ such that $\mathbb{N}$ is strictly
            smaller than $X$, but $X$ is strictly smaller
            than $\mathbb{R}$. Continuing, you can ask the
            same thing about $\mathbb{R}$ and
            $\mathcal{P}(\mathbb{R})$, and so on.
            This is called the continuum hypothesis.
            It turns out to be independent of
            the standard axioms of mathematics.
        \subsection{The Axiom of Choice}
        \subsection{Cardinality}
            We begin by talking about cardinality. This is the
            \textit{size} of a set. For an infinite set it
            doesn't make sense to talk about the \textit{number}
            of elements, but we can specify what it means two sets
            to have the same size. Sets $A$ and $B$ are equivalent
            if there exists a bijection $f:A\rightarrow{B}$.
            We then say that $A$ and $B$ have the same cardinality.
            The notation is written as $|A|=|B|$ or
            $\Card(A)=\Card(B)$. A finite set is a set $A$ such that
            there is a bijection between $A$ and $\mathbb{Z}_{n}$.
            We can then view the elements of $A$ as
            $A=\{a_{1},\hdots,a_{n}\}$. A countable set is a set
            $A$ such that there is a bijection between $A$ and
            $\mathbb{N}$. Here, $\mathbb{N}$ is the set of all
            natural numbers, or positive integers.
            \begin{lexample}
                There are many commonly discussed sets that are
                countably infinite. $\mathbb{N}$ is a trivial
                such example, but also $\mathbb{N}_{e}$ and
                $\mathbb{N}_{o}$, the sets of even and odd positive
                integers, respectively. For consider as bijections
                the following functions:
                \par
                \begin{subequations}
                    \begin{minipage}[b]{0.49\textwidth}
                        \centering
                        \begin{equation}
                            f_{e}(n)=2n
                        \end{equation}
                    \end{minipage}
                    \hfill
                    \begin{minipage}[b]{0.49\textwidth}
                        \centering
                        \begin{equation}
                            f_{0}(n)=2n-1
                        \end{equation}
                    \end{minipage}
                    \par\hfill\par
                    $\mathbb{Z}$ is also countable, as shown in
                    Fig.~\ref{fig:MEASURE_THEORY:BIJECTION_N_AND_Z}.
                    An explicit bijection for $\mathbb{Z}$ is:
                    \begin{equation}
                        f(n)=
                        \begin{cases}
                            \frac{n}{2},&n\mod{2}=0\\
                            \frac{1-n}{2},&n\mod{2}=1
                        \end{cases}
                    \end{equation}
                \end{subequations}
            \end{lexample}
            $\mathbb{Q}$ is also countable. We may intuitively
            think of $\mathbb{N}$ as being smaller than $\mathbb{Q}$,
            since there are simple \textit{surjections} that can be
            constructed from $\mathbb{Q}$ to $\mathbb{N}$. There
            is also a surjection from $\mathbb{N}$ onto
            $\mathbb{Q}^{+}$, as is shown in
            Fig.~\ref{fig:MEASURE_THEORY:BIJECTION_N_AND_Q_Plus}.
            \newpage
            \begin{figure}[H]
                \centering
                \captionsetup{type=figure}
                \subimport{../../../tikz/}{Surjection_From_N_to_Z}
                \caption{Diagram of a Bijection Between
                         $\mathbb{N}$ and $\mathbb{Z}$.}
                \label{fig:MEASURE_THEORY:BIJECTION_N_AND_Z}
            \end{figure}
            To construct such a surjection, write out all of the
            positive rational numbers in a grid so that $a_{nm}$ is
            the number $n/m$. Then zig-zag along the diagonals
            to construct the function. Thus there is a surjection
            $f:\mathbb{Q}^{+}\rightarrow\mathbb{N}$
            and a surjection $g:\mathbb{N}\rightarrow\mathbb{Q}^{+}$. The
            Cantor-Schr\"{o}eder-Bernstein theorem says that if there
            is surjection from $A$ to $B$ and a surjection from $B$
            to $A$, then there is a bijection between $A$ and $B$.
            Therefore there is a bijection between $\mathbb{N}$ and
            $\mathbb{Q}^{+}$, and $\mathbb{Q}^{+}$ is countable.
            \begin{figure}[H]
                \centering
                \captionsetup{type=figure}
                \resizebox{0.7\textwidth}{!}{%
                    \subimport{../../../tikz/}
                              {Surjection_From_N_to_Q_Plus.tex}
                }
                \caption{Diagram of a Surjection from
                         $\mathbb{N}$ onto $\mathbb{Q}^{+}$.}
                \label{fig:MEASURE_THEORY:BIJECTION_N_AND_Q_Plus}
            \end{figure}
            We can modify
            Fig.~\ref{fig:MEASURE_THEORY:BIJECTION_N_AND_Q_Plus}
            slightly to create a surjection between $\mathbb{N}$ and
            $\mathbb{Q}$, see
            Fig.~\ref{fig:MEASURE_THEORY:BIJECTION_N_AND_Q}.
            It is important to note that this bijection will not
            preserve the order of the rational numbers. The bijection
            will have to jump around back and forth. Any such
            bijection will be forced to do this, as the rationals are
            everywhere dense on $\mathbb{R}$. Any monotonic sequence of
            $\mathbb{Q}$ cannot possibly be a bijection.
            \begin{figure}[H]
                \centering
                \captionsetup{type=figure}
                \resizebox{\textwidth}{!}{%
                    \subimport{../../../tikz/}
                              {Surjection_From_N_to_Q.tex}
                }
                \caption{Diagram of a Surjection from
                         $\mathbb{N}$ onto $\mathbb{Q}$.}
                \label{fig:MEASURE_THEORY:BIJECTION_N_AND_Q}
            \end{figure}
            \begin{theorem}
                If $A$ is a countable infinite set and
                $B\subseteq{A}$, then either $B$ is finite or
                $B$ is countable.
            \end{theorem}
            \begin{proof}
                As $A$ is countable, there is a bijection
                $a:\mathbb{N}\rightarrow{A}$. Define the following:
                \begin{equation}
                    K=\{n\in\mathbb{N}:a_{n}\in{B}\}
                \end{equation}
                As $B\subseteq{A}$,
                this set contains a subsequence of points in
                $\mathbb{N}$ that get mapped into $B$. If $K$ is finite,
                then $B$ is finite, and if not then $K$ is countably
                infinite, and thus $B$ is countably infinite.
            \end{proof}
            \begin{theorem}
                If $A$ is an infinite set, then there exists a
                countable subset $B\subseteq{A}$.
            \end{theorem}
            \begin{proof}
                If $A$ is infinite then there is an
                $a_{1}\in{A}$. But, as $A$ is infinite,
                $A\setminus\{a_{1}\}$ is infinite, and there
                is an $a_{2}\in{A}\setminus\{a_{1}\}$. Continuing
                we obtain a sequence of distinct elements in $A$.
                Let $B=\{a_{n}:n\in\mathbb{N}\}$. Then
                $B\subseteq{A}$, and $B$ is countable.
            \end{proof}
            \begin{lexample}
                Suppose we have a collection of disjoint intervals
                of $\mathbb{R}$. This collection is either finite
                or countable. For in every interval, choose a
                rational number $q_{n}$. Let
                $A=\{q_{1},q_{2},\hdots\}$. Then
                $A\subseteq\mathbb{Q}$, and thus $A$ is either
                finite or countable. But this is also an enumeration
                of the intervals in the collection, and thus the
                collection is either finite or countable.
            \end{lexample}
            Given a countable collection of sets
            $A=\{\mathcal{A}_{1},\mathcal{A}_{2},\hdots\}$ such
            that, for all $n\in\mathbb{N}$, $\mathcal{A}_{n}$ is
            also a countable set, then the union is countable. That is:
            \begin{equation}
                B=\bigcup_{n=1}^{\infty}\mathcal{A}_{n}
            \end{equation}
            is a countable set. The proof of this is a mimicry of
            the proof of the countability of $\mathbb{Q}$. Not
            every set is either finite or countable. The real numbers,
            $\mathbb{R}$, is an example of an \textit{uncountable}
            set. First, some notes on the power set of a set.
            \begin{ldefinition}{Power Set}{Measure_Theory:Power_Set}
                The power set of a set $\Omega$, denoted
                $\mathcal{P}(\Omega)$, is the set of all subsets of
                $\Omega$:
                \begin{equation}
                    \mathcal{P}(\Omega)=
                    \{A:A\subseteq\Omega\}
                \end{equation}
            \end{ldefinition}
            \begin{lexample}
                \begin{subequations}
                    If $\Omega=\{1,2\}$, then the power set is:
                    \begin{equation}
                        \mathcal{P}(\Omega)=
                        \big\{\emptyset,\{1\},\{2\},\{1,2\}\big\}
                    \end{equation}
                    We must consider the empty set, since for any set
                    $A$, $\emptyset\subseteq{A}$.
                    If $\Omega=\{1,2,3\}$, then:
                    \begin{equation}
                        \mathcal{P}(\Omega)=
                        \big\{\emptyset,\{1\},\{2\},\{3\},\{1,2\},
                          \{1,3\},\{2,3\},\{1,2,3\}\big\}
                    \end{equation}
                    We see that, in the first example, a set with
                    2 elements has a power set with 4 elements. In the
                    second example we see that a set with 3 elements has
                    a power set with 8 elements. This pattern continues
                    for finite sets. If $A$ has $n$ elements, then
                    $\mathcal{P}(A)$ has $2^{n}$ elements. If
                    $A$ is an infinite set, then $\mathcal{P}(A)$ is
                    uncountable. Indeed:
                    \begin{equation}
                        \Card\big(\mathcal{P}(\mathbb{N})\big)=
                        \Card(\mathbb{R})
                    \end{equation}
                    We can show this by using the binary representation
                    of real numbers. We construct a bijection as
                    follows: If $A\subseteq\mathbb{N}$, then
                    let $r_{A}=0.n_{1}n_{2}\hdots$ where:
                    \begin{equation}
                        n_{i}=
                        \begin{cases}
                            0,&i\notin{A}\\
                            1,&i\in{A}
                        \end{cases}
                    \end{equation}
                    The function
                    $f:\mathcal{P}(\mathbb{N})\rightarrow[0,1]$
                    defined by $f(A)=r_{A}$ is thus a bijection.
                    That is, every element of $[0,1]$ gets mapped to in
                    a one-to-one manner. The potentially tricky numbers are
                    0 and 1, but $f(\emptyset)=0$, and $f(\mathbb{N})=1$.
                    Thus $\mathcal{P}(\mathbb{N})$ and $[0,1]$ are of the
                    same cardinality. But $(0,1)$ and $\mathbb{R}$
                    are of the same cardinality. To see this, consider
                    the graph of the function
                    $g:(0,1)\rightarrow\mathbb{R}$ defined as:
                    \begin{equation}
                        g(x)=\frac{2x-1}{x(1-x)}
                    \end{equation}
                    This is a bijection between the unit interval
                    $(0,1)$ and $\mathbb{R}$. One can also use the
                    \textit{stereographic projection} to show this.
                    But also $[0,1]$ and $(0,1)$ have the same cardinality.
                    For this, consider the following function:
                    \begin{equation}
                        f(x)=
                        \begin{cases}
                            \frac{1}{2},&x=0\\
                            \frac{1}{2^{n+2}},&x=\frac{1}{2^{n}}\\
                            x,&\textrm{Otherwise}
                        \end{cases}
                    \end{equation}
                    A graph of this is shown in
                    Fig.~\ref{fig:MEASURE_THEORY:Bijection_Closed_Interval_to_Open}.
                    Therefore, $\mathbb{R}$ and
                    $\mathcal{P}(\mathbb{N})$ have the same cardinality.
                    This can then be used to show that $\mathbb{R}$ is
                    uncountable.
                \end{subequations}
            \end{lexample}
            \begin{figure}[H]
                \centering
                \captionsetup{type=figure}
                \subimport{../../../tikz/}
                    {Bijction_Closed_to_Open_Interval}
                \caption{Bijection from $[0,1]$ to $(0,1)$.}
                \label{fig:MEASURE_THEORY:Bijection_Closed_Interval_to_Open}
            \end{figure}
            The power set of any set is strictly larger than the
            original set. If $\Omega$ is finite with $n$ elements, it
            can be shown that $\mathcal{P}(\Omega)$ has $2^{n}$
            elements. For infinite sets, there is a trivial surjection
            from $\mathcal{P}(\Omega)$ onto $\Omega$: for any element
            $x$, let $f(\{x\})=x$. This shows that
            $\Card(\Omega)\leq\Card(\mathcal{P}(\Omega))$. We now show
            that the inequality is strict.
            \begin{theorem}
                If $\Omega$ is a set, then there is no bijection
                $f:\Omega\rightarrow\mathcal{P}(\Omega)$
            \end{theorem}
            \begin{proof}
                For suppose not, and let
                $f:\Omega\rightarrow\mathcal{P}(\Omega)$ be such a
                bijection. Define:
                \begin{equation}
                    A=\{x\in\Omega:x\in{f}(x)\}
                \end{equation}
                Then $A\subseteq\Omega$, and thus
                $A\in\mathcal{P}(\Omega)$. But then the complement of
                $A$ is also an element of $\mathcal{P}(\Omega)$. But
                $f$ is a bijection and thus there is an $x\in\Omega$
                such that $f(x)=A^{C}$. If $x\in{f}(x)$, then
                $x\in{A}$, a contradiction as $f(x)=A^{C}$, and thus
                $x\in{A}^{C}$ as well. Therefore $x\notin{f}(x)$. But
                then $x\in{A}^{C}$. But, from the definition of $A$,
                since $x\in{A}^{C}$ and $f(x)=A^{C}$, $x\in{f}(x)$
                and thus $x\in{A}$, a contradiction. Thus there is no
                $x$ such that $f(x)=A^{C}$. Therefore, $f$ is not a
                bijection.
            \end{proof}
            From this we conclude that $\mathcal{P}(\mathbb{N})$
            is an uncountable infinite set. But since $\mathbb{R}$
            and $\mathcal{P}(\mathbb{N})$ have the same cardinality,
            $\mathbb{R}$ is also uncountable.
            If a set $A$ has the same cardinality as $\mathbb{R}$,
            we say that $A$ has the cardinality of the continuum.
            \begin{lexample}
                There is a bijection between the open unit
                square $(0,1)\times(0,1)$ and the open unit interval
                $(0,1)$. For an element $(x,y)\in(0,1)\times(0,1)$,
                let $z\in(0,1)$ be defined as
                $z=0.x_{1}y_{1}x_{2}y_{2}x_{3}y_{3}\dots$ This is
                a bijection, for all $(x,y)$ in the square there is
                a corresponding $z\in(0,1)$, and for all
                $z\in(0,1)$ there is a corresponding element of
                $(0,1)\times(0,1)$. We can say that $(x,y)$ can
                be coded into $z$, and $z$ can be decoded into
                $(x,y)$. Hence, $(0,1)\times(0,1)$ has the cardinality
                of the continuum. By stereographic projection and induction
                we obtain:
                \par\hfill\par
                \begin{subequations}
                    \begin{minipage}[b]{0.49\textwidth}
                        \begin{equation}
                            \Card(\mathbb{R}^{2})=\Card(\mathbb{R})
                        \end{equation}
                    \end{minipage}
                    \hfill
                    \begin{minipage}[b]{0.49\textwidth}
                        \begin{equation}
                            \Card(\mathbb{R}^{n})=\Card(\mathbb{R})
                        \end{equation}
                    \end{minipage}
                    \par
                \end{subequations}
            \end{lexample}
            \begin{lexample}
                Consider the set of all real-valued sequences. We've seen
                that any real number can be represented as a function
                $f:\mathbb{N}\rightarrow\{0,1\}$. A real-valued sequence
                is a function $a:\mathbb{N}\rightarrow\mathbb{R}$, and
                thus the set of real-valued sequences can be seen as the
                set of functions whose domain is $\mathbb{N}$ and whose
                range is the set of all functions
                $f:\mathbb{N}\rightarrow\{0,1\}$. So given a sequence
                $a$, the image of $a_{n}$, for $n\in\mathbb{N}$, is a
                function $f_{n}:\mathbb{N}\rightarrow\{0,1\}$. Therefore
                the set of all real-valued sequences can be represented
                as the set of all functions
                $g:\mathbb{N}\times\mathbb{N}\rightarrow\{0,1\}$, where
                $g(n,m)=f_{n}(m)$. But $\mathbb{N}\times\mathbb{N}$ is
                countable, and thus the set of all functions of the form
                $g:\mathbb{N}\times\mathbb{N}\rightarrow\{0,1\}$ has the
                same cardinality as the set of all functions of the form
                $f:\mathbb{N}\rightarrow\{0,1\}$. But this has the
                cardinality of the continuum. Therefore, the set of all
                real-valued sequences has the cardinality of the continuum.
            \end{lexample}
        \subsection{Set Operations}
            There are several operations that can be performed on sets,
            most notable are union, intersection, difference, and symmetric
            difference.
            \begin{ldefinition}{Set Union and Intersection}{Measure_Theory:Union}
                The union of two sets $A$ and $B$, denoted
                $A\cup{B}$, is the set:
                \begin{subequations}
                    \begin{equation}
                        A\cup{B}=\{x:x\in{A}\textrm{ or }x\in{B}\}
                    \end{equation}
                    The intersction of two sets $A$ and $B$, denoted
                    $A\cap{B}$, is the set:
                    \begin{equation}
                        A\cap{B}=\{x:x\in{A}\textrm{ and }x\in{B}\}
                    \end{equation}
                \end{subequations}
            \end{ldefinition}
            These two concepts can be visualized with Venn diagrams, as shown in
            Fig.~\ref{fig:MEASURE_THEORY_union_intersection_venn_diagram}.
            \begin{figure}[H]
                \centering
                \captionsetup{type=figure}
                \begin{subfigure}[b]{0.49\textwidth}
                    \centering
                    \subimport{../../../tikz/}{Venn_Diagram_Union}
                    \subcaption{Set Union}
                \end{subfigure}
                \begin{subfigure}[b]{0.49\textwidth}
                    \centering
                    \subimport{../../../tikz/}{Venn_Diagram_Intersection}
                    \subcaption{Set Intersection}
                \end{subfigure}
                \caption{Venn Diagrams Depicting the Union and Intersection
                         of the Sets $A$ and $B$.}
                \label{fig:MEASURE_THEORY_union_intersection_venn_diagram}
            \end{figure}
            \begin{ftheorem}{Laws of Union and Intersection}
                            {MEASURE_THEORY_UNION_LAWS_UNION_INTERSECTION}
                If $A$, $B$, and $C$ are sets, then the following are true:
                \par\hfill\par
                \begin{subequations}
                    \underline{Commutative Laws:}
                    \begin{align}
                        A\cup{B}&=B\cup{A}\\
                        A\cap{B}&=B\cap{A}
                    \end{align}
                    \underline{Associative Laws:}
                    \begin{align}
                        A\cup(B\cup{C})&=(A\cup{B})\cup{C}\\
                        A\cap(B\cap{B})&=(A\cap{B})\cap{C}
                    \end{align}
                    \underline{Distributive Laws:}
                    \begin{align}
                        A\cup(B\cap{C})&=(A\cup{B})\cap(A\cup{C})\\
                        A\cap(B\cup{C})&=(A\cap{B})\cup(A\cap{C})
                    \end{align}
                \end{subequations}
            \end{ftheorem}
            \begin{bproof}
                If $x\in{A}\cup{B}$, then either $x\in{A}$ or $x\in{B}$,
                or both. But then either $x\in{B}$ or $x\in{A}$, or both,
                and thus $x\in{B}\cup{A}$. Thus $A\cup{B}\subseteq{B}\cup{A}$.
                Similarly, $B\cup{A}\subseteq{A}\cup{B}$, and thus
                $A\cup{B}=B\cup{A}$. By a similar argument,
                $A\cap{B}=B\cap{A}$.
                \par\hfill\par
                If $x\in{A}\cup(B\cup{C})$, then either
                $x\in{A}$ or $x\in{B}\cup{C}$, or both.
                If $x\in{A}$, then $x\in{A}\cup{B}$, and
                therefore $x\in(A\cup{B})\cup{C}$. If not,
                then $x\in{B}\cup{C}$, and thus either
                $x\in{B}$ or $x\in{C}$, or both.
                If $x\in{B}$, then $x\in{A}\cup{B}$, and
                thus $x\in(A\cup{B})\cup{C}$. If not
                then $x\in{C}$, and therefore
                $x\in(A\cup{B})\cup{C}$. Therefore
                $A\cup(B\cup{C})\subseteq(A\cup{B})\cup{C}$.
                Similarly, $(A\cup{B})\cup{C}\subseteq{A}\cup(B\cup{C})$.
                Thus, $A\cup(B\cup{C})=(A\cup{B})\cup{C}$. By a similar
                argument, $A\cap(B\cap{C})=(A\cap{B})\cap{C}$.
                \par\hfill\par
                If $x\in{A}\cup(B\cap{C})$, then either $x\in{A}$
                or $x\in{B}\cap{C}$, or both. If $x\in{A}$, then
                $x\in{A}\cup{B}$ and $x\in{A}\cup{C}$, and therefore
                $x\in(A\cup{B})\cap(A\cup{C})$. If $x\in{B}\cap{C}$,
                then $x\in{B}$ and $x\in{C}$, and therefore
                $x\in(A\cup{B})\cap(A\cup{C})$. Thus,
                $A\cup(B\cap{C})\subseteq(A\cup{B})\cap(A\cup{C})$.
                Similarly,
                $(A\cup{B})\cap(A\cup{C})\subseteq{A}\cup(B\cap{C})$.
            \end{bproof}
            If $A$ and $B$ are sets, and if $C\subseteq{A}\cup{B}$, then
            either $C\subseteq{A}$ or $C\subseteq{B}$, or both. It is
            possible that $C\subseteq{A}\cup{B}$ and yet $C$ and $B$ have no
            elements in common, as long as $C\subseteq{A}$. As an example,
            take $A$ and $B$ to be disjoint sets. Then $A\subseteq{A}\cup{B}$,
            yet $A$ and $B$ have no elements in common. If
            $C\subseteq{A}\cap{B}$, then it must be true that
            $C\subseteq{A}$ and $C\subseteq{B}$.
            \begin{ldefinition}{Set Difference and Symmetric Difference}
                The set difference of a set $A$ with respect to a set $B$,
                denoted $B\setminus{A}$, is the set:
                \begin{equation}
                    B\setminus{A}=\{x\in{B}:x\notin{A}\}
                \end{equation}
                The symmetric difference of $A$ and $B$, denoted
                $A\ominus{B}$, is the set:
                \begin{equation}
                    A\ominus{B}=(A\cup{B})\setminus(A\cap{B})
                \end{equation}
            \end{ldefinition}
            As with the notions of unions and intersections, set differences and
            symmetric differences can be visualized using Venn diagrams.
            \begin{figure}[H]
                \centering
                \captionsetup{type=figure}
                \begin{subfigure}[b]{0.49\textwidth}
                    \centering
                    \subimport{../../../tikz/}{Venn_Diagram_Set_Difference}
                    \subcaption{Set Difference}
                \end{subfigure}
                \begin{subfigure}[b]{0.49\textwidth}
                    \centering
                    \subimport{../../../tikz/}{Venn_Diagram_Symmetric_Difference}
                    \subcaption{Symmetric Difference}
                \end{subfigure}
                \caption{Venn Diagrams Depicting the Set Difference and
                         Symmetric Difference of the Sets $A$ and $B$.}
                \label{fig:MEASURE_THEORY_Difference_sym_venn_diagram}
            \end{figure}
            \begin{theorem}
                \label{thm:MEASURE_THEORY_SET_DIFFERENCE_AS_INTERSECTION}
                If $A$, $B$, and $C$ are sets, and if $A\subseteq{C}$
                and $B\subseteq{C}$, then:
                \begin{equation}
                    B\setminus{A}=B\cap(C\setminus{A})
                \end{equation}
            \end{theorem}
            \begin{proof}
                For if $x\in{B}\setminus{A}$, then
                $x\in{B}$ and $x\notin{A}$. But
                $B\subseteq{C}$, and thus if $x\in{B}$, then $x\in{C}$.
                But if $x\notin{A}$, then $x\in{C}\setminus{A}$. Therefore
                $B\setminus{A}\subseteq{B}\cap(C\setminus{A})$.
                Similarly, $B\cap(C\setminus{A})\subseteq{B}\setminus{A}$,
                and therefore $B\setminus{A}={B}\cap(C\setminus{A})$.
            \end{proof}
            While set difference appears similar to subtraction that one finds in
            basic arithmetic, the two have their differences. For any two real
            numbers $a$ and $b$, $b=a-(a-b)$. For sets this is not true. For let
            $A$ be the empty set, and let $B$ be non-empty. Then
            $A\setminus(A\setminus{B})=\emptyset$, which is not $B$.
            Also, while it may seems convincing that
            $A\setminus(B\setminus{A})=A\setminus{B}$, this is not true. For
            let $A$ be a non-empty set and let $B=A$. Then
            $A\setminus(B\setminus{A})=A$, but $A\setminus{B}=\emptyset$.
            The concept of set difference can then be used to define the
            concept of complement.
            \begin{ldefinition}{Complement}
                The complement of a set $A$ with respect to a set
                $\Omega$, denoted $A^{C}$, is the set:
                \begin{equation}
                    A^{C}=\Omega\setminus{A}
                \end{equation}
            \end{ldefinition}
            Thm.~\ref{thm:MEASURE_THEORY_SET_DIFFERENCE_AS_INTERSECTION} can then
            be translated into the notation of complements as follows:
            \begin{theorem}
                If $A$, $B$, and $\Omega$ are sets, $A,B\subseteq\Omega$,
                and if $A^{C}$ is the complement of $A$ with respect
                to $\Omega$, then:
                \begin{equation}
                    B\setminus{A}=B\cap{A}^{C}
                \end{equation}
            \end{theorem}
            \begin{proof}
                By the definition of complement, $A^{C}=\Omega\setminus{A}$.
                As $A\subseteq\Omega$ and $B\subseteq\Omega$, by
                Thm.~\ref{thm:MEASURE_THEORY_SET_DIFFERENCE_AS_INTERSECTION},
                $B\setminus{A}=B\cap(\Omega\setminus{A})$, and therefore
                $B\setminus{A}=B\cap{A}^{C}$.
            \end{proof}
            \begin{ftheorem}{DeMorgan's Laws}{MEASURE_DEMORGAN}
                If $A$ and $B$ are sets, then:
                \begin{subequations}
                    \begin{align}
                        \big(A\cap{B}\big)^{C}
                        &=A^{C}\cup{B}^{C}\\
                        \big(A\cup{B}\big)^{C}
                        &=A^{C}\cap{B}^{C}
                    \end{align}
                \end{subequations}
            \end{ftheorem}
            With this, we can prove some results about set differences.
            \begin{theorem}
                If $A$ and $B$ are sets, then:
                \begin{equation}
                    A=\big(A\cap{B}\big)\cup\big(A\setminus{B}\big)
                \end{equation}
            \end{theorem}
            \begin{proof}
                For:
                \begin{subequations}
                    \begin{align}
                        \big(A\cap{B})\cup\big(A\setminus{B}\big)
                        &=\big(A\cap{B}\big)\cup\big(A\cap{B}^{C}\big)\\
                        &=A\cap(B\cup{B}^{C})\\
                        &=A\cap\Omega\\
                        &=A
                    \end{align}
                \end{subequations}
            \end{proof}
            \begin{theorem}
                If $A$, $B$, and $C$ are sets, then:
                \begin{equation}
                    A\cap\big(B\setminus{C}\big)
                    =\big(A\cap{B}\big)\cap\big(A\setminus{C}\big)
                \end{equation}
            \end{theorem}
            \begin{proof}
                For:
                \begin{subequations}
                    \begin{align}
                        A\cap\big(B\setminus{C}\big)
                        &=A\cap\big(B\cap{C}^{C}\big)\\
                        &=\big(A\cap{A}\big)\cap\big(B\cap{C}^{C}\big)\\
                        &=\big(A\cap{B}\big)\cap\big(A\cap{C}^{C}\big)\\
                        &=\big(A\cap{B}\big)\cap\big(A\setminus{C}\big)
                    \end{align}
                \end{subequations}
            \end{proof}
            Intersections do distribute over set differences.
            \begin{theorem}
                If $A$, $B$, and $C$ are sets, then:
                \begin{equation}
                    A\cap(B\setminus{C})=
                    (A\cap{B})\setminus(A\cap{C})
                \end{equation}
            \end{theorem}
            \begin{proof}
                For:
                \begin{subequations}
                    \begin{align}
                        \big(A\cap{B}\big)\setminus\big(A\cap{C}\big)
                        &=\big(A\cap{B}\big)\cap\big(A\cap{C}\big)^{C}\\
                        &=\big(A\cap{B}\big)\cap\big(A^{C}\cup{C}^{C}\big)\\
                        &=\Big(\big(A\cap{B}\big)\cap{A}^{C}\Big)
                            \cup\Big(\big({A}\cap{B}\big)\cap{C}^{C}\Big)\\
                        &=\Big(\big(A\cap{A}^{C}\big)\cap{B}\Big)\cup
                            \Big(\big(A\cap{B}\big)\cap{C}^{C}\Big)\\
                        &=\emptyset\cup
                            \Big(\big(A\cap{B}\big)\cap{C}^{C}\Big)\\
                        &=\big(A\cap{B}\big)\cap{C}^{C}\\
                        &=A\cap\big(B\cap{C}^{C}\big)\\
                        &=A\cap\big(B\setminus{C}\big)
                    \end{align}
                \end{subequations}
            \end{proof}
            Unions do not, however. For let $A$ be non-empty and let
            $A=B=C$. Then $A\cup(B\setminus{C})=A$, but
            $(A\cup{B})\setminus(A\cup{C})=\emptyset$.
            \begin{enumerate}
                \item DeMorgan's Laws:
                      \subitem $(A\cup{B})^{C}=A^{C}\cap{B}^{C}$
                      \subitem $(A\cap{B})^{C}=A^{C}\cup{B}^{C}$
            \end{enumerate}
            DeMorgan's Laws hold for arbitrary collections
            of set. If $I$ is some indexing set:
            \begin{align}
                \Big(\bigcup_{\alpha\in{I}}A_{\alpha}\Big)^{C}
                &=\bigcap_{\alpha\in{I}}A_{\alpha}^{C}\\
                \Big(\bigcap_{\alpha\in{I}}A_{\alpha}\Big)^{C}
                &=\bigcup_{\alpha\in{I}}A_{\alpha}^{C}
            \end{align}
            The set operations thus define binary operations
            on the power set of a set $\Omega$. It's important
            to note the notation. An element of $\Omega$ may
            be anything, while an element of
            $\mathcal{P}(\Omega)$ is a subset of $\Omega$.
            That is, the \textit{points} of $\mathcal{P}(\Omega)$
            are themselves sets. Thus, union, intersection,
            etc., define binary operations on
            $\mathcal{P}(\Omega)$. Given two subsets of
            $\Omega$, $A$ and $B$, $A\cup{B}$ is another
            subset of $\Omega$, as is $A\cap{B}$, and so on.
            The complement can also be seen as a unary operator
            on $\mathcal{P}(\Omega)$. From the various theorems
            presented, we have the following:
            \begin{enumerate}
                \item Union is commutative and associative.
                \item Intersection is commutative and
                      associative.
                \item Union distributes over intersection.
                \item Intersection distributes over union.
                \item DeMorgan's Laws hold.
                \item Set difference is not commutative,
                      nor is it associative.
            \end{enumerate}
        \subsection{Sets, Functions, and Countability}
            A quick review of the notations of various sets:
            \begin{table}[H]
                \centering
                \begin{tabular}{|l|l|l|}
                    \hline
                    Notation:&Description:&Importance:\\
                    \hline
                    $\mathbb{N}$&The Natural Numbers
                    &Used a lot.\\
                    \hline
                    $\mathbb{Z}$&The integers.&Never used.\\
                    \hline
                    $\mathbb{Z}_{n}$&Integers from $1$ to $n$.
                    &Mentioned occasionally.\\
                    \hline
                    $\mathbb{Q}$&The Rational Numbers&
                    Good for examples/counterexamples.\\
                    \hline
                    $\mathbb{R}$&The Real Numbers.&
                    Primary set of concern.\\
                    \hline
                    $\mathbb{C}$&The Complex Numbers&
                    Rarely used.\\
                    \hline
                \end{tabular}
            \end{table}
            From set theory, a function $f$ from a set $X$ to a
            set $Y$ is a subset of $X\times{Y}$ such that, for
            all $x\in{X}$, there is a unique $y\in{Y}$
            such that $(x,y)\in{f}$. We often call $X$ the
            domain, $Y$ the range or co-domain, and write
            $f:X\rightarrow{Y}$ to indicate this. The
            \textit{image} of $x\in{X}$ is often written
            as $y=f(x)$. Requiring that $y$ be unique for each
            $x$ is equivalent to the \textit{vertical line test}
            one might find in a calculus course.
            \begin{definition}
                The image of $S\subseteq{X}$
                by a function $f:X\rightarrow{Y}$
                is the set:
                \begin{equation}
                    f(S)=\{f(x)\in{Y}:x\in{X}\}
                \end{equation}
            \end{definition}
            The image of $S\subseteq{X}$ is the set of all
            points in $Y$ that $S$ gets mapped to by $f$.
            \begin{definition}
                The pre-image of $S\subseteq{Y}$ by a function
                $f:X\rightarrow{Y}$ is the set:
                \begin{equation}
                    f^{-1}(S)=\{x\in{X}:f(x)\in{S}\}
                \end{equation}
            \end{definition}
            An injective function is a function such
            that $f(x_{1})=f(x_{2})$ if and only if $x_{1}=x_{2}$.
            A surjective function is a function $f:X\rightarrow{Y}$
            such that $f(X)=Y$. That is, every point $y\in{Y}$
            gets mapped to by at least one point in $X$. Or, in
            more familiar notation, for all $y\in{Y}$ there is
            and $x\in{X}$ such that $y=f(x)$. A bijective
            function is a function that is both injective and
            surjective. Sets $X$ and $Y$ such that there
            exists a bijective function $f:X\rightarrow{Y}$ are
            called \textit{equivalent}. Such sets can be said
            to have the same size. We say that $X$ is strictly
            smaller than $Y$ if there is an injective function
            $f:X\rightarrow{Y}$, but no bijective function.
            \begin{definition}
                A finite set is a set $X$ such that there
                exists an $n\in\mathbb{N}$ such that there is
                a bijective function
                $f:\mathbb{Z}_{n}\rightarrow{S}$
            \end{definition}
            \begin{definition}
                A countable set is a set
                $X$ such that there exists a bijective
                function $f:\mathbb{N}\rightarrow{X}$.
            \end{definition}
            Being countable means you can write
            the elements out in a list, or a
            one-to-one correspondence with all of
            the positive integers. Many sets are countable,
            including the whole numbers, integers, rational
            numbers, and \textit{algebraic} numbers. The
            union of finitely many countable sets is also
            countable, as is the union of countably many
            countable sets.
            \begin{example}
                The set of all positive even integers is
                countable. For let $\mathbb{N}_{e}$ be the
                set of all even integers and define
                $f:\mathbb{N}\rightarrow\mathbb{N}_{e}$ be
                $f(n)=2n$ for all $n\in\mathbb{N}$. This is
                a bijection, and thus $\mathbb{N}_{e}$ is
                countable. The set of all odd positive integers
                is countable, as shown by letting
                $f(n)=2n-1$. Even though the set of even
                integers may seem ``smaller,'' than the set of
                all integers, they are equivalent. The set of
                all integers $\mathbb{Z}$ is also countable.
                For let $f:\mathbb{N}\rightarrow\mathbb{Z}$
                be defined as:
                \begin{equation}
                    f(n)=
                    \begin{cases}
                        \frac{1}{2}(n-1),&n\textrm{ odd}\\
                        -\frac{n}{2},&n\textrm{ even}
                    \end{cases}
                \end{equation}
            \end{example}
            Any set that is infinite (Not finite) contains a
            countable subset. Thus, $\mathbb{N}$ can be
            considered as the \textit{smallest} infinite set.
            \begin{theorem}
                If $A$ is an infinite set, then there exists
                $S\subseteq{A}$ such that $S$ is countabl e.
            \end{theorem}
            \begin{proof}
                For as $A$ is infinite, for all $n\in\mathbb{N}$
                there exists a set $B\subseteq{A}$ such that
                $|B|=n$. For all $n\in\mathbb{N}$,
                define the following:
                \begin{equation}
                    \mathcal{S}_{n}=\{B\subseteq{A}:|B|=n\}
                \end{equation}
                Let $\mathcal{S}$ be defined as:
                \begin{equation}
                    \mathcal{S}=\{\mathcal{S}_{n}:n\in\mathbb{N}\}
                \end{equation}
                Then $\mathcal{S}$ is countable, for
                $a:\mathbb{N}\rightarrow\mathcal{S}$ defined
                by $a_{n}=\mathcal{S}_{n}$ is a bijection.
                By the axiom of choice, there is a function:
                \begin{equation}
                    \alpha:\mathcal{S}\rightarrow
                    \bigcup_{n=1}^{\infty}\mathcal{S}_{n}
                \end{equation}
                Such that, for all $x\in\mathcal{S}$,
                $\alpha(x)\in{x}$. But then, for all
                $x\in\mathcal{S}$, $\alpha(x)$ is a subset
                of $A$. But for all $x\in\mathcal{S}$, there
                is an $n\in\mathbb{N}$ such that
                $a_{n}=x$. Thus, let $S$ be the following:
                \begin{equation}
                    S=\bigcup_{n=1}^{\infty}\alpha(a_{n})
                \end{equation}
            \end{proof}
            \begin{table}[H]
                \captionsetup{type=table}
                \centering
                \begin{tabular}{ccccc}
                    $u_{11}$&$u_{12}$&$u_{13}$
                    &$u_{14}$&$\hdots$\\
                    $u_{21}$&$u_{22}$&$u_{23}$
                    &$u_{24}$&$\hdots$\\
                    $u_{31}$&$u_{32}$&$u_{33}$
                    &$u_{34}$&$\hdots$\\
                    $u_{41}$&$u_{42}$&$u_{43}$
                    &$u_{44}$&$\hdots$\\
                    $\vdots$&$\vdots$&$\vdots$
                    &$\vdots$&$\ddots$
                \end{tabular}
                \caption{Construction of a Bijection on the
                         Countable Union of Countably Infinite
                         Sets.}
                \label{table:Countable_Union_of_Countable}
            \end{table}
            Where $u_{nm}$ is the $m^{th}$ element of
            $\mathcal{U}_{n}$.
            Using the \textit{diagonal argument},
            we obtain:
            In the absence of the requirement that
            $a\cap{b}=\emptyset$ for all pairs in $\mathcal{U}$,
            we still have that the union is, at most, countable.
            The mapping we found would be a
            \textit{surjection}, rather than a bijection.
            The union is then either finite or countable. The
            Cantor-Schr\"{o}der-Bernstein Theorem can often be
            used to help identify the size of a set. This says
            that if $A$ and $B$ are sets such that there exists
            a surjective function $f:A\rightarrow{B}$ and a
            surjective function $g:B\rightarrow{A}$, then there
            is a bijective function $h:A\rightarrow{B}$. The
            requirement that $f$ and $g$ both be surjective
            can be replaced with the requirement that they both
            be injective. This is similar to saying that if
            $\Card(A)\leq\Card(B)$ and $\Card(B)\leq\Card(A)$,
            then $\Card(A)=\Card(B)$. Here, $\Card(A)$ denotes
            the \textit{cardinality} of the set $A$.
            \begin{definition}
                An uncountable set is a set that is
                neither finite nor countable.
            \end{definition}
            \begin{theorem}
                $\mathbb{Q}$ is countable.
            \end{theorem}
            \begin{proof}
                For we have that:
                \begin{equation}
                    \mathbb{Q}=
                    \bigcup_{n=-\infty}^{\infty}
                    \Big\{\frac{n}{m}:m\in\mathbb{N}\Big\}
                \end{equation}
                And this is the union of countably
                many countable sets, and is thus countable.
            \end{proof}
            \begin{theorem}
                $\mathbb{R}$ is uncountable.
            \end{theorem}
            \textit{Sketch of Proof.} We'll show that the unit
            interval $(0,1)$ is uncountable. Suppose not.
            Let $r_{ij}$ be the $j^{th}$ decimal of the $i^{th}$
            element in the list. We construct the real number
            $d$ as follows: If $d_{j}$ denotes the $j^{th}$
            decimal in $d$, let $d_{j}=r_{jj}+1$ if
            $r_{jj}\ne{9}$, and $d_{j}=0$ otherwise. Then
            $d\in(0,1)$, but $d$ is not on the list. For it's not
            the $n^{th}$ element, for it differs in the
            $n^{th}$ decimal place. Thus there is no bijection.
            Therefore, $(0,1)$ is uncountable. By extension,
            $\mathbb{R}$ is uncountable.
            \par\hfill\par
            \vspace{-2ex}
            For a set $X$, we often write
            $\mathcal{P}(X)$ to denote the
            \textit{power set} of $X$. This is the
            set of all subsets of $X$.
            For any set $X$ you can show that $X$ is
            strictly smaller than $\mathcal{P}(X)$.
            For example, $\mathcal{P}(\mathbb{N})$
            can be shown to be equivalent to $\mathbb{R}$.
            Since $\mathbb{N}$ is stricly smaller than
            $\mathbb{R}$, one might ask if there exists
            a set $X$ such that $\mathbb{N}$ is strictly
            smaller than $X$, but $X$ is strictly smaller
            than $\mathbb{R}$. Continuing, you can ask the
            same thing about $\mathbb{R}$ and
            $\mathcal{P}(\mathbb{R})$, and so on.
            This is called the continuum hypothesis.
            It turns out to be independent of
            the standard axioms of mathematics.
        \section{Set Theory}
            \begin{definition}
                A set is a collection of distinct objects,
                none of which are the set itself.
            \end{definition}
            As neither "Collection," nor "Objects," have been
            define, the above definition is logically meaningless.
            \begin{definition}
                The objects in a set are called the elements of
                the set. If $x$ is an element of $A$,
                we write $x\in A$.
            \end{definition}
            \begin{definition}
                The empty set $\emptyset$ is the set containing
                no elements. It is unique.
            \end{definition}
            \begin{definition}
                A set $A$ is said to be a subset of a $B$ if
                and only if $x\in{A}\Rightarrow{x}\in{B}$.
                This is denoted $A\subset B$.
            \end{definition}
            \begin{corollary}
                For any set $A$, $\emptyset\subset{A}$
                and $A\subset{A}$.
            \end{corollary}
            \begin{proof}
                Suppose not. Then
                $\exists{x}\in\emptyset:x\notin A$.
                A contradiction. Suppose $A\not\subset A$.
                Then $\exists{x}\in{A}:x\notin{A}$,
                a contradiction.
            \end{proof}
            \begin{definition}
                $A\subset B$ is said to be a proper subset
                of $B$ if and only if there is an $x\in B$
                such that $x\notin A$.
            \end{definition}
            \begin{definition}
                Two sets $A$ and $B$ are said to be equal
                if and only if $x\in{A}\Leftrightarrow{x}\in B$.
            \end{definition}
            \begin{theorem}
                Two sets $A$ and $B$ are equal if and only
                if $A\subset{B}$ and $B\subset{A}$.
            \end{theorem}
            \begin{proof}
                $[A=B]\Leftrightarrow%
                 \big[[x\in{A}\Rightarrow{x}\in{B}]\big]%
                 \land\big[[y\in{B}\Rightarrow{y}\in{A}]\big]%
                 \Leftrightarrow[A\subset{B}\land{B}\subset{A}]$. 
            \end{proof}
            \begin{definition}
                The set difference of a set $A$ with respect
                to a set $B$, denoted $B\setminus{A}$, is the set
                $B\setminus{A}=\{x\in{B}:x\notin{A}\}$.
            \end{definition}
            \begin{theorem}
                If $A$ and $B$ are sets and $A\subset B$,
                then $B\setminus(B\setminus A)=A$.
            \end{theorem}
            \begin{proof}
                $[x\in B\setminus(B\setminus{A})]%
                 \Rightarrow[x\in{B}\land{x}\notin%
                 \{x\in{B}:x\notin{A}\}]%
                 \Rightarrow[x\in{A}\subset{B}]$.
                 $[x\in{A}]\Rightarrow[x\notin{B}\setminus{A}]%
                 \Rightarrow[x\in{B}\setminus(B\setminus{A})]$.
            \end{proof}
            A universe set is the set under consideration
            from which all subsets are drawn.
            \begin{definition}
                If $\mathcal{U}$ is a universe set,
                $A\subset\mathcal{U}$, then the complement of
                $A$ is $\mathcal{U}\setminus{A}$ and is
                denoted $A^{C}$.
            \end{definition}
            The previous theorem shows that $(A^c)^{C}=A$.
            \begin{definition}
                If $A$ and $B$ are sets, then
                $A\cup{B}=\{x: x\in A \lor x\in B\}$
                is called their union.
            \end{definition}
            \begin{corollary}
                If $A$ and $B$ are sets, then $A\subset{A}\cup{B}$.
            \end{corollary}
            \begin{proof}
                $[x\in A]\Rightarrow[x\in A\lor x\in B]%
                 \Rightarrow[x\in{A}\cup{B}]$.
            \end{proof}
            \begin{theorem}
                If $A$ and $B$ are sets, $A=A\cup B$
                if and only if $B\subset A$.
            \end{theorem}
            \begin{proof}
                $[B\subset A]\Rightarrow\big[[x\in A\cup B]%
                 \Rightarrow[x\in A]\big]%
                 \Rightarrow[A\cup B \subset A],[A\subset A\cup B]%
                 \Rightarrow[A=A\cup B]$.
                $[x\in B] \Rightarrow [x\in A\cup B]%
                 \Rightarrow [x\in A]$
            \end{proof}
            \begin{definition}
                If $A$ and $B$ are sets, then
                $A\cap{B}=\{x:x\in{A}\land{x}\in{B}\}$
                is called their intersection.
            \end{definition}
            \begin{corollary}
                If $A$ and $B$ are sets,
                $A\cap{B}\subset{A}$ and $A\cap{B}\subset{B}$.
            \end{corollary}
            \begin{proof}
                $[x\in A\cap B]\Rightarrow[x\in A\land x\in B]%
                 \Rightarrow\big[[A\cap B \subset A]%
                 \land[A\cap B \subset B]\big]$.
            \end{proof}
            \begin{theorem}
                If $A$ and $B$ are sets, then $A=A\cap{B}$
                if and only if $A\subset{B}$.
            \end{theorem}
            \begin{proof}
            $[A=A\cap B]\Rightarrow [x\in A\Rightarrow x\in A \cap B]\Rightarrow [x\in B]$. $[A\subset B]\Rightarrow [x\in A\Rightarrow x\in B]\Rightarrow [x\in A\cap B]\Rightarrow [A=A\cap B]$.
            \end{proof}
            \begin{theorem}
            If $A,B$, and $C$ are sets, then the following are true:
            \begin{enumerate}
            \item $A\cap (B\cup C) = (A\cap B)\cup (A\cap C)$
            \item $A\cup (B\cup C) = (A\cup B)\cap (A\cup C)$
            \end{enumerate}
            \end{theorem}
            \begin{proof}
            In order,
            \begin{enumerate}
            \item $[x\in A\cap (B\cup C)]\Rightarrow \big[[x\in A] \land [x\in B\cup C]\big]\Rightarrow \big[[x\in A\land x\in B]\lor [x\in A\land x\in C]\big]\Rightarrow [x\in (A\cap B)\cup (A\cap C)]$. $[x\in (A\cap B)\cup(A\cap C)]\Rightarrow \big[[x\in A\land x\in C]\lor [x\in A \land x\in C]\big]\Rightarrow \big[[x\in A]\land [x\in B\lor x\in C]\big]\Rightarrow [x\in A\cap(B\cup C)]$.
            \item $[x\in A\cup (B\cap C)]\Rightarrow \big[[x\in A]\lor [x\in B\cap C]\big] \Rightarrow \big[[x\in A \lor x\in B]\land [x\in A$ or $x\in C]\big]\Rightarrow [x\in (A\cap B)\cup (A\cap C)]$. $[x\in (A\cup B)\cap (A\cup C)]\Rightarrow \big[[x\in A\lor B]\land [x\in A\lor B]\big]\Rightarrow \big[[x\in A]\lor[x\in B\land C]\big]\Rightarrow [x\in A\cap(B\cup C)]$.
            \end{enumerate}
            \end{proof}
            \begin{theorem}[DeMorgan's Laws]
            If $A$ and $B$ are subsets of some universe $\mathcal{U}$, then the following are true:
            \begin{enumerate}
            \item $(A\cup B)^c = A^c \cap B^c$
            \item $(A\cap B)^c = A^c \cup B^c$
            \end{enumerate}
            \end{theorem}
            \begin{proof}
            In order,
            \begin{enumerate}
            \item $[x\in (A\cup B)^c]\Rightarrow [x\in A^c\land x\in B^c]\Rightarrow [x\in A^c\cap B^c]$. $[x\in A^c \cap B^c]\Rightarrow [x\in A^c\land x\in B^c]\Rightarrow [x\notin A\cup B]\Rightarrow [x\in (A\cup B)^c]$.
            \item $[x\in (A\cap B)^c]\Rightarrow [x\in A^c\lor x\in B^c]\Rightarrow [x\in A^c \cup B^c]$. $[x\in A^c \cup B^c]\Rightarrow [x\notin A\lor x\notin B]\Rightarrow [x\notin A\cap B]\Rightarrow [x\in (A\cap B)^c]$.
            \end{enumerate}
            \end{proof}
            \begin{definition}
            If $A$ is a set and $a,b\in A$, then the ordered pair $(a,b)$ is the set $\{\{a\},\{a,b\}\}$.
            \end{definition}
            \begin{remark}
            This definition is due to Kuratowski. Note that $(a,b)$ and $(b,a)$ are not necessarily equal.
            \end{remark}
            \begin{definition}
            The Cartesian Product of $A$ and $B$ is defined as $A\times B = \{(a,b):a\in A, b\in B\}$.
            \end{definition}
            \begin{definition}
            The power set of a set $A$ is the set $\mathcal{P}(A) = \{\mathcal{U}:\mathcal{U}\subset A\}$. That is, it is the set of all subsets of $A$.
            \end{definition}
        \subsection{Equivalence and Transfinite Cardinal Numbers}
            \begin{definition}
            $A$ and $B$ are called equivalent if and only if there is a bijective function $f:A\rightarrow B$. We write $A\sim B$.
            \end{definition}
            \begin{theorem}
            Equivalence has the following properties:
            \begin{enumerate}
            \item $A\sim A$ for any set $A$.
            \item If $A\sim B$, then $B\sim A$.
            \item If $A\sim B$ and $B\sim C$, then $A\sim C$.
            \end{enumerate}
            \end{theorem}
            \begin{proof}
            In order,
            \begin{enumerate}
            \item For let $f$ be the identity mapping. That is, for all $x\in A$, $f(x) = x$. This is bijective and thus $A\sim A$.
            \item If $A\sim B$, there is a bijective function $f:A\rightarrow B$. Then $f^{-1}:B\rightarrow A$ is bijective, and $B\sim A$.
            \item Let $f:A\rightarrow B$ and $g:B\rightarrow C$ be bijections. Then $g\circ f:A\rightarrow C$ is a bijection, and thus $A\sim C$.
            \end{enumerate}
            \end{proof}
            \begin{theorem}
            If $A\sim C$ and $B\sim D$, where $A,B$ and $C,D$ are disjoint, then $A\cup B \sim C\cup D$
            \end{theorem}
            \begin{proof}
            Let $f:A\rightarrow C$ and $g:B\rightarrow D$ be isomorphisms. Let $h:A\cup B \rightarrow C\cup D$ be defined by $h(x) = \begin{cases} f(x), & x\in A\\ g(x), & x\in B\end{cases}$. As $A$ and $B$ are disjoint, this is indeed a function and it is bijective as $C$ and $D$ are disjoint. Therefore, etc.
            \end{proof}
            \begin{definition}
            The set $\mathbb{Z}_n$ is defined for all $n\in \mathbb{N}$ as $\{k\in \mathbb{N}: k\leq n\}$.
            \end{definition}
            \begin{definition}
            A set $A$ is a said to be finite if and only if there is some $n\in \mathbb{N}$ such that there is a bijection $f:\mathbb{Z}_n \rightarrow A$.
            \end{definition}
            \begin{definition}
            If $A$ is a set that is equivalent to $\mathbb{Z}_n$ for some $n\in \mathbb{N}$, then the cardinality of $A$, denoted $|A|$, is $n$.
            \end{definition}
            \begin{theorem}
            For two finite sets $A$ and $B$, $A\sim B$ if and only if $|A|=|B|$.
            \end{theorem}
            \begin{proof}
            $[|A|=|B|=n]\Rightarrow[A\sim \mathbb{Z}_n]\land[B\sim \mathbb{Z}_n]\Rightarrow [A\sim B]$. $[A\sim B]\Rightarrow [\exists \underset{Bijective}{f:A\rightarrow B}]\Rightarrow [f(A) = B]\Rightarrow [|A|=|B|]$.
            \end{proof}
            \begin{definition}
            A set $A$ is said to be infinite if and only if there is a proper subset $B\underset{Proper}\subset A$ such that $B\sim A$.
            \end{definition}
            \begin{theorem}
            Infinite sets are not finite.
            \end{theorem}
            \begin{proof}
            Suppose not. Let $A$ be an infinite set and suppose there is an $n\in \mathbb{N}$ such that $A\sim \mathbb{Z}_n$. But as $A$ is an infinite set, there is a proper subset $B$ such that $B\sim A$. But then $B\sim \mathbb{Z}_n$. But as $B$ is a proper subset, there is at least one point in $A$ not contained in $B$. But then $|B|<n$, a contradiction. Thus $A$ is not finite.
            \end{proof}
            \begin{corollary}
            If $A$ is an infinite set, then for every $n\in \mathbb{N}$ there is a subset $B\subset A$ such that $B\sim \mathbb{Z}_n$.
            \end{corollary}
            \begin{proof}
            Suppose not. Then there is a least $n\in \mathbb{N}:B\subset A\Rightarrow |B|<n$. But then $A$ has at most $n$ elements, a contradiction.
            \end{proof}
            \begin{definition}
            A set $A$ is called countable if and only if $A\sim \mathbb{N}$.
            \end{definition}
            \begin{theorem}
            A set $A$ is infinite if and only if it contains a proper subset $B$ such that $B\sim \mathbb{N}$.
            \end{theorem}
            \begin{proof}
            If $A$ has a proper subset $B$ such that $B\sim \mathbb{N}$, then $A$ is not finite and is thus infinite. If $A$ is infinite, then for all $n\in \mathbb{N}$ there is a set $A_n\subset A$ such that $A_n \sim \mathbb{Z}_n$. Let $B = \{a_n: a_n \in A_n, a_n \notin A_{n-1}\}$. Note that $a_{n} = a_{m}$ if and only if $m= n$. Let $f:\mathbb{N} \rightarrow B$ be defined by $n\mapsto a_n$. This is bijective, and thus $B\sim \mathbb{N}$.
            \end{proof}
            \begin{remark}
            This shows that $\mathbb{N}$ is, in a sense, the "Smallest," infinite set. $|\mathbb{N}|$ is denoted $\aleph_0$.
            \end{remark}
            \begin{definition}
            A set is called uncountable if and only if it is infinite and not countable.
            \end{definition}
            \begin{lemma}
            If $B\subset A$, $f:A\rightarrow B$ is injective, then there is a bijection $g:A\rightarrow B$
            \end{lemma}
            \begin{proof}
            Let $Y = A\setminus B$, and inductively define $f^{k+1}(Y) = f(f^{k}(Y))$. Let $X = Y\cup (\cup_{k=0}^{\infty} f^{k}(Y))$. As $Y\cap B = \emptyset$, then $f(Y)\cap Y= \emptyset$. As $f$ is an injection, $f(f(Y))\cap f(Y)=\emptyset$, and similarly $f(f(Y))\cap Y = \emptyset$. Inductively, $f^{n}(Y)\cap f^{m}(Y) = \emptyset$, for $n\ne m$. It then also follows that $f(X) = \cup_{k=1}^{\infty} f^{k}(Y)$. Thus $A\setminus X = [B\cup Y]\setminus [Y\cup f(X)] = B\setminus f(X)$. Let $g(x) = \begin{cases} f(x), & x\in X \\ x, & x \in B\setminus f(X)\end{cases}$. This is a bijections from $A$ to $B$.
            \end{proof}
            \begin{theorem}[Cantor-Schr\"{o}der-Bernstein Theorem]
            If $A_1 \subset A$, $B_1 \subset B$, and $A\sim B_1$, $B \sim A_1$, then $A\sim B$.
            \end{theorem}
            \begin{proof}
            Let $f:A\rightarrow B_1$ and $g:B\rightarrow A_1$ be bijections.Then $(g\circ f):A\rightarrow A_1$ is an injection from $A$ into $A_1$. Thus, there is a bijection $h:A\rightarrow A_1$. Thus, $A\sim A_1 \sim B\Rightarrow A\sim B$.
            \end{proof}
            \begin{theorem}
            $\mathbb{N}\times \mathbb{N}$ is countable.
            \end{theorem}
            \begin{proof}
            For $f:\mathbb{N} \rightarrow \mathbb{N}\times \mathbb{N}$ defined by $f(n) = (0,n)$ shows there is a subset $N_1$ of $\mathbb{N} \times \mathbb{N}$ such that $\mathbb{N}\sim N_1$. And $g:\mathbb{N}\times \mathbb{N} \rightarrow \mathbb{N}$ defined by $g(n,m) =n+2^{n+m}$ shows that there is a subset $M_1 \subset \mathbb{N}$ such that $\mathbb{N} \times \mathbb{N} \sim M_1$. By the Cantor-Schr\"{o}der-Bernstein Theorem, $\mathbb{N} \sim \mathbb{N}\times \mathbb{N}$.
            \end{proof}
            \begin{lemma}
            If $A$ is infinite and $f:A\rightarrow \mathbb{N}$ is injective, then $A$ is countable.
            \end{lemma}
            \begin{proof}
            As $A$ is infinite and $A\sim f(A)$, $f(A)$ is infinite. But as $f(A)\subset \mathbb{N}$ and $f(A)$ is infinite, $f(A)\sim \mathbb{N}$. Thus, $A\sim \mathbb{N}$. 
            \end{proof}
            \begin{theorem}
            $\mathbb{Q}$ is countable.
            \end{theorem}
            \begin{proof}
            For each $x\in \mathbb{Q}$, $x\ne 0$, let $p_x,q_x\in\mathbb{Z}:q_x>0$ be the unique integers such that $x = \frac{p_x}{q_x}$ and $g.c.d.(|p_x|,|q_x|)=1$. Define $f:\mathbb{Q}\rightarrow \mathbb{N}\times \mathbb{N}$ as $f(x) = \begin{cases}(p_x,q_x), & x\ne 0 \\ 0, & x=0\end{cases}$. This shows there is a subset $N_1$ of $\mathbb{N}\times \mathbb{N}$ such that $\mathbb{Q}\sim N_1$. But $N_1$ is an infinite subset of $\mathbb{N}\times\mathbb{N}$ as $\mathbb{Q}$ is infinite and $f$ is injective. Define $g$ as $n+2^{n+m}:(n,m)\in N_1$. This is an injective function into $\mathbb{N}$, and thus $N_1 \sim \mathbb{N}$. Therefore $\mathbb{Q}\sim \mathbb{N}$.
            \end{proof}
            \begin{definition}
            If $A$ and $B$ are sets, we say that $|A|<|B|$ if there is an injective function $f:A\rightarrow B$, yet no bijection.
            \end{definition}
            \begin{theorem}[Cantor's Theorem]
            For a set $M$, $|M|<|\mathcal{P}(M)|$.
            \end{theorem}
            \begin{proof}
            For let $M$ be a set with cardinality $|M|$. Let $U_m \subset M$ such that $U_m \sim M$. Such a set exists, for example, the singletons of $\mathcal{P}(M)$. Thus, $M$ is split into two distinct sets $Class\ I=\{x\in M: \textrm{There is a subset } X\subset U_m\textrm{ such that }x\in X\}$, and $Class\ II=M-Class\ I$. Let $L = Class\ II$. $L\subset M$, and thus $L\in \mathcal{P}(M)$. However, $L \notin U_m$ for if it were, then the element $m_1$ paired with it in $M$ is of Class II (For it cannot be of Class I as $m_1$ would not appear in $L$). If $m_1$ were in Class II, then by definition $m_1 \notin L$. But as $m_1 \in L$, we see that $L\notin U_m$. Thus, $|U_m| <|\mathcal{P}(M)|$, and therefore $|M|<|\mathcal{P}(M)|$.
            \end{proof}
            \begin{theorem}
            The set $R=\{x\in \mathbb{R}:0<x<1\}$ is equivalent to $\mathcal{P}(\mathbb{N})$.
            \end{theorem}
            \begin{proof}
            For every real number has a binary representation (Proof of this is omitted). That is, for every real number $r$, $ r = \sum_{n=-\infty}^{\infty} \frac{a_n}{2^n}$, where $a_n = 0$ or $1$. As $0<x<1$, this sum is just $\sum_{n=1}^{\infty} \frac{a_n}{2^n}$. Let $f:\mathcal{P}(\mathbb{N})\rightarrow R$ be defined by the following: If $N\subset \mathcal{P}(\mathbb{N})$ and $n\in N$, then $a_n = 1$, other wise $n=0$. Then every real number is matched to a subset of $\mathcal{P}(\mathbb{N})$, moreover this is done bijectively. Thus, $\mathcal{P}(\mathbb{N})\sim R$.
            \end{proof}
            \begin{theorem}
            $\mathbb{R} \sim \mathcal{P}(\mathbb{N})$.
            \end{theorem}
            \begin{proof}
            It suffices to show that $R\sim \mathbb{R}$, where $R$ is from the previous theorem. Let $f(x) = \begin{cases} \frac{x(1-x)}{2x-1}, & x \ne \frac{1}{2} \\ 0, & x = \frac{1}{2}\end{cases}$.
            \end{proof}
            \begin{remark}
            There is something called the continuum hypothesis which states that there is no set $S$ such that $|\mathbb{N}| < |S| < |\mathbb{R}|$. If this is accepted, we may write $|\mathbb{R}| = \aleph_1 = 2^{\aleph_0}$. Much like the axiom of choice, this is independent of the rest of set theory. Its acceptance or negation is possible without contradiction. We will not need to worry about this.
            \end{remark}
\end{document}