\section{Relations of Order}
    \begin{definition}
        Given a set $A$, a total order on $A$ is a relation $\leq$ with the
        following properties: For all $a,b,c\in A$,
        \begin{enumerate}
            \item   $a\leq b$ and $b\leq a$ if and only if $a=b$.
                    \hfill[Antisymmetry]
            \item   If $a\leq{b}$ and $b\leq c$, then $a\leq{c}$.
                    \hfill[Transitivity]
            \item Either $a\leq b$, or $b\leq a$, or both.\hfill[Totality]
        \end{enumerate}
    \end{definition}
    If $a\leq b$, we may also write $b\geq a$.
    \begin{definition}
        Given a set $A$, a strict relation of order is a relation $<$ with the
        following properties: For all $a,b,c\in A$,
        \begin{enumerate}
            \item   Precisely one of the following is true: $a<b$, $b<a$, $a=b$.
                    \hfill [Trichotomy]
            \item   If $a<b$ and $b<c$, then $a<c$.
                    \hfill[Transitivity]
        \end{enumerate}
    \end{definition}
    \begin{definition}
        An ordered field is a field $\langle F,+,\cdot \rangle$ with a total
        order $\leq$ with the following properties: For all $a,b,c\in F$,
        \begin{enumerate}
            \item   If $a\leq b$, then $a+c\leq b+c$
            \item   If $0 \leq a$ and $0\leq b$, then $0\leq a\cdot b$
            \item   $0\leq 1$
        \end{enumerate}
    \end{definition}
    If $a\leq b$ and $a\ne b$, we write $a<b$.
    \begin{theorem}
        In a field, $(ab)^{2}=a^{2}b^{2}$.
    \end{theorem}
    \begin{proof}
        For $(ab)^{2}=(ab)(ab)=(a)(b)(a)(b)=(a)(a)(b)(b)=a^{2}b^{2}$.
    \end{proof}
    \begin{theorem}
        In an ordered field, if $0\leq a$ and $0\leq b$, then $0\leq a+b$.
    \end{theorem}
    \begin{proof}
        For as $0\leq a$, $0+b\leq a+b$. But $0+b = b$ and $0\leq b$. From
        transitivity, $0\leq a+b$.
    \end{proof}
    \begin{theorem}
        In an ordered field, if $0\leq x$, then $-x\leq 0$.
    \end{theorem}
    \begin{proof}
        For $0\leq x$, and thus $(-x)=0+(-x)\leq x+(-x)=0$. From transitivity,
        $(-x)\leq 0$.
    \end{proof}
    \begin{theorem}
        In a field, $(-1)^2 = 1$.
    \end{theorem}
    \begin{proof}
        For $(-1)^2 +(-1) = (-1)(-1+1) = (-1)\cdot 0 = 0$. As additive inverses
        are unique, $(-1)^2 = 1$.
    \end{proof}
    \begin{theorem}
        In an ordered field, $0\leq x^2$.
    \end{theorem}
    \begin{proof}
        If $0 \leq x$, we are done. Suppose $x\leq 0$. Then
        $0\leq(-x)=(-1)x$, and thus $0\leq (-1)^2 x^2=x^2$.
    \end{proof}
    \begin{theorem}
        In an ordered field, $a\leq b$ if and only if $0 \leq b-a$
    \end{theorem}
    \begin{proof}
        For suppose $a\leq b$. Then $0=a+(-a)\leq b-a\Rightarrow 0 \leq b-a$.
        If $0\leq b-a$, then $a=0+a \leq (b-a)+a = b\Rightarrow a\leq b$.
    \end{proof}
    \begin{theorem}
        If $a\leq b$, then $-b\leq -a$.
    \end{theorem}
    \begin{proof}
        For then $0 \leq b-a$, and thus $-(b-a)=a-b\leq 0$, and therefore
        $-b\leq-a$.
    \end{proof}
    \begin{theorem}
        In an ordered field, if $a\leq b$ and $c\leq d$, then $a+c \leq b+d$.
    \end{theorem}
    \begin{proof}
        For $0\leq b-a$ and $0\leq d-c$. Thus,
        $0\leq(b-a)+(d-c)=(b+d)-(a+c)$, and therefore $a+c \leq b+d$.
    \end{proof}
    \begin{theorem}
        In an ordered field, if $0\leq a$ and $b\leq 0$, then $ab\leq 0$.
    \end{theorem}
    \begin{proof}
        For as $b\leq 0$, $0\leq -b$, and thus $0\leq -ba$, and therefore
        $-(-ba) = ba \leq 0$.
    \end{proof}
    \begin{theorem}
        If $0< a$, then $0<\frac{1}{a}$.
    \end{theorem}
    \begin{proof}
        For $\frac{1}{a}\ne 0$ as it is invertible, and $0$ is not. But
        $0\leq1=a\cdot \frac{1}{a}$ and $0<a$ and thus $\frac{1}{a} \not <0$.
        Therefore $0<\frac{1}{a}$.
    \end{proof}
    \begin{theorem}
        In an ordered field, if $0<a\leq b$, then
        $0<\frac{1}{b}\leq\frac{1}{a}$.
    \end{theorem}
    \begin{proof}
        As $a\leq b$:
        \begin{equation}
            \frac{1}{b}
            =a\cdot\frac{1}{ba}\leq{b}\cdot\frac{1}{ba}
            =\frac{1}{a}
        \end{equation}
        Thus, $0< \frac{1}{b}\leq \frac{1}{a}$.
    \end{proof}
    \begin{theorem}
        In an ordered field, if $0 \leq a \leq b$, then $a^2 \leq b^2$.
    \end{theorem}
    \begin{proof}
        For as $0\leq{a}\leq{b}$, $a\cdot{a}\leq{b}\cdot{a}$. Thus,
        $a^{2}\leq{b}\cdot{a}$. But also $a\cdot{b}\leq{b}\cdot{b}$. Thus,
        $a\cdot{b}\leq{b}^{2}$. By transitivity, $a^{2}\leq{b}^{2}$.
    \end{proof}
    \begin{theorem}
        If $1\leq a$, then $a \leq a^2$. If $0\leq a \leq 1$, then
        $a^2 \leq a$.
    \end{theorem}
    \begin{proof}
        For as $1\leq a$, $a=1\cdot a \leq a^2$. If $0\leq a \leq 1$, then
        $a^2 \leq 1\cdot a = a$.
    \end{proof}
    \begin{definition}
        A partial ordering on a set $A$ is a relation $\leq$ such that if
        $A\leq{B}$ and $B\leq{C}$, then $A\leq{C}$, and if $A\leq{B}$ and
        $B\leq{A}$, then $A=B$.
    \end{definition}
    Set inclusion is a partial ordering on the power set of a set. The union
    of two sets $A$ and $B$ is the set $A\cup{B}$ containing all of the
    elements of $A$ and all of the elements of $B$. The intersection of $A$
    and $B$ is the set $A\cap{B}$ containing only the elements that are in
    both $A$ and $B$.