%------------------------------------------------------------------------------%
\begingroup
    \ifcsname\PATH\endcsname
        \newcommand{\PATH}{books/Foundations/Logic_and_Sets}
        \newcommand{\OLDPATH}{\PATH}
    \else
        \newcommand{\OLDPATH}{\PATH}
        \renewcommand{\PATH}{books/Foundations/Logic_and_Sets}
    \fi
    \chapter{Propositional Logic}
        \label{chapt:Propositional_Logic}%
        We'll begin our discussion of logic with the most primitive kind:
        propositional logic\index{Logic!Propositional}. This form, also called
        \textit{sentential logic}\index{Logic!Sentential}, deals with the
        structure of sentences. We'll learn how the English language is used to
        formulate arguments and deduce new facts. It is for this reason that one
        should start with the foundations of logic, for at the heart of
        mathematics is the concept of \textit{definition}, \textit{theorem}, and
        \textit{proof}. The word definition, it is hoped, is understood as a
        primitive word in the English language (much like the word \textit{the})
        but theorem and proof need an explaination if we are to use them
        consistently. We'll discuss propositions, predicates, and the concepts
        of implication and negation. Later we will study mathematical logic when
        developing Boolean algebras\index{Boolean Algebra} in
        Book~\ref{book:Foundations} and Stone spaces in
        Book~\ref{book:Topology}, culminating in Stone's representation
        theorem\index{Theorem!Stone's Representation}, connecting logic, set
        theory, topology, and measure theory in a remarkable way.
        We'll begin our discussion of logic with the most primitive kind:
propositional\index{Logic!Propositional}.%
\footnote{%
    Also called \textit{sentential logic}\index{Logic!Sentential}.%
}
This deals with the structure of sentences, how the English language is used to
formulate arguments and deduce new facts. It is for this reason one should start
with the foundations of logic for at the heart of mathematics is the concept of
\textit{definition-theorem-proof}. The word definition, it is hoped, is
understood as a primitive of the English language (much like the word
\textit{the}) but theorem and proof require an explanation if we are to use them
consistently. We'll discuss propositions, predicates, implications, negations,
and overall what we are to consider valid reasoning. We return to logic later
in Books~\ref{book:Foundations} and \ref{book:Topology} when discussing
Boolean algebras\index{Boolean Algebra} and Stone spaces%
\index{Stone Space}\index{Topological Space!Stone Space}, culminating in Stone's
representation theorem.
\section{What is Logic?}
    It may seem strange to begin a study of mathematics with the development of
    logic as one might think such conversations should reside in philosophy.
    Indeed, most of classical logic was developed by philosophers rather than
    mathematicians. Many problems, which we will discuss in
    Chapt.~\ref{chapt:Zermelo_Fraenkel_Set_Theory}, arose in the early 1900s
    with the very core of mathematics. Arguments once considered sound were
    shattered and contradictions were discovered. On the other hand
    other methods of proof that are very intuitive were shown to be able to
    prove the existence of non-intuitive and almost impossible objects. This
    motivates us to develop the \textit{axioms}%
    \footnote{%
        The word axiom will be defined soon enough (Def.~\ref{def:Axiom}).
    }
    of logic and explore what valid arguments should look like.
    \begin{example}
        \label{ex:Logic_IVP}%
        A student of calculus has likely heard of the intermediate value
        theorem\index{Theorem!Intermediate Value}. Fear not those who haven't,
        we shall draw a picture. Given a \textit{continuous}%
        \footnote{%
            Intuitively a \textit{curve} one can draw from left to right without
            lifting up their pencil.%
        }
        function $f$ of real numbers, if $f(0)$ is negative and $f(1)$
        positive,%
        \footnote{%
            $f(0)$ means the evaluation of $f$ at zero. If $f$ has the formula
            $f(x)=x+1$, then $f(0)=1$.
        }
        then there is some point in the middle which evaluates to zero. The
        proof is quite simple: We first look at what happens to the point
        $\frac{1}{2}$. If $f$ is zero here we are done, otherwise if $f$ is
        positive then we may suspect there's a value in
        between 0 and $\frac{1}{2}$ which evaluates to zero and if $f$ is
        negative at this point, then there's probably a zero between
        $\frac{1}{2}$ and 1. In either case we divide the range of possibilities
        in half and see what happens at $\frac{1}{4}$ in the first case and
        $\frac{3}{4}$ in the latter. We continue \textit{inductively} (whatever
        this means) and obtain a \textit{sequence} of real numbers which we then
        show \textit{converges}. Invoking continuity, $f$ then evaluates to
        zero at this limit and we are done (see Fig.~\ref{fig:Sketch_of_IVP}).
    \end{example}
    \begin{figure}[H]
        \centering
        \captionsetup{type=figure}
        \if\compilefigures1
            \includegraphics{images/Intermediate_Value_Theorem_Sketch.pdf}
        \fi
        \caption{Sketch for the Intermediate Value Theorem}
        \label{fig:Sketch_of_IVP}
    \end{figure}
    We can see why this may work. After a few iterations we've narrowed down a
    zero point to a small range between $x_{3}$ and $x_{5}$ and this is a
    nice algorithm we can tell a computer to execute to arbitrary precision%
    \footnote{%
        This is known as the \textit{bisection} method%
        \index{Root Finding!Bisection}\index{Bisection Method} of root finding.%
    }
    but what went into the proof? If we are to phrase this with absolute rigor,
    what definitions, assumptions, and previous theorems are we relying on? For
    starters, the existence of \textit{real numbers}, a notion of
    \textit{continuity}, and the definition of a \textit{sequence}. Our
    exposition of logic is to make clear what is required for valid proofs.
    \begin{example}
        \label{ex:Logic_Gauss_Sum}%
        It has been alleged, though as always one should exhibit skepticism,
        that in 1784 at the age of seven the great mathematician Carl
        Friedrich Gauss\index{Gauss, Carl Friedrich} (1777-1855 C.E.)
        demonstrated that $1+2+\dots+99+100=5050$ \cite[p.~12-13]{von1856gauss}.
        It has also been claimed he had revelations about the normal
        distribution while counting the number of steps on his way to school,
        and that he corrected his fathers mathematical calculations at the age
        of three.%
        \footnote{%
            Many such claims were made by Gauss' biographer
            \textit{Wolfgang Sartorious von Waltershausen}, born 30 years after
            Gauss, publishing these stories in
            \textit{Gauss: zum Ged\"{a}chtnis}. Gauss apparently told some of
            these stories to von Waltershausen at old age in great excitement.
            It would be great to accept the tale as a feel-good story, but
            shortly after J\'{a}nos Bolyai
            (1802-1860 C.E.)\index{Bolyai, J\'{a}nos} published his account of
            absolute geometry in 1831 creating a common foundation for Euclidean
            and hyperbolic geometry, Gauss wrote that he was unable to give
            praise since he had developed these results 30 years prior but
            never published, offering no evidence of the claim. With such
            comments it is hard to tell what is factual about Gauss.
        }
        Alas, mathematicians are a few tales away from forming the religion of
        Gauss. Nevertheless, let's see what the argument is. We rearrange this
        sum as follows:
        \begin{table}[H]
            \centering
            \captionsetup{type=table}
            \begin{tabular}{ccccccccccc}
                &&$1$&$+$&$2$&$+$&$\cdots$&$+$&$99$&$+$&$100$\\
                \hline\\
                $=$&&$1$&$+$&$2$&$+$&$\cdots$&$+$&$49$&$+$&$50$\\
                &$+$&$100$&$+$&$99$&$+$&$\cdots$&$+$&$52$&$+$&$51$\\
                \hline\\
                $=$&&$101$&$+$&$101$&$+$&$\cdots$&$+$&$101$&$+$&$101$\\
                \hline\\
                $=$&&$50$&$\times$&$101$\\
                \hline\\
                $=$&&$5050$
            \end{tabular}
            \caption{Gauss' Sum of 1 to 100}
        \end{table}
        While this seems to be a concrete proof of the claim, is it valid? Can
        we generalize it? What assumptions about integers and arithmetic are we
        making?
    \end{example}
    \begin{example}
        \label{ex:Logic_Euler_Sum}%
        Gauss is often called the prince of Mathematics, and the king is the
        great Leonhard Euler\index{Euler, Leonhard} (1707-1783 C.E.). He too
        studied sums and considered the bizarre series $1+2+3+4+\cdots$ arriving
        at the answer $\minus\frac{1}{12}$.%
        \footnote{%
            While this sum was studied in the $18^{th}$ century, there seems to
            be ambiguity as to whether or not Euler earns the credit on this
            one. He writes this sum and others, such as $1+2+4+8+\cdots$ which
            he claims sums to $\minus{1}$ and $1+3+9+27+\cdots$ arriving at
            $\minus\frac{1}{2}$, in his text \textit{De Seriebus Divergentibus}
            (English: \textit{On Divergent Series}). He also computes Grandi's
            summation using the geometric series
            \cite[p.~206-208]{euler2012seriebus}. He does not sum $1+2+3+\cdots$
            to $\minus\frac{1}{12}$ like we do, only mentions it is probably
            negative.
        }
        Let's see how we can obtain this. First consider Grandi's
        series\index{Grandi's series}, named after Luigi Guido Grandi%
        \index{Grandi, Luigi Guido} (1671-1742 C.E.). We have:
        \begin{equation}
            G=1-1+1-1+1-1+\cdots
        \end{equation}
        Is there a meaningful number to assign to $G$? Suppose there is and
        write:
        \begin{table}[H]
            \centering
            \captionsetup{type=table}
            \begin{tabular}{rrrrrrrrrrrrr}
                $1-G$&$=$&$1$&$-$&$\Big[$$1$&$-$&$1$&$+$&$1$&$-$&$\cdots$
                    &$\Big]$\\[1ex]
                \hline\\
                    &$=$&$1$&$-$&$1$&$+$&$1$&$-$&$1$&$+$&$\cdots$\\[1ex]
                \hline\\
                &$=$&$G$
            \end{tabular}
            \caption{Grandi's Series $1-1+1-1+\cdots$}
        \end{table}
        \begin{minipage}[c]{0.58\textwidth}
            And hence we have $1-G=G$. Adding $G$ to both sides gives us $2G=1$,
            and after dividing by two we obtain $G=\frac{1}{2}$.
            Next, we consider the series:
            \begin{equation}
                T=1-2+3-4+5-6+\cdots
            \end{equation}
            This series was considered by Euler, writing it sums to
            $\frac{1}{4}$, though admits this is paradoxical since the first
            100 terms add to $\minus{50}$ and the first 101 terms sum to $+51$,
            neither of which is close to one-fourth. Nevertheless, we treat this
            series as we did Grandi's and see what can be done:
        \end{minipage}
        \hfill
        \begin{minipage}[c]{0.4\textwidth}
            \centering
            \begin{figure}[H]
                \centering
                \captionsetup{type=figure}
                \if\compilefigures1
                    \includegraphics{photos/GuidoGrandi.jpg}
                \fi
                \caption{Luigi Guido Grandi}
                \label{photo:GuidoGrandi}
            \end{figure}
        \end{minipage}
        \begin{table}[H]
            \centering
            \captionsetup{type=table}
            \begin{tabular}{ccccccccccc}
                $2T$&$=$&$T$&$+$&$T$\\
                    \hline\\
                    &$=$&$1$&$-$&$2$&$+$&$3$&$-$&$4$&$+$&$\cdots$\\
                    &   &   &$+$&$1$&$-$&$2$&$+$&$3$&$-$&$\cdots$\\
                \hline\\
                    &$=$&$1$&$-$&$1$&$+$&$1$&$-$&$1$&$+$&$\cdots$\\
                \hline\\
                &$=$&$G$\\
                \hline\\
                &$=$&$\frac{1}{2}$
            \end{tabular}
            \caption{The Sum of $1-2+3-4+\cdots$}
        \end{table}
        That is, adding $T$ to itself gives us back Grandi's series which we
        know to be $G=\frac{1}{2}$. Hence, $2T=\frac{1}{2}$. Dividing by two
        gives us Euler's solution: $T=\frac{1}{4}$. We now return to Euler's sum
        which we'll denote $S$. We subtract the series $T$ which we computed
        above, obtaining:
        \begin{table}[H]
            \centering
            \captionsetup{type=table}
            \begin{tabular}{ccccccccccccc}
                $S-T$&$=$&&       &$1$&$+$&$2$&$+$&$3$&$+$&$4$&$\cdots$\\
                     &&$-$&$\Big[$&$1$&$-$&$2$&$+$&$3$&$-$&$4$&$\cdots$&$\Big]$
                \\[1ex]
                \hline\\
                     &$=$&&&$0$&$+$&$4$&$+$&$0$&$+$&$8$&$\cdots$\\[1ex]
                \hline\\
                &$=$&&$4\Big[$&$1$&$+$&$2$&$+$&$3$&$+$&$4$&$\cdots$&$\Big]$
                    \\[1ex]
                \hline\\
                &$=$&&$4S$
            \end{tabular}
            \caption{The Sum of $1+2+3+4+\cdots$}
        \end{table}
        So $S-T=4S$. But $T=\frac{1}{4}$ and hence $3S=\minus\frac{1}{4}$.
        Thus $S=\minus\frac{1}{12}$.%
        \footnote{%
            The method of summation presented here is an augmentation of
            Srinivasa Ramanujan's (1887-1920 C.E.)\index{Ramanujan, Srinivasa}
            \cite[Chapt.~VIII p.~3]{RamanujanNotebooksI}.
        }
    \end{example}
    We now contemplate the previous examples and ask which are valid. It is
    tempting to say Ex.~\ref{ex:Logic_IVP} and Ex.~\ref{ex:Logic_Gauss_Sum} were
    presented with accurate proofs, whereas Ex.~\ref{ex:Logic_Euler_Sum} is
    garbage, but why? We ``proved'' Gauss' and Euler's sum in the same manner:
    Rearranged terms, used \textit{dot dot dot} notation to indicate some
    pattern, added things together in a convincing way, and then simplified. One
    might suggest that Gauss' proof involved a finite scheme and that if asked
    one could laboriously write out the numbers indicated by the dots, whereas
    Euler's sum is infinite. But if we were to perform Gauss' problem with a
    number so large that it would take longer than the age of the universe to
    write down, even with the aid of computer, would we reject his method of
    proof then? There is some solace, and we can show that the Euler sum is
    invalid if we accept that $1\ne{0}$. Consider the sum $1+1+1+1+\cdots$ which
    we shall denote $B$ for bad. Using Grandi's series we subtract and obtain:
    \begin{table}[H]
        \centering
        \captionsetup{type=table}
        \begin{tabular}{ccccccccccccc}
            $B-G$&$=$&&       &$1$&$+$&$1$&$+$&$1$&$+$&$1$&$\cdots$\\
                 &&$-$&$\Big[$&$1$&$-$&$1$&$+$&$1$&$-$&$1$&$\cdots$&$\Big]$
            \\[1ex]
            \hline\\
                 &$=$&&&$0$&$+$&$2$&$+$&$0$&$+$&$2$&$\cdots$\\[1ex]
            \hline\\
            &$=$&&$2\Big[$&$1$&$+$&$1$&$+$&$1$&$+$&$1$&$\cdots$&$\Big]$
                \\[1ex]
            \hline\\
            &$=$&&$2B$
        \end{tabular}
        \caption{The Sum of $1+1+1+1+\cdots$}
    \end{table}
    And so we obtain $B-G=2B$, so $B=\minus{G}$. But we know Grandi's series is
    $G=\frac{1}{2}$, and hence $B=\minus\frac{1}{2}$. We can also do the
    following:
    \begin{equation}
        1+B=1+(1+1+1+1+\cdots)=1+1+1+1+\cdots=B
    \end{equation}
    And hence $1+B=B$, so $1=0$ which is a contradiction.
    \subsection{Truth}
        Since the aim of mathematics is to prove the validity of mathematical
        statements, we should start with a definition of truth. We run into a
        wall instantly since this is essentially an impossible task. Any
        definition will be circular, and it is a theorem of
        Alfred Tarski\index{Tarski, Alfred} (1901-1983 C.E.) that if one has
        defined arithmetic, then one cannot use arithmetic to define
        truth.\index{Theorem!Tarski's Undefineability Theorem}%
        \index{Tarski's Undefineability Theorem}%
        \footnote{%
            This is known as Tarski's Undefineability Theorem.%
        }
        That is, if we take upon the assumption of the existence of the
        natural numbers 0, 1, 2, $\dots$ with the familiar notion of addition,%
        \footnote{%
            i.e. $1+1=2$ and other mathematical gems.
        }
        then \textit{arithmetic}%
        \footnote{%
            Pronounced \textit{eh-rith-meh-tik}.
        }
        truth cannot be defined using this arithmetic.%
        \footnote{%
            Pronounced \textit{ah-rith-muh-tik}. Isn't English fun?
        }
        Let us consult the dictionary. The Oxford English dictionary defines
        truth to mean \textit{in accordance with fact or reality}
        \cite{OEDTrueDef}, Merriam-Webster states that truth is
        \textit{the body of real things, events, and facts}
        \cite{MerriamWebsterTruthDef}, and Cambridge claims it is
        \textit{the quality of being true} \cite{CambridgeTruthDef}. As
        hypothesized these definitions are circular and rely on other predefined
        terms. We propose the following work around: Truth\index{Truth} is a
        primitive notion that needs no definition. We can then define
        false\index{False} to mean \textit{not} true.
        \begin{figure}[H]
            \centering
            \captionsetup{type=figure}
            \if\compilefigures1
                \includegraphics[scale=0.4]{photos/AlfredTarski1968.jpeg}
            \fi
            \caption{Alfred Tarski (1968)}
            \label{photo:Alfred_Tarski}
        \end{figure}
        Tarski's result came about in the 1930's when he tried to mathematically
        work out the \textit{liar's paradox}\index{Paradox!Liar's}%
        \index{Liar's Paradox} \cite{TarskiUndefinability}. Consider the
        following sentence:
        \begin{equation}
            \text{This sentence is false.}
        \end{equation}
        Similar statements have been considered throughout the ages, including
        the variant known as Epimenides' paradox\index{Paradox!Epimenides'}.
        Epimenides of Cnossos (\textit{c.} 600 B.C.E.), who was from Crete,
        proclaims \textit{Cretans are always liars} \cite{KingJamesBible}. The
        question was considered again 200 years later when Eubulides of Miletus
        (\textit{c.} 400 B.C.E.) considered the sentence \textit{I am lying.}
        Further still in the Book of Psalms king David says
        \textit{I said in my haste, all men are liars} \cite{KingJamesBible}.
        Needless to say, the paradox is quite old and well studied. Now we ask,
        is the statement \textit{true} or \textit{false} (assuming such notions
        are defined)? Let's work through it and suppose truth. If this sentence
        is false is true, then the sentence is false even though we just claimed
        it to be true. Hence, it must be false. But if this sentence is false is
        false, then the sentence is true, but we just showed it cannot be true.
        So, which one is it? There are two interpretations: The statement is
        \textit{neither} true nor false, and the sentence is \textit{both} true
        and false. Suppose we accept that the statement is neither true nor
        false. This leads to another sentence where we cannot make such a
        conclusion:
        \begin{equation}
            \text{This sentence is not true.}
        \end{equation}
        If this is neither true nor false, then it is not true, and hence true,
        bringing us back to the paradox. Now we claim it is both true and false,
        leading us to:
        \begin{equation}
            \text{The sentence is false and not true.}
        \end{equation}
        The problem intensifies if we consider pairs of sentences:
        \twocolumneq[\par]{%
            \label{eqn:That_Sentence_Is_True}%
            \text{Statement \ref{eqn:That_Statement_Is_False} is true.}
        }
        {%
            \label{eqn:That_Statement_Is_False}%
            \text{Statement \ref{eqn:That_Sentence_Is_True} is false.}
        }
        and now we go round and round in an endless circle. As Alfred Tarski
        pointed out, the problem arises in languages in which statements are
        allowed to be self-referential. To see this is indeed a self
        referencing claim we write $P$ for the proposition and arrive at the
        equation:
        \begin{equation}
            P=P\text{ is false}
        \end{equation}
        if we substitute $P$, we obtain:
        \begin{equation}
            P=(P\text{ is false})\text{ is false}
             =\big((P\text{ is false})\text{ is false}\big)\text{ is false}
        \end{equation}
        While it may seem like this is an unnecessary discussion, the liar's
        paradox plays a role in mathematics. For one it motivates Tarski's
        theorem on the defineability of truth, and more famously it allowed Kurt
        G\"{o}del\index{G\"{o}del, Kurt} (1906-1978 C.E.) to prove his
        \textit{incompleteness theorems}, which really shook most of modern
        mathematics. Indeed, this theorem allegedly made Albert
        Einstein\index{Einstein, Albert} believe there could be no
        \textit{theory of everything}\index{Theory of Everything}.%
        \footnote{%
            A theory of physics that could solve all problems great and small.
            While no direct quotation from Einstein could be found, Stephen
            Hawking wrote that G\"{o}del's theorem convinced him no such theory
            can exist \cite{hawking2002godel}. Einstein was a close friend to
            G\"{o}del, which is probably connected to how this claim originated.
        }
        For the sake of moving on to mathematics we accept truth to be a
        primitive notion and acknowledge that the foundations of this concept
        are very shaky.
        \par\hfill\par
        Let us examine a few more paradoxes of the English language. The main
        troubles of set theory will have to wait until we've developed more
        vocabulary. The first to discuss is
        \textit{Berry's paradox}\index{Paradox!Berry's}\index{Berry's Paradox},
        named after G. G. Berry\index{Berry, G. G.} (1867-1928 C.E.), a junior
        librarian at one of Oxford's library who relayed the paradox to Bertrand
        Russell \cite[p.~63]{CamCompBertRuss03}, who published it in 1906.%
        \footnote{%
            Berry's original paradox dealt with Cantor's theory of
            \textit{ordinal} numbers, not integers.
        }
        The paradox arises from the following sentence:
        \begin{center}
            \textit{The smallest positive integer not defineable in less than}
            \textit{60 letters}
        \end{center}
        This statement is itself only 57 characters long. The English language
        has 26 letters, a space bar, and 10 numerical symbols (in addition to
        grammatical symbols like commas), so let's suppose there are 50 distinct
        characters allowed in a sentence. There are then a total of
        $50^{60}\approx{8}\times{10}^{101}$ combinations. Just about all of
        these sentences are absolute gibberish, for example:
        \begin{center}
            \textit{qjasneofiq923m woasmd fd/'?maojs 3m ansdjf aia sdf iquer sj}
        \end{center}
        One of my more poetic works, called
        \textit{bashing my keyboard and counting to 60}. Some of these
        combinations of characters do indeed correspond to integers. For
        example, \textit{The smallest positive integer} is 1. In almost every
        system of arithmetic studied there exists a \textit{well-ordering}%
        \index{Well Ordering} property of the integers. If you are given a
        collection of positive integers and told it contains some number $n$,
        then there is a \textit{smallest} such. The proof is quite simple, ask
        yourself \textit{is 1 in the collection}? If yes, you are done since 1
        is the smallest positive integer, if not, proceed. Then ask
        \textit{is 2 in the collection}? Again, if yes then you are done since
        2 is the smallest positive integer greater than 1, but you've already
        checked that 1 is not in your collection. You continue until finally you
        hit an integer where the answer is \textit{yes}. You are guarenteed this
        process will stop since by hypothesis the collection has some integer
        $n$, and hence you need at most $n$ iterations of this procedure. This
        \textit{proof by algorithm} is not a proof unless we have some
        \textit{axiom} that such arguments are legal. Nevertheless it is
        intuitive and gives means of proving the claim in some theories (like
        Zermelo-Fraenkel Set Theory) and justifies accepting the principle as
        valid without requiring proof in others (like Peano Arithmetic).
        \par\hfill\par
        We use the well-ordering principle to formulate Berry's paradox. In
        mathematics we describe collections of objects with sentences. This need
        not be with English, but the problem will exist in any human language.
        For example,
        \textit{the collection of all integers which are divisible by two} is a
        description of the \textit{even} integers. So now we propose
        \textit{the set of all integers not defineable in less than 60 letters}.
        Since there are at most $\approx{10}^{102}$ such integers that are
        defineable in less than 60 characters, and since most presume there are
        infinitely many numbers, we conclude the collection of integers defined
        by this sentence is non-empty. Then by the well ordering principle there
        is a least such element. But then this least element satisfies the
        criterion \textit{The smallest positive integer not defineable in less}
        \textit{than 60 letters}, which is less than 60 letters. But we've
        described it in fewer than 60 characters, a contradiction.%
        \footnote{%
            Some number theorists use this paradox to prove all integers are
            interesting. If not, then there is a least such integer that is
            not interesting. But being the smallest boring integer is pretty
            interesting! Hence, all integers are interesting.%
        }
        \footnote{%
            There is a semantical equivalent more familiar to most (though
            almost all are unbothered by it). In the classic Disney movie
            \textit{Aladdin}, whilst singing the song
            \textit{A Whole New World}, princess Jasmine proclaims
            \textit{indescribable feeling}, and yet she just described it.
        }
        \par\hfill\par
        The resolution has already been alluded to. There is an ambiguity with
        the word \textit{defineable}. Does this sentence indeed define an
        integer? It seems the paradox merely gives a proof that it does not. One
        proposed alternative to this solution is to create a hierarchy. Hence
        \textit{The smallest positive integer that is not $defineable_{0}$}
        \textit{in less than 60 letters} is a number that is $defineable_{1}$ in
        less than 60 letters. Like the liar's paradox, Berry's has a role in
        mathematics. In 1989 the American mathematics George
        Boolos\index{Boolos, George}%
        \footnote{Note to be confused with George Boole.}
        used the paradox to prove G\"{o}del's incompleteness theorem in a
        different manner.
        \par\hfill\par
        Next is the \textit{Grelling-Nelson Paradox}%
        \index{Grelling-Nelson Paradox}\index{Paradox!Grelling-Nelson}. This is
        less mathematical than the previous one but has a familiar ring to it.
        It is named after the German logicians Kurt
        Grelling\index{Grelling, Kurt} (1886-1943 C.E.) and Leonard
        Nelson\index{Nelson, Leonard} (1882-1927 C.E.).%
        \footnote{%
            Nelson's great grandfather is the mathematician
            \textit{Johann Peter Gustav Lejeune Dirichlet}.%
            \index{Dirichlet, Peter Gustav Lejeune}
        }
        Label an adjective of the English language as \textit{autological} if
        whatever the adjective is describing also holds for the adjective
        itself. For example, \textit{polysyllabic} describes words with many
        syllables, of which polysyllabic is such a word. Even more creative,
        \textit{pentasyllabic} words have five syllables, which pentasyllabic
        happens to have. The word \textit{word} is autological, as is the word
        \textit{English} (so long as it's read in English). Label an adjective
        \textit{heterological} otherwise. The words \textit{monosyllabic} and
        \textit{long} are heterological. Now we consider the word heterological.
        Since this is an adjective it is valid to ask if it is autological or
        heterological. If we suppose it is heterological, then it describes
        itself and is thus autological, a contradiction. If it is autological
        then it describes itself, but heterological words do not describe
        themselves, a contradiction.
        \par\hfill\par
        The game  being played here is similar to the circularity of the
        liar's paradox. Since this problem is purely semantical, we move on to
        \textit{Curry's paradox}\index{Curry's Paradox}\index{Paradox!Curry's}.%
        \footnote{%
            Also known as L\"{o}b's paradox\index{Paradox!L\"{o}b's}%
            \index{L\"{o}b's Paradox}
        }
        Like the liar's paradox, this problem has mathematical use and is often
        seen as a simplification of the Kleene-Rosser paradox%
        \index{Kleene-Rosser Paradox}\index{Paradox!Kleene-Rosser} which was
        used to prove certain formal systems are inconsistent.%
        \footnote{%
            Much the way Russell's paradox proved na\"{i}ve set theory to
            be inconsistent.
        }
        The problem is named after the American mathematician
        Haskell Curry\index{Haskell Curry}%
        \footnote{%
            Programming enthusiasts should note the language
            \textit{Haskell} is named after Curry.
        }
        (1900-1982 C.E.).%
        \footnote{%
            Curry's paradox shows the inconsistentency of na\"{i}ve set theory,
            the original \textit{lambda calculus}, and Curry's
            \textit{combinatory logic}.
        }
        Let $P$ be any sentence, and let $Q$ be the sentence
        \textit{if this sentence is true, then P is true}. If $Q$ is true, then
        \textit{if this sentence is true, then P is true} is a true statment.
        Hence, $P$ is true. From this \textit{anything} can be proven. Like the
        liar's paradox the problem is in the allowance of self-referencing
        sentences. $Q$ can be written as the statement \textit{if Q, then P},
        which is certainly self-referential.
        \par\hfill\par
        We've quite a lot of cleaning up to do before we can delve into the core
        of mathematics. One might think this is a waste of time, but an
        inconsistent theory is truly horrible. The
        \textit{principle of explosion}\index{Principle of Explosion}%
        \footnote{We will prove the principle later in this chapter.}
        allows one to prove, given an inconsistent set of assumptions, that
        \textit{everything} is provably true and false. This is rather boring
        and something one would hope to avoid. While it will only play a small
        role throughout our investigations, we should discuss Bertrand
        Russell's\index{Russell, Bertrand} \textit{type theory}%
        \index{Type Theory}, invented in collaboration with
        Alfred North Whitehead\index{Whitehead, Alfred North}
        (1861-1947 C.E.).%
        \footnote{%
            Not to be confused with the great topologist J. H. C. Whitehead.
        }
        This was one of the first attempts at resolving these paradoxes. Here
        everything has a \textit{type}, which may be thought of as a
        non-negative integer. When discussing containment, for example
        \textit{the collection A is contained in the collection B}, we only
        give this meaning if $B$ is of type 1 greater than $A$. Operations can
        only be applied to objects of the correct type. This has applications in
        computer science and programming. Programming languages like Python
        allow one to define functions that take in undefined inputs. The program
        will crash if an input is given that the function cannot safely handle.
        For example, consider the following code which is meant to take in a
        real number and add 1 to it:
        \par
        \begin{tcblisting}{
            before=\par\vspace{2ex},
            boxsep=0.5\topsep,
            after=\par\vspace{2ex},
            top=0ex,
            bottom=0ex,
            title=Python Code with Ambiguous Input,
            listing only,
            listing options={
                language=Python,
                basicstyle=\ttfamily,
                keywordstyle=\color{blue}\ttfamily,
                stringstyle=\color{red}\ttfamily,
                commentstyle=\color{green}\ttfamily,
                morecomment={[l][\color{magenta}]{\#}},
                numbers=left,
                framexleftmargin=-16ex,
                numbersep=-16ex,
                xleftmargin=-16ex
            }
        }
            def f(x):
                return x+1
        \end{tcblisting}
        If we enter $x=1$, then type $f(x)$, this will return 2 without error.
        However Python allows strings and if we enter $x=\textit{``Bob''}$ the
        program will crash. This implementation of type checking is called
        \textit{Duck typing}\index{Duck Typing}\index{Python}%
        \footnote{%
            If your input looks like a duck and sounds like a duck, it is
            probably a duck.
        }
        in which the type of the input is checked at run time and if the type
        looks correct, the program will attempt to execute it. Contrast this
        with languages which require types to be declared prior to compilation
        or executation. Writing this in C we have:%
        \index{C Programming Language}%
        \footnote{%
            For simplicity, \textit{double} in C simply means a real number.
        }
        \begin{tcblisting}{
            before=\par\vspace{2ex},
            boxsep=0.5\topsep,
            after=\par\vspace{2ex},
            top=0ex,
            bottom=0ex,
            title=C Code with Declared Input,
            listing only,
            listing options={
                language=C,
                basicstyle=\ttfamily,
                keywordstyle=\color{blue}\ttfamily,
                stringstyle=\color{red}\ttfamily,
                commentstyle=\color{green}\ttfamily,
                morecomment={[l][\color{magenta}]{\#}},
                numbers=left,
                framexleftmargin=-16ex,
                numbersep=-16ex,
                xleftmargin=-16ex
            }
        }
            double f(double x){
                return x+1;
            }
        \end{tcblisting}
        If we try to use this function on anything that is not a real number
        the program will refuse to compile.%
        \footnote{%
            Unless you're using a \textit{really} outdated compiler.
        }
        The entirety of type theory is laid out in the three volume
        treatise \textit{Principia Mathematica}. We will adopt Zermelo-Fraenkel
        set theory\index{Zermelo-Fraenkel Set Theory} in addition to the logical
        axioms of Hilbert\index{Hilbert, David} to formulate mathematics, both
        of which were introduced 20 years after Russell and Whitehead's efforts.
    \subsection{Sets}
        The main objects in mathematics are \textit{sets}. This development came
        about in the 1800's with figures like Georg Cantor\index{Cantor, Georg},
        Augustus De Morgan\index{De Morgan, Augustus}, and Bernard
        Bolzano\index{Bolzano, Bernard} making the first strides in the theory.
        The early history is intuitive, but vague. This led to Russell's
        paradox\index{Paradox!Russell's Paradox}\index{Russell's Paradox} which
        ultimately showed stricter axioms of set theory were needed. We
        currently only need a definition, and we will delve into the axioms in
        Chapt.~\ref{chapt:Zermelo_Fraenkel_Set_Theory}. Let's look to the
        primary source for a definition, posited by Georg Cantor
        (1845-1918 C.E.). He wrote \cite{Cantor1895}:
        \begin{center}
            \textit{A set is a gathering together into a whole of definite}
            \textit{distinct objects of our perception or of our thought, which}
            \textit{are called the elements of the set.}
            \par\hfill
            \textit{Beitr\"{a}ge zur Begr\"{u}ndung der Transfiniten}
            \textit{Mengenlehre}%
            \footnote{%
                English:
                \textit{Contributions in Support of Transfinite Set Theory}.%
            }
            \par\hfill
            \textit{Georg Cantor, 1985 C.E.}
        \end{center}
        Felix Hausdorff\index{Hausdorff, Felix} (1868-1942 C.E.) posits
        \cite[p.~11]{HausdorffSetTheory}:
        \begin{center}
            \textit{A set is formed by the grouping together of single objects}
            \textit{into a whole. A set is a plurality thought of as a unit.}
            \par\hfill
            \textit{Mengenlehre}%
            \footnote{%
                English:
                \textit{Set Theory}.%
            }
            \par\hfill
            \textit{Felix Hausdorff, 1927 C.E.}
        \end{center}
        Both are beautifully phrased, but circular since the terms
        \textit{gathering}, \textit{objects}, and \textit{grouping} are not
        defined. This form of circularity was addressed by Alfred Tarski in his
        1946 book \textit{Introduction to Logic and the Methodology of the}
        \textit{Deductive Science}. He expresses the need for primitive
        undefined notions that we take for granted and use freely. We collect
        the smallest number of primitives possible, motivated by intuition, and
        then define other terms in our theory by means of sentences involving
        these primitives and previously defined terms.
        \par\hfill\par
        I do not wish to imply the circularity of the foundations of
        mathematics was created with the advent of set theory. In what is
        perhaps the most important textbook ever written,
        \textit{The Elements}\index{Elements, The (Euclid)} by Euclid of
        Alexandria (\textit{c.} 300 B.C.E), we find the first known work
        that employs the \textit{axiomatic method}\index{Axiomatic Method}. It
        starts with definitions, \textit{postulates}, and
        \textit{common notions}, and proceeds to prove a plethora of important
        theorems in a logical manner deriving results from these primitives and
        previously proved theorems. In modern language postulates and common
        notions are known as \textit{axioms}, which are statements that we
        accept as true without evidence or proof. It is not without flaw since
        his primitive definitions are circular. For example, the
        first definition is of a point:
        \begin{center}
            \textit{A point is that which has no part.}
            \par
            \hfill\textit{The Elements}\par
            \hfill\textit{Euclid of Alexandria, c. 300 B.C.E.}
        \end{center}
        The word part is never defined. Similarly, a line is defined as
        \textit{breadthless length}. This is not to detract from his efforts but
        to show the problem of \textit{infinite regress}\index{Infinite Regress}
        is unavoidable unless we assert that certain terms need no definition.
        For us, the word set will have no real definition. Nevertheless, we
        write the following:
        \begin{fdefinition}{Set}{Set}
            A \gls{set} is a collection of objects called the elements of the
            set.\index{Set}\index{Set!Definition}
        \end{fdefinition}
        The circularity we pointed out in Cantor's definition arises here since
        neither \textit{collection} nor \textit{object} have been defined. To
        begin doing mathematics we need a \textit{thing}. Sets act as our thing.
        We know they exist%
        \footnote{%
            In the sense of Plato's realism.%
            \index{Plato}\index{Platonic Realism}
            Whether sets exist in any real sense is another question.
        }
        but cannot define them very well. Instead we describe how they behave
        and how to obtain new sets from pre-existing ones via \textit{axioms}.
        Before doing so we should first get familiar with the notation. We
        cannot build set theory just yet since we've yet to develop logic. In
        the most elementary systems such as Peano arithmetic there is a notion
        of set, and hence we need to define this first. Pedagogically it is poor
        to proceed without examples, so we provide some now.
        \begin{fnotation}{Element Notation}{Element_Notation}
            If $A$ is a \gls{set} and if $x$ is an element\index{Set!Element of}
            of $A$, then we denote this by writing
            \glslink{containmentsymb}{$x\in{A}$}. If $x$ is not an element
            of $A$, we write $x\notin{A}$.\index{Containment $\in$}%
            \index{Element}\index{Element!Notation}
        \end{fnotation}
        \begin{example}
            The first three letters of the Latin alphabet can be expressed in
            set notation as follows. If we let the symbol $A$ denote this set, we
            may write:
            \begin{equation}
                A=\{\,a,\,b,\,c\,\}
            \end{equation}
            This is the standard for smaller sets; separate the elements by
            commas and enclose with braces. Letting $B$ denote the first three
            positive integers, we get:
            \begin{equation}
                B=\{\,1,\,2,\,3\,\}
            \end{equation}
            Using element notation (Not.~\ref{not:Element_Notation}) we have
            $1\in{B}$, but $4\notin{B}$. That is, $B$ contains the number 1 but
            does not contain the number 4. Similarly, $a\in{A}$ and
            $d\notin{A}$. The symbol $\in$ reads \textit{is in}, or
            \textit{is an element of}, or \textit{is contained inside of}. Thus
            $a\in{A}$ reads $a$ is an element of $A$, or simply $a$ is in $A$.
            The notation $\notin$ is the negation of this: \textit{not in} or
            \textit{not an element of}. Hence $4\notin{B}$ reads
            4\textit{ is not an element of B}, or 4\textit{ is not in B}.
        \end{example}
        \begin{example}
            For the working mathematician sets are allowed to contain almost
            anything they like. For example, we could consider the set of all
            cities of Earth and label this $C$. Then Boston is an element of
            $C$, but Massachusetts is not since Boston is a city, but
            Massachusetts is not (it's a state). Similarly, London would be an
            element of $C$, but England would not be.
        \end{example}
        As we've seen, it is common to denote small sets be enumerating all of
        the elements, separating with commas, and then enclosing in braces.
        Rigorously we need to justify such notation with axioms, such as the
        \textit{axiom of pairing}, which asserts the existence of sets like this
        but for now it's safe to just present this as what finite sets are. It
        is important to note the distinction between $A$ and $\{A\}$ since these
        are not the same thing%
        \footnote{%
            Provably so using the axiom of a regularity, so this claim is by no
            means obvious.
        }
        much the way there is a difference between a human and the house that
        human lives in.
        \begin{example}
            \label{ex:Everything_is_a_Set}%
            In the set theory that we will be working with, Zermelo-Fraenkel set
            theory\index{Zermelo-Fraenkel Set Theory}, abbreviated
            \gls{ZFC}, \textit{everything} is a set. This will be explained
            later, but we quite literally mean everything. The integers will be
            defined via John von Neumann's\index{von Neumann, John}
            construction. We start with the empty set\index{Empty Set}
            $\emptyset$ which is the set that contains nothing, often denoted
            $\emptyset=\{\,\}$, and this will be our zero. We proceed and define
            $1=\{\emptyset\}$, $2=\{\emptyset,\{\emptyset\}\}$,
            $3=\{\emptyset,\{\emptyset\},\{\emptyset,\{\emptyset\}\}\}$, and so
            on. This can be written more attractively as $0=\{\}$, $1=\{0\}$,
            $2=\{0,1\}$, and $3=\{0,1,2\}$, and hence $n$ is just a set
            containing $n$ things using our usual intuitive notion of counting.
            Moreover \textit{functions} are defined as sets, as are
            \textit{ordered pairs} $(a,b)$, and even \textit{orderings}. The
            examples presented here, such as sets of letters, sets of cities,
            etc., are for the sake of intuition and familiarization. No claim is
            made that these are actual sets in a mathematically rigorous sense.
            In ZFC everything is a set, the elements of a set are sets
            themselves, as are the elements of those elements, and so on,
            creating massive towers of sets. The axioms of ZFC are laid out to
            ensure these towers don't collapse on themselves with
            contradictions.
        \end{example}
        While in \gls{ZFC} we do adopt the idea that everything is a set, there
        seems to be universal agreement about notation for sets. Capital Latin
        letters like $A,B,C,X,Y,Z$ denote sets that we actually want to think of
        as collections of things, whereas lowercase Latin letters like
        $a,b,c,x,y,z$ denote elements of these sets which we think of as
        ordinary mathematical objects like numbers, or functions, or whatever.
        Because of this it often causes one discomfort where in the proof of
        something we need to write $x\in{y}$. This is especially true in the
        development of arithmetic. The benefit of the construction described
        above is we now have a simple means of \textit{ordering} the integers
        and we can write $m<n$ (read as \textit{m is less than n}) if it is true
        that $m\in{n}$. But this leads to the bizarre notation $1\in{2}$ or
        $0\in{3}$, which may cause some anxiety.
        \begin{example}
            The collection of all non-negative integers $0,1,2,\dots$ constitute
            a set which is often denoted $\mathbb{N}$. If we include the
            negatives we also get a set, labelled $\mathbb{Z}$. The rational
            numbers form a set, as do the real numbers, and these are written
            $\mathbb{Q}$ and $\mathbb{R}$, respectively. Lastly, the complex
            numbers also create a set $\mathbb{C}$. It is not obvious how to
            make these familiar things into sets without circularity, something
            we wish to avoid in foundations. Such constructions are another aim
            of Book~\ref{book:Foundations}.
        \end{example}
        Elaborating on the discussion in Ex.~\ref{ex:Everything_is_a_Set}, there
        are other theories for the foundations of mathematics that allow for
        primitive notions such as classes and universes. Some of these theories
        are extremely weak (cannot prove much) but very safe (there is likely no
        contradiction), whereas some are very user friendly but almost certainly
        fallacious. Peano's axioms are an example of a weak system of which the
        axioms are so basic and obvious that no one is likely to ever find a
        contradiction, but they cannot assert the existence of negative
        integers, let alone the reals. On the other hand, any theory that
        allows one to say the \textit{collection} of all sets whether the
        collection is a class, or a universe, or whatever, is one that should be
        treated with skepticism. The theory of Zermelo and Fraenkel is a healthy
        middle ground. Strong enough to do most mathematics, and no
        contradiction found yet, though much of the $20^{th}$ century was spent
        searching to no avail.
        \subsubsection{A Brief History}
            Considering collections of things and naming them accordingly dates
            back to antiquity. Aggragates of points are defined in Euclid's
            elements and large assemblages have been used in mathematics since.
            The term \textit{set} seems to appear first in the works of
            Bernard Placidus Johann Nepomuk Bolzano (1781-1848 C.E.), a Bohemian
            who coined the German term \textit{Menge}, which translates to set
            in English\index{Bolzano, Bernard}. The phrase appears in his
            \textit{Paradoxien des Unendiichen}%
            \footnote{%
                English: \textit{The Paradoxes of the Infinite}.
            }
            which was posthumously published in 1851. At the time Bolzano was
            known mostly for his philosophical works, including his 1837
            \textit{Wissenschaftslehre}%
            \footnote{%
                English: \textit{Theory of Science}.
            }
            which attempts to provide a logical foundation to the natural
            sciences. It is a shame, then, that he did not enjoy the fame as a
            mathematician accredited to him today since his groundbreaking works
            in real analysis were not published until the 1880's when the
            Austrian mathematician Otto Stolz\index{Stolz, Otto}
            (1842-1905 C.E.) happened across Bolzano's journals.
            \par\hfill\par
            Alluded to in Bolzano's text is \textit{Galileo's Paradox}%
            \index{Galileo's Paradox}\index{Paradox!Galileo's} put forward by
            the famed Italian astronomer Galileo Galilei\index{Galilei, Galileo}
            (1564-1642 C.E.)%
            \footnote{%
                In one of the \textit{faux} coincidences of history, Galileo
                Galilei died the same year Sir Isaac Newton was born
                (1642-1727 C.E.). One must use the \textit{Julian} calendar for
                Newton and the \textit{Gregorian} for Galileo. Using the
                Julian for Galileo puts his death in 1641, and Newton was born
                in 1643 using the Gregorian, somewhat ruining this
                \textit{coincidence}. However, $17^{th}$ century England used
                the Julian, and $17^{th}$ century Italy used the Gregorian, so
                the claim is not entirely bogus.
            }
            in his work \textit{Discorsi e Dimostrazioni}
            \textit{Matematiche Intorno a Due Nuove Scienze},%
            \footnote{%
                English: \textit{Discourses and Mathematical Demonstrations}
                \textit{Relating to Two New Sciences}.
            }
            usually referred to as \textit{two new sciences}. The paradox is
            formally resolved in set theory, most notably in the works of Georg
            Cantor in the late 1800's. Quite impressive, then, that Galileo
            pondered such problems two centuries ahead of his time.
            \begin{figure}[H]
                \centering
                \captionsetup{type=figure}
                \if\compilefigures1
                    \includegraphics{images/Galileo_Paradox_001.pdf}
                \fi
                \caption{Lines for Galileo's Paradox}
                \label{fig:Galileo_Paradox_001}
            \end{figure}
            Consider the two lines drawn in Fig.~\ref{fig:Galileo_Paradox_001}.
            One is longer than the other but both contain infinitely many
            points. Since one is greater we are forced to conclude, as Galileo
            wrote, that we have something greater than infinity since the
            infinity of points in the longer is greater than the infinity of
            points in the shorter \cite{GalileoTwoNewSciences}. The conclusion
            is somewhat correct, there are different infinities in set theory.
            The infinity of the real numbers is strictly greater than the
            infinity of the natural numbers, for example. This will be made very
            clear when we discuss bijective functions and cardinalities, but for
            now it may seem as mathematical gibberish.%
            \footnote{%
                These conclusions are due to Cantor. Some, like Hilbert,
                defended his work while many, like Kronecker and Wittgenstein,
                lambasted his writings and his character. Some solace: Most of
                Cantor's works were accepted as true by the mid $20^{th}$
                century and his results have become standard in $21^{st}$
                century analysis, topology, and measure theory courses.
            }
            Galileo is wrong in claiming one line has more points than the other
            since both have the same infinity of points. That is, we can match
            up every point of the long line to a unique point in the short line,
            showing the longer one can't be greater in quantity.
            \begin{figure}[H]
                \centering
                \captionsetup{type=figure}
                \if\compilefigures1
                    \includegraphics{images/Galileo_Paradox_002.pdf}
                \fi
                \caption{Solution to Galileo's Paradox}
                \label{fig:Galileo_Paradox_002}
            \end{figure}
            To construct this one-to-one correspondence requires a bit of
            perspective. Suppose we place an observer at the point $O$ shown in
            Fig.~\ref{fig:Galileo_Paradox_002}. The observer would not see the
            longer line, but if he or she were able to then both would appear to
            be the same length. To get our one-to-one correspondence we draw a
            straight line from this observer to the first line and then continue
            outwards to the longer one. This takes a point in the short line
            uniquely to a point in the long one, and every point in the long
            line is hit by some point in the short one. So we've matched every
            point in the short line to every point in the long line and hence
            the claim cannot be made that one has \textit{more} elements than
            the other. What sets them apart (their \textit{length}) is not a
            set-theoretic concept, but rather a \textit{measure} theoretic one.%
            \footnote{%
                In a late night musing I stumbled across this same problem. I
                learned of the solution from my mathematical mentor James
                \textit{Kiwi} Graham-Eagle who presented me with
                Fig.~\ref{fig:Galileo_Paradox_002}.
            }
            \par\hfill\par
            Galileo's second paradox is presented in a similar manner. He writes
            that there are infinitely many integers, and infinitely many square
            integers. Every square integer is also an integer, for example 1, 4,
            9, 16, 25, and so on. The converse is false, not every integer is a
            square integer since 2, 3, 5, 6, 7, 8, 10, and so on are not
            squares. From this argument Galileo concluded there are
            \textit{more} integers than square integers. The following
            observation then troubled him. For every integer%
            \footnote{%
                Here we loosely use integer to mean non-negative natural number:
                0, 1, 2, 3, etc.
            }
            there is a unique square number, namely given $n$ there is $n^{2}$.
            Given this there cannot be more integers than squares, a
            contradiction.
            \par\hfill\par
            The amazing part about this paradox is that it already contains its
            own solution. The \textit{function} which sends $n$ to $n^{2}$ is a
            \textit{bijection} between the integers and the squares. In set
            theory if a bijection exists, then the two sets have the same
            size. The problem can be reduced significantly if we note that not
            every integer is even, but every even integer is an integer. One
            might wrongly conclude then that there are more integers then even
            integers. But for every integer there is precisely one even integer,
            namely given $n$ we have $2n$. Even easier, all positive integers
            are integers, but not all integers are positive since zero is not
            positive. We might then conclude that the set of positive integers
            has infinity points, whereas the set of all integers has infinity
            + 1 points. But for every integer there is precisely one positive
            integer, namely given $n$ we have $n+1$.
            \par\hfill\par
            All three examples have the same problem. We take an infinite set,
            remove points%
            \footnote{%
                Perhaps infinitely many, as Galileo did.
            }
            but arrive at a set that is the same \textit{size} as what we
            started with. As stated, this problem is handled in set theory. By
            definition two sets are of the same size if a one-to-one
            correspondence (like $n\mapsto{n}^{2}$ or $n\mapsto{2n}$ or
            $n\mapsto{n}+1$)%
            \footnote{%
                The symbol $\mapsto$ reads aloud as \textit{maps to}.
            }
            exists between them. What's even more troubling%
            \footnote{%
                Unbeknownst to Galileo but knowst to Cantor.
            }
            is that the set of all \textit{rational} numbers is the same size as
            all of these. We'll explore this rigorously in
            Chapt.~\ref{chapt:Cardinality}.
    \subsection{Predicates and Propositions}
        Propositions and predicates will be two more of our primitive notions
        which we will vaguely define, but mostly rely on intuition.
        \begin{fdefinition}{Predicate}{Predicate}
            A \gls{predicate} $P$ on a \gls{set} $A$ is a sentence such that for
            all $x\in{A}$ one may state that $P(x)$ is either true or false.%
            \index{Predicate}\index{Predicate!Definition}
        \end{fdefinition}
        We have technically left the realm of propositional logical and entered
        \textit{predicate logic}\index{Predicate Logic}\index{Logic!Predicate}.
        This is because we needed to \textit{quantify} over all of the elements
        of the set $A$. That is, we've used the phrase \textit{for all}, a type
        of quantifier that is dealt with in predicate logic, not propositional.
        In strict propositional logic we would define \textit{propositional}
        \textit{variables}\index{Propositional Variable}, which is a primitive
        taken to be a variable which can either be true or false. These are also
        called \textit{atomic formulae}\index{Atomic Formula}, and from some set
        $A$ of atomic formulae one obtains a language by combining these
        primitive variables with logical \textit{connectives} such as $\neg$,
        $\land$, $\lor$, $\Rightarrow$, and $\Leftrightarrow$, each of which
        we'll discuss in the next section. So if $A$ is our set of propositional
        variables, and if $p,q\in{A}$, then $p\land{q}$ is something valled a
        \textit{well-formed formula}. A language is then obtained by considering
        the set of all well-formed formulae that are obtained using the
        following three rules:
        \begin{itemize}
            \item[1.)] If $p\in{A}$ is an atomic formula, then it is a
                       well-formed formula.
            \item[2.)] If $p\in{A}$ is an atomic formula, then $\neg{p}$ is a
                       well-formed formula. 
            \item[3.)] If $p,q$ are well-formed formulae, and if $b$ is a
                       logical connective, then $pbq$ is a well-formed formula.  
        \end{itemize}
        We will not completely abandon propositional logic but since we will,
        as all mathematicians must, accept predicate logic into our vocabulary
        there is no harm in introducing Def.~\ref{def:Predicate} now. It is,
        after all, a primitive definition. For pure propositional logic one may
        consider as the set of propositional variables the sentences
        $P(x)$ for all $x\in{A}$. These are then just sentences which one may
        ask \textit{is it true or false}, which is precisely what a
        propositional variable is.
        \par\hfill\par
        Predicates are the main tool used in set theory for defining and
        building new sets via the \textit{axiom schema of specification}%
        \index{Axiom!Schema of Specification}. Many describe predicates as
        \textit{functions} from a set $A$ to the Boolean-valued set
        $\{\text{True},\,\text{False}\}$, which is fine provided the notion
        function has been defined. To do so now would requires us to make
        \textit{function} another primitive, and thus one has the following
        problem: Do we accept \textit{predicate} as a primitive, or
        \textit{function}? Since predicates are essentially just sentences,
        we'll adopt these as primitives and later \textit{define} functions
        using elementary notions from the axioms of set theory.
        \par\hfill\par
        To see if our definition matches other standards, let's consult a
        dictionary and a textbook. Merriam-Webster writes that a predicate is
        \textit{something that is affirmed or denied of the subject in a}
        \textit{proposition in logic} \cite{MerriamWebsterPredicateDef}. This is
        a linguistic definition for the use of word predicate in English,
        whereas mathematicians often use overly convoluted formulas to define
        things. Looking to Cunningham's \textit{A Logical Introduction to Proof}
        we have that \textit{A predicate is just a statement proclaiming that}
        \textit{certain variables satisfy a property}
        \cite{Cunningham2010}. He then uses the example \textit{x is tall} where
        $x$ is taken to be a variable. Thus our adopted definition is not at all
        different from the linguistical or mathematical ones, and it truly is a
        primitive notion.
        \begin{example}
            Let $A$ be the set $A=\{1,2,3\}$ and let $P$ be the predicate
            \textit{x is greater than 2}. To be precise, $P$ is a predicate on
            the set of all positive integers 1, 2, 3, 4, etc. The set of values
            in $A$ which satisfy $P$ is $B=\{3\}$. If we let $Q$ represent
            \textit{x is negative}, then there are no elements in $A$ that
            satify $Q$ and hence the resulting set is the empty set $\emptyset$.
        \end{example}
        \begin{example}
            Let $X$ be any set and let $P$ be the sentence $x$ is not equal to
            $x$. There are no elements which satisfy this criterion and hence
            the sentence is always false for any input $x$. In other words, the
            set of elements which satisfies $P$ is the empty set $\emptyset$.
            A sentence that is always false is called a
            \textit{contradiction}.\index{Contradiction}
        \end{example}
        This is the primary means of forming sets in mathematics. We take a set
        we have proven exists and then apply some sentence to it and collect all
        elements of the set satisfying our statement. To do this requires
        justification, and this is the \textit{axiom schema of specification}.
        For example, if we have $A=\{1,2,3\}$ and $P$ is the predicate on $A$
        claiming \textit{x is greater than 2}, then we form a new set $B$ by
        collecting the elements of $A$ which satisfy $P$. We denote this using
        the so-called \textit{set-builder notation}\index{Set-Builder Notation}%
        \index{Set!Set-Builder Notation}:
        \begin{equation}
            B=\{\,x\in{A}\;|\;P(x)\,\}
        \end{equation}
        In more relaxed settings one uses the \textit{axiom of unrestricted}
        \textit{comprehension}\index{Axiom!of Unrestricted Comprehension} which
        allows one to form a set by any sentence. That is, we need not first
        consider some set $A$ and then apply our sentence $P$ to the elements of
        $A$ and extract the elements $x\in{A}$ which satisfy $P$, instead we
        ask for the \textit{set of all things satisfying P}. That is, we write:
        \begin{equation}
            B=\{\,x\;|\;P(x)\,\}
        \end{equation}
        Unfortunately this leads to contradiction and must be avoided.
        Nevertheless it is common in textbooks on measure theory or analysis,
        where convoluted sets need to be constructed, to appeal to unrestricted
        comprehension regardless of the consequences. It is fortunate enough
        that almost all of these constructions can be reformulated in a manner
        consistent with \gls{ZFC}.
        \par\hfill\par
        Moving on to propositions, we first express the Aristotelian
        definition put forward by Aristotle\index{Aristotle} (384-322 B.C.E.).
        \begin{equation}
            \text{A proposition is a sentence which affirms or denies a }
            \text{predicate.}
        \end{equation}
        Examples of propositions are found in the so-called
        \textit{Socrates syllogism}\index{Socrates!Syllogism of}.
        \begin{example}
            We wish to conclude the ancient greek philosopher
            Socrates\index{Socrates} was mortal. We start with the following
            proposition: \textit{All men are mortal}. We assert this sentence is
            true and may be used later in the proofs of other claims. Second,
            we state \textit{Socrates is a man}. This is another proposition
            that we accept as true. To conclude Socrates is mortal we need some
            \textit{rule of inference} that allows us to tie these two
            propositions together. One such rule, known as
            \textit{modus ponens}\index{Modus Ponens}%
            \index{Axiom!of Modus Ponens}, does what we need. It states
            that if $P$ and $Q$ are propositions, if $P$ implies $Q$, and if $P$
            is true, then $Q$ is true. Using this, since all men are mortal, and
            since Socrates is a man, we conclude that Socrates is mortal.
        \end{example}
        An easier proof that Socrates is mortal goes as follows: He's dead.
        Nevertheless, we are starting to see what is needed to construct valid
        proofs. We formalize the Aristotelian view slightly, adopting the
        following definition:%
        \footnote{%
            Though recognizing that this is again another primitive.
        }
        \begin{fdefinition}{Proposition}{Proposition}
            A \gls{proposition} is a \gls{predicate} on a \gls{set} $A$
            evaluated at a particular element $x\in{A}$, which is then affirmed
            or denied to be true.%
            \index{Proposition}\index{Proposition!Definition}
        \end{fdefinition}
        What if we have a proposition that takes in several variables? Moreover,
        how do we respect the order of the input when dealing with more than one
        variable? From how we've defined predicates and propositions the input
        is a single element of the set $A$. To speak of \textit{multivariable}
        predicates and propositions requires order, which sets do not have.
        Indeed, the only distinguishing feature of a set is the elements it
        contains and hence $\{1,2,3\}$ and $\{3,2,1\}$ are identical. Ordering
        sets is a big part of elementary set theory and is dealt with by
        constructing the \textit{Cartesian Product}\index{Cartesian Product} of
        two sets. Since we do not have such notions at the moment our predicates
        must act on single variables.
        \begin{example}
            Let's use the previous example of Socrates to motivate what we mean.
            Let $P$ be the predicate \textit{x is a man}.%
            \footnote{%
                To be precise, it is a predicate on the set of all humans who
                have ever lived.
            }
            If we input Socrates we obtain $P(\text{Socrates})=\text{True}$. If
            we input Hatshepsut, we get $P(\text{Hatshepsut})=\text{False}$.
            This is how we distinguish a predicate from a proposition. $P$ is
            a predicate, whereas $P(\text{Hatshepsut})=\text{False}$ is
            proposition.
        \end{example}
        \begin{example}
            Suppose we have invented the notion of ordered pairs somehow, and
            we denote this by $(x,y)$ using parentheses. So $(x,y)$ and $(y,x)$
            are distinct things whenever $x$ and $y$ are different. With this,
            let us consider the predicate \textit{x is a man, y is a woman} and
            let us label this $P\big((x,y)\big)$.%
            \footnote{%
                As pointed out in Halmos' text, while $P\big((x,y)\big)$ is the
                \textit{correct} notation, $P$ accepts ordered pairs as input,
                everyone just writes $P(x,y)$ \cite[p.~32]{Halmos1974}.
            }
            \footnote{%
                To be precise, $P$ is a predicate on the set of all
                \textit{ordered pairs} of humans.
            }
            The \textit{order} of the inputs now
            matters. For example,
            $P(\text{Socrates},\text{Hatshepsut})=\text{True}$, but
            $P(\text{Hatshepsut},\text{Socrates})=\text{False}$.
        \end{example}
    \subsection{Rules of Inference}
        Given a collection of propositions we often wish to derive new ones.
        Indeed, that is the entirety of mathematics: Proving new theorems. We
        must precisely state which rules we accept as valid and then attempt to
        stay consistent with them. The first we have already discussed,
        \textit{modus ponens}. This rule applies to \textit{implications}, one
        of the two primitives we adopt in our lanuage of deducing new claims,
        the second being negation which we will discuss soon enough.
        \begin{fdefinition}{Implication}{Implication}
            An \gls{implication} on a \gls{predicate} $Q$ on \gls{set} $A$ by a
            predicate $P$ on $A$ is the sentence that if $P$, then $Q$. We
            denote this by $P\Rightarrow{Q}$.
            \index{Implication}\index{Implication!Definition}
        \end{fdefinition}
        $P$ is called the \textit{hypothesis}\index{Hypothesis}, and $Q$ is the
        \textit{conclusion}\index{Conclusion}. $P$ is also called the
        \textit{antecedent}\index{Antecedent} and $Q$ the consequent%
        \index{Consequent}, and moreover $P$ is also called the
        \textit{premise}\index{Premise}
        \begin{example}
            \label{ex:Implication_Mammals}%
            We've no way of proving if $P\Rightarrow{Q}$ is true for any two
            predicates since we've no axioms to draw from yet. Nevertheless let
            us use intuition to see what this \textit{should} mean. Let $P$ be
            the predicate \textit{x is a bear}, where $P$ acts on the set
            of all animals, and let $Q$ be the predicate \textit{x is a mammal}
            (again acting on the set of all animals). From biology we know that
            if $x$ is a bear, then $x$ is a mammal. We say $P$ implies $Q$ and
            write $P\Rightarrow{Q}$.
        \end{example}
        We describe implication via \textit{truth tables}\index{Truth Table}.
        Truth tables for a finite collection of propositions exhaust all
        possible combinations of True and False, and then apply these
        combinations to the logical question at hand. Implication is depicted in
        Tab.~\ref{tab:Truth_Table_Implication}. Given $n$ propositions there are
        $2^{n}$ possible scenarios and so these truth tables get very big very
        fast. We can see that there are $2^{n}$ possibilities since there are
        $n$ choices to be made from \textit{True} and \textit{False}, so there
        are $2\cdot{2}\cdots{2}$ possibilities with $n$ 2's.%
        \footnote{Of course, we've yet to define what $2^{n}$ means.}
        \par\hfill\par
        It is often easier and more useful to write these tables using zeros and
        ones. For one this hints at a means for computers to be able to
        understand and manipulate logical statements and proofs, but it is also
        less cumbersome and allows us to systematically run through the
        possibilities. That is, we start with all propositions set to 0 and then
        flick the right-most one to 1, then the second right-most, and so on.
        The symbol 0 denotes false, and 1 represents truth. The alternative
        truth table for implication is shown in
        Tab.~\ref{tab:Alternate_Truth_Table_Implication}.
        \par
        \begin{minipage}[b]{0.49\textwidth}
            \centering
            \begin{table}[H]
                \centering
                \captionsetup{type=table}
                \begin{tabular}{l|l|l}
                    \multicolumn{1}{c|}{$P$}&\multicolumn{1}{c|}{$P$}&
                    \multicolumn{1}{c}{$P\Rightarrow{Q}$}\\
                    \hline
                    False&False&True\\
                    False&True&True\\
                    True&False&False\\
                    True&True&True
                \end{tabular}
                \caption{Truth Table for Implication}
                \label{tab:Truth_Table_Implication}
            \end{table}
        \end{minipage}\hfill
        \begin{minipage}[b]{0.49\textwidth}
            \centering
            \begin{table}[H]
                \centering
                \captionsetup{type=table}
                \begin{tabular}{c|c|c}
                    $P$&$Q$&$P\Rightarrow{Q}$\\
                    \hline
                    0&0&1\\
                    0&1&1\\
                    1&0&0\\
                    1&1&1
                \end{tabular}
                \caption{Alternate Table for Implication}
                \label{tab:Alternate_Truth_Table_Implication}
            \end{table}
        \end{minipage}
        \par\vspace{2.5ex}
        This shows that the only possible way for $P\Rightarrow{Q}$ to be false
        is if $P$ is true, yet $Q$ is false. This differs from everyday language
        and there is a common tendency to confuse the order of implication.
        Let's spell this out by looking at examples.
        \begin{example}
            \label{ex:Mathematical_Implication}%
            Let $P$ be the proposition \textit{I am late for work} and let $Q$
            denote \textit{I will be fired}, and let's exhaust the four
            possible scenarios of $P\Rightarrow{Q}$. That is, we consider the
            claim \textit{if I am late for work, then I will be fired}. Suppose
            I was not late for work, and I was not fired. Is $P\Rightarrow{Q}$
            true? Well, the criterion for $P$ was not satisfied and hence the
            statement is \textit{not} false, and so we claim it is true. Next, I
            was not late for work yet I was still fired (harsh). Again, the
            criterion for $P$ was not satisfied and so the claim is not false,
            and hence we accept it is true. Third, I was late for work and I was
            not fired (nice boss). Here we see that $P\Rightarrow{Q}$ is
            \textit{false}. The criterion for $P$ was satisfied, yet $Q$ was not
            and therefore $P\Rightarrow{Q}$ is a false statement. Lastly, I was
            late for work and I was fired. This is perhaps the easiest one to
            handle since it is verbatim what one thinks of when they hear
            \textit{if, then} claims. Here, $P\Rightarrow{Q}$ is true.
        \end{example}
        \begin{example}
            \label{ex:False_Antecedent}%
            Given any two propositions $P$ and $Q$ (i.e. predicates evaluated at
            certain elements of our set) where the hypothesis, also called the
            antecedent, is false, then $P\Rightarrow{Q}$ is true regardless of
            $Q$. Let $A$ be the set of all integers and cities of Earth, and let
            $P(x)$ the predicate \textit{x is odd}, and $Q(x)$ be set to
            \textit{x is the capital of France}. Suppose we input 2 into $P$ and
            \textit{London} for $Q$, so we have in plain English the statement
            \textit{if 2 is odd, then London is the capital of France}. This is
            a true statement according to \ref{tab:Truth_Table_Implication}
            since the hypothesis $P$ is false. This is at odds with use in
            natural language in which one thinks $P$ implies $Q$ should be some
            kind of causational argument.
        \end{example}
        \begin{example}
            \label{ex:Vacuous_Truth}%
            The argument shown in Ex.~\ref{ex:False_Antecedent} is very common
            in mathematics and is called a \textit{vacuous} truth.%
            \index{Vacuous Truth}\index{Truth!Vacuous} Proof via vacuous truth
            most often occurs when one has a proposition that needs proving and
            they need to handle the special case of the empty set $\emptyset$.
            For example, a \textit{subset} of a given set $B$ is another set $A$
            such that for every element $x\in{A}$ it is true that $x\in{B}$. We
            wish to prove the statement that if $A$ is a set, then $\emptyset$
            is a subset of $A$. That is, we need to prove
            \textit{if} $x\in\emptyset$, \textit{then} $x\in{A}$. But the empty
            set, as the name implies, contains no elements $(\emptyset=\{\,\})$
            so the hypothesis $x\in\emptyset$ is always false. We'd then say
            that the empty set is a subset of $A$ \textit{vacuously}.
        \end{example}
        In Ex.~\ref{ex:Mathematical_Implication} and
        Tab.~\ref{tab:Truth_Table_Implication} we see the only scenario where
        $P\Rightarrow{Q}$ is concluded to be false is when $P$ is true, yet $Q$
        is false. This is at odds with our use of \textit{if, then} in natural
        language, and further examples will be given when discussing modus
        ponens. For mathematical statements this truth table is adequate.
        \par\hfill\par
        Given this definition of implication the axiom of \textit{modus ponens},
        short for \textit{modus ponendo ponens} which in Latin means
        \textit{mode that by affirmings affirms}, seems redundantly obvious.
        Nevertheless, we must write it out. First, we must define
        \textit{axiom} and \textit{proof}. Proof is another primitive, but axiom
        is not. We may define axiom in terms of previous notions.
        \begin{fdefinition}{Proof}{Proof}
            A \gls{proof} of a \gls{proposition} $P$ is a valid argument
            affirming or denying $P$.\index{Proof}
        \end{fdefinition}
        Valid arguments are formed by combining previously proved propositions
        with definitions to arrive at a conclusion using the allowed rules of
        inference. As of yet we have not claimed what rules of inference we will
        accept nor do we have any propositions to build from. Both of these are
        addressed by \textit{axioms}. In the presentation of a proof one must
        justify each step without error.
        \begin{example}
            In elementary arithmetic, if $a$ is a real number and $a\ne{0}$,
            then we may divide by $a$, or take its \textit{reciprocal}%
            \index{Reciprocal}. Using this fact let's present the classic proof
            that $1=2$.
            \begin{align}
                1.)\quad&
                \text{Let $a$ and $b$ be non-zero real numbers with $a=b$}
                \tag{Hypothesis}\\
                2.)\quad
                &\text{If $a=b$, then $a^{2}=ab$}
                \tag{Multiply both sides by $a$}\\
                3.)\quad
                &\text{If $a^{2}=ab$, then $a^{2}-b^{2}=ab-b^{2}$}
                \tag{Subtract $b^{2}$ from both sides}\\
                4.)\quad
                &\text{But $a^{2}-b^{2}=(a-b)(a+b)$}
                \tag{Factoring}\\
                5.)\quad
                &\text{And $ab-b^{2}=(a-b)b$}
                \tag{Factoring}\\
                6.)\quad
                &\text{Therefore $(a-b)(a+b)=(a-b)b$}
                \tag{Laws of equality}\\
                7.)\quad
                &\text{Therefore $a+b=b$}
                \tag{Divide by $a-b$}\\
                8.)\quad
                &\text{But $a=b$, so $a+b=b+b=2b$}
                \tag{Hypothesis and arithmetic}\\
                9.)\quad
                &\text{Hence $2b=b$}
                \tag{Laws of equality}\\
                10.)\quad
                &\textit{But $b\ne{0}$, and therefore $2=1$}
                \tag{Divide by $b$}
            \end{align}
            If this proof is valid, then we have successfully proven arithmetic
            to be inconsistent and there is no further need to read this book.
            The flaw is in step 7 since we cannot divide by $a-b$. We declared
            $a=b$ in the hypothesis and hence $a-b=0$ and division by zero is
            not possibile in ordinary arithmetic.
        \end{example}
        \begin{example}
            \label{ex:Missing_Square_Puzzle}%
            \index{Missing Square Puzzle}%
            This next example shows why \textit{proof by picture}%
            \index{Proof!by Picture} is dangerous, as we will \textit{prove}
            $31.5=32.5$. Consider the two triangles shown in
            Fig.~\ref{fig:Missing_Square_Puzzle}. Both are inscribed in a
            $13\times{5}$ grid of squares. The triangle on the left covers
            $32.5$ blocks, whereas the one on the right covers all but one of
            those and so it contains $31.5$ squares. However, both objects are
            comprised of the same pieces so they should have identical area, and
            therefore $31.5=32.5$. What's wrong?
            \begin{figure}[H]
                \centering
                \captionsetup{type=figure}
                \includegraphics{images/Missing_Square_Puzzle_001.pdf}
                \caption{The Missing Square Puzzle}
                \label{fig:Missing_Square_Puzzle}
            \end{figure}
            The problem is that I've lied, neither object is a triangle at all,
            they're actually four-sided figures. The green and red triangles are
            not similar so we actually have a triangle with a \textit{dent} in
            it, and this is where the missing \textit{one} has gone. We can see
            this if we zoom into the \textit{pseudo-triangle} on the left and
            fill in the missing region
            (See Fig.~\ref{fig:Missing_Square_Puzzle_Solution}).
            \begin{figure}[H]
                \centering
                \captionsetup{type=figure}
                \includegraphics{images/Missing_Square_Puzzle_002.pdf}
                \caption{The Solution to the Missing Square Puzzle}
                \label{fig:Missing_Square_Puzzle_Solution}
            \end{figure}
            The root of the deception can be shown if we place the red triangle
            on top of the green one and show that they are not similar (have
            different angles). This is done in
            Fig.~\ref{fig:Missing_Square_Puzzle_Deception}.
        \end{example}
        \begin{figure}[H]
            \centering
            \captionsetup{type=figure}
            \includegraphics{images/Missing_Square_Puzzle_003.pdf}
            \caption{The Deception of the Missing Square Puzzle}
            \label{fig:Missing_Square_Puzzle_Deception}
        \end{figure}
        Ex.~\ref{ex:Missing_Square_Puzzle} is not intended to suggests that
        pictures have no place in mathematics. Indeed, intuition about complex
        concepts often comes from a nice drawing and in some settings a proof by
        picture is completely valid. There are two very well known proofs of
        \textit{Pythagoras' Theorem}\index{Pythagoras' Theorem}%
        \index{Theorem!Pythagoras'} dating to antiquity. One is usually
        attributed to Pythagoras of Samos (\textit{c.} 570-495 B.C.E.) himself,
        and the other is by Euclid. Pythagoras' proof is with pictures and is
        very clear and easy to grasp
        (See Fig.~\ref{fig:Proof_Pythagoras_Theorem}), whereas Euclid's proof
        appears after 46 other theorems had been rigorously proved using
        axioms, definitions, and previous theorems, but is perhaps not the
        easiest to comprehend.%
        \footnote{%
            The proof is not hard, just laborious.
        }
        \begin{figure}[H]
            \centering
            \captionsetup{type=figure}
            \includegraphics{images/Pythagoras_Theorem_Proof.pdf}
            \caption{Proof of Pythagoras' Theorem}
            \label{fig:Proof_Pythagoras_Theorem}
        \end{figure}
        A healthy middle ground is to present pictures for intuition, but only
        use rigor in the presentation of a proof. With this, we move on to
        axioms.
        \begin{fdefinition}{Axiom}{Axiom}
            An \gls{axiom} is a \gls{proposition} that is affirmed to be true
            without \gls{proof}.\index{Axiom}
        \end{fdefinition}
        There are two uses of the word \textit{axiom} in mathematics, and
        unfortunately they are different. We will use axiom in accordance with
        Def.~\ref{def:Axiom}, which is the original meaning of the word. The
        second meaning is essentially a synonymn for \textit{definition} or
        \textit{defining properties}. For example, in group theory one accepts
        the following \textit{axioms} of multiplication:
        \par
        \begin{subequations}
            \begin{minipage}[b]{0.49\textwidth}
                \begin{align}
                    a\cdot(b\cdot{c})&=(a\cdot{b})\cdot{c}\\
                    a\cdot{1}&=a
                \end{align}
            \end{minipage}
            \hfill
            \begin{minipage}[b]{0.49\textwidth}
                \begin{align}
                    1\cdot{a}&=a\\
                    a/a&=1
                \end{align}
            \end{minipage}
        \end{subequations}
        \par\vspace{2.5ex}
        These are not fundamental truths of every type of multiplication
        possible, but rather the defining properties of something called a
        group. Another common example is that of a \textit{vector space}, where
        the defining properties are called the \textit{axioms of vector spaces}.
        To be consistent with mathematical vocabulary this collection of
        properties should be called the \textit{definition} of vector spaces. If
        presented with some mathematical object one must then prove it
        satisfies the various properties, and then we may use the theory of
        vector spaces freely in our investigation of this object. This is
        contrasted with actual axioms which we take to always be true in every
        scenario. Throughout we will use axiom and definition as distinct
        notions. For the sake of completeness it should be mentioned that
        Def.~\ref{def:Axiom} refers to \textit{logical axioms}%
        \index{Axiom!Logical Axiom}, whereas the latter are called
        \textit{non-logical axioms}\index{Axiom!Non-Logical Axiom}.
        \par\hfill\par
        The word \textit{axiomatize}\index{Axiom!Axiomatize} will appear, which
        may cause confusion since it has a slightly different meaning than the
        other two. This is the process of taking a complicated mathematical
        structure, like the arithmetic of real numbers, and stripping away it's
        key properties such as associativity: $a+(b+c)=(a+b)+c$, or
        commutativity: $a+b=b+a$. We often give a name to an object that has
        these properties (groups, vector spaces, topological spaces, etc.) and
        have thus \textit{axiomatized} the fundamental traits. Axiomatize has no
        formal definition and is only used in discussions, and not in proofs or
        theorems.
        \par\hfill\par
        In a good system the axioms should be intuitively obvious and
        noncontroversial since you must convince other mathematicians they are
        valid. Usually axioms are based on intuition or modeling the real world.
        In Euclid's \textit{Elements}\index{Elements, The (Euclid)} there are 10
        axioms,%
        \footnote{%
            Five are called postulates and the others common notions.%
        }
        most of which are not too awe inspiring:
        \begin{center}
            \begin{enumerate}
                \item \textit{To draw a straight line from any point}
                      \textit{to any point.}
                \item \textit{To produce a finite straight line continuously in}
                      \textit{a straight line.}
                \item \textit{To describe a circle with any centre and}
                      \textit{distance.}
                \item \textit{That all right angles are equal to one another.}
                \item \textit{That if a straight line falling on two straight}
                      \textit{lines make the interior angles on the same side}
                      \textit{less than two right angles, the two straight}
                      \textit{lines, if produced indefinitely, meet on that}
                      \textit{side on which are the angles less than two right}
                      \textit{angles.}
            \end{enumerate}
            \hfill\textit{The Elements,}%
            \footnote{%
                This is from the Heath translation written by the Englishman
                Sir Thomas Little Heath\index{Heath, Sir Thomas Little}
                (1861-1940 C.E.), hence \textit{center} is spelled
                \textit{centre}.
            }\par
            \hfill\textit{Euclid of Alexandria, c. 300 B.C.E.}
        \end{center}
        The first three are modeling constructions using a straigth-edge and
        compass, and the fourth says there is one type of right angle. Only the
        fifth postulate, known famously as \textit{Euclid's Fifth Axiom}%
        \index{Axiom!Euclid's Fifth}, is non-obvious, perhaps because it is very
        wordy. Drawing what it describes shows that is indeed \textit{obvious}.
        \begin{figure}[H]
            \centering
            \captionsetup{type=figure}
            \includegraphics{images/Euclids_Fifth.pdf}
            \caption{Euclid's Fifth Axiom}
            \label{fig:Euclids_Fifth_Axiom}
        \end{figure}
        The fifth axiom asserts that the point $C$ in
        Fig.~\ref{fig:Euclids_Fifth_Axiom} exists. These five constitute the
        study of \textit{Euclidean geometry}\index{Geometry!Euclidean}%
        \index{Euclidean Geometry}, the first four of which make up
        \textit{absolute geometry}\index{Geometry!Absolute}%
        \index{Absolute Geometry}. For almost two thousand years a handful of
        mathematicians tried to prove that absolute and Euclidean geometry are
        the same. That is, the fifth postulate can be proved from
        the other four. Many minds of antiquity such as Claudius Ptolemny%
        \index{Ptolemy, Claidus} (\textit{c.} 100-170 C.E.), Proclus
        Lycaeous\index{Lycaeous, Proclus} (\textit{c.} 412-485 C.E.), Omar
        Khayy\'{a}m\index{Khayy\'{a}m, Omar} (1050-1123 C.E.), and others
        made the attempt, but all merely introduced an alternative axiom either
        explicitly or implicity which was equivalent to Euclid's. Euclid himself
        was hesitant in his use of the postulate, avoiding it for 30 theorems
        into book one until he needed it. So the question remains, can the fifth
        be proven from the first four? The answer is \textit{no}. We know this
        because there are \textit{models}\index{Model} of the first four where
        the fifth one fails.
        \par\hfill\par
        In the $18^{th}$ century the Swiss mathematician Johann Heinrich
        Lambert\index{Lambert, Johann Heinrich} (1728-1777 C.E.) made
        investigations into a quadrilateral where three of the angles are right.
        If one could prove the fourth angle must also be right, then with this
        Euclid's fifth could be proven.%
        \footnote{%
            Omar Khayy\'{a}m did this but claimed the fourth angle
            being right was self-evident.
        }
        Lambert discarded the possibility of the fourth angle being obtuse, but
        then made investigations into the acute case. He was never able to prove
        this is impossible, and indeed it's not. This leads to another
        \textit{model} of absolute geometry, \textit{hyperbolic geometry}%
        \index{Geometry!Hyperbolic}, in which the fourth angle is allowed to be
        acute. Such a quadrilateral is shown in
        Fig.~\ref{fig:Lambert_Khayyam_Quadrilateral}.
        \begin{figure}[H]
            \centering
            \captionsetup{type=figure}
            \if\compilefigures1
                \includegraphics{images/Lambert_Khayyam_Quadrilateral.pdf}
            \fi
            \caption{A Lambert-Khayy\'{m} Quadrilateral}
            \label{fig:Lambert_Khayyam_Quadrilateral}
        \end{figure}
        The revelation that Euclid's fifth is not provable from the other four
        is made more intuitive by the work of John
        Playfair\index{Playfair, John} (1748-1819 C.E.):
        \begin{center}
            \textit{In a plane, given a line and a point not on it, at most one}
            \textit{line parallel to the given line can be drawn through the}
            \textit{point}
            \par
            \hfill\textit{Elements of Geometry}\par
            \hfill\textit{John Playfair, 1795 C.E.}
        \end{center}
        This statement is known as \textit{Playfair's axiom}%
        \index{Axiom!Playfair's}. In combination with the first four axioms this
        is equivalent to Euclid's fifth. The first four can prove parallel lines
        exist, and hence we cannot claim no such line exists, but we may change
        Playfair's axiom to state that \textit{many} parallels exist, again
        giving us the model of hyperbolic geometry. Perhaps we wish to claim
        there are no parallel lines. This leads to \textit{elliptic geometry}%
        \index{Geometry!Elliptic}\index{Elliptic Geometry}, which is the
        geometry of a sphere. Elliptic geometry does not obey Euclid's third
        axiom since there is a maximum radius possible (half of the
        circumference of the sphere), showing us Euclid's third is independent
        of the other four.
        \par\hfill\par
        There are two things to take away from this discussion: Consistency and
        independence. If we accept a handful of axioms, we hope no
        inconsistencies are found. That is, there is no statement which can be
        proven from the axioms to be both true and false. Such systems are easy
        to construct, take as your first axiom the sentence \textit{P is true},
        and for the second \textit{P is false}, where $P$ is any claim you wish.
        This axiomatic system is inconsistent (by design). Most systems that are
        inconsistent are far more subtle.%
        \footnote{%
            Combinatory logic, na\"{i}ve set theory, etc.
        }
        Unfortunately for most useful axiomatic systems%
        \footnote{%
            A system where elementary arithmetic (addition and multiplication)
            is possible.
        }
        consistency is an inapproachable problem. One can prove consistency if
        and only if the system is actually inconsistent, paradoxically.%
        \footnote{%
            This is one of G\"{o}del's Theorems.
        }
        The second problem of independence is far more approachable and asks if,
        given a collection of axioms, are any of them redundant? That is, can
        one or more of them be proven from the others? Examples of independent
        axioms are Euclid's fifth and the axiom of choice (which is independent
        of the other axioms of ZFC). On the other hand, as we will see
        momentarily, Hilbert's first axiom of logic is redundant, provable from
        the other three.
        \par\hfill\par
        A final note before moving on from Euclid is the degree of rigor one
        accepts as valid. Let's prove the first theorem in Euclid's book and see
        if you accept the argument. We are to construct, given two points, an
        equilateral triangle with side lengths equal to the distance between the
        two starting points. Label the points $A$ and $B$. From Euclid's first
        axiom we may draw a line from $A$ to $B$. From the third axiom way may
        construct the circle centered at $A$ with radius $\overline{AB}$, and
        similarly the circle centered at $B$ with radius $\overline{BA}$.
        Labeling the point of intersection $C$, we may construct a line from $A$
        to $C$ and a line from $B$ to $C$ using Euclid's first. But then the
        triangle $\Delta{ABC}$ is an equilateral triangle. Since $C$ lies on the
        circle centered at $A$ with radius $\overline{AB}$, the length of
        $\overline{AC}$ is equal to $\overline{AB}$ per the definition of a
        circle. Similarly, the length of $\overline{BC}$ is equal to that of
        $\overline{BA}$. The construction is shown in
        Fig.~\ref{fig:Euclid_I_1}.
        \par\hfill\par
        \begin{figure}
            \centering
            \captionsetup{type=figure}
            \includegraphics{images/Euclid_Book_I_Prop_I.pdf}
            \caption{Proof of Euclid I.1}
            \label{fig:Euclid_I_1}
        \end{figure}
        Simple, clear, and definitely doable with a ruler, compass, and steady
        hands. But why does the point $C$ exist? The only axiom which asserts
        the existence of points is the fifth, and that has no use here. Indeed,
        the existence of the point $C$ is independent of the first five since
        we can construct a \textit{model} satisfying Euclid's axioms in which
        the point $C$ need not always exist. A sixth axiom along the lines of
        \textit{If two circles are placed such that distance between their}
        \textit{centers is equal to their radii, then the circles intersect} is
        required.
        \par\hfill\par
        Another critique was put forward by Zeno of Sidon\index{Zeno of Sidon}%
        \footnote{%
            Not to be confused with Zeno of Elea, famous for
            \textit{Zeno's Paradox}\index{Zeno's Paradox}\index{Paradox!Zeno's}.
        }
        who noted that none of the postulates, common notions, nor definitions
        provide a means for proving that two distinct lines can't intersect in
        more than one point. Indeed, it is not shown that two distinct lines can
        intersect along an entire line segment. Zeno describes the following
        construction shown in Fig.~\ref{fig:Zeno_Euclid_I_1} where the line
        segments $\overline{BC}$ and $\overline{AC}$ share the entire segment
        $\overline{DC}$ in common. A seventh axiom stating that distinct
        straight lines interesect in at most one point is needed.%
        \begin{figure}[H]
            \centering
            \captionsetup{type=figure}
            \includegraphics{images/Euclid_Book_I_Prop_I_Zeno.pdf}
            \caption{Zeno's Construction of Euclid I.1}
            \label{fig:Zeno_Euclid_I_1}
        \end{figure}
        This seventh axiom is not redundant since in elliptic/spherical geometry
        every pair of distinct lines intersects in \textit{two} different point.
        While spherical geometry does not obey the third axiom, and we've used
        the third axiom in this proof, it does obey an augmentation which states
        that \textit{given a point P and a point Q, to construct a circle about}
        \textit{P of radius} $\overline{PQ}$. Before we allowed arbitrary radius
        which is why the third fails in spherical geometry, whereas now we've
        tamed the size of the radius. This augmented version is what we used in
        the proof, so it still holds in spherical.%
        \footnote{%
            Unless the point $P$ is the north pole and the point $Q$ is the
            south pole, in which case a \textit{circle} degenerates to a single
            point.
        }
        Similarly, since the fifth axiom was never used, these results hold in
        hyperbolic geometry. Hence we have two different models of geometry we
        may apply this construction to and examine how the results differ from
        intuition (See Fig.~\ref{fig:Euclid_Spherical_I_I}).%
        \footnote{%
            The figure is \textit{slightly} misleading. For the reader
            interested in spherical geometry, straight lines and circles rarely
            coincide. The three points $A$, $B$, and $C$ were specifically
            chosen to make the computations easier, meaning the program to draw
            this was simpler. In this case, the straight lines and circles do
            indeed lie on top of each other, showing spherical geometry to be a
            rather bizarre study.
        }
        \begin{figure}[H]
            \centering
            \captionsetup{type=figure}
            \includegraphics{images/Euclid_Book_I_Prop_I_Spherical.pdf}
            \caption{Spherical Construction of Euclid I.1}
            \label{fig:Euclid_Spherical_I_I}
        \end{figure}
        Here, the red and blue curves represent the \textit{circles} centered at
        $A$ and $B$, respectively, whereas the black curves represent
        \textit{straight lines}. The green represents the equilateral triangle.
        The take-away is this: Be careful not to let intuition get the better of
        you, and pay attention to the assumption you make when using various
        axioms. It is very tempting to jump to Euclid's defense saying these
        critiques are absurd and the construction is valid,%
        \footnote{%
            Indeed, if you buy a ruler and compass and follow the construction
            on a piece of paper you will get the best looking equilateral
            triangle possible by hand.
        }
        but to do so would allow faulty arguments to be accepted as valid, and
        also deprive us of two beautiful forms of geometry.
        \par\hfill\par
        We are now in the position to state our first axiom. In most textbooks
        it is called a \textit{rule of inference} but for us there will be no
        difference.
        \begin{faxiom}{Modus Ponens}{Modus Ponens}
            If $A$ is a set, if $P$ and $Q$ are predicates on $A$, if
            $P\Rightarrow{Q}$ is true, if $x\in{A}$, and if $P(x)$ is
            true, then $Q(x)$ is true.\index{Axiom!of Modus Ponens}
        \end{faxiom}
        \begin{example}
            Let $A$ be the set of all animals, and $P$ be the predicate
            \textit{x is a bear}. Furthermore, let $Q$ be the predicate
            \textit{x is a mammal}. As previously discussed
            (Ex.~\ref{ex:Implication_Mammals}), biology tells us
            $P\Rightarrow{Q}$ is a true statement as all bears are mammals. That
            is, \textit{if x is a bear, then x is a mammal}. Hence if we stumble
            across a bear, by modus ponens we may conclude we've found a mammal.
        \end{example}
        \begin{example}
            This example is found in \cite{McGee1985ModusPonens}. For
            background, in 1980 C.E. there were three presidential candidates
            for the United States of America: Reagan (of the Republican party),
            Carter (of the Democratic party), and Anderson (running as an
            independent, but affiliated with the Republican party). So
            altogether there were two Republicans running and one Democrat. The
            argument goes as follows. The first proposition: If a Republican
            wins the election, then if Reagan does not win, then Anderson will
            win. This is true since there are only two Republican option, hence
            if Reagan doesn't win but it is true that a Republican does, the
            only option left is Anderson and hence Anderson will win. The second
            proposition: A Republican will win the election. The conclusion: If
            Reagan doesn't win, then Anderson will win. For some more feedback,
            while a Republican (Reagan) was predicted to win, and did indeed so,
            Carter was the sure runner up as Anderson did not garner much
            support. Because of this, the conclusion seems faulty and seems more
            likely that if Reagan doesn't win, then Carter will. Given the first
            two propositions, the third one is a valid conclusion. Since we've
            already stated the first proposition is valid, all eyes point to the
            second. There's no proof of this, and McGee's paper relies on proof
            by probability, a logical fallacy known as
            \textit{Appeal to Probability}\index{Fallacy!Appeal to Probability}%
            \index{Appeal to Probability}. Polls showed Reagan was likely to win
            and this is used to justify the second proposition is true. If by
            any other means one was able to logically prove the second
            proposition, then the conclusion, regardless of how inaccurate it
            may sound, is valid, further demonstrating the occasional disconnect
            between natural language and formal ones.
        \end{example}
        The next two commonly accepted rules of inference, known as
        \textit{modus tollens}\index{Axiom!of Modus Tollens} and
        \textit{contraposition}\index{Axiom!of Contraposition} are widely used
        in the mathematical world and relate to a statements contrapositive.
        The contrapositive is related to the \textit{negation} of a proposition,
        and so we define this now.
        \begin{fdefinition}{Negation}{Negation}
            The negation of a \gls{predicate} $P$ is the predicate \textit{not}
            $P$, denoted $\neg{P}$.
        \end{fdefinition}
        While we continue to use the word \textit{predicate} in these
        definitions it is worth noting some textbooks call these
        \textit{propositional variables}\index{Propositional Variable}, but the
        intent is the same: A sentence in some collection of variables which
        allows one to proclaim True of False, depending on input. Negation is a
        \textit{unary} operation acting on a single variable. The truth table
        for $\neg$ is shown below in Tab.~\ref{tab:Truth_Table_Negation}.
        \begin{table}[H]
            \centering
            \captionsetup{type=table}
            \begin{tabular}{c|c}
                $P$&$\neg{P}$\\
                \hline
                0&1\\
                1&0
            \end{tabular}
            \caption{Truth Table for Negation}
            \label{tab:Truth_Table_Negation}
        \end{table}
        That is, we take true to false and false to true. Negation and
        implication form our two primitive logical operations, and the other
        familiar terms (disjunction, conjunction, equivalence) can be expressed
        using these two. With negation defined, we now present a
        \textit{logical fallacy}\index{Logical Fallacy}, a form of reasoning
        that is invalid and leads to contradiction. This is the fallacy of
        \textit{affirming the consequent}%
        \index{Logical Fallacy!Affirming the Consequent}.
        \begin{example}
            \label{ex:Affirming_the_Consequent}%
            The fallacy of affirming the consequent is also known as the
            inverse fallacy\index{Logical Fallacy!Inverse Fallacy}, and in
            mathematics it is called the \textit{converse}
            fallacy\index{Logical Fallacy!of the Converse}.
            \textit{Modus ponens} tells us that if $P$ and $Q$ are statements
            with $P\Rightarrow{Q}$, and if $P$ is true, then $Q$ is true. The
            \textit{converse}\index{Converse} of the statement
            \textit{if P, then Q} is the sentence \textit{if Q, then P}. The
            validity of $P\Rightarrow{Q}$ does \textbf{not} verify the converse,
            much to the dismay of mathematicians. Theorems (which are just
            propositions that are \textit{mathematical} in nature) are often
            held in higher regard if they express the \textit{equivalence} of
            two propositions: $P\Rightarrow{Q}$ and $Q\Rightarrow{P}$. That is,
            both the statement and its converse are true. There are everyday and
            mathematical examples showing that this line of reasoning if faulty.
            Previously we discussed the classification of bears:
            \textit{if an animal is a bear, then it is a mammal}. The converse
            states if an animal is a mammal, then it is a bear. Humans are a
            counterexample to this claim since humans are mammals but are not
            bears. In the mathematical world we can consider the proposition
            \textit{if n is an odd integer, then n is not 2}. The converse
            states that if $n$ is not 2, then $n$ is an odd integer. But 4 is
            not 2, yet 4 is not an odd integer.
        \end{example}
        The invalidity of affirming the consequent can be see from truth tables:
        \begin{table}[H]
            \centering
            \captionsetup{type=table}
            \begin{tabular}{c|c|c|c}
                $P$&$Q$&$P\Rightarrow{Q}$&$Q\Rightarrow{P}$\\
                \hline
                0&0&1&1\\
                0&1&1&0\\
                1&0&0&1\\
                1&1&1&1
            \end{tabular}
            \caption{Truth Table for the Converse}
            \label{tab:Truth_Table_Converse}
        \end{table}
        Since the columns for $P\Rightarrow{Q}$ and $Q\Rightarrow{P}$ are
        different, we see that these are different claims. Negation further
        allows us to define the \textit{contrapositive}\index{Contrapositive} of
        $P\Rightarrow{Q}$, which is the new statement
        $\neg{Q}\Rightarrow\neg{P}$. As it turns out this is not new
        sentence at all and is equivalent to $P\Rightarrow{Q}$. Note
        $P\Rightarrow{Q}$ is false only when $P$ is true, yet $Q$ is false.
        Similarly, $\neg{Q}\Rightarrow\neg{P}$ is false only if $\neg{Q}$ is
        true and $\neg{P}$ is false. But if $\neg{Q}$ is true, then $Q$ is
        false (Def.~\ref{def:Negation}) and if $\neg{P}$ is false, then $P$ is
        true (Def.~\ref{def:Negation}). Thus $\neg{Q}\Rightarrow\neg{P}$ is only
        false when $P$ is true and $Q$ is false. We can further examine this via
        truth tables:
        \begin{table}[H]
            \centering
            \captionsetup{type=table}
            \begin{tabular}{c|c|c|c|c|c}
                $P$&$Q$&$\neg{P}$&$\neg{Q}$&$P\Rightarrow{Q}$
                    &$\neg{Q}\Rightarrow\neg{P}$\\
                \hline
                0&0&1&1&1&1\\
                0&1&1&0&1&1\\
                1&0&0&1&0&0\\
                1&1&0&0&1&1
            \end{tabular}
            \caption{Truth Table for the Contrapositive}
            \label{tab:Truth_Table_for_Contrapositive}
        \end{table}
        \begin{example}
            Suppose $a$ and $b$ are variables representing real numbers and $P$
            is the proposition $a<1/2$ and $b<1/2$, and let $Q$ be the
            proposition $a+b<1$. What is the contrapositive of
            $P\Rightarrow{Q}$? This would be $\neg{Q}\Rightarrow\neg{P}$, where
            $\neg{Q}$ is the negation of $Q$ which reads $a+b\geq{1}$.
            Similarly, $\neg{P}$ is the statement $a\geq{1}/2$ or $b\geq{1}/2$.
            Thus, the contrapositive says that if $a+b\geq{1}$, then either
            $a\geq{1}/2$ or $b\geq{1}/2$ (or both). While the contrapositive of
            a statement is always equivalent to the original statement, the
            converse need not be. Indeed, this statement is true (once one knows
            the order structure of real numbers), but the converse is not. The
            converse states that if $a+b<1$, then $a<1/2$ and $b<1/2$, but
            letting $a=2$ and $b=\minus{3}$ contradicts this claim.
        \end{example}
        The axiom of \textit{modus tollens} states that if
        $P\Rightarrow{Q}$, and if $\neg{Q}$ is true, then $\neg{P}$ is true. We
        are not directly adopting this axiom since it is provable from the
        axiom of \textit{modus ponens} if one accepts the standard axioms of set
        theory. Indeed, \textit{modus tollens} is implied by the
        \textit{law of the excluded middle}\index{Law of the Excluded Middle},
        which is in turn implied by the
        \textit{axiom of choice}\index{Axiom!of Choice}, two topics that will be
        discussed in Chapt.~\ref{chapt:Zermelo_Fraenkel_Set_Theory}. Similar to
        \textit{modus tollens}, the axiom of contraposition states that
        $P\Rightarrow{Q}$ is equivalent to $\neg{Q}\Rightarrow\neg{P}$. That is,
        if the statement $P\Rightarrow{Q}$ is true, then the contrapositive
        $\neg{Q}\Rightarrow\neg{P}$ is also true.
        \begin{example}
            Consider once again the description of Socrates. We start with the
            proposition \textit{all humans are mammals}. If Socrates is a human,
            then Socrates is a mammal. Therefore if Socrates is \textit{not} a
            mammal, then Socrates is \textit{not} a human. We could also say
            that all bears are mammals, and hence if you come across an animal
            that is \textit{not} a mammal, then this animal is \textit{not} a
            bear. Reasoning like this follows from contraposition.
        \end{example}
        Before moving to Hilbert systems it is worth while mentioning a few
        \textit{invalid} forms of reasoning. We do not include these as rules of
        inference since they lead to contradiction, although students often make
        these fallacious mistakes when exploring proofs for the first time.%
        \footnote{%
            I know I did.
        }
        We've discussed \textit{affirming the consequent} and now identify
        \textit{denying the antecedent}%
        \index{Logical Fallacy!Denying the Antecedent}, which is another type of
        converse fallacy and is the most common to make. Denying the antecedent
        goes as follows, if $P$ implies $Q$, and \textit{not} $P$, then
        \textit{not} $Q$. This is false, and we will provide plenty of examples
        to indicate this.
        \begin{example}
            Harking back to a previous example, consider the statement
            \textit{if I am late to work, then I will be fired}. Now suppose I
            was not late to work. Does this mean I was not fired? No! Perhaps I
            was lazy on the job, or uttered too many vulgarities (I do have a
            sailor's mouth). Knowing that I was not late tells us nothing about
            whether or not I was fired. It is only if I \textit{was} late that
            we can then appropriately apply \textit{modus ponens} and conclude
            that I was fired.
        \end{example}
        \begin{example}
            Consider the proposition
            \textit{If n is an odd integer, then n is not 2}. Now suppose we are
            told than $n$ is \textit{not} an odd integer. Can we conclude that
            $n$ is 2? No! It may be 4, or 6, or any other even integer.
        \end{example}
        We can consider $P\Rightarrow{Q}$ and $\neg{P}\Rightarrow\neg{Q}$ by
        means of truth table. The statement $\neg{P}\Rightarrow\neg{Q}$ is
        called the \textit{inverse} of $P\Rightarrow{Q}$.
        \begin{table}[H]
            \centering
            \captionsetup{type=table}
            \begin{tabular}{c|c|c|c|c|c}
                $P$&$Q$&$\neg{P}$&$\neg{Q}$&$P\Rightarrow{Q}$
                                           &$\neg{P}\Rightarrow\neg{Q}$\\
                \hline
                0&0&1&1&1&1\\
                0&1&1&0&1&0\\
                1&0&0&1&0&1\\
                1&1&0&0&1&1
            \end{tabular}
            \caption{Truth Table for the Inverse}
            \label{tab:Truth_Table_Inverse}
        \end{table}
        The next fallacy is known as the
        \textit{fallacy of the undistributed middle}%
        \index{Logical Fallacy!of the Undistributed Middle}. This is the first
        argument that takes three propositions. It falsely concludes that
        if $P_{1}\Rightarrow{Q}$, and if $P_{2}\Rightarrow{Q}$, then
        $P_{1}\Rightarrow{P}_{2}$. This is false as the truth table below
        demonstates.
        \begin{table}[H]
            \centering
            \captionsetup{type=table}
            \begin{tabular}{c|c|c|c|c|c}
                $P_{1}$&$P_{2}$&$Q$&$P_{1}\Rightarrow{Q}$&$P_{2}\Rightarrow{Q}$
                                   &$P_{1}\Rightarrow{P}_{2}$\\
                \hline
                0&0&0&1&1&1\\
                0&0&1&1&1&1\\
                0&1&0&1&0&1\\
                0&1&1&1&1&1\\
                1&0&0&0&1&0\\
                1&0&1&1&1&0\\
                1&1&0&0&0&1\\
                1&1&1&1&1&1
            \end{tabular}
            \caption{Fallacy of the Undistributed Middle}
            \label{tab:Fallcy_of_Undistributed_Middle}
        \end{table}
        There's a column on this table where $P_{1}\Rightarrow{Q}$ is true, and
        $P_{2}\Rightarrow{Q}$ is true, yet $P_{1}\Rightarrow{P}_{2}$ is false.
        Namely, choose $P_{1}=True$, $P_{2}=False$, and $Q=True$.
        \begin{example}
            Consider the claim \textit{all mathematicians love geometry} which
            one could only hope is true. Consider also
            \textit{all physicists love geometry}. Given that these two
            statements are true, it would be wrong to conclude that all
            mathematicians are physicists.
        \end{example}
        \begin{example}
            We can consider a mathematical proposition. If $n$ is divisible by
            2, then $n$ is even, and if $n$ is divisible by 4, then $n$ is even.
            We cannot conclude that if $n$ is divisible by 2, then $n$ is
            divisible by 4 since the number 6 serves as a counterexample. That
            is, $6=2\cdot{3}$ and hence 6 is divisible by 2, but it is not
            divisible by 4.
        \end{example}
        Lastly, we discuss the difference between a \textit{valid} argument and
        a \textit{sound} one. A valid argument is one that proves a claim from
        hypothesized propositions by correctly using the rules of inference. A
        sound argument is a valid argument of which the hypothesized
        premises are true.
        \begin{example}
            Consider the proposition \textit{all birds can fly}. We invoke
            \textit{modus ponens} and arrive at the following absurdity:
            \begin{subequations}
                \begin{align}
                    &\text{All birds can fly}.\\
                    &\text{Penguins are birds}.\\
                    &\text{Therefore, penguins can fly}.
                \end{align}
            \end{subequations}
            It is currently believed that penguins are incapable of flight no
            matter how hard they may try, and hence we have used our rules of
            inference correctly, but we've arrived at a false claim. That is,
            our argument is valid, but it cannot be sound. And indeed, the flaw
            is that the proposition \textit{all birds can fly} is false since
            penguins serve as a counterexample.
        \end{example}
        Another example is known as the \textit{masked-man fallacy}%
        \index{Logical Fallacy!Masked-Man Fallacy}.
        \begin{example}
            \label{ex:Masked_Man_Fallacy}%
            Suppose I have a friend \textit{Bob}. Since he is my friend, the
            proposition \textit{I know Bob} is true. Suppose further that there
            is a man wearing a mask. Since he is wearing a mask, I do not know
            who he is and hence the proposition
            \textit{I do not know the masked man} is true. We obtain the
            following:
            \begin{subequations}
                \begin{align}
                    &\text{I know Bob}.\\
                    &\text{I do not know that masked man}.\\
                    &\text{Therefore, Bob is not the masked man}.
                \end{align}
            \end{subequations}
            Our argument is valid and follows from \textit{modus tollens}. That
            is, we have the proposition
            \textit{if the man is Bob, then I know him}. Hence, if I don't know
            the man, then it is not Bob. However this argument is not sound
            since it is perfectly possible for Bob to be wearing the mask. The
            flaw comes from the proposition
            \textit{I do not know the masked man}. In truth, it may be possible
            that I do. This knowledge is not available to me and cannot be used
            or refuted in the argument.
        \end{example}
        \section{Hilbert Systems}
    In the set theory we will be working with there are a few words and symbols
    that are left undefined. As stated, this is unavoidable since defining
    everything would be circular, and we try to use the fewest number of
    \textit{primitive}. The main undefined symbol in set theory is that of
    \textit{containment} (\gls{containmentsymb})
    (see Not.~\ref{not:Element_Notation}), a type of
    \textit{\gls{predicate}}\index{Predicate} of the form \textit{x is in A}.
    Other common symbols such as subset (\gls{subseteq}) and equality
    (\gls{equalsymb}) are then defined in terms of this. In a similar manner
    there are other commonly used symbols in mathematical logic such as
    \textit{disjunction} (\gls{disjunctionsymb}), \textit{conjunction}
    (\gls{conjunctionsymb}), and \textit{equivalence} (\gls{equivalencesymb})
    that we need not accept as primitives, but rather can define in terms of
    implication (\gls{implicationsymb}) and negation (\gls{negationsymb}). We
    start with conjunction, which gives meaning the logical term \textit{and}.
    \subsection{Connectives}
        The symbol $\land$ is used to represent the word \textit{and} in a
        mathematical way. 
        \begin{fdefinition}{Conjunction}{Conjunction}
            The conjunction of propositions $P$ and $Q$ is the statement $P$ and
            $Q$ defined by the formula:\index{Conjunction}
            \begin{equation*}
                P\land{Q}\equiv\neg\big(P\Rightarrow\neg{Q}\big)
            \end{equation*}
        \end{fdefinition}
        Before justifying this definition, we wish to get an idea as to what
        \textit{and} should mean. Given two propositions $P$ and $Q$, $P$ and
        $Q$ should be considered if and only if both $P$ is true and $Q$ is
        true. That is, both are true simultaneously.
        \par\hfill\par
        \begin{fdefinition}{Disjunction}{Disjunction}
            The disjunction of propositions $P$ and $Q$ is the statement $P$ or
            $Q$ defined by the formula:
            \begin{equation*}
                P\lor{Q}\equiv(P\Rightarrow{Q})\Rightarrow{P}
            \end{equation*}
        \end{fdefinition}
    We have relied on the word \textit{statement} being already defined, and
    similarly for the words \textit{parameter} or \textit{variable}. For most
    this is not an issue, but it may irk others. From our undefined symbol $\in$
    we build new symbols by expressing them in terms of a
    \textit{formula}\index{Formula}, which is simply a finite
    sequence of symbols. Here the word \textit{sequence} is meant to imply that
    the \textit{order} in which we combine these symbols is important and that
    rearranging said order may create a different inequivalent formula. We build
    formulas by defining a few symbols that stand as placeholders for standard
    words in English. There are four symbols, called
    \textit{\glspl{connective}}\index{Connective (Logic)}, that we use.
    From this we see that we have introduced 6 new words that are undefined but
    require comment. The words are \textit{and, or, if, then, true}, and
    \textit{false}. There are other symbols we could adopt, such as
    \textit{equivalence}:
    \begin{equation*}
        a\Leftrightarrow{b}
    \end{equation*}
    But from how we shall define these notions, this new symbol is equivalent to
    a combination of the previous ones:
    \begin{equation*}
        \Big(\big(a\Leftrightarrow{b}\big)\Rightarrow
             \big((a\Rightarrow{b})\land(b\Rightarrow{a})\big)\Big)
        \land\Big(\big((a\Rightarrow{b})\land(b\Rightarrow{A})\big)
            \Rightarrow\big(a\Leftrightarrow{b}\big)\Big)
    \end{equation*}
    That is, $a\Leftrightarrow{b}$ if and only if $a$ is true if and only if $b$
    is true. Similarly, we could define \textit{does not imply}:
    \begin{equation*}
        a\not\Rightarrow{b}
    \end{equation*}
    But this is the same as:
    \begin{equation*}
        a\not\Rightarrow{b}\Longleftrightarrow
        \neg(a\Rightarrow{b})
        \Longleftrightarrow
        a\land\neg{b}
    \end{equation*}
    The words true and false are assumed to be well defined. They are also
    assumed to be opposites of each other (which we will define in terms of
    negation). We will use truth tables\index{Truth Table} to define what
    various connectives mean when it is known that certain propositions are true
    or false. In such tables the symbol 0 represents that a proposition is false
    and the symbol 1 represents truth.
    \par\hfill\par
    The other four words can be ambiguous in their everyday usage which we
    cannot allow for in mathematics. As such we must specify what we mean when
    we use these words and rid of any such ambiguity.
    \subsection{Conjunction}
        The conjunction connective (\gls{conjunctionsymb}) is used to denote the
        word \textit{and}. Given two propositions $P$ and $Q$, $P\land{Q}$ is a true
        statement if and only if both $P$ and $Q$ are true. That is, we
        associate to $\land$ the following truth table:
        \begin{table}[H]
            \centering
            \captionsetup{type=table}
            \begin{tabular}{ccc}
                $P$&$Q$&$P\land{Q}$\\
                \hline
                0&0&0\\
                0&1&0\\
                1&0&0\\
                1&1&1
            \end{tabular}
            \caption{Truth Table for Conjunction}
            \label{tab:Truth_Table_for_Conjunction}
        \end{table}
        There are several \textit{axioms} of conjunctions that are intuitively
        obvious, but must be stated since their use is wide spread.
        \begin{faxiom}{Axioms of Conjunction}{Axioms_of_Conjunction}
            If $P$ and $Q$ are propositions, then the following are true:
            \begin{align}
                P\land{Q}&\Longleftrightarrow{Q}\land{P}
                \tag{Commutativity of Conjunction}\\
                P\land\textrm{True}&\Longleftrightarrow\textrm{P}
                \tag{Identity of Conjunction}
            \end{align}
        \end{faxiom}
    \subsection{Disjunction}
        The disjunction connective (\gls{disjunctionsymb}) represents the word
        \textit{or}. Given two propositions $P$ and $Q$, $P\lor{Q}$ is true if
        and only if $P$ is true, or $Q$ is true, or both $P$ and $Q$ are true.
        There is an unfortunate ambiguity in English as to whether $P$ or $Q$
        means $P$ is true or $Q$ is true, but not both, or whether it means
        $P$ is true or $Q$ is true, or \textit{both} are true. The convention is
        to adopt the latter definition. That is, $P\lor{Q}$ has the following
        truth table:
        \begin{table}[H]
            \centering
            \captionsetup{type=table}
            \begin{tabular}{ccc}
                $P$&$Q$&$P\lor{Q}$\\
                \hline
                0&0&0\\
                0&1&1\\
                1&0&1\\
                1&1&1
            \end{tabular}
            \caption{Truth Table for Disjunction}
            \label{tab:Truth_Table_for_Disjunction}
        \end{table}
        There is another connective called the \textit{exlusive} or, which is
        defined to be false if both $P$ and $Q$ are true. The symbol $\lor$ is
        strictly used to denote the inclusive or. That is, the word or as
        represented by the truth table in
        Tab.~\ref{tab:Truth_Table_for_Disjunction}.
    \subsection{Implication}
        Examining, we see that there are scenarios where $P\Rightarrow{Q}$
        is true and $Q\Rightarrow{P}$ is false, and similarly where
        $P\Rightarrow{Q}$ is false and $Q\Rightarrow{P}$ is true. Propositions
        $P$ and $Q$ such that $P\Rightarrow{Q}$ and $Q\Rightarrow{P}$ are called
        \textit{equivalent}, and great deal of mathematics is devoted to the
        search for equivalencies of statements. This is denoted by the
        connective $P\Leftrightarrow{Q}$. Equivalence has the following truth
        table:
        \begin{table}[H]
            \centering
            \captionsetup{type=table}
            \begin{tabular}{cccccc}
                $P$&$Q$&$P\Rightarrow{Q}$&$Q\Rightarrow{P}$
                   &$P\Leftrightarrow{Q}$
                   &$(P\Rightarrow{Q})\land(Q\Rightarrow{P})$\\
                \hline
                0&0&1&1&1&1\\
                0&1&1&0&0&0\\
                1&0&0&1&0&0\\
                1&1&1&1&1&1
            \end{tabular}
            \caption{Truth Table for Equivalence}
            \label{tab:Truth_Table_for_Equivalence}
        \end{table}
        \subsection{Misc}
        \begin{table}[H]
            \centering
            \captionsetup{type=table}
            \begin{tabular}{c c c c c c}
                \hline
                $p$&$q$&$r$&$\neg{q}$&$p\lor\neg{q}$&$(p\lor\neg{q})\land{r}$\\
                \hline
                0&0&0&1&1&0\\
                0&0&1&1&1&1\\
                0&1&0&0&0&0\\
                0&1&1&0&0&0\\
                1&0&0&1&1&0\\
                1&0&1&1&1&1\\
                1&1&0&0&1&0\\
                1&1&1&0&1&1\\
                \hline
            \end{tabular}
            \caption{Truth Table for $(p\lor\neg{q})\land{r}$}
            \label{tab:Truth_Table_Example}
        \end{table}
        \begin{theorem}
            If $a\rightarrow{b}$, if $\neg{c}\rightarrow\neg{b}$, and if
            $\neg{c}$, then $\neg{a}$.
        \end{theorem}
        \begin{proof}
            For if $a\rightarrow{b}$, then $\neg{b}\rightarrow\neg{a}$. But
            $\neg{c}\rightarrow\neg{b}$. But if $\neg{c}\rightarrow\neg{b}$ and
            $\neg{b}\rightarrow\neg{a}$, then $\neg{c}\rightarrow\neg{a}$. Thus
            $a\rightarrow{b}$, $\neg{c}\rightarrow\neg{b}$, and thus
            $\neg{c}\Rightarrow\neg{a}$.
        \end{proof}
        \begin{problem}
            If $a\rightarrow{b}$, if $\neg{c}\rightarrow\neg{b}$, and if
            $\neg{c}$, then $\neg{a}$.
        \end{problem}
        \begin{proof}
            For if $\neg c \rightarrow \neg b$, then $b\rightarrow c$. But if
            $a\rightarrow b$ and $b\rightarrow c$, then $a\rightarrow c$.
            Therefore $a\rightarrow c$. But if $a\rightarrow c$, then
            $\neg c \rightarrow \neg a$. Therefore,
            $a\rightarrow b,\neg c\rightarrow\neg b,\neg c\Rightarrow\neg a$.
        \end{proof}
        \begin{ftheorem}{Law of Syllogism}{Law_of_Syllogism}
            If $P$, $Q$, and $R$ are propositions, if $P\Rightarrow{Q}$, and if
            $Q\Rightarrow{R}$, then $P\Rightarrow{R}$.
        \end{ftheorem}
    \chapter{Predicate Calculus}
        \label{chapt:Predicate_Calculus}%
        \section{Quantifiers}
    There are two more symbols called
    \textit{\glspl{quantifier}}\index{Quantifier}.
    \begin{equation*}
        \forall_{x}\quad\textrm{For all }x
        \quad\quad\quad\quad
        \exists_{x}\quad\textrm{There exists }x
    \end{equation*}
    Quantifiers, together with connectives, the word \textit{set}, and the
    $\in$ symbol are combined to define new terms and new symbols. The rest
    of mathematics rests on trusting ones intuition behind these notions.
    \begin{example}
        Let $\mathbb{R}$ denote the set of real numbers. The symbols
        $\forall_{R\in\mathbb{R}}(n^{2}\geq{0})$ can then be read in English
        as \textit{For all real numbers x, the square of x is non-negative},
        which is indeed a true statement. We can combine quantifiers to
        create more complicated statements, such as:
        \begin{equation}
            \forall_{x\in\mathbb{R}}(x\ne{0})\exists_{y\in\mathbb{R}}(xy=1)
        \end{equation}
        This reads that for all non-zero real numbers $x$, there exists a
        real numbers $y$ such that the product $xy$ is equal to 1. This is
        also a true statement.
    \end{example}
    \begin{example}
        The order of quantifiers is very important and often can not be
        interchanged. Considering the previous example, if we switch the
        order of the quantifiers we get:
        \begin{equation}
            \exists_{y\in\mathbb{R}}(xy=1)\forall_{x\in\mathbb{R}}(x\ne{0})
        \end{equation}
        This states that there exists a real number $y$ such that, for every
        non-zero real number $x$, it is true that $xy=1$. But this is
        certainly not true because if $x=1$ and $z=\minus{1}$, we obtain
        $(1)y=1$ and $(\minus{1})y=1$, and from this we conclude that
        $\minus{1}=1$, which is false. Hence, the order of the quantifiers
        is important.
    \end{example}
    \begin{example}
        Quantifiers can be combined with connectives to make longer and more
        complicated statements. For example, suppose $P$ is the proposition
        \textit{true if n is an even integer, false otherwise}. Furthermore,
        let $Q$ be the proposition \textit{true if n is a square integer},
        \textit{false otherwise}. Lastly, let $r$ be the proposition
        \textit{true if n is divisible by 4, false otherwise}.
        Consider then the following statement:
        \begin{equation}
            \forall_{n\in\mathbb{Z}}(p(n)\land{q(n)}\Rightarrow{r}(n))
        \end{equation}
        This reads in English as \textit{for all integers n, if n is an}
        \textit{integer, and if n is a square, then n is divisible by 4}.
    \end{example}
    \subsection{Negating Quantifiers}
        The negation of the statement \textit{for all x, P(x) is true} implies
        this is false. Thus there must exist one $x$ such that $P(x)$ is false,
        and from this we see that negating the $\forall$ quantifier produces the
        $\exists$ quantifier.
        \begin{example}
            Let $P$ be the proposition \textit{true if $x^{2}=2$} and consider
            the following statement:
            \begin{equation}
                \exists_{x\in\mathbb{Q}}\big(P(x)\big)
            \end{equation}
            This reads in plain English as the statement \textit{there exists a}
            \textit{rational number x whose square is equal to 2}. This has been
            known to be false since the ancient Greeks, and thus it's negation
            is true. We can write the negation as follows:
            \begin{equation}
                \neg\Big(\exists_{x\in\mathbb{Q}}\big(p(x)\big)\Big)
                \Longleftrightarrow\forall_{x\in\mathbb{Q}}\big(\neg{P}(x)\big)
            \end{equation}
            This now says that for all rational numbers $x$, the square of $x$
            is not equal to 2.
        \end{example}
    \chapter{Zermelo-Fraenkel Set Theory}
        \label{chapt:Zermelo_Fraenkel_Set_Theory}%
        We'll develop mathematics from an axiomatic view built on set theory,
        adopting as truths the few postulates of Zermelo and Fraenkel. We'll
        then add the axiom of choice and proceed from there to define many
        familiar concepts and prove some basic results that are often taken for
        granted. The existence of many types of sets will be proven, rather than
        accepting these things as trivial truths.
        %------------------------------------------------------------------------------%
\section{The Axioms of Zermelo and Fraenkel}
    The first thing to do is define what \textit{sets}\index{Set} are.
    \begin{fdefinition}{Sets}{Sets}
        A \gls{set} is a collection of objects (or elements), none of which is
        the set itself.
    \end{fdefinition}
    If we wish to stand on a truly solid foundation, it seems we're off to a bad
    start. In defining sets we used the words \textit{collection} and
    \textit{objects}, neither of which have been defined. This is the problem
    found in Chapt.~\ref{chapt:Logic} when defining connectives. To begin
    stating definitions and theorems we need the existence of a \textit{thing}.
    Sets act as our thing. We know they exist, but we don't know how to define
    them all to well. Nevertheless, we can describe how they behave and what
    they can do, as well as how to obtain new sets from pre-existing ones.
    \begin{fnotation}{Element Notation}{Element_Notation}
        If $A$ is a \gls{set} and if $x$ is an element\index{Element (Sets)} of
        $A$, then we denote this by writing $x\in{A}$. If $x$ is not an element
        of $A$, we write $x\notin{A}$.
    \end{fnotation}
    We do not yet know that sets exist. Pedagogically it seems poor to wait
    for examples, so we'll speak loosely for the moment so we may familiarize
    ourselves with the notation.
    \begin{fexample}{Using Element Notation}{Using_Element_Notation}
        Given a set $A$ that contains only a few objects, we can represent $A$
        by listing out the elements, separated by commas, and enclosing them in
        braces. Suppose $A$ is the set that contains three distinct objects
        labelled $a$, $b$, and $c$. We then write:
        \begin{equation}
            A=\big\{\,a,\,b,\,c\,\big\}
        \end{equation}
        If we are told that there is a fourth object $d$ that is different from
        $a$, $b$, and $c$, then we can use the notation defined in
        Not.~\ref{not:Element_Notation} to write the following:
        \par\hfill\par
        \begin{subequations}
            \begin{minipage}[b]{0.49\textwidth}
                \begin{equation}
                    a\in{A}
                \end{equation}
            \end{minipage}
            \hfill
            \begin{minipage}[b]{0.49\textwidth}
                \begin{equation}
                    d\notin{A}
                \end{equation}
            \end{minipage}
        \end{subequations}
        \par\vspace{2.5ex}
        The notation $a\in{A}$ should be read as \textit{a is an element of A},
        or \textit{a is contained in A}, or simply \textit{a is in A}.
        Similarly, the notation $d\notin{A}$ should be read as
        \textit{d is not an element of A}, or \textit{d is not contained in A}.
        \par\hfill\par
        $A$ is an example of a \textit{finite} set\index{Finite Set}, moreover
        it contains only three elements. For larger sets we rely on other
        methods to write them down. One such means is to indicate a pattern and
        use an ellipses to show that it goes on. Such a description is vague and
        lacks rigor, but can be helpful when the pattern is obvious. The set of
        all \textit{natural} numbers\index{Natural Numbers}, or non-negative
        integers (denoted $\mathbb{N}$) can be loosely represented by writing:
        \begin{equation}
            \label{eqn:Natural_Numbers_Ellipses}%
            \mathbb{N}=\big\{\,0,\,1,\,2,\,3,\,4,\,5,\,\dots\,\big\}
        \end{equation}
        Using our developed notation, we can write:
        \par\hfill\par
        \begin{subequations}
            \begin{minipage}[b]{0.49\textwidth}
                \begin{equation}
                    23\in\mathbb{N}
                \end{equation}
            \end{minipage}
            \hfill
            \begin{minipage}[b]{0.49\textwidth}
                \begin{equation}
                    \minus{4}\notin\mathbb{N}
                \end{equation}
            \end{minipage}
        \end{subequations}
        \par\vspace{2.5ex}
        Letting $\mathbb{Z}_{n}$ denote all non-negative integers between 0 and
        $n-1$, we have:
        \begin{equation}
            \label{eqn:Z_n_Ellipses}%
            \mathbb{Z}_{n}=\big\{0,\,1,\,2,\,\dots,\,n-1\,\big\}
        \end{equation}
        Thus $17\in\mathbb{Z}_{18}$ but $19\notin\mathbb{Z}_{18}$. Lastly, we
        present the integers\index{Integers}.
        \begin{equation}
            \label{eqn:Integers_Ellipses}%
            \mathbb{Z}=\big\{\,\dots,\,\minus{3},\,\minus{2},\,\minus{1},
                             \,0,\,1,\,2,\,3,\dots\,\big\}
        \end{equation}
    \end{fexample}
    In our definition of a set (Def.~\ref{def:Sets}) we explicitly required
    that sets cannot contain themselves. That is, if $A$ is a set, then
    $A\notin{A}$. This requirement was introduced to avoid paradoxes discovered
    by Bertrand Russell\index{Russell, Bertrand} in 1901. Allow us to neglect
    this requirement for a moment and reveal why it is essential. Recall from
    logic that a system of mathematics is inconsistent if one can prove a
    contradiction within the theory. In Naive Set Theory\index{Naive Set Theory}
    we allow the \textit{axiom of unrestricted comprehension}%
    \index{Axiom!of Unrestricted Comprehension}. This allows us to constructs
    sets as any definable collection. That is, if we have a proposition $P$,
    then we can define a set $A$ as the set of all objects that satisfy $P$.
    We can write:
    \begin{equation}
        A=\big\{\,x\;|\;P(x)\,\big\}
    \end{equation}
    Problems with such a loose definition arise instantly. Let $P$ be the
    proposition \textit{true if x is a set, false otherwise}. Then
    $A=\{\,x\;|\;P(x)\,\}$ can be read in plain English as the
    \textit{set of all sets}\index{Set!of All Sets}. A natural question would be
    whether or not $A$ then contains itself. That is, is $A\in{A}$? Russell's
    paradox arises by defining proper sets to be sets $B$ such that
    $B\notin{B}$, and improper sets to be sets $B$ such that $B\in{B}$. Using
    the \textit{Law of the Excluded Middle}\index{Law of the Excluded Middle}
    (which we will prove later), one has that every set is either proper or
    improper.
    \begin{ftheorem}{Russell's Paradox}{Russells_Paradox}
        Naive Set Theory is inconsistent.\index{Russell's Paradox}
    \end{ftheorem}
    \begin{bproof}
        For let $P$ be the proposition \textit{true if} $x\notin{x}$,
        \textit{false otherwise}. Let $A$ be the set defined by this
        proposition:
        \begin{equation}
            A=\big\{\,x\;|\;P(x)\,\big\}
        \end{equation}
        That is, $A$ is the set of all sets that do not contain themselves.
        Suppose $A\in{A}$. If $A\in{A}$ then $P(A)$ is true. That is, $A$ is a
        proper set. But proper sets do not contain themselves and $A\in{A}$, a
        contradiction. Thus $A\notin{A}$. But if $A\notin{A}$ than $P(A)$ is
        false. But if $P(A)$ is false, than $A$ is a improper set. But then
        $A\in{A}$, a contradiction as $A\notin{A}$. Thus $A\in{A}$ if and only
        if $A\notin{A}$, a contradiction. Therefore, Naive Set Theory is
        inconsistent.
    \end{bproof}
    Our development of Zermelo-Fraenkel Set Theory is to avoid this paradox and
    attempt to develop a consistent system of mathematics. The proof of
    Russell's Paradox (Thm.~\ref{thm:Russells_Paradox}) relied on the
    \textit{Law of the Excluded Middle}\index{Law of the Excluded Middle} which
    states that, given a proposition $P$, either $P$ is true or its negation is
    true. Thus we have shown that the axiom of unrestricted
    comprehension\index{Axiom!of Unrestricted Comprehension} and the law of the
    excluded middle are not compatible. This is quite unfortunate as the law of
    the excluded middle is essential in mathematics as it allows one to prove
    things via contradiction\index{Proof by Contradiction}. That is, given some
    statement we assume the opposite is true and arrive at a contradiction thus
    showing the negation of our statement is false. We then invoke the law of
    the excluded middle to show that our original statement is true. The axioms
    of Zermelo and Fraenkel, together with the axiom of choice (a system
    commonly abbreviated as ZFC) are able to prove the validity of the law of
    the excluded middle. That is, if ZFC is consistent, then so is the law of
    the excluded middle. This is one of the reasons for studying ZFC in detail.
    \par\hfill\par
    The first collection of axioms were proposed in 1908 by
    Ernst Zermelo\index{Zermelo, Ernst}. Subtle problems were pointed out by
    Abraham Fraenkel\index{Fraenkel, Abraham} in 1920, and in 1921 the system of
    Zermelo-Fraenkel Set Theory\index{Zermelo-Fraenkel Set Theory} came to be.
    The requirement that a set does not contain itself is sufficient to avoid
    Russell's paradox. This is equivalent to the
    \textit{axiom of regularity}\index{Axiom!of Regularity}. We will prove the
    equivalence of this axiom with our definition once we have obtained the law
    of the excluded middle.
    \subsection{Subsets and Equality}
        To delve more into set theory it would be convenient to know that at
        least \textit{one} set exists. The axiom of the empty
        set\index{Axiom!of the Empty Set} gives us such an existance.
        \begin{faxiom}{Axiom of the Empty Set}{Axiom_of_the_Empty_Set}
            There exists a set $\emptyset$ (the \gls{empty set}) such that for
            all $x$ it is true that $x\notin\emptyset$.\index{Empty Set}
            \begin{equation*}
                \exists_{\emptyset}:\forall_{x}\big(\neg(x\in\emptyset)\big)
            \end{equation*}
        \end{faxiom}
        The empty set is the set that contains no elements. As such, some choose
        to write $\emptyset=\{\}$. Note that the empty set is different from the
        set $\{\emptyset\}$. The empty set contains no elements whereas
        $\{\emptyset\}$ contains one elements (it contains the empty set).
        Indeed, the equality of $\emptyset$ and $\{\emptyset\}$ would violate
        our requirement that sets do not contain themselves. Any set that
        contains \textit{something} is called non-empty.
        \begin{fdefinition}{Non-Empty Set}{Non_Empty_Set}
            A \gls{non-empty set} is a \gls{set} $A$ such that there exists an
            $x$ such that $x\in{A}$.\index{Non-Empty Set}
        \end{fdefinition}
        \begin{example}
            The terminology is somewhat redundant, and essentially every set we
            deal with is non-empty. Indeed, there is only one empty set
            (see Thm.~\ref{thm:Empty_Set_is_Unique}). Thus, every other set one
            thinks of ($\mathbb{N},\mathbb{Z},\mathbb{Q},\mathbb{R},\mathbb{C}$,
            etc.) is non-empty.
        \end{example}
        \begin{example}
            It's possible to write down some formula for a set that ultimately
            leads to the empty set. For consider the \textit{set of all}
            \textit{rational numbers whose square is two}. This set turns out to
            be empty since there is no rational that satisfies this criterion.
            That is, $\sqrt{2}$ is known to be an irrational number. Thus, the
            set specified by our proposition is the empty set.
        \end{example}
        \begin{example}
            Going in the other direction, it is possible to write a formula for
            a set that appears empty, but is indeed not. The set of all
            $p\textrm{-Sylow}$ subgroups\index{Sylow Subgroups} of a non-empty
            finite group (Discussed in Book~\ref{book:Algebra}) is a non-empty
            set, but there's no reason to believe so from the start.
        \end{example}
        A set is entirely determined by its elements, and as such repetition and
        order cannot be accounted for. That is, the sets $\{a,b\}$ and
        $\{a,a,b\}$ must be considered the same since they contain precisely the
        same elements. This will be made clear once equality has been defined.
        In a similar manner, sets have no sense of order and thus $\{a,b\}$ and
        $\{b,a\}$ are equivalent. It then becomes a task to invent some new
        object that does have a notion of order. To do this requires the concept
        of a \textit{function}\index{Function}, and it is our current aim to
        develop this topic.
        \par\hfill\par
        To rigorously show that the examples in the previous paragraph are equal
        requires a definition of equality. This is the
        \textit{axiom of extensionality}\index{Axiom!of Extensionality}. First,
        we define the familiar symbol for equality\index{Equality} in terms of
        containment.
        \begin{fnotation}{Equality}{Equality}
            If $A$ and $B$ are sets, then $A=B$ if and only if for all sets
            $C$, $C\in{A}$ if and only if $C\in{B}$, and for all sets $D$,
            $A\in{D}$ if and only if $B\in{D}$.
            \begin{equation*}
                \forall_{A}\forall_{B}(A=B)
                \Longleftrightarrow\Big(
                    \forall_{C}(C\in{A}\Leftrightarrow{C}\in{B})
                    \land\forall_{D}(A\in{D}\Leftrightarrow{B}\in{D})\Big)
            \end{equation*}
        \end{fnotation}
        \begin{example}
            Consider the set of all planets in the solar system, and consider
            the set of the eight largest objects in the solar system other than
            the sun. These two sets are equal since the eight largest objects
            (other than the sun) are the eight planets (sorry Pluto), and the
            set of planets form the eight largest objects. The tricky part is
            to check that for any set one can name, it is true that if the set
            of planets lies in the set, then the set of the eight largest
            objects not equal to the sun lie in this set as well, and vice
            versa. This is almost an impossible task, and so we rely on the
            \textit{axiom of extensionality} the demonstration of equality.
        \end{example} 
        The axiom of extensionality says that to check for equality it suffices
        to show that for all $C$, $C\in{A}$ if and only if $C\in{B}$.
        That is, there is no need to check that for all $D$, $A\in{D}$ if and
        only if $B\in{D}$. For simplicity, the axiom of extensionality may be
        taken as the definition of equality.
        \begin{faxiom}{Axiom of Extensionality}{Axiom_of_Extensionality}
            If $A$ and $B$ are sets, and if for all $x$ it is true that
            $x\in{A}$ if and only if $x\in{B}$, then $A=B$. That is, $A$ and $B$
            are equal sets.\index{Axiom!of Extensionality}
            \begin{equation*}
                \forall_{x}\forall_{y}\Big(\forall_{z}(z\in{x}\Leftrightarrow
                z\in{y})\Longleftrightarrow\big(x=y\big)\Big)
            \end{equation*}
        \end{faxiom}
        \begin{example}
            Returning to our example of planets, we have seen that the set of
            all planets and the set of the eight largest objects other than the
            sun contain precisely the same elements. By the axiom of
            extensionality, we thus have equality amongst these two.
        \end{example}
        We won't adopt this axiom directly, but restate equality as a definition
        using the language of subsets\index{Set!Subset}. This will make proving
        various things easier in the proceeding sections. The notions are
        equivalent. Subsets are sets that are defined in terms of another given
        set by simply removing some (or none, or all) of the elements.
        \begin{fdefinition}{Subsets}{Subsets}
            A \gls{subset} of a \gls{set} $B$ is a set $A$ such that for all
            $x\in{A}$ it is true that $x\in{B}$. If $A$ is a subset of $B$ we
            write $A\subseteq{B}$. Otherwise, we write $A\nsubseteq{B}$.
            \index{Subset}
            \begin{equation*}
                \forall_{A}\forall_{B}\Big(\big(A\subseteq{B}\big)
                \Longleftrightarrow
                \forall_{x}\big(x\in{A}\Rightarrow{x}\in{B}\big)\Big)
            \end{equation*}
        \end{fdefinition}
        We can often visualize sets as blobs in the plane. Using such a visual,
        we can envision subsets as well (Fig.~\ref{fig:Subset_Blobs}). Given a
        blob $B$, a subset of $B$ is another blob $A$ that is entirely contained
        within $B$.
        \begin{figure}[H]
            \centering
            %--------------------------------Dependencies----------------------------------%
%   tikz                                                                       %
%-------------------------------Main Document----------------------------------%
\begin{tikzpicture}[line width=0.2mm, scale=1.2]

    % Coordinates for the bigger blob.
    \coordinate (P1) at ( 0.0, -2.0);
    \coordinate (P2) at ( 1.0, -1.0);
    \coordinate (P3) at ( 1.5,  1.0);
    \coordinate (P4) at ( 0.0,  2.0);
    \coordinate (P5) at (-3.0,  0.0);

    % Coordinates for the inner blob.
    \coordinate (Q1) at ( 0.0, -1.0);
    \coordinate (Q2) at ( 1.0,  0.0);
    \coordinate (Q3) at ( 0.5,  0.5);
    \coordinate (Q4) at (-0.5,  0.5);
    \coordinate (Q5) at (-1.0,  0.0);

    % Coordindates to label things.
    \coordinate (A) at (-0.1, -0.2);
    \coordinate (B) at (-1.5,  0.5);

    % Draw the bigger blob.
    \draw[fill=red, opacity=0.4] (P1) to [out=0,    in=-120] (P2)
                                      to [out=60,   in=-45]  (P3)
                                      to [out=135,  in=0]    (P4)
                                      to [out=-180, in=70]   (P5)
                                      to [out=-110, in=-180] cycle;

    % Draw the inner blob.
    \draw[fill=cyan, opacity=0.8] (Q1) to [out=0,    in=-120]  (Q2)
                                       to [out=60,   in=20]    (Q3)
                                       to [out=-160, in=45]    (Q4)
                                       to [out=-135, in=90]    (Q5)
                                       to [out=-90,  in=180]   cycle;

    % Labels for the two blobs.
    \node at (A) {$A$};
    \node at (B) {$B$};
\end{tikzpicture}

            \caption{Visualizing Subsets as Blobs}
            \label{fig:Subset_Blobs}
        \end{figure}
        \begin{example}
            Consider the set of natural numbers $\mathbb{N}$ and the set of
            integers $\mathbb{Z}$ (loosely defined in
            Eqn.~\ref{eqn:Natural_Numbers_Ellipses} and
            Eqn.~\ref{eqn:Integers_Ellipses}, respectively). It can be seen that
            every natural number is also an integer, and thus we have:
            \begin{equation}
                \mathbb{N}\subseteq\mathbb{Z}
            \end{equation}
            Letting $\mathbb{Q}$ denote the rational numbers $p/q$, where
            $p,q\in\mathbb{Z}$ and $q$ is non-zero, we can see that $\mathbb{Q}$
            contains $\mathbb{Z}$ as a subset. That is, setting $q=1$ and
            allowing $p$ to vary over $\mathbb{Z}$ gives us every integer. Thus:
            \begin{equation}
                \mathbb{Z}\subseteq\mathbb{Q}
            \end{equation}
            We can continue with the real numbers and the complex numbers as
            well, creating a chain of subsets:
            \begin{equation}
                \mathbb{N}\subseteq\mathbb{Z}\subseteq\mathbb{Q}
                \subseteq\mathbb{R}\subseteq\mathbb{C}
            \end{equation}
        \end{example}
        We can use subsets to define equality and to provide examples
        of the \textit{axiom schema of specification}%
        \index{Axiom!Schema of Specification}. It is important to note the
        distinction between the symbols $\in$ and $\subseteq$. The symbol $\in$
        is used to denote that some object $x$ is an \textit{element}%
        \index{Element (Sets)} of some set. That is, $x\in{A}$ indicates that
        $x$ is an element of $A$. This does not necessarily imply
        $x\subseteq{A}$, but this \textit{does} imply that $\{x\}\subseteq{A}$.
        That is, if $x\in{A}$, then the set that contains only $x$ is a subset
        of $A$. Moreover, the notions are not mutually exclusive. It is possible
        for $A$ to be a set such that $x\in{A}$ and $x\subseteq{A}$. For let
        $A=\{\emptyset\}$. For any set $A$ it is true that
        $\emptyset\subseteq{A}$ (see Thm.~\ref{thm:Emptyset_Is_Subset}). But
        from how $A$ is defined, we have that $\emptyset\in{A}$. Thus it is true
        that both $\emptyset\in{A}$ and $\emptyset\subseteq{A}$.
        \begin{fexample}{Elementary Examples of Subsets}
                        {Elementary_Examples_of_Subsets}
            Let $A$ and $B$ be the sets defined by:
            \par\hfill\par
            \begin{subequations}
                \begin{minipage}[b]{0.49\textwidth}
                    \begin{equation}
                        A=\big\{\,a,\,b,\,c\,\big\}
                    \end{equation}
                \end{minipage}
                \hfill
                \begin{minipage}[b]{0.49\textwidth}
                    \begin{equation}
                        B=\big\{\,a,\,b,\,c,\,d\,\big\}
                    \end{equation}
                \end{minipage}
            \end{subequations}
            \par\vspace{2.5ex}
            where we assume that $a$, $b$, $c$, and $d$ are distinct objects.
            From the definition of subsets (Def.~\ref{def:Subsets}):
            \par\hfill\par
            \begin{subequations}
                \begin{minipage}[b]{0.49\textwidth}
                    \begin{equation}
                        A\subseteq{B}
                    \end{equation}
                \end{minipage}
                \hfill
                \begin{minipage}[b]{0.49\textwidth}
                    \begin{equation}
                        B\nsubseteq{A}
                    \end{equation}
                \end{minipage}
            \end{subequations}
            \par\vspace{2.5ex}
            This is true since from the definition of $A$ and $B$, every element
            of $A$ is also an element of $B$. The converse of this is not true
            since there is an element of $B$ that is not an element of $A$
            (namely, the element $d$). That is, $d\in{B}$ but $d\notin{A}$ and
            therefore $B\nsubseteq{A}$.
        \end{fexample}
        The example shown in Ex.~\ref{ex:Elementary_Examples_of_Subsets} shows
        how we can define equality of sets. We see that $A\subseteq{B}$, but
        $B\nsubseteq{A}$. If we have two sets $A$ and $B$ such that
        $A\subseteq{B}$ and $B\subseteq{A}$ then it would be impossible to
        discern between the two. This gives us our new definition of equality.
        We now prove this equivalence with the axiom of extensionality%
        \index{Axiom!of Extensionality} (Ax.~\ref{ax:Axiom_of_Extensionality}).
        \begin{theorem}
            \label{thm:Equivalent_Def_of_Equality}%
            If $A$ and $B$ are sets, then $A=B$ if and only if $A\subseteq{B}$
            and $B\subseteq{A}$.
        \end{theorem}
        \begin{proof}
            For by the axiom of extensionality
            (Ax.~\ref{ax:Axiom_of_Extensionality}), $A=B$ if and only if, for
            all $x$ it is true that $x\in{A}$ if and only if $x\in{B}$. But then
            $x\in{A}$ implies that $x\in{B}$, and thus $A\subseteq{B}$
            (Def.~\ref{def:Subsets}). But also $x\in{B}$ implies $x\in{A}$, and
            therefore $B\subseteq{A}$. Moreover, if $A\subseteq{B}$ and
            $B\subseteq{A}$, then for all $x\in{A}$ it is true that $x\in{B}$
            and for all $x\in{B}$ it is true that $x\in{A}$
            (Def.~\ref{def:Subsets}), and therefore $x\in{A}$ if and only if
            $x\in{B}$. Thus, $A=B$ if and only if $A\subseteq{B}$ and
            $B\subseteq{A}$.
        \end{proof}
        With this, we can redefine the notion of
        \textit{equal sets}\index{Equal Sets}.
        \begin{fdefinition}{Equal Sets}{Equal_Sets}
            \Glspl{equal set} are \glspl{set} $A$ and $B$, denoted $A=B$, such
            that $A\subseteq{B}$ and $B\subseteq{A}$.\index{Equal Sets}
            \begin{equation*}
                \forall_{A}\forall_{B}(A=B)
                \Leftrightarrow
                (A\subseteq{B}\land{B}\subseteq{A})
            \end{equation*}
        \end{fdefinition}
        Def.~\ref{def:Equal_Sets} is justified by
        Thm.~\ref{thm:Equivalent_Def_of_Equality}, and thus there is no
        contradiction with the axiom of extensionality
        (Ax.~\ref{ax:Axiom_of_Extensionality}). If $A$ and $B$ are not equal, we
        write $A\ne{B}$. 
        \begin{lexample}{More Examples of Subsets}{More_Examples_of_Subsets}
            Using the notation from Ex.~\ref{ex:Using_Element_Notation}, for all
            $n\in\mathbb{N}$ we have:
            \begin{equation}
                \mathbb{Z}_{n}\subseteq\mathbb{N}
            \end{equation}
            Let's define $\mathbb{N}_{e}$ and $\mathbb{N}_{o}$ to be the sets of
            even\index{Even Integers} and odd\index{Odd Integers} non-negative
            integers, respectively:
            \par
            \begin{subequations}
                \begin{minipage}[b]{0.49\textwidth}
                    \begin{equation}
                        \label{eqn:Even_Pos_Ints_Ellipses}%
                        \mathbb{N}_{e}=\big\{\,0,\,2,\,4,\,6,\,8,\,\dots\,\big\}
                    \end{equation}
                \end{minipage}
                \hfill
                \begin{minipage}[b]{0.49\textwidth}
                    \begin{equation}
                        \label{eqn:Odd_Pos_Ints_Ellipses}%
                        \mathbb{N}_{o}=\big\{\,1,\,3,\,5,\,7,\,9,\,\dots\,\big\}
                    \end{equation}
                \end{minipage}
            \end{subequations}
            \par\vspace{2.5ex}
            From this we see the following two expressions are true:
            \par\hfill\par
            \begin{subequations}
                \begin{minipage}[b]{0.49\textwidth}
                    \begin{equation}
                        \mathbb{N}_{o}\subseteq\mathbb{N}
                    \end{equation}
                \end{minipage}
                \hfill
                \begin{minipage}[b]{0.49\textwidth}
                    \begin{equation}
                        \mathbb{N}_{e}=\mathbb{N}
                    \end{equation}
                \end{minipage}
            \end{subequations}
            \par\vspace{2.5ex}
            Moreover we see that $\mathbb{N}_{o}$ and $\mathbb{N}_{e}$ have no
            elements in common. That is, they are \textit{disjoint}%
            \index{Disjoint Sets}. We can represent this symbolically by
            writing:
            \par\hfill\par
            \begin{subequations}
                \begin{minipage}[b]{0.49\textwidth}
                    \begin{equation}
                        \mathbb{N}_{o}\nsubseteq\mathbb{N}_{e}
                    \end{equation}
                \end{minipage}
                \hfill
                \begin{minipage}[b]{0.49\textwidth}
                    \begin{equation}
                        \mathbb{N}_{e}\nsubseteq\mathbb{N}_{o}
                    \end{equation}
                \end{minipage}
            \end{subequations}
            \par\vspace{2.5ex}
            We can also think of trivial examples. We see that:
            \par\hfill\par
            \begin{subequations}
                \begin{minipage}[b]{0.49\textwidth}
                    \begin{equation}
                        \mathbb{Z}_{3}\subseteq\mathbb{Z}_{4}
                    \end{equation}
                \end{minipage}
                \hfill
                \begin{minipage}[b]{0.49\textwidth}
                    \begin{equation}
                        \mathbb{Z}_{4}\nsubseteq\mathbb{Z}_{3}
                    \end{equation}
                \end{minipage}
            \end{subequations}
            \par\vspace{2.5ex}
            This is because every element of $\mathbb{Z}_{3}$ is contained in
            $\mathbb{Z}_{4}$, but $3\in\mathbb{Z}_{4}$ but
            $3\notin\mathbb{Z}_{3}$. It may seem like bad notation to write
            $3\notin\mathbb{Z}_{3}$, but since we want $\mathbb{Z}_{n}$ to have
            $n$ elements, and since we started counting at zero, we have that
            $n\notin\mathbb{Z}_{n}$ for all $n\in\mathbb{N}$. Such counting
            schemes are common in computer science, but there's disagreement in
            mathematics as to whether $0\in\mathbb{N}$ or not. We will use the
            \textit{axiom of infinity}\index{Axiom!of Infinity} to prove the
            existence of $\mathbb{N}$, and in doing so it will be natural to
            define $\mathbb{N}$ as a set that contains $0$.
        \end{lexample}
        While Def.~\ref{def:Equal_Sets} is indeed equivalent to the axiom of
        extensionality, this definition creates a few problems. As discussed
        previously, sets have no notion of order and cannot account for
        repetition. For let $A$, $B$, and $C$ be the sets defined by:
        \par
        \begin{subequations}
            \begin{minipage}[b]{0.31\textwidth}
                \begin{equation}
                    A=\big\{\,a,\,b\,\big\}
                \end{equation}
            \end{minipage}
            \hfill
            \begin{minipage}[b]{0.36\textwidth}
                \begin{equation}
                    B=\big\{\,a,\,a,\,b\,\big\}
                \end{equation}
            \end{minipage}
            \hfill
            \begin{minipage}[b]{0.31\textwidth}
                \begin{equation}
                    C=\big\{\,b,\,a\,\big\}
                \end{equation}
            \end{minipage}
        \end{subequations}
        \par\vspace{2.5ex}
        All three of these sets are equal by both the definition of equality
        (Def.~\ref{def:Equal_Sets})\index{Equal Sets} and the axiom of
        extensionality\index{Axiom!of Extensionality}. It seems clear that
        $A\subseteq{B}$, but it is also true that $B\subseteq{A}$. This is
        because $B$ contains only the elements $a$ and $b$. While $a$ is
        included twice, repetition cannot be accounted for and $B$ is entirely
        determined by $a$ and $b$. But $A$ also contains $a$ and $b$, and
        therefire $B\subseteq{A}$. By the definition of equality
        (Def.~\ref{def:Equal_Sets}), we have that $A=B$. In a similar manner,
        $A=C$. From the definition of subsets, for any set $A$ we see that
        $A\subseteq{A}$ (see Thm.~\ref{thm:Reflexivity_of_Inclusion}). It would
        be nice to distinguish between subsets that aren't the entire set
        itself. These are called proper subsets\index{Set!Subset!Proper}, and
        we can define them in terms of equality.
        \begin{fdefinition}{Proper Subsets}{Proper_Subsets}
            A \gls{proper subset} of a \gls{set} $B$ is a set $A$ such that
            $A\subseteq{B}$ and $A\ne{B}$. We write $A\subsetneq{B}$
            to denote that $A$ is a proper subset of $B$.\index{Proper Subset}
            \begin{equation*}
                \forall_{A}\forall_{B}(A\subsetneq{B})
                \Leftrightarrow(A\subseteq{B}\land{A}\ne{B})
            \end{equation*}
        \end{fdefinition}
        The symbols $\subseteq$ and $\subsetneq$ are analogous to the notations
        of inequalities that one finds in calculus: $\leq$ and $<$. In many
        texts, the two symbols $\subseteq$ and $\subset$ are taken to be
        identical, which may cause confusion. In an attempt to reduce confusion,
        $\subseteq$ will denote any subset, $\subsetneq$ denotes a proper
        subset, and the symbol $\subset$ will be avoided.
        \begin{lexample}{Proper Subsets}{Proper_Subsets}
            Let $A$ and $B$ be sets defined as follows:
            \par
            \begin{subequations}
                \begin{minipage}[b]{0.49\textwidth}
                    \centering
                    \begin{equation}
                        A=\big\{\,a,\,b,\,c\,\big\}
                    \end{equation}
                \end{minipage}
                \hfill
                \begin{minipage}[b]{0.49\textwidth}
                    \centering
                    \begin{equation}
                        B=\big\{\,a,\,b,\,c,\,d\,\big\}
                    \end{equation}
                \end{minipage}
            \end{subequations}
            \par\vspace{2.5ex}
            Then $A\subseteq{B}$, since every element of $A$ is an element of
            $B$, but $B\nsubseteq{A}$ since $d\in{B}$ and $d\notin{A}$.
            Therefore $A\ne{B}$, and thus $A$ is a proper subset of $B$. We can
            denote this by writing $A\subsetneq{B}$.
        \end{lexample}
        \begin{example}
            Returning to more concrete examples, $\mathbb{N}$ is a proper subset
            of $\mathbb{Z}$. To see this, note that $\minus{1}\in\mathbb{Z}$ but
            $\minus{1}\notin\mathbb{N}$. Indeed, none of the negative integers
            are natural numbers, but they are integers. We can write this by:
            \begin{equation}
                \mathbb{N}\subsetneq\mathbb{Z}
            \end{equation}
            Similarly, $\mathbb{Q}$ contains numbers that are not integers,
            for example $1/2$. Thus, $\mathbb{Z}$ is also a proper subset of
            $\mathbb{Q}$. Lastly, since $\sqrt{2}$ is not a rational number, the
            set of rational numbers must then be a proper subset of the set of
            real numbers.
        \end{example}
        We now introduce the \textit{axiom schema of specification}.
        \begin{faxiom}{Axiom Schema of Specification}
                      {Axiom_Schema_of_Specification}
            If $A$ is a set and if $P$ is a proposition, then there exists a set
            $B$ such that $x\in{B}$ if and only if $x\in{A}$ and $P(x)$ is true.
            We can write this as:\index{Axiom!Schema of Specification}
            \begin{equation*}
                B=\big\{\,x\in{A}\;|\;P(x)\,\big\}
            \end{equation*}
            Using our formal language, we have:
            \begin{equation*}
                \forall_{A}\forall_{P}\exists_{B}:
                \forall_{x}\Big(x\in{B}\Leftrightarrow
                \big(x\in{A}\land{P}(x)\big)\Big)
            \end{equation*}
        \end{faxiom}
        Ax.~\ref{ax:Axiom_Schema_of_Specification} is different from the
        inconsistent axiom of unrestricted comprehension%
        \index{Axiom!of Unrestricted Comprehension} in that we can only speak of
        elements that are already defined and contained in some other set. That
        is, this new axiom does not allow us to talk about the
        \textit{set of all sets}\index{Set!of All Sets}, and so we have avoided
        the crux of Russell's paradox.
        \par\hfill\par
        This allows us to use the Set-Builder method of constructing sets. We
        loosely defined then natural numbers $\mathbb{N}$ and the integers
        $\mathbb{Z}$ (From the German \textit{Zahl}) by
        Eqns.~\ref{eqn:Natural_Numbers_Ellipses} and
        \ref{eqn:Integers_Ellipses}, respectively. It would be more
        difficult (but not impossible) to describe the set of rational numbers%
        \index{Rational Numbers} in such a way. Instead, we use set builder
        notation if it is known that $\mathbb{Q}$ is contained in some larger
        set $\mathbb{R}$ (the \textit{real} numbers)\index{Real Numbers}.
        \begin{equation}
            \mathbb{Q}=\Big\{\;\frac{p}{q}\in\mathbb{R}\;\big|\;
                                p,\,q\in\mathbb{Z}\textrm{ and }q\ne{0}\;\Big\}
        \end{equation}
        That is, the rational numbers are the set of all real numbers which can
        be written as the ratios of integers with non-zero denominator. The
        Axiom Schema of Specification states that this is is a valid method of
        describing sets. It is also known as the axiom of separation%
        \index{Axiom!of Separation}.
        \begin{example}
            We can describe the sets $\mathbb{Z}$, $\mathbb{N}$,
            $\mathbb{N}_{e}$, and $\mathbb{N}_{o}$ using set-builder notation if
            we assume these belong to some larger set $\mathbb{R}$. We define
            $\mathbb{Z}$ by:
            \index{Natural Numbers}\index{Even Integers}\index{Odd Integers}%
            \index{Integers}
            \begin{equation}
                \mathbb{Z}=
                \big\{\,n\in\mathbb{R}\;|\;n\textrm{ is an integer}\,\big\}
            \end{equation}
            From here we can define $\mathbb{N}$ by:
            \begin{equation}
                \mathbb{N}=\{\,n\in\mathbb{Z}\;|\;n\geq{0}\,\}
            \end{equation}
            Furthermore, $\mathbb{N}_{e}$ and $\mathbb{N}_{0}$ can be described
            as follows:
            \par
            \begin{subequations}
                \begin{minipage}[b]{0.495\textwidth}
                    \centering
                    \begin{equation}
                        \label{eqn:Even_Pos_Ints_Set_Builder}%
                        \mathbb{N}_{e}=
                        \big\{n\in\mathbb{N}\;|\;n\textrm{ is even}\big\}
                    \end{equation}
                \end{minipage}
                \hfill
                \begin{minipage}[b]{0.495\textwidth}
                    \centering
                    \begin{equation}
                        \label{eqn:Odd_Pos_Ints_Set_Builder}%
                        \mathbb{N}_{o}=
                        \big\{n\in\mathbb{N}\;|\;n\textrm{ is odd}\big\}
                    \end{equation}
                \end{minipage}
            \end{subequations}
            \par\vspace{2.5ex}
            Such notation is justified by the axiom schema of specification%
            \index{Axiom!Schema of Specification}.
        \end{example}
        We are not adopting these definitions since they lack rigor. These
        examples build intuition behind the notation and the axioms, but we will
        develop arithmetic from an axiomatic viewpoint.
    \subsection{Ordered Pairs and Unions}
        We now wish to solve the issue previously raised that sets do
        not have order. We'll develop a new object, called ordered pairs%
        \index{Ordered Pair}, that can distinguish such things. The definition
        we'll adopt is due to Kuratowski\index{Kuratowski, Kazimierz} and uses
        the following form:
        \begin{equation}
            (a,\,b)=\Big\{\,\big\{\,a\,\big\},\,\big\{\,a,\,b\,\big\}\,\Big\}
        \end{equation}
        We now prove such a set exists within the framework of ZFC.
        \begin{faxiom}{Axiom of Pairing}{Axiom_of_Pairing}
            If $A$ and $B$ are sets, then there exists a set $\mathcal{C}$
            such that $A\in\mathcal{C}$ and $B\in\mathcal{C}$.
            \index{Axiom!of Pairing}
            \begin{equation*}
                \forall_{A}\forall_{B}\exists_{\mathcal{C}}:
                (A\in\mathcal{C}\land{B}\in\mathcal{C})
            \end{equation*}
        \end{faxiom}
        The set hypothesized to exist in this axiom may be very large, we have
        no way of knowing. What we want from this is a set that contains two
        elements $A$ and $B$, and only those elements. We obtain this by
        combining pairing with specification.
        \begin{theorem}
            \label{thm:Existence_of_Set_Built_from_Two_Sets}%
            If $A$ and $B$ are sets, then there exists a set $D$ such that,
            for all $x$ it is true that $x\in{D}$ if and only if $x=A$ or
            $x=B$. That is:
            \begin{equation}
                D=\{\,A,\,B\,\}
            \end{equation}
        \end{theorem}
        \begin{proof}
            By the axiom of pairing (Ax.~\ref{ax:Axiom_of_Pairing}) there
            exists a set $\mathcal{C}$ such that $A\in\mathcal{C}$ and
            $B\in\mathcal{C}$. Let $P$ be the proposition
            \textit{true if} $x=A$ \textit{or} $x=B$, \textit{false otherwise}.
            By the axiom schema of specification
            (Ax.~\ref{ax:Axiom_Schema_of_Specification}), there is a set
            $D$ such that $x\in{D}$ if and only if $x\in\mathcal{C}$ and
            $P(x)$ is true. But then $x\in{D}$ if and only if
            $x\in\mathcal{C}$ and $x=A$ or $x\in\mathcal{C}$ and $x=B$.
            But $A\in\mathcal{C}$ and $B\in\mathcal{C}$, and thus
            $x\in{D}$ if and only if $x=A$ or $x=B$.
        \end{proof}
        By the axiom of extensionality\index{Axiom!of Extensionality}
        (Ax.~\ref{ax:Axiom_of_Extensionality}), the set hypothesized in
        Thm.~\ref{thm:Existence_of_Set_Built_from_Two_Sets} is unique, and thus
        there is no trouble in \textit{defining} the symbol $\{A,B\}$ to be the
        unique set that contains the elements $A$ and $B$ and only those
        elements. That is, we develop the new notation:
        \begin{fnotation}{Finite Set Notation}{Finite_Set_Notation}
            If $A$ and $B$ are sets, then $\{A,B\}$ is the unique set such that
            for all $x$, $x\in\{A,B\}$ if and only if $x=A$ or $x=B$.
            \begin{equation*}
                \forall_{x}\Big(\big(x\in\{A,B\}\big)
                \Leftrightarrow\big((x=A)\lor(x=B)\big)\Big)
            \end{equation*}
        \end{fnotation}
        \begin{theorem}
            \label{thm:Existence_of_Set_Containing_Set}%
            If $A$ is a set, then there is a set $B$ such that $x\in{B}$ if
            and only if $x=A$. That is, there exists a set $B$ such that:
            \begin{equation}
                B=\{\,A\,\}
            \end{equation}
        \end{theorem}
        \begin{proof}
            For since $A$ is a set, by
            Thm.~\ref{thm:Existence_of_Set_Built_from_Two_Sets} there exists
            a set $B=\{A,\,A\}$. But then $x\in{B}$ if and only if $x=A$.
        \end{proof}
        We can apply Not.~\ref{not:Finite_Set_Notation} to a single set $A$ and
        similarly define what the notation $\{A\}$ means. With this, we can now
        prove the existence of ordered pairs.
        \begin{ltheorem}{Existence of Ordered Pairs}{Existence_of_Ordered_Pairs}
            If $A$ and $B$ are sets, then there is a set $(A,\,B)$ such that,
            for all $x$ it is true that $x\in(A,\,B)$ if only if $x=\{A\}$
            or $x=\{A,B\}$.
        \end{ltheorem}
        \begin{proof}
            For by Thm.~\ref{thm:Existence_of_Set_Containing_Set}, there is
            a set $\{A\}$ such that $x\in\{A\}$ if and only if $x=A$.
            But by Thm.~\ref{thm:Existence_of_Set_Built_from_Two_Sets}, there
            is a set $\{A,\,B\}$ such that $x\in\{A,\,B\}$ and if only
            if $x=A$ or $x=B$. But again by
            Thm.~\ref{thm:Existence_of_Set_Built_from_Two_Sets}, since
            $\{A\}$ and $\{A,\,B\}$ are sets, there is a set $(A,\,B)$ such
            that $x\in(A,\,B)$ if and only if $x=\{A\}$ or $x=\{A,\,B\}$.
        \end{proof}
        Thm.~\ref{thm:Existence_of_Ordered_Pairs} asserts the existence of
        ordered pairs\index{Ordered Pair}, as defined by Kuratowski, and allows
        us to provide the following definition.
        \begin{fdefinition}{Ordered Pairs}{Ordered_Pairs}
            The \gls{ordered pair} of a \gls{set} $x$ with respect to a set
            $y$ is the set:\index{Ordered Pair}
            \begin{equation*}
                (x,\,y)=\big\{\,\{\,x\,\},\,\{\,x,\,y\,\}\,\big\}
            \end{equation*}
        \end{fdefinition}
        Kuratowski\index{Kuratowski, Kazimierz} first put forward this definition
        in 1921 and this does precisely what we want it to do and orders
        elements. That is, if $x$ and $y$ are distinct, then
        $(x,\,y)\ne(y,\,x)$. The caveat with this definition is the following
        reduction:
        \begin{equation}
            (x,\,x)
            =\big\{\,\{\,x\,\},\,\{\,x,\,x\,\}\,\big\}
            =\big\{\,\{\,x\,\},\{\,x\,\}\,\big\}
            =\big\{\,\{\,x\,\}\,\big\}
        \end{equation}
        Prior to Kuratowski there existed a definition due to Norbert Wiener%
        \index{Wiener, Norbert}, put forward in 1914. His definition grew out of
        Bertrand Russell's\index{Russell, Bertrand} Type
        Theory\index{Type Theory} which was an attempt to rid set theory of the
        paradoxes he discovered. Wiener writes:
        \begin{equation}
            (x,\,y)_{W}=\Big\{\,\big\{\,\{\,x\,\},\,\emptyset\,\big\},\,
                                \big\{\,\{\,y\,\}\,\big\}\Big\}
        \end{equation}
        Returning to Kuratowski's definition (Def.~\ref{def:Ordered_Pairs}),
        consider the ordered pair $(1,\,2)$, where we take for granted that
        $1\ne{2}$. We have:
        \begin{equation}
            (1,\,2)=\big\{\,\{\,1\,\},\,\{\,1,\,2\,\}\,\big\}
        \end{equation}
        Swapping and computing $(2,\,1)$, we obtain:
        \begin{equation}
            (2,\,1)=\big\{\,\{\,2\,\},\,\{\,2,\,1\,\}\,\big\}
        \end{equation}
        We know that sets cannot distinguish order, so
        $\{\,1,\,2\,\}=\{\,2,\,1\,\}$. Thus:
        \par
        \begin{subequations}
            \begin{minipage}[b]{0.49\textwidth}
                \begin{equation}
                    (1,\,2)=\big\{\,\{\,1\,\},\,\{\,1,\,2\,\}\,\big\}
                \end{equation}
            \end{minipage}
            \hfill
            \begin{minipage}[b]{0.49\textwidth}
                \begin{equation}
                    (2,\,1)=\big\{\,\{\,2\,\},\,\{\,1,\,2\,\}\,\big\}
                \end{equation}
            \end{minipage}
        \end{subequations}
        \par\vspace{2.5ex}
        Combining these equations, we now have that:
        \begin{equation}
            (1,\,2)\ne(2,\,1)
        \end{equation}
        To see this, note that both sets contain the element $\{1,\,2\}$, but
        $\{1\}$ is an element of $(1,\,2)$ and not an element of $(2,\,1)$,
        and thus $(1,\,2)\nsubseteq(2,\,1)$. Similarly, $\{2\}$ is an element
        of $(2,\,1)$ but not an element of $(1,\,2)$, and therefore
        $(2,\,1)\nsubseteq(1,\,2)$. From the definition of equality
        (Def.~\ref{def:Equal_Sets}), we have that these sets are not equal.
        \par\hfill\par
        There's is an unfortunate doubling of notation that occurs in
        mathematics, and $(a,b)$ has two common meanings. The first meaning is
        the ordered pair which we've just defined, and the second is the
        \textit{open interval}\index{Open Interval} defined in the context of a
        \textit{partially ordered set}\index{Partially Ordered Set}.
        The most common example is when discussing the real numbers
        $\mathbb{R}$, $(a,b)$ denotes the set of all real numbers $x$ such that
        $a<x$ and $x<b$. Hopefully it will be clear what the notation means when
        a theorem or example is being presented, but we will be explicit when
        ambiguity can arise.
        \par\hfill\par
        The natural thing from here is to construct the
        \textit{Cartesian Product}\index{Cartesian Product} of two sets. This is
        the set of all ordered pairs\index{Ordered Pair} $(a,\,b)$ where $a$
        belongs to some set $A$ and $b$ belongs to another set $B$. To prove
        such a set exists requires two more axioms.
        \begin{faxiom}{Axiom of Union}{Axiom_of_Union}
            If $\mathcal{O}$ is a set, then there exists a set $\mathcal{F}$
            such that, for all $A$ such that $A\in\mathcal{O}$ and for all
            $x$ such that $x\in{A}$, it is true that $x\in\mathcal{F}$.%
            \index{Axiom!of Union}
            \begin{equation*}
                \forall_{\mathcal{O}}\exists_{\mathcal{F}}\forall_{x}:
                \big(\exists_{A\in\mathcal{O}}(x\in{A})\big)
                \Rightarrow{x}\in\mathcal{F}
            \end{equation*}
        \end{faxiom}
        This states that, given a collection of sets $\mathcal{O}$, there exists
        a larger set which contains the elements of the constituent sets of
        $\mathcal{O}$. One question that arises is
        \textit{what happens if our collection is empty}? That is, if
        $\mathcal{O}=\emptyset$, is there any meaning behind the equation
        (the notation $\cup$ is defined in Def.~\ref{def:Union_over_a_Set}):
        \begin{equation}
            \mathcal{F}=\bigcup_{\mathcal{U}\in\mathcal{O}}\mathcal{U}
        \end{equation}
        There is, and $\mathcal{F}$ will be the empty set. That is,
        $\mathcal{F}=\emptyset$. This is true in a vacuous sense and can be
        proved via contradiction with the law of the excluded middle.
        \begin{ltheorem}{Existence of the Union of Sets}{Existence_of_Unions}
            If $\mathcal{O}$ is a set, then there exists a set $\mathcal{F}$
            such that, for all $x$ it is true that $x\in\mathcal{F}$ if and
            only if there is a set $A\in\mathcal{O}$ such that $x\in{A}$.
        \end{ltheorem}
        \begin{proof}
            For by the axiom of union (Ax.~\ref{ax:Axiom_of_Union}), there
            exists a set $\mathcal{A}$ such that, for all $A\in\mathcal{O}$
            and for all $x\in{A}$, it is true that $x\in\mathcal{A}$. Let
            $P$ be the proposition \textit{true if there exists a set}
            $A\in\mathcal{O}$ \textit{such that} $x\in{A}$,
            \textit{false otherwise}. Then, by the axiom schema of specification
            (Ax.~\ref{ax:Axiom_Schema_of_Specification}) there exists a set
            $\mathcal{F}$ such that:
            \begin{equation}
                \mathcal{F}=\big\{\,x\in\mathcal{A}\;|\;P(x)\,\big\}
            \end{equation}
            But then $x\in\mathcal{F}$ if and only if $x\in\mathcal{A}$ and
            $P(x)$ is true. But if $P(x)$ is true, then $x\in\mathcal{A}$, and
            thus $x\in\mathcal{F}$ if and only if there is a set
            $A\in\mathcal{O}$ such that $x\in{A}$.
        \end{proof}
        We define the set $\mathcal{F}$ described in
        Thm.~\ref{thm:Existence_of_Unions} as the \textit{union}%
        \index{Union (Sets)} over the set $\mathcal{O}$. This set is often
        called the index set\index{Index Set} for which we take the union over.
        \begin{fdefinition}{Union over a Set}{Union_over_a_Set}
            The \gls{union over a set} $\mathcal{O}$ is the set:
            \index{Union (Sets)}
            \begin{equation*}
                \bigcup_{\mathcal{U}\in\mathcal{O}}\mathcal{U}
                =\big\{\,x\;|\;\textrm{There exists a set }
                         \mathcal{U}\in\mathcal{O}\textrm{ such that }
                         x\in\mathcal{U}\big\}
            \end{equation*}
            Using our formal language:
            \begin{equation*}
                \forall_{x}\Big(
                    x\in\bigcup_{\mathcal{U}\in\mathcal{O}}\mathcal{U}
                    \Leftrightarrow\exists_{\mathcal{U}\in\mathcal{O}}:
                    x\in\mathcal{U}\Big)
            \end{equation*}
        \end{fdefinition}
        This is very convenient if we have a collection of sets defined, but it
        would be nice to form the union over two given sets without considering
        them as part of a larger collection. This can be done by combining the
        axiom of union\index{Axiom!of Union} with
        pairing\index{Axiom!of Pairing}.
        \begin{theorem}
            \label{thm:Union_of_Two_Sets_Exists}%
            If $A$ and $B$ are sets, then there exists a set $C$ such that
            $x\in{C}$ if and only if either $x\in{A}$ or $x\in{B}$.
        \end{theorem}
        \begin{proof}
            For by Thm.~\ref{thm:Existence_of_Set_Built_from_Two_Sets},
            there exists a set $\mathcal{O}$ such that $y\in\mathcal{O}$ if and
            only if $y=A$ or $y=B$. That is, $\mathcal{O}=\{A,\,B\}$.
            But by Thm.~\ref{thm:Existence_of_Unions} there exists a set
            $C$ such that $x\in{C}$ if and only if there exist a set
            $F\in\mathcal{O}$ such that $x\in{F}$. But then $x\in{C}$ if and
            only if either $x\in{A}$ or $x\in{B}$.
        \end{proof}
        This allows us to define our first \textit{operation} of two sets.
        \begin{fdefinition}{Union of Two Sets}{Union_of_Two_Sets}
            The \gls{union of two sets} $A$ and $B$ is the set $A\cup{B}$
            defined by:
            \begin{equation*}
                A\cup{B}=\big\{\,x\;|\;x\in{A}\textrm{ or }x\in{B}\,\big\}
            \end{equation*}
        \end{fdefinition}
        In our definition of the union over a set and the union of two sets
        we have slightly abused our set-builder notation. The axiom schema
        of specification allows us to write a set as
        $A=\{\,x\in{B}\,|\,P(x)\,\}$, given some set $B$ that is already known
        to exists, and some proposition $P$. These two definitions
        (Def.~\ref{def:Union_over_a_Set} and \ref{def:Union_of_Two_Sets})
        are justified by the theorems we have proven, and so there is no
        contradiction.
        \begin{example}
            Again using the notation found in Eqn.~\ref{eqn:Z_n_Ellipses}, if we
            let $\mathbb{Z}_{n}$ denote the integers between $0$ and $n-1$, we
            have the following: If $n$ is less than $m$, then:
            \begin{equation}
                \mathbb{Z}_{n}\cup\mathbb{Z}_{m}=\mathbb{Z}_{m}
            \end{equation}
            This is because every element of $\mathbb{Z}_{n}$ is already
            and element of $\mathbb{Z}_{m}$, and thus taking the union adds
            nothing new to $\mathbb{Z}_{m}$, so the resulting set is
            $\mathbb{Z}_{m}$ itself.
        \end{example}
        \begin{example}
            Denoting the even and odd non-negative integers by $\mathbb{N}_{e}$
            and $\mathbb{N}_{o}$, respectively, we see that:
            \begin{equation}
                \mathbb{N}_{e}\cup\mathbb{N}_{o}=\mathbb{N}
            \end{equation}
            This is because every non-negative integer $n\in\mathbb{N}$ is
            either even or odd, and thus either $n\in\mathbb{N}_{e}$ or
            $n\in\mathbb{N}_{o}$. Taking the union therefore gives the entire
            set $\mathbb{N}$. The union does not add anything more than
            $\mathbb{N}$ since $\mathbb{N}_{e}\subseteq\mathbb{N}$ and
            $\mathbb{N}_{o}\subseteq\mathbb{N}$.
        \end{example}
        \begin{fexample}{Union of Two Sets}{Union_of_Two_Sets}
            Let $A$ and $B$ be the sets defined by:
            \par\hfill\par
            \begin{subequations}
                \begin{minipage}[b]{0.49\textwidth}
                    \begin{equation}
                        A=\{\,a,\,b,\,c\,\}
                    \end{equation}
                \end{minipage}
                \hfill
                \begin{minipage}[b]{0.49\textwidth}
                    \begin{equation}
                        B=\{\,c,\,1,\,2\,\}
                    \end{equation}
                \end{minipage}
            \end{subequations}
            \par\vspace{2.5ex}
            The union of $A$ and $B$ is the set that contains all of the
            elements of $A$ and all of the elements of $B$, and only such
            elements. That is:
            \begin{equation}
                A\cup{B}=\{\,a,\,b,\,c,\,1,\,2\,\}
            \end{equation}
            Even though $c\in{A}$ and $c\in{B}$, $c$ only appears once in the
            union. This is because sets cannot account for repetition, so
            including $c$ twice would be redundant.
        \end{fexample}
        The union of two sets can again be visualized by considering blobs
        in the plane. Let $A$ and $B$ be two circles that overlap somewhere in
        the middle. The union $A\cup{B}$ can then be represented by shading in
        the region covered by either $A$ or $B$
        (see Fig.~\ref{fig:Union_of_Two_Sets}). Such a drawing is called a
        \textit{Venn diagram}.
        \begin{figure}[H]
            \centering
            \captionsetup{type=figure}
            \documentclass[crop,class=article]{standalone}
%----------------------------Preamble-------------------------------%
\usepackage{tikz}                       % Drawing/graphing tools.
%--------------------------Main Document----------------------------%
\begin{document}
    \begin{tikzpicture}
        \draw (-2.5,-2) rectangle (2.5,2);
        \fill[cyan] (-0.8cm,0) circle (1.5cm);
        \fill[cyan] (0.8cm,0) circle (1.5cm);
        \draw (-0.8cm,0) circle (1.5cm);
        \draw (0.8cm,0) circle (1.5cm);
        \node at (-1,1.1) {$A$};
        \node at (1,1.1) {$B$};
        \node at (-1,1.8) {$A\cup{B}$};
    \end{tikzpicture}
\end{document}
            \caption{The Union of Two Sets}
            \label{fig:Union_of_Two_Sets}
        \end{figure}
        Fig.~\ref{fig:Union_of_Two_Sets} can be extended to an
        arbitrary collection of sets. For the sake of simplicity, a Venn
        diagram for the union of three sets is shown in
        Fig.~\ref{fig:Union_of_Three_Sets}. We can combine the axiom schema of
        specification (Ax.~\ref{ax:Axiom_Schema_of_Specification}) with the
        existence of the union of two sets to define intersections.
        \begin{figure}[H]
            \centering
            \captionsetup{type=figure}
            \begin{tikzpicture}
    % Coordinates for the centers of the circles.
    \coordinate (C1) at (-1.3,  0.00);
    \coordinate (C2) at ( 1.3,  0.00);
    \coordinate (C3) at ( 0.0, -2.15);

    % Coordinates for the labels.
    \coordinate (A) at (-1.3, 1.2);
    \coordinate (B) at ( 1.3, 1.2);
    \coordinate (C) at ( 0.0, -3.5);
    \coordinate (U) at ( 0.0, 2.5);

    % Rectangle indicating the universe set.
    \draw (-4, -4.5) rectangle (4, 3.2);

    % Fill in the circle with cyan.
    \draw[fill=cyan, draw=none] (C1) circle (2);
    \draw[fill=cyan, draw=none] (C2) circle (2);
    \draw[fill=cyan, draw=none] (C3) circle (2);

    % Give outlines to the circles.
    \draw (C1) circle (2);
    \draw (C2) circle (2);
    \draw (C3) circle (2);

    % Labels.
    \node at (A) {$A$};
    \node at (B) {$B$};
    \node at (C) {$C$};
    \node at (U)
        {\Large{$\underset{{\mathcal{U}\in\{A,B,C\}}}{\bigcup}\mathcal{U}$}};
\end{tikzpicture}
            \caption{The Union of Three Sets}
            \label{fig:Union_of_Three_Sets}
        \end{figure}
        \begin{theorem}
            If $A$ and $B$ are sets, then there exists a set $\mathcal{C}$
            such that, for all $x$ it is true that $x\in\mathcal{C}$ if and
            only if $x\in{A}$ and $x\in{B}$.
        \end{theorem}
        \begin{proof}
            For by Thm.~\ref{thm:Union_of_Two_Sets_Exists}, there exists
            a set $A\cup{B}$ such that, for all $x$ is is true that
            $x\in{A}\cup{B}$ if and only if $x\in{A}$ or $x\in{B}$. Let
            $P$ be the proposition \textit{True if} $x\in{A}$ \textit{and}
            $x\in{B}$, \textit{false otherwise}. Then by the axiom schema
            of specification (Ax.~\ref{ax:Axiom_Schema_of_Specification})
            there is a set $\mathcal{C}$ such that:
            \begin{equation}
                \mathcal{C}=\big\{\,x\in{A}\cup{B}\;|\;P(x)\,\big\}
            \end{equation}
            But then $x\in\mathcal{C}$ if and only if $x\in{A}\cup{B}$ and
            $x\in{A}$ and $x\in{B}$. But if $x\in{A}$ and $x\in{B}$, then
            $x\in{A}$, and thus $x\in{A}\cup{B}$
            (Def.~\ref{def:Union_of_Two_Sets}). That is, $P(x)$ implies that
            $x\in{A}\cup{C}$. Therefore, $x\in\mathcal{C}$
            if and only if $x\in{A}$ and $x\in{B}$.
        \end{proof}
        \begin{fdefinition}{Intersection of Two Sets}
                           {Intersection_of_Two_Sets}
            The \gls{intersection of two sets} $A$ and $B$ is the set:
            \begin{equation*}
                A\cap{B}
                =\big\{\,x\in{A}\cup{B}\;|\;
                    a\in{A}\textrm{ and }b\in{B}\,\big\}
            \end{equation*}
        \end{fdefinition}
        We'll need one brief theorem about intersections to allow us to prove
        that certain sets are not equal.
        \begin{theorem}
            \label{thm:Lemma_for_Anti_Russells_Paradox}%
            If $A$ and $B$ are sets, if $x\in{B}$, and if $x\notin{A}\cap{B}$,
            then $x\notin{A}$.
        \end{theorem}
        \begin{proof}
            For if $x\notin{A}\cap{B}$ then either $x\notin{A}$ or $x\notin{B}$
            (Def.~\ref{def:Intersection_of_Two_Sets}). But $x\in{B}$, and
            therefore $x\notin{A}$.
        \end{proof}
        \begin{fexample}{Intersections of Two Sets}
                        {Intersections_of_Two_Sets}
            If we let $A$ and $B$ be the sets defined by:
            \par\hfill\par
            \begin{subequations}
                \begin{minipage}[b]{0.49\textwidth}
                    \begin{equation}
                        A=\{\,a,\,b,\,c\,\}
                    \end{equation}
                \end{minipage}
                \hfill
                \begin{minipage}[b]{0.49\textwidth}
                    \begin{equation}
                        B=\{\,c,\,1,\,2\,\}
                    \end{equation}
                \end{minipage}
            \end{subequations}
            \par\vspace{2.5ex}
            we have that the intersection is then:
            \begin{equation}
                A\cap{B}=\{\,1\,\}
            \end{equation}
            Letting $\mathbb{N}_{e}$ and $\mathbb{N}_{o}$ once again denote
            the even and odd non-negative integers, respectively, we have:
            \begin{equation}
                \mathbb{N}_{e}\cap\mathbb{N}_{o}=\emptyset
            \end{equation}
            This is precisely what was meant earlier when it was claimed that
            $\mathbb{N}_{e}$ and $\mathbb{N}_{o}$ are disjoint%
            \index{Disjoint Sets}. If $n,m\in\mathbb{N}$ and if $n<m$, we have:
            \begin{equation}
                \mathbb{Z}_{n}\cap\mathbb{Z}_{m}=\mathbb{Z}_{n}
            \end{equation}
            This is because every element of $\mathbb{Z}_{n}$ is an element
            of $\mathbb{Z}_{m}$. This creates a more general theorem that if
            $A\subseteq{B}$, then $A\cap{B}=A$. Intersections thus seem like
            antonyms of unions.
        \end{fexample}
        Similar to how unions can be visualized with Venn diagrams
        (Fig.~\ref{fig:Union_of_Two_Sets}), so can the intersection of
        two sets. We draw two circles that overlap slightly, and consider the
        region contained in both (see Fig.~\ref{fig:Intersection_of_Two_Sets}).
        \begin{figure}[H]
            \centering
            \documentclass[crop,class=article]{standalone}
%----------------------------Preamble-------------------------------%
\usepackage{tikz}                       % Drawing/graphing tools.
%--------------------------Main Document----------------------------%
\begin{document}
    \begin{tikzpicture}
        \draw (-2.5,-2) rectangle (2.5,2);
        \draw (-0.8cm,0) circle (1.5cm);
        \draw (0.8cm,0) circle (1.5cm);
        \draw[fill=cyan]
            (0,-1.26886) arc(-57.77:57.77:1.5)
                         arc(122.231:237.7690:1.5);
        \node at (-1,1.1) {$A$};
        \node at (1,1.1) {$B$};
        \node at (-1,1.8) {$A\cap{B}$};
    \end{tikzpicture}
\end{document}
            \caption{Venn Diagram for Intersection}
            \label{fig:Intersection_of_Two_Sets}
        \end{figure}
        We can extend this further and define the intersection over any
        collection of sets.
        \begin{theorem}
            \label{thm:Existence_of_Arbitrary_Intersetions}%
            If $\mathcal{O}$ is a set, then there exists a set
            $\mathcal{C}$ such that, for all $x$ it is true that
            $x\in\mathcal{C}$ if and only if for all
            $\mathcal{U}\in\mathcal{O}$ it is true that $x\in\mathcal{U}$.
            That is:
            \begin{equation}
                \mathcal{C}=\bigcap_{\mathcal{U}\in\mathcal{O}}\mathcal{U}
            \end{equation}
        \end{theorem}
        \begin{proof}
            For by Thm.~\ref{thm:Existence_of_Unions} there is a set
            $\bigcup\mathcal{U}$ such that for all $x$ it is true that
            $x\in\bigcup\mathcal{U}$ if and only if for all
            $\mathcal{U}\in\mathcal{O}$ it is true that $x\in\mathcal{U}$. Let
            $P$ be the proposition \textit{True if for all}
            $\mathcal{U}\in\mathcal{O}$ \textit{it is true that}
            $x\in\mathcal{U}$, \textit{false otherwise}. Then by the
            axiom schema of specification
            (Ax.~\ref{ax:Axiom_Schema_of_Specification}), there exists the set:
            \begin{equation}
                \mathcal{C}
                =\Big\{\,x\in\bigcup_{\mathcal{U}\in\mathcal{O}}\mathcal{U}
                    \;|\;P(x)\,\Big\}
            \end{equation}
            But then $x\in\mathcal{C}$ if and only if $x\in\bigcup\mathcal{U}$
            and $P(x)$ is true. But if $P(x)$ is true, then for all
            $\mathcal{U}\in\mathcal{O}$ it is true that $x\in\mathcal{U}$. But
            then there is a $\mathcal{U}\in\mathcal{O}$ such that
            $x\in\mathcal{U}$, and thus $P(x)$ implies that
            $x\in\bigcup\mathcal{U}$. Therefore $x\in\mathcal{C}$ if and only if
            for all $\mathcal{U}\in\mathcal{O}$ it is true that
            $x\in\mathcal{U}$.
        \end{proof}
        It is common to consider some \textit{universal} set, of which all
        other sets of current consideration are drawn from. Using this the
        definition of arbitrary intersection is defined as the subset of
        this universal set such that every element of this subset is
        contained in every element of the arbitrary collection. One may then
        ask what would happen if the collection is empty. Using this
        definition the intersection would be the entire universal set
        in a vacuous sense. That is, there would be no $x$ in the universe
        that fails the definition of the intersection over an empty
        collection, and thus the intersection is everything. Letting $X$
        denote our universe, we obtain:
        \begin{equation}
            \emptyset=\bigcup_{\mathcal{U}\in\emptyset}\mathcal{U}
                \subseteq\bigcap_{\mathcal{U}\in\emptyset}\mathcal{U}
            =X
        \end{equation}
        It seems like unions should always be bigger. Indeed, for any
        non-empty collection, the intersection over the collection is a
        subset of the union over the collection. Because of this we do
        not adopt this definition of the intersection over a collection,
        but rather require in our construction the use of the union over
        the collection, and then use the axiom schema of specification to
        pick the subset of all elements of the union that belong to every
        element of the collection. Thus:
        \begin{equation}
            \bigcap_{\mathcal{U}\in\emptyset}\mathcal{U}
            \subseteq\bigcup_{\mathcal{U}\in\emptyset}\mathcal{U}
            =\emptyset
        \end{equation}
        And from this we conclude the intersection is empty as well.
        \begin{fdefinition}{Intersection Over a Collection}
                           {Intersection_Over_a_Collection}
            The \gls{intersection over a set} $\mathcal{O}$
            of sets is the set:
            \begin{equation*}
                \bigcap_{E\in\mathcal{O}}E
                =\Big\{\,x\in\bigcup_{E\in\mathcal{O}}E\;\Big|\;x\in{E}
                    \textrm{ for all }E\in\mathcal{O}\,\Big\}
            \end{equation*}
        \end{fdefinition}
        We can extend our Venn diagram for larger collections as well
        (see Fig.~\ref{fig:Intersection_of_Three_Sets}).
        \begin{figure}[H]
            \centering
            \begin{tikzpicture}
    % Coordinates for the centers of the circles.
    \coordinate (C1) at (-1.3,  0.00);
    \coordinate (C2) at ( 1.3,  0.00);
    \coordinate (C3) at ( 0.0, -2.15);

    \coordinate (O)  at ( 0.0000, -0.7500);
    \coordinate (P1) at ( 0.6817, -0.2697);
    \coordinate (P2) at (-0.6817, -0.2697);
    \coordinate (P3) at ( 0.0000, -1.5198);

    % Coordinates for the labels.
    \coordinate (A) at (-1.3, 1.2);
    \coordinate (B) at ( 1.3, 1.2);
    \coordinate (C) at ( 0.0, -3.5);
    \coordinate (U) at ( 0.0, 2.5);

    % Rectangle indicating the universe set.
    \draw (-4, -4.5) rectangle (4, 3.3);

    % Fill in the intersection with cyan.
    \draw[fill=cyan, draw=none] (P1) arc(70.07:109.93:2)
                                     arc(187.8:229.50:2)
                                     arc(310.54:352.25:2);

    % Give outlines to the circles.
    \draw (C1) circle (2);
    \draw (C2) circle (2);
    \draw (C3) circle (2);

    % Labels.
    \node at (A) {$A$};
    \node at (B) {$B$};
    \node at (C) {$C$};
    \node at (U)
        {\Large{$\underset{\mathcal{U}\in\{A,B,C\}}{\bigcap}\mathcal{U}$}};
\end{tikzpicture}
            \caption{Intersection of Three Sets}
            \label{fig:Intersection_of_Three_Sets}
        \end{figure}
        Next on the list of axioms is that of \textit{regularity}.
        \begin{faxiom}{Axiom of Regularity}{Axiom_of_Regularity}
            If $A$ is a non-empty set, then there is an element $B\in{A}$
            such that $A\cap{B}=\emptyset$.\index{Axiom!of Regularity}
            \begin{equation*}
                \forall_{A}(\exists_{x\in{A}})\Rightarrow
                \exists_{y}:\Big((y\in{A})\land
                \big((y\cap{A})=\emptyset\big)\Big)
            \end{equation*}
        \end{faxiom}
        This axiom is often seen as unnecessary by many working mathematicians
        and indeed it's use seems to only lie in set theory and foundations.
        That is, unlike the axioms of choice and union which are widely
        applicable to analysis and topology, regularity seems to only be useful
        to set theoriests. Nevertheless, it is vital to support the claim that
        ZFC is a good system to base mathematics on. We will combine this with
        pairing to prove that for any set $A$ it is true that $A\notin{A}$. That
        is, Zermelo-Fraenkel set theory is free of Russell's
        paradox\index{Russell's Paradox}.
        \begin{theorem}
            \label{thm:Anti_Russells_Paradox}%
            If $A$ is a set, then $A\notin{A}$.
        \end{theorem}
        \begin{proof}
            For if $A$ is a set, then $\{A\}$ is a set
            (Thm.~\ref{thm:Existence_of_Set_Containing_Set}). But since
            $A\in\{A\}$, $\{A\}$ is a non-empty set
            (Def.~\ref{def:Non_Empty_Set}). Thus by the axiom of regularity
            (Ax.~\ref{ax:Axiom_of_Regularity}), there is a set $B\in\{A\}$ such
            that $B\cap\{A\}=\emptyset$. But $B\in\{A\}$ if and only if
            $B=A$, and therefore $A\cap\{A\}=\emptyset$. Thus, by the axiom of
            the empty set (Ax.~\ref{ax:Axiom_of_the_Empty_Set}), for all $x$ it
            is true that $x\notin{A}\cap\{A\}$ and therefore
            $A\notin{A}\cap\{A\}$. But $A\in\{A\}$ and therefore
            $A\notin{A}$ (Thm.~\ref{thm:Lemma_for_Anti_Russells_Paradox}).
        \end{proof}
        \begin{theorem}
            \label{thm:Containment_NEqual_Underlying_Set}%
            If $A$ and $B$ are sets and if $A\in{B}$, then $A\ne{B}$.
        \end{theorem}
        \begin{proof}
            For $A\notin{A}$ (Thm.~\ref{thm:Anti_Russells_Paradox}) and
            $A\in{B}$ and therefore it is not true that for all $x$, $x\in{A}$
            if and only if $x\in{B}$. Therefore, by the Axiom of
            Extensionality (Ax.~\ref{ax:Axiom_of_Extensionality}), $A\ne{B}$.
        \end{proof}
        \begin{theorem}
            If $A$ is a set, then $A\ne\{A\}$.
        \end{theorem}
        \begin{proof}
            \label{thm:Cor_of_Containment_NEqual_Underlying_Set}%
            For if $A$ is a set, then $A\notin{A}$
            (Thm.~\ref{thm:Anti_Russells_Paradox}). But if $A$ is a set, then
            $\{A\}$ is a set (Thm.~\ref{thm:Existence_of_Set_Containing_Set}).
            But $A\in\{A\}$, and thus $A\ne\{A\}$
            (Thm.~\ref{thm:Containment_NEqual_Underlying_Set}).
        \end{proof}
        This quick theorem will eventually prove the well known result that
        $0\ne{1}$. It also shows that there is no set of all
        sets\index{Set!of All Sets}.
    \subsection{The Axiom of the Power Set}
        Continuing in our goal of constructing order, we move on to the
        Cartesian product of two sets $A$ and $B$. This is the collection of
        all ordered pairs $(a,b)$ such that $a\in{A}$ and $b\in{B}$. To prove
        such a set exists requires the \textit{axiom of the power set}.
        \begin{faxiom}{Axiom of the Power Set}{Axiom_of_the_Power_Set}
            If $X$ is a set, then there exists a set $\mathscr{P}$ such that,
            for all $A\subseteq{X}$, it is true that $A\in\mathscr{P}$.
            \index{Axiom!of the Power Set}
            \begin{equation*}
                \forall_{X}\exists_{\mathscr{P}}:
                \forall_{A}(A\subseteq{X})\Rightarrow{A}\in\mathscr{P}
            \end{equation*}
        \end{faxiom}
        Again, much like the axiom of union and the axiom of pairing, this
        set may be bigger than we would like. We wish to find a set, called
        the \textit{power set}, the contains all of the subsets of a given
        set $A$ and nothing else. Combining the axiom of the power set
        with the axiom schema of specification gives us such existence.
        \begin{ltheorem}{Existence of the Power Set}
                        {Existence_of_the_Power_Set}
            If $A$ is a set, then there exists a set $\mathcal{P}(A)$
            such that, for all $x$ it is true that $x\in\mathcal{P}(A)$ if and
            only if $x\subseteq{A}$.
        \end{ltheorem}
        \begin{proof}
            For by the axiom of the power set
            (Ax.~\ref{ax:Axiom_of_the_Power_Set}), there is a set
            $\mathscr{P}(A)$ such that for all $x\subseteq{A}$ it is true
            that $x\in\mathscr{P}(A)$. Let $P$ be the proposition
            \textit{true if} $x\subseteq{A}$, \textit{false otherwise}. By the
            axiom schema of specification
            (Ax.~\ref{ax:Axiom_Schema_of_Specification}), there is a set
            $\mathcal{P}(A)$ such that:
            \begin{equation}
                \mathcal{P}(A)=\{\,x\in\mathscr{P}(A)\;|\;P(x)\,\}
            \end{equation}
            But if $P(x)$ is true, then $x$ is a subset of $A$, and therefore
            $x\in\mathscr{P}(A)$. Thus $x\in\mathcal{P}(A)$ if and only if
            $x\subseteq{A}$.
        \end{proof}
        With this we now define the \textit{power set} of a given set.
        \begin{fdefinition}{Power Set}{Power_Set}
            The \gls{power set} of a \gls{set} $X$ is the set $\mathcal{P}(X)$
            defined by:\index{Power Set}
            \begin{equation*}
                \mathcal{P}(X)=\{\,A\;|\;A\subseteq{X}\,\}
            \end{equation*}
            That is, the set of all subsets of $X$.
            \begin{equation*}
                \forall_{A}\forall_{B}\Big(B\in\mathcal{P}(A)\Longleftarrow
                    B\subseteq{A}\Big)
            \end{equation*}
        \end{fdefinition}
        Thm.~\ref{thm:Existence_of_the_Power_Set} justifies such a definition.
        The power set of a set is a crucial construction for when one discusses
        the \textit{cardinality} of sets, denoted $\textrm{Card}(A)$. This
        describes the \textit{size} of a set in a very precise manner. A theorem
        that will eventually be proved known as
        \textit{Cantor's Theorem}\index{Cantor's Power Set Theorem} shows that
        the power set of a set is always strictly \textit{larger} than the
        original set. That is:
        \begin{equation}
            \textrm{Card}(A)<\textrm{Card}\big(\mathcal{P}(A)\big)
        \end{equation}
        This will be made precise soon enough. The axiom of the power set allows
        us to build \textit{larger} sets from a given set. That is, given a set
        $A$ it is true that $A\ne\mathcal{P}(A)$ (otherwise we would violate the
        axiom of regularity). But since $A\in\mathcal{P}(A)$ we see that the
        power set is in a sense \textit{larger}.
        \begin{example}
            If $A=\{1,2\}$, then the power set is:
            \begin{equation}
                \mathcal{P}(A)=\big\{\,\emptyset,\,\{1\},\,\{2\},\,
                    \{1,2\}\,\big\}
            \end{equation}
            We must consider the empty set since $\emptyset\subseteq{A}$.
            Now suppose $A=\{1,2,3\}$:
            \begin{equation}
                \mathcal{P}(A)=\big\{\,\emptyset,\,\{1\},\,\{2\},\,\{3\},\,
                    \{1,2\},\,\{1,3\},\,\{2,3\},\,\{1,2,3\}\,\big\}
            \end{equation}
            We see that a set with 2 elements has a power set with 4 elements
            and a set with 3 elements has a power set with 8. This pattern
            continues for finite sets and if $A$ has $n$ elements, then
            $\mathcal{P}(A)$ has $2^{n}$ elements.
        \end{example}
        \begin{example}
            When we consider the case of an \textit{infinite} set $A$ we have
            that $\mathcal{P}(A)$ is a strictly larger set and this creates a
            paradoxical heirarchy of infinities. The smallest heirarchy is
            that of the \textit{countable} infinite sets, like $\mathbb{N}$.
            Everything larger is called \textit{uncountable}. It will be
            shown that the following is true:
            \begin{equation}
                \Card\big(\mathcal{P}(\mathbb{N})\big)=
                \Card(\mathbb{R})
            \end{equation}
            where again $\mathbb{N}$ denotes the non-negative integers and
            $\mathbb{R}$ denotes the set of all \textit{real} numbers. We
            can loosely show this by using the binary representation of real
            numbers. A real number may be thought of as an infinite decimal.
            For example, $\pi=3.1415926\dots$ and $1=1.000\dots$ We can
            also represent real numbers as a sequence of zeroes and ones and
            this is the \textit{binary} representation. For
            $A\subseteq\mathbb{N}$ and let $r_{A}=0.n_{1}n_{2}\hdots$ where:
            \begin{equation}
                n_{i}=
                \begin{cases}
                    0,&i\notin{A}\\
                    1,&i\in{A}
                \end{cases}
            \end{equation}
            Thus for each $A\in\mathcal{P}(\mathbb{N})$ there is a real
            number $r_{A}$ such that $0\leq{r}_{A}\leq{1}$ that is
            associated with it, and moreover to every real number between
            zero and one there is a subset of $\mathbb{N}$ associated with
            it. The tricky numbers to see are zero and one, but note that
            $r_{\emptyset}$ is associated to 0 and $r_{\mathbb{N}}$ gets
            paired with 1. To show that $\mathbb{R}$ and
            $\mathcal{P}(\mathbb{N})$ are the same size requires us to
            refine this association so that every element of
            $\mathcal{P}(\mathbb{N})$ uniquely corresponds to an element of
            $\mathbb{R}$, and vice-versa.
        \end{example}
    \subsection{Cartesian Products and Functions}
        In the previous section we introduced ordered pairs and the notion of
        the power set. We can use both of these concepts to define and prove
        the existence of \textit{Cartesian products}. Intuitively we want to
        define $A\times{B}$ to be the set of all ordered pairs $(a,b)$ where
        $a\in{A}$ and $b\in{B}$:
        \begin{equation}
            A\times{B}=\big\{\,(a,\,b)\;|\;a\in{A}\textrm{ and }b\in{B}\,\big\}
        \end{equation}
        But recalling Def.~\ref{def:Ordered_Pairs}, ordered pairs are sets
        of the form $\{\{a\},\{a,b\}\}$. Thus elements of $A\times{B}$ are
        contained in the power set of the power set of $A\cup{B}$:
        \begin{equation}
            A\times{B}\subseteq\mathcal{P}\big(\mathcal{P}(A\cup{B})\big)
        \end{equation}
        We can combine the axiom of the power set with the axiom schema of
        specification to obtain the existence of the Cartesian product of
        two sets.
        \begin{theorem}
            \label{thm:Ordered_Pair_Subset_of_Power_Set}%
            If $A$ and $B$ are sets, if $a\in{A}$ and $b\in{B}$, then
            $(a,b)\subseteq\mathcal{P}(A\cup{B})$.
        \end{theorem}
        \begin{proof}
            For if $a\in{A}$ and $b\in{B}$, then
            $(a,b)=\{\,\{\,a\,\},\,\{\,a,\,b\,\}\,\}$
            (Def.~\ref{def:Ordered_Pairs}). But if $a\in{A}$, then $a\in{A}$
            or $a\in{B}$, and thus $a\in{A}\cup{B}$
            (Def.~\ref{def:Union_of_Two_Sets}). But then
            $\{\,a\,\}\subseteq{A}\cup{B}$ (Def.~\ref{def:Subsets}). But if
            $b\in{B}$, then $b\in{A}$ or $b\in{B}$, and thus $b\in{A}\cup{B}$
            (Def.~\ref{def:Union_of_Two_Sets}). But then
            $\{\,a,\,b\,\}\subseteq{A}\cup{B}$ (Def.~\ref{def:Subsets}).
            But then $\{\,a\,\}\subseteq{A}\cup{B}$ and
            $\{\,a,\,b\,\}\subseteq{A}\cup{B}$, and thus
            $(a,b)\subseteq\mathcal{P}(A\cup{B})$ (Def.~\ref{def:Power_Set}).
        \end{proof}
        \begin{ltheorem}{Existence of the Cartesian Product}
                        {Existence_of_the_Cartesian_Product}
            If $A$ and $B$ are sets, then there exists a set $A\times{B}$
            such that, for all $z$, $z\in{A}\times{B}$ if and only if there
            is an $a\in{A}$ and $b\in{B}$ such that $z=(a,b)$.
        \end{ltheorem}
        \begin{proof}
            For if $A$ and $B$ are sets, then by the axiom of union
            (Ax.~\ref{ax:Axiom_of_Union}), $A\cup{B}$ is a set. But if
            $A\cup{B}$ is a set, by the axiom of the power set
            (Ax.~\ref{ax:Axiom_of_the_Power_Set}), $\mathcal{P}(A\cup{B})$ is
            a set, where $\mathcal{P}(X)$ denotes the power set of $X$. But if
            $\mathcal{P}(A\cup{B})$ is a set, then
            $\mathcal{P}(\mathcal{P}(A\cup{B}))$ is a set
            (Ax.~\ref{ax:Axiom_of_the_Power_Set}). But then
            $z\in\mathcal{P}(\mathcal{P}(A\cup{B}))$ if and only if
            $z\subseteq\mathcal{P}(A\cup{B})$ (Def.~\ref{def:Power_Set}).
            But if $a\in{A}$ and $b\in{B}$, then
            $(a,b)\subseteq\mathcal{P}(A\cup{B})$
            (Thm.~\ref{thm:Ordered_Pair_Subset_of_Power_Set}), and therefore
            $(a,b)\in\mathcal{P}(\mathcal{P}(A\cup{B}))$
            (Def.~\ref{def:Power_Set}). Let $P$ be the proposition
            \textit{True if there exists} $a\in{A}$ \textit{and}
            $b\in{B}$ \textit{such that} $z=(a,b)$, \textit{false otherwise}.
            Then by the axiom schema of specification
            (Ax.~\ref{ax:Axiom_Schema_of_Specification}), there exists a
            set $A\times{B}$ such that:
            \begin{equation}
                A\times{B}=
                \{\,z\in\mathcal{P}\big(\mathcal{P}(A\cup{B})\big)\;|\;
                    P(z)\,\}
            \end{equation}
            But it was proved that $P(z)$ implies that
            $z\in\mathcal{P}(\mathcal{P}(A\cup{B}))$. Thus $z\in{A}\times{B}$
            if and only if there exists $a\in{A}$ and $b\in{B}$
            such that $z=(a,b)$.
        \end{proof}
        \begin{fdefinition}{Cartesian Product of Two Sets}
                           {Cartesian_Product_of_Two_Sets}
            The \gls{Cartesian product} of two \glspl{set} $A$ and $B$ is the
            set
            \begin{equation*}
                A\times{B}
                =\{\,(a,\,b)\;|\;a\in{A}\textrm{ and }b\in{B}\,\}
            \end{equation*}
        \end{fdefinition}
        Note that since, in general, $(a,b)\ne(b,a)$, it is generally true that
        $A\times{B}\ne{B}\times{A}$. Indeed, equality occurs if and only if
        $A=B$ (or if either set is empty).
        \begin{fexample}{Basic Cartesian Products}{Basic_Cartesian_Products}
            Let $A$ and $B$ be sets defined as follows:
            \par
            \begin{subequations}
                \begin{minipage}[b]{0.49\textwidth}
                    \centering
                    \begin{equation}
                        A=\{\,1,\,2,\,3\,\}
                    \end{equation}
                \end{minipage}
                \hfill
                \begin{minipage}[b]{0.49\textwidth}
                    \centering
                    \begin{equation}
                        B=\{\,a,\,b\,\}
                    \end{equation}
                \end{minipage}
            \end{subequations}
            \par\vspace{2.5ex}
            Let's compute $A\times{B}$ and $B\times{A}$. From the definition
            (Def.~\ref{def:Cartesian_Product_of_Two_Sets}) we have:
            \begin{equation}
                A\times{B}=\{\,(a,b)\;|\;a\in{A}\textrm{ and }b\in{B}\,\}
            \end{equation}
            Using this, we can compute:
            \begin{equation}
                A\times{B}=\big\{\,(1,a),\,(2,a),\,(3,a),\,
                                   (1,b),\,(2,b),\,(3,b)\,\big\}
            \end{equation}
            Computing $B\times{A}$, we have:
            \begin{equation}
                B\times{A}=\big\{\,(a,\,1),\,(a,\,2),\,(a,\,3),\,
                                   (b,\,1),\,(b,\,2),\,(b,\,3)\,\big\}
            \end{equation}
            Now if we suppose that $a$ is not equal to 1, then we see that
            $(a,1)$ is a different element than $(1,a)$, and thus $A\times{B}$
            is not equal to $B\times{A}$. Next, compute $A\times{A}$:
            \begin{equation}
                \begin{split}
                    A\times{A}=\Big\{\,(1,1),\,(1,2),\,&(1,3),
                                       (2,1),\,(2,2),\,\\&(2,3),
                                       (3,1),\,(3,2),\,(3,3)\,\Big\}
                \end{split}
            \end{equation}
            And finally $B\times{B}$:
            \begin{equation}
                B\times{B}=\big\{\,(a,\,a),\,(a,\,b),
                                 \,(b,\,a),\,(b,\,b)\,\big\}
            \end{equation}
            Equality of $A\times{B}$ and $B\times{A}$ is achieved if and only
            if $A=B$, or if either set is the empty set.
        \end{fexample}
        Note that in Ex.~\ref{ex:Basic_Cartesian_Products}, the \textit{size}
        of the Cartesian product of two sets was simply the product of the
        number of elements of the constituent sets. That is, we see that $A$
        has three elements and $B$ has two elements, but also that
        $A\times{B}$ has six elements. Moreover, $A\times{A}$ has nine
        elements and $B\times{B}$ has four. This pattern holds for the
        Cartesian products of any two \textit{finite} sets.
        \par\hfill\par
        It is common to consider the Cartesian product of a set with itself.
        That is, given a set $A$, we are often interested in $A\times{A}$. We
        denote this by writing $A^{2}$. One such example is when we consider
        the set of real numbers $\mathbb{R}$. The Cartesian product
        $\mathbb{R}^{2}$ is called the \textit{Euclidean Plane},
        or the \textit{Cartesian Plane}, after Euclid of Alexandria%
        \index{Euclide of Alexandria} and Ren\'{e} Descartes%
        \index{Descartes, Ren\'{e}}. This is because $\mathbb{R}^{2}$ is used to
        model both planar geometry and analytical geometry, of which Euclid and
        Descartes were pioneers of, respectively. The term Cartesian products
        is in honor of Ren\'{e} Descartes, as well.
        Let $\mathbb{R}$ denote the set of real numbers, and let
        $A=\mathbb{R}$ and $B=\mathbb{R}$. Then we have:
        \begin{equation}
            A\times{B}=\mathbb{R}\times\mathbb{R}\equiv\mathbb{R}^{2}
        \end{equation}
        Where the symbol $\equiv$ again means that $\mathbb{R}^{2}$ is
        defined by this expression. Using the definition of Cartesian
        products (Def.~\ref{def:Cartesian_Product_of_Two_Sets}), we obtain:
        \begin{equation}
            \mathbb{R}^{2}=\{\;(x,y)\,:\,x\in\mathbb{R}
                               \textrm{ and }y\in\mathbb{R}\;\}
        \end{equation}
        That is, $\mathbb{R}^{2}$ is the set of all ordered pairs of real
        numbers. The first term is called the $x$ coordinate, and similarly the
        second term is called the $y$ coordinate. We envision this as a
        \textit{plane} of points, each one corresponding to an ordered pair
        $(x,y)$. This is depicted in Fig.~\ref{fig:Cartesian_Plane}.
        \begin{figure}[H]
            \centering
            %--------------------------------Dependencies----------------------------------%
%   tikz                                                                       %
%       arrows.meta                                                            %
%-------------------------------Main Document----------------------------------%
\begin{tikzpicture}[%
    >=Latex,
    line width=0.2mm,
    line cap=round,
    font=\Large
]
    % Coordinates for the points.
    \coordinate (x) at (2.2, 0.0);
    \coordinate (y) at (0.0, 2.9);
    \coordinate (z) at (2.2, 2.9);

    % Draw a grid.
    \draw[style=help lines] (-0.3, -0.3) grid (7.9, 7.9);

    % Axes.
    \begin{scope}[thick]
        \draw[->] (-0.3, 0) to (8.4, 0) node [above] {$\mathbb{R}$};
        \draw[->] (0, -0.3) to (0, 8.4) node [right] {$\mathbb{R}$};
    \end{scope}

    % Draw dashed lines to the point.
    \begin{scope}[densely dashed]
        \draw (x) to (z);
        \draw (y) to (z);
    \end{scope}

    % Draw dots marking the various points.
    \draw[fill=black] (x) circle (0.6mm);
    \draw[fill=black] (y) circle (0.6mm);
    \draw[fill=black] (z) circle (0.6mm);

    % Label the points x, y, and the dot (x,y) in the plane.
    \node at (x) [below=0.1]     {$x$};
    \node at (y) [left=0.1]      {$y$};
    \node at (z) [above right]   {$(x,\,y)$};
\end{tikzpicture}
            \caption{The Cartesian Plane $\mathbb{R}^{2}$}
            \label{fig:Cartesian_Plane}
        \end{figure}
        Consider further the set $\mathbb{N}^{2}$. That is, letting
        $\mathbb{N}$ denote the set of natural numbers
        (Eqn.~\ref{eqn:Natural_Numbers_Ellipses}), letting $A=\mathbb{N}$ and
        $B=\mathbb{N}$ we have:
        \begin{equation}
            A\times{B}=\mathbb{N}\times\mathbb{N}\equiv\mathbb{N}^{2}
        \end{equation}
        Again using the definition of Cartesian products
        (Def.~\ref{def:Cartesian_Product_of_Two_Sets}), we have:
        \begin{equation}
            \mathbb{N}^{2}=
            \{\,(n,\,m)\;|\;n\in\mathbb{N}\textrm{ and }m\in\mathbb{N}\,\}
        \end{equation}
        We can visualize this as a subset of $\mathbb{R}^{2}$ by drawing a
        lattice of points in the Cartesian plane
        (Fig.~\ref{fig:Lattice_Cart_Prod_of_N_with_N}).
        \begin{figure}[H]
            \centering
            %--------------------------------Dependencies----------------------------------%
%   tikz                                                                       %
%       arrows.meta                                                            %
%-------------------------------Main Document----------------------------------%
\begin{tikzpicture}[%
    >=Latex,
    line width=0.2mm,
    line cap=round
]

    % Axes.
    \begin{scope}[thick, font=\Large]
        \draw[->] (0, 0) to (8.4, 0) node [above] {$\mathbb{N}$};
        \draw[->] (0, 0) to (0, 8.4) node [right] {$\mathbb{N}$};
    \end{scope}

    \foreach\x in{1, 2, 3, 4, 5, 6, 7, 8}{
        \foreach\y in{1, 2, 3, 4, 5, 6, 7, 8}{
            \draw[fill=black] (\x, \y) circle (0.2mm);
        }
        \draw (\x, -0.1) to (\x, 0.1) node [below=1ex] {$\x$};
        \draw (-0.1, \x) to (0.1, \x) node [left=1ex]  {$\x$};
    }
\end{tikzpicture}
            \caption{The Lattice $\mathbb{N}^{2}$}
            \label{fig:Lattice_Cart_Prod_of_N_with_N}
        \end{figure}
        This can then be consider a subset of the Euclidean plane
        $\mathbb{R}^{2}$. That is, at every ordered pair of integers $(m,n)$,
        we place a point in the Euclidean plane whose $x$ coordinate is $m$ and
        whose $y$ coordinate is $n$. We can also be more abstract and general in
        our examples. Consider the following sets:
        \par
        \begin{subequations}
            \begin{minipage}[b]{0.49\textwidth}
                \centering
                \begin{equation}
                    A=\{\,\textrm{Point, Line 1, Line 2}\,\}
                \end{equation}
            \end{minipage}
            \hfill
            \begin{minipage}[b]{0.49\textwidth}
                \centering
                \begin{equation}
                    B=\{\,\textrm{Point, Line}\,\}
                \end{equation}
            \end{minipage}
        \end{subequations}
        \par\vspace{2.5ex}
        We can visually represent the Cartesian product $A\times{B}$ by
        drawing $A$ in green and $B$ in red, as shown in
        Fig.~\ref{fig:Cartesian_Product_Example}. The Cartesian Product
        $A\times{B}$ is the set formed by connecting all of the points
        from $A$ and $B$ in the plane. This is shown in blue.
        \begin{figure}[H]
            \centering
            %--------------------------------Dependencies----------------------------------%
%   tikz                                                                       %
%       arrows.meta                                                            %
%-------------------------------Main Document----------------------------------%
\begin{tikzpicture}[%
    >=Latex,
    line width=0.2mm,
    line cap=round
]

    % Draw green to indicate the set A.
    \begin{scope}[green]

        % Draw some points.
        \draw[fill=green] (1, 0) circle (0.3mm);
        \draw[fill=green] (2, 0) circle (0.3mm);
        \draw[fill=green] (5, 0) circle (0.3mm);
        \draw[fill=green] (6, 0) circle (0.3mm);
        \draw[fill=green] (7, 0) circle (0.3mm);

        % Draw lines.
        \draw (2, 0) to (5, 0);
        \draw (6, 0) to (7, 0);
    \end{scope}

    % Draw red to denote the set B.
    \begin{scope}[red]

        % Draw in some points.
        \draw[fill=red] (0, 1) circle (0.3mm);
        \draw[fill=red] (0, 2) circle (0.3mm);
        \draw[fill=red] (0, 5) circle (0.3mm);

        % Draw a line.
        \draw (0, 2) to (0, 5);
    \end{scope}

    % Use blue to mark AxB (Cartesian product).
    \begin{scope}[blue]

        % Fill in points.
        \draw[fill=blue] (1, 1) circle (0.3mm);
        \draw[fill=blue] (1, 2) circle (0.3mm);
        \draw[fill=blue] (1, 5) circle (0.3mm);
        \draw[fill=blue] (2, 1) circle (0.3mm);
        \draw[fill=blue] (5, 1) circle (0.3mm);
        \draw[fill=blue] (6, 1) circle (0.3mm);
        \draw[fill=blue] (7, 1) circle (0.3mm);
        \draw[fill=blue] (2, 2) circle (0.3mm);
        \draw[fill=blue] (2, 5) circle (0.3mm);
        \draw[fill=blue] (5, 2) circle (0.3mm);
        \draw[fill=blue] (5, 5) circle (0.3mm);
        \draw[fill=blue] (6, 2) circle (0.3mm);
        \draw[fill=blue] (7, 2) circle (0.3mm);
        \draw[fill=blue] (6, 5) circle (0.3mm);
        \draw[fill=blue] (7, 5) circle (0.3mm);

        % Draw lines.
        \draw (1, 2) to (1, 5);
        \draw (2, 1) to (5, 1);
        \draw (6, 1) to (7, 1);

        % Fill in rectangles.
        \draw[fill=blue, opacity=0.4] (2, 2) to (5, 2) to (5, 5)
                                             to (2, 5) to cycle;
        \draw[fill=blue, opacity=0.4] (6, 2) to (7, 2) to (7, 5)
                                             to (6, 5) to cycle;
        \draw (2, 2) to (5, 2) to (5, 5) to (2, 5) to cycle;
        \draw (6, 2) to (7, 2) to (7, 5) to (6, 5) to cycle;
    \end{scope}
\end{tikzpicture}
            \caption[Cartesian Product of Two Sets]
                {The Cartesian Product of Two Sets. $A$ is
                 in \textcolor{green}{Green},
                 $B$ is in \textcolor{red}{red}, and
                 $A\times{B}$ is in \textcolor{blue}{blue}.}
            \label{fig:Cartesian_Product_Example}
        \end{figure}
        Cartesian products are not \textit{associative}. That is, given three
        sets $A$, $B$, and $C$, there is no clear way to take the Cartesian
        product of these since:
        \begin{equation}
            A\times(B\times{C})\ne(A\times{B})\times{C}
        \end{equation}
        To see this, note that the elements of $A\times(B\times{C})$ are
        ordered pairs of the form $\big(a,\,(b,\,c)\big)$, whereas elements of
        $(A\times{B})\times{C}$ are of the form $\big((a,\,b),\,c\big)$. When
        we write $A\times{B}\times{C}$ we really want ordered \textit{triples}
        of the form $(a,\,b,\,c)$.
        Much the way ordered pairs have been
        defined, we can modify Kuratowski's approach and define ordered
        triples and ordered $n$ tuples. Rather than doing this we will use the
        language of functions to define higher order Cartesian products.
        \begin{fdefinition}{Functions}{Function}
            A \gls{function} from a \gls{set} $A$ to a set $B$ is a \gls{subset}
            $f\subseteq{A}\times{B}$, denoted $f:A\rightarrow{B}$, such that
            for all $x\in{A}$ there is a unique $y\in{B}$ such that
            $(x,y)\in{f}$. $A$ is called the domain of $f$
            and $B$ is called the codomain.
        \end{fdefinition}
        We're used to hearing that a function is a rule that assigns to an
        input value $x$ some output value $f(x)$. It may seem hard to justify,
        then, why we've defined a function as a subset of the Cartesian
        product. But note the requirement that, for each $x\in{A}$, there is a
        \textit{unique} $y\in{B}$ such that $(x,y)\in{f}$. We call this unique
        element the \textit{image} of $x$ under the function $f$ and write
        $y=f(x)$. The condition that there is a unique such value $y$ to each
        $x$ is called the \textit{vertical line test} when graphing functions
        of the form $f:\mathbb{R}\rightarrow\mathbb{R}$
        (Fig.~\ref{fig:Function_R_to_R_Subset_Cart_Prod}). Simply, given such
        a function, if one draws a vertical line in the plane, then it must
        intersect the graph of $f$ once and only once. This provides a
        quick means of discerning functions from non-functions.
        \begin{lexample}{The Square Function}{Square_Function}
            If we can come up with some rule that assigns to every element
            $a\in{A}$ a unique element of $B$, then we can use this rule to
            define a function $f:A\rightarrow{B}$. Such a rule often comes
            in the form of a \textit{formula}. We write the unique element that
            $a$ corresponds to as $f(b)$. For example, let $A=\mathbb{R}$ and
            let $B=\mathbb{R}$. We can define a function by the squaring
            formula:
            \begin{equation}
                f(x)=x\cdot{x}=x^{2}
            \end{equation}
            Once we know that $x^{2}$ gives a unique number
            (which will require some notion of arithmetic), we can define
            the function $f:\mathbb{R}\rightarrow\mathbb{R}$ by:
            \begin{equation}
                f=\{\,(x,\,x^{2})\in\mathbb{R}^{2}\;|\;x\in\mathbb{R}\,\}
            \end{equation}
            Usually we'll define functions by their formula's, rather than
            expressing them explicitly as subsets of the Cartesian product.
        \end{lexample}
        In the field of mathematical analysis we are often concerned with
        functions involving real numbers. For the sake of intuition, let
        us consider functions of the form $f:\mathbb{R}\rightarrow\mathbb{R}$.
        Any curve that we draw left-to-right, without picking up the pencil,
        will be a valid function.
        (See Fig.~\ref{fig:Function_R_to_R_Subset_Cart_Prod}).
        \begin{figure}[H]
            \centering
            %--------------------------------Dependencies----------------------------------%
%   xcolor                                                                     %
%   amssymb                                                                    %
%   tikz                                                                       %
%       arrows.meta                                                            %
%       patterns                                                               %
%-------------------------------Main Document----------------------------------%
\begin{tikzpicture}[%
    >=Latex,
    line width=0.2mm,
    line cap=round,
    scale=1.2
]
    % Coorindates for the curve.
    \coordinate (P1) at (-3.85, -2.00);
    \coordinate (P2) at (-2.00, -3.00);
    \coordinate (P3) at ( 0.00,  0.00);
    \coordinate (P4) at ( 2.00,  3.00);
    \coordinate (P5) at ( 3.85,  3.80);

    \draw[%
        pattern=north west lines,
        pattern color=Green!80!Black,
        opacity=0.5,
        draw=white
    ]   (-3.9, -3.9) rectangle (3.9, 3.9);

    \begin{scope}[thick, font=\Large]
        \draw[<->] (-4.2, 0) to (4.2, 0) node [above] {$\mathbb{R}$};
        \draw[<->] (0, -4.2) to (0, 4.2) node [right] {$\mathbb{R}$};
    \end{scope}

    \draw[draw=blue] (P1) to [out=-30, in=150]  (P2)
                          to [out=-30, in=210]  (P3)
                          to [out=30,  in=180]  (P4)
                          to [out=0,   in=200]  (P5);
    \draw[fill=white, draw=white] 
        (1.3, 2) rectangle node {$\textcolor{blue}{f}$} (1.6, 1.5);
\end{tikzpicture}
            \caption[Example of a Function $f:\mathbb{R}\rightarrow\mathbb{R}$]
                    {Example of a function $f:\mathbb{R}\rightarrow\mathbb{R}$.
                     The Cartesian product $\mathbb{R}\times\mathbb{R}$ is
                     shown in \textcolor{green!80!black}{green}, and the
                     function $f\subseteq\mathbb{R}\times\mathbb{R}$ is shown
                     in \textcolor{blue}{blue}.}
            \label{fig:Function_R_to_R_Subset_Cart_Prod}
        \end{figure}
        Let $g\subseteq\mathbb{R}\times\mathbb{R}$ be defined as follows:
        \begin{equation}
            g=\big\{\,(x,\,y)\in\mathbb{R}^{2}\;|\;y^{2}=x\,\big\}
        \end{equation}
        It is tempting to label $g$ by writing $g(x)=\sqrt{x}$, but $g$ is
        not a function for it fails two of the requirements of a function.
        Firstly, for any $x>0$, there are two values $y_{1}$ and $y_{2}$
        whose square is equal to $x$. Indeed, if $y_{1}$ is one such value,
        then setting $y_{2}=\minus{y}_{1}$ will result in a second
        distinct value. Thus $g$ does not have the uniqueness property
        required for functions. Moreover, if $x<0$, then there is no such
        value $y\in\mathbb{R}$ such that $(x,y)\in{g}$, and thus $g$ also
        lacks the existence property. In terms of the vertical line test,
        there are points $x$ such that the vertical line through
        $(x,\,0)$ intersects $g$ twice, and there are points such that the
        vertical line does not intersect at all. The graph of $g$ is shown
        in Fig.~\ref{fig:SQRT_Not_a_Function}.
        \begin{figure}[H]
            \centering
            %--------------------------------Dependencies----------------------------------%
%   xcolor                                                                     %
%   amssymb                                                                    %
%   tikz                                                                       %
%       arrows.meta                                                            %
%-------------------------------Main Document----------------------------------%
\begin{tikzpicture}[%
    >=Latex,
    line width=0.2mm,
    line cap=round,
    scale=1.2
]
    % Coorindates for the curve.
    \coordinate (P1) at (-3.85, -2.00);
    \coordinate (P2) at (-2.00, -3.00);
    \coordinate (P3) at ( 0.00,  0.00);
    \coordinate (P4) at ( 2.00,  3.00);
    \coordinate (P5) at ( 3.85,  3.80);

    % Draw a green mesh indicating the Cartesian plane.
    \foreach\x in {-40, -39, ..., 39}{
        \draw[draw=green, line width=0.1mm] (\x/10, -4) to (-4, \x/10);
        \draw[draw=green, line width=0.1mm] (4, \x/10)  to (\x/10, 4);
    }
    \draw[draw=green, line width=0.1mm] (4, 4)  to (4, 4);

    \begin{scope}[thick, font=\Large]
        \draw[<->] (-4.3,  0.0) to (4.3, 0.0) node [above] {$\mathbb{R}$};
        \draw[<->] ( 0.0, -4.3) to (0.0, 4.3) node [right] {$\mathbb{R}$};
    \end{scope}

    \draw[draw=red] (3.880000, -1.969772) to (3.840000, -1.959592)
                                          to (3.800000, -1.949359)
                                          to (3.760000, -1.939072)
                                          to (3.720000, -1.928730)
                                          to (3.680000, -1.918333)
                                          to (3.640000, -1.907878)
                                          to (3.600000, -1.897367)
                                          to (3.560000, -1.886796)
                                          to (3.520000, -1.876166)
                                          to (3.480000, -1.865476)
                                          to (3.440000, -1.854724)
                                          to (3.400000, -1.843909)
                                          to (3.360000, -1.833030)
                                          to (3.320000, -1.822087)
                                          to (3.280000, -1.811077)
                                          to (3.240000, -1.800000)
                                          to (3.200000, -1.788854)
                                          to (3.160000, -1.777639)
                                          to (3.120000, -1.766352)
                                          to (3.080000, -1.754993)
                                          to (3.040000, -1.743560)
                                          to (3.000000, -1.732051)
                                          to (2.960000, -1.720465)
                                          to (2.920000, -1.708801)
                                          to (2.880000, -1.697056)
                                          to (2.840000, -1.685230)
                                          to (2.800000, -1.673320)
                                          to (2.760000, -1.661325)
                                          to (2.720000, -1.649242)
                                          to (2.680000, -1.637071)
                                          to (2.640000, -1.624808)
                                          to (2.600000, -1.612452)
                                          to (2.560000, -1.600000)
                                          to (2.520000, -1.587451)
                                          to (2.480000, -1.574802)
                                          to (2.440000, -1.562050)
                                          to (2.400000, -1.549193)
                                          to (2.360000, -1.536229)
                                          to (2.320000, -1.523155)
                                          to (2.280000, -1.509967)
                                          to (2.240000, -1.496663)
                                          to (2.200000, -1.483240)
                                          to (2.160000, -1.469694)
                                          to (2.120000, -1.456022)
                                          to (2.080000, -1.442221)
                                          to (2.040000, -1.428286)
                                          to (2.000000, -1.414214)
                                          to (1.960000, -1.400000)
                                          to (1.920000, -1.385641)
                                          to (1.880000, -1.371131)
                                          to (1.840000, -1.356466)
                                          to (1.800000, -1.341641)
                                          to (1.760000, -1.326650)
                                          to (1.720000, -1.311488)
                                          to (1.680000, -1.296148)
                                          to (1.640000, -1.280625)
                                          to (1.600000, -1.264911)
                                          to (1.560000, -1.249000)
                                          to (1.520000, -1.232883)
                                          to (1.480000, -1.216553)
                                          to (1.440000, -1.200000)
                                          to (1.400000, -1.183216)
                                          to (1.360000, -1.166190)
                                          to (1.320000, -1.148913)
                                          to (1.280000, -1.131371)
                                          to (1.240000, -1.113553)
                                          to (1.200000, -1.095445)
                                          to (1.160000, -1.077033)
                                          to (1.120000, -1.058301)
                                          to (1.080000, -1.039230)
                                          to (1.040000, -1.019804)
                                          to (1.000000, -1.000000)
                                          to (0.960000, -0.979796)
                                          to (0.920000, -0.959166)
                                          to (0.880000, -0.938083)
                                          to (0.840000, -0.916515)
                                          to (0.800000, -0.894427)
                                          to (0.760000, -0.871780)
                                          to (0.720000, -0.848528)
                                          to (0.680000, -0.824621)
                                          to (0.640000, -0.800000)
                                          to (0.600000, -0.774597)
                                          to (0.560000, -0.748331)
                                          to (0.520000, -0.721110)
                                          to (0.480000, -0.692820)
                                          to (0.440000, -0.663325)
                                          to (0.400000, -0.632456)
                                          to (0.360000, -0.600000)
                                          to (0.320000, -0.565685)
                                          to (0.280000, -0.529150)
                                          to (0.240000, -0.489898)
                                          to (0.200000, -0.447214)
                                          to (0.160000, -0.400000)
                                          to (0.120000, -0.346410)
                                          to (0.080000, -0.282843)
                                          to (0.040000, -0.200000)
                                          to (0.000000, 0.000000) 
                                          to (0.040000, 0.200000)
                                          to (0.080000, 0.282843)
                                          to (0.120000, 0.346410)
                                          to (0.160000, 0.400000)
                                          to (0.200000, 0.447214)
                                          to (0.240000, 0.489898)
                                          to (0.280000, 0.529150)
                                          to (0.320000, 0.565685)
                                          to (0.360000, 0.600000)
                                          to (0.400000, 0.632456)
                                          to (0.440000, 0.663325)
                                          to (0.480000, 0.692820)
                                          to (0.520000, 0.721110)
                                          to (0.560000, 0.748331)
                                          to (0.600000, 0.774597)
                                          to (0.640000, 0.800000)
                                          to (0.680000, 0.824621)
                                          to (0.720000, 0.848528)
                                          to (0.760000, 0.871780)
                                          to (0.800000, 0.894427)
                                          to (0.840000, 0.916515)
                                          to (0.880000, 0.938083)
                                          to (0.920000, 0.959166)
                                          to (0.960000, 0.979796)
                                          to (1.000000, 1.000000)
                                          to (1.040000, 1.019804)
                                          to (1.080000, 1.039230)
                                          to (1.120000, 1.058301)
                                          to (1.160000, 1.077033)
                                          to (1.200000, 1.095445)
                                          to (1.240000, 1.113553)
                                          to (1.280000, 1.131371)
                                          to (1.320000, 1.148913)
                                          to (1.360000, 1.166190)
                                          to (1.400000, 1.183216)
                                          to (1.440000, 1.200000)
                                          to (1.480000, 1.216553)
                                          to (1.520000, 1.232883)
                                          to (1.560000, 1.249000)
                                          to (1.600000, 1.264911)
                                          to (1.640000, 1.280625)
                                          to (1.680000, 1.296148)
                                          to (1.720000, 1.311488)
                                          to (1.760000, 1.326650)
                                          to (1.800000, 1.341641)
                                          to (1.840000, 1.356466)
                                          to (1.880000, 1.371131)
                                          to (1.920000, 1.385641)
                                          to (1.960000, 1.400000)
                                          to (2.000000, 1.414214)
                                          to (2.040000, 1.428286)
                                          to (2.080000, 1.442221)
                                          to (2.120000, 1.456022)
                                          to (2.160000, 1.469694)
                                          to (2.200000, 1.483240)
                                          to (2.240000, 1.496663)
                                          to (2.280000, 1.509967)
                                          to (2.320000, 1.523155)
                                          to (2.360000, 1.536229)
                                          to (2.400000, 1.549193)
                                          to (2.440000, 1.562050)
                                          to (2.480000, 1.574802)
                                          to (2.520000, 1.587451)
                                          to (2.560000, 1.600000)
                                          to (2.600000, 1.612452)
                                          to (2.640000, 1.624808)
                                          to (2.680000, 1.637071)
                                          to (2.720000, 1.649242)
                                          to (2.760000, 1.661325)
                                          to (2.800000, 1.673320)
                                          to (2.840000, 1.685230)
                                          to (2.880000, 1.697056)
                                          to (2.920000, 1.708801)
                                          to (2.960000, 1.720465)
                                          to (3.000000, 1.732051)
                                          to (3.040000, 1.743560)
                                          to (3.080000, 1.754993)
                                          to (3.120000, 1.766352)
                                          to (3.160000, 1.777639)
                                          to (3.200000, 1.788854)
                                          to (3.240000, 1.800000)
                                          to (3.280000, 1.811077)
                                          to (3.320000, 1.822087)
                                          to (3.360000, 1.833030)
                                          to (3.400000, 1.843909)
                                          to (3.440000, 1.854724)
                                          to (3.480000, 1.865476)
                                          to (3.520000, 1.876166)
                                          to (3.560000, 1.886796)
                                          to (3.600000, 1.897367)
                                          to (3.640000, 1.907878)
                                          to (3.680000, 1.918333)
                                          to (3.720000, 1.928730)
                                          to (3.760000, 1.939072)
                                          to (3.800000, 1.949359)
                                          to (3.840000, 1.959592)
                                          to (3.880000, 1.969772);
    \draw[fill=white, draw=white] 
        (1.3, 2.0) rectangle node {$\textcolor{red}{g}$} (1.6, 1.5);
\end{tikzpicture}
            \caption[Example of a Non-Function]
                {$g\subseteq\mathbb{R}\times\mathbb{R}$ is not a function
                 since it fails the vertical line test.}
            \label{fig:SQRT_Not_a_Function}
        \end{figure}
        We need not only consider functions of the form
        $f:\mathbb{R}\rightarrow\mathbb{R}$, nor functions
        $f:\mathcal{U}\rightarrow\mathcal{V}$, where $\mathcal{U}$ and
        $\mathcal{V}$ are subsets of $\mathbb{R}$, and we can allow for
        arbitrary abstract functions. Let $A$ and $B$ be defined as follows:
        \par
        \begin{subequations}
            \begin{minipage}[b]{0.49\textwidth}
                \centering
                \begin{equation}
                    A=\{\,1,\,2,\,3,\,4\,\}
                \end{equation}
            \end{minipage}
            \hfill
            \begin{minipage}[b]{0.49\textwidth}
                \centering
                \begin{equation}
                    B=\{\,a,\,b,\,c\,\}
                \end{equation}
            \end{minipage}
        \end{subequations}
        \par\vspace{2.5ex}
        Similar to the vertical line test, we can devise a visual to
        discerning functions from non-functions for abstract sets.
        We represent the elements of $A$ and $B$ as points in some blob
        in the plane, and then draw arrows between the points
        $x\in{A}$ and $y\in{b}$ indicating that $(x,\,y)\in{f}$.
        This allows us to determine if a given $f\subseteq{A}\times{B}$ is a
        functions ore not. Every point in $A$ must be mapped to a unique point
        in $B$. That is, every point in $A$ must have one and only one arrow
        connecting it to some point in $B$. Examples of valid functions
        are shown in Fig.~\ref{fig:Abstract_Functions}, and non-functions
        are shown in Fig.~\ref{fig:Abstract_Non_Functions}.
        \begin{figure}[H]
            \centering
            \begin{subfigure}[b]{0.49\textwidth}
                \centering
                \resizebox{\textwidth}{!}{%
                    \input{tikz/Function_Example_002.tex}
                }
                \subcaption{A Valid Function.}
            \end{subfigure}
            \begin{subfigure}[b]{0.49\textwidth}
                \centering
                \resizebox{\textwidth}{!}{%
                    \input{tikz/Function_Example_003.tex}
                }
                \subcaption{Another Valid Function.}
            \end{subfigure}
            \caption{Visual for Abstract Functions}
            \label{fig:Abstract_Functions}
        \end{figure}
        \begin{figure}[H]
            \centering
            \begin{subfigure}[b]{0.49\textwidth}
                \centering
                \resizebox{\textwidth}{!}{%
                    %--------------------------------Dependencies----------------------------------%
%   tikz                                                                       %
%       arrows.meta                                                            %
%-------------------------------Main Document----------------------------------%
\begin{tikzpicture}[%
    >=latex,
    line width=0.2mm,
    line cap=round,
    scale=1.2
]
    % Coorindates.
    \coordinate (a) at ( 1.5,  0.75);
    \coordinate (b) at ( 1.5, -0.00);
    \coordinate (c) at ( 1.5, -0.75);
    \coordinate (1) at (-1.5,  1.20);
    \coordinate (2) at (-1.5,  0.40);
    \coordinate (3) at (-1.5, -0.40);
    \coordinate (4) at (-1.5, -1.20);
    \coordinate (A) at (-1.5,  2.50);
    \coordinate (B) at ( 1.5,  2.50);

    % Ellipses representing the sets A and B.
    \draw[thick] (-1.5, 0.0) ellipse (1 and 2);
    \draw[thick] ( 1.5, 0.0) ellipse (1 and 2);

    % Draw circles for the various points.
    \draw[fill=black] (a) circle (0.4mm);
    \draw[fill=black] (b) circle (0.4mm);
    \draw[fill=black] (c) circle (0.4mm);
    \draw[fill=black] (1) circle (0.4mm);
    \draw[fill=black] (2) circle (0.4mm);
    \draw[fill=black] (3) circle (0.4mm);
    \draw[fill=black] (4) circle (0.4mm);

    % Draw paths indicating mappings.
    \begin{scope}[->]
        \draw[shorten >=0.8mm] (1) to (a);
        \draw[shorten >=0.8mm] (2) to (b);
        \draw[shorten >=0.8mm] (3) to (c);
    \end{scope}

    % Labels.
    \node at (A)         {$A$};
    \node at (B)         {$B$};
    \node at (a) [right] {$a$};
    \node at (b) [right] {$b$};
    \node at (c) [right] {$c$};
    \node at (1) [left]  {$1$};
    \node at (2) [left]  {$2$};
    \node at (3) [left]  {$3$};
    \node at (4) [left]  {$4$};
\end{tikzpicture}
                }
                \subcaption{Fails Existence.}
            \end{subfigure}
            \begin{subfigure}[b]{0.49\textwidth}
                \centering
                \resizebox{\textwidth}{!}{%
                    \input{tikz/Non_Function_Example_003.tex}
                }
                \subcaption{Fails Uniqueness.}
            \end{subfigure}
            \caption{Non-Functions}
            \label{fig:Abstract_Non_Functions}
        \end{figure}
        It is possible to count the total number of functions from $A$ to $B$.
        Since every element of $A$ needs to be mapped to some element of $B$,
        and since there are 4 elements in $A$ and 3 elements in $B$, the total
        number of functions $f:A\rightarrow{B}$ is $4^{3}=64$. On the other
        hand, the total number of subsets of $A\times{B}$ is $2^{12}=4096$
        (we will justify this when we discuss the \textit{cardinality} of
        sets). Thus, if we were to randomly pick a subset of $A\times{B}$, the
        odds are that it is almost certainly \textit{not} a function
        (1.5625\%). Thus, functions are very special subsets.
        There is a frequent need to discuss the \textit{set of all functions}
        from a given set $A$ into another set $B$. To ensure we don't create
        a function version of Russell's paradox, we prove such a set exists.
        \begin{theorem}
            If $A$ and $B$ are sets, then there exists a set $\mathcal{F}$ such
            that, for all $f$ it is true that $f\in\mathcal{F}$ if and only if
            $f$ is a function from $A$ to $B$, $f:A\rightarrow{B}$.
        \end{theorem}
        \begin{proof}
            For if $A$ and $B$ are sets, then by
            Thm.~\ref{thm:Existence_of_the_Cartesian_Product} the set
            $A\times{B}$ exists. But by the axiom of the power set
            (Ax.~\ref{ax:Axiom_of_the_Power_Set}) the power set of $A\times{B}$,
            $\mathcal{P}(A\times{B})$, exists. Let $P$ be the proposition
            \textit{True if} $f$ \textit{is a function from} $A$ \textit{to}
            $B$, \textit{false otherwise}. Then by axiom schema of specification
            (Ax.~\ref{ax:Axiom_Schema_of_Specification}), there is a set
            $\mathcal{F}$ such that:
            \begin{equation}
                \mathcal{F}=\big\{\,f\in\mathcal{P}(A\times{B})\;|
                    \;P(f)\,\big\}
            \end{equation}
            But then for all $f$, $f\in\mathcal{F}$ if and only if
            $f\in\mathcal{F}$ and $P(f)$ is true. But if $P(f)$ is true then
            $f$ is a function from $A$ to $B$, and thus by the definition of a
            function (Def.~\ref{def:Function}) $f\subseteq{A}\times{B}$. But
            then by the definition of the power set (Def.~\ref{def:Power_Set})
            we have that $f\in\mathcal{P}(A\times{B})$. Thus $P(f)$ implies
            $f\in\mathcal{F}$. Therefore $f\in\mathcal{F}$ if and only if
            $P(f)$. That is, $f\in\mathcal{F}$ if and only if $f$ is a function
            from $A$ to $B$.
        \end{proof}
        There is non-standard notation when discussing the set of all functions
        from a given set $A$ to a set $B$:
        \begin{fnotation}{Set of All Functions}{Set_of_All_Functions}
            If $A$ and $B$ are sets, the set of all functions from $A$ to $B$,
            $f:A\rightarrow{B}$, is denoted as either $\mathcal{F}(A,B)$ or
            $B^{A}$.
        \end{fnotation}
        The notation $B^{A}$ is common in many areas such as topology and
        algebra, especially when $A=B$. The \textit{topological space} $I^{I}$,
        which is the set of all functions from the \textit{closed unit inverval}
        to itself, is often used to construct examples and counterexamples.
        In analysis the notation $\mathcal{F}(A,B)$ seems to be more common,
        in particular $\mathcal{C}(A,B)$ is often used to denote the set of all
        \textit{continuous} functions from $A$ to $B$, provided the word
        continuous has meaning. Since the notation is not universal nor standard
        across the various disciplines, an attempt will be made to specify what
        $B^{A}$ or $\mathcal{F}(A,B)$ means before using it in a theorem or
        counterexample.
        \begin{fdefinition}{Image of a Point}{Image_of_Point}
            The image of an element $x$ in a set $A$ under a function
            $f:A\rightarrow{B}$ is the unique value $y\in{B}$ such that
            $(x,y)\in{f}$. We write $y=f(x)$.
        \end{fdefinition}
        This allows us to define functions by simply specifying what the
        image of each $x\in{A}$ is. Restating our previous claim, if we can
        define some formula such that for each $x\in{A}$ there is a unique
        $f(x)\in{B}$ such that the formula takes $x$ to $f(x)$, then we can
        define $f$ as the set of all such ordered pairs $(x,f(x))$, and this
        will be a function.
        \begin{fnotation}{Image Notation}{Image_Notation}
            If $A$ and $B$ are sets, if $f:A\rightarrow{B}$ is a function,
            if $x\in{A}$ and if $y=f(x)\in{B}$, then we denote this by
            writing $x\overset{f}{\longmapsto}{y}$ or just $x\mapsto{y}$.
        \end{fnotation}
        Throughout we will almost exclusively use the notation $y=f(x)$ rather
        than $x\mapsto{y}$. The reasons are purely aesthetic and both notations
        are common in mathematics. In a similar manner, we can define the image
        of an entire subset.
        \begin{theorem}
            If $A$ and $B$ are sets, if $f:A\rightarrow{B}$ is a function,
            and if $\mathcal{U}\subseteq{A}$, then there is a set
            $\mathcal{V}\subseteq{B}$ such that, for all $y$ it is true that
            $y\in\mathcal{V}$ if and only if $y\in{B}$ and such that there is
            an $x\in\mathcal{U}$ such that $y=f(x)$.
        \end{theorem}
        \begin{proof}
            For let $P$ be the proposition \textit{True if there exists}
            $x\in\mathcal{U}$ \textit{such that} $y=f(x)$,
            \textit{false otherwise}. By the axiom schema of specification
            (Ax.~\ref{ax:Axiom_Schema_of_Specification}) there is a set
            $\mathcal{V}$ such that, for all $y$ it is true that
            $y\in\mathcal{V}$ if and only if $y\in{B}$ and $P(y)$ is true. That
            is, $y\in\mathcal{V}$ if and only if $y\in{B}$ and if there is an
            $x\in\mathcal{U}$ such that $y=f(x)$.
        \end{proof}
        \begin{fdefinition}{Image of a Subset}{Image_of_Subset}
            The image of a subset $\mathcal{U}$ of a set $A$ under a function
            $f:A\rightarrow{B}$ is the set:
            \begin{equation*}
                f\big(\mathcal{U}\big)=
                    \{\,y\in{B}\;|\;\textrm{There exists }x\in\mathcal{U}
                                    \textrm{ such that }y=f(x)\,\}
            \end{equation*}
            That is, the set of all points in $B$ that are the
            image of points in $\mathcal{U}$.
        \end{fdefinition}
        This definition of the image of a subset was given in such a manner
        so that it only relies on the axiom schema of specification to
        justify it's existence. We could also use the notation:
        \begin{equation}
            f\big(\mathcal{U})=\{\,f(x)\in{B}\;|\;x\in\mathcal{U}\,\}
        \end{equation}
        Writing the definition of the image of a subset in such a way is
        justified by the \textit{axiom schema of replacement}, but we've not yet
        included this axiom in our system. This axiom deals with
        \textit{class functions} and will be dealt with later.
        \begin{example}
            If $f:\mathbb{R}\rightarrow\mathbb{R}$ is the function $f(x)=x^{2}$,
            then $f(\mathbb{R})=[0,\infty)$, where $[0,\infty)$ is defined by:
            \begin{equation}
                [0,\infty)=\{\,x\in\mathbb{R}\;|\;x\geq{0}\}
            \end{equation}
            To see this, not that every positive real number $y$ gets mapped to
            by at least one real number (notably, the positive square root
            $\sqrt{y}$). Zero is mapped to as well since $0^{2}=0$. However,
            none of the negative numbers are the image of any element of
            $\mathbb{R}$ since the square of a real number is always
            non-negative.
        \end{example}
        We can visualize functions and images by using blobs in the plane. Given
        some sub-blob of a set $A$, the image of this will be another sub-blob
        of $B$. Note that if $f:A\rightarrow{B}$ is a function, it does
        \textbf{not} need to be true that $f(A)=B$. These are special functions
        that are called \textit{surjective}\index{Function!Surjective} and are
        discussed in Chapt.~\ref{chapt:Function_Theory}. Such a drawing of the
        general case is shown in Fig.~\ref{fig:Image_of_Point_and_Subset}.
        \begin{figure}[H]
            \centering
            \captionsetup{type=figure}
            \begin{tikzpicture}[>=Latex]
    \coordinate (U1) at (-5.0, -2.0);
    \coordinate (U2) at (-3.5, -2.0);
    \coordinate (U3) at (-0.5, -0.5);
    \coordinate (U4) at (-2.0,  2.0);
    \coordinate (U5) at (-3.3,  1.6);
    \coordinate (U6) at (-4.0,  2.0);
    \coordinate (U7) at (-5.0,  0.0);

    \coordinate (V1) at (5.0,  2.0);
    \coordinate (V2) at (4.0,  2.0);
    \coordinate (V3) at (2.0,  0.0);
    \coordinate (V4) at (2.0, -2.0);
    \coordinate (V5) at (4.0, -1.0);
    \coordinate (V6) at (5.0, -1.0);

    \coordinate (S1) at (-4.0,  0.0);
    \coordinate (S2) at (-3.0, -1.0);
    \coordinate (S3) at (-2.5,  0.0);
    \coordinate (S4) at (-3.0,  0.5);

    \coordinate (T1) at (3.0,  0.0);
    \coordinate (T2) at (3.5, -0.8);
    \coordinate (T3) at (4.5,  0.0);
    \coordinate (T4) at (3.5,  0.8);

    \coordinate (x)  at (-2.8, -0.3);
    \coordinate (fx) at (3.4,  -0.4);

    \draw[fill=blue,opacity=0.5,draw=black,thick]
        (U1)    to[out=0,  in=-150] (U2)
                to[out=30, in=-90]  (U3)
                to[out=90, in=-60]  (U4)
                to[out=120, in=-30] (U5)
                to[out=150,in=10]   (U6)
                to[out=-170,in=90]  (U7)
                to[out=-90,in=180]  cycle;

    \draw[fill=red!80!white,opacity=0.5,draw=black,thick]
        (V1)    to[out=180, in=0]       (V2)
                to[out=180, in=60]      (V3)
                to[out=-120,in=120]     (V4)
                to[out=-60, in=180]     (V5)
                to[out=0,   in=-120]    (V6)
                to[out=60,  in=0]       cycle;

    \draw[fill=blue!80!white] (S1)  to[out=-150,in=180] (S2)
                                    to[out=0,   in=-90] (S3)
                                    to[out=90,  in=-60] (S4)
                                    to[out=120, in=30]  cycle;

    \draw[fill=red!80!white] (T1)   to[out=-150,in=180] (T2)
                                    to[out=0,   in=-90] (T3)
                                    to[out=90,  in=0]   (T4)
                                    to[out=180, in=30]  cycle;

    \draw[fill=black] (x)  circle (0.3mm);
    \draw[fill=black] (fx) circle (0.3mm);
    \node at (-2.5, 1.0) {\Large{$A$}};
    \node at ( 4.6, 1.3) {\Large{$B$}};
    \node at (-3.3, 0.0) {\large{$\mathcal{U}$}};
    \node at ( 3.7, 0.2) {\large{$f(\mathcal{U})$}};
    \node at (x)  [below] {$x$};
    \node at (fx) [right] {$f(x)$};
    \draw[->,shorten >= 1.5mm,shorten <= 1.5mm]
        (x) to[out=-30,in=-150] node[below]{$f$} (fx);
\end{tikzpicture}
            \caption{Image of a Subset and of a Point under a Function}
            \label{fig:Image_of_Point_and_Subset}
        \end{figure}
        If we consider a function $f:A\rightarrow{B}$ and the image of the
        entire set $A$ we obtain the \textit{range} of $f$. That is, the range
        is the set $f(A)\subseteq{B}$. In a similar manner to the forward image
        of a function, we can define the pre-image. First, we prove such things
        exist.
        \begin{theorem}
            \label{thm:Existence_of_Pre_Image}%
            If $A$ and $B$ are sets, if $f:A\rightarrow{B}$ is a function from
            $A$ to $B$, and if $\mathcal{V}\subseteq{B}$, then there is a set
            $\mathcal{U}\subseteq{A}$ such that for all $x$ it is true that
            $x\in\mathcal{U}$ if and only if $x\in{A}$ and $f(x)\in\mathcal{V}$.
        \end{theorem}
        \begin{proof}
            For let $P$ be the proposition \textit{True if} $x\in{A}$
            \textit{and} $f(x)\in\mathcal{V}$, \textit{false otherwise}. Then
            by the axiom schema of specification
            (Ax.~\ref{ax:Axiom_Schema_of_Specification}) there is a set
            $\mathcal{U}$ such that:
            \begin{equation*}
                \mathcal{U}=\big\{\,x\in{A}\;|\;P(x)\,\big\}
            \end{equation*}
            But $P(x)$ implies $x\in{A}$ and thus $x\in\mathcal{U}$ if and only
            if $x\in{A}$ and $f(x)\in\mathcal{V}$.
        \end{proof}
        \begin{fdefinition}{Pre-Image of a Subset}{Pre_Image_of_Subset}
            The \gls{pre-image} of a \gls{subset} $\mathcal{V}\subseteq{B}$
            under a \gls{function} $f:A\rightarrow{B}$ is the set:
            \begin{equation}
                f^{\minus{1}}(\mathcal{V})
                =\big\{\,x\in{X}\;|\;f(x)\in\mathcal{V}\,\big\}
            \end{equation}
        \end{fdefinition}
        The pre-image of a set behaves a lot differently than the image, and
        this will be explored in detail when functions are discussed. The cause
        of the discrepancy is the requirement that elements of $A$ map uniquely
        to elements of $B$, but a single element in $B$ can be the image of
        many different points in $A$. This gives rise to the notion of a
        \textit{fiber} of a point in $B$.
        \begin{theorem}
            If $A$ and $B$ are sets, if $f:A\rightarrow{B}$ is a function, and
            if $b\in{A}$, then there is a set $\mathcal{U}\subseteq{A}$ such
            for all $x\in{A}$ it is true that $x\in\mathcal{U}$ if and only if
            $f(x)=b$.
        \end{theorem}
        \begin{proof}
            For by Thm.~\ref{thm:Existence_of_Set_Containing_Set}, $\{b\}$ is
            a set and $\{b\}\subseteq{B}$ (Def.~\ref{def:Subsets}). But if
            $\{b\}$ is a subset of $B$, then there is a set
            $\mathcal{U}\subseteq{A}$ such that for all $x\in{A}$ it is true
            that $x\in\mathcal{U}$ if and only if $f(x)\in\{b\}$
            (Thm.~\ref{thm:Existence_of_Pre_Image}). But $f(x)\in\{b\}$ if and
            only if $f(x)=b$ (Thm.~\ref{thm:Existence_of_Set_Containing_Set}).
            Thus, for all $x\in{A}$, $x\in\mathcal{U}$ if and only if $f(x)=b$.
        \end{proof}
        \begin{fdefinition}{Fiber of an Element}{Fiber_of_Element}
            The fiber of an element $b$ in a set $B$ under a function
            $f:A\rightarrow{B}$ from a set $A$ to a set $B$ is the pre-image
            of the set $\{b\}$. That is:
            \begin{equation*}
                f^{\minus{1}}(\{b\})
                =\big\{\,a\in{A}\:|\;f(a)=b\,\big\}
            \end{equation*}
        \end{fdefinition}
        Since the set $b\in{B}$ we have that $\{b\}\subseteq{B}$
        (Def.~\ref{def:Subsets}) and thus there is no need to prove again that
        the fiber of an element $b\in{B}$ is a valid set, since
        Thm.~\ref{thm:Existence_of_Pre_Image} applies.
    \subsection{The Axiom of Choice and Diaconescu's Theorem}
        The next two axioms to be introduced are the most controversial of those
        listed in ZFC: The \textit{axiom of infinity} and the
        \textit{axiom of choice}. While the axiom of infinity only has a
        small number of critics, the axiom of choice is far more contentious.
        Choice is equivalent to many other statements that come across in
        almost all forms of mathematics (analysis, algebra, topology, etc.).
        Many of which are theorems we would \textit{want} to be true, and so
        accepting the axiom of choice allows us to prove them. In particular,
        the axioms presented thus far can be combined with the axiom of choice
        to prove the \textit{Law of the Excluded Middle}, a result known as
        Diaconescu's theorem, and this is our current goal.
        \begin{faxiom}{Axiom of Choice}{Axiom_of_Choice}
            If $\mathcal{O}$ is a non-empty set such that for all
            $\mathcal{U}\in\mathcal{O}$ it is true that $\mathcal{U}$ is
            non-empty, and if $\mathcal{F}$ is the union over $\mathcal{O}$:
            \index{Axiom!of Choice}
            \begin{equation*}
                \mathcal{F}=\bigcup_{\mathcal{U}\in\mathcal{O}}\mathcal{U}
            \end{equation*}
            then there is a function $f:\mathcal{O}\rightarrow\mathcal{F}$ such
            that, for all $x\in\mathcal{O}$, $f(x)\in{x}$
        \end{faxiom}
        Such a function is called a choice function. The axiom can be made
        obviously true if we word it one way, and obviously false if we word it
        another. To convince one that is it true will require talking about
        products. The Cartesian product has been defined using ordered pairs
        as defined by Kuratowski and allows us to order two elements. Given two
        sets $A$ and $B$ we can define an equivalent notion of the Cartesian
        product using the set of all functions from $\mathbb{Z}_{2}=\{0,1\}$
        into $A\cup{B}$ with a particular property:
        \begin{equation}
            A\times{B}=
            \big\{\,f:\mathbb{Z}_{2}\rightarrow{A}\cup{B}\;|\;
                f(0)\in{A}\textrm{ and }f(1)\in{B}\,\big\}
        \end{equation}
        To see why this is equivalent to the actual Cartesian product, note that
        the Cartesian product $A\times{B}$ is the set of all ordered pairs whose
        first entry lies in $A$ and whose second entry lies in $B$. Given
        $a\in{A}$ and $b\in{B}$, let $f:\mathbb{Z}_{2}\rightarrow{A}\cup{B}$ be
        the function such that $f(0)=a$ and $f(1)=b$. Then we can identify the
        ordered pair $(a,b)$ with $f$. Indeed, $(a,b)=(f(0),f(1))$ making our
        identification very explicit. We can now generalize to a collection of
        $n$ different sets and define the ordered $n$ tuple over a collection of
        $n$ sets to be the set of all functions from $\mathbb{Z}_{n}$ into the
        union over this collection with a similar property:
        \begin{equation}
            \prod_{k\in\mathbb{Z}_{n}}A_{k}
            =\big\{f:\mathbb{Z}_{n}\rightarrow\bigcup_{k\in\mathbb{Z}_{n}}^{n}
                A_{k}\;|\;f(k)\in{A}_{k}
                \textrm{for all }k\in\mathbb{Z}_{n}\big\}
        \end{equation}
        Given the function $f$ that maps $k$ to $a_{k}\in{A}_{k}$, we can
        identify $f$ with the ordered $n$ tuple:
        \begin{equation}
            f=(a_{0},\,a_{1},\,\dots,\,a_{k},\,\dots,\,a_{n-1})
            =(f(0),\,f(1),\,\dots,\,f(k),\,\dots,\,f(n-1))
        \end{equation}
        And thus we have a more general way of defining products. Note that
        swapping the order of the product is equivalent to changing functions,
        and thus we have that two $n$ tuples are equal if and only if all of
        their entries are equal. What's nice about our function definition is
        that it allows one to define products over \textit{arbitrary}
        collections. This is crucial for topology and analysis as we often wish
        to speak of \textit{infinite dimensional} spaces that are constructed
        using these abstract products. Given a set $I$, often called the
        \textit{index set}, such that for all $\mathcal{U}\in{I}$ it is true
        $\mathcal{U}$ is a set, we can form the product over $I$ by defining
        this to be the collection of all functions from $I$ into the union over
        $I$.
        \begin{equation}
            \prod_{i\in{I}}A_{i}
            =\big\{\,f:I\rightarrow\bigcup_{i\in{I}}A_{i}\;|\;
                f(i)\in{A}_{i}\textrm{ for all }i\in{I}\,\big\}
        \end{equation}
        The axiom of choice is thus equivalent to the statement
        \textit{The infinite product of non-empty sets is non-empty}. These
        functions that identify $k$ with the $k^{th}$ set are precisely
        choice functions. Phrasing it like this we see that the axiom of choice
        is somewhat obvious. The infinite product of non-empty sets is most
        likely enormous! Claiming it's non-empty thus seems trivial. It's thus
        unfortunate that this claim can not be proven with the other axioms
        we've developed. As stated before, the axiom of choice is equivalent to
        many other statements such as \textit{Zorn's lemma, Tychonoff's theorem,
        the well ordering theorem, every vector space has a basis, every set has
        a group structure}, and countless more. Many of these theorems have many
        applications to algebra, analysis, and topology, and since we would like
        to use them to prove other things we are forced to accept the axiom of
        choice. Many theorems in real analysis hide the use of the axiom of
        choice by constructing sequences \textit{by induction}. An attempt will
        be made to be very clear whenever the axiom of choice is used in a
        proof.
        \par\hfill\par
        We conclude this section by presenting Diaconescu's theorem.
        \begin{ftheorem}{Diaconescu's Theorem}{Diaconescus_Theorem}
            If $P$ is a proposition on sets and if $x$ is a set, then either
            $P(x)$ is true or $P(x)$ is false. That is, $P\lor\neg{P}$ is true.
        \end{ftheorem}
        \begin{bproof}
            Let $0=\emptyset$. By
            Thm.~\ref{thm:Existence_of_Set_Containing_Set}, we have that the set
            $\{0\}$ exists. Let $1=\{0\}$. Then since $0\in{1}$, $0\ne{1}$
            (Thm.~\ref{thm:Containment_NEqual_Underlying_Set}). Since $0$ and
            $1$ are sets, by Thm.~\ref{thm:Existence_of_Set_Built_from_Two_Sets}
            we have that the set $\{0,1\}$ exists. Let $Q$ be the proposition
            \textit{true if P(x) or } $x=0$, \textit{false otherwise}. By the
            axiom schema of specification
            (Ax.~\ref{ax:Axiom_Schema_of_Specification}) there exists a set
            $\mathcal{U}$ such that:
            \begin{equation}
                \mathcal{U}=\{\,x\in\{\,0,\,1\,\}\;|\;Q(x)\,\}
            \end{equation}
            Similarly, let $Q$ be the proposition \textit{true if P(x) or}
            $x=1$, \textit{false otherwise}. By the axiom schema of
            specification we have that the following set exists:
            \begin{equation}
                \mathcal{V}=\{\,x\in\{\,0,\,1\,\}\;|\;R(x)\,\}
            \end{equation}
            By Thm.~\ref{thm:Existence_of_Set_Built_from_Two_Sets}, we have that
            the set $\{\,\mathcal{U},\,\mathcal{V}\,\}$ exists. By the axiom of
            choice (Ax.~\ref{ax:Axiom_of_Choice}), there exists a function
            $f:\{\mathcal{U},\mathcal{V}\}\rightarrow%
             \bigcup\{\mathcal{U},\mathcal{V}\}$ such that
            $f(\mathcal{U})\in\mathcal{U}$ and $f(\mathcal{V})\in\mathcal{V}$.
            But then, by the definition of $\mathcal{U}$, either
            $f(\mathcal{U})=0$ or $P(x)$ is true. Similarly, either
            $f(\mathcal{V})=1$ or $P(x)$ is true. But since $0\ne{1}$, either
            $f(\mathcal{U})\ne{f}(\mathcal{V})$ or $P(x)$ is true. Again by the
            axiom of extensionality, and by the definition of $\mathcal{U}$ and
            $\mathcal{V}$, if $P(x)$ is true then $\mathcal{U}=\mathcal{V}$. But
            then $f(\mathcal{U})=f(\mathcal{V})$. But then, by the
            contrapositive, $\neg{P}(x)$ implies that
            $f(\mathcal{U})\ne{f}(\mathcal{V})$. But by extensionality, either
            $f(\mathcal{U})=f(\mathcal{V})$ or
            $f(\mathcal{U})\ne{f}(\mathcal{V})$, and thus either $P(x)$ or
            $\neg{P}(x)$. That is, $P\lor\neg{P}$ is true.
        \end{bproof}
        We can now prove things via \textit{proof by contradiction}. While we
        have made great efforts to justify every step of a proof thus far, we
        will often omit mention of Diaconescu's theorem as a justification for
        the law of the excluded middle and simply use it freely. We may rest
        easy knowing that we've proved the equivalence within the framework of
        ZFC. There are two more axioms remaining, that of infinity and
        replacement. The axiom of infinity is best introduced when we construct
        the natural numbers, and from there build the real numbers, and thus we
        shall delay its development briefly. The axiom of replacement needs a
        notion of class and thus will also be postponed.
        \section{The Structure of Sets}
    We've developed two \textit{operations} on sets thus far, that of union and
    intersections. Several of the properties of these two operations give rise
    to a structure known as a \textit{Boolean algebra}. Many theorems about sets
    can thus be proven in an algebraic setting by using the structure of a
    Boolean algebra, and thus it is our current goal to prove the basics about
    unions and operations, and then to develop the notion of a Boolean algebra.
    \subsection{Basic Theorems}
        With the law of the excluded middle in our toolbelt, we can now rapidly
        prove many basic and familiar results.
        \begin{theorem}
            \label{thm:Emptyset_Is_Subset}%
            If $A$ is a set, then $\emptyset\subseteq{A}$.
        \end{theorem}
        \begin{proof}
            For suppose not. Then there is an $x\in\emptyset$ such that
            $x\notin{A}$, a contradiction as for all $x$, it is true that
            $x\notin\emptyset$ (Def.~\ref{ax:Axiom_of_the_Empty_Set}).
            Therefore $\emptyset\subseteq{A}$.
        \end{proof}
        \begin{theorem}
            \label{thm:Subset_is_Transitive}%
            If $A$, $B$, and $C$ are sets, if $A\subseteq{B}$, and if
            $B\subseteq{C}$, then $A\subseteq{C}$.
        \end{theorem}
        \begin{proof}
            For suppose not. Then there is an $x\in{A}$ such that $x\notin{C}$.
            But $A$ is a subset of $B$ and thus $x\in{B}$
            (Def.~\ref{def:Subsets}). But $B$ is a subset of $C$ and therefore
            $x\in{C}$ (Def.~\ref{def:Subsets}). But $x\notin{C}$, a
            contradiction. Therefore, $A\subseteq{C}$.
        \end{proof}
        \begin{theorem}
            \label{thm:Set_Is_Subset_Of_Self}%
            If $A$ is a set, then $A\subseteq{A}$.
        \end{theorem}
        \begin{proof}
            Suppose not. Then there is an $x\in{A}$ such that $x\notin{A}$, a
            contradiction. Therefore $A\subseteq{A}$.
        \end{proof}
        We can now rigorously restate our claim that the empty set is unique.
        \begin{theorem}
            If $\emptyset'$ is a set with no elements,
            then $\emptyset=\emptyset'$.
        \end{theorem}
        \begin{proof}
            For suppose not. But $\emptyset'$ is a set, and thus
            $\emptyset\subseteq\emptyset'$ (Thm.~\ref{thm:Emptyset_Is_Subset}).
            Therefore $\emptyset'\nsubseteq\emptyset$. But then there is an $x$
            such that $x\in\emptyset'$ and $x\notin\emptyset$. But $\emptyset'$
            contains no elements, a contradiction. Thus
            $\emptyset'\subseteq\emptyset$. Therefore,
            $\emptyset=\emptyset'$ (Def.~\ref{def:Equal_Sets}).
        \end{proof}
        We can use this to define \textit{the} empty set.
        \begin{fdefinition}{The Empty Set}{Empty_Set}
            The empty set is the unique set $\emptyset$ such that for all $x$
            it is true that $x\notin\emptyset$.
        \end{fdefinition}
        \begin{theorem}
            \label{thm:Subsets_of_Equal_Sets}%
            If $A$, $B$, and $C$ are sets, if $A=B$, and if $C\subseteq{A}$,
            then $C\subseteq{B}$.
        \end{theorem}
        \begin{proof}
            For if $A=B$, then $A\subseteq{B}$ (Def.~\ref{def:Equal_Sets}). But
            if $C\subseteq{A}$ and $A\subseteq{B}$, then $C\subseteq{B}$
            (Thm.~\ref{thm:Subset_is_Transitive}). Therefore, etc.
        \end{proof}
        \begin{theorem}
            \label{thm:Equality_Reflexive}%
            If $A$ is a set, then $A=A$.
        \end{theorem}
        \begin{proof}
            For if $A$ is a set then $A\subseteq{A}$
            (Thm.~\ref{thm:Set_Is_Subset_Of_Self}). Thus,
            $A=A$ (Def.~\ref{def:Equal_Sets}).
        \end{proof}
        \begin{theorem}
            \label{thm:Equality_Symmetric}%
            If $A$ and $B$ are sets and if $A=B$, then $B=A$.
        \end{theorem}
        \begin{proof}
            For suppose not. If $B\ne{A}$, then either $B\nsubseteq{A}$ or
            $A\nsubseteq{B}$. But $A=B$, and thus $A\subseteq{B}$  and
            $B\subseteq{A}$ (Def.~\ref{def:Equal_Sets}),
            a contradiction. Therefore, etc.
        \end{proof}
        \begin{theorem}
            \label{thm:Equality_Transitive}%
            If $A$, $B$, and $C$ are sets, if $A=B$, and if $B=C$, then $A=C$.
        \end{theorem}
        \begin{proof}
            For if $B=C$, then $C\subseteq{B}$ (Def.~\ref{def:Equal_Sets}). But
            if $A=B$, then $B=A$ (Thm.~\ref{thm:Equality_Symmetric}). But if
            $B=A$ and and $C\subseteq{B}$, then $C\subseteq{A}$
            (Thm.~\ref{thm:Subsets_of_Equal_Sets}). And if $A=B$, then
            $A\subseteq{B}$ (Def.~\ref{def:Equal_Sets}). But if $B=C$ and
            $A\subseteq{B}$, then $A\subseteq{C}$
            (Thm.~\ref{thm:Subsets_of_Equal_Sets}). But it was proved that
            $C\subseteq{A}$, and thus $A=C$ (Def.~\ref{def:Equal_Sets}).
        \end{proof}
        These three properties,
        Thms.~\ref{thm:Equality_Reflexive}-%
        \ref{thm:Equality_Transitive}, are the key ingredients to define
        \textit{equivalence relations}. Equivalence relations are used to model
        the notion of equality in more abstract settings and are fundamental in
        the study of algebra and topology.
        \begin{theorem}
            \label{thm:Prop_Subset_Not_Equal}%
            If $A$ and $B$ are sets, and if $A\subsetneq{B}$, then there is an
            $x\in{B}$ such that $x\notin{A}$.
        \end{theorem}
        \begin{proof}
            For suppose not. Then for all $x\in{B}$ it is true that $x\in{A}$.
            But then $B\subseteq{A}$ (Def.~\ref{def:Subsets}).
            But $A\subseteq{B}$ and thus $A=B$ (Def.~\ref{def:Equal_Sets}).
            But $A\subsetneq{B}$ and therefore $A\ne{B}$, a contradiction.
            Therefore, etc.
        \end{proof}
        Theorem \ref{thm:Prop_Subset_Not_Equal} can be used as an equivalent
        definition of a proper subset. That is, a proper subset is a subset that
        is missing at least one element.
    \subsection{Operations on Sets}
        Similar to the arithmetic of real numbers, there are standard operations
        that can be performed on sets to obtain new sets. The four most common
        operations are union, intersection, set difference, and symmetric
        difference. As stated before, we wish to build the structure of sets in
        an algebraic manner. To do this requires the notion that the operations
        of intersection and unions are \textit{commutative},
        \textit{distributive}, have \textit{identities}, and have
        \textit{complements}.
        \begin{ltheorem}{Commutative Law of Unions}{Commutative_Law_of_Unions}
            If $A$ and $B$ are sets, then $A\cup{B}=B\cup{A}$.
        \end{ltheorem}
        \begin{proof}
            For if $x\in{A}\cup{B}$, then either $x\in{A}$
            or $x\in{B}$, or both
            (Def.~\ref{def:Union_of_Two_Sets}). But then either
            $x\in{B}$ or $x\in{A}$, or both, and therefore
            $x\in{B}\cup{A}$ (Def.~\ref{def:Union_of_Two_Sets}).
            But then for all $x\in{A}\cup{B}$ it is true that
            $x\in{B}\cup{A}$, and therefore
            $A\cup{B}\subseteq{B}\cup{A}$
            (Def.~\ref{def:Subsets}). Similarly,
            $B\cup{A}\subseteq{A}\cup{B}$, and thus
            $A\cup{B}=B\cup{A}$ (Def.~\ref{def:Equal_Sets}).
            Therefore, etc.
        \end{proof}
        When taking the union of two sets, we obtain a \textit{larger} set, in
        a sense. Again relying on the analogy of arithmetic, given two
        non-negative integers $a$ and $b$, it is true that $a\leq{a}+b$.
        Equality is obtained if and only if either $a$ or $b$ is equal to zero.
        The empty set thus acts as the \textit{zero} of unions. Also, given
        three non-negative integers $a$, $b$, and $c$, if $b\leq{c}$, then
        $a+b\leq{a}+c$. A similar result will hold for sets and unions.
        \begin{theorem}
            \label{thm:Union_is_Bigger}%
            If $A$ and $B$ are sets, then $A\subseteq{A}\cup{B}$.
        \end{theorem}
        \begin{proof}
            For suppose not. Then there is an $x\in{A}$ such
            that $x\notin{A}\cup{B}$. But if $x\in{A}$, then
            $x\in{A}$ or $x\in{B}$ and thus $x\in{A}\cup{B}$
            (Def.~\ref{def:Union_of_Two_Sets}), a
            contradiction. Therefore, etc.
        \end{proof}
        \begin{theorem}
            \label{thm:Union_With_Lesser_Set}%
            If $A$, $B$, and $C$ are sets, and if
            $B\subseteq{C}$, then
            $A\cup{B}\subseteq{A}\cup{C}$.
        \end{theorem}
        \begin{proof}
            For if $x\in{A}\cup{B}$, then either $x\in{A}$,
            or $x\in{B}$, or both
            (Def.~\ref{def:Union_of_Two_Sets}). But $B$ is a
            subset of $C$, and therefore if $x\in{B}$, then
            $x\in{C}$ (Def.~\ref{def:Subsets}).
            Thus, if $x\in{A}$ or $x\in{B}$, then
            $x\in{A}$ or $x\in{C}$, and therefore
            $x\in{A}\cup{C}$ (Def.~\ref{def:Union_of_Two_Sets}).
            Thus, $A\cup{B}\subseteq{A}\cup{C}$
            (Def.~\ref{def:Subsets}). Therefore, etc.
        \end{proof}
        \begin{theorem}
            If $A$, $B$, $C$, and $D$ are sets, if
            $A\subseteq{C}$, and if $B\subseteq{D}$, then
            $A\cup{B}\subseteq{C}\cup{D}$.
        \end{theorem}
        \begin{proof}
            For if $B\subseteq{D}$, then
            $A\cup{B}\subseteq{A}\cup{D}$
            (Thm.~\ref{thm:Union_With_Lesser_Set}).
            But $A\cup{D}=D\cup{A}$
            (Thm.~\ref{thm:Commutative_Law_of_Unions}).
            But if $A\subseteq{C}$, then
            $D\cup{A}\subseteq{D}\cup{C}$
            (Thm.~\ref{thm:Union_With_Lesser_Set}). But
            $D\cup{C}=C\cup{D}$
            (Thm.~\ref{thm:Commutative_Law_of_Unions}).
            And if $A\cup{B}\subseteq{A}\cup{D}$ and
            $A\cup{D}\subseteq{C}\cup{D}$, then
            $A\cup{B}\subseteq{C}\cup{D}$
            (Thm.~\ref{thm:Subset_is_Transitive}).
            Therefore, etc.
        \end{proof}
        Taking the union of subsets is redundant, as we
        simply obtain the larger set. This starts to break
        down the analogy between sets and arithmetic, since
        there is only one \textit{zero}. That is, there is
        only one number $b$ such that $a+b=a$, and that is
        $b=0$. While any subset acts as a \textit{zero} of a
        given set, the empty set has the property that it
        acts as a zero for \textit{every} set. It is the only
        set with this property, and thus the analogy with
        arithmetic is restored.
        \begin{theorem}
            \label{thm:Union_With_Subset}%
            If $A$ and $B$ are sets, and if
            $A\subseteq{B}$, then $A\cup{B}=B$.
        \end{theorem}
        \begin{proof}
            For if $A$ and $B$ are sets, then
            $B\subseteq{A}\cup{B}$
            (Thm.~\ref{thm:Union_is_Bigger}).
            But if $A\subseteq{B}$, then for all $x\in{A}$,
            it is true that $x\in{B}$
            (Def.~\ref{def:Subsets}). Thus if $x\in{A}$ or if
            $x\in{B}$, then $x\in{B}$. But then, for all
            $x\in{A}\cup{B}$, it is true that $x\in{B}$, and
            therefore $A\cup{B}\subseteq{B}$
            (Def.~\ref{def:Subsets}). Thus,
            $A\cup{B}=B$ (Def.~\ref{def:Equal_Sets}).
            Therefore, etc.
        \end{proof}
        \begin{theorem}
            \label{thm:Union_with_Emptyset}%
            If $A$ is a set, then $A=\emptyset\cup{A}$.
        \end{theorem}
        \begin{proof}
            For $\emptyset\subseteq{A}$
            (Thm.~\ref{thm:Emptyset_Is_Subset}) and
            therefore $\emptyset\cup{A}=A$
            (Thm.~\ref{thm:Union_With_Subset}).
        \end{proof}
        \begin{theorem}
            \label{thm:Empty_Set_Is_Zero_for_Unions}%
            If $A$ is a set such that, for any set $B$, it is
            true that $A\cup{B}=B$, then $A$ is the
            empty set.
        \end{theorem}
        \begin{proof}
            For suppose not. If $A\ne\emptyset$, then there
            is an $x\in{A}$ (Def.~\ref{ax:Axiom_of_the_Empty_Set}).
            But then $B=\{A\}$ is a set
            (Def.~\ref{def:Sets}). But then $x\in{A}\cup{B}$
            (Def.~\ref{def:Union_of_Two_Sets}). But $x\notin{B}$,
            and thus $A\cup{B}\ne{B}$
            (Def.~\ref{def:Equal_Sets}), a contradiction
            since $A$ is such that for any set $B$, it is
            true that $A\cup{B}=B$. Therefore, etc.
        \end{proof}
        Thm.~\ref{thm:Empty_Set_Is_Zero_for_Unions} proves
        the assertion that the empty set is the zero of set
        union. The converse of
        Thm.~\ref{thm:Union_With_Subset} can be proved as
        well.
        \begin{theorem}
            \label{thm:Conv_Union_Is_Bigger}%
            If $A$ and $B$ are sets, and if
            $A\cup{B}\subseteq{A}$, then $A\cup{B}=A$.
        \end{theorem}
        \begin{proof}
            For $A\subseteq{A}\cup{B}$
            (Thm.~\ref{thm:Union_is_Bigger}). But by
            hypothesis, $A\cup{B}\subseteq{A}$. But then
            $A=A\cup{B}$ (Def.~\ref{def:Equal_Sets}).
            Therefore, etc.
        \end{proof}
        \begin{theorem}
            \label{thm:Union_is_Equal}%
            If $A$ and $B$ are sets, and if
            $A\cup{B}\subseteq{A}$, then $B\subseteq{A}$.
        \end{theorem}
        \begin{proof}
            For if $A\cup{B}\subseteq{A}$, then
            $A\cup{B}=A$
            (Thm.~\ref{thm:Conv_Union_Is_Bigger}). And also,
            $B\subseteq{A}\cup{B}$
            (Thm.~\ref{thm:Union_is_Bigger}). But if
            $A\cup{B}=A$ and $B\subseteq{A}\cup{B}$, then
            $B\subseteq{A}$
            (Thm.~\ref{thm:Subsets_of_Equal_Sets}).
            Therefore, etc.
        \end{proof}
        We'll wrap up unions by showing that the operation
        is associative. Once again relying on the analogy
        of arithmetic, given three real numbers $a$, $b$,
        and $c$, it is true that $a+(b+c)=(a+b)+c$. This
        is called the associative law of addition. Combining
        this law with the commutative law shows that the
        order in which three real numbers are added is
        irrelevant. Applying induction, we see that given
        any finite collection of real numbers, the order in
        which we add them is again irrelevant. The same holds
        true for the union of sets.
        \begin{theorem}
            \label{thm:Redundant_Union}%
            If $A$, $B$, and $C$ are sets and if $A\subseteq{B}$, then
            $A\cup(B\cup{C})=B\cup{C}$
        \end{theorem}
        \begin{proof}
            For $B\cup{C}\subseteq{A}\cup(B\cup{C})$
            (Thm.~\ref{thm:Union_is_Bigger}). But
            $A\cup(B\cup{C})=(A\cup{B})\cup{C}$
            (Thm.~\ref{thm:Associative_Law_of_Unions}).
            And since $A$ is a subset of $B$, $A\cup{B}=B$
            (Thm.~\ref{thm:Union_With_Subset}), and thus
            $(A\cup{B})\cup{C}=B\cup{C}$. Thus, $B\cup{C}=A\cup(B\cup{C})$
            (Thm.~\ref{thm:Equality_Transitive}). Therefore, etc.
        \end{proof}
        \begin{figure}[H]
            \centering
            \captionsetup{type=figure}
            \centering
            %--------------------------------Dependencies----------------------------------%
%   tikz                                                                       %
%-------------------------------Main Document----------------------------------%
\begin{tikzpicture}[line width=0.2mm]

    % Coordinates for the two circles.
    \coordinate (A) at ( 0.5, -0.3);
    \coordinate (B) at ( 0.0,  0.0);
    \coordinate (I) at (-2.3,  1.5);

    % Draw a rectangle for the universe set.
    \draw (-3,-2.3) rectangle (2.5,2.3);

    \draw[fill=cyan] (A) circle (0.85);
    \draw (B) circle (2);

    \node at (A) [above=0.1cm] {$A$};
    \node at (B) [above=1.0cm] {$B$};
    \node at (I) {$A\cap{B}$};
\end{tikzpicture}
            \caption{Visual for Thm.~\ref{thm:Intersection_of_Subset}.}
            \label{fig:Union_Intersection_venn_diagram}
        \end{figure}
        \begin{ltheorem}{Commutative Law of Intersections}{Commut_Law_Intersec}
            If $A$ and $B$ are sets, then $A\cap{B}=B\cap{A}$.
        \end{ltheorem}
        \begin{proof}
            For if $x\in{A}\cap{B}$, then $x\in{A}$ and
            $x\in{B}$. But then $x\in{B}$ and $x\in{A}$,
            and therefore $x\in{B}\cap{A}$
            (Def.~\ref{def:Intersection_of_Two_Sets}). But then
            for all $x\in{A}\cap{B}$ it is true that
            $x\in{B}\cap{A}$, and therefore
            $A\cup{B}\subseteq{B}\cup{A}$
            (Def.~\ref{def:Subsets}). Similarly,
            $B\cap{A}\subseteq{A}\cap{B}$, and thus
            $A\cap{B}=B\cap{A}$ (Def.~\ref{def:Equal_Sets}).
            Therefore, etc.
        \end{proof}
        \begin{theorem}
            \label{thm:Intersection_is_Smaller}%
            If $A$ snd $B$ are sets, then
            $A\cap{B}\subseteq{A}$.
        \end{theorem}
        \begin{proof}
            If $x\in{A}\cap{B}$, then $x\in{A}$ and
            $x\in{B}$, and thus $x\in{A}$. Therefore, etc.
        \end{proof}
        \begin{theorem}
            \label{thm:Intersection_with_Lesser_Set}%
            If $A$, $B$, and $C$ are sets, and if
            $B\subseteq{C}$, then
            $A\cap{B}\subseteq{A}\cap{C}$.
        \end{theorem}
        \begin{proof}
            For if $x\in{A}\cap{B}$, then $x\in{A}$ and
            $x\in{B}$ (Def.~\ref{def:Intersection_of_Two_Sets}).
            But $B$ is a subset of $C$, and thus if
            $x\in{B}$, then $x\in{C}$
            (Def.~\ref{def:Subsets}). But then $x\in{A}$ and
            $x\in{C}$, and therefore $x\in{A}\cap{C}$
            (Def.~\ref{def:Intersection_of_Two_Sets}). But
            then $A\cap{B}\subseteq{A}\cap{C}$
            (Def.~\ref{def:Subsets}). Therefore, etc.
        \end{proof}
        \begin{theorem}
            \label{thm:Intersection_is_Equal}%
            If $A$ and $B$ are sets, and if
            $A=A\cap{B}$, then $A\subseteq{B}$.
        \end{theorem}
        \begin{proof}
            For suppose not. Then there is an $x\in{A}$ such
            that $x\notin{B}$. But since $A=A\cap{B}$,
            if $x\in{A}$ then $x\in{A}\cap{B}$
            (Def.~\ref{def:Equal_Sets}). But if
            $x\in{A}\cap{B}$, then $x\in{B}$
            (Thm.~\ref{thm:Intersection_is_Smaller}),
            a contradiction. Therefore, etc.
        \end{proof}
        \begin{theorem}
            \label{thm:Intersection_of_Subset}%
            If $A$ and $B$ are sets, and if
            $A\subseteq{B}$, then $A\cap{B}=A$.
        \end{theorem}
        \begin{proof}
            For $A\cap{B}\subseteq{A}$
            (Thm.~\ref{thm:Intersection_is_Smaller}). But
            since $A$ is a subset of $B$, if $x\in{A}$, then
            $x\in{B}$ (Def.~\ref{def:Subsets}). But then
            $x\in{A}\cap{B}$
            (Def.~\ref{def:Intersection_of_Two_Sets}). Therefore,
            $A\subseteq{A}\cap{B}$ (Def~\ref{def:Subsets})
            and thus $A=A\cap{B}$ (Def~\ref{def:Equal_Sets}).
            Therefore, etc.
        \end{proof}
        \begin{theorem}
            \label{thm:Conv_Intersection_is_Smaller}%
            If $A$ and $B$ are sets, and if
            $A\subseteq{A}\cap{B}$, then $A=A\cap{B}$.
        \end{theorem}
        \begin{proof}
            For $A\cap{B}\subseteq{A}$
            (Thm.~\ref{thm:Intersection_is_Smaller}). But
            by hypothesis, $A\subseteq{A}\cap{B}$, and thus
            $A=A\cap{B}$ (Def.~\ref{def:Equal_Sets}).
            Therefore, etc.
        \end{proof}
        \begin{theorem}
            If $A$ is a set, then $\emptyset\cap{A}=\emptyset$.
        \end{theorem}
        \begin{proof}
            For $\emptyset\subseteq{A}$ (Thm.~\ref{thm:Emptyset_Is_Subset}), and
            therefore $\emptyset\cap{A}=\emptyset$
            (Thm.~\ref{thm:Intersection_of_Subset}).
        \end{proof}
        \begin{theorem}
            \label{thm:Redundant_Intersection}%
            If $A$, $B$, and $C$ are sets and if
            $B\subseteq{A}$, then
            $A\cap(B\cap{C})=B\cap{C}$.
        \end{theorem}
        \begin{proof}
            For $A\cup(B\cup{C})\subseteq{B}\cup{C}$
            (Thm.~\ref{thm:Intersection_is_Smaller}). But
            $A\cap(B\cap{C})=(A\cap{B})\cap{C}$
            (Thm.~\ref{thm:Assoc_Law_Intersec}).
            And since $B$ is a subset of $A$,
            $A\cap{B}=A$
            (Thm.~\ref{thm:Intersection_of_Subset}),
            and thus $(A\cap{B})\cap{C}=B\cap{C}$. Thus,
            $B\cap{C}=A\cap(B\cap{C})$
            (Thm.~\ref{thm:Equality_Transitive}).
            Therefore, etc.
        \end{proof}
        \begin{theorem}
            \label{thm:First_Pseudo_Dist_Law_Union}%
            If $A$, $B$, and $C$ are sets, then
            $(B\cap{C})\subseteq(A\cup{B})\cap(A\cup{C})$.
        \end{theorem}
        \begin{proof}
            For $B\subseteq{A}\cup{B}$
            (Thm.~\ref{thm:Union_is_Bigger}). But then
            $B\cap{C}\subseteq(A\cup{B})\cap{C}$
            (Thm.~\ref{thm:Intersection_with_Lesser_Set}).
            But $C\subseteq{A}\cup{C}$
            (Thm.~\ref{thm:Union_is_Bigger}), and thus
            $(A\cup{B})\cap{C}%
            \subseteq(A\cup{B})\cap{A}\cup{C}$
            (Thm.~\ref{thm:Intersection_with_Lesser_Set}).
            But it was just proved that
            $B\cap{C}\subseteq(A\cup{B})\cap{C}$, and
            therefore by transivity,
            $(B\cap{C})\subseteq(A\cup{B})\cap(A\cup{C})$
            (Thm.~\ref{thm:Subset_is_Transitive}).
            Therefore, etc.
        \end{proof}
        \begin{ltheorem}{Distributive Law of Unions}
            {Distributive_Law_Union}
            If $A$, $B$, and $C$ are sets, then
            $A\cup(B\cap{C})=(A\cup{B})\cap(A\cup{C})$.
        \end{ltheorem}
        \begin{proof}
            For $(B\cap{C})\subseteq(A\cup{B})\cap(A\cup{C})$
            (Thm.~\ref{thm:First_Pseudo_Dist_Law_Union}).
            But then:
            \begin{equation}
                A\cup(B\cap{C})\subseteq
                A\cup\Big((A\cup{B})\cap(A\cup{C})\Big)
            \end{equation}
            But $A\cup((A\cup{B})\cap(A\cup{C}))%
                =(A\cup{B})\cap(A\cup{C})$, and therefore:
            \begin{equation}
                A\cup(B\cap{C})\subseteq
                (A\cup{B})\cap(A\cup{C})
            \end{equation}
        \end{proof}
        \begin{ltheorem}{Distributive Law of Intersections}
            {Distributive_Law_Intersections}
            If $A$, $B$, and $C$ are sets, then
            $A\cap(B\cup{C})=(A\cap{B})\cup(A\cap{C})$.
        \end{ltheorem}
        \begin{proof}
            Hi
        \end{proof}
        If $A$ and $B$ are sets, and if
        $C\subseteq{A}\cup{B}$, then
        either $C\subseteq{A}$ or $C\subseteq{B}$, or both.
        It is possible that $C\subseteq{A}\cup{B}$ and yet
        $C$ and $B$ have no elements in common, as long
        as $C\subseteq{A}$. As an example,
        take $A$ and $B$ to be disjoint sets. Then
        $A\subseteq{A}\cup{B}$, yet $A$ and $B$ have no
        elements in common. If $C\subseteq{A}\cap{B}$, then
        it must be true that $C\subseteq{A}$ and
        $C\subseteq{B}$.
        Using arithmetic as an analogy, the empty set
        acts somewhat like a zero element. It is an identity
        element under set unions, and collapses everything
        down to zero under set intersections. Continuing
        with this analogy, we discuss set difference.
        \begin{ldefinition}{Set Difference}{Set_Difference}
            The set difference of a set $A$ with respect to
            a set $B$, denoted $B\setminus{A}$, is the set:
            \begin{equation}
                B\setminus{A}=\{x\in{B}:x\notin{A}\}
            \end{equation}
        \end{ldefinition}
        \begin{ldefinition}{Symmetric Difference}
            {Symmetric_Difference}
            The symmetric difference of $A$ and $B$, denoted
            $A\ominus{B}$, is the set:
            \begin{equation}
                A\ominus{B}
                =(A\cup{B})\setminus(A\cap{B})
            \end{equation}
        \end{ldefinition}
        While set difference appears similar to subtraction,
        the two have their differences. For any two real
        numbers $a$ and $b$, it is always true that
        $b=a-(a-b)$. For sets this is not true. For let $A$
        be the empty set, and let $B$ be non-empty.
        Then $A\setminus(A\setminus{B})=\emptyset$, which
        is not $B$. Set differences can not be easily
        simplified. The notion is not associative, nor is it
        commutative. If there is a larger \textit{universe}
        set, then set difference can be related to
        intersection.
        \begin{theorem}
            \label{thm:Set_Difference_As_Intersection}%
            If $A$, $B$, and $C$ are sets, and if
            $A\subseteq{C}$ and $B\subseteq{C}$, then:
            \begin{equation}
                B\setminus{A}=B\cap(C\setminus{A})
            \end{equation}
        \end{theorem}
        \begin{proof}
            For if $x\in{B}\setminus{A}$, then
            $x\in{B}$ and $x\notin{A}$. But
            $B\subseteq{C}$, and thus if $x\in{B}$, then
            $x\in{C}$. But if $x\notin{A}$, then
            $x\in{C}\setminus{A}$. Therefore
            $B\setminus{A}\subseteq{B}\cap(C\setminus{A})$.
            Similarly,
            $B\cap(C\setminus{A})\subseteq{B}\setminus{A}$,
            and therefore
            $B\setminus{A}={B}\cap(C\setminus{A})$.
        \end{proof}
        Similar to unions and intersections,
        set differences and symmetric differences can be
        visualized by Venn diagrams, as shown in
        Fig.~\ref{fig:Difference_Sym_Venn_Diagram}.
        \begin{figure}[H]
            \centering
            \captionsetup{type=figure}
            \begin{subfigure}[b]{\textwidth}
                \centering
                \documentclass[crop,class=article]{standalone}
%----------------------------Preamble-------------------------------%
\usepackage{tikz}                       % Drawing/graphing tools.
%--------------------------Main Document----------------------------%
\begin{document}
    \begin{tikzpicture}
        \draw (-2.5,-2) rectangle (2.5,2);
        \fill[cyan] (-0.8cm,0) circle (1.5cm);
        \fill[white] (0.8cm,0) circle (1.5cm);
        \draw (-0.8cm,0) circle (1.5cm);
        \draw (0.8cm,0) circle (1.5cm);
        \node at (-1,1.1) {$A$};
        \node at (1,1.1) {$B$};
        \node at (-1,1.75) {$A\setminus{B}$};
    \end{tikzpicture}
\end{document}
                \subcaption{Set Difference}
            \end{subfigure}
            \begin{subfigure}[b]{\textwidth}
                \centering
                %--------------------------------Dependencies----------------------------------%
%   tikz                                                                       %
%-------------------------------Main Document----------------------------------%
\begin{tikzpicture}[line width=0.2mm]

    % Coordinates for the centers of the circles.
    \coordinate (C1) at (-1.3, 0);
    \coordinate (C2) at ( 1.3, 0);

    % Coordinates for the labels.
    \coordinate (A) at (-1.3, 1.2);
    \coordinate (B) at ( 1.3, 1.2);
    \coordinate (S) at ( 0.0, 2.5);

    % Rectangle indicating the universe set.
    \draw (-4, -3) rectangle (4, 3);

    % Fill in the circle with cyan.
    \draw[fill=cyan, draw=none] (C1) circle (2);
    \draw[fill=cyan, draw=none] (C2) circle (2);

    % Fill in the circle with cyan.
    \draw[fill=white, draw=none] (0, -1.51987) arc(-49.46:49.46:2)
                                               arc(130.54:229.46:2);

    % Give outlines to the circles.
    \draw (C1) circle (2);
    \draw (C2) circle (2);

    % Labels.
    \node at (A) {$A$};
    \node at (B) {$B$};
    \node at (S) {$A\ominus{B}$};
\end{tikzpicture}
                \subcaption{Symmetric Difference}
            \end{subfigure}
            \caption[Venn Diagrams for Set Difference
                    and Symmetric Difference]
                    {Venn Diagrams Depicting the Set
                    Difference and Symmetric Difference
                    of the Sets $A$ and $B$.}
            \label{fig:Difference_Sym_Venn_Diagram}
        \end{figure}
        The concept of set difference can then be used to define the
        concept of complement.
        Thm.~\ref{thm:Set_Difference_As_Intersection} can then be
        translated into the notation of complements as follows:
        \begin{theorem}
            If $A$, $B$, and $\Omega$ are sets,
            $A,B\subseteq\Omega$, and if $A^{C}$ is the
            complement of $A$ with respect to $\Omega$, then:
            \begin{equation}
                B\setminus{A}=B\cap{A}^{C}
            \end{equation}
        \end{theorem}
        \begin{proof}
            By the definition of complement,
            $A^{C}=\Omega\setminus{A}$.
            As $A\subseteq\Omega$ and $B\subseteq\Omega$, by
            Thm.~\ref{thm:Set_Difference_As_Intersection},
            $B\setminus{A}=B\cap(\Omega\setminus{A})$,
            and therefore $B\setminus{A}=B\cap{A}^{C}$.
        \end{proof}
        The main result about complements are known as
        DeMorgan's Laws. The laws relate unions and
        intersections by means of complements. The general
        laws hold for arbitrary unions and arbitrary
        intersections, as will be shown later.
        \begin{ftheorem}{DeMorgan's Laws}{MEASURE_DEMORGAN}
            If $A$, $B$, and $\Omega$ are sets, if
            $A\subseteq\Omega$ and $B\subseteq\Omega$, then:
            \begin{subequations}
                \begin{align}
                    \big(A\cap{B}\big)^{C}
                    &=A^{C}\cup{B}^{C}\\
                    \big(A\cup{B}\big)^{C}
                    &=A^{C}\cap{B}^{C}
                \end{align}
            \end{subequations}
        \end{ftheorem}
        With this, we can prove some results about
        set differences.
        \begin{theorem}
            If $A$ and $B$ are sets, then:
            \begin{equation}
                A=\big(A\cap{B}\big)
                    \cup\big(A\setminus{B}\big)
            \end{equation}
        \end{theorem}
        \begin{proof}
            For let $\Omega=A\cup{B}$. Then
            $A\subseteq\Omega$ and $B\subseteq\Omega$,
            and thus:
            \begin{subequations}
                \begin{align}
                    \big(A\cap{B})\cup\big(A\setminus{B}\big)
                    &=\big(A\cap{B}\big)
                        \cup\big(A\cap{B}^{C}\big)\\
                    &=A\cap(B\cup{B}^{C})\\
                    &=A\cap\Omega
                \end{align}
            \end{subequations}
            But by Thm.~\ref{thm:Intersection_is_Smaller},
            $A\cap\Omega=A$. Therefore, etc.
        \end{proof}
        \begin{theorem}
            If $A$, $B$, and $C$ are sets, then:
            \begin{equation}
                A\cap\big(B\setminus{C}\big)
                =\big(A\cap{B}\big)\cap\big(A\setminus{C}\big)
            \end{equation}
        \end{theorem}
        \begin{proof}
            For:
            \begin{subequations}
                \begin{align}
                    A\cap\big(B\setminus{C}\big)
                    &=A\cap\big(B\cap{C}^{C}\big)\\
                    &=\big(A\cap{A}\big)
                        \cap\big(B\cap{C}^{C}\big)\\
                    &=\big(A\cap{B}\big)
                        \cap\big(A\cap{C}^{C}\big)\\
                    &=\big(A\cap{B}\big)
                        \cap\big(A\setminus{C}\big)
                \end{align}
            \end{subequations}
        \end{proof}
        Intersections do distribute over set differences.
        \begin{theorem}
            If $A$, $B$, and $C$ are sets, then:
            \begin{equation}
                A\cap(B\setminus{C})=
                (A\cap{B})\setminus(A\cap{C})
            \end{equation}
        \end{theorem}
        \begin{proof}
            For:
            \begin{subequations}
                \begin{align}
                    \big(A\cap{B}\big)\setminus
                        \big(A\cap{C}\big)
                    &=\big(A\cap{B}\big)
                        \cap\big(A\cap{C}\big)^{C}\\
                    &=\big(A\cap{B}\big)
                        \cap\big(A^{C}\cup{C}^{C}\big)\\
                    &=\big[\big(A\cap{B}\big)\cap{A}^{C}\big]
                        \cup\big[\big({A}\cap{B}\big)
                        \cap{C}^{C}\big]\\
                    &=\big[\big(A\cap{A}^{C}\big)\cap{B}\big]
                        \cup\big[\big(A\cap{B}\big)
                        \cap{C}^{C}\big]\\
                    &=\emptyset\cup\big[\big(A\cap{B}\big)
                        \cap{C}^{C}\big]\\
                    &=\big(A\cap{B}\big)\cap{C}^{C}\\
                    &=A\cap\big(B\cap{C}^{C}\big)\\
                    &=A\cap\big(B\setminus{C}\big)
                \end{align}
            \end{subequations}
            Therefore, etc.
        \end{proof}
        Unions do not, however. For let $A$ be non-empty
        and let $A=B=C$. Then $A\cup(B\setminus{C})=A$, but
        $(A\cup{B})\setminus(A\cup{C})=\emptyset$.
        \begin{theorem}
            If $A$ and $B$ are sets and $A\subset B$,
            then $B\setminus(B\setminus A)=A$.
        \end{theorem}
        \begin{proof}
            For:
            \begin{align}
                Yo
            \end{align}
            $[x\in B\setminus(B\setminus{A})]%
            \Rightarrow[x\in{B}\land{x}\notin%
            \{x\in{B}:x\notin{A}\}]%
            \Rightarrow[x\in{A}\subset{B}]$.
            $[x\in{A}]\Rightarrow[x\notin{B}\setminus{A}]%
            \Rightarrow[x\in{B}\setminus(B\setminus{A})]$.
        \end{proof}
        The previous theorem shows that $(A^C)^{C}=A$.
        % Untrue garbage.
        % If $A$ and $B$ are sets, and if $C\subseteq{A}\cup{B}$, then
        % either $C\subseteq{A}$ or $C\subseteq{B}$, or both. It is
        % possible that $C\subseteq{A}\cup{B}$ and yet $C$ and $B$ have no
        % elements in common, as long as $C\subseteq{A}$. As an example,
        % take $A$ and $B$ to be disjoint sets. Then $A\subseteq{A}\cup{B}$,
        % yet $A$ and $B$ have no elements in common. If
        % $C\subseteq{A}\cap{B}$, then it must be true that
        % $C\subseteq{A}$ and $C\subseteq{B}$.
        % As with the notions of unions and intersections, set differences and
        % symmetric differences can be visualized using Venn diagrams.
        \begin{theorem}
            \label{thm:MEASURE_THEORY_SET_DIFFERENCE_AS_INTERSECTION}
            If $A$, $B$, and $C$ are sets, and if $A\subseteq{C}$
            and $B\subseteq{C}$, then:
            \begin{equation}
                B\setminus{A}=B\cap(C\setminus{A})
            \end{equation}
        \end{theorem}
        \begin{proof}
            For if $x\in{B}\setminus{A}$, then
            $x\in{B}$ and $x\notin{A}$. But
            $B\subseteq{C}$, and thus if $x\in{B}$, then $x\in{C}$.
            But if $x\notin{A}$, then $x\in{C}\setminus{A}$. Therefore
            $B\setminus{A}\subseteq{B}\cap(C\setminus{A})$.
            Similarly, $B\cap(C\setminus{A})\subseteq{B}\setminus{A}$,
            and therefore $B\setminus{A}={B}\cap(C\setminus{A})$.
        \end{proof}
        While set difference appears similar to subtraction that one finds in
        basic arithmetic, the two have their differences. For any two real
        numbers $a$ and $b$, $b=a-(a-b)$. For sets this is not true. For let $A$
        be the empty set, and let $B$ be non-empty. Then
        $A\setminus(A\setminus{B})=\emptyset$, which is not $B$.
        Also, while it may seems convincing that
        $A\setminus(B\setminus{A})=A\setminus{B}$, this is not true. For
        let $A$ be a non-empty set and let $B=A$. Then
        $A\setminus(B\setminus{A})=A$, but $A\setminus{B}=\emptyset$.
        The concept of set difference can then be used to define the
        concept of complement.
        %------------------------------------------------------------------------------%
\section{Relations}
    \label{Section:ZFC:Elementary_Set_Theory:Relations}%
    \begin{fdefinition}{Relation on a Set}{Relation_on_a_Set}
        A \gls{relation} on a \gls{set} $A$ is a \gls{subset} $R$ of the
        \gls{Cartesian product} $A\times{A}$.
    \end{fdefinition}
    We use a special notation for relations on a set.
    \begin{fnotation}{Relation Notation}{Relation_Notation}
        If $A$ is a set, if $R$ is a relation on $A$, and if $(a,b)\in{R}$, we
        write $aRb$.
        \begin{equation*}
            \forall_{x}\forall_{y}\big(aRb\big)\Leftrightarrow
            \big((a,b)\in{R}\big)
        \end{equation*}
    \end{fnotation}
    For a relation $R$ it is not necessary true that $aRb$ implies $bRa$, nor is
    it necessarily true that $aRa$. These are called symmetric and reflexive
    relations, respectively.
    \begin{lexample}{Examples of Relations}{Examples_of_Relations}
        Let $A=\mathbb{R}$ and consider the relation of equality. That is, let
        $R_{=}\subseteq\mathbb{R}^{2}$ be defined by:
        \begin{equation}
            R_{=}=\{\,(x,y)\in\mathbb{R}^{2}\;|\;x=y\,\}
        \end{equation}
        Then $R_{=}$ is a relation on $\mathbb{R}^{2}$. Rather than writing
        $(x,y)\in{R_{=}}$ or $xR_{=}y$ we commonly write $x=y$. Note that this
        relation is defined entirely by the \textit{diagonal} of the Cartesian
        product $\mathbb{R}\times\mathbb{R}$. Another simple relation is that of
        ordering. Let $R_{<}$ be defined as follows:
        \begin{equation}
            R_{<}=\{\,(x,y)\in\mathbb{R}^{2}\;|\;x<y\,\}
        \end{equation}
        This is also a relation since it is a subset of the Cartesian product,
        but it is a slightly more complicated one. There are many
        \textit{off-diagonal} elements of this relation.
    \end{lexample}
    \begin{theorem}
        If $B$ is a set, if $A\subseteq{B}$, and if $R$ is a relation on $B$,
        then there is a relation $R_{A}$ such that $R_{A}$ is a relation on
        $A$ and $R_{A}\subseteq{R}$.
    \end{theorem}
    \begin{proof}
        For let $P$ be the proposition \textit{True if} $(x,y)\in{A}^{2}$,
        \textit{false otherwise}. By the axiom schema of specification
        (Ax.~\ref{ax:Axiom_Schema_of_Specification}) there is a set:
        \begin{equation}
            R_{A}=\big\{\,(x,y)\in{R}\;|\;P\big((x,y)\big)\,\big\}
        \end{equation}
        But then $(x,y)\in{R}_{A}$ if and only if $(x,y)\in{R}$ and
        $(x,y)\in{A}^{2}$.
    \end{proof}
    This set is called the \textit{restriction} of $R$ to the subset $A$.
    \begin{fdefinition}{Restriction of a Relation}{Restriction_of_a_Relation}
        The restriction of a relation $R$ on a set $B$ to a subset $A$ is the
        set $R_{A}$ defined by:
        \begin{equation*}
            R_{A}=\big\{\,(x,y)\in{R}\;|\;(x,y)\in{A}^{2}\,\big\}
        \end{equation*}
    \end{fdefinition}
    There are many basic properties that relations have, and we prove them now.
    \begin{theorem}
        \label{thm:Cartesian_Product_Is_Relation}%
        If $A$ is a set, then $A\times{A}$ is a relation on $A$.
    \end{theorem}
    \begin{proof}
        For if $A$ is a set, then
        $A\times{A}\subseteq{A}\times{A}$. Therefore, etc.
    \end{proof}
    \begin{theorem}
        \label{thm:Empty_Set_Is_Relation}%
        If $A$ is a set, and then $\emptyset$ is a relation
        on $A$.
    \end{theorem}
    \begin{proof}
        For if $A$ is a set, then
        $\emptyset\subseteq{A}\times{A}$. Therefore, etc.
    \end{proof}
    \begin{theorem}
        Set inclusion $\subseteq$ is a relation. Proper set inclusion
        $\subsetneq$ is a relation. These define partial orderings.
    \end{theorem}
    \begin{fdefinition}{Domain of a Relation}{Domain_of_a_Relation}
        The \gls{domain (relation)} of a \gls{relation} $R$ on a \gls{set} $A$
        is the set:
        \begin{equation*}
            \textrm{dom}(R)=\big\{a\in{A}\;|\;\exists{b}\in{A}
                \textrm{ such that }aRb\big\}
        \end{equation*}
    \end{fdefinition}
    \begin{fdefinition}{Range of a Relation}{Range_of_a_Relation}
        The \gls{range (relation)} of a \gls{relation} $R$ on a \gls{set} $A$ is
        the set:
        \begin{equation*}
            \textrm{ran}(R)=\big\{b\in{A}\;|\;\exists{a}\in{A}
                \textrm{ such that }aRb\big\}
        \end{equation*}
    \end{fdefinition}
    \begin{fdefinition}{Field of a Relation}{Field_of_a_Relation}
        The \gls{field (relation)} of a \gls{relation} $R$ on a set $A$ is the
        set:
        \begin{equation*}
            \textrm{field}(R)=\textrm{dom}(R)\cup\textrm{ran}(R)
        \end{equation*}
        Where $\textrm{dom}(R)$ is the \gls{domain (relation)} of $R$ and
        $\textrm{ran}(R)$ is the \gls{range (relation)} of $R$.
    \end{fdefinition}
    These provide the two most basic examples of relations on a
    set. The empty set is the relation that says no two elements
    are related. Indeed, even single elements are unrelated to
    themselves. The second, the entire Cartesian product
    $A\times{A}$, says that everything is related. These are the
    two extreme cases, but provide useful examples and
    counterexamples in various contexts. More useful is that the
    union and intersection of relations is also a relation. We
    prove this now.
    \begin{theorem}
        \label{thm:Intersection_of_Relations_Is_Relation}%
        If $A$ is a set and if $R_{1}$ and $R_{2}$ are relations
        on $A$, then $R_{1}\cap{R}_{2}$ is a relation on $A$.
    \end{theorem}
    \begin{proof}
        For let $R=R_{1}\cap{R}_{2}$ and suppose $R$ is not a
        relation on $A$. Then there is an $x\in{R}$ such that
        $x\notin{A}\times{A}$. But if $x\in{R}$ then
        $x\in{R}_{1}$ and $x\in{R}_{2}$. But for all
        $x\in{R}_{1}$, $x\in{A}\times{A}$, since $R_{1}$ is a
        relation on $A$, a contradiction as
        $x\notin{A}\times{A}$. Therefore, $R$ is a relation on
        $A$.
    \end{proof}
    \begin{theorem}
        \label{thm:Set_Theory_Union_of_Relations_Is_Relation}
        If $A$ is a set and if $R_{1}$ and $R_{2}$ are relations
        on $A$, then $R_{1}\cup{R}_{2}$ is a relation on $A$.
    \end{theorem}
    \begin{proof}
        For let $R=R_{1}\cup{R}_{2}$ and suppose $R$ is not a
        relation on $A$. Then there is an $x\in{R}$ such that
        $x\notin{A}\times{A}$. But if $x\in{R}$ then
        $x\in{R}_{1}$ or $x\in{R}_{2}$. But for all $x\in{R}_{1}$
        and for all $x\in{R}_{2}$,
        $x\in{A}\times{A}$, since $R_{1}$ and $R_{2}$ are
        relations on $A$, a contradiction. Therefore, etc.
    \end{proof}
    \begin{theorem}
        If $A$ is a set and $R$ is a relation on $A$, then there
        is a relation $\mathcal{U}$ on $A$ such that
        $R\cap\mathcal{U}=R$.
    \end{theorem}
    \begin{proof}
        For let $\mathcal{U}={A}\times{A}$. Then by
        Thm.~\ref{thm:Cartesian_Product_Is_Relation}, $A\times{A}$ is
        a relation on $A$. But since $R$ is a relation,
        $R\subseteq{A}\times{A}$. But then
        $R\cap\mathcal{U}=R$. Therefore, etc.
    \end{proof}
    \begin{theorem}
        If $A$ is a set and $R$ is a relation on $A$, then there
        is a relation $\mathcal{U}$ on $A$ such that
        $R\cup\mathcal{U}=R$
    \end{theorem}
    \begin{proof}
        For let $\mathcal{U}=\emptyset$. Then by
        Thm.~\ref{thm:Empty_Set_Is_Relation},
        $\mathcal{U}$ is a relation. But if $R$ is a set, then
        $R\cup\emptyset=R$. Thus, $R\cup\mathcal{U}=R$.
        Therefore, etc.
    \end{proof}
    Since a general relation is simply a subset of $A\times{A}$,
    there's not much structure on them, and thus there's not a lot
    that can be said about them. We can add more constraints to
    certain relations to get the more familiar properties
    we're used to.
    \begin{fdefinition}{Reflexive Relations}{Reflexive_Relations}
        A reflexive relation on a set $A$ is a
        relation $R$ on $A$ such that for all $a\in{A}$
        it is true that $aRa$.
    \end{fdefinition}
    A reflexive relation on $A$ is simply any subset of
    $A\times{A}$ that contains the entire \textit{diagonal}. That,
    all of the pairs $(a,a)$. A reflexive relation can contain more
    than this, however. The only strict requirement is that
    $aRa$ for all $a\in{A}$.
    \begin{theorem}
        If $A$ is a set, and if $R_{1}$ and $R_{2}$ are reflexive
        relations on $A$, then $R_{1}\cap{R}_{2}$ is a reflexive
        relation on $A$.
    \end{theorem}
    \begin{proof}
        For let $R=R_{1}\cap{R}_{2}$. Then by
        Thm.~\ref{thm:Intersection_of_Relations_Is_Relation}, $R$ is a relation.
        Suppose $R$ is not reflexive.
        Then there is an $a\in{A}$ such that $(a,a)\notin{R}$. But
        if $a\in{A}$, then $(a,a)\in{R}_{1}$, since $R_{1}$ is
        reflexive. Similarly, $(a,a)\in{R}_{2}$ since $R_{2}$ is
        reflexive. But if $(a,a)\in{R}_{1}$ and $(a,a)\in{R}_{2}$,
        then $(a,a)\in{R}$ since $R=R_{1}\cap{R}_{2}$, a
        contradiction. Therefore, $R$ is reflexive.
    \end{proof}
    \begin{theorem}
        If $A$ is a set, if $R_{1}$ is a reflexive relation on
        $A$, and if $R_{2}$ is a relation on $A$, then
        $R_{1}\cup{R}_{2}$ is a reflexive relation on $A$.
    \end{theorem}
    \begin{proof}
        For let $R=R_{1}\cup{R}_{2}$. Since $R_{1}$ and $R_{2}$ are
        relations, by
        Thm.~\ref{thm:Set_Theory_Union_of_Relations_Is_Relation},
        $R$ is a relation. Suppose it is not reflexive.
        Then there is an $a\in{A}$ such that
        $(a,a)\notin{R}$. But if $a\in{A}$ then $(a,a)\in{R}_{1}$
        since $R_{1}$ is reflexive. But if $(a,a)\in{R}_{1}$ then
        $(a,a)\in{R}_{1}\cup{R}_{2}$, a contradiction.
        Therefore, etc.
    \end{proof}
    Given an arbitrary relation $R$ on a set $A$, it may not be
    true that $R$ is reflexive. It may often be useful to add in
    only the necessary points of $A$ that will make $R$
    reflexive. This is called the reflexive closure of $R$.
    \begin{fdefinition}{Reflexive Closure of a Relation}
                       {Reflexive_Closure_of_Relation}
        The reflexive closure of a relation $R$ on a set $A$
        is the set:
        \begin{equation}
            S=R\cup\{(a,a):a\in{A}\}
        \end{equation}
    \end{fdefinition}
    \begin{theorem}
        If $A$ is a set, $R$ is a relation on $A$, and if $S$ is the
        reflexive closure of $R$, then $S$ is a reflexive relation on $A$.
    \end{theorem}
    \begin{theorem}
        \label{thm:Set_Theory_Refl_Clos_Is_Smallest_Refl_With_R}
        If $A$ is a set, if $R$ is a relation on $A$, if
        $S$ is the reflexive closure of $R$, and if $T$ is a
        reflexive relation on $A$ such that $R\subseteq{T}$, then
        $S\subseteq{T}$.
    \end{theorem}
    \begin{proof}
        For if $x\in{S}$, then either $x\in{R}$ or there is an
        $a\in{A}$ such that $x=(a,a)$. But if $x\in{R}$, then
        $x\in{T}$ since $R\subseteq{T}$. If $x\notin{R}$ then
        there is an $a\in{A}$ such that $x=(a,a)$. But $T$ is
        reflexive, and therefore $(a,a)\in{T}$. But then
        $x\in{T}$. Therefore, $S\subseteq{T}$.
    \end{proof}
    Thm.~\ref{thm:Set_Theory_Refl_Clos_Is_Smallest_Refl_With_R}
    says that the reflexive closure of a relation $R$ is, in a sense,
    the \textit{smallest} relation that is reflexive and contains
    $R$ as a subset.
    \begin{theorem}
        If $A$ is a set, $R_{1}$ and $R_{2}$ are relations on $A$,
        and if $S_{1}$ and $S_{2}$ are the reflexive closures of
        $R_{1}$ and $R_{2}$, respectively, then the reflexive closure
        of $R_{1}\cap{R}_{2}$ is:
        \begin{equation}
            S=S_{1}\cap{S}_{2}
        \end{equation}
    \end{theorem}
    \begin{proof}
        By the definition of reflexive closure, we have:
        \begin{align}
            S_{1}&=R_{1}\cup\{(a,a):a\in{A}\}
            \tag{Def.~\ref{def:Reflexive_Closure_of_Relation}}\\
            S_{1}&=R_{2}\cup\{(a,a):a\in{A}\}
            \tag{Def.~\ref{def:Reflexive_Closure_of_Relation}}\\
            \nonumber
            S_{1}\cap{S}_{2}&=
            (R_{1}\cup\{(a,a):a\in{A}\})
            \cap(R_{2}\cup\{(a,a):a\in{A}\})\\
            &=(R_{1}\cap{R}_{2})
            \cup\{(a,a):a\in{A}\}
            \tag{Distributive Law}
        \end{align}
        But by the definition of the transitive closure of
        $R_{1}\cap{R}_{2}$:
        \begin{equation}
            S=(R_{1}\cap{R}_{2})\cup\{(a,a):a\in{A}\}
            \tag{Def.~\ref{def:Reflexive_Closure_of_Relation}}
        \end{equation}
        Therefore, etc.
    \end{proof}
    \begin{fdefinition}{Symmetric Relation}{Symmetric_Relation}
        A symmetric relation on a set $A$ is a
        relation $R$ on $A$ such that for all $a,b\in{A}$
        such that $aRb$, it is true that $bRa$.
    \end{fdefinition}
    \begin{theorem}
        If $A$ is a set, if $S_{1}$ and $S_{2}$ are symmetric relations
        on $A$, then $S_{1}\cap{S}_{2}$ is a symmetric relation on $A$.
    \end{theorem}
    \begin{proof}
        For since $S_{1}$ and $S_{2}$ are relations, $S_{1}\cap{S}_{2}$ is a
        relation (Thm.~\ref{thm:Intersection_of_Relations_Is_Relation}). Suppose
        it is not symmetric. Then there is an $(x,y)\in{S}_{1}\cap{S}_{2}$ such
        that $(y,x)\notin{S}_{1}\cap{S}_{2}$. But if
        $(x,y)\in{S}_{1}\cap{S}_{2}$, then $(x,y)\in{S}_{1}$ and
        $(x,y)\in{S}_{2}$ (Def.~\ref{def:Intersection_of_Two_Sets}). But $S_{1}$
        is symmetric and if $(x,y)\in{S}_{1}$, then $(y,x)\in{S}_{1}$
        (Def.~\ref{def:Symmetric_Relation}). Similarly $(y,x)\in{S}_{2}$, and
        therefore $(y,x)\in{S}_{1}\cap{S}_{2}$
        (Def.~\ref{def:Intersection_of_Two_Sets})), a contradiction. Therefore,
        $S_{1}\cap{S}_{2}$ is symmetric.
    \end{proof}
    \begin{fdefinition}{Transitive Relation}{Transitive_Relation}
        A transitive relation on a set $A$ is a relation $R$ on $A$
        such that for all $a,b,c\in{A}$ such that $aRb$ and $bRc$,
        is it true that $aRc$.
    \end{fdefinition}
    \begin{theorem}
        \label{thm:Entire_Cartesian_is_Transitive}%
        If $A$ is a set, then $A\times{A}$ is a transitive relation on $A$.
    \end{theorem}
    \begin{proof}
        For suppose not. Then there exists $a,b,c\in{A}$ such that
        $(a,b)\in{A}\times{A}$ and $(b,c)\in{A}\times{A}$, yet
        $(a,c)\in{A}\times{A}$. But if $a\in{A}$ and $c\in{A}$, then
        $(a,c)\in{A}\times{A}$ (Def.~\ref{def:Cartesian_Product_of_Two_Sets}), a
        contradiction. Therefore $A\times{A}$ is a transitive relation on $A$.
    \end{proof}
    Using the Cartesian product definition of a relation, we can visualize the
    requirement imposed on transitive relations in the diagram below
    (Fig.~\ref{fig:Transitive_Relation_Diagram}).
    \begin{figure}[H]
        \centering
        \captionsetup{type=figure}
        \begin{tikzpicture}
    \foreach\x in {0,1,2,3,4,5,6,7,8,9}{
        \foreach\y in {0,1,2,3,4,5,6,7,8,9}{
            \draw[fill=black] (\x,\y) circle (0.1);
        }
        \node at (\x, -1) {$\x$};
        \node at (-1, \x) {$\x$};
    }

    \draw[fill=red,draw=red,opacity=0.4] ( 1.7, -0.3) rectangle (2.3, 9.3);
    \draw[fill=red,draw=red,opacity=0.4] (-0.3,  5.7) rectangle (9.3, 6.3);

    \draw[draw=blue,fill=cyan,opacity=0.5] (2, 4) circle (0.2);
    \draw[draw=blue,fill=cyan,opacity=0.5] (4, 6) circle (0.2);
    \draw[draw=blue,fill=cyan,opacity=0.5] (2, 6) circle (0.2);
\end{tikzpicture}
        \caption{Diagram for a Transitive Relation}
        \label{fig:Transitive_Relation_Diagram}
    \end{figure}
    Given a point $(a,b)$ that is in the relation and another point $(b,c)$, for
    the relation to be transitive requires $(a,c)$ to be contained in it. That
    is, if we take the first coordinate from the first element and the second
    coordinate from the second element and then combine them to form a new
    ordered pair, this element must also be in the relation.
    \begin{theorem}
        If $A$ is a set, if $T_{1}$ and $T_{2}$ are transitive relations on $A$,
        and if $R=T_{1}\cap{T}_{2}$, then $R$ is a transitive relation.
    \end{theorem}
    \begin{proof}
        For since $T_{1}$ and $T_{2}$ are relations, $T_{1}\cap{T}_{2}$ is a
        relation (Thm.~\ref{thm:Intersection_of_Relations_Is_Relation}). Suppose
        it is not transitive. Then there are $(x,y),(y,z)\in{R}$ such that
        $(x,z)\notin{R}$ (Def.~\ref{def:Transitive_Relation}). But if
        $(x,y),(y,z)\in{R}$, then by the definition of intersection,
        $(x,y),(y,z)\in{T}_{1}$ and $(x,y),(y,z)\in{T}_{1}$
        (Def.~\ref{def:Intersection_of_Two_Sets}). But $T_{1}$ is transitive, and
        thus if $xT_{1}y$ and $yT_{1}z$, then $xT_{1}z$. But similarly $T_{2}$
        is transitive, and therefore $xT_{2}z$. But then $(x,z)\in{T}_{1}$ and
        $(x,z)\in{T}_{2}$, and thus $(x,z)\in{T}_{1}\cap{T}_{2}$, a
        contradiction. Therefore, $R$ is transitive.
    \end{proof}
    \begin{example}
        The requirement that both relations $T_{1}$ and $T_{2}$ are transitive
        cannot be weakened. For consider the relations $S$ and $T$ on
        $\mathbb{Z}_{3}$ defined by:
        \par\hfill\par
        \begin{subequations}
            \begin{minipage}[b]{0.49\textwidth}
                \centering
                \begin{equation}
                    S=\big\{\,(0,1),\,(1,2)\,\}
                \end{equation}
            \end{minipage}
            \hfill
            \begin{minipage}[b]{0.49\textwidth}
                \centering
                \begin{equation}
                    T=\big\{\,(0,1),\,(1,2),\,(0,1)\,\}
                \end{equation}
            \end{minipage}
        \end{subequations}
        \par\vspace{2.5ex}
        Then $T$ is transitive and $S$ is not. Moreover $S\subseteq{T}$, and
        hence $S\cap{T}=S$ (Thm.~\ref{thm:Intersection_with_Subset}), and
        therefore the intersection is not transitive. This example is
        demonstrated in
        Fig.~\subref{fig:Trans_Intersect_Non_Trans_May_Not_Be_Trans}. The
        opposite is possible, and to construct an example we need only find a
        transitve relation $T$ and a non-transitive relation $S$ such that
        $T\subseteq{S}$. Define:
        \par\hfill\par
        \begin{subequations}
            \begin{minipage}[b]{0.49\textwidth}
                \centering
                \begin{equation}
                    T=\big\{\,(0,0)\,\}
                \end{equation}
            \end{minipage}
            \hfill
            \begin{minipage}[b]{0.49\textwidth}
                \centering
                \begin{equation}
                    S=\big\{\,(0,0),\,(0,1),\,(1,2)\,\}
                \end{equation}
            \end{minipage}
        \end{subequations}
        \par\vspace{2.5ex}
        Then $T$ is transitive, $S$ is not, and $S\cap{T}=T$
        (See \subref{fig:Trans_Int_Trans_May_Not_Be_Trans}).
    \end{example}
    \begin{figure}[H]
        \centering
        \captionsetup{type=figure}
        \begin{subfigure}[b]{0.49\textwidth}
            \centering
            \begin{tikzpicture}
    \foreach\x in {0,1,2}{
        \foreach\y in {0,1,2}{
            \draw[fill=black] (\x,\y) circle (0.4mm);
        }
    }
    \node at (-1, 2) {$a$};
    \node at (-1, 1) {$b$};
    \node at (-1, 0) {$c$};
    \node at (2, 3) {$a$};
    \node at (1, 3) {$b$};
    \node at (0, 3) {$c$};
    \draw[draw=blue,fill=none] (0,1) circle (0.3);
    \draw[draw=blue,fill=none] (1,0) circle (0.3);
    \draw[draw=blue,fill=none] (2,2) circle (0.3);
    \draw[draw=red,fill=none] (-0.4243, 0.5757) rectangle (0.4243, 1.4243);
    \draw[draw=red,fill=none] ( 0.5757,-0.4243) rectangle (1.4243, 0.4243);
\end{tikzpicture}
            \subcaption{The intersection is not transitive}
            \label{fig:Trans_Intersect_Non_Trans_May_Not_Be_Trans}
        \end{subfigure}
        \hfill
        \begin{subfigure}[b]{0.49\textwidth}
            \centering
            \begin{tikzpicture}
    \foreach\x in {0,1,2}{
        \foreach\y in {0,1,2}{
            \draw[fill=black] (\x,\y) circle (0.4mm);
        }
        \node at (\x, -1) {$\x$};
        \node at (-1, \x) {$\x$};
    }
    \draw[draw=blue,fill=none] (0,2) circle (0.3);
    \draw[draw=red,fill=none] (-0.4, 1.6) rectangle (0.4, 2.4);
    \draw[draw=red,fill=none] (-0.4, 0.6) rectangle (0.4, 1.4);
    \draw[draw=red,fill=none] ( 0.6,-0.4) rectangle (1.4, 0.4);
\end{tikzpicture}
            \subcaption{The intersection is transitive}
            \label{fig:Trans_Int_Trans_May_Not_Be_Trans}
        \end{subfigure}
        \label{fig:Intersection_of_Transitive_and_Non_Transitive_Relations}
        \caption{The Intersection of Transitive and Non-Transitive Relations}
    \end{figure}
    We can strengthened our claim that the intersection of two transitive
    relations is again transitive and show that any arbitrary intersection will
    again be transitive.
    \begin{theorem}
        \label{thm:Intersection_of_Transitive_is_Transitive}%
        If $A$ is a set, if $\mathcal{P}(A\times{A})$ denotes the power set of
        $A\times{A}$, if $\mathcal{O}\subseteq\mathcal{P}(A\times{A})$ is such
        that for all $\mathcal{U}\in\mathcal{O}$ it is true that $\mathcal{U}$
        is a transitive relation on $A$, if $\mathcal{T}$ is defined by:
        \begin{equation}
            \mathcal{T}=\bigcap_{\mathcal{U}\in\mathcal{O}}\mathcal{U}
        \end{equation}
        Then $\mathcal{T}$ is a transitive relation on $A$.
    \end{theorem}
    \begin{proof}
        For suppose not. Then there exists $a,b,c\in{A}$ such that
        $(a,b)\in\mathcal{T}$ and $(b,c)\in\mathcal{T}$, yet
        $(a,c)\notin\mathcal{T}$. But if $(a,b)\in\mathcal{T}$, then for all
        $\mathcal{U}\in\mathcal{O}$ it is true that $(a,b)\in\mathcal{U}$
        (Def.~\ref{def:Intersection_Over_a_Collection}). Similarly, for all
        $\mathcal{U}\in\mathcal{O}$ it is true that $(b,c)\in\mathcal{U}$.
        But by hypothesis, for all $\mathcal{U}\in\mathcal{O}$ it is true that
        $\mathcal{U}$ is a transitive relation and thus if $(a,b)\in\mathcal{U}$
        and $(b,c)\in\mathcal{U}$, then it is true that $(a,c)\in\mathcal{U}$
        (Def.~\ref{def:Transitive_Relation}). But then for all
        $\mathcal{U}\in\mathcal{O}$ it is true that $(a,c)\in\mathcal{U}$, and
        therefore $(a,c)\in\mathcal{T}$
        (Def.~\ref{def:Intersection_Over_a_Collection}), a contradiction.
        Therefore, $\mathcal{T}$ is a transitive relation on $A$.
    \end{proof}
    This allows us to define the transitive closure of any relation $R$ on a set
    $A$. It is, in a sense, the \textit{smallest} transitive relation that
    contains $R$.
    \begin{theorem}
        If $A$ is a set and if $R$ is a relation on $A$, then there exists a
        transitive relation $\mathcal{T}$ on $A$ such that
        $R\subseteq\mathcal{T}$ and such that for transitive relations $T$ on
        $A$ such that $R\subseteq{T}$ it is true that $\mathcal{T}\subseteq{T}$.
    \end{theorem}
    \begin{proof}
        For let $P$ be the proposition \textit{True if} $S$
        \textit{is a transitive relation on} $A$ \textit{such that}
        $R\subseteq{S}$, \textit{false otherwise}. Then by the axiom schema of
        specification (Ax.~\ref{ax:Axiom_Schema_of_Specification}) there exists
        a set:
        \begin{equation}
            \mathcal{O}=\big\{\,S\in\mathcal{P}(A\times{A})\;|\;P(S)\,\big\}
        \end{equation}
        But then for all $S\in\mathcal{O}$, $P(S)$ is true and therefore
        $R\subseteq{S}$ and $S$ is transitive. Moreover, $\mathcal{O}$ is
        non-empty since by Thm.~\ref{thm:Entire_Cartesian_is_Transitive},
        $A\times{A}$ is a transitive relation. Define $\mathcal{T}$ by:
        \begin{equation}
            \mathcal{T}=\bigcap_{S\in\mathcal{O}}S
        \end{equation}
        Then by Thm.~\ref{thm:Intersection_of_Transitive_is_Transitive},
        $\mathcal{T}$ is a transitive relation. Moreover, suppose $S$ is a
        transitive relation such that $R\subseteq{S}$. But if $S$ is a relation
        on $A$, then $S\subseteq{A}\times{A}$
        (Def.~\ref{def:Relation_on_a_Set}) and therefore
        $S\in\mathcal{P}(A\times{A})$ (Def.~\ref{def:Power_Set}). But if $S$ is
        a transitive relation and if $R\subseteq{S}$, then $P(S)$ is true, and
        therefore $S\in\mathcal{P}$. Thus, $\mathcal{T}\subseteq{S}$.
    \end{proof}
    \begin{fdefinition}{Transitive Closure}{Transitive_Closure}
        The transitive closure of a relation $R$ on a set
        $A$ is the the set $R^{t}\subseteq{A}\times{A}$ defined by:
        \begin{equation}
            R^{t}
        \end{equation}
    \end{fdefinition}
    \begin{fdefinition}{Asymmetric Relation}{Assymetric_Relation}
        An asymmetric relation on a set $A$ is a relation $R$
        on $A$ such that for all $a,b\in{A}$ such that $aRb$
        it is true that $(b,a)\notin{R}$.
    \end{fdefinition}
    \begin{fdefinition}{Total Relation}{Total_Relation}
        A total relation on a set $A$ is a relation $R$ on $A$ such
        that for all $a,b\in{A}$ it is true that either
        $aRb$ or $bRa$, or both.
    \end{fdefinition}
    The notion of equality can be defined as a relation
    with the following properties:
    \begin{enumerate}
        \item Equality is Reflexive: $a=a$ for all $a\in{A}$.
        \item Equality is Symmetric: $a=b$ if and only if $b=a$.
        \item Equality is Transitive: If $a=b$ and $b=c$, then $a=c$.
        \item The relation is uniquely defined by the set
              $\{(a,a)\in A\times A:a\in A\}$.
    \end{enumerate}
    That is, equality can be seen as the \textit{diagonal} in the
    Cartesian product $A\times{A}$.
    \begin{fdefinition}{Antisymmetric Relation}
        An antisymmetric relation on a set $A$ is a relation $R$ on $A$
        such that for all $a,b\in{A}$ such that $aRb$ and $bRa$, it
        is true that $a=b$.
    \end{fdefinition}
    \begin{fdefinition}{Equivalence Relation}{Equivalence_Relation}
        An equivalence relation on a set $A$ is a relation $R$ on $A$ such that
        $R$ is reflexive, symmetric, and transitive.
    \end{fdefinition}
    Equivalence relations attempt to model equality. They are fundamental in
    mathematics as they allow us to define \textit{equivalence classes}, which
    are used to define quotients. There are many examples such as quotient
    topologies, quotient groups, quotient rings, and quotient modules, all of
    which will be discussed later.
    \begin{fdefinition}{Equivalence Class}{Equivalence_Class}
        The equivalence class of an element $x$ in a set $A$ by an
        equivalence relation $R$ is the set:
        \begin{equation*}
            [x]=\{\,y\in{A}\;|\;xRy\,\}
        \end{equation*}
    \end{fdefinition}
    It's important to note that the term class here is different from the notion
    of a collection of sets. And equivalence class of an element $x$ in a set
    $A$ under an equivalene relation $R$ will indeed be a set in $ZFC$.
    \begin{theorem}
        \label{thm:Equivalence_Classes_Disjoint_or_Equal}%
        If $A$ is a set, if $R$ is an equivalence relation on $A$, and if
        $x,y\in{A}$, then either $[x]=[y]$ or $[x]\cap[y]=\emptyset$.
    \end{theorem}
    \begin{proof}
        For suppose not and suppose $[x]\ne[y]$ and that
        $[x]\cap[y]\ne\emptyset$. That is, suppose:
        \begin{equation*}
            \neg([x]=[y])\land\neg([x]\cap[y]=\emptyset)
        \end{equation*}
        If $[x]\cap[y]$ is non-empty then there is a
        $z\in{A}$ such that $z\in[x]$ and $z\in[y]$
        (Def.~\ref{def:Non_Empty_Set}). But if $z\in[x]$, then $xRz$
        (Def.~\ref{def:Equivalence_Class}). But also
        $z\in[y]$ and therefore $yRz$. But $R$ is an equivalence relation and
        is therefore symmetric (Def.~\ref{def:Equivalence_Relation}) and thus
        if $yRz$ then $zRy$ (Def.~\ref{def:Symmetric_Relation}). But an
        equivalence relation is also transitive, and thus if $xRz$ and $zRy$,
        then $xRy$ (Def.~\ref{def:Transitive_Relation}). But if $[x]\ne[y]$ then
        either $[x]\nsubseteq[y]$ or $[y]\nsubseteq[x]$. Suppose
        $[x]\nsubseteq[y]$ and let $a\in[x]$ be such that $a\notin[y]$. But
        if $a\in[x]$ then $xRa$ (Def.~\ref{def:Equivalence_Class}). But since
        equivalence relations are symmetric, if $xRa$, then $aRx$. But it was
        proven that $xRy$ and since equivalence relations are transitive, if
        $aRx$ and $xRy$, then $aRy$. But again if $aRy$, then $yRa$ and
        therefore $a\in[y]$, a contradiction. Therefore $[x]\subseteq[y]$.
        Similarly, $[y]\subseteq[x]$ and therefore $[x]=[y]$, a contradiction.
        By the law of the excluded middle, the negation is true:
        \begin{equation*}
            \neg\big(\neg([x]=[y])\land\neg([x]\cap[y]=\emptyset)\big)
            =([x]=[y])\lor([x]\cap[y]=\emptyset)
        \end{equation*}
        Thus, either $[x]=[y]$ or $[x]\cap[y]=\emptyset$.
    \end{proof}
    \begin{fdefinition}{Quotient Set}{Quotient_Set}
        The quotient set of a set $A$ by an equivalence relation $R$ on $A$ is
        the set:
        \begin{equation*}
            A/R=\{\,[x]\in\mathcal{P}(A)\;|\;x\in{A}\,\}
        \end{equation*}
        Where $[x]$ is the equivalence class of $x$ under $R$.
    \end{fdefinition}
    \begin{example}
        The definition of the quotient set comes naturally when one considers
        functions between sets. Suppose $A$ and $B$ are sets, and suppose
        $f:A\rightarrow{B}$ is a function. In general, it may not be true that
        $f(a_{1})=f(a_{2})$ implies that $a_{1}=a_{2}$, and so we wish to find a
        subset of $A$ with this property. The quotient set does this. Let
        $R$ be the relation:
        \begin{equation}
            R=\{\,(a,b)\in{A}^{2}\;|\;f(a)=f(b)\,\}
        \end{equation}
        If we form the quotient set $A/R$ and consider the projective mapping
        $\pi:A\rightarrow{A}/R$ that sends $a\in{A}$ to its equivalence class.
        That is, $\pi(a)=[a]$. We then seek a function
        $\tilde{f}:A/R\rightarrow{B}$ such that $\tilde{f}\circ{\pi}=f$.
        That is, we wish to make the diagram below \textit{commute}.
        \begin{figure}[H]
            \centering
            \begin{tikzpicture}[%
    >=latex,
    every path/.style={->},
    line width=0.2mm,
    line cap=round
]
    \node (A) at (0.0,  0.0) {$A$};
    \node (AR) at (0.0, -2.0) {$A/R$};
    \node (B) at  (2.0,  0.0) {$B$};
    \path (A) edge node [above]        {$f$}         (B);
    \path (AR) edge node [below right] {$\tilde{f}$} (B);
    \path (A) edge node [left]         {$\pi$}       (AR);
\end{tikzpicture}
            \label{fig:Comm_Diagram_Quotient_Set}
            \caption{Commutative Diagram for the Quotient Set}
        \end{figure}
        So we need to map $[x]$ to $f(x)$. That is, $\tilde{f}([x])=f(x)$. For
        this problem to be well posed requires that the equivalence class that
        make up the elements of $A/R$ come from equivalence relations. That is,
        that the relation $R$ is transitive, symmetric, and reflexive.
    \end{example}
    \begin{theorem}
        If $A$ is a set and if $R$ is an equivalence relation on $A$, then
        $A/R$ is a partition of $A$.
    \end{theorem}
    \begin{proof}
        For by Thm.~\ref{thm:Equivalence_Classes_Disjoint_or_Equal}, if
        $\mathcal{U},\mathcal{V}\in{A}/R$, then either
        $\mathcal{U}=\mathcal{V}$ or $\mathcal{U}\cap\mathcal{V}=\emptyset$.
        But also, for all $x\in{A}$, there is a $\mathcal{U}\in{A}/R$ such that
        $x\in\mathcal{U}$ since $x\in[x]$ and $[x]\in{R}/A$. Therefore,
        $A/R$ is a partition of $A$.
    \end{proof}
        \chapter{The Real Numbers}
        Now that most of ZFC has been laid out, we take the time to discuss the
        next axiom on the list:
        \textit{the axiom of infinity}\index{Axiom!of Infinity}. This axiom,
        while controversial, posits the existence of an infinite set and allows
        us to construct the natural numbers and the real numbers in the
        framework of set theory.
    \chapter{Function Theory}
        \label{chapt:Function_Theory}%
        Functions serve as a basic tool for studying mathematics, so much so
        that it is often taken as fundamental and no definition is given. We've
        adopted the definition that a function\index{Function} from a set $A$ to
        a set $B$, $f:A\rightarrow{B}$, is a subset of the Cartesian product
        $A\times{B}$ with a few properties (see Def.~\ref{def:Function}). We now
        take the time to examine the implications of this definition.
        \section{Definitions}
    Given a function $f:X\rightarrow{Y}$, and any non-empty subset
    $S\subseteq{X}$, the image $f(S)$ is non-empty. This is not true for the
    pre-image of a function. For let $f:\mathbb{R}\rightarrow\mathbb{R}$ be
    defined by $f(x)=1$ for all $x\in\mathbb{R}$. Then, for any subset
    $S\subset\mathbb{R}$
    such that $1\notin{S}$, we have that $f^{\minus{1}}(S)=\emptyset$.
    There are many examples of functions, but certain ones are easier
    to study than others. We give some of these special functions names.
    \begin{ldefinition}{Injective Functions}{Injective_Function}
        An \gls{injective function} is a function
        $f:X\rightarrow{Y}$ such that, for all
        $x,y\in{X}$ such that $x\ne{y}$, it is true that
        $f(x)\ne{f}(y)$.
    \end{ldefinition}
    That is, an injective function is a function
    $f:X\rightarrow{Y}$ such that $f(x_{1})=f(x_{2})$
    if and only if $x_{1}=x_{2}$. Such functions are also
    called \textit{one-to-one}.
    \begin{lexample}{}{Natural_Log_Is_Injective}
        Consider the natural logarithm
        $\ln:\mathbb{R}^{+}\rightarrow\mathbb{R}$. This is an injective
        function. For let $x,y\in\mathbb{R}^{+}$ be such that
        $x\ne{y}$. Suppose $\ln(x)=\ln(y)$. But then:
        \begin{equation}
            \ln(x)-\ln(y)=\ln\Big(\frac{x}{y}\Big)=0
        \end{equation}
        Recall the definition of the natural logarithm:
        \begin{equation}
            \ln(t)=\int_{1}^{t}\frac{1}{x}\diff{x}
        \end{equation}
        But then $\ln(t)=0$ if and only if $t=1$. Thus $x=y$, a
        contradiction. Therefore $\ln$ is an injective function. Not
        every function is injective, for define
        $f:\mathbb{R}\rightarrow\mathbb{R}$ by $f(x)=x^{2}$. Then, for
        all $x\in\mathbb{R}^{+}$, $f(\minus{x})=f(x)$, and thus $f$
        cannot be an injective function.
    \end{lexample}
    One might think that most functions are not injective,
    and indeed for the \textit{finite} case, this is true.
    For let $A$ and $B$ be finite sets with $n$ and $m$
    elements, respectively. If $m<n$, there can't be
    any injective function. Consider the case when $n=m$.
    Then we are simply counting the number of ways to
    permute the elements of $A$. This is $n!$. On the
    other hand, the total number of functions is
    $n^{n}$. Thus, the ratio of the number of injective
    functions to the number of functions is
    $n!/n^{n}$, and this decays to zero rapidly as
    $n$ get's large. Finally, if $m>n$, then the total
    number of injective functions is
    $n!\binom{m}{n}$, where $\binom{m}{n}$ is the
    binomial coefficient. The total number of functions
    is $n^{m}$. The ratio is thus:
    \begin{equation}
        \frac{n!\binom{m}{n}}{n^{m}}=\frac{n!\frac{m!}{n!(m-n)!}}{n^{m}}
                                    =\frac{m!}{(m-n)!n^{m}}
    \end{equation}
    And again, this decays rapidly to zero and $n$ and $m$
    get large. Later, when we define infinite sets
    and the notion of Cardinality, we'll show that this
    trend continues. That is, in a sense, \textit{most}
    functions from a set $A$ to a sufficiently large set
    $B$ are not injective. Next, we define
    \textit{surjective} functions.
    \begin{ldefinition}{Surjective Functions}{Surjective_Function}
        A \gls{surjective function} is a function
        $f:X\rightarrow{Y}$ such that $f(X)=Y$.
        That is, for all $y\in{Y}$, there is an
        $x\in{X}$ such that $f(x)=y$.
    \end{ldefinition}
    That is, every point $y\in{Y}$ gets mapped to by
    at least one point in $X$. It may also be true that
    many points in $X$ map to the same point in $Y$.
    The notions of surjective functions and injective
    functions are distinct, and neither implies the
    other. Surjective functions are also called
    \textit{onto}.
    \begin{ldefinition}{Bijective Functions}{Bijective_Function}
        A \gls{bijective function} is a function
        that is both injective and surjective.
    \end{ldefinition}
    \begin{theorem}
        \label{thm:Image_of_Empty_Set_Is_Empty}%
        If $A$ and $B$ are sets, and if $f:A\rightarrow{B}$
        is a function, then:
        \begin{equation}
            f(\emptyset)=\emptyset
        \end{equation}
    \end{theorem}
    \begin{theorem}
        If $A$ and $B$ are sets, and if $f:A\rightarrow{B}$
        is a function, then:
        \begin{equation}
            f^{-1}(\emptyset)=\emptyset
        \end{equation}
    \end{theorem}
    \begin{theorem}
        If $X$ and $Y$ are sets, if $A\subseteq{X}$, and if
        $f:X\rightarrow{Y}$ is a function such that
        $f(A)=\emptyset$, then $A=\emptyset$.
    \end{theorem}
    \begin{proof}
        For suppose not. If $A\ne\emptyset$, then there is an $x\in{A}$.
        But then $f(x)\in{f}(A)$, a contradiction as $f(A)=\emptyset$.
    \end{proof}
    \begin{theorem}
        If $X$ and $Y$ are sets, if $B$ is a subset of $Y$,
        and if $f:X\rightarrow{Y}$ is a function, then:
        \begin{equation}
            f\big(f^{-1}(B)\big)\subseteq{B}
        \end{equation}
    \end{theorem}
    \begin{proof}
        For if $y\in{f(f^{-1}(B))}$, then there is an
        $x\in{f^{-1}(B)}$ such that $y=f(x)$. But if
        $x\in{f^{-1}(B)}$, then $f(x)\in{B}$. Thus,
        $y\in{B}$. Therefore, etc.
    \end{proof}
    \begin{theorem}
        If $X$ and $Y$ are non-empty sets and if there exists
        $y_{1},y_{2}\in{Y}$ such that $y_{1}\ne{y}_{2}$, then
        there is a function $f:X\rightarrow{Y}$ and a
        $B\subseteq{Y}$ such that:
        \begin{equation}
            f\big(f^{-1}(B)\big)\ne{B}
        \end{equation}
    \end{theorem}
    \begin{proof}
        \begin{subequations}
            For if $X$ and $Y$ are non-empty, let $f:X\rightarrow{Y}$
            be defined by:
            \begin{equation}
                f=\{(x,y_{1}):x\in{X}\}
            \end{equation}
            Then $f$ is a function, since $f\subseteq{X}\times{Y}$
            as $y_{1}\in{Y}$. Moreover, for all $x\in{X}$ there is a
            unique $y\in{Y}$ such that $(x,y)\in{f}$. Thus, $f$ is a
            function from $X$ to $Y$. However since for all
            $x\in{X}$, $f(x)=y_{1}$, we have that:
            \begin{equation}
                f^{-1}(\{y_{2}\})=\emptyset
            \end{equation}
            For suppose $x\in{f}^{-1}(\{y_{2}\})$.
            Then $f(x)=y_{2}$, but for all $x\in{X}$, $f(x)=y_{1}$,
            and $y_{1}\ne{y}_{2}$. Thus
            $f^{-1}(\{y_{2}\})=\emptyset$. But by
            Thm.~\ref{thm:Image_of_Empty_Set_Is_Empty},
            $f(\emptyset)=\emptyset$. Therefore:
            \begin{equation}
                f\big(f^{-1}(\{y_{2}\})\big)=\emptyset
            \end{equation}
            But $\{y_{2}\}\ne\emptyset$ and
            $\{y_{2}\}\subseteq{Y}$. Therefore, etc.
        \end{subequations}
    \end{proof}
    \begin{theorem}
        If $X$ and $Y$ are sets, if $A$ is a subset of $X$,
        and if $f:X\rightarrow{Y}$ is a function, then:
        \begin{equation}
            A\subseteq{f^{-1}}\big(f(A)\big)
        \end{equation}
    \end{theorem}
    \begin{proof}
        For if $x\in{A}$, then there is a $y\in{f}(A)$ such that
        $f(x)=y$. But then $x\in{f^{-1}(f(A))}$. Therefore, etc.
    \end{proof}
    \begin{theorem}
        If $X$ and $Y$ are sets, if $A_{1}$ and $A_{2}$ are
        subsets of $X$ such that $A_{1}\subseteq{A}_{2}$,
        and if $f:X\rightarrow{Y}$ is a function, then:
        \begin{equation}
            f(A_{1})\subseteq{f}(A_{2})
        \end{equation}
    \end{theorem}
    \begin{proof}
        For if $y\in{f}(A_{1})$, then there is an $x\in{A}_{1}$
        such that $f(x)=y$. But $A_{1}\subseteq{A}_{2}$, and
        therefore $x\in{A}_{2}$. But if $x\in{A}_{2}$, then
        $f(x)\in{f}(A_{2})$. Thus, $y\in{f}(A_{2})$. Therefore, etc.
    \end{proof}
    \begin{theorem}
        If $X$ and $Y$ are sets, if $B_{1}$ and $B_{2}$ are subsets of
        $Y$ such that $B_{1}\subseteq{B}_{2}$, and if $f:X\rightarrow{Y}$
        is a function, then:
        \begin{equation}
            f^{-1}(B_{1})\subseteq{f^{-1}}(B_{2})
        \end{equation}
    \end{theorem}
    \begin{proof}
        For if $x\in{f}^{-1}(B_{1})$, then there is a
        $y\in{B}_{1}$ such that $f(x)=y$. But
        $B_{1}\subseteq{B}_{2}$, and therefore $y\in{B}_{2}$.
        Thus, $x\in{f}^{-1}(B_{2})$. Therefore, etc.
    \end{proof}
    \begin{theorem}
    If $f:A\rightarrow B$, $A_1,A_2\subset A$, then $f(A_1 \cup A_2) = f(A_1)\cup f(A_2)$.
    \end{theorem}
    \begin{proof}
    $[y\in f(A_1\cup A_2)]\Rightarrow [\exists x\in A_1 \cup A_2:y=f(x)]\Rightarrow [y \in f(A_1)\cup f(A_2)]$. $[y\in f(A_1)\cup f(A_2)]\Rightarrow \big[[\exists x\in A_1] \lor[\exists x\in A_2]: y=f(x)\big]\Rightarrow [x\in A_1\cup A_2]\Rightarrow [f(x)\in f(A_1\cup A_2)]$
    \end{proof}
    \begin{theorem}
        If $f:A\rightarrow B$, $A_{1},A_{}2\subset A$, then
        $f(A_{1}\cap{A}_{2})\subset{f}(A_{1})\cap{f}(A_{2})$.
    \end{theorem}
    \begin{proof}
        $[y\in f(A_1 \cap A_2)]\Rightarrow [\exists x\in A_1 \cap A_2:y=f(x)]\Rightarrow [x\in A_1 \land x \in A_2] \Rightarrow[y \in f(A_1)\cap f(A_2)]$.
    \end{proof}
    \begin{theorem}
        If $A$ and $B$ are sets, $f:A\rightarrow{B}$ is a function,
        and $B_{1},B_{2}\subseteq{B}$, then:
        \begin{equation}
            f^{-1}(B_{1}\cup{B}_{2})=f^{-1}(B_{1})\cup{f}^{-1}(B_{2})
        \end{equation}
    \end{theorem}
    \begin{proof}
        For if $x\in{B}_{1}\cup{B}_{2}$, then
        $f(x)\in{B}_{1}\cup{B}_{2}$. but then either
        $f(x)\in{B}_{1}$ or $f(x)\in{B}_{2}$, and therefore
        $x\in{f}^{\minus{1}}(B_1)\cup{f}^{\minus{1}}(B_2)$. But if
        $x\in{f}^{\minus{1}}(B_{1})\cup{f}^{\minus{1}}(B_2)$, then
        $f(x)\in{B}_{1}$ or $f(x)\in{B}_{2}$. Therefore
        $f(x)\in{B}_{1}\cup{B}_{2}$. Thus, $x\in{f}^{-1}(B_1\cup{B}_2)$.
    \end{proof}
    \begin{theorem}
        If $A$ and $B$ are sets, $f:A\rightarrow{B}$ is a function,
        and $B_{1},B_{2}\subseteq{B}$, then:
        \begin{equation}
            f^{-1}(B_{1}\cap{B}_{2})=f^{-1}(B_{1})\cap{f}^{-1}(B_{2})
        \end{equation}
    \end{theorem}
    \begin{proof}
        $[x\in f^{-1}(B_1\cap B_2)]\Rightarrow [f(x) \in B_1 \cap B_2]\Rightarrow [f(x)\in B_1\land f(x) \in B_2 ]\Rightarrow [x\in f^{-1}(B_1)\cap f^{-1}(B_2)]$. $[x\in f^{-1}(B_1)\cap f^{-1}(B_2)]\Rightarrow [x\in f^{-1}(B_1)\land x\in f^{-1}(B_2)]\Rightarrow [f(x) \in B_1\land f(x) \in B_2]\Rightarrow [f(x)\in B_1\cap B_2]\Rightarrow [x\in f^{-1}(B_1\cap B_2)]$.
    \end{proof}
    \begin{theorem}
    If $f:A\rightarrow B$, $B_1 \subset B$, then $f^{-1}(B\setminus B_1) = f^{-1}(B)\setminus f^{-1}(B_1)$.
    \end{theorem}
    \begin{proof}
    $[x\in f^{-1}(B\setminus B_1)]\Leftrightarrow [f(x)\notin B_1]\Leftrightarrow [x\in f^{-1}(B)\setminus f^{-1}(B_1)]$
    \end{proof}
    If $f:A\rightarrow B$, the image of $A$ under $f$
    is often called the range (A is often called the domain).
    \begin{ldefinition}{Permutations}{Permutations}
        A permutation on a set $A$ is a bijective function
        $f:A\rightarrow{A}$.
    \end{ldefinition}
    \begin{theorem}
    If $f:A\rightarrow B$ is bijective, then $f^{-1}$ is bijective.
    \end{theorem}
    \begin{proof}
    $[f^{-1}(y_1) = f^{-1}(y_2)]\Rightarrow [\exists x\in A:[f(x) = y_1]\land [f(x)=y_2]]\Rightarrow [y_1=y_2]$. By definition, $f^{-1}$ is surjective.
    \end{proof}
    \begin{definition}
    If $f:A\rightarrow B$ and $g:B\rightarrow C$, then $g\circ f:A\rightarrow C$ is defined by the image $g(f(x)), x\in A$. 
    \end{definition}
    \begin{theorem}
    If $f:A\rightarrow B$, $g:B\rightarrow C$, and $\mathcal{V}\subset C$, then $(g\circ g)^{-1}(\mathcal{V}) = f^{-1}(g^{-1}(\mathcal{V}))$.
    \end{theorem}
    \begin{proof}
    $[x\in (g\circ f)^{-1}(\mathcal{V})]\Leftrightarrow [g(f(x))\in \mathcal{V}] \Leftrightarrow [f(x)\in g^{-1}(\mathcal{V})]\Leftrightarrow [x\in f^{-1}(g^{-1}(\mathcal{V}))]$.
    \end{proof}
    \begin{theorem}
    If $f:A\rightarrow B$ is bijective, $g:B\rightarrow C$ is bijective, then $g\circ f$ is bijective.
    \end{theorem}
    \begin{proof}
    $\big[[f(A) = B]\land [g(B) = C]\big]\Rightarrow [g(f(A)) = g(B) = C]$. $[g(f(x_1))=g(f(x_2))]\Leftrightarrow [f(x_1)=f(x_2)]\Leftrightarrow [x_1=x_2]$.
    \end{proof}
    \begin{theorem}
    If $f:A\rightarrow B$ is bijective, $A_1\subset A$, and $f(A_1) = B$, then $A_1=A$.
    \end{theorem}
    \begin{proof}
    $\Big[\big[[A_1^c \ne \emptyset]\Rightarrow [f(A_1^c) \ne \emptyset]\big]\land[f(A_1)\cap f(A_1^c) = \emptyset]\Big]\Rightarrow [\exists y\in B:y\notin f(A_1)]$, a contradiction.
    \end{proof}
        \section{Binary Operations}
    Binary operations are the standard tools that one uses when they develope
    arithmetic. As such, the most familiar examples of binary operations are
    those of addition, multiplication, and subtraction with real numbers.
    On the other hand, division is \textit{not} a binary operation on the real
    numbers since division by zero is undefined. To make this explicit we need
    to give a rigorous definition to binary operations. We can do this with the
    language of functions\index{Function} and by using the Cartesian product
    \index{Cartesian Product} of a set $A$ with itself.
    \begin{fdefinition}{Binary Operation}{Binary_Operation}
        A \gls{binary operation} on a set $A$ is a function
        $*:A\times{A}\rightarrow{A}$.
    \end{fdefinition}
    \begin{example}
        Let $\mathbb{R}$ be the set of real numbers and $+$ denote the addition
        of two real numbers. Then $+$ is a binary operation on $\mathbb{R}$.
        Similarly, if $\cdot$ denotes the multiplication of two real numbers,
        than it two is a binary operation on $\mathbb{R}$. For division, $\div$,
        we are lacking the requirement that \textit{for all}
        $(a,b)\in\mathbb{R}^{2}$ there is a unique $c\in\mathbb{R}$ such that
        $a\div{b}=c$, since if $b=0$ our expression is undefined. That is, this
        is not a function from $\mathbb{R}^{2}$ to $\mathbb{R}$. If we consider
        all of the non-zero elements, then division is a binary operation. That
        is, division is a binary operation on $\mathbb{R}\setminus\{0\}$.
    \end{example}
    \begin{lexample}{Binary Operation on the Set of Functions}
                    {Binary_Operation_on_the_Set_of_Functions}
        If $A$ is a set, and if $\mathcal{F}(A,A)$ denotes the set of all
        functions $f:A\rightarrow{A}$, and if $\circ$ denotes function
        composition, then $\circ$ is a binary operation on $\mathcal{F}(A,A)$.
        That is, for any two functions $f,g\in\mathcal{F}(A,A)$, the composition
        $g\circ{f}:A\rightarrow{A}$ is again an element of $\mathcal{F}(A,A)$
    \end{lexample}
    Just like functions, there are three important conditions that a binary
    operation must satisfy. Given any ordered pair $(a,b)\in{A}^{2}$, it must
    be true that $*(a,b)$ is defined. This comes from the definition of a
    function on a set (Def.~\ref{def:Function}). Next, the image of $(a,b)$ must
    be unique. That is, if $*(a,b)=c$ and $*(a,b)=d$, then $c=d$. Note that this
    is not the same as requiring that $*(a,b)=*(b,a)$, and in general this is
    not true. Such binary operations are called
    \textit{commutative}\index{Commutative Operation}. Lastly, for any
    $(a,b)\in{A}^{2}$, $*(a,b)$ must be an element of $A$. That is,
    $*(a,b)\in{A}$. All of these requirements come from the definition of a
    function, so in a sense it is redundant to repeat these. In practice one
    defines a binary operation by a formula $\varphi$, and it then becomes
    necessary to show that this formula satisfies these properties before we can
    rightly call it a binary operation.
    \begin{example}
        Let $A=\mathbb{Z}_{2}$ and consider all of the binary operations on
        $\mathbb{Z}_{2}$. We can count these by constructing tables:
        \begin{table}[H]
            \centering
            \begin{tabular}{c|c}
                $(x,y)$&$*(x,y)$\\
                \hline
                $(0,0)$&0\\
                $(0,1)$&0\\
                $(1,0)$&1\\
                $(1,1)$&0
            \end{tabular}
            \label{tab:Binary_Operation_on_Z_2}
            \caption{Simple Binary Operation on $\mathbb{Z}_{2}$}
        \end{table}
        This is one such binary operation, there are 15 others. To see this,
        recall that the number of functions from a set $A$ to a set $B$, where
        both $A$ and $B$ are finite sets with $m$ and $n$ elements,
        respectively, is $n^{m}$. Since $\mathbb{Z}_{2}$ has 2 elements, and
        since a binary operation is a function
        $*\mathbb{Z}_{2}\times\mathbb{Z}_{2}\rightarrow\mathbb{Z}_{2}$, the
        total number of binary operations is $2^{(2^{2})}=2^{4}=16$. In general,
        if $A$ has $n$ elements, and if $B$ is the set of all binary operations
        on $A$, then:
        \begin{equation}
            \textrm{Card}(B)=n^{(n^2)}
        \end{equation}
    \end{example}
    \begin{example}
        Let's consider some formula that take in numbers and return numbers, and
        see if they can define operations on various sets. Suppose we have:
        \begin{equation}
            a*b=\{\,r\in\mathbb{R}\;|\;r^{2}=|ab|\,\}
        \end{equation}
        Where $|ab|$ denotes the absolute value of $a$ times $b$. If we take the
        positive square root we can write this as $a*b=\sqrt{|ab|}$. If we
        consider this formula on the rational numbers $\mathbb{Q}$, does it
        define a function? One might recall that $\sqrt{2}$ is not a rational
        number. That is, it is \textit{irrational}. Thus $1*2$ is not a rational
        number, and so $*$ is not a binary operation on $\mathbb{Q}$. It is a
        binary operation on $\mathbb{R}$, however. Suppose we change the formula
        to state:
        \begin{equation}
            a*b=\{\,r\in\mathbb{R}\:|\;r^{2}-ab=0\,\}
        \end{equation}
        and where we consider this formula to take inputs from $\mathbb{R}$.
        This is not a binary operation since it is poorly defined. That is,
        should $1*1=1$, or should $1*1=\minus{1}$? The formula is ambgious and
        thus $*$ is not a binary operation.
    \end{example}
    \begin{example}
        If we consider subtraction on the integers $\mathbb{Z}$, this is a
        binary operation. The operation is well defined and returns an integer
        for all integer inputs. If instead we consider subtraction on
        $\mathbb{N}$, this is \textit{not} a binary operation since it may take
        in non-negative integers and return a negative integer. For example,
        $1-2=\minus{1}$, and $\minus{1}\notin\mathbb{N}$. A simple fix for this
        is considering again the absolute value function. If we define
        $n*m=|n-m|$, then $*$ is indeed a binary operation on $\mathbb{N}$.
    \end{example}
    \begin{fnotation}{Binary Operation}{Binary_Operation}
        If $A$ is a set and if $*:A\times{A}\rightarrow{A}$ is a binary
        operation on $A$, for any ordered pair $(a,b)\in{A}^{2}$, the image
        of $*(a,b)$ is denoted $a*b$.
    \end{fnotation}
    It is occasionally useful to think of binary operations purely as functions,
    and so we will use function notation at these times. For the most part we
    will stick with notation defined in Not.~\ref{not:Binary_Operation}. There
    are several types of binary operations worth studying, and several key
    properties that these operations can have. One of the most fundamental is
    the existence of a \textit{unital} element, also known as an identity.
    \begin{fdefinition}{Left Unital Element}{Left_Unital_Element}
        A left unital element in a \gls{set} $A$ under a \gls{binary operation}
        $*$ on $A$ is an element $e_{L}\in{A}$ such that, for all $a\in{A}$ it
        is true that $e_{L}*a=a$.
    \end{fdefinition}
    \begin{example}
        From the definition of a left unital element
        (Def.~\ref{def:Left_Unital_Element}) it would seem natural to define a
        right unital element. The importance is to note that left and right
        unital elements need not be equal. Indeed, if $A$ is a set and $*$ is
        a binary operation, given a left identity $e_{L}$ and a right identity
        $e_{R}$ it will be true that $e_{R}=e_{L}$ and thus all left and right
        unital elements will be the same
        (see Thm.~\ref{thm:left_and_right_identity_implies_identity}). Thus to
        find counterexamples to the claim that the existence of a left unital
        element implies the existence of a right unital element we need to think
        of strange operations. Let $A=\mathbb{R}$ and let $*$ be defined by
        $a*b=b$ for all $a,b\in\mathbb{R}$. Then every element of $\mathbb{R}$
        is a left unital element. Moreover, none of the element of $\mathbb{R}$
        are right unital elements.
    \end{example}
    \begin{fdefinition}{Right Unital Element}{Right_Unital_Element}
        A right unital element of a \gls{set} $A$ under a \gls{binary operation}
        $*$ is an element $e_{R}$ such that for all $a\in{A}$ it is true that
        $a*e_{R}=a$.
    \end{fdefinition}
    \begin{example}
        Consider $\mathbb{R}$ with the operation $*$ defined by $a*b=a+b+1$.
        This operation has a right unital element, $\minus{1}$. For:
        \begin{equation}
            a*(\minus{1})=a+(\minus{1})+1=a+0=a
        \end{equation}
        And this is true for all $a\in\mathbb{R}$, so $\minus{1}$ is a right
        unital element. It turns out this is also a left unital element, and
        hence a unital element, and this can be proven if addition is known to
        be a \textit{commutative} operation.
    \end{example}
    \begin{theorem}
        \label{thm:left_and_right_identity_implies_identity}%
        If $A$ is a set, if $*$ is a binary operation on $A$, if $e_{L}$ is a
        left unital element of $A$, and if $e_{R}$ is a right unital element of
        $a$, then $e_{L}=e_{R}$.
    \end{theorem}
    \begin{proof}
        For:
        \begin{equation}
            e_{L}=e_{L}*e_{R}=e_{R}
        \end{equation}
        And thus $e_{L}=e_{R}$.
    \end{proof}
    \begin{example}
        Consider a non-empty set $A$ and the set of all functions from $A$ to
        itself, $\mathcal{F}(A,A)$. Let $\circ$ denote the binary operation of
        function composition. Then $\mathcal{F}(A,A)$ has a right identity under
        $\circ$, and a left identity. For
        the identity function $\textrm{id}_{A}$ acts as a right identity:
        \begin{equation}
            (f\circ\textrm{id}_{A})(x)
            =f\big(\textrm{id}_{A}(x)\big)
            =f(x)
        \end{equation}
        And thus $\textrm{id}_{A}$ is a right identity. By
        Thm.~\ref{thm:left_and_right_identity_implies_identity}, any left
        identity must also be a right identity, and so the likely candidate to
        check is $\textrm{id}_{A}$. And indeed we have:
        \begin{equation}
            (\textrm{id}_{A}\circ{f})(x)
            =\textrm{id}_{A}\big(f(x)\big)
            =f(x)
        \end{equation}
        And thus $\textrm{id}_{A}$ is a left identity as well.
    \end{example}
    \begin{fdefinition}{Unital Element}{Unital_Element}
        A \gls{unital element} of a \gls{set} $A$ under a \gls{binary operation}
        $*$ is an element $e\in{A}$ that is both a right unital element and a
        left unital element.
    \end{fdefinition}
    \begin{example}
        Let $\mathbb{R}$ be the set of real numbers and let $+$ be the usual
        notion of addition. Then 0 is a unital element of $\mathbb{R}$ with
        respect to this operation. That is, for any real number $x$ we have
        $x+0=0+x=x$. For multiplication the unital element is 1. This is because
        $1\cdot{x}=x\cdot{1}=x$. Subtraction has a right unital element, and
        again it is 0 since $x-0=x$, but no left identity. To see this, suppose
        $e-x=x$ for all $x$. Applying some algebra we have that $e=2x$, meaning
        there is no constant $e\in\mathbb{R}$ such that for all $x$, $e-x=x$.
        Since subtraction has no left unital element, it has no unital element
        either.
    \end{example}
    \begin{theorem}
        \label{thm:Unital_Elements_are_Unique}%
        If $A$ is a set, if $*$ is a binary operation on $A$, and if $e$ and
        $e'$ are unital elements of $A$, then $e=e'$
    \end{theorem}
    \begin{proof}
        For:
        \begin{equation}
            e=e*e'=e'
        \end{equation}
        And thus by transitivity, $e=e'$.
    \end{proof}
    The next thing to discuss is that of inverses. There are five types, but in
    practice only one of these is discussed.
    \begin{fdefinition}{Weak Right Inverse}{Weak_Right_Inverse}
        A weak right inverse of an element $a$ in a \gls{set} $A$ under a
        \gls{binary operation} $*$ on $A$ is an element $b\in{A}$ such that
        $a*b$ is a right unital element.
    \end{fdefinition}
    This definition will not recieve much use until we discuss
    groups\index{Group}. A group is a set with a binary operation $*$ that has
    a unital element, inverse elements, and is associatied (to be defined soon).
    As it turns out these conditions are stronger than necessary and it suffices
    to check that there are weak right inverses and a right unital element. The
    next thing to define is right inverses.
    \begin{fdefinition}{Right Inverse}{Right_Inverse}
        A right inverse of an element $a$ in a \gls{set} $A$ under a
        \gls{binary operation} is an element $b\in{A}$ such that $a*b$ is a
        \gls{unital element}.
    \end{fdefinition}
    Here, we've simply strengthened the requirement that $a*b$ not only be a
    right unital element, but also a left unital element as well. A right
    inverse is therefore necessarily a weak right inverse.
    \begin{fdefinition}{Weakly Left Invertible}{Weakly_Left_Invertible}
        A weakly left invertible element of a \gls{set} $A$ under a
        \gls{binary operation} $*$ is an element $a\in{A}$ such that there
        exists a $b\in{A}$ such that $b*a$ is a left unital element.
    \end{fdefinition}
    \begin{fdefinition}{Left Invertible Element}{Left_Inverse}
        A left invertible element of a \gls{set} $A$ under a
        \gls{binary operation} is an element $a\in{A}$ such that there exists a
        $b\in{A}$ such that $b*a$ is a \gls{unital element}.
    \end{fdefinition}
    \begin{fdefinition}{Invertible Element}{Invertible_Element}
        An invertible element of a a \gls{set} $A$ under a
        \gls{binary operation} is an element $a\in{A}$ that is both
        left invertible and right invertible.
    \end{fdefinition}
    \begin{theorem}
        \label{thm:Unital_Elements_Are_Invertible}%
        If $A$ is a set, if $*$ is a binary operation on $A$, and if $e$ is a
        unital element of $A$, then $e$ is an invertible element.
    \end{theorem}
    \begin{proof}
        For since $e$ is a unital element, it is true that $e=e*e$
        (Def.~\ref{def:Unital_Element}). Therefore $e$ is invertible element
        (Def.~\ref{def:Invertible_Element}).
    \end{proof}
    \begin{fdefinition}{Commutative Operation}{Commutative_Operation}
        A \gls{commutative operation} on a \gls{set} $A$ is a
        \gls{binary operation} $*$ such that for all $(a,b)\in{A}^{2}$ it is
        true that $a*b=b*a$.
    \end{fdefinition}
    \begin{fdefinition}{Associative Operation}{Associative_Operation}
        A \gls{associative operation} on a \gls{set} $A$ is a
        \gls{binary operation} $*$ such that, for all $a,b,c\in{A}$ it is true
        that $a*(b*c)=(a*b)*c$.
    \end{fdefinition}
    \begin{example}
        Consider a finite set $A$ and consider the set of all functions from
        $\mathbb{Z}_{n}$ to $A$. That is, $\mathcal{F}_{n}(\mathbb{Z}_{n},A)$.
        Define $A[x]$ by:
        \begin{equation}
            \mathcal{F}=\bigcup_{n\in\mathbb{N}}\mathcal{F}_{n}
        \end{equation}
        That is, the set of all finite sequences in $A$. We can form an
        associative operation on this set by defining the concatenation
        operation. Given $f,g\in\mathcal{F}$, suppose
        $f\in\mathcal{F}(\mathbb{Z}_{m},A)$ and
        $g\in\mathcal{F}(\mathbb{Z}_{n},A)$. We define
        $f*g\in\mathcal{F}(\mathbb{Z}_{m+n},A)$ as follows:
        \begin{equation}
            (f*g)(k)=
            \begin{cases}
                f(k),&k\in\mathbb{Z}_{m}\\
                g(k-m),&k\in\mathbb{Z}_{m+n}\textrm{ and }k\geq{m}
            \end{cases}
        \end{equation}
        That is, given two sequences $f_{0},f_{1},\dots,f_{m-1}$ and
        $g_{0},g_{1},\dots,g_{n-1}$, we concatenate them to form the sequence
        $f_{0},\dots,f_{m-1},g_{0},\dots,g_{n-1}$. This operation is associative
        since if $f,g,h\in\mathcal{F}$, then:
        \begin{subequations}
            \begin{align}
                f*(g*h)&=(f_{0},f_{1},\dots,f_{m-1})
                    *(g_{0},g_{1},\dots,g_{m-1},h_{0},h_{1},\dots,h_{r-1})\\
                &=f_{0},f_{1},\dots,f_{m-1},
                    g_{0},g_{1},\dots,g_{m-1},h_{0},h_{1},\dots,h_{r-1}\\
                &=(f_{0},f_{1},\dots,f_{m-1},
                    g_{0},g_{1},\dots,g_{m-1})*(h_{0},h_{1},\dots,h_{r-1})\\
                &=(f*g)*h
            \end{align}
        \end{subequations}
        If $A$ has more than one point than $*$ is not commutative. For let
        $f,g:\mathbb{Z}_{1}\rightarrow{A}$ be defined by $f(0)=a$ and $g(0)=b$,
        respectively. Then $f*g=a,b$ but $g*f=b,a$, and thus $f*g\ne{g}*f$.
        There is, however, an identity. Consider a $\mathbb{Z}_{0}$, which is
        the empty set. Any function from $\mathbb{Z}_{0}$ to $A$ is therefore
        the \textit{empty sequence}. If we concatenate $f$ with the empty
        sequence we get back $f$, and this then acts as our unital element.
    \end{example}
    \begin{theorem}
        \label{thm:Assoc_Op_Inverses_are_Unique}%
        If $A$ is a set, if $*$ is a binary operation on $A$, if $a$ is an
        invertible element, and if $b,c\in{A}$ are such that $b*a$ and $a*c$
        are unital elements, then $b=c$.
    \end{theorem}
    \begin{proof}
        For:
        \begin{align}
            b&=b*(a*c)
            \tag{$a*b$ is a unital element}\\
            &=(b*a)*c
            \tag{Associativity}\\
            &=c
            \tag{$b*a$ is a unital element}
        \end{align}
        And therefore $b=c$.
    \end{proof}


        \section{Boolean Algebras}
    \label{sec:Boolean_Algebra}%
    We now attempt to make set theory more algebraic. We wish to model
    as an object the triple $(\mathcal{P}(X),\cup,\cap)$, where
    $\mathcal{P}(X)$ is the \gls{power set} of some set, and $\cup$ and
    $\cap$ and union and intersection, respectively. These can be seen as
    binary operations on $\mathcal{P}(X)$. We take a few of the properties
    of this structure and state them as the definition for our new object:
    \textit{Boolean Algebras}.
    \begin{fdefinition}{Complement in a Boolean Algebra}
                       {Complement}
        A complement of a \gls{set} $A$ with respect to two
        \glspl{binary operation} $*$ and $\circ$ is an element
        $a^{\minus{1}}\in{A}$ such that:
        \begin{equation*}
            a*a^{\minus{1}}=a^{\minus{1}}*a=e_{\circ}
            \quad\quad
            a\circ{a}^{\minus{1}}=a^{\minus{1}}\circ{a}=e_{*}
        \end{equation*}
        Where $e_{\circ}$ and $e_{*}$ are the \glspl{unital element} of
        $\circ$ and $*$, respectively.
    \end{fdefinition}
    \begin{fdefinition}{Boolean Algebras}{Boolean_Algebra}
        A Boolean algebra is a set $A$ with two
        \glslink{commutative operation}{commutative} \glspl{binary operation}
        $\circ$ and $*$ on $A$ with \glspl{unital element} $e_{*}$ and
        $e_{\circ}$, respectively, such that:
        \begin{itemize}
            \item[1.)]  $\circ$ \glslink{distributive operation}{distributes}
                        over $*$ and $*$ distributes over $\circ$.
            \item[2.)]  For all $a\in{A}$ there is a complement of $a$.
        \end{itemize}
    \end{fdefinition}
    The second property is known as the complement property and it is very
    different from the notion of inverses. An inverse of an element $a$ with
    respective to an operation $\cdot$ is such that $a\cdot{b}$ is a unital
    element with respect to the operation $\cdot$. A complement produces a
    unital element with respect the the \textit{other} operation. That is,
    $a*a^{\minus{1}}$ is a unital element of $\circ$, and not $*$. Similarly,
    $a\circ{a}^{\minus{1}}$ is a unital element of $*$ and not $\circ$.
    \begin{theorem}
        If $(A,\circ,*)$ is a Boolean algebra and if $b\in{X}$ is a
        unital element of $\circ$, then $b=e_{\circ}$.
    \end{theorem}
    \begin{proof}
        For unital elements are unique
        (Thm.~\ref{thm:Unital_Elements_are_Unique}), and therefore
        $b=e_{\circ}$.
    \end{proof}
    \begin{theorem}
        If $(X,\circ,*)$ is a Boolean algebra and if $b\in{X}$ is a
        unital element of $*$, then $b=e_{*}$.
    \end{theorem}
    \begin{proof}
        For unital elements are unique
        (Thm.~\ref{thm:Unital_Elements_are_Unique}), and therefore $b=e_{*}$.
    \end{proof}
    \begin{theorem}
        \label{thm:Bool_Alg_Boundary_of_Circ}%
        If $(X,\circ,*)$ is a Boolean algebra, if $e_{*}$ is the unital
        element of $*$, and if $a\in{X}$, then $a\circ{e}_{*}=e_{*}$.
    \end{theorem}
    \begin{proof}
        For if $a\in{X}$ then there is an $a^{\minus{1}}\in{X}$ such that
        $a\circ{a}^{\minus{1}}=e_{*}$ (Def.~\ref{def:Boolean_Algebra}).
        But then:
        \par\vspace{-2.5ex}
        \begin{minipage}[t]{0.51\textwidth}
            \centering
            \begin{align}
                e_{*}&=a\circ{a}^{\minus{1}}
                \tag{Complement}\\
                &=a\circ(a^{\minus{1}}*e_{*})
                \tag{Identity}\\
                &=(a\circ{a}^{\minus{1}})*(a\circ{e}_{*})
                \tag{Distributivity}
            \end{align}
        \end{minipage}
        \hfill
        \begin{minipage}[t]{0.47\textwidth}
            \centering
            \begin{align}
                &=e_{*}*(a\circ{e}_{*})
                \tag{Complement}\\
                &=a\circ{e}_{*}
                \tag{Identity}
            \end{align}
        \end{minipage}
        \par\vspace{2.5ex}
        And therefore $e_{*}=a\circ{e}_{*}$.
    \end{proof}
    This theorem is equivalent to the notion that a Boolean algebra is a
    bounded lattice\index{Bounded Lattice} and the $e_{*}$ is a boundary. The
    theorem holds for $\circ$ as well.
    \begin{theorem}
        \label{thm:Bool_Alg_Boundary_of_Star}%
        If $(X,\circ,*)$ is a Boolean algebra, if $e_{*}$ is the unital
        element of $*$, and if $a\in{X}$, then $a*{e}_{\circ}=e_{\circ}$.
    \end{theorem}
    \begin{proof}
        For if $a\in{X}$ then there is an $a^{\minus{1}}\in{X}$ such that
        $a*{a}^{\minus{1}}=e_{\circ}$ (Def.~\ref{def:Boolean_Algebra}).
        But then:
        \par\vspace{-2.5ex}
        \begin{minipage}[t]{0.51\textwidth}
            \centering
            \begin{align}
                e_{\circ}&=a*{a}^{\minus{1}}
                \tag{Complement}\\
                &=a*(a^{\minus{1}}\circ{e}_{\circ})
                \tag{Identity}\\
                &=(a*a^{\minus{1}})\circ(a*e_{*})
                \tag{Distributivity}
            \end{align}
        \end{minipage}
        \hfill
        \begin{minipage}[t]{0.47\textwidth}
            \centering
            \begin{align}
                &=e_{\circ}\circ(a*e_{\circ})
                \tag{Complement}\\
                &=a*e_{\circ}
                \tag{Identity}
            \end{align}
        \end{minipage}
        \par\vspace{2.5ex}
        And therefore $e_{\circ}=a*e_{\circ}$.
    \end{proof}
    \begin{theorem}
        If $(A,\circ,*)$ is a Boolean algebra, if $e_{\circ}$ and $e_{*}$
        are the unital elements of $\circ$ and $*$, respectively, then
        $e_{\circ}$ is the complement of $e_{*}$ and $e_{*}$ is the
        complement of $e_{\circ}$.
    \end{theorem}
    \begin{proof}
        From identity:
        \par\vspace{-2.5ex}
        \begin{subequations}
            \begin{minipage}[b]{0.49\textwidth}
                \centering
                \begin{equation}
                    e_{\circ}\circ{e}_{*}=e_{*}\circ{e}_{\circ}=e_{*}
                    \tag{Identity}
                \end{equation}
            \end{minipage}
            \hfill
            \begin{minipage}[b]{0.49\textwidth}
                \centering
                \begin{equation}
                    e_{\circ}*{e}_{*}=e_{*}*e_{\circ}=e_{\circ}
                    \tag{Identity}
                \end{equation}
            \end{minipage}
        \end{subequations}
        \par\vspace{2.5ex}
        Thus, $e_{*}$ is a complement of $e_{\circ}$ and $e_{\circ}$ is a
        complement of $e_{*}$ (Def.~\ref{def:Complement}).
    \end{proof}
    Every element of a Boolean algebra is idempotent with respect to both
    operations.
    \begin{theorem}
        \label{thm:Bool_Alg_Idempotent_of_Star}%
        If $(A,\circ,*)$ is a Boolean algebra and if $a\in{A}$, then $a*a=a$.
    \end{theorem}
    \begin{proof}
        For:
        \par\vspace{-2.5ex}
        \begin{subequations}
            \begin{minipage}[t]{0.49\textwidth}
                \centering
                \begin{align}
                    a&=a*e_{*}
                    \tag{Identity}\\
                    &=a*(a\circ{a}^{\minus{1}})
                    \tag{Complement}\\
                    &=(a*a)\circ(a*a^{\minus{1}})
                    \tag{Distributivity}
                \end{align}
            \end{minipage}
            \hfill
            \begin{minipage}[t]{0.49\textwidth}
                \centering
                \begin{align}
                    &=(a*a)\circ{e}_{\circ}
                    \tag{Complement}\\
                    &=a*a\tag{Identity}
                \end{align}
            \end{minipage}
        \end{subequations}
        \par\vspace{2.5ex}
        And therefore $a=a*a$.
    \end{proof}
    \begin{theorem}
        \label{thm:Bool_Alg_Idempotent_of_Circ}%
        If $(A,\circ,*)$ is a Boolean algebra and if $a\in{A}$, then
        $a\circ{a}=a$.
    \end{theorem}
    \begin{proof}
        For:
        \par\vspace{-2.5ex}
        \begin{subequations}
            \begin{minipage}[t]{0.49\textwidth}
                \centering
                \begin{align}
                    a&=a\circ{e}_{\circ}
                    \tag{Identity}\\
                    &=a\circ(a*a^{\minus{1}})
                    \tag{Complement}\\
                    &=(a\circ{a})*(a\circ{a}^{\minus{1}})
                    \tag{Distributivity}
                \end{align}
            \end{minipage}
            \hfill
            \begin{minipage}[t]{0.49\textwidth}
                \centering
                \begin{align}
                    &=(a\circ{a})*{e}_{*}
                    \tag{Complement}\\
                    &=a\circ{a}\tag{Identity}
                \end{align}
            \end{minipage}
        \end{subequations}
        \par\vspace{2.5ex}
        And therefore $a=a\circ{a}$.
    \end{proof}
    \begin{theorem}
        \label{thm:Bool_Alg_aob_equal_a_acb_equal_a_implies_a_equal_b}%
        If $(A,\circ,*)$ is a Boolean algebra, if $a,b\in{A}$, if $a\circ{b}=a$,
        and if $a*b=a$, then $b=a$.
    \end{theorem}
    \begin{proof}
        For:
        \par
        \begin{minipage}[b]{0.49\textwidth}
            \centering
            \begin{align}
                b&=b*e_{*}\tag{Identity}\\
                 &=b*(a\circ a^{-1})\tag{Complement}\\
                 &=(b*a)\circ(b* a^{-1})\tag{Distributivity}\\
                 &=a\circ (b* a^{-1})\tag{Hypothesis}
            \end{align}
        \end{minipage}
        \hfill
        \begin{minipage}[b]{0.49\textwidth}
            \centering
            \begin{align}
                &=(a\circ b)*(a\circ a^{-1})\tag{Distributivity}\\
                &=(a\circ{b})*e_{*}\tag{Complement}\\
                &=a\circ{b}\tag{Identity}\\
                &=a\tag{Hypothesis}
            \end{align}
        \end{minipage}
        \par\vspace{2.5ex}
        And therefore $a=b$.
    \end{proof}
    \begin{theorem}
        If $(A,\circ,*)$ is a Boolean algebra, if $a\in{A}$ is such that
        $a=a^{\minus{1}}$, then $a=e_{\circ}=e_{*}$.
    \end{theorem}
    \begin{proof}
        For let $a\in{A}$ and let $a=a^{\minus{1}}$. Then by
        Thm.~\ref{thm:Bool_Alg_Idempotent_of_Star}:
        \begin{equation}
            a=a*a=a*a^{-1}=e_{\circ}
        \end{equation}
        Similarly, $a=e_{*}$.
    \end{proof}
    \begin{theorem}
        If $(A,\circ,*)$ is a Boolean algebra, if $a\in{A}$, and if $b,c\in{A}$
        are complements of $a$, then $b=c$.
    \end{theorem}
    \begin{proof}
        For:
        \par
        \begin{minipage}[t]{0.49\textwidth}
            \centering
            \begin{align}
                b&=b*e
                \tag{Identity}\\
                &=b*(a\circ{c})
                \tag{Complement}\\
                &=(b*a)\circ(b*c)
                \tag{Distributivity}\\
                &=e_{\circ}\circ(b*c)
                \tag{Complement}\\
                &=(c*a)\circ(b*c)
                \tag{Complement}
            \end{align}
        \end{minipage}
        \hfill
        \begin{minipage}[t]{0.49\textwidth}
            \centering
            \begin{align}
                &=(c*a)\circ(c*b)
                \tag{Commutativity}\\
                &=c\circ(a*b)
                \tag{Distributivity}\\
                &=c\circ{e}_{\circ}
                \tag{Complement}\\
                &=c
                \tag{Identity}
            \end{align}
        \end{minipage}
        \par\vspace{2.5ex}
        Therefore, $b=c$.
    \end{proof}
    \begin{theorem}
        If $(A,\circ,*)$ is a Boolean algebra and if $a\in{A}$,
        then $(a^{\minus{1}})^{\minus{1}}=a$.
    \end{theorem}
    \begin{proof}
        For:
        \begin{align}
            a&=a*e_{*}
            \tag{Identity}\\
            &=a*\big(a^{\minus{1}}\circ(a^{\minus{1}})^{\minus{1}})
            \tag{Complement}\\
            &=(a\circ{a}^{\minus{1}})*
                \big(a\circ(a^{\minus{1}})^{\minus{1}}\big)
            \tag{Distributivity}\\
            &=e_{*}*\big(a\circ(a^{\minus{1}})^{\minus{1}}\big)
            \tag{Complement}\\
            &=a\circ(a^{\minus{1}})^{\minus{1}}
            \tag{Identity}
        \end{align}
        And similarly $a*(a^{\minus{1}})^{\minus{1}}=a$. But if $a*b=a$ and
        $a\circ{b}=a$, then $a=b$
        (Thm.~\ref{thm:Bool_Alg_aob_equal_a_acb_equal_a_implies_a_equal_b}).
        Therefore, $a=(a^{\minus{1}})^{\minus{1}}$.
    \end{proof}
    \begin{ltheorem}{Absorption Laws}{Absorption_Law}
        If $(A,\circ,*)$ is a Boolean algebra, if $a\in{A}$ and if $b\in{A}$,
        then $a*(a\circ{b})=a$ and $a\circ(a*{b})=a$.
    \end{ltheorem}
    \begin{proof}
        For:
        \par
        \begin{minipage}[t]{0.59\textwidth}
            \centering
            \begin{align}
                a*(a\circ{b})&=(a*e_{*})\circ(a*b)
                \tag{Identity}\\
                &=a*(e_{*}\circ{b})
                \tag{Distributivity}
            \end{align}
        \end{minipage}
        \hfill
        \begin{minipage}[t]{0.39\textwidth}
            \centering
            \begin{align}
                &=a*e_{*}
                \tag{Thm.~\ref{thm:Bool_Alg_Boundary_of_Circ}}\\
                &=a
                \tag{Identity}
            \end{align}
        \end{minipage}
        \par\vspace{2.5ex}
        And therefore $a*(a\circ{b})=a$. Similarly:
        \par
        \begin{minipage}[t]{0.59\textwidth}
            \centering
            \begin{align}
                a\circ(a*b)&=(a\circ{e}_{\circ})*(a\circ{b})
                \tag{Identity}\\
                &=a\circ(e_{\circ}*b)
                \tag{Distributivity}
            \end{align}
        \end{minipage}
        \hfill
        \begin{minipage}[t]{0.39\textwidth}
            \begin{align}
                &=a\circ{e}_{\circ}
                \tag{Thm.~\ref{thm:Bool_Alg_Boundary_of_Star}}\\
                &=a
                \tag{Identity}
            \end{align}
        \end{minipage}
        \par\vspace{2.5ex}
        And therefore $a\circ(a*b)=a$.
    \end{proof}
    We can weaken the hypothesis of
    Thm.~\ref{thm:Bool_Alg_aob_equal_a_acb_equal_a_implies_a_equal_b} to obtain
    a more general result.
    \begin{theorem}
        \label{thm:Bool_Alg_aob_equal_acb_implies_a_equal_b}
        If $(A,\circ,*)$ is a Boolean algebra, if $a,b\in{A}$, and if
        $a*b=a\circ{b}$, then $a=b$.
    \end{theorem}
    \begin{proof}
        For:
        \begin{align}
            a&=a*e_{*}
            \tag{Identity}\\
            &=a*(b\circ{b}^{\minus{1}})
            \tag{Complement}\\
            &=(a*b)\circ(a*b^{\minus{1}})
            \tag{Distributivity}\\
            &=(a\circ{b})\circ(a*b^{\minus{1}})
            \tag{Hypothesis}\\
            &=((a\circ{b})\circ{a})*
                \big((a\circ{b})\circ{b}^{\minus{1}}\big)
            \tag{Distributivity}\\
            &=\big((a\circ{b})\circ{a}\big)*
                \big(a\circ(b\circ{b}^{\minus{1}})\big)
            \tag{Associativity}\\
            &=\big((a\circ{b})\circ{a}\big)*(a\circ{e}_{*})
            \tag{Complement}\\
            &=\big((a\circ{b})\circ{a}\big)*e_{*}
            \tag{Thm.~\ref{thm:Bool_Alg_Boundary_of_Circ}}\\
            &=(a\circ{b})\circ{a}
            \tag{Identity}\\
            &=(a\circ{a})\circ{b}
            \tag{Associativity and Commutativity}\\
            &=a\circ{b}
            \tag{Thm~\ref{thm:Bool_Alg_Idempotent_of_Circ}}
        \end{align}
        Thus $a=a\circ{b}$. But $a\circ{b}=a*{b}$, and so $a=a*{b}$. By
        Thm.~\ref{thm:Bool_Alg_aob_equal_a_acb_equal_a_implies_a_equal_b},
        $a=b$.
    \end{proof}
            \begin{definition} For $a\in S$, an inverse, or normal inverse, of the First Operation is an element $b\in S$ such that $a\circ b=e_{\circ}$. An inverse of the Second Operation is similarly defined. The normal inverses are denoted $a^{*}$ and $a^{\circ}$.
            \end{definition}
            \begin{theorem} If $a\in S$ has a normal inverse for either operation, than it is unique.
            \end{theorem}
            \begin{proof} For suppose not. Let $a\in S$ have a normal inverse for the First Operation. That is, there is an $a^{\circ}\in S$ such that $a\circ a^{\circ}=e_{\circ}$ and let $a'^{\circ}$ be a second normal inverse not equal to the first. But then $a^{\circ}=a^{\circ}\circ e_{\circ}=a^{\circ}\circ (a\circ a'^{\circ})$ and from associativity we have $a^{\circ}=(a^{\circ}\circ a)\circ a'^{\circ}=a'^{\circ}$. Thus, the normal inverse is unique. Similarly if there is an inverse for the Second Operation
            \end{proof}
            \begin{theorem} If $a\in S$ has a normal inverse, say $a'$, for one operation, then $a^{-1}=a'^{-1}$.
            \end{theorem}
            \begin{proof} For let $a\in S$ have a normal inverse $a'$ for the First Operation. That is, $a\circ a' = e_{\circ}$. But $a' \circ a'^{-1}=e_{*}$, and from theorem 1.3 $a\circ e_{*}=e_{*}$. So $a\circ (a' \circ a'^{-1})=e_{*}$. And from theorem 1.4, $a\circ a=a$, so we have $(a\circ a)\circ (a'\circ a'^{-1}=a\circ (a\circ a')\circ a'^{-1}=a\circ a'^{-1}=e_{*}$. But $a\circ a^{-1}=e_{\circ}$. And pseudo-inverses are unique. Thus, $a^{-1}=a'^{-1}$. 
            \end{proof}
            \begin{theorem} The identities have normal inverses for their respective operations.
            \end{theorem}
            \begin{proof} As normal inverses are unique, it suffices to find inverses for both identities. But $e_{\circ}\circ e_{\circ}=e_{\circ}$, so $e_{\circ}$ is its own inverse for the First Operation. Similarly, $e_{*}*e_{*}=e_{*}$.
            \end{proof}
            \begin{theorem} \textbf{(The Not-A-Field Theorem)} Only the identities have normal inverses.
            \end{theorem}
            \begin{proof} For suppose not. Suppose $a\in S,\ a\ne e_{\circ},\ a\ne e_{*}$ and a has an inverse for the First Operation. That is $\exists a^{\circ}\in S|\ a\circ a^{\circ}=e_{\circ}$. But by theorem 1.4, $a\circ a^{\circ}=(a\circ a)\circ a^{\circ}$. By associativity, we have $e_{\circ}=a\circ a^{\circ} = a\circ (a\circ a^{\circ})=a\circ e_{\circ}=a$. Thus, $a=e_{\circ}$. But by hypothesis, $a\ne e_{\circ}$. Thus, there is no inverse for $a$. Similarly, a has no inverse for the Second Operation.
            \end{proof}
            \begin{theorem}
            There exist pseudo-fields with only one element.
            \end{theorem}
            \begin{proof}
            For let $e_{\circ} = e_{*}$, and let no other elements be in the set. 
            \end{proof}
            \begin{theorem}
            A pseud-field has one element if and only if $e_{\circ} = e_{*}$.
            \end{theorem}
            \begin{proof}
            For suppose there is another element $a \ne e_{\circ}$. But then $a \circ e_{\circ} = a$, but also $a \circ e_{\circ} = a \circ e_{*} = e_{*}$. So $a = e_{*}$. If there is only one element, then clearly $e_{\circ} = e_{*}$ as otherwise there would be two elements.
            \end{proof}
            \begin{definition} A generating set on a pseudo-field is a subset $g_S \subset S$ such that every element of $S$ can be written as a finite combination of elements in $g_S$ using $\circ$ or $*$.
            \end{definition}
            \begin{theorem}
            The number of elements in a finite pseudo-field is a power of 2.
            \end{theorem}
            \begin{proof}
            Consider the set of all generators $g_S$ on $S$. Clearly for all such generators, $1\leq |g_S|\leq |S|$. Let $G$ be the smallest generator, such that $|G| \leq |g_S|$ for any other given generator. 
            \end{proof}
        \section{Sequences and Matrices}
    Matrices\index{Matrix} are the fundamental object studied in
    linear algebra\index{Linear Algebra}, and are used in the study of general
    algebra as well. To discuss the more interesting properties requires some
    notion of arithmetic that we do not yet posses. In particular, matrices are
    most interesting when there is an underlying \textit{ring}\index{Ring}
    structure. For now we simply introduce the set theoretic definition of a
    matrix, relate this to the familiar \textit{grid of numbers} definition, and
    provide examples.
    \begin{fdefinition}{Matrix}
        An $n\times{m}$, $n,m\in\mathbb{N}$, matrix on a set $X$ is a function
        $A:\mathbb{Z}_{n}\times\mathbb{Z}_{m}\rightarrow{X}$.
    \end{fdefinition}
    \chapter{Arithmetic}
        \section{Elementary Number Theory}
    \begin{theorem}
        \label{thm:Equiv_Classes_Form_Partition}%
        If $A$ is a set, if $R$ is an equivalence relation on $A$, and
        if $A/R$ is the quotient set, then $A/R$ is a partition of $A$.
    \end{theorem}
    \begin{proof}
        For all $x\in{A}$ it is true that $[x]\in{A}/R$ and since $R$ is
        an equivalence relation we have $x\in[x]$, hence $A/R$ covers
        $A$. Moreover, if $\mathcal{U},\mathcal{V}\in{A}/R$ and if
        $\mathcal{U}\cap\mathcal{V}$ is non-empty, then there is an
        $x\in{R}$ such that $[x]\in\mathcal{U}$ and $[x]\in\mathcal{V}$.
        But if $[y]\in\mathcal{U}$ and $[z]\in\mathcal{V}$, then
        $yRx$ and $xRz$. But then $yRz$ since $R$ is an equivalence
        relation and therefore $[y]\in\mathcal{V}$. Similarly,
        $[z]\in\mathcal{U}$. Hence, either $\mathcal{U}=\mathcal{V}$ or
        they are disjoint. Therefore, $A/R$ is a partition of $A$.
    \end{proof}
    \begin{theorem}
        If $A$ is a set, and if $\mathcal{O}\subseteq\powset{X}$ is a
        partition of $A$, then there is an equivalence relation $R$ on
        $A$ such that $\mathcal{O}=A/R$.
    \end{theorem}
    \begin{proof}
        For let $R\subseteq{A}\times{A}$ be defined by:
        \begin{equation}
            R=\{\,(x,y)\in{A}\times{A}\;|\;
                \exists_{\mathcal{U}\in\mathcal{O}}
                (x,y\in\mathcal{U})\,\}
        \end{equation}
        then $R$ is an equivalence relation. Since $\mathcal{O}$ is a
        partition, for all $x\in{A}$ there is a
        $\mathcal{U}\in\mathcal{O}$ such that $x\in\mathcal{U}$. But
        then $x\in\mathcal{U}$ and $x\in\mathcal{U}$, and therefore
        $(x,x)\in{R}$. That is, $xRx$. Moreover, if $xRy$ then there is
        a $\mathcal{U}\in\mathcal{O}$ such that $x\in\mathcal{U}$ and
        $y\in\mathcal{U}$. But then $y\in\mathcal{U}$ and
        $x\in\mathcal{U}$ and hence $yRx$. Lastly, if $xRy$ and $yRz$,
        then there is a set $\mathcal{U}\in\mathcal{O}$ and a set
        $\mathcal{V}\in\mathcal{O}$ such that $x,y\in\mathcal{U}$ and
        $y,z\in\mathcal{V}$. But $\mathcal{O}$ is a partition, and hence
        either $\mathcal{U}\cap\mathcal{V}=\emptyset$ or
        $\mathcal{U}=\mathcal{V}$. But
        $\mathcal{U}\cap\mathcal{V}\ne\emptyset$ since $y\in\mathcal{U}$
        and $y\in\mathcal{V}$. Therefore, $\mathcal{U}=\mathcal{V}$ and
        thus, since $z\in\mathcal{V}$, it is true that
        $z\in\mathcal{U}$. That is, $xRz$. Hence, $R$ is an equivalence
        relation. Moreover, by definition, $A/R=\mathcal{O}$.
    \end{proof}
    \begin{theorem}
        \label{thm:Fibers_of_Func_Form_Equiv_Relation}%
        If $A$ and $B$ are sets, if $f:A\rightarrow{B}$ is a function,
        and if $R\subseteq{A}\times{A}$ is the relation defined by:
        \begin{equation}
            R=\{\,(x,y)\in{A}\times{A}\;|\;f(x)=f(y)\,\}
        \end{equation}
        then $R$ is an equivalence relation on $A$.
    \end{theorem}
    \begin{proof}
        For all $x\in{A}$ it is true that $f(x)=f(x)$, and hence $xRx$.
        Moreover, if $x,y\in{A}$ and $xRy$, then $f(x)=f(y)$. But
        equality is reflexive, and hence $f(y)=f(x)$. But then $yRx$.
        Lastly, by the transitivity of equality, if $xRy$ and $yRz$,
        then $f(x)=f(y)$ and $f(y)=f(z)$, hence $f(x)=f(z)$. But then
        $xRz$. Thus, $R$ is and equivalence relation.
    \end{proof}
    \begin{theorem}
        \label{thm:Weak_Euc_Division_Alg}%
        If $n,m\in\mathbb{N}^{+}$, then there exists $q,r\in\mathbb{N}$
        such that $n=q\cdot{m}+r$ with $r<m$.
    \end{theorem}
    \begin{proof}
        For let $G\subseteq\mathbb{N}$ be defined by:
        \begin{equation}
            G=\{\,k\in\mathbb{N}\;|\;n-k\cdot{m}\geq{0}\,\}
        \end{equation}
        if $n<m$, then $G=\{0\}$ and hence choosing $r=n$ and $q=0$
        works. If $n=m$, $q=1$ and $r=0$ does the trick. Otherwise, $G$
        is non-empty and bounded since it is bounded by $n$, and hence
        there is a greatest element. Let $q\in{G}$ be the greatest
        element, and let $r=n-q\cdot{m}$. Since $q\in{G}$, $r>0$.
        Moreover, $r<m$. For if not, then
        $r-m=n-q\cdot{m}-m=n-(q+1)\cdot{m}\geq{0}$, which contradicts
        the maximality of $q$. But then $n=kq+r$ and $r<m$.
    \end{proof}
    There's a strengthening of this theorem, which dates back to
    antiquity.
    \begin{ftheorem}{Euclid's Division Algorithm}
                    {Euclid_Division_Algorithm}
        If $m,n\in\mathbb{Z}$, and if $m\ne{0}$, then there exists
        unique $q\in\mathbb{Z}$ and $r\in\mathbb{N}$ such that $r<|m|$
        and:
        \begin{equation*}
            n=mq+r
        \end{equation*}
    \end{ftheorem}
    \begin{bproof}
        For if $n=0$, then let $r=0$ and $q=0$. If $n>0$, and if $m>0$,
        then by Thm.~\ref{thm:Weak_Euc_Division_Alg} there exists
        $q,r$ such that $n=q\cdot{m}+r$ with $r<m$. If $n>0$ and $m<0$,
        then $\minus{m}>0$. But then by
        Thm.~\ref{thm:Weak_Euc_Division_Alg} there exists $q,r$ such
        that $n=q\cdot(\minus{m})+r$ with $r<\minus{m}$. But then
        $n=(\minus{q})\cdot{m}+r$, and $r<|m|$. If $n<0$ and $m>0$,
        then $\minus{n}>0$ and hence by
        Thm.~\ref{thm:Weak_Euc_Division_Alg} there exists
        $q,r\in\mathbb{N}$ such that $r<m$ and $\minus{n}=q\cdot{m}+r$.
        But then:
        \begin{equation}
            n=(\minus{q})\cdot{m}-r=(\minus{q}-1)\cdot{m}+(m-r)
        \end{equation}
        but since $0\leq{r}<m$ we have $0\leq{m}-r<m$. Lastly, if $n<0$
        and $m<0$, then $\minus{n}>0$ and $\minus{m}>0$. But then by
        Thm.~\ref{thm:Weak_Euc_Division_Alg} there exists $q,r$ with
        $\minus{n}=q\cdot(\minus{m})+r$ and $r<\minus{m}$. But then
        $n=q\cdot{m}-r$, and hence $n=(q+1)\cdot{m}+(\minus{r}-m)$. But 
        $r<\minus{m}$, and therefore $0<\minus{r}-m$ and
        $\minus{r}-m<|m|$. Moreover, $q$ and $r$ are unique. For suppose
        $n=q_{0}\cdot{m}+r_{0}$ and $n=q_{1}\cdot{m}+r_{1}$. If
        $q_{0}\ne{q}_{1}$, then either $q_{0}<q_{1}$ or $q_{1}<q_{0}$.
        Suppose $q_{0}<q_{1}$ and let $k=q_{1}-q_{0}$. But then
        \begin{equation}
            n=q_{0}\cdot{m}+r_{0}=(q_{1}-k)\cdot{m}+r_{0}
                =q_{1}\cdot{m}+r_{0}-k\cdot{m}
                =q_{1}\cdot{m}+r_{1}
        \end{equation}
        and therefore $r_{1}=r_{0}-k\cdot{m}$, so
        $r_{0}=r_{1}+k\cdot{m}$, a contradiction since $r_{0}<m$. Hence,
        $q_{0}=q_{1}$. But then taking the difference we obtain
        $r_{0}=r_{1}$. Thus, they are unique.
    \end{bproof}
    \begin{fdefinition}{Divisor of an Integer}{Divisor_of_Integer}
        A divisor of an integer $n\in\mathbb{Z}$ is an integer
        $m\in\mathbb{Z}$ such that there exists and integer
        $k\in\mathbb{Z}$ where $k\cdot{m}=n$. We denote this by $m|n$.
    \end{fdefinition}
    \begin{theorem}
        \label{thm:One_is_Divisor}%
        If $n\in\mathbb{Z}$, then $1|n$.
    \end{theorem}
    \begin{proof}
        For $n=1\cdot{n}$, and hence $1$ is a divisor of $n$
        (Def.~\ref{def:Divisor_of_Integer}).
    \end{proof}
    \begin{theorem}
        \label{thm:Greater_is_not_divisor_of_lesser}%
        If $n,m\in\mathbb{N}^{+}$ and $m<n$, then $n$ does not divide
        $m$.
    \end{theorem}
    \begin{proof}
        For suppose not. If $n|m$, then there exists $k\in\mathbb{Z}$
        such that $m=k\cdot{n}$ (Def.~\ref{def:Divisor_of_Integer}). But
        $n,m\in\mathbb{N}^{+}$ and therefore $n>0$ and $m>0$. But
        $m=k\cdot{n}$, and hence $k>0$. But then
        $m=k\cdot{n}\leq{1}\cdot{n}=n$, a contradiction since $m<n$.
    \end{proof}
    Well ordering of $\mathbb{N}$. Well ordering of countable set.
    \begin{fdefinition}{Greatest Common Divisor}{GCD}
        A greatest common divisor of two integers
        $n,m\in\mathbb{Z}\setminus\{0\}$ is an integer
        $k\in\mathbb{N}^{+}$ such that $k|n$, $k|m$, and for all
        $j\in\mathbb{N}^{+}$ where $j|n$ and $j|m$ it is true that
        $j\leq{k}$.
    \end{fdefinition}
    We say \textit{a} greatest divisor since we don't know
    \textit{a priori} if there are many, or if there is one at all.
    \begin{theorem}
        \label{thm:GCD_Existence_Theorem}%
        If $n,m\in\mathbb{Z}\setminus\{0\}$, then there exists a
        greatest common divisor.
    \end{theorem}
    \begin{proof}
        For let $G\subseteq\mathbb{N}^{+}$ be defined by:
        \begin{equation}
            G=\{\,k\in\mathbb{N}^{+}\;|\;
                k\textrm{ divides }n\textrm{ and }
                k\textrm{ divides }m\,\}
        \end{equation}
        Then $G$ is non-empty since $1\in{G}$. That is, $1|n$ and $1|m$
        (Thm.~\ref{thm:One_is_Divisor}). If $k\in{G}$, then
        $k$ divides $|n|$ and $|m|$. But since
        $n,m\in\mathbb{Z}\setminus\{0\}$ it is true that
        $|n|,|m|\in\mathbb{N}^{+}$. But then if $k$ divides $|n|$ and $|m|$,
        then $k<|n|$ and $k<|m|$
        (Thm.~\ref{thm:Greater_is_not_divisor_of_lesser}). Hence, $G$ is
        bounded above. But then there is a greatest element $k\in{G}$.
        But then $k|m$ and $k|n$, and for all $j\in\mathbb{N}^{+}$ such
        that $j|m$ and $j|n$ it is true that $j\leq{k}$. Hence, $k$ is
        a greatest common divisor (Def.~\ref{def:GCD}).
    \end{proof}
    \begin{theorem}
        \label{thm:GCD_Unique}%
        If $n,m\in\mathbb{Z}\setminus\{0\}$, if $k$ is a greatest common
        divisor of $n$ and $m$, and if $j$ is a greatest common
        divisor of $n$ and $m$, then $k=j$.
    \end{theorem}
    \begin{proof}
        For if not, then either $j<k$ or $k<j$. But if $j<k$ then $j$ is
        not the greatest common divisor since there exists an element
        $k\in\mathbb{N}^{+}$ such that $k|m$, $k|n$, and such that
        $j<k$, a contradiction (Def.~\ref{def:GCD}). Similarly
        $k\not<j$, and hence $j=k$.
    \end{proof}
    Now that we know there's a unique greatest common divisor, we
    use the following notation:
    \begin{fnotation}{Greatest Common Divisor}{GCD}
        The greatest common divisor of $n,m\in\mathbb{Z}\setminus\{0\}$
        is denoted $\GCD(n,m)$.
    \end{fnotation}
    The algorithmic way to go about showing there exists a unique
    greatest common divisor is to apply Euclid's algorithm. We perform
    division with remainder repeatedly until we're left with no
    remaining term. Suppose we're given $n,m\in\mathbb{N}^{+}$ and want
    to compute $\GCD(n,m)$. We do:
    \par
    \begin{subequations}
        \begin{minipage}[b]{0.49\textwidth}
            \centering
            \begin{align}
                n&=q_{0}m+r_{0}\\
                m&=q_{1}r_{0}+r_{2}\\
                r_{0}&=q_{2}r_{1}+r_{2}
            \end{align}
        \end{minipage}
        \hfill
        \begin{minipage}[b]{0.49\textwidth}
            \centering
            \begin{align}
                r_{n-3}&=q_{n-1}r_{n-2}+r_{n-1}\\
                r_{n-2}&=q_{n}r_{n-1}+r_{n}\\
                r_{n-1}&=q_{n+1}r_{n}
            \end{align}
        \end{minipage}
    \end{subequations}
    \par\vspace{2.5ex}
    The $\GCD$ is the last non-zero remainder term.
    \begin{example}
        Let's compute the $\GCD$ of $34$ and $51$. We have:
        \twocolumneq{34=1\cdot{34}+17}{34=2\cdot{17}}
        and so $\GCD(34,51)=17$. Perhaps bigger numbers will better
        demonstrate the algorithm. Let's compute $\GCD(57970,10353)$. We
        obtain:
        \par
        \begin{subequations}
            \begin{minipage}[b]{0.49\textwidth}
                \centering
                \begin{align}
                    57970&=5\cdot{10353}+6205\\
                    10353&=1\cdot{6205}+4148\\
                    6205&=1\cdot{4148}+2057
                \end{align}
            \end{minipage}
            \hfill
            \begin{minipage}[b]{0.49\textwidth}
                \centering
                \begin{align}
                    4148&=2\cdot{2057}+34\\
                    2057&=60\cdot{34}+17\\
                    34&=2\cdot{17}
                \end{align}
            \end{minipage}
        \end{subequations}
        \par\vspace{2.5ex}
        after running the gauntlet, the last equation has zero
        remainder: $34=2\cdot{17}+0$ and hence $\GCD(57970,10353)=17$.
    \end{example}
    \begin{fdefinition}{Least Common Multiple}{LCM}
        A least common multiple of integers
        $n,m\in\mathbb{Z}\setminus\{0\}$ is an integer
        $k\in\mathbb{N}^{+}$ where $n|k$, $m|k$, and for all
        $j\in\mathbb{N}^{+}$ with $n|k$ and $m|k$ it is true that
        $k\leq{j}$.
    \end{fdefinition}
    Again, we use the phrasing \textit{a} least common multiple since
    the existence and uniqueness of such a concept has not yet been
    proved. We now perform this task.
    \begin{theorem}
        \label{thm:Integer_Divides_Multiple_of_Self}%
        If $n,m\in\mathbb{Z}$, if $k=n\cdot{m}$, then $n$ divides $k$.
    \end{theorem}
    \begin{proof}
        For by hypothesis there exists an $m\in\mathbb{Z}$ such that
        $n\cdot{m}=k$ and hence $n$ divides $k$
        (Def.~\ref{def:Divisor_of_Integer}).
    \end{proof}
    \begin{theorem}
        \label{thm:LCM_Existence_Theorem}%
        If $n,m\in\mathbb{Z}\setminus\{0\}$, then there is a least
        common multiple of $n$ and $m$.
    \end{theorem}
    \begin{proof}
        For let $G\subseteq\mathbb{N}^{+}$ be defined by:
        \begin{equation}
            G=\{\,k\in\mathbb{N}^{+}\;|\;
                n\textit{ divides }k\textrm{ and }
                m\textit{ divides }k\,\}
        \end{equation}
        Then $G$ is non-empty since $|n|\cdot|m|\in{G}$
        (Thm.~\ref{thm:Integer_Divides_Multiple_of_Self}). But if $G$ is
        a non-empty subset of $\mathbb{N}^{+}$, then there is a least
        element $k\in{G}$. But then $n$ divides $k$, $m$ divides $k$,
        and for all $j$ with $n|j$ and $m|j$ it is true that
        $k\leq{j}$. Hence, $k$ is a least common multiple of $n$ and $m$
        (Def.~\ref{def:LCM}).
    \end{proof}
    Much like the greatest common divisor, we now prove that least
    common multiples are unique and then assign a notation to the
    concept.
    \begin{theorem}
        \label{thm:LCM_Unique}%
        If $n,m\in\mathbb{Z}^{+}$, if $k$ is a least common multiple of
        $n$ and $m$, and if $j$ is a least common multiple of $n$ and
        $m$, then $k=j$.
    \end{theorem}
    \begin{proof}
        For if not, then either $j<k$ or $k<j$, violating minimality.
    \end{proof}
    \begin{fnotation}{Least Common Multiple}{LCM}
        The least common multiple of $n,m\in\mathbb{Z}\setminus\{0\}$ is
        denoted $\LCM(n,m)$.
    \end{fnotation}
    One of the most useful theorems in number theory is from the French
    mathematician \'{E}tienne B\'{e}zout. The theorem, known as
    B\'{e}zout's identity, was proved in the $18^{th}$ century in the
    setting of polynomials, but when applied to integers it allows to us
    make many theorems of great historical importance into short
    corollaries. For example, as we will show, the effort in proving
    Euclid's prime number lemma and the Chinese remainder theorem are
    greatly diminshed by applying the identity. Later, in the context of
    \textit{rings}, a generalization will be given.
    \begin{ftheorem}{B\'{e}zout's Identity}{Bezout_Identity}
        If $a,b\in\mathbb{N}\setminus\{0\}$, and if $d=\GCD(a,b)$, then
        there exist $n,m\in\mathbb{Z}$ such that:
        \begin{equation*}
            a\cdot{n}+b\cdot{m}=d
        \end{equation*}
    \end{ftheorem}
    \begin{bproof}
        For let $G\subseteq\mathbb{Z}\setminus\{0\}$ be defined by:
        \begin{equation}
            G=\{\,k\in\mathbb{N}\setminus\{0\}\;|\;
                \textrm{ There exists }n,m\in\mathbb{Z}
                \textrm{ such that }k=an+bm\big)\,\}
        \end{equation}
        then $G$ is non-empty since $a^{2}+b^{2}\in{G}$. But then there
        is a least element $d\in{G}$. By Euclid's division algorithm,
        there exists $q\in\mathbb{Z}$ and $r\in\mathbb{N}$ such that
        $r<d$ and $a=q\cdot{d}+r$
        (Thm.~\ref{thm:Euclid_Division_Algorithm}). But $d\in{G}$, and
        hence there are $n,m\in\mathbb{Z}$ such that
        $d=a\cdot{n}+b\cdot{m}$. But then:
        \begin{equation}
            r=a-qd=a-q(na+mb)=a-aqn-bqm=a(1-qn)+b\cdot(\minus{q}m)
        \end{equation}
        But if $r>0$, then $r\in{G}$ and $r<d$, a contradiction since
        $d$ is the least element of $G$. Hence, $r=0$ and $d$ divides
        $a$ (Def.~\ref{def:Divisor_of_Integer}). Similarly, $d$ divides
        $b$. If $c\in\mathbb{N}^{+}$ is such that $c|a$ and $c|b$, then
        there exists $s,t\in\mathbb{Z}$ such that $a=c\cdot{s}$ and
        $b=c\cdot{t}$. But then:
        \begin{equation}
            d=a\cdot{n}+b\cdot{m}=(c\cdot{s})n+(c\cdot{t})m=c(sn+tm)
        \end{equation}
        and therefore $d$ divides $c$. But it $d$ divides $c$, then
        $d$ is not greater than $c$
        (Thm.~\ref{thm:Greater_is_not_divisor_of_lesser}), and therefore
        $c\leq{d}$. Thus, $d$ is the greatest common divisor
        (Def.~\ref{def:GCD}).
    \end{bproof}
    \begin{theorem}
        \label{thm:Divisor_of_AB_Divides_GCD}%
        If $a,b,n\in\mathbb{Z}\setminus\{0\}$, and if $n|a$ and $n|b$,
        then $n|\GCD(a,b)$.
    \end{theorem}
    \begin{proof}
        By B\'{e}zout's identity there exists $s,t\in\mathbb{Z}$ such
        that $as+bt=\GCD(a,b)$ (Thm.~\ref{thm:Bezout_Identity}). But
        $n|a$ and $n|b$, and thus there exists $u,v\in\mathbb{Z}$ such
        that $nu=a$ and $nv=b$ (Def.~\ref{def:Divisor_of_Integer}). But
        then $n(us+vt)=\GCD(a,b)$. Therefore, $n$ divides $\GCD(a,b)$
        (Def.~\ref{def:Divisor_of_Integer}).
    \end{proof}
    \begin{theorem}
        \label{thm:n_Div_AB_then_A_Div_AS_BT}%
        If $a,b,s,t,n\in\mathbb{Z}$, if $n|a$, and if $n|b$, then $a$
        divides $as+bt$.
    \end{theorem}
    \begin{proof}
        For if $n$ divides $a$ and $b$, then there exists
        $u,v\in\mathbb{Z}$ such that $nu=a$ and $nv=b$
        (Def.~\ref{def:Divisor_of_Integer}). But then
        $as+bt=nus+nvt=n(us+bt)$, and thus $n$ divides $as+bt$
        (Def.~\ref{def:Divisor_of_Integer}).
    \end{proof}
    As claimed, many great theorems from antiquity become corollaries of
    B\'{e}zout's identity. We now demonstrate this.
    \begin{ftheorem}{Euclid's Prime Number Lemma}
                    {Euclid_Prime_Number_Lemma}
        If $a,b,p\in\mathbb{Z}\setminus\{0\}$, if $\GCD(a,p)=1$, and if
        $p$ divides $a\cdot{b}$, then $p$ divides $b$.
    \end{ftheorem}
    \begin{bproof}
        Since $\GCD(a,p)=1$, by B\'{e}zout's identity there exist
        integers $n,m\in\mathbb{Z}$ such that $an+pm=1$
        (Thm.~\ref{thm:Bezout_Identity}). But then $ban+bpm=b$. But $p$
        divides $ab$ and hence there is a $k\in\mathbb{Z}$ such that
        $kp=ab$ (Def.~\ref{def:Divisor_of_Integer}) and therefore
        $kpn+bpm=b$. But then $p(kn+bm)=b$, and hence $p$ divides $b$
        (Def.~\ref{def:Divisor_of_Integer}).
    \end{bproof}
    Euclid's phrasing of this theorem, as presented in his
    \textit{elements}, goes as follows:
    \begin{theorem}
        \label{thm:Prime_Div_AB_then_PdivA_or_PdivB}%
        If $a,b\in\mathbb{N}^{+}$, if $p\in\mathbb{N}^{+}$ is prime, and
        if $p$ divides $a\cdot{b}$, then either $p$ divides $a$ or $p$
        divides $b$.
    \end{theorem}
    \begin{proof}
        For if $p$ does not divide $a$, then $\GCD(a,p)=1$, and thus
        $p$ divides $b$ (Thm.~\ref{thm:Euclid_Prime_Number_Lemma}).
        Similary, if $p$ does not divide $b$ then it divides $a$.
    \end{proof}
    \begin{theorem}
        If $m,n\in\mathbb{N}^{+}$, if $l$ is the least common multiple of
        $n$ and $m$, and if $d$ is the greatest common divisor of $n$ and
        $m$, then $l\cdot{d}=n\cdot{m}$.
    \end{theorem}
    \begin{ftheorem}{Fundamental Theorem of Arithmetic}
                    {Fundamental_Theorem_of_Arithmetic}
        If $n\in\mathbb{N}^{+}$, then there exists a unique
        $n\in\mathbb{N}^{+}$, a unique strictly increasing sequence
        $P:\mathbb{Z}_{n}\rightarrow\mathbb{N}^{+}$ and a unique
        sequence $N:\mathbb{Z}_{n}\rightarrow\mathbb{N}^{+}$ such that
        for all $k\in\mathbb{Z}_{n}$ it is true that $P_{k}$ is prime
        and:
        \begin{equation*}
            n=\prod_{k\in\mathbb{Z}_{n}}P_{k}^{N_{k}}
        \end{equation*}
    \end{ftheorem}
    That is, every integer has a unique prime factorization. If $n$ is a
    prime, the factorization is simply
    $P:\mathbb{Z}_{1}\rightarrow\mathbb{N}^{+}$ defined by $P_{0}=n$ and
    $N:\mathbb{Z}_{1}\rightarrow\mathbb{N}^{+}$ with $N_{0}=1$.
    \begin{theorem}
        \label{thm:Composite_N_Exists_AB_N_Div_AB_and_N_NDiv_A_or_B}%
        If $n\in\mathbb{N}^{+}$ is not prime, then there exists
        integers $a,b\in\mathbb{N}^{+}$ such that $n$ divides
        $ab$, but $n$ does not divide $a$ and $n$ does not divide $b$.
    \end{theorem}
    \begin{proof}
        For of $n$ is composite, then there is a prime $p$ that divides
        $n$. Let $Q=n/p$. By the fundamental theorem of arithmetic,
        there exists a subset
        $S\subseteq\mathbb{N}^{+}\times\mathbb{N}^{+}$ such that for all
        $(q,n)\in{S}$ it is true that $q$ is prime and:
        \begin{equation}
            Q=\prod_{(p,n)\in{S}}q^{n}
        \end{equation}
        Since there are infinitely many primes, there is a prime $P$ not
        contained in the projection of $S$ into the first variable. Let
        $a=p$ and $b=Q\cdot{P}$. Then $n$ does not divide $a$ or $b$,
        but it does divide $a\cdot{b}$.
    \end{proof}
    \begin{example}
        The smallest example we have to work with is 4, and so we direct
        our attention there. 4 divides 12, and $12=4\cdot{3}$. Following
        the proof of
        Thm.~\ref{thm:Composite_N_Exists_AB_N_Div_AB_and_N_NDiv_A_or_B},
        we remove a prime from 4 and are left with 2. We then pick a
        prime that is not in the prime factorization of 4 and multiply
        by this. That is, we have $p=2$, $Q=2$, and $P=3$. We set
        $a=p=2$ and $b=Q\cdot{P}=2\cdot{3}=6$. Then 4 does not divide 2
        since $2<4$ and moreover 4 does not divide 6, but it does divide
        $2\cdot{6}=12$.
    \end{example}
    \begin{ftheorem}{Chinese Remainder Theorem}
                    {Chinese_Remainder_Theorem}
        If $n\in\mathbb{N}^{+}$ is an integer, if
        $N:\mathbb{Z}_{n}\rightarrow\mathbb{N}$ is such that for all
        $i,j\in\mathbb{Z}_{n}$ with $i\ne{j}$ it is true that
        $\GCD(N_{i},N_{j})=1$, and if
        $A:\mathbb{Z}_{n}\rightarrow\mathbb{N}$ is a sequence of
        integers such that $A_{i}<N_{i}$ for all $i\in\mathbb{Z}_{n}$,
        then there is a unique $x\in\mathbb{N}$ such that for all
        $i\in\mathbb{Z}_{n}$ it is true that $x\equiv{A}_{i}\mod{N}_{i}$
        and $x<\prod_{k\in\mathbb{Z}_{n}}N_{k}$
    \end{ftheorem}
    \begin{bproof}
        By induction. The base case, since $N_{1}$ and $N_{2}$ are
        relatively prime, by B\'{e}zout's identity there exist integers
        $m_{1},m_{2}\in\mathbb{Z}$ such that $m_{1}N_{1}+m_{2}N_{2}=1$.
        Let $x=m_{2}N_{2}A_{1}+m_{1}N_{1}A_{2}$. In the induction case,
        there is an $x$ such that $x\equiv{A}_{i}\mod{N}_{i}$ for
        $i\in\mathbb{Z}_{n-1}$ and $x\equiv{A}_{n}A_{n+1}\mod{N}_{n}N_{n+1}$
        since $N_{n}$ and $N_{n+1}$ are relatively prime.
    \end{bproof}
    We can get some use out of this by studying the Euler totient
    function.
    \begin{theorem}
        \label{thm:Rel_Prime_to_Prod_of_Rel_Primes}%
        If $n,m\in\mathbb{N}$, if $\GCD(n,m)=1$, if $k\in\mathbb{N}$,
        and if $k<mn$, then $\GCD(k,n\cdot{m})=1$ if and only if
        $\GCD(k,n)=1$ and $\GCD(k.m)=1$
    \end{theorem}
    \begin{proof}
        Somehow the Chinese remainder theorem does this. Moving on for
        now.
    \end{proof}
    $S_{1}\subseteq{S}_{2}$, we see that
    $\varphi(a)$ divides $\varphi(b)$.  \begin{fdefinition}{Euler Totient Function}{Euler_Totient_Func}
        The Euler totient function is the function
        $\varphi:\mathbb{N}^{+}\rightarrow\mathbb{N}^{+}$ defined by:
        \begin{equation*}
            \varphi(n)=\cardinality{\{\,k\in\mathbb{N}\;|\;
                k\leq{n}\textrm{ and }\GCD(k,n)=1\,\}}
        \end{equation*}
    \end{fdefinition}
    Since there are so many different uses of the symbol $\varphi$, when
    a theorem pertains to the Euler totient function we will explicitly
    say so.
    \begin{theorem}
        \label{thm:Euler_Totient_of_Prime}%
        If $p\in\mathbb{N}$ is a prime number and if $\varphi$ is the
        Euler totient function, then $\varphi(p)=p-1$.
    \end{theorem}
    \begin{proof}
        For all $k\in\mathbb{N}^{+}$ such that $k<p$ it is true that
        $\GCD(k,p)=1$ since $p$ is prime. Hence, $\varphi(p)$ is equal
        to $\cardinality{\mathbb{Z}_{p}\setminus\{0\}}$ which  is $p-1$.
    \end{proof}
    \begin{theorem}
        \label{thm:Euler_Totient_Multiplicative}%
        If $a,b\in\mathbb{N}$, if $\varphi$ is the Euler totient
        function, if $\GCD(a,b)=1$, then:
        \begin{equation}
            \varphi(a\cdot{b})=\varphi(a)\cdot\varphi(b)
        \end{equation}
    \end{theorem}
    \begin{proof}
        For since $\GCD(n,m)=1$, for all $k<mn$ it is true that
        $\GCD(k,mn)=1$ if and only if $\GCD(k,m)=1$ and $\GCD(k,n)=1$
        (Thm.~\ref{thm:Rel_Prime_to_Prod_of_Rel_Primes}). Hence, there
        are $\varphi(a)\cdot\varphi(b)$ such elements.
    \end{proof}
    \begin{theorem}
        \label{thm:Euler_Totient_Powers_of_Primes}%
        If $p\in\mathbb{N}$ is prime, if $\varphi$ is the Euler totient
        function, and if $n\in\mathbb{N}^{+}$, then
        $\varphi(p^{n})=p^{n-1}(p-1)$.
    \end{theorem}
    \begin{proof}
        For if $p$ is prime, and if $m\in\mathbb{Z}_{p}\setminus\{0\}$,
        then since the only factors of $p^{n}$ are powers of $p$,
        $\GCD(p^{n},m)=p^{k}$ for some $k\in\mathbb{Z}_{n}$. There are
        $p\cdot{p}^{n-1}$ elements that are multiples of $p$, and hence
        $p^{n}-p^{n-1}$ elements that are coprime. Hence,
        $\varphi(p^{n})=p^{n}-p^{n-1}=p^{n-1}(p-1)$.
    \end{proof}
    We can combine the fundamental theorem of arithmetic together with
    these theorems to quickly compute the Euler totient function of a
    given value.
    \begin{example}
        The prime factorization of 12 is $2^{2}\cdot{3}$. Thus, we can
        compute $\varphi$ as follows:
        \begin{equation}
            \varphi(12)=\varphi(2^{2}\cdot{3})
                =\varphi(2^{2})\cdot\varphi(3)
                =2^{2-1}(2-1)\cdot(3-1)=2\cdot{1}\cdot{2}=4
        \end{equation}
        we can also just count out the relatively prime elements of
        $\mathbb{N}$ that are less than 12, and we obtain 1, 5, 7, and
        11. Powers of primes are particularly easy:
        \begin{equation}
            \varphi(16)=\varphi(2^{4})=2^{4-1}(2-1)=2^{3}=8
        \end{equation}
        the relatively prime elements are 1, 3, 5, 7, 9, 11, 13, and 15.
        That is, all of the odd numbers less than 16. Lastly, let's try
        $\varphi(75)$. We obtain:
        \begin{equation}
            \varphi(75)=\varphi(5^{2}*3)=5^{2-1}(5-1)\cdot(3-1)
                =5\cdot{4}\cdot{2}=40
        \end{equation}
    \end{example}
    \begin{theorem}
        \label{thm:SQRT_Primes_are_Irrational}%
        If $p\in\mathbb{N}^{+}$ is a prime number, then $\sqrt{p}$ is
        irrational.
    \end{theorem}
    \begin{proof}
        For suppose not. If $\sqrt{p}$ is rational, then there exists
        $a,b\in\mathbb{Z}$ such that $b\ne{0}$, $\GCD(a,b)=1$, and
        $\sqrt{p}=a/b$. But then $a^{2}=pb^{2}$ and thus $p$ divides
        $a\cdot{a}$ (Def.~\ref{def:Divisor_of_Integer}). Since $p$ is
        prime, by Euclid's prime number lemma $p$ divides $a$
        (Thm.~\ref{thm:Euclid_Prime_Number_Lemma}). But then there is a
        $k\in\mathbb{Z}$ such that $a=p\cdot{k}$. But then
        $a^{2}=p^{2}k^{2}=pb^{2}$, and hence $pk^{2}=b^{2}$. But then
        $p$ divides $b^{2}$ (Def.~\ref{def:Divisor_of_Integer}) and thus
        since $p$ is prime, $p$ divides $b$
        (Thm.~\ref{thm:Euclid_Prime_Number_Lemma}). But then $p$ divides
        $a$ and $b$, a contradiction since $\GCD(a,b)=1$ and $1<p$.
        Hence, $\sqrt{p}$ is irrational.
    \end{proof}
    Only finitely many $n$ have $\varphi(n)=N$ for a fixed $N$, where
    $\varphi$ is the Euler totient function.
    \begin{theorem}
        \label{thm:A_DIV_B_then_EulerTotA_Div_EulerTotB}%
        If $a,b\in\mathbb{N}^{+}$, if $a|b$, and if $\varphi$ is the
        Euler totient function, then $\varphi(a)$ divides $\varphi(b)$.
    \end{theorem}
    \begin{proof}
        For by the fundamental theorem of arithmetic
        (Thm.~\ref{thm:Fundamental_Theorem_of_Arithmetic}) there exist
        integers $m,n\in\mathbb{N}^{+}$, sequences
        $P,M:\mathbb{Z}_{m}\rightarrow\mathbb{N}^{+}$ and
        $Q,N:\mathbb{Z}_{n}\rightarrow\mathbb{N}^{+}$ such that for all
        $j\in\mathbb{Z}_{m}$ and $k\in\mathbb{Z}_{n}$, $P_{j}$ and
        $Q_{k}$ are prime, and:
        \twocolumneq{a=\prod_{j\in\mathbb{Z}_{m}}P_{j}^{M_{j}}}
                    {b=\prod_{k\in\mathbb{Z}_{n}}Q_{k}^{N_{k}}}
        But since $a$ divides $b$ there exists a $k\in\mathbb{Z}$ such
        that $b=a\cdot{k}$ (Def.~\ref{def:Divisor_of_Integer}). But
        since $P$ is a strictly increasing sequence and hence the
        $P_{j}$ are distinct, and since all of the $P_{j}$ are prime,
        all of the $P_{j}^{M_{j}}$ are relatively prime. Hence:
        \begin{align*}
            \varphi(a)&=\varphi\Big(
                \prod_{j\in\mathbb{Z}_{m}}P_{j}^{M_{j}}
            \Big)\\
            &=\prod_{j\in\mathbb{Z}_{m}}\varphi(P_{j}^{M_{j}})
                \tag{Thm.~\ref{thm:Euler_Totient_Multiplicative}}\\
            &=\prod_{j\in\mathbb{Z}_{m}}P_{j}^{N_{j}-1}(P_{j}-1)
                \tag{Thm.~\ref{thm:Euler_Totient_Powers_of_Primes}}
        \end{align*}
        and
        \begin{align*}
            \varphi(b)&=\varphi\Big(
                \prod_{k\in\mathbb{Z}_{n}}Q_{k}^{M_{k}}
            \Big)\\
            &=\prod_{k\in\mathbb{Z}_{n}}\varphi(Q_{k}^{M_{k}})
                \tag{Thm.~\ref{thm:Euler_Totient_Multiplicative}}\\
            &=\prod_{k\in\mathbb{Z}_{n}}Q_{k}^{N_{k}-1}(Q_{k}-1)
                \tag{Thm.~\ref{thm:Euler_Totient_Powers_of_Primes}}
        \end{align*}
        since $a$ divides $b$, and from the uniqueness of the sequences
        $P$ and $Q$, we see that $\varphi(a)$ divides $\varphi(b)$.
    \end{proof}
    \subsection{Exam I}
        \begin{problem}
            Find an integer $n$ such that $\gcd(n,4)=2$ and
            $\gcd(n,6)=3$, or prove that no such integer exists.
        \end{problem}
        \begin{proof}[Solution 1]
            If $\gcd(n,4)=2$, then ${2}\vert{n}$, and thus
            $\exists_{k\in\mathbb{Z}}:n=2k$. But
            $\gcd(n,6)=\gcd(2k,2\cdot 3)=2\gcd(k,3)$. But
            $\gcd(n,6)=3$, and therefore $2\gcd(k,3)=3$, a
            contradiction as $3$ is odd. No such $n$ exists.
        \end{proof}
        \begin{proof}[Solution 2]
            If $\gcd(n,4)=2$, then ${2}\vert{n}$, and thus
            $\exists_{j\in\mathbb{Z}}:n=2j$. If $\gcd(n,6)=3$,
            then ${3}\vert{n}$. Therefore
            $\exists_{k\in\mathbb{Z}}:n=3k$. But then $2j=3k$.
            As $3$ is odd, $k$ must be even. Therefore,
            $\exists_{m\in\mathbb{Z}}:k=2m$. But then
            $n=3k=3(2m)=6m$. Thus, ${6}\vert{n}$. But then
            $\gcd(n,6)=6$, a contradiction as $\gcd(n,6)=3$.
        \end{proof}
        \begin{proof}[Solution 3]
            If $\gcd(n,4)=2$, then ${2}\vert{n}$, and thus
            $\exists_{k\in\mathbb{Z}}:n=2k$. But $\gcd(n,6)=3$,
            and therefore $\exists_{x,y\in\mathbb{Z}}:nx+6y=3$.
            But $nx+6y=2kx+6y=2(kx+3y)$, and $nx+6y=3$, and
            therefore $2(nx+3y)=3$, a contradiction as $3$ is
            odd. No such $n$ exists.
        \end{proof}
        \begin{problem}
            Prove or disprove the following:
            \begin{enumerate}
                \begin{multicols}{2}
                    \item ${20}\vert{300}$
                    \item If $a>0$, then ${a}\vert{1}$
                    \item $\forall_{a,b>0}$, either
                        ${a}\vert{b}$ or ${b}\vert{a}$
                    \item $\forall_{a,b,c>0}$, if ${a}\vert{b}$
                        and ${a}\vert{(b+c)}$,
                        then ${a}\vert{(c-b)}$
                    \item $\forall_{a,b,c>0}$, if ${a}\vert{b}$
                        and ${a}\vert{c}$, then 
                        ${a}\vert{(b^{2}+c^{2})}$
                    \item $\forall_{a,b,c>0}$, if ${a}\vert{b}$
                        and $a\vert{(b^{2}+c^{2})}$, then
                        ${a}\vert{c}$
                    \item $\forall_{a,b,c>0}$, if ${a}\vert{b}$
                        and ${b}\vert{c}$, then $a\leq c$
                    \item If $a,b,c>0$, then
                        $\gcd(a,bc)\geq\gcd(a,b)$
                    \item If $a,b,c>0$, then
                        $\gcd(a,c-a)=\gcd(a+c,c)$
                    \item If $p$ is prime and
                        ${p^{3}}\vert{abc}$, then ${p}\vert{a}$
                    \item If $a+b$ is prime, then $ab$ is even.
                    \item If $a$ and $b$ are composite, then
                        $a+b$ is composite.
                    \item If $p$ is prime and ${p}\vert{a^{2}}$,
                        then $p^{2}\vert{a^{2}}$
                    \item If $0<b<a$, then $a^{2}-b^{2}$ is
                        composite.
                \end{multicols}
            \end{enumerate}
        \end{problem}
        \begin{proof}[Solution]
            \
            \begin{enumerate}
                \item True, for $300=20\cdot 15$
                \item False, for $2>0$, but $2$ does not divide
                    $1$
                \item False, for $5>0$ and $7>0$ but $5$ does
                    not divide $7$ and $7$ does not
                    divide $5$ for they are prime.
                \item True. If ${a}\vert{b}$, then
                    $\exists_{n\in\mathbb{Z}}:b=na$. If
                    ${a}\vert{(b+c)}$, then
                    $\exists_{m\in\mathbb{Z}}:b+c=ma$. But we
                    have that $c=ma-b=ma-na=a(m-n)$,
                    and therefore ${a}\vert{c}$. But then
                    $b-c=a(2n-m)$, so ${a}\vert{(b-c)}$
                \item True. If ${a}\vert{b}$ then
                    $\exists_{n\in\mathbb{Z}}:b=an$.
                    If ${a}\vert{c}$, then
                    $\exists_{m\in\mathbb{Z}}:c=am$. But then
                    $b^{2}+c^{2}=a^{2}n^{2}+a^{2}m^{2}%
                     =a(an^{2}+am^{2})$, and therefore
                    ${a}\vert{(b^{2}+c^{2})}$
                \item False. Let $a=4$, $b=8$, and $c=6$.
                    Then $b=2a$, $b^{2}+c^{2}=25a$, but $4$
                    does not divide $6$.
                \item True. If $a,b,c>0$ and ${a}\vert{b}$,
                    then $\exists_{n\in\mathbb{N}}:b=na$,
                    and therefore $a\leq b$. If
                    ${b}\vert{c}$, then
                    $\exists_{m\in\mathbb{N}}:c=mb$. But then
                    $b\leq c$. But $a\leq b$, and therefore
                    $a\leq c$
                \item True. If ${n}\vert{a}$ and ${n}\vert{b}$,
                    then ${n}\vert{a}$ and ${n}\vert{bc}$, and
                    therefore $\gcd(a,b)\leq\gcd(a,bc)$
                \item True. If ${n}\vert{a}$ and
                    ${n}\vert{(c-a)}$, then ${n}\vert{c}$. But
                    then ${n}\vert{(a+c)}$. If ${n}\vert{c}$
                    and ${n}\vert{(a+c)}$, then ${n}\vert{c}$.
                    But then ${n}\vert{(c-a)}$, and therefore
                    $\gcd(a,c-a)=\gcd(a+c,c)$
                \item False. Let $a=6$ and $c=10$. Then
                    $\gcd(a,b)=\gcd(6,10)=2$, and
                    $\gcd(a+c,c-a)=\gcd(16,4)=4$.
                \item False. Let $p=5$, $a=2$, $b=5$, and $c=25$.
                    Then $p$ is prime, ${p^{3}}\vert{abc}$, but
                    $5$ does not divide $2$
                \item False. Let $a=b=1$. Then $a+b=2$, which
                    is prime, but $ab=1$, which is odd.
                \item False. Let $a=9$, and $b=8$. Then $a$ and
                    $b$ are composite, but $a+b=17$,
                    which is prime.
                \item True. If ${p}\vert{a^{2}}$, then
                    $\exists_{n\in\mathbb{Z}}:a^{2}=np$. But, as
                    $p$ is prime, $a$ does not divide $p$, and
                    therefore $a=\frac{n}{a}p$. That is,
                    ${p}\vert{a}$. Therefore, ${p}\vert{a^{2}}$
                \item False. Let $a=9$ and $b=8$. Then
                    $9^{2}-8^{2}=81-64=17$, which is prime.
            \end{enumerate}
        \end{proof}
        \begin{problem}
            Use Euclid's Algorithm to compute $\gcd(201,62)$.
        \end{problem}
        \begin{proof}[Solution]
            \begin{align*}
                201&=62\cdot 3+15\\
                62&=15\cdot 5+2\\
                15&=2\cdot 7+1\\
                2&=1\cdot 2+0
            \end{align*}
            $\gcd(201,62)=1$
        \end{proof}
        \begin{problem}
            Find all integer solutions to $201x+62y=1$
        \end{problem}
        \begin{proof}[Solution 1]
            From the previous problem, we have:
            \begin{equation*}
                3+\frac{1}{4+\frac{1}{7}}=\frac{94}{29}
            \end{equation*}
            So $201(29)+62(-94)=1$. The general solution
            is therefore $x=29+62k$ and $y=-94-201k$ for
            all $k\in\mathbb{Z}$.
        \end{proof}
        \begin{proof}[Solution 2]
            From the previous problem, we have:
            \begin{align*}
                1&=15-2\cdot7&
                &=201\cdot(1+28)+62\cdot(-3-7-84)\\
                &=(201-63\cdot3)-(62-15\cdot4)\cdot7&
                &=201\cdot29+62\cdot(-94)\\
                &=(201-62\cdot3)-(62-(201-62\cdot3)\cdot4)\cdot7
            \end{align*}
            The general solution is $x=29+62k$ and $y=-94-201k$
        \end{proof}
        \begin{problem}
            Solve the following:
            \begin{enumerate}
                \begin{multicols}{2}
                    \item ${300^{3}+400^{4}}\mod{6}$
                    \item ${300^{3}+400^{4}}\mod{5}$
                    \item ${3^{1}}\mod{10}$
                    \item Last digit of $333^{222}$
                    \item ${1212^{11}}\mod{13}$
                    \item If $m$ is odd and $66\equiv{4}\mod{m}$,
                        what is $m$?
                    \item ${(21)(34)+765}\mod{9}$
                    \item ${48^{237}}\mod{4}$
                    \item ${3+3^{3}+3^{5}+3^{7}+3^{9}}\mod{8}$
                    \item If $2x\equiv{5}\mod{21}$, what is
                        ${x}\mod{21}$?
                \end{multicols}
            \end{enumerate}
        \end{problem}
        \begin{proof}[Solution]
            \par\hfill\par
            \begin{enumerate}
                \item We have
                    ${6}\vert{300}\Rightarrow%
                     300^{3}\equiv{0}\mod{6}$.
                    Also
                    $400\equiv{4}\mod{6}\Rightarrow%
                     400^{4}\equiv{4^{4}}\mod{6}%
                     ={256}\mod{6}\equiv{4}$
                \item
                    ${5}\vert{300}\Rightarrow{300^{3}}%
                     \equiv{0}\mod{5}$,
                    ${5}\vert{400}\Rightarrow{400^{4}}%
                     \equiv{0}\mod{5}$.
                    ${300^{3}+400^{4}}\equiv{0}\mod{5}$
                \item
                    ${3}\cdot{7}={21}\equiv{1}\mod{10}%
                     \Rightarrow{3^{-1}}\equiv{7}\mod{10}$
                \item
                    ${333}\equiv{3}\mod{10}\Rightarrow%
                     {333^{222}}\equiv{3^{222}}\mod{10}$. But
                    $3^{222}=9(3^{2})^{110}$, and
                    $9^{110}={81^{55}}\equiv{1}\mod{10}$.
                    So, ${333^{222}}\equiv{9}\mod{10}$
                \item
                    ${1212}\equiv{3}\mod{13}$, and
                    $3^{11}=9\cdot((3^{3})^{3}={9}\cdot{27}^{3}$.
                    But ${27}\equiv{1}\mod{13}$. So
                    ${1212^{11}}\equiv{9}\mod{13}$
                \item ${62}\equiv{0}\mod{m}$. But
                    $62={31}\cdot{2}$. $m=31$
                \item ${21}\equiv{3}\mod{9}$,
                    ${34}\equiv{7}\mod{9}$, and
                    ${765}\equiv{0}\mod{9}$. So we have
                    ${3}\cdot{7}={21}\equiv{3}\mod{9}$
                \item ${48}\equiv{0}\mod{4}$.
                \item $3^{2}\equiv{1}\mod{8}$,
                    $3^{5}\equiv{{3}\cdot{3^{4}}}\mod{8}%
                     \equiv{3}\mod{8}$,
                    $3^{7}\equiv{{3}\cdot{3^{6}}}\mod{8}%
                     \equiv{3}\mod{8}$, and finally
                    ${3^{9}}\equiv{{3}\cdot{3^{8}}}\mod{8}%
                     \equiv{3}\mod{8}$. So we have
                    $3+3+3+3+3={15}\equiv{7}\mod{8}$
                \item If ${2x}\equiv{5}\mod{21}$, then
                    $x\equiv{{5}\cdot{2^{-1}}}\mod{21}$.
                    But ${2^{-1}}\equiv{11}\mod{21}$, so
                    ${x}\equiv{{5}\cdot{11}}\mod{21}$ and
                    ${5}\cdot{11}={55}\equiv{13}\mod{21}$.
            \end{enumerate}
        \end{proof}
        \begin{problem}
            Find all integers $n,m\geq{0}$ such that
            $5^{n}-4^{m}=1$.
        \end{problem}
        \begin{proof}[Solution]
            $n=m=1$ is a solution since
            $5-4=1$. Suppose there is another solution.
            Note that $5^{0}-4^{0}=1-1=0$,
            $5^{1}-4^{0}=5-1=4$, and $5^{0}-4^{1}=1-4=-3$.
            If $m\geq{1}$ and $n\geq{2}$, we have
            $5^{n}-4^{m}>5^{n}-1\geq25-4=21>1$. If $m\geq{2}$,
            then $4^{m}$ is divisible by 8, and thus
            $4^{m}\mod{8}=0$. If $(n,m)$ is a solution, then
            $1=5^{n}-4^{n}\equiv{5^{n}}\mod{8}$, and thus
            $5^{n}\equiv{1}\mod{8}$. If $n$ is even, then
            $5^{2k}=25^{k}\equiv{1}\mod{8}$. If $n$ is odd, then
            $5^{2k+1}\equiv{5}\mod{8}$. Thus $n$ must be even if it
            is a solution. But if $5^{n}-4^{m}=1$,
            then $5^{n}-4^{m}\equiv{1}\mod{3}$. But
            $5^{n}-4^{m}\equiv{(-1)^{n}-(1)^{m}}\mod{3}$. But $n$ is
            even, and thus $5^{n}-4^{m}\equiv{0}\mod{8}$. But then
            $1\equiv{0}\mod{3}$, a contradiction. Thus, there is
            no other solution. $n=m=1$ is the only solution.
        \end{proof}
        \section{Modulo Arithmetic}
    \begin{theorem}
        \label{thm:Modulo_n_is_Equiv_Relation}%
        If $n\in\mathbb{N}$, and if
        $R\subseteq\mathbb{Z}\times\mathbb{Z}$ is defined by:
        \begin{equation}
            R=\{\,(a,b)\in\mathbb{Z}^{2}\;|\;
                n\textrm{ divides }b-a\,\}
        \end{equation}
        then $R$ is an equivalence relation on $\mathbb{Z}$.
    \end{theorem}
    \begin{proof}
        For $aRa$ since $a-a=0$ and $n$ divides 0. If $aRb$, then $n$
        divides $b-a$ and hence there is a $k\in\mathbb{Z}$ such that
        $nk=b-a$ (Def.~\ref{def:Divisor_of_Integer}). But then
        $n(\minus{k})=a-b$, and thus $n$ divides $a-b$
        (Def.~\ref{def:Divisor_of_Integer}), hence $bRa$. Lastly, if
        $aRb$ and $bRc$, then there exist integers $j,k\in\mathbb{Z}$
        such that $nj=b-a$ and $nk=c-b$. But then:
        \begin{equation}
            n(k+j)=nk+nk=(c-b)+(b-a)=c-a
        \end{equation}
        and therefore $aRc$.
    \end{proof}
    \begin{fdefinition}{Ring of Integers Modulo $n$}{Ring_Ints_Mod_N}
        The ring of integers modulo $n\in\mathbb{N}$ is the quotient set
        $\mathbb{Z}/R$ where $R$ is the equivalence relation
        $R\subseteq\mathbb{Z}^{2}$ defined by:
        \begin{equation*}
            R=\{\,(a,b)\in\mathbb{Z}^{2}\;|\;n\textrm{ divides }b-a\,\}
        \end{equation*}
        We denote this $\mathbb{Z}/n\mathbb{Z}$.
    \end{fdefinition}
    \begin{theorem}
        \label{thm:Z_n_is_Bij_onto_Z_mod_n}%
        If $n\in\mathbb{N}^{+}$, if $\mathbb{Z}/n\mathbb{Z}$ is the ring
        of integers modulo $n$, and if the function
        $\pi:\mathbb{Z}\rightarrow\mathbb{Z}/n\mathbb{Z}$ is the
        canonical projection map: $\pi(n)=[n]$, then
        $\pi|_{\mathbb{Z}_{n}}$ is bijective.
    \end{theorem}
    \begin{proof}
        For let $x\in\mathbb{Z}/n\mathbb{Z}$. Then there exists a
        representative $k\in\mathbb{Z}$ such that $[k]=x$. But by
        Euclid's division algorithm there exists $q\in\mathbb{Z}$ and
        $r\in\mathbb{N}$ such that $r<n$ and $k=qn+r$. But then
        $qn=k-r$ and hence $n$ divides $k-r$
        (Def.~\ref{def:Divisor_of_Integer}). But then $r\in[k]$, and
        hence $\pi(r)=\pi(k)=x$. And since $r<n$, $r\in\mathbb{Z}_{n}$.
        Hence, $\pi|_{\mathbb{Z}_{n}}$ is surjective. Moreover, if
        $a,b\in\mathbb{Z}_{n}$ and if $a\ne{b}$, then either $a<b$ or
        $b<a$. Suppose $a<b$. If $\pi(a)=\pi(b)$, then $[a]=[b]$ and
        hence $n$ divides $b-a$. But since $a,b\in\mathbb{Z}_{n}$, $a<n$
        and $b<n$, and thus $b-a<n$. But since $a<b$,
        $b-a\in\mathbb{N}^{+}$. But then $n$ the greater divides $b-a$
        the lesser, a contradiction
        (Thm.~\ref{thm:Greater_is_not_divisor_of_lesser}). Hence,
        $\pi|_{\mathbb{Z}_{n}}$ is injective, and is therefore a
        bijection. 
    \end{proof}
    In the case of $n=0$ we see that $\mathbb{Z}/0\mathbb{Z}$ reduces to
    just $\mathbb{Z}$. That is, the equivalence classes are just
    $[x]=\{x\}$ and thus
    $\pi:\mathbb{Z}\rightarrow\mathbb{Z}/0\mathbb{Z}$ is a bijection. We
    now wish to endow $\mathbb{Z}/n\mathbb{Z}$ with an arithmetic. We do
    so by borrowing the arithmetic from $\mathbb{Z}$ by using Euclid's
    division algorithm. We define this as follows.
    \begin{fdefinition}{Addition Modulo $n$}{Addition_Mod_n}
        The additive operation $\tilde{+}$ on $\mathbb{Z}/n\mathbb{Z}$
        (relabelled $Z_{n}$ for brevity) for $n\in\mathbb{N}^{+}$ is the
        set:
        \begin{equation*}
            \tilde{+}=\big\{\,\big((x,y),z\big)
                \in\big(Z_{n}\times{Z}_{n}\big)\times{Z}_{n}\;\big|\;
                \exists_{a,b\in\mathbb{Z}}
                \big(x=[a],y=[b],z=[a+b]\big)\big\}
        \end{equation*}
    \end{fdefinition}
    We only used the labelling $Z_{n}$ so that the equation didn't
    run off the page, and will not use it regularly, but rather stick
    to $\mathbb{Z}/n\mathbb{Z}$. The definition seems strange, but
    recall a function $f:A\rightarrow{B}$ is a subset
    $f\subseteq{A}\times{B}$ with certain properties, and a binary
    operation on $A$ is a function $*:A\times{A}\rightarrow{A}$. Hence,
    a binary operation is a particular subset of
    $(A\times{A})\times{A}$. We are now tasked with showing the
    definition given by Def.~\ref{def:Addition_Mod_n} forms a valid well
    defined binary operation on $\mathbb{Z}/n\mathbb{Z}$. To show that
    it is a binary operation amounts to showing that for every pair of
    elements $(x,y)$ there is a unique $z$ corresponding to this.
    \begin{theorem}
        \label{thm:Mod_Addition_is_Bin_Op}%
        If $n\in\mathbb{N}^{+}$, and if $\tilde{+}$ is the additive
        operation on $\mathbb{Z}/n\mathbb{Z}$, the $\tilde{+}$ is a
        binary operation.
    \end{theorem}
    \begin{proof}
        For if $x,y\in\mathbb{Z}/n\mathbb{Z}$, then there exists
        $a,b\in\mathbb{Z}$ such that $x=[a]$ and $y=[b]$. But then
        $\big((x,y)[a+b]\big)\in\tilde{+}$
        (Def.~\ref{def:Addition_Mod_n}). If $z\in\mathbb{Z}$ is such
        that $\big((x,y),z\big)\in\tilde{+}$, then there exists
        $\alpha,\beta,\gamma\in\mathbb{Z}$ such that
        $x=[\alpha]$, $y=[\beta]$, and $z=[\alpha+\beta]$. But if
        $x=[\alpha]$ and $x=[a]$, then $n$ divides $a-\alpha$ and hence
        there is a $j\in\mathbb{Z}$ such that $jn=a-\alpha$
        (Def.~\ref{def:Divisor_of_Integer}). Similarly there is a
        $k\in\mathbb{Z}$ such that $kn=b-\beta$. But then:
        \begin{equation}
            a+b-(\alpha+\beta)=(a-\alpha)+(b-\beta)=jn+kn=(j+k)n
        \end{equation}
        and hence $n$ divides $a+b-(\alpha+\beta)$
        (Def.~\ref{def:Divisor_of_Integer}). But then
        $[a+b]=[\alpha+\beta]$, and hence for all
        $x,y\in\mathbb{Z}/n\mathbb{Z}$ there is a unique
        $z\in\mathbb{Z}/n\mathbb{Z}$ such that
        $\big((x,y),z\big)\in\tilde{+}$, and hence $\tilde{+}$ is
        function. Moreover, since the domain of $\tilde{+}$ is
        $\mathbb{Z}/n\mathbb{Z}\times\mathbb{Z}/n\mathbb{Z}$ and the
        range is $\mathbb{Z}/n\mathbb{Z}$, $\tilde{+}$ is a binary
        operation.
    \end{proof}
    Since $\mathbb{Z}/n\mathbb{Z}$ can be put into a bijection with
    $\mathbb{Z}_{n}$ for $n\in\mathbb{N}^{+}$
    (Thm.~\ref{thm:Z_n_is_Bij_onto_Z_mod_n}), and since $\tilde{+}$
    gives a binary operation on $\mathbb{Z}/n\mathbb{Z}$
    (Thm.~\ref{thm:Mod_Addition_is_Bin_Op}) it custom to pull this
    binary operation back to $\mathbb{Z}_{n}$ and relabel it simply as
    $+$. It's always poor to use the same symbol for two different
    things that are frequently used in the same context, but alas it is
    the standard. In this case it is rather justifiable since
    $[j+k]$ is simply the remainder term of $j+k$ after division by $n$.
    \begin{example}
        The most common example one comes across of modulo arithmetic is
        in $\mathbb{Z}_{12}$ since this represents a clock. If it is 11
        A.M. and you wait for 3 hours, the time will then be 2 P.M. and
        hence $11+3=2$, quite paradoxical. One might claim ``Aha! I use
        military time!'' but then we simply apply the argument to
        $\mathbb{Z}_{24}$ and ask what time is 23 hours + 3 hours? That
        answer is 2 in the morning, hence $23+3=2$. There's no mystery
        once one realizes we are simply using the arithmetic of
        $\mathbb{Z}/n\mathbb{Z}$.
    \end{example}
    With modulo addition defined, we now turn to multiplication. Similar
    to addition, we build this new operation by borrowing from our
    familiar multiplicative operation on $\mathbb{Z}$ and thus push
    down to the equivalence classes in $\mathbb{Z}/n\mathbb{Z}$.
    \begin{fdefinition}{Multiplication Modulo $n$}{Multiplication_Mod_n}
        The multiplicative operation $\tilde{\cdot}$ on
        $\mathbb{Z}/n\mathbb{Z}$
        (relabelled $Z_{n}$ for brevity) for $n\in\mathbb{N}^{+}$
        is the set:
        \begin{equation*}
            \tilde{\cdot}=\big\{\,\big((x,y),z\big)
                \in\big(Z_{n}\times{Z}_{n}\big)\times{Z}_{n}\;\big|\;
                \exists_{a,b\in\mathbb{Z}}
                \big(x=[a],y=[b],z=[a\cdot{b}]\big)\big\}
        \end{equation*}
    \end{fdefinition}
    \begin{theorem}
        \label{thm:Mult_Mod_n_is_Bin_Op}%
        If $n\in\mathbb{N}^{+}$, if $\tilde{\cdot}$ is the
        multiplicative operation on $\mathbb{Z}/n\mathbb{Z}$, then
        $\tilde{\cdot}$ is a binary operation on
        $\mathbb{Z}/n\mathbb{Z}$.
    \end{theorem}
    \begin{proof}
        For if $x,y\in\mathbb{Z}/n\mathbb{Z}$ then there exist
        $n,m\in\mathbb{Z}$ such that $x=[n]$ and $y=[m]$. But then
        $\big((x,y),[n\cdot{m}]\big)\in\tilde{\cdot}$
        (Def.~\ref{def:Multiplication_Mod_n}). If
        $\big((x,y),z\big)\in\tilde{\cdot}$, then there exists
        $\alpha,\beta\in\mathbb{Z}/n\mathbb{Z}$ such that
        $x=[\alpha]$, $y=[\beta]$, and $z=[\alpha\cdot\beta]$
        (Def.~\ref{def:Multiplication_Mod_n}). But if $x=[\alpha]$ and
        $x=[a]$, then $[a]=[\alpha]$ and hence $n$ divides $a-\alpha$.
        Similarly, $n$ divides $b-\beta$. But then there exists
        $j,k\in\mathbb{Z}$ such that $jn=a-\alpha$ and $kn=b-\beta$
        (Def.~\ref{def:Divisor_of_Integer}). But then
        $a=jn+\alpha$ and $b-kn+\beta$, hence:
        \begin{subequations}
            \begin{align}
                (a\cdot{b})-(\alpha\cdot\beta)
                    &=(\alpha+jn)(\beta+kn)-(\alpha\cdot\beta)\\
                    &=\alpha\beta+jn\beta+kn\alpha-\alpha\cdot\beta\\
                    &=jn\beta+kn\alpha\\
                    &=n(j\beta+k\alpha)
            \end{align}
        \end{subequations}
        and hence $n$ divides $a\cdot{b}-\alpha\cdot\beta$. But then
        $[a\cdot{b}]=[\alpha\cdot\beta]$ and therefore for all
        $x,y\in\mathbb{Z}/n\mathbb{Z}$ there is a unique
        $z\in\mathbb{Z}/n\mathbb{Z}$ such that
        $\big((x,y),z\big)\in\tilde{\cdot}$. Thus, $\tilde{\cdot}$ is a
        binary operation on $\mathbb{Z}/n\mathbb{Z}$.
    \end{proof}
    While we occasional use $\mathbb{Z}_{n}$, $+$, and $\cdot$ in place
    of $\mathbb{Z}/n\mathbb{Z}$, $\tilde{+}$, and $\tilde{\cdot}$, it is
    worthwhile to note that the elements of $\mathbb{Z}/n\mathbb{Z}$ are
    \textit{not} the same as the elements of $\mathbb{Z}_{n}$.
    $\mathbb{N}$ was constructed from the axiom of infinity and the
    elements look like
    $\emptyset,\{\emptyset\},\{\emptyset,\{\emptyset\}\}$, and so forth.
    Meanwhile $\mathbb{Z}/n\mathbb{Z}$ was constructed from an
    equivalence relation, and hence the elements of
    $\mathbb{Z}/n\mathbb{Z}$ are elements of the \textit{power set} of
    $\mathbb{N}$. Indeed, we know precisely what these elements are:
    \begin{equation}
        \begin{split}
            \mathbb{Z}/n\mathbb{Z}=
            \Big\{\,&\{\,0,\,\pm{n},\,\pm{2n},\,\pm{3n},\,\dots\,\},\\
                &\{\,1,\,1\pm{n},\,1\pm{2n},\,1\pm{3n},\,\dots\,\},\\
                &\{\,2,\,2\pm{n},\,2\pm{2n},\,2\pm{3n},\,\dots\,\},\\
                &\dots,\\
                &\{\,n-1,\,n-1\pm{n},\,n-1\pm{2n},\,n-1\pm{3n},\,
                    \dots\,\}\Big\}
        \end{split}
    \end{equation}
    set theoretically these are different.
    Thm.~\ref{thm:Z_n_is_Bij_onto_Z_mod_n} tells us they're the same
    size and the way we've defined modulo arithmetic mimics the notion
    division with remainder in $\mathbb{Z}_{n}$. That is,
    writing $a+b=qn+r$ with $0\leq{r}<r$, we have defined
    $[a]\tilde{+}[b]=[r]$, and similary for $a\cdot{b}=sn+t$, we have
    $[a]\tilde{\cdot}[b]=[t]$. It is in this sense that we are justified
    in using the same notation.
    \begin{example}
        A common application of modulo arithmetic that one sees in
        elementary number theory is the computation of the last few
        digits of large powers of numbers. For example, consider $3^120$
        and suppose we want to know the last digit of this number.
        That's equivalent to asking what is the smallest representative
        of the equivalence class of $3^{120}$ in the ring of integers
        modulo 10. First we reduce the problem and recognize that
        $120=2\cdot{60}$, and hence
        $3^{120}=3^{2\cdot{60}}=(3^{2})^{60}$. Well $3^{2}=9$, which
        is equivalent to $\minus{1}$ in $\mathbb{Z}/10\mathbb{Z}$, and
        so we next consider $(\minus{1})^{60}$. But this is just 1.
        But 1 is congruent to 1 in $\mathbb{Z}/10\mathbb{Z}$ and there
        we have it: The last digit of $3^{120}$ is 1. Indeed, using your
        favorite computer language, we can compute and obtain:
        \begin{equation*}
            3^{120}=
            1797010299914431210413179829509605039731475627537851106401
        \end{equation*}
        and so our calculation was correct. Let us try $2^{2000}$ and
        attain the last 2 digits (But perhaps not compute the actual
        value). We note that $2000=10*200$ and every computer scientist
        recognizes instantly that $2^{10}=1024$, which is congruent to
        24 in $\mathbb{Z}/100\mathbb{Z}$. So we are left with
        $24^{200}$. We look at the exponent again, note that it is equal
        to $2\cdot{100}$ and $24^{2}$ seems a far simpler computation.
        We get $24^{2}=576$, the last two digits of which are 76, and so
        we're down to $76^{100}$. We break this into $(76^{2})^{50}$ and
        upon computing we get $76^{2}=5776$ and so the last two digits
        are once again 76, meaning we can continue decomposing away all
        of the powers of 2, and we are left with
        $75^{25}=76^{24}\cdot{76}$. But then removing the powers of 2
        away from 24 ($24=2^{3}\cdot{3}$), we are left with
        $76^{3}\cdot{76}=76^{4}$ which then reduces to $76$. The last
        two digits of $2^{2000}$ are 76.
    \end{example}
    \begin{theorem}
        \label{thm:Equiv_Class_of_0_is_Add_Identity_Mod_n}%
        If $n\in\mathbb{N}^{+}$, if $[0]\in\mathbb{Z}/n\mathbb{Z}$ is
        the equivalence class of 0 in the ring of integers modulo $n$,
        and if $x\in\mathbb{Z}/n\mathbb{Z}$, then $x\tilde{+}[0]=x$.
    \end{theorem}
    \begin{proof}
        For if $x\in\mathbb{Z}/n\mathbb{Z}$ then there is a
        $k\in\mathbb{Z}$ such that $x=[k]$. But then:
        \begin{equation}
            x\tilde{+}[0]=[k]\tilde{+}[0]=[k+0]=[k]=x
        \end{equation}
    \end{proof}
    \begin{theorem}
        \label{thm:Mod_Add_is_Assoc}%
        If $n\in\mathbb{N}^{+}$, if $x,y,z\in\mathbb{Z}/n\mathbb{Z}$,
        then $(x\tilde{+}y)\tilde{+}z=x\tilde{+}(y\tilde{+}z)$.
    \end{theorem}
    \begin{proof}
        For if $x,y,z\in\mathbb{Z}/n\mathbb{Z}$, then there exists
        $i,j,k\in\mathbb{Z}$ such that $x=[i]$, $y=[j]$, and $z=[k]$.
        But then:
        \par
        \begin{subequations}
            \begin{minipage}[t]{0.54\textwidth}
                \centering
                \begin{align}
                    (x\tilde{+}y)\tilde{+}z
                    &=([i]\tilde{+}[j])\tilde{+}[k]
                        \tag{Hypothesis}\\
                    &=[i+j]\tilde{+}[k]
                        \tag{Def.~\ref{def:Addition_Mod_n}}\\
                    &=[(i+j)+k]
                        \tag{Def.~\ref{def:Addition_Mod_n}}\\
                    &=[i+(j+k)]
                        \tag{Associativity}
                \end{align}
            \end{minipage}
            \hfill
            \begin{minipage}[t]{0.44\textwidth}
                \begin{align}
                    &=[i]\tilde{+}[j+l]
                        \tag{Def.~\ref{def:Addition_Mod_n}}\\
                    &=[i]\tilde{+}([j]\tilde{+}[k])
                        \tag{Def.~\ref{def:Addition_Mod_n}}\\
                    &=x\tilde{+}(y\tilde{+}z)
                        \tag{Hypothesis}
                \end{align}
            \end{minipage}
        \end{subequations}
        \par\vspace{2.5ex}
        proving the claim.
    \end{proof}
    \begin{theorem}
        \label{thm:Additive_Inv_Mod_n}%
        If $n\in\mathbb{N}^{+}$, and if $x\in\mathbb{Z}/n\mathbb{Z}$,
        then there is a $y\in\mathbb{Z}/n\mathbb{Z}$ such that
        $x\tilde{+}y=[0]$.
    \end{theorem}
    \begin{proof}
        For if $x\in\mathbb{Z}/n\mathbb{Z}$, then there is a
        $k\in\mathbb{Z}$ such that $x=[k]$. Let $y=[\minus{k}]$. But
        then:
        \begin{equation}
            x\tilde{+}y=[k]\tilde{+}[\minus{k}]
            =[k+(\minus{k})]=[0]
        \end{equation}
        proving the claim.
    \end{proof}
    \begin{theorem}
        \label{thm:Equiv_Class_of_1_is_Mult_Identity_Mod_n}%
        If $n\in\mathbb{N}^{+}$, if $[1]\in\mathbb{Z}/n\mathbb{Z}$ is
        the equivalence class of 1 in the ring of integers modulo $n$,
        and if $x\in\mathbb{Z}/n\mathbb{Z}$, then $x\tilde{\cdot}[1]=x$.
    \end{theorem}
    \begin{proof}
        For if $x\in\mathbb{Z}/n\mathbb{Z}$, then there is a
        $k\in\mathbb{Z}$ such that $x=[k]$. But then:
        \begin{equation}
            x\tilde{\cdot}[1]=[k]\tilde{\cdot}[1]=[k\cdot{1}]=[k]=x
        \end{equation}
    \end{proof}
    \begin{theorem}
        \label{thm:Mod_Mult_is_Assoc}%
        If $n\in\mathbb{N}^{+}$, if $x,y,z\in\mathbb{Z}/n\mathbb{Z}$,
        then $(x\tilde{\cdot}y)\tilde{\cdot}z%
        =x\tilde{\cdot}(y\tilde{\cdot}z)$.
    \end{theorem}
    \begin{proof}
        For if $x,y,z\in\mathbb{Z}/n\mathbb{Z}$, then there exists
        $i,j,k\in\mathbb{Z}$ such that $x=[i]$, $y=[j]$, and $z=[k]$.
        But then:
        \par
        \begin{subequations}
            \begin{minipage}[t]{0.54\textwidth}
                \centering
                \begin{align}
                    (x\tilde{\cdot}y)\tilde{\cdot}z
                    &=([i]\tilde{\cdot}[j])\tilde{\cdot}[k]
                        \tag{Hypothesis}\\
                    &=[i\cdot{j}]\tilde{\cdot}[k]
                        \tag{Def.~\ref{def:Multiplication_Mod_n}}\\
                    &=[(i\cdot{j})\cdot{k}]
                        \tag{Def.~\ref{def:Multiplication_Mod_n}}\\
                    &=[i\cdot(j\cdot{k})]
                        \tag{Associativity}
                \end{align}
            \end{minipage}
            \hfill
            \begin{minipage}[t]{0.44\textwidth}
                \begin{align}
                    &=[i]\tilde{\cdot}[j\cdot{l}]
                        \tag{Def.~\ref{def:Multiplication_Mod_n}}\\
                    &=[i]\tilde{\cdot}([j]\tilde{\cdot}[k])
                        \tag{Def.~\ref{def:Multiplication_Mod_n}}\\
                    &=x\tilde{\cdot}(y\tilde{\cdot}z)
                        \tag{Hypothesis}
                \end{align}
            \end{minipage}
        \end{subequations}
        \par\vspace{2.5ex}
        proving the claim.
    \end{proof}
    \begin{theorem}
        \label{thm:Invertible_Mod_n_iff_Relatively_Prime}%
        If $n\in\mathbb{N}^{+}$, if $x\in\mathbb{Z}/n\mathbb{Z}$, then
        there exists a $y\in\mathbb{Z}/n\mathbb{Z}$ such that
        $x\tilde{\cdot}{y}=[1]$ if and only if there is a
        $k\in\mathbb{Z}$ such that $x=[k]$ and $\GCD(k,n)=1$.
    \end{theorem}
    \begin{proof}
        For if $k$ is a representative for $x$ and $\GCD(k,n)=1$, then
        by B\'{e}zout's identity there exist $a,b\in\mathbb{Z}$ such
        that $a\cdot{k}+b\cdot{n}=1$ (Thm.~\ref{thm:Bezout_Identity}).
        But then $b\cdot{n}=1-a\cdot{k}$ and hence $n$ divides
        $1-a\cdot{k}$ (Def.~\ref{def:Divisor_of_Integer}). But then by
        the definition of the ring of integers modulo $n$,
        $1\in[a\cdot{k}]$ (Def.~\ref{def:Ring_Ints_Mod_N} and
        hence $[a]\tilde{\cdot}[k]=[1]$. In the other direction, if
        $x$ has an inverse $y$, then there are representatives $k,m$
        such that $x=[k]$ and $y=[m]$. But then
        $x\tilde{\cdot}{y}=[k]\tilde{\cdot}[m]=[k\cdot{m}]$. But by
        hypothesis $x\tilde{\cdot}{y}=[1]$ and hence $[k\cdot{m}]=[1]$.
        But then $n$ divides $1-km$ and hence there is a
        $j\in\mathbb{Z}$ such that $jn=1-km$
        (Def.~\ref{def:Divisor_of_Integer}). But then $nj+km=1$ and
        therefore $\GCD(k,n)=1$.
    \end{proof}
    Hence every non-zero element of $\mathbb{Z}/p\mathbb{Z}$ for some prime
    $p$ is invertible.
    \begin{example}
        Other tricks that use modulo arithmetic appear in disguise when one
        studies divisibility tricks. If $a=\sum{a}_{n}10^{n}$ is a finite
        sum, then $a$ is congruent of $\sum{a}_{n}$. Simply use the
        additivity of modulo addition, and use the fact that
        $10^{n}\equiv{1}\mod{9}$. Another trick comes from studying if a
        number is divisible by 3. Applying the same trick, we see that if
        3 divides $n$, then it divides the sum of its digits (in base 10).
        Conversely, if 3 divides the sum of the digits of $n$ (in base 10),
        then 3 divides $n$.
    \end{example}
    \begin{example}
        Looking back at a previous example, let's compute the last digit of
        $9^{n}$ for any $n\in\mathbb{N}$. We note that
        $9\equiv\minus{1}\mod{10}$ and hence we only need to consider
        $(\minus{1})^{n}$. But $(\minus{1})^{n}$ is 1 if $n$ is even and
        $\minus{1}$ is $n$ is odd. Therefore the last digit of $9^{n}$ is
        9 if $n$ is odd and 1 if $n$ is even. And indeed, the pattern holds
        true: 1, 9, 81, 729, 6561, and so on.
    \end{example}
    Come back to excercises (DF Chapt 1).
    \chapter{Cardinality}
        \section{Cardinality}
        Functions can also be called maps or mappings. The unique point
        $b\in{B}$ such that $(a,b)\in{f}$ is often called the image of
        $a$ under $f$. We sometimes write $a\mapsto{b}$, but most often
        will write $f(a)=b$.
        The image of a subset $A\subseteq{X}$ is the set of all points
        that get mapped onto by the function $f$ by the elements in $A$.
        In a similar manner we can define the opposite of this notion,
        called the pre-image.
        \begin{axiom}
            If $X$ is a non-empty set such that for all $x\in{X}$,
            $x\ne\emptyset$, then there is a function
            $f:X\rightarrow\bigcup_{x\in{X}}x$ such that,
            for all $x\in{X}$, $f(x)\in{x}$.
        \end{axiom}
        This is called the axiom of choice. It can be made into
        a blatantly obvious statement, by choosing a more careful
        wording, however many of the results it gives are far
        from intuitive. This says that, given a collection of sets,
        each of which is non-empty, one may choose a single element from
        each set. This choosing is the function $f$, and is often called
        a \textit{choice function}. For those interested, the axiom
        of choice is consistent with modern set theory (Called
        Zermelo-Fraenkel set theory, or ZF). It may thus be rejected
        or accepted without logical contradiction.
        \begin{theorem}
            \label{theorem:Set_Theory_Image_of_Empty_Set_Is_Empty}
            If $A$ and $B$ are sets, and if $f:A\rightarrow{B}$
            is a function, then:
            \begin{equation}
                f(\emptyset)=\emptyset
            \end{equation}
        \end{theorem}
        \begin{proof}
            For suppose not. Let $y\in{f}(\emptyset)$.
            But then there is an
            $x\in\emptyset$ such that $f(x)=y$, a contradiction
            ince for all $x$, $x\notin\emptyset$.
            Thus, $f(\emptyset)=\emptyset$.
        \end{proof}
        \begin{theorem}
            If $A$ and $B$ are sets, and if $f:A\rightarrow{B}$ is a function, then:
            \begin{equation}
                f^{-1}(\emptyset)=\emptyset
            \end{equation}
        \end{theorem}
        \begin{proof}
            For suppose not. Then there is an $x\in{X}$ such that
            $f(x)\in\emptyset$, a contradiction since for all $x$,
            $f(x)\notin\emptyset$. Therefore, etc.
        \end{proof}
        \begin{theorem}
            If $X$ and $Y$ are sets, if $A\subseteq{X}$, and if
            $f:X\rightarrow{Y}$ is a function such that
            $f(A)=\emptyset$, then $A=\emptyset$.
        \end{theorem}
        \begin{proof}
            For suppose not. If $A\ne\emptyset$, then there is an
            $x\in{A}$. But then $f(x)\in{f}(A)$, a contradiction as
            $f(A)=\emptyset$. Therefore, etc.
        \end{proof}
        \begin{theorem}
            If $X$ and $Y$ are sets, if $B$ is a subset of $Y$,
            and if $f:X\rightarrow{Y}$ is a function, then:
            \begin{equation}
                f\big(f^{-1}(B)\big)\subseteq{B}
            \end{equation}
        \end{theorem}
        \begin{proof}
            For if $y\in{f(f^{-1}(B))}$, then there is an
            $x\in{f^{-1}(B)}$ such that $y=f(x)$. But if
            $x\in{f^{-1}(B)}$, then $f(x)\in{B}$. Thus,
            $y\in{B}$. Therefore, etc.
        \end{proof}
        \begin{theorem}
            If $X$ and $Y$ are non-empty sets and if there exists
            $y_{1},y_{2}\in{Y}$ such that $y_{1}\ne{y}_{2}$, then
            there is a function $f:X\rightarrow{Y}$ and a
            $B\subseteq{Y}$ such that:
            \begin{equation}
                f\big(f^{-1}(B)\big)\ne{B}
            \end{equation}
        \end{theorem}
        \begin{proof}
            \begin{subequations}
                For if $X$ and $Y$ are non-empty, let $f:X\rightarrow{Y}$
                be defined by:
                \begin{equation}
                    f=\{(x,y_{1}):x\in{X}\}
                \end{equation}
                Then $f$ is a function, since $f\subseteq{X}\times{Y}$
                as $y_{1}\in{Y}$. Moreover, for all $x\in{X}$ there is a
                unique $y\in{Y}$ such that $(x,y)\in{f}$. Thus, $f$ is a
                function from $X$ to $Y$. However since for all
                $x\in{X}$, $f(x)=y_{1}$, we have that:
                \begin{equation}
                    f^{-1}(\{y_{2}\})=\emptyset
                \end{equation}
                For suppose $x\in{f}^{-1}(\{y_{2}\})$.
                Then $f(x)=y_{2}$, but for all $x\in{X}$, $f(x)=y_{1}$,
                and $y_{1}\ne{y}_{2}$. Thus
                $f^{-1}(\{y_{2}\})=\emptyset$. But by
                Thm.~\ref{theorem:Set_Theory_Image_%
                          of_Empty_Set_Is_Empty},
                $f(\emptyset)=\emptyset$. Therefore:
                \begin{equation}
                    f\big(f^{-1}(\{y_{2}\})\big)=\emptyset
                \end{equation}
                But $\{y_{2}\}\ne\emptyset$ and
                $\{y_{2}\}\subseteq{Y}$. Therefore, etc.
            \end{subequations}
        \end{proof}
        \begin{theorem}
            If $X$ and $Y$ are sets, if $A$ is a subset of $X$,
            and if $f:X\rightarrow{Y}$ is a function, then:
            \begin{equation}
                A\subseteq{f^{-1}}\big(f(A)\big)
            \end{equation}
        \end{theorem}
        \begin{proof}
            For if $x\in{A}$, then there is a
            $y\in{f}(A)$ such that $f(x)=y$. But then
            $x\in{f^{-1}(f(A))}$. Therefore, etc.
        \end{proof}
        \begin{theorem}
            If $X$ and $Y$ are sets, if $A_{1}$ and $A_{2}$ are
            subsets of $X$ such that $A_{1}\subseteq{A}_{2}$,
            and if $f:X\rightarrow{Y}$ is a function, then:
            \begin{equation}
                f(A_{1})\subseteq{f}(A_{2})
            \end{equation}
        \end{theorem}
        \begin{proof}
            For if $y\in{f}(A_{1})$, then there is an $x\in{A}_{1}$
            such that $f(x)=y$. But $A_{1}\subseteq{A}_{2}$, and
            therefore $x\in{A}_{2}$. But if $x\in{A}_{2}$, then
            $f(x)\in{f}(A_{2})$. Thus, $y\in{f}(A_{2})$. Therefore, etc.
        \end{proof}
        \begin{theorem}
            If $X$ and $Y$ are sets, if $B_{1}$ and $B_{2}$ are subsets of
            $Y$ such that $B_{1}\subseteq{B}_{2}$, and if $f:X\rightarrow{Y}$
            is a function, then:
            \begin{equation}
                f^{-1}(B_{1})\subseteq{f^{-1}}(B_{2})
            \end{equation}
        \end{theorem}
        \begin{proof}
            For if $x\in{f}^{-1}(B_{1})$, then there is a
            $y\in{B}_{1}$ such that $f(x)=y$. But
            $B_{1}\subseteq{B}_{2}$, and therefore $y\in{B}_{2}$.
            Thus, $x\in{f}^{-1}(B_{2})$. Therefore, etc.
        \end{proof}
        \begin{theorem}
        If $f:A\rightarrow B$, $A_1,A_2\subset A$, then $f(A_1 \cup A_2) = f(A_1)\cup f(A_2)$.
        \end{theorem}
        \begin{proof}
        $[y\in f(A_1\cup A_2)]\Rightarrow [\exists x\in A_1 \cup A_2:y=f(x)]\Rightarrow [y \in f(A_1)\cup f(A_2)]$. $[y\in f(A_1)\cup f(A_2)]\Rightarrow \big[[\exists x\in A_1] \lor [\exists x\in A_2]: y=f(x)\big]\Rightarrow [x\in A_1\cup A_2]\Rightarrow [f(x)\in f(A_1\cup A_2)]$
        \end{proof}
        \begin{theorem}
        If $f:A\rightarrow B$, $A_1,A_2\subset A$, then $f(A_1\cap A_2)\subset f(A_1)\cap f(A_2)$.
        \end{theorem}
        \begin{proof}
        $[y\in f(A_1 \cap A_2)]\Rightarrow [\exists x\in A_1 \cap A_2:y=f(x)]\Rightarrow [x\in A_1 \land x \in A_2] \Rightarrow[y \in f(A_1)\cap f(A_2)]$.
        \end{proof}
        \begin{theorem}
        If $f:A\rightarrow B$, $B_1,B_2\subset B$, then $f^{-1}(B_1\cup B_2) = f^{-1}(B_1)\cup f^{-1}(B_2)$.
        \end{theorem}
        \begin{proof}
        $[x\in B_1\cup B_2]\Rightarrow [f(x)\in B_1\cup B_2]\Rightarrow [f(x)\in B_1\lor f(x)\in B_2]\Rightarrow [x\in f^{-1}(B_1)\cup f^{-1}(B_2)]$. $[x \in f^{-1}(B_1)\cup f^{-1}(B_2)]\Rightarrow [f(x)\in B_1\lor f(x) \in B_2]\Rightarrow [f(x) \in B_1\cup B_2]\Rightarrow [x\in f^{-1}(B_1\cup B_2)]$.
        \end{proof}
        \begin{theorem}
        If $f:A\rightarrow B$, $B_1,B_2\subset B$, then $f^{-1}(B_1\cap B_2) = f^{-1}(B_1)\cap f^{-1}(B_2)$.
        \end{theorem}
        \begin{proof}
        $[x\in f^{-1}(B_1\cap B_2)]\Rightarrow [f(x) \in B_1 \cap B_2]\Rightarrow [f(x)\in B_1\land f(x) \in B_2 ]\Rightarrow [x\in f^{-1}(B_1)\cap f^{-1}(B_2)]$. $[x\in f^{-1}(B_1)\cap f^{-1}(B_2)]\Rightarrow [x\in f^{-1}(B_1)\land x\in f^{-1}(B_2)]\Rightarrow [f(x) \in B_1\land f(x) \in B_2]\Rightarrow [f(x)\in B_1\cap B_2]\Rightarrow [x\in f^{-1}(B_1\cap B_2)]$.
        \end{proof}
        \begin{theorem}
        If $f:A\rightarrow B$, $B_1 \subset B$, then $f^{-1}(B\setminus B_1) = f^{-1}(B)\setminus f^{-1}(B_1)$.
        \end{theorem}
        \begin{proof}
        $[x\in f^{-1}(B\setminus B_1)]\Leftrightarrow [f(x)\notin B_1]\Leftrightarrow [x\in f^{-1}(B)\setminus f^{-1}(B_1)]$
        \end{proof}
        If $f:A\rightarrow B$, the image of $A$ under $f$
        is often called the range (A is often called the domain).
        \begin{ldefinition}{Permutations}{Permutations}
            A permutation on a set $A$ is a bijective function
            $f:A\rightarrow{A}$.
        \end{ldefinition}
        \begin{theorem}
        If $f:A\rightarrow B$ is bijective, then $f^{-1}$ is bijective.
        \end{theorem}
        \begin{proof}
        $[f^{-1}(y_1) = f^{-1}(y_2)]\Rightarrow [\exists x\in A:[f(x) = y_1]\land [f(x)=y_2]]\Rightarrow [y_1=y_2]$. By definition, $f^{-1}$ is surjective.
        \end{proof}
        \begin{definition}
        If $f:A\rightarrow B$ and $g:B\rightarrow C$, then $g\circ f:A\rightarrow C$ is defined by the image $g(f(x)), x\in A$.
        \end{definition}
        \begin{theorem}
        If $f:A\rightarrow B$, $g:B\rightarrow C$, and $\mathcal{V}\subset C$, then $(g\circ g)^{-1}(\mathcal{V}) = f^{-1}(g^{-1}(\mathcal{V}))$.
        \end{theorem}
        \begin{proof}
        $[x\in (g\circ f)^{-1}(\mathcal{V})]\Leftrightarrow [g(f(x))\in \mathcal{V}] \Leftrightarrow [f(x)\in g^{-1}(\mathcal{V})]\Leftrightarrow [x\in f^{-1}(g^{-1}(\mathcal{V}))]$.
        \end{proof}
        \begin{theorem}
        If $f:A\rightarrow B$ is bijective, $g:B\rightarrow C$ is bijective, then $g\circ f$ is bijective.
        \end{theorem}
        \begin{proof}
        $\big[[f(A) = B]\land [g(B) = C]\big]\Rightarrow [g(f(A)) = g(B) = C]$. $[g(f(x_1))=g(f(x_2))]\Leftrightarrow [f(x_1)=f(x_2)]\Leftrightarrow [x_1=x_2]$.
        \end{proof}
        \begin{theorem}
        If $f:A\rightarrow B$ is bijective, $A_1\subset A$, and $f(A_1) = B$, then $A_1=A$.
        \end{theorem}
        \begin{proof}
        $\Big[\big[[A_1^c \ne \emptyset]\Rightarrow [f(A_1^c) \ne \emptyset]\big]\land[f(A_1)\cap f(A_1^c) = \emptyset]\Big]\Rightarrow [\exists y\in B:y\notin f(A_1)]$, a contradiction.
        \end{proof}
        \begin{lexample}{}{Image_is_Nonempty}
            Given a function $f:X\rightarrow{Y}$, and any
            non-empty subset $S\subseteq{X}$, the image
            $f(S)$ is non-empty. This is not true for the
            pre-image of a function. For let
            $f:\mathbb{R}\rightarrow\mathbb{R}$ be defined by
            $f(x)=1$ for all $x\in\mathbb{R}$. Then, for any
            subset $S\subset\mathbb{R}$ such that
            $1\notin{S}$, we have that
            $f^{\minus{1}}(S)=\emptyset$.
        \end{lexample}
        There are many examples of functions, but certain
        ones are easier to study than others. We give some
        of these special functions names.
        \begin{ldefinition}{Injective Functions}
              {Funct_Analysis_Injective_Function}
            An \gls{injective function} is a function
            $f:X\rightarrow{Y}$ such that, for all
            $x,y\in{X}$ such that $x\ne{y}$, it is true that
            $f(x)\ne{f}(y)$.
        \end{ldefinition}
        That is, an injective function is a function
        $f:X\rightarrow{Y}$ such that $f(x_{1})=f(x_{2})$
        if and only if $x_{1}=x_{2}$. Such functions are also
        called \textit{one-to-one}.
        \begin{lexample}{}{Natural_Log_is_Injective}
            Consider the natural logarithm
            $\ln:\mathbb{R}^{+}\rightarrow\mathbb{R}$. This
            is an injective function. For let
            $x,y\in\mathbb{R}^{+}$ be such that $x\ne{y}$.
            Suppose $\ln(x)=\ln(y)$. But then:
            \begin{equation}
                \ln(x)-\ln(y)=\ln\Big(\frac{x}{y}\Big)=0
            \end{equation}
            Recall the definition of the natural logarithm:
            \begin{equation}
                \ln(t)=\int_{1}^{t}\frac{1}{x}\diff{x}
            \end{equation}
            But then $\ln(t)=0$ if and only if $t=1$. Thus
            $x=y$, a contradiction. Therefore $\ln$ is an
            injective function. Not every function is
            injective, for define
            $f:\mathbb{R}\rightarrow\mathbb{R}$ by
            $f(x)=x^{2}$. Then, for all $x\in\mathbb{R}^{+}$,
            $f(\minus{x})=f(x)$, and thus $f$ cannot be an
            injective function.
        \end{lexample}
        One might think that most functions are not injective,
        and indeed for the \textit{finite} case, this is true.
        For let $A$ and $B$ be finite sets with $n$ and $m$
        elements, respectively. If $m<n$, there can't be
        any injective function. Consider the case when $n=m$.
        Then we are simply counting the number of ways to
        permute the elements of $A$. This is $n!$. On the
        other hand, the total number of functions is
        $n^{n}$. Thus, the ratio of the number of injective
        functions to the number of functions is
        $n!/n^{n}$, and this decays to zero rapidly as
        $n$ get's large. Finally, if $m>n$, then the total
        number of injective functions is
        $n!\binom{m}{n}$, where $\binom{m}{n}$ is the
        binomial coefficient. The total number of functions
        is $n^{m}$. The ratio is thus:
        \begin{equation}
            \frac{n!\binom{m}{n}}{n^{m}}=
            \frac{n!\frac{m!}{n!(m-n)!}}{n^{m}}
            =\frac{m!}{(m-n)!n^{m}}
        \end{equation}
        And again, this decays rapidly to zero and $n$ and $m$
        get large. Later, when we define infinite sets
        and the notion of Cardinality, we'll show that this
        trend continues. That is, in a sense, \textit{most}
        functions from a set $A$ to a sufficiently large set
        $B$ are not injective.
        \begin{ldefinition}{Inverse of Injective Functions}
                           {Inverse_Function}
            The inverse of an injective function $f:A\rightarrow{B}$
            is the function $f^{-1}:f(A)\rightarrow{A}$ defined by
            $f^{-1}(y)=x$, where $x\in{A}$ is the unique element
            such that $f(x)=y$.
        \end{ldefinition}
        Next, we define \textit{surjective} functions.
        \begin{ldefinition}{Surjective Functions}
              {Funct_Analysis_Surjective_Function}
            A \gls{surjective function} is a function
            $f:X\rightarrow{Y}$ such that $f(X)=Y$.
            That is, for all $y\in{Y}$, there is an
            $x\in{X}$ such that $f(x)=y$.
        \end{ldefinition}
        That is, every point $y\in{Y}$ gets mapped to by
        at least one point in $X$. It may also be true that
        many points in $X$ map to the same point in $Y$.
        The notions of surjective functions and injective
        functions are distinct, and neither implies the
        other. Surjective functions are also called
        \textit{onto}.
        \begin{ldefinition}{Bijective Functions}
              {Funct_Analysis_Bijective_Function}
            A \gls{bijective function} is a function
            that is both injective and surjective.
        \end{ldefinition}
        Sets $X$ and $Y$ such that there
        exists a bijective function $f:X\rightarrow{Y}$ are
        called \textit{equivalent}. Such sets can be said
        to have the same size. We say that $X$ is strictly
        smaller than $Y$ if there is an injective function
        $f:X\rightarrow{Y}$, but no bijective function.
        Being countable means you can write
        the elements out in a list, or a
        one-to-one correspondence with all of
        the positive integers. Many sets are countable,
        including the whole numbers, integers, rational
        numbers, and \textit{algebraic} numbers. The
        union of finitely many countable sets is also
        countable, as is the union of countably many
        countable sets.
        \begin{example}
            The set of all positive even integers is
            countable. For let $\mathbb{N}_{e}$ be the
            set of all even integers and define
            $f:\mathbb{N}\rightarrow\mathbb{N}_{e}$ be
            $f(n)=2n$ for all $n\in\mathbb{N}$. This is
            a bijection, and thus $\mathbb{N}_{e}$ is
            countable. The set of all odd positive integers
            is countable, as shown by letting
            $f(n)=2n-1$. Even though the set of even
            integers may seem ``smaller,'' than the set of
            all integers, they are equivalent. The set of
            all integers $\mathbb{Z}$ is also countable.
            For let $f:\mathbb{N}\rightarrow\mathbb{Z}$
            be defined as:
            \begin{equation}
                f(n)=
                \begin{cases}
                    \frac{1}{2}(n-1),&n\textrm{ odd}\\
                    -\frac{n}{2},&n\textrm{ even}
                \end{cases}
            \end{equation}
        \end{example}
        Any set that is infinite (Not finite) contains a
        countable subset. Thus, $\mathbb{N}$ can be
        considered as the \textit{smallest} infinite set.
        \begin{theorem}
            If $A$ is an infinite set, then there exists
            $S\subseteq{A}$ such that $S$ is countabl e.
        \end{theorem}
        \begin{proof}
            For as $A$ is infinite, for all $n\in\mathbb{N}$
            there exists a set $B\subseteq{A}$ such that
            $|B|=n$. For all $n\in\mathbb{N}$,
            define the following:
            \begin{equation}
                \mathcal{S}_{n}=\{B\subseteq{A}:|B|=n\}
            \end{equation}
            Let $\mathcal{S}$ be defined as:
            \begin{equation}
                \mathcal{S}=\{\mathcal{S}_{n}:n\in\mathbb{N}\}
            \end{equation}
            Then $\mathcal{S}$ is countable, for
            $a:\mathbb{N}\rightarrow\mathcal{S}$ defined
            by $a_{n}=\mathcal{S}_{n}$ is a bijection.
            By the axiom of choice, there is a function:
            \begin{equation}
                \alpha:\mathcal{S}\rightarrow
                \bigcup_{n=1}^{\infty}\mathcal{S}_{n}
            \end{equation}
            Such that, for all $x\in\mathcal{S}$,
            $\alpha(x)\in{x}$. But then, for all
            $x\in\mathcal{S}$, $\alpha(x)$ is a subset
            of $A$. But for all $x\in\mathcal{S}$, there
            is an $n\in\mathbb{N}$ such that
            $a_{n}=x$. Thus, let $S$ be the following:
            \begin{equation}
                S=\bigcup_{n=1}^{\infty}\alpha(a_{n})
            \end{equation}
        \end{proof}
        \begin{theorem}
            \label{thm:Funct_Countable_Union_of_Countable}
            If $A$ is a countable set such that for all
            $\mathcal{U}\in{A}$, $\mathcal{U}$ is a
            countable set, and if for all $a,b\in{A}$,
            $a\cap{b}=\emptyset$, then
            $\bigcup_{\mathcal{U}\in{A}}\mathcal{U}$
            is countable set.
        \end{theorem}
        \textit{Sketch of Proof.} The proof of
        Thm.~\ref{thm:Funct_Countable_Union_of_Countable}
        follows in the same manner
        as proving that the rationals are countable. Since
        there are countably many sets, write them out in
        a list $\mathcal{U}_{1}$, $\mathcal{U}_{2}$, and
        so on. Then write out the elements in a table as
        follows:
        \begin{table}[H]
            \captionsetup{type=table}
            \centering
            \begin{tabular}{ccccc}
                $u_{11}$&$u_{12}$&$u_{13}$
                &$u_{14}$&$\hdots$\\
                $u_{21}$&$u_{22}$&$u_{23}$
                &$u_{24}$&$\hdots$\\
                $u_{31}$&$u_{32}$&$u_{33}$
                &$u_{34}$&$\hdots$\\
                $u_{41}$&$u_{42}$&$u_{43}$
                &$u_{44}$&$\hdots$\\
                $\vdots$&$\vdots$&$\vdots$
                &$\vdots$&$\ddots$
            \end{tabular}
            \caption{Construction of a Bijection on the
                     Countable Union of Countably Infinite
                     Sets.}
            \label{table:Func_Countable_Union_of_Countable}
        \end{table}
        Where $u_{nm}$ is the $m^{th}$ element of
        $\mathcal{U}_{n}$.
        Using the \textit{diagonal argument},
        we obtain:
        \begin{table}[H]
            \captionsetup{type=table}
            \centering
            \begin{tabular}{|c|c|c|c|c|c|c|c|c|c|c|}
                \hline
                $\mathbb{N}$&1&2&3&4&5&6&7&8&9&$\hdots$\\
                \hline
                $\bigcup_{\mathcal{U}\in{A}}\mathcal{U}$&
                $u_{11}$&$u_{12}$&$u_{21}$&$u_{13}$&
                $u_{22}$&$u_{31}$&$u_{14}$&$u_{23}$&
                $u_{32}$&$\hdots$\\
                \hline
            \end{tabular}
            \caption{The Bijection Between $\mathbb{N}$ and
                     $\bigcup_{\mathcal{U}\in{A}}\mathcal{U}$}
            \label{table:Func_Bijection_on_Countable_Union}
        \end{table}
        In the absence of the requirement that
        $a\cap{b}=\emptyset$ for all pairs in $\mathcal{U}$,
        we still have that the union is, at most, countable.
        The mapping we found would be a
        \textit{surjection}, rather than a bijection.
        The union is then either finite or countable. The
        Cantor-Schr\"{o}der-Bernstein Theorem can often be
        used to help identify the size of a set. This says
        that if $A$ and $B$ are sets such that there exists
        a surjective function $f:A\rightarrow{B}$ and a
        surjective function $g:B\rightarrow{A}$, then there
        is a bijective function $h:A\rightarrow{B}$. The
        requirement that $f$ and $g$ both be surjective
        can be replaced with the requirement that they both
        be injective. This is similar to saying that if
        $\mathrm{Card}(A)\leq\mathrm{Card}(B)$ and
        $\mathrm{Card}(B)\leq\mathrm{Card}(A)$,
        then $\mathrm{Card}(A)=\mathrm{Card}(B)$. Here, $\mathrm{Card}(A)$
        denotes the \textit{cardinality} of the set $A$.
        \begin{theorem}
            $\mathbb{R}$ is uncountable.
        \end{theorem}
        \textit{Sketch of Proof.} We'll show that the unit
        interval $(0,1)$ is uncountable. Suppose not.
        Let $r_{ij}$ be the $j^{th}$ decimal of the $i^{th}$
        element in the list. We construct the real number
        $d$ as follows: If $d_{j}$ denotes the $j^{th}$
        decimal in $d$, let $d_{j}=r_{jj}+1$ if
        $r_{jj}\ne{9}$, and $d_{j}=0$ otherwise. Then
        $d\in(0,1)$, but $d$ is not on the list. For it's not
        the $n^{th}$ element, for it differs in the
        $n^{th}$ decimal place. Thus there is no bijection.
        Therefore, $(0,1)$ is uncountable. By extension,
        $\mathbb{R}$ is uncountable.
        \par\hfill\par
        \vspace{-2ex}
        For a set $X$, we often write
        $\mathcal{P}(X)$ to denote the
        \textit{power set} of $X$. This is the
        set of all subsets of $X$.
        For any set $X$ you can show that $X$ is
        strictly smaller than $\mathcal{P}(X)$.
        For example, $\mathcal{P}(\mathbb{N})$
        can be shown to be equivalent to $\mathbb{R}$.
        Since $\mathbb{N}$ is stricly smaller than
        $\mathbb{R}$, one might ask if there exists
        a set $X$ such that $\mathbb{N}$ is strictly
        smaller than $X$, but $X$ is strictly smaller
        than $\mathbb{R}$. Continuing, you can ask the
        same thing about $\mathbb{R}$ and
        $\mathcal{P}(\mathbb{R})$, and so on.
        This is called the continuum hypothesis.
        It turns out to be independent of
        the standard axioms of mathematics.
        We begin by talking about cardinality. This is the
        \textit{size} of a set. For an infinite set it
        doesn't make sense to talk about the \textit{number}
        of elements, but we can specify what it means two
        sets to have the same size.
        \begin{ldefinition}{Equivalent Sets}{Equivalent_Sets}
            Equivalent sets are set $A$ and $B$ such that
            there exists a bijection $f:A\rightarrow{B}$.
        \end{ldefinition}
        The notion of equivalent sets defines an equivalence
        relation on sets. That is, the notion is reflexive,
        symmetric, and transitive.
        \begin{theorem}
            If $A$ is a set, then $A$ is equivalent to $A$.
        \end{theorem}
        \begin{proof}
            For let $\mathrm{id}_{A}:A\rightarrow{A}$
            be the identity mapping, $\mathrm{id}_{A}(x)=x$,
            then $\mathrm{id}_{A}$ is a bijection, and thus
            $A$ is equivalent to $A$.
        \end{proof}
        \begin{theorem}
            If $A$ and $B$ are sets and if $A$ is equivalent
            to $B$, then $B$ is equivalent to $A$.
        \end{theorem}
        \begin{proof}
            For if $A$ is equivalent to $B$, then there is
            a bijection $f:A\rightarrow{B}$. But if $f$ is a
            bijection, then the inverse function
            $f^{-1}:B\rightarrow{A}$ is well-defined and is
            a bijection. Thus $B$ is equivalent to $A$.
        \end{proof}
        \begin{theorem}
            If $A$, $B$, and $C$ are sets, if $A$ is
            equivalent to $B$, and if $B$ is equivalent to
            $C$, then $A$ is equivalent to $C$.
        \end{theorem}
        \begin{proof}
            For if $A$ is equivalent to $B$, then there is
            a bijection $f:A\rightarrow{B}$. But if $B$ is
            equivalent ot $C$, then there is a bijection
            $g:B\rightarrow{C}$. But then
            $g\circ{f}:A\rightarrow{C}$ is a bijection, and
            thus $A$ and $C$ are equivalent.
        \end{proof}
        A bijection is a function that is both injective and
        surjective. Thus, two equivalent sets can be put
        into a one-to-one correspondence and can be said to
        have the same size. We then say that $A$ and $B$
        have the same cardinality. The notation is written
        as $|A|=|B|$ or $\mathrm{Card}(A)=\mathrm{Card}(B)$. Cardinality
        splits sets into one of three categories.
        \begin{ldefinition}{Finite Sets}{Finite_Sets}
            A finite set is a set $A$ such that there exists
            an $n\in\mathbb{N}$ such that there is a
            bijection $f:\mathbb{Z}_{n}\rightarrow{A}$, or
            such that $A=\emptyset$.
        \end{ldefinition}
        Sets that are not finite are called infinite. There
        are two types of infinite sets. Let $\mathbb{N}$
        denote the set of positive integers, or
        \textit{natural} numbers.
        \begin{ldefinition}{Countably Infinite Sets}
              {Countably_Infinite}
            A countably infinite set is a set $A$ such that
            is a bijection $f:\mathbb{N}\rightarrow{A}$.
        \end{ldefinition}
        Combining the notions of finite sets and countably
        infinite sets, we get the notion of
        \textit{countable} sets.
        \begin{ldefinition}{Countable Sets}
              {Countable_Sets}
            A countable set is a set $A$ such that $A$ is
            either finite or countably infinite.
        \end{ldefinition}
        Countable sets are also called \textit{listable}.
        This is because if $A$ is a countably infinite set,
        and if $a:\mathbb{N}\rightarrow{A}$ is a bijection,
        we can write $A$ as:
        \begin{equation}
            A=\{\;a_{n}\,:\,n\in\mathbb{N}\;\}
            =\{\,a_{1},\,a_{2},\,\dots,\,a_{k},\,\dots\,\}
        \end{equation}
        If $A$ is finite, and if
        $a:\mathbb{Z}_{n}\rightarrow{A}$ is a
        bijection, then we can write:
        \begin{equation}
            A=\{\;a_{n}\,:\,n\in\mathbb{Z}_{n}\;\}
             =\{\,a_{1},\,\dots,\,a_{n}\,\}
        \end{equation}
        Recall that functions $a:\mathbb{N}\rightarrow{A}$
        are called \textit{sequences}, and the image of
        $n\in\mathbb{N}$ is written $a_{n}$, rather than
        $a(n)$.
        \begin{lexample}{}{Countable_Sets}
            There are many commonly discussed sets that are
            countably infinite. $\mathbb{N}$ is a trivial
            such example, but also $\mathbb{N}_{e}$ and
            $\mathbb{N}_{o}$, the sets of even and odd
            positive integers, respectively. For consider as
            bijections the following functions:
            \par
            \begin{subequations}
                \begin{minipage}[b]{0.49\textwidth}
                    \centering
                    \begin{equation}
                        f_{e}(n)=2n
                    \end{equation}
                \end{minipage}
                \hfill
                \begin{minipage}[b]{0.49\textwidth}
                    \centering
                    \begin{equation}
                        f_{0}(n)=2n-1
                    \end{equation}
                \end{minipage}
                \par\vspace{2.5ex}
                The set of all integers, $\mathbb{Z}$ is also
                countable, as shown in
                Fig.~\ref{fig:Bijection_N_and_Z}.
                One bijection is:
                \begin{equation}
                    f(n)=
                    \begin{cases}
                        \frac{n}{2},&n\mod{2}=0\\
                        \frac{1-n}{2},&n\mod{2}=1
                    \end{cases}
                \end{equation}
            \end{subequations}
            Any subset of $\mathbb{Z}$ is countable,
            and this is true of any countable set.
        \end{lexample}
        \begin{figure}[H]
            \centering
            \captionsetup{type=figure}
            %--------------------------------Dependencies----------------------------------%
%   amssymb                                                                    %
%   tikz                                                                       %
%       arrows.meta                                                            %
%-------------------------------Main Document----------------------------------%
\begin{tikzpicture}[%
    >=latex
]
    \draw[<->, thick] (-5, 0) to (5, 0) node[below] {$\mathbb{Z}$};
    \foreach\x in {-4, -3, -2, -1, 0, 1, 2, 3, 4}{%
        \draw (\x, -0.1) to (\x, 0.1);
        \node at (\x, -0.4) {\x};
    }
    \draw[->] (0, 0.2) arc(180:0:0.5 and 0.4);
    \draw[->] (1, -0.6) arc(0:-180:1 and 0.6);
    \draw[->] (-1, 0.2) arc(180:0:1.5 and 0.7);
    \draw[->] (2, -0.6) arc(0:-180:2 and 0.8);
    \draw[->] (-2, 0.2) arc(180:0:2.5 and 0.9);
    \draw[->] (3, -0.6) arc(0:-180:3 and 1);
    \draw[->] (-3, 0.2) arc(180:0:3.5 and 1.1);
    \draw[->] (4, -0.6) arc(0:-180:4 and 1.2);
\end{tikzpicture}
            \caption{Diagram of a Bijection Between
                     $\mathbb{N}$ and $\mathbb{Z}$.}
            \label{fig:Bijection_N_and_Z}
        \end{figure}
        One of the standard results about countable sets is
        that their subsets are also countable. This theorem
        relies, in a very subtle way, the use of the axiom
        of choice. There are a few stepping stones to get
        there. We will accept the various
        Cantor-Schr\"{o}eder-Bernstein theorems, which say
        the following:
        \begin{ltheorem}
              {First Cantor-Schr\"{o}eder-Bernstein Theorem}
              {First_Cantor_Schroeder_Bernstein}
            If $A$ and $B$ are sets such that there is an injective
            function $f:A\rightarrow{B}$ and an injective function
            $g:B\rightarrow{A}$, then there is a bijective function
            $h:A\rightarrow{B}$.
        \end{ltheorem}
        \begin{ltheorem}
              {Second Cantor-Schr\"{o}eder-Bernstein Theorem}
              {Second_Cantor_Schroeder_Bernstein}
            If $A$ and $B$ are sets such that there is a surjective
            function $f:A\rightarrow{B}$ and a surjective function
            $g:B\rightarrow{A}$, then there is a bijective function
            $h:A\rightarrow{B}$.
        \end{ltheorem}
        \par\hfill\par
        Using cardinalities, this says that if
        $\mathrm{Card}(A)\leq\mathrm{Card}(B)$ and $\mathrm{Card}(B)\leq\mathrm{Card}(A)$, then
        $\mathrm{Card}(A)=\mathrm{Card}(B)$. With this notation it becomes more
        intuitive. We will use this to prove that various sets are
        countable. Many sets that appear to be larger than $\mathbb{N}$
        can shown to to be the same size as $\mathbb{N}$ by finding
        a simple injective function, without finding an explicit
        bijection.
        \begin{ltheorem}
              {Third Cantor-Schr\"{o}eder-Bernstein Theorem}
              {Third_Cantor_Schroeder_Bernstein}
            If $A$, $B$, and $C$ are sets such that
            $A\subseteq{B}\subseteq{C}$, and if $A$ and $C$ are equivalent
            sets, then $B$ and $C$ are equivalent sets.
        \end{ltheorem}
        \par\hfill\par
        This says that if $\mathrm{Card}(A)\leq\mathrm{Card}(B)\leq\mathrm{Card}(C)$,
        and if $\mathrm{Card}(A)=\mathrm{Card}(C)$, then $\mathrm{Card}(B)=\mathrm{Card}(C)$.
        \begin{theorem}
            \label{thm:Measure_Theory_NxN_Is_Countable}
            $\mathbb{N}\times\mathbb{N}$ is countably infinite.
        \end{theorem}
        \begin{proof}
            There is a trivial injection
            $f:\mathbb{N}\rightarrow\mathbb{N}\times\mathbb{N}$
            defined by:
            \begin{equation}
                f(n)=(n,0)
            \end{equation}
            There is also an injection
            $g:\mathbb{N}\times\mathbb{N}\rightarrow\mathbb{N}$
            defined by:
            \begin{equation}
                g(n.m)=2^{n}3^{m}
            \end{equation}
            Since 2 and 3 are co-prime, if
            $g(n_{1},m_{1})=g(n_{2},m_{2})$, then
            $(n_{1},m_{1})=(n_{2},m_{2})$. Thus, $g$ is an injection.
            By the Cantor-Schr\"{o}eder-Bernstein Theorem, there is a
            bijection $h:\mathbb{N}\rightarrow\mathbb{N}\times\mathbb{N}$.
        \end{proof}
        One can intuitively see that the set of all positive
        rational numbers $\mathbb{Q}^{+}$ is countable by examining
        the zig-zag pattern shown in
        Fig.~\ref{fig:Bijection_N_and_Q_Plus}.
        Thm.~\ref{thm:Measure_Theory_NxN_Is_Countable} also
        shows this in a more rigorous way that. We can create
        a one-to-one correspondence with
        $\mathbb{N}\times\mathbb{N}$ by mapping
        $pq^{\minus{1}}\mapsto(p,q)$. Thus $\mathbb{Q}^{+}$
        and $\mathbb{N}\times\mathbb{N}$ are equivalent sets.
        But $\mathbb{N}\times\mathbb{N}$ and $\mathbb{N}$
        are equivalent sets, and therefore $\mathbb{Q}^{+}$
        is countable.
        Thm.~\ref{thm:Measure_Theory_NxN_Is_Countable} can also be used
        to show that the countable union of countable sets is also
        countable.
        \begin{ltheorem}{Equivalence of Countable Sets}
              {Countable_iff_exists_inj_to_N}
            A set $A$ is countable if and only if there is an injective
            function $f:A\rightarrow\mathbb{N}$.
        \end{ltheorem}
        Thm.~\ref{thm:Countable_iff_exists_inj_to_N} seems
        intuitively obvious, the injective function is
        simply the listing function. For a finite set, this
        is precisely how one constructs such an injection.
        For an infinite set $A$, this is equivalent to
        showing that any infinite subset of $\mathbb{N}$ is
        equivalent to $\mathbb{N}$. The standard proof
        using \textit{induction}, but actually has the axiom
        of choice underlying it.
        \begin{theorem}
            If $\mathcal{A}$ is a countably infinite set
            such that, for all $A\in\mathcal{A}$, $A$ is
            a non-empty countable set, then the set:
            \begin{equation}
                S=\bigcup_{A\in\mathcal{A}}A
            \end{equation}
            Is a countable set.
        \end{theorem}
        \begin{proof}
            If $\mathcal{F}$ is finite, then we are done. Suppose not.
            Let $A:\mathbb{N}\rightarrow\mathcal{A}$ be a bijection,
            and define:
            \begin{equation}
                S=\bigcup_{n\in\mathbb{N}}A_{n}
            \end{equation}
            Also, let:
            \begin{equation}
                \mathcal{F}_{n}
                =\{f:A_{n}\rightarrow\mathbb{N}:
                    f\textrm{ is injective}\}
            \end{equation}
            Since, for all $n\in\mathbb{N}$, $A_{n}$ is
            non-empty and countable, $\mathcal{F}_{n}$
            is non-empty. Let:
            \begin{equation}
                \mathcal{F}
                =\bigcup_{n\in\mathbb{N}}\mathcal{F}_{n}
            \end{equation}
            Thus, by the axiom of choice, there is a function
            $F:\mathbb{N}\rightarrow\mathcal{F}$ such that, for all
            $n\in\mathbb{N}$, $F_{n}\in\mathcal{F}_{n}$. For
            $x\in{S}$, let:
            \begin{equation}
                \varphi_{x}
                =\inf\{n\in\mathbb{N}:x\in{A}_{n}\}
            \end{equation}
            By the well-ordering of $\mathbb{N}$, for all
            $x\in{S}$, $\varphi_{x}$ is well defined. Let
            $\phi:S\rightarrow\mathbb{N}\times\mathbb{N}$
            be defined by:
            \begin{equation}
                \phi(x)
                =\big(\varphi_{x},F_{\varphi_{x}}(x)\big)
            \end{equation}
            Then $\phi$ is an injection. For if
            $\big(\varphi_{x},F_{\varphi_{x}}(x)\big)=%
             \big(\varphi_{y},F_{\varphi_{x}}(y)\big)$, then
            $\varphi_{x}=\varphi_{y}$, and thus
            $F_{\varphi(x)}(x)=F_{\varphi(x)}(y)$. But
            $F_{\varphi_{x}}$ is an injection, and
            thus $x=y$. Therefore $\phi$ is an injection.
            But $\mathbb{N}\times\mathbb{N}$ and $\mathbb{N}$
            are equivalent sets, and thus there's an
            injection $f:\mathbb{N}\times\mathbb{N}$. And
            the composition of injective functions is again
            injective, and thus
            $\phi\circ{f}:S\rightarrow\mathbb{N}$ is an
            injective function. But by
            Thm.~\ref{thm:Countable_iff_exists_inj_to_N},
            if there is an injective function
            $f:S\rightarrow\mathbb{N}$, then $S$ is
            countable. Therefore, etc.
        \end{proof}
        \begin{theorem}
            If $X$ is infinite, then there exists a
            countably infinite set $A\subseteq{X}$.
        \end{theorem}
        \begin{proof}
            If $A$ is finite, then we are done. Suppose not.
            For $n\in\mathbb{N}$, let:
            \begin{equation}
                A_{n}
                =\{g:\mathbb{Z}_{n}\rightarrow{A}:f\textrm{ is inective}\}
            \end{equation}
            Also, define:
            \begin{equation}
                \mathcal{F}=\bigcup_{n\in\mathbb{N}}A_{n}
            \end{equation}
            But by the axiom of choice, there is a function
            $f:\mathbb{N}\rightarrow\mathcal{F}$ such that
            $f_{n}\in{A}_{n}$. But then, for all
            $n\in\mathbb{N}$, the range of $f_{n}$ is finite.
            \begin{equation}
                A=\bigcup_{n\in\mathbb{N}}f_{n}
                    \Big(\mathbb{Z}_{n}\Big)
            \end{equation}
            Then $A\subseteq{X}$ is countably infinite.
        \end{proof}
        The set of rational numbers, $\mathbb{Q}$, is also
        countable. We may intuitively think of $\mathbb{N}$
        as being smaller than $\mathbb{Q}$, since there are
        simple \textit{surjections} that can be constructed
        from $\mathbb{Q}$ to $\mathbb{N}$. There is also a
        surjection from $\mathbb{N}$ onto $\mathbb{Q}^{+}$,
        as is shown in Fig.~\ref{fig:Bijection_N_and_Q_Plus}.
        To construct such a surjection, write out all of the
        positive rational numbers in a grid so that $a_{nm}$
        is the number $n/m$. Then zig-zag along the diagonals
        to construct the function. Thus there is a surjection
        $f:\mathbb{Q}^{+}\rightarrow\mathbb{N}$
        and a surjection
        $g:\mathbb{N}\rightarrow\mathbb{Q}^{+}$.
        \begin{figure}[H]
            \centering
            \captionsetup{type=figure}
            \resizebox{0.7\textwidth}{!}{%
                %--------------------------------Dependencies----------------------------------%
%   tikz                                                                       %
%       arrows.meta                                                            %
%-------------------------------Main Document----------------------------------%
\begin{tikzpicture}[%
    >=latex
]
    \foreach\y in {1, 2, 3, 4, 5, 6}{%
        \foreach\x in {1, 2, 3, 4, 5, 6}{%
            \node (\x\y) at (\x, 7-\y) {$\frac{\x}{\y}$};
        }
    }
    \foreach\x in {1, 2, 3, 4, 5, 6}{%
        \node at (7, \x) {$\cdots$};
        \node at (\x, 0) {$\vdots$};
    }
    \node at (7, 0) {$\ddots$};
    \draw[->] (11) to (12);
    \draw[->] (12) to (21);
    \draw[->] (21) to (31);
    \draw[->] (31) to (22);
    \draw[->] (22) to (13);
    \draw[->] (13) to (14);
    \draw[->] (14) to (23);
    \draw[->] (23) to (32);
    \draw[->] (32) to (41);
    \draw[->] (41) to (51);
    \draw[->] (51) to (42);
    \draw[->] (42) to (33);
    \draw[->] (33) to (24);
    \draw[->] (24) to (15);
    \draw[->] (15) to (16);
    \draw[->] (16) to (25);
    \draw[->] (25) to (34);
    \draw[->] (34) to (43);
    \draw[->] (43) to (52);
    \draw[->] (52) to (61);
    \draw[->] (61) to (62);
    \draw[->] (62) to (53);
    \draw[->] (53) to (44);
    \draw[->] (44) to (35);
    \draw[->] (35) to (26);
    \draw[->] (36) to (45);
    \draw[->] (45) to (54);
    \draw[->] (54) to (63);
    \draw[->] (64) to (55);
    \draw[->] (55) to (46);
    \draw[->] (56) to (65);
\end{tikzpicture}
            }
            \caption{Diagram of a Surjection from
                     $\mathbb{N}$ onto $\mathbb{Q}^{+}$.}
            \label{fig:Bijection_N_and_Q_Plus}
        \end{figure}
        We can modify Fig.~\ref{fig:Bijection_N_and_Q_Plus}
        slightly to create a surjection between $\mathbb{N}$
        and $\mathbb{Q}$, see
        Fig.~\ref{fig:Bijection_N_and_Q}.
        It is important to note that this bijection will not
        preserve the order of the rational numbers. The
        bijection will have to jump around back and forth.
        Any such bijection will be forced to do this, as the
        rationals are everywhere dense on $\mathbb{R}$. Any
        monotonic sequence of $\mathbb{Q}$ cannot possibly
        be a bijection.
        \begin{theorem}
            If $A$ is a countably infinite set and
            $B\subseteq{A}$, then $B$ is countable.
        \end{theorem}
        \begin{proof}
            As $A$ is countably infinite, there is a bijection
            $a:\mathbb{N}\rightarrow{A}$. Define:
            \begin{equation}
                K=\{n\in\mathbb{N}:a_{n}\in{B}\}
            \end{equation}
            As $B\subseteq{A}$,
            this set contains a subsequence of points in
            $\mathbb{N}$ that get mapped into $B$. If $K$ is finite,
            then $B$ is finite, and if not then $K$ is countably
            infinite, and thus $B$ is countably infinite.
        \end{proof}
        \begin{figure}[H]
            \centering
            \captionsetup{type=figure}
            \resizebox{\textwidth}{!}{%
                %--------------------------------Dependencies----------------------------------%
%   amsmath                                                                    %
%   tikz                                                                       %
%       arrows.meta                                                            %
%-------------------------------Main Document----------------------------------%
\begin{tikzpicture}[%
    >=latex
]
    \foreach\y in {1, 2, 3, 4}{%
        \foreach\x in {-4, -3, -2, -1, 0, 1, 2, 3, 4}{%
            \node (\x\y) at (\x, 7-\y) {$\frac{\x}{\y}$};
        }
    }
    \foreach\x in {-4, -3, -2, -1, 0, 1, 2, 3, 4}{%
        \node at (\x, 2) {$\vdots$};
    }
    \foreach\y in {3, 4, 5, 6}{%
        \node at (5, \y) {$\cdots$};
        \node at (-5, \y) {$\cdots$};
    }
    \node at (5, 2) {$\ddots$};
    \node at (-5, 2) {$\reflectbox{\ensuremath{\ddots}}$};
    \draw[->] (01) to (11);
    \draw[->] (11) to (12);
    \draw[->] (12) to (02);
    \draw[->] (02) to (-12);
    \draw[->] (-12) to (-11);
    \draw[->] (-1, 6.3) to (-1, 6.6)
                        to (2, 6.6)
                        to (2, 6.3);
    \draw[->] (21) to (22);
    \draw[->] (22) to (23);
    \draw[->] (23) to (13);
    \draw[->] (13) to (03);
    \draw[->] (03) to (-13);
    \draw[->] (-13) to (-23);
    \draw[->] (-23) to (-22);
    \draw[->] (-22) to (-21);
    \draw[->] (-2, 6.3) to (-2, 6.8)
                        to (3, 6.8)
                        to (3, 6.3);
    \draw[->] (31) to (32);
    \draw[->] (32) to (33);
    \draw[->] (33) to (34);
    \draw[->] (34) to (24);
    \draw[->] (24) to (14);
    \draw[->] (14) to (04);
    \draw[->] (04) to (-14);
    \draw[->] (-14) to (-24);
    \draw[->] (-24) to (-34);
    \draw[->] (-34) to (-33);
    \draw[->] (-33) to (-32);
    \draw[->] (-32) to (-31);
    \draw[->] (-3, 6.3) to (-3, 7)
                        to (4, 7)
                        to (4, 6.3);
    \draw[->] (41) to (42);
    \draw[->] (42) to (43);
    \draw[->] (43) to (44);
    \draw[->] (44) to (4, 2.3);
    \draw[->] (-4, 2.3) to (-44);
    \draw[->] (-44) to (-43);
    \draw[->] (-43) to (-42);
    \draw[->] (-42) to (-41);
\end{tikzpicture}
            }
            \caption{Diagram of a Surjection from
                     $\mathbb{N}$ onto $\mathbb{Q}$.}
            \label{fig:Bijection_N_and_Q}
        \end{figure}
        \begin{theorem}
            If $A$ is an infinite set, then there exists a
            countable subset $B\subseteq{A}$.
        \end{theorem}
        \begin{proof}
            If $A$ is infinite then there is an
            $a_{1}\in{A}$. But, as $A$ is infinite,
            $A\setminus\{a_{1}\}$ is infinite, and there
            is an $a_{2}\in{A}\setminus\{a_{1}\}$. Continuing
            we obtain a sequence of distinct elements in $A$.
            Let $B=\{a_{n}:n\in\mathbb{N}\}$. Then
            $B\subseteq{A}$, and $B$ is countable.
        \end{proof}
        \begin{lexample}{}{Disjoint_Union_of_Intervals_is_Countable}
            Suppose we have a collection of disjoint intervals
            of $\mathbb{R}$. This collection is either finite
            or countable. For in every interval, choose a
            rational number $q_{n}$. Let
            $A=\{q_{1},q_{2},\hdots\}$. Then
            $A\subseteq\mathbb{Q}$, and thus $A$ is either
            finite or countable. But this is also an enumeration
            of the intervals in the collection, and thus the
            collection is either finite or countable.
        \end{lexample}
        Given a countable collection of sets
        $A=\{\mathcal{A}_{1},\mathcal{A}_{2},\hdots\}$ such
        that, for all $n\in\mathbb{N}$, $\mathcal{A}_{n}$ is
        also a countable set, then the union is countable. That is:
        \begin{equation}
            B=\bigcup_{n=1}^{\infty}\mathcal{A}_{n}
        \end{equation}
        is a countable set. The proof of this is a mimicry of
        the proof of the countability of $\mathbb{Q}$. Not
        every set is either finite or countable. The real numbers,
        $\mathbb{R}$, is an example of an \textit{uncountable}
        set. First, some notes on the power set of a set.
        \begin{lexample}{}{Basic_Power_Set}
            \begin{subequations}
                If $\Omega=\{1,2\}$, then the power set is:
                \begin{equation}
                    \mathcal{P}(\Omega)=
                    \big\{\emptyset,\{1\},\{2\},\{1,2\}\big\}
                \end{equation}
                We must consider the empty set, since for any set
                $A$, $\emptyset\subseteq{A}$. As another example,
                let $\Omega=\{1,2,3\}$. Then:
                \begin{equation}
                    \mathcal{P}(\Omega)=
                    \big\{\emptyset,\{1\},\{2\},\{3\},\{1,2\},
                      \{1,3\},\{2,3\},\{1,2,3\}\big\}
                \end{equation}
                We see that, in the first example, a set with
                2 elements has a power set with 4 elements. In the
                second example we see that a set with 3 elements has
                a power set with 8 elements. This pattern continues
                for finite sets. If $A$ has $n$ elements, then
                $\mathcal{P}(A)$ has $2^{n}$ elements. If
                $A$ is an infinite set, then $\mathcal{P}(A)$ is
                uncountable. Indeed:
                \begin{equation}
                    \mathrm{Card}\big(\mathcal{P}(\mathbb{N})\big)=
                    \mathrm{Card}(\mathbb{R})
                \end{equation}
                We can show this by using the binary representation
                of real numbers. We construct a bijection as
                follows: If $A\subseteq\mathbb{N}$, then
                let $r_{A}=0.n_{1}n_{2}\hdots$ where:
                \begin{equation}
                    n_{i}=
                    \begin{cases}
                        0,&i\notin{A}\\
                        1,&i\in{A}
                    \end{cases}
                \end{equation}
                The function
                $f:\mathcal{P}(\mathbb{N})\rightarrow[0,1]$
                defined by $f(A)=r_{A}$ is thus a bijection.
                That is, every element of $[0,1]$ gets mapped to in
                a one-to-one manner. The potentially tricky numbers are
                0 and 1, but $f(\emptyset)=0$, and $f(\mathbb{N})=1$.
                Thus $\mathcal{P}(\mathbb{N})$ and $[0,1]$ are of the
                same cardinality. But $(0,1)$ and $\mathbb{R}$
                are of the same cardinality. To see this, consider
                the graph of the function
                $g:(0,1)\rightarrow\mathbb{R}$ defined as:
                \begin{equation}
                    g(x)=\frac{2x-1}{x(1-x)}
                \end{equation}
                This is a bijection between the unit interval
                $(0,1)$ and $\mathbb{R}$. One can also use the
                \textit{stereographic projection} to show this.
                But also $[0,1]$ and $(0,1)$ have the same cardinality.
                For this, consider the following function:
                \begin{equation}
                    f(x)=
                    \begin{cases}
                        \frac{1}{2},&x=0\\
                        \frac{1}{2^{n+2}},&x=\frac{1}{2^{n}}\\
                        x,&\textrm{Otherwise}
                    \end{cases}
                \end{equation}
                A graph of this is shown in
                Fig.~\ref{fig:Measure_Theory_Bijection_Closed_I_to_Open}.
                Therefore, $\mathbb{R}$ and
                $\mathcal{P}(\mathbb{N})$ have the same cardinality.
                This can then be used to show that $\mathbb{R}$ is
                uncountable.
            \end{subequations}
        \end{lexample}
        \begin{figure}[H]
            \centering
            \captionsetup{type=figure}
            \documentclass[crop,class=article]{standalone}
%----------------------------Preamble-------------------------------%
\usepackage{tikz}                       % Drawing/graphing tools.
\usetikzlibrary{arrows.meta}            % Latex and Stealth arrows.
%--------------------------Main Document----------------------------%
\begin{document}
    \begin{tikzpicture}[>=Latex, scale=2]
        \draw[->] (-0.15in, 0) to (1.1in, 0) node[above] {$x$};
        \draw[->] (0, -0.15in) to (0, 1.1in) node[right] {$y$};
        \draw (0, 0) to (1in, 1in);
        \draw[fill=black, draw=black] (0, 0.5in) circle (0.3mm);
        \foreach\x in{1in, 0.5in, 0.25in, 0.125in, 0.0625in, 0.03125in}{
            \draw[fill=white, draw=black] (\x, \x) circle (0.3mm);
            \draw[fill=black, draw=black] (\x, 0.25*\x) circle (0.3mm);
        }
    \end{tikzpicture}
\end{document}
            \caption{Bijection from $[0,1]$ to $(0,1)$.}
            \label{fig:Measure_Theory_Bijection_Closed_I_to_Open}
        \end{figure}
        The power set of any set is strictly larger than the
        original set. If $\Omega$ is finite with $n$ elements, it
        can be shown that $\mathcal{P}(\Omega)$ has $2^{n}$
        elements. For infinite sets, there is a trivial surjection
        from $\mathcal{P}(\Omega)$ onto $\Omega$: for any element
        $x$, let $f(\{x\})=x$. This shows that
        $\mathrm{Card}(\Omega)\leq\mathrm{Card}(\mathcal{P}(\Omega))$. We now show
        that the inequality is strict.
        \begin{theorem}
            If $\Omega$ is a set, then there is no bijection
            $f:\Omega\rightarrow\mathcal{P}(\Omega)$
        \end{theorem}
        \begin{proof}
            For suppose not, and let
            $f:\Omega\rightarrow\mathcal{P}(\Omega)$ be such a
            bijection. Define:
            \begin{equation}
                A=\{x\in\Omega:x\in{f}(x)\}
            \end{equation}
            Then $A\subseteq\Omega$, and thus
            $A\in\mathcal{P}(\Omega)$. But then the complement of
            $A$ is also an element of $\mathcal{P}(\Omega)$. But
            $f$ is a bijection and thus there is an $x\in\Omega$
            such that $f(x)=A^{C}$. If $x\in{f}(x)$, then
            $x\in{A}$, a contradiction as $f(x)=A^{C}$, and thus
            $x\in{A}^{C}$ as well. Therefore $x\notin{f}(x)$. But
            then $x\in{A}^{C}$. But, from the definition of $A$,
            since $x\in{A}^{C}$ and $f(x)=A^{C}$, $x\in{f}(x)$
            and thus $x\in{A}$, a contradiction. Thus there is no
            $x$ such that $f(x)=A^{C}$. Therefore, $f$ is not a
            bijection.
        \end{proof}
        From this we conclude that $\mathcal{P}(\mathbb{N})$
        is an uncountable infinite set. But since $\mathbb{R}$
        and $\mathcal{P}(\mathbb{N})$ have the same cardinality,
        $\mathbb{R}$ is also uncountable.
        If a set $A$ has the same cardinality as $\mathbb{R}$,
        we say that $A$ has the cardinality of the continuum.
        \begin{lexample}{}{Bijection_from_closed_to_open_interval}
            There is a bijection between the open unit
            square $(0,1)\times(0,1)$ and the open unit interval
            $(0,1)$. For an element $(x,y)\in(0,1)\times(0,1)$,
            let $z\in(0,1)$ be defined as
            $z=0.x_{1}y_{1}x_{2}y_{2}x_{3}y_{3}\dots$ This is
            a bijection, for all $(x,y)$ in the square there is
            a corresponding $z\in(0,1)$, and for all
            $z\in(0,1)$ there is a corresponding element of
            $(0,1)\times(0,1)$. We can say that $(x,y)$ can
            be coded into $z$, and $z$ can be decoded into
            $(x,y)$. Hence, $(0,1)\times(0,1)$ has the cardinality
            of the continuum. By stereographic projection and induction
            we obtain:
            \par\hfill\par
            \begin{subequations}
                \begin{minipage}[b]{0.49\textwidth}
                    \centering
                    \begin{equation}
                        \mathrm{Card}(\mathbb{R}^{2})=\mathrm{Card}(\mathbb{R})
                    \end{equation}
                \end{minipage}
                \hfill
                \begin{minipage}[b]{0.49\textwidth}
                    \centering
                    \begin{equation}
                        \mathrm{Card}(\mathbb{R}^{n})=\mathrm{Card}(\mathbb{R})
                    \end{equation}
                \end{minipage}
                \par
            \end{subequations}
        \end{lexample}
        \begin{lexample}{}{Space_of_Sequences}
            Consider the set of all real-valued sequences. We've seen
            that any real number can be represented as a function
            $f:\mathbb{N}\rightarrow\{0,1\}$. A real-valued sequence
            is a function $a:\mathbb{N}\rightarrow\mathbb{R}$, and
            thus the set of real-valued sequences can be seen as the
            set of functions whose domain is $\mathbb{N}$ and whose
            range is the set of all functions
            $f:\mathbb{N}\rightarrow\{0,1\}$. So given a sequence
            $a$, the image of $a_{n}$, for $n\in\mathbb{N}$, is a
            function $f_{n}:\mathbb{N}\rightarrow\{0,1\}$. Therefore
            the set of all real-valued sequences can be represented
            as the set of all functions
            $g:\mathbb{N}\times\mathbb{N}\rightarrow\{0,1\}$, where
            $g(n,m)=f_{n}(m)$. But $\mathbb{N}\times\mathbb{N}$ is
            countable, and thus the set of all functions of the form
            $g:\mathbb{N}\times\mathbb{N}\rightarrow\{0,1\}$ has the
            same cardinality as the set of all functions of the form
            $f:\mathbb{N}\rightarrow\{0,1\}$. But this has the
            cardinality of the continuum. Therefore, the set of all
            real-valued sequences has the cardinality of the continuum.
        \end{lexample}

    \renewcommand{\PATH}{\OLDPATH}
\endgroup