\section{Hilbert Systems}
    In the set theory we will be working with there are a few words and symbols
    that are left undefined. As stated, this is unavoidable since defining
    everything would be circular, and we try to use the fewest number of
    \textit{primitive}. The main undefined symbol in set theory is that of
    \textit{containment} (\gls{containmentsymb})
    (see Not.~\ref{not:Element_Notation}), a type of
    \textit{\gls{predicate}}\index{Predicate} of the form \textit{x is in A}.
    Other common symbols such as subset (\gls{subseteq}) and equality
    (\gls{equalsymb}) are then defined in terms of this. In a similar manner
    there are other commonly used symbols in mathematical logic such as
    \textit{disjunction} (\gls{disjunctionsymb}), \textit{conjunction}
    (\gls{conjunctionsymb}), and \textit{equivalence} (\gls{equivalencesymb})
    that we need not accept as primitives, but rather can define in terms of
    implication (\gls{implicationsymb}) and negation (\gls{negationsymb}). We
    start with conjunction, which gives meaning the logical term \textit{and}.
    \subsection{Connectives}
        The symbol $\land$ is used to represent the word \textit{and} in a
        mathematical way. 
        \begin{fdefinition}{Conjunction}{Conjunction}
            The conjunction of propositions $P$ and $Q$ is the statement $P$ and
            $Q$ defined by the formula:\index{Conjunction}
            \begin{equation*}
                P\land{Q}\equiv\neg\big(P\Rightarrow\neg{Q}\big)
            \end{equation*}
        \end{fdefinition}
        Before justifying this definition, we wish to get an idea as to what
        \textit{and} should mean. Given two propositions $P$ and $Q$, $P$ and
        $Q$ should be considered if and only if both $P$ is true and $Q$ is
        true. That is, both are true simultaneously.
        \par\hfill\par
    We have relied on the word \textit{statement} being already defined, and
    similarly for the words \textit{parameter} or \textit{variable}. For most
    this is not an issue, but it may irk others. From our undefined symbol $\in$
    we build new symbols by expressing them in terms of a
    \textit{formula}\index{Formula}, which is simply a finite
    sequence of symbols. Here the word \textit{sequence} is meant to imply that
    the \textit{order} in which we combine these symbols is important and that
    rearranging said order may create a different inequivalent formula. We build
    formulas by defining a few symbols that stand as placeholders for standard
    words in English. There are four symbols, called
    \textit{\glspl{connective}}\index{Connective (Logic)}, that we use.
    \begin{align*}
        a\land{b}\quad
        &\textrm{True if and only if }a
        \textrm{ is true and }b\textrm{ is true}
        \tag{Conjunction}\\
        a\lor{b}\quad
        &\textrm{True if and only if }a
        \textrm{ is true or }b\textrm{ is true, or both}
        \tag{Disjunction}
    \end{align*}
    From this we see that we have introduced 6 new words that are undefined but
    require comment. The words are \textit{and, or, if, then, true}, and
    \textit{false}. There are other symbols we could adopt, such as
    \textit{equivalence}:
    \begin{equation*}
        a\Leftrightarrow{b}
    \end{equation*}
    But from how we shall define these notions, this new symbol is equivalent to
    a combination of the previous ones:
    \begin{equation*}
        \Big(\big(a\Leftrightarrow{b}\big)\Rightarrow
             \big((a\Rightarrow{b})\land(b\Rightarrow{a})\big)\Big)
        \land\Big(\big((a\Rightarrow{b})\land(b\Rightarrow{A})\big)
            \Rightarrow\big(a\Leftrightarrow{b}\big)\Big)
    \end{equation*}
    That is, $a\Leftrightarrow{b}$ if and only if $a$ is true if and only if $b$
    is true. Similarly, we could define \textit{does not imply}:
    \begin{equation*}
        a\not\Rightarrow{b}
    \end{equation*}
    But this is the same as:
    \begin{equation*}
        a\not\Rightarrow{b}\Longleftrightarrow
        \neg(a\Rightarrow{b})
        \Longleftrightarrow
        a\land\neg{b}
    \end{equation*}
    The words true and false are assumed to be well defined. They are also
    assumed to be opposites of each other (which we will define in terms of
    negation). We will use truth tables\index{Truth Table} to define what
    various connectives mean when it is known that certain propositions are true
    or false. In such tables the symbol 0 represents that a proposition is false
    and the symbol 1 represents truth.
    \par\hfill\par
    The other four words can be ambiguous in their everyday usage which we
    cannot allow for in mathematics. As such we must specify what we mean when
    we use these words and rid of any such ambiguity.
    \subsection{Conjunction}
        The conjunction connective (\gls{conjunctionsymb}) is used to denote the
        word \textit{and}. Given two propositions $P$ and $Q$, $P\land{Q}$ is a true
        statement if and only if both $P$ and $Q$ are true. That is, we
        associate to $\land$ the following truth table:
        \begin{table}[H]
            \centering
            \captionsetup{type=table}
            \begin{tabular}{ccc}
                $P$&$Q$&$P\land{Q}$\\
                \hline
                0&0&0\\
                0&1&0\\
                1&0&0\\
                1&1&1
            \end{tabular}
            \caption{Truth Table for Conjunction}
            \label{tab:Truth_Table_for_Conjunction}
        \end{table}
        There are several \textit{axioms} of conjunctions that are intuitively
        obvious, but must be stated since their use is wide spread.
        \begin{faxiom}{Axioms of Conjunction}{Axioms_of_Conjunction}
            If $P$ and $Q$ are propositions, then the following are true:
            \begin{align}
                P\land{Q}&\Longleftrightarrow{Q}\land{P}
                \tag{Commutativity of Conjunction}\\
                P\land\textrm{True}&\Longleftrightarrow\textrm{P}
                \tag{Identity of Conjunction}
            \end{align}
        \end{faxiom}
    \subsection{Disjunction}
        The disjunction connective (\gls{disjunctionsymb}) represents the word
        \textit{or}. Given two propositions $P$ and $Q$, $P\lor{Q}$ is true if
        and only if $P$ is true, or $Q$ is true, or both $P$ and $Q$ are true.
        There is an unfortunate ambiguity in English as to whether $P$ or $Q$
        means $P$ is true or $Q$ is true, but not both, or whether it means
        $P$ is true or $Q$ is true, or \textit{both} are true. The convention is
        to adopt the latter definition. That is, $P\lor{Q}$ has the following
        truth table:
        \begin{table}[H]
            \centering
            \captionsetup{type=table}
            \begin{tabular}{ccc}
                $P$&$Q$&$P\lor{Q}$\\
                \hline
                0&0&0\\
                0&1&1\\
                1&0&1\\
                1&1&1
            \end{tabular}
            \caption{Truth Table for Disjunction}
            \label{tab:Truth_Table_for_Disjunction}
        \end{table}
        There is another connective called the \textit{exlusive} or, which is
        defined to be false if both $P$ and $Q$ are true. The symbol $\lor$ is
        strictly used to denote the inclusive or. That is, the word or as
        represented by the truth table in
        Tab.~\ref{tab:Truth_Table_for_Disjunction}.
    \subsection{Implication}
        Examining, we see that there are scenarios where $P\Rightarrow{Q}$
        is true and $Q\Rightarrow{P}$ is false, and similarly where
        $P\Rightarrow{Q}$ is false and $Q\Rightarrow{P}$ is true. Propositions
        $P$ and $Q$ such that $P\Rightarrow{Q}$ and $Q\Rightarrow{P}$ are called
        \textit{equivalent}, and great deal of mathematics is devoted to the
        search for equivalencies of statements. This is denoted by the
        connective $P\Leftrightarrow{Q}$. Equivalence has the following truth
        table:
        \begin{table}[H]
            \centering
            \captionsetup{type=table}
            \begin{tabular}{cccccc}
                $P$&$Q$&$P\Rightarrow{Q}$&$Q\Rightarrow{P}$
                   &$P\Leftrightarrow{Q}$
                   &$(P\Rightarrow{Q})\land(Q\Rightarrow{P})$\\
                \hline
                0&0&1&1&1&1\\
                0&1&1&0&0&0\\
                1&0&0&1&0&0\\
                1&1&1&1&1&1
            \end{tabular}
            \caption{Truth Table for Equivalence}
            \label{tab:Truth_Table_for_Equivalence}
        \end{table}
        \subsection{Misc}
        \begin{table}[H]
            \centering
            \captionsetup{type=table}
            \begin{tabular}{c c c c c c}
                \hline
                $p$&$q$&$r$&$\neg{q}$&$p\lor\neg{q}$&$(p\lor\neg{q})\land{r}$\\
                \hline
                0&0&0&1&1&0\\
                0&0&1&1&1&1\\
                0&1&0&0&0&0\\
                0&1&1&0&0&0\\
                1&0&0&1&1&0\\
                1&0&1&1&1&1\\
                1&1&0&0&1&0\\
                1&1&1&0&1&1\\
                \hline
            \end{tabular}
            \caption{Truth Table for $(p\lor\neg{q})\land{r}$}
            \label{tab:Truth_Table_Example}
        \end{table}
        \begin{theorem}
            If $a\rightarrow{b}$, if $\neg{c}\rightarrow\neg{b}$, and if
            $\neg{c}$, then $\neg{a}$.
        \end{theorem}
        \begin{proof}
            For if $a\rightarrow{b}$, then $\neg{b}\rightarrow\neg{a}$. But
            $\neg{c}\rightarrow\neg{b}$. But if $\neg{c}\rightarrow\neg{b}$ and
            $\neg{b}\rightarrow\neg{a}$, then $\neg{c}\rightarrow\neg{a}$. Thus
            $a\rightarrow{b}$, $\neg{c}\rightarrow\neg{b}$, and thus
            $\neg{c}\Rightarrow\neg{a}$.
        \end{proof}
        \begin{problem}
            If $a\rightarrow{b}$, if $\neg{c}\rightarrow\neg{b}$, and if
            $\neg{c}$, then $\neg{a}$.
        \end{problem}
        \begin{proof}
            For if $\neg c \rightarrow \neg b$, then $b\rightarrow c$. But if
            $a\rightarrow b$ and $b\rightarrow c$, then $a\rightarrow c$.
            Therefore $a\rightarrow c$. But if $a\rightarrow c$, then
            $\neg c \rightarrow \neg a$. Therefore,
            $a\rightarrow b,\neg c\rightarrow\neg b,\neg c\Rightarrow\neg a$.
        \end{proof}
        \begin{ftheorem}{Law of Syllogism}{Law_of_Syllogism}
            If $P$, $Q$, and $R$ are propositions, if $P\Rightarrow{Q}$, and if
            $Q\Rightarrow{R}$, then $P\Rightarrow{R}$.
        \end{ftheorem}