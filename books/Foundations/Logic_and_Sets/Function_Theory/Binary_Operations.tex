\section{Binary Operations}
    Binary operations are the standard tools used to develop arithmetic. As
    such, the most familiar examples of binary operations are addition,
    multiplication, and subtraction with real numbers. Division is \textit{not}
    a binary operation on the real numbers since division by zero is undefined.
    To make this explicit we need to give a rigorous definition to binary
    operations. We can do this with the language of functions\index{Function}
    and Cartesian products\index{Cartesian Product}.
    \begin{fdefinition}{Binary Operation}{Binary_Operation}
        A \gls{binary operation} on a \gls{set} $A$ is a function
        $*:A\times{A}\rightarrow{A}$.\index{Binary Operation}
    \end{fdefinition}
    \begin{example}
        Let $\mathbb{R}$ be the set of real numbers and $+$ denote the addition
        of two real numbers. Then $+$ is a binary operation on $\mathbb{R}$.
        Similarly, if $\cdot$ denotes the multiplication of two real numbers,
        than it too is a binary operation on $\mathbb{R}$. For division we are
        lacking the requirement that \textit{for all} $(a,b)\in\mathbb{R}^{2}$
        there is a unique $c\in\mathbb{R}$ such that $a\div{b}=c$, since if
        $b=0$ our expression is undefined. That is, this is not a function from
        $\mathbb{R}^{2}$ to $\mathbb{R}$. If we consider
        all of the non-zero elements, then division is a binary operation. That
        is, division is a binary operation on $\mathbb{R}\setminus\{0\}$.
    \end{example}
    \begin{lexample}{Binary Operation on the Set of Functions}
                    {Binary_Operation_on_the_Set_of_Functions}
        If $A$ is a set, and if $\mathcal{F}(A,A)$ denotes the set of all
        functions $f:A\rightarrow{A}$, and if $\circ$ denotes function
        composition, then $\circ$ is a binary operation on $\mathcal{F}(A,A)$.
        That is, for any two functions $f,g\in\mathcal{F}(A,A)$, the composition
        $g\circ{f}:A\rightarrow{A}$ is again an element of $\mathcal{F}(A,A)$
    \end{lexample}
    Just like functions, there are three important conditions that a binary
    operation must satisfy. Given any ordered pair $(a,b)\in{A}\times{A}$, it
    must be true that $*(a,b)$ is defined. This comes from the definition of a
    function on a set (Def.~\ref{def:Function}). Next, the image of $(a,b)$ must
    be unique. That is, if $*(a,b)=c$ and $*(a,b)=d$, then $c=d$. Note that this
    is not the same as requiring that $*(a,b)=*(b,a)$, and in general this is
    not true. Such binary operations are called
    \textit{commutative}\index{Binary Operation!Commutative}. Lastly, for any
    $(a,b)\in{A}\times{A}$, $*(a,b)$ must be an element of $A$. That is,
    $*(a,b)\in{A}$. All of these requirements come from the definition of a
    function, so in a sense it is redundant to repeat these. In practice one
    defines a binary operation by a formula $\varphi$, and it then becomes
    necessary to show that this formula satisfies these properties before we can
    rightly call it a binary operation.
    \begin{example}
        Let $A=\mathbb{Z}_{2}$ and consider all of the binary operations on
        $\mathbb{Z}_{2}$. We can count these by constructing tables:
        \begin{table}[H]
            \centering
            \begin{tabular}{c|c}
                $(x,y)$&$*(x,y)$\\
                \hline
                $(0,0)$&0\\
                $(0,1)$&0\\
                $(1,0)$&1\\
                $(1,1)$&0
            \end{tabular}
            \label{tab:Binary_Operation_on_Z_2}
            \caption{Simple Binary Operation on $\mathbb{Z}_{2}$}
        \end{table}
        This is one such binary operation, there are 15 others. To see this,
        recall that the number of functions from a set $A$ to a set $B$, where
        both $A$ and $B$ are finite sets with $m$ and $n$ elements,
        respectively, is $n^{m}$. Since $\mathbb{Z}_{2}$ has 2 elements, and
        since a binary operation is a function
        $*\mathbb{Z}_{2}\times\mathbb{Z}_{2}\rightarrow\mathbb{Z}_{2}$, the
        total number of binary operations is $2^{(2^{2})}=2^{4}=16$. In general,
        if $A$ has $n$ elements, and if $B$ is the set of all binary operations
        on $A$, then:
        \begin{equation}
            \textrm{Card}(B)=n^{(n^2)}
        \end{equation}
    \end{example}
    \begin{example}
        Let's consider formulas that take in numbers and return numbers and see
        if they can define operations on various sets. Suppose we have:
        \begin{equation}
            a*b=\{\,r\in\mathbb{R}\;|\;r^{2}=|ab|\,\}
        \end{equation}
        Where $|ab|$ denotes the absolute value of $a$ times $b$. If we take the
        positive square root we can write this as $a*b=\sqrt{|ab|}$. If we
        consider this formula on the rational numbers $\mathbb{Q}$, does it
        define a binary operation? One might recall that $\sqrt{2}$ is not a
        rational number, and thushus $1*2$ is not well-defined. Hence, $*$ is
        not a binary operation on $\mathbb{Q}$. It is a binary operation on
        $\mathbb{R}$, however. Suppose we change the formula to state:
        \begin{equation}
            a*b=\{\,r\in\mathbb{R}\:|\;r^{2}-a^{2}b^{2}=0\,\}
        \end{equation}
        and where we consider this formula to take inputs from $\mathbb{R}$.
        This is not a binary operation since it is poorly defined. That is,
        should $1*1=1$, or should $1*1=\minus{1}$? The formula is ambgious and
        thus $*$ is not a binary operation.
    \end{example}
    \begin{example}
        If we consider subtraction on the integers $\mathbb{Z}$, this is a
        binary operation. The operation is well defined and returns an integer
        for all integer inputs. If instead we consider subtraction on
        $\mathbb{N}$, this is \textit{not} a binary operation since it may take
        in non-negative integers and return a negative integer. For example,
        $1-2=\minus{1}$, and $\minus{1}\notin\mathbb{N}$. A simple fix for this
        is considering again the absolute value function. If we define
        $n*m=|n-m|$, then $*$ is indeed a binary operation on $\mathbb{N}$.
    \end{example}
    \begin{fnotation}{Binary Operation}{Binary_Operation}
        If $A$ is a set and if $*:A\times{A}\rightarrow{A}$ is a binary
        operation on $A$, for any ordered pair $(a,b)\in{A}^{2}$, the image
        of $*(a,b)$ is denoted $a*b$.
    \end{fnotation}
    Given a binary operation $*$ on a set $A$, and given three distinct elements
    $a,b,c\in{A}$, the expression $a*b*c$ is ambiguous. A binary operation takes
    in two elements at a time, and thus the question arises as to whether this
    should denote $(a*b)*c$ or $a*(b*c)$. To rid ourselves of such problems, we
    consider \textit{associative} operations.
    \begin{fdefinition}{Associative Operation}{Associative_Operation}
        A \gls{associative operation} on a \gls{set} $A$ is a
        \gls{binary operation} $*$ such that, for all $a,b,c\in{A}$ it is true
        that $a*(b*c)=(a*b)*c$.\index{Binary Operation!Associative}
    \end{fdefinition}
    \begin{example}
        The usual arithmetic operations addition and multiplication are
        associative. Subtraction is not, for $a-(b-c)=a+(\minus{b})+c$, whereas
        $(a-b)-c=a+(\minus{b})+(\minus{c})$, and these are only equal if $c=0$.
        Similarly, division is not associative for $(1/2)/3=1/6$, where as
        $1/(2/3)=3/2$, and these are not equal.
    \end{example}
    \begin{example}
        We can place an arithmetic structure on the set of $n\times{n}$ matrices
        over a set $A$ if $A$ has the structure of a \textit{ring}. This is the
        standard arithmetic of matrices that one may be familiar with when one
        considers real valued entries. Both matrix addition and multiplication
        are associative.
    \end{example}
    \begin{example}
        Consider a finite set $A$ and consider the set of all functions from
        $\mathbb{Z}_{n}$ to $A$. That is, $\mathcal{F}_{n}(\mathbb{Z}_{n},A)$.
        Define $A[x]$ by:
        \begin{equation}
            \mathcal{F}=\bigcup_{n\in\mathbb{N}}\mathcal{F}_{n}
        \end{equation}
        That is, the set of all finite sequences in $A$. We can form an
        associative operation on this set by defining the concatenation
        operation. Given $f,g\in\mathcal{F}$, suppose
        $f\in\mathcal{F}(\mathbb{Z}_{m},A)$ and
        $g\in\mathcal{F}(\mathbb{Z}_{n},A)$. We define
        $f*g\in\mathcal{F}(\mathbb{Z}_{m+n},A)$ as follows:
        \begin{equation}
            (f*g)(k)=
            \begin{cases}
                f(k),&k\in\mathbb{Z}_{m}\\
                g(k-m),&k\in\mathbb{Z}_{m+n}\textrm{ and }k\geq{m}
            \end{cases}
        \end{equation}
        That is, given two sequences $f_{0},f_{1},\dots,f_{m-1}$ and
        $g_{0},g_{1},\dots,g_{n-1}$, we concatenate them to form the sequence
        $f_{0},\dots,f_{m-1},g_{0},\dots,g_{n-1}$. This operation is associative
        since if $f,g,h\in\mathcal{F}$, then:
        \begin{subequations}
            \begin{align}
                f*(g*h)&=(f_{0},f_{1},\dots,f_{m-1})
                    *(g_{0},g_{1},\dots,g_{m-1},h_{0},h_{1},\dots,h_{r-1})\\
                &=f_{0},f_{1},\dots,f_{m-1},
                    g_{0},g_{1},\dots,g_{m-1},h_{0},h_{1},\dots,h_{r-1}\\
                &=(f_{0},f_{1},\dots,f_{m-1},
                    g_{0},g_{1},\dots,g_{m-1})*(h_{0},h_{1},\dots,h_{r-1})\\
                &=(f*g)*h
            \end{align}
        \end{subequations}
        If $A$ has more than one point than $*$ is not commutative. For let
        $f,g:\mathbb{Z}_{1}\rightarrow{A}$ be defined by $f(0)=a$ and $g(0)=b$,
        respectively. Then $f*g=a,b$ but $g*f=b,a$, and thus $f*g\ne{g}*f$.
        There is, however, an identity. Consider $\mathbb{Z}_{0}$, which is the
        empty set. Any function from $\mathbb{Z}_{0}$ to $A$ is therefore the
        \textit{empty sequence}. If we concatenate $f$ with the empty sequence
        we get back $f$, and this then acts as our unital element.
    \end{example}
    \begin{fdefinition}{Idempotent}{Idempotent}
        An idemopotent element of a \gls{set} $A$ under a \gls{binary operation}
        $*$ is an element $a\in{A}$ such that $a*a=a$.\index{Idempotent}
    \end{fdefinition}
    It is occasionally useful to think of binary operations purely as functions,
    and so we will use function notation at these times. For the most part we
    will stick with notation defined in Not.~\ref{not:Binary_Operation}. There
    are several types of binary operations worth studying, and several key
    properties that these operations can have. One of the most fundamental is
    the existence of a \textit{unital} element, also known as an identity.
    \begin{fdefinition}{Left Unital Element}{Left_Unital_Element}
        A left unital element in a \gls{set} $A$ under a \gls{binary operation}
        $*$ on $A$ is an element $e_{L}\in{A}$ such that, for all $a\in{A}$ it
        is true that $e_{L}*a=a$.\index{Unital Element!Left Unital}
    \end{fdefinition}
    \begin{theorem}
        If $A$ is a set, if $*$ is a binary operation on $A$, and if $e_{L}$
        is a left unital element of $A$, then $e_{L}$ is idempotent.
    \end{theorem}
    \begin{proof}
        For $e_{L}=e_{L}*e_{L}$ (Def.~\ref{def:Left_Unital_Element}), and thus
        $e_{L}$ is idemopotent (Def.~\ref{def:Idempotent}).
    \end{proof}
    \begin{example}
        From the definition of a left unital element
        (Def.~\ref{def:Left_Unital_Element}) it would seem natural to define a
        right unital element. The importance is to note that the existence of a
        left unital element does not imply the existence of a right. Indeed, if
        $A$ is a set and $*$ is a binary operation, given a left identity
        $e_{L}$ and a right identity $e_{R}$ it will be true that $e_{R}=e_{L}$
        and thus all left and right unital elements will be the same
        (see Thm.~\ref{thm:left_and_right_identity_implies_identity}). Thus to
        find counterexamples to the claim that the existence of a left unital
        element implies the existence of a right unital element we need to think
        of strange operations. Let $A=\mathbb{R}$ and let $*$ be defined by
        $a*b=b$ for all $a,b\in\mathbb{R}$. Then every element of $\mathbb{R}$
        is a left unital element. Moreover, none of the elements of $\mathbb{R}$
        are right unitals.
    \end{example}
    \begin{fdefinition}{Right Unital Element}{Right_Unital_Element}
        A right unital element of a \gls{set} $A$ under a \gls{binary operation}
        $*$ is an element $e_{R}$ such that for all $a\in{A}$ it is true that
        $a*e_{R}=a$.\index{Unital Element!Right Unital}
    \end{fdefinition}
    \begin{example}
        Consider $\mathbb{R}$ with the operation $*$ defined by $a*b=a+b+1$.
        This operation has a right unital element, $\minus{1}$. For:
        \begin{equation}
            a*(\minus{1})=a+(\minus{1})+1=a+0=a
        \end{equation}
        And this is true for all $a\in\mathbb{R}$, so $\minus{1}$ is a right
        unital element. It turns out this is also a left unital element, and
        hence a unital element, and this can be proven if addition is known to
        be an \textit{associative} operation.
    \end{example}
    \begin{theorem}
        \label{thm:left_and_right_identity_implies_identity}%
        If $A$ is a set, if $*$ is a binary operation on $A$, if $e_{L}$ is a
        left unital element of $A$, and if $e_{R}$ is a right unital element of
        $a$, then $e_{L}=e_{R}$.
    \end{theorem}
    \begin{proof}
        For $e_{L}=e_{L}*e_{R}=e_{R}$
        (Defs.~\ref{def:Left_Unital_Element} and \ref{def:Right_Unital_Element})
        and thus $e_{L}=e_{R}$.
    \end{proof}
    \begin{example}
        Consider a non-empty set $A$ and the set of all functions from $A$ to
        itself, $\mathcal{F}(A,A)$. Let $\circ$ denote the binary operation of
        function composition. Then $\mathcal{F}(A,A)$ has a right identity under
        $\circ$, and a left identity. For
        the identity function $\textrm{id}_{A}$ acts as a right identity:
        \begin{equation}
            (f\circ\textrm{id}_{A})(x)
            =f\big(\textrm{id}_{A}(x)\big)
            =f(x)
        \end{equation}
        And thus $\textrm{id}_{A}$ is a right identity. By
        Thm.~\ref{thm:left_and_right_identity_implies_identity}, any left
        identity must also be a right identity, and so the likely candidate to
        check is $\textrm{id}_{A}$. And indeed we have:
        \begin{equation}
            (\textrm{id}_{A}\circ{f})(x)
            =\textrm{id}_{A}\big(f(x)\big)
            =f(x)
        \end{equation}
        And thus $\textrm{id}_{A}$ is a left identity as well.
    \end{example}
    \begin{theorem}
        If $A$ is a set, if $*$ is a binary operation on $A$, if $e_{L}$ and
        $e_{L}'$ are left unital elements, and if $e_{R}$ is a right unital
        element, then $e_{L}=e_{L}'$
    \end{theorem}
    \begin{proof}
        For if $e_{L}$ and $e_{L}'$ are left unitals and $e_{R}$ is a right
        unital, then $e_{L}=e_{R}$ and $e_{L}'=e_{R}$
        (Thm.~\ref{thm:left_and_right_identity_implies_identity}). By the
        transitivity of equality, $e_{L}=e_{L}'$.
    \end{proof}
    The same is true of right identities and thus if we find a left identity and
    a right identity, then they're the same and they are \textit{the} identity.
    \begin{fdefinition}{Unital Element}{Unital_Element}
        A \gls{unital element} of a \gls{set} $A$ under a \gls{binary operation}
        $*$ is an element $e\in{A}$ that is both a right unital element and a
        left unital element.\index{Unital Element}
    \end{fdefinition}
    \begin{example}
        Let $\mathbb{R}$ be the set of real numbers and let $+$ be the usual
        notion of addition. Then 0 is a unital element of $\mathbb{R}$ with
        respect to this operation. That is, for any real number $x$ we have
        $x+0=0+x=x$. For multiplication the unital element is 1. This is because
        $1\cdot{x}=x\cdot{1}=x$. Subtraction has a right unital element, and
        again it is 0 since $x-0=x$, but no left identity. To see this, suppose
        $e-x=x$ for all $x$. Applying some algebra we have that $e=2x$, meaning
        there is no constant $e\in\mathbb{R}$ such that for all $x$, $e-x=x$.
        Since subtraction has no left unital element, it has no unital element
        either.
    \end{example}
    \begin{theorem}
        \label{thm:Unital_Elements_are_Unique}%
        If $A$ is a set, if $*$ is a binary operation on $A$, and if $e$ and
        $e'$ are unital elements of $A$, then $e=e'$
    \end{theorem}
    \begin{proof}
        For then $e$ is a right unital element and $e'$ is a left unital element
        (Def.~\ref{def:Unital_Element}). But then $e=e'$
        (Thm.~\ref{thm:left_and_right_identity_implies_identity}).
    \end{proof}
    \begin{theorem}
        If $A$ is a set, if $*$ is a binary operation on $A$, if $e$ is a unital
        element, and $e_{R}$ is a right unital element, then $e=e_{R}$.
    \end{theorem}
    \begin{proof}
        For if $e$ is a unital element, then it is a left unital element
        (Def.~\ref{def:Unital_Element}). But if $e$ is a left unital element and
        $e_{R}$ is a right unital element, then $e=e_{R}$
        (Thm.~\ref{thm:left_and_right_identity_implies_identity}).
    \end{proof}
    Thus, if one has an operation that contains a unital element $e$, then there
    is only one right unital element, and only one left unital element, and that
    is $e$. The next thing to discuss is the notion of inverses.
    \begin{fdefinition}{Weakly Right Invertible}{Weakly_Right_Invertible}
        A weakly right invertible element in a \gls{set} $A$ under a
        \gls{binary operation} $*$ on $A$ is an element $a\in{A}$ such that
        there is a $b\in{A}$ such that $a*b$ is a right unital element.
        \index{Invertible Element!Weakly Right Invertible}
    \end{fdefinition}
    That is to say, an element $a\in{A}$ is weakly right invertible if there is
    a $b\in{A}$ such that, for all $r\in{A}$ the following is true:
    \begin{equation}
        r*(a*b)=r
    \end{equation}
    The reason for the word weakly is because we do not require $r*(a*b)=r$.
    The justification for \textit{right} is because we also do not require
    $r*(b*a)=r$. That is, $a$ is weakly right invertible if there is some
    other element $b$ that we can multiply on the right of $a$ such that the
    product acts as a right identity. One would think that such a restriction
    would be useless, but it turns out the notions allow one to determine if
    something is a \textit{group}.
    \begin{theorem}
        If $A$ is a set, if $*$ is an associative operation on $A$, if $a\in{A}$
        is weakly right invertible, and if $b$ is such that $a*b$ is a right
        unital element, then $b*a$ is idempotent.
    \end{theorem}
    \begin{proof}
        For:
        \begin{align}
            b*a&=\big(b*(a*b)\big)*a
            \tag{$a*b$ is a right unital element %
                 (Def.~\ref{def:Right_Unital_Element})}\\
            &=\big((b*a)*b\big)*a
            \tag{Associativity}\\
            &=(b*a)*(b*a)
            \tag{Associativity}
        \end{align}
        And therefore $b*a$ is idempotent (Def.~\ref{def:Idempotent}). 
    \end{proof}
    \begin{theorem}
        \label{thm:ab_right_unital_implies_ba_right_unital}%
        If $A$ is a set, if $*$ is an associative binary operation on $A$, if
        $a,b\in{A}$ are weakly right invertible, and if $a*b$ is a right unital
        element, then $b*a$ is a right unital element.
    \end{theorem}
    \begin{proof}
        For if $b$ is weakly right invertible, then there is a $c\in{A}$ such
        that $b*c$ is a right unital element
        (Def.~\ref{def:Weakly_Right_Invertible}). But then:
        \begin{align}
            b*a&=b*\big(a*(b*c)\big)
            \tag{$b*c$ is a right unital element %
                 (Def.~\ref{def:Right_Unital_Element})}\\
            &=b*\big((a*b)*c\big)
            \tag{Associativity}\\
            &=\big(b*(a*b)\big)*c
            \tag{Associativity}\\
            &=b*c
            \tag{$a*b$ is a right unital element %
                 (Def.~\ref{def:Right_Unital_Element})}
        \end{align}
        And thus by transitivity, $b*a=b*c$. But $b*c$ is a right unital element
        and therefore $b*a$ is a right unital element.
    \end{proof}
    \begin{theorem}
        If $A$ is a set, if $*$ is an associative binary operation on $A$, if
        $e\in{A}$ is a left unital element, and if $a\in{A}$ is weakly right
        invertible, then $e$ is a unital element.
    \end{theorem}
    \begin{proof}
        For if $a$ is weakly right invertible, then there is a $b\in{A}$ such
        that $a*b$ is a right unital element
        (Def.~\ref{def:Weakly_Right_Invertible}). But if $e$ is a left unital
        element and $a*b$ is a right unital element, then $e=a*b$
        (Thm.~\ref{thm:left_and_right_identity_implies_identity}). Therefore,
        $e$ is a unital element.
    \end{proof}
    \begin{theorem}
        \label{thm:right_unit_with_right_inv_almost_implies_unit}%
        If $A$ is a set, if $*$ is an associative binary operation on $A$, if
        $e$ is a right unital element, and if $a,b\in{A}$ are weakly right
        invertible elements such that $a*b=e$, then $e*a=a$.
    \end{theorem}
    \begin{proof}
        For:
        \begin{align}
            e*a&=(a*b)*a
            \tag{Hypothesis}\\
            &=a*(b*a)
            \tag{Associativity}\\
            &=a
            \tag{Thm.~\ref{thm:ab_right_unital_implies_ba_right_unital}}
        \end{align}
        And therefore $e*a=a$.
    \end{proof}
    \begin{theorem}
        \label{thm:right_unit_with_right_inv_implies_unit}%
        If $A$ is a set, if $*$ is an associative binary operation on $A$, if
        $e\in{A}$ is a unique right unital element, and if for all $a\in{A}$ it
        is true that $a$ is weakly right invertible, then $e$ is a unital
        element.
    \end{theorem}
    \begin{proof}
        For suppose not. Then there is an $a\in{A}$ such that $e*a\ne{a}$. But
        since $a\in{A}$, $a$ is weakly right invertible and thus there is a
        $b\in{A}$ such that $a*b$ is a right unital element
        (Def.~\ref{def:Weakly_Right_Invertible}). But by hypothesis $e$ is the
        unique right unital element, and thus $a*b=e$. But since $b\in{A}$,
        by hypothesis $b$ is weakly right invertible. But then $a$ and $b$ are
        weakly right invertible elements such that $a*b=e$, and thus by
        Thm.~\ref{thm:right_unit_with_right_inv_almost_implies_unit}, $e*a=a$,
        a contradiction. Therefore, $e$ is a unital element.
    \end{proof}
    We've almost set up the definition of a group. We have that a set with an
    associative binary operation that contains a unique right unital element
    and such that all elements are weakly right invertible will necessarily have
    a unique unital element. Next we need to show that every element will also
    be weakly \textit{left} invertible, and we'll have a group. It is crucial to
    note the requirement that the right unital element in
    Thm.~\ref{thm:right_unit_with_right_inv_implies_unit} is unique. If it is
    not, we may not have a unital element at all. Consider again the operation
    $a*b=a$ on some set with at least two element. Then every element is a
    right unital element, and similarly every element is weakly right invertible
    since, for any $a,b$, we have:
    \begin{equation}
        a*(b*b)=a*b=a
    \end{equation}
    and thus $b$ is weakly right invertible, and is it's own weak inverse. This
    structure has no left unital element since $e*a=e$ for any $e\in{A}$, and
    thus $e$ cannot be a left unital. The uniqueness of $e$ in
    Thm.~\ref{thm:right_unit_with_right_inv_implies_unit} is what prevents such
    pathological examples from appearing.
    \begin{fdefinition}{Right Invertible}{Right_Invertible}
        A right invertible element of a \gls{set} $A$ under a
        \gls{binary operation} is an element $a\in{A}$ such that there exists
        $b\in{A}$ such that $a*b$ is a \gls{unital element}.
        \index{Invertible Element!Right Invertible}
    \end{fdefinition}
    \begin{theorem}
        \label{thm:r_inv_implies_weak_r_inv}%
        If $A$ is a set, if $*$ is a binary operation on $A$, and if $a\in{A}$
        is right invertible, then it is weakly right invertible.
    \end{theorem}
    \begin{proof}
        For then there is a $b\in{A}$ such that $a*b$ is a unital element
        (Def.~\ref{def:Right_Invertible}). But then unital elements are
        right unital elements (Def.~\ref{def:Unital_Element}), and thus $a*b$
        is a right unital element. Therefore, $a$ is weakly right invertible.
    \end{proof}
    Nothing to deep here and we've simply strengthened the requirement that
    $a*b$ not only be a right unital element, but also a left unital element as
    well.
    \begin{theorem}
        \label{thm:assoc_right_idem_inv_is_unit}%
        If $A$ is a set, if $*$ is an associative binary operation on $A$, if
        $a$ is right invertible and idempotent, then $a$ is a unital element.
    \end{theorem}
    \begin{proof}
        For if $a$ is right invertible then there is a $b\in{A}$ such that
        $a*b$ is a unital element.
        \begin{align}
            a&=a*(a*b)
            \tag{$a*b$ is a unital element (Def.~\ref{def:Unital_Element})}\\
            &=(a*a)*b
            \tag{Associativity}\\
            &=a*b
            \tag{$a$ is idempotent (Def.~\ref{def:Idempotent})}
        \end{align}
        And thus $a=a*b$. But $a*b$ is a unital element, and thus so is $a$.
    \end{proof}
    \begin{fdefinition}{Weakly Left Invertible}{Weakly_Left_Invertible}
        A weakly left invertible element of a \gls{set} $A$ under a
        \gls{binary operation} $*$ is an element $a\in{A}$ such that there
        exists a $b\in{A}$ such that $b*a$ is a left unital element.
        \index{Invertible Element!Weakly Left Invertible}
    \end{fdefinition}
    All of the theorems about weakly right invertible elements apply to weakly
    left invertible elements, but we need to swap all of the orders of
    multiplication.
    \begin{theorem}
        If $A$ is a set, if $*$ is an associative operation on $A$, if $a\in{A}$
        is weakly left invertible, and if $b\in{A}$ is such that $b*a$ is a left
        unital element, then $a*b$ is idempotent.
    \end{theorem}
    \begin{proof}
        For:
        \begin{align}
            a*b&=a*\big((b*a)*b\big)
            \tag{$b*a$ is a left unital element %
                 (Def.~\ref{def:Left_Unital_Element})}\\
            &=a*\big(b*(a*b)\big)
            \tag{Associativity}\\
            &=(a*b)*(a*b)
            \tag{Associativity}
        \end{align}
        And therefore $a*b$ is idempotent (Def.~\ref{def:Idempotent}). 
    \end{proof}
    \begin{theorem}
        \label{thm:ba_left_unital_implies_ab_left_unital}%
        If $A$ is a set, if $*$ is an associative binary operation on $A$, if
        $a,b\in{A}$ are weakly left invertible, and if $b*a$ is a left unital
        element, then $a*b$ is a left unital element.
    \end{theorem}
    \begin{proof}
        For if $b$ is weakly left invertible, then there is a $c\in{A}$ such
        that $c*b$ is a left unital element
        (Def.~\ref{def:Weakly_Left_Invertible}). But then:
        \begin{align}
            a*b&=\big((c*b)*a\big)*b
            \tag{$c*b$ is a left unital element %
                (Def.~\ref{def:Left_Unital_Element})}\\
            &=\big(c*(b*a)\big)*b
            \tag{Associativity}\\
            &=c*\big((b*a)*b\big)
            \tag{Associativity}\\
            &=c*b
            \tag{$b*a$ is a left unital element %
                 (Def.~\ref{def:Left_Unital_Element})}
        \end{align}
        And thus by transitivity, $a*b=c*b$. But $c*b$ is a left unital element
        and therefore $a*b$ is a left unital element.
    \end{proof}
    \begin{theorem}
        If $A$ is a set, if $*$ is an associative binary operation on $A$, if
        $e\in{A}$ is a left unital element, and if $a\in{A}$ is weakly right
        invertible, then $e$ is a unital element.
    \end{theorem}
    \begin{proof}
        For if $a$ is weakly left invertible, then there is a $b\in{A}$ such
        that $b*a$ is a left unital element
        (Def.~\ref{def:Weakly_Right_Invertible}). But if $e$ is a right unital
        element and $b*a$ is a left unital element, then $e=a*b$
        (Thm.~\ref{thm:left_and_right_identity_implies_identity}). Therefore,
        $e$ is a unital element.
    \end{proof}
    \begin{theorem}
        \label{thm:left_unit_with_left_inv_almost_implies_unit}%
        If $A$ is a set, if $*$ is an associative binary operation on $A$, if
        $e$ is a left unital element, and if $a,b\in{A}$ are weakly left
        invertible elements such that $b*a=e$, then $a*e=a$.
    \end{theorem}
    \begin{proof}
        For:
        \begin{align}
            a*e&=a*(b*a)
            \tag{Hypothesis}\\
            &=(a*b)*a
            \tag{Associativity}\\
            &=a
            \tag{Thm.~\ref{thm:ba_left_unital_implies_ab_left_unital}}
        \end{align}
        And therefore $e*a=a$.
    \end{proof}
    \begin{theorem}
        \label{thm:left_unit_with_left_inv_implies_unit}%
        If $A$ is a set, if $*$ is an associative binary operation on $A$, if
        $e\in{A}$ is a unique left unital element, and if for all $a\in{A}$ it
        is true that $a$ is weakly left invertible, then $e$ is a unital
        element.
    \end{theorem}
    \begin{proof}
        For suppose not. Then there is an $a\in{A}$ such that $a*e\ne{a}$. But
        since $a\in{A}$, $a$ is weakly left invertible and thus there is a
        $b\in{A}$ such that $b*a$ is a left unital element
        (Def.~\ref{def:Weakly_Left_Invertible}). But by hypothesis $e$ is the
        unique right unital element, and thus $b*a=e$. But since $b\in{A}$,
        by hypothesis $b$ is weakly left invertible. But then $a$ and $b$ are
        weakly left invertible elements such that $b*a=e$, and thus by
        Thm.~\ref{thm:left_unit_with_left_inv_almost_implies_unit}, $e*a=a$,
        a contradiction. Therefore, $e$ is a unital element.
    \end{proof}
    \begin{theorem}
        \label{thm:existence_of_weak_left_and_weak_right_implies_unital}%
        If $A$ is a set, if $*$ is an associative binary operation on $A$, if
        $a\in{A}$ is a weakly left invertible element, and if $b\in{A}$ is a
        weakly right invertible element, then there exists a unique $e\in{A}$
        such that $e$ is a unital element.
    \end{theorem}
    \begin{proof}
        For if $a$ is weakly left invertible, then there exists $r_{L}\in{A}$
        such that $a_{L}*a$ is a left unital element
        (Def.~\ref{def:Left_Unital_Element}). But if $b$ is weakly right
        invertible then there is a $b_{R}$ such that $b*b_{R}$ is a right
        unital element. But then there exists a left unital element
        $b_{L}*b$ and a right unital element $a*a_{R}$, and thus
        $b_{L}*b=a*a_{R}$
        (Thm.~\ref{thm:left_and_right_identity_implies_identity}), and thus
        there exists a unital element (Def.~\ref{def:Unital_Element}). And
        unital elements are unique (Thm.~\ref{thm:Unital_Elements_are_Unique}),
        completing the proof.
    \end{proof}
    \begin{theorem}
        \label{thm:unique_r_id_and_weak_r_inv_implies_weak_l_inv}%
        If $A$ is a set, if $*$ is a binary operation on $A$, if there is a
        unique right unital element $e\in{A}$, and if for all $a$ it is true
        that $a$ is weakly right invertible, then for all $a\in{A}$ it is true
        that $a$ is weakly left invertible.
    \end{theorem}
    \begin{proof}
        For suppose not and suppose there is an $a\in{A}$ such that $a$ is not
        weakly left invertible. But $a$ is weakly right invertible and thus
        there is a $b\in{A}$ such that $a*b$ is a right unital element. But
        $e$ is the unique right unital element, and thus $a*b=e$. But if
        $a*b$ is a right unital element, then $b*a$ is a right unital element
        (Thm.~\ref{thm:right_unit_with_right_inv_almost_implies_unit}), and
        thus $b*a=e$. But since every element of $A$ is weakly right invertible,
        and since $e$ is the unique right unital element, it is true that
        $e$ is a unital element
        (Thm.~\ref{thm:right_unit_with_right_inv_implies_unit}). But then
        $b*a$ is a unital element, and is therefore a left unital element
        (Def.~\ref{def:Unital_Element}), a contradiction. Thus, $a$ is weakly
        left invertible.
    \end{proof}
    \begin{fdefinition}{Left Invertible}{Left_Invertible}
        A left invertible element of a \gls{set} $A$ under a
        \gls{binary operation} is an element $a\in{A}$ such that there exists a
        $b\in{A}$ such that $b*a$ is a \gls{unital element}.
        \index{Invertible Element!Left Invertible}
    \end{fdefinition}
    \begin{theorem}
        If $A$ is a set, if $*$ is a binary operation on $A$, and if $a\in{A}$
        is left invertible, then it is weakly left invertible.
    \end{theorem}
    \begin{proof}
        For then there is a $b\in{A}$ such that $b*a$ is a unital element
        (Def.~\ref{def:Left_Invertible}). But then unital elements are
        right unital elements (Def.~\ref{def:Unital_Element}), and thus $b*a$
        is a left unital element. Therefore, $a$ is weakly left invertible.
    \end{proof}
    Again, a rather obvious theorem that comes straight from the definition.
    We use the notions of right and left invertible to define invertible.
    \begin{theorem}
        \label{thm:assoc_left_idem_inv_is_unit}%
        If $A$ is a set, if $*$ is an associative binary operation on $A$, and
        if $a$ is left invertible and idemptotent, then $a$ is a unital element.
    \end{theorem}
    \begin{proof}
        For if $a$ is left invertible, there is a $b\in{A}$ such that $b*a$ is
        unital. But then:
        \begin{align}
            a&=(b*a)*a
            \tag{$b*a$ is a unital element (Def.~\ref{def:Unital_Element})}\\
            &=b*(a*a())
            \tag{Associativity}\\
            &=b*a
            \tag{$a$ is idempotent (Def.~\ref{def:Idempotent})}
        \end{align}
        And thus $a=b*a$. But $b*a$ is unital, and thus so is $a$.
    \end{proof}
    \begin{fdefinition}{Invertible Element}{Invertible_Element}
        An invertible element of a a \gls{set} $A$ under a
        \gls{binary operation} is an element $a\in{A}$ that is both
        left invertible and right invertible.
        \index{Invertible Element}
    \end{fdefinition}
    \begin{theorem}
        \label{thm:Unital_Elements_Are_Invertible}%
        If $A$ is a set, if $*$ is a binary operation on $A$, and if $e$ is a
        unital element of $A$, then $e$ is an invertible element.
    \end{theorem}
    \begin{proof}
        For since $e$ is a unital element, it is true that $e=e*e$
        (Def.~\ref{def:Unital_Element}). Therefore $e$ is invertible element
        (Def.~\ref{def:Invertible_Element}).
    \end{proof}
    \begin{theorem}
        \label{thm:Weak_r_inv_and_weak_l_inv_implies_inv}%
        If $A$ is a set, if $*$ is an associative binary operation on $A$, and
        if $a\in{A}$ is weakly left invertible and weakly right invertible, then
        $a$ is invertible.
    \end{theorem}
    \begin{proof}
        For if $a$ is weakly right invertible and weakly left invertible, then
        there is a unique unital element $e\in{A}$
        (Thm.~\ref{thm:existence_of_weak_left_and_weak_right_implies_unital}).
        But since $a$ is weakly right invertible, there is a $b$ such that
        $a*b$ is a right unital element
        (Def.~\ref{def:Weakly_Right_Invertible}). And since $e$ is the unique
        unital element, it is the unique right unital element and thus $a*b=e$.
        But since $a$ is weakly left invertible there is a $c\in{A}$ such that
        $c*a$ is a left unital element (Def.~\ref{def:Weakly_Left_Invertible}),
        and again since $e$ is unique we have that $c*a=e$. But then $a$ has a
        right inverse and a left inverse
        (Defs.~\ref{def:Right_Invertible} and \ref{def:Left_Invertible}), and is
        therefore invertible (Def.~\ref{def:Invertible_Element})
    \end{proof}
    \begin{theorem}
        \label{thm:unqiue_right_unit_and_weak_r_inv_implies_inv}%
        If $A$ is a set, if $*$ is an associative binary operation on $A$, if
        $e$ is a unique right unital element, and if for all $a\in{A}$ it is
        true that $a$ is weakly right invertible, then for all $a\in{A}$ it is
        true that $a$ is invertible.
    \end{theorem}
    \begin{proof}
        For by Thm.~\ref{thm:unique_r_id_and_weak_r_inv_implies_weak_l_inv},
        for all $a\in{A}$ it is true that $a$ is weakly left invertible.
        But then for all $a\in{A}$, $a$ is weakly left invertible and weakly
        right invertible, and is therefore invertible
        (Thm.~\ref{thm:Weak_r_inv_and_weak_l_inv_implies_inv}).
    \end{proof}
    \begin{theorem}
        \label{thm:Assoc_Op_Inverses_are_Unique}%
        If $A$ is a set, if $*$ is a associative binary operation on $A$, if $a$
        is an invertible element, and if $b,c\in{A}$ are such that $b*a$ and
        $a*c$ are unital elements, then $b=c$.
    \end{theorem}
    \begin{proof}
        For:
        \begin{align}
            b&=b*(a*c)
            \tag{$a*b$ is a unital element}\\
            &=(b*a)*c
            \tag{Associativity}\\
            &=c
            \tag{$b*a$ is a unital element}
        \end{align}
        And therefore $b=c$.
    \end{proof}
    \begin{theorem}
        \label{thm:Assoc_Op_Inverses_are_Unique_Right}%
        If $A$ is a set, if $*$ is an associative binary operation on $A$, if
        $a\in{A}$ is invertible, and if $b,c\in{A}$ are such that $a*b$ and
        $a*c$ are unital elements, then $b=c$.
    \end{theorem}
    \begin{proof}
        For if $a$ is invertible, there is an $l\in{A}$ such that $l*a$ is a
        unital element (Def.~\ref{def:Unital_Element}). But then $l*a$ is a
        unital element and $a*b$ is a unital element, and thus $l=b$
        (Thm.~\ref{thm:Assoc_Op_Inverses_are_Unique}). Similarly $l=c$, and thus
        by the transitivity of equality, $b=c$.
    \end{proof}
    \begin{theorem}
        \label{thm:Assoc_Op_Inverses_are_Unique_Left}%
        If $A$ is a set, if $*$ is an associative binary operation on $A$, if
        $a\in{A}$ is invertible, and if $b,c\in{A}$ are such that $b*a$ and
        $c*a$ are unital elements, then $b=c$.
    \end{theorem}
    \begin{proof}
        For if $a$ is invertible, there is an $r\in{A}$ such that $r*a$ is a
        unital element (Def.~\ref{def:Unital_Element}). But then $a*r$ is a
        unital element and $b*a$ is a unital element, and thus $b=r$
        (Thm.~\ref{thm:Assoc_Op_Inverses_are_Unique}). Similarly $r=c$, and thus
        by the transitivity of equality, $b=c$.
    \end{proof}
    Combining Thms.~\ref{thm:Assoc_Op_Inverses_are_Unique_Right}-%
    \ref{thm:Assoc_Op_Inverses_are_Unique_Left} we have that invertible elements
    have a unique \textit{inverse} element. Depending on the operation under
    discussing, the inverse of $a$ is denoted as either $\minus{a}$ or
    $a^{\minus{1}}$.
    \begin{theorem}
        \label{thm:assoc_op_prod_of_inv_is_inv}%
        If $A$ is a set, if $*$ is a binary operation on $A$, if $a,b\in{A}$ are
        invertible elements, then $a*b$ is invertible and
        $(a*b)^{\minus{1}}=b^{\minus{1}}*a^{\minus{1}}$.
    \end{theorem}
    \begin{proof}
        For if $a$ is invertible, then there is an $a^{\minus{1}}$ such that
        $a*a^{\minus{1}}$ is a unital element
        (Def.~\ref{def:Invertible_Element}). But unital elements are unique
        (Thm.~\ref{thm:Unital_Elements_are_Unique}). Let $e\in{A}$ be the unital
        element. But $b$ is also invertible and thus there is a
        $b^{\minus{1}}\in{A}$ such that $b*b^{\minus{1}}=e$. Thus:
        \begin{align}
            e&=a*a^{\minus{1}}
            \tag{Inverse Property}\\
            &=(a*e)*a^{\minus{1}}
            \tag{Identity}\\
            &=\big(a*(b*b^{\minus{1}}\big)*a^{\minus{1}}
            \tag{Inverse Property}\\
            &=\big((a*b)*b^{\minus{1}}\big)*a^{\minus{1}}
            \tag{Associativity}\\
            &=(a*b)*(b*b^{\minus{1}})
            \tag{Associativity}
        \end{align}
        And therefore $(a*b)*(b^{\minus{1}}*a^{\minus{1}})$ is a unital element,
        and thus $a*b$ is right invertible (Def.~\ref{def:Right_Invertible}).
        And similarly, $(b^{\minus{1}}*a^{\minus{1}})*(a*b)=e$, and thus
        $a*b$ is left invertible (Def.~\ref{def:Left_Invertible}). Therefore,
    \end{proof}
    \begin{fdefinition}{Commuting Elements}{Commuting_Elements}
        Commuting elements in a \gls{set} $A$ with respect to a
        \gls{binary operation} $*$ on $A$ are elements $a,b\in{A}$ such that
        $a*b=b*a$.
    \end{fdefinition}
    \begin{fdefinition}{Commutative Operation}{Commutative_Operation}
        A \gls{commutative operation} on a \gls{set} $A$ is a
        \gls{binary operation} $*$ such that for all $(a,b)\in{A}^{2}$ it is
        true that $a*b=b*a$.\index{Binary Operation!Commutative}
    \end{fdefinition}
    \begin{fdefinition}{Distributive Operation}{Distributive_Operation}
        A distributive operation over a \gls{binary operation} $+$ on a
        \gls{set} $A$ is a binary operation $*$ on $A$ such that, for all
        $a,b,c\in{A}$ the following is true:
        \index{Binary Operation!Distributive}
        \begin{equation*}
            a*(b+c)=(a*b)+(a*c)
        \end{equation*}
    \end{fdefinition}