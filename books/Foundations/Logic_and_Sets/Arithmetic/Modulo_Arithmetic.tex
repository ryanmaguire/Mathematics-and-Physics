\section{Modulo Arithmetic}
    \begin{theorem}
        \label{thm:Modulo_n_is_Equiv_Relation}%
        If $n\in\mathbb{N}$, and if
        $R\subseteq\mathbb{Z}\times\mathbb{Z}$ is defined by:
        \begin{equation}
            R=\{\,(a,b)\in\mathbb{Z}^{2}\;|\;
                n\textrm{ divides }b-a\,\}
        \end{equation}
        then $R$ is an equivalence relation on $\mathbb{Z}$.
    \end{theorem}
    \begin{proof}
        For $aRa$ since $a-a=0$ and $n$ divides 0. If $aRb$, then $n$
        divides $b-a$ and hence there is a $k\in\mathbb{Z}$ such that
        $nk=b-a$ (Def.~\ref{def:Divisor_of_Integer}). But then
        $n(\minus{k})=a-b$, and thus $n$ divides $a-b$
        (Def.~\ref{def:Divisor_of_Integer}), hence $bRa$. Lastly, if
        $aRb$ and $bRc$, then there exist integers $j,k\in\mathbb{Z}$
        such that $nj=b-a$ and $nk=c-b$. But then:
        \begin{equation}
            n(k+j)=nk+nk=(c-b)+(b-a)=c-a
        \end{equation}
        and therefore $aRc$.
    \end{proof}
    \begin{fdefinition}{Ring of Integers Modulo $n$}{Ring_Ints_Mod_N}
        The ring of integers modulo $n\in\mathbb{N}$ is the quotient set
        $\mathbb{Z}/R$ where $R$ is the equivalence relation
        $R\subseteq\mathbb{Z}^{2}$ defined by:
        \begin{equation*}
            R=\{\,(a,b)\in\mathbb{Z}^{2}\;|\;n\textrm{ divides }b-a\,\}
        \end{equation*}
        We denote this $\mathbb{Z}/n\mathbb{Z}$.
    \end{fdefinition}
    \begin{theorem}
        \label{thm:Z_n_is_Bij_onto_Z_mod_n}%
        If $n\in\mathbb{N}^{+}$, if $\mathbb{Z}/n\mathbb{Z}$ is the ring
        of integers modulo $n$, and if the function
        $\pi:\mathbb{Z}\rightarrow\mathbb{Z}/n\mathbb{Z}$ is the
        canonical projection map: $\pi(n)=[n]$, then
        $\pi|_{\mathbb{Z}_{n}}$ is bijective.
    \end{theorem}
    \begin{proof}
        For let $x\in\mathbb{Z}/n\mathbb{Z}$. Then there exists a
        representative $k\in\mathbb{Z}$ such that $[k]=x$. But by
        Euclid's division algorithm there exists $q\in\mathbb{Z}$ and
        $r\in\mathbb{N}$ such that $r<n$ and $k=qn+r$. But then
        $qn=k-r$ and hence $n$ divides $k-r$
        (Def.~\ref{def:Divisor_of_Integer}). But then $r\in[k]$, and
        hence $\pi(r)=\pi(k)=x$. And since $r<n$, $r\in\mathbb{Z}_{n}$.
        Hence, $\pi|_{\mathbb{Z}_{n}}$ is surjective. Moreover, if
        $a,b\in\mathbb{Z}_{n}$ and if $a\ne{b}$, then either $a<b$ or
        $b<a$. Suppose $a<b$. If $\pi(a)=\pi(b)$, then $[a]=[b]$ and
        hence $n$ divides $b-a$. But since $a,b\in\mathbb{Z}_{n}$, $a<n$
        and $b<n$, and thus $b-a<n$. But since $a<b$,
        $b-a\in\mathbb{N}^{+}$. But then $n$ the greater divides $b-a$
        the lesser, a contradiction
        (Thm.~\ref{thm:Greater_is_not_divisor_of_lesser}). Hence,
        $\pi|_{\mathbb{Z}_{n}}$ is injective, and is therefore a
        bijection. 
    \end{proof}
    In the case of $n=0$ we see that $\mathbb{Z}/0\mathbb{Z}$ reduces to
    just $\mathbb{Z}$. That is, the equivalence classes are just
    $[x]=\{x\}$ and thus
    $\pi:\mathbb{Z}\rightarrow\mathbb{Z}/0\mathbb{Z}$ is a bijection. We
    now wish to endow $\mathbb{Z}/n\mathbb{Z}$ with an arithmetic. We do
    so by borrowing the arithmetic from $\mathbb{Z}$ by using Euclid's
    division algorithm. We define this as follows.
    \begin{fdefinition}{Addition Modulo $n$}{Addition_Mod_n}
        The additive operation $\tilde{+}$ on $\mathbb{Z}/n\mathbb{Z}$
        (relabelled $Z_{n}$ for brevity) for $n\in\mathbb{N}^{+}$ is the
        set:
        \begin{equation*}
            \tilde{+}=\big\{\,\big((x,y),z\big)
                \in\big(Z_{n}\times{Z}_{n}\big)\times{Z}_{n}\;\big|\;
                \exists_{a,b\in\mathbb{Z}}
                \big(x=[a],y=[b],z=[a+b]\big)\big\}
        \end{equation*}
    \end{fdefinition}
    We only used the labelling $Z_{n}$ so that the equation didn't
    run off the page, and will not use it regularly, but rather stick
    to $\mathbb{Z}/n\mathbb{Z}$. The definition seems strange, but
    recall a function $f:A\rightarrow{B}$ is a subset
    $f\subseteq{A}\times{B}$ with certain properties, and a binary
    operation on $A$ is a function $*:A\times{A}\rightarrow{A}$. Hence,
    a binary operation is a particular subset of
    $(A\times{A})\times{A}$. We are now tasked with showing the
    definition given by Def.~\ref{def:Addition_Mod_n} forms a valid well
    defined binary operation on $\mathbb{Z}/n\mathbb{Z}$. To show that
    it is a binary operation amounts to showing that for every pair of
    elements $(x,y)$ there is a unique $z$ corresponding to this.
    \begin{theorem}
        \label{thm:Mod_Addition_is_Bin_Op}%
        If $n\in\mathbb{N}^{+}$, and if $\tilde{+}$ is the additive
        operation on $\mathbb{Z}/n\mathbb{Z}$, the $\tilde{+}$ is a
        binary operation.
    \end{theorem}
    \begin{proof}
        For if $x,y\in\mathbb{Z}/n\mathbb{Z}$, then there exists
        $a,b\in\mathbb{Z}$ such that $x=[a]$ and $y=[b]$. But then
        $\big((x,y)[a+b]\big)\in\tilde{+}$
        (Def.~\ref{def:Addition_Mod_n}). If $z\in\mathbb{Z}$ is such
        that $\big((x,y),z\big)\in\tilde{+}$, then there exists
        $\alpha,\beta,\gamma\in\mathbb{Z}$ such that
        $x=[\alpha]$, $y=[\beta]$, and $z=[\alpha+\beta]$. But if
        $x=[\alpha]$ and $x=[a]$, then $n$ divides $a-\alpha$ and hence
        there is a $j\in\mathbb{Z}$ such that $jn=a-\alpha$
        (Def.~\ref{def:Divisor_of_Integer}). Similarly there is a
        $k\in\mathbb{Z}$ such that $kn=b-\beta$. But then:
        \begin{equation}
            a+b-(\alpha+\beta)=(a-\alpha)+(b-\beta)=jn+kn=(j+k)n
        \end{equation}
        and hence $n$ divides $a+b-(\alpha+\beta)$
        (Def.~\ref{def:Divisor_of_Integer}). But then
        $[a+b]=[\alpha+\beta]$, and hence for all
        $x,y\in\mathbb{Z}/n\mathbb{Z}$ there is a unique
        $z\in\mathbb{Z}/n\mathbb{Z}$ such that
        $\big((x,y),z\big)\in\tilde{+}$, and hence $\tilde{+}$ is
        function. Moreover, since the domain of $\tilde{+}$ is
        $\mathbb{Z}/n\mathbb{Z}\times\mathbb{Z}/n\mathbb{Z}$ and the
        range is $\mathbb{Z}/n\mathbb{Z}$, $\tilde{+}$ is a binary
        operation.
    \end{proof}
    Since $\mathbb{Z}/n\mathbb{Z}$ can be put into a bijection with
    $\mathbb{Z}_{n}$ for $n\in\mathbb{N}^{+}$
    (Thm.~\ref{thm:Z_n_is_Bij_onto_Z_mod_n}), and since $\tilde{+}$
    gives a binary operation on $\mathbb{Z}/n\mathbb{Z}$
    (Thm.~\ref{thm:Mod_Addition_is_Bin_Op}) it custom to pull this
    binary operation back to $\mathbb{Z}_{n}$ and relabel it simply as
    $+$. It's always poor to use the same symbol for two different
    things that are frequently used in the same context, but alas it is
    the standard. In this case it is rather justifiable since
    $[j+k]$ is simply the remainder term of $j+k$ after division by $n$.
    \begin{example}
        The most common example one comes across of modulo arithmetic is
        in $\mathbb{Z}_{12}$ since this represents a clock. If it is 11
        A.M. and you wait for 3 hours, the time will then be 2 P.M. and
        hence $11+3=2$, quite paradoxical. One might claim ``Aha! I use
        military time!'' but then we simply apply the argument to
        $\mathbb{Z}_{24}$ and ask what time is 23 hours + 3 hours? That
        answer is 2 in the morning, hence $23+3=2$. There's no mystery
        once one realizes we are simply using the arithmetic of
        $\mathbb{Z}/n\mathbb{Z}$.
    \end{example}
    With modulo addition defined, we now turn to multiplication. Similar
    to addition, we build this new operation by borrowing from our
    familiar multiplicative operation on $\mathbb{Z}$ and thus push
    down to the equivalence classes in $\mathbb{Z}/n\mathbb{Z}$.
    \begin{fdefinition}{Multiplication Modulo $n$}{Multiplication_Mod_n}
        The multiplicative operation $\tilde{\cdot}$ on
        $\mathbb{Z}/n\mathbb{Z}$
        (relabelled $Z_{n}$ for brevity) for $n\in\mathbb{N}^{+}$
        is the set:
        \begin{equation*}
            \tilde{\cdot}=\big\{\,\big((x,y),z\big)
                \in\big(Z_{n}\times{Z}_{n}\big)\times{Z}_{n}\;\big|\;
                \exists_{a,b\in\mathbb{Z}}
                \big(x=[a],y=[b],z=[a\cdot{b}]\big)\big\}
        \end{equation*}
    \end{fdefinition}
    \begin{theorem}
        \label{thm:Mult_Mod_n_is_Bin_Op}%
        If $n\in\mathbb{N}^{+}$, if $\tilde{\cdot}$ is the
        multiplicative operation on $\mathbb{Z}/n\mathbb{Z}$, then
        $\tilde{\cdot}$ is a binary operation on
        $\mathbb{Z}/n\mathbb{Z}$.
    \end{theorem}
    \begin{proof}
        For if $x,y\in\mathbb{Z}/n\mathbb{Z}$ then there exist
        $n,m\in\mathbb{Z}$ such that $x=[n]$ and $y=[m]$. But then
        $\big((x,y),[n\cdot{m}]\big)\in\tilde{\cdot}$
        (Def.~\ref{def:Multiplication_Mod_n}). If
        $\big((x,y),z\big)\in\tilde{\cdot}$, then there exists
        $\alpha,\beta\in\mathbb{Z}/n\mathbb{Z}$ such that
        $x=[\alpha]$, $y=[\beta]$, and $z=[\alpha\cdot\beta]$
        (Def.~\ref{def:Multiplication_Mod_n}). But if $x=[\alpha]$ and
        $x=[a]$, then $[a]=[\alpha]$ and hence $n$ divides $a-\alpha$.
        Similarly, $n$ divides $b-\beta$. But then there exists
        $j,k\in\mathbb{Z}$ such that $jn=a-\alpha$ and $kn=b-\beta$
        (Def.~\ref{def:Divisor_of_Integer}). But then
        $a=jn+\alpha$ and $b-kn+\beta$, hence:
        \begin{subequations}
            \begin{align}
                (a\cdot{b})-(\alpha\cdot\beta)
                    &=(\alpha+jn)(\beta+kn)-(\alpha\cdot\beta)\\
                    &=\alpha\beta+jn\beta+kn\alpha-\alpha\cdot\beta\\
                    &=jn\beta+kn\alpha\\
                    &=n(j\beta+k\alpha)
            \end{align}
        \end{subequations}
        and hence $n$ divides $a\cdot{b}-\alpha\cdot\beta$. But then
        $[a\cdot{b}]=[\alpha\cdot\beta]$ and therefore for all
        $x,y\in\mathbb{Z}/n\mathbb{Z}$ there is a unique
        $z\in\mathbb{Z}/n\mathbb{Z}$ such that
        $\big((x,y),z\big)\in\tilde{\cdot}$. Thus, $\tilde{\cdot}$ is a
        binary operation on $\mathbb{Z}/n\mathbb{Z}$.
    \end{proof}
    While we occasional use $\mathbb{Z}_{n}$, $+$, and $\cdot$ in place
    of $\mathbb{Z}/n\mathbb{Z}$, $\tilde{+}$, and $\tilde{\cdot}$, it is
    worthwhile to note that the elements of $\mathbb{Z}/n\mathbb{Z}$ are
    \textit{not} the same as the elements of $\mathbb{Z}_{n}$.
    $\mathbb{N}$ was constructed from the axiom of infinity and the
    elements look like
    $\emptyset,\{\emptyset\},\{\emptyset,\{\emptyset\}\}$, and so forth.
    Meanwhile $\mathbb{Z}/n\mathbb{Z}$ was constructed from an
    equivalence relation, and hence the elements of
    $\mathbb{Z}/n\mathbb{Z}$ are elements of the \textit{power set} of
    $\mathbb{N}$. Indeed, we know precisely what these elements are:
    \begin{equation}
        \begin{split}
            \mathbb{Z}/n\mathbb{Z}=
            \Big\{\,&\{\,0,\,\pm{n},\,\pm{2n},\,\pm{3n},\,\dots\,\},\\
                &\{\,1,\,1\pm{n},\,1\pm{2n},\,1\pm{3n},\,\dots\,\},\\
                &\{\,2,\,2\pm{n},\,2\pm{2n},\,2\pm{3n},\,\dots\,\},\\
                &\dots,\\
                &\{\,n-1,\,n-1\pm{n},\,n-1\pm{2n},\,n-1\pm{3n},\,
                    \dots\,\}\Big\}
        \end{split}
    \end{equation}
    set theoretically these are different.
    Thm.~\ref{thm:Z_n_is_Bij_onto_Z_mod_n} tells us they're the same
    size and the way we've defined modulo arithmetic mimics the notion
    division with remainder in $\mathbb{Z}_{n}$. That is,
    writing $a+b=qn+r$ with $0\leq{r}<r$, we have defined
    $[a]\tilde{+}[b]=[r]$, and similary for $a\cdot{b}=sn+t$, we have
    $[a]\tilde{\cdot}[b]=[t]$. It is in this sense that we are justified
    in using the same notation.
    \begin{example}
        A common application of modulo arithmetic that one sees in
        elementary number theory is the computation of the last few
        digits of large powers of numbers. For example, consider $3^120$
        and suppose we want to know the last digit of this number.
        That's equivalent to asking what is the smallest representative
        of the equivalence class of $3^{120}$ in the ring of integers
        modulo 10. First we reduce the problem and recognize that
        $120=2\cdot{60}$, and hence
        $3^{120}=3^{2\cdot{60}}=(3^{2})^{60}$. Well $3^{2}=9$, which
        is equivalent to $\minus{1}$ in $\mathbb{Z}/10\mathbb{Z}$, and
        so we next consider $(\minus{1})^{60}$. But this is just 1.
        But 1 is congruent to 1 in $\mathbb{Z}/10\mathbb{Z}$ and there
        we have it: The last digit of $3^{120}$ is 1. Indeed, using your
        favorite computer language, we can compute and obtain:
        \begin{equation*}
            3^{120}=
            1797010299914431210413179829509605039731475627537851106401
        \end{equation*}
        and so our calculation was correct. Let us try $2^{2000}$ and
        attain the last 2 digits (But perhaps not compute the actual
        value). We note that $2000=10*200$ and every computer scientist
        recognizes instantly that $2^{10}=1024$, which is congruent to
        24 in $\mathbb{Z}/100\mathbb{Z}$. So we are left with
        $24^{200}$. We look at the exponent again, note that it is equal
        to $2\cdot{100}$ and $24^{2}$ seems a far simpler computation.
        We get $24^{2}=576$, the last two digits of which are 76, and so
        we're down to $76^{100}$. We break this into $(76^{2})^{50}$ and
        upon computing we get $76^{2}=5776$ and so the last two digits
        are once again 76, meaning we can continue decomposing away all
        of the powers of 2, and we are left with
        $75^{25}=76^{24}\cdot{76}$. But then removing the powers of 2
        away from 24 ($24=2^{3}\cdot{3}$), we are left with
        $76^{3}\cdot{76}=76^{4}$ which then reduces to $76$. The last
        two digits of $2^{2000}$ are 76.
    \end{example}
    \begin{theorem}
        \label{thm:Equiv_Class_of_0_is_Add_Identity_Mod_n}%
        If $n\in\mathbb{N}^{+}$, if $[0]\in\mathbb{Z}/n\mathbb{Z}$ is
        the equivalence class of 0 in the ring of integers modulo $n$,
        and if $x\in\mathbb{Z}/n\mathbb{Z}$, then $x\tilde{+}[0]=x$.
    \end{theorem}
    \begin{proof}
        For if $x\in\mathbb{Z}/n\mathbb{Z}$ then there is a
        $k\in\mathbb{Z}$ such that $x=[k]$. But then:
        \begin{equation}
            x\tilde{+}[0]=[k]\tilde{+}[0]=[k+0]=[k]=x
        \end{equation}
    \end{proof}
    \begin{theorem}
        \label{thm:Mod_Add_is_Assoc}%
        If $n\in\mathbb{N}^{+}$, if $x,y,z\in\mathbb{Z}/n\mathbb{Z}$,
        then $(x\tilde{+}y)\tilde{+}z=x\tilde{+}(y\tilde{+}z)$.
    \end{theorem}
    \begin{proof}
        For if $x,y,z\in\mathbb{Z}/n\mathbb{Z}$, then there exists
        $i,j,k\in\mathbb{Z}$ such that $x=[i]$, $y=[j]$, and $z=[k]$.
        But then:
        \par
        \begin{subequations}
            \begin{minipage}[t]{0.54\textwidth}
                \centering
                \begin{align}
                    (x\tilde{+}y)\tilde{+}z
                    &=([i]\tilde{+}[j])\tilde{+}[k]
                        \tag{Hypothesis}\\
                    &=[i+j]\tilde{+}[k]
                        \tag{Def.~\ref{def:Addition_Mod_n}}\\
                    &=[(i+j)+k]
                        \tag{Def.~\ref{def:Addition_Mod_n}}\\
                    &=[i+(j+k)]
                        \tag{Associativity}
                \end{align}
            \end{minipage}
            \hfill
            \begin{minipage}[t]{0.44\textwidth}
                \begin{align}
                    &=[i]\tilde{+}[j+l]
                        \tag{Def.~\ref{def:Addition_Mod_n}}\\
                    &=[i]\tilde{+}([j]\tilde{+}[k])
                        \tag{Def.~\ref{def:Addition_Mod_n}}\\
                    &=x\tilde{+}(y\tilde{+}z)
                        \tag{Hypothesis}
                \end{align}
            \end{minipage}
        \end{subequations}
        \par\vspace{2.5ex}
        proving the claim.
    \end{proof}
    \begin{theorem}
        \label{thm:Additive_Inv_Mod_n}%
        If $n\in\mathbb{N}^{+}$, and if $x\in\mathbb{Z}/n\mathbb{Z}$,
        then there is a $y\in\mathbb{Z}/n\mathbb{Z}$ such that
        $x\tilde{+}y=[0]$.
    \end{theorem}
    \begin{proof}
        For if $x\in\mathbb{Z}/n\mathbb{Z}$, then there is a
        $k\in\mathbb{Z}$ such that $x=[k]$. Let $y=[\minus{k}]$. But
        then:
        \begin{equation}
            x\tilde{+}y=[k]\tilde{+}[\minus{k}]
            =[k+(\minus{k})]=[0]
        \end{equation}
        proving the claim.
    \end{proof}
    \begin{theorem}
        \label{thm:Equiv_Class_of_1_is_Mult_Identity_Mod_n}%
        If $n\in\mathbb{N}^{+}$, if $[1]\in\mathbb{Z}/n\mathbb{Z}$ is
        the equivalence class of 1 in the ring of integers modulo $n$,
        and if $x\in\mathbb{Z}/n\mathbb{Z}$, then $x\tilde{\cdot}[1]=x$.
    \end{theorem}
    \begin{proof}
        For if $x\in\mathbb{Z}/n\mathbb{Z}$, then there is a
        $k\in\mathbb{Z}$ such that $x=[k]$. But then:
        \begin{equation}
            x\tilde{\cdot}[1]=[k]\tilde{\cdot}[1]=[k\cdot{1}]=[k]=x
        \end{equation}
    \end{proof}
    \begin{theorem}
        \label{thm:Mod_Mult_is_Assoc}%
        If $n\in\mathbb{N}^{+}$, if $x,y,z\in\mathbb{Z}/n\mathbb{Z}$,
        then $(x\tilde{\cdot}y)\tilde{\cdot}z%
        =x\tilde{\cdot}(y\tilde{\cdot}z)$.
    \end{theorem}
    \begin{proof}
        For if $x,y,z\in\mathbb{Z}/n\mathbb{Z}$, then there exists
        $i,j,k\in\mathbb{Z}$ such that $x=[i]$, $y=[j]$, and $z=[k]$.
        But then:
        \par
        \begin{subequations}
            \begin{minipage}[t]{0.54\textwidth}
                \centering
                \begin{align}
                    (x\tilde{\cdot}y)\tilde{\cdot}z
                    &=([i]\tilde{\cdot}[j])\tilde{\cdot}[k]
                        \tag{Hypothesis}\\
                    &=[i\cdot{j}]\tilde{\cdot}[k]
                        \tag{Def.~\ref{def:Multiplication_Mod_n}}\\
                    &=[(i\cdot{j})\cdot{k}]
                        \tag{Def.~\ref{def:Multiplication_Mod_n}}\\
                    &=[i\cdot(j\cdot{k})]
                        \tag{Associativity}
                \end{align}
            \end{minipage}
            \hfill
            \begin{minipage}[t]{0.44\textwidth}
                \begin{align}
                    &=[i]\tilde{\cdot}[j\cdot{l}]
                        \tag{Def.~\ref{def:Multiplication_Mod_n}}\\
                    &=[i]\tilde{\cdot}([j]\tilde{\cdot}[k])
                        \tag{Def.~\ref{def:Multiplication_Mod_n}}\\
                    &=x\tilde{\cdot}(y\tilde{\cdot}z)
                        \tag{Hypothesis}
                \end{align}
            \end{minipage}
        \end{subequations}
        \par\vspace{2.5ex}
        proving the claim.
    \end{proof}
    \begin{theorem}
        \label{thm:Invertible_Mod_n_iff_Relatively_Prime}%
        If $n\in\mathbb{N}^{+}$, if $x\in\mathbb{Z}/n\mathbb{Z}$, then
        there exists a $y\in\mathbb{Z}/n\mathbb{Z}$ such that
        $x\tilde{\cdot}{y}=[1]$ if and only if there is a
        $k\in\mathbb{Z}$ such that $x=[k]$ and $\GCD(k,n)=1$.
    \end{theorem}
    \begin{proof}
        For if $k$ is a representative for $x$ and $\GCD(k,n)=1$, then
        by B\'{e}zout's identity there exist $a,b\in\mathbb{Z}$ such
        that $a\cdot{k}+b\cdot{n}=1$ (Thm.~\ref{thm:Bezout_Identity}).
        But then $b\cdot{n}=1-a\cdot{k}$ and hence $n$ divides
        $1-a\cdot{k}$ (Def.~\ref{def:Divisor_of_Integer}). But then by
        the definition of the ring of integers modulo $n$,
        $1\in[a\cdot{k}]$ (Def.~\ref{def:Ring_Ints_Mod_N} and
        hence $[a]\tilde{\cdot}[k]=[1]$. In the other direction, if
        $x$ has an inverse $y$, then there are representatives $k,m$
        such that $x=[k]$ and $y=[m]$. But then
        $x\tilde{\cdot}{y}=[k]\tilde{\cdot}[m]=[k\cdot{m}]$. But by
        hypothesis $x\tilde{\cdot}{y}=[1]$ and hence $[k\cdot{m}]=[1]$.
        But then $n$ divides $1-km$ and hence there is a
        $j\in\mathbb{Z}$ such that $jn=1-km$
        (Def.~\ref{def:Divisor_of_Integer}). But then $nj+km=1$ and
        therefore $\GCD(k,n)=1$.
    \end{proof}
    Hence every non-zero element of $\mathbb{Z}/p\mathbb{Z}$ for some prime
    $p$ is invertible.
    \begin{example}
        Other tricks that use modulo arithmetic appear in disguise when one
        studies divisibility tricks. If $a=\sum{a}_{n}10^{n}$ is a finite
        sum, then $a$ is congruent of $\sum{a}_{n}$. Simply use the
        additivity of modulo addition, and use the fact that
        $10^{n}\equiv{1}\mod{9}$. Another trick comes from studying if a
        number is divisible by 3. Applying the same trick, we see that if
        3 divides $n$, then it divides the sum of its digits (in base 10).
        Conversely, if 3 divides the sum of the digits of $n$ (in base 10),
        then 3 divides $n$.
    \end{example}
    \begin{example}
        Looking back at a previous example, let's compute the last digit of
        $9^{n}$ for any $n\in\mathbb{N}$. We note that
        $9\equiv\minus{1}\mod{10}$ and hence we only need to consider
        $(\minus{1})^{n}$. But $(\minus{1})^{n}$ is 1 if $n$ is even and
        $\minus{1}$ is $n$ is odd. Therefore the last digit of $9^{n}$ is
        9 if $n$ is odd and 1 if $n$ is even. And indeed, the pattern holds
        true: 1, 9, 81, 729, 6561, and so on.
    \end{example}
    Come back to excercises (DF Chapt 1).