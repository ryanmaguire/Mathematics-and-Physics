%------------------------------------------------------------------------------%
\begingroup
    \ifcsname\PATH\endcsname
        \newcommand{\PATH}{books/Foundations/Logic}
        \newcommand{\OLDPATH}{\PATH}
    \else
        \newcommand{\OLDPATH}{\PATH}
        \renewcommand{\PATH}{books/Foundations/Logic}
    \fi
    \chapter{Propositional Logic}
        \label{chapt:Propositional_Logic}%
        We'll begin our discussion of logic with the most primitive kind:
        propositional logic\index{Logic!Propositional}. This form, also called
        \textit{sentential logic}\index{Logic!Sentential}, deals with the
        structure of sentences. That is, how do we use the English language to
        formulate arguments and how can various arguments be used to deduce new
        facts. It is for this reason that one should start with the foundations
        of logic, for at the heart of mathematics is the concept of
        \textit{definition}, \textit{theorem}, and \textit{proof}. The word
        definition, it is hoped, is understood as a primitive word in the
        English language (much like the word \textit{the}), but theorem and
        proof need an explaination if we are to consistently use the notions.
        \par\hfill\par
        We'll build logic rather rapidly, starting with the concept of a
        \textit{proposition}\index{Proposition}, which will be
        another one of our primitive words. The development here will be self
        contained, but rather brief. Unlike other parts of this work, which
        contain theorems, examples, and discussions not for their usefulness,
        but for the sake of mathematics, this part on foundational logic will
        primarily develop what is needed to proceed on to set theory. This is
        not to say that mathematical logic\index{Mathematical Logic} will have
        no further part, on the contrary we will develop
        \textit{Boolean Algebras}\index{Boolean Algebra} later in
        Book~\ref{book:Foundations}, and in Book~\ref{book:Topology} we will
        discuss \textit{Stone Spaces}\index{Stone Space} which are integral to
        one of the most fundamental theorems of mathematics: The
        \textit{Stone Representation Theorem}%
        \index{Theorem!Stone Representation Theorem}%
        \index{Stone Representation Theorem}, connecting logic, set theory,
        topology, and measure theory in a remarkable way. All that is meant is,
        in comparison to other areas, logic will be the most underdeveloped and
        the curious should consult the plethora of texts available.
        \section{What is Logic?}
    It may seem strange to start a work on mathematics with an entire
    development of logic, as one might think such conversations should reside
    in philosophy. And indeed, much of classical logic was developed by
    philosophers, rather than mathematicians. Many problems, which we will
    discuss in Chapt.~\ref{chapt:Zermelo_Fraenkel_Set_Theory}, arose in the
    early 1900s with the very core of mathematics. Arguments once considered
    sound were shattered, and contradictions were discovered. On the other hand,
    other methods of proof that are very intuitive were shown to be able to
    prove the existence of non-intuitive and almost impossible objects.
    \begin{example}
        A student of calculus most likely knows well the
        \textit{intermediate value theorem}%
        \index{Theorem!Intermediate Value Theorem}%
        \index{Intermediate Value Theorem}. To those who don't, fear not, we
        shall draw pictures. Given a \textit{continuous} function $f$ of real
        numbers (roughly speaking, a curve one can draw from left to right
        without lifting up ones pencil), if $0$ evaluates to a negative number
        and $1$ evaluates to a positive number, then there is some number in the
        middle which evaluates to zero. The method of proof is quite simple:
        We first look at what happens at the point $\frac{1}{2}$. If $f$
        is zero at this point, we are done and the theorem is proved, otherwise
        if $f$ evaluates to a positive number then we may suspect there's a
        point in between 0 and $\frac{1}{2}$ that evaluates to zero. If $f$ is
        negative at the point, then there's probably a zero in between
        $\frac{1}{2}$ and 1. In either case, we divide the range of
        possibilities in half once again and see what happens at $\frac{1}{4}$
        in the first case, and $\frac{3}{4}$ in the latter. We continue
        \textit{inductively} (whatever this means) and obtain a
        \textit{sequence} of real numbers which we then show
        \textit{converge} to some real number between 0 and 1. We then invoke
        continuity to show that $f$ evaluates to zero at this point, and we are
        done (See Fig.~\ref{fig:Sketch_of_IVP}).
        \begin{figure}[H]
            \centering
            \captionsetup{type=figure}
            \begin{tikzpicture}[>=Latex, scale=7]
    \draw[->, thick] (-0.1, 0.0) to (1.1, 0.0) node[below] {$x$};
    \draw[->, thick] ( 0,  -0.4) to (0.0, 0.5) node[right] {$y$};

    \coordinate (z0) at (0.000000, -0.100000);
    \coordinate (z1) at (1.000000,  0.101170);
    \coordinate (z2) at (0.500000, -0.100000);
    \coordinate (z3) at (0.750000, -0.147061);
    \coordinate (z4) at (0.870000,  0.222959);
    \coordinate (z5) at (0.810000,  0.056956);

    \coordinate (x0) at (0.00, 0.0);
    \coordinate (x1) at (1.00, 0.0);

    \coordinate (y0) at (0.00, 0.3);
    \coordinate (y1) at (1.00, 0.3);
    \coordinate (y2) at (0.50, 0.3);
    \coordinate (y3) at (0.75, 0.3);
    \coordinate (y4) at (0.87, 0.3);
    \coordinate (y5) at (0.81, 0.3);

    \draw[semithick]
        (0.000000, -0.100000) to (0.010000, -0.190882) to
        (0.020000, -0.134156) to (0.030000, -0.298796) to
        (0.040000, -0.161042) to (0.050000, -0.190845) to
        (0.060000, -0.151818) to (0.070000, -0.105578) to
        (0.080000, -0.294114) to (0.090000, -0.221376) to
        (0.100000, -0.348381) to (0.110000, -0.185051) to
        (0.120000, -0.137027) to (0.130000, -0.071993) to
        (0.140000, -0.083851) to (0.150000, -0.131908) to
        (0.160000, -0.207415) to (0.170000, -0.155534) to
        (0.180000, -0.225251) to (0.190000, -0.111510) to
        (0.200000, -0.160976) to (0.210000, -0.098903) to
        (0.220000, -0.066714) to (0.230000, -0.099304) to
        (0.240000, -0.037003) to (0.250000, -0.116941) to
        (0.260000, -0.065740) to (0.270000, -0.046159) to
        (0.280000, -0.029719) to (0.290000,  0.037735) to
        (0.300000, -0.048490) to (0.310000, -0.048847) to
        (0.320000, -0.120539) to (0.330000, -0.135940) to
        (0.340000, -0.088400) to (0.350000, -0.117398) to
        (0.360000, -0.054753) to (0.370000, -0.110149) to
        (0.380000, -0.090130) to (0.390000, -0.115973) to
        (0.400000, -0.117710) to (0.410000, -0.118431) to
        (0.420000, -0.120026) to (0.430000, -0.119288) to
        (0.440000, -0.103887) to (0.450000, -0.111766) to
        (0.460000, -0.102198) to (0.470000, -0.113391) to
        (0.480000, -0.110579) to (0.490000, -0.105565) to
        (0.500000, -0.100000) to (0.510000, -0.098346) to
        (0.520000, -0.093914) to (0.530000, -0.093459) to
        (0.540000, -0.083951) to (0.550000, -0.068383) to
        (0.560000, -0.079468) to (0.570000, -0.066709) to
        (0.580000, -0.095622) to (0.590000, -0.096420) to
        (0.600000, -0.097357) to (0.610000, -0.127241) to
        (0.620000, -0.082486) to (0.630000, -0.123949) to
        (0.640000, -0.055269) to (0.650000, -0.084457) to
        (0.660000, -0.071498) to (0.670000, -0.097302) to
        (0.680000, -0.125052) to (0.690000, -0.106186) to
        (0.700000, -0.072916) to (0.710000, -0.062225) to
        (0.720000, -0.025907) to (0.730000, -0.111287) to
        (0.740000, -0.108351) to (0.750000, -0.147061) to
        (0.760000, -0.121475) to (0.770000, -0.033252) to
        (0.780000, -0.037532) to (0.790000,  0.056384) to
        (0.800000,  0.041977) to (0.810000,  0.056956) to
        (0.820000,  0.100997) to (0.830000,  0.010700) to
        (0.840000,  0.091665) to (0.850000,  0.039524) to
        (0.860000,  0.121069) to (0.870000,  0.222959) to
        (0.880000,  0.132402) to (0.890000,  0.244225) to
        (0.900000, -0.009427) to (0.910000,  0.111595) to
        (0.920000,  0.029857) to (0.930000,  0.159189) to
        (0.940000,  0.224706) to (0.950000,  0.176094) to
        (0.960000,  0.165249) to (0.970000,  0.083951) to
        (0.980000,  0.032372) to (0.990000,  0.114170) to
        (1.000000,  0.101170);

        \foreach\i in {1, 2, 3, 4, 5}{
            \draw[fill=black] (z\i) circle (0.07mm);
            \draw[densely dashed] (z\i) to (y\i) node[above] {$x_{\i}$};
        }
        \draw[densely dashed] (z0) to (y0) node [above left] {$x_{0}$};

        \draw[fill=black] (z0) circle (0.07mm);
\end{tikzpicture}
            \caption{Sketch of the Intermediate Value Theorem}
            \label{fig:Sketch_of_IVP}
        \end{figure}
    \end{example}
    We can see why this may work. After a few iterations we've narrowed down the
    zero point to a very small range between $x_{3}$ and $x_{5}$, and this is a
    nice algorithm that we can tell a computer to execute to arbitrary
    precision, but what went into the proof? That is, if we were to phrase this
    with absolute precision, what definitions, assumptions, and previous
    theorems are we relying on? For one, the existence of \textit{real number},
    a notion of \textit{continuity}, and the definition of a \textit{sequence}.
    Our exposition of logic is to make clear what is required for valid proofs.
    \subsection{Misc}
        \begin{table}[H]
            \centering
            \captionsetup{type=table}
            \begin{tabular}{c c c c c c}
                \hline
                $p$&$q$&$r$&$\neg{q}$&$p\lor\neg{q}$&$(p\lor\neg{q})\land{r}$\\
                \hline
                0&0&0&1&1&0\\
                0&0&1&1&1&1\\
                0&1&0&0&0&0\\
                0&1&1&0&0&0\\
                1&0&0&1&1&0\\
                1&0&1&1&1&1\\
                1&1&0&0&1&0\\
                1&1&1&0&1&1\\
                \hline
            \end{tabular}
            \caption{Truth Table for $(p\lor\neg{q})\land{r}$}
            \label{tab:Truth_Table_Example}
        \end{table}
        \begin{theorem}
            If $a\rightarrow{b}$, if $\neg{c}\rightarrow\neg{b}$, and if
            $\neg{c}$, then $\neg{a}$.
        \end{theorem}
        \begin{proof}
            For if $a\rightarrow{b}$, then $\neg{b}\rightarrow\neg{a}$. But
            $\neg{c}\rightarrow\neg{b}$. But if $\neg{c}\rightarrow\neg{b}$ and
            $\neg{b}\rightarrow\neg{a}$, then $\neg{c}\rightarrow\neg{a}$. Thus
            $a\rightarrow{b}$, $\neg{c}\rightarrow\neg{b}$, and thus
            $\neg{c}\Rightarrow\neg{a}$.
        \end{proof}
        \begin{problem}
            If $a\rightarrow{b}$, if $\neg{c}\rightarrow\neg{b}$, and if
            $\neg{c}$, then $\neg{a}$.
        \end{problem}
        \begin{proof}
            For if $\neg c \rightarrow \neg b$, then $b\rightarrow c$. But if
            $a\rightarrow b$ and $b\rightarrow c$, then $a\rightarrow c$.
            Therefore $a\rightarrow c$. But if $a\rightarrow c$, then
            $\neg c \rightarrow \neg a$. Therefore,
            $a\rightarrow b, \neg c \rightarrow \neg b, \neg c \Rightarrow \neg a$.
        \end{proof}
    \chapter{Connectives}
        \label{chapt:Connectives}%
        \section{Connectives}
    In the set theory we will be working with there are a few words and symbols
    that are left undefined. This is certainly unavoidable since defining all
    words would ultimately be circular, and thus we try to leave the fewest
    number of things undefined as possible. Everything else is then built from
    this collection of notions that we agree are obvious from an intuitive point
    of view. Our first undefined symbol is $\in$ (\textit{containment}). This is
    a type of \textit{\gls{predicate}}\index{Predicate}.
    \begin{fdefinition}{Predicate}{Predicate}
        A \gls{predicate} is a statement that takes in some parameter or
        variable $x$ and returns either \textit{True} or \textit{True}.
    \end{fdefinition}
    We have relied on the word \textit{statement} being already defined, and
    similarly for the words \textit{parameter} or \textit{variable}. For most
    this is not an issue, but it may irk others. From our undefined symbol $\in$
    we build new symbols by expressing them in terms of a
    \textit{formula}\index{Formula}, which is simply a finite
    sequence of symbols. Here the word \textit{sequence} is meant to imply that
    the \textit{order} in which we combine these symbols is important and that
    rearranging said order may create a different inequivalent formula. We build
    formulas by defining a few symbols that stand as placeholders for standard
    words in English. There are four symbols, called
    \textit{\glspl{connective}}\index{Connective (Logic)}, that we use.
    \begin{align*}
        a\land{b}\quad
        &\textrm{True if and only if }a
        \textrm{ is true and }b\textrm{ is true}
        \tag{Conjunction}\\
        a\lor{b}\quad
        &\textrm{True if and only if }a
        \textrm{ is true or }b\textrm{ is true, or both}
        \tag{Disjunction}\\
        \neg{a}\quad
        &\textrm{True if and only if }a\textrm{ is false}
        \tag{Negation}\\
        a\Rightarrow{b}\quad
        &a\textrm{ is true implies that }b\textrm{ is true}
        \tag{Implication}
    \end{align*}
    From this we see that we have introduced 6 new words that are undefined but
    require comment. The words are \textit{and, or, if, then, true}, and
    \textit{false}. There are other symbols we could adopt, such as
    \textit{equivalence}:
    \begin{equation*}
        a\Leftrightarrow{b}
    \end{equation*}
    But from how we shall define these notions, this new symbol is equivalent to
    a combination of the previous ones:
    \begin{equation*}
        \Big(\big(a\Leftrightarrow{b}\big)\Rightarrow
             \big((a\Rightarrow{b})\land(b\Rightarrow{a})\big)\Big)
        \land\Big(\big((a\Rightarrow{b})\land(b\Rightarrow{A})\big)
            \Rightarrow\big(a\Leftrightarrow{b}\big)\Big)
    \end{equation*}
    That is, $a\Leftrightarrow{b}$ if and only if $a$ is true if and only if $b$
    is true. Similarly, we could define \textit{does not imply}:
    \begin{equation*}
        a\not\Rightarrow{b}
    \end{equation*}
    But this is the same as:
    \begin{equation*}
        a\not\Rightarrow{b}\Longleftrightarrow
        \neg(a\Rightarrow{b})
        \Longleftrightarrow
        a\land\neg{b}
    \end{equation*}
    The words true and false are assumed to be well defined. They are also
    assumed to be opposites of each other (which we will define in terms of
    negation). We will use truth tables\index{Truth Table} to define what
    various connectives mean when it is known that certain propositions are true
    or false. In such tables the symbol 0 represents that a proposition is false
    and the symbol 1 represents truth.
    \par\hfill\par
    The other four words can be ambiguous in their everyday usage which we
    cannot allow for in mathematics. As such we must specify what we mean when
    we use these words and rid of any such ambiguity.
    \subsection{Conjunction}
        The conjunction connective is the $\land$ symbol, which denotes the word
        \textit{and}. Given two propositions $P$ and $Q$, $P\land{Q}$ is a true
        statement if and only if both $P$ and $Q$ are true. That is, we
        associate to $\land$ the following truth table:
        \begin{table}[H]
            \centering
            \captionsetup{type=table}
            \begin{tabular}{ccc}
                $P$&$Q$&$P\land{Q}$\\
                \hline
                0&0&0\\
                0&1&0\\
                1&0&0\\
                1&1&1
            \end{tabular}
            \caption{Truth Table for Conjunction}
            \label{tab:Truth_Table_for_Conjunction}
        \end{table}
        There are several \textit{axioms} of conjunctions that are intuitively
        obvious, but must be stated since their use is wide spread.
        \begin{faxiom}{Axioms of Conjunction}{Axioms_of_Conjunction}
            If $P$ and $Q$ are propositions, then the following are true:
            \begin{align}
                P\land{Q}&\Longleftrightarrow{Q}\land{P}
                \tag{Commutativity of Conjunction}\\
                P\land\textrm{True}&\Longleftrightarrow\textrm{P}
                \tag{Identity of Conjunction}
            \end{align}
        \end{faxiom}
    \subsection{Disjunction}
        The disjunction connective is the $\lor$ symbol which represents
        \textit{or}. Given two propositions $P$ and $Q$, $P\lor{Q}$ is true if
        and only if $P$ is true, or $Q$ is true, or both $P$ and $Q$ is true.
        There is an unfortunate ambiguity in English as to whether $P$ or $Q$
        means $P$ is true or $Q$ is true, but not both, or whether it means
        $P$ is true or $Q$ is true, or \textit{both} are true. The convention is
        to adopt the latter definition. That is, $P\lor{Q}$ has the following
        truth table:
        \begin{table}[H]
            \centering
            \captionsetup{type=table}
            \begin{tabular}{ccc}
                $P$&$Q$&$P\lor{Q}$\\
                \hline
                0&0&0\\
                0&1&1\\
                1&0&1\\
                1&1&1
            \end{tabular}
            \caption{Truth Table for Disjunction}
            \label{tab:Truth_Table_for_Disjunction}
        \end{table}
        There is another connective called the \textit{exlusive} or, which is
        defined to be false if both $P$ and $Q$ are true. The symbol $\lor$ is
        strictly used to denote the inclusive or. That is, the word or as
        represented by the truth table in
        Tab.~\ref{tab:Truth_Table_for_Disjunction}.
    \subsection{Implication}
        The implication connective is denoted by $P\Rightarrow{Q}$. This is read
        as \textit{if P, then Q} for two given propositions $P$ and $Q$. Here,
        $P$ is called the \textit{hypothesis} and $Q$ is called the
        \textit{conclusion}. There is potential ambiguity as to the meaning of
        if-then statements. If $P$ is a proposition that is false, and $Q$ is a
        propositions that is true, should $P\Rightarrow{Q}$ be a true statement
        or a false one? The convention is to take this as true. That is, we
        adopt the following truth table.
        \begin{table}[H]
            \centering
            \captionsetup{type=table}
            \begin{tabular}{ccc}
                $P$&$Q$&$P\Rightarrow{Q}$\\
                \hline
                0&0&1\\
                0&1&1\\
                1&0&0\\
                1&1&1
            \end{tabular}
            \caption{Truth Table for Implication}
            \label{tab:Truth_Table_for_Implication}
        \end{table}
        The converse of the statement $P\Rightarrow{Q}$ is the statement
        $Q\Rightarrow{P}$. The validity of the converse is not implied by the
        truth or falsehood of $P\Rightarrow{Q}$, and we need only look at the
        truth table to see this.
        \begin{table}[H]
            \centering
            \captionsetup{type=table}
            \begin{tabular}{cccc}
                $P$&$Q$&$P\Rightarrow{Q}$&$Q\Rightarrow{P}$\\
                \hline
                0&0&1&1\\
                0&1&1&0\\
                1&0&0&1\\
                1&1&1&1
            \end{tabular}
            \caption{Truth Table for the Converse}
            \label{tab:Truth_Table_for_Converse}
        \end{table}
        Examining, we see that there are scenarios where $P\Rightarrow{Q}$
        is true and $Q\Rightarrow{P}$ is false, and similarly where
        $P\Rightarrow{Q}$ is false and $Q\Rightarrow{P}$ is true. Propositions
        $P$ and $Q$ such that $P\Rightarrow{Q}$ and $Q\Rightarrow{P}$ are called
        \textit{equivalent}, and great deal of mathematics is devoted to the
        search for equivalencies of statements. This is denoted by the
        connective $P\Leftrightarrow{Q}$. Equivalence has the following truth
        table:
        \begin{table}[H]
            \centering
            \captionsetup{type=table}
            \begin{tabular}{cccccc}
                $P$&$Q$&$P\Rightarrow{Q}$&$Q\Rightarrow{P}$
                   &$P\Leftrightarrow{Q}$
                   &$(P\Rightarrow{Q})\land(Q\Rightarrow{P})$\\
                \hline
                0&0&1&1&1&1\\
                0&1&1&0&0&0\\
                1&0&0&1&0&0\\
                1&1&1&1&1&1
            \end{tabular}
            \caption{Truth Table for Equivalence}
            \label{tab:Truth_Table_for_Equivalence}
        \end{table}
    \subsection{Negation}
        Negation is the first connective we come across that is a \textit{unary}
        operation. That is, it only takes in one proposition rather than two.
        Given a proposition $P$, $\neg{P}$ is true if and only if $P$ if false.
        That is, we have the following truth table.
        \begin{table}[H]
            \centering
            \captionsetup{type=table}
            \begin{tabular}{cc}
                $P$&$\neg{P}$\\
                \hline
                0&1\\
                1&0
            \end{tabular}
            \caption{Truth Table for Negation}
            \label{tab:Truth_Table_for_Negation}
        \end{table}
        The negation connective allows us to define the
        \textit{contrapositive}\index{Contrapositive} of the proposition
        $P\Rightarrow{Q}$, which is the new proposition
        $\neg{Q}\Rightarrow\neg{P}$. As it turns out, this is not a new
        proposition at all and is equivalent to $P\Rightarrow{Q}$. To see this,
        note that $P\Rightarrow{Q}$ is only false when $P$ is true, yet $Q$ is
        false. Similarly, $\neg{Q}\Rightarrow\neg{P}$ is only false when
        $\neg{Q}$ is true and $\neg{P}$ is false. But if $\neg{Q}$ is true, then
        $Q$ is false, and if $\neg{P}$ is false, then $P$ is true. Thus
        $\neg{Q}\Rightarrow\neg{P}$ is only false when $P$ is true and $Q$ is
        false. We can further examine this by the truth tables of these
        statements.
        \begin{table}[H]
            \centering
            \captionsetup{type=table}
            \begin{tabular}{cccccc}
                $P$&$Q$&$\neg{P}$&$\neg{Q}$
                    &$P\Rightarrow{Q}$&$\neg{Q}\Rightarrow\neg{P}$\\
                \hline
                0&0&1&1&1&1\\
                0&1&1&0&1&1\\
                1&0&0&1&0&0\\
                1&1&0&0&1&1
            \end{tabular}
            \caption{Truth Table for the Contrapositive}
            \label{tab:Truth_Table_for_Contrapositive}
        \end{table}
        \begin{example}
            Suppose $a$ and $b$ are variables representing real numbers and $P$
            is the proposition $a<1/2$ and $b<1/2$, and let $Q$ be the
            proposition $a+b<1$. What is the contrapositive of
            $P\Rightarrow{Q}$? This would be $\neg{Q}\Rightarrow\neg{P}$, and
            $\neg{Q}$ is the negation of $Q$, which reads $a+b\geq{1}$.
            Similarly, $\neg{P}$ is the statement $a\geq{1}/2$ or $b\geq{1}/2$.
            Thus, the contrapositive says that if $a+b\geq{1}$, then either
            $a\geq{1}/2$ or $b\geq{1}/2$ (or both). While the contrapositive of
            a statement is always equivalent to the original statement, the
            converse need not be. Indeed, this statement is true (once one knows
            the order structure of real numbers), but the converse is not. The
            converse states that if $a+b<1$, then $a<1/2$ and $b<1/2$, but
            letting $a=2$ and $b=\minus{3}$ contradicts this claim.
        \end{example}
    \chapter{Predicate Calculus}
        \label{chapt:Predicate_Calculus}
        \section{Quantifiers}
    There are two more symbols called
    \textit{\glspl{quantifier}}\index{Quantifier}.
    \begin{equation*}
        \forall_{x}\quad\textrm{For all }x
        \quad\quad\quad\quad
        \exists_{x}\quad\textrm{There exists }x
    \end{equation*}
    Quantifiers, together with connectives, the word \textit{set}, and the
    $\in$ symbol are combined to define new terms and new symbols. The rest
    of mathematics rests on trusting ones intuition behind these notions.
    \begin{example}
        Let $\mathbb{R}$ denote the set of real numbers. The symbols
        $\forall_{R\in\mathbb{R}}(n^{2}\geq{0})$ can then be read in English
        as \textit{For all real numbers x, the square of x is non-negative},
        which is indeed a true statement. We can combine quantifiers to
        create more complicated statements, such as:
        \begin{equation}
            \forall_{x\in\mathbb{R}}(x\ne{0})\exists_{y\in\mathbb{R}}(xy=1)
        \end{equation}
        This reads that for all non-zero real numbers $x$, there exists a
        real numbers $y$ such that the product $xy$ is equal to 1. This is
        also a true statement.
    \end{example}
    \begin{example}
        The order of quantifiers is very important and often can not be
        interchanged. Considering the previous example, if we switch the
        order of the quantifiers we get:
        \begin{equation}
            \exists_{y\in\mathbb{R}}(xy=1)\forall_{x\in\mathbb{R}}(x\ne{0})
        \end{equation}
        This states that there exists a real number $y$ such that, for every
        non-zero real number $x$, it is true that $xy=1$. But this is
        certainly not true because if $x=1$ and $z=\minus{1}$, we obtain
        $(1)y=1$ and $(\minus{1})y=1$, and from this we conclude that
        $\minus{1}=1$, which is false. Hence, the order of the quantifiers
        is important.
    \end{example}
    \begin{example}
        Quantifiers can be combined with connectives to make longer and more
        complicated statements. For example, suppose $P$ is the proposition
        \textit{true if n is an even integer, false otherwise}. Furthermore,
        let $Q$ be the proposition \textit{true if n is a square integer},
        \textit{false otherwise}. Lastly, let $r$ be the proposition
        \textit{true if n is divisible by 4, false otherwise}.
        Consider then the following statement:
        \begin{equation}
            \forall_{n\in\mathbb{Z}}(p(n)\land{q(n)}\Rightarrow{r}(n))
        \end{equation}
        This reads in English as \textit{for all integers n, if n is an}
        \textit{integer, and if n is a square, then n is divisible by 4}.
    \end{example}
    \subsection{Negating Quantifiers}
        The negation of the statement \textit{for all x, P(x) is true} implies
        this is false. Thus there must exist one $x$ such that $P(x)$ is false,
        and from this we see that negating the $\forall$ quantifier produces the
        $\exists$ quantifier.
        \begin{example}
            Let $P$ be the proposition \textit{true if $x^{2}=2$} and consider
            the following statement:
            \begin{equation}
                \exists_{x\in\mathbb{Q}}\big(P(x)\big)
            \end{equation}
            This reads in plain English as the statement \textit{there exists a}
            \textit{rational number x whose square is equal to 2}. This has been
            known to be false since the ancient Greeks, and thus it's negation
            is true. We can write the negation as follows:
            \begin{equation}
                \neg\Big(\exists_{x\in\mathbb{Q}}\big(p(x)\big)\Big)
                \Longleftrightarrow\forall_{x\in\mathbb{Q}}\big(\neg{P}(x)\big)
            \end{equation}
            This now says that for all rational numbers $x$, the square of $x$
            is not equal to 2.
        \end{example}
    \renewcommand{\PATH}{\OLDPATH}
\endgroup