\section{What is Logic?}
    It may seem strange to start a work on mathematics with an entire
    development of logic, as one might think such conversations should reside
    in philosophy. And indeed, much of classical logic was developed by
    philosophers, rather than mathematicians. Many problems, which we will
    discuss in Chapt.~\ref{chapt:Zermelo_Fraenkel_Set_Theory}, arose in the
    early 1900s with the very core of mathematics. Arguments once considered
    sound were shattered, and contradictions were discovered. On the other hand,
    other methods of proof that are very intuitive were shown to be able to
    prove the existence of non-intuitive and almost impossible objects.
    \begin{example}
        A student of calculus most likely knows well the
        \textit{intermediate value theorem}%
        \index{Theorem!Intermediate Value Theorem}%
        \index{Intermediate Value Theorem}. To those who don't, fear not, we
        shall draw pictures. Given a \textit{continuous} function $f$ of real
        numbers (roughly speaking, a curve one can draw from left to right
        without lifting up ones pencil), if $0$ evaluates to a negative number
        and $1$ evaluates to a positive number, then there is some number in the
        middle which evaluates to zero. The method of proof is quite simple:
        We first look at what happens at the point $\frac{1}{2}$. If $f$
        is zero at this point, we are done and the theorem is proved, otherwise
        if $f$ evaluates to a positive number then we may suspect there's a
        point in between 0 and $\frac{1}{2}$ that evaluates to zero. If $f$ is
        negative at the point, then there's probably a zero in between
        $\frac{1}{2}$ and 1. In either case, we divide the range of
        possibilities in half once again and see what happens at $\frac{1}{4}$
        in the first case, and $\frac{3}{4}$ in the latter. We continue
        \textit{inductively} (whatever this means) and obtain a
        \textit{sequence} of real numbers which we then show
        \textit{converge} to some real number between 0 and 1. We then invoke
        continuity to show that $f$ evaluates to zero at this point, and we are
        done (See Fig.~\ref{fig:Sketch_of_IVP}).
        \begin{figure}[H]
            \centering
            \captionsetup{type=figure}
            \begin{tikzpicture}[>=Latex, scale=7]
    \draw[->, thick] (-0.1, 0.0) to (1.1, 0.0) node[below] {$x$};
    \draw[->, thick] ( 0,  -0.4) to (0.0, 0.5) node[right] {$y$};

    \coordinate (z0) at (0.000000, -0.100000);
    \coordinate (z1) at (1.000000,  0.101170);
    \coordinate (z2) at (0.500000, -0.100000);
    \coordinate (z3) at (0.750000, -0.147061);
    \coordinate (z4) at (0.870000,  0.222959);
    \coordinate (z5) at (0.810000,  0.056956);

    \coordinate (x0) at (0.00, 0.0);
    \coordinate (x1) at (1.00, 0.0);

    \coordinate (y0) at (0.00, 0.3);
    \coordinate (y1) at (1.00, 0.3);
    \coordinate (y2) at (0.50, 0.3);
    \coordinate (y3) at (0.75, 0.3);
    \coordinate (y4) at (0.87, 0.3);
    \coordinate (y5) at (0.81, 0.3);

    \draw[semithick]
        (0.000000, -0.100000) to (0.010000, -0.190882) to
        (0.020000, -0.134156) to (0.030000, -0.298796) to
        (0.040000, -0.161042) to (0.050000, -0.190845) to
        (0.060000, -0.151818) to (0.070000, -0.105578) to
        (0.080000, -0.294114) to (0.090000, -0.221376) to
        (0.100000, -0.348381) to (0.110000, -0.185051) to
        (0.120000, -0.137027) to (0.130000, -0.071993) to
        (0.140000, -0.083851) to (0.150000, -0.131908) to
        (0.160000, -0.207415) to (0.170000, -0.155534) to
        (0.180000, -0.225251) to (0.190000, -0.111510) to
        (0.200000, -0.160976) to (0.210000, -0.098903) to
        (0.220000, -0.066714) to (0.230000, -0.099304) to
        (0.240000, -0.037003) to (0.250000, -0.116941) to
        (0.260000, -0.065740) to (0.270000, -0.046159) to
        (0.280000, -0.029719) to (0.290000,  0.037735) to
        (0.300000, -0.048490) to (0.310000, -0.048847) to
        (0.320000, -0.120539) to (0.330000, -0.135940) to
        (0.340000, -0.088400) to (0.350000, -0.117398) to
        (0.360000, -0.054753) to (0.370000, -0.110149) to
        (0.380000, -0.090130) to (0.390000, -0.115973) to
        (0.400000, -0.117710) to (0.410000, -0.118431) to
        (0.420000, -0.120026) to (0.430000, -0.119288) to
        (0.440000, -0.103887) to (0.450000, -0.111766) to
        (0.460000, -0.102198) to (0.470000, -0.113391) to
        (0.480000, -0.110579) to (0.490000, -0.105565) to
        (0.500000, -0.100000) to (0.510000, -0.098346) to
        (0.520000, -0.093914) to (0.530000, -0.093459) to
        (0.540000, -0.083951) to (0.550000, -0.068383) to
        (0.560000, -0.079468) to (0.570000, -0.066709) to
        (0.580000, -0.095622) to (0.590000, -0.096420) to
        (0.600000, -0.097357) to (0.610000, -0.127241) to
        (0.620000, -0.082486) to (0.630000, -0.123949) to
        (0.640000, -0.055269) to (0.650000, -0.084457) to
        (0.660000, -0.071498) to (0.670000, -0.097302) to
        (0.680000, -0.125052) to (0.690000, -0.106186) to
        (0.700000, -0.072916) to (0.710000, -0.062225) to
        (0.720000, -0.025907) to (0.730000, -0.111287) to
        (0.740000, -0.108351) to (0.750000, -0.147061) to
        (0.760000, -0.121475) to (0.770000, -0.033252) to
        (0.780000, -0.037532) to (0.790000,  0.056384) to
        (0.800000,  0.041977) to (0.810000,  0.056956) to
        (0.820000,  0.100997) to (0.830000,  0.010700) to
        (0.840000,  0.091665) to (0.850000,  0.039524) to
        (0.860000,  0.121069) to (0.870000,  0.222959) to
        (0.880000,  0.132402) to (0.890000,  0.244225) to
        (0.900000, -0.009427) to (0.910000,  0.111595) to
        (0.920000,  0.029857) to (0.930000,  0.159189) to
        (0.940000,  0.224706) to (0.950000,  0.176094) to
        (0.960000,  0.165249) to (0.970000,  0.083951) to
        (0.980000,  0.032372) to (0.990000,  0.114170) to
        (1.000000,  0.101170);

        \foreach\i in {1, 2, 3, 4, 5}{
            \draw[fill=black] (z\i) circle (0.07mm);
            \draw[densely dashed] (z\i) to (y\i) node[above] {$x_{\i}$};
        }
        \draw[densely dashed] (z0) to (y0) node [above left] {$x_{0}$};

        \draw[fill=black] (z0) circle (0.07mm);
\end{tikzpicture}
            \caption{Sketch of the Intermediate Value Theorem}
            \label{fig:Sketch_of_IVP}
        \end{figure}
    \end{example}
    We can see why this may work. After a few iterations we've narrowed down the
    zero point to a very small range between $x_{3}$ and $x_{5}$, and this is a
    nice algorithm that we can tell a computer to execute to arbitrary
    precision, but what went into the proof? That is, if we were to phrase this
    with absolute precision, what definitions, assumptions, and previous
    theorems are we relying on? For one, the existence of \textit{real number},
    a notion of \textit{continuity}, and the definition of a \textit{sequence}.
    Our exposition of logic is to make clear what is required for valid proofs.