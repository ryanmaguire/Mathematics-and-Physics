\documentclass[crop=false,class=article,oneside]{standalone}
%----------------------------Preamble-------------------------------%
%---------------------------Packages----------------------------%
\usepackage{geometry}
\geometry{b5paper, margin=1.0in}
\usepackage[T1]{fontenc}
\usepackage{graphicx, float}            % Graphics/Images.
\usepackage{natbib}                     % For bibliographies.
\bibliographystyle{agsm}                % Bibliography style.
\usepackage[french, english]{babel}     % Language typesetting.
\usepackage[dvipsnames]{xcolor}         % Color names.
\usepackage{listings}                   % Verbatim-Like Tools.
\usepackage{mathtools, esint, mathrsfs} % amsmath and integrals.
\usepackage{amsthm, amsfonts, amssymb}  % Fonts and theorems.
\usepackage{tcolorbox}                  % Frames around theorems.
\usepackage{upgreek}                    % Non-Italic Greek.
\usepackage{fmtcount, etoolbox}         % For the \book{} command.
\usepackage[newparttoc]{titlesec}       % Formatting chapter, etc.
\usepackage{titletoc}                   % Allows \book in toc.
\usepackage[nottoc]{tocbibind}          % Bibliography in toc.
\usepackage[titles]{tocloft}            % ToC formatting.
\usepackage{pgfplots, tikz}             % Drawing/graphing tools.
\usepackage{imakeidx}                   % Used for index.
\usetikzlibrary{
    calc,                   % Calculating right angles and more.
    angles,                 % Drawing angles within triangles.
    arrows.meta,            % Latex and Stealth arrows.
    quotes,                 % Adding labels to angles.
    positioning,            % Relative positioning of nodes.
    decorations.markings,   % Adding arrows in the middle of a line.
    patterns,
    arrows
}                                       % Libraries for tikz.
\pgfplotsset{compat=1.9}                % Version of pgfplots.
\usepackage[font=scriptsize,
            labelformat=simple,
            labelsep=colon]{subcaption} % Subfigure captions.
\usepackage[font={scriptsize},
            hypcap=true,
            labelsep=colon]{caption}    % Figure captions.
\usepackage[pdftex,
            pdfauthor={Ryan Maguire},
            pdftitle={Mathematics and Physics},
            pdfsubject={Mathematics, Physics, Science},
            pdfkeywords={Mathematics, Physics, Computer Science, Biology},
            pdfproducer={LaTeX},
            pdfcreator={pdflatex}]{hyperref}
\hypersetup{
    colorlinks=true,
    linkcolor=blue,
    filecolor=magenta,
    urlcolor=Cerulean,
    citecolor=SkyBlue
}                           % Colors for hyperref.
\usepackage[toc,acronym,nogroupskip,nopostdot]{glossaries}
\usepackage{glossary-mcols}
%------------------------Theorem Styles-------------------------%
\theoremstyle{plain}
\newtheorem{theorem}{Theorem}[section]

% Define theorem style for default spacing and normal font.
\newtheoremstyle{normal}
    {\topsep}               % Amount of space above the theorem.
    {\topsep}               % Amount of space below the theorem.
    {}                      % Font used for body of theorem.
    {}                      % Measure of space to indent.
    {\bfseries}             % Font of the header of the theorem.
    {}                      % Punctuation between head and body.
    {.5em}                  % Space after theorem head.
    {}

% Italic header environment.
\newtheoremstyle{thmit}{\topsep}{\topsep}{}{}{\itshape}{}{0.5em}{}

% Define environments with italic headers.
\theoremstyle{thmit}
\newtheorem*{solution}{Solution}

% Define default environments.
\theoremstyle{normal}
\newtheorem{example}{Example}[section]
\newtheorem{definition}{Definition}[section]
\newtheorem{problem}{Problem}[section]

% Define framed environment.
\tcbuselibrary{most}
\newtcbtheorem[use counter*=theorem]{ftheorem}{Theorem}{%
    before=\par\vspace{2ex},
    boxsep=0.5\topsep,
    after=\par\vspace{2ex},
    colback=green!5,
    colframe=green!35!black,
    fonttitle=\bfseries\upshape%
}{thm}

\newtcbtheorem[auto counter, number within=section]{faxiom}{Axiom}{%
    before=\par\vspace{2ex},
    boxsep=0.5\topsep,
    after=\par\vspace{2ex},
    colback=Apricot!5,
    colframe=Apricot!35!black,
    fonttitle=\bfseries\upshape%
}{ax}

\newtcbtheorem[use counter*=definition]{fdefinition}{Definition}{%
    before=\par\vspace{2ex},
    boxsep=0.5\topsep,
    after=\par\vspace{2ex},
    colback=blue!5!white,
    colframe=blue!75!black,
    fonttitle=\bfseries\upshape%
}{def}

\newtcbtheorem[use counter*=example]{fexample}{Example}{%
    before=\par\vspace{2ex},
    boxsep=0.5\topsep,
    after=\par\vspace{2ex},
    colback=red!5!white,
    colframe=red!75!black,
    fonttitle=\bfseries\upshape%
}{ex}

\newtcbtheorem[auto counter, number within=section]{fnotation}{Notation}{%
    before=\par\vspace{2ex},
    boxsep=0.5\topsep,
    after=\par\vspace{2ex},
    colback=SeaGreen!5!white,
    colframe=SeaGreen!75!black,
    fonttitle=\bfseries\upshape%
}{not}

\newtcbtheorem[use counter*=remark]{fremark}{Remark}{%
    fonttitle=\bfseries\upshape,
    colback=Goldenrod!5!white,
    colframe=Goldenrod!75!black}{ex}

\newenvironment{bproof}{\textit{Proof.}}{\hfill$\square$}
\tcolorboxenvironment{bproof}{%
    blanker,
    breakable,
    left=3mm,
    before skip=5pt,
    after skip=10pt,
    borderline west={0.6mm}{0pt}{green!80!black}
}

\AtEndEnvironment{lexample}{$\hfill\textcolor{red}{\blacksquare}$}
\newtcbtheorem[use counter*=example]{lexample}{Example}{%
    empty,
    title={Example~\theexample},
    boxed title style={%
        empty,
        size=minimal,
        toprule=2pt,
        top=0.5\topsep,
    },
    coltitle=red,
    fonttitle=\bfseries,
    parbox=false,
    boxsep=0pt,
    before=\par\vspace{2ex},
    left=0pt,
    right=0pt,
    top=3ex,
    bottom=1ex,
    before=\par\vspace{2ex},
    after=\par\vspace{2ex},
    breakable,
    pad at break*=0mm,
    vfill before first,
    overlay unbroken={%
        \draw[red, line width=2pt]
            ([yshift=-1.2ex]title.south-|frame.west) to
            ([yshift=-1.2ex]title.south-|frame.east);
        },
    overlay first={%
        \draw[red, line width=2pt]
            ([yshift=-1.2ex]title.south-|frame.west) to
            ([yshift=-1.2ex]title.south-|frame.east);
    },
}{ex}

\AtEndEnvironment{ldefinition}{$\hfill\textcolor{Blue}{\blacksquare}$}
\newtcbtheorem[use counter*=definition]{ldefinition}{Definition}{%
    empty,
    title={Definition~\thedefinition:~{#1}},
    boxed title style={%
        empty,
        size=minimal,
        toprule=2pt,
        top=0.5\topsep,
    },
    coltitle=Blue,
    fonttitle=\bfseries,
    parbox=false,
    boxsep=0pt,
    before=\par\vspace{2ex},
    left=0pt,
    right=0pt,
    top=3ex,
    bottom=0pt,
    before=\par\vspace{2ex},
    after=\par\vspace{1ex},
    breakable,
    pad at break*=0mm,
    vfill before first,
    overlay unbroken={%
        \draw[Blue, line width=2pt]
            ([yshift=-1.2ex]title.south-|frame.west) to
            ([yshift=-1.2ex]title.south-|frame.east);
        },
    overlay first={%
        \draw[Blue, line width=2pt]
            ([yshift=-1.2ex]title.south-|frame.west) to
            ([yshift=-1.2ex]title.south-|frame.east);
    },
}{def}

\AtEndEnvironment{ltheorem}{$\hfill\textcolor{Green}{\blacksquare}$}
\newtcbtheorem[use counter*=theorem]{ltheorem}{Theorem}{%
    empty,
    title={Theorem~\thetheorem:~{#1}},
    boxed title style={%
        empty,
        size=minimal,
        toprule=2pt,
        top=0.5\topsep,
    },
    coltitle=Green,
    fonttitle=\bfseries,
    parbox=false,
    boxsep=0pt,
    before=\par\vspace{2ex},
    left=0pt,
    right=0pt,
    top=3ex,
    bottom=-1.5ex,
    breakable,
    pad at break*=0mm,
    vfill before first,
    overlay unbroken={%
        \draw[Green, line width=2pt]
            ([yshift=-1.2ex]title.south-|frame.west) to
            ([yshift=-1.2ex]title.south-|frame.east);},
    overlay first={%
        \draw[Green, line width=2pt]
            ([yshift=-1.2ex]title.south-|frame.west) to
            ([yshift=-1.2ex]title.south-|frame.east);
    }
}{thm}

%--------------------Declared Math Operators--------------------%
\DeclareMathOperator{\adjoint}{adj}         % Adjoint.
\DeclareMathOperator{\Card}{Card}           % Cardinality.
\DeclareMathOperator{\curl}{curl}           % Curl.
\DeclareMathOperator{\diam}{diam}           % Diameter.
\DeclareMathOperator{\dist}{dist}           % Distance.
\DeclareMathOperator{\Div}{div}             % Divergence.
\DeclareMathOperator{\Erf}{Erf}             % Error Function.
\DeclareMathOperator{\Erfc}{Erfc}           % Complementary Error Function.
\DeclareMathOperator{\Ext}{Ext}             % Exterior.
\DeclareMathOperator{\GCD}{GCD}             % Greatest common denominator.
\DeclareMathOperator{\grad}{grad}           % Gradient
\DeclareMathOperator{\Ima}{Im}              % Image.
\DeclareMathOperator{\Int}{Int}             % Interior.
\DeclareMathOperator{\LC}{LC}               % Leading coefficient.
\DeclareMathOperator{\LCM}{LCM}             % Least common multiple.
\DeclareMathOperator{\LM}{LM}               % Leading monomial.
\DeclareMathOperator{\LT}{LT}               % Leading term.
\DeclareMathOperator{\Mod}{mod}             % Modulus.
\DeclareMathOperator{\Mon}{Mon}             % Monomial.
\DeclareMathOperator{\multideg}{mutlideg}   % Multi-Degree (Graphs).
\DeclareMathOperator{\nul}{nul}             % Null space of operator.
\DeclareMathOperator{\Ord}{Ord}             % Ordinal of ordered set.
\DeclareMathOperator{\Prin}{Prin}           % Principal value.
\DeclareMathOperator{\proj}{proj}           % Projection.
\DeclareMathOperator{\Refl}{Refl}           % Reflection operator.
\DeclareMathOperator{\rk}{rk}               % Rank of operator.
\DeclareMathOperator{\sgn}{sgn}             % Sign of a number.
\DeclareMathOperator{\sinc}{sinc}           % Sinc function.
\DeclareMathOperator{\Span}{Span}           % Span of a set.
\DeclareMathOperator{\Spec}{Spec}           % Spectrum.
\DeclareMathOperator{\supp}{supp}           % Support
\DeclareMathOperator{\Tr}{Tr}               % Trace of matrix.
%--------------------Declared Math Symbols--------------------%
\DeclareMathSymbol{\minus}{\mathbin}{AMSa}{"39} % Unary minus sign.
%------------------------New Commands---------------------------%
\DeclarePairedDelimiter\norm{\lVert}{\rVert}
\DeclarePairedDelimiter\ceil{\lceil}{\rceil}
\DeclarePairedDelimiter\floor{\lfloor}{\rfloor}
\newcommand*\diff{\mathop{}\!\mathrm{d}}
\newcommand*\Diff[1]{\mathop{}\!\mathrm{d^#1}}
\renewcommand*{\glstextformat}[1]{\textcolor{RoyalBlue}{#1}}
\renewcommand{\glsnamefont}[1]{\textbf{#1}}
\renewcommand\labelitemii{$\circ$}
\renewcommand\thesubfigure{%
    \arabic{chapter}.\arabic{figure}.\arabic{subfigure}}
\addto\captionsenglish{\renewcommand{\figurename}{Fig.}}
\numberwithin{equation}{section}

\renewcommand{\vector}[1]{\boldsymbol{\mathrm{#1}}}

\newcommand{\uvector}[1]{\boldsymbol{\hat{\mathrm{#1}}}}
\newcommand{\topspace}[2][]{(#2,\tau_{#1})}
\newcommand{\measurespace}[2][]{(#2,\varSigma_{#1},\mu_{#1})}
\newcommand{\measurablespace}[2][]{(#2,\varSigma_{#1})}
\newcommand{\manifold}[2][]{(#2,\tau_{#1},\mathcal{A}_{#1})}
\newcommand{\tanspace}[2]{T_{#1}{#2}}
\newcommand{\cotanspace}[2]{T_{#1}^{*}{#2}}
\newcommand{\Ckspace}[3][\mathbb{R}]{C^{#2}(#3,#1)}
\newcommand{\funcspace}[2][\mathbb{R}]{\mathcal{F}(#2,#1)}
\newcommand{\smoothvecf}[1]{\mathfrak{X}(#1)}
\newcommand{\smoothonef}[1]{\mathfrak{X}^{*}(#1)}
\newcommand{\bracket}[2]{[#1,#2]}

%------------------------Book Command---------------------------%
\makeatletter
\renewcommand\@pnumwidth{1cm}
\newcounter{book}
\renewcommand\thebook{\@Roman\c@book}
\newcommand\book{%
    \if@openright
        \cleardoublepage
    \else
        \clearpage
    \fi
    \thispagestyle{plain}%
    \if@twocolumn
        \onecolumn
        \@tempswatrue
    \else
        \@tempswafalse
    \fi
    \null\vfil
    \secdef\@book\@sbook
}
\def\@book[#1]#2{%
    \refstepcounter{book}
    \addcontentsline{toc}{book}{\bookname\ \thebook:\hspace{1em}#1}
    \markboth{}{}
    {\centering
     \interlinepenalty\@M
     \normalfont
     \huge\bfseries\bookname\nobreakspace\thebook
     \par
     \vskip 20\p@
     \Huge\bfseries#2\par}%
    \@endbook}
\def\@sbook#1{%
    {\centering
     \interlinepenalty \@M
     \normalfont
     \Huge\bfseries#1\par}%
    \@endbook}
\def\@endbook{
    \vfil\newpage
        \if@twoside
            \if@openright
                \null
                \thispagestyle{empty}%
                \newpage
            \fi
        \fi
        \if@tempswa
            \twocolumn
        \fi
}
\newcommand*\l@book[2]{%
    \ifnum\c@tocdepth >-3\relax
        \addpenalty{-\@highpenalty}%
        \addvspace{2.25em\@plus\p@}%
        \setlength\@tempdima{3em}%
        \begingroup
            \parindent\z@\rightskip\@pnumwidth
            \parfillskip -\@pnumwidth
            {
                \leavevmode
                \Large\bfseries#1\hfill\hb@xt@\@pnumwidth{\hss#2}
            }
            \par
            \nobreak
            \global\@nobreaktrue
            \everypar{\global\@nobreakfalse\everypar{}}%
        \endgroup
    \fi}
\newcommand\bookname{Book}
\renewcommand{\thebook}{\texorpdfstring{\Numberstring{book}}{book}}
\providecommand*{\toclevel@book}{-2}
\makeatother
\titleformat{\part}[display]
    {\Large\bfseries}
    {\partname\nobreakspace\thepart}
    {0mm}
    {\Huge\bfseries}
\titlecontents{part}[0pt]
    {\large\bfseries}
    {\partname\ \thecontentslabel: \quad}
    {}
    {\hfill\contentspage}
\titlecontents{chapter}[0pt]
    {\bfseries}
    {\chaptername\ \thecontentslabel:\quad}
    {}
    {\hfill\contentspage}
\newglossarystyle{longpara}{%
    \setglossarystyle{long}%
    \renewenvironment{theglossary}{%
        \begin{longtable}[l]{{p{0.25\hsize}p{0.65\hsize}}}
    }{\end{longtable}}%
    \renewcommand{\glossentry}[2]{%
        \glstarget{##1}{\glossentryname{##1}}%
        &\glossentrydesc{##1}{~##2.}
        \tabularnewline%
        \tabularnewline
    }%
}
\newglossary[not-glg]{notation}{not-gls}{not-glo}{Notation}
\newcommand*{\newnotation}[4][]{%
    \newglossaryentry{#2}{type=notation, name={\textbf{#3}, },
                          text={#4}, description={#4},#1}%
}
%--------------------------LENGTHS------------------------------%
% Spacings for the Table of Contents.
\addtolength{\cftsecnumwidth}{1ex}
\addtolength{\cftsubsecindent}{1ex}
\addtolength{\cftsubsecnumwidth}{1ex}
\addtolength{\cftfignumwidth}{1ex}
\addtolength{\cfttabnumwidth}{1ex}

% Indent and paragraph spacing.
\setlength{\parindent}{0em}
\setlength{\parskip}{0em}
%----------------------------GLOSSARY-------------------------------%
\makeglossaries
\loadglsentries{../../glossary}
\loadglsentries{../../acronym}
%--------------------------Main Document----------------------------%
\begin{document}
    \pagenumbering{roman}
    \title{Functional Analysis}
    \author{Ryan Maguire}
    \date{\vspace{-5ex}}
    \maketitle
    \tableofcontents
    \clearpage
    \pagenumbering{arabic}
    \section{Notes from Dartmouth S2019}
        \subsection{Metric Spaces}
            \begin{ldefinition}{Pseudo-Metric}
                A pseudo-metric on a set $X$ is a function
                $\rho:X\times{X}\rightarrow[0,\infty)$ such that,
                for all $x,y,z\in{X}$, it is true that:
                \begin{align}
                    \rho(x,y)&=\rho(y,x)
                    \tag{Symmetry}\\
                    \rho(x,z)&\leq\rho(x,y)+\rho(y,z)
                    \tag{Triangle Inequality}
                \end{align}
            \end{ldefinition}
            \begin{ldefinition}{Pseudo-Metric Space}
                A pseudo-metric space, $(X,\rho)$, is a set
                $X$ and a pseudo-metric $\rho$ on $X$.
            \end{ldefinition}
            \begin{theorem}
                There exist pseudo-metric spaces $(X,\rho)$
                such that for all $x\in{X}$, $\rho(x,x)>0$.
            \end{theorem}
            \begin{proof}
                For let $X=\mathbb{R}$ and define
                $\rho:\mathbb{R}^{2}\rightarrow[0,\infty)$ by:
                \begin{equation}
                    \rho(x,y)=1+|x|+|y|
                \end{equation}
                Then $\rho$ is a pseudo-metric. For it is symmetric,
                since:
                \begin{equation}
                    \rho(x,y)=1+|x|+|y|=1+|y|+|x|=\rho(y,x)
                \end{equation}
                Moreover, it obeys the triangle inequality:
                \begin{subequations}
                    \begin{align}
                        \rho(x,z)&=1+|x|+|z|\\
                        &\leq{1}+|x|+|z|+2|y|+1\\
                        &=(1+|x|+|y|)+(1+|y|+|z|)\\
                        &=\rho(x,y)+\rho(y,z)
                    \end{align}
                    Thus, $\rho$ is a pseudo-metric. However, for
                    all $x\in\mathbb{R}$:
                    \begin{equation}
                        \rho(x,x)=1+|x|+|x|=1+2|x|\geq{1}>0
                    \end{equation}
                    Thus, there are no $x\in{X}$ such that
                    $\rho(x,x)=0$. Therefore, etc
                \end{subequations}
            \end{proof}
            If we require $\rho(x,x)=0$ for all $x\in{X}$, we
            can still have the case where elements cannot be
            distinguished from. That is, there may be
            $x,y\in{X}$ such that $x\ne{y}$, but
            $\rho(x,y)=0$.
            \begin{theorem}
                There exist pseudo-metric spaces $(X,\rho)$
                such that for all $x\in{X}$, $\rho(x,x)=0$,
                and there are distinct elements $x,y\in{X}$
                such that $\rho(x,y)=0$.
            \end{theorem}
            \begin{proof}
                For let $X$ have at least two distinct elements,
                and let $\rho:X^{2}\rightarrow[0,\infty)$
                be defined by:
                \begin{equation}
                    \rho(x,y)=0
                \end{equation}
                Then $\rho$ is a pseudo-metric. Symmetry and
                the triangle inequality are both trivial.
                However, since there are at least two distinct
                elements in $X$, we have unique points such that
                $\rho(x,y)=0$. Therefore, etc.
            \end{proof}
            \begin{ldefinition}{Metric Space}
                A metric space is a pseudo-metric space $(X,d)$
                such that:
                \begin{equation}
                    d(x,y)=0\Longleftrightarrow{x}=y
                    \tag{Definiteness}
                \end{equation}
            \end{ldefinition}
            \begin{ldefinition}{Semi-Norm}
                A semi-norm on a vector space $V$ over a field
                $\mathbb{F}\subseteq\mathbb{C}$ is a function
                $\norm{\cdot}:V\rightarrow[0,\infty)$ such that, for
                all $v\in{V}$ and $\alpha\in\mathbb{F}$,
                it is true that:
                \begin{align}
                    \norm{\alpha{v}}
                    &=|\alpha|\norm{v}
                    \tag{Homogeneity}\\
                    \norm{v+w}
                    &\leq\norm{v}+\norm{w}
                    \tag{Triangle Inequality}
                \end{align}
            \end{ldefinition}
            \begin{theorem}
                If $V$ is a vector space over a field
                $\mathbb{F}\subseteq\mathbb{C}$, and if
                $\norm{\cdot}$ is semi-norm on $V$, then:
                \begin{equation}
                    \norm{\mathbf{0}}=0
                \end{equation}
            \end{theorem}
            \begin{proof}
                For:
                \begin{equation}
                    \norm{\mathbf{0}}
                    =\norm{0\mathbf{0}}
                    =|0|\norm{\mathbf{0}}=0
                \end{equation}
                Therefore, etc.
            \end{proof}
            \begin{ldefinition}{Norm}
                A norm on a vector space $V$ over a
                field $\mathbb{F}\subseteq\mathbb{C}$
                is a semi-norm $\norm{\cdot}$ such that:
                \begin{equation}
                    \norm{\mathbf{x}}\Longrightarrow\mathbf{x}
                    =\mathbf{0}
                \end{equation}
            \end{ldefinition}
            \begin{lexample}
                Suppose that $\norm{\cdot}_{0}$ is a semi-norm
                on a vector space $V$. Define the following:
                \begin{equation}
                    N=\{v\in{V}:\norm{v}_{0}=0\}
                \end{equation}
                If follows from the definition of a semi-norm that
                $N$ is a subspace of $V$. Thus we can define a
                function on the quotient space
                $\norm{\cdot}:V/N\rightarrow[0,\infty)$ by:
                \begin{equation}
                    \norm{v+N}=\norm{v}_{0}
                \end{equation}
                We can then verify that this is well defined and
                that $\norm{\cdot}$ is a norm on $V/N$.
            \end{lexample}
            If $\norm{\cdot}$ is a norm on $V$, then we get an
            associated metric $\rho$ via:
            \begin{equation}
                \rho(v,u)=\norm{v-u}
            \end{equation}
            \begin{lexample}
                Let $(X,\mathcal{M},\mu)$ be a measure space.
                That is, $X$ is a set, $\mathcal{M}$ is
                $\sigma\textrm{-Algebra}$, and $\mu$ is a measure
                on $X$. Let $1\leq{p}<\infty$. Then:
                \begin{equation}
                    \mathcal{L}^{p}(X)=
                    \{f:X\rightarrow\mathbb{C}:
                    f\textrm{ is measurable and}
                    \int_{X}|f|^{p}\diff{\mu}<\infty.\}
                \end{equation}
                The set $\mathcal{L}^{p}(X)$ is a vector space.
                We define the semi-norm on $\mathcal{L}^{p}(X)$
                to be:
                \begin{equation}
                    \norm{f}_{p}=
                    \Big(\int_{X}|f|^{p}\diff\mu\Big)^{1/p}
                \end{equation}
                This is not a norm, since there are many
                functions such that $\norm{f}_{p}=0$, yet $f\ne{0}$.
                However, if $\norm{f}_{p}=0$, then $f=0$ $\mu$
                almost-everywhere. So we create equivalence
                classes by s comparing functions that are $\mu$
                almost-everywhere. The final thing to check is
                the triangle-inequality. It is not obvious and is a
                consequence of Minkowski's Inequality. We get
                a normed vector space by considering $N$ to be
                the set of functions $f$ such that $\norm{f}_{p}=0$,
                and we define:
                \begin{equation}
                    L^{p}(X)=\mathcal{L}^{p}(X)/N
                \end{equation}
                The analyst Halmos said that the only important
                values of $p$ are 1, 2, and $\infty$. If
                $f:X\rightarrow\mathcal{C}$ is
                measurable, then we define:
                \begin{equation}
                    \norm{f}_{\infty}=
                    \inf\{c\geq{0}:\mu\Big(\{x:|f(x)|>c\}\big)=0\}
                \end{equation}
                With the convention that $\inf\{\emptyset\}=\infty$.
                This defines a semi-norm on:
                \begin{equation}
                    \mathcal{L}^{\infty}(X)
                    =\{f:\norm{f}_{\infty}<\infty\}
                \end{equation}
                Homogeneity pops out rather quickly, but the
                triangle-inequality is still tricky. We call
                $\norm{\cdot}_{\infty}$ the essential supremum
                of $f$.
            \end{lexample}
            \begin{theorem}
                If $f:X\rightarrow\mathbb{C}$ is measurable, and if:
                \begin{equation}
                    E=\{p:\norm{f}_{p}<\infty,p\in[1,\infty)\}
                \end{equation}
                Then $E$ is connected.
            \end{theorem}
            \begin{lexample}
                Let $X$ be a finite set, let
                $\mathcal{M}=\mathcal{P}(X)$, and let $\mu$ be the
                counting measure on $X$. A function on
                $X$ is an n-tuple $x=(x_{1},\dots,x_{n})$. Then:
                \begin{equation}
                    \norm{x}_{p}=
                    \begin{cases}
                        \Big(\sum_{k=1}^{n}|x_{k}|^{p}\Big)^{1/p},
                        &1\leq{p}<\infty\\
                        \max\{|x|,x\in{X}\}
                    \end{cases}
                \end{equation}
                $\norm{\cdot}_{p}$ is a norm on $X$.
            \end{lexample}
            \begin{lexample}
                Let $X=\mathbb{N}$, the set of natural numbers. Let
                $\mathcal{M}=\mathcal{P}(X)$, and let $\mu$ be
                the counting measure. Then functions are sequences
                $a:\mathbb{N}\rightarrow\mathbb{R}$, or
                $a:\mathbb{N}\rightarrow\mathbb{C}$. Then:
                \begin{equation}
                    \norm{x}_{p}=
                    \begin{cases}
                        \Big(\sum_{n=1}^{\infty}
                            |x_{n}|^{p}\Big)^{1/p},
                        &1\leq{p}<\infty\\
                        \max\{|x|,x\in{X}\}
                    \end{cases}
                \end{equation}
                This defines a norm. Recall that a series is
                absolutely convergent if $\sum|a_{n}|<\infty$.
                Given an absolutely convergent series, the original
                series is also convergent. For this space we use
                the following notation:
                \begin{equation}
                    \ell^{p}=
                    \{a:\mathbb{R}\rightarrow\mathbb{R}:
                        \sum_{n=1}^{\infty}|a_{n}|^{p}<\infty\}
                \end{equation}
            \end{lexample}
            In general, if $X$ is a set then we can equip $X$
            with the counting measure and then if $f$ is any
            bounded function on $f$, then:
            \begin{equation}
                \norm{f}_{\infty}=\sup\{|f(x):x\in{X}\}
            \end{equation}
            For the space of sequences, we write $\ell^{\infty}(X)$.
            \begin{lexample}
                If $X$ is any set, then we define the following
                metric:
                \begin{equation}
                    \rho(x,y)=
                    \begin{cases}
                        1,&x\ne{y}\\
                        0,&x=1
                    \end{cases}
                \end{equation}
                This is often called the discrete metric. It is
                indeed a metric, and $(X,\rho)$ is a metric space.
            \end{lexample}
            \begin{lexample}
                Let $(X,\rho)$ be a metric space. If $Y\subseteq{X}$
                is a non-empty subset of $X$, we can define a new
                metric on $Y$
                be restricting $\rho$ to $Y\times{Y}$. We call
                this the metric subspace.
            \end{lexample}
            \begin{ldefinition}{Strongly Equivalent Metrics}
                Strongly equilalent metrics on a set $X$ are metrics
                $\rho_{1}$ and $\rho_{2}$ such that there
                exists $c,d\in\mathbb{R}^{+}$ such that, for
                all $x,y\in{X}$:
                \begin{equation}
                    c\rho_{1}(x,y)\leq
                        \rho_{2}(x,y)\leq{d}\rho_{1}(x,y)
                \end{equation}
            \end{ldefinition}
            The definition of strongly equivalent metrics is indeed
            symmetric. For since $c,d\in\mathbb{R}^{+}$,
            $c^{-1}$ and $d^{-1}$ are well defined and positive,
            and thus:
            \begin{equation}
                \frac{1}{d}\rho_{2}(x,y)\leq\rho_{1}(x,y)
                \leq\frac{1}{c}\rho_{1}(x,y)
            \end{equation}
            \begin{theorem}
                If $p,q\in[1,\infty)$, then $\norm{\cdot}_{p}$ and
                $\norm{\cdot}_{q}$ are strongly equivalent.
            \end{theorem}
            \begin{proof}
                It suffices to show that for all
                $p\in[1,\infty)$ there exist
                $c,d\in\mathbb{R}^{+}$ such that:
                \begin{equation}
                    c\norm{x}_{p}\leq\norm{x}_{2}\leq{d}\norm{x}_{p}
                \end{equation}
                Also note that:
                \begin{equation}
                    \partial{\overline{B}_{1}(\mathbf{0})}=
                    \{\mathbf{x}\in\mathbb{R}^{n}:\norm{x}_{2}=1\}
                \end{equation}
                Is a closed and bounded subset of $\mathbb{R}^{n}$.
            \end{proof}
            \begin{lexample}
                Let $(X,\rho)$ be a metric space on $X$ and define:
                \begin{equation}
                    d(x,y)=\frac{\rho(x,y)}{1+\rho(x,y)}
                \end{equation}
                Then $(X,d)$ is a metric space. Definiteness
                and symmetry come rather immediately from the
                definition and the fact that $\rho$ is a metric.
                The only thing to check is the
                triangle inequality.
            \end{lexample}
            \begin{ldefinition}{Open Ball in a Metric Space}
                The open ball about a point $x$ in a metric space
                $(X,\rho)$ of radius $r\in\mathbb{R}$ is the set:
                \begin{equation}
                    B_{r}(x)=\{y\in{X}:\rho(x,y)<r\}
                \end{equation}
            \end{ldefinition}
            \begin{ldefinition}{Open Subsets of a Metric Space}
                An open subset of a metric space $(X,\rho)$ is a set
                $\mathcal{U}\subseteq{X}$ such that for all
                $x\in\mathcal{U}$ there is an $\varepsilon>0$
                such that:
                \begin{equation}
                    B_{\varepsilon}(x)\subseteq\mathcal{U}
                \end{equation}
            \end{ldefinition}
            \begin{theorem}
                Open balls are open.
            \end{theorem}
            \begin{ldefinition}{Neighborhoods}
                A neighborhood of a point $x$ in a metric space
                $(X,\rho)$ is a subset $D\subseteq{X}$ such
                that there is an open subset
                $\mathcal{U}\subseteq{D}$ such that
                $x\in\mathcal{U}$.
            \end{ldefinition}
            \begin{theorem}
                IF $(X,\rho)$ is a metric space, then $X$ is
                an open subset.
            \end{theorem}
            \begin{theorem}
                If $(X,\rho)$ is a metric space, then
                $\emptyset$ is an open subset.
            \end{theorem}
            \begin{theorem}
                If $\mathcal{U}_{I}$ is a collection of open subsets
                of a metric space $(X,\rho)$, and if
                $\mathcal{U}$ is defined by:
                \begin{equation}
                    \mathcal{U}=\bigcup_{i\in{I}}\mathcal{U}_{i}
                \end{equation}
                Then $\mathcal{U}$ is an open subset.
            \end{theorem}
            \begin{theorem}
                If $(X,\rho)$ is a metric space, and if
                $\mathcal{U}$ and $\mathcal{V}$ are open
                subsets, then the set $\mathcal{D}$ defined by:
                \begin{equation}
                    \mathcal{D}=\mathcal{U}\cap\mathcal{V}
                \end{equation}
                Is an open subset of $X$.
            \end{theorem}
            \begin{theorem}
                If $(X,\rho)$ is a metric space and
                $(\mathcal{E},\rho_{\mathcal{E}})$ is a subspace
                of $(X,\rho)$, then a set
                $\mathcal{U}\subseteq\mathcal{E}$
                is open in $\mathcal{E}$ if and only if there is
                an open set $\mathcal{U}$ in $X$ such that:
                \begin{equation}
                    \mathcal{V}=\mathcal{U}\cap\mathcal{E}
                \end{equation}
            \end{theorem}
            \begin{ldefinition}{Topology}
                A topology on a set $X$ is a subset
                $\tau\subseteq\mathcal{P}(X)$ such that:
                $\emptyset\in\tau$ and $X\in\tau$, for any finite
                subset $\mathcal{C}\subseteq\tau$, it is true that:
                \begin{subequations}
                    \begin{equation}
                        \bigcap_{C\in\mathcal{C}}C\in\tau
                    \end{equation}
                    And for any subset $\mathcal{O}\subseteq\tau$
                    it is true that:
                    \begin{equation}
                        \bigcup_{\mathcal{U}\in\mathcal{O}}
                        \mathcal{U}\in\tau
                    \end{equation}
                \end{subequations}
                That is, $\tau$ is closed to finite intersections
                and arbitrary unions.
            \end{ldefinition}
            \begin{lexample}
                If $(X,\rho)$ is a metric space, then the set:
                \begin{equation}
                    \tau=\{\mathcal{U}\subseteq{X}:
                        \mathcal{U}\textrm{ is open}\}
                \end{equation}
                is a topology on $X$. This is the \textit{metric}
                topology. Not every topological space can be
                formed from a metric space. If
                $(\mathcal{E},\rho_{\mathcal{E}})$ is a subspace
                of $(X,\rho)$, then the \textit{subspace topology},
                or the relative topology, is the set:
                \begin{equation}
                    \tau_{\mathcal{E}}
                    =\{\mathcal{E}\cap\mathcal{U}:
                        \mathcal{U}\in\tau\}
                \end{equation}
                This is a topology on $\mathcal{E}$.
            \end{lexample}
            \begin{ldefinition}{Closed Subsets}
                A closed subset of a metric space $(X,\rho)$
                is a set $\mathcal{C}\subseteq{X}$ such that
                $X\setminus\mathcal{C}$ is an open subset of $X$.
            \end{ldefinition}
            \begin{theorem}
                If $(X,\rho)$ is a metric space, then $X$ is closed.
            \end{theorem}
            \begin{theorem}
                If $(X,\rho)$ is a metric space, then
                $\emptyset$ is closed.
            \end{theorem}
            \begin{theorem}
                If $(X,\rho)$ is a metric space, and if
                $\mathcal{C}_{i}$ is a finite collection of
                closed subsets, then the set
                $\mathcal{C}$ defined by:
                \begin{equation}
                    \mathcal{C}=\cup_{i=1}^{n}\mathcal{C}_{i}
                \end{equation}
                is a closed subset of $X$.
            \end{theorem}
            \begin{ldefinition}{Closure of a Set}
                If $(X,\rho)$ is a metric space and
                $\mathcal{E}\subseteq{X}$, then the closure
                of $\mathcal{E}$ is the set:
                \begin{equation}
                    \overline{\mathcal{E}}=
                    \bigcap\{\mathcal{F}\subseteq{X}:
                        \mathcal{E}\subseteq\mathcal{F}
                        \land\mathcal{F}\textrm{ is closed.}\}
                \end{equation}
            \end{ldefinition}
            The closure of a set $\mathcal{E}$ is the
            smallest closed set that contains $\mathcal{E}$.
            \begin{theorem}
                If $(X,\rho)$ is a metric space, and if
                $\mathcal{E}$ is a non-empty subset of $X$, then
                $x\in\overline{\mathcal{E}}$ if and only if for
                all $\varepsilon>0$:
                \begin{equation}
                    B_{\varepsilon}(x)\cap\mathcal{E}\ne\emptyset
                \end{equation}
            \end{theorem}
            \begin{ldefinition}
                  {Convergent Sequences in a Metric Space}
                A convergent sequence in a metric space
                $(X,\rho)$ is a sequence
                $a:\mathbb{N}\rightarrow{X}$ such that there is
                an $x\in{X}$ such that for all $\varepsilon>0$
                there exists an $N\in\mathbb{N}$ such that, for
                all $n\in\mathbb{N}$ and $n>N$, it is true that
                $d(x,a_{n})<\varepsilon$. We write
                $a_{n}\rightarrow{x}$.
            \end{ldefinition}
            \begin{ldefinition}{Limits of Convergent Sequences}
                A limit of a convergent sequence
                $a:\mathbb{N}\rightarrow{X}$ in a metric space
                $(X,\rho)$ is a point $x\in{X}$ such that
                $a_{n}\rightarrow{x}$.
            \end{ldefinition}
            \begin{theorem}
                If $(X,\rho)$ is a metric space, if
                $a:\mathbb{N}\rightarrow{X}$ is a convergent
                sequence, and if $x$ and $y$ are limits of $a$,
                then $x=y$.
            \end{theorem}
            \begin{ldefinition}{Equivalent Metrics}
                Equivalent metrics on a set $X$ and metrics
                $\rho$ and $d$ such that they generate the
                same topology.
            \end{ldefinition}
            \begin{theorem}
                If $X$ is a set and if $\rho$ and $d$ are
                equivalent metrics on $X$, then for all $x\in{X}$
                and for all $r\in\mathbb{R}^{+}$ there exists
                $r_{1},r_{2}\in\mathbb{R}^{+}$ such that:
                \begin{align}
                    B_{r_{1}}^{\rho}(x)&\subseteq{B}_{r}^{d}(x)\\
                    B_{r_{2}}^{d}(x)&\subseteq{B}_{r}^{\rho}(x)
                \end{align}
            \end{theorem}
            \begin{lexample}
                If $(X,\rho)$ is a metric space and if $d$
                is defined by:
                \begin{equation}
                    d(x,y)=\frac{\rho(x,y)}{1+\rho(x,y)}
                \end{equation}
                The $d$ is a metric on $X$. Moreover, $d$
                is equivalent to $\rho$. This shows that the
                notion of boundedness is not a topological one,
                but a metric property. For, given any metric
                $\rho$, $d$ is bounded. For all $x,y\in{X}$,
                $0\leq{d}(x,y)<1$. Letting $\rho$ be the
                standard metric on $\mathbb{R}$,
                $\rho(x,y)=|x-y|$, we see that the topology
                generated by this unbounded metric is equivalent
                to the topology generated by the metric:
                \begin{equation}
                    d(x,y)=\frac{|x-y|}{1+|x-y|}
                \end{equation}
            \end{lexample}
            \begin{theorem}
                If $(X,\rho)$ is a metric space, and if
                $d:X^{2}\rightarrow[0,\infty)$ is defined by:
                \begin{equation}
                    d(x,y)=\frac{\rho(x,y)}{1+\rho(x,y)}
                \end{equation}
                Then $\rho$ and $d$ are equivalent.
            \end{theorem}
            \begin{proof}
                For let $r>0$. For all $x,y\in{X}$,
                $d(x,y)\leq\rho(x,y)$, and therefore:
                \begin{equation}
                    B_{r}^{\rho}(x)\subseteq{B}_{r}^{d}(x)
                \end{equation}
                If $\rho(x,y)\leq{1}$, then
                $\rho(x,y)\leq{2}d(x,y)$. Let
                $r_{1}=\min\{r/2,1\}$, then:
                \begin{equation}
                    B_{r/2}^{d}(x)\subseteq{B}_{r}^{\rho}(x)
                \end{equation}
                Therefore, etc.
            \end{proof}
            \begin{ldefinition}
                  {Continuous Functions Between Metric Spaces}
                A continuous function from a metric space
                $(X,\rho)$ to a metric space $(Y,d)$ is a function
                $f:X\rightarrow{Y}$ such that, for all $x\in{X}$
                and for all $\varepsilon>0$, there is a
                $\delta>0$ such that:
                \begin{equation}
                    f\big(B_{\delta}^{\rho}(x)\big)\subset
                    B_{\varepsilon}^{d}\big(f(x)\big)
                \end{equation}
            \end{ldefinition}
            \begin{theorem}
                If $(X,\rho)$ and $(Y,d)$ are metric spaces, and
                if $f:X\rightarrow{Y}$ is a function, then
                the following are equivalent:
                \begin{enumerate}
                    \item $f$ is continuous at $x_{0}\in{X}$.
                    \item If $x_{n}\rightarrow{x_{0}}$ then
                          $f(x_{n})\rightarrow{f}(x_{0})$
                    \item If $\mathcal{V}$ is a neighborhood of $f(x_{0})$,
                          then $f^{-1}(\mathcal{V})$ is a neighborhodd of
                          $x_{0}$.
                \end{enumerate}
            \end{theorem}
            \begin{theorem}
                If $(X,\rho)$ and $(Y,d)$ are metric spaces, and if
                $f:X\rightarrow{Y}$ is a function that is
                continuous at $x_{0}\in{X}$, then for
                all $\mathcal{V}\subseteq{Y}$ such that
                $\mathcal{V}$ is open and $f(x_{0})\in\mathcal{V}$,
                then $f^{-1}(\mathcal{V})$ is an open subset
                of $x_{0}$.
            \end{theorem}
            \begin{ldefinition}{Uniformly Continuous Functions}
                A uniformly continuous function from a metric space
                $(X,\rho)$ to a metric space $(Y,d)$ if a function
                $f:X\rightarrow{Y}$ such that for all
                $\varepsilon>0$ there exists a $\delta>0$ such
                that, for all $x,y\in{X}$ such that
                $\rho(x,y)<\delta$, it is true that
                $d(f(x),f(y))<\varepsilon$.
            \end{ldefinition}
            \begin{theorem}
                If $f:X\rightarrow{Y}$ is uniformly continuous,
                then $f$ is continuous.
            \end{theorem}
            The converse is false. For define $f(x)=x^{2}$.
            \begin{ldefinition}{Cauchy Sequences}
                A Cauchy sequence in a metric space $(X,d)$ is a
                sequence $a:\mathbb{N}\rightarrow{X}$ such that,
                for all $\varepsilon>0$ there is an
                $N\in\mathbb{N}$ such that, for all
                $n,m\in\mathbb{N}$ such that $n,m>N$,
                it is true that $d(x_{n},x_{m})<\varepsilon$.
            \end{ldefinition}
            \begin{lexample}
                Let $X=(0,2)$ with the usual metric, and let
                $a:\mathbb{N}\rightarrow{X}$ be defined by:
                \begin{equation}
                    a_{n}=\frac{1}{n}
                \end{equation}
                Then $a$ is a Cauchy sequence since:
                \begin{equation}
                    |a_{n}-a_{m}|=\frac{|n-m|}{nm}
                    <\frac{2}{\min(n,m)}
                \end{equation}
                And this converges to zero. However the sequence
                doesn't converge, since we took zero away.
            \end{lexample}
            \begin{lexample}
                Let $X=C([0,3])$ and let:
                \begin{equation}
                    \norm{f}_{1}=\int_{0}^{3}|f(x)|\diff{x}
                \end{equation}
                Then $\norm{\cdot}_{1}$ is a norm on the set of
                continuous functions, and thus induces a metric.
                Let $f_{n}$ be defined by:
                \begin{equation}
                    f_{n}(x)=
                    \begin{cases}
                        1,&x\leq{x}<2-\frac{1}{n}\\
                        Bob,\\
                        0,&x\geq{2}
                    \end{cases}
                \end{equation}
                Then $f_{n}$ is Cauchy, but does not converge.
            \end{lexample}
            \begin{ldefinition}{Complete Metric Spaces}
                A complete metric space is a metric space
                $(X,d)$ such that, for all Cauchy sequences
                $a:\mathbb{N}\rightarrow{X}$, $a$ is a convergent
                sequence.
            \end{ldefinition}
            \begin{theorem}
                If $(X,d)$ is a metric space, if
                $a:\mathbb{N}\rightarrow{X}$ is a Cauchy
                sequence, and if there is a convergent subseqence
                of $a$, then $a$ is a convergent sequence.
            \end{theorem}
            \begin{theorem}
                A normed vector space $(V,\norm{\cdot})$ is
                complete if and only if every absolutely
                convergent series converges.
            \end{theorem}
            \begin{proof}
                Suppose $V$ is complete and let $u_{n}$
                be absolutely convergent. That is, the sequence
                of partial sums:
                \begin{equation}
                    S_{N}=\sum_{n=1}^{N}\norm{u_{n}}
                \end{equation}
                Converges in $\mathbb{R}$. But then:
                \begin{equation}
                    \underset{N\rightarrow\infty}{\lim}
                    \sum_{k=N}^{\infty}\norm{u_{k}}=0
                \end{equation}
                Define:
                \begin{equation}
                    s_{n}=\sum_{k=1}^{n}u_{k}
                \end{equation}
                But if $m\geq{n}$, then:
                \begin{equation}
                    \norm{s_{n}-s_{m}}
                    \leq\sum_{k=n+1}^{m}\norm{u_{k}}
                    \leq\sum_{k=n+1}^{\infty}\norm{u_{k}}
                \end{equation}
                Thus, if $\varepsilon>0$ there is an $N\in\mathbb{N}$
                such that, for all $n\geq{N}$, it is true that:
                \begin{equation}
                    \sum_{k=n+1}^{\infty}\norm{u_{k}}<\varepsilon
                \end{equation}
                Therefore $s_{n}$ is a Cauchy sequence, and therefore
                there is an $s\in{V}$ such that $s_{n}\rightarrow{s}$.
                Proving the converse, suppose $u_{n}$ is a Cauchy
                sequence. Then there is an $N_{1}\in\mathbb{N}$ such
                that, for all $n,m\geq{N}_{1}$, we have
                $\norm{u_{n}-u_{m}}<1/2$. But then there is also an
                $N_{2}\in\mathbb{N}$ such that $N_{2}>N_{1}$, and for
                all $n,m>N_{2}$, $\norm{u_{n}-u_{m}}$. Continuing
                inductively, we find a sequence $u_{n_{k}}$ such that:
                \begin{equation}
                    \norm{u_{n_{k+1}}-u_{n_{k}}}<\frac{1}{2^{k}}
                \end{equation}
                Let $v_{k}=u_{n_{k+1}}-u_{n_{k}}$, and note that:
                \begin{equation}
                    \sum_{n=1}^{\infty}\norm{v_{n}}<\infty
                \end{equation}
                But then there is a $v\in{V}$ such that:
                \begin{equation}
                    \sum_{n=1}^{\infty}v_{n}=v
                \end{equation}
                But:
                \begin{align}
                    v&=\underset{N\rightarrow\infty}{\lim}
                        \sum_{k=1}^{N}v_{k}\\
                    &=\underset{N\rightarrow\infty}{\lim}
                        v_{n_{N+1}}-u_{n_{1}}
                \end{align}
                Therefore $u_{n_{k}}\rightarrow{v}+u_{n_{1}}$.
                But $u_{n}$ is Cauchy and thus if there is a
                convergent subsequence, then it is a convergent
                sequence. Therefore, $(X,\norm{\cdot})$ is complete.
            \end{proof}
            \begin{theorem}
                If $(X,\mathcal{M},\mu)$ is a measure space, and
                if $1\leq{p}\leq\infty$.
            \end{theorem}
            \begin{proof}
                Suppose that $f_{n}$ is a sequence of functions
                in $L^{P}(X,\mathcal{M},\mu)$ such that:
                \begin{equation}
                    \sum_{k=1}^{\infty}\norm{f_{n}}_{p}=B<\infty
                \end{equation}
                Define $G,G_{n}:X\rightarrow[0,\infty]$ be
                defined by:
                \begin{align}
                    G(x)&=\sum_{k=1}^{\infty}|f_{k}(x)|\\
                    G_{n}(x)&=\sum_{k=1}^{n}|f_{k}(x)|
                \end{align}
                Then, from the triangle inequality, we have that:
                \begin{equation}
                    \norm{G_{n}}_{p}\leq
                    \sum_{k=1}^{n}\norm{f_{k}}_{p}
                    \leq{B}
                \end{equation}
                Thus, by the monotone convergence theorem, we have:
                \begin{equation}
                    \int_{X}G(x)^{p}\diff{\mu}
                    =\underset{n\rightarrow\infty}{\lim}
                    \int_{X}G_{n}(x)^{p}\diff{\mu}\leq{B}^{p}
                \end{equation}
                Therefore $G\in\mathcal{L}^{p}(X)$, and thus
                $G(x)<\infty$ $\mu$ almost-everywhere. But then the
                original series converges $\mu$ almost everywhere.
                Define $F$ be:
                \begin{equation}
                    F(x)=
                    \begin{cases}
                        \sum_{n=1}^{\infty}f_{n}(x),
                        &|\sum_{n=1}^{\infty}f_{n}(x)|<\infty\\
                        0,&\textrm{Otherwise}
                    \end{cases}
                \end{equation}
                Then $|F(x)|\leq{G}(x)$, and thus
                $F\in\mathcal{L}^{p}(X)$. Moreover:
                \begin{equation}
                    \big|F(x)-\sum_{k=1}^{n}f_{k}(x)\big|^{p}
                    \leq{2}^{p}G(x)^{p}
                \end{equation}
                Therefore, by the Lebesgue Dominated Convergence
                Theorem, we have that:
                \begin{equation}
                    \norm{F-\sum_{k=1}^{n}f_{k}(x)}^{p}_{p}
                    \rightarrow{0}
                \end{equation}
                Therefore, $F\in{L}^{p}(X,\mathcal{M},\mu)$.
            \end{proof}
            \begin{ldefinition}
                  {Supremum Norm of Bounded Continuous Function}
                The supremum norm on set $C_{b}(X)$ of bounded
                continuous functions on a metric space $(X,d)$ is:
                \begin{equation}
                    \norm{f}_{\infty}=\sup_{x\in{X}}|f(x)|
                \end{equation}
            \end{ldefinition}
            From this, we can see thatn $f_{n}\rightarrow{f}$ if
            and only if $f_{n}\rightarrow{f}$ uniformly on $X$.
            \begin{theorem}
                $C_{b}(X)$ is complete in the supremum norm.
            \end{theorem}
            \begin{proof}
                Suppose that $f_{n}$ is a Cauchy sequence in
                $C_{b}(X)$. For for all $f_{n}$ and $x\in{X}$,
                $f_{n}(x)$ is a Cauchy sequence in $\mathbb{C}$.
                But $\mathbb{C}$ is complete, and thus there is a
                $c_{x}\in\mathbb{C}$ such that
                $f_{n}(x)\rightarrow{c}_{x}$. Let
                $f(x)=c_{x}$ for all $x\in{X}$. Then
                $f_{n}\rightarrow{f}$. For, let $\varepsilon>0$.
                Then there exists $N\in\mathbb{N}$ sch that, for
                all $n,m>{N}$ implies that:
                \begin{equation}
                    |f_{n}(x)-f_{m}(x)|<\varepsilon/2
                \end{equation}
                But then:
                \begin{equation}
                    |f_{n}(x)-f(x)|=
                    \underset{m\rightarrow\infty}{\lim}
                    |f_{n}(x)-f_{m}(x)|
                    \leq\frac{\varepsilon}{2}<\varepsilon
                \end{equation}
                But the uniform limit of continuous functions is
                continuous. Therefore, etc.
            \end{proof}
            \begin{theorem}
                If $(X,d)$ is a complete metric space, if
                $(E,d')$ is a subspace of $(X,d)$, and if
                $E$ is closed, then $(E,d')$ is complete.
            \end{theorem}
            \begin{proof}
                Suppose $E$ is closed and suppose $(x_{n})$ is a
                Cauchy sequence in $E$. Then $x_{n}$ is a Cauchy
                sequence in $X$, but $X$ is complete. Therefore
                there is an $x\in{X}$ such that
                $x_{n}\rightarrow{x}$. But $E$ is closed,
                and therefore $x\in{E}$. Now suppose $E$ is
                complete. Suppose $x_{n}$ is a sequence in $E$
                and that $x_{n}\rightarrow{y}$ in $X$. But
                convergent sequences are Cauchy sequences, and
                thus $x_{n}$ is a Cauchy sequence. But $E$ is
                complete and therefore $y\in{E}$.
                Therefore, $E$ is closed.
            \end{proof}
            \begin{ldefinition}{Bounded Metric Spaces}
                A bounded metric space is a metric space
                $(X,d)$ such that there exists an $x\in{X}$
                and an $r>0$ such that:
                \begin{equation}
                    X\subseteq{B}_{r}^{(X,d)}(x)
                \end{equation}
            \end{ldefinition}
            \begin{ldefinition}{Diameter of a Metric Space}
                The diameter of a bounded metric space $(X,d)$ is:
                \begin{equation}
                    \diam(X)=\sup_{x\in{X}}\{d(x,y):x,y\in{X}\}
                \end{equation}
            \end{ldefinition}
            Every bounded metric space is contained in some
            open ball.
            \begin{theorem}
                If $(X,d)$ is a metric space, and then it
                is complete if and only if for any sequence
                of non-empty closed sets
                $F:\mathbb{N}\rightarrow\mathcal{P}(X)$ such that
                $F_{n+1}\subseteq{F}_{n}$ and
                $\diam(F_{n})\rightarrow{0}$,
                there is an $x\in{X}$ such that:
                \begin{equation}
                    \{x\}=\cap_{n=1}^{\infty}F_{n}
                \end{equation}
            \end{theorem}
            \begin{proof}
                For suppose $(X,d)$ is complete, and let
                $F:\mathbb{N}\rightarrow\mathcal{P}(X)$ be a
                sequence of non-empty subsets of $X$. Then, for
                all $n\in\mathbb{N}$, $F_{n}$ is non-empty, and
                thus there is a sequence
                $a:\mathbb{N}\rightarrow{X}$ such that, for all
                $n\in\mathbb{N}$, $x_{n}\in{F_{n}}$. But then:
                \begin{equation}
                    d(a_{n},a_{m})\leq\diam(F_{\max\{n,m\}})
                \end{equation}
                But $\diam(F_{n})\rightarrow{0}$, and therefore
                $a$ is a Cauchy sequence. But $(X,d)$ is complete,
                and therefore there is an $x\in{X}$ such that
                $a_{n}\rightarrow{x}$. Moreover, there is an
                $N\in\mathbb{N}$ such that $x\in\overline{F}_{N}$.
                But $F_{N}$ is closed, and thus $x\in{F}_{N}$.
                But for all $n>N$, $F_{n}\subseteq{F}_{N}$.
                Therefore:
                \begin{equation}
                    x\in\cap_{n=1}^{\infty}F_{n}
                \end{equation}
                If $y\in\cap_{n=1}^{\infty}F_{n}$, then
                $d(x,y)\leq\diam(F_{n})$ for all $n\in\mathbb{N}$.
                But $\diam(F_{n})\rightarrow{0}$, and thus
                $d(x,y)=0$. Therefore, $x=y$. Going the other
                way, suppose $X$ has the nested set property and
                let $a:\mathbb{N}\rightarrow{X}$ be a Cauchy
                sequence in $X$. Let
                $F:\mathbb{N}\rightarrow\mathcal{P}(X)$
                be defined by:
                \begin{equation}
                    F_{n}=\overline{\{a_{k}:k\geq{n}\}}
                \end{equation}
                Then, for all $n\in\mathbb{N}$, $F_{n}$
                is non-empty, and $F_{n+1}\subseteq{F}_{n}$.
                Moreover, $\diam(F_{n})\rightarrow{0}$. Thus, by
                the nested sequence property, there is an
                $x\in{X}$ such that $x\in\cap_{n=1}^{\infty}F_{n}$.
                But then:
                \begin{equation}
                    d(a_{n},x)\leq\diam(F_{n})\rightarrow{0}
                \end{equation}
                and therefore $a_{n}\rightarrow{x}$. Thus, $a$ is
                a Cauchy sequence and $(X,d)$ is complete.
            \end{proof}
        \subsection{Compactness}
            \begin{ldefinition}{Covers}
                A cover of a subset $\mathcal{E}\subseteq{X}$ of
                a set $X$ is a subset
                $\mathcal{O}\subseteq\mathcal{P}(X)$ such that:
                \begin{equation}
                    \mathcal{E}\subseteq
                    \bigcup_{\mathcal{U}\in\mathcal{O}}
                        \mathcal{U}
                \end{equation}
            \end{ldefinition}
            \begin{ldefinition}{Sub-Cover}
                A sub-cover of a cover $\mathcal{O}$ of a subset
                $E\subseteq{X}$ of a set $X$ is a subset
                $\Delta\subseteq\mathcal{O}$ such that:
              \begin{equation}
                    \mathcal{E}\subseteq
                    \bigcup_{\mathcal{U}\in\Delta}
                        \mathcal{U}
                \end{equation}
            \end{ldefinition}
            \begin{ldefinition}{Open Covers}
                An open cover of a metric space $(X,d)$ is a cover
                $\mathcal{O}\subseteq\mathcal{P}(X)$ of $X$ such
                that, for all $\mathcal{U}\in\mathcal{O}$,
                $\mathcal{U}$ is open.
            \end{ldefinition}
            \begin{ldefinition}{Compact Sets}
                A compact metric space is a metric space $(X,d)$
                such that for any open cover $\mathcal{O}$ of
                $X$, there is a finite
                sub-cover $\Delta\subseteq\mathcal{O}$.
            \end{ldefinition}
            \begin{lexample}
                Let $X=[0,1)$ with the usual topology, and let:
                \begin{equation}
                    \mathcal{U}_{x}=[0,x)
                    \quad\quad
                    x\in(0,1)
                \end{equation}
                Then $\mathcal{O}=\{\mathcal{U}_{x}:x\in(0,1)\}$
                is an open cover of $X$, but there is no finite
                sub-cover. For given any finite sub-cover,
                there is a greatest $x$ such that
                $\mathcal{U}_{x}$ is contained in the sub-cover.
                But then for all $y\in(x,1)$, $y$ is not in
                the sub-cover. As a trivial example, any
                finite metric space is compact.
            \end{lexample}
            \begin{theorem}
                If $K$ is a subspace of $X$, then $K$ is compact
                if and only if every open cover of $K$ has a
                finite sub-cover.
            \end{theorem}
            \begin{proof}
                For suppose $(K,d_{K})$ is compact, and let
                $\mathcal{O}$ be an open cover of $K$. Then:
                \begin{equation}
                    \mathcal{O}_{K}=\{K\cup\mathcal{U}:
                        \mathcal{U}\in\mathcal{O}\}
                \end{equation}
                Is an open cover of $K$. But $K$ is compact,
                and thus there is a finite sub-cover
                $\Delta_{K}$. But then:
                \begin{equation}
                    \Delta=\{\mathcal{U}\in\mathcal{U}:
                             \mathcal{U}\cap{K}\in\Delta_{K}\}
                \end{equation}
                And this is a finite sub-cover.
            \end{proof}
            \begin{ldefinition}{Finite Intersection Property}
                A set with the finite intersection property
                in a metric space $(X,d)$ is a collection of sets
                $\mathscr{F}\subseteq\mathcal{P}(X)$ such that,
                for any sequence
                $F:\mathbb{Z}_{n}\rightarrow\mathscr{F}$, 
                it is true that $\cap_{k=1}^{n}F_{k}\ne\emptyset$.
            \end{ldefinition}
            \begin{theorem}
                A metric space $(X,d)$ is compact if and only if
                every collection $\mathscr{F}$ of closed sets in
                $X$ with the
                finite intersection property is such that:
                \begin{equation}
                    \bigcap_{\mathcal{C}\in\mathcal{F}}
                        \mathcal{C}\ne\emptyset
                \end{equation}
            \end{theorem}
            \begin{lexample}
                Let $F_{n}=[n,\infty)$, and let
                $\mathscr{F}=\{F_{n}:n\in\mathbb{N}\}$.
                Then $\mathscr{F}$ has the finite intersection
                property. However, the intersection over the
                entire set is empty, and hence $\mathbb{R}$
                (With the standard metric) is not compact.
            \end{lexample}
            \begin{ldefinition}{Totally Bounded Metric Space}
                A totally bounded metric space is a metric
                space $(X,d)$ such that, for all $\varepsilon>0$,
                there exists an $n\in\mathbb{N}$ and a sequence
                $a:\mathbb{Z}_{n}\rightarrow{X}$ such that:
                \begin{equation}
                    X=\cup_{k=1}^{n}B_{\varepsilon}^{(X,d)}(a_{k})
                \end{equation}
            \end{ldefinition}
            \begin{ldefinition}{$\varepsilon\textrm{-Nets}$}
                An $\varepsilon\textrm{-Net}$ of a subspace
                $(\mathcal{E},d_{\mathcal{E}})$ of a
                metric space $(X,d)$ is a finite collection:
                \begin{equation}
                    E=\{B_{\varepsilon}^{(X,d)}(x_{k}):
                        k\in\mathbb{Z}_{n}\}
                \end{equation}
                Such that $E$ is an open cover of $\mathcal{E}$.
            \end{ldefinition}
            \begin{lexample}
                Let $X=\ell^{2}$ and let:
                \begin{equation}
                    e_{n}(x)=
                    \begin{cases}
                        1,&k=n\\
                        0,&k\ne{n}
                    \end{cases}
                \end{equation}
                Then $e_{n}\in\ell^{2}$ and $\norm{e}_{2}=1$,
                but for all $n\ne{m}$,
                $\norm{e_{n}-e_{m}}_{2}=\sqrt{2}$. Let:
                \begin{equation}
                    B_{1}=\{f\in\ell^{2}:\norm{f}_{2}\leq{1}\}
                \end{equation}
                Then $B_{1}$ is bounded, but if
                $\varepsilon=\sqrt{2}/2$ then no finite collection
                of $\varepsilon$ balls can cover $B_{1}$ since
                each ball can contain at most one of the
                $e_{n}$. Thus any cover is infinite.
            \end{lexample}
            \begin{theorem}
                A subset of $(\mathbb{R}^{n},\norm{\cdot}_{2})$
                is totally bounded if and only if it's bounded.
            \end{theorem}
            \begin{proof}
                Totally bounded implies bounded, so it suffices
                to show that if $\mathbb{R}^{n}$ is bounded then
                it is totally bounded. Let
                $\mathcal{E}\subseteq\mathbb{R}^{n}$ be bounded.
                Then there is an $r>0$ such that:
                \begin{equation}
                    \mathcal{E}\subseteq[-r,r]^{n}
                \end{equation}
                Then, compactness, stuff like that.
            \end{proof}
            This works for any norm on $\mathbb{R}^{n}$, since all
            norm's on $\mathbb{R}^{n}$ are strongly equivalent.
            \begin{ldefinition}{Sequential Compactness}
                A sequentially compact metric space is a metric
                space $(X,d)$ such that, for all
                $a:\mathbb{N}\rightarrow{X}$, there is a convergent
                subsequence of $a$.
            \end{ldefinition}
            \begin{lexample}
                Let $X\subseteq\mathbb{R}$ be defined by:
                \begin{equation}
                    X=\{\frac{1}{n}:n\in\mathbb{N}\}\cup\{0\}
                \end{equation}
                Then $X$ is sequentially compact, with respect
                to the subspace metric.
            \end{lexample}
            \begin{theorem}
                IF $(X,d)$ is a metric space, then the following are
                equivalent:
                \begin{enumerate}
                    \item $X$ is compact.
                    \item $X$ is complete and totally bounded.
                    \item $X$ is sequentially compact.
                \end{enumerate}
            \end{theorem}
            \begin{proof}
                Suppose $(X,d)$ is not compact, and let
                $\mathcal{U}_{i}$ be an open cover with no finite
                subcover. If $X$ is totally bounded, then there is a
                finite covering of $1/2$ balls. But then at least
                one of these isn't covered by finitely many of the
                $\mathcal{U}_{i}$. Let $F_{1}$ be the closure of
                this. Then $F_{1}$ is totally bounded, and has
                a finite covering of $1/4$ balls. One of these
                must not be covered by finitely many of the
                $\mathcal{U}_{i}$. Let $F_{2}'$ be the closure
                of such a ball, and let $F_{2}=F_{1}\cap{F}_{2}'$.
                Then $F_{2}$ is closed, non-empty, and
                $\diam(F_{2})\leq{1/2}$. Continuing, we
                obtain a sequence of non-empty closed sets
                $F_{n}$ such that for all $n\in\mathbb{N}$,
                $F_{n+1}\subseteq{F}_{n}$ and
                $\diam(F_{n})<1/2^{n}$. Thus, if $X$ is complete,
                there is a unique $x$ that lies in the intersection
                of all of the $F_{n}$. But then there is a
                $\mathcal{U}_{i}$ such that $x\in\mathcal{U}_{i}$,
                and thus eventually $F_{n}\subset\mathcal{U}_{i}$,
                a contradiction. Thus, $X$ is compact. Now, suppose
                $X$ is compact and let $a:\mathbb{N}\rightarrow{X}$
                be a sequence in $X$. Let:
                \begin{equation}
                    F_{n}=\overline{\{x_{k}:l\geq{n}\}}
                    \quad\quad
                    \mathscr{F}=\{F_{n}:n\in\mathbb{N}\}
                \end{equation}
                Then $\mathscr{F}$ has the finite intersection
                property. Since $X$ is compact, the intersection of
                the $F_{n}$ is non-empty. Let $x$ be contained in
                the intersection. Then:
                \begin{equation}
                    B_{1}(x)\cap\{x_{k}:k\geq{1}\}\ne\emptyset
                \end{equation}
                Pick $n_{1}$ such that $x_{n_{1}}\in{B}_{1}(x)$.
                Then there is an $n_{2}>n_{1}$ such that
                $x_{n_{2}}\in{B}_{1/2}(x)$. Continuing, we obtain
                a subsequence $n_{k}$ such that
                $x_{k}\in{B}_{1/k}(x)$, and thus
                $x_{k}\rightarrow{x}$. Finally, we show that
                sequential compactness implies that $X$ is
                complete and totally bounded. For suppose $X$ is
                not totally bounded. Then there exists
                $\varepsilon>0$ such that $X$ has no finite
                covering of $\varepsilon$ balls. We can thus
                obtain a sequence $a:\mathbb{N}\rightarrow{X}$
                such that, for all $n\ne{m}$,
                $d(a_{n},a_{m})\geq\varepsilon$. But this
                has no convergence subsequence, for any convergent
                subsequence would be a Cauchy sequence. Moreover,
                $X$ is complete. For suppose not, and let
                $a:\mathbb{N}\rightarrow{X}$ be a Cauchy sequence
                and suppose it does not converge. But then there
                is no convergent subsequence, since Cauchy
                sequences with convergent subsequences converge.
                Thus, $X$ is complete.
            \end{proof}
            \begin{ltheorem}{Heine-Borel Theorem}
                A subset $\mathcal{E}\subseteq\mathbb{R}^{n}$ is
                compact with respect to the standard topology if and
                only if $\mathcal{E}$ is closed and bounded.
            \end{ltheorem}
            This theorem does not generalize to other spaces. For
            consider $\ell^{2}$ and the closed unit ball about the
            origin. This is closed and bounded, but it is not
            compact. This is simply because it is not totally
            bounded, nor is it sequentially compact.
            \begin{ltheorem}{Extreme Value Theorem}
                If $(X,d)$ is a compact metric space and if
                $f:X\rightarrow\mathbb{R}$ is continuous, then
                $f$ attains it's maximum and minimum. In particular,
                if $f:X\rightarrow\mathbb{C}$ is continuous, then
                $f$ is bounded.
            \end{ltheorem}
            \begin{proof}
                Note that if $f:X\rightarrow\mathbb{C}$ is
                continuous, then $|f|:X\rightarrow\mathbb{R}$ is
                continuous, so we only need to prove the first
                statement. For if $X$ is compact, then $f(X)$ is
                compact, for $f$ is continuous. But then $f(X)$
                is closed and bounded. Let:
                \begin{equation}
                    M=\underset{x\in{X}}\sup\{f(x)\}
                \end{equation}
                Then, since $f(X)$ is bounded, $M\in\mathbb{R}$.
                But then there is a sequence
                $a:\mathbb{N}\rightarrow{X}$ such that
                $f(a_{n})\rightarrow{M}$. But if $X$ is
                compact, then it is sequentially compact, and
                thus there is an $x\in{X}$ an a subsequence
                $a_{k}$ such that $a_{k_{n}}\rightarrow{x}$.
                But then $f(x)=M$. Similarly for the minimum value.
            \end{proof}
            \begin{ldefinition}{Lebesgue Number}
                A Lebesgue Number of an open cover
                $\mathcal{O}$ of a metric space $(X,d)$ is
                a non-zero number $d>0$ such that, for all
                $x\in{X}$, there exists
                a $\mathcal{U}\in\mathcal{O}$ such that:
                \begin{equation}
                    B_{d}^{(X,d)}(x)\subseteq\mathcal{U}
                \end{equation}
            \end{ldefinition}
            \begin{lexample}
                Let $X=\mathbb{R}$, and let $d$ be the
                standard metric. Let
                $\mathcal{O}=\{\mathcal{U}_{i}:i=1,2,3\}$ where:
                \begin{equation}
                    \mathcal{U}_{1}=(-\infty,1)
                    \quad\quad
                    \mathcal{U}_{2}=(0,2)
                    \quad\quad
                    \mathcal{U}_{3}=(1,\infty)
                \end{equation}
                Then $d=1/2$ is a Lebesgue number of this cover.
                Letting $X=(0,1)$ with the standard metric, for all
                $x\in{X}$ the is a $\delta_{x}>0$ such that:
                \begin{equation}
                    B_{\delta_{x}}^{(X,d)}(x)
                    \subseteq{X}
                \end{equation}
                And thus these open balls are a covering of the unit
                interval, but this covering has no Lebesgue number.
            \end{lexample}
            \begin{ltheorem}{Lebesgue Covering Lemma}
                If $(X,d)$ is a compact metric space, and if
                $\mathcal{O}$ is an open covering of $X$, then
                $\mathcal{O}$ has a Lebesgue number.
            \end{ltheorem}
            \begin{proof}
                Suppose not. Suppose $(X,d)$ is compact, and suppose
                that $\mathcal{O}$ is a covering of $X$ with no
                Lebesgue number. But then, for all $n\in\mathbb{N}$,
                there is an $a_{n}$ such that, for all
                $\mathcal{U}\in\mathcal{O}$:
                \begin{equation}
                    B_{1/n}^{(X,d)}(a_{n})\not\subset\mathcal{U}
                \end{equation}
                But $X$ is compact, and thus there is a convergent
                subsequence such that $a_{k_{n}}\rightarrow{X}$. 
                But then there is a $\mathcal{U}\in\mathcal{O}$ such
                that $x\in\mathcal{U}$. But $\mathcal{U}$ is open,
                and thus there is an $r>0$ such that:
                \begin{equation}
                    B_{r}^{(X,d)}(x)\subseteq\mathcal{U}
                \end{equation}
                Let $N\in\mathbb{N}$ be such that, for all
                $k_{n}>N$, $d(x_{k_{n}},x)<r/2$. Let
                $n>N$ be such that $1/k_{n}<r/2$. But then:
                \begin{equation}
                    B_{1/k_{n}}(a_{k_{n}})\subseteq\mathcal{U}
                \end{equation}
                A contradiction.
            \end{proof}
            \begin{theorem}
                If $(X.d)$ is a compact metric space, if $(Y,\rho)$
                is a metric space, and if $f:X\rightarrow{Y}$ is a
                continuous function, then $f$ is
                uniformly continuous.
            \end{theorem}
            \begin{proof}
                For let $\varepsilon>0$. since $f$ is
                continuous, for all $x\in{X}$ there is a
                $\delta_{x}$ such that, for all $y\in{X}$ such
                that $d(x,y)<\delta_{x}$, it is true that
                $\rho(f(x),f(y))<\varepsilon/2$. But then:
                \begin{equation}
                    X\subseteq
                        \bigcup_{x\in{X}}B_{\delta_{x}}^{(X,d)}(x)
                \end{equation}
                But $X$ is compact, and thus this covering has a
                Lebesgue number. Let $\delta$ be such a
                Lebesgue number. But then if $d(x,y)<\delta$,
                then there is a $z\in{X}$ such that
                $x,y\in{B}_{\delta_{z}}(z)$. But then:
                \begin{equation}
                    \rho(f(x),f(x))\leq
                    \rho(f(x),f(z))+\rho(f(z),f(y))
                    <\varepsilon
                \end{equation}
            \end{proof}
    \section{Arzela-Ascoli Theorem}
        \begin{ltheorem}{Arzela-Ascoli Theorem}
            If $X$ is a compact metric space, if
            $F_{n}\in{C}(X)$ is a sequence of equicontinuous
            point-wise bounded functions, then $F_{n}$ has a
            uniformly convergent subsequence.
        \end{ltheorem}
        \begin{theorem}
            If $X$ is a compact metric space and
            $\mathscr{F}\subseteq{C}(X)$ is a closed subset with
            respect to the uniform norm, and if
            $\mathscr{F}$ is equicontinuous on $X$ and point-wise
            bounded, then $\mathscr{F}$ is compact.
        \end{theorem}
        \begin{proof}
            It suffices to show that $\mathscr{F}$ is sequentially
            compact. Let $F_{n}$ be s sequence in $\mathscr{F}$.
            The by the Arzela-Ascoli theorem, there is a uniformly
            convergent subsequence $F_{k_{n}}$. But $\mathscr{F}$
            is closed, and thus the limit function is contained
            in $\mathscr{F}$. Thus, $\mathscr{F}$ is sequentially
            compact. But sequentially compact metric spaces are
            compact. Therefore, etc.
        \end{proof}
        \begin{theorem}
            If $X$ is a compact metric space and
            $\mathscr{F}\subseteq{C}(X)$ is a closed subset with
            respect to the uniform norm, and if
            $\mathscr{F}$ is equicontinuous on $X$ and point-wise
            bounded, then $\mathscr{F}$ is uniformly bounded.
        \end{theorem}
        \begin{proof}
            For $\mathscr{F}$ is compact by the previous theorem.
            But then $\mathscr{F}$ is bounded with respect to
            $\norm{\cdot}_{\infty}$. Therefore, $\mathscr{F}$ is
            uniformly bounded.
        \end{proof}
        \begin{theorem}
            If $X$ is a compact metric space and if
            $\mathscr{F}\subseteq{C}(X)$ is closed, equicontinuous,
            and uniformly bounded on $X$, then $\mathscr{F}$ is
            compact.
        \end{theorem}
        \begin{proof}
            For suppose $\mathscr{F}$ is compact. Then $\mathscr{F}$
            is closed and uniformly bounded. Thus it suffices to
            show that $\mathscr{F}$ is equicontinuous. Suppose not.
            Then there is a point $x\in{X}$ such that
            $\mathscr{F}$ is not equicontinuous at $x$. Thus,
            there exists an $\varepsilon>0$ such that, for all
            $\delta>0$, there are points $x,y$ such that
            $d(x,y)<\delta$, but $|f(x)-f(y)|\geq\varepsilon$
            for some $f\in\mathscr{F}$. Thus, for all
            $n\in\mathbb{N}$, there is an $x_{n}\in{X}$ such that
            $d(x,x_{n})<1/n$, and
            $|f_{n}(x)-f_{n}(x_{n})|\geq\varepsilon_{0}$. But if
            $\mathscr{F}$ is compact, then $f_{n}$ has a convergent
            subsequence $f_{k_{n}}$. Let $f$ be the limit.
            Since $\mathscr{F}$ is compact, $f\in\mathscr{F}$.
            But then $f_{k_{n}}(x_{k_{n}})\rightarrow{f}(x)$. But
            then there is an $N\in\mathbb{N}$ such that,
            for $k_{n}>N$,
            $\norm{f_{k_{n}}-f}_{\infty}<\varepsilon_{0}/3$.
            But then:
            \begin{align}
                |f(x_{k_{n}})-f(x)|&=
                |f(x_{k_{n}})-f_{k_{n}}(x_{k_{n}})
                +f_{k_{n}}(x_{k_{n}})-f_{k_{n}}(x)
                +f_{k_{n}}(x)-f(x)|\\
                &\geq|f_{k_{n}}(x_{k_{n}})-f_{k_{n}}(x)|+
                |f(x_{k_{n}})-f_{k_{n}}(x_{k_{n}})
                +f_{k_{n}}(x)-f(x)|\\
                &>\varepsilon
            \end{align}
            A contradiction.
        \end{proof}
        \begin{ldefinition}{Baire Space}
            A Baire space is a metric space $(X,d)$ such that, for
            countable collection of open and dense sets, the
            intersection is also dense.
        \end{ldefinition}
        This is a topological property, and so Baire spaces can
        be defined for a more general topological space. The
        interior of a set in a topological space is:
        \begin{equation}
            \Int(A)=
            \bigcup\{\mathcal{U}\in\tau:\mathcal{U}\subseteq{A}\}
        \end{equation}
        \begin{theorem}
            A metric space $(X,d)$ is a Baire space if and only
            if given a countable collection $F_{n}$ of closed
            sets such that the union over all of $F_{n}$ has
            non-empty interior, then at least one of the $F_{n}$
            has non-empty interior.
        \end{theorem}
        \begin{theorem}
            There exist countable Baire spaces.
        \end{theorem}
        Suppose $\mathcal{U}\subseteq{X}$ is open at
        $x_{0}\in\mathcal{U}$. There there is a $\delta>0$ such
        $B_{\delta}(x)\subseteq\mathcal{U}$. Then:
        \begin{equation}
            \overline{B_{\delta/2}(x)}\subseteq{B}_{\delta}(x)
        \end{equation}
        Thus, $\overline{B_{\delta/2}(x)}\subseteq\mathcal{U}$
        and the diameter is less than $2\delta$.
        \begin{ltheorem}{Baire Category Theorem}
            Every complete metric space is a Baire space.
        \end{ltheorem}
        \begin{proof}
            Suppose $\mathcal{O}_{n}\subseteq{X}$ is open and
            dense for all $n\in\mathbb{N}$. Let $x_{0}\in{X}$ and
            $r_{0}>0$. It will suffice to show that:
            \begin{equation}
                B_{r_{0}}(x_{0})\cap\bigcap_{n\in\mathbb{N}}
                    \mathcal{O}_{n}\ne\emptyset
            \end{equation}
            Inductively, we create a sequence of points $x_{k}$
            and real numbers $r_{k}>0$ such that $r_{k}$ is strictly
            monotonically decreasing, and thus that:
            \begin{equation}
                \overline{B_{r_{k+1}}(x_{k+1})}
                \subseteq{B}_{r_{k}}(x_{k})\cap\mathcal{O}_{k+1}
            \end{equation}
        \end{proof}
        Consider the set of all lines through the origin with
        rational slope. The complete of any given line is the
        union of two open half planes, which are open and dense
        subsets of $\mathbb{R}^{2}$. Since we have only a countable
        collection of such lines, the intersection of the complement
        is dense in $\mathbb{R}^{2}$. Baire's Category Theorem
        holds even if $(X,d)$ is not complete, but is equivalent
        to a complete metric. For example, let $X=(0,1)$ and let
        $d(x,y)=|x-y|$ be the standard metric. This is not a
        complete space, but is homeomorphic to $\mathbb{R}$, 
        which is a complete metric space. Using this homeomorphism,
        we can find a metric $\tilde{d}$ on $(0,1)$ that is complete
        and which is equivalent to the original metric. Thus,
        $(0,1)$ is a Baire space.
        \begin{theorem}
            If $V$ is a non-empty open subset of a complete metric
            space $(X,d)$, then there is a metric $\tilde{d}$ such
            that $(V,\tilde{d})$ is complete.
        \end{theorem}
        Hence, $V$ is a Baire space. Then, given a set
        $F_{n}$ of closed subsets of $X$ such that:
        \begin{equation}
            V=\bigcup_{n=1}^{\infty}(V\cap{F}_{n})
        \end{equation}
        Then some $F_{n}\cap{V}$ has non-empty interior in $V$,
        and hence in $X$.
        \begin{theorem}
            If $X$ is a Baire space and if $f_{n}$ is a sequence
            of continuous function in $C(X)$ which converges
            point-wise to $f:X\rightarrow\mathbb{C}$, then
            the set:
            \begin{equation}
                \{x\in{X}:f\textrm{ is continuous at }x\}
            \end{equation}
            Is dense in $X$.
        \end{theorem}
        \begin{proof}
            Let $\varepsilon>0$ and define:
            \begin{align}
                A_{N}(\varepsilon)&=
                \{x:|f_{n}(x)-f_{m}(x)|\leq\varepsilon,
                    n,m\in\mathbb{N}\}\\
                &=\bigcap_{n,m\geq{N}}
                    \{x:|f_{n}(x)-f_{m}(x)|\leq\varepsilon\}
            \end{align}
            Then $A_{N}(\varepsilon)$ is closed. But also:
            \begin{equation}
                X=\bigcup_{N=1}^{\infty}A_{N}(\varepsilon)
            \end{equation}
            Thus, by the Baire category theorem, we have:
            \begin{equation}
                \mathcal{U}(\varepsilon)=
                \bigcup_{N=1}^{\infty}\Int(A_{N}(\varepsilon))
            \end{equation}
            Is non-empty and open. Moreover,
            $\mathcal{U}(\varepsilon)$ is dense. But then:
            \begin{equation}
                \mathcal{C}=\bigcap_{n=1}^{\infty}
                \mathcal{U}(\frac{1}{n})
            \end{equation}
            Is dense in $X$, and $f$ is continuous at all
            $x\in\mathcal{C}$.
        \end{proof}
    \section{Lecture 7-ish}
        The Baire Category Theorem says that every complete metric
        space is a Baire space. The notion of Baire space is a
        topological property, and not a metric property. Thus, even
        if $(X,d)$ is not complete but is equivalent to a complete
        metric space $(X,\tilde{d})$, then $(X,d)$ is a Baire space.
        A topological space is called completely metrizable if there
        is a metric on the space that is complete and generates the
        topology. Given a complete metric space $(X,d)$, every
        non-empty open set $\mathcal{V}$ has a metric
        $d_{\mathcal{V}}$ such that $(\mathcal{V},d_{\mathcal{V}})$
        is complete, and is therefore a Baire space. Thus, if:
        \begin{equation}
            \mathcal{V}=\bigcup_{n\in\mathbb{N}}\Big(
                \mathcal{V}\cap{F}_{n}\Big)
        \end{equation}
        Where $F_{n}$ is closed for all $n\in\mathbb{N}$, then
        for some $N\in\mathbb{N}$, $\mathcal{V}\cap{F}_{N}$ has
        interior.
        \begin{theorem}
            If $X$ is a Baire space, and if
            $F_{n}$ is a sequence of continuous functions that
            converges point-wise to $f:X\rightarrow\mathbb{C}$, then
            the set $\mathcal{D}$ defined by:
            \begin{equation}
                \mathcal{D}=
                    \{x\in{X}:\textrm{$f$ is continuous as $x$}\}
            \end{equation}
            Then $\mathcal{D}$ is dense in $X$.
        \end{theorem}
        \begin{proof}
            For let $\varepsilon>0$, and let:
            \begin{equation}
                A_{N}(\varepsilon)=
                \{x\in{X}:|f_{n}(x)-f_{m}(x)|\leq\varepsilon,n,m>N\}
            \end{equation}
            Then, for all $N\in\mathbb{N}$, $A_{N}(\varepsilon)$ is
            closed. Let $\mathcal{U}(\varepsilon)$ be defined by:
            \begin{equation}
                \mathcal{U}=\bigcup_{n\in\mathbb{N}}
                    \Int\Big(A_{N}(\varepsilon)\Big)
            \end{equation}
            Then $\mathcal{U}(\varepsilon)$ is open and dense. It
            is open for it is the union of open sets. For let
            $\mathcal{V}$ be a non-empty subset. Then:
            \begin{equation}
                \mathcal{V}=\bigcup_{n\in\mathbb{N}}
                    \Big(A_{n}(\varepsilon)\cap\mathcal{V}\Big)
            \end{equation}
            Hence there exists an $N\in\mathbb{N}$ such that:
            \begin{equation}
                A_{N}(\varepsilon)\cap\mathcal{V}\ne\emptyset
            \end{equation}
            And this has interior, and therefore:
            \begin{equation}
                \Int(A_{N}(\varepsilon))\cap\mathcal{V}\ne\emptyset
            \end{equation}
            Therefore,
            $\mathcal{V}\cap\mathcal{U}(\varepsilon)\ne\emptyset$.
            Now, define:
            \begin{equation}
                \mathcal{V}=\bigcap_{n\in\mathbb{N}}
                    \mathcal{U}\big(\frac{1}{n}\big)
            \end{equation}
            And therefore $\mathcal{C}$ is dense in $X$. We
            now want to show that $f$ is continuous for all
            $x\in\mathcal{C}$. For let $x_{0}\in\mathcal{C}$ and
            let $\varepsilon>0$. Let $k\in\mathbb{N}$ be such that
            $k^{\minus{1}}<\varepsilon$. Then
            $x_{0}\in\mathcal{U}(k^{\minus{1}})$ and thus there is
            an $N\in\mathbb{N}$ such that:
            \begin{equation}
                x_{0}\in\Int\big(A_{N}(k^{\minus{1}})\big)
            \end{equation}
            But $f_{N}$ is continuous, and thus there is a
            neighborhood $\omega$ of $x_{0}$ such that, for all
            $y\in\omega$:
            \begin{equation}
                |f_{N}(x_{0})-f_{N}(y)|<\varepsilon/3
            \end{equation}
            Shrink $\omega$ so that it resides inside of
            $\Int(A_{N}(k^{\minus{1}})$. Then:
            \begin{equation}
                |f_{n}(y)-f_{N}(y)|<k^{\minus{1}}
                \quad\quad
                n\geq{N}
            \end{equation}
            But then, use the Cauchy trick and you're down.
        \end{proof}
    \section{Normed Vector Spaces}
        \begin{ldefinition}{Normed Vector Spaces}
            A normed vector space over a field
            $\mathbb{F}\subseteq\mathbb{C}$, denoted
            $(V,\norm{\cdot})$ is a vector space $V$ over
            $\mathbb{F}$ and a norm $\norm{\cdot}$ on $V$.
        \end{ldefinition}
        \begin{ldefinition}{Banach Space}
            A Banach space is a normed vector space
            $(V,\norm{\cdot})$ such that the metric $d$ induced by
            the norm $\norm{\cdot}$ is complete on $V$.
        \end{ldefinition}
        Normed spaces are special. Give $\mathbf{v}\in{V}$, and
        for $r>0$, we have:
        \begin{equation}
            B_{r}^{(V,\norm{\cdot})}(\mathbf{x})=
            B_{r}^{(V,\norm{\cdot})}(\mathbf{0})+\mathbf{x}
        \end{equation}
        That is, open balls about arbitrary points are merely
        translations of an open ball about the origin.
        \begin{equation}
            |\norm{\mathbf{v}}-\norm{\mathbf{u}}|
            \leq\norm{\mathbf{v}-\mathbf{u}}
        \end{equation}
        And thus the map $\mathbf{v}\mapsto\norm{\mathbf{v}}$ is
        continuous. The closure of an open ball is the closed ball.
        \begin{equation}
            \overline{B_{r}^{(V,\norm{\cdot})}(\mathbf{x})}
            =\{\mathbf{y}\in{V}:
                \norm{\mathbf{x}-\mathbf{y}}\leq{r}\}
        \end{equation}
        We can also multiply open balls by constants, to get
        the following:
        \begin{equation}
            \varepsilon{B}_{r}^{(V,\norm{\cdot})}(\mathbf{x})=
            B_{\varepsilon{r}}^{(V,\norm{\cdot})}(\mathbf{x})
        \end{equation}
        \begin{theorem}
            Suppose $X$ and $Y$ are normed vector spaces over
            $\mathbb{F}\subseteq\mathbb{C}$. Let $T:X\rightarrow{Y}$
            be a linear transformation. Then the following
            are equivalent:
            \begin{enumerate}
                \item $T$ is continuous.
                \item $T$ is continuous at some $x_{0}\in{X}$.
                \item There is an $\alpha>0$ such that
                      $\norm{Tx}\leq\alpha\norm{x}$.
            \end{enumerate}
        \end{theorem}
        \begin{proof}
            Suppose $T$ is continuous at $x_{0}$. Then there is a
            $\delta>0$ such that:
            \begin{equation}
                T\Big(\overline{B_{\delta}(x_{0})}\big)
                \subseteq{B}_{1}(T(x_{0}))
            \end{equation}
            But:
            \begin{align}
                T\Big(\overline{B_{\delta}(x_{0})}\big)
                &=T\Big(\overline{B_{\delta}(0)}\big)+T(x_{0})\\
                B_{1}(T(x_{0}))=
                B_{1}(0)+T(x_{0})
            \end{align}
            Now suppose $z\ne{0}$. Then:
            \begin{equation}
                \norm{T(z)}=
                \norm{\frac{\norm{z}}{\delta}
                      T\Big(\frac{\delta{z}}{\norm{z}}\Big)}
                \leq\frac{1}{\delta}\norm{z}
            \end{equation}
            Let $\alpha=\delta^{\minus{1}}$.
            Proving the next one:
            \begin{equation}
                \norm{T(x)-T(y)}=\norm{T(x-y)}
                \leq\alpha\norm{x-y}
            \end{equation}
            And so we have continuity.
        \end{proof}
        There are linear maps that are not bounded. Let
        $\ell_{1}^{0}=\Span\{e_{k}:x\in\mathbb{N}\}$. Map
        $e_{k}\rightarrow{k}e_{k}$. Let
        $\norm{\cdot}_{a}$ and $\norm{\cdot}_{b}$ be norms on
        $X$ that induce the same topology on $X$. Consider the map
        $id:(X,\norm{\cdot}_{a})\rightarrow(X,\norm{\cdot}_{b})$.
        Since the topologies are the same, $id$ is continuous.
        Then there is a $c\geq{0}$ such that:
        \begin{equation}
            \norm{x}_{b}\leq{c}\norm{x}_{a}
        \end{equation}
        We can go the other way as well, and thus we see that
        equivalence implies strongly equivalent. This is not true
        in a general metric space.
        \begin{ldefinition}{}
            Let $\mathscr{L}(X,Y)$ be the set of bounded linear
            transformation $T:X\rightarrow{Y}$. The operator
            norm on $T$ is:
            \begin{equation}
                \norm{T}=\sup\{\norm{T}(x):\norm{x}\leq{1}\}
            \end{equation}
        \end{ldefinition}
        \begin{theorem}
            The operator norm on $\mathscr{L}(X,Y)$ is a norm.
        \end{theorem}
        \begin{proof}
            For we have:
            \begin{equation}
                \norm{S\circ{T}}\leq\norm{S}\norm{T}
            \end{equation}
        \end{proof}
        \begin{ldefinition}{Algebra}
            An algebra over a field $\mathbb{F}$ is a vector
            space $A$ over $\mathbb{F}$ such that $A$ has a ring
            structure $(A,\times,+)$ such that:
            \begin{equation}
                \lambda(xy)=(\lambda{x})y
                =x(\lambda(y))
                \quad\quad
                x,y\in{A}
                \quad\lambda\in\mathbb{F}
            \end{equation}
        \end{ldefinition}
        \begin{lexample}
            $\mathbb{R}[x]$, $\mathbb{C}[x]$, $M_{n}(\mathbb{F})$,
            and $C_{b}(X)$.
        \end{lexample}
        \begin{ldefinition}{Normed Algebra}
            A normed algebra is a normed space $(A,\norm{\cdot})$
            such that $A$ is an algebra and such that,
            for all $x,y\in{A}$:
            \begin{equation}
                \norm{xy}\leq\norm{x}\norm{y}
            \end{equation}
        \end{ldefinition}
        \begin{ldefinition}{Banach Algebra}
            A Banach Algebra is a normed algebra $(A,\norm{\cdot})$
            such that $(A,\norm{\cdot})$ is a Banach space.
        \end{ldefinition}
        \begin{theorem}
            If $Y$ is a Banach space, then $\mathscr{L}(X,Y)$
            is a Banach space.
        \end{theorem}
        \begin{theorem}
            If $X$ is a Banach Algebra, then
            $\mathscr{L}(A)$ is a Banach algebra.
        \end{theorem}
        \begin{proof}
            For suppose $T_{n}$ is Cauchy in
            $\mathscr{L}(X,Y)$. Then for all $x\in{X}$,
            $T_{n}(x)$ is Cauchy in $Y$, and thus
            $T_{n}(x)$ converges to some $y\in{Y}$. Let
            $T:X\rightarrow{Y}$ be this limit function. Then
            $T:X\rightarrow{Y}$ is a linear map. But since
            $T_{n}$ is Cauchy, it is uniformly bounded. But then
            there is an $M\in\mathbb{R}^{+}$ such that:
            \begin{equation}
                \norm{T_{n}}\leq{M}
            \end{equation}
            And thus:
            \begin{equation}
                \norm{T{x}}=
                \underset{n\rightarrow\infty}{\lim}
                \norm{T_{n}x}\leq
                \lim\underset{n}{\sup}\norm{T_{n}}\norm{x}
                \leq{M}\norm{x}
            \end{equation}
            Therefore, etc.
        \end{proof}
    \section{Lecture 9, I Think}
        Let $X_{\lambda}$ be a Banach space for all
        $\lambda\in\Lambda$. Then the product is:
        \begin{equation}
            \prod_{\lambda\in\Lambda}X_{\lambda}
            =\{f:\Lambda\rightarrow\bigcup_{\lambda\in\Lambda}
            X_{\lambda}:f(\lambda)\in{X}_{\lambda}\}
        \end{equation}
        Generally, we think of the indexing set to be finite,
        $\Lambda=\{1,\dots,n\}$. The product space is then:
        \begin{equation}
            \prod_{\lambda=1}^{n}X_{\lambda}=
            X_{1}\times\dots{X}_{n}
        \end{equation}
        Functions are therefore $n$ tuples. There's no reason to
        expect that this will be a Banach space in any reasonable
        way. Thus we define the Banach Space Direct-Product.
        \begin{ldefinition}{Banach Space Directo Product}
            The Banach Space Direct Product of a set of Banach
            spaces $X_{\lambda}$ indexed over $\Lambda$ is:
            \begin{equation}
                \prod_{\lambda\in\Lambda}^{*}X_{\lambda}
                =\{f\in\prod_{\lambda\in\Lambda}X_{\lambda}:
                \underset{\lambda\in\Lambda}{\sup}
                f(\lambda)<\infty\}
            \end{equation}
        \end{ldefinition}
        Then $\norm{x}=\sup_{\lambda}\norm{x_{\lambda}}$ is a norm
        on the product space.
        \begin{ltheorem}{Open Mapping Theorem}
            If $X$ and $Y$ are Banach spaces and if
            $T\in\mathcal{L}(X,Y)$ is surjective, then $T$ is an
            open map.
        \end{ltheorem}
        \begin{proof}
            It will suffice to find $r>0$ such that:
            \begin{equation}
                B_{r}^{Y}\subseteq
                T\Big(B_{1}^{X}(0)\Big)
            \end{equation}
            By homogeneity, $T(B_{\delta}^{X}(0))$ is a
            neighborhood of $0$ for all $\delta>0$. By linearity,
            $T(B_{\delta}(x))$ is a neighborhood of
            $T(x)$ for all $x\in{X}$ and for all $\delta>0$. But
            if $V\subseteq{X}$ is open, and $x\in{V}$, then there
            is a $\delta>0$ such that $B_{\delta}(x)\subseteq{V}$.
            Thus, $T(B_{\delta}(x))$ is a neighborhood of $T(x)$
            in $T(V)$. There is al an $r>0$ such that:
            \begin{equation}
                B_{r}^{Y}(0)\subseteq
                \overline{T(B_{1}^{X}(0)))}
            \end{equation}
            For let $\alpha\in(0,1)$. Note that:
            \begin{equation}
                T(B_{\alpha}(x))=\alpha{T}(B_{1}(0))
            \end{equation}
            And also:
            \begin{equation}
                B_{\alpha{r}}(0)\subseteq
                \overline{\alpha{T}(B_{1}(0))}
            \end{equation}
            Let $y\in{B}_{r}(0)$. Then there is a
            $y_{1}\in{T(B_{1}(0))}$ such that:
            \begin{equation}
                \norm{y-y_{1}}<\frac{r}{2}
            \end{equation}
            But then $y-y_{1}\in{B}_{r/2}(0)$. But then there is a
            $y_{2}\in{T}(B_{1/2}(0))$ such that:
            \begin{equation}
                \norm{y_{2}}<\frac{r}{4}
            \end{equation}
            That is:
            \begin{equation}
                \norm{y-y_{1}-y_{2}}<\frac{r}{2^{2}}
            \end{equation}
            Continuing we obtain a sequence $y_{n}$ such that:
            \begin{equation}
                y_{n}\in{T}(B_{1/2^{n}}(0))
            \end{equation}
            And such that:
            \begin{equation}
                \norm{y-\sum_{k=1}^{n}y_{k}}<
                \frac{r}{2^{n}}
            \end{equation}
            Note that there exists $x_{n}\in{X}$ such that
            $T(x_{n})=y_{n}$ and:
            \begin{equation}
                \norm{x_{n}}<\frac{1}{2^{n-1}}
            \end{equation}
            But then there is an $x\in{X}$ such that:
            \begin{equation}
                x=\sum_{n=1}^{\infty}x_{n}
            \end{equation}
            Since $X$ is complete and since this series converges.
            But $T$ is continuous, and therefore $T(x)=y$. But:
            \begin{equation}
                \norm{x}\leq\sum_{n=1}^{\infty}\norm{x_{n}}
                <\sum_{n=0}^{\infty}\frac{1}{2^{n}}=2
            \end{equation}
            That is,:
            \begin{equation}
                B_{r}(0)\subseteq{T}(B_{2}(0))
            \end{equation}
            And therefore:
            \begin{equation}
                B_{r/2}(0)\subseteq
                T_{1}(B_{1}(0))
            \end{equation}
            If $T$ is surjective, we can write:
            \begin{equation}
                Y=\bigcup_{n\in\mathbb{N}}
                \overline{T(B_{n}(0))}
            \end{equation}
            But $Y$ is a Banach space, and thus by the Baire
            category theorem, there is an $n\in\mathbb{N}$ such
            that $\overline{T(B_{n}(0))}$ has interior. Therefore,
            etc.
        \end{proof}
        \begin{lexample}
            Recall that:
            \begin{equation}
                \ell_{0}^{p}=
                \{x\in\ell^{p}:\exists{N\in\mathbb{N}},
                    \forall_{n>N},x_{n}=0\}
            \end{equation}
            Let $\ell_{0}^{p}$ be defined by:
            \begin{equation}
                \ell_{0}^{p}=\Span\{e_{n}:n\in\mathbb{N}\}
            \end{equation}
            Define $T:\ell_{0}^{2}\rightarrow\ell_{0}^{p}$ by:
            \begin{equation}
                T(e_{n})=\frac{1}{n}e_{n}
            \end{equation}
            Then $T$ is bounded and $\norm{T}\leq{1}$. Note that
            $T$ is bijective and has an inverse:
            \begin{equation}
                T^{\minus{1}}(e_{n})=ne_{n}
            \end{equation}
            And this is not bounded, so
            $T^{\minus{1}}\notin\mathcal{L}(\ell_{0}^{p})$.
            This can happen since $\ell_{0}^{P}$ is not complete.
        \end{lexample}
        \begin{ltheorem}{Inverse Mapping Theorem}
            If $X$ and $Y$ are Banach spaces and
            $T\in\mathcal{L}(X,Y)$ is bijective, then
            $T^{\minus{1}}\in\mathcal{L}(Y,X)$.
        \end{ltheorem}
        \begin{proof}
            Note that $T^{\minus{1}}$ is linear. Since $T$ has to be
            open by the open mapping theorem, $T^{\minus{1}}$ is
            continuous. But continuous linear functions are bounded.
            Therefore, $T^{\minus{1}}$ is bounded.
        \end{proof}
        \begin{ltheorem}{Closed Graph Theorem}
            If $X$ and $Y$ are Banach spaces and if
            $T:X\rightarrow{Y}$ is linear, then
            $T\in\mathcal{L}(X,Y)$ if and only if the graph of
            $T$ is closed in $X\times{Y}$.
        \end{ltheorem}
        \begin{proof}
            Note that $X\times{Y}$ is a Banach space and
            $(x_{n},y_{n})\rightarrow(x,y)$ if and only if
            $x_{n}\rightarrow{x}$ and $y_{n}\rightarrow{y}$.
            If $T$ is bounded, then the graph of $T$ is closed.
            Now, suppose that the graph of $T$ is closed. But
            then the graph is a Banach space. The projection map
            $P:T\rightarrow{X}$ given by $P((x,T(x))=x$ is
            a bounded bijection. Hence, $P^{\minus{1}}$ is bounded.
            Let $P_{2}$ be defined by $P_{2}(x,T(x))=T(x)$. Then
            $P_{2}$ is bounded. But:
            \begin{equation}
                T(x)=P_{2}\circ{P}_{1}^{\minus{1}}(x)
            \end{equation}
            And therefore $T$ is bounded.
        \end{proof}
        \begin{lexample}
            Suppose $T_{n}\in\mathcal{L}(X,Y)$ and suppose, for
            all $x\in{X}$:
            \begin{equation}
                T(x)=\underset{n\rightarrow\infty}{\lim}T_{n}(x)
            \end{equation}
            Then we have that $T$ is linear. Is
            $T\in\mathcal{L}(X,Y)$? We have:
            \begin{equation}
                \norm{T}=
                \norm{\underset{n\rightarrow\infty}{\lim}T_{n}}
                \leq\underset{n\rightarrow\infty}{\lim}\sup_{n}
                \norm{T_{n}}
            \end{equation}
        \end{lexample}
        \begin{ltheorem}{Principle of Uniform Boundedness}
            If $X$ and $Y$ are Banach spaces, if
            $T_{\lambda}\in\mathcal{L}(X,Y)$ for all
            $\lambda\in\Lambda$, and if for all $x\in{X}$ we have:
            \begin{equation}
                \norm{\{\norm{T_{\lambda}(x)}:\lambda\in\Lambda\}}
                <\infty
            \end{equation}
            Then $\{\norm{T_{\lambda}}:\lambda\in\Lambda\}$ is
            bounded.
        \end{ltheorem}
        \begin{ltheorem}{Banach-Stiennhaus Theorem}
            If $X$ and $Y$ are Banach spaces, and if
            $T_{n}\in\mathcal{L}(X,Y)$ converges point-wise to
            $T$, then $T\in\mathcal{L}(X,Y)$.
        \end{ltheorem}
\end{document}