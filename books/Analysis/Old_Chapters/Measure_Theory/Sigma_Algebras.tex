\section{\texorpdfstring{$\sigma$}{Sigma}-Algebras}
    \subsection{Set Rings}
        Given a set $\Omega$, $\mathcal{P}(\Omega)$ is the
        set of all subsets of $\Omega$. Often this is too
        much, and too difficult to handle. Indeed, even
        $\mathcal{P}(\mathbb{R})$ is quite large and hard
        to get a grasp on. We wish to speak of collections
        of sets that have some structure on them.
        The first thing we will talk about is a set ring.
        \begin{ldefinition}{Set Ring}{Set_Ring}
            A set ring of a set $\Omega$ is a nonempty subset
            $\mathcal{R}\subseteq\mathcal{P}(\Omega)$ such
            that for all
            $A,B\in\mathcal{R}$, $A\cup{B}\in\mathcal{R}$,
            and $A\setminus{B}\in\mathcal{R}$.
        \end{ldefinition}
        \begin{lexample}{}{Set Ring}
            If $\Omega$ is a set, then
            $\mathcal{P}(\Omega)$ is a set ring of
            $\Omega$. So is the set $R=\{\emptyset$.
            For any $A\subset\Omega$, the set
            $R=\{A\}$ is also a set ring. If
            $\Omega=\{1,2,3\}$, then
            $R=\{\emptyset,\{1\},\{2,3\},\{1,2,3\}\}$ is
            a set ring on $\Omega$.
        \end{lexample}
        \begin{lexample}{}{Bigger_Set_Ring}
            If $\Omega$ is an infinite set, and if
            $\mathcal{E}=\big\{\{x\}:x\in\Omega\big\}$, then the
            smallest set ring that contains $\mathcal{E}$ is the set of
            all finite subsets of $\Omega$. For the union of two finite
            sets is finite, as is the set difference of two finite sets,
            and thus this satisfies a set ring. Moreover, if $\mathcal{R}$
            is a set ring that contains $\mathcal{E}$ then it contains the
            union of any finite collection of elements in $\mathcal{E}$.
            But $\mathcal{E}$ is the set of all of the singletons of
            $\Omega$, and any finite subset of $\Omega$ can be written
            as the union of finitely many singletons. Thus, $\mathcal{R}$
            is the smallest set ring that contains $\mathcal{E}$.
        \end{lexample}
        \begin{theorem}
            If $\Omega$ is a set, if $R$ is a set ring
            on $\Omega$, and if $A$ is a finite subset of
            $R$, then $\cup_{\alpha\in{A}}\alpha$ is an
            element of $R$.
        \end{theorem}
        \begin{proof}
            Apply induction to the closure of unions.
        \end{proof}
        \begin{theorem}
            If $\Omega$ is a set, if $R$ is a set ring on
            $\Omega$, and if $A,B\in{R}$, then
            $A\cup{B}\in{R}$.
        \end{theorem}
        \begin{proof}
            For $A\cap{B}=A\setminus(A\setminus{B})$, and
            from the closure of set difference,
            $A\cap{B}\in{R}$.
        \end{proof}
        \begin{theorem}
            If $\Omega$ is a set, if $R$ is a set ring
            on $\Omega$, and if $A$ is a finite subset of
            $R$, then $\cap_{\alpha\in{A}}\alpha$ is an
            element of $R$.
        \end{theorem}
        \begin{proof}
            Apply induction to the closure of intersections.
        \end{proof}
        \begin{theorem}
            If $\Omega$ is a set, if $R$ is a set ring on
            $\Omega$, if $A,B\subset\Omega$, and if
            $A\setminus{B}$, $B\setminus{A}$, and
            $A\cap{B}$ are elements of $R$, then
            $A,B\in{R}$.
        \end{theorem}
        Thus, the set ring generated by the set $\{A,B\}$ and
        the set ring generated by
        $\{A\setminus{B},B\setminus{A},A\cap{B}\}$ are the
        same.
        \begin{theorem}
            If $\Omega$ is a set and $R$ is a set ring
            of $\Omega$, then $\emptyset\in{R}$.
        \end{theorem}
        \begin{proof}
            For as $R$ is non-empty, there is an element
            $A\in{R}$. If $A=\emptyset$, then we are done.
            If not, as $R$ is closed under set difference,
            $A\setminus{A}\in{R}$. But
            $A\setminus{A}=\emptyset$.
        \end{proof}
        From this, if we have a collection $R$ of subsets of
        $\Omega$ and we wish to check if $R$ is a set ring
        of $\Omega$, there are several redundant operations
        we don't need to check. Since, for any set $A$,
        we have:
        \begin{align}
            A\setminus\emptyset&=A\\
            A\setminus{A}&=\emptyset\\
            \emptyset\setminus{A}&=\emptyset\\
            A\cup{A}&=A\\
            A\cup\emptyset&=A\\
            \emptyset\cup\emptyset&=\emptyset
        \end{align}
        Using our previous example $\Omega=\{1,2,3\}$,
        we can check laboriously that
        $R=\{\emptyset,\{1\},\{2,3\},\{1,2,3\}\}$ is a
        set ring on $\Omega$. The set
        $\{\emptyset,\{1\},\{2\},\{1,2,3\}\}$ is not
        a set ring, for $\{1,2\}=\{1\}\cup\{2\}$ is not
        an element.
        \begin{theorem}
            If $\Omega$ is a set, and if $A$ and $B$ are
            disjoint subsets of $\Omega$, then
            $R=\{\emptyset,A,B,A\cup{B}\}$ is a set ring
            on $\Omega$.
        \end{theorem}
        \begin{theorem}
            If $\Omega$ is a set, if $A$ and $B$ are
            disjoint subsets of $\Omega$, and if
            $R$ is a set ring such that $A,B\in{R}$,
            then $\{emptyset,A,B,A\cup{B}\}\subset{R}$.
        \end{theorem}
        As such, the set ring $\{\emptyset,A,B,A\cup{B}\}$
        is called the set ring generated by $A$ and $B$. We
        can continue and consider the case of three mutually
        disjoint subsets.
        \begin{theorem}
            If $\Omega$ is a set, and $A_{1},A_{2},A_{3}$ are
            mutually disjoint subsets of $\Omega$, then:
            \begin{equation}
                R=\{\emptyset,A_{1},A_{2},A_{3},
                    A_{1}\cup{A}_{2},A_{1}\cup{A}_{3},
                    A_{2}\cup{A}_{3},
                    A_{1}\cup{A}_{2}\cup{A}_{3}\}
            \end{equation}
            is a set ring on $\Omega$.
        \end{theorem}
        Indeed, we may generalize further.
        \begin{theorem}
            If $\Omega$ is a set and if
            $A$ is a subset of $\mathcal{P}(\Omega)$ of
            $n$ elements such that, for all
            $a,b\in{A}$, $a\cap{B}=\emptyset$, then:
            \begin{equation}
                R=\{\cup_{i\in{I}}A_{i}:
                I\in\mathcal{P}(\mathbb{Z}_{n})\}
            \end{equation}
            Is a set ring on $\Omega$.
        \end{theorem}
        \begin{theorem}
            If $\Omega$ is a set, then the set of all
            finite subsets of $\Omega$ is a set ring on
            $\Omega$.
        \end{theorem}
        A left semi-interval of $\mathbb{R}$ is an interval
        of the form $[a,b)$ where $a\leq{b}$. If $a=b$, this
        is the empty set. The set of all left semi-intervals
        is not a set ring on $\mathbb{R}$ since the union
        of two semi-intervals need not be a semi-interval.
        We need to add more sets to allow this to be a
        set ring. The collection of all finite unions of
        semi-intervals of $\mathbb{R}$ is a set ring.
        First, note the following:
        \begin{equation}
            \Big(\bigcup_{n=1}^{N}[a_{n},b_{n})\Big)
            \setminus[c,d)=\bigcup_{n=1}^{N}
            \Big([a_{n},b_{n})\setminus[c,d)]
        \end{equation}
        This is again the finite union of intervals. By
        induction we see that this collection is a ring on
        $\mathbb{R}$. We have seen that a set ring is
        closed to unions and set differences, and this
        implies that rings are closed under intersections and
        closed under symmetric differences. As it turns out,
        this is an equivalent definition of a set ring.
        \begin{theorem}
            If $\Omega$ is a set and
            $R\subset\mathcal{P}(\Omega)$, then $R$ is
            a set ring of $\Omega$ if and only if $R$ is
            closed under symmetric differences and
            intersections.
        \end{theorem}
        If $R$ is a set ring on $\Omega$, and if
        $A\in{R}$, let $\chi_{A}:\Omega\rightarrow[0,1]$ be
        the indicator function defined as follows:
        \begin{equation}
            \chi_{A}(\omega)=
            \begin{cases}
                0,&\omega\notin{A}\\
                1,&\omega\in{A}
            \end{cases}
        \end{equation}
        Then we have:
        \begin{align}
            \chi_{A\cap{B}}(\omega)
            &=\chi_{A}(\omega)\chi_{B}(\omega)\\
            \chi_{A\ominus{B}}&=
            \big(\chi_{A}(\omega)+\chi_{B}(\omega)\big)
            \mod{2}
        \end{align}
        These two operations satisfy the axioms of a ring.
        That is, a ring in the algebraic sense of the word:
        A set with two operations that behave certain
        properties. It is because of this that set rings
        have earned their name.
    \subsection{Set Algebras}
        \begin{definition}
            A set algebra on a set $\Omega$ is a set ring
            on $\Omega$ such that $\Omega\in\mathcal{A}$.
        \end{definition}
        \begin{lexample}{}{Set_Algebra}
            Let $\Omega=\{1,2,3,4\}$ and
            $R=\{\emptyset,\{1\},\{2,3\}\}$. Then $R$
            is a set ring, but it is not a set algebra
            since $\Omega\notin{R}$.
        \end{lexample}
        \begin{lexample}{}{Big_Set_Algebra}
            If $\Omega$ is an infinite set, and if
            $\mathcal{E}=\big\{\{x\}:x\in\Omega\big\}$, then
            the smallest set algebra that contains $\mathcal{E}$
            is the set of all finite and co-finite subsets of
            $\Omega$. There are a few cases to check. The finite
            union of finite subsets is finite, the finite union of
            co-finite subsets is co-finite, and the finite union
            of finite and co-finite is again co-finite. For set
            difference, the difference of finite with finite is
            again finite, and the difference of co-finite with
            co-finite is either co-finite or finite. The
            difference of co-finite with finite is co-finite,
            and the difference of finite with co-finite is finite.
            Thus, this set is a set algebra on $\Omega$. Moreoever
            it is the smallest set algebra that will contain $\mathcal{E}$.
        \end{lexample}
        From the definition, we see that a set algebra
        is closed under complements. indeed, this creates
        and equivalent definition for set algebras.
        \begin{theorem}
            If $\Omega$ is a set and
            $\mathcal{A}\subseteq\mathcal{P}(\Omega)$,
            then $\mathcal{A}$ is a set algebra on $\Omega$
            if and only if $\Omega\in\mathcal{A}$, and
            $\mathcal{A}$ is closed under union and
            complement.
        \end{theorem}
        \begin{theorem}
            If $\Omega$ is a set and $R$ is a set ring
            on $\Omega$, and if $\mathcal{A}$ is a set
            algebra on $\Omega$ such that
            $R\subset\mathcal{A}$, then for all $A\in{R}$,
            $A\in\mathcal{A}$ and $A^{C}\in\mathcal{A}$.
        \end{theorem}
        This then defines the \textit{smallest} set algebra
        that contains a set ring.
        \begin{theorem}
            If $\Omega$ is a set and $R$ is a set ring on
            $\Omega$, then:
            \begin{equation}
                \mathcal{A}=\{A,A^{C}:A\in{R}\}
            \end{equation}
            Is a set algebra on $\Omega$.
        \end{theorem}
        \begin{theorem}
            If $\Omega$ is a set and $A$ and $B$ are
            disjoint subset of $A$, then:
            \begin{equation}
                \mathcal{A}=
                    \{\emptyset,A,B,A\cup{B},
                      \Omega,A^{C},B^{C},A^{C}\cap{B}^{C}\}
            \end{equation}
            is a set algebra on $\Omega$.
        \end{theorem}
        For non-disjoint $A$ and $B$, the smallest
        set algebra becomes more complicated. We saw that
        the collection of all finite subsets of a set is
        a set ring on the set. The collection of all finite
        subsets, and their complements, is a set algebra.
    \subsection{\texorpdfstring{$\sigma$}{Sigma}-Rings}
        If $\Omega$ is a set, then $R\subset\mathcal{P}(\Omega)$
        is called a set ring on $\Omega$ if, for all
        $A,B\in{R}$, $A\cup{B}\in{R}$ and
        $A\setminus{B}\in{R}$. From this, given a ring $R$ on
        $\Omega$, the empty set is included, that is
        $\emptyset\in{R}$, and if $A,B\in{R}$, then
        $A\cap{B}\in{R}$. By induction, for any finite collection
        of elements in $R$, the union of these subsets is also
        contained in $R$, as well as the intersection. A set
        algebra on $\Omega$ is a ring $\mathcal{A}$
        on $\Omega$ such that $\Omega\in\mathcal{A}$. That is,
        $\mathcal{A}\subset\mathcal{P}(\Omega)$, and
        $\mathcal{A}$ is closed under union, set difference, and
        $\Omega\in\mathcal{A}$. There is an equivalent definition:
        $\Omega\in\mathcal{A}$, for all $A\in\mathcal{A}$,
        $A^{C}\in\mathcal{A}$, and for all $A,B\in\mathcal{A}$,
        $A\cup{B}\in\mathcal{A}$. The complement of $A$,
        $A^{C}$, is defined as $\Omega\setminus{A}$. The
        equivalence of the two definitions comes from DeMorgan's
        laws, since
        $A\setminus{B}=A\cap{B}^{C}=(A^{C}\cup{B})^{C}$. We now
        talk about $\sigma$-Ring.
        \begin{definition}
            A $\sigma$-Ring on a set $\Omega$ is a set
            $\sigma\subset\mathcal{P}(\Omega)$ such that,
            for all countable subsets of $\sigma$, the union
            $\bigcup_{i=1}^{\infty}A_{i}\in\sigma$, and for all
            $A,B\in\sigma$, $A\setminus{B}\in\sigma$.
        \end{definition}
        The requirement that the collections be countable is
        important to note. A \textit{topology} is a subset
        of $\mathcal{P}(\Omega)$ with the property that it is
        closed under arbitrary unions. $\sigma$-Rings need only
        be closed under countable unions.
        \begin{example}
            Every $\sigma$-Ring is a set ring, but not every
            set ring is a $\sigma$-ring. Let $\Omega$ be
            uncountable, and let $R$ be the set of all finite
            subsets of $\Omega$. Then $R$ is a ring, but it is
            not a $\sigma$-ring. For, as $\Omega$ is uncountably
            infinite, it has a countable subset $A$, and we
            may subscript the elements as $a_{n}$. But
            $\bigcup_{n=1}^{\infty}\{a_{n}\}$ is not a finite
            subset of $\Omega$, and is therefore not contained
            in $R$. Thus, $R$ is not closed under countable unions
            and $R$ is not a $\sigma$-ring. However, if we let
            $\sigma$ be the set of all \textit{countable} subsets
            of $\Omega$, the $\sigma$ is indeed a $\sigma$-ring.
        \end{example}
        \begin{lexample}{}{Ring_From_Semi-Intervals}
            The collection of all semi-intervals and finite
            unions of semi-intervals defines a ring on
            $\mathbb{R}$. It is tempting to think tha the
            collection of all countable unions of semi-intervals
            is a $\sigma$-ring on $\mathbb{R}$, but this is not
            the case. The Cantor set is an example of a subset
            that can be constructed by a countable number of
            steps of removing intervals from a given interval,
            but the resulting set is not the countable union of
            semi-intervals. To construct the Cantor set, consider
            the interval $[0,1]$. From this, remove
            $(\frac{1}{3},\frac{2}{3})$. Continuing removing the
            middle third from each sub-interval obtained. The
            resulting set contains no interval as a subset, and
            thus cannot be the union of countably many intervals,
            or semi-intervals.
        \end{lexample}
    \subsection{Dynkin System}
        A Dynkin system on a set $\Omega$ is a subset
        $\mathcal{D}\subset\mathcal{P}(\Omega)$ such that
        $\Omega\in\mathcal{D}$, if $A,B\in\Omega$ and if
        $A\subseteq{B}$, then $A\setminus{B}\in\mathcal{D}$,
        and for all countable collections of elements of
        $\mathcal{D}$ such that
        $A_{1}\subset{A}_{2}\subset\hdots$,
        $\cup_{n=1}^{\infty}A_{n}\in\mathcal{D}$. There is
        an equivalent defintion for Dynkin Systems.
        $\Omega\in\mathcal{D}$, $A\in\mathcal{D}$ implies
        $A^{C}\in\mathcal{D}$, and for all countable disjoint
        collections of elements in $\mathcal{D}$, the union
        is also contained in $\mathcal{D}$. These requirements
        are weaker than those of a $\sigma$-algebra. Any
        $\sigma$-algebra is a Dynkin system, but not every
        Dynkin system is a $\sigma$-algebra.
        \begin{ldefinition}{Dynkin System}
            A Dynkin System on a set $\Omega$ is a subset
            $\mathcal{D}\subseteq\mathcal{P}(\Omega)$ such that:
            \begin{enumerate}
                \item $\Omega\in\mathcal{D}$.
                \item For all $A,B\in\mathcal{D}$ such that $A\subseteq{B}$,
                      $B\setminus{A}\in\mathcal{D}$.
                \item For any sequence $A_{n}\in\mathcal{D}$ such that
                      $A_{n}\subseteq{A}_{n+1}$,
                      $\cup_{n=1}^{\infty}A_{n}\in\mathcal{D}$
            \end{enumerate}
        \end{ldefinition}
        \begin{theorem}
            If $\Omega$ is a set and $\mathcal{D}\subseteq\mathcal{P}(\Omega)$
            is such that $\Omega\in\mathcal{D}$, for all $A\in\mathcal{D}$,
            $A^{C}\in\mathcal{D}$, and if for all sequences $A_{n}\in\mathcal{D}$
            such that $A_{n}\cap{A}_{m}=\emptyset$ for all $n\ne{m}$,
            $\cup_{n=1}^{\infty}A_{n}\in\mathcal{D}$, then
            $\mathcal{D}$ is a Dynkin System on $\Omega$.
        \end{theorem}
        \begin{theorem}
            If $\mathcal{D}$ is a Dynkin system on a set
            $\Omega$, and if $\mathcal{D}$ is closed with
            respect to intersections, then $\mathcal{D}$
            is a $\sigma$-algebra on $\Omega$.
        \end{theorem}
        \begin{theorem}
            If $\Omega$ is a set, if
            $\mathcal{E}\subset\mathcal{P}(\Omega)$ is closed
            to intersections, and if $\mathcal{D}$ is the
            Dynkin System generated by $\mathcal{E}$, then
            $\mathcal{D}$ is a $\sigma$-algebra.
        \end{theorem}
        \begin{theorem}[Dynkin's Theorem]
            If $\Omega$ is a set, $\mathcal{C}\subseteq\mathcal{P}(\Omega)$
            is intersection-stable, and if $\mathcal{D}$ is the smallest
            Dynkin system that contains $\mathcal{C}$, then $\mathcal{D}$
            is also intersection-stable.
        \end{theorem}
        \begin{proof}
            For let:
            \begin{equation}
                \mathcal{D}_{1}=
                \{D\in\mathcal{D}:\forall_{C\in\mathcal{C}},D\cap{D}\in\mathcal{D}\}
            \end{equation}
            Then $\mathcal{D}_{1}$ is a Dynkin system, and thus
            $\mathcal{D}_{1}=\mathcal{D}$. Now define:
            \begin{equation}
                \mathcal{D}_{2}=\{
                    D\in\mathcal{D}:\forall_{A\in\mathcal{D}},D\cap{A}\in\mathcal{D}
                \}
            \end{equation}
            Then $\mathcal{C}\subseteq\mathcal{D}_{2}$ and $\mathcal{D}_{2}$ is a
            Dynkin System, and thus $\mathcal{D}_{2}=\mathcal{D}$.
        \end{proof}
        \begin{theorem}
            If $\Omega$ is a set, $\mathcal{C}\subseteq\mathcal{P}(\Omega)$
            is intersection-stable, and if $\mathcal{D}$ is the smallest
            Dynkin system that contains $\mathcal{C}$, then $\mathcal{D}$
            is a $\sigma\textrm{-Algebra}$ on $\Omega$.
        \end{theorem}
        Since semi-intervals are closed to intersections,
        the Borel $\sigma$-algebra is equivalently the
        Dynkin system generated by semi-intervals.
    \subsection{\texorpdfstring{$\sigma$}{Sigma}-Algebras}
        In an analogous manner to how set rings and set algebras
        were defined, there is something called a $\sigma$-algebra.
        This notion will be one of the central themes of measure
        theory.
        \begin{definition}
            A $\sigma$-algebra on a set $\Omega$ is a
            $\sigma$-ring on $\Omega$ such that
            $\Omega\in\mathcal{A}$
        \end{definition}
        That is, given any countable collection of elements in
        $\mathcal{A}$, the union is also contained in
        $\mathcal{A}$. In addition, $\mathcal{A}$ is closed under
        set differences and $\Omega\in\mathcal{A}$.
        \begin{example}
            The first trivial example is the power set
            $\mathcal{P}(\Omega)$. Also the set
            $\{\emptyset,\Omega\}$ defines a $\sigma$-algebra on
            $\Omega$. The set of all countable subsets defines
            a $\sigma$-ring, and the set of all countable and
            co-countable (Sets whose complement is countable)
            will define a $\sigma$-algebra.
        \end{example}
        We can equivalently define a $\sigma$-algebra to be a
        collection of sets $\mathcal{A}$ such that
        $\Omega\in\mathcal{A}$, for all $A\in\mathcal{A}$,
        $A^{C}\in\mathcal{A}$, and $\mathcal{A}$ is closed under
        countable unions. Being closed under countable unions
        implies that it is closed under finite unions as well.
        For let $A_{1}=A$, and for all $n>1$, let $A_{n}=B$.
        Then $\bigcup_{n=1}^{\infty}A_{n}=A\cup{B}$. By induction,
        a $\sigma$-algebra is closed under any finite union.
    \subsection{Borel \texorpdfstring{$\sigma$}{Sigma}-Algebra}
        One of the most important types of $\sigma$-algebras
        is the Borel $\sigma$-algebra. We first define the
        Borel $\sigma$-algebra on $\mathbb{R}$.
        \begin{definition}
            The Borel $\sigma$-algebra on $\mathbb{R}$, denoted
            $\mathcal{B}$, is the $\sigma$-algebra generated
            by the set $\{[a,b):a,b\in\mathbb{R}\}$.
        \end{definition}
        That is, the Borel $\sigma$-algebra on $\mathbb{R}$ is
        the \textit{smallest} $\sigma$-algebra that contains
        all of the semi-intervals. We can equivalently say that
        $\mathcal{B}$ is the $\sigma$-algebra generated by all
        \textit{open} intervals. If we know that every open
        subset of $\mathbb{R}$ is the countable union of open
        subsets, than we can equivalently say that
        $\mathcal{B}$ is the $\sigma$-algebra generated by all
        \textit{open subsets} of $\mathbb{R}$. Writing $[a,b)$
        as the countable intersection of open intervals, or
        $(a,b)$ as the countable union of semi-intervals comes
        from the Archimedean property of the real numbers.
        Thus, the smallest $\sigma$-algebra that contains all
        semi-intervals is the smallest $\sigma$-algebra that
        contains all open intervals, which
        is the smallest $\sigma$-algebra that contains all open
        subsets of $\mathbb{R}$. Similarly, this will contain all
        of the \textit{closed} intervals, intervals of the form
        $[a,b]$. We say that a set $\mathcal{U}\subset\mathbb{R}$
        is open if, for all $x\in\mathcal{U}$, there is an $r>0$
        such that $(x-r,x+r)\subset\mathcal{U}$. That is, every
        point in $\mathcal{U}$ can be surrounded by an interval
        that is entirely contained in $\mathcal{U}$. Thus, any
        open set can be written as:
        \begin{equation}
            \mathcal{U}=
                \bigcup_{x\in\mathcal{U}}(\alpha_{x},\beta_{x})
        \end{equation}
        This union is not countable, for any open set must
        contain an interval, an intervals are uncountable large.
        This is simply because $(a,b)$ is equivalent to $(0,1)$.
        By adjusting the size of $\alpha_{x}$ and $\beta_{x}$ to
        be rational numbers, we can written $\mathcal{U}$ as the
        union of intervals with rational endpoints. But there are
        only countably many such intervals, and thus
        $\mathcal{U}$ is the union of countably many open
        intervals. Thus, any open set is the union of countably
        many open intervals. From this, the smallest
        $\sigma$-algebra that contains open intervals will contain
        all open sets, since $\sigma$-algebras are closed under
        countable unions. Borel sets are elements of the
        Borel $\sigma$-algebra $\mathcal{B}$. Since all open
        sets are Borel sets, and as $\sigma$-algebras are closed
        under complenents, all closed sets are also Borel sets.
        This is because the complement of an open set is a closed
        set, and vice versa. Thus, equivalently, $\mathcal{B}$ is
        the smallest $\sigma$-algebra containing all closed sets.
        A $G_{\delta}$ sets is a subset that is the countable
        intersection of open sets. As open sets are not
        necessarily closed under countable intersections, not
        all $G_{\delta}$ sets are open. There is another notion,