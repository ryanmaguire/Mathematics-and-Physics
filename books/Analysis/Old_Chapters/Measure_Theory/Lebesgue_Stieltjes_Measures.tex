\section{Lebesgue-Stieltjes Measures}
    A Lebesgue-Stieltjes measure is any measure on the
    Borel $\sigma\text{-Algebra}$ $\mathcal{B}$ such that,
    for any finite semi-interval $[a,b)$,
    $\mu\big(\mu[a,b)\big)<\infty$. $\mu(\mathbb{R})$ may
    be infinite. Recall that the Borel $\sigma\text{-Algebra}$
    is the smallest $\sigma\text{-Algebra}$ on $\mathbb{R}$
    that contains all semi-intervals $[a,b)$. A pure point
    measure on $\mathbb{R}$, indexing over the rational
    numbers, would be such a measure. If we have a
    Lebesgue-Stieltjes measure, we wish to find a function
    $F_{\mu}:\mathbb{R}\rightarrow\mathbb{R}$ such that, for
    all semi-intervals $[a,b)$,
    $\mu([a,b))=F_{\mu}(b)-F_{\mu}(a)$. In probability, this
    is called the cummulative probability function. For now
    we wish to show that there is indeed such a function that
    does this. Consider the case when $\mu(\mathbb{R})<\infty$.
    Let $F_{\mu}(x)=\mu(-\infty,x)$. Then:
    \begin{align}
        \mu([a,b))
        &=\mu((-\infty,b)\setminus(-\infty,a))\\
        &=\mu((-\infty,b))-\mu((-\infty,a))\\
        &=F_{\mu}(b)-F_{\mu}(a)
    \end{align}
    In the more general case when that measure of the entire
    real line is infinite we still want to find a function
    such that:
    \begin{align}
        \mu\big([0,x)\big)&=F_{\mu}(x)-F_{\mu}(0)&x>0\\
        \mu\big([x,0)\big)&=F_{\mu}(0)-F_{\mu}(x)&x<0
    \end{align}
    We can define the following:
    \begin{equation}
        F_{\mu}(x)=
        \begin{cases}
            \mu\big([0,x)\big)+C,&x>0\\
            -\mu\big([x,0)\big)+C,&x<0\\
            C,&x=0
        \end{cases}
    \end{equation}
    Then $F_{\mu}$ is a function that satisfies our criterion.
    Indeed, $F_{\mu}$ is defined uniquely up to an additive
    constant. Any such function is non-decreasing since, for
    any $x<y$, $F_{\mu}(y)-F_{\mu}(x)=\mu([x,y))\geq{0}$.
    In addition, $F_{\mu}$ is left-continuous. That is,
    for all $a\in\mathbb{R}$:
    \begin{equation}
        \underset{x\rightarrow{a}^{-}}{\lim}F_{\mu}(x)
        =F_{\mu}(a)
    \end{equation}
    \begin{theorem}
        If $\mu$ is a Lebesgue-Stieltjes measure on the
        Borel $\sigma\text{-Algebra}$ of $\mathbb{R}$, and if
        $F_{\mu}$ is the function thing, then $F_{\mu}$ is
        left-continuous.
    \end{theorem}
    \begin{proof}
        For let $a\in\mathbb{R}$ and let
        $x:\mathbb{N}\rightarrow\mathbb{R}$ be a monotonic
        increasing sequence such that $x_{n}\rightarrow{a}$.
        But then, for all $n\in\mathbb{N}$,
        $[x_{n+1},a)\subset[x_{n},a)$. But then:
        \begin{equation}
            \mu\Big(\bigcap_{n=1}^{\infty}[x_{n},a)\Big)
            =\underset{N\rightarrow\infty}{\lim}
            \mu\big([x_{N},a)\big)
        \end{equation}
        But $\cap_{n=1}^{\infty}[x_{n},a)=\emptyset$ as
        $x_{n}\rightarrow{a}$. Therefore:
        \begin{equation}
            \underset{N\rightarrow\infty}{\lim}
            \mu\big([x_{n},a)\big)=0
        \end{equation}
        But from the definition of $F_{\mu}$,
        \begin{equation}
            \mu\big([x_{n},a)\big)=F_{\mu}(a)-F_{\mu}(x_{n})
        \end{equation}
        Thus, $F_{\mu}(x_{n})\rightarrow{F}_{\mu}(a)$.
    \end{proof}
    Such a function may not be right-continuous. The requirement
    that the sequence $x$ be increasing was necessary for the
    proof. Howver, the right-hand limit does exists.
    \begin{theorem}
        If blah blah, right hand limit exists.
    \end{theorem}
    \begin{proof}
        For:
        \begin{equation}
            \{a\}=\bigcap_{n=1}^{\infty}[a,a+\frac{1}{n})
        \end{equation}
        And thus, as $\mu$ is a Lebesgue-Stieltjes measure,
        and thus $\mu([a,b))<\infty$ for all finite semi-intervals,
        we may apply continuity from above and obtain:
        \begin{align}
            \mu(\{a\})&=
            \underset{N\rightarrow\infty}{\lim}
            \mu\big([a,a+\frac{1}{n}\big)\\
            &=\underset{x\rightarrow{a}^{+}}{\lim}
            F_{\mu}(x)-F_{\mu}(a)
        \end{align}
    \end{proof}
    If $\mu$ has no points of positive measure, then
    $F_{\mu}$ will be continuous.
    \begin{ftheorem}{Carath\'{e}odory Extension Theorem}{}
        If $F:\mathbb{R}\rightarrow\mathbb{R}$ is a non-decreasing
        left-continuous function then there exists a unique
        Lebesgue-Stieltjes measure $\mu$ such that, for all
        $a,b\in\mathbb{R}$, $a<b$:
        \begin{equation}
            \mu\big([a,b)\big)=F(b)-F(a)
        \end{equation}
    \end{ftheorem}
    In particular, using $F(x)=x$, we see that there is a unique
    measure on the Borel $\sigma\text{-Algebra}$ such that
    $\mu([a,b))=b-a$. This measure is called the Lebesgue measure
    on $\mathbb{R}$. We define $\mu^{*}$ on a set
    $A\subseteq\mathbb{R}$ to be:
    \begin{equation}
        \mu^{*}(A)=
        \inf\Bigg\{\sum_{i=1}^{\infty}(b_{i}-a_{i}):
        A\subseteq\bigcup_{i=1}^{\infty}[a_{i},b_{i})\Bigg\}
    \end{equation}
    If $A$ is countable, then $\mu^{*}(A)$ is zero. For let
    $a:\mathbb{N}\rightarrow{A}$ be bijection, and let
    $\varepsilon>0$. Then:
    \begin{align}
        \mu^{*}(A)&\leq
        \sum_{n=1}^{\infty}
        \Big((a_{n}+\frac{\varepsilon}{2^{n+1}})-
        (a_{n}-\frac{\varepsilon}{2^{n+1}})\Big)\\
        &=\sum_{n=1}^{\infty}\frac{\varepsilon}{2^{n}}\\
        &=\varepsilon
    \end{align}
    Taking the infininum, we see that $\mu^{*}(A)=0$. This
    function is defined on all of $\mathcal{P}(\mathbb{R})$,
    however it is not a measure. The restriction of
    $\mu^{*}$ to the Borel $\sigma\text{-Algebra}$ is
    a measure.