\chapter{Numerical Analysis}
    \section{Power Series}
    \section{Asymptotic Expansions}
    \section{Stationary Phase Approximation}
        Suppose $g$ is an analytical function about
        the origin (i.e. it has a convergent MacLaurin series),
        and consider the integral:
        \begin{equation}
            I(k) = \int_{a}^{b}e^{ikg(x)}dx
        \end{equation}
        Suppose that there is a $c\in(a,b)$ such
        that $g'(c)=0$ and $g''(c)\ne 0$. Then:
        \begin{align}
            \nonumber I(k)&=e^{ikg(c)}\int_{a}^{b}e^{ik[g(x)-g(c)]}dx\\
            &=e^{ikg(c)}\int_{a}^{b}e^{ik[\frac{g''(c)}{2}(x-c)^{2}+\hdots]}dx
        \end{align}
        Higher terms are extremely oscillatory, and so we neglect them.
        Note that higher terms can indeed cancel each
        other out, meaning these neglected terms may
        not be negligible. For example, if
        $g(x)=-sin(\pi x)$, then $\exp(ig(x))$ is never
        too oscillatory. However, so long as the interval
        $[a,b]$ is small enough, the approximation is still
        valid. The previously mentioned $g(x)$ is how
        one approximates the $J_{0}(x)$ Bessel function.
        Out integral then becomes:
        \begin{align}
            I(k)&\approx e^{ikg(c)}
                    \int_{a}^{b}e^{ik\frac{g''(c)}{2}(x-c)^{2}}dx\\
                &\approx e^{ikg(c)}
                    \int_{\infty}^{\infty}
                    \exp(ik\frac{g''(c)}{2}(x-c)^{2})\diff{x}\\
                &=e^{ikg(c)}\sqrt{\frac{2\pi i}{kg''(c)}}
        \end{align}
        We can use this for our double integral,
        and make it a single integral. The first and
        second integrals of $\psi$ are nasty, however.
        \begin{equation}
            \begin{split}
                \frac{\partial\psi}{\partial\phi}
                &=kD\Big[\frac{2D\rho\cos(B)
                    \sin(\phi)+2\rho\rho_{0}
                \sin(\phi-\phi_{0})}{2D^2\sqrt{1+2\cos(B)
                \frac{\rho_{0}\cos(\phi_{0})-\rho\cos(\phi)}{D}+
                \frac{\rho^{2}+\rho_{0}^{2}-
                2\rho\rho_{0}\cos(\phi-\phi_{0})}{D^2}}}\\
                &\quad\quad\quad\quad\quad
                -\frac{\rho\cos(B)\sin(\phi)}{D}\Big]
            \end{split}
        \end{equation}
        The second derivative is equally bad.
        Solving for $\frac{\partial\psi}{\partial\phi}=0$
        must be done iteratively by successive approximations.
        A further approximation can be made as $\psi$
        is analytic in $\phi$. Let $\phi_{s}$ be
        the solution to $\frac{\partial\psi}{\partial\phi}$
        and let $\phi_{s_{n}}$ be a sequence such that
        $\phi_{s_{n}}\rightarrow\phi_{s}$.
    \subsection{Root Finding}
        \begin{figure}[H]
            \centering
            \captionsetup{type=figure}
            \resizebox{\textwidth}{!}{%
                \includegraphics{../images/NewtonRoots.png}}
            \caption{Newton Fractal for $z^{3}-1=0$}
            \label{fig:Diff_Theory_Newton_Fractal}
        \end{figure}
        The Python code used to generated this is given below.
        It can be altered to try various functions, and this
        is encouraged.
        \newpage
        \begin{lstlisting}[%
            language=python,
            basicstyle=\footnotesize\ttfamily,
            frame=single,
            gobble=16
        ]
            from PIL import Image
            import numpy as np

            # Set range for x and y axes.
            x_min, x_max = -1.0, 1.0
            y_min, y_max = -1.0, 1.0

            # Colors for the roots (Red, Green, Blue).
            colors = [(255, 0, 30), (0, 255, 30), (0, 30, 255)]

            size = 1024
            img = Image.new("RGB", (size, size), (255, 255, 255))

            # List the roots of z^3-1
            roots = [1.0+0.0j, -0.5+0.8660254037844386j, -0.5-0.8660254037844386j]
            roots = np.array(roots)
            for y in range(size):
                z_y = y * (y_max - y_min)/(size - 1) + y_min
                for x in range(size):
                    z_x = x * (x_max - x_min)/(size - 1) + x_min
                    z = complex(z_x, z_y)

                    # Allow 40 iterations for Newton-Raphson.
                    for iters in range(40):
                        # Perfrom Newton-Raphson on z^3 - 1 (Simplifying as well).
                        root = (2.0*z*z*z + 1)/(3.0*z*z)

                        # Checks for convergence
                        if abs(root - z) < 10e-10:
                            break
                        z = root

                    ind = np.min((
                        np.abs(root-roots) == np.min(np.abs(root-roots))
                    ).nonzero())
                    col = colors[ind]
                    col = [k for k in col]

                    # Create a gradient in color to emphasize rate of convergence.
                    col[ind] -= 10*iters
                    col = tuple([k for k in col])
                    img.putpixel((x, y), col)
            img.save("NewtonRoots.png", "PNG")
        \end{lstlisting}
