\section{\texorpdfstring{$\sigma$}{Sigma}-Rings}
    If $\Omega$ is a set, then $R\subset\mathcal{P}(\Omega)$
    is called a set ring on $\Omega$ if, for all
    $A,B\in{R}$, $A\cup{B}\in{R}$ and
    $A\setminus{B}\in{R}$. From this, given a ring $R$ on
    $\Omega$, the empty set is included, that is
    $\emptyset\in{R}$, and if $A,B\in{R}$, then
    $A\cap{B}\in{R}$. By induction, for any finite collection
    of elements in $R$, the union of these subsets is also
    contained in $R$, as well as the intersection. A set
    algebra on $\Omega$ is a ring $\mathcal{A}$
    on $\Omega$ such that $\Omega\in\mathcal{A}$. That is,
    $\mathcal{A}\subset\mathcal{P}(\Omega)$, and
    $\mathcal{A}$ is closed under union, set difference, and
    $\Omega\in\mathcal{A}$. There is an equivalent definition:
    $\Omega\in\mathcal{A}$, for all $A\in\mathcal{A}$,
    $A^{C}\in\mathcal{A}$, and for all $A,B\in\mathcal{A}$,
    $A\cup{B}\in\mathcal{A}$. The complement of $A$,
    $A^{C}$, is defined as $\Omega\setminus{A}$. The
    equivalence of the two definitions comes from DeMorgan's
    laws, since
    $A\setminus{B}=A\cap{B}^{C}=(A^{C}\cup{B})^{C}$. We now
    talk about $\sigma$-Ring.
    \begin{definition}
        A $\sigma$-Ring on a set $\Omega$ is a set
        $\sigma\subset\mathcal{P}(\Omega)$ such that,
        for all countable subsets of $\sigma$, the union
        $\bigcup_{i=1}^{\infty}A_{i}\in\sigma$, and for all
        $A,B\in\sigma$, $A\setminus{B}\in\sigma$.
    \end{definition}
    The requirement that the collections be countable is
    important to note. A \textit{topology} is a subset
    of $\mathcal{P}(\Omega)$ with the property that it is
    closed under arbitrary unions. $\sigma$-Rings need only
    be closed under countable unions.
    \begin{example}
        Every $\sigma$-Ring is a set ring, but not every
        set ring is a $\sigma$-ring. Let $\Omega$ be
        uncountable, and let $R$ be the set of all finite
        subsets of $\Omega$. Then $R$ is a ring, but it is
        not a $\sigma$-ring. For, as $\Omega$ is uncountably
        infinite, it has a countable subset $A$, and we
        may subscript the elements as $a_{n}$. But
        $\bigcup_{n=1}^{\infty}\{a_{n}\}$ is not a finite
        subset of $\Omega$, and is therefore not contained
        in $R$. Thus, $R$ is not closed under countable unions
        and $R$ is not a $\sigma$-ring. However, if we let
        $\sigma$ be the set of all \textit{countable} subsets
        of $\Omega$, the $\sigma$ is indeed a $\sigma$-ring.
    \end{example}
    \begin{lexample}
        The collection of all semi-intervals and finite
        unions of semi-intervals defines a ring on
        $\mathbb{R}$. It is tempting to think tha the
        collection of all countable unions of semi-intervals
        is a $\sigma$-ring on $\mathbb{R}$, but this is not
        the case. The Cantor set is an example of a subset
        that can be constructed by a countable number of
        steps of removing intervals from a given interval,
        but the resulting set is not the countable union of
        semi-intervals. To construct the Cantor set, consider
        the interval $[0,1]$. From this, remove
        $(\frac{1}{3},\frac{2}{3})$. Continuing removing the
        middle third from each sub-interval obtained. The
        resulting set contains no interval as a subset, and
        thus cannot be the union of countably many intervals,
        or semi-intervals.
    \end{lexample}