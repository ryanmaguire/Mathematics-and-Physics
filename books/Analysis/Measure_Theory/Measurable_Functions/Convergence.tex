\section{Convergence of Measurable Functions}
    \begin{ldefinition}{Measure Space}
        A Measure Space on a set $\Omega$, denoted $(\Omega,\mathcal{A},\mu)$,
        is a $\sigma\textrm{-Algebra}$ on $\Omega$ and a measure
        $\mu:\Omega\rightarrow\mathbb{R}$.
    \end{ldefinition}
    A function $f:\Omega\rightarrow\mathbb{R}$ is $\mathcal{A}-\mathcal{B}$
    measurable, or simply measurable, if for all $B\in\mathcal{B}$, the
    pre-image is in $\mathcal{A}$. That is, $f^{-1}(B)\in\mathcal{A}$.
    \begin{theorem}
        If $f,g:\Omega\rightarrow\mathbb{R}$ are measurable functions, and if
        $A$ is defined by:
        \begin{equation}
            A=\{\omega\in\Omega:f(\omega)\ne{g}(\omega)\}
        \end{equation}
        then $A$ is $\mathcal{A}$ measurable.
    \end{theorem}
    \begin{definition}
        Two functions are equal $\mu$ almost everywhere if:
        \begin{equation}
            \mu(\omega\in\Omega:f(\omega)\ne{g}(\omega))=0
        \end{equation}
    \end{definition}
    \begin{definition}
        A sequence of measurable functions
        $f_{n}:\mathbb{R}\rightarrow\mathbb{R}$ converges
        to a function $f:\Omega\rightarrow\mathbb{R}$
        if there is a set $E$ such that $\mu(E^{C})=0$,
        and $f_{n}(\omega)\rightarrow{f}(\omega)$ for all
        $\omega\in{E}$.
    \end{definition}
    \begin{theorem}
        If $f_{n}\rightarrow{f}$ almost every and
        $f_{n}\rightarrow{g}$ almost everywhere, then
        $f=g$ almost everywher.
    \end{theorem}
    \begin{theorem}
        If $f_{n}\rightarrow{f}$ almost everywhere, and
        if $f=g$ almost everywhere, then
        $f_{n}\rightarrow{g}$ almost everywhere.
    \end{theorem}
    \begin{definition}
        A function $f:\Omega\rightarrow\mathbb{R}$
        converges to $f:\Omega\rightarrow\mathbb{R}$
        uniformly if for all $\varepsilon>0$ there is
        an $N\in\mathbb{N}$ such that, for all
        $\omega\in\Omega$,
        $|f(\omega)-f_{n}(\omega)|<\varepsilon$.
    \end{definition}
    \begin{example}
        Consider $f_{n}(\omega)=\omega^{n}$ for
        $\omega\in[0,a]$, where $a<1$. Then this
        converges uniformly to zero since, all all
        $\omega\in[0,a]$:
        \begin{equation}
            |\omega^{n}-0|=\omega^{n}\leq{a}^{n}
        \end{equation}
        But since $0\leq{a}<1$, $a^{n}$ converges to
        zero. Thus $f_{n}\rightarrow{0}$ uniformly.
    \end{example}
    We can define non-uniform converges by considering
    the logical negation of the definition for
    uniform convergence, but we can simplify this as
    well.
    \begin{theorem}
        A sequence of functions $f_{n}$ converges
        non-uniformly to a function $f$ if
        $f_{n}\rightarrow{f}$ point-wise, and there
        exists a $\delta>0$, a strictly increasing
        sequence $n_{k}$, and a sequence $\omega_{k}$
        such that
        $|f_{n_{k}}(\omega_{k})-f(\omega_{k})|>\delta$
    \end{theorem}
    \begin{example}
        If we define $f_{n}(\omega)=\omega^{n}$ on the
        interval $[0,1]$, then the convergence is no
        longer uniform. Indeed, the limit is
        discontinuous.
    \end{example}
    \begin{definition}
        A sequence $f_{n}$ converges to $f$ almost
        uniformly if, for all $\varepsilon>0$ there is
        a set $E$ such that $\mu(E^{C})<\varepsilon$,
        and $f_{n}$ converges to $f$ uniformly on
        $E$.
    \end{definition}
    \begin{example}
        If we again let $f_{n}(\omega)=\omega^{n}$ on
        $[0,1]$, then $f_{n}\rightarrow{0}$
        almost uniformly. For let $\varepsilon>0$. Then
        $f_{n}\rightarrow{0}$ uniformly on the set
        $[0,1-\varepsilon]$, and the measure of the
        compliment of this is less than$ \varepsilon$.
    \end{example}
    There is a difference between convergence almost
    everywhere and convergence almost uniformly. For
    convergence almost everywhere, we may remove a set
    of measure zero and expect that there is point-wise
    convergence on the remaining set. For almost uniform
    convergence we may remove a set of arbitrarily small
    measure, but not necessarily measure zero, and expect
    uniform convergence on the remaining set.
    \begin{theorem}
        If $f_{n}\rightarrow{f}$ almost uniformly,
        then $f_{n}\rightarrow{f}$ almost everywhere.
    \end{theorem}
    \begin{proof}
        For all $n\in\mathbb{N}$ there is a set
        $E_{n}$ such that $\mu(E_{n}^{C})<1/n$, and
        $f_{n}\rightarrow{f}$ uniformly on $E_{n}$.
        But then $f_{n}\rightarrow{f}$ on
        $\cup_{n=1}^{\infty}E_{n}$. But the complement
        of this set has measure zero. Therefore, etc.
    \end{proof}
    The converse is not true, in general. For let
    $f_{n}(\omega)$ be defined as follows:
    \begin{equation}
        f_{n}(\omega)=
        \begin{cases}
            0,&\omega\leq{n}\\
            1,&\omega>n
        \end{cases}
    \end{equation}
    The $f_{n}\rightarrow{0}$ almost everywhere, and
    indeed $f_{n}\rightarrow{0}$ point-wise. But
    the convergence is not uniform, nor is it
    almost uniform. There is no way to remove a set of
    finite measure and have uniform convergence on the
    resulting set. Similar to where continuity from above
    failed, the fact that $\mu(\mathbb{R})$ is infinite
    is why this failed. If we can limit the measure on
    the set, then convergence almost everywhere implies
    almost uniform convergence.
    \begin{ftheorem}{Egorov's Theorem}
                    {Measure_Theory_Egorov_Theorem}
        If $(\Omega,\mathcal{A},\mu)$ is a measure
        space and if $\mu(\Omega)<\infty$, then
        convergence $\mu$-almost everywhere implies
        $\mu$-almost uniform convergence.
    \end{ftheorem}
    \begin{bproof}
        For if $f_{n}(\omega)\rightarrow{f(\omega)}$
        $\mu$-almost everywhere, then there is a set
        $E$ such that $\mu(E^{C})=0$ and
        $f_{n}(\omega)\rightarrow{f(\omega)}$ for all
        $\omega\in{E}$. This means that for all
        $\delta>0$ and for all $\omega\in{E}$ there is
        an $N\in\mathbb{N}$ such that for all $n>N$,
        $|f_{n}(\omega)-f(\omega)|<\delta$. Let $A_{nm}$
        be defined as:
        \begin{equation}
            A_{Nm}=\bigcup_{n=N}^{\infty}\Big\{
                \omega\in\Omega:
                |f_{n}(\omega)-f(\omega)|\geq\frac{1}{m}|
                \Big\}
        \end{equation}
        Define $B_{m}$ as:
        \begin{equation}
            B_{m}=\bigcap_{N=1}^{\infty}A_{Nm}
        \end{equation}
        But since $\mu(\Omega)<\infty$, the measure
        $\mu$ is continuous from above. Therefore:
        \begin{equation}
            \mu(B_{m})
                =\underset{N\rightarrow\infty}{\lim}
                \mu(A_{Nm})
        \end{equation}
        But $B_{m}\subseteq{E^{c}}$, and thus
        $\mu(B_{m})=0$. But then
        $\mu(A_{Nm})\rightarrow{0}$.
    \end{bproof}
    So we have shown that, even though convergence
    almost everywhere and almost uniform convergence
    are diferent concepts, on sets of finite measure
    they are equivalent. In probably the total measure
    of the entire set if 1, and so finite. Thus, in
    probabability spaces, almost everywhere convergence
    and almost uniform convergence will always be
    equivalent. Thus it is common to use the term
    convergence almost surely, and forgot the differences
    between the two properties. There is a third type of
    convergence called convergence in measure.
    \begin{ldefinition}{Convergence in Measure}
        A sequence of functions $f_{n}$ convergence
        in measure to $f$ is, for all $\delta>0$, the
        following is true:
        \begin{equation}
            \underset{n\rightarrow\infty}{\lim}
            \mu\Big(
                \big\{
                    \omega:|f_{n}(\omega)-f(\omega)|
                    <\delta
                \big\}
            \Big)=0
        \end{equation}
    \end{ldefinition}
    \begin{example}
        Let $\Omega=[0,1]$, and let $\mathcal{B}$ be the Borel
        $\sigma-\textrm{Algebra}$ on $[0,1]$. Finally, let $\mu$ be the standard
        Lebesgue-Measure. Define the following:
        \begin{equation}
            f_{1}=
            \begin{cases}
                1,&x<\frac{1}{2}\\
                0,&x\geq\frac{1}{2}
            \end{cases}
        \end{equation}
        Define $f_{2}=1-f_{1}$. The split the interval into fourths and define
        $f_{3}$ as 1 in $[0,1/4)$ and zero otherwise, and continue the pattern
        for $f_{4}$, $f_{5}$, $f_{6}$, and $f_{7}$. This sequence of functions
        converges nowhere since there will be 1's and 0's oscillating back and
        forth, and thus there is no limit. However, $f_{n}$ converges in
        measure to 0.
    \end{example}
    \begin{theorem}
        If $f_{n}\rightarrow{f}$ in measure $\mu$, and if $g=f$ almost
        everywhere, then $f_{n}\rightarrow{g}$ in measure $\mu$.
    \end{theorem}
    \begin{proof}
        For all $\delta>0$,
        $\mu(\{\omega:|f_{n}(\omega)-f(\omega)|>\delta\}$
        tends to zero as $n\rightarrow\infty$. But:
        \begin{equation}
            \begin{split}
                \{\omega:|f_{n}(\omega)-f(\omega)|
                &>\delta\}\\
                &=\Big(
                    \{\omega:|f_{n}(\omega)-f(\omega)|>\delta\}
                    \bigcap
                    \{\omega:f(\omega)=g(\omega)\}
                \Big)\\
                &\bigcup\Big(
                    \{\omega:|f_{n}(\omega)-f(\omega)|>\delta\}
                    \bigcap
                    \{\omega:f(\omega)\ne{g}(\omega)\}
                \Big)
            \end{split}
        \end{equation}
    \end{proof}
    \begin{theorem}
        If $f_{n}\rightarrow{f}$ in measure $\mu$ and if $f_{n}\rightarrow{g}$
        in measure $\mu$, and $f=g$ $\mu$ almost everywhre.
    \end{theorem}
    \begin{proof}
        For let $A=\{\omega:f(\omega)\ne{g}(\omega)\}$. Then
        $A=\{\omega:|f(\omega)-g(\omega)|>0\}$. Thus we may write:
        \begin{equation}
            A=\bigcup_{n=1}^{\infty}
            \Big\{\omega:|f(\omega)-g(\omega)|>\frac{1}{n}
            \Big\}
        \end{equation}
        We now show that $\{\omega:|f(\omega)-g(\omega)|>\frac{1}{n}\}$ has
        measure zero for all $n\in\mathbb{N}$. By subadditivity, this will
        imply $A$ has measure zero. From the triangle inequality:
        \begin{equation}
            |f(\omega)-g(\omega)|\leq
            |f(\omega)-f_{n}(\omega)|+
            |g(\omega)-f_{n}(\omega)|
        \end{equation}
        If $|f(\omega)-g(\omega)|\geq{1/m}$, then at least one of the two
        numbers here must be greater than $1/2m$. Thus, either
        $|f(\omega)-f_{n}(\omega)|\geq\frac{1}{2m}$ or
        $|g(\omega)-f_{n}(\omega)|\geq\frac{1}{2m}$, or both. Therefore:
        \begin{equation}
            \{\omega:|f(\omega)-g(\omega)|>\frac{1}{n}\}
            \subseteq\{\omega:|f(\omega)-f_{n}(\omega)|>\frac{1}{2n}\}
            \bigcup\{\omega:|g(\omega)-f_{n}(\omega)|>\frac{1}{2n}\}
        \end{equation}
        But the two sets on the left have measures that tend to zero as
        $n\rightarrow\infty$, and thus the set on the left has measure zero.
        Thus, by subadditivity their union has measure zero, and therefore
        $\mu(A)=0$. Thus, $f=g$ $\mu$ almost everywhere.
    \end{proof}
    \begin{theorem}
        If $f_{n}\rightarrow{f}$ in measure $\mu$ and $g_{n}\rightarrow{g}$ in
        measure $\mu$, then $f_{n}+g_{n}\rightarrow{f+g}$ in measure $\mu$.
    \end{theorem}
    \begin{theorem}
        If $f_{n}\rightarrow{f}$ in measure $\mu$, then there exists a
        subsequence of $f_{n}$ that converges to $f$ almost uniformly.
    \end{theorem}
    \begin{proof}
        For all $\delta>0$ the limit of $\mu(\{|f_{n}-f|\geq\delta\})$ tends
        to zero as $n\rightarrow\infty$. Thus, there is an index $n_{1}$ such
        that $\mu(\{|f_{n_{1}}-f|\geq{1}\})<1$. Choosing $\delta=1/2$, we find
        an index $n_{2}$ such that $\mu(\{|f_{n_{2}}-f|\geq1/2\})<1/2$. Carrying
        on, we obtain a sequence $n_{k}$ such that, for all $k\in\mathbb{N}$,
        $\mu(\{|f_{n_{k}}-f|\geq1/k\})<1/2^{k}$. Let
        $E_{k}=\{|f_{n_{k}}-f|\geq1/k\}^{C}$. $\mu(E_{k}^{C})<1/2^{k}$, and thus
        for all $\varepsilon>0$ there is an $N\in\mathbb{N}$ such that, for all
        $n>N$, $\mu(E_{n}^{C})<\varepsilon$.
    \end{proof}
    The $\sigma-\textrm{Algebra}$ generated by a set $\mathcal{E}$ is the
    intersection of all possible $\sigma-\textrm{Algebra's}$ that contain all
    elements of $\mathcal{E}$. Given a function $f:\Omega\rightarrow\mathbb{R}$
    and a $\sigma-\textrm{Algebra}$ $\mathcal{A}$ on $\Omega$, it is often a
    good strategy to look at the set:
    \begin{equation}
        \mathcal{B}_{f,\mathcal{A}}
            =\big\{B\subseteq\mathbb{R}:f^{-1}(B)\in\mathcal{A}\big\}
    \end{equation}
    And then show that all intervals of the form $(a,b)$ are contained within
    $\mathcal{B}_{f,\mathcal{A}}$, thus implying that $f$ is
    $\mathcal{A}-\mathcal{B}$ measurable, where $\mathcal{B}$ is the Borel
    $\sigma-\textrm{Algebra}$ on $\mathbb{R}$. Recapping, we have now discussed
    three different types of convergence: Almost uniform convergence,
    convergence almost everywhere, and convergence in measure.
    \begin{theorem}
        If $f_{n}\rightarrow{f}$ almost uniformly, then $f_{n}\rightarrow{f}$
        in measure.
    \end{theorem}