\section{Definitions and Properties}
    We wish to eventually talk about what it means for a function to be
    \textit{measurable}. First we do a quick review of functions.
    \begin{lexample}
        If we let $f:\mathbb{R}\rightarrow\mathbb{R}$ be defined by
        $f(x)=x^{2}$, then $f([1,2])=[1,4]$, and
        $f^{-1}([1,4])=[1,2]\cup[-2,-1]$. As another example we can consider
        $f(x)=\sin(x)$. Then $f^{-1}(\{0\})=\{n\pi:n\in\mathbb{N}\}$ and
        $f^{-1}([-1,1])=\mathbb{R}$.
    \end{lexample}
    \begin{theorem}
        If $X$ and $Y$ are sets, $f:X\rightarrow{Y}$, and if $A,B\subset{X}$,
        then:
        \begin{equation}
            f(A\cup{B})=f(A)\cup{f}(B)
        \end{equation}
    \end{theorem}
    \begin{theorem}
        If $X$ and $Y$ are sets, $f:X\rightarrow{Y}$, and if $A,B\subset{X}$,
        then:
        \begin{equation}
            f(A\cap{B})\subseteq{f(A)\cap{f}(B)}
        \end{equation}
    \end{theorem}
    For pre-images, we get equality:
    \begin{theorem}
        If $X$ and $Y$ are sets, $f:X\rightarrow{Y}$, and if $A,B\subset{X}$,
        then:
        \begin{equation}
            f^{-1}(A\cup{B})=f^{-1}(A)\cup{f}^{-1}(B)
        \end{equation}
    \end{theorem}
    \begin{theorem}
        If $X$ and $Y$ are sets, $f:X\rightarrow{Y}$, and if $A,B\subset{X}$,
        then:
        \begin{equation}
            f^{-1}(A\cap{B})=f^{-1}(A)\cap{f}^{-1}(B)
        \end{equation}
    \end{theorem}
    \begin{theorem}
        If $X$ and $Y$ are sets, $f:X\rightarrow{Y}$, and if $A\subset{X}$,
        then:
        \begin{equation}
            f^{-1}(A^{C})=f^{-1}(A)^{C}
        \end{equation}
    \end{theorem}
    \begin{theorem}
        If $X$ and $Y$ are sets,if
        $f:X\rightarrow{Y}$ is a function, and if
        $A\subseteq{X}$, then:
        \begin{equation}
            A\subseteq{f^{-1}\Big(f\big(A\big)\Big)}
        \end{equation}
    \end{theorem}
    \begin{theorem}
        If $X$ and $Y$ are sets, if
        $f:X\rightarrow{Y}$ is an injective function,
        and if $A\subseteq{X}$, then:
        \begin{equation}
            A=f^{-1}\Big(f\big(A\big)\Big)
        \end{equation}
    \end{theorem}
    Recalling some definitions, a measure on a
    $\sigma\textrm{-Algebra}$ $\mathcal{A}$ is a function
    $\mu:\mathcal{A}\rightarrow\mathbb{R}$ such that
    $\mu(A)\geq{0}$, $\mu(\emptyset)=0$, and given a countable
    collection of disjoint sets $A_{i}\in\mathcal{A}$,
    $\mu(\cup_{i=1}^{\infty}A_{i})=\sum_{n=1}^{\infty}\mu(A_{i})$.
    \begin{theorem}
        If $\Omega$ is a set and $\mathcal{A}$ is a
        $\sigma\text{-Algebra}$ on $\Omega$, and if
        $\mu:\mathcal{A}\rightarrow\mathbb{R}$ is a function such that:
        \begin{enumerate}
            \item $\mu(A)\geq{0}$
            \item $\mu(\emptyset)=0$
            \item $\mu$ is finitely additive
            \item $\mu$ is continuous from below
        \end{enumerate}
        Then $\mu$ is a measure.
    \end{theorem}
    \begin{proof}
        All that is necessary to show is countable additivity.
        Let $A_{n}$ be a countable collection of disjoint elements
        of $\mathcal{A}$, and let $B_{n}=\cup_{k=1}^{n}A_{k}$.
        Then:
        \begin{align}
            \mu(B_{n})&=\mu(\cup_{n=1}^{n}A_{k})\\
            &=\sum_{k=1}^{n}\mu(A_{k})
        \end{align}
        But by definition, for all $n\in\mathbb{N}$,
        $B_{n}\subseteq{B_{n+1}}$, and therefore by continuity from
        below, we have:
        \begin{equation}
            \mu(\cup_{n=1}^{\infty}B_{k})
            =\lim_{n\rightarrow\infty}\mu(\cup_{k=1}^{n}B_{k})
        \end{equation}
        And therefore
        \begin{equation}
            \mu(\cup_{n=1}^{\infty}A_{k})=\sum_{k=1}^{\infty}\mu(A_{k})
        \end{equation}
    \end{proof}
        The Borel $\sigma\text{-Algebra}$ on $\mathbb{R}$ is the smallest set
        that makes open sets, elements of the standard topology on $\mathbb{R}$,
        measurable. In an analogous manner to how continuous functions are
        defined for topological spaces, measurable functions can also be
        defined.
    \subsection{Measurable Functions}
        \begin{ldefinition}{Measurable Functions}
            A measure function from a measurable space $(A,\mathcal{A})$ to a
            measurable space $(B,\mathcal{B})$ is a function $f:A\rightarrow{B}$
            such that, for all $\mathcal{U}\in\mathcal{A}$,
            $f^{-1}(\mathcal{U})\in\mathcal{B}$.
        \end{ldefinition}
        That is, the pre-image of measurable sets is measurable. This is similar
        to continuous functions where the pre-image of open sets is open. Such
        functions are also called $\mathcal{A}-\mathcal{B}$ measurable.
        \begin{lexample}
            If $\mathcal{A}$ is a $\sigma\text{-Algebra}$ on $\Omega$, $\Omega$
            and if $\mathcal{B}=\{\emptyset,\Omega\}$, then any function
            $f:\omega\rightarrow\Omega$ will be $\mathcal{A}-\mathcal{B}$
            measurable. If $\mathcal{A}=\mathcal{P}(\Omega)$ and if
            $\mathcal{B}$ is a $\sigma\text{-Algebra}$ on $\Omega$, then again
            any function $f:\Omega\rightarrow\Omega$ will be
            $\mathcal{A}-\mathcal{B}$ measurable. There are similar notions in
            topology called the discrete and chaotic topologies which make all
            functions continuous.
        \end{lexample}
        \begin{theorem}
            If $A$ and $B$ are sets, if $\mathcal{B}$ is a
            $\sigma\text{-Algebra}$ on $B$, and if $f:A\rightarrow{B}$ is a
            function, then the set $\mathcal{A}$ defined by:
            \begin{equation}
                \mathcal{A}=\{f^{-1}(\mathcal{U}):\mathcal{U}\in\mathcal{B}\}
            \end{equation}
            Is a $\sigma\text{-Algebra}$ on $A$.
        \end{theorem}
        \begin{proof}
            It is true that $\emptyset\in\mathcal{A}$, since
            $\empty\in\mathcal{B}$ and $f^{-1}(\emptyset)=\emptyset$. Also,
            $B\in\mathcal{B}$, and $A=f^{-1}(B)$, and therefore
            $A\in\mathcal{A}$. If $A\in\mathcal{A}$, then there is a
            $B\in\mathcal{B}$ such that $A=f^{-1}(B)$, and thus:
            \begin{equation}
                A^{C}=f^{-1}(B)^{C}=f^{-1}(B^{C})
            \end{equation}
            But if $B\in\mathcal{B}$, then $B^{C}\in\mathcal{B}$, and thus
            $A^{C}\in\mathcal{A}$. Finally, for any countable collection of
            sets $A_{n}\in\mathcal{A}$, there is a countable collection of sets
            $B_{n}$ such that $A_{n}=f^{-1}(B_{n})$ But then:
            \begin{equation}
                \cup_{n=1}^{\infty}A_{n}=\cup_{n=1}^{\infty}f^{-1}(B_{n})
                    =f^{-1}(\cup_{n=1}^{\infty}B_{n})
            \end{equation}
            But $\mathcal{B}$ is a $\sigma\text{-Algebra}$, and thus
            $\cup_{n=1}^{\infty}B_{n}\in\mathcal{B}$. Therefore
            $\cup_{n=1}^{\infty}A_{n}\in\mathcal{A}$.
        \end{proof}
        It's worth noting that $\mathcal{A}$ is the smallest
        $\sigma\text{-Algebra}$ on $A$ that will make
        $f$ $\mathcal{A}-\mathcal{B}$ measurable. Removing any set from
        $\mathcal{A}$ will result in $f:A\rightarrow{B}$ being non-measurable
        with respect to $\mathcal{A}$ and $\mathcal{B}$.
        \begin{theorem}
            If $A$ and $B$ are sets, $f:A\rightarrow{B}$ a function, and if
            $\mathcal{A}$ is a $\sigma\text{-Algebra}$ on $A$, then the set
            $\mathcal{B}$ defined by:
            \begin{equation}
                \mathcal{B}=\{B\subset{B}:f^{-1}(B)\in\mathcal{A}\}
            \end{equation}
            Is a $\sigma\text{-Algebra}$ on $B$.
        \end{theorem}
        \begin{proof}
            $\emptyset$ and $B$ are elements since
            $f^{-1}(\emptyset)=\emptyset\in\mathcal{A}$, and
            $f^{-1}(B)=A\in\mathcal{A}$.
        \end{proof}
        \begin{theorem}
            If $f:\Omega\rightarrow\mathbb{R}$, if $a\in\mathbb{R}$, if
            $\mathcal{B}$ is the Borel $\sigma\text{-Algebra}$ on $\mathbb{R}$,
            and if $\mathcal{A}$ is defined by:
            \begin{equation}
                \mathcal{A}=\{\omega\in\Omega:f(\omega)<a\}
            \end{equation}
            then $f$ is $\mathcal{A}-\mathcal{B}$ measurable.
        \end{theorem}
        \begin{theorem}
            If $\Omega$ is a set, and if $\mathcal{A}$ is a
            $\sigma\text{-Algebra}$ on $\Omega$, and if
            $f:\Omega\rightarrow\mathbb{R}$ is a function such that, for all
            $a\in\mathbb{R}$, $\{\omega\in\Omega:f(\omega)<a\}\in\mathcal{A}$,
            then $f$ is $\mathcal{A}-\mathcal{B}$ measurable, where
            $\mathcal{B}$ is the Borel $\sigma\text{-Algebra}$.
        \end{theorem}
        \begin{theorem}
            If $f:\mathbb{R}\rightarrow\mathbb{R}$ is continuous, then it is
            Borel measurable.
        \end{theorem}
        \begin{theorem}
            If $\Omega$ is a set, $\mathcal{A}$ is a $\sigma\text{-Algebra}$ on
            $\Omega$, and if $f:\Omega\rightarrow\mathbb{R}$ is
            $\mathcal{A}-\mathcal{B}$ measurable, where $\mathcal{B}$ is the
            Borel $\sigma\text{-Algebra}$, and if
            $g:\mathbb{R}\rightarrow\mathbb{R}$ is $\mathcal{B}-\mathcal{B}$
            measurable, then $g\circ{f}:\Omega\rightarrow\mathbb{R}$ is
            $\mathcal{A}-\mathcal{B}$ measurable.
        \end{theorem}
        \begin{proof}
            For if $B\in\mathcal{B}$, then $g^{-1}(B)\in\mathcal{B}$, for $g$ is
            $\mathcal{B}-\mathcal{B}$ measurable. but then, as $f$ is
            $\mathcal{A}-\mathcal{B}$ measurable,
            $f^{-1}(g^{-1}(B))\in\mathcal{A}$. Therefore, $g\circ{f}$ is
            $\mathcal{A}-\mathcal{B}$ measurable.
        \end{proof}
        In particular, if we have two measurable functions on $\mathbb{R}$, then
        the composition of these two function is also measurable. This is
        analogous to the fact that the composition of continuous functions is
        continuous. The sum, difference, and product of measurable functions is
        also measurable. We now define the Borel $\sigma\text{-Algebra}$ for
        $\mathbb{R}^{2}$. This is denoted $\mathcal{B}_{2}$. It is defined
        similarly to $\mathcal{B}$: It is the smallest $\sigma\text{-Algebra}$
        that contains all open subsets of $\mathbb{R}^{2}$. We can also limit
        this to all open rectangles in the plane, or all open discs.
        \begin{theorem}
            If $\Omega$ is a set, $\mathcal{A}$ a $\sigma\text{-Algebra}$ on
            $\Omega$, if $f,g:\Omega\rightarrow\mathbb{R}$ are
            $\mathcal{A}-\mathcal{B}$ measurable functions, and if
            $\vec{h}:\Omega\rightarrow\mathbb{R}^{2}$ is defined by
            $\vec{h}(\boldsymbol{\omega})=(f(\omega),g(\omega))$, then
            $\vec{h}$ is $\mathcal{A}-\mathcal{B}_{2}$ measurable.
        \end{theorem}
        This theorem goes the other way as well. $\vec(h)$ is measurable if
        and only if $f$ and $g$ are measurable.
        \begin{theorem}
            If $\Omega$ is a set, $\mathcal{A}$ a $\sigma\text{-Algebra}$ on
            $\Omega$, if $\mathcal{B}$ is the Borel $\sigma\text{-Algebra}$ on
            $\mathbb{R}$, and if $f,g:\Omega\rightarrow\mathbb{R}$ are
            $\mathcal{A}-\mathcal{B}$ measurable functions, then $f+g$ is
            $\mathcal{A}-\mathcal{B}$ measurable.
        \end{theorem}
        \begin{proof}
            For let $\varphi:\mathbb{R}^{2}\rightarrow\mathbb{R}$ be defined by
            $\varphi(x,y)=x+y$. Then $\varphi$ is continuous, and is therefore
            $\mathcal{B}_{2}-\mathcal{B}$ measurable. Let
            $h:\Omega\rightarrow\mathbb{R}^{2}$ be defined by
            $h(\omega)=(f(\omega),g(\omega))$. Then $h$ is
            $\mathcal{A}-\mathcal{B}_{2}$ measurable. But by taking the
            composition, we have that $f+g=\varphi\circ{h}$ is
            $\mathcal{A}-\mathcal{B}$ measurable.
        \end{proof}
        We can do the same thing with multiplication by defining
        $\varphi(x,y)=x\cdot{y}$.
        \begin{theorem}
            If $f,g:\Omega\rightarrow\mathbb{R}$ are $\mathcal{A}-\mathcal{B}$
            measurable, and if:
            \begin{equation}
                A=\{\omega\in\Omega:f(\omega)<g(\omega)\}
            \end{equation}
            $\Pi$ open set (Half plane along diagonal. Draw this). Do the same
            thing with the line $L$. They're measurable, yadda yadda.
        \end{theorem}
    \subsection{Sequences of Measurable Functions}
        If $f_{n}:\omega\rightarrow\mathbb{R}$ is a sequence of
        $\mathcal{A}-\mathcal{B}$ measurable functions, and if
        $f_{n}\rightarrow{f}$, then $f$ is $\mathcal{A}-\mathcal{B}$ measurable. 
        \begin{theorem}
            Let $F(\omega)=\sup\{f_{n}(\omega):n\in\mathbb{N}\}$.
            Then $F$ is measurable.
        \end{theorem}
        \begin{proof}
            For let $a\in\mathbb{R}$. Then:
            \begin{equation}
                \{\omega:F(\omega)\leq{a}\}=
                \bigcap_{n=1}^{\infty}\{\omega:f_{n}(\omega)\leq{a}\}
            \end{equation}
            Then $F(\omega)\leq{a}$ if and only if
            $f_{n}(\omega)\leq{a}$ for all $n\in\mathbb{N}$.
        \end{proof}
        Similarly, $F(\omega)=\inf\{f_{n}(\omega)\}$ is measurable.
        \begin{theorem}
            If $f_{n}:\Omega\rightarrow\mathbb{R}$ are
            $\mathcal{A}-\mathcal{B}$ functions, then
            $\underset{n\rightarrow{\infty}}{\overline{\lim}}f_{n}$ and
            $\underset{n\rightarrow{\infty}}{\underline{\lim}}f_{n}$
            are $\mathcal{A}-\mathcal{B}$ measurable.
        \end{theorem}