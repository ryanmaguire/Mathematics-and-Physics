\documentclass[crop=false,class=book,oneside]{standalone}
%----------------------------Preamble-------------------------------%
%---------------------------Packages----------------------------%
\usepackage{geometry}
\geometry{b5paper, margin=1.0in}
\usepackage[T1]{fontenc}
\usepackage{graphicx, float}            % Graphics/Images.
\usepackage{natbib}                     % For bibliographies.
\bibliographystyle{agsm}                % Bibliography style.
\usepackage[french, english]{babel}     % Language typesetting.
\usepackage[dvipsnames]{xcolor}         % Color names.
\usepackage{listings}                   % Verbatim-Like Tools.
\usepackage{mathtools, esint, mathrsfs} % amsmath and integrals.
\usepackage{amsthm, amsfonts, amssymb}  % Fonts and theorems.
\usepackage{tcolorbox}                  % Frames around theorems.
\usepackage{upgreek}                    % Non-Italic Greek.
\usepackage{fmtcount, etoolbox}         % For the \book{} command.
\usepackage[newparttoc]{titlesec}       % Formatting chapter, etc.
\usepackage{titletoc}                   % Allows \book in toc.
\usepackage[nottoc]{tocbibind}          % Bibliography in toc.
\usepackage[titles]{tocloft}            % ToC formatting.
\usepackage{pgfplots, tikz}             % Drawing/graphing tools.
\usepackage{imakeidx}                   % Used for index.
\usetikzlibrary{
    calc,                   % Calculating right angles and more.
    angles,                 % Drawing angles within triangles.
    arrows.meta,            % Latex and Stealth arrows.
    quotes,                 % Adding labels to angles.
    positioning,            % Relative positioning of nodes.
    decorations.markings,   % Adding arrows in the middle of a line.
    patterns,
    arrows
}                                       % Libraries for tikz.
\pgfplotsset{compat=1.9}                % Version of pgfplots.
\usepackage[font=scriptsize,
            labelformat=simple,
            labelsep=colon]{subcaption} % Subfigure captions.
\usepackage[font={scriptsize},
            hypcap=true,
            labelsep=colon]{caption}    % Figure captions.
\usepackage[pdftex,
            pdfauthor={Ryan Maguire},
            pdftitle={Mathematics and Physics},
            pdfsubject={Mathematics, Physics, Science},
            pdfkeywords={Mathematics, Physics, Computer Science, Biology},
            pdfproducer={LaTeX},
            pdfcreator={pdflatex}]{hyperref}
\hypersetup{
    colorlinks=true,
    linkcolor=blue,
    filecolor=magenta,
    urlcolor=Cerulean,
    citecolor=SkyBlue
}                           % Colors for hyperref.
\usepackage[toc,acronym,nogroupskip,nopostdot]{glossaries}
\usepackage{glossary-mcols}
%------------------------Theorem Styles-------------------------%
\theoremstyle{plain}
\newtheorem{theorem}{Theorem}[section]

% Define theorem style for default spacing and normal font.
\newtheoremstyle{normal}
    {\topsep}               % Amount of space above the theorem.
    {\topsep}               % Amount of space below the theorem.
    {}                      % Font used for body of theorem.
    {}                      % Measure of space to indent.
    {\bfseries}             % Font of the header of the theorem.
    {}                      % Punctuation between head and body.
    {.5em}                  % Space after theorem head.
    {}

% Italic header environment.
\newtheoremstyle{thmit}{\topsep}{\topsep}{}{}{\itshape}{}{0.5em}{}

% Define environments with italic headers.
\theoremstyle{thmit}
\newtheorem*{solution}{Solution}

% Define default environments.
\theoremstyle{normal}
\newtheorem{example}{Example}[section]
\newtheorem{definition}{Definition}[section]
\newtheorem{problem}{Problem}[section]

% Define framed environment.
\tcbuselibrary{most}
\newtcbtheorem[use counter*=theorem]{ftheorem}{Theorem}{%
    before=\par\vspace{2ex},
    boxsep=0.5\topsep,
    after=\par\vspace{2ex},
    colback=green!5,
    colframe=green!35!black,
    fonttitle=\bfseries\upshape%
}{thm}

\newtcbtheorem[auto counter, number within=section]{faxiom}{Axiom}{%
    before=\par\vspace{2ex},
    boxsep=0.5\topsep,
    after=\par\vspace{2ex},
    colback=Apricot!5,
    colframe=Apricot!35!black,
    fonttitle=\bfseries\upshape%
}{ax}

\newtcbtheorem[use counter*=definition]{fdefinition}{Definition}{%
    before=\par\vspace{2ex},
    boxsep=0.5\topsep,
    after=\par\vspace{2ex},
    colback=blue!5!white,
    colframe=blue!75!black,
    fonttitle=\bfseries\upshape%
}{def}

\newtcbtheorem[use counter*=example]{fexample}{Example}{%
    before=\par\vspace{2ex},
    boxsep=0.5\topsep,
    after=\par\vspace{2ex},
    colback=red!5!white,
    colframe=red!75!black,
    fonttitle=\bfseries\upshape%
}{ex}

\newtcbtheorem[auto counter, number within=section]{fnotation}{Notation}{%
    before=\par\vspace{2ex},
    boxsep=0.5\topsep,
    after=\par\vspace{2ex},
    colback=SeaGreen!5!white,
    colframe=SeaGreen!75!black,
    fonttitle=\bfseries\upshape%
}{not}

\newtcbtheorem[use counter*=remark]{fremark}{Remark}{%
    fonttitle=\bfseries\upshape,
    colback=Goldenrod!5!white,
    colframe=Goldenrod!75!black}{ex}

\newenvironment{bproof}{\textit{Proof.}}{\hfill$\square$}
\tcolorboxenvironment{bproof}{%
    blanker,
    breakable,
    left=3mm,
    before skip=5pt,
    after skip=10pt,
    borderline west={0.6mm}{0pt}{green!80!black}
}

\AtEndEnvironment{lexample}{$\hfill\textcolor{red}{\blacksquare}$}
\newtcbtheorem[use counter*=example]{lexample}{Example}{%
    empty,
    title={Example~\theexample},
    boxed title style={%
        empty,
        size=minimal,
        toprule=2pt,
        top=0.5\topsep,
    },
    coltitle=red,
    fonttitle=\bfseries,
    parbox=false,
    boxsep=0pt,
    before=\par\vspace{2ex},
    left=0pt,
    right=0pt,
    top=3ex,
    bottom=1ex,
    before=\par\vspace{2ex},
    after=\par\vspace{2ex},
    breakable,
    pad at break*=0mm,
    vfill before first,
    overlay unbroken={%
        \draw[red, line width=2pt]
            ([yshift=-1.2ex]title.south-|frame.west) to
            ([yshift=-1.2ex]title.south-|frame.east);
        },
    overlay first={%
        \draw[red, line width=2pt]
            ([yshift=-1.2ex]title.south-|frame.west) to
            ([yshift=-1.2ex]title.south-|frame.east);
    },
}{ex}

\AtEndEnvironment{ldefinition}{$\hfill\textcolor{Blue}{\blacksquare}$}
\newtcbtheorem[use counter*=definition]{ldefinition}{Definition}{%
    empty,
    title={Definition~\thedefinition:~{#1}},
    boxed title style={%
        empty,
        size=minimal,
        toprule=2pt,
        top=0.5\topsep,
    },
    coltitle=Blue,
    fonttitle=\bfseries,
    parbox=false,
    boxsep=0pt,
    before=\par\vspace{2ex},
    left=0pt,
    right=0pt,
    top=3ex,
    bottom=0pt,
    before=\par\vspace{2ex},
    after=\par\vspace{1ex},
    breakable,
    pad at break*=0mm,
    vfill before first,
    overlay unbroken={%
        \draw[Blue, line width=2pt]
            ([yshift=-1.2ex]title.south-|frame.west) to
            ([yshift=-1.2ex]title.south-|frame.east);
        },
    overlay first={%
        \draw[Blue, line width=2pt]
            ([yshift=-1.2ex]title.south-|frame.west) to
            ([yshift=-1.2ex]title.south-|frame.east);
    },
}{def}

\AtEndEnvironment{ltheorem}{$\hfill\textcolor{Green}{\blacksquare}$}
\newtcbtheorem[use counter*=theorem]{ltheorem}{Theorem}{%
    empty,
    title={Theorem~\thetheorem:~{#1}},
    boxed title style={%
        empty,
        size=minimal,
        toprule=2pt,
        top=0.5\topsep,
    },
    coltitle=Green,
    fonttitle=\bfseries,
    parbox=false,
    boxsep=0pt,
    before=\par\vspace{2ex},
    left=0pt,
    right=0pt,
    top=3ex,
    bottom=-1.5ex,
    breakable,
    pad at break*=0mm,
    vfill before first,
    overlay unbroken={%
        \draw[Green, line width=2pt]
            ([yshift=-1.2ex]title.south-|frame.west) to
            ([yshift=-1.2ex]title.south-|frame.east);},
    overlay first={%
        \draw[Green, line width=2pt]
            ([yshift=-1.2ex]title.south-|frame.west) to
            ([yshift=-1.2ex]title.south-|frame.east);
    }
}{thm}

%--------------------Declared Math Operators--------------------%
\DeclareMathOperator{\adjoint}{adj}         % Adjoint.
\DeclareMathOperator{\Card}{Card}           % Cardinality.
\DeclareMathOperator{\curl}{curl}           % Curl.
\DeclareMathOperator{\diam}{diam}           % Diameter.
\DeclareMathOperator{\dist}{dist}           % Distance.
\DeclareMathOperator{\Div}{div}             % Divergence.
\DeclareMathOperator{\Erf}{Erf}             % Error Function.
\DeclareMathOperator{\Erfc}{Erfc}           % Complementary Error Function.
\DeclareMathOperator{\Ext}{Ext}             % Exterior.
\DeclareMathOperator{\GCD}{GCD}             % Greatest common denominator.
\DeclareMathOperator{\grad}{grad}           % Gradient
\DeclareMathOperator{\Ima}{Im}              % Image.
\DeclareMathOperator{\Int}{Int}             % Interior.
\DeclareMathOperator{\LC}{LC}               % Leading coefficient.
\DeclareMathOperator{\LCM}{LCM}             % Least common multiple.
\DeclareMathOperator{\LM}{LM}               % Leading monomial.
\DeclareMathOperator{\LT}{LT}               % Leading term.
\DeclareMathOperator{\Mod}{mod}             % Modulus.
\DeclareMathOperator{\Mon}{Mon}             % Monomial.
\DeclareMathOperator{\multideg}{mutlideg}   % Multi-Degree (Graphs).
\DeclareMathOperator{\nul}{nul}             % Null space of operator.
\DeclareMathOperator{\Ord}{Ord}             % Ordinal of ordered set.
\DeclareMathOperator{\Prin}{Prin}           % Principal value.
\DeclareMathOperator{\proj}{proj}           % Projection.
\DeclareMathOperator{\Refl}{Refl}           % Reflection operator.
\DeclareMathOperator{\rk}{rk}               % Rank of operator.
\DeclareMathOperator{\sgn}{sgn}             % Sign of a number.
\DeclareMathOperator{\sinc}{sinc}           % Sinc function.
\DeclareMathOperator{\Span}{Span}           % Span of a set.
\DeclareMathOperator{\Spec}{Spec}           % Spectrum.
\DeclareMathOperator{\supp}{supp}           % Support
\DeclareMathOperator{\Tr}{Tr}               % Trace of matrix.
%--------------------Declared Math Symbols--------------------%
\DeclareMathSymbol{\minus}{\mathbin}{AMSa}{"39} % Unary minus sign.
%------------------------New Commands---------------------------%
\DeclarePairedDelimiter\norm{\lVert}{\rVert}
\DeclarePairedDelimiter\ceil{\lceil}{\rceil}
\DeclarePairedDelimiter\floor{\lfloor}{\rfloor}
\newcommand*\diff{\mathop{}\!\mathrm{d}}
\newcommand*\Diff[1]{\mathop{}\!\mathrm{d^#1}}
\renewcommand*{\glstextformat}[1]{\textcolor{RoyalBlue}{#1}}
\renewcommand{\glsnamefont}[1]{\textbf{#1}}
\renewcommand\labelitemii{$\circ$}
\renewcommand\thesubfigure{%
    \arabic{chapter}.\arabic{figure}.\arabic{subfigure}}
\addto\captionsenglish{\renewcommand{\figurename}{Fig.}}
\numberwithin{equation}{section}

\renewcommand{\vector}[1]{\boldsymbol{\mathrm{#1}}}

\newcommand{\uvector}[1]{\boldsymbol{\hat{\mathrm{#1}}}}
\newcommand{\topspace}[2][]{(#2,\tau_{#1})}
\newcommand{\measurespace}[2][]{(#2,\varSigma_{#1},\mu_{#1})}
\newcommand{\measurablespace}[2][]{(#2,\varSigma_{#1})}
\newcommand{\manifold}[2][]{(#2,\tau_{#1},\mathcal{A}_{#1})}
\newcommand{\tanspace}[2]{T_{#1}{#2}}
\newcommand{\cotanspace}[2]{T_{#1}^{*}{#2}}
\newcommand{\Ckspace}[3][\mathbb{R}]{C^{#2}(#3,#1)}
\newcommand{\funcspace}[2][\mathbb{R}]{\mathcal{F}(#2,#1)}
\newcommand{\smoothvecf}[1]{\mathfrak{X}(#1)}
\newcommand{\smoothonef}[1]{\mathfrak{X}^{*}(#1)}
\newcommand{\bracket}[2]{[#1,#2]}

%------------------------Book Command---------------------------%
\makeatletter
\renewcommand\@pnumwidth{1cm}
\newcounter{book}
\renewcommand\thebook{\@Roman\c@book}
\newcommand\book{%
    \if@openright
        \cleardoublepage
    \else
        \clearpage
    \fi
    \thispagestyle{plain}%
    \if@twocolumn
        \onecolumn
        \@tempswatrue
    \else
        \@tempswafalse
    \fi
    \null\vfil
    \secdef\@book\@sbook
}
\def\@book[#1]#2{%
    \refstepcounter{book}
    \addcontentsline{toc}{book}{\bookname\ \thebook:\hspace{1em}#1}
    \markboth{}{}
    {\centering
     \interlinepenalty\@M
     \normalfont
     \huge\bfseries\bookname\nobreakspace\thebook
     \par
     \vskip 20\p@
     \Huge\bfseries#2\par}%
    \@endbook}
\def\@sbook#1{%
    {\centering
     \interlinepenalty \@M
     \normalfont
     \Huge\bfseries#1\par}%
    \@endbook}
\def\@endbook{
    \vfil\newpage
        \if@twoside
            \if@openright
                \null
                \thispagestyle{empty}%
                \newpage
            \fi
        \fi
        \if@tempswa
            \twocolumn
        \fi
}
\newcommand*\l@book[2]{%
    \ifnum\c@tocdepth >-3\relax
        \addpenalty{-\@highpenalty}%
        \addvspace{2.25em\@plus\p@}%
        \setlength\@tempdima{3em}%
        \begingroup
            \parindent\z@\rightskip\@pnumwidth
            \parfillskip -\@pnumwidth
            {
                \leavevmode
                \Large\bfseries#1\hfill\hb@xt@\@pnumwidth{\hss#2}
            }
            \par
            \nobreak
            \global\@nobreaktrue
            \everypar{\global\@nobreakfalse\everypar{}}%
        \endgroup
    \fi}
\newcommand\bookname{Book}
\renewcommand{\thebook}{\texorpdfstring{\Numberstring{book}}{book}}
\providecommand*{\toclevel@book}{-2}
\makeatother
\titleformat{\part}[display]
    {\Large\bfseries}
    {\partname\nobreakspace\thepart}
    {0mm}
    {\Huge\bfseries}
\titlecontents{part}[0pt]
    {\large\bfseries}
    {\partname\ \thecontentslabel: \quad}
    {}
    {\hfill\contentspage}
\titlecontents{chapter}[0pt]
    {\bfseries}
    {\chaptername\ \thecontentslabel:\quad}
    {}
    {\hfill\contentspage}
\newglossarystyle{longpara}{%
    \setglossarystyle{long}%
    \renewenvironment{theglossary}{%
        \begin{longtable}[l]{{p{0.25\hsize}p{0.65\hsize}}}
    }{\end{longtable}}%
    \renewcommand{\glossentry}[2]{%
        \glstarget{##1}{\glossentryname{##1}}%
        &\glossentrydesc{##1}{~##2.}
        \tabularnewline%
        \tabularnewline
    }%
}
\newglossary[not-glg]{notation}{not-gls}{not-glo}{Notation}
\newcommand*{\newnotation}[4][]{%
    \newglossaryentry{#2}{type=notation, name={\textbf{#3}, },
                          text={#4}, description={#4},#1}%
}
%--------------------------LENGTHS------------------------------%
% Spacings for the Table of Contents.
\addtolength{\cftsecnumwidth}{1ex}
\addtolength{\cftsubsecindent}{1ex}
\addtolength{\cftsubsecnumwidth}{1ex}
\addtolength{\cftfignumwidth}{1ex}
\addtolength{\cfttabnumwidth}{1ex}

% Indent and paragraph spacing.
\setlength{\parindent}{0em}
\setlength{\parskip}{0em}
%----------------------------GLOSSARY-------------------------------%
\makeglossaries
\loadglsentries{../../glossary}
\loadglsentries{../../acronym}
%--------------------------Main Document----------------------------%
\begin{document}
    \ifx\ifmathcourses\undefined
        \pagenumbering{roman}
        \title{Measure Theory}
        \author{Ryan Maguire}
        \date{\vspace{-5ex}}
        \maketitle
        \tableofcontents
        \clearpage
        \chapter*{Measure Theory}
        \addcontentsline{toc}{chapter}{Measure Theory}
        \markboth{}{MEASURE THEORY}
        \vspace{10ex}
        \setcounter{chapter}{1}
        \pagenumbering{arabic}
    \else
        \chapter{Measure Theory}
    \fi
    \section{A Review of Set Theory}
        \subsection{Cardinality}
            We begin by talking about cardinality. This is the
            \textit{size} of a set. For an infinite set it
            doesn't make sense to talk about the \textit{number}
            of elements, but we can specify what it means two sets
            to have the same size. Sets $A$ and $B$ are equivalent
            if there exists a bijection $f:A\rightarrow{B}$.
            We then say that $A$ and $B$ have the same cardinality.
            The notation is written as $|A|=|B|$ or
            $\Card(A)=\Card(B)$. A finite set is a set $A$ such that
            there is a bijection between $A$ and $\mathbb{Z}_{n}$.
            We can then view the elements of $A$ as
            $A=\{a_{1},\hdots,a_{n}\}$. A countable set is a set
            $A$ such that there is a bijection between $A$ and
            $\mathbb{N}$. Here, $\mathbb{N}$ is the set of all
            natural numbers, or positive integers.
            \begin{lexample}
                There are many commonly discussed sets that are
                countably infinite. $\mathbb{N}$ is a trivial
                such example, but also $\mathbb{N}_{e}$ and
                $\mathbb{N}_{o}$, the sets of even and odd positive
                integers, respectively. For consider as bijections
                the following functions:
                \par
                \begin{subequations}
                    \begin{minipage}[b]{0.49\textwidth}
                        \centering
                        \begin{equation}
                            f_{e}(n)=2n
                        \end{equation}
                    \end{minipage}
                    \hfill
                    \begin{minipage}[b]{0.49\textwidth}
                        \centering
                        \begin{equation}
                            f_{0}(n)=2n-1
                        \end{equation}
                    \end{minipage}
                    \par\hfill\par
                    $\mathbb{Z}$ is also countable, as shown in
                    Fig.~\ref{fig:MEASURE_THEORY:BIJECTION_N_AND_Z}.
                    An explicit bijection for $\mathbb{Z}$ is:
                    \begin{equation}
                        f(n)=
                        \begin{cases}
                            \frac{n}{2},&n\mod{2}=0\\
                            \frac{1-n}{2},&n\mod{2}=1
                        \end{cases}
                    \end{equation}
                \end{subequations}
            \end{lexample}
            $\mathbb{Q}$ is also countable. We may intuitively think of
            $\mathbb{N}$ as being smaller than $\mathbb{Q}$, since there
            are simple \textit{surjections} that can be constructed
            from $\mathbb{Q}$ to $\mathbb{N}$. There is also a
            surjection from $\mathbb{N}$ onto $\mathbb{Q}^{+}$, as is
            shown in
            Fig.~\ref{fig:MEASURE_THEORY:BIJECTION_N_AND_Q_Plus}.
            \newpage
            \begin{figure}[H]
                \centering
                \captionsetup{type=figure}
                \subimport{../../tikz/}{Surjection_From_N_to_Z}
                \caption{Diagram of a Bijection Between
                         $\mathbb{N}$ and $\mathbb{Z}$.}
                \label{fig:MEASURE_THEORY:BIJECTION_N_AND_Z}
            \end{figure}
            To construct such a surjection, write out all of the
            positive rational numbers in a grid so that $a_{nm}$ is
            the number $n/m$. Then zig-zag along the diagonals
            to construct the function. Thus there is a surjection
            $f:\mathbb{Q}^{+}\rightarrow\mathbb{N}$
            and a surjection $g:\mathbb{N}\rightarrow\mathbb{Q}^{+}$. The
            Cantor-Schr\"{o}eder-Bernstein theorem says that if there
            is surjection from $A$ to $B$ and a surjection from $B$
            to $A$, then there is a bijection between $A$ and $B$.
            Therefore there is a bijection between $\mathbb{N}$ and
            $\mathbb{Q}^{+}$, and $\mathbb{Q}^{+}$ is countable.
            \begin{figure}[H]
                \centering
                \captionsetup{type=figure}
                \resizebox{0.7\textwidth}{!}{%
                    \subimport{../../tikz/}
                              {Surjection_From_N_to_Q_Plus.tex}
                }
                \caption{Diagram of a Surjection from
                         $\mathbb{N}$ onto $\mathbb{Q}^{+}$.}
                \label{fig:MEASURE_THEORY:BIJECTION_N_AND_Q_Plus}
            \end{figure}
            We can modify
            Fig.~\ref{fig:MEASURE_THEORY:BIJECTION_N_AND_Q_Plus}
            slightly to create a surjection between $\mathbb{N}$ and
            $\mathbb{Q}$, see
            Fig.~\ref{fig:MEASURE_THEORY:BIJECTION_N_AND_Q}.
            It is important to note that this bijection will not
            preserve the order of the rational numbers. The bijection
            will have to jump around back and forth. Any such
            bijection will be forced to do this, as the rationals are
            everywhere dense on $\mathbb{R}$. Any monotonic sequence of
            $\mathbb{Q}$ cannot possibly be a bijection.
            \begin{figure}[H]
                \centering
                \captionsetup{type=figure}
                \resizebox{\textwidth}{!}{%
                    \subimport{../../tikz/}
                              {Surjection_From_N_to_Q.tex}
                }
                \caption{Diagram of a Surjection from
                         $\mathbb{N}$ onto $\mathbb{Q}$.}
                \label{fig:MEASURE_THEORY:BIJECTION_N_AND_Q}
            \end{figure}
            \begin{theorem}
                If $A$ is a countable infinite set and
                $B\subseteq{A}$, then either $B$ is finite or
                $B$ is countable.
            \end{theorem}
            \begin{proof}
                As $A$ is countable, there is a bijection
                $a:\mathbb{N}\rightarrow{A}$. Define the following:
                \begin{equation}
                    K=\{n\in\mathbb{N}:a_{n}\in{B}\}
                \end{equation}
                As $B\subseteq{A}$,
                this set contains a subsequence of points in
                $\mathbb{N}$ that get mapped into $B$. If $K$ is finite,
                then $B$ is finite, and if not then $K$ is countably
                infinite, and thus $B$ is countably infinite.
            \end{proof}
            \begin{theorem}
                If $A$ is an infinite set, then there exists a
                countable subset $B\subseteq{A}$.
            \end{theorem}
            \begin{proof}
                If $A$ is infinite then there is an
                $a_{1}\in{A}$. But, as $A$ is infinite,
                $A\setminus\{a_{1}\}$ is infinite, and there
                is an $a_{2}\in{A}\setminus\{a_{1}\}$. Continuing
                we obtain a sequence of distinct elements in $A$.
                Let $B=\{a_{n}:n\in\mathbb{N}\}$. Then
                $B\subseteq{A}$, and $B$ is countable.
            \end{proof}
            \begin{lexample}
                Suppose we have a collection of disjoint intervals
                of $\mathbb{R}$. This collection is either finite
                or countable. For in every interval, choose a
                rational number $q_{n}$. Let
                $A=\{q_{1},q_{2},\hdots\}$. Then
                $A\subseteq\mathbb{Q}$, and thus $A$ is either
                finite or countable. But this is also an enumeration
                of the intervals in the collection, and thus the
                collection is either finite or countable.
            \end{lexample}
            Given a countable collection of sets
            $A=\{\mathcal{A}_{1},\mathcal{A}_{2},\hdots\}$ such
            that, for all $n\in\mathbb{N}$, $\mathcal{A}_{n}$ is
            also a countable set, then the union is countable. That is:
            \begin{equation}
                B=\bigcup_{n=1}^{\infty}\mathcal{A}_{n}
            \end{equation}
            is a countable set. The proof of this is a mimicry of
            the proof of the countability of $\mathbb{Q}$. Not
            every set is either finite or countable. The real numbers,
            $\mathbb{R}$, is an example of an \textit{uncountable}
            set. First, some notes on the power set of a set.
            \begin{ldefinition}{Power Set}
                The power set of a set $\Omega$, denoted
                $\mathcal{P}(\Omega)$, is the set of all subsets of
                $\Omega$:
                \begin{equation}
                    \mathcal{P}(\Omega)=
                    \{A:A\subseteq\Omega\}
                \end{equation}
            \end{ldefinition}
            \begin{lexample}
                \begin{subequations}
                    If $\Omega=\{1,2\}$, then the power set is:
                    \begin{equation}
                        \mathcal{P}(\Omega)=
                        \big\{\emptyset,\{1\},\{2\},\{1,2\}\big\}
                    \end{equation}
                    We must consider the empty set, since for any set
                    $A$, $\emptyset\subseteq{A}$.
                    If $\Omega=\{1,2,3\}$, then:
                    \begin{equation}
                        \mathcal{P}(\Omega)=
                        \big\{\emptyset,\{1\},\{2\},\{3\},\{1,2\},
                          \{1,3\},\{2,3\},\{1,2,3\}\big\}
                    \end{equation}
                    We see that, in the first example, a set with
                    2 elements has a power set with 4 elements. In the
                    second example we see that a set with 3 elements has
                    a power set with 8 elements. This pattern continues
                    for finite sets. If $A$ has $n$ elements, then
                    $\mathcal{P}(A)$ has $2^{n}$ elements. If
                    $A$ is an infinite set, then $\mathcal{P}(A)$ is
                    uncountable. Indeed:
                    \begin{equation}
                        \Card\big(\mathcal{P}(\mathbb{N})\big)=
                        \Card(\mathbb{R})
                    \end{equation}
                    We can show this by using the binary representation
                    of real numbers. We construct a bijection as
                    follows: If $A\subseteq\mathbb{N}$, then
                    let $r_{A}=0.n_{1}n_{2}\hdots$ where:
                    \begin{equation}
                        n_{i}=
                        \begin{cases}
                            0,&i\notin{A}\\
                            1,&i\in{A}
                        \end{cases}
                    \end{equation}
                    The function
                    $f:\mathcal{P}(\mathbb{N})\rightarrow[0,1]$
                    defined by $f(A)=r_{A}$ is thus a bijection.
                    That is, every element of $[0,1]$ gets mapped to in
                    a one-to-one manner. The potentially tricky numbers are
                    0 and 1, but $f(\emptyset)=0$, and $f(\mathbb{N})=1$.
                    Thus $\mathcal{P}(\mathbb{N})$ and $[0,1]$ are of the
                    same cardinality. But $(0,1)$ and $\mathbb{R}$
                    are of the same cardinality. To see this, consider
                    the graph of the function
                    $g:(0,1)\rightarrow\mathbb{R}$ defined as:
                    \begin{equation}
                        g(x)=\frac{2x-1}{x(1-x)}
                    \end{equation}
                    This is a bijection between the unit interval
                    $(0,1)$ and $\mathbb{R}$. One can also use the
                    \textit{stereographic projection} to show this.
                    But also $[0,1]$ and $(0,1)$ have the same cardinality.
                    For this, consider the following function:
                    \begin{equation}
                        f(x)=
                        \begin{cases}
                            \frac{1}{2},&x=0\\
                            \frac{1}{2^{n+2}},&x=\frac{1}{2^{n}}\\
                            x,&\textrm{Otherwise}
                        \end{cases}
                    \end{equation}
                    A graph of this is shown in
                    Fig.~\ref{fig:MEASURE_THEORY:Bijection_Closed_Interval_to_Open}.
                    Therefore, $\mathbb{R}$ and
                    $\mathcal{P}(\mathbb{N})$ have the same cardinality.
                    This can then be used to show that $\mathbb{R}$ is
                    uncountable.
                \end{subequations}
            \end{lexample}
            \begin{figure}[H]
                \centering
                \captionsetup{type=figure}
                \subimport{../../tikz/}{Bijction_Closed_to_Open_Interval}
                \caption{Bijection from $[0,1]$ to $(0,1)$.}
                \label{fig:MEASURE_THEORY:Bijection_Closed_Interval_to_Open}
            \end{figure}
            The power set of any set is strictly larger than the
            original set. If $\Omega$ is finite with $n$ elements, it
            can be shown that $\mathcal{P}(\Omega)$ has $2^{n}$
            elements. For infinite sets, there is a trivial surjection
            from $\mathcal{P}(\Omega)$ onto $\Omega$: for any element
            $x$, let $f(\{x\})=x$. This shows that
            $\Card(\Omega)\leq\Card(\mathcal{P}(\Omega))$. We now show
            that the inequality is strict.
            \begin{theorem}
                If $\Omega$ is a set, then there is no bijection
                $f:\Omega\rightarrow\mathcal{P}(\Omega)$
            \end{theorem}
            \begin{proof}
                For suppose not, and let
                $f:\Omega\rightarrow\mathcal{P}(\Omega)$ be such a
                bijection. Define:
                \begin{equation}
                    A=\{x\in\Omega:x\in{f}(x)\}
                \end{equation}
                Then $A\subseteq\Omega$, and thus
                $A\in\mathcal{P}(\Omega)$. But then the complement of
                $A$ is also an element of $\mathcal{P}(\Omega)$. But
                $f$ is a bijection and thus there is an $x\in\Omega$
                such that $f(x)=A^{C}$. If $x\in{f}(x)$, then
                $x\in{A}$, a contradiction as $f(x)=A^{C}$, and thus
                $x\in{A}^{C}$ as well. Therefore $x\notin{f}(x)$. But
                then $x\in{A}^{C}$. But, from the definition of $A$,
                since $x\in{A}^{C}$ and $f(x)=A^{C}$, $x\in{f}(x)$
                and thus $x\in{A}$, a contradiction. Thus there is no
                $x$ such that $f(x)=A^{C}$. Therefore, $f$ is not a
                bijection.
            \end{proof}
            From this we conclude that $\mathcal{P}(\mathbb{N})$
            is an uncountable infinite set. But since $\mathbb{R}$
            and $\mathcal{P}(\mathbb{N})$ have the same cardinality,
            $\mathbb{R}$ is also uncountable.
            If a set $A$ has the same cardinality as $\mathbb{R}$,
            we say that $A$ has the cardinality of the continuum.
            \begin{lexample}
                There is a bijection between the open unit
                square $(0,1)\times(0,1)$ and the open unit interval
                $(0,1)$. For an element $(x,y)\in(0,1)\times(0,1)$,
                let $z\in(0,1)$ be defined as
                $z=0.x_{1}y_{1}x_{2}y_{2}x_{3}y_{3}\dots$ This is
                a bijection, for all $(x,y)$ in the square there is
                a corresponding $z\in(0,1)$, and for all
                $z\in(0,1)$ there is a corresponding element of
                $(0,1)\times(0,1)$. We can say that $(x,y)$ can
                be coded into $z$, and $z$ can be decoded into
                $(x,y)$. Hence, $(0,1)\times(0,1)$ has the cardinality
                of the continuum. By stereographic projection and induction
                we obtain:
                \par\hfill\par
                \begin{subequations}
                    \begin{minipage}[b]{0.49\textwidth}
                        \begin{equation}
                            \Card(\mathbb{R}^{2})=\Card(\mathbb{R})
                        \end{equation}
                    \end{minipage}
                    \hfill
                    \begin{minipage}[b]{0.49\textwidth}
                        \begin{equation}
                            \Card(\mathbb{R}^{n})=\Card(\mathbb{R})
                        \end{equation}
                    \end{minipage}
                    \par
                \end{subequations}
            \end{lexample}
            \begin{lexample}
                Consider the set of all real-valued sequences. We've seen
                that any real number can be represented as a function
                $f:\mathbb{N}\rightarrow\{0,1\}$. A real-valued sequence
                is a function $a:\mathbb{N}\rightarrow\mathbb{R}$, and
                thus the set of real-valued sequences can be seen as the
                set of functions whose domain is $\mathbb{N}$ and whose
                range is the set of all functions
                $f:\mathbb{N}\rightarrow\{0,1\}$. So given a sequence
                $a$, the image of $a_{n}$, for $n\in\mathbb{N}$, is a
                function $f_{n}:\mathbb{N}\rightarrow\{0,1\}$. Therefore
                the set of all real-valued sequences can be represented
                as the set of all functions
                $g:\mathbb{N}\times\mathbb{N}\rightarrow\{0,1\}$, where
                $g(n,m)=f_{n}(m)$. But $\mathbb{N}\times\mathbb{N}$ is
                countable, and thus the set of all functions of the form
                $g:\mathbb{N}\times\mathbb{N}\rightarrow\{0,1\}$ has the
                same cardinality as the set of all functions of the form
                $f:\mathbb{N}\rightarrow\{0,1\}$. But this has the
                cardinality of the continuum. Therefore, the set of all
                real-valued sequences has the cardinality of the continuum.
            \end{lexample}
        \subsection{Set Operations}
            There are several operations that can be performed on sets,
            most notable are union, intersection, difference, and symmetric
            difference.
            \begin{ldefinition}{Set Union and Intersection}
                \begin{subequations}
                    The union of two sets $A$ and $B$, denoted
                    $A\cup{B}$, is the set:
                    \begin{equation}
                        A\cup{B}=\{x:x\in{A}\textrm{ or }x\in{B}\}
                    \end{equation}
                    The intersction of two sets $A$ and $B$, denoted
                    $A\cap{B}$, is the set:
                    \begin{equation}
                        A\cap{B}=\{x:x\in{A}\textrm{ and }x\in{B}\}
                    \end{equation}
                \end{subequations}
            \end{ldefinition}
            These two concepts can be visualized with Venn diagrams, as shown in
            Fig.~\ref{fig:MEASURE_THEORY_union_intersection_venn_diagram}.
            \begin{figure}[H]
                \centering
                \captionsetup{type=figure}
                \begin{subfigure}[b]{0.49\textwidth}
                    \centering
                    \subimport{../../tikz/}{Venn_Diagram_Union}
                    \subcaption{Set Union}
                \end{subfigure}
                \begin{subfigure}[b]{0.49\textwidth}
                    \centering
                    \subimport{../../tikz/}{Venn_Diagram_Intersection}
                    \subcaption{Set Intersection}
                \end{subfigure}
                \caption{Venn Diagrams Depicting the Union and Intersection
                         of the Sets $A$ and $B$.}
                \label{fig:MEASURE_THEORY_union_intersection_venn_diagram}
            \end{figure}
            \begin{ftheorem}{Laws of Union and Intersection}
                            {MEASURE_THEORY_UNION_LAWS_UNION_INTERSECTION}
                If $A$, $B$, and $C$ are sets, then the following are true:
                \par\hfill\par
                \begin{subequations}
                    \underline{Commutative Laws:}
                    \begin{align}
                        A\cup{B}&=B\cup{A}\\
                        A\cap{B}&=B\cap{A}
                    \end{align}
                    \underline{Associative Laws:}
                    \begin{align}
                        A\cup(B\cup{C})&=(A\cup{B})\cup{C}\\
                        A\cap(B\cap{B})&=(A\cap{B})\cap{C}
                    \end{align}
                    \underline{Distributive Laws:}
                    \begin{align}
                        A\cup(B\cap{C})&=(A\cup{B})\cap(A\cup{C})\\
                        A\cap(B\cup{C})&=(A\cap{B})\cup(A\cap{C})
                    \end{align}
                \end{subequations}
            \end{ftheorem}
            \begin{bproof}
                If $x\in{A}\cup{B}$, then either $x\in{A}$ or $x\in{B}$,
                or both. But then either $x\in{B}$ or $x\in{A}$, or both,
                and thus $x\in{B}\cup{A}$. Thus $A\cup{B}\subseteq{B}\cup{A}$.
                Similarly, $B\cup{A}\subseteq{A}\cup{B}$, and thus
                $A\cup{B}=B\cup{A}$. By a similar argument,
                $A\cap{B}=B\cap{A}$.
                \par\hfill\par
                If $x\in{A}\cup(B\cup{C})$, then either
                $x\in{A}$ or $x\in{B}\cup{C}$, or both.
                If $x\in{A}$, then $x\in{A}\cup{B}$, and
                therefore $x\in(A\cup{B})\cup{C}$. If not,
                then $x\in{B}\cup{C}$, and thus either
                $x\in{B}$ or $x\in{C}$, or both.
                If $x\in{B}$, then $x\in{A}\cup{B}$, and
                thus $x\in(A\cup{B})\cup{C}$. If not
                then $x\in{C}$, and therefore
                $x\in(A\cup{B})\cup{C}$. Therefore
                $A\cup(B\cup{C})\subseteq(A\cup{B})\cup{C}$.
                Similarly, $(A\cup{B})\cup{C}\subseteq{A}\cup(B\cup{C})$.
                Thus, $A\cup(B\cup{C})=(A\cup{B})\cup{C}$. By a similar
                argument, $A\cap(B\cap{C})=(A\cap{B})\cap{C}$.
                \par\hfill\par
                If $x\in{A}\cup(B\cap{C})$, then either $x\in{A}$
                or $x\in{B}\cap{C}$, or both. If $x\in{A}$, then
                $x\in{A}\cup{B}$ and $x\in{A}\cup{C}$, and therefore
                $x\in(A\cup{B})\cap(A\cup{C})$. If $x\in{B}\cap{C}$,
                then $x\in{B}$ and $x\in{C}$, and therefore
                $x\in(A\cup{B})\cap(A\cup{C})$. Thus,
                $A\cup(B\cap{C})\subseteq(A\cup{B})\cap(A\cup{C})$.
                Similarly,
                $(A\cup{B})\cap(A\cup{C})\subseteq{A}\cup(B\cap{C})$.
            \end{bproof}
            If $A$ and $B$ are sets, and if $C\subseteq{A}\cup{B}$, then
            either $C\subseteq{A}$ or $C\subseteq{B}$, or both. It is
            possible that $C\subseteq{A}\cup{B}$ and yet $C$ and $B$ have no
            elements in common, as long as $C\subseteq{A}$. As an example,
            take $A$ and $B$ to be disjoint sets. Then $A\subseteq{A}\cup{B}$,
            yet $A$ and $B$ have no elements in common. If
            $C\subseteq{A}\cap{B}$, then it must be true that
            $C\subseteq{A}$ and $C\subseteq{B}$.
            \begin{ldefinition}{Set Difference and Symmetric Difference}
                The set difference of a set $A$ with respect to a set $B$,
                denoted $B\setminus{A}$, is the set:
                \begin{equation}
                    B\setminus{A}=\{x\in{B}:x\notin{A}\}
                \end{equation}
                The symmetric difference of $A$ and $B$, denoted
                $A\ominus{B}$, is the set:
                \begin{equation}
                    A\ominus{B}=(A\cup{B})\setminus(A\cap{B})
                \end{equation}
            \end{ldefinition}
            As with the notions of unions and intersections, set differences and
            symmetric differences can be visualized using Venn diagrams.
            \begin{figure}[H]
                \centering
                \captionsetup{type=figure}
                \begin{subfigure}[b]{0.49\textwidth}
                    \centering
                    \subimport{../../tikz/}{Venn_Diagram_Set_Difference}
                    \subcaption{Set Difference}
                \end{subfigure}
                \begin{subfigure}[b]{0.49\textwidth}
                    \centering
                    \subimport{../../tikz/}{Venn_Diagram_Symmetric_Difference}
                    \subcaption{Symmetric Difference}
                \end{subfigure}
                \caption{Venn Diagrams Depicting the Set Difference and
                         Symmetric Difference of the Sets $A$ and $B$.}
                \label{fig:MEASURE_THEORY_Difference_sym_venn_diagram}
            \end{figure}
            \begin{theorem}
                \label{thm:MEASURE_THEORY_SET_DIFFERENCE_AS_INTERSECTION}
                If $A$, $B$, and $C$ are sets, and if $A\subseteq{C}$
                and $B\subseteq{C}$, then:
                \begin{equation}
                    B\setminus{A}=B\cap(C\setminus{A})
                \end{equation}
            \end{theorem}
            \begin{proof}
                For if $x\in{B}\setminus{A}$, then
                $x\in{B}$ and $x\notin{A}$. But
                $B\subseteq{C}$, and thus if $x\in{B}$, then $x\in{C}$.
                But if $x\notin{A}$, then $x\in{C}\setminus{A}$. Therefore
                $B\setminus{A}\subseteq{B}\cap(C\setminus{A})$.
                Similarly, $B\cap(C\setminus{A})\subseteq{B}\setminus{A}$,
                and therefore $B\setminus{A}={B}\cap(C\setminus{A})$.
            \end{proof}
            While set difference appears similar to subtraction that one finds in
            basic arithmetic, the two have their differences. For any two real
            numbers $a$ and $b$, $b=a-(a-b)$. For sets this is not true. For let
            $A$ be the empty set, and let $B$ be non-empty. Then
            $A\setminus(A\setminus{B})=\emptyset$, which is not $B$.
            Also, while it may seems convincing that
            $A\setminus(B\setminus{A})=A\setminus{B}$, this is not true. For
            let $A$ be a non-empty set and let $B=A$. Then
            $A\setminus(B\setminus{A})=A$, but $A\setminus{B}=\emptyset$.
            The concept of set difference can then be used to define the
            concept of complement.
            \begin{ldefinition}{Complement}
                The complement of a set $A$ with respect to a set
                $\Omega$, denoted $A^{C}$, is the set:
                \begin{equation}
                    A^{C}=\Omega\setminus{A}
                \end{equation}
            \end{ldefinition}
            Thm.~\ref{thm:MEASURE_THEORY_SET_DIFFERENCE_AS_INTERSECTION} can then
            be translated into the notation of complements as follows:
            \begin{theorem}
                If $A$, $B$, and $\Omega$ are sets, $A,B\subseteq\Omega$,
                and if $A^{C}$ is the complement of $A$ with respect
                to $\Omega$, then:
                \begin{equation}
                    B\setminus{A}=B\cap{A}^{C}
                \end{equation}
            \end{theorem}
            \begin{proof}
                By the definition of complement, $A^{C}=\Omega\setminus{A}$.
                As $A\subseteq\Omega$ and $B\subseteq\Omega$, by
                Thm.~\ref{thm:MEASURE_THEORY_SET_DIFFERENCE_AS_INTERSECTION},
                $B\setminus{A}=B\cap(\Omega\setminus{A})$, and therefore
                $B\setminus{A}=B\cap{A}^{C}$.
            \end{proof}
            \begin{ftheorem}{DeMorgan's Laws}{MEASURE_DEMORGAN}
                If $A$ and $B$ are sets, then:
                \begin{subequations}
                    \begin{align}
                        \big(A\cap{B}\big)^{C}
                        &=A^{C}\cup{B}^{C}\\
                        \big(A\cup{B}\big)^{C}
                        &=A^{C}\cap{B}^{C}
                    \end{align}
                \end{subequations}
            \end{ftheorem}
            With this, we can prove some results about set differences.
            \begin{theorem}
                If $A$ and $B$ are sets, then:
                \begin{equation}
                    A=\big(A\cap{B}\big)\cup\big(A\setminus{B}\big)
                \end{equation}
            \end{theorem}
            \begin{proof}
                For:
                \begin{subequations}
                    \begin{align}
                        \big(A\cap{B})\cup\big(A\setminus{B}\big)
                        &=\big(A\cap{B}\big)\cup\big(A\cap{B}^{C}\big)\\
                        &=A\cap(B\cup{B}^{C})\\
                        &=A\cap\Omega\\
                        &=A
                    \end{align}
                \end{subequations}
            \end{proof}
            \begin{theorem}
                If $A$, $B$, and $C$ are sets, then:
                \begin{equation}
                    A\cap\big(B\setminus{C}\big)
                    =\big(A\cap{B}\big)\cap\big(A\setminus{C}\big)
                \end{equation}
            \end{theorem}
            \begin{proof}
                For:
                \begin{subequations}
                    \begin{align}
                        A\cap\big(B\setminus{C}\big)
                        &=A\cap\big(B\cap{C}^{C}\big)\\
                        &=\big(A\cap{A}\big)\cap\big(B\cap{C}^{C}\big)\\
                        &=\big(A\cap{B}\big)\cap\big(A\cap{C}^{C}\big)\\
                        &=\big(A\cap{B}\big)\cap\big(A\setminus{C}\big)
                    \end{align}
                \end{subequations}
            \end{proof}
            Intersections do distribute over set differences.
            \begin{theorem}
                If $A$, $B$, and $C$ are sets, then:
                \begin{equation}
                    A\cap(B\setminus{C})=
                    (A\cap{B})\setminus(A\cap{C})
                \end{equation}
            \end{theorem}
            \begin{proof}
                For:
                \begin{subequations}
                    \begin{align}
                        \big(A\cap{B}\big)\setminus\big(A\cap{C}\big)
                        &=\big(A\cap{B}\big)\cap\big(A\cap{C}\big)^{C}\\
                        &=\big(A\cap{B}\big)\cap\big(A^{C}\cup{C}^{C}\big)\\
                        &=\Big(\big(A\cap{B}\big)\cap{A}^{C}\Big)
                            \cup\Big(\big({A}\cap{B}\big)\cap{C}^{C}\Big)\\
                        &=\Big(\big(A\cap{A}^{C}\big)\cap{B}\Big)\cup
                            \Big(\big(A\cap{B}\big)\cap{C}^{C}\Big)\\
                        &=\emptyset\cup
                            \Big(\big(A\cap{B}\big)\cap{C}^{C}\Big)\\
                        &=\big(A\cap{B}\big)\cap{C}^{C}\\
                        &=A\cap\big(B\cap{C}^{C}\big)\\
                        &=A\cap\big(B\setminus{C}\big)
                    \end{align}
                \end{subequations}
            \end{proof}
            Unions do not, however. For let $A$ be non-empty and let
            $A=B=C$. Then $A\cup(B\setminus{C})=A$, but
            $(A\cup{B})\setminus(A\cup{C})=\emptyset$.
            \begin{enumerate}
                \item DeMorgan's Laws:
                      \subitem $(A\cup{B})^{C}=A^{C}\cap{B}^{C}$
                      \subitem $(A\cap{B})^{C}=A^{C}\cup{B}^{C}$
            \end{enumerate}
            DeMorgan's Laws hold for arbitrary collections
            of set. If $I$ is some indexing set:
            \begin{align}
                \Big(\bigcup_{\alpha\in{I}}A_{\alpha}\Big)^{C}
                &=\bigcap_{\alpha\in{I}}A_{\alpha}^{C}\\
                \Big(\bigcap_{\alpha\in{I}}A_{\alpha}\Big)^{C}
                &=\bigcup_{\alpha\in{I}}A_{\alpha}^{C}
            \end{align}
            The set operations thus define binary operations
            on the power set of a set $\Omega$. It's important
            to note the notation. An element of $\Omega$ may
            be anything, while an element of
            $\mathcal{P}(\Omega)$ is a subset of $\Omega$.
            That is, the \textit{points} of $\mathcal{P}(\Omega)$
            are themselves sets. Thus, union, intersection,
            etc., define binary operations on
            $\mathcal{P}(\Omega)$. Given two subsets of
            $\Omega$, $A$ and $B$, $A\cup{B}$ is another
            subset of $\Omega$, as is $A\cap{B}$, and so on.
            The complement can also be seen as a unary operator
            on $\mathcal{P}(\Omega)$. From the various theorems
            presented, we have the following:
            \begin{enumerate}
                \item Union is commutative and associative.
                \item Intersection is commutative and
                      associative.
                \item Union distributes over intersection.
                \item Intersection distributes over union.
                \item DeMorgan's Laws hold.
                \item Set difference is not commutative,
                      nor is it associative.
            \end{enumerate}
    \section{\texorpdfstring{$\sigma$}{Sigma}-Algebras}
        \subsection{Set Rings}
            Given a set $\Omega$, $\mathcal{P}(\Omega)$ is the
            set of all subsets of $\Omega$. Often this is too
            much, and too difficult to handle. Indeed, even
            $\mathcal{P}(\mathbb{R})$ is quite large and hard
            to get a grasp on. We wish to speak of collections
            of sets that have some structure on them.
            The first thing we will talk about is a set ring.
            \begin{ldefinition}{Set Ring}
                A set ring of a set $\Omega$ is a nonempty subset
                $\mathcal{R}\subseteq\mathcal{P}(\Omega)$ such that:
                \begin{enumerate}
                    \item For all $A,B\in\mathcal{R}$,
                          $A\cup{B}\in\mathcal{R}$
                    \item For all $A,B\in\mathcal{R}$,
                          $A\setminus{B}\in\mathcal{R}$
                \end{enumerate}
            \end{ldefinition}
            \begin{example}{Set Ring}
                If $\Omega$ is a set, then
                $\mathcal{P}(\Omega)$ is a set ring of
                $\Omega$. So is the set $R=\{\emptyset$.
                For any $A\subset\Omega$, the set
                $R=\{A\}$ is also a set ring. If
                $\Omega=\{1,2,3\}$, then
                $R=\{\emptyset,\{1\},\{2,3\},\{1,2,3\}\}$ is
                a set ring on $\Omega$.
            \end{example}
            \begin{lexample}
                If $\Omega$ is an infinite set, and if
                $\mathcal{E}=\big\{\{x\}:x\in\Omega\big\}$, then the
                smallest set ring that contains $\mathcal{E}$ is the set of
                all finite subsets of $\Omega$. For the union of two finite
                sets is finite, as is the set difference of two finite sets,
                and thus this satisfies a set ring. Moreover, if $\mathcal{R}$
                is a set ring that contains $\mathcal{E}$ then it contains the
                union of any finite collection of elements in $\mathcal{E}$.
                But $\mathcal{E}$ is the set of all of the singletons of
                $\Omega$, and any finite subset of $\Omega$ can be written
                as the union of finitely many singletons. Thus, $\mathcal{R}$
                is the smallest set ring that contains $\mathcal{E}$.
            \end{lexample}
            \begin{theorem}
                If $\Omega$ is a set, if $R$ is a set ring
                on $\Omega$, and if $A$ is a finite subset of
                $R$, then $\cup_{\alpha\in{A}}\alpha$ is an
                element of $R$.
            \end{theorem}
            \begin{proof}
                Apply induction to the closure of unions.
            \end{proof}
            \begin{theorem}
                If $\Omega$ is a set, if $R$ is a set ring on
                $\Omega$, and if $A,B\in{R}$, then
                $A\cup{B}\in{R}$.
            \end{theorem}
            \begin{proof}
                For $A\cap{B}=A\setminus(A\setminus{B})$, and
                from the closure of set difference,
                $A\cap{B}\in{R}$.
            \end{proof}
            \begin{theorem}
                If $\Omega$ is a set, if $R$ is a set ring
                on $\Omega$, and if $A$ is a finite subset of
                $R$, then $\cap_{\alpha\in{A}}\alpha$ is an
                element of $R$.
            \end{theorem}
            \begin{proof}
                Apply induction to the closure of intersections.
            \end{proof}
            \begin{theorem}
                If $\Omega$ is a set, if $R$ is a set ring on
                $\Omega$, if $A,B\subset\Omega$, and if
                $A\setminus{B}$, $B\setminus{A}$, and
                $A\cap{B}$ are elements of $R$, then
                $A,B\in{R}$.
            \end{theorem}
            Thus, the set ring generated by the set $\{A,B\}$ and
            the set ring generated by
            $\{A\setminus{B},B\setminus{A},A\cap{B}\}$ are the
            same.
            \begin{theorem}
                If $\Omega$ is a set and $R$ is a set ring
                of $\Omega$, then $\emptyset\in{R}$.
            \end{theorem}
            \begin{proof}
                For as $R$ is non-empty, there is an element
                $A\in{R}$. If $A=\emptyset$, then we are done.
                If not, as $R$ is closed under set difference,
                $A\setminus{A}\in{R}$. But
                $A\setminus{A}=\emptyset$.
            \end{proof}
            From this, if we have a collection $R$ of subsets of
            $\Omega$ and we wish to check if $R$ is a set ring
            of $\Omega$, there are several redundant operations
            we don't need to check. Since, for any set $A$,
            we have:
            \begin{align}
                A\setminus\emptyset&=A\\
                A\setminus{A}&=\emptyset\\
                \emptyset\setminus{A}&=\emptyset\\
                A\cup{A}&=A\\
                A\cup\emptyset&=A\\
                \emptyset\cup\emptyset&=\emptyset
            \end{align}
            Using our previous example $\Omega=\{1,2,3\}$,
            we can check laboriously that
            $R=\{\emptyset,\{1\},\{2,3\},\{1,2,3\}\}$ is a
            set ring on $\Omega$. The set
            $\{\emptyset,\{1\},\{2\},\{1,2,3\}\}$ is not
            a set ring, for $\{1,2\}=\{1\}\cup\{2\}$ is not
            an element.
            \begin{theorem}
                If $\Omega$ is a set, and if $A$ and $B$ are
                disjoint subsets of $\Omega$, then
                $R=\{\emptyset,A,B,A\cup{B}\}$ is a set ring
                on $\Omega$.
            \end{theorem}
            \begin{theorem}
                If $\Omega$ is a set, if $A$ and $B$ are
                disjoint subsets of $\Omega$, and if
                $R$ is a set ring such that $A,B\in{R}$,
                then $\{emptyset,A,B,A\cup{B}\}\subset{R}$.
            \end{theorem}
            As such, the set ring $\{\emptyset,A,B,A\cup{B}\}$
            is called the set ring generated by $A$ and $B$. We
            can continue and consider the case of three mutually
            disjoint subsets.
            \begin{theorem}
                If $\Omega$ is a set, and $A_{1},A_{2},A_{3}$ are
                mutually disjoint subsets of $\Omega$, then:
                \begin{equation}
                    R=\{\emptyset,A_{1},A_{2},A_{3},
                        A_{1}\cup{A}_{2},A_{1}\cup{A}_{3},
                        A_{2}\cup{A}_{3},
                        A_{1}\cup{A}_{2}\cup{A}_{3}\}
                \end{equation}
                is a set ring on $\Omega$.
            \end{theorem}
            Indeed, we may generalize further.
            \begin{theorem}
                If $\Omega$ is a set and if
                $A$ is a subset of $\mathcal{P}(\Omega)$ of
                $n$ elements such that, for all
                $a,b\in{A}$, $a\cap{B}=\emptyset$, then:
                \begin{equation}
                    R=\{\cup_{i\in{I}}A_{i}:
                    I\in\mathcal{P}(\mathbb{Z}_{n})\}
                \end{equation}
                Is a set ring on $\Omega$.
            \end{theorem}
            \begin{theorem}
                If $\Omega$ is a set, then the set of all
                finite subsets of $\Omega$ is a set ring on
                $\Omega$.
            \end{theorem}
            A left semi-interval of $\mathbb{R}$ is an interval
            of the form $[a,b)$ where $a\leq{b}$. If $a=b$, this
            is the empty set. The set of all left semi-intervals
            is not a set ring on $\mathbb{R}$ since the union
            of two semi-intervals need not be a semi-interval.
            We need to add more sets to allow this to be a
            set ring. The collection of all finite unions of
            semi-intervals of $\mathbb{R}$ is a set ring.
            First, note the following:
            \begin{equation}
                \Big(\bigcup_{n=1}^{N}[a_{n},b_{n})\Big)
                \setminus[c,d)=\bigcup_{n=1}^{N}
                \Big([a_{n},b_{n})\setminus[c,d)]
            \end{equation}
            This is again the finite union of intervals. By
            induction we see that this collection is a ring on
            $\mathbb{R}$. We have seen that a set ring is
            closed to unions and set differences, and this
            implies that rings are closed under intersections and
            closed under symmetric differences. As it turns out,
            this is an equivalent definition of a set ring.
            \begin{theorem}
                If $\Omega$ is a set and
                $R\subset\mathcal{P}(\Omega)$, then $R$ is
                a set ring of $\Omega$ if and only if $R$ is
                closed under symmetric differences and
                intersections.
            \end{theorem}
            If $R$ is a set ring on $\Omega$, and if
            $A\in{R}$, let $\chi_{A}:\Omega\rightarrow[0,1]$ be
            the indicator function defined as follows:
            \begin{equation}
                \chi_{A}(\omega)=
                \begin{cases}
                    0,&\omega\notin{A}\\
                    1,&\omega\in{A}
                \end{cases}
            \end{equation}
            Then we have:
            \begin{align}
                \chi_{A\cap{B}}(\omega)
                &=\chi_{A}(\omega)\chi_{B}(\omega)\\
                \chi_{A\ominus{B}}&=
                \big(\chi_{A}(\omega)+\chi_{B}(\omega)\big)
                \mod{2}
            \end{align}
            These two operations satisfy the axioms of a ring.
            That is, a ring in the algebraic sense of the word:
            A set with two operations that behave certain
            properties. It is because of this that set rings
            have earned their name.
        \subsection{Set Algebras}
            \begin{definition}
                A set algebra on a set $\Omega$ is a set ring
                on $\Omega$ such that $\Omega\in\mathcal{A}$.
            \end{definition}
            \begin{example}
                Let $\Omega=\{1,2,3,4\}$ and
                $R=\{\emptyset,\{1\},\{2,3\}\}$. Then $R$
                is a set ring, but it is not a set algebra
                since $\Omega\notin{R}$.
            \end{example}
            \begin{lexample}
                If $\Omega$ is an infinite set, and if
                $\mathcal{E}=\big\{\{x\}:x\in\Omega\big\}$, then
                the smallest set algebra that contains $\mathcal{E}$
                is the set of all finite and co-finite subsets of
                $\Omega$. There are a few cases to check. The finite
                union of finite subsets is finite, the finite union of
                co-finite subsets is co-finite, and the finite union
                of finite and co-finite is again co-finite. For set
                difference, the difference of finite with finite is
                again finite, and the difference of co-finite with
                co-finite is either co-finite or finite. The
                difference of co-finite with finite is co-finite,
                and the difference of finite with co-finite is finite.
                Thus, this set is a set algebra on $\Omega$. Moreoever
                it is the smallest set algebra that will contain $\mathcal{E}$.
            \end{lexample}
            From the definition, we see that a set algebra
            is closed under complements. indeed, this creates
            and equivalent definition for set algebras.
            \begin{theorem}
                If $\Omega$ is a set and
                $\mathcal{A}\subseteq\mathcal{P}(\Omega)$,
                then $\mathcal{A}$ is a set algebra on $\Omega$
                if and only if $\Omega\in\mathcal{A}$, and
                $\mathcal{A}$ is closed under union and
                complement.
            \end{theorem}
            \begin{theorem}
                If $\Omega$ is a set and $R$ is a set ring
                on $\Omega$, and if $\mathcal{A}$ is a set
                algebra on $\Omega$ such that
                $R\subset\mathcal{A}$, then for all $A\in{R}$,
                $A\in\mathcal{A}$ and $A^{C}\in\mathcal{A}$.
            \end{theorem}
            This then defines the \textit{smallest} set algebra
            that contains a set ring.
            \begin{theorem}
                If $\Omega$ is a set and $R$ is a set ring on
                $\Omega$, then:
                \begin{equation}
                    \mathcal{A}=\{A,A^{C}:A\in{R}\}
                \end{equation}
                Is a set algebra on $\Omega$.
            \end{theorem}
            \begin{theorem}
                If $\Omega$ is a set and $A$ and $B$ are
                disjoint subset of $A$, then:
                \begin{equation}
                    \mathcal{A}=
                        \{\emptyset,A,B,A\cup{B},
                          \Omega,A^{C},B^{C},A^{C}\cap{B}^{C}\}
                \end{equation}
                is a set algebra on $\Omega$.
            \end{theorem}
            For non-disjoint $A$ and $B$, the smallest
            set algebra becomes more complicated. We saw that
            the collection of all finite subsets of a set is
            a set ring on the set. The collection of all finite
            subsets, and their complements, is a set algebra.
        \subsection{\texorpdfstring{$\sigma$}{Sigma}-Rings}
            If $\Omega$ is a set, then $R\subset\mathcal{P}(\Omega)$
            is called a set ring on $\Omega$ if, for all
            $A,B\in{R}$, $A\cup{B}\in{R}$ and
            $A\setminus{B}\in{R}$. From this, given a ring $R$ on
            $\Omega$, the empty set is included, that is
            $\emptyset\in{R}$, and if $A,B\in{R}$, then
            $A\cap{B}\in{R}$. By induction, for any finite collection
            of elements in $R$, the union of these subsets is also
            contained in $R$, as well as the intersection. A set
            algebra on $\Omega$ is a ring $\mathcal{A}$
            on $\Omega$ such that $\Omega\in\mathcal{A}$. That is,
            $\mathcal{A}\subset\mathcal{P}(\Omega)$, and
            $\mathcal{A}$ is closed under union, set difference, and
            $\Omega\in\mathcal{A}$. There is an equivalent definition:
            $\Omega\in\mathcal{A}$, for all $A\in\mathcal{A}$,
            $A^{C}\in\mathcal{A}$, and for all $A,B\in\mathcal{A}$,
            $A\cup{B}\in\mathcal{A}$. The complement of $A$,
            $A^{C}$, is defined as $\Omega\setminus{A}$. The
            equivalence of the two definitions comes from DeMorgan's
            laws, since
            $A\setminus{B}=A\cap{B}^{C}=(A^{C}\cup{B})^{C}$. We now
            talk about $\sigma$-Ring.
            \begin{definition}
                A $\sigma$-Ring on a set $\Omega$ is a set
                $\sigma\subset\mathcal{P}(\Omega)$ such that,
                for all countable subsets of $\sigma$, the union
                $\bigcup_{i=1}^{\infty}A_{i}\in\sigma$, and for all
                $A,B\in\sigma$, $A\setminus{B}\in\sigma$.
            \end{definition}
            The requirement that the collections be countable is
            important to note. A \textit{topology} is a subset
            of $\mathcal{P}(\Omega)$ with the property that it is
            closed under arbitrary unions. $\sigma$-Rings need only
            be closed under countable unions.
            \begin{example}
                Every $\sigma$-Ring is a set ring, but not every
                set ring is a $\sigma$-ring. Let $\Omega$ be
                uncountable, and let $R$ be the set of all finite
                subsets of $\Omega$. Then $R$ is a ring, but it is
                not a $\sigma$-ring. For, as $\Omega$ is uncountably
                infinite, it has a countable subset $A$, and we
                may subscript the elements as $a_{n}$. But
                $\bigcup_{n=1}^{\infty}\{a_{n}\}$ is not a finite
                subset of $\Omega$, and is therefore not contained
                in $R$. Thus, $R$ is not closed under countable unions
                and $R$ is not a $\sigma$-ring. However, if we let
                $\sigma$ be the set of all \textit{countable} subsets
                of $\Omega$, the $\sigma$ is indeed a $\sigma$-ring.
            \end{example}
            \begin{example}
                The collection of all semi-intervals and finite
                unions of semi-intervals defines a ring on
                $\mathbb{R}$. It is tempting to think tha the
                collection of all countable unions of semi-intervals
                is a $\sigma$-ring on $\mathbb{R}$, but this is not
                the case. The Cantor set is an example of a subset
                that can be constructed by a countable number of
                steps of removing intervals from a given interval,
                but the resulting set is not the countable union of
                semi-intervals. To construct the Cantor set, consider
                the interval $[0,1]$. From this, remove
                $(\frac{1}{3},\frac{2}{3})$. Continuing removing the
                middle third from each sub-interval obtained. The
                resulting set contains no interval as a subset, and
                thus cannot be the union of countably many intervals,
                or semi-intervals.
            \end{example}
        \subsection{Dynkin System}
            A Dynkin system on a set $\Omega$ is a subset
            $\mathcal{D}\subset\mathcal{P}(\Omega)$ such that
            $\Omega\in\mathcal{D}$, if $A,B\in\Omega$ and if
            $A\subseteq{B}$, then $A\setminus{B}\in\mathcal{D}$,
            and for all countable collections of elements of
            $\mathcal{D}$ such that
            $A_{1}\subset{A}_{2}\subset\hdots$,
            $\cup_{n=1}^{\infty}A_{n}\in\mathcal{D}$. There is
            an equivalent defintion for Dynkin Systems.
            $\Omega\in\mathcal{D}$, $A\in\mathcal{D}$ implies
            $A^{C}\in\mathcal{D}$, and for all countable disjoint
            collections of elements in $\mathcal{D}$, the union
            is also contained in $\mathcal{D}$. These requirements
            are weaker than those of a $\sigma$-algebra. Any
            $\sigma$-algebra is a Dynkin system, but not every
            Dynkin system is a $\sigma$-algebra.
            \begin{ldefinition}{Dynkin System}
                A Dynkin System on a set $\Omega$ is a subset
                $\mathcal{D}\subseteq\mathcal{P}(\Omega)$ such that:
                \begin{enumerate}
                    \item $\Omega\in\mathcal{D}$.
                    \item For all $A,B\in\mathcal{D}$ such that $A\subseteq{B}$,
                          $B\setminus{A}\in\mathcal{D}$.
                    \item For any sequence $A_{n}\in\mathcal{D}$ such that
                          $A_{n}\subseteq{A}_{n+1}$,
                          $\cup_{n=1}^{\infty}A_{n}\in\mathcal{D}$
                \end{enumerate}
            \end{ldefinition}
            \begin{theorem}
                If $\Omega$ is a set and $\mathcal{D}\subseteq\mathcal{P}(\Omega)$
                is such that $\Omega\in\mathcal{D}$, for all $A\in\mathcal{D}$,
                $A^{C}\in\mathcal{D}$, and if for all sequences $A_{n}\in\mathcal{D}$
                such that $A_{n}\cap{A}_{m}=\emptyset$ for all $n\ne{m}$,
                $\cup_{n=1}^{\infty}A_{n}\in\mathcal{D}$, then
                $\mathcal{D}$ is a Dynkin System on $\Omega$.
            \end{theorem}
            \begin{theorem}
                If $\mathcal{D}$ is a Dynkin system on a set
                $\Omega$, and if $\mathcal{D}$ is closed with
                respect to intersections, then $\mathcal{D}$
                is a $\sigma$-algebra on $\Omega$.
            \end{theorem}
            \begin{theorem}
                If $\Omega$ is a set, if
                $\mathcal{E}\subset\mathcal{P}(\Omega)$ is closed
                to intersections, and if $\mathcal{D}$ is the
                Dynkin System generated by $\mathcal{E}$, then
                $\mathcal{D}$ is a $\sigma$-algebra.
            \end{theorem}
            \begin{theorem}[Dynkin's Theorem]
                If $\Omega$ is a set, $\mathcal{C}\subseteq\mathcal{P}(\Omega)$
                is intersection-stable, and if $\mathcal{D}$ is the smallest
                Dynkin system that contains $\mathcal{C}$, then $\mathcal{D}$
                is also intersection-stable.
            \end{theorem}
            \begin{proof}
                For let:
                \begin{equation}
                    \mathcal{D}_{1}=
                    \{D\in\mathcal{D}:\forall_{C\in\mathcal{C}},D\cap{D}\in\mathcal{D}\}
                \end{equation}
                Then $\mathcal{D}_{1}$ is a Dynkin system, and thus
                $\mathcal{D}_{1}=\mathcal{D}$. Now define:
                \begin{equation}
                    \mathcal{D}_{2}=\{
                        D\in\mathcal{D}:\forall_{A\in\mathcal{D}},D\cap{A}\in\mathcal{D}
                    \}
                \end{equation}
                Then $\mathcal{C}\subseteq\mathcal{D}_{2}$ and $\mathcal{D}_{2}$ is a
                Dynkin System, and thus $\mathcal{D}_{2}=\mathcal{D}$.
            \end{proof}
            \begin{theorem}
                If $\Omega$ is a set, $\mathcal{C}\subseteq\mathcal{P}(\Omega)$
                is intersection-stable, and if $\mathcal{D}$ is the smallest
                Dynkin system that contains $\mathcal{C}$, then $\mathcal{D}$
                is a $\sigma\textrm{-Algebra}$ on $\Omega$.
            \end{theorem}
            Since semi-intervals are closed to intersections,
            the Borel $\sigma$-algebra is equivalently the
            Dynkin system generated by semi-intervals.
        \subsection{\texorpdfstring{$\sigma$}{Sigma}-Algebras}
            In an analogous manner to how set rings and set algebras
            were defined, there is something called a $\sigma$-algebra.
            This notion will be one of the central themes of measure
            theory.
            \begin{definition}
                A $\sigma$-algebra on a set $\Omega$ is a
                $\sigma$-ring on $\Omega$ such that
                $\Omega\in\mathcal{A}$
            \end{definition}
            That is, given any countable collection of elements in
            $\mathcal{A}$, the union is also contained in
            $\mathcal{A}$. In addition, $\mathcal{A}$ is closed under
            set differences and $\Omega\in\mathcal{A}$.
            \begin{example}
                The first trivial example is the power set
                $\mathcal{P}(\Omega)$. Also the set
                $\{\emptyset,\Omega\}$ defines a $\sigma$-algebra on
                $\Omega$. The set of all countable subsets defines
                a $\sigma$-ring, and the set of all countable and
                co-countable (Sets whose complement is countable)
                will define a $\sigma$-algebra.
            \end{example}
            We can equivalently define a $\sigma$-algebra to be a
            collection of sets $\mathcal{A}$ such that
            $\Omega\in\mathcal{A}$, for all $A\in\mathcal{A}$,
            $A^{C}\in\mathcal{A}$, and $\mathcal{A}$ is closed under
            countable unions. Being closed under countable unions
            implies that it is closed under finite unions as well.
            For let $A_{1}=A$, and for all $n>1$, let $A_{n}=B$.
            Then $\bigcup_{n=1}^{\infty}A_{n}=A\cup{B}$. By induction,
            a $\sigma$-algebra is closed under any finite union.
        \subsection{Borel \texorpdfstring{$\sigma$}{Sigma}-Algebra}
            One of the most important types of $\sigma$-algebras
            is the Borel $\sigma$-algebra. We first define the
            Borel $\sigma$-algebra on $\mathbb{R}$.
            \begin{definition}
                The Borel $\sigma$-algebra on $\mathbb{R}$, denoted
                $\mathcal{B}$, is the $\sigma$-algebra generated
                by the set $\{[a,b):a,b\in\mathbb{R}\}$.
            \end{definition}
            That is, the Borel $\sigma$-algebra on $\mathbb{R}$ is
            the \textit{smallest} $\sigma$-algebra that contains
            all of the semi-intervals. We can equivalently say that
            $\mathcal{B}$ is the $\sigma$-algebra generated by all
            \textit{open} intervals. If we know that every open
            subset of $\mathbb{R}$ is the countable union of open
            subsets, than we can equivalently say that
            $\mathcal{B}$ is the $\sigma$-algebra generated by all
            \textit{open subsets} of $\mathbb{R}$. Writing $[a,b)$
            as the countable intersection of open intervals, or
            $(a,b)$ as the countable union of semi-intervals comes
            from the Archimedean property of the real numbers.
            Thus, the smallest $\sigma$-algebra that contains all
            semi-intervals is the smallest $\sigma$-algebra that
            contains all open intervals, which
            is the smallest $\sigma$-algebra that contains all open
            subsets of $\mathbb{R}$. Similarly, this will contain all
            of the \textit{closed} intervals, intervals of the form
            $[a,b]$. We say that a set $\mathcal{U}\subset\mathbb{R}$
            is open if, for all $x\in\mathcal{U}$, there is an $r>0$
            such that $(x-r,x+r)\subset\mathcal{U}$. That is, every
            point in $\mathcal{U}$ can be surrounded by an interval
            that is entirely contained in $\mathcal{U}$. Thus, any
            open set can be written as:
            \begin{equation}
                \mathcal{U}=
                    \bigcup_{x\in\mathcal{U}}(\alpha_{x},\beta_{x})
            \end{equation}
            This union is not countable, for any open set must
            contain an interval, an intervals are uncountable large.
            This is simply because $(a,b)$ is equivalent to $(0,1)$.
            By adjusting the size of $\alpha_{x}$ and $\beta_{x}$ to
            be rational numbers, we can written $\mathcal{U}$ as the
            union of intervals with rational endpoints. But there are
            only countably many such intervals, and thus
            $\mathcal{U}$ is the union of countably many open
            intervals. Thus, any open set is the union of countably
            many open intervals. From this, the smallest
            $\sigma$-algebra that contains open intervals will contain
            all open sets, since $\sigma$-algebras are closed under
            countable unions. Borel sets are elements of the
            Borel $\sigma$-algebra $\mathcal{B}$. Since all open
            sets are Borel sets, and as $\sigma$-algebras are closed
            under complenents, all closed sets are also Borel sets.
            This is because the complement of an open set is a closed
            set, and vice versa. Thus, equivalently, $\mathcal{B}$ is
            the smallest $\sigma$-algebra containing all closed sets.
            A $G_{\delta}$ sets is a subset that is the countable
            intersection of open sets. As open sets are not
            necessarily closed under countable intersections, not
            all $G_{\delta}$ sets are open. There is another notion,
    \section{Measures}
        \subsection{A Review Infinite Series}
            Given a sequence of real numbers,
            $a:\mathbb{N}\rightarrow\mathbb{R}$, the sum of this
            sequence is defined as the limit of
            finite partial sums. That is:
            \begin{equation}
                \sum_{n=1}^{\infty}a_{n}=
                \underset{N\rightarrow\infty}{\lim}
                    \sum_{n=1}^{N}a_{n}
            \end{equation}
            In general, this limit may not in general exists. If it
            does, we say the series converges. If the limit does
            not exists, we do not define the sum and instead just
            have a meaningless combination of symbols. If the
            sequence is positive, then the sequence of partial sums
            will be increasing. If this sequence is bounded, then
            the limit exists. This comes from the fact that bounded
            monotonic sequences converge, a result that stems from
            the least upper bound property of $\mathbb{R}$.
            Moreover, if $a:\mathbb{N}\rightarrow\mathbb{R}$ is a
            sequence of positive real numbers, and if
            $f:\mathbb{N}\rightarrow\mathbb{N}$ is any bijective
            function, then the following is true:
            \begin{equation}
                \sum_{n=1}^{\infty}a_{n}
                =\sum_{n=1}^{\infty}a_{f(n)}
            \end{equation}
            We can also split the sequence into a grid,
            and take the
            double sum, obtaining the same result. If
            $A_{1},A_{2},\hdots$ are disjoint sets whose union is
            $\mathbb{N}$, and if $b_{nm}$ is the $n^{th}$ element
            of $A_{m}$, then:
            \begin{equation}
                \sum_{i=1}^{\infty}a_{i}=
                \sum_{n=1}^{\infty}\sum_{m=1}^{\infty}b_{nm}
            \end{equation}
            We should be precise in what we mean. The double
            sum is the \textit{limit of a limit}.
            \begin{equation}
                \sum_{n=1}^{\infty}\sum_{m=1}^{\infty}a_{nm}
                =\underset{N\rightarrow\infty}{\lim}\sum_{n=1}^{N}
                \Big(\underset{M\rightarrow\infty}{\lim}
                \sum_{m=1}^{M}a_{nm}\Big)
            \end{equation}
            We use infinite series to define \textit{measures} on
            $\sigma$-algebra.
        \subsection{Measure Functions}
            A set function on a collection of sets $\mathcal{E}$
            is a function $\mu:\mathcal{E}\rightarrow\mathbb{R}$.
            For example, if we consider the set of all
            semi-intervals of the form $[a,b)$, where
            $a,b\in\mathbb{R}$ and $a\leq{b}$, then we can define
            $\mu([a,b))=b-a$. This gives rise to the notion of
            a measure function.
            A measure function on a collection of set
            $\mathcal{E}$ is a function
            $\mu:\mathcal{E}\rightarrow\mathbb{R}$ such that:
            \begin{enumerate}
                \item If $\emptyset\in\mathcal{E}$, then
                      $\mu(\emptyset)=0$
                \item For all $A\in\mathcal{E}$, $\mu(A)\geq{0}$
                \item For any countable collection of pair-wise
                      disjoint sets whose
                      union also lies in $\mathcal{E}$,
                      $\mu(\cup_{n=1}^{\infty}A_{n})=%
                       \sum_{n=1}^{\infty}\mu(A_{n})$
            \end{enumerate}
            It helps if we don't have to consider the case where
            $\mu(\emptyset)$ is undefined, or where we don't have
            closure under countable unions, so we discuss measure
            functions on $\sigma$-algebras.
            \begin{definition}
                A measure on a $\sigma$-algebra
                $\mathcal{A}$ is a function
                $\mu:\mathcal{A}\rightarrow\mathbb{R}$ such that:
                \begin{enumerate}
                    \item $\mu(\emptyset)=0$
                    \item For all $A\in\mathcal{A}$,
                          $\mu(A)\geq{0}$
                    \item For any countable collection of pairwise
                          disjoint elements of $\mathcal{A}$,
                          $\mu(\cup_{n=1}^{\infty}A_{n})=%
                           \sum_{n=1}^{\infty}\mu(A_{n})$
                \end{enumerate}
            \end{definition}
            \begin{example}
                Let $\Omega$ be a set, and let
                $\mathcal{A}=\mathcal{P}(\Omega)$. Then
                $\mathcal{A}$ is a $\sigma$-algebra on $\Omega$.
                If $\omega_{1},\hdots,\omega_{n}\in\Omega$ and if
                $p_{1},\hdots,p_{n}\in\mathbb{R}^{+}$, then:
                \begin{equation}
                    \mu(A)=\sum_{k=1}^{n}p_{k}\chi_{A}(\omega_{k})
                \end{equation}
                Where $\xi_{A}$ is the indicator function:
                \begin{equation}
                    \chi_{A}(\omega)=
                    \begin{cases}
                        0,&\omega\notin{A}\\
                        1,&\omega\in{A}
                    \end{cases}
                \end{equation}
                This is an example of a \textit{point measure}
                on $\mathcal{A}$. It defines a measure function.
            \end{example}
    \section{Lecture 3}
        A $\sigma\text{-Algebra}$ on a set $\Omega$ is a subset
        $\mathcal{A}$ of $\mathcal{P}(\Omega)$ such that
        $\Omega\in\mathcal{A}$ and for any countable collection of
        elements $A_{i}\in\mathcal{A}$, the union
        $\bigcup_{i=1}^{\infty}A_{i}$ is also contained in
        $\mathcal{A}$. $\mathcal{A}$ does not have to consist of
        countably many elements. The sequence of subset $A_{i}$ does
        not have to exhaust the entirety of $\mathcal{A}$, much the
        way that any sequence of real numbers will not exhaust the
        entire of $\mathbb{R}$. Going in the other direction,
        $\sigma\text{-Algebras}$ can be finite. If $\Omega$ is a
        set, and if $A\subset\Omega$ is non-empty, then
        $\mathcal{A}=\{\emptyset,A,A^{C},\Omega\}$ defines a
        $\sigma\text{-algebra}$ on $\Omega$. A measure on a
        $\sigma\text{-Algebra}$ $\mathcal{A}$ is a function
        $\mu:\mathcal{A}\rightarrow\mathbb{R}$ such that, for all
        $A\in\mathcal{A}$, $\mu(A)\geq{0}$, $\mu(\emptyset)=0$, and
        given a mutually disjoint countable collection of elements of
        $\mathcal{A}$, the following holds:
        \begin{equation}
            \mu\Big(\bigcup_{i=1}^{\infty}A_{i}\Big)
            =\sum_{n=1}^{\infty}\mu(A_{i})
        \end{equation}
        \begin{example}
            A pure point measure is a measure that assigns to a
            collection of elements $\omega_{j}\in\Omega$ a positive
            real number $p_{j}$, and then the measure of any set
            $A$ is:
            \begin{equation}
                \mu(A)=\sum_{j:\omega_{j}\in{A}}p_{j}
            \end{equation}
        \end{example}
        \subsection{Properties of Measure}
            \subsubsection{Monotonicity}
                If $A$ and $B$ are elements of a $\sigma\text{-Algebra}$
                $\mathcal{A}$, if $\mu$ is a measure on
                $\mathcal{A}$, and if $A\subseteq{B}$, then
                $\mu(A)\leq\mu(B)$. This is the monotonic property
                of measures.
                \begin{theorem}
                    If $\Omega$ is a set, $\mathcal{A}$ is a
                    $\sigma\text{-Algebra}$ on $\Omega$, if
                    $\mu$ is a measure on $\mathcal{A}$, and if
                    $A,B$ are elements of $\mathcal{A}$ such that
                    $A\subseteq{B}$, then $\mu(A)\leq\mu(B)$.
                \end{theorem}
                \begin{proof}
                    For as $\mathcal{A}$ is a $\sigma\text{-Algebra}$
                    on $\Omega$, and as $A,B\in\mathcal{A}$,
                    $B\setminus{A}\in\mathcal{A}$. But, as
                    $A\subseteq{B}$, $B=(B\setminus{A})\cup{A}$.
                    But then, as measures are additive and positive:
                    \begin{align}
                        \mu(B)&=\mu\big((B\setminus{A})\cup{A}\big)\\
                        &=\mu(B\setminus{A})+\mu(A)\\
                        &\geq\mu(A)
                    \end{align}
                \end{proof}
                \begin{theorem}
                    If $\Omega$ is a set, $\mathcal{A}$ is a
                    $\sigma\text{-Algebra}$ on $\Omega$, if
                    $\mu$ is a measure on $\mathcal{A}$, and if
                    $A,B$ are elements of $\mathcal{A}$ such that
                    $A\subseteq{B}$ and $\mu(A),\mu(B)<\infty$,
                    then $\mu\big(B\setminus{A}\big)=\mu(B)-\mu(A)$.
                \end{theorem}
            \subsubsection{Continuity Theorems}
                \begin{theorem}[Continuity from Below]
                    If $\Omega$ is a set, $\mathcal{A}$ is a
                    $\sigma\text{-Algebra}$ on $\Omega$, if
                    $\mu$ is a measure on $\mathcal{A}$, and if
                    $A_{i}$ is a sequence of elements in $\mathcal{A}$
                    such that, for all $i\in\mathbb{N}$,
                    $A_{i}\subseteq{A}_{i+1}$, then:
                    \begin{equation}
                        \mu\Big(\bigcup_{i=1}^{\infty}A_{i}\Big)
                        =\underset{N\rightarrow\infty}{\lim}
                        \mu(A_{N})
                    \end{equation}
                \end{theorem}
                \begin{proof}
                    For let $A=\cup_{n=1}^{\infty}A_{n}$ and let
                    $B_{n}=A_{n+1}\setminus{A}_{n}$. Then, as
                    $A_{n}\subseteq{A}_{n+1}$, for all
                    $i,j\in\mathbb{N}$, $B_{i}\cap{B}_{j}=\emptyset$.
                    But $A=A_{1}\cup\Big(\cup_{n=1}^{\infty}B_{n}\Big)$
                    and this is the countable union of mutually
                    disjoint sets, and therefore, using the
                    telescoping series:
                    \begin{align}
                        \mu(A)&=\mu(A_{1})+
                        \sum_{n=1}^{\infty}\mu(B_{n})\\
                        &=\mu(A_{1})+\sum_{n=1}^{\infty}
                        \Big(\mu(A_{n+1})-\mu(A_{n})\Big)\\
                        &=\mu(A_{1})+
                        \underset{N\rightarrow\infty}{\lim}
                        \Big(\mu(A_{N})-\mu(A_{1})\Big)\\
                        &=\underset{N\rightarrow\infty}{\lim}
                        \mu(A_{N})
                    \end{align}
                \end{proof}
                \begin{theorem}[Continuity from Above]
                    If $\Omega$ is a set, $\mathcal{A}$ is a
                    $\sigma\text{-Algebra}$ on $\Omega$, if
                    $\mu$ is a measure on $\mathcal{A}$, and if
                    $A_{i}$ is a sequence of elements in $\mathcal{A}$
                    such that, for all $i\in\mathbb{N}$,
                    $A_{i+1}\subseteq{A}_{i}$ and there exists an
                    $n\in\mathbb{N}$ such that $\mu(A_{n})$ is finite,
                    then:
                    \begin{equation}
                        \mu\Big(\bigcap_{n=1}^{\infty}A_{n}\Big)
                        =\underset{N\rightarrow\infty}{\lim}
                        \mu(A_{N})
                    \end{equation}
                \end{theorem}
                \begin{proof}
                    For let $A=\cap_{n=1}^{\infty}A_{n}$ and let
                    $B_{n}=A_{n}\setminus{A}_{n+1}$. Then:
                    \begin{equation}
                        A_{1}=
                        A\cup\big(\bigcup_{n=1}^{\infty}B_{n}\Big)
                    \end{equation}
                    And this is the union of countably many disjoint
                    sets. Therefore:
                    \begin{align}
                        \mu(A_{1})&=
                        \mu(A)+\sum_{n=1}^{\infty}\mu(B_{n})\\
                        &=\mu(A)+\sum_{n=1}^{\infty}
                        \Big(\mu(A_{n}-\mu(A_{n+1})\Big)\\
                        &=\mu(A)+\mu(A_{1})-
                        \underset{N\rightarrow\infty}{\lim}\mu(A_{N})
                    \end{align}
                    Subtracting by $\mu(A_{1})$ obtains the result.
                \end{proof}
                If $\mu(A_{i})=\infty$ for all $i\in\mathbb{N}$, then
                the above theorem may not be true. For consider
                the collection of sets $A_{n}=[n,\infty)$. The
                measure of each $A_{n}$ is infinite, but the
                intersection of the entire collection is empty.
                Thus the measure of the intersection is zero.
                \begin{theorem}[Countable Sub-Additivity]
                    If $\Omega$ is a set, $\mathcal{A}$ a
                    $\sigma\text{-Algebra}$ on $\mathcal{A}$, and
                    if $A_{i}$ is a countable collection of elements
                    of $\mathcal{A}$, then:
                    \begin{equation}
                        \mu\Big(\bigcup_{n=1}^{\infty}A_{n}\Big)
                        \leq\sum_{n=1}^{\infty}\mu(A_{n})
                    \end{equation}
                \end{theorem}
                \begin{proof}
                    For if $A_{1},A_{2}\in\mathcal{A}$, then:
                    \begin{equation}
                        \mu(A_{1}\cup{A}_{2})=
                        \mu(A_{1}\setminus{A}_{2})+
                        \mu(A_{2}\setminus{A}_{1})+
                        \mu(A_{1}\cap{A}_{2})
                    \end{equation}
                    But also:
                    \begin{align}
                        \mu(A_{1})&=
                        \mu(A_{1}\setminus{A}_{2})+
                        \mu(A_{1}\cap{A}_{2})\\
                        \mu(A_{2})&=
                        \mu(A_{2}\setminus{A}_{1})+
                        \mu(A_{1}\cap{A}_{2})
                    \end{align}
                    And therefore:
                    \begin{equation}
                        \mu(A_{1})+\mu(A_{2})=
                        \mu(A_{1}\cup{A}_{2})+\mu(A_{1}\cap{A}_{2})
                    \end{equation}
                    We now prove by induction. Suppose this is true
                    of a collection of $N$ elements. Given a collection
                    $A_{i}$ of $N+1$ elements, let
                    $B=\cup_{n=1}^{N}A_{i}$. But then:
                    \begin{align}
                        \mu(A_{N+1}\cup{B})&
                        \leq\mu(A_{N+1})+\mu(B)\\
                        &\leq\mu(A_{N+1})+\sum_{n=1}^{N}A_{n}\\
                        &=\sum_{n=1}^{N+1}\mu(A_{n})
                    \end{align}
                \end{proof}
    \section{Lebesgue-Stieltjes Measures}
        A Lebesgue-Stieltjes measure is any measure on the
        Borel $\sigma\text{-Algebra}$ $\mathcal{B}$ such that,
        for any finite semi-interval $[a,b)$,
        $\mu\big(\mu[a,b)\big)<\infty$. $\mu(\mathbb{R})$ may
        be infinite. Recall that the Borel $\sigma\text{-Algebra}$
        is the smallest $\sigma\text{-Algebra}$ on $\mathbb{R}$
        that contains all semi-intervals $[a,b)$. A pure point
        measure on $\mathbb{R}$, indexing over the rational
        numbers, would be such a measure. If we have a
        Lebesgue-Stieltjes measure, we wish to find a function
        $F_{\mu}:\mathbb{R}\rightarrow\mathbb{R}$ such that, for
        all semi-intervals $[a,b)$,
        $\mu([a,b))=F_{\mu}(b)-F_{\mu}(a)$. In probability, this
        is called the cummulative probability function. For now
        we wish to show that there is indeed such a function that
        does this. Consider the case when $\mu(\mathbb{R})<\infty$.
        Let $F_{\mu}(x)=\mu(-\infty,x)$. Then:
        \begin{align}
            \mu([a,b))
            &=\mu((-\infty,b)\setminus(-\infty,a))\\
            &=\mu((-\infty,b))-\mu((-\infty,a))\\
            &=F_{\mu}(b)-F_{\mu}(a)
        \end{align}
        In the more general case when that measure of the entire
        real line is infinite we still want to find a function
        such that:
        \begin{align}
            \mu\big([0,x)\big)&=F_{\mu}(x)-F_{\mu}(0)&x>0\\
            \mu\big([x,0)\big)&=F_{\mu}(0)-F_{\mu}(x)&x<0
        \end{align}
        We can define the following:
        \begin{equation}
            F_{\mu}(x)=
            \begin{cases}
                \mu\big([0,x)\big)+C,&x>0\\
                -\mu\big([x,0)\big)+C,&x<0\\
                C,&x=0
            \end{cases}
        \end{equation}
        Then $F_{\mu}$ is a function that satisfies our criterion.
        Indeed, $F_{\mu}$ is defined uniquely up to an additive
        constant. Any such function is non-decreasing since, for
        any $x<y$, $F_{\mu}(y)-F_{\mu}(x)=\mu([x,y))\geq{0}$.
        In addition, $F_{\mu}$ is left-continuous. That is,
        for all $a\in\mathbb{R}$:
        \begin{equation}
            \underset{x\rightarrow{a}^{-}}{\lim}F_{\mu}(x)
            =F_{\mu}(a)
        \end{equation}
        \begin{theorem}
            If $\mu$ is a Lebesgue-Stieltjes measure on the
            Borel $\sigma\text{-Algebra}$ of $\mathbb{R}$, and if
            $F_{\mu}$ is the function thing, then $F_{\mu}$ is
            left-continuous.
        \end{theorem}
        \begin{proof}
            For let $a\in\mathbb{R}$ and let
            $x:\mathbb{N}\rightarrow\mathbb{R}$ be a monotonic
            increasing sequence such that $x_{n}\rightarrow{a}$.
            But then, for all $n\in\mathbb{N}$,
            $[x_{n+1},a)\subset[x_{n},a)$. But then:
            \begin{equation}
                \mu\Big(\bigcap_{n=1}^{\infty}[x_{n},a)\Big)
                =\underset{N\rightarrow\infty}{\lim}
                \mu\big([x_{N},a)\big)
            \end{equation}
            But $\cap_{n=1}^{\infty}[x_{n},a)=\emptyset$ as
            $x_{n}\rightarrow{a}$. Therefore:
            \begin{equation}
                \underset{N\rightarrow\infty}{\lim}
                \mu\big([x_{n},a)\big)=0
            \end{equation}
            But from the definition of $F_{\mu}$,
            \begin{equation}
                \mu\big([x_{n},a)\big)=F_{\mu}(a)-F_{\mu}(x_{n})
            \end{equation}
            Thus, $F_{\mu}(x_{n})\rightarrow{F}_{\mu}(a)$.
        \end{proof}
        Such a function may not be right-continuous. The requirement
        that the sequence $x$ be increasing was necessary for the
        proof. Howver, the right-hand limit does exists.
        \begin{theorem}
            If blah blah, right hand limit exists.
        \end{theorem}
        \begin{proof}
            For:
            \begin{equation}
                \{a\}=\bigcap_{n=1}^{\infty}[a,a+\frac{1}{n})
            \end{equation}
            And thus, as $\mu$ is a Lebesgue-Stieltjes measure,
            and thus $\mu([a,b))<\infty$ for all finite semi-intervals,
            we may apply continuity from above and obtain:
            \begin{align}
                \mu(\{a\})&=
                \underset{N\rightarrow\infty}{\lim}
                \mu\big([a,a+\frac{1}{n}\big)\\
                &=\underset{x\rightarrow{a}^{+}}{\lim}
                F_{\mu}(x)-F_{\mu}(a)
            \end{align}
        \end{proof}
        If $\mu$ has no points of positive measure, then
        $F_{\mu}$ will be continuous.
        \begin{ftheorem}{Carath\'{e}odory Extension Theorem}{}
            If $F:\mathbb{R}\rightarrow\mathbb{R}$ is a non-decreasing
            left-continuous function then there exists a unique
            Lebesgue-Stieltjes measure $\mu$ such that, for all
            $a,b\in\mathbb{R}$, $a<b$:
            \begin{equation}
                \mu\big([a,b)\big)=F(b)-F(a)
            \end{equation}
        \end{ftheorem}
        In particular, using $F(x)=x$, we see that there is a unique
        measure on the Borel $\sigma\text{-Algebra}$ such that
        $\mu([a,b))=b-a$. This measure is called the Lebesgue measure
        on $\mathbb{R}$. We define $\mu^{*}$ on a set
        $A\subseteq\mathbb{R}$ to be:
        \begin{equation}
            \mu^{*}(A)=
            \inf\Bigg\{\sum_{i=1}^{\infty}(b_{i}-a_{i}):
            A\subseteq\bigcup_{i=1}^{\infty}[a_{i},b_{i})\Bigg\}
        \end{equation}
        If $A$ is countable, then $\mu^{*}(A)$ is zero. For let
        $a:\mathbb{N}\rightarrow{A}$ be bijection, and let
        $\varepsilon>0$. Then:
        \begin{align}
            \mu^{*}(A)&\leq
            \sum_{n=1}^{\infty}
            \Big((a_{n}+\frac{\varepsilon}{2^{n+1}})-
            (a_{n}-\frac{\varepsilon}{2^{n+1}})\Big)\\
            &=\sum_{n=1}^{\infty}\frac{\varepsilon}{2^{n}}\\
            &=\varepsilon
        \end{align}
        Taking the infininum, we see that $\mu^{*}(A)=0$. This
        function is defined on all of $\mathcal{P}(\mathbb{R})$,
        however it is not a measure. The restriction of
        $\mu^{*}$ to the Borel $\sigma\text{-Algebra}$ is
        a measure.
    \section{Measurable Functions}
        We wish to eventually talk about what it means for a
        function to be \textit{measurable}. First we do a quick review
        of function. If $f:X\rightarrow{Y}$ is a function and if
        $A\subseteq{X}$, the imsge of $A$ under $f$ is the set
        $f(A)=\{f(x):x\in{A}\}$. The notation is a little strange, but
        it has become the standard. For a subset $B\subseteq{Y}$, the
        \textit{pre-image} of $B$ is the set
        $f^{-1}(B)=\{x\in{X}:f(x)\in{B}\}$.
        \begin{lexample}
            If we let $f:\mathbb{R}\rightarrow\mathbb{R}$ be defined
            by $f(x)=x^{2}$, then $f([1,2])=[1,4]$, and
            $f^{-1}([1,4])=[1,2]\cup[-2,-1]$. As another example we
            can consider $f(x)=\sin(x)$. Then
            $f^{-1}(\{0\})=\{n\pi:n\in\mathbb{N}\}$ and
            $f^{-1}([-1,1])=\mathbb{R}$.
        \end{lexample}
        \begin{theorem}
            If $X$ and $Y$ are sets, $f:X\rightarrow{Y}$,
            and if $A,B\subset{X}$, then:
            \begin{equation}
                f(A\cup{B})=f(A)\cup{f}(B)
            \end{equation}
        \end{theorem}
        \begin{theorem}
            If $X$ and $Y$ are sets, $f:X\rightarrow{Y}$,
            and if $A,B\subset{X}$, then:
            \begin{equation}
                f(A\cap{B})\subseteq{f(A)\cap{f}(B)}
            \end{equation}
        \end{theorem}
        For pre-images, we get equality:
        \begin{theorem}
            If $X$ and $Y$ are sets, $f:X\rightarrow{Y}$,
            and if $A,B\subset{X}$, then:
            \begin{equation}
                f^{-1}(A\cup{B})=f^{-1}(A)\cup{f}^{-1}(B)
            \end{equation}
        \end{theorem}
        \begin{theorem}
            If $X$ and $Y$ are sets, $f:X\rightarrow{Y}$,
            and if $A,B\subset{X}$, then:
            \begin{equation}
                f^{-1}(A\cap{B})=f^{-1}(A)\cap{f}^{-1}(B)
            \end{equation}
        \end{theorem}
        \begin{theorem}
            If $X$ and $Y$ are sets, $f:X\rightarrow{Y}$,
            and if $A\subset{X}$, then:
            \begin{equation}
                f^{-1}(A^{C})=f^{-1}(A)^{C}
            \end{equation}
        \end{theorem}
        \begin{theorem}
            If $X$ and $Y$ are sets,if
            $f:X\rightarrow{Y}$ is a function, and if
            $A\subseteq{X}$, then:
            \begin{equation}
                A\subseteq{f^{-1}\Big(f\big(A\big)\Big)}
            \end{equation}
        \end{theorem}
        \begin{theorem}
            If $X$ and $Y$ are sets, if
            $f:X\rightarrow{Y}$ is an injective function,
            and if $A\subseteq{X}$, then:
            \begin{equation}
                A=f^{-1}\Big(f\big(A\big)\Big)
            \end{equation}
        \end{theorem}
    \section{Lecture 4}
        Recalling some definitions, a measure on a $\sigma\textrm{-Algebra}$
        $\mathcal{A}$ is a function
        $\mu:\mathcal{A}\rightarrow\mathbb{R}$ such that
        $\mu(A)\geq{0}$, $\mu(\emptyset)=0$, and given a countable
        collection of disjoint sets $A_{i}\in\mathcal{A}$,
        $\mu(\cup_{i=1}^{\infty}A_{i})=\sum_{n=1}^{\infty}\mu(A_{i})$.
        \begin{theorem}
            If $\Omega$ is a set and $\mathcal{A}$ is a
            $\sigma\text{-Algebra}$ on $\Omega$, and if
            $\mu:\mathcal{A}\rightarrow\mathbb{R}$ is a function such that:
            \begin{enumerate}
                \item $\mu(A)\geq{0}$
                \item $\mu(\emptyset)=0$
                \item $\mu$ is finitely additive
                \item $\mu$ is continuous from below
            \end{enumerate}
            Then $\mu$ is a measure.
        \end{theorem}
        \begin{proof}
            All that is necessary to show is countable additivity.
            Let $A_{n}$ be a countable collection of disjoint elements
            of $\mathcal{A}$, and let $B_{n}=\cup_{k=1}^{n}A_{k}$.
            Then:
            \begin{align}
                \mu(B_{n})&=\mu(\cup_{n=1}^{n}A_{k})\\
                &=\sum_{k=1}^{n}\mu(A_{k})
            \end{align}
            But by definition, for all $n\in\mathbb{N}$,
            $B_{n}\subseteq{B_{n+1}}$, and therefore by continuity from
            below, we have:
            \begin{equation}
                \mu(\cup_{n=1}^{\infty}B_{k})
                =\lim_{n\rightarrow\infty}\mu(\cup_{k=1}^{n}B_{k})
            \end{equation}
            And therefore
            \begin{equation}
                \mu(\cup_{n=1}^{\infty}A_{k})=
                \sum_{k=1}^{\infty}\mu(A_{k})
            \end{equation}
        \end{proof}
        \subsection{A Review of Continuous Functions}
            Consider a function $f:\mathbb{R}\rightarrow\mathbb{R}$.
            Such a function is continuous at a point
            $x_{0}\in\mathbb{R}$ if, for all $\varepsilon>0$ there exists
            $\delta>0$ such that, for all $x\in\mathbb{R}$ such that
            $|x-x_{0}|>\delta$, we have that
            $|f(x)-f(x_{0})|<\varepsilon$. We can write this
            equivalently by saying that for all
            $x\in(x_{0}-\delta,x_{0}+\delta)$, it is true that
            $f(x)\in(f(x_{0})-\varepsilon,f(x_{0})+\varepsilon)$.
            We can also write the following:
            \begin{equation}
                f\Big((x_{0}-\delta,x_{0}+\delta)\Big)\subseteq
                \Big((f(x_{0}-\varepsilon,f(x_{0})+\varepsilon)\Big)
            \end{equation}
            And finally, from this we can write:
            \begin{equation}
                (x_{0}-\delta,x_{0}+\delta)\subseteq
                f^{-1}\Big(
                    \big(f(x_{0})-\varepsilon,f(x_{0})+\varepsilon\big)
                \Big)
            \end{equation}
            This gives us the following theorem about continuous
            functions.
            \begin{theorem}
                A function $f:\mathbb{R}\rightarrow\mathbb{R}$ is
                continuous at every point $x\in\mathbb{R}$ if and only
                if for every open set $\mathcal{U}\subseteq\mathbb{R}$,
                the pre-image $f^{-1}(\mathcal{U})$ is an open subset
                of $\mathbb{R}$.
            \end{theorem}
            \begin{proof}
                Going one way, every open interval is an open set.
                Thus, if $f$ is such that for every open set
                $\mathcal{U}\subseteq\mathbb{R}$, the pre-image of
                $\mathcal{U}$ is also open, then the pre-image of every
                open interval is open. Thus, let $\varepsilon>0$ be
                given. Then
                $f^{-1}(f(x_{0}-\varepsilon,f(x_{0})+\varepsilon)$ is
                open. But as $x_{0}$ is contained in this set, and as
                this set is open, there is a $\delta>0$ such that:
                \begin{equation}
                    (x_{0}-\delta,x_{0}+\delta)\subseteq
                    f^{-1}\big(
                        f(x_{0}-\varepsilon,f(x_{0})+\varepsilon\big)
                \end{equation}
                Thus, for all $x$ such that
                $|x-x_{0}|<\delta$, $|f(x)-f(x_{0})|<\varepsilon$.
                Now for the other direction. Suppose
                $f:\mathbb{R}\rightarrow\mathbb{R}$ is continuous at
                every point $x\in\mathbb{R}$ and let
                $\mathcal{U}\subseteq\mathbb{R}$ be an open set, and
                let $\mathcal{V}=f^{-1}(\mathcal{U})$. Then, for all
                $x\in\mathcal{V}$, $f(x)\in\mathcal{U}$. But, as
                $\mathcal{U}$ is open, there is a $\varepsilon>0$
                such that
                $(f(x)-\varepsilon,f(x)+\varepsilon)\subseteq\mathcal{U}$.
                But, as $f$ is continuous, there is a $\delta>0$ such
                that, for for all $|x-x_{0}|<\delta$,
                $|f(x)-f(x_{0})|<\varepsilon$. But then
                $(x-\delta,x+\delta)\subseteq\mathcal{V}$, and thus
                $\mathcal{V}$ is open.
            \end{proof}
            In a course on topology, one takes this as the definition of
            continuity.
            \newpage
            \begin{ldefinition}{Topology}
                A topology on a set $X$ is a subset
                $\tau\subseteq\mathcal{P}(X)$ such that:
                \begin{enumerate}
                    \item $\emptyset\in\tau$
                    \item $X\in\tau$
                    \item For any arbitrary collection of elements of
                          $\tau$, the union is an element of $\tau$.
                    \item For any finite collection of elements in
                          $\tau$, the intersection is an element of
                          $\tau$.
                \end{enumerate}
            \end{ldefinition}
            \begin{ldefinition}
                  {Continuous Functions Between Topological Spaces}
                A continuous function from a topological space
                $(X,\tau_{X})$ to a topological space
                $(Y,\tau_{Y})$ is a function $f:X\rightarrow{Y}$ such that
                for all $\mathcal{U}\in\tau_{Y}$,
                $f^{-1}(\mathcal{U})\in\tau_{X}$.
            \end{ldefinition}
            The Borel $\sigma\text{-Algebra}$ on $\mathbb{R}$ is the
            smallest set that makes open sets, elements of the standard
            topology on $\mathbb{R}$, measurable. In an analogous manner
            to how continuous functions are defined for topological
            spaces, measurable functions can also be defined.
        \subsection{Measurable Functions}
            \begin{ldefinition}{Measurable Functions}
                A measure function from a measurable space
                $(A,\mathcal{A})$ to a measurable space
                $(B,\mathcal{B})$ is a function $f:A\rightarrow{B}$
                such that, for all $\mathcal{U}\in\mathcal{A}$,
                $f^{-1}(\mathcal{U})\in\mathcal{B}$.
            \end{ldefinition}
            That is, the pre-image of measurable sets is measurable.
            This is similar to continuous functions where the pre-image
            of open sets is open. Such functions are also called
            $\mathcal{A}-\mathcal{B}$ measurable.
            \begin{lexample}
                If $\mathcal{A}$ is a $\sigma\text{-Algebra}$ on $\Omega$,
                $\Omega$ and if $\mathcal{B}=\{\emptyset,\Omega\}$,
                then any function $f:\omega\rightarrow\Omega$ will be
                $\mathcal{A}-\mathcal{B}$ measurable. If
                $\mathcal{A}=\mathcal{P}(\Omega)$ and if
                $\mathcal{B}$ is a $\sigma\text{-Algebra}$ on $\Omega$,
                then again any function $f:\Omega\rightarrow\Omega$ will
                be $\mathcal{A}-\mathcal{B}$ measurable. There are similar
                notions in topology called the discrete and chaotic
                topologies which make all functions continuous.
            \end{lexample}
            \begin{theorem}
                If $A$ and $B$ are sets, if $\mathcal{B}$ is a
                $\sigma\text{-Algebra}$ on $B$, and if $f:A\rightarrow{B}$
                is a function, then the set $\mathcal{A}$ defined by:
                \begin{equation}
                    \mathcal{A}=
                    \{f^{-1}(\mathcal{U}):\mathcal{U}\in\mathcal{B}\}
                \end{equation}
                Is a $\sigma\text{-Algebra}$ on $A$.
            \end{theorem}
            \begin{proof}
                It is true that $\emptyset\in\mathcal{A}$, since
                $\empty\in\mathcal{B}$ and $f^{-1}(\emptyset)=\emptyset$.
                Also, $B\in\mathcal{B}$, and $A=f^{-1}(B)$, and therefore
                $A\in\mathcal{A}$. If $A\in\mathcal{A}$, then
                there is a $B\in\mathcal{B}$ such that
                $A=f^{-1}(B)$, and thus:
                \begin{equation}
                    A^{C}=f^{-1}(B)^{C}=f^{-1}(B^{C})
                \end{equation}
                But if $B\in\mathcal{B}$, then $B^{C}\in\mathcal{B}$,
                and thus $A^{C}\in\mathcal{A}$. Finally, for any
                countable collection of sets $A_{n}\in\mathcal{A}$,
                there is a countable collection of sets $B_{n}$ such that
                $A_{n}=f^{-1}(B_{n})$ But then:
                \begin{equation}
                    \cup_{n=1}^{\infty}A_{n}=
                    \cup_{n=1}^{\infty}f^{-1}(B_{n})
                    =f^{-1}(\cup_{n=1}^{\infty}B_{n})
                \end{equation}
                But $\mathcal{B}$ is a $\sigma\text{-Algebra}$, and
                thus $\cup_{n=1}^{\infty}B_{n}\in\mathcal{B}$. Therefore
                $\cup_{n=1}^{\infty}A_{n}\in\mathcal{A}$.
            \end{proof}
            It's worth noting that $\mathcal{A}$ is the smallest
            $\sigma\text{-Algebra}$ on $A$ that will make
            $f$ $\mathcal{A}-\mathcal{B}$ measurable. Removing any set
            from $\mathcal{A}$ will result in $f:A\rightarrow{B}$ being
            non-measurable with respect to $\mathcal{A}$ and
            $\mathcal{B}$.
            \begin{theorem}
                If $A$ and $B$ are sets, $f:A\rightarrow{B}$ a function,
                and if $\mathcal{A}$ is a $\sigma\text{-Algebra}$ on
                $A$, then the set $\mathcal{B}$ defined by:
                \begin{equation}
                    \mathcal{B}=
                    \{B\subset{B}:f^{-1}(B)\in\mathcal{A}\}
                \end{equation}
                Is a $\sigma\text{-Algebra}$ on $B$.
            \end{theorem}
            \begin{proof}
                $\emptyset$ and $B$ are elements since
                $f^{-1}(\emptyset)=\emptyset\in\mathcal{A}$, and
                $f^{-1}(B)=A\in\mathcal{A}$.
            \end{proof}
            \begin{theorem}
                If $f:\Omega\rightarrow\mathbb{R}$, if $a\in\mathbb{R}$,
                if $\mathcal{B}$ is the Borel $\sigma\text{-Algebra}$ on
                $\mathbb{R}$, and if $\mathcal{A}$ is defined by:
                \begin{equation}
                    \mathcal{A}=\{\omega\in\Omega:f(\omega)<a\}
                \end{equation}
                then $f$ is $\mathcal{A}-\mathcal{B}$ measurable.
            \end{theorem}
            \begin{theorem}
                If $\Omega$ is a set, and if $\mathcal{A}$ is a
                $\sigma\text{-Algebra}$ on $\Omega$, and if
                $f:\Omega\rightarrow\mathbb{R}$ is a function such that,
                for all $a\in\mathbb{R}$,
                $\{\omega\in\Omega:f(\omega)<a\}\in\mathcal{A}$, then
                $f$ is $\mathcal{A}-\mathcal{B}$ measurable, where
                $\mathcal{B}$ is the Borel $\sigma\text{-Algebra}$.
            \end{theorem}
            \begin{theorem}
                If $f:\mathbb{R}\rightarrow\mathbb{R}$ is continuous,
                then it is Borel measurable.
            \end{theorem}
            \begin{theorem}
                If $\Omega$ is a set, $\mathcal{A}$ is a
                $\sigma\text{-Algebra}$ on $\Omega$, and if
                $f:\Omega\rightarrow\mathbb{R}$ is
                $\mathcal{A}-\mathcal{B}$ measurable, where
                $\mathcal{B}$ is the Borel $\sigma\text{-Algebra}$, and
                if $g:\mathbb{R}\rightarrow\mathbb{R}$ is
                $\mathcal{B}-\mathcal{B}$ measurable, then
                $g\circ{f}:\Omega\rightarrow\mathbb{R}$ is
                $\mathcal{A}-\mathcal{B}$ measurable.
            \end{theorem}
            \begin{proof}
                For if $B\in\mathcal{B}$, then
                $g^{-1}(B)\in\mathcal{B}$, for $g$ is
                $\mathcal{B}-\mathcal{B}$ measurable. but then,
                as $f$ is $\mathcal{A}-\mathcal{B}$ measurable,
                $f^{-1}(g^{-1}(B))\in\mathcal{A}$. Therefore,
                $g\circ{f}$ is $\mathcal{A}-\mathcal{B}$ measurable.
            \end{proof}
            In particular, if we have two measurable functions on
            $\mathbb{R}$, then the composition of these two functions
            is also measurable. This is analogous to the fact that the
            composition of continuous functions is continuous. The sum,
            difference, and product of measurable functions is also
            measurable. We now define the Borel $\sigma\text{-Algebra}$ for
            $\mathbb{R}^{2}$. This is denoted $\mathcal{B}_{2}$. It is
            defined similarly to $\mathcal{B}$: It is the smallest
            $\sigma\text{-Algebra}$ that contains all open subsets of
            $\mathbb{R}^{2}$. We can also limit this to all open rectangles
            in the plane, or all open discs.
            \begin{theorem}
                If $\Omega$ is a set, $\mathcal{A}$ a $\sigma\text{-Algebra}$
                on $\Omega$, if $f,g:\Omega\rightarrow\mathbb{R}$ are
                $\mathcal{A}-\mathcal{B}$ measurable functions, and if
                $\vec{h}:\Omega\rightarrow\mathbb{R}^{2}$ is defined by
                $\vec{h}(\boldsymbol{\omega})=(f(\omega),g(\omega))$,
                then $\vec{h}$ is $\mathcal{A}-\mathcal{B}_{2}$ measurable.
            \end{theorem}
            This theorem goes the other way as well. $\vec(h)$ is
            measurable if and only if $f$ and $g$ are measurable.
            \begin{theorem}
                If $\Omega$ is a set, $\mathcal{A}$ a $\sigma\text{-Algebra}$
                on $\Omega$, if $\mathcal{B}$ is the Borel
                $\sigma\text{-Algebra}$ on $\mathbb{R}$, and if
                $f,g:\Omega\rightarrow\mathbb{R}$ are
                $\mathcal{A}-\mathcal{B}$ measurable functions, then
                $f+g$ is $\mathcal{A}-\mathcal{B}$ measurable.
            \end{theorem}
            \begin{proof}
                For let $\varphi:\mathbb{R}^{2}\rightarrow\mathbb{R}$
                be defined by $\varphi(x,y)=x+y$. Then
                $\varphi$ is continuous, and is therefore
                $\mathcal{B}_{2}-\mathcal{B}$ measurable. Let
                $h:\Omega\rightarrow\mathbb{R}^{2}$ be defined by
                $h(\omega)=(f(\omega),g(\omega))$. Then $h$ is
                $\mathcal{A}-\mathcal{B}_{2}$ measurable. But by taking
                the composition, we have that
                $f+g=\varphi\circ{h}$ is $\mathcal{A}-\mathcal{B}$ measurable.
            \end{proof}
            We can do the same thing with multiplication by defining
            $\varphi(x,y)=x\cdot{y}$.
            \begin{theorem}
                If $f,g:\Omega\rightarrow\mathbb{R}$ are
                $\mathcal{A}-\mathcal{B}$ measurable, and if:
                \begin{equation}
                    A=\{\omega\in\Omega:f(\omega)<g(\omega)\}
                \end{equation}
                $\Pi$ open set (Half plane along diagonal. Draw this).
                Do the same thing with the line $L$. They're measurable,
                yadda yadda.
            \end{theorem}
        \subsection{Sequences of Measurable Functions}
            If $f_{n}:\omega\rightarrow\mathbb{R}$ is a sequence of
            $\mathcal{A}-\mathcal{B}$ measurable functions, and if
            $f_{n}\rightarrow{f}$, then $f$ is $\mathcal{A}-\mathcal{B}$
            measurable. 
            \begin{theorem}
                Let $F(\omega)=\sup\{f_{n}(\omega):n\in\mathbb{N}\}$.
                Then $F$ is measurable.
            \end{theorem}
            \begin{proof}
                For let $a\in\mathbb{R}$. Then:
                \begin{equation}
                    \{\omega:F(\omega)\leq{a}\}=
                    \bigcap_{n=1}^{\infty}\{\omega:f_{n}(\omega)\leq{a}\}
                \end{equation}
                Then $F(\omega)\leq{a}$ if and only if
                $f_{n}(\omega)\leq{a}$ for all $n\in\mathbb{N}$.
            \end{proof}
            Similarly, $F(\omega)=\inf\{f_{n}(\omega)\}$ is measurable.
            \begin{theorem}
                If $f_{n}:\Omega\rightarrow\mathbb{R}$ are
                $\mathcal{A}-\mathcal{B}$ functions, then
                $\underset{n\rightarrow{\infty}}{\overline{\lim}}f_{n}$ and
                $\underset{n\rightarrow{\infty}}{\underline{\lim}}f_{n}$
                are $\mathcal{A}-\mathcal{B}$ measurable.
            \end{theorem}
    \section{Convergence of Measurable Functions}
        \begin{ldefinition}{Measure Space}
            A Measure Space on a set $\Omega$, denoted
            $(\Omega,\mathcal{A},\mu)$, is a
            $\sigma\textrm{-Algebra}$ on $\Omega$ and a
            measure $\mu:\Omega\rightarrow\mathbb{R}$.
        \end{ldefinition}
        A function $f:\Omega\rightarrow\mathbb{R}$ is
        $\mathcal{A}-\mathcal{B}$ measurable, or simply
        measurable, if for all $B\in\mathcal{B}$, the
        pre-image is in $\mathcal{A}$. That is,
        $f^{-1}(B)\in\mathcal{A}$.
        \begin{theorem}
            If $f,g:\Omega\rightarrow\mathbb{R}$ are
            measurable functions, and if
            $A=\{\omega\in\Omega:f(\omega)\ne{g}(\omega)\}$,
            then $A\in\mathcal{A}$.
        \end{theorem}
        \begin{definition}
            Two functions are equal $\mu$-Almost everywhere
            if the following is true:
            \begin{equation}
                \mu(\omega\in\Omega:f(\omega)\ne{g}(\omega))
                =0
            \end{equation}
        \end{definition}
        \begin{definition}
            A sequence of measurable functions
            $f_{n}:\mathbb{R}\rightarrow\mathbb{R}$ converges
            to a function $f:\Omega\rightarrow\mathbb{R}$
            if there is a set $E$ such that $\mu(E^{C})=0$,
            and $f_{n}(\omega)\rightarrow{f}(\omega)$ for all
            $\omega\in{E}$.
        \end{definition}
        \begin{theorem}
            If $f_{n}\rightarrow{f}$ almost every and
            $f_{n}\rightarrow{g}$ almost everywhere, then
            $f=g$ almost everywher.
        \end{theorem}
        \begin{theorem}
            If $f_{n}\rightarrow{f}$ almost everywhere, and
            if $f=g$ almost everywhere, then
            $f_{n}\rightarrow{g}$ almost everywhere.
        \end{theorem}
        \begin{definition}
            A function $f:\Omega\rightarrow\mathbb{R}$
            converges to $f:\Omega\rightarrow\mathbb{R}$
            uniformly if for all $\varepsilon>0$ there is
            an $N\in\mathbb{N}$ such that, for all
            $\omega\in\Omega$,
            $|f(\omega)-f_{n}(\omega)|<\varepsilon$.
        \end{definition}
        \begin{example}
            Consider $f_{n}(\omega)=\omega^{n}$ for
            $\omega\in[0,a]$, where $a<1$. Then this
            converges uniformly to zero since, all all
            $\omega\in[0,a]$:
            \begin{equation}
                |\omega^{n}-0|=\omega^{n}\leq{a}^{n}
            \end{equation}
            But since $0\leq{a}<1$, $a^{n}$ converges to
            zero. Thus $f_{n}\rightarrow{0}$ uniformly.
        \end{example}
        We can define non-uniform converges by considering
        the logical negation of the definition for
        uniform convergence, but we can simplify this as
        well.
        \begin{theorem}
            A sequence of functions $f_{n}$ converges
            non-uniformly to a function $f$ if
            $f_{n}\rightarrow{f}$ point-wise, and there
            exists a $\delta>0$, a strictly increasing
            sequence $n_{k}$, and a sequence $\omega_{k}$
            such that
            $|f_{n_{k}}(\omega_{k})-f(\omega_{k})|>\delta$
        \end{theorem}
        \begin{example}
            If we define $f_{n}(\omega)=\omega^{n}$ on the
            interval $[0,1]$, then the convergence is no
            longer uniform. Indeed, the limit is
            discontinuous.
        \end{example}
        \begin{definition}
            A sequence $f_{n}$ converges to $f$ almost
            uniformly if, for all $\varepsilon>0$ there is
            a set $E$ such that $\mu(E^{C})<\varepsilon$,
            and $f_{n}$ converges to $f$ uniformly on
            $E$.
        \end{definition}
        \begin{example}
            If we again let $f_{n}(\omega)=\omega^{n}$ on
            $[0,1]$, then $f_{n}\rightarrow{0}$
            almost uniformly. For let $\varepsilon>0$. Then
            $f_{n}\rightarrow{0}$ uniformly on the set
            $[0,1-\varepsilon]$, and the measure of the
            compliment of this is less than$ \varepsilon$.
        \end{example}
        There is a difference between convergence almost
        everywhere and convergence almost uniformly. For
        convergence almost everywhere, we may remove a set
        of measure zero and expect that there is point-wise
        convergence on the remaining set. For almost uniform
        convergence we may remove a set of arbitrarily small
        measure, but not necessarily measure zero, and expect
        uniform convergence on the remaining set.
        \begin{theorem}
            If $f_{n}\rightarrow{f}$ almost uniformly,
            then $f_{n}\rightarrow{f}$ almost everywhere.
        \end{theorem}
        \begin{proof}
            For all $n\in\mathbb{N}$ there is a set
            $E_{n}$ such that $\mu(E_{n}^{C})<1/n$, and
            $f_{n}\rightarrow{f}$ uniformly on $E_{n}$.
            But then $f_{n}\rightarrow{f}$ on
            $\cup_{n=1}^{\infty}E_{n}$. But the complement
            of this set has measure zero. Therefore, etc.
        \end{proof}
        The converse is not true, in general. For let
        $f_{n}(\omega)$ be defined as follows:
        \begin{equation}
            f_{n}(\omega)=
            \begin{cases}
                0,&\omega\leq{n}\\
                1,&\omega>n
            \end{cases}
        \end{equation}
        The $f_{n}\rightarrow{0}$ almost everywhere, and
        indeed $f_{n}\rightarrow{0}$ point-wise. But
        the convergence is not uniform, nor is it
        almost uniform. There is no way to remove a set of
        finite measure and have uniform convergence on the
        resulting set. Similar to where continuity from above
        failed, the fact that $\mu(\mathbb{R})$ is infinite
        is why this failed. If we can limit the measure on
        the set, then convergence almost everywhere implies
        almost uniform convergence.
        \begin{ftheorem}{Egorov's Theorem}
                        {Measure_Theory_Egorov_Theorem}
            If $(\Omega,\mathcal{A},\mu)$ is a measure
            space and if $\mu(\Omega)<\infty$, then
            convergence $\mu$-almost everywhere implies
            $\mu$-almost uniform convergence.
        \end{ftheorem}
        \begin{bproof}
            For if $f_{n}(\omega)\rightarrow{f(\omega)}$
            $\mu$-almost everywhere, then there is a set
            $E$ such that $\mu(E^{C})=0$ and
            $f_{n}(\omega)\rightarrow{f(\omega)}$ for all
            $\omega\in{E}$. This means that for all
            $\delta>0$ and for all $\omega\in{E}$ there is
            an $N\in\mathbb{N}$ such that for all $n>N$,
            $|f_{n}(\omega)-f(\omega)|<\delta$. Let $A_{nm}$
            be defined as:
            \begin{equation}
                A_{Nm}=\bigcup_{n=N}^{\infty}\Big\{
                    \omega\in\Omega:
                    |f_{n}(\omega)-f(\omega)|\geq\frac{1}{m}|
                    \Big\}
            \end{equation}
            Define $B_{m}$ as:
            \begin{equation}
                B_{m}=\bigcap_{N=1}^{\infty}A_{Nm}
            \end{equation}
            But since $\mu(\Omega)<\infty$, the measure
            $\mu$ is continuous from above. Therefore:
            \begin{equation}
                \mu(B_{m})
                    =\underset{N\rightarrow\infty}{\lim}
                    \mu(A_{Nm})
            \end{equation}
            But $B_{m}\subseteq{E^{c}}$, and thus
            $\mu(B_{m})=0$. But then
            $\mu(A_{Nm})\rightarrow{0}$.
        \end{bproof}
        So we have shown that, even though convergence
        almost everywhere and almost uniform convergence
        are diferent concepts, on sets of finite measure
        they are equivalent. In probably the total measure
        of the entire set if 1, and so finite. Thus, in
        probabability spaces, almost everywhere convergence
        and almost uniform convergence will always be
        equivalent. Thus it is common to use the term
        convergence almost surely, and forgot the differences
        between the two properties. There is a third type of
        convergence called convergence in measure.
        \begin{ldefinition}{Convergence in Measure}
            A sequence of functions $f_{n}$ convergence
            in measure to $f$ is, for all $\delta>0$, the
            following is true:
            \begin{equation}
                \underset{n\rightarrow\infty}{\lim}
                \mu\Big(
                    \big\{
                        \omega:|f_{n}(\omega)-f(\omega)|
                        <\delta
                    \big\}
                \Big)=0
            \end{equation}
        \end{ldefinition}
        \begin{example}
            Let $\Omega=[0,1]$, and let $\mathcal{B}$ be
            the Borel $\sigma-\textrm{Algebra}$ on
            $[0,1]$. Finally, let $\mu$ be the standard
            Lebesgue-Measure. Define the following:
            \begin{equation}
                f_{1}=
                \begin{cases}
                    1,&x<\frac{1}{2}\\
                    0,&x\geq\frac{1}{2}
                \end{cases}
            \end{equation}
            Define $f_{2}=1-f_{1}$. The split the
            interval into fourths and define
            $f_{3}$ as 1 in $[0,1/4)$ and zero otherwise,
            and continue the pattern for $f_{4}$, $f_{5}$,
            $f_{6}$, and $f_{7}$. This sequence of functions
            converges nowhere since there will be 1's and
            0's oscillating back and forth, and thus there
            is no limit. However, $f_{n}$ converges in
            measure to 0.
        \end{example}
        \begin{theorem}
            If $f_{n}\rightarrow{f}$ in measure $\mu$,
            and if $g=f$ almost everywhere, then
            $f_{n}\rightarrow{g}$ in measure $\mu$.
        \end{theorem}
        \begin{proof}
            For all $\delta>0$,
            $\mu(\{\omega:|f_{n}(\omega)-f(\omega)|>\delta\}$
            tends to zero as $n\rightarrow\infty$. But:
            \begin{equation}
                \begin{split}
                    \{\omega:|f_{n}(\omega)-f(\omega)|
                    &>\delta\}\\
                    &=\Big(
                        \{\omega:|f_{n}(\omega)-f(\omega)|>\delta\}
                        \bigcap
                        \{\omega:f(\omega)=g(\omega)\}
                    \Big)\\
                    &\bigcup\Big(
                        \{\omega:|f_{n}(\omega)-f(\omega)|>\delta\}
                        \bigcap
                        \{\omega:f(\omega)\ne{g}(\omega)\}
                    \Big)
                \end{split}
            \end{equation}
        \end{proof}
        \begin{theorem}
            If $f_{n}\rightarrow{f}$ in measure $\mu$ and if
            $f_{n}\rightarrow{g}$ in measure $\mu$, and
            $f=g$ $\mu$ almost everywhre.
        \end{theorem}
        \begin{proof}
            For let $A=\{\omega:f(\omega)\ne{g}(\omega)\}$.
            Then $A=\{\omega:|f(\omega)-g(\omega)|>0\}$.
            Thus we may write:
            \begin{equation}
                A=\bigcup_{n=1}^{\infty}
                \Big\{\omega:|f(\omega)-g(\omega)|>\frac{1}{n}
                \Big\}
            \end{equation}
            We now show that
            $\{\omega:|f(\omega)-g(\omega)|>\frac{1}{n}\}$
            has measure zero for all $n\in\mathbb{N}$. By
            subadditivity, this will imply $A$ has measure zero.
            From the triangle inequality:
            \begin{equation}
                |f(\omega)-g(\omega)|\leq
                |f(\omega)-f_{n}(\omega)|+
                |g(\omega)-f_{n}(\omega)|
            \end{equation}
            If $|f(\omega)-g(\omega)|\geq{1/m}$, then at
            least one of the two numbers here must be greater
            than $1/2m$. Thus, either
            $|f(\omega)-f_{n}(\omega)|\geq\frac{1}{2m}$ or
            $|g(\omega)-f_{n}(\omega)|\geq\frac{1}{2m}$, or
            both. Therefore:
            \begin{equation}
                \{\omega:|f(\omega)-g(\omega)|>\frac{1}{n}\}
                \subseteq
                \{\omega:|f(\omega)-f_{n}(\omega)|>\frac{1}{2n}
                \}\bigcup
                \{\omega:|g(\omega)-f_{n}(\omega)|>\frac{1}{2n}
                \}
            \end{equation}
            But the two sets on the left have measures that
            tend to zero as $n\rightarrow\infty$, and thus
            the set on the left has measure zero. Thus, by
            subadditivity their union has measure zero, and
            therefore $\mu(A)=0$. Thus, $f=g$ $\mu$ almost
            everywhere.
        \end{proof}
        \begin{theorem}
            If $f_{n}\rightarrow{f}$ in measure $\mu$ and
            $g_{n}\rightarrow{g}$ in measure $\mu$, then
            $f_{n}+g_{n}\rightarrow{f+g}$ in measure $\mu$.
        \end{theorem}
        \begin{theorem}
            If $f_{n}\rightarrow{f}$ in measure $\mu$, then
            there exists a subsequence of $f_{n}$ that converges
            to $f$ almost uniformly.
        \end{theorem}
        \begin{proof}
            For all $\delta>0$ the limit of
            $\mu(\{|f_{n}-f|\geq\delta\})$ tends to zero
            as $n\rightarrow\infty$. Thus, there is an
            index $n_{1}$ such that
            $\mu(\{|f_{n_{1}}-f|\geq{1}\})<1$. Choosing
            $\delta=1/2$, we find an index $n_{2}$ such that
            $\mu(\{|f_{n_{2}}-f|\geq1/2\})<1/2$.
            Carrying on, we obtain a sequence $n_{k}$ such
            that, for all $k\in\mathbb{N}$,
            $\mu(\{|f_{n_{k}}-f|\geq1/k\})<1/2^{k}$.
            Let $E_{k}=\{|f_{n_{k}}-f|\geq1/k\}^{C}$.
            $\mu(E_{k}^{C})<1/2^{k}$, and thus for all
            $\varepsilon>0$ there is an $N\in\mathbb{N}$
            such that, for all $n>N$,
            $\mu(E_{n}^{C})<\varepsilon$.
        \end{proof}
    \section{Lecture 6}
        The $\sigma-\textrm{Algebra}$ generated by
        a set $\mathcal{E}$ is the intersection of all
        possible $\sigma-\textrm{Algebra's}$ that contain
        all elements of $\mathcal{E}$. Given a function
        $f:\Omega\rightarrow\mathbb{R}$ and a
        $\sigma-\textrm{Algebra}$ $\mathcal{A}$ on
        $\Omega$, it is often a good strategy to look at
        the set:
        \begin{equation}
            \mathcal{B}_{f,\mathcal{A}}=
                \big\{
                    B\subseteq\mathbb{R}:
                    f^{-1}(B)\in\mathcal{A}
                \big\}
        \end{equation}
        And then show that all intervals of the form
        $(a,b)$ are contained within
        $\mathcal{B}_{f,\mathcal{A}}$, thus implying
        that $f$ is $\mathcal{A}-\mathcal{B}$ measurable,
        where $\mathcal{B}$ is the Borel
        $\sigma-\textrm{Algebra}$ on $\mathbb{R}$.
        Recapping, we have now discussed three different
        types of convergence: Almost uniform convergence,
        convergence almost everywhere, and convergence in
        measure.
        \begin{theorem}
            If $f_{n}\rightarrow{f}$ almost uniformly,
            then $f_{n}\rightarrow{f}$ in measure.
        \end{theorem}
    \section{Integration}
        The integral of a constant function on an interval
        $[a,b]$ is defined by the signed area under the
        rectangle formed by the function. That is, if
        $f(x)=C$, we define the integral on $[a,b]$ to
        be $C(b-a)$. Given a piece-wise constant function,
        we can define the integral as the sum over the
        various regions. Given an arbitrary function, the
        only reasonable way to define the integral is to
        take a limit of approximations using piece-wise
        constant functions. This is how the Riemann integral
        is defined. The Riemann integral makes sense if
        the given function is continuous. By making the
        partition small enough, approximating a continuous
        function on a small interval by a constant can be
        reasonable. But consider the function:
        \begin{equation}
            f(x)=
            \begin{cases}
                0,&x\notin\mathbb{Q}\\
                1,&x\in\mathbb{Q}
            \end{cases}
        \end{equation}
        Given any interval $(a,b)$, $f$ takes on the values
        0 and 1 and thus it is not reasonable to approximate
        this function by a constant anywhere. However,
        the measure, or length, of $\mathbb{Q}$ is zero, and
        the height of the function on $\mathbb{Q}$ is 1.
        Thus it may be reasonable to define the area under
        this function as zero. While the Riemann integral
        cannot handle such function, the Lebesgue integral
        can. Given a measurable space
        $(\Omega,\mathcal{A},\mu)$, and a measurable
        function $f:\Omega\rightarrow\mathbb{R}$, where
        $f$ is $\mathcal{A}-\mathcal{B}$ measurable,
        $\mathcal{B}$ being the Borel
        $\sigma-\textrm{Algebra}$, it is possible to define
        the integral of $f$.
        \begin{ldefinition}
              {Support of a Real Valued Function}
            The support of a real-valued function
            $f:\Omega\rightarrow\mathbb{R}$ is the set:
            \begin{equation}
                \supp(f)=\{\omega\in\Omega:f(\omega)\ne{0}\}
            \end{equation}
        \end{ldefinition}
        \begin{ldefinition}{Simple Function}
            A simple function from a measurable space
            $(\Omega,\mathcal{A},\mu)$ to the real line
            $\mathbb{R}$ is a function
            $f:\Omega\rightarrow\mathbb{R}$ such that
            $f$ is $\mathcal{A}-\mathcal{B}$ measurable,
            where $\mathcal{B}$ is the Borel
            $\sigma-\textrm{Algebra}$, the range of $f$,
            $f(\Omega)$, is finite, and the measure of the
            support of $f$ is finite.
        \end{ldefinition}
        \begin{lexample}
            Let $(\Omega,\mathcal{A},\mu)$ be a measurable
            space, and let $E\subseteq\Omega$ be measurable
            and of finite measure. Define the following:
            \begin{equation}
                \chi_{E}(\omega)=
                \begin{cases}
                    0,&\omega\notin{E}\\
                    1,&\omega\in{E}
                \end{cases}
            \end{equation}
            This is called the indicator function of
            $E$. It is a simple function on the measurable
            space $(\Omega,\mathcal{A},\mu)$ since
            $\mu(E)<\infty$. To see that it is measurable,
            note that the pre-image is either
            $\emptyset$, $E$, $E^{C}$, or $\Omega$, and thus
            $\chi_{E}$ is measurable. Finally, it takes on
            only two values and thus it's range is finite.
        \end{lexample}
        \begin{lexample}
            Let $(\Omega,\mathcal{A},\mu)$ be a measurable
            space, and let $B_{1},\dots,B_{n}$ be measurable
            subsets of $\Omega$. Furthermore, let
            $a_{1},\dots,a_{n}$ be real numbers. If we let
            $\chi_{B_{i}}$ denote that indicator
            function of $B_{i}$, then we see that their sum
            is also a simple function. That is, define the
            following:
            \begin{equation}
                f(\omega)=\sum_{k=1}^{n}
                    a_{k}\chi_{B_{k}}(\omega)
            \end{equation}
            THen $f:\Omega\rightarrow\mathbb{R}$ is a simple
            function. Since the sum of measurable functions
            is measurable, we see that $f$ is measurable.
            We also have that the support is finite since:
            \begin{equation}
                \supp(f)\subseteq\bigcup_{k=1}^{n}B_{k}
            \end{equation}
            Finally, there are $2^{n}$ ways, at most, to
            combine the various real numbers
            $a_{1},\dots,a_{n}$, and thus the range of
            $f$ has, at most, $2^{n}$ elements. Therefore
            $f$ is simple.
        \end{lexample}
        Now, let $(\Omega,\mathcal{A},\mu)$ be a
        measurable space, and let
        $f:\Omega\rightarrow\mathbb{R}$ be a simple function.
        If $f$ is simple than it's range is finite. Let
        $a_{1},\cdots,a_{n}$ be the distinct elements of
        $f(\Omega)$ and define the following:
        \begin{equation}
            E_{k}=f^{-1}\big(\{a_{k}\}\big)
        \end{equation}
        Since $f$ is simple, it is measurable, and thus
        $E_{k}\in\mathcal{A}$ for all $k\in\mathbb{Z}_{n}$.
        Moreover, for $i\ne{j}$,
        $E_{i}\cap{E}_{j}=\emptyset$. But, since $f$ is
        simple, the measure of its support is finite, and
        thus for all $k\in\mathbb{Z}_{n}$,
        the masure of $E_{k}$ is also finite.
        We can thus obtain the following for $f$:
        \begin{equation}
            f(\omega)=
                \sum_{k=1}^{n}a_{k}\chi_{E_{k}}(\omega)
        \end{equation}
        This is called the \textrm{Canonical Representation}
        of $f$.
        \begin{ldefinition}{Integral of a Simple Function}
            The integral of a simple function
            $f:\Omega\rightarrow\mathbb{R}$ on a measurable
            space $(\Omega,\mathcal{A},\mu)$ with canonical
            form:
            \begin{equation}
                f(\omega)=\sum_{k=1}^{n}
                    a_{k}\chi_{E_{k}}(\omega)
            \end{equation}
            Is the real number $\int_{\Omega}f\diff{\mu}$
            define by:
            \begin{equation}
                \int_{\Omega}f\diff{\mu}=
                    \sum_{k=1}^{n}a_{k}\mu(E_{k})
            \end{equation}
        \end{ldefinition}
        The first thing to check to ensure that this is
        a good definition of integration is that the
        sum of two integrals is the integral of the sum
        of the two functions. We first prove a result that
        will make this definition more flexible.
        \begin{theorem}
            If $(\Omega,\mathcal{A},\mu)$ is a measurable
            space, and if $B_{1},\dots,B_{n}$ are measurable
            subsets of $\Omega$ that are pairwise disjoint
            and such that, for all $k\in\mathbb{Z}_{n}$,
            $\mu(B_{k})<\infty$, if
            $\lambda_{1},\dots,\lambda_{n}$ are real
            numbers, and if $f:\Omega\rightarrow\mathbb{R}$
            is defined by:
            \begin{equation}
                f(\omega)=\sum_{k=1}^{n}
                    \lambda_{k}\chi_{B_{k}}(\omega)
            \end{equation}
            Then the integral of $f$ is given by:
            \begin{equation}
                \int_{\Omega}f\diff{\mu}
                    =\sum_{k=1}^{n}\lambda_{k}
                    \mu(B_{k})
            \end{equation}
        \end{theorem}
        \begin{proof}
            For every $\omega\in\Omega$ falls in only
            one of the $B_{k}$. But if $\omega\in{B}_{k}$,
            then $f(\omega)=\lambda_{k}$. Let
            $a_{1},\hdots,a_{n}$ be the distinct values of
            $f$ and defin $\mathcal{J}_{k}$ as:
            \begin{equation}
                \mathcal{J}_{k}=\{j:\lambda_{j}=a_{k}\}
            \end{equation}
            Also, let $\mathcal{J}_{0}$ be defined as:
            \begin{equation}
                \mathcal{J}_{0}=\{j:\lambda_{j}=0\}
            \end{equation}
            If $E_{k}=f^{-1}(\{a_{k}\}$, then:
            \begin{equation}
                E_{k}=\bigcup_{j\in\mathcal{J}_{k}}B_{j}
            \end{equation}
            But since the $B_{j}$ are pair-wise disjoint,
            we have that:
            \begin{equation}
                \mu(E_{k})=\sum_{j\in\mathcal{J}_{k}}
                    \mu(B_{j})
            \end{equation}
            But then:
            \begin{align}
                \int_{\Omega}f\diff{\mu}
                &=\sum_{k=1}^{n}a_{k}\mu(E_{k})\\
                &=\sum_{k=1}^{n}a_{k}\Big(
                        \sum_{j\in\mathcal{J}_{k}}\mu(B_{j})
                    \Big)\\
                &=\sum_{k=1}^{n}\sum_{j\in\mathcal{J}_{k}}
                    a_{k}\mu(B_{j})\\
                &=\sum_{k=1}^{n}\sum_{j\in\mathcal{J}_{k}}
                    \lambda_{j}\mu(B_{j})\\
                &=\sum_{k=1}^{m}\lambda_{k}\mu(B_{k})
            \end{align}
        \end{proof}
        This theorem will make it easier to prove the
        additive property of integrals. We are still only
        talking about the integral of simple functions.
        \begin{theorem}
            If $(\Omega,\mathcal{A},\mu)$ is a measurable
            space and $f:\Omega\rightarrow\mathbb{R}$ and
            $g:\Omega\rightarrow\mathbb{R}$ are simple
            functions, then:
            \begin{equation}
                \int_{\Omega}(f+g)\diff{\mu}
                =\int_{\Omega}f\diff{\mu}
                +\int_{\Omega}g\diff{\mu}
            \end{equation}
        \end{theorem}
        \begin{proof}
            For let $f$ and $g$ have the following
            canonical representations:
            \begin{align}
                f(\omega)&=\sum_{k=1}^{n}\alpha_{k}
                    \chi_{A_{k}}(\omega)\\
                g(\omega)&=\sum_{k=1}^{m}\beta_{k}
                    \chi_{B_{k}}(\omega)
            \end{align}
            Define the following:
            \begin{align}
                A_{n+1}
                =\Big(\bigcup_{k=1}^{m}B_{k}\Big)\setminus
                    \Big(\bigcup_{j=1}^{n}A_{j}\Big)\\
                B_{m+1}
                =\Big(\bigcup_{k=1}^{n}A_{k}\Big)\setminus
                    \Big(\bigcup_{j=1}^{m}B_{j}\Big)
            \end{align}
            From this, we have the following:
            \begin{equation}
                \bigcup_{k=1}^{n+1}A_{k}=
                \bigcup_{j=1}^{m+1}B_{j}
            \end{equation}
            Let $\alpha_{n+1}=\beta_{m+1}=0$. Then we have:
            \begin{align}
                f(\omega)&=\sum_{k=1}^{n+1}\alpha_{k}
                    \chi_{A_{k}}(\omega)\\
                g(\omega)&=\sum_{k=1}^{m+1}\beta_{k}
                    \chi_{B_{k}}(\omega)
            \end{align}
            Then, for all $i$, we have:
            \begin{align}
                A_{i}&=A_{i}\bigcap
                    \Big(\bigcup_{j=1}^{m+1}B_{j}\Big)\\
                &=\bigcup_{j=1}^{m+1}
                    \Big(A_{i}\cap{B}_{j}\Big)\\
                B_{i}&=B_{i}\bigcap
                    \Big(\bigcup_{k=1}^{n+1}A_{k}\Big)\\
                &=\bigcup_{k=1}^{n+1}
                    \Big(A_{k}\cap{B}_{i}\Big)
            \end{align}
            And these are all pair-wise disjoint sets.
            So, we have:
            \begin{align}
                f(\omega)&=
                \sum_{i=1}^{n+1}\sum_{j=1}^{m+1}\alpha_{i}
                    \chi_{A_{i}\cap{B}_{j}}(\Omega)\\
                g(\omega)&=
                \sum_{i=1}^{n+1}\sum_{j=1}^{m+1}\beta_{j}
                    \chi_{A_{i}\cap{B}_{j}}(\Omega)
            \end{align}
            Summing these two functions, we get:
            \begin{equation}
                f(\omega)+g(\omega)=
                \sum_{i=1}^{n+1}\sum_{j=1}^{m+1}
                    (\alpha_{i}+\beta_{j})
                    \chi_{A_{i}\cap{B}_{j}}(\Omega)
            \end{equation}
            But by the previous theorem:
            \begin{align}
                \int_{\Omega}(f+g)\diff{\mu}
                &=\sum_{i=1}^{n+1}\sum_{j=1}^{m+1}
                    (\alpha_{i}+\beta_{j})
                    \mu(A_{i}\cap{B}_{j})\\
                &=\sum_{i=1}^{n+1}\sum_{j=1}^{m+1}
                    \alpha_{i}\mu(A_{i}\cap{B}_{j})+
                    \sum_{i=1}^{n+1}\sum_{j=1}^{m+1}
                    \beta_{j}\mu(A_{i}\cap{B}_{j})\\
                &=\sum_{i=1}^{n+1}\alpha_{i}\mu(A_{i})+
                    \sum_{j=1}^{m+1}\beta_{j}\mu(B_{j})\\
                &=\int_{\Omega}f\diff{\mu}+
                    \int_{\Omega}g\diff{\mu}
            \end{align}
        \end{proof}
        The integrals of simple functions also have the
        property of homogeneity.
        \begin{theorem}
            If $(\Omega,\mathcal{A},\mu)$ is a measurable
            space, $f:\Omega\rightarrow\mathbb{R}$ is
            a simple function, and if $c\in\mathbb{R}$,
            then:
            \begin{equation}
                \int_{\Omega}cf\diff{\mu}=
                c\int_{\Omega}f\diff{\mu}
            \end{equation}
        \end{theorem}
        \begin{theorem}
            If $(\Omega,\mathcal{A},\mu)$ is a measurable
            space, $f:\omega\rightarrow\mathbb{R}$ is
            a simple function, if $A_{1},\dots,A_{n}$
            are measurable subsets of $\Omega$ with
            finite measure, and if $f$ is such that:
            \begin{equation}
                f(\omega)=\sum_{k=1}^{n}a_{k}
                    \chi_(A_{k})(\omega)
            \end{equation}
            Then:
            \begin{equation}
                \int_{\Omega}f\diff{\mu}=
                \sum_{k=1}^{n}a_{k}\mu(A_{k})
            \end{equation}
        \end{theorem}
        \subsection{Further Properties of the Integral}
            So far we have defined the integral of a simple
            function over the entire of a given space
            $\Omega$. We often wish to evaluate the integral
            of a function on a subset of $\Omega$, rather
            than the entire of it. We can do this by defining
            the following:
            \begin{equation}
                f_{E}(\omega)=
                \begin{cases}
                    f(\omega),&\omega\in{E}\\
                    0,&\omega\notin{E}
                \end{cases}
            \end{equation}
            We need some properties of $f_{E}$.
            $f_{E}$ is measurable, has finite range, and
            has support of finite measure, and is therefore
            simple. $f_{E}$ can have only one more value
            (That is, zero) than $f$. Finally,
            $\supp(f_{E})\subseteq{\supp(f)}$. We define
            the integral on $E\in\mathcal{A}$ as follows:
            \begin{equation}
                \int_{E}f\diff{\mu}=
                \int_{\Omega}f_{E}\diff{\mu}
            \end{equation}
            Since $f_{E}$ is also simple, the right hand
            side of this equation is well defined.
            \begin{theorem}
                If $(\Omega,\mathcal{A},\mu)$ is a measurable
                space, if $E_{1},E_{2}$ are disjoint
                measurable subsets of $\Omega$, and if
                $f:\Omega\rightarrow\mathbb{R}$ is a simple
                function, then:
                \begin{equation}
                    \int_{E_{1}\cup{E}_{2}}f\diff{\mu}=
                    \int_{E_{1}}f\diff{\mu}+
                    \int_{E_{2}}f\diff{\mu}
                \end{equation}
            \end{theorem}
            This is similar to a notion that is found when
            studying the Riemann integral. That is:
            \begin{equation}
                \int_{a}^{b}f(x)\diff{x}=
                \int_{a}^{c}f(x)\diff(x)+
                \int_{c}^{b}f(x)\diff{x}
            \end{equation}
            \begin{theorem}
                If if $f:\omega\rightarrow\mathbb{R}$
                is a simple function, and if
                $f(\omega)\geq{0}$ for all $\omega\in\Omega$,
                then:
                \begin{equation}
                    \int_{\Omega}f\diff{\mu}\geq{0}
                \end{equation}
            \end{theorem}
            \begin{theorem}
                If $f:\Omega\rightarrow\mathbb{R}$ is
                simple, then:
                \begin{equation}
                    \int_{\Omega}f\diff{\mu}=0
                \end{equation}
                If and only if $f=0$ $\mu$ almost everywhere.
            \end{theorem}
            \begin{ftheorem}
                  {Triangle Inequality
                   for Simple Functions}{}
                If $f$ is a simple function, then:
                \begin{equation}
                    \Big|\int_{\Omega}f\diff{\mu}\Big|
                    \leq\int_{\Omega}|f|\diff{\mu}
                \end{equation}
            \end{ftheorem}
        \subsection{Limit Theorems for Simple Functions}
            \begin{theorem}
                If $f_{n}:\Omega\rightarrow\mathbb{R}$ is
                a sequence of simple functions such that
                $f_{n}\rightarrow{f}$ uniformly,
                where $f$ is a simple function,
                and if there is a measurable set $E$
                such that $\supp(f_{n})\subseteq{E}$ and
                $\supp(f)\subseteq{E}$, and if
                $\mu(E)<\infty$, then:
                \begin{equation}
                    \underset{n\rightarrow\infty}{\lim}
                    \int_{\Omega}f_{n}\diff{\mu}
                    =\int_{\Omega}f\diff{\mu}
                \end{equation}
            \end{theorem}
            \begin{proof}
                For:
                \begin{align}
                    \Big|
                        \int_{\Omega}f_{n}\diff{\mu}-
                        \int_{\Omega}f\diff{\mu}
                    \Big|
                    &=\Big|\int_{\Omega}(f_{n}-f)\diff{\mu}
                        \Big|\\
                    &\leq\int_{\Omega}|f_{n}-f|\diff{\mu}\\
                    &=\int_{E}|f_{n}-f|\diff{\mu}\\
                    &\leq\int_{E}\varepsilon\diff{\mu}\\
                    &=\varepsilon\int_{E}\diff{\mu}\\
                    &=\varepsilon\mu(E)
                \end{align}
                Since $\mu(E)<\infty$, this can be made
                arbitrarily small.
            \end{proof}
            \begin{theorem}
                If $f_{n}\rightarrow{f}$ uniformly, and if
                $f_{n}$ and $f$ are simple, then
                $f_{n}$ is uniformly bounded.
            \end{theorem}
            \begin{theorem}[Bounded Convergence Theorem]
                If $f_{n}\rightarrow{f}$, $f_{n}$ are
                simple and $f$ is simple, if
                $f_{n}$ are uniformly bounded, and if
                there is a measurable set $E$ of finite
                measure such that $\supp(f_{n})\subseteq{E}$
                and $\supp(f)\subseteq{E}$, then:
                \begin{equation}
                    \underset{n\rightarrow\infty}{\lim}
                    \int_{\Omega}f_{n}\diff{\mu}
                    =\int_{\Omega}f\diff{\mu}
                \end{equation}
            \end{theorem}
            \begin{lexample}
                The additional assumptions are indeed
                necessary, and without them these results
                may not hold. For let $f_{n}$ be defined as:
                \begin{equation}
                    f_{n}(\omega)=
                    \begin{cases}
                        n,&0\leq\omega\leq\frac{1}{n}\\
                        0,&\textrm{Otherwise}
                    \end{cases}
                \end{equation}
                Then $\int_{\mathbb{R}}f_{n}\diff{\mu}=1$
                for all $n$, but the limit function is
                $f=0$, and this has integral zero. It is
                also not guarenteed that the results
                fail, for let:
                \begin{equation}
                    f_{n}(\omega)=
                    \begin{cases}
                        n,&0\leq\omega\leq\frac{1}{n^{2}}\\
                        0,&\textrm{Otherwise}
                    \end{cases}
                \end{equation}
                Then the limit function is zero, and the
                integral is $\frac{1}{n}$, which does indeed
                converge to zero. For the requirement that
                $\supp(f_{n})$ and $\supp(f)$ be contained in
                one set, consider the following function:
                \begin{equation}
                    f_{n}(\omega)=
                    \begin{cases}
                        1,&n\leq\omega\leq{n+1}\\
                        0,&\textrm{Otherwise}
                    \end{cases}
                \end{equation}
                Then $f_{n}$ is uniformly bounded,
                converges to $0$, but the integral is
                1 for all $n$.
            \end{lexample}
            \begin{theorem}[Monotone Convergence Theorem]
                If $f_{n}$, $f$ are simple functions, if
                $f_{n}\rightarrow{f}$, if $f_{n}\leq{f}_{n+1}$,
                then:
                \begin{equation}
                    \int_{\Omega}f_{n}\diff{\mu}
                    \rightarrow\int_{\Omega}f\diff{\mu}
                \end{equation}
            \end{theorem}
            \begin{theorem}
                \label{thm:MEASURE_THEORY_LIM_INT_MONO_SIMPLE_FUNCS}
                If $f_{n}$ and $g_{n}$ are simple and monotonically
                increasing, and if:
                \begin{equation}
                    \underset{n\rightarrow\infty}{\lim}f_{n}(\omega)
                    =\underset{n\rightarrow\infty}{\lim}g_{n}(\omega)
                \end{equation}
                Then:
                \begin{equation}
                    \underset{n\rightarrow\infty}{\lim}
                    \int_{\Omega}f_{n}\diff\mu
                    =\underset{n\rightarrow\infty}{\lim}
                    \int_{\Omega}g_{n}\diff\mu
                \end{equation}
            \end{theorem}
            This theorem will allow us to extend the definition
            of the integral to a more general class of functions.
        \subsection{Integration of Non-Negative Measurable Functions}
            Consider a measure space $(\Omega,\mathcal{A},\mu)$ and
            let $f:\Omega\rightarrow\mathbb{R}$ be an
            $\mathcal{A}-\mathcal{B}$ measurable function, where
            $\mathcal{B}$ is the Borel $\sigma\textrm{-Algebra}$ on
            $\mathbb{R}$. If there exist a sequence of simple
            functions $f_{n}$ that are monotonically increasing
            and such that $f_{n}\rightarrow{f}$, then we define the
            integral of $f$ as follows:
            \begin{equation}
                \int_{\Omega}f\diff{\mu}=
                \underset{n\rightarrow\infty}{\lim}
                \int_{\Omega}f_{n}\diff{\mu}
            \end{equation}
            Because of
            Thm.~\ref{thm:MEASURE_THEORY_LIM_INT_MONO_SIMPLE_FUNCS}
            this is a well defined concept, since for any
            two sequences of simples functions that are monotonically
            increasing and converge to $f$, the limit of the
            integrals is the same, thus giving a consitent definition
            to the integral of $f$. The first question that then
            arises is which measurable functions can be approximated
            arbitrarily well by a sequence of monotonically
            increasing simple functions? From the definition we will
            need that $f$ is bounded below. For now we will discuss
            measurable functions that are non-negative. Let
            $f:\Omega\rightarrow\mathbb{R}$ be any non-negative
            function, it need not be measurable. We wish to construct
            a sequence of simple functions $f_{n}$ such that
            $f_{n}$ is monotonically increasing, and for all
            $\omega\in\Omega$, $f_{n}(\omega)\rightarrow{f}(\omega)$.
            We construct such a sequence as follows, dividing
            the $[0,n)$ into $n2^{n}-1$ parts
            $[\frac{i}{2^{n}},\frac{i+1}{2^{n}})$ and define the
            following sets:
            \begin{align}
                E_{n}^{C}
                &=\{\omega\in\Omega:f(\omega)\geq{n}\}\\
                E_{n,i}
                &=\{\omega:\frac{i}{2^{n}}\leq{f}(\omega)
                    \leq\frac{i+1}{2^{n}}\}\\
                &=f^{-1}\big([\frac{i}{2^{n}},\frac{i+1}{2^{n}})\big)
            \end{align}
            Then, for every fixed $n\in\mathbb{N}$, the sets
            $E_{n,i}$ are pairwise disjoint. Defined $f_{n}$ as
            follows:
            \begin{equation}
                f_{n}(\omega)=
                \begin{cases}
                    n,&f(\omega)\geq{n}\\
                    \frac{i}{2^{n}},&\frac{i}{2^{n}}
                        \leq{f}(\omega)
                        \leq\frac{i+1}{2^{n}}
                \end{cases}
            \end{equation}
            Then, using the sets $E_{n,i}$ and $E_{n}^{C}$, we
            can rewrite $f_{n}$ as follows:
            \begin{equation}
                f_{n}(\omega)=
                n\chi_{E_{n}^{C}}(\omega)+
                \sum_{i=0}^{n2^{n}-1}
                    \frac{i}{2^{n}}\chi_{E_{n,i}}(\omega)
            \end{equation}
            We now have that $f_{n}$ is monotonically increasing
            and tends to $f$. For is $f(\omega)=\infty$, then:
            \begin{equation}
                w\in\cap_{n=1}^{\infty}E_{n}^{C}
                \Rightarrow
                f_{n}(\omega)=n\rightarrow\infty
            \end{equation}
            If $f(\omega)\in\mathbb{R}$, then there is an
            $N\in\mathbb{N}$ such that $f(\omega)<N$. But then, for
            all $n>N$, $f(\omega)<n$ and thus there is an
            $i\in\mathbb{Z}_{n2^{n}-1}$ such that:
            \begin{equation}
                \frac{i}{2^{n}}\leq{f}(\omega)\leq\frac{i+1}{2^{n}}
            \end{equation}
            But $f_{n}(\omega)=\frac{i}{2^{n}}$ and thus:
            \begin{equation}
                0\leq{f}(\omega)-f_{n}(\omega)\leq\frac{1}{2^{n}}
            \end{equation}
            And therefore $f_{n}\rightarrow{f}$. Finally,
            $f_{n}$ is monotonically increasing. Now let's see what
            we can add to this if we know that $f$ is measurable.
            Since $f$ is measurable, the pre-image
            $f^{-1}([n,\infty))\in\mathcal{A}$, since
            $[n,\infty)$ is a Borel set for all $n\in\mathbb{N}$.
            Moreover, for all $n$ and $i$, $E_{n,i}\in\mathcal{A}$.
            Then all of the indicator functions $\chi_{E_{n,i}}$ are
            measurable, and thus $f_{n}$ is measurable for all
            $n$. However, the support of the $f_{n}$ may not be
            finite. Simply take $f(\omega)=1$ for all $\omega$, and
            let $\Omega=\mathbb{R}$. If the $\mu(\Omega)<\infty$,
            then $\mu(\supp(f_{n}))<\infty$. This case is particularly
            important when studying probability theory.
            \begin{ldefinition}{$\sigma\textrm{-Finite}$}
                A $\sigma\textrm{-Finite}$ measure on a
                $\sigma\textrm{-Algebra}$ $\mathcal{A}$ of a set
                $\Omega$ is a measure $\mu$ such that there is a
                sequence of sets $\Omega_{n}$ such that
                $\Omega_{n}\subseteq\Omega_{n+1}$,
                $\Omega=\cup_{n=1}^{\infty}\Omega_{n}$, and for all
                $n\in\mathbb{N}$, $\mu(\Omega_{n})<\infty$.
            \end{ldefinition}
            \begin{lexample}
                Let $\Omega=\mathbb{R}$ and consider the Borel
                $\sigma\textrm{-Algebra}$ $\mathcal{B}$ on
                $\mathbb{R}$. Then the standard Lebesgue-Measure
                is $\sigma\textrm{-finite}$ since we can write:
                \begin{equation}
                    \mathbb{R}=\cup_{n=1}^{\infty}[-n,n]
                \end{equation}
                And $\mu([-n,n])=2n$, which is finite.
            \end{lexample}
            Define $\tilde{f}_{n}$ as follows:
            \begin{equation}
                f_{n}(\omega)
                =f_{n}(\omega)\cdot\chi_{\Omega_{n}}(\omega)
            \end{equation}
            Then we have that $\tilde{f}_{n}$ is simple, measurable,
            takes on finitely many values, and the measure of it's
            support is finite. Thus we have that if
            $(\Omega,\mathcal{A},\mu)$ is a measure space and if
            $\mu$ is $\sigma\textrm{-Finite}$, then for any
            non-negative measurable function
            $f:\Omega\rightarrow\mathbb{R}$, the integral of $f$ is
            well defined. We write:
            \begin{equation}
                \int_{\Omega}f\diff{\mu}=
                \underset{n\rightarrow\infty}{\lim}
                \int_{\Omega}f_{n}\diff{\mu}
            \end{equation}
        \subsection{Properties of the Integral
                    of Non-Negative Functions}
        \begin{theorem}[Homogeneity of the Integral]
            If $f$ is a non-negative measurable function, and if
            $c>0$, then:
            \begin{equation}
                \int_{\Omega}(cf)\diff{\mu}=
                c\int_{\Omega}f\diff{\mu}
            \end{equation}
        \end{theorem}
        \begin{proof}
            For let $f_{n}$ be a sequence of simple functions such
            that $f_{n}\rightarrow{f}$ and $f_{n}$ is monotonically
            increasing. Then:
            \begin{equation}
                \int_{\Omega}(cf)\diff{\mu}
                =\underset{n\rightarrow\infty}{\lim}
                \int_{\Omega}(cf_{n})\diff{\mu}
                =c\underset{n\rightarrow\infty}{\lim}
                \int_{\Omega}f_{n}\diff{\mu}
                =c\int_{\Omega}f\diff{\mu}
            \end{equation}
        \end{proof}
        \begin{theorem}[Additivity of the Integral]
            If $f$ and $g$ are non-negative and measurable,
            then:
            \begin{equation}
                \int_{\Omega}(f+g)\diff{\mu}
                =\int_{\Omega}f\diff{\mu}
                +\int_{\Omega}g\diff{\mu}
            \end{equation}
        \end{theorem}
        \begin{proof}
            For let $f_{n}$ and $g_{n}$ be simple functions such
            that $f_{n}\rightarrow{f}$, $g_{n}\rightarrow{g}$, and
            such that $f_{n}$ and $g_{n}$ are monotonically
            increasing. Then:
            \begin{align}
                \int_{\Omega}(f+g)\diff{\mu}
                &=\underset{n\rightarrow\infty}{\lim}
                \int_{\Omega}(f_{n}+g_{n})\diff{\mu}\\
                &=\underset{n\rightarrow\infty}{\lim}
                \int_{\Omega}f_{n}\diff{\mu}+
                \underset{n\rightarrow\infty}{\lim}
                \int_{\Omega}g_{n}\diff{\mu}\\
                &=\int_{\Omega}f\diff{\mu}+\int_{\Omega}g\diff{\mu}
            \end{align}
        \end{proof}
        \begin{theorem}
            If $f$ is non-negative and measurable, and if
            $E_{1},E_{2}$ are disjoint sets, then:
            \begin{equation}
                \int_{E_{1}\cup{E}_{2}}f\diff{\mu}
                =\int_{E_{1}}f\diff{\mu}+\int_{E_{2}}f\diff{\mu}
            \end{equation}
        \end{theorem}
        \begin{theorem}
            If $f$ is a non-negative measurable function, then:
            \begin{equation}
                \int_{\Omega}f\diff{\mu}\geq{0}
            \end{equation}
        \end{theorem}
        \begin{theorem}
            If $f$ is a non-negative measurable function, then:
            \begin{equation}
                \int_{\Omega}f\diff{\mu}=0
                \Longleftrightarrow{f=0}
                \quad\mu\textrm{-almost everywhere}
            \end{equation}
        \end{theorem}
        \begin{ldefinition}{Summable Functions}
            A non-negative summable function is a non-negative
            and measurable function from
            a measure space $(\Omega,\mathcal{A},\mu)$ where
            $\mu$ is $\sigma\textrm{-Finite}$ to $\mathbb{R}$
            such that:
            \begin{equation}
                \int_{\Omega}f\diff{\mu}<\infty
            \end{equation}
        \end{ldefinition}
        \begin{ltheorem}{Chebyshev's Inequality}
            If $f$ is a non-negative measurable function, then for
            all $a\in\mathbb{R}^{+}$:
            \begin{equation}
                \mu\Big(\{\omega:f(\omega)\geq{a}\}\Big)
                <\frac{1}{a}\int_{\Omega}f\diff{\mu}
            \end{equation}
        \end{ltheorem}
        \begin{proof}
            For define the following:
            \begin{align}
                E_{1}&=\{\omega:f(\omega)\geq{a}\}\\
                E_{2}&=\{\omega:f(\omega)<a\}
            \end{align}
            Then $E_{1}$ and $E_{2}$ are disjoint, and therefore:
            \begin{equation}
                \int_{\Omega}f\diff{\mu}=
                \int_{E_{1}}f\diff{\mu}+
                \int_{E_{2}}f\diff{\mu}
                \geq\int_{E_{1}}f\diff{\mu}
                \geq\int_{\Omega}a\diff{\mu}
                =a\mu(E_{1})
            \end{equation}
            Dividing by $a$ completes the proof.
        \end{proof}
        \begin{theorem}
            If $f$ is a non-negative summable function, then
            for all $a\in\mathbb{R}^{+}$:
            \begin{equation}
                \underset{a\rightarrow\infty}{\lim}
                \mu\Big(\{\omega:f(\omega)\geq{a}\}\Big)=0
            \end{equation}
        \end{theorem}
        \begin{ltheorem}{Monotone Convergence Theorem}
            If $f$ is a non-negative measurable function and if
            $f_{n}$ is a sequence of non-negative measurable
            functions that are monotonically increasing and such
            that $f_{n}\rightarrow{f}$, then:
            \begin{equation}
                \underset{n\rightarrow\infty}{\lim}
                \int_{\Omega}f_{n}\diff{\mu}
                =\int_{\Omega}f\diff{\mu}
            \end{equation}
        \end{ltheorem}
        \begin{proof}
            For all $n\in\mathbb{N}$, there is a function
            $g_{n,k}$ such that $g_{n,k}$ is simple and:
            \begin{equation}
                \int_{\Omega}f_{n}\diff{\mu}=
                \underset{k\rightarrow\infty}{\lim}
                \int_{\Omega}g_{n,k}\diff{\mu}
            \end{equation}
            Define $F_{n}$ as follows:
            \begin{equation}
                F_{n}(\omega)=
                \max\{g_{j,n}(\omega):
                    \omega\in\Omega,0\leq{j}\leq{n}\}
            \end{equation}
            Then, for all $n\in\mathbb{N}$, $F_{n}$ is simple.
            For it is the maximum of finitely many measurable
            functions, and is therefore measurable. Moreover:
            \begin{equation}
                \supp(F_{n})\subseteq
                \bigcup_{k=1}^{n}\supp(f_{k,n})
            \end{equation}
            And finally, $F_{n}$ is monotonically increasing from
            it's definition. Now we must show that
            $F_{n}\rightarrow{f}$. For:
            \begin{equation}
                g_{k,n}\leq{F}_{k}\leq{f}_{k}
            \end{equation}
            Since $g_{n,k}$ increases monotonically to $f_{n}$.
            Taking the limit as $k\rightarrow\infty$, we obtain:
            \begin{equation}
                f_{n}\leq\underset{k\rightarrow\infty}{\lim}F_{k}
                \leq{f}
            \end{equation}
            Then taking the limit on $n$, we see that
            $F_{n}\rightarrow{f}$. Integrating this inequality, we
            get:
            \begin{equation}
                \int_{\Omega}g_{k,n}\diff{\mu}
                \leq\int_{\Omega}F_{k}\diff{\mu}
                \leq\int_{\Omega}f\diff{\mu}
            \end{equation}
            Taking the limit as $k\rightarrow\infty$, we get:
            \begin{equation}
                \underset{k\rightarrow\infty}{\lim}
                \int_{\Omega}g_{k,n}\diff{\mu}
                \leq\int_{\Omega}f\diff{\mu}
                \leq\underset{k\rightarrow\infty}{\lim}
                \int_{\Omega}f_{k}\diff{\mu}
            \end{equation}
            Finally, taking the limit on $n$, we get:
            \begin{equation}
                \underset{n\rightarrow\infty}{\lim}
                \int_{\Omega}f_{n}\diff{\mu}
                \leq\int_{\Omega}f\diff{\mu}
                \leq\underset{k\rightarrow\infty}{\lim}
                \int_{\Omega}f_{k}\diff{\mu}
            \end{equation}
            This completes the proof.
        \end{proof}
    \section{Lecture 8}
        Let $(\Omega,\mathcal{A},\mu)$ be a measure space and let
        $f\geq{0}$ be measurable. From before we were able to define
        the integral of $f$ is $\mu$ is $\sigma\textrm{-finite}$. We
        approximate $f$ with an increasing sequence of simple functions
        that are also non-negative. The integral of $f$ is defined as
        the limit of the integrals of the approximating simple functions.
        That is, we define the integral to be:
        \begin{equation}
            \int_{\Omega}f\diff{\mu}=
            \underset{n\rightarrow\infty}{\lim}
            \int_{\Omega}f_{n}\diff{\mu}
        \end{equation}
        We have seen from a previous theorem that the value of the integral
        is independent of the approximating sequence. That is, for
        $f_{n}$ and $g_{n}$ are a sequence of simple functions that
        are monotonically increasing to $f$, then:
        \begin{equation}
            \underset{n\rightarrow\infty}{\lim}
            \int_{\Omega}f_{n}\diff{\mu}=
            \underset{n\rightarrow\infty}{\lim}
            \int_{\Omega}g_{n}\diff{\mu}
        \end{equation}
        We then proved the monotone convergence theorem.
        \begin{ltheorem}{Monotone Convergence Theorem}
            If $f_{n}$ is a sequence of positive measurable functions,
            not necessarily simple, and if $f_{n}$ is monotonically
            increasing, then:
            \begin{equation}
                \underset{n\rightarrow\infty}{\lim}
                \int_{\Omega}f_{n}\diff{\mu}
                =\int_{\Omega}
                \underset{n\rightarrow\infty}{\lim}f_{n}\diff{\mu}
            \end{equation}
        \end{ltheorem}
        Note that we are still only talking about non-negative measurable
        functions. We have yet to discuss functions that are possibly
        negative.
        \begin{ltheorem}{Fatou's Theorem}
            If $f_{n}$ is a sequence of non-negative measurable functions,
            then:
            \begin{equation}
                \int_{\Omega}
                \underset{n\rightarrow\infty}{\underline{\lim}}
                f_{n}\diff{\mu}
                \leq
                \underset{n\rightarrow\infty}{\underline{\lim}}
                \int_{\Omega}f_{n}\diff{\mu}
            \end{equation}
            Where $\underline{\lim}$ denotes the limit-inferior.
        \end{ltheorem}
        \begin{proof}
            For:
            \begin{equation}
                0\leq\inf_{k\geq{n}}f_{k}(\omega)
                \leq{f}_{k}(\omega)
            \end{equation}
            And therefore:
            \begin{equation}
                \int_{\Omega}\inf_{k\geq{n}}f_{k}\diff{\mu}
                \leq\int_{\Omega}f_{k}\diff{\mu}
            \end{equation}
            And therefore:
            \begin{equation}
                \int_{\Omega}\inf_{k\geq{n}}f_{k}\diff{\mu}
                \leq\inf_{k\geq{n}}\int_{\Omega}f_{k}\diff{\mu}
            \end{equation}
            But:
            \begin{equation}
                \underset{n\rightarrow\infty}{\lim}
                \int_{\Omega}\inf_{k\geq{n}}f_{k}\diff{\mu}
                \leq
                \underset{n\rightarrow\infty}{\lim}
                \inf_{k\geq{n}}\int_{\Omega}f_{k}\diff{\mu}
                =\underset{n\rightarrow\infty}{\underline{\lim}}
                \int_{\Omega}f_{n}\diff{\mu}
            \end{equation}
            Therefore, etc.
        \end{proof}
        \begin{theorem}
            If $f_{n}$ is a sequence of non-negative measurable functions,
            then the function $f$ defined by:
            \begin{equation}
                f=\underset{N\rightarrow\infty}{\lim}
                \sum_{n=0}^{N}f_{n}
            \end{equation}
            Is measurable.
        \end{theorem}
        \begin{theorem}
            If $f_{n}$ is a sequence of non-negative measurable functions
            and if $f$ is defined by:
            \begin{equation}
                f=\underset{N\rightarrow\infty}{\lim}
                \sum_{n=0}^{N}f_{n}
            \end{equation}
            Then:
            \begin{equation}
                \int_{\Omega}f\diff{\mu}=
                \underset{N\rightarrow\infty}{\lim}
                \sum_{n=0}^{N}\int_{\Omega}f_{n}\diff{\mu}
            \end{equation}
        \end{theorem}
        \begin{theorem}
            If $(\Omega,\mathcal{A},\mu)$ is a measure space,
            if $f$ is measurable and non-negative, and if
            $\nu:\mathcal{A}\rightarrow\mathbb{R}$ is defined by:
            \begin{equation}
                \nu(E)=\int_{E}f\diff{\mu}=
                \int_{\Omega}f_{E}\diff{\mu}
            \end{equation}
            Then $\nu$ is a measure on $\mathcal{A}$.
        \end{theorem}
        \begin{proof}
            For $\mu(\emptyset)=0$ by definition. Since $f$ is positive,
            for all $E\in\mathcal{A}$:
            \begin{equation}
                \nu(E)=\int_{\Omega}f_{E}\diff{\mu}\geq{0}
            \end{equation}
            And finally, if $E_{n}$ are pairwise disjoint then:
            \begin{equation}
                \nu\Big(\bigcup_{n=1}^{\infty}E_{n}\Big)=
                \int_{\bigcup_{n=1}^{\infty}E_{n}}f\diff{\mu}
                =\sum_{n=1}^{\infty}\int_{E_{n}}f\diff{\mu}
                =\sum_{n=1}^{\infty}\nu(E_{n})
            \end{equation}
            Therefore, etc.
        \end{proof}
        \begin{theorem}
            If $(\Omega,\mathcal{A},\mu)$ is a measure space,
            if $f$ is measurable and non-negative, if
            $\nu:\mathcal{A}\rightarrow\mathbb{R}$ is defined by:
            \begin{equation}
                \nu(E)=\int_{E}f\diff{\mu}=
                \int_{\Omega}f_{E}\diff{\mu}
            \end{equation}
            And if $E\in\mathcal{A}$ is such that $\mu(E)=0$, then
            $\nu(E)=0$.
        \end{theorem}
        \begin{ldefinition}{Absolute Continuity}
            An absolutely continuous measure $\nu$ with respect
            to a measure space $(\Omega,\mathcal{A},\mu)$ is a meausre
            $\nu$ on $\mathcal{A}$ such that for all $E\in\mathcal{A}$
            such that $\mu(E)=0$, it is true that $\nu(E)=0$. This is
            denoted $\nu<<\mu$.
        \end{ldefinition}
        \begin{ltheorem}{Radon-Nikodym Theorem}
            If $(\Omega,\mathcal{A},\mu)$ is a measure space and if
            $\nu$ is absolutely continuous with respect to
            $(\Omega,\mathcal{A},\nu)$. then there is a measurable
            non-negative function $f$ such that, for all $E\in\mathcal{A}$:
            \begin{equation}
                \nu(E)=\int_{E}f\diff{\mu}
            \end{equation}
        \end{ltheorem}
        The function $f$ in the previous theorem is often called the
        density of $\nu$ against $\mu$, or the
        Radon-Nikodym derivative of $\nu$ with respect to $\mu$. The
        function $f$ is unique $\mu$ almost everywhere.
    \section{Integral of Signed Functions}
        Given a function $f:\Omega\rightarrow\mathbb{R}$, we can define
        the following two functions:
        \begin{equation}
            f^{+}(\omega)=
            \begin{cases}
                f(\omega),&f(\omega)\geq{0}\\
                0,&f(\omega)<0
            \end{cases}
        \end{equation}
        \begin{equation}
            f^{+}(\omega)=
            \begin{cases}
                0,&f(\omega)\geq{0}\\
                -f(\omega),&f(\omega)<0
            \end{cases}
        \end{equation}
        From these definitions we see that:
        \begin{equation}
            f=f^{+}-f^{-}
        \end{equation}
        There are two useful formala for computed $f^{+}$ and $f^{-1}$:
        \begin{align}
            f^{+}&=\frac{|f|+f}{2}\\
            f^{-}&=\frac{|f|-f}{2}
        \end{align}
        \begin{theorem}
            If $f$ is measurable, then $f^{+}$ and $f^{-}$ are measurable.
        \end{theorem}
        \begin{ldefinition}{Integral of Signed Function}
            The integral of a measurable function $f$ such that either
            the integral of $f^{+}$ or the integral of $f^{-}$, or both,
            is finite, is the difference:
            \begin{equation}
                \int_{\Omega}f\diff{\mu}=
                \int_{\Omega}f^{+}\diff{\mu}-\int_{\Omega}f^{-}\diff{\mu}
            \end{equation}
        \end{ldefinition}
        \begin{ldefinition}{Summable Function}
            A summable function is a function $f$ such:
            \begin{equation}
                \int_{\Omega}f^{+}\diff{\mu}<\infty
            \end{equation}
            \begin{equation}
                \int_{\Omega}f^{-}\diff{\mu}<\infty
            \end{equation}
        \end{ldefinition}
        \begin{theorem}
            A function $f$ is summable if and only if:
            \begin{equation}
                \int_{\Omega}|f|\diff{\mu}<\infty
            \end{equation}
        \end{theorem}
        \begin{ltheorem}{Homogeneity of the Integral of Signed Functions}
            If $f$ is a signed integrable function, and if $c$ is a
            real number, then:
            \begin{equation}
                \int_{\Omega}(cf)\diff{\mu}=
                c\int_{\Omega}f\diff{\mu}
            \end{equation}
        \end{ltheorem}
        \begin{ltheorem}{Additivity of the Integral of Signed Functions}
            If $f$ and $g$ are summable functions, then:
            \begin{equation}
                \int_{\Omega}(f+g)\diff{\mu}
                =\int_{\Omega}f\diff{\mu}+\int_{\Omega}g\diff{\mu}
            \end{equation}
        \end{ltheorem}
        \begin{proof}
            For:
            \begin{equation}
                (f+g)^{+}=
                \frac{|f+g|+f+g}{2}\leq
                \frac{|f|+|g|+f+g}{2}=f^{+}+g^{+}
            \end{equation}
            Similarly:
            \begin{equation}
                (f+g)^{-}\leq{f}^{-}+g^{-}
            \end{equation}
            And therefore $f+g$ is summable. Evaluating the integral:
            \begin{equation}
                \int_{\Omega}(f+g)\diff{\mu}=
                \int_{\Omega}(f+g)^{+}\diff{\mu}-
                \int_{\Omega}(f+g)^{-}\diff{\mu}
            \end{equation}
            But we have:
            \begin{align}
                f+g&=(f+g)^{+}-(f+g)^{-}\\
                &=(f^{+}-f^{-})+(g^{+}-g^{-})
            \end{align}
            Rearranging, we have:
            \begin{equation}
                (f+g)^{+}+f^{-}+g^{-}=
                (f+g)^{-}+f^{+}+g^{+}
            \end{equation}
            Computing the integral, we have:
            \begin{equation}
                \begin{split}
                    \int_{\Omega}(f+g)^{+}\diff{\mu}+
                    \int_{\Omega}f^{-}&\diff{\mu}+
                    \int_{\Omega}g^{-}\diff{\mu}\\
                    &=\int_{\Omega}(f+g)^{-}\diff{\mu}+
                    \int_{\Omega}f^{+}\diff{\mu}+
                    \int_{\Omega}g^{+}\diff{\mu}
                \end{split}
            \end{equation}
            Rearranging this, we obtain:
            \begin{equation}
                \begin{split}
                    \int_{\Omega}(f+g)^{+}\diff{\mu}-
                    \int_{\Omega}&(f+g)^{-}\diff{\mu}\\
                    &=\int_{\Omega}f^{+}\diff{\mu}-
                    \int_{\Omega}f^{-}\diff{\mu}+
                    \int_{\Omega}g^{+}\diff{\mu}-
                    \int_{\Omega}g^{-}\diff{\mu}
                \end{split}
            \end{equation}
            This completes the proof.
        \end{proof}
        \begin{theorem}
            If $f$ is integrable and if $E_{1}$ and $E_{2}$ are disjoint,
            then:
            \begin{equation}
                \int_{E_{1}\cup{E}_{2}}f\diff{\mu}=
                \int_{E_{1}}f\diff{\mu}+\int_{E_{2}}f\diff{\mu}
            \end{equation}
        \end{theorem}
        \begin{theorem}
            If $f$ and $g$ are summable, and if $f\geq{g}$, then:
            \begin{equation}
                \int_{\Omega}f\diff{\mu}\geq\int_{\Omega}g\diff{\mu}
            \end{equation}
        \end{theorem}
        \begin{theorem}
            If $f=0$ $\mu$ almost everywhere, then:
            \begin{equation}
                \int_{\Omega}f\diff{\mu}=0
            \end{equation}
        \end{theorem}
        \begin{theorem}
            If $f$ is an integrable signed function such that:
            \begin{equation}
                \int_{\Omega}f\diff{\mu}=0
            \end{equation}
            Then $f=0$ $\mu$ almost everywhere.
        \end{theorem}
        \begin{ltheorem}{The Triangle Inequality for Integrals}
            If $f$ is an integrable signed function, then:
            \begin{equation}
                \Big|\int_{\Omega}f\diff{\mu}\Big|
                \leq\int_{\Omega}|f|\diff{\mu}
            \end{equation}
        \end{ltheorem}
        \begin{ltheorem}{Monotone Convergence for Signed Functions}
            If $F$ is a summable function, if $f_{n}$ is a sequence of
            measurable functions that is monotonically increasing and such
            that, for all $n\in\mathbb{N}$, $F\leq{f}_{n}$, then:
            \begin{equation}
                \underset{n\rightarrow\infty}{\lim}
                \int_{\Omega}f_{n}\diff{\mu}
                =\int_{\Omega}\underset{n\rightarrow\infty}{\lim}
                f\diff{\mu}
            \end{equation}
        \end{ltheorem}
        \begin{proof}
            For let $\tilde{f}_{n}$ be defined by:
            \begin{equation}
                \tilde{f}_{n}(\omega)=f_{n}(\omega)-F(\omega)
            \end{equation}
            Then for all $n\in\mathbb{N}$ and for all $\omega$,
            $\tilde{f}_{n}(\omega)\geq{0}$. But then by the monotone
            convergence theorem:
            \begin{equation}
                \underset{n\rightarrow\infty}{\lim}
                \int_{\Omega}(f_{n}-F)\diff{\mu}=
                \int_{\Omega}\underset{n\rightarrow\infty}{\lim}
                (f_{n}-F)\diff{\mu}
            \end{equation}
            But $F$ is summable, and thus we may cancel this from both
            sides. Therefore, etc.
        \end{proof}
        Without the requirement that there is a summable \textit{floor}
        for the sequence of functions $f_{n}$, the theorem may not
        hold. For consider the sequence defined by:
        \begin{equation}
            f_{n}(\omega)=\frac{\minus{1}}{n\omega}
        \end{equation}
        Then $f_{n}\rightarrow{0}$ on $(0,1)$, but the integral of
        $f_{n}$ is infinite for all $n$.
        There is an equivalent theorem with a summable majorant, rather
        than a summable minorant. Here we'd have a sequence of
        monotonically decreasing functions with a summable \textit{roof}.
        \begin{ltheorem}{Fatou's First Theorem for Signed Functions}
            If $f_{n}$ is a sequence of measurable functions such that
            there is a summable function $F$ such that $f_{n}\geq{F}$,
            then:
            \begin{equation}
                \int_{\Omega}
                \underset{n\rightarrow\infty}{\underline{\lim}}
                f_{n}\diff{\mu}
                \leq\underset{n\rightarrow\infty}{\underline{\lim}}
                \int_{\Omega}f_{n}\diff{\mu}
            \end{equation}
        \end{ltheorem}
        \begin{ltheorem}{Fatou's Second Theorem for Signed Functions}
            If $f_{n}$ is a sequence of measurable functions such that
            there is a summable function $F$ such that
            $f_{n}\leq{F}$, then:
            \begin{equation}
                \underset{n\rightarrow\infty}{\overline{\lim}}
                \int_{\Omega}f_{n}\diff{\mu}
                \leq\int_{\Omega}
                \underset{n\rightarrow\infty}{\overline{\lim}}
                f_{n}\diff{\mu}
            \end{equation}
            Where $\overline{\lim}$ denotes the limit superior.
        \end{ltheorem}
        \begin{ltheorem}{Dominated Convergence Theorem}
            If $f_{n}$ is a sequence of functions such that there is a
            summable minorant $F_{1}$ and a summable majorant
            $F_{2}$, that is $F_{1}\leq{f}_{n}\leq{F}_{2}$, and if
            $f_{n}\rightarrow{f}$, then:
            \begin{equation}
                \underset{n\rightarrow\infty}{\lim}
                \int_{\Omega}f_{n}\diff{\mu}
                =\int_{\Omega}
                \underset{n\rightarrow\infty}{\lim}
                f_{n}\diff{\mu}
            \end{equation}
            That is, the limit of the integrals exists.
        \end{ltheorem}
        \begin{proof}
            For if $f_{n}$ has a summable majorant and a summable minorant,
            then both of Fatou's theorem's apply. That is:
            \begin{align}
                \int_{\Omega}
                \underset{n\rightarrow\infty}{\underline{\lim}}
                f_{n}\diff{\mu}
                &\leq\underset{n\rightarrow\infty}{\underline{\lim}}
                \int_{\Omega}f_{n}\diff{\mu}\\
                \underset{n\rightarrow\infty}{\overline{\lim}}
                \int_{\Omega}f_{n}\diff{\mu}
                &\leq\int_{\Omega}
                \underset{n\rightarrow\infty}{\overline{\lim}}
                f_{n}\diff{\mu}
            \end{align}
            But the limit of $f_{n}$ exists, so we have:
            \begin{equation}
                \int_{\Omega}\underset{n\rightarrow\infty}{\lim}
                f_{n}\diff{\mu}
                \leq\underset{n\rightarrow\infty}{\underline{\lim}}
                \int_{\Omega}f_{n}\diff{\mu}
                \leq\underset{n\rightarrow\infty}{\overline{\lim}}
                \int_{\Omega}f_{n}\diff{\mu}
                \leq\int_{\Omega}\underset{n\rightarrow\infty}{\lim}
                f_{n}\diff{\mu}
            \end{equation}
            Therefore:
            \begin{equation}
                \underset{n\rightarrow\infty}{\underline{\lim}}
                \int_{\Omega}f_{n}\diff{\mu}
                =\underset{n\rightarrow\infty}{\overline{\lim}}
                \int_{\Omega}f_{n}\diff{\mu}
            \end{equation}
            Therefore the limit exists, and by the inequalities:
            \begin{equation}
                \underset{n\rightarrow\infty}{\lim}
                \int_{\Omega}f_{n}\diff{\mu}
                =\int_{\Omega}
                \underset{n\rightarrow\infty}{\lim}
                f_{n}\diff{\mu}
            \end{equation}
        \end{proof}
        We can relax the requirements of the monotone convergence theorems,
        Fatou's theorems, and the dominated convergence theorem to be
        true on all but a set of measure zero, and the results are still
        valid.
        \begin{ltheorem}{Generalized Monotone Converence Theorem}
            If $f_{n}$ is a sequence of measurable functions such that
            $f_{n}(\omega)\leq{f}_{n+1}(\omega)$ $\mu$ almost everywhere,
            and if $F$ is a summable function such that
            $F\leq{F}_{n}(\omega)$ $\mu$ almost everywhere, then:
            \begin{equation}
                \underset{n\rightarrow\infty}{\lim}
                \int_{\Omega}f_{n}\diff{\mu}
                =\int_{\Omega}f\diff{\mu}
            \end{equation}
        \end{ltheorem}
        \begin{theorem}
            For define $E_{n}$ be:
            \begin{equation}
                E_{n}=\{\omega:f_{n}(\omega)\not\leq{f}_{n+1}(\omega)\}
            \end{equation}
            And let $E$ be defined by:
            \begin{equation}
                E=\Big(\bigcup_{n=1}^{\infty}E_{n}\Big)^{C}
            \end{equation}
            Then, as the countable union of sets of measure zero has
            measure zero, $\mu(E^{C})=0$. 
        \end{theorem}
    \section{Product Measures}
        Let $(\Omega_{1},\mathcal{A}_{1},\mu_{1})$ and
        $(\Omega_{2},\mathcal{A}_{2},\mu_{2})$ be measure spaces. We
        wish to define a \textit{natural} measure space
        on the Cartesian product $\Omega_{1}\times\Omega_{2}$.
        Let $\mathcal{P}$ be defined by:
        \begin{equation}
            \mathcal{P}=
            \{A_{1}\times{A}_{2}:
                A_{1}\in\mathcal{A}_{1},A_{2}\in\mathcal{A}_{2}\}
        \end{equation}
        Then $\mathcal{P}$ is a semi-ring, but is not a
        $\sigma\textrm{-Algebra}$ on $\Omega_{1}\times\Omega_{2}$
        This is because the union of two rectangles may not be a
        rectangle. Similarly, the difference of two rectangles may not
        be a rectangle. However, the intersection of two rectangles is
        a rectangle, and hence this is a semi-ring.
        We defined the product $\sigma\textrm{-Algebra}$ to be the
        $\sigma\textrm{-Algebra}$ that is generated by $\mathcal{P}$. 
        \begin{ltheorem}{Carath\'{e}odory Extension Theorem}
            If $(\Omega_{1},\mathcal{A},\mu_{1})$ and
            $(\Omega_{2},\mathcal{A}_{2},\mu_{2})$ are measure spaces,
            if $\mathcal{A}$ is the product $\sigma\textrm{-Algebra}$
            on $\Omega_{1}\times\Omega_{2}$, then there is a unique
            measure $\mu$ on $\mathcal{A}$ such that, for all
            $A_{1}\in\mathcal{A}_{1}$ and all
            $A_{2}\in\mathcal{A}_{2}$:
            \begin{equation}
                \mu(A_{1}\times{A}_{2})
                =\mu_{1}(A_{1})\cdot\mu_{2}(A_{2})
            \end{equation}
        \end{ltheorem}
        \begin{ltheorem}{Funini's Theorem}
            If $f:\Omega_{1}\times\Omega_{2}\rightarrow\mathbb{R}$
            is a non-negative function that is
            $\mathcal{A}-\mathcal{B}$ measurable, where
            $\mathcal{A}$ is the product $\sigma\textrm{-Algebra}$,
            then:
            \begin{equation}
                \int_{\Omega_{1}\times\Omega_{2}}f\diff{\mu}=
                \int_{\Omega_{1}}\Big(
                    \int_{\Omega_{2}}f\diff{\mu_{2}}
                \Big)\diff{\mu}_{1}
                =\int_{\Omega_{2}}
                    \Big(\int_{\Omega_{1}}f\diff{\mu_{1}}\Big)
                    \diff{\mu}_{2}
            \end{equation}
        \end{ltheorem}
        As a summary, when is the following true?
        \begin{equation}
            \underset{n\rightarrow\infty}{\lim}
            \int_{\Omega}f_{n}\diff{\mu}
            \overset{?}{=}\int_{\Omega}
            \underset{n\rightarrow\infty}{\lim}f_{n}\diff{\mu}
        \end{equation}
        There are two special cases when equality can be guarenteed.
        The first is monotone convergence. If
        $f_{n}\rightarrow{f}$, where $f_{n+1}(x)\leq{f}_{n}(x)$ for
        all $n$, and if $f_{n}(x)\geq{F}$, where $F$ is a
        summable minorant, or if $f_{n}\rightarrow{f}$,
        $f_{n+1}(x)\leq{f}_{n}(x)$, and if
        $f_{n}(x)\leq{F}$, where $F$ is a summable majorant, then
        equality holds. The next case is by dominated convergence.
        If the limit of $f_{n}$ exists almost everywhere, and if
        $|f_{n}|\leq{F}$, where $F$ is summable, then by Fatou's
        Lemma:
        \begin{equation}
            \underset{n\rightarrow\infty}{\underline{\lim}}
            \int_{\Omega}f_{n}\diff{\mu}
            \geq\int_{\Omega}
            \underset{n\rightarrow\infty}{\underline{\lim}}
            f_{n}\diff{\mu}
        \end{equation}
        And also:
        \begin{equation}
            \underset{n\rightarrow\infty}{\overline{\lim}}
            \int_{\Omega}f_{n}\diff{\mu}
            \leq\int_{\Omega}
            \underset{n\rightarrow\infty}{\overline{\lim}}
            f_{n}\diff{\mu}
        \end{equation}
    \section{Probablity Spaces}
    \section{Random Variables}
        Let $(\Omega,\mathcal{A},\mu)$ be a probability space.
        A probability space is a measure space such that
        $\mu(\Omega)=1$. Let $f:\Omega\rightarrow\mathbb{R}$ be
        $\mathcal{A}-\mathcal{B}$ measurable, where $\mathcal{B}$ is
        the Borel $\sigma\textrm{-Algebra}$. Such functions are called
        random-variables on $\Omega$. While there's nothing random
        about this, we use such functions to model problems in
        probability theory. The probability of an event
        $A\in\mathcal{A}$ is simply $\mu(A)$. The associated
        $\sigma\textrm{-Algebra}$ is defined as:
        \begin{equation}
            \mathcal{A}_{f}=\{f^{-1}(B):B\in\mathcal{B}\}
        \end{equation}
        This is also called the $\sigma\textrm{-Algebra}$ of events
        bearing on $f$. This is a $\sigma\textrm{-Algebra}$ on
        $\Omega$.
        \begin{ldefinition}{Distribution of a Random Variable}
            The distribution of a random variable
            $f:\Omega\rightarrow\mathbb{R}$ on a probability space
            $(\Omega,\mathcal{A},\mu)$ is the image measure
            $\mu_{f}$ of $f$.
        \end{ldefinition}
        The image measure is the measure:
        \begin{equation}
            \mu_{f}(B)=\mu(f^{-1}(B))
            =\mu(\{\omega\in\Omega:f(\omega)\in{B}\})
        \end{equation}
        This is a Lebesgue-Stieljes Measure on the Borel
        $\sigma\textrm{-Algebra}$ on $\mathbb{R}$.
        \begin{equation}
            \mu_{f}(\mathbb{R})=\mu(f^{-1}(\mathbb{R}))
            =\mu(\Omega)=1
        \end{equation}
        \begin{ldefinition}{Cumulative Distribution Function}
            The Cumulative Distribution Function of a random variable
            $f:\Omega\rightarrow\mathbb{R}$ on a probability space
            $(\Omega,\mathcal{A},\mu)$ is the function
            $F:\mathbb{R}\rightarrow\mathbb{R}$ defined by:
            \begin{equation}
                F(x)=\mu_{f}\big((-\infty,a)\big)
            \end{equation}
            Where $\mu_{f}$ is the distribution of $f$.
        \end{ldefinition}
        Some facts about the cumulative distribution function:
        It is non-decreasing on $\mathbb{R}$, left continuous, and
        $F(\minus\infty)-F(\infty)=1$. By the Caratheodory extension
        theorem, and function $F$ that satisfies these three
        conditions is the cumulative distribution function of some
        Lebesgue-Stieljes probability measure on $\mathbb{R}$. From this
        we also have that every Lebesgue-Stieljes probability measure
        on $\mathbb{R}$ is a distribution for a random variable.
        \begin{example}
            Let $\Omega=\mathbb{R}$, let $\mathcal{A}=\mathcal{B}$,
            where $\mathcal{B}$ is the Borel $\sigma\textrm{-Algebra}$,
            and let $\mu$ be a Lebesgue-Stieljes probability measure
            on $\mathbb{R}$. Define the random variable
            $f:\Omega\rightarrow\mathbb{R}$ by
            $f(\omega)=\omega$. The inverse of any Borel set is itself,
            and thus we see that the distribution and the random
            variable coincide.
        \end{example}
        \begin{example}
            Let $\Omega=[0,1]$, $\mathcal{B}$ be the Borel
            $\sigma\textrm{-Algebra}$, and define
            $f_{1},f_{2}:\Omega\rightarrow\mathbb{R}$ by:
            \begin{equation}
                f_{1}(\omega)=\omega
                \quad\quad
                f_{2}(\omega)=1-\omega
            \end{equation}
            These two functions, while different, will have the same
            cumulative distribution function. For we have:
            \begin{equation}
                F_{1}(u)=\mu_{f_{1}}\big((\minus\infty,u)\big)=
                \mu\big(f^{-1}(\minus\infty,u)\big)
            \end{equation}
            We can evaluate this case by case to get:
            \begin{equation}
                F_{1}(u)=
                \begin{cases}
                    \mu(\emptyset)=0,&u\leq{0}\\
                    \mu\big([0,u)\big)]u,0<u<1\\
                    \mu([0,1])=1,1\leq{u}
                \end{cases}
            \end{equation}
            Looking at $F_{2}$, we have:
            \begin{equation}
                F_{2}(u)=\mu_{f_{2}}\big((\minus\infty,u)\big)
                =\mu\big(f_{2}^{\minus{1}}(\minus\infty,u)\big)
            \end{equation}
            Again, evaluating case by case, we get:
            \begin{equation}
                F_{2}(u)=
                \begin{cases}
                    \mu(\emptyset)=0,&u\leq{0}\\
                    \mu\big((1-u,1]\big)]u,0<u<1\\
                    \mu([0,1])=1,1\leq{u}
                \end{cases}
            \end{equation}
            Thus, $F_{1}=F_{2}$.
        \end{example}
        \begin{ldefinition}{Random Vector}
            A random vector on a probability space
            $(\Omega,\mathcal{A},\mu)$ is an
            $\mathcal{A}-\mathcal{B}_{n}$ measurable function
            $\mathbf{f}:\Omega\rightarrow\mathbb{R}^{n}$, where
            $\mathcal{B}_{n}$ is the Borel $\sigma\textrm{-Algebra}$
            on $\mathbb{R}^{n}$.
        \end{ldefinition}
        As a comment, if $f:\Omega\rightarrow\mathbb{R}$ is
        $\mathcal{A}-\mathcal{B}$ measurable, then
        $\mathcal{A}_{f}\subseteq\mathcal{A}$. The associated
        $\sigma\textrm{-Algebra}$ of a random vector
        $\mathbf{f}:\Omega\rightarrow\mathbb{R}^{n}$ is:
        \begin{equation}
            A_{\mathbf{f}}
            =\{\mathbf{f}^{\minus{1}}(B):B\in\mathcal{B}_{n}\}
        \end{equation}
        \begin{theorem}
            If $(\Omega,\mathcal{A},\mu)$ is a probability space,
            $\mathcal{B}_{n}$ is the Borel $\sigma\textrm{-Algebra}$
            on $\mathbb{R}^{n}$, and if
            $\mathbf{f}:\Omega\rightarrow\mathbb{R}^{n}$ is a random
            vector such that:
            \begin{equation}
                \mathbf{f}(\omega)=(f_{1}(\omega),\dots,f_{n}(\omega))
            \end{equation}
            Then:
            \begin{equation}
                \mathcal{A}_{\mathbf{f}}=
                \sigma\big(
                    \mathcal{A}_{f_{1}},\dots,\mathcal{A}_{f_{n}}\big)
            \end{equation}
            Where this is the $\sigma\textrm{-Algebra}$ generated by
            these sets.
        \end{theorem}
        \begin{proof}
            For any $f_{j}$,
            $\mathcal{A}_{f_{j}}\subseteq\mathcal{A}_{\mathbf{f}}$,
            and thus the generated $\sigma\textrm{-Algebra}$ is
            contained in $\mathcal{A}_{\mathbf{f}}$. Going the other
            ways, let $\tilde{\mathcal{B}}$ be the set of subsets
            $B\subseteq\mathbb{R}^{n}$ such that:
            \begin{equation}
                \mathbf{f}^{\minus{1}}(B)\in
                \sigma\big(
                    \mathcal{A}_{f_{1}},\dots,\mathcal{A}_{f_{n}}\big)
            \end{equation}
            But then for any sequence $B_{1},\dots,B_{n}\in\mathcal{B}$,
            $B_{1}\times\cdots\times{B}_{n}$ is contained in
            $\tilde{\mathcal{B}}$. But $\mathcal{B}_{n}$ is the
            smallest such $\sigma\textrm{-Algebra}$ to contain such
            sets, and thus
            $\mathcal{B}_{n}\subseteq\tilde{\mathcal{B}}$.
        \end{proof}
        \begin{ldefinition}{Distribution of a Random Vector}
            The distribution of a random vector
            $\mathbf{f}:\Omega\rightarrow\mathbb{R}^{n}$ on a
            measure space $(\Omega,\mathcal{A},\mu)$ is the measure:
            \begin{equation}
                \mu_{\mathbf{f}}(B)=
                \mu(\mathbf{f}^{\minus{1}}(B))
            \end{equation}
            Which is the joint distribution of
            $f_{1},\dots,f_{n}$, where:
            \begin{equation}
                \mathbf{f}(\omega)=(f_{1}(\omega),\dots,f_{n}(\omega))
            \end{equation}
        \end{ldefinition}
        The individual distributions can be computed in terms of the
        joint distribution. This is because:
        \begin{equation}
            \mu_{f_{1}}(B)=
            \mu(f_{1}^{\minus{1}}(B))=
            \mu\big(\mathbf{f}^{\minus{1}}(
                B\times\mathbb{R}^{n-1})\big)=
            \mu_{\mathbf{f}}\big(B\times\mathbb{R}^{n-1}\big)
        \end{equation}
        The joint distribution can not, in general, be computed
        in terms of the individual distributions. There is a special
        exception to this rule, and that is when the random variables
        are independent. That is, if the associated
        $\sigma\textrm{-Algebras}$ are independent. So events that
        bear on $f_{1},\dots,f_{n}$ are independent. If
        $E_{j}\in\mathcal{A}_{f_{j}}$, then:
        \begin{equation}
            \mu\Big(\bigcap_{k=1}^{n}E_{k}\Big)=
            \prod_{k=1}^{n}\mu(E_{k})
        \end{equation}
        \begin{theorem}
            A sequence of random variables $f_{1},\dots,f_{n}$ are
            independent if and only if the joint distribution is
            the product measure of the individual distributions.
        \end{theorem}
        \begin{proof}
            For let $B_{k}\in\mathcal{B}$ and let:
            \begin{equation}
                E_{k}=f_{k}^{\minus{1}}(B_{k})
            \end{equation}
            But then:
            \begin{subequations}
                \begin{align}
                    \mu\Big(\bigcap_{k=1}^{n}E_{k}\Big)&=
                    \mu\Big(\bigcap_{k=1}^{n}
                        f_{k}^{\minus{1}}(B_{k})\Big)\\
                    &=\mu\big(\mathbf{f}^{\minus{1}}
                        (B_{1}\times\dots\times{B}_{n})\big)\\
                    &=\mu_{\mathbf{f}}(B_{1}\times\dots\times{B}_{n})\\
                    &=\prod_{k=1}^{n}\mu(E_{n})\\
                    &=\prod_{k=1}^{n}\mu\big(f^{\minus{1}}(B_{k})\big)\\
                    &=\prod_{k=1}^{n}\mu_{f_{k}}(B_{k})
                \end{align}
            \end{subequations}
        \end{proof}
        Let $\mu_{1},\dots,\mu_{n}$ be probability Lebesgue-stieljes
        measures on $\mathbb{R}$, and let $\mu$ be the product
        measure. Consider the probability space
        $(\mathbb{R}^{n},\mathcal{B}_{n},\mu)$ and the projection
        mappings $\pi_{k}:\mathbb{R}^{n}\rightarrow\mathbb{R}$:
        \begin{equation}
            \pi_{k}(\omega_{1},\dots,\omega_{n})=\omega_{k}
        \end{equation}
        \begin{theorem}
            Let $f_{n}$ be an infinite sequence of random variables
            on a probability space $(\Omega,\mathcal{A},\mu)$. Let
            $\mathcal{A}_{f_{n}}$ be the associated
            $\sigma\textrm{-Algebras}$. For every $\omega\in\Omega$,
            let:
            \begin{equation}
                F_{\inf}(\Omega)=
                \underset{n\rightarrow\infty}{\underline{\lim}}
                f_{n}(\omega)
                \quad\quad
                F_{\sup}(\Omega)=
                \underset{n\rightarrow\infty}{\overline{\lim}}
                f_{n}(\omega)
            \end{equation}
            Then $F_{\inf}$ and $F_{\sup}$ are measurable with
            respect to the terminal $\sigma\textrm{-Algebra}$.
        \end{theorem}
        \begin{proof}
            For $F_{\inf}$ is measurable if and only if for all
            $u\in\mathbb{R}$, we have
            $F^{\minus{1}}\big((\minus\infty,u)\big)\in\mathcal{F}$.
            But:
            \begin{subequations}
                \begin{align}
                    F^{\minus{1}}\big((\minus\infty,u)\big)
                    &=\{\omega:F(\omega)\leq{u}\}\\
                    &=\{\omega:\underline{\lim}f_{n}(\omega)\leq{u}\}\\
                    &=\{\omega:\underset{n}{\sup}
                        \underset{k\geq{n}}{\lim}f_{k}(\omega)\}\\
                    &=\bigcap_{n=1}^{\infty}\Big\{\omega:
                        \underset{n\geq{k}}{\inf}f_{k}(\omega)\leq{u}
                    \Big\}\\
                    &=\bigcap_{n=N}^{\infty}\Big\{\omega:
                        \underset{n\geq{k}}{\inf}f_{k}(\omega)\leq{u}
                    \Big\}
                \end{align}
            \end{subequations}
        \end{proof}
        \begin{theorem}
            If $\mathcal{F}$ is a self-independent
            $\sigma\textrm{-Algebra}$, if $F$ is measurable with
            respect to $\mathcal{F}$, then $F$ is constant almost
            everywhere.
        \end{theorem}
        \begin{proof}
            For since $\mathcal{F}$ is self independent:
            \begin{equation}
                \mu(\{\omega:F(\omega)<u\})=0
                \quad\textrm{or}\quad
                \mu(\{\omega:F(\omega)<u\})=1
            \end{equation}
            Define $A$ and $B$ as follows:
            \begin{align}
                A&=\{u\in\mathbb{F}:\mu(\{\omega:F(\omega)<u\})=0\}\\
                B&=\{u\in\mathbb{F}:\mu(\{\omega:F(\omega)<u\})=1\}\\
            \end{align}
            This separates the real line into two parts. By
            Dedekind's Axiom there is a $c\in\mathbb{R}$ such that,
            for all $a\in{A}$, and for all $b\in{B}$,
            $a\leq{c}\leq{b}$. But then:
            \begin{equation}
                \mu(\{u:F(u)<c+\frac{1}{n}\})=1
            \end{equation}
            From continuity from above, we're done.
        \end{proof}
        \begin{theorem}
            If $(\Omega,\mathcal{A},\mu)$ is a probability space,
            $f_{n}$ is a sequence of independent random variables,
            then the limit inferior and the limit superior are
            constants $\mu$ almost everywhere.
        \end{theorem}
        \begin{proof}
            For the limit inferior and limit superior are measurable
            with respect to the terminal $\sigma\textrm{-Algebra}$.
            By the Kolmogorov zero-one law, $\mathcal{F}$ is
            self-independent if $\mathcal{A}_{f_{n}}$ are independent.
            Thus, by the previous theorem, these functions are constants
            almost everywhere.
        \end{proof}
        Thus the limit of random-variables is entirely not random, but
        constant functions.
        \begin{theorem}
            If $f_{n}$ is a sequence of random variables, then the
            limit of $f_{n}$ almost surely exists, or almost never
            exists.
        \end{theorem}
        \begin{proof}
            For since the limit superior and limit inferior are
            constants almost everywhere, then eithere they agree,
            in which there's convergence almost surely, or they do
            not agree, in which there's convergence almost never.
        \end{proof}
        \begin{ldefinition}{Expectation Value}
            The expectation value of a summable random variable
            $f:\Omega\rightarrow\mathbb{R}$ on a measure space
            $(\Omega,\mathcal{A},\mu)$ is the real number
            $E(f)$ defined by:
            \begin{equation}
                E(f)=\int_{\Omega}f\diff{\mu}
            \end{equation}
        \end{ldefinition}
        The expectation can be expressed in terms of the distribution
        by using the measure transformation theorem. If
        $g:\mathbb{R}\rightarrow\mathbb{R}$ is a real valued function,
        then:
        \begin{equation}
            \int_{\Omega}g\diff{\mu}=
            \int_{\mathbb{R}}g\circ{f}\diff{\mu_{f}}
        \end{equation}
        Now we apply this in the simple case when $g(u)=u$. Then:
        \begin{equation}
            E(f)=\int_{\Omega}f\diff{\mu}
            =\int_\mathbb{R}u\diff{\mu_{f}}
        \end{equation}
        Where we assume that $f$ is summable against $\mu$. Thus,
        $u$ is summable against $\mu_{f}$. So, we have that:
        \begin{equation}
            \int_{\mathbb{R}}|u|\diff{\mu_{f}}<\infty
        \end{equation}
        \begin{ldefinition}{Variance}
            The variance of a random variable
            $f:\Omega\rightarrow\mathbb{R}$ on a measure space
            $(\Omega,\mathcal{A},\mu)$, is the real number
            $Var(f)$ defined by:
            \begin{equation}
                Var(f)=E\big(f-E(f)\big)^{2}
                =\int_{\Omega}\big(f-E(f)\big)^{2}\diff{\mu}
            \end{equation}
        \end{ldefinition}
        \begin{theorem}
            \begin{equation}
                Var(f)=E(f^{2})-E(f)^{2}
            \end{equation}
        \end{theorem}
\end{document}