\documentclass[crop=false,class=article,oneside]{standalone}
%----------------------------Preamble-------------------------------%
%---------------------------Packages----------------------------%
\usepackage{geometry}
\geometry{b5paper, margin=1.0in}
\usepackage[T1]{fontenc}
\usepackage{graphicx, float}            % Graphics/Images.
\usepackage{natbib}                     % For bibliographies.
\bibliographystyle{agsm}                % Bibliography style.
\usepackage[french, english]{babel}     % Language typesetting.
\usepackage[dvipsnames]{xcolor}         % Color names.
\usepackage{listings, lstlinebgrd}      % Verbatim-Like Tools.
\usepackage{mathtools, esint, mathrsfs} % amsmath and integrals.
\usepackage{amsthm, amsfonts}           % Fonts and theorems.
\usepackage{tabularx}
\usepackage{tcolorbox}                  % Frames around theorems.
\usepackage{upgreek}                    % Non-Italic Greek.
\usepackage{paracol}                    % Two-column styling.
\usepackage{wrapfig}                    % Wrap text around figure.
\usepackage{fmtcount, etoolbox}         % For the \book{} command.
\usepackage[newparttoc]{titlesec}       % Formatting chapter, etc.
\usepackage{titletoc}                   % Allows \book in toc.
\usepackage[nottoc]{tocbibind}          % Bibliography in toc.
\usepackage[titles]{tocloft}            % ToC formatting.
\usepackage{multicol, enumitem}         % Multi-column/enumerate.
\usepackage{import}                     % Import external files.
\usepackage{pgfplots, tikz}             % Drawing/graphing tools.
\usetikzlibrary{
    calc,                   % Calculating right angles and more.
    angles,                 % Drawing angles within triangles.
    arrows.meta,            % Latex and Stealth arrows.
    quotes,                 % Adding labels to angles.
    positioning,            % Relative positioning of nodes.
    decorations.markings,   % Adding arrows in the middle of a line.
    patterns,
    arrows,
    shapes,
    shapes.geometric,
    cd,
    hobby,
    babel
}                                       % Libraries for tikz.
\pgfplotsset{compat=1.9}                % Version of pgfplots.
\usepackage[font=scriptsize,
            labelformat=simple,
            labelsep=colon]{subcaption} % Subfigure captions.
\usepackage[font={scriptsize},
            hypcap=true,
            labelsep=colon]{caption}    % Figure captions.
\usepackage{hyperref}                   % Allows for hyperlinks.
\hypersetup{
    colorlinks=true,
    linkcolor=blue,
    filecolor=magenta,
    urlcolor=Cerulean,
    citecolor=SkyBlue
}                           % Colors for hyperref.
\usepackage[toc,acronym,nogroupskip]{glossaries} % Glossaries and acronyms.
\usepackage[subpreambles=false]{standalone}      % Complileable sub files.

% Various font stuff from kiwi.
% Use this for Times text and Computer Modern math
%\usepackage{times}

% Quite nice
%\usepackage[charter, greekfamily=, greekuppercase=italicized]{mathdesign}
%\usepackage[utopia, greekuppercase=italicized]{mathdesign}    % Math is narrower

% Use this for Times text and math
%\usepackage{newtxtext}
%\usepackage[libertine,cmintegrals]{newtxmath}
%\usepackage{fix-cm}

%\usepackage{txfontsb}
% or
%\usepackage{mathptmx}

%\usepackage[scaled=0.92]{helvet}
%\renewcommand{\rmdefault}{ptm}

%\usepackage{mathpazo}    % add possibly `sc` and `osf` options
%\usepackage{eulervm}

%\usepackage{fourier}
%\renewcommand{\rmdefault}{ptm}
%\usepackage{mathptm}

%\usepackage{fontspec}
%\setmainfont{lmodern}

%\usepackage[varg]{txfonts}
%\usepackage{fouriernc}
%\usepackage{mathpazo}

%\usepackage{bookman}
%\usepackage[scaled]{uarial}
%\usepackage[scaled]{helvet}
%\renewcommand*\familydefault{\sfdefault}
%\usepackage[math]{anttor}

%\newcommand\fgeorgia{\fontfamily{jvn}\selectfont}
%\newcommand\ftimes{\fontfamily{ptm}\selectfont}
%\newcommand\fhelvetica{\fontfamily{phv}\selectfont}
%\newcommand\fcourier{\fontfamily{pcr}\selectfont}
%\newcommand\fbookman{\fontfamily{pbk}\selectfont}
%\newcommand\fnewcentury{\fontfamily{pnc}\selectfont}
%\newcommand\fpalatino{\fontfamily{ppl}\selectfont}
%\newcommand\favantgarde{\fontfamily{pag}\selectfont}
%\newcommand\fnormal{\normalfont}
%\newcommand\fsize[1]{\ifnum#1>0\fontsize{#1}{#1}\selectfont\else\normalsize\fi}
%------------------------Theorem Styles-------------------------%
% Define theorem style for default spacing and normal font.
\newtheoremstyle{normal}
    {\topsep}               % Amount of space above the theorem.
    {\topsep}               % Amount of space below the theorem.
    {}                      % Font used for body of theorem.
    {}                      % Measure of space to indent.
    {\bfseries}             % Font of the header of the theorem.
    {}                      % Punctuation between head and body.
    {.5em}                  % Space after theorem head.
    {}

% Define theorem style for default spacing with italicized font.
\newtheoremstyle{normalit}{\topsep}{\topsep}
                {\itshape}{}{\bfseries}{}{.5em}{}

% Italic header environment.
\newtheoremstyle{thmit}{\topsep}{\topsep}{}{}{\itshape}{}{0.5em}{}

% Define italicized environments.
\theoremstyle{normalit}
\newtheorem{theorem}{Theorem}[section]
\newtheorem{lemma}{Lemma}[section]
\newtheorem{corollary}{Corollary}[section]
\newtheorem{proposition}{Proposition}[section]
\newtheorem*{theorem*}{Theorem}

% Define environments with italic headers.
\theoremstyle{thmit}
\newtheorem*{solution}{Solution}
\newtheorem*{fsolution}{Solution}

% Define default environments.
\theoremstyle{normal}
\newtheorem{example}{Example}[section]
\newtheorem{definition}{Definition}[section]
\newtheorem{problem}{Problem}[section]
\newtheorem{question}{Question}[section]
\newtheorem{remark}{Remark}[section]
\newtheorem{properties}{Properties}[section]
\newtheorem{notation}{Notation}[section]
\newtheorem{axiom}{Axiom}[section]
\newtheorem*{properties*}{Properties}
\newtheorem*{remark*}{Remark}
\newtheorem*{definition*}{Definition}
\theoremstyle{plain}

% Define framed environment.
\tcbuselibrary{most}
\newtcbtheorem[use counter*=theorem]{ftheorem}{Theorem}%
    {colback=green!5,colframe=green!35!black,
     fonttitle=\bfseries\upshape}{th}

\newtcbtheorem[use counter*=example]{fdefinition}{Definition}%
    {fonttitle=\bfseries\upshape,
     colback=blue!5!white,colframe=blue!75!black}{def}

\newtcbtheorem[use counter*=example]{fexample}{Example}%
    {fonttitle=\bfseries\upshape,
     colback=red!5!white,colframe=red!75!black}{ex}

\newtcbtheorem[use counter*=notation]{fnotation}{Notation}%
    {fonttitle=\bfseries\upshape,
     colback=SeaGreen!5!white,colframe=SeaGreen!75!black}{ex}

\newtcbtheorem[use counter*=corollary]{fcorollary}{Corollary}%
    {fonttitle=\bfseries\upshape,
     colback=Orchid!5!white,colframe=Orchid!75!black}{ex}

\newenvironment{bproof}{\textit{Proof.}}{\hfill$\square$}
\tcolorboxenvironment{bproof}{blanker,breakable,left=5mm,
                             before skip=10pt,after skip=10pt,
                             borderline west={1mm}{0pt}{red}}
\tcolorboxenvironment{fsolution}
    {enhanced jigsaw,colframe=cyan,interior hidden,breakable}

%--------------------Declared Math Operators--------------------%
\DeclareMathOperator{\Refl}{Refl}           % Reflection operator.
\DeclareMathOperator{\Span}{Span}           % Span of a set of vectors.
\DeclareMathOperator{\Card}{Card}           % Cardinality of set.
\DeclareMathOperator{\Ord}{Ord}             % Ordinal of ordered set.
\DeclareMathOperator{\Tr}{Tr}               % Trace of matrix.
\DeclareMathOperator{\adjoint}{adj}         % Adjoint of matrix.
\DeclareMathOperator{\rk}{rk}               % Rank of operator.
\DeclareMathOperator{\nul}{nul}             % Null space of operator.
\DeclareMathOperator{\sgn}{sgn}             % Sign of a number.
\DeclareMathOperator{\multideg}{mutlideg}   % Multi-Degree (Graphs).
\DeclareMathOperator{\GCD}{GCD}             % Greatest common denominator.
\DeclareMathOperator{\LM}{LM}               % Leading monomial
\DeclareMathOperator{\LC}{LC}               % Leading coefficient.
\DeclareMathOperator{\LT}{LT}               % Leading term.
\DeclareMathOperator{\LCM}{LCM}             % Least common multiple.
\DeclareMathOperator{\Mon}{Mon}             % Monomial.
\DeclareMathOperator{\Spec}{Spec}           % Spectrum.
\DeclareMathOperator{\proj}{proj}           % Projection.
\DeclareMathOperator{\comp}{comp}           % Component.
\DeclareMathOperator{\sinc}{sinc}           % Sinc function.
\DeclareMathOperator{\Ima}{Im}              % Image of operator.
\DeclareMathOperator{\Prin}{Prin}           % Principal value.
\DeclareMathOperator{\Mod}{mod}             % Modulus.
%------------------------New Commands---------------------------%
\DeclarePairedDelimiter\norm{\lVert}{\rVert}
\DeclarePairedDelimiter\ceil{\lceil}{\rceil}
\DeclarePairedDelimiter\floor{\lfloor}{\rfloor}
\newcommand*\diff{\mathop{}\!\mathrm{d}}
\newcommand*\Diff[1]{\mathop{}\!\mathrm{d^#1}}
\renewcommand{\mod}{\ \Mod}
\renewcommand*{\glstextformat}[1]{\textcolor{RoyalBlue}{#1}}
\renewcommand{\glsnamefont}[1]{\textbf{#1}}
\renewcommand\labelitemii{$\circ$}
\renewcommand\thesubfigure{\arabic{chapter}.\arabic{figure}}
\renewcommand\thesubfigure{%
    \arabic{chapter}.\arabic{figure}.\arabic{subfigure}}
\addto\captionsenglish{\renewcommand{\figurename}{Fig.}}
%------------------------Book Command---------------------------%
\makeatletter
\renewcommand\@pnumwidth{1cm}
\newcounter{book}
\renewcommand\thebook{\@Roman\c@book}
\newcommand\book{%
    \if@openright
        \cleardoublepage
    \else
        \clearpage
    \fi
    \thispagestyle{plain}%
    \if@twocolumn
        \onecolumn
        \@tempswatrue
    \else
        \@tempswafalse
    \fi
    \null\vfil
    \secdef\@book\@sbook
}
\def\@book[#1]#2{%
    \ifnum \c@secnumdepth >-3\relax
        \refstepcounter{book}%
        \addcontentsline{toc}{book}{
            \bookname\ \thebook:\hspace{1em}#1
        }
    \else
        \addcontentsline{toc}{book}{#1}%
    \fi
    \markboth{}{}%
    {\centering
     \interlinepenalty \@M
     \normalfont
     \ifnum \c@secnumdepth >-2\relax
       \huge\bfseries \bookname\nobreakspace\thebook
       \par
       \vskip 20\p@
     \fi
     \Huge \bfseries #2\par}%
    \@endbook}
\def\@sbook#1{%
    {\centering
     \interlinepenalty \@M
     \normalfont
     \Huge \bfseries #1\par}%
    \@endbook}
\def\@endbook{
    \vfil\newpage
        \if@twoside
            \if@openright
                \null
                \thispagestyle{empty}%
                \newpage
            \fi
        \fi
        \if@tempswa
            \twocolumn
        \fi
}
\newcommand*\l@book[2]{%
    \ifnum \c@tocdepth >-2\relax
        \addpenalty{-\@highpenalty}%
        \addvspace{2.25em \@plus\p@}%
        \setlength\@tempdima{3em}%
        \begingroup
            \parindent \z@ \rightskip \@pnumwidth
            \parfillskip -\@pnumwidth
            {
                \leavevmode
                \Large \bfseries #1\hfil \hb@xt@\@pnumwidth{
                    \hss #2
                }
            }
            \par
            \nobreak
            \global\@nobreaktrue
            \everypar{\global\@nobreakfalse\everypar{}}%
        \endgroup
    \fi}
\newcommand\bookname{Book}
\renewcommand{\thebook}{\texorpdfstring{\Numberstring{book}}{book}}
\providecommand*{\toclevel@book}{-2}
\makeatother
\titlecontents{chapter}[0pt]
    {\bfseries}
    {\chaptername\ \thecontentslabel:\quad}
    {}
    {\hfill\contentspage}
\titleformat{\part}[display]
    {\Large\bfseries}
    {\partname\nobreakspace\thepart}
    {0mm}
    {\Huge\bfseries}
    \titlecontents{part}[0pt]
    {\large\bfseries}
    {\partname\ \thecontentslabel: \quad}
    {}
    {\hfill\contentspage}
\newcommand{\MarkRightAngle}[4][.3cm]
    {\coordinate (tempa) at ($(#3)!#1!(#2)$);
     \coordinate (tempb) at ($(#3)!#1!(#4)$);
     \coordinate (tempc) at ($(tempa)!0.5!(tempb)$);%midpoint
     \draw (tempa) -- ($(#3)!2!(tempc)$) -- (tempb);}
%--------------------------LENGTHS------------------------------%
% Spacings for the Table of Contents.
\addtolength{\cftsecnumwidth}{1ex}
\addtolength{\cftsubsecindent}{1ex}
\addtolength{\cftsubsecnumwidth}{1ex}
\addtolength{\cftfignumwidth}{1ex}
\addtolength{\cfttabnumwidth}{1ex}

% Spacing for multi-column and enumerate environments.
\setlength{\multicolsep}{6pt}
\setlist[enumerate]{itemsep=0pt,topsep=3pt}

% Indent and paragraph spacing.
\setlength{\parindent}{0em}
\setlength{\parskip}{0em}
%--------------------------Main Document----------------------------%
\begin{document}
    \title{Functional Analysis: Homework}
    \author{Ryan Maguire}
    \date{\vspace{-5ex}}
    \maketitle
    \tableofcontents
    \clearpage
    \section{Homework I}
        \begin{problem}
            Show that, for $1\leq{p}\leq{q}\leq\infty$, that
            $\norm{\cdot}_{p}$ and $\norm{\cdot}_{q}$ are strongly
            equivalent.
        \end{problem}
        \begin{solution}[1]
            First, if $X$ is a set, $d_{1},d_{2},d_{3}$ are
            metrics on $X$, if $d_{1}$ is strongly equivalent to
            $d_{2}$, and $d_{2}$ is strongly equivalent to $d_{3}$,
            then $d_{1}$ is strongly equivalent to $d_{3}$.
            For if $d_{1}$ is strongly equivalent to $d_{2}$, then
            there are $\alpha,\beta>0$ such that, for all
            $x,y\in{X}$:
            \begin{equation}
                \alpha{d}_{1}(x,y)\leq{d}_{2}(x,y)
                \leq\beta{d}_{1}(x,y)
            \end{equation}
            But we have seen that strongly equivalent is a symmetric
            relation, and therefore if $d_{2}$ and $d_{3}$ are
            strongly equivalent then there are $a,b>0$ such that:
            \begin{equation}
                ad_{3}(x,y)\leq{d}_{2}(x,y)\leq{b}d_{3}(x,y)
            \end{equation}
            Therefore:
            \begin{equation}
                \frac{\alpha}{b}d_{1}(x,y)\leq{d}_{3}(x,y)
                \leq\frac{\beta}{a}d_{1}(x,y)
            \end{equation}
            Thus, strongly equivalent is a transitive relation.
            Let $n\in\mathbb{N}$, $p\in[1,\infty)$, and let
            $\mathbf{x}\in\mathbb{R}^{n}$. Then, since
            $\norm{\mathbf{x}}_{\infty}$ is the supremum norm,
            it is greater than or equal to all of the compononents
            of $\mathbf{x}$. Thus:
            \begin{equation}
                n\norm{\mathbf{x}}_{\infty}^{p}=
                \sum_{k=1}^{n}\norm{\mathbf{x}}_{\infty}^{p}
                \geq\sum_{k=1}^{n}|x_{k}|^{p}
                =\norm{\mathbf{x}}_{p}^{p}
            \end{equation}
            Taking $p^{th}$ roots, we have:
            \begin{equation}
                n^{\frac{1}{p}}\norm{\mathbf{x}}_{\infty}
                \geq\norm{\mathbf{x}}_{p}
            \end{equation}
            Going the other way:
            \begin{equation}
                \norm{\mathbf{x}}_{\infty}^{p}\leq
                \sum_{k=1}^{n}|x_{k}|^{p}
            \end{equation}
            Taking $p^{th}$ roots again, we obtain the following:
            \begin{equation}
                \norm{\mathbf{x}}_{\infty}\leq\norm{\mathbf{x}}_{p}
                \leq{n}^{\frac{1}{p}}\norm{\mathbf{x}}_{\infty}
            \end{equation}
            Thus, for all $p\in[1,\infty)$, $\norm{\cdot}_{p}$
            is strongly equivalent to $\norm{\cdot}_{\infty}$.
            But strongly equivalent is a transitive relation,
            thus for all $p,q\in[1,\infty]$, $\norm{\cdot}_{p}$
            is strongly equivalent to $\norm{\cdot}_{q}$.
        \end{solution}
        \begin{solution}[2]
            Let $p\in[1,\infty]$, and let Let
            $f:S^{n}\rightarrow\mathbb{R}$ be defined by:
            \begin{equation}
                f(\mathbf{x})=\norm{\mathbf{x}}_{p}
            \end{equation}
            But $S^{n}$ is a closed and bounded subset of
            $\mathbb{R}^{n}$, and is therefore, by the Heine-Borel
            theorem, it is compact. But continuous functions
            attain their maximum and minimum on compact sets, by
            the extreme value theorem. Thus, there are
            $\mathbf{x}_{\min}$ and $\mathbf{x}_{\max}$ such
            that, for all $\mathbf{x}\in{S}^{n}$:
            \begin{equation}
                \norm{\mathbf{x}_{\min}}_{p}\leq
                \norm{\mathbf{x}}_{p}\leq
                \norm{\mathbf{x}_{\max}}_{p}
            \end{equation}
            But also $\norm{\mathbf{x}_{\min}}_{p}>0$, for
            $\mathbf{x}_{\min}\in{S}^{n}$, and therefore
            $\mathbf{x}_{\min}\ne\mathbf{0}$. Moreover:
            \begin{equation}
                \norm{e_{i}}_{p}=1
            \end{equation}
            For all $i\in\mathbb{Z}_{n}$. But also, for all
            $\mathbf{x}\in{S}^{n}$, we have:
            \begin{equation}
                \norm{\mathbf{x}}_{2}=1
            \end{equation}
            Therefore, for all $\mathbf{x}\in{S}^{n}$:
            \begin{equation}
                \frac{\norm{\mathbf{x}_{\min}}_{p}}
                     {\norm{\mathbf{x}_{\max}}_{p}}
                \norm{\mathbf{x}}_{p}\leq
                \norm{\mathbf{x}}_{2}\leq
                \frac{\norm{\mathbf{x}_{\max}}_{p}}
                     {\norm{\mathbf{x}_{\min}}_{p}}
                \norm{\mathbf{x}}_{p}
            \end{equation}
            For the general
            $\mathbf{x}\in\mathbb{R}^{n}\setminus\{\mathbf{0}\}$,
            we can scale back to the unit $n$ sphere. Thus,
            $\norm{\cdot}_{p}$ and $\norm{\cdot}_{2}$ are
            strongly equivalent for all $p\in[1,\infty]$. Since
            strongly equivalent is a transitive relation,
            $\norm{\cdot}_{p}$ and $\norm{\cdot}_{q}$ are
            strongly equivalent for all $p,q\in[1,\infty]$.
        \end{solution}
        \begin{problem}
            Give an example of a metric on $\mathbb{R}^{n}$
            that is not strongly equivalent to $\norm{\cdot}_{p}$
            for any $p\in[1,\infty]$.
        \end{problem}
        \begin{solution}
            The discrete metric is not strongly equivalent to
            any of the metrics induced by $\norm{\cdot}_{p}$.
            For suppose not. Then:
            \begin{equation}
                \alpha\norm{\mathbf{x}}_{p}\leq
                d(\mathbf{x},\mathbf{0})\leq{1}
            \end{equation}
            Where $d$ is the discrete metric.
            But $\norm{\mathbf{x}}_{p}$ is unbounded, and thus if
            $\alpha\norm{\mathbf{x}}_{p}\leq{1}$, then $\alpha=0$.
            Thus there is no $\alpha>0$ that satisfies
            the inequality.
        \end{solution}
        \begin{problem}
            Show that, if $a,b\geq{0}$ and $0<\lambda<1$, then:
            \begin{equation}
                a^{\lambda}b^{1-\lambda}
                \leq\lambda{a}+(1-\lambda)b
            \end{equation}
        \end{problem}
        \begin{solution}
            If $a=0$ or $b=0$, then we are done. Suppose not.
            Define $t=ab^{\minus{1}}$.
            Then, if $\lambda\in(0,1)$, and $t\geq{1}$, we have:
            \begin{subequations}
                \begin{align}
                    \lambda(t^{\lambda-1}-1)&\geq0\\
                    \Rightarrow\int_{1}^{t}\lambda
                        \Big(\tau^{\lambda-1}-1\Big)\diff{\tau}
                        &\geq{0}\\
                    \Rightarrow
                    (t^{\lambda}-\lambda{t})-(1-\lambda)&\geq{0}
                \end{align}
            \end{subequations}
            For $t\in(0,1)$, we have:
            \begin{subequations}
                \begin{align}
                    \lambda(t^{\lambda-1}-1)&\leq{0}\\
                    \Rightarrow\int_{t}^{1}\lambda
                        \Big(\tau^{\lambda-1}-1\Big)\diff{\tau}
                        &\leq{0}\\
                    \Rightarrow
                    (1-\lambda)-(t^{\lambda}-\lambda{t})&\leq{0}
                \end{align}
            \end{subequations}
            Combining these two, we have for $t\in(0,\infty)$:
            \begin{equation}
                t^{\lambda}-\lambda{t}\leq{1}-\lambda
            \end{equation}
            But $t=ab^{\minus{1}}$, and thus multiplying through
            by $b$:
            \begin{equation}
                a^{\lambda}b^{\lambda-1}\leq
                \lambda{a}+(1-\lambda)b
            \end{equation}
        \end{solution}
        \begin{problem}
            Prove H\"{o}lder's Inequality: If $p\in(1,\infty)$,
            $p^{\minus{1}}+q^{\minus{1}}=1$, then:
            \begin{equation}
                \norm{fg}_{1}\leq\norm{f}_{p}\norm{g}_{q}
            \end{equation}
        \end{problem}
        \begin{solution}
            For let:
            \par
            \begin{subequations}
                \begin{minipage}[b]{0.49\textwidth}
                    \centering
                    \begin{equation}
                        \tilde{f}=\frac{f}{\norm{f}_{q}}
                    \end{equation}
                \end{minipage}
                \hfill
                \begin{minipage}[b]{0.49\textwidth}
                    \centering
                    \begin{equation}
                        \tilde{g}=\frac{g}{\norm{g}_{q}}
                    \end{equation}
                \end{minipage}
            \end{subequations}
            \par\hfill\par
            Then $\norm{\tilde{f}}_{p}=1$ and
            $\norm{\tilde{g}}_{q}=1$. But if $p\in(1,\infty)$,
            then $p^{\minus{1}}<1$, and thus
            by the previous problem, and since
            $1-p^{\minus{1}}=q^{\minus{1}}$, we have:
            \begin{equation}
                |\tilde{f}(x)\tilde{g}(x)|\leq
                    p^{\minus{1}}|\tilde{f}(x)|^{p}+
                    q^{\minus{1}}|\tilde{g}(x)|^{q}
            \end{equation}
            Integrating, we get:
            \begin{subequations}
                \begin{align}
                    \norm{\tilde{f}\tilde{g}}_{1}
                    &=\int_{\mathbb{R}}
                        |\tilde{f}(x)\tilde{g}(x)|\diff{x}\\
                    &\leq\int_{\mathbb{R}}\Big(
                        p^{\minus{1}}|\tilde{f}(x)|^{p}+
                        q^{\minus{1}}|\tilde{g}(x)|^{q}\Big)
                        \diff{x}\\
                    &=p^{\minus{1}}\norm{\tilde{f}}_{p}^{p}+
                    q^{\minus{1}}\norm{\tilde{g}}_{q}^{q}\\
                    &=p^{\minus{1}}+q^{\minus{1}}
                \end{align}
            \end{subequations}
            But $p^{\minus{1}}+q^{\minus{1}}=1$, and thus
            $\norm{\tilde{f}\tilde{g}}_{1}=1$. But from the
            definition of $\tilde{f}$ and $\tilde{g}$, we can
            multiply through by $\norm{f}_{p}\norm{g}_{q}$ to obtain:
            \begin{equation}
                \norm{fg}_{1}\leq\norm{f}_{p}\norm{g}_{q}
            \end{equation}
        \end{solution}
        \begin{problem}
            Prove Minkowski's Inequality
        \end{problem}
        \begin{problem}
            A limit point of a subspace $(E,d_{E})$ of a metric
            space $(X,d)$ is a point $x\in{X}$ such that there
            exists a sequence $a:\mathbb{N}\rightarrow{E}$ such
            that $a_{n}\rightarrow{x}$. Prove that $E$ is closed
            if and only if it has all of it's limit points.
        \end{problem}
        \begin{solution}
            Suppose $E$ is closed and let $x$ be a limit point
            of $E$. Suppose $x\in{E}^{C}$. But if $E$ is closed,
            then $E^{C}$ is open, and thus there is an $r>0$
            such that:
            \begin{equation}
                B_{r}^{(X,d)}(x)\subseteq{E}^{C}
            \end{equation}
            But if $x$ is a limit point of $E$ then there is a
            sequence $a:\mathbb{N}\rightarrow{E}$ such that
            $a_{n}\rightarrow{x}$. But if $a_{n}\rightarrow{x}$ then
            there is an $N\in\mathbb{N}$ such that, for all
            $n\in\mathbb{N}$ such that $n>N$, it is true that
            $d(x,a_{n})<r/2$. But then, for all $n>N$, we have that:
            \begin{equation}
                a_{n}\in{B}_{r}^{(X,d)}(x)
            \end{equation}
            And thus $a_{n}\in{E}^{C}$. But $a_{n}\in{E}$, a
            contradiction. Thus, $x\in{E}$. Now, suppose $x$ has
            all of it's limit points and suppose $E$
            is not closed. Then $E^{C}$ is not open. But then
            there is an $x\in{E}^{C}$ such that, for all
            $\varepsilon>0$:
            \begin{equation}
                B_{\varepsilon}^{(X,d)}(x)\cap{E}\ne\emptyset
            \end{equation}
            Define the following:
            \begin{equation}
                A_{n}=\Big\{y\in{E}:d(x,y)<\frac{1}{n}\Big\}
            \end{equation}
            Then for all $n\in\mathbb{N}$, $A_{n}$ is non-empty.
            By choice there is a sequence
            $a:\mathbb{N}\rightarrow{E}$ such that, for all
            $n\in\mathbb{N}$, $a_{n}\in{A}_{n}$. But then, for all
            $n\in\mathbb{N}$, $d(a_{n},x)<n^{\minus{1}}$. Thus
            $a_{n}\rightarrow{x}$ and therefore $x$ is a limit point
            of $E$. But $x\in{E}^{C}$, a contradiction as $E$
            contains all of its limit points. Therefore, $E$ is
            closed.
        \end{solution}
        \begin{problem}
            State and prove a result characterizing open sets in
            a metric space in terms of sequences, similar to
            the previous problem.
        \end{problem}
        \begin{solution}
            A subset $\mathcal{U}\subseteq{X}$ of a metric space
            $(X,d)$ is open if and only if for any convergence
            sequence $a:\mathbb{N}\rightarrow{X}$ such that the
            limit of $a$ is in $\mathcal{U}$, there is an
            $N\in\mathbb{N}$ such that, for all $n\in\mathbb{N}$ and
            $n>N$, it is true that $a_{n}\in\mathcal{U}$. For
            suppose $\mathcal{U}$ is open, and let
            $a:\mathbb{N}\rightarrow{X}$ be a convergent sequence
            such that there is an $x\in\mathcal{U}$ such that
            $a_{n}\rightarrow{x}$. But if $\mathcal{U}$ is open
            then there is an $\varepsilon>0$ such that:
            \begin{equation}
                B_{\varepsilon}^{(X,d)}(x)\subseteq\mathcal{U}
            \end{equation}
            But if $a_{n}\rightarrow{x}$ then there is an
            $N\in\mathbb{N}$ such that, for all $n\in\mathbb{N}$
            such that $n>N$, it is true that
            $d(x,a_{n})<\varepsilon$. But then for all $n>N$,
            $n\in\mathbb{N}$, we have
            $a_{n}\in{B}_{\varepsilon}^{(X,d)}(x)$, and thus
            $a_{n}\in\mathcal{U}$. Now suppose for any sequence
            that converges to a point in $\mathcal{U}$, the sequence
            is eventually contained within $\mathcal{U}$. Suppose
            $\mathcal{U}$ is not open. Then there is an
            $x\in\mathcal{U}$ such that, for all $\varepsilon>0$
            there is a $y\in\mathcal{U}^{C}$ such that
            $d(x,y)<\varepsilon$. Define the following:
            \begin{equation}
                A_{n}=
                \Big\{y\in\mathcal{U}^{C}:d(x,y)<\frac{1}{n}\Big\}
            \end{equation}
            Then for all $n\in\mathbb{N}$, $A_{n}$ is non-empty.
            By choice there is a sequence
            $a:\mathbb{N}\rightarrow\mathcal{U}^{C}$ such
            that $a_{n}\in{A}_{n}$. But then $a_{n}\rightarrow{x}$.
            But if $a_{n}\rightarrow{x}$, then there is an
            $N\in\mathbb{N}$ such that for all $n\in\mathbb{N}$ such
            that $n>N$ it is true that $a_{n}\in\mathcal{U}$, a
            contradiction. Therefore, $\mathcal{U}$ is open.
        \end{solution}
        \begin{problem}
            Let $\rho$ and $\sigma$ be metrics on $X$ and show that
            $\rho$ and $\sigma$ are equivalent if and only if
            they have the same convergent sequences.
        \end{problem}
        \begin{solution}
            For a sequence $a:\mathbb{N}\rightarrow{X}$ converges to
            $x\in{X}$ if and only if for all open subsets
            $\mathcal{U}\subseteq{X}$ such that $x\in\mathcal{U}$,
            there is an $N\in\mathbb{N}$ such that, for all
            $n\in\mathbb{N}$ and $n>N$, it is true that
            $a_{n}\in\mathcal{U}$. Going one way, if
            $a_{n}\rightarrow{x}$ then for all $\varepsilon>0$
            there is an $N\in\mathbb{N}$ such that for all
            $n\in\mathbb{N}$ and $n>N$, it is true that
            $d(x,a_{n})<\varepsilon$. Let $\mathcal{U}$ be an open
            subset such that $x\in\mathcal{U}$. But then there is
            an $\varepsilon>0$ such that:
            \begin{equation}
                B_{\varepsilon}^{(X,d)}(x)\subseteq\mathcal{U}
            \end{equation}
            But there is an $N\in\mathbb{N}$ such that, for all
            $n>N$, $n\in\mathbb{N}$, we have
            $a_{n}\in{B}_{\varepsilon}^{(X,d)}(x)$. Therefore
            $a_{n}\in\mathcal{U}$. Going the other way, let
            $a:\mathbb{N}\rightarrow{X}$ be a sequence such that,
            for every open set $\mathcal{U}\subseteq{X}$ such
            that $x\in\mathcal{U}$, there is an $N\in\mathbb{N}$
            such that, for all $n\in\mathbb{N}$ and $n>N$, it is
            true that $a_{n}\in\mathcal{U}$. Let:
            \begin{equation}
                A_{k}=B_{k^{\minus{1}}}^{(X,d)}(x)
            \end{equation}
            Given $\varepsilon>0$ there is a $k\in\mathbb{N}$
            such that $k^{\minus{1}}<\varepsilon$. But $A_{k}$ is
            open and $x\in{A}_{k}$, and thus there is an
            $N\in\mathbb{N}$ such that for all $n>N$ and
            $n\in\mathbb{N}$, we have that $a_{n}\in{A}_{k}$. But
            then $d(x,a_{n})<\varepsilon$, so $a_{n}\rightarrow{x}$.
            This converts the notion of convergence from a metric
            space property to a topological property. If $(X,\rho)$
            and $(X,\sigma)$ are equivalent then
            they have the same open sets, and thus convergence of
            sequences is preserved. For suppose
            $a:\mathbb{N}\rightarrow{X}$ converges to $x$ with
            respect to $\rho$. Then, for all open subsets
            $\mathcal{U}$ of $(X,d)$ such that $x\in\mathcal{U}$,
            there is an $N\in\mathbb{N}$ such that, for all
            $n\in\mathbb{N}$ and $n>N$, it is true that
            $a_{n}\in\mathcal{U}$. But if $\mathcal{U}$ is open in
            $(X,\rho)$, then it is open in $(X,\sigma)$, for the
            two metrics are equivalent. Thus $a_{n}\rightarrow{x}$
            with respect to $\sigma$.
        \end{solution}
        \begin{problem}
            Let $(X,\mathcal{M},\mu)$ be a measure space and define
            $\mathcal{U}\subseteq{L}^{1}(X)$ by:
            \begin{equation}
                \mathcal{U}=
                    \Big\{f\in{L}^{1}(X):
                        \int_{X}\Re(f)\diff{\mu}<1\Big\}
            \end{equation}
            Prove that $\mathcal{U}$ is open with respect to the
            metric induced by $\norm{\cdot}_{1}$.
        \end{problem}
        \begin{solution}
            For let $f\in{L}^{1}(X)$ and let $M=\norm{f}_{1}$. As
            $f\in{L}^{1}(X)$, $M<\infty$. Define:
            \begin{equation}
                \alpha=\int_{X}\Re(f)\diff{\mu}
            \end{equation}
            And let $\varepsilon=\min\{1/2M,1-\alpha\}$.
            Then if $\norm{f-g}_{1}<\varepsilon$ we have:
            \begin{equation}
                \int_{X}\Re(g)\diff{\mu}=
                \int_{X}\Re(g-f+f)\diff{\mu}=
                \int_{X}\Re(g-f)\diff{\mu}+\int_{X}\Re(f)\diff{\mu}
            \end{equation}
            We can simplify this further to get:
            \begin{equation}
                \int_{X}\Re(g)\diff{\mu}
                =\int_{X}\Re(g-f)\diff{\mu}+\alpha
                <\varepsilon+\alpha\leq{1}
            \end{equation}
            Therefore:
            \begin{equation}
                B_{\varepsilon}^{(L^{1}(X),\norm{\cdot}_{1})}(f)
                \subseteq\mathcal{U}
            \end{equation}
            And thus $\mathcal{U}$ is open.
        \end{solution}
        \begin{problem}
            For a metric space $(X,d)$, define
            $\dist:X\times\mathcal{P}(X)\setminus\{\emptyset\}%
             \rightarrow[0,\infty)$ by:
            \begin{equation}
                \dist(x,A)=\inf\{d(x,y):y\in{A}\}
            \end{equation}
            Prove that $\dist(x,A)=0$ if and only if
            $x\in\overline{A}$. Show that, for a fixed non-empty
            $A\subseteq{X}$, $\dist(x,A)$ is continuous. Prove that
            if $A,B\subseteq{X}$ are closed disjoint non-empty
            subsets, then there is a continuous function
            $f:X\rightarrow[0,1]$ such that $f(x)=0$
            if and only if $x\in{A}$ and $f(y)=1$ if
            and only if $x\in{B}$.
        \end{problem}
        \begin{solution}
            If $x\in\overline{A}$, then for all $\varepsilon>0$:
            \begin{equation}
                B_{\varepsilon}^{(X,d)}(x)\cap{A}\ne\emptyset
            \end{equation}
            Thus $\dist(x,A)<\varepsilon$ for all positive
            $\varepsilon$, and therefore $\dist(x,A)=0$. If
            $\dist(x,A)=0$ then for all $\varepsilon>0$ there is a
            $y\in{A}$ such that $d(x,y)<\varepsilon$. Therefore
            $x$ is a limit point of $A$, and thus $x\in\overline{A}$.
            \par\hfill\par
            Let $\varepsilon>0$ and let $x\in{X}$, and let
            $\delta=\varepsilon$. Then:
            \begin{subequations}
                \begin{align}
                    \dist(y,A)&=\inf\{d(y,z):z\in{a}\}\\
                    &\leq\inf\{d(x,y)+d(x,z):z\in{A}\}\\
                    &=d(x,y)+\inf\{d(x,z):z\in{A}\}\\
                    &=d(x,y)+\dist(x,A)
                \end{align}
            \end{subequations}
            Similarly:
            \begin{equation}
                \dist(x,A)\leq{d}(x,y)+\dist(y,A)
            \end{equation}
            And therefore, if $d(x,y)<\varepsilon$:
            \begin{equation}
                \big|\dist(x,A)-\dist(y,A)\big|\leq{d}(x,y)
                <\varepsilon
            \end{equation}
            Finally, let:
            \begin{equation}
                f(x)=\frac{\dist(x,B)}{\dist(x,A)+\dist(x,B)}
            \end{equation}
            As $A$ and $B$ are disjoint and closed, $f(x)$ is well
            defined for all $x\in{X}$. Moreover,
            $0\leq{f(x)}\leq{1}$. If $f(x)=0$, then $\dist(x,B)=0$,
            and thus $x\in\overline{B}$. But $B$ is closed, and
            therefore $x\in{B}$. If $f(x)=1$, then
            $\dist(x,A)=0$, and thus $x\in\overline{A}$. Again, as
            $A$ is closed, $x\in{A}$. But $\dist(x,B)$ is
            continuous, and $\dist(x,A)+\dist(x,B)$ is
            continuous, and the quotient of continuous functions
            is continuous. Thus, $f$ is continuous.
        \end{solution}
        \begin{problem}
            Show that a Cauchy sequence with a convergent
            subsequence is convergent.
        \end{problem}
        \begin{solution}
            For let $a:\mathbb{N}\rightarrow{X}$ be a Cauchy
            sequence and let $k:\mathbb{N}\rightarrow\mathbb{N}$ be
            such that $a\circ{k}$ is a convergent subsequence, and
            let $x$ be the limit. That is, $k$ is a stricly
            monotonically increasing sequence of natural numbers.
            Let $\varepsilon>0$. Then there is an
            $N_{1}\in\mathbb{N}$ such that, for all $k_{n}>N_{1}$,
            $n\in\mathbb{N}$, it is true that
            $d(x,a_{k_{n}})<\varepsilon/2$. But $a$ is
            Cauchy, and thus there is an $N_{2}\in\mathbb{N}$
            such that, for all $n,m\in\mathbb{N}$ such that
            $n,m>N_{2}$, it is true that
            $d(a_{n},a_{m})<\varepsilon/2$. Let
            $N=\max\{N_{1},N_{2}\}$. Then for all $n>N$,
            $n\in\mathbb{N}$, $k_{n}>N$ since $k$ is increasing,
            and thus:
            \begin{equation}
                d(a_{n},x)\leq
                d(a_{n},a_{k_{n}})+d(a_{k_{n}},x)<\varepsilon
            \end{equation}
            Therefore, $a_{n}\rightarrow{x}$.
        \end{solution}
        \begin{problem}
            Let $F:\mathbb{N}\times{X}\rightarrow\mathbb{C}$ be a
            sequence of continuous functions and let
            $f:X\rightarrow\mathbb{C}$ be such that
            $F_{n}(x)\rightarrow{f}$ uniformly. Show that $f$ is
            continuous.
        \end{problem}
        \begin{solution}
            For let $\varepsilon>0$. As $F_{n}\rightarrow{f}$
            uniformly, there is an $N\in\mathbb{N}$ such that for
            all $x\in{X}$, it is true that:
            \begin{equation}
                |F_{N}(x)-f(x)|<\frac{\varepsilon}{3}
            \end{equation}
            But $F_{N}(x)$ is continuous, and thus for $x\in{X}$
            there is a $\delta>0$ such that $d(x,x_{0})<\delta$
            implies that:
            \begin{equation}
                |F_{N}(x)-F_{N}(x_{0})|<\frac{\varepsilon}{3}
            \end{equation}
            But then:
            \begin{subequations}
                \begin{align}
                    |f(x)-f(x_{0})|&\leq
                    |f(x)-F_{N}(x)|+|F_{N}(x)-F_{N}(x_{0})|+
                    |F_{N}(x_{0})-f(x_{0})|\\
                    &<\varepsilon
                \end{align}
            \end{subequations}
            Thus, $f$ is continuous.
        \end{solution}
        \begin{problem}
            Let $X$ be a metric space. Recall that we say
            $f:X\rightarrow\mathbb{C}$ is bounded if
            $\norm{f}_{\infty}<\infty$. A sequence of functions
            $f_{n}:X\rightarrow{D}$ is uniformly bounded if there
            is an $M$ such that $\norm{f_{n}}_{\infty}<M$ for
            all $n$. Also, $f_{n}$ is uniformly Cauchy if for
            all $\varepsilon>0$ there is an $N\in\mathbb{N}$
            such that $n,m>N$ implies
            $|f_{n}(x)-f_{m}(x)|<\varepsilon$ for all $x\in{X}$.
            Show that a uniformly Cauchy sequence $f_{n}$ of
            bounded functions is uniformly bounded.
        \end{problem}
        \begin{solution}
            For let $F:\mathbb{N}\times{X}\rightarrow\mathbb{C}$
            be a sequence of functions such that, for all
            $n\in\mathbb{N}$, there is an $M_{n}\in\mathbb{R}^{+}$
            such that:
            \begin{equation}
                \norm{F_{n}}_{\infty}<M_{n}
            \end{equation}
            And such that $F$ is uniformly Cauchy. Let
            $\varepsilon=1$. Then, as $F$ is uniformly Cauchy,
            there is an $N\in\mathbb{N}$ such that, for all
            $n,m\in\mathbb{N}$ such that $n,m>N$, and for all
            $x\in{X}$, it is true that:
            \begin{equation}
                |F_{n}(x)-F_{m}(x)|<\varepsilon
            \end{equation}
            Then, for all $n>N$ and for all $x\in{X}$:
            \begin{subequations}
                \begin{align}
                    |F_{n}(x)|&=
                    |F_{n}(x)+F_{N+1}(x)-F_{N+1}(x)|\\
                    &\leq|F_{n}(x)-F_{N+1}(x)|+|F_{N+1}(x)|\\
                    &<\varepsilon+M_{N+1}\\
                    &=M_{N+1}+1
                \end{align}
            \end{subequations}
            Let:
            \begin{equation}
                M=\max\Big(
                    \{M_{n}:n\in\mathbb{Z}_{N}\}\cup\{M_{N+1}+1\}
                \Big)
            \end{equation}
            Then for all $n\in\mathbb{N}$, and for all $x\in{X}$:
            \begin{equation}
                |F_{n}(x)|\leq{M}
            \end{equation}
            Therefore, $F$ is uniformly bounded.
        \end{solution}
        \begin{problem}
            We say that $D$ is dense in $X$ if $\overline{D}=X$.
            Show that $D$ is dense if and only if $D$ meets every
            non-empty open subset.
        \end{problem}
        \begin{solution}
            Suppose $D$ is a dense subset of $X$ and let
            $\mathcal{U}\subseteq{X}$ be an open subset. Suppose
            $\mathcal{U}\cap{D}=\emptyset$. Let $x\in\mathcal{U}$. As
            $\mathcal{U}$ is open, there is an $r>0$ such that:
            \begin{equation}
                B_{r}^{(X,d)}(x)\subseteq\mathcal{U}
            \end{equation}
            But if $D$ is dense in $X$, then $x$ is a limit point
            of $D$. Thus there is a sequence
            $a:\mathbb{N}\rightarrow{D}$ such
            that $a_{n}\rightarrow{x}$. But if $a_{n}\rightarrow{x}$,
            then there is an $N\in\mathbb{N}$ such that, for all
            $n\in\mathbb{N}$ and $n>N$, we have:
            \begin{equation}
                a_{n}\in{B}_{r}^{(X,d)}(x)
            \end{equation}
            But then, $a_{n}\in\mathcal{U}$, a contradiction as
            as $a_{n}\in{D}$ and $\mathcal{U}$ and $\mathcal{D}$
            are disjoint. Therefore, etc. Now suppose
            $D$ meets every open set. Suppose
            $x\notin\overline{D}$. Define the following:
            \begin{equation}
                A_{n}=\Big\{y\in{B}_{1/n}^{(X,d)}(x):y\in{D}\Big\}
            \end{equation}
            Then $A_{n}$ is non-empty for all $n\in\mathbb{N}$,
            since $D$ meets every open set. By choice there is
            a sequence $a:\mathbb{N}\rightarrow{D}$ such
            that $a_{n}\in{A}_{n}$ for all $n\in\mathbb{N}$. But
            then $x$ is a limit point of $D$, a contradiction.
            Thus, $\overline{D}=X$.
        \end{solution}
        \begin{problem}
            Let $(x_{n})$ be a sequence in a complete metric
            space $(X,\rho)$. Suppose that
            $\rho(x_{n},x_{n+1})<2^{\minus{n}}$ for all
            $n\in\mathbb{N}$. Conclude that $(x_{n})$ is
            convergent. What if instead we have that
            $\rho(x_{n},x_{n+1})<1/n$?
        \end{problem}
        \begin{solution}
            For let $\varepsilon>0$. Let $N\in\mathbb{N}$ such that
            $2^{1-N}<\varepsilon$. But then for $n,m>N$:
            \begin{equation}
                \rho(x_{n},x_{m})\leq
                \sum_{k=\min(n,m)}^{\max(n,m)}d(x_{k},x_{k+1})
                \leq\sum_{k=N}^{\infty}d(x_{k},x_{k_{n+1}})
                \leq\sum_{k=N}^{\infty}\frac{1}{2^{k}}
            \end{equation}
            But by applying the geometric series, we have:
            \begin{equation}
                \sum_{k=N}^{\infty}\frac{1}{2^{k}}
                =2^{1-N}<\varepsilon
            \end{equation}
            Thus $x_{n}$ is Cauchy, and Cauchy sequences
            converge in a complete metric space. Therefore, etc.
            If we replace $2^{\minus{n}}$ with $n^{\minus{1}}$,
            the result may not hold. For let $X=\mathbb{R}$, which
            is complete with the standard metric, and let
            $a:\mathbb{N}\rightarrow\mathbb{R}$ be defined
            by $a_{n}=\ln(n)$. Applying some calculus, we have:
            \begin{equation}
                d(a_{n+1},a_{n})=\ln(n+1)-\ln(n)
                =\int_{n}^{n+1}\frac{1}{x}\diff{x}<\frac{1}{n}
            \end{equation}
            But $\ln(n)$ is not a convergent sequence.
        \end{solution}
        \begin{problem}
            A metric space is separable if it has a countable dense
            subset. Show that a metric space $X$ is separable if
            and only if there is a countable family $\mathcal{D}$
            of open sets such that every open set in $X$ is the union
            of open sets in $\mathcal{D}$.
        \end{problem}
        \begin{solution}
            For let $(X,d)$ be a separable metric space, and let
            $A$ be a countable dense subset. Let:
            \begin{equation}
                \mathcal{D}=\bigcup_{n\in\mathbb{N}}
                \bigcup_{a\in{A}}B_{n^{\minus{1}}}^{(X,d)}(a)
            \end{equation}
            Then $\mathcal{D}$ is countable, and
            for all $\mathcal{U}\in\mathcal{D}$, $\mathcal{U}$
            is open. Let $\mathcal{O}$ be an open subset of $X$.
            Define:
            \begin{equation}
                \mathcal{V}=\{\mathcal{U}\in\mathcal{D}:
                    \mathcal{U}\subseteq\mathcal{O}\}
            \end{equation}
            Then:
            \begin{equation}
                \bigcup_{\mathcal{U}\in\mathcal{V}}\mathcal{U}
                \subseteq\mathcal{O}
            \end{equation}
            Suppose the converse is false, and let
            $x\in\mathcal{O}$ be such that it is not contained
            in this union. But $\mathcal{O}$ is open, and thus
            there is an $r>0$ such that:
            \begin{equation}
                B_{r}^{(X,d)}(x)\subseteq\mathcal{O}
            \end{equation}
            But by the Archimedean principle, there is an
            $n\in\mathbb{N}$ such that $n^{\minus{1}}<r/2$. But $A$
            is dense in $\mathcal{O}$ and thus there is a $y\in{A}$
            such that $d(x,y)<n^{-1}$. But then:
            \begin{equation}
                x\in{B}_{n^{\minus{1}}}^{(X,d)}(y)
                \subseteq{B}_{r}^{(X,d)}(x)\subseteq\mathcal{O}
            \end{equation}
            And thus $x$ is contained in this union, a contradiction.
            Therefore, etc. Going the other way, suppose $(X,d)$
            is a metric space such that there exists a countable set
            $\mathcal{D}$ of open subsets of $X$ such that, for all
            open sets $\mathcal{O}$, $\mathcal{O}$ is the union
            of elements of $\mathcal{D}$. That is, there is a sequence
            $A:\mathbb{N}\rightarrow\mathcal{P}(X)$ such that:
            \begin{equation}
                \mathcal{D}=\{A_{n}:n\in\mathbb{N}\}
            \end{equation}
            But then by choice there is a sequence:
            \begin{equation}
                a:\mathbb{N}\rightarrow\bigcup_{n\in\mathbb{N}}A_{n}
            \end{equation}
            Such that, for all $n\in\mathbb{N}$, $a_{n}\in{A}_{n}$.
            Let $y\in{X}$ and let $\varepsilon>0$, define:
            \begin{equation}
                \mathcal{V}=B_{\varepsilon}^{(X,d)}(y)
            \end{equation}
            But then $\mathcal{V}$ is an open subset of
            $(X,d)$ and is therefore the union of elements of
            $\mathcal{D}$. That is, there is a sequence
            $k:\mathbb{N}\rightarrow\mathbb{N}$ such that:
            \begin{equation}
                \mathcal{V}=
                \bigcup_{n\in\mathbb{N}}A_{k_{n}}
            \end{equation}
            Where we write $k_{n}$ to denote $k(n)$. But then:
            \begin{equation}
                d(y,a_{k_{n}})<\varepsilon
            \end{equation}
            For all $n\in\mathbb{N}$. Thus the set:
            \begin{equation}
                \mathcal{A}=\{a_{n}:n\in\mathbb{N}\}
            \end{equation}
            Is a a countable dense subset, and $(X,d)$ is separable.
        \end{solution}
    \section{Homework II}
        \begin{problem}
            Show that $X$ is compact if and only if every collection
            of closed sets $\mathcal{F}$ with the finite intersection
            property is such that:
            \begin{equation}
                \bigcap_{F\in\mathcal{F}}F\ne\emptyset
            \end{equation}
        \end{problem}
        \begin{solution}
            For suppose $X$ is compact and suppose there is a
            collection $\mathcal{F}$ of closed sets with the finite
            intersection property such that:
            \begin{equation}
                \bigcap_{F\in\mathcal{F}}F=\emptyset
            \end{equation}
            But, for all $F\in\mathcal{F}$, $F$ is closed, and
            therefore $F^{C}$ is open. But then:
            \begin{equation}
                X=\emptyset^{C}
                =\Big(\bigcap_{F\in\mathcal{F}}F\Big)^{C}
                =\bigcup_{F\in\mathcal{F}}F^{C}
            \end{equation}
            And thus:
            \begin{equation}
                \mathcal{O}=
                    \{F^{C}:F\in\mathcal{F}\}
            \end{equation}
            Is an open cover of $X$. But $X$ is compact, and
            therefore there is a finite subcover
            $\Delta\subseteq\mathcal{O}$. But then:
            \begin{equation}
                \emptyset=
                X^{C}=\Big(\bigcup_{\mathcal{U}\in\Delta}
                    \mathcal{U}\Big)^{C}=
                    \bigcap_{\mathcal{U}\in\Delta}
                    \mathcal{U}^{C}
            \end{equation}
            But $\mathcal{U}^{C}\in\mathcal{F}$ for all
            $\mathcal{U}\in\Delta$. And $\Delta$ is finite. Thus
            there is a finite subset of $\mathcal{F}$ such that
            the intersection is empty, a contradiction as
            $\mathcal{F}$ has the finite intersection property.
            Therefore, etc. Now suppose $X$ is such that every
            collection of closed sets $\mathcal{F}$ with the
            finite intersection property is such that the
            intersection over all elements is non-empty. Suppose
            $X$ is not compact. Then there is an open cover
            $\mathcal{O}$ such that there is no finite subcover.
            Let:
            \begin{equation}
                \mathcal{F}=\{\mathcal{U}^{C}:
                    \mathcal{U}\in\mathcal{O}\}
            \end{equation}
            But then for any finite subset, the intersection is
            non-empty. For if not then $\mathcal{O}$ has a finite
            subcover, which it does not. But then $\mathcal{F}$ is
            a collection of closed sets with
            the finite intersection property, and therefore:
            \begin{equation}
                \bigcap_{F\in\mathcal{F}}F\ne\emptyset
            \end{equation}
            But:
            \begin{equation}
                \emptyset=X^{C}=\Big(
                    \bigcup_{\mathcal{U}\in\mathcal{O}}\mathcal{U}
                \Big)^{C}
                =\bigcap_{F\in\mathcal{F}}F
            \end{equation}
            A contradiction. Therefore, $X$ is compact.
        \end{solution}
        \begin{problem}
            Show that $E\subseteq{X}$ is totally bounded if and
            only if there is a finite $\varepsilon\textrm{-net}$
            for all $\varepsilon>0$.
        \end{problem}
        \begin{solution}
            If $E\subseteq{X}$ is totally bounded, then for all
            $\varepsilon>0$ there are finitely many points
            $x_{k}$, $k\in\mathbb{Z}_{n}$ such that:
            \begin{equation}
                E\subseteq\bigcup_{k=1}^{n}
                    B_{\varepsilon}^{(X,d)}(x_{k})
            \end{equation}
            But then:
            \begin{equation}
                \mathcal{O}=\{B_{\varepsilon}^{(X,d)}(x_{k}):
                    k\in\mathbb{Z}_{n}\}
            \end{equation}
            Is a finite $\varepsilon\textrm{-net}$ of $E$. If,
            for all $\varepsilon>0$, there is a finite
            $\varepsilon\textrm{-net}$ of $E$, then there are
            finitely many points $x_{k}$, $k\in\mathbb{Z}_{n}$
            such that:
            \begin{equation}
                \mathcal{O}=\{B_{\varepsilon}^{(X,d)}(x_{k}):
                    k\in\mathbb{Z}_{n}\}
            \end{equation}
            Is an open cover of $E$. But then for all
            $\varepsilon>0$ there are finitely many open balls that
            cover $E$, and therefore $E$ is totally bounded.
        \end{solution}
        \begin{problem}
            Suppose $(X,d_{X})$ is compact and that
            $f:(X,d_{X})\rightarrow(Y,d_{Y})$ is continuous.
            Show that $f(X)$ is compact.
        \end{problem}
        \begin{solution}
            For let $\mathcal{O}$ be an open cover of $f(X)$.
            But $f$ is continuous, and thus for all
            $\mathcal{U}\in\mathcal{O}$,
            $f^{\minus{1}}(\mathcal{U})$ is an open subset of
            $X$. But then:
            \begin{equation}
                \Delta=\{f^{\minus{1}}(\mathcal{U}):
                    \mathcal{U}\in\mathcal{O}\}
            \end{equation}
            Is an open cover of $X$. But $X$ is compact, and thus
            there is a finite sub-cover $\Lambda$. But then:
            \begin{equation}
                \mathscr{O}=
                \{\mathcal{U}:f^{\minus{1}}(\mathcal{U})\in\Lambda\}
            \end{equation}
            Is is a finite subcover of $f(X)$, and therefore
            $f(X)$ is compact.
        \end{solution}
        \begin{problem}
            Let $X=(0,1)$ and let $\delta_{x}>0$ be such that:
            \begin{equation}
                y\in{B}^{(X,||)}_{\delta_{x}}(x)
                \Longrightarrow
                \Big|\frac{1}{x}-\frac{1}{y}\Big|<1
            \end{equation}
            Show that:
            \begin{equation}
                \mathcal{O}=
                \{B^{(X,||)}_{\delta_{x}}(x):x\in{X}\}
            \end{equation}
            Has no Lebesgue number.
        \end{problem}
        \begin{solution}
            Suppose not, and let $d>0$ be a Lebesgue number. Then
            for all $x\in(0,1)$, there is a
            $\mathcal{U}\in\mathcal{O}$ such that:
            \begin{equation}
                B_{d}^{(X,||)}(x)\subseteq\mathcal{U}
            \end{equation}
            Let $n\in\mathbb{N}$ be such that $n^{\minus{1}}<d$.
            Let $x=n^{\minus{1}}/2$. Then $x\in(0,1)$.
            But $d>n^{\minus{1}}$, and thus:
            \begin{equation}
                B_{d}^{(X,||)}(x)=(0,x+d)
            \end{equation}
            But since $x\in(0,1)$,
            there is a $y\in(0,1)$ such that:
            \begin{equation}
                B_{d}^{(X,||)}(x)\subseteq
                B_{\delta_{y}}^{(X,||)}(y)
            \end{equation}
            But then for all $z\in(0,x+d)$, we have:
            \begin{equation}
                \Big|\frac{1}{z}-\frac{1}{y}\Big|<1
            \end{equation}
            Let $N\in\mathbb{N}$ be such that
            $N>x^{\minus{1}}+y^{\minus{1}}+2$.
            But then $N^{\minus{1}}\in(0,x+d)$, and thus:
            \begin{equation}
                \Big|\frac{1}{N^{\minus{1}}}-\frac{1}{y}\Big|<1
            \end{equation}
            But:
            \begin{equation}
                \Big|\frac{1}{N^{\minus{1}}}-\frac{1}{y}\Big|
                =|N-y^{\minus{1}}|>2
            \end{equation}
            A contradiction. Thus, $d$ is not a Lebesgue number.
        \end{solution}
        \begin{problem}
            Show that a compact metric space has a countable
            dense subset.
        \end{problem}
        \begin{solution}
            For if $(X,d)$ is compact, then it is complete and
            totally bounded. But if it is totally bounded, for all
            $n\in\mathbb{N}$ there exists an $N\in\mathbb{N}$ and
            a sequence $a:\mathbb{Z}_{N}\rightarrow{X}$ such that:
            \begin{equation}
                X=\bigcup_{k=1}^{N}B_{n^{\minus{1}}}^{(X,d)}(a_{k})
            \end{equation}
            Define the following:
            \begin{equation}
                A_{n}=\bigcup_{N\in\mathbb{N}}
                    \Big\{a:\mathbb{Z}_{N}\rightarrow{X}:
                    X=\bigcup_{k=1}^{N}
                    B_{n^{\minus{1}}}^{(X,d)}(a_{k})\Big\}
            \end{equation}
            Then, for all $n\in\mathbb{N}$, $A_{n}$ is non-empty.
            Then by choice there is a sequence:
            \begin{equation}
                f:\mathbb{N}\rightarrow
                \bigcup_{n\in\mathbb{N}}A_{n}
            \end{equation}
            Such that, for all $n$, $f_{n}\in{A}_{n}$. Let:
            \begin{equation}
                \mathcal{D}=\bigcup_{n\in\mathbb{N}}
                    \textrm{Im}(f_{n})
            \end{equation}
            Where $\textrm{Im}$ denotes the image of $f_{n}$. From
            construction, for all $n\in\mathbb{N}$,
            $\textrm{Im}(f_{n})$ is finite, and thus $\mathcal{D}$
            is the countable union of countable sets, and is
            therefore countable. Moreover,
            $\overline{\mathcal{D}}=X$. For let $x\in{X}$ and let
            $\varepsilon>0$. By the
            Archimedean property, there and an $n\in\mathbb{N}$
            such that $n^{\minus{1}}<\varepsilon$. But:
            \begin{equation}
                X=\bigcup_{y\in{f_{n}}}
                    B_{n^{\minus{1}}}^{(X,d)}(y)
            \end{equation}
            And thus there is a $y\in{f}_{n}$ such that
            $d(x,y)<n^{\minus{1}}$. But if $y\in{f}_{n}$, the
            $y\in\mathcal{D}$. Thus, $x\in\overline{\mathcal{D}}$.
            Therefore $\mathcal{D}$ is a countable dense subset.
        \end{solution}
        \begin{problem}
            Show that the family of functions $\mathcal{F}$ defined
            on $[0,1]$ by $f_{n}(x)=x^{n}$,
            is equicontinuous at each $x\in[0,1)$. 
        \end{problem}
        \begin{solution}[1]
            If $F:\mathbb{N}\times{X}\rightarrow{Y}$ is a
            sequence of continuous functions such that
            $F_{n}\rightarrow{f}$ uniformly, then
            $F$ is point-wise equicontinuous. For let $\varepsilon>0$
            and let $x\in{X}$. But $F_{n}\rightarrow{f}$ unifomly,
            and $F_{n}$ is continuous for all $n\in\mathbb{N}$,
            and therefore $f$ is continuous. But then there is
            a $\delta_{1}>0$ such that, for all $x_{0}\in{X}$
            such that $d_{X}(x,x_{0})<\delta_{1}$, we have that:
            \begin{equation}
                d_{Y}\big(f(x),f(x_{0})\big)
                <\frac{\varepsilon}{3}
            \end{equation}
            But $F_{n}\rightarrow{f}$ uniformly, and thus there is
            an $N\in\mathbb{N}$ such that, for all
            $n>N$ and $n\in\mathbb{N}$, it is true that:
            \begin{equation}
                d_{Y}\big(F_{n}(x),F_{n}(x_{0})\big)
                <\frac{\varepsilon}{3}
            \end{equation}
            But then, for all $n>N$, and for all $x_{0}\in{X}$
            such that $d_{X}(x,x_{0})<\delta_{1}$, we have that:
            \begin{equation}
                \begin{split}
                    d_{Y}\big(F_{n}(x),F_{n}(x_{0})\big)
                    \leq{d}_{Y}&\big(F_{n}(x),f(x)\big)+\\
                    &d_{Y}\big(f(x),f(x_{0})\big)+
                    d_{Y}\big(f(x_{0}),F_{n}(x_{0})\big)<\varepsilon
                \end{split}
            \end{equation}
            But $F$ is continuous, and thus for all
            $n\in\mathbb{Z}_{N}$ there is a $\delta_{n}$ such that
            $d_{X}(x,x_{0})<\delta_{n}$ implies that:
            \begin{equation}
                d_{Y}\big(F_{n}(x),F_{n}(x_{0})\big)<\varepsilon
            \end{equation}
            Let:
            \begin{equation}
                \delta=\min\Big(\{\delta_{0}\}\cup
                    \{\delta_{n}:n\in\mathbb{Z}_{N}\}\Big)
            \end{equation}
            Now, for all $x_{0}<1$, $f_{n}(x)=x^{n}$ tends
            to zero uniformly on $[0,x_{0}]$. Therefore, etc.
        \end{solution}
        \begin{solution}[2]
            For let $\varepsilon>0$, and let $x\in[0,1)$. If
            $x=0$, Let
            $\delta=\varepsilon\min\{\varepsilon,\tfrac{1}{2}\}$.
            Then, for $0\leq{x}_{0}<\delta$, and for all
            $n\in\mathbb{N}$:
            \begin{equation}
                \big|x_{0}^{n}\big|<
                \delta^{n}\leq\varepsilon\Big(\frac{1}{2}\Big)^{n}
                <\varepsilon
            \end{equation}
            Otherwise, let $\delta_{1}=\tfrac{1}{2}\min\{x,1-x\}$
            and let $y=\delta_{1}+x$. Then $0<y<1$. By the mean
            value theorem, for all $x_{0}$ there is a
            $c_{x_{0}}$ such that $|x-c_{x_{0}}|<|x-x_{0}|$ and
            such that:
            \begin{equation}
                \big|x^{n}-x_{0}^{n}\big|
                =nc_{x_{0}}^{n-1}|x-x_{0}|
            \end{equation}
            But then:
            \begin{equation}
                \big|x^{n}-x_{0}^{n}\big|<
                ny^{n}\delta
            \end{equation}
            But, since $0<y<1$, $ny^{n}$ is bounded. For let
            $f:[0,\infty)\rightarrow\mathbb{R}$ be defined by:
            \begin{equation}
                f(x)=\frac{x}{y^{1-x}}
            \end{equation}
            Then by L'H\"{o}pital, we have:
            \begin{equation}
                \underset{x\rightarrow\infty}{\lim}f(x)
                =\underset{x\rightarrow\infty}{\lim}
                    \frac{x}{y^{1-x}}
                =\underset{x\rightarrow\infty}{\lim}
                    \frac{\minus{1}}{y^{1-x}\ln(y)}
                =\underset{x\rightarrow\infty}{\lim}
                    \frac{\minus{y}^{x-1}}{\ln(y)}
            \end{equation}
            But $0<y<1$, and therefore $y^{x-1}\rightarrow{0}$
            as $x\rightarrow{0}$. Therefore:
            \begin{equation}
                \underset{x\rightarrow\infty}{\lim}f(x)=0
            \end{equation}
            But then for any sequence
            $a:\mathbb{N}\rightarrow\mathbb{R}$ such that
            $a_{n}\rightarrow\infty$, we have
            $f(a_{n})\rightarrow{0}$. Therefore,
            $ny^{n-1}$ converges to zero. But convergent sequences
            are bounded sequences, and therefore there is an
            $M\in\mathbb{R}^{+}$ such that, for all
            $n\in\mathbb{N}$:
            \begin{equation}
                \big|ny^{n-1}\big|\leq{M}
            \end{equation}
            Let $\delta=\min\{\tfrac{\varepsilon}{M},\delta_{1}\}$.
            Then for all $x_{0}\in(0,1)$ such that
            $|x-x_{0}|<\delta$, we have:
            \begin{equation}
                \big|x^{n}-x_{0}^{n}\big|=
                nc_{x_{0}}^{n-1}|x-x_{0}|<
                ny^{n-1}\delta<\varepsilon
            \end{equation}
        \end{solution}
        \begin{problem}
            Show that an equicontinuous family of functions on a
            compact metric space is uniformly equicontinuous.
        \end{problem}
        \begin{solution}
            For let $(X,d_{X})$ be a compact metric space and let
            $\mathcal{F}$ be a family of equicontinuous functions
            to a metric space $(Y,d_{Y})$.
            Let $\varepsilon>0$. Then, as $\mathcal{F}$ is
            equicontinuous, for all $x\in{X}$ there exists a
            $\delta_{x}>0$ such that, for all $f\in\mathcal{F}$,
            we have:
            \begin{equation}
                x_{0}\in{B}_{\delta_{x}}^{(X,d_{X})}(x)
                \Longrightarrow
                f(x_{0})\in{B}_{\varepsilon/2}^{(Y,d_{Y})}
                \big(f(x)\big)
            \end{equation}
            But then:
            \begin{equation}
                \mathcal{O}=\Big\{
                    B_{\delta_{x}}^{(X,d_{X})}(x):x\in{X}\Big\}
            \end{equation}
            Is an open cover of $X$. But $(X,d)$ is compact, and
            therefore this cover has a Lebesgue number $\delta>0$.
            If $x\in{X}$, then there is a $y\in{X}$ such that:
            \begin{equation}
                B_{\delta}^{(X,d_{X})}(x)\subseteq
                B_{\delta_{y}}^{(X,d_{X})}(y)
            \end{equation}
            But then, if $d_{X}(x,x_{0})<\delta$, then:
            \begin{equation}
                x_{0}\in{B}_{\delta}^{(X,d_{X})}(x)
                \Rightarrow
                x_{0}\in{B}_{\delta_{y}}^{(X,d_{X})}(y)
                \Rightarrow
                f(x_{0})\in
                B_{\varepsilon/2}^{(Y,d_{Y})}\big(f(y)\big)
            \end{equation}
            And therefore:
            \begin{equation}
                d_{Y}\big(f(x),f(x_{0})\big)\leq
                d_{Y}\big(f(x),f(y)\big)+
                d_{Y}\big(f(y),f(x_{0})\big)<\varepsilon
            \end{equation}
            Thus, $\mathcal{F}$ is uniformly equicontinuous.
        \end{solution}
        \begin{problem}
            Show that a subset of a compact metric space is compact
            if and only if it is closed.
        \end{problem}
        \begin{solution}
            For let $(X,d)$ be a compact metric space and let
            $(E,d_{E})$ be a compact subspace. Suppose $E$ is not
            closed. Then $E^{C}$ is not open, and therefore
            there is an $x\in{E}^{C}$ such that, for all
            $\varepsilon>0$:
            \begin{equation}
                B_{\varepsilon}^{(X,d)}(x)\cap
                E\ne\emptyset
            \end{equation}
            Let:
            \begin{equation}
                \mathcal{O}=\Big\{\textrm{Cl}\Big(
                    B_{\varepsilon}^{(X,d)}(x)\Big)^{C}:
                    \varepsilon\in\mathbb{R}^{+}\Big\}
            \end{equation}
            Where $\textrm{Cl}$ denotes the closure of a set.
            Then $\mathcal{O}$ is an open cover of $E$. But
            $(E,d_{E})$ is compact, and thus there is a finite
            subcover $\Delta$. But then there is a least
            $r\in\mathbb{R}^{+}$ such that:
            \begin{equation}
                \textrm{Cl}
                \Big(B_{r}^{(X,d)}(x)\Big)^{C}\in\Delta
            \end{equation}
            But then, for all $0<\varepsilon<r$, we have:
            \begin{equation}
                B_{\varepsilon}^{(X,d)}(x)\cap
                E=\emptyset
            \end{equation}
            A contradiction. Therefore, $E$ is closed. Suppose
            $(X,d)$ is compact and $E\subseteq{X}$ is closed.
            Suppose $(E,d_{E})$ is not compact. Then there is an
            open cover $\mathcal{O}_{E}$ of $E$ with no finite
            subcover. But $E$ is closed, and thus $E^{C}$ is open.
            But then:
            \begin{equation}
                \mathcal{O}_{X}=\mathcal{O}_{E}\cup
                \big\{E^{C}\big\}
            \end{equation}
            Is an open cover of $X$. But $(X,d)$ is compact,
            and therefore there is a finite subcover $\Delta_{X}$.
            But then:
            \begin{equation}
                \Delta_{E}=\Delta_{X}\setminus\big\{E^{C}\big\}
            \end{equation}
            Is a finite subcover of $\mathcal{O}_{E}$, a
            contradiction. Therefore, $(E,d_{E})$ is compact.
        \end{solution}
\end{document}