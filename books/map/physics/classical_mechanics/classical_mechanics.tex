\section{Notes}
    \subsection{Old Notes on Lagrangians}
        The Lagrangian is defined as $\mathcal{L}=T-V$,
        where $T$ is the kinetic energy and $V$ is the
        potential energy. Hamilton's Principle,
        which we take upon as an axiom of nature,
        states that the action of a system is an extremum.
        The action is defined as:
        \begin{equation*}
            S=\int_{t_1}^{t_2}\mathcal{L}(x,\dot{x},t)dt
        \end{equation*}
        This means that $\mathcal{L}$ satisfies the Euler-Lagrange equation:
        \begin{equation*} 
            \frac{\partial \mathcal{L}}{\partial{x}}
            -
            \frac{d}{dt}\big(
                \frac{\partial\mathcal{L}}{\partial\dot{x}}
            \big)
            =0
        \end{equation*}
        This is the equation of motion in a one-dimensional
        Cartesian system. For general coordinates, we use:
        \begin{equation*}
            \frac{\partial\mathcal{L}}{\partial q}
            -\frac{d}{dt}\big(
                \frac{\partial\mathcal{L}}{\partial\dot{q}}
            \big)
            =0.
        \end{equation*}
        Let's suppose we have a nice ``Physics I,'' style Lagrangian.
        By that I mean that $T = T(t)$, and $V \ne V(t)$.
        That is, the potential energy is a function of position,
        and not of time. Then the equation of motion is:
        \begin{equation*}
            \frac{\partial}{\partial q}(T-V)-\frac{d}{dt}
            \big(\frac{\partial}{\partial\dot{q}}(T-V)\big)=0
            \Rightarrow
            \frac{\partial{V}}{\partial{q}}
            +\frac{d}{dt}\big(
                \frac{\partial{T}}{\partial\dot{q}}
            \big)
            =0
        \end{equation*}
        Since $\frac{\partial{T}}{\partial{q}}=0$
        and $\frac{\partial{V}}{\partial\dot{q}}=0$.
        In a ``Physics I'' problem, $T=\frac{1}{2}m\dot{q}^2$.
        This is the kinetic energy. So we have:
        \begin{equation}
        \nonumber m\ddot{q} = -\frac{\partial V}{\partial q}
        \end{equation}
        Recall that Newton's Second Law says that:
        \begin{equation*}
            m\ddot{x}=-\frac{\partial{V}}{\partial{x}}    
        \end{equation*}
        This final result then looks very
        much like Newton's Second law.
        We thus define the following:
        \begin{definition}
            The Generalized Momentum of a system is
            defined as $\frac{\partial\mathcal{L}}{\partial\dot{q}}$.
        \end{definition}
        \begin{definition}
            The Generalized Force of a system is defined as
            $\frac{\partial\mathcal{L}}{\partial q}$.
        \end{definition}
        When $\mathcal{L}$ is the nice ``Physics I,''
        Lagrangian that we're familiar with,
        we see that Newton's Second Law appears.
        That is, the time derivative of momentum is equal
        to minus the gradient of the potential energy.
        For any Lagrangian, if we use generalized momentum
        and generalized force, then the mathematics becomes
        very similar to Newton's Second Law.
        We obtain the ``Generalized,'' Newton's Second Law:
        \begin{equation*}
            \textit{Generalized Force}
            =
            \frac{d}{dt}\big(
                \textit{Generalized Momentum}
            \big)
        \end{equation*}
        Let's consider the example $q=\theta$.
        That is, our generalized coordinate is the angle
        made between the particle and the $x$ axis.
        The generalized momentum is just angular momentum,
        and the generalized force is angular force
        (Also known as torque). Let's consider the
        following system:
        \begin{equation*}
            \frac{\partial\mathcal{L}}{\partial\theta}
            =mr^{2}\ddot{\theta}
        \end{equation*}
        This is particle going around in a circle.
        Then we have from the equation of motion that:
        \begin{equation*}
            \frac{d}{dt}\big(
                \frac{\partial\mathcal{L}}{\partial\dot{\theta}}
            \big)
            =mr^{2}\ddot{\theta}
            \Rightarrow
            \frac{\partial\mathcal{L}}{\partial\dot{\theta}}
            =mr^{2}\dot{\theta}+c
            \Rightarrow
            \mathcal{L}
            =\frac{1}{2}mr^2\dot{\theta}+g(\theta)+c\dot{\theta}
        \end{equation*}
        Remember from calculus that since we are taking partial
        derivatives, when we perform integration we get a function
        of integration, and not a just constant of integration.
        This function of integration is the $g(\theta)$
        we have in the equation above. Let's further add the
        requirement that $\ddot{\theta}=0$. That is,
        there is no angular acceleration.
        This represent a particle going around in a circle at
        constant angular velocity. With this information,
        we can determine $g(\theta)$. We have that
        \begin{equation*}
            \frac{\partial\mathcal{L}}{\partial\theta}
            =0
            \Rightarrow
            g'(\theta)=0
        \end{equation*}
        which means $g(\theta)$ is a constant. The Lagrangian of
        this problem is
        $\mathcal{L}%
         =\frac{1}{2}mr^{2}\dot{\theta}^{2}%
         +c_{1}\dot{\theta}+c_{2}$.
        Now what of the generalized momentum? We have that:
        \begin{equation*}
            \frac{d}{dt}\big(
                \frac{\partial\mathcal{L}}{\partial\dot{\theta}}
             \big)
            =\frac{\partial\mathcal{L}}{\partial\theta}=0
            \Rightarrow
            \frac{\partial\mathcal{L}}{\partial\dot{\theta}}
            =constant
        \end{equation*}
        But our generalized momentum is simply the
        angular momentum. That is, we have shown
        that angular momentum is conserved.