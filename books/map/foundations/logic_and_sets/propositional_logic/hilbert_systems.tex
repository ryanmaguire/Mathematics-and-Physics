\section{Hilbert Systems}
    We are now ready to state Hilbert's axioms and start proving some basic
    theorems. The axioms are strange at first glance and are not very intuitive.
    Fortunately, if we look at truth tables, we'll see why we're accepting them.
    We begin with Hilbert's second axiom. As mentioned, the first axiom is
    redundent.
    \begin{faxiom}{Hilbert's Second Axiom}{Hilberts_Second_Axiom}
        If $P$ and $Q$ are \glspl{predicate} on a \gls{set} $A$, then:%
        \index{Hilbert's Second Axiom}\index{Axiom!Hilbert's Second}
        \begin{equation*}
            P\Rightarrow(Q\Rightarrow{P})
        \end{equation*}
    \end{faxiom}
    Let's try to explain away this axiom. If $Q\Rightarrow{P}$ is only false
    when $Q$ is true and $P$ is false
    (See Tab.~\ref{tab:Alternate_Truth_Table_Implication}). But if $P$ is
    true, then this scenario never occurs and hence $Q\Rightarrow{P}$ is true.
    That is, if $P$, then $Q\Rightarrow{P}$. This is what
    Ax.~\ref{ax:Hilberts_Second_Axiom} says. Now since we are accepting this as
    an axiom, the truth table better be a column of all ones. This is shown in
    Tab.~\ref{tab:Truth_Table_Hilbert_Second_Axiom}.
    \begin{table}[H]
        \centering
        \captionsetup{type=table}
        \begin{tabular}{c|c|c|c}
            $P$&$Q$&$Q\Rightarrow{P}$&$P\Rightarrow(Q\Rightarrow{P})$\\
            \hline
            0&0&1&1\\
            0&1&0&1\\
            1&0&1&1\\
            1&1&1&1
        \end{tabular}
        \caption{Truth Table for Hilbert's Second Axiom}
        \label{tab:Truth_Table_Hilbert_Second_Axiom}
    \end{table}
    The column for $P\Rightarrow(Q\rightarrow{P})$ is always one, meaning this
    statement is always true, justifying our accepting it as an axiom. Note the
    parentheses. These are not redundent. Let's look at the truth table for
    $(P\Rightarrow{Q})\Rightarrow{P}$
    (Tab.~\ref{tab:Truth_Table_P_implies_Q_implies_P}).
    \begin{table}
        \centering
        \captionsetup{type=table}
        \begin{tabular}{c|c|c|c}
            $P$&$Q$&$P\Rightarrow{Q}$&$(P\Rightarrow{Q})\Rightarrow{P}$\\
            \hline
            0&0&1&0\\
            0&1&1&1\\
            1&0&0&1\\
            1&1&1&1
        \end{tabular}
        \caption{Truth Table for $(P\Rightarrow{Q})\Rightarrow{P}$}
        \label{tab:Truth_Table_P_implies_Q_implies_P}
    \end{table}
    By comparing the last column in
    Tab.~\ref{tab:Truth_Table_P_implies_Q_implies_P} with the last one in
    Tab.~\ref{tab:Truth_Table_Hilbert_Second_Axiom} we see that
    $(P\Rightarrow{Q})\Rightarrow{P}$ and $P\Rightarrow(Q\Rightarrow{P})$ are
    different statements. Indeed, $P\Rightarrow(Q\Rightarrow{P})$ is always
    true, but $(P\Rightarrow{Q})\Rightarrow{P})$ is not. This shows us that
    implication is not an \textit{associative}\index{Associative Operation}
    operation. Contrast this with the familiar rules of arithmetic where
    $a+(b+c)=(a+b)+c$ is always true. Because of this we may simply write
    $a+b+c$ without ambiguity. We are not afforded such a luxury here and
    because of this we must keep track of parentheses. To make this easier we
    often nest parentheses by size. For example:
    \begin{equation}
        \Bigg(
            \bigg(
                \Big(
                    \big(
                        (\cdots)\cdots
                    \big)\cdots
                \Big)\cdots
            \bigg)\cdots
        \Bigg)
    \end{equation}
    We can also see why $(P\Rightarrow{Q})\Rightarrow{P}$ is not true on an
    intuitive level. Choose a statement $P\Rightarrow{Q}$ that is always true by
    choosing $P$ to always be false. That is, a \textit{vacuous} truth. For
    example, \textit{if 1+1=fish, then London is the capital of England}.
    Moving on, we now present Hilbert's third axiom, which was an axiom included
    Jan \L{u}kasiewicz's\index{\L{u}kasiewicz, Jan} (1878-1956 C.E.) system. It
    allows us to deal with three propositions simultaneously.
    \begin{faxiom}{Hilbert's Third Axiom}{Hilberts_Third_Axiom}
        \label{thm:Lukasiewicz_Second_Axiom}%
        If $P$, $Q$, and $R$ are predicates on a set $A$, then:%
        \index{Hilbert's Third Axiom}\index{Axiom!Hilbert's Third}
        \begin{equation*}
            (P\Rightarrow(Q\Rightarrow{R}))\Rightarrow
            ((P\Rightarrow{Q})\Rightarrow(P\Rightarrow{R}))
        \end{equation*}
    \end{faxiom}
    Let's justify this via truth table:
    \begin{table}[H]
        \centering
        \captionsetup{type=table}
        \begin{tabular}{c|c|c|c|c|c|c}
            $P$&$Q$&$R$&$P\Rightarrow{Q}$&$P\Rightarrow{R}$&$Q\Rightarrow{R}$
                &$P\Rightarrow(Q\Rightarrow{R})$\\
            \hline
            0&0&0&1&1&1&1\\
            0&0&1&1&1&1&1\\
            0&1&0&1&1&0&1\\
            0&1&1&1&1&1&1\\
            1&0&0&0&0&1&1\\
            1&0&1&0&1&1&1\\
            1&1&0&1&0&0&0\\
            1&1&1&1&1&1&1
        \end{tabular}
        \caption{Truth Table for Hilbert's Third Axiom (a)}
        \label{tab:Truth_Table_for_Hilberts_Third_a}
    \end{table}
    And, since we're running out of space, we continue with a new table on the
    next page. By examining Tab.~\ref{tab:Truth_Table_for_Hilberts_Third_b} we
    see that, as hoped, the last column is all ones meaning this is a valid
    axiom we can accept without proof.
    \begin{table}[H]
        \centering
        \captionsetup{type=table}
        \begin{tabular}{c|c}
            $(P\Rightarrow{Q})\Rightarrow(P\Rightarrow{R})$&
            $(P\Rightarrow(Q\Rightarrow{R}))\Rightarrow%
             ((P\Rightarrow{Q})\Rightarrow(P\Rightarrow{R}))$\\
            \hline
            1&1\\
            1&1\\
            1&1\\
            1&1\\
            1&1\\
            1&1\\
            0&1\\
            1&1
        \end{tabular}
        \caption{Truth Table for Hilbert's Third Axiom (b)}
        \label{tab:Truth_Table_for_Hilberts_Third_b}
    \end{table}
    The fourth and final axiom is related to the contrapositive of
    $P\Rightarrow{Q}$.
    \begin{faxiom}{Hilbert's Fourth Axiom}{Hilberts_Fourth_Axiom}
        If $P$ and $Q$ are \glspl{predicate} on a \gls{set} $A$, then:%
        \index{Hilbert's Fourth Axiom}\index{Axiom!Hilbert's Fourth}
        \begin{equation*}
            (\neg{Q}\Rightarrow\neg{P})\Rightarrow(P\Rightarrow{Q})
        \end{equation*}
    \end{faxiom}
    Again, since we are accepting this as an axiom, it's truth table better have
    a column of ones. Let's look
    (Tab.~\ref{tab:Truth_Table_Hiberts_Third_Axiom}).
    \begin{table}[H]
        \centering
        \captionsetup{type=table}
        \begin{tabular}{c|c|c|c|c|c|c}
            $P$&$Q$&$\neg{P}$&$\neg{Q}$&$P\Rightarrow{Q}$&
                $\neg{Q}\Rightarrow\neg{P}$&
                $(\neg{Q}\rightarrow\neg{P})\Rightarrow(P\Rightarrow{Q})$\\
            \hline
            0&0&1&1&1&1&1\\
            0&1&1&0&1&1&1\\
            1&0&0&1&0&0&1\\
            1&1&0&0&1&1&1
        \end{tabular}
        \caption{Truth Table for Hilbert's Third Axiom}
        \label{tab:Truth_Table_Hiberts_Third_Axiom}
    \end{table}
    And all is well with the world. Now one may wonder why we're looking at such
    basic axioms, and why are they named after David Hilbert (1862-1943 C.E.),
    a mathematician associated with analysis and geometry. Well around the
    time Hilbert proposed these, the \textit{Grundlagenkrise der Mathematik}%
    \footnote{English: Foundational crisis in mathematics}
    was occuring. Russel's paradox had really stumped many school's of
    mathematics, and while Zermelo and Fraenkel were putting together their
    set theory, Hilbert proposed a research program, now known as the
    \textit{Hilbert Program}\index{Hilbert Program}, to perform all known
    mathematics under a single \textit{finite} set of axioms and primitive
    symbols. He desired a proof that any mathematical statement could be proven
    from these axioms (\textit{completeness}) and that the system itself was
    consistent. Unfortunately, G\"{o}del's theorems (published in 1931 C.E.)
    showed both of these to be impossible. Nevertheless, Hilbert's axioms are
    fairly basic, can prove quite a lot, and when combined with \gls{ZFC} and
    predicate logic, can prove \textit{most} of mathematics.%
    \footnote{%
        Category theory is probably the only branch missing because it requires
        proper classes, which we'll discuss later.
    }
    \par\hfill\par
    We are now ready for our first theorem! As mentioned, Hilbert's first
    axiom is redundant. We should show this, deriving from Hilbert's other
    axioms, together with modus ponens. The statement is obvious, far more so
    than the three axioms. It says if $P$ is true, then $P$ is true. That is,
    $P\Rightarrow{P}$. Trivial as far as the English language is concerned, and
    indeed the truth table is a column of ones. Nevertheless, we prove it.
    \begin{ftheorem}{Hilbert's First Axiom}{Hilberts_First_Axiom}
        If $P$ is a \gls{predicate} on a \gls{set} $A$, then:
        \begin{equation*}
            P\Rightarrow{P}
        \end{equation*}
    \end{ftheorem}
    \begin{bproof}
        Let $Q$ be the predicate $P\Rightarrow{P}$, defined on $A$. Then by
        Hilbert's second axiom (Ax.~\ref{ax:Hilberts_Second_Axiom}), we have:
        \begin{equation}
            P\Rightarrow(Q\Rightarrow{P})
        \end{equation}
        But by Hilbert's third axiom (Ax.~\ref{ax:Hilberts_Third_Axiom}) we
        obtain:
        \begin{equation}
            \big(
                P\Rightarrow(Q\Rightarrow{P})
            \big)\Rightarrow
            \big(
                (P\Rightarrow{Q})\Rightarrow(P\Rightarrow{P})
            \big)
        \end{equation}
        But as claimed, $P\Rightarrow(Q\Rightarrow{P})$ and hence by
        modus ponens (Ax.~\ref{ax:Modus Ponens}), since
        $(P\Rightarrow(Q\Rightarrow{P}))\Rightarrow%
         ((P\Rightarrow{Q})\Rightarrow(P\Rightarrow{P}))$
        is true, and since $P\Rightarrow(Q\Rightarrow{P})$ is true, we have that
        $(P\Rightarrow{Q})\Rightarrow(P\Rightarrow{P})$ is true. But $Q$ is the
        statement $P\Rightarrow{P}$, and hence we have
        $P\Rightarrow(P\Rightarrow{P})$. But by Hilbert's second axiom, this is
        true. But if $P\Rightarrow(P\Rightarrow{P})$ is true and
        $(P\Rightarrow{Q})\Rightarrow(P\Rightarrow{P})$ is true, then
        $P\Rightarrow{P}$ by modus ponens. Hence, $P\Rightarrow{P}$.
    \end{bproof}
    The next obvious theorem to present should be the theorem of
    \textit{hypothetical syllogism}\index{Theorem!Hypothetical Syllogism}, which
    states that if $P\Rightarrow{Q}$ and $Q\Rightarrow{R}$, then
    $P\Rightarrow{R}$. In other words, implication is a \textit{transitive}
    notion. We do not yet have a mathematical meaning for \textit{and} so we'll
    have to wait. We can, however, present two alternative forms of hypothetical
    syllogism that are defineable in pure propositional logic.
    \begin{theorem}
        \label{thm:hypothetical_syllogism_1}%
        If $P$, $Q$, and $R$ are predicates on a set $A$, then:
        \begin{equation*}
            (Q\Rightarrow{R})\Rightarrow
            \big(
                (P\Rightarrow{Q})\Rightarrow(P\Rightarrow{R})
            \big)
        \end{equation*}
    \end{theorem}
    \begin{proof}
        By Hilbert's third axiom (Ax.~\ref{ax:Hilberts_Third_Axiom}), we
        have:
        \begin{equation}
            \label{eqn:instance_of_Hilbert_1_hypothetical_syllogism_1}%
            (P\Rightarrow(Q\Rightarrow{R}))\Rightarrow
            ((P\Rightarrow{Q})\Rightarrow(P\Rightarrow{R}))
        \end{equation}
        By Hilbert's second axiom (Ax.~\ref{ax:Hilberts_Second_Axiom}) we have:
        \begin{equation}
            \label{eqn:Instance_of_Hilbert_2_for_hypothetical_syllogism_1}%
            \begin{split}
                &\Big(
                    \big(
                        P\Rightarrow(Q\Rightarrow{R})
                    \big)
                    \Rightarrow
                    \big(
                        (P\Rightarrow{Q})\Rightarrow(P\Rightarrow{R})
                    \big)
                \Big)\\
                \Rightarrow&
                \bigg(
                    (Q\Rightarrow{R})\Rightarrow
                    \Big(
                        \big(
                            P\Rightarrow(Q\Rightarrow{R})
                        \big)
                        \Rightarrow
                        \big(
                            (P\Rightarrow{Q})\Rightarrow(P\Rightarrow{R})
                        \big)
                    \Big)
                \bigg)
            \end{split}
        \end{equation}
        That is, let $\alpha$ be the predicate
        $(P\Rightarrow(Q\Rightarrow{R}))\Rightarrow%
         ((P\Rightarrow{Q})\Rightarrow(P\Rightarrow{R}))$,
        and define $\beta$ by $Q\Rightarrow{R}$. Then
        Eqn.~\ref{eqn:Instance_of_Hilbert_2_for_hypothetical_syllogism_1} simply
        reads $\alpha\Rightarrow(\beta\Rightarrow\alpha)$, which is an instance
        of Hilbert's second axiom. But the left hand side of
        Eqn.~\ref{eqn:Instance_of_Hilbert_2_for_hypothetical_syllogism_1} is
        just Eqn.~\ref{eqn:instance_of_Hilbert_1_hypothetical_syllogism_1},
        which I claimed was true via Hilbert's third axiom. Hence by modus
        ponens (Ax.~\ref{ax:Modus Ponens}), since the left hand side of the
        implication
        Eqn.~\ref{eqn:Instance_of_Hilbert_2_for_hypothetical_syllogism_1} is
        true, the right hand side of
        Eqn.~\ref{eqn:Instance_of_Hilbert_2_for_hypothetical_syllogism_1} must
        be true. hence:
        \begin{equation}
            (Q\Rightarrow{R})\Rightarrow
            \Big(
                \big(
                    P\Rightarrow(Q\Rightarrow{R})
                \big)
                \Rightarrow
                \big(
                    (P\Rightarrow{Q})\Rightarrow(P\Rightarrow{R})
                \big)
            \Big)
        \end{equation}
        But also by Hilbert's third axiom, we have:
        \begin{equation}
            \label{eqn:hypothetical_syllogism_long_eqn}%
            \begin{split}
                &\bigg(
                    (Q\Rightarrow{R})\Rightarrow
                    \Big(
                        \big(
                            P\Rightarrow(Q\Rightarrow{R})
                        \big)
                        \Rightarrow
                        \big(
                            (P\Rightarrow{Q})\Rightarrow(P\Rightarrow{R})
                        \big)
                    \Big)
                \bigg)
                \Rightarrow\\
                &\bigg(
                    \Big(
                        (Q\Rightarrow{R})\Rightarrow
                        \big(
                            P\Rightarrow(Q\Rightarrow{R})
                        \big)
                    \Big)
                    \Rightarrow
                    \Big(
                        (Q\Rightarrow{R})
                        \Rightarrow
                        \big(
                            (P\Rightarrow{Q})\Rightarrow(P\Rightarrow{R})
                        \big)
                    \Big)
                \bigg)
            \end{split}
        \end{equation}
        But then again by modus ponens, since the left side of this implication
        is true, we have:
        \begin{equation}
            \label{eqn:hypothetical_syllogism_not_long_eqn}%
            \Big(
                (Q\Rightarrow{R})\Rightarrow
                \big(
                    P\Rightarrow(Q\Rightarrow{R})
                \big)
            \Big)
            \Rightarrow
            \Big(
                (Q\Rightarrow{R})
                \Rightarrow
                \big(
                    (P\Rightarrow{Q})\Rightarrow(P\Rightarrow{R})
                \big)
            \Big)
        \end{equation}
        But by Hilbert's second axiom:
        \begin{equation}
            (Q\Rightarrow{R})\Rightarrow
            \big(
                P\Rightarrow(Q\Rightarrow{R})
            \big)
        \end{equation}
        Combining this with Eqn.~\ref{eqn:hypothetical_syllogism_not_long_eqn}
        and invoking modus ponens, we have:
        \begin{equation}
            (Q\Rightarrow{R})
            \Rightarrow
            \big(
                (P\Rightarrow{Q})\Rightarrow(P\Rightarrow{R})
            \big)
        \end{equation}
        which is what we wanted to prove.
    \end{proof}
    Now let's look at what this is saying. I have some statement $Q$ and I know
    that whenever $Q$ is true, some other statement $R$ is true. That is,
    $Q\Rightarrow{R}$. Now suppose I have a third statement $P$ and I know that
    $P$ implies $Q$, $P\Rightarrow{Q}$. Well if $P$ implies $Q$ it seems only
    natural that $P$ should imply $R$, since $Q$ implies $R$. That is precisely
    what Thm.~\ref{thm:hypothetical_syllogism_1} is spelling out. If $Q$ implies
    $R$, then if $P$ implies $Q$, then $P$ implies $R$. It is an implicational
    version of the hypothetical syllogism that does not need the word
    \textit{and}. It seems we should be able to flip the order of the $P$ and
    $Q$ a bit and still obtain a valid statement. Let's work towards making this
    a theorem.
    \begin{theorem}
        \label{thm:precursor_to_hypothetical_syllogism_2}%
        If $P$ and $Q$ are predicates on a set $A$, then:
        \begin{equation*}
            P\Rightarrow
            \big(
                (P\Rightarrow{Q})\Rightarrow{Q}
            \big)
        \end{equation*}
    \end{theorem}
    \begin{proof}
        By Hilbert's third axiom, we have:
        \begin{equation}
            \label{eqn:hilberts_third_in_precursor_to_hypothetical_syllogism_2}%
            \big(
                (P\Rightarrow{Q})\Rightarrow(P\Rightarrow{Q})
            \big)
            \Rightarrow
            \Big(
                \big(
                    (P\Rightarrow{Q})\Rightarrow{P}
                \big)
                \Rightarrow
                \big(
                    (P\Rightarrow{Q})\Rightarrow{Q}
                \big)
            \Big)
        \end{equation}
        To see this note that if we define $\alpha$ by $P\Rightarrow{Q}$ then
        Eqn.~\ref{eqn:hilberts_third_in_precursor_to_hypothetical_syllogism_2}
        reads as
        $(\alpha\Rightarrow(P\Rightarrow{Q}))\Rightarrow%
         ((\alpha\Rightarrow{P})\Rightarrow(\alpha\Rightarrow{Q}))$,
        which is Hilbert's third. But by Hilbert's first axiom
        (Thm.~\ref{thm:Hilberts_First_Axiom}) we have:
        \begin{equation}
            (P\Rightarrow{Q})\Rightarrow(P\Rightarrow{Q})
        \end{equation}
        But then by modus ponens (Ax.~\ref{ax:Modus Ponens}), the previous two
        implications yield:
        \begin{equation}
            \label{eqn:eqn2_precursor_to_hypothetical_syllogism_2}%
            \big(
                (P\Rightarrow{Q})\Rightarrow{P}
            \big)
            \Rightarrow
            \big(
                (P\Rightarrow{Q})\Rightarrow{Q}
            \big)
        \end{equation}
        By Thm.~\ref{thm:hypothetical_syllogism_1} we have
        $(\beta\Rightarrow\gamma)\Rightarrow%
         ((\alpha\Rightarrow\beta)\Rightarrow(\alpha\Rightarrow\gamma))$ for any
        predicates $\alpha,\beta,\gamma$ on $A$. Letting $\beta$ denote
        $((P\Rightarrow{Q})\Rightarrow{P})$, $\gamma$ denote
        $((P\Rightarrow{Q})\Rightarrow{Q})$, and letting $\alpha$ simply be $P$,
        this becomes:
        \begin{equation}
            \begin{split}
                &
                \Big(
                    \big(
                        (P\Rightarrow{Q})\Rightarrow{P}
                    \big)\Rightarrow
                    \big(
                        (P\Rightarrow{Q})\Rightarrow{Q}
                    \big)
                \Big)
                \Rightarrow\\
                &\bigg(
                    \Big(
                        P\Rightarrow
                        \big(
                            (P\Rightarrow{Q})\Rightarrow{P}
                        \big)
                    \Big)
                    \Rightarrow
                    \Big(
                        P\Rightarrow
                        \big(
                            (P\Rightarrow{Q})\Rightarrow{Q}
                        \big)
                    \Big)
                \bigg)
            \end{split}
        \end{equation}
        But Eqn.~\ref{eqn:eqn2_precursor_to_hypothetical_syllogism_2} says the
        left hand side of this implication is true, hence by modus ponens so is
        the right side. That is:
        \begin{equation}
            \Big(
                P\Rightarrow
                \big(
                    (P\Rightarrow{Q})\Rightarrow{P}
                \big)
            \Big)
            \Rightarrow
            \Big(
                P\Rightarrow
                \big(
                    (P\Rightarrow{Q})\Rightarrow{Q}
                \big)
            \Big)
        \end{equation}
        But by Hilbert's second axiom (Ax.~\ref{ax:Hilberts_Second_Axiom}) we
        have:
        \begin{equation}
            P\Rightarrow\big((P\Rightarrow{Q})\Rightarrow{P}\big)
        \end{equation}
        Combining these previous two implications with modus ponens give us:
        \begin{equation}
            P\Rightarrow
            \big(
                (P\Rightarrow{Q})\Rightarrow{Q}
            \big)
        \end{equation}
        Which is what we wanted to prove.
    \end{proof}
    \begin{theorem}
        \label{thm:precursor_to_hypothetical_syllogism_2}%
        If $P$, $Q$, and $R$ are predicates on a set $A$, then:
        \begin{equation*}
            \big(P\Rightarrow(Q\Rightarrow{R})\big)
            \Rightarrow\big(Q\Rightarrow(P\Rightarrow{R})\big)
        \end{equation*}
    \end{theorem}
    \begin{proof}
        By Hilbert's third axiom (Ax.~\ref{ax:Hilberts_Third_Axiom}), we have:
        \begin{equation}
            \label{eqn:precursor_to_hypothetical_syllogism_2_eqn1}%
            \big(
                P\Rightarrow(Q\Rightarrow{R})
            \big)
            \Rightarrow
            \big(
                (P\Rightarrow{Q})\Rightarrow(P\Rightarrow{R})
            \big)
        \end{equation}
        But by Thm.~\ref{thm:hypothetical_syllogism_1}, we have:
        \begin{equation}
            \label{eqn:precursor_to_hypothetical_syllogism_2_eqn2}%
            \big(
                (P\Rightarrow{Q})\Rightarrow(P\Rightarrow{R})
            \big)
            \Rightarrow
            \Big(
                \big(
                    Q\Rightarrow(P\Rightarrow{Q})
                \big)
                \Rightarrow
                \big(
                    Q\Rightarrow(P\Rightarrow{R})
                \big)
            \Big)
        \end{equation}
        Let $\alpha$, $\beta$, and $\gamma$ denote the predicates
        $P\Rightarrow(Q\Rightarrow{R})$,
        $(P\Rightarrow{Q})\Rightarrow(P\Rightarrow{R})$, and
        $(Q\Rightarrow(P\Rightarrow{Q}))\Rightarrow%
         (Q\Rightarrow(P\Rightarrow{R}))$, respectively. Then again by
        Thm.~\ref{thm:hypothetical_syllogism_1}, we have:
        \begin{equation}
            (\beta\Rightarrow\gamma)\Rightarrow
            \big(
                (\alpha\Rightarrow\beta)\Rightarrow(\alpha\Rightarrow\gamma)
            \big)
        \end{equation}
        But Eqn.~\ref{eqn:precursor_to_hypothetical_syllogism_2_eqn1} claims
        $\beta\Rightarrow\gamma$ is true. Hence by modus ponens
        (Ax.~\ref{ax:Modus Ponens}) we have:
        \begin{equation}
            (\alpha\Rightarrow\beta)\Rightarrow(\alpha\Rightarrow\gamma)
        \end{equation}
        But Eqn.~\ref{eqn:precursor_to_hypothetical_syllogism_2_eqn2} claims
        $\alpha\Rightarrow\beta$ is true, and hence by modus ponens we have
        $\alpha\Rightarrow\gamma$ is true. But by the definition of $\alpha$ and
        $\gamma$ this says:
        \begin{equation}
            \big(
                P\Rightarrow(Q\Rightarrow{R})
            \big)
            \Rightarrow
            \Big(
                \big(
                    Q\Rightarrow(P\Rightarrow{Q})
                \big)
                \Rightarrow
                \big(
                    Q\Rightarrow(P\Rightarrow{R})
                \big)
            \Big)
        \end{equation}
        But by Hilbert's third axiom (Ax.~\ref{ax:Hilberts_Third_Axiom}) we
        distribute the implication symbol to obtain:
        \begin{equation}
            \begin{split}
                &
                \bigg(
                    \big(
                        P\Rightarrow(Q\Rightarrow{R})
                    \big)
                    \Rightarrow
                    \Big(
                        \big(
                            Q\Rightarrow(P\Rightarrow{Q})
                        \big)
                        \Rightarrow
                        \big(
                            Q\Rightarrow(P\Rightarrow{R})
                        \big)
                    \Big)
                \bigg)\Rightarrow\\
                &
                \bigg(
                    \Big(
                        \big(
                            P\Rightarrow(Q\Rightarrow{R})
                        \big)
                        \Rightarrow
                        \big(
                            Q\Rightarrow(P\Rightarrow{Q})
                        \big)
                    \Big)
                    \Rightarrow\\
                &\hspace{2ex}
                    \Big(
                        \big(
                            P\Rightarrow(Q\Rightarrow{R})
                        \big)
                        \Rightarrow
                        \big(
                            Q\Rightarrow(P\Rightarrow{R})
                        \big)
                    \Big)
                \bigg)
            \end{split}
        \end{equation}
        Since the left hand side is true, by modus ponens the right hand side
        must be. Hence:
        \begin{equation}
            \label{eqn:precursor_to_hypothetical_syllogism_2_almost_done}%
            \begin{split}
                &\Big(
                    \big(
                        P\Rightarrow(Q\Rightarrow{R})
                    \big)
                    \Rightarrow
                    \big(
                        Q\Rightarrow(P\Rightarrow{Q})
                    \big)
                \Big)
                \Rightarrow\\
                &\Big(
                    \big(
                        P\Rightarrow(Q\Rightarrow{R})
                    \big)
                    \Rightarrow
                    \big(
                        Q\Rightarrow(P\Rightarrow{R})
                    \big)
                \Big)
            \end{split}
        \end{equation}
        But by Hilbert's second axiom, $Q\Rightarrow(P\Rightarrow{Q})$ and
        \begin{equation}
            \big(Q\Rightarrow(P\Rightarrow{Q})\big)
            \Rightarrow
            \Big(
                \big(
                    P\Rightarrow(Q\Rightarrow{R})
                \big)\Rightarrow
                \big(
                    Q\Rightarrow(P\Rightarrow{Q})
                \big)
            \Big)
        \end{equation}
        Hence, by modus ponens:
        \begin{equation}
            \big(
                P\Rightarrow(Q\Rightarrow{R})
            \big)\Rightarrow
            \big(
                Q\Rightarrow(P\Rightarrow{Q})
            \big)
        \end{equation}
        But then by
        Eqn.~\ref{eqn:precursor_to_hypothetical_syllogism_2_almost_done} and
        modus ponens we get:
        \begin{equation}
            \big(
                P\Rightarrow(Q\Rightarrow{R})
            \big)
            \Rightarrow
            \big(
                Q\Rightarrow(P\Rightarrow{R})
            \big)
        \end{equation}
        Which is what we wanted to prove.
    \end{proof}
    And finally, the \textit{reordering} of the hypothetical syllogism that was
    alluded to. If $P\Rightarrow{Q}$, it seems likely that
    $Q\Rightarrow{R}$ implies $P\Rightarrow{R}$. This is another way of phrasing
    that if $P\Rightarrow{Q}$ and $Q\Rightarrow{R}$, then $P\Rightarrow{R}$
    without the need of \textit{and}. Now we might note that the previous proofs
    have been horrendously long and yet only proved very simple claims. This is
    the first instance of benefitting from the fruits of our labors. Most of the
    proof of the following theorem is performed by invoking the results of
    previous theorems, rather than providing any new actual work.
    \begin{theorem}
        \label{thm:hypothetical_syllogism_2}%
        If $P$, $Q$ and $R$ are predicates on a set $A$, then:
        \begin{equation*}
            (P\Rightarrow{Q})\Rightarrow
            \big(
                (Q\Rightarrow{R})\Rightarrow(P\Rightarrow{R})
            \big)
        \end{equation*}
    \end{theorem}
    \begin{proof}
        For by Thm.~\ref{thm:hypothetical_syllogism_1}:
        \begin{equation}
            (Q\Rightarrow{R})\Rightarrow
            \big(
                (P\Rightarrow{Q})\Rightarrow(P\Rightarrow{R})
            \big)
        \end{equation}
        But by Thm.~\ref{thm:precursor_to_hypothetical_syllogism_2}:
        \begin{equation}
            \begin{split}
                &
                \Big(
                    (Q\Rightarrow{R})\Rightarrow
                    \big(
                        (P\Rightarrow{Q})\Rightarrow(P\Rightarrow{R})
                    \big)
                \Big)
                \Rightarrow\\
                &
                \Big(
                    (P\Rightarrow{Q})\Rightarrow
                    \big(
                        (Q\Rightarrow{R})\Rightarrow(P\Rightarrow{R})
                    \big)
                \Big)
            \end{split}
        \end{equation}
        Hence by modus ponens (Ax.~\ref{ax:Modus Ponens}):
        \begin{equation}
            (P\Rightarrow{Q})\Rightarrow
            \big(
                (Q\Rightarrow{R})\Rightarrow(P\Rightarrow{R})
            \big)
        \end{equation}
        which is what we wanted to prove.
    \end{proof}
    \subsection{Connectives}
        As stated, we attempt to collect the smallest number of
        \textit{primitive} symbols possible when creating our mathematical
        language. The main undefined symbol in set theory is that of
        \textit{containment} (\gls{containmentsymb}) (see
        Not.~\ref{not:Element_Notation}), which is a type of \gls{predicate} of
        the form \textit{x is in A}. Other common symbols such as subset
        (\gls{subseteq}) and equality (\gls{equalsymb}) are then defined in
        terms of this. In a similar manner there are other commonly used symbols
        in mathematical logic such as \textit{disjunction}
        (\gls{disjunctionsymb}), \textit{conjunction} (\gls{conjunctionsymb}),
        and \textit{equivalence} (\gls{equivalencesymb}) that we need not accept
        as primitives, but rather can define in terms of implication
        (\gls{implicationsymb}) and negation (\gls{negationsymb}). We start with
        conjunction, which gives meaning the logical term \textit{and}.
        The symbol $\land$ is used to represent the word \textit{and} in a
        mathematical way. We want it to represent the following truth table:
        \begin{table}[H]
            \centering
            \captionsetup{type=table}
            \begin{tabular}{ccc}
                $P$&$Q$&$P\land{Q}$\\
                \hline
                0&0&0\\
                0&1&0\\
                1&0&0\\
                1&1&1
            \end{tabular}
            \caption{Truth Table for Conjunction}
            \label{tab:Truth_Table_for_Conjunction}
        \end{table}
        We could simply accept conjunction as another primitive, but it is
        possible to avoid this by defining it in terms of negation and
        implication. To define new symbols in terms of old we need some
        notation. There are several standards for the \textit{definition} symbol
        such as $:=$, $=:$, and $\doteq$. We will adopt the following, which
        seems to have widespread use in various areas of mathematics.
        \begin{fnotation}{Definition Symbol}{Definition_Symbol}
            We denote that a new symbol $x$ is defined by a combination of
            previously defined symbols $a_{0},\dots,a_{n}$ via some formula $F$
            by writing:
            \begin{equation*}
                x\equiv{F}(a_{1},\dots,a_{n})
            \end{equation*}
        \end{fnotation}
        We adopt this symbol (\gls{equiv}) as a new primitive, and hence we have
        four: Three logical ones, and one set theoretic
        (implication \gls{implicationsymb}, negation \gls{negationsymb},
        definition \gls{equiv}, and set containment \gls{containmentsymb}). We
        now wish to define conjunction in terms of negation and implication. We
        do this first by looking at the following truth table:
        \begin{table}[H]
            \centering
            \captionsetup{type=table}
            \begin{tabular}{ccccccc}
                $P$&$Q$&$\neg{P}$&$\neg{Q}$&$P\Rightarrow\neg{Q}$
                    &$\neg(P\Rightarrow\neg{Q})$&$P\land{Q}$\\
                \hline
                0&0&1&1&1&0&0\\
                0&1&1&0&1&0&0\\
                1&0&0&1&1&0&0\\
                1&1&0&0&0&1&1
            \end{tabular}
            \caption{Defining Conjunction by Implication and Negation}
            \label{tab:Def_Conj_by_Imp_and_Neg}
        \end{table}
        Here we see that $\neg(P\Rightarrow\neg{Q})$ and $P\land{Q}$ have the
        exact same truth table. Using this we need not create a new primitive
        but can simply adopt the following definition of conjunction.
        \begin{fdefinition}{Conjunction}{Conjunction}
            The conjunction of propositions $P$ and $Q$ is the statement $P$ and
            $Q$ defined by the formula:\index{Conjunction}
            \begin{equation*}
                P\land{Q}\equiv\neg\big(P\Rightarrow\neg{Q}\big)
            \end{equation*}
        \end{fdefinition}
        Before justifying this definition, we wish to get an idea as to what
        \textit{and} should mean. Given two propositions $P$ and $Q$, $P$ and
        $Q$ should be considered if and only if both $P$ is true and $Q$ is
        true. That is, both are true simultaneously.
        \par\hfill\par
        \begin{fdefinition}{Disjunction}{Disjunction}
            The disjunction of propositions $P$ and $Q$ is the statement $P$ or
            $Q$ defined by the formula:
            \begin{equation*}
                P\lor{Q}\equiv(P\Rightarrow{Q})\Rightarrow{P}
            \end{equation*}
        \end{fdefinition}
        We have relied on the word \textit{statement} being already defined, and
        similarly for the words \textit{parameter} or \textit{variable}. For most
        this is not an issue, but it may irk others. From our undefined symbol $\in$
        we build new symbols by expressing them in terms of a
        \textit{formula}\index{Formula}, which is simply a finite
        sequence of symbols. Here the word \textit{sequence} is meant to imply that
        the \textit{order} in which we combine these symbols is important and that
        rearranging said order may create a different inequivalent formula. We build
        formulas by defining a few symbols that stand as placeholders for standard
        words in English. There are four symbols, called
        \textit{\glspl{connective}}\index{Connective (Logic)}, that we use.
        From this we see that we have introduced 6 new words that are undefined but
        require comment. The words are \textit{and, or, if, then, true}, and
        \textit{false}. There are other symbols we could adopt, such as
        \textit{equivalence}:
        \begin{equation*}
            a\Leftrightarrow{b}
        \end{equation*}
        But from how we shall define these notions, this new symbol is equivalent to
        a combination of the previous ones:
        \begin{equation*}
            \Big(\big(a\Leftrightarrow{b}\big)\Rightarrow
                \big((a\Rightarrow{b})\land(b\Rightarrow{a})\big)\Big)
            \land\Big(\big((a\Rightarrow{b})\land(b\Rightarrow{A})\big)
                \Rightarrow\big(a\Leftrightarrow{b}\big)\Big)
        \end{equation*}
        That is, $a\Leftrightarrow{b}$ if and only if $a$ is true if and only if $b$
        is true. Similarly, we could define \textit{does not imply}:
        \begin{equation*}
            a\not\Rightarrow{b}
        \end{equation*}
        But this is the same as:
        \begin{equation*}
            a\not\Rightarrow{b}\Longleftrightarrow
            \neg(a\Rightarrow{b})
            \Longleftrightarrow
            a\land\neg{b}
        \end{equation*}
        The words true and false are assumed to be well defined. They are also
        assumed to be opposites of each other (which we will define in terms of
        negation). We will use truth tables\index{Truth Table} to define what
        various connectives mean when it is known that certain propositions are true
        or false. In such tables the symbol 0 represents that a proposition is false
        and the symbol 1 represents truth.
        \par\hfill\par
        The other four words can be ambiguous in their everyday usage which we
        cannot allow for in mathematics. As such we must specify what we mean when
        we use these words and rid of any such ambiguity.
        The conjunction connective (\gls{conjunctionsymb}) is used to denote the
        word \textit{and}. Given two propositions $P$ and $Q$, $P\land{Q}$ is a true
        statement if and only if both $P$ and $Q$ are true. That is, we
        associate to $\land$ the following truth table:
        There are several \textit{axioms} of conjunctions that are intuitively
        obvious, but must be stated since their use is wide spread.
        \begin{faxiom}{Axioms of Conjunction}{Axioms_of_Conjunction}
            If $P$ and $Q$ are propositions, then the following are true:
            \begin{align}
                P\land{Q}&\Longleftrightarrow{Q}\land{P}
                \tag{Commutativity of Conjunction}\\
                P\land\textrm{True}&\Longleftrightarrow\textrm{P}
                \tag{Identity of Conjunction}
            \end{align}
        \end{faxiom}
    \subsection{Disjunction}
        The disjunction connective (\gls{disjunctionsymb}) represents the word
        \textit{or}. Given two propositions $P$ and $Q$, $P\lor{Q}$ is true if
        and only if $P$ is true, or $Q$ is true, or both $P$ and $Q$ are true.
        There is an unfortunate ambiguity in English as to whether $P$ or $Q$
        means $P$ is true or $Q$ is true, but not both, or whether it means
        $P$ is true or $Q$ is true, or \textit{both} are true. The convention is
        to adopt the latter definition. That is, $P\lor{Q}$ has the following
        truth table:
        \begin{table}[H]
            \centering
            \captionsetup{type=table}
            \begin{tabular}{ccc}
                $P$&$Q$&$P\lor{Q}$\\
                \hline
                0&0&0\\
                0&1&1\\
                1&0&1\\
                1&1&1
            \end{tabular}
            \caption{Truth Table for Disjunction}
            \label{tab:Truth_Table_for_Disjunction}
        \end{table}
        There is another connective called the \textit{exlusive} or, which is
        defined to be false if both $P$ and $Q$ are true. The symbol $\lor$ is
        strictly used to denote the inclusive or. That is, the word or as
        represented by the truth table in
        Tab.~\ref{tab:Truth_Table_for_Disjunction}.
    \subsection{Implication}
        Examining, we see that there are scenarios where $P\Rightarrow{Q}$
        is true and $Q\Rightarrow{P}$ is false, and similarly where
        $P\Rightarrow{Q}$ is false and $Q\Rightarrow{P}$ is true. Propositions
        $P$ and $Q$ such that $P\Rightarrow{Q}$ and $Q\Rightarrow{P}$ are called
        \textit{equivalent}, and great deal of mathematics is devoted to the
        search for equivalencies of statements. This is denoted by the
        connective $P\Leftrightarrow{Q}$. Equivalence has the following truth
        table:
        \begin{table}[H]
            \centering
            \captionsetup{type=table}
            \begin{tabular}{cccccc}
                $P$&$Q$&$P\Rightarrow{Q}$&$Q\Rightarrow{P}$
                   &$P\Leftrightarrow{Q}$
                   &$(P\Rightarrow{Q})\land(Q\Rightarrow{P})$\\
                \hline
                0&0&1&1&1&1\\
                0&1&1&0&0&0\\
                1&0&0&1&0&0\\
                1&1&1&1&1&1
            \end{tabular}
            \caption{Truth Table for Equivalence}
            \label{tab:Truth_Table_for_Equivalence}
        \end{table}
        \subsection{Misc}
        \begin{table}[H]
            \centering
            \captionsetup{type=table}
            \begin{tabular}{c c c c c c}
                \hline
                $p$&$q$&$r$&$\neg{q}$&$p\lor\neg{q}$&$(p\lor\neg{q})\land{r}$\\
                \hline
                0&0&0&1&1&0\\
                0&0&1&1&1&1\\
                0&1&0&0&0&0\\
                0&1&1&0&0&0\\
                1&0&0&1&1&0\\
                1&0&1&1&1&1\\
                1&1&0&0&1&0\\
                1&1&1&0&1&1\\
                \hline
            \end{tabular}
            \caption{Truth Table for $(p\lor\neg{q})\land{r}$}
            \label{tab:Truth_Table_Example}
        \end{table}
        \begin{theorem}
            If $a\rightarrow{b}$, if $\neg{c}\rightarrow\neg{b}$, and if
            $\neg{c}$, then $\neg{a}$.
        \end{theorem}
        \begin{proof}
            For if $a\rightarrow{b}$, then $\neg{b}\rightarrow\neg{a}$. But
            $\neg{c}\rightarrow\neg{b}$. But if $\neg{c}\rightarrow\neg{b}$ and
            $\neg{b}\rightarrow\neg{a}$, then $\neg{c}\rightarrow\neg{a}$. Thus
            $a\rightarrow{b}$, $\neg{c}\rightarrow\neg{b}$, and thus
            $\neg{c}\Rightarrow\neg{a}$.
        \end{proof}
        \begin{problem}
            If $a\rightarrow{b}$, if $\neg{c}\rightarrow\neg{b}$, and if
            $\neg{c}$, then $\neg{a}$.
        \end{problem}
        \begin{proof}
            For if $\neg c \rightarrow \neg b$, then $b\rightarrow c$. But if
            $a\rightarrow b$ and $b\rightarrow c$, then $a\rightarrow c$.
            Therefore $a\rightarrow c$. But if $a\rightarrow c$, then
            $\neg c \rightarrow \neg a$. Therefore,
            $a\rightarrow b,\neg c\rightarrow\neg b,\neg c\Rightarrow\neg a$.
        \end{proof}
        \begin{ftheorem}{Law of Syllogism}{Law_of_Syllogism}
            If $P$, $Q$, and $R$ are propositions, if $P\Rightarrow{Q}$, and if
            $Q\Rightarrow{R}$, then $P\Rightarrow{R}$.
        \end{ftheorem}