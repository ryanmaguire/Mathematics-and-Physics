\section{Zorn's Lemma}
    A relation on a set $X$ is a subset $R\subseteq{X}\times{X}$. Given an
    element $(x,y)\in{R}$, we often write $xRy$ to denote this. Here we'll write
    $x\leq{y}$.
    \begin{ldefinition}{Ordered Sets}{Ordered_Set}
        An ordered set is a set $X$ and a relation $\leq$ on $X$, denoted
        $(X,\leq)$ such that the following are true:
        \begin{enumerate}
            \item   For all $x\in{X}$, $x\leq{x}$.
            \item   For all $x,y\in{X}$ such that $x\leq{y}$ and $y\leq{x}$, it
                    is true that $x=y$.
            \item   For all $x,y,z\in{X}$ such that $x\leq{y}$ and $y\leq{z}$,
                    it is true that $x\leq{z}$.
        \end{enumerate}
    \end{ldefinition}
    \begin{ldefinition}{Majorants in Ordered Sets}{Majorant_in_Ord_Set}
        A majorant of a subset $Y\subseteq{X}$ of and ordered set $(X,\leq)$ is
        an element $x\in{X}$ such that, for all $y\in{Y}$, it is true that
        $y\leq{x}$.
    \end{ldefinition}
    \begin{lexample}{}{}
        Let $(X,d)$ be a metric space, and let $x_{0}\in{X}$. Define the
        following:
        \begin{equation}
            \mathscr{N}(x_{0})=
            \big\{\mathcal{V}\subseteq{X}:\mathcal{V}
                \textrm{ is a neighborhood of $x_{0}$}\big\}
        \end{equation}
        We can order $\mathscr{N}$ by reverse containment. That is, We have the
        following relation:
        \begin{equation}
            \leq=\big\{(\mathcal{U},\mathcal{V})\in
                \mathscr{N}(x_{0})\times\mathscr{N}(x_{0})
                :\mathcal{V}\subseteq\mathcal{U}\big\}
        \end{equation}
        That is, we write $\mathcal{U}\leq\mathcal{V}$ if $\mathcal{V}$ is a
        subset of $\mathcal{U}$. Note that, for all $x_{0}$, $\mathscr{x_{0}}$
        has a least element, or a minorant, but $X$ is such an element. But, if
        $\{x_{0}\}$ is not open, then there is no majorant.
    \end{lexample}
    \begin{ldefinition}{Totally Ordered Sets}{Tot_Ord_Set}
        A totally ordered set is an ordered set $(X,\leq)$ such that, for all
        $x,y\in{X}$, either $x\leq{y}$ or $y\leq{x}$.
    \end{ldefinition}
    \begin{ldefinition}{Maximal Element}{Maximal_Element}
        A maximal element of a subset $Y\subseteq{X}$ of a totally ordered set
        $(X,\leq)$ is an element $y$ such that:
        \begin{equation}
            \{y'\in{Y}:y\leq{y}'\}=\{y\}
        \end{equation}
        Note that $y$ is not necessary a majorant for $Y$ nor is $y$ necessarily
        unique.
    \end{ldefinition}
    \begin{ldefinition}{Inductively Ordered Sets}{Induct_Ordered_Set}
        An inductively ordered set is an ordered set $)X,\leq)$ such that, for
        all totally ordered subsets $S\subseteq{X}$, there is a majorant
        $x\in{X}$ of $S$.
    \end{ldefinition}
    That is, there exists $x\in{X}$ such that, for all $y\in{S}$, $y\leq{x}$.
    Let $X=\mathbb{R}$ and consider the set $\mathscr{N}(0)$. Then
    $\mathscr{N}(0)$ is not inductively ordered.
    \begin{ltheorem}{Zorn's Lemma}{Zorns_Lemma}
        If $(X,\leq)$ is an inductively ordered set, then there is a maximal
        element $x\in{X}$.
    \end{ltheorem}