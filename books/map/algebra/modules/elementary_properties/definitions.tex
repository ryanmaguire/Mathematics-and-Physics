Modules generalize the notion of a vector space that one might find in a linear
algebra or multivariable calculus course. More than being a generalization for
the sake of generalizing, they are found abundantly in the wild and are used in
both geometry and algebra. This chapter is devoted to developing the basics of
modules, providing definitions, theorems, and plenty of examples.
\section{Definitions}
    The first thing to do is to define a module\index{Module} over a given ring.
    \begin{fdefinition}{Modules}{Module}
        A \gls{module}\index{Module} on a \gls{ring} $\ring[R]{R}$ is an
        \gls{Abelian group} $\monoid[][+]{M}$ with a \gls{function}
        $\star:R\times{M}\rightarrow{M}$ such that for all $r_{1},r_{2}\in{R}$
        and for all $m_{1},m_{2}\in{M}$, the following are true:
        \begin{align}
            r_{1}\star(m_{1}+m_{2})
                &=(r_{1}\star{m}_{1})+(r_{1}\star{m}_{2})
                \tag{Scalar Distributivity}\\
            (r_{1}+_{R}r_{2})\star{m}_{1}
                &=(r_{1}\star{m}_{1})+_{M}(r_{2}\star{m}_{1})
                \tag{Module Distributivity}\\
            r_{1}\star(r_{2}\star{m}_{1})
                &=(r_{1}\cdot_{R}r_{2})\star{m}_{1}
                \tag{Associativity}\\
            1\star{m}_{1}&=m_{1}
            \tag{Identity}
        \end{align}
        Where 1 is the unital element of $R$. We denote a module by
        $\module{M}$.
    \end{fdefinition}
    Note that, while we've called these properties \textit{associative} and
    \textit{distributive}, we are not using these words as defined in
    Def.~\ref{def:Associative_Operation} and
    Def.~\ref{def:Distributive_Operation}, respectively.
    Associativity is a property of a binary operation, and $\star$ is not a
    binary operation. Recall that a binary operation on a set $A$ is a function
    $*:A\times{A}\rightarrow{A}$. Here, we've defined a function
    $\star:R\times{M}\rightarrow{M}$. Since, in general, it may not be true that
    $R=M$, it is not generally true that $\star$ is a binary operation. These
    equations and definitions to seem to \textit{mimic} binary operations, and
    hence it seems appropriate to attribute these notions with the same words.
    Thus, we use reuse the terms associativity and distributivity and hope that
    no confusion arises from this.
    \begin{example}
        If $\ring[R]{R}$ is a \textit{field}\index{Field}, and if
        $\module{M}$ is a module over this field, then it will
        also be a \textit{vector space}\index{Vector Space}. That is, modules
        are the ring-analog of vector spaces. Vectors spaces will be discussed
        in detail later.
    \end{example}
    \begin{example}
        \label{ex:Ring_is_Module_over_Self}%
        If $\ring{R}$ is a ring, then it can be thought of as a module over
        itself. That is, define $\module{R}$ trivially as follows:
        \twocolumneq{r_{1}\boldsymbol{+}r_{2}=r_{1}+r_{2}}
                    {r_{1}\star{r}_{2}=r_{1}\cdot{r}_{2}}
        From this we have that $R$ is a module over itself. This follows since
        by the definition of a \gls{ring}, $\monoid[][+]{R}$ is an Abelian group
        and $\monoid[][\cdot]{R}$ is a \gls{monoid} such that $\cdot$ is a
        distributive operation over $+$ (Here we are using the word distributive
        correctly. That is, as defined by
        Def.~\ref{def:Distributive_Operation}). Hence we obtain scalar and
        module distributivity, as well as associativity
        (monoids are associative). The identity property is simply a restatement
        of the fact that 1 is a unital element of the monoid
        $\monoid[][\cdot]{R}$.
    \end{example}
    For the sake of using Ex.~\ref{ex:Ring_is_Module_over_Self} in later
    theorems, we take the time to prove it.
    \begin{theorem}
        \label{thm:Ring_is_Module_over_itself}%
        If $\ring{R}$ is a ring, then $\ring{R}$ is a module over $R$.
    \end{theorem}
    \begin{proof}
        For if $\ring{R}$ is a ring, then $\monoid[][+]{R}$ is an Abelian group
        and $\monoid[][\cdot]{R}$ is a monoid such that $\cdot$ distributed over
        $+$ (Def.~\ref{def:Ring}). But then $\cdot$ is an associative binary
        operation (Def.~\ref{def:Monoid}), and hence for all $r_{1},r_{2}\in{R}$
        and for all $m_{1},m_{2}\in{R}$, we have:
        \begin{align}
            r_{1}\cdot(m_{1}+m_{2})&=r_{1}\cdot{m}_{1}+r_{1}\cdot{m}_{2}
                \tag{Associativity}\\
            (r_{1}+r_{2})\cdot{m}_{1}&=r_{1}\cdot{m}_{1}+r_{2}\cdot{m}_{1}
                \tag{Distributivity}\\
            r_{1}\cdot(r_{2}\cdot{m}_{1})&=(r_{1}\cdot{r}_{2})\cdot{m}_{1}
                \tag{Associativity}\\
            1\cdot{m}_{1}&=m_{1}
                \tag{Identity}
        \end{align}
        Thus, $\ring{R}$ is a module over $\ring{R}$ (Def.~\ref{def:Module}).
    \end{proof}
    \begin{example}
        Another example arises from the polynmial ring $R[x]$ over a commutative
        ring $\ring[R]{R}$. Define $\star:R\times{R}[x]\rightarrow{R}[x]$ by:
        \begin{equation}
            r\star{m}(x)=r\star\Big(\sum_{k=1}^{n}a_{k}x^{k}\Big)
                        =\sum_{k=1}^{n}(r\cdot_{R}{a}_{k})x^{k}
        \end{equation}
        where $\cdot_{R}$ is ring multiplication and the $a_{k}$ are elements of
        $R$. Stated another way, given a finitely supported sequence
        $a:\mathbb{N}\rightarrow{R}$, we define
        $r\star{a}:\mathbb{N}\rightarrow{R}$ by:
        \begin{equation}
            (r\star{a})_{n}=r\cdot_{R}a_{n}
            \quad\quad
            n\in\mathbb{N}
        \end{equation}
        Module addition is then just sequence addition. Given two finitely
        supported sequences $a,b:\mathbb{N}\rightarrow{R}$, we define
        $(a+b):\mathbb{N}\rightarrow{R}$ by:
        \begin{equation}
            (a+b)_{n}=a_{n}+_{R}b_{n}
        \end{equation}
        Since $a$ and $b$ are finitely supported there exists $N_{1},N_{2}$
        such that for all $n\in\mathbb{N}$ with $n>N_{1}$ it is true that
        $a_{n}=0$, and similarly for all $n>N_{2}$ it is true that $b_{n}=0$.
        Thus, for all $n\in\mathbb{N}$ such that $n>\maximum{N_{1},N_{2}}$
        we have that $a_{n}+_{R}b_{n}=0+0=0$, hence $a+b$ is finitely supported.
        Hence $+$ is a well defined binary operation on $R[x]$, and since
        $\ring{R}$ is a commutative ring, $+_{R}$ is a commutative operation,
        which in turn implies that $+$ is a commutative operation, and thus
        $\monoid[][+]{R[x]}$ an Abelian group. Then $\module[]{R}$ is a module
        over $\ring[R]{R}$.
    \end{example}
    \begin{example}
        We can also think about modules on $R[x]$ itself, since $R[x]$ is a
        ring. If we consider the case where $R$ is itself a field, then any
        module over $R[x]$ will be equivalent to a vector space over $R$ that is
        equipped with a \textit{linear transformation} from the vector space
        to itself.
    \end{example}
    \begin{example}
        \label{Abelian_Groups_as_Z_Modules}%
        If $\monoid{G}$ is an Abelian group, then it may be thought of as a
        module over the ring of integers $\ring{\mathbb{Z}}$. Define
        $\star:\mathbb{N}\times{G}\rightarrow{G}$ by:
        \begin{equation}
            1\star{g}=g
            \quad\quad
            (n+1)\star{g}=(n\star{g})*g
        \end{equation}
        For negative elements, define the following:
        \begin{equation}
            (\minus{n})\star{g}=(n\star{g})^{\minus{1}}
        \end{equation}
        where $(n\star{g})^{\minus{1}}$ is the group inverse element of
        $n\star{g}$. This is well defined since $n\star{g}$ is an element of $G$
        and $G$ is a group. Trivially, define:
        \begin{equation}
            0\star{g}=e
        \end{equation}
        where $e$ is the unital element of the group $\monoid{G}$. With this we
        $(G,*,\star)$ is a module over $\ring{\mathbb{Z}}$.
    \end{example}
    It is worthwhile to prove the claim that $(G,*,\star)$ is a module over
    $\ring{\mathbb{Z}}$ since this result is used alot. The proof is straight
    forward, but laborious.
    \begin{ftheorem}{Abelian Groups as $\mathbb{Z}$-Modules}
                    {Abelian_Groups_as_Z_Modules}
        If $\monoid{G}$ is an Abelian group, then there is a function
        $\star:\mathbb{Z}\times{G}\rightarrow{G}$ such that
        $(G,*,\star)$ is a module over $\ring{\mathbb{Z}}$.
    \end{ftheorem}
    \begin{subequations}
        \begin{align}
            (n+m)\star{g}&=(n+m-1+1)\star{g}\\
                         &=\big((n+m-1)\star{g}\big)*g\\
                         &=\Big((n\star{g})*\big((m-1)\star{g}\big)\Big)*g\\
                         &=(n\star{g})*\Big(\big((m-1)\star{g})*g\Big)\\
                         &=(n\star{g})*\big((m-1+1)\star{g}\big)\\
                         &=(n\star{g})*(m\star{g})
        \end{align}
    \end{subequations}
    For negatives we have:
    \begin{subequations}
        \begin{align}
            \big((\minus{n})+(\minus{m})\big)\star{g}
                &=\big((n+m)\star{g}\big)^{\minus{1}}\\
                &=\big((n\star{g})*(m\star{g})\big)^{\minus{1}}\\
                &=(n\star{g})^{\minus{1}}*(m\star{g})^{\minus{1}}
        \end{align}
    \end{subequations}
    This last equality comes from the fact tha $(G,*)$ is an Abelian group, and
    therefore $(a*b)^{\minus{1}}=a^{\minus{1}}*b^{\minus{1}}$. Continuing, we
    obtain:
    \begin{subequations}
        \begin{align}
            \big((\minus{n})+(\minus{m})\big)\star{g}
            &=(n\star{g})^{\minus{1}}*(m\star{g})^{\minus{1}}\\
            &=\big((\minus{n})\star{g}\big)*\big((\minus{m})\star{g}\big)
        \end{align}
    \end{subequations}
    And thus we have the distributive law holds again. Next we need to check for
    when we have one positive and one negative. We get:
    \begin{subequations}
        \begin{align}
            \big((n+1)+(\minus{m})\big)\star{g}
            &=\Big(\big(n+(\minus{m})\big)+1\Big)\star{g}\\
            &=\Big(\big(n+(\minus{m})\big)\star{g}\Big)*g\\
            &=\Big((n\star{g})*\big((\minus{m})\star{g}\big)\Big)*g
        \end{align}
    \end{subequations}
    Since $(G,*)$ is an Abelian group, we can simplify this further to get:
    \begin{subequations}
        \begin{align}
            \big((n+1)+(\minus{m})\big)\star{g}
            &=\Big((n\star{g})*\big((\minus{m})\star{g}\big)\Big)*g\\
            &=\Big((n\star{g})*g\Big)*\big((\minus{m})\star{g}\big)\\
            &=\Big((n+1)\star{g})\Big)*\big((\minus{m})\star{g}\big)
        \end{align}
    \end{subequations}
    If either $n$ or $m$ is zero, the identity holds trivially. Thus we have the
    preservation of module distributivity. For the scalar distributive law, we
    have by induction (and since $(G,*)$ is Abelian):
    \begin{subequations}
        \begin{align}
            (n+1)\star(g*h)
            &=\big(n\star(g*h)\big)*(g*h)\\
            &=(n\star{g})*(n\star{g})*(g*h)\\
            &=\big((n\star{g})*g\big)*\big((n\star{h})*h\big)\\
            &=\big((n+1)\star{g}\big)*\big((n+1)\star{h}\big)
        \end{align}
    \end{subequations}
    And therefore the distributive law over modules holds for positive integers.
    For negative integers we have:
    \begin{subequations}
        \begin{align}
            (\minus{n})\star(g*h)
            &=\big(n\star(g*h)\big)^{\minus{1}}\\
            &=\big((n\star{g})*(n\star{h})\big)^{\minus{1}}\\
            &=(n\star{g})^{\minus{1}}*(n\star{h})^{\minus{1}}\\
            &=\big((\minus{n})\star{g}\big)*\big((\minus{n})\star{g}\big)
        \end{align}
    \end{subequations}
    For $n=0$ this is trivial since the product is simply the identity element
    of $G$. The last thing to check is the compatibility of ring multiplication
    with $\star$. If $n,m\in\mathbb{N}$, we have:
    \begin{subequations}
        \begin{align}
            n\star\big((m+1)\star{g}\big)
            &=n\star\big((m\star{g})*g\big)\\
            &=\big(n\star(m\star{g})\big)*(n\star{g})\\
            &=\big((n\cdot{m})\star{g}\big)*(n\star{g})\\
            &=(n\cdot{m}+n)\star{g}\\
            &=\big(n\cdot(m+1)\big)\star{g}
        \end{align}
    \end{subequations}
    The structure of a $\mathbb{Z}$ module is different from a vector space.
    In $\mathbb{Z}$, $\{2\}$ is a linearly independent set that cannot be
    extended to a basis. This is because for all $n\in\mathbb{Z}$ such that
    $n\ne{2}$ we have that $0=x\cdot{2}+(\minus{2})\cdot{x}$, and thus the set
    $\{2,x\}$ is linearly dependent. This is contrary to the usual notions of
    vector spaces where a linearly independent set can always be extended to a
    basis. Moreover, $\{2,3\}$ is a generated set of $\mathbb{Z}$ but no subset
    of $\{2,3\}$ is a basis. To see that it generates $\mathbb{Z}$, note that
    $1=1\cdot{3}-1\cdot{2}$, and 1 generated $\mathbb{Z}$. We've already seen
    that $\{2\}$ is not a basis and cannot be extended to form one, and neither
    can 3.
    \begin{theorem}
        \label{thm:Scalar_Mult_in_Module_Defines_Group_Endo}%
        If $(M,\,\boldsymbol{+},\,\star)$ is a module over a ring
        $(R,\,+,\,\cdot\,)$ and if $r\in{R}$, then the function
        $f:M\rightarrow{M}$ defined by:
        \begin{equation}
            f(m)=r\star{m}
        \end{equation}
        is a group endomorphism on the group $(M,\boldsymbol{+})$.
    \end{theorem}
    \begin{proof}
        For let $m_{1},m_{2}\in{M}$. But since $(M,\,\boldsymbol{+},\star)$
        is a module, $\star$ left-distributes over $\boldsymbol{+}$
        (Def~\ref{def:Module}). Therefore:
        \begin{equation}
            f(m_{1}\boldsymbol{+}m_{2})
            =r\star(m_{1}\boldsymbol{+}m_{2})
            =(r\star{m}_{1})\boldsymbol{+}(r\star{m}_{2})
            =f(m_{1})\boldsymbol{+}f(m_{2})
        \end{equation}
        Thus, $f$ is a group endomorphism on $(M,\boldsymbol{+})$.
    \end{proof}
    \begin{theorem}
        If $(M,\,\boldsymbol{+},\,\star)$ is a module over a ring
        $(R,\,+,\,\cdot\,)$, if $\textrm{End}(M)$ is the set of all
        group endomorphisms on $(M,\boldsymbol{+})$, if $\boldsymbol{+}'$
        denotes function addition, and if $\circ$ denotes function composition,
        then there exists a ring homomorphism
        $\varphi:R\rightarrow(\textrm{End}(M),\boldsymbol{+},\circ)$.
    \end{theorem}
    \begin{proof}
        For let $\varphi:R\rightarrow\textrm{End}(M,\,\boldsymbol{+},\star)$ be
        defined by the function that maps $r\in{R}$ to the function
        $\varphi_{r}$ by:
        \begin{equation}
            \varphi_{r}(m)=r\star{m}
            \quad\quad
            m\in{M}
        \end{equation}
        By Thm.~\ref{thm:Scalar_Mult_in_Module_Defines_Group_Endo}, for all
        $r\in{R}$, $\varphi_{r}\in\textrm{End}(M,\,\boldsymbol{+},\star)$.
        Moreover, $\varphi$ is a ring homomorphism. For if
        $r_{1},r_{2}\in{R}$, then for all $m\in{M}$:
        \begin{align}
            \varphi_{r_{1}+r_{2}}(m)
            &=(r_{1}+r_{2})\star{m}
            \tag{Definition of $\varphi$}\\
            &=(r_{1}\star{m})\boldsymbol{+}(r_{2}\star{m})
            \tag{Module Distributivity}\\
            &=\varphi_{r_{1}}(m)\boldsymbol{+}\varphi_{r_{2}}(m)
            \tag{Definition of $\varphi$}\\
            &=(\varphi_{r_{1}}\boldsymbol{+}'\varphi_{r_{2}})(m)
            \tag{Definition of $\boldsymbol{+}'$}
        \end{align}
        And thus $\varphi$ preserves addition. Moreover:
        \begin{align}
            \varphi_{r_{1}\cdot{r}_{2}}(m)
            &=(r_{1}\cdot{r}_{2})\star{m}
            \tag{Definition of $\varphi$}\\
            &=r_{1}\star(r_{2}\star{m})
            \tag{Associativity}\\
            &=r_{1}\star\big(\varphi_{r_{2}}(m)\big)
            \tag{Definition of $\varphi$}\\
            &=\varphi_{r_{1}}\big(\varphi_{r_{2}}(m)\big)
            \tag{Definition of $\varphi$}\\
            &=(\varphi_{r_{1}}\circ\varphi_{r_{2}})(m)
            \tag{Definition of $\circ$}
        \end{align}
        And thus $\varphi$ preserves multiplication. Lastly:
        \begin{align}
            \varphi_{1_{R}}(m)&=1_{R}\star{m}
            \tag{Definition of $\varphi$}\\
            &=m
            \tag{Identity}
        \end{align}
        And thus $\varphi_{1}=\textrm{id}_{M}$, and $\textrm{id}_{M}$ is
        the unital element of $M$. Thus $\varphi$ preserves identity,
        and therefore $\varphi$ is a ring homomorphism
        (Def.~\ref{def:Ring_Homomorphism}).
    \end{proof}
    The converse of this theorem is true as well. That is, if
    $(M,\boldsymbol{+})$ is an Abelian group and if
    $\varphi:(R,+,\cdot)\rightarrow(\textrm{End}(M),\boldsymbol{+},\circ)$ is
    a ring homomorphism, then there exists an operation
    $\star:R\times{M}\rightarrow{M}$ that makes $(M,\boldsymbol{+},\circ)$ a
    module over $(R,+,\cdot)$.
    \begin{theorem}
        If $(R,+,\cdot\,)$ is a ring, if $(M,\boldsymbol{+})$ is an Abelian
        group, and if
        $\varphi:(R,+,\cdot)\rightarrow(\textrm{End}(M),\boldsymbol{+},\circ)$
        is a ring homomorphism, then there is a function
        $\star:R\times{M}\rightarrow{M}$ such that $(M,\boldsymbol{+},\star)$ is
        a module over $(R,+,\cdot)$.
    \end{theorem}
    \begin{proof}
        For let $\star:R\times{M}\rightarrow{M}$ be defined by:
        \begin{equation}
            r\star{m}=\varphi_{r}(m)
        \end{equation}
        Where $\varphi_{r}$ is the endomorphism that $r\in{R}$ gets mapped to
        by $\varphi$. Then $(M,\boldsymbol{+},\star)$ is a module over
        $(R,+,\cdot\,)$. For if $1$ is the unital element of $R$, then
        $\varphi_{1}=\textrm{id}_{M}$, since $\varphi$ is a ring homomorphism.
        But then:
        \begin{equation}
            1\star{m}=\varphi_{1}(m)=\textrm{id}_{M}(m)=m
        \end{equation}
        If $r_{1},r_{2}\in{R}$ and if $m\in{M}$, then:
        \begin{equation}
            (r_{1}+r_{2})\star{m}=\varphi_{r_{1}+r_{2}}(m)
        \end{equation}
        But $\varphi$ is a ring homomorphism and therefore
        $\varphi_{r_{1}+r_{2}}=\varphi_{r_{1}}\boldsymbol{+}'\varphi_{r_{2}}$,
        and thus:
        \begin{equation}
            (r_{1}+r_{2})\star{m}=
            (\varphi_{r_{1}}\boldsymbol{+}'\varphi_{r_{2}})(m)
            =\varphi_{r_{1}}(m)\boldsymbol{+}\varphi_{r_{2}}(m)
            =(r_{1}\star{m})\boldsymbol{+}(r_{2}\star{m})
        \end{equation}
        And thus we have module distributivity. For scalars, if $r\in{R}$ and if
        $m_{1},m_{2}\in{M}$, then since for all $r\in{R}$ it is true that
        $\varphi_{r}$ is an endomorphism on $(M,\boldsymbol{+})$, we have:
        \begin{equation}
            r\star(m_{1}\boldsymbol{+}m_{2})
            =\varphi_{r}(m_{1}\boldsymbol{+}m_{2})
            =\varphi_{r}(m_{1})\boldsymbol{+}\varphi_{r}(m_{2})
            =(r\star{m}_{1})\boldsymbol{+}(r\star{m}_{2})
        \end{equation}
        And thus the distributive law for scalars is upheld. Lastly, to check
        for the compatibility of multiplication. If $r_{1},r_{2}\in{R}$ and if
        $m\in{M}$, then:
        \begin{equation}
            r_{1}\star(r_{2}\star{m})
            =r_{1}\star(\varphi_{r_{2}}(m))
            =\varphi_{r_{1}}\big(\varphi_{r_{2}}(m)\big)
            =(\varphi_{r_{1}}\circ\varphi_{r_{2}})(m)
        \end{equation}
        But $\varphi$ is a ring homomorphism, and therefore
        $\varphi_{r_{1}\cdot{r_{2}}}=\varphi_{r_{1}}\circ\varphi_{r_{2}}$.
        Thus, we obtain:
        \begin{equation}
            r\star(m_{1}\boldsymbol{+}m_{2})
            =(r\star{m}_{1})\boldsymbol{+}(r\star{m}_{2})
            =\varphi_{r_{1}\cdot{r_{2}}}(m)
            =(r_{1}\cdot{r}_{2})\star{m}
        \end{equation}
        Thus, $(M,\boldsymbol{+},\star)$ is a module over $(R,+,\cdot\,)$.
    \end{proof}
    \begin{lexample}{Another Example of a Module}{Another_Example_of_a_Module}
        Let $(V,\boldsymbol{+},\boldsymbol{\cdot}\,)$ be a finite dimensional
        vector space over the field $(\mathbb{F},+,\cdot\,)$ and let
        $T:V\rightarrow{V}$ be a linear operator and let $\mathbb{F}[x]$ be the
        polynomial ring with coefficients in $\mathbb{F}$. We can define a
        module structure over $V$ by letting
        $\star:\mathbb{F}[x]\times{V}\rightarrow{V}$ be defined by:
        \begin{equation}
            f\star{v}=\sum_{k=0}^{n}a_{k}T^{k}(v)
        \end{equation}
        where $a_{k}$ are the coefficients of the polynomial $f\in\mathbb{F}[x]$
        and where $T^{k}$ is the $k^{th}$ composition of $T$ with itself. That
        is, $T^{2}=T\circ{T}$, $T^{n+1}=T^{n}\circ{T}$. Letting
        $\boldsymbol{+}'$ denote polynomial addition, we have that
        $(V,\boldsymbol{+}',\star)$ is a module over $\mathbb{F}[x]$ with its
        usual ring structure.
    \end{lexample}
    \begin{fdefinition}{Module Homomorphism}{Module_Homomorphism}
        A \gls{module homomorphism}\index{Module Homomorphism} from a
        \gls{module} $(M_{1},\boldsymbol{+},\star)$ to another module
        $(M_{2},\boldsymbol{+}',\diamond)$ over a \gls{ring}
        $(R,\,+,\,\cdot\,)$ is a \gls{function} $f:M_{1}\rightarrow{M}_{2}$
        such that, for all $x,y\in{M}_{1}$, and for all $r\in{R}$, the
        following are true:
        \begin{align}
            f(x\boldsymbol{+}y)&=f(\mathbf{x})\boldsymbol{+}'f(\mathbf{y})
            \tag{Preservation of Addition}\\
            f(r\star{x})&=r\diamond{f}(x)
            \tag{Preservation of Scalar Multiplication}
        \end{align}
    \end{fdefinition}
    \begin{example}
        If $(M_{1},\boldsymbol{+},\star)$ and $(M_{2},\boldsymbol{+}',\diamond)$
        are modules over $(R,+,\cdot\,)$, if if $f:M_{1}\rightarrow{M}_{2}$ is
        the zero map: $f(m)=0$ for all $m\in{M}_{1}$, then $f$ is a module
        homomorphism. As a non-trivial example, given a field
        $(\mathbb{F},+,\cdot\,)$ and two vector spaces
        $(V_{1},\boldsymbol{+},\boldsymbol{\cdot})$,
        $(V_{1},\boldsymbol{+}',\boldsymbol{\cdot}')$ any linear transformation
        $T:V_{1}\rightarrow{V}_{2}$ is a module homomorphism. Preservation of
        addition comes from the fact that $T$ is linear. That is:
        \begin{equation}
            T(\mathbf{x}\boldsymbol{+}\mathbf{y})
            =T(x)\boldsymbol{+}T(y)
        \end{equation}
        The preservation of scalar multiplication also holds since for linear
        transformations we have:
        \begin{equation}
            T(a\boldsymbol{\cdot}\mathbf{x})
            =a\boldsymbol{\cdot}{T}(\mathbf{x})
        \end{equation}
        Thus $T$ is a module homomorphism. Remember that any vector space over
        a field can be thought of as a module over the underlying field, since
        any field is also a ring. It is in this sense that $T$ may be thought of
        as a module homomorphism.
    \end{example}
    \begin{example}
        We've seen that any ring $(R,+,\cdot\,)$ can be seen as a module over
        itself. We've also seen that the ring of endomorphisms over $R$ 
        can be given a module structure as well by letting $\boldsymbol{+}'$
        denote the sum of two endomorphisms and $\star$ denoting the product
        by scalar elements of $R$. That is, $r\star{f}$ is the map defined by
        $(r\star{f})(s)=r\cdot{f}(s)$, for all $s\in{R}$. There is bijective
        homomorphism between these two modules. For let
        $\varphi:R\rightarrow\textrm{End}(R)$ be defined by:
        \begin{equation}
            \big(\varphi(r)\big)(s)=r\cdot{s}
        \end{equation}
        Then $\varphi$ is a module homomorphism. For we have:
        \begin{equation}
            \big(\varphi(r+s)\big)(t)=(r+s)\cdot{t}
                =(r\cdot{t})+(s\cdot{t})
                =\big(\varphi(r)\big)(t)+\big(\varphi(s)\big)(t)
        \end{equation}
        And thus addition is preserved. Moreover, if $r,s\in{S}$, then:
        \begin{equation}
            \big(\varphi(r\cdot{s})\big)(t)
            =(r\cdot{s})\cdot{t}
            =r\cdot(s\cdot{t})
            =r\cdot\big(\varphi(r)\big)(t)
        \end{equation}
        And thus $\varphi$ is a module homomorphism. Moreover, the identity is
        mapped to the identity, for:
        \begin{equation}
            \big(\varphi(1)\big)(s)=1\cdot{s}=\textrm{id}_{R}(s)
        \end{equation}
        We will later show that this homomorphism is bijective.
    \end{example}
    And from this we can define a module isomorphism\index{Module Isomorphism}.
    \begin{fdefinition}{Module Isomorphism}{Module_Isomorphism}
        A \gls{module isomorphism} from a \gls{module}
        $(M_{1},\boldsymbol{+},\star)$ to a module
        $(M_{2},\boldsymbol{+}',\diamond)$ over a \gls{ring}
        $(R,\,+,\,\cdot\,)$ is a \gls{module homomorphism} $f$ such that $f$ is
        a \gls{bijective function}.
    \end{fdefinition}
    \begin{example}
        Using the previous example we see with a ring $(R,+,\cdot\,)$ acting as
        a module over itself, and the ring of endomorphisms
        $(\textrm{End}(R),\boldsymbol{+}',\circ)$ given the module structure
        $(r\star{f})(s)=r\cdot{f}(s)$, are homomorphic. The homomorphism given,
        mapping $r$ to the endomorphism $f(s)=r\cdot{s}$, is bijective. It is
        injective for:
        \begin{subequations}
            \begin{equation}
                \big(\varphi(r)\big)(t)=\big(\varphi(s)\big)(t)
                \Leftrightarrow
                r\cdot{t}=s\cdot{t}
                \Leftrightarrow
                (r-s)\cdot{t}=0
            \end{equation}
        \end{subequations}
        But if $(r-s)\cdot{t}=0$ for all $t\in{R}$, then $r-s=0$
        (Thm.~\ref{thm:Mult_by_Zero_Always_Zero_Implies_Zero_in_Ring}) and
        therefore $r=s$. Moreover every element of $\textrm{End}(R)$ is mapped
        to. For let $f\in\textrm{End}(R)$. Then since $f$ is a homomorphism,
        $f(1)=1$, so we have:
        \begin{equation}
            f(s)=f(s\cdot{1})=s\cdot{f}(1)=s\cdot{r}
        \end{equation}
    \end{example}
    \begin{fdefinition}{Submodule}{Submodule}
        A \gls{submodule}\index{Submodule} of a \gls{module}
        $(M,\boldsymbol{+},\star)$ over a ring $(R,+,\cdot)$ is a \gls{subgroup}
        $(N,\boldsymbol{+})$ of $(M,\boldsymbol{+})$ such that, for all
        $n\in{N}$ and for all $r\in{R}$ it is true that $r\star{n}\in{N}$.
    \end{fdefinition}
    \begin{fdefinition}{Linear Transformation}{Linear_Transformation}
        A linear transformation from a module $(M_{1},\boldsymbol{+},\star)$ to
        a module $(M_{2},\boldsymbol{+}',\diamond)$ over a ring
        $(R,+,\cdot\,)$ is a function $T:M_{1}\rightarrow{M}_{2}$ such that,
        for all $m_{1},m_{2}\in{M}_{1}$ and for all $r_{1},r_{2}\in{R}$, the
        following is true:
        \begin{equation*}
            f\big((r_{1}\star{m}_{1})\boldsymbol{+}(r_{2}\star{m}_{2})\big)
            =\big(r_{1}\diamond{f}(m_{1})\big)\boldsymbol{+}'
                \big(r_{2}\diamond{f}(m_{2})\big)
        \end{equation*}
    \end{fdefinition}
    \begin{fdefinition}{Invariant Subspace}{Invariant_Subspace}
        And invariant subspace of a vector space $(V,\boldsymbol{+},\cdot\,)$
        over a field $(\mathbb{F},+,\cdot\,)$ under a linear transformation
        $T:V\rightarrow{V}$ is a subspace $W\subseteq{V}$ such that the image
        of $W$ under $T$ is a subset of $W$. That is, $T(W)\subseteq{W}$.
    \end{fdefinition}
    \begin{theorem}
        If $(V,\boldsymbol{+},\boldsymbol{\cdot})$ is a vector space over a
        field $(\mathbb{F},+,\cdot\,)$, if $(\mathbb{F}[x],+',*\,)$ is
        the ring of polynomials over $\mathbb{F}$, if $T:V\rightarrow{V}$ is a
        linear transformation, if
        $\star:\mathbb{F}[x]\times{V}\rightarrow{V}$ is defined by:
        \begin{equation}
            f\star{v}=\sum_{k=1}^{n}a_{k}T^{k}(v)
        \end{equation}
        where $a_{k}$ are the coefficients of $f\in\mathbb{F}[x]$ and
        $T^{k}$ is the $k^{th}$ composition of $T$ with itself, then a subset
        $W\subseteq{V}$ is a submodule of $V$ over $\mathbb{F}[x]$ if and only
        if $W$ is an invariant subspace of $V$ under $T$.
    \end{theorem}
    \begin{proof}
        For let $W$ be a submodule of $V$, let $w\in{W}$, and let $f$ be the
        polynomial $f(x)=x$. But then $f\star{w}=T(w)$, and since $W$ is a
        submodule, it is true that $f\star{w}\in{W}$. But then $T(w)\in{W}$,
        and therefore $T(W)\subseteq{W}$. Thus, $W$ is an invariant subspace
        under $T$ (Def.~\ref{def:Invariant_Subspace}). Now suppose $W$ is an
        invariant subspace of $V$ under $T$. Then for all $n\in\mathbb{N}$,
        $T^{n}(W)\subseteq{W}$ since by induction
        $T^{n}(W)=T(T^{n-1}(W))\subseteq{T}(W)\subseteq{W}$. Let
        $f\in\mathbb{F}[x]$ with coefficients $a_{k}$ and let $w\in{W}$. Since,
        for all $n\in\mathbb{N}$ it is true that $T^{n}(w)\in{W}$, and since
        $W$ is a subspace of $V$, for all $k$ it is true that
        $a_{k}\cdot{T}^{n}(w)\in{W}$. Therefore $f\star{w}\in{W}$. That is,
        $W$ is a submodule of $V$ over $\mathbb{F}[x]$.
    \end{proof}
    If we have a finite dimensional vector space $V$ and a subspace $W$ with
    basis $\mathscr{B}'$, we can extend this to a basis $\mathscr{B}$. The
    representiting matrix then has the form:
    \begin{equation}
        [T]_{\mathscr{B}}^{\mathscr{B}}=
        \begin{bmatrix}
            [T|_{W}]_{\mathscr{B}'}^{\mathscr{B}'}&*\\
            0&*
        \end{bmatrix}
    \end{equation}
    \begin{theorem}
        If $(R_{1},+,\cdot\,)$ and $(R_{2},+',*)$ are rings, and if
        $(M,\boldsymbol{+},\star)$ is a module over $(R_{2},+',*)$, and if
        $\varphi:R_{1}\rightarrow{R}_{2}$ is a ring homomorphism, then
        there is a function $\diamond:R_{1}\times{M}\rightarrow{M}$ such that
        $(M,\boldsymbol{+},\diamond)$ is a module over $(R,+,\cdot\,)$.
    \end{theorem}
    \begin{proof}
        For let $\diamond:R_{1}\times{M}\rightarrow{M}$ be defined by:
        \begin{equation}
            r\diamond{m}=\varphi(r)\star{m}
        \end{equation}
        The $(M,\boldsymbol{+},\diamond)$ is a module over $(R,+,\cdot\,)$.
        For since $(M,\boldsymbol{+},\diamond)$ is a module over $(R,+',*)$,
        $(M,\boldsymbol{+})$ is an Abelian group (Def.~\ref{def:Module}).
        Moreover, if $r_{1},r_{2}\in{R}_{1}$, and if $m\in{M}$, then:
        \begin{align}
            (r_{1}+r_{2})\diamond{m}&=\varphi(r_{1}+r_{2})\star{m}
            \tag{Definition of $\diamond$}\\
            &=\Big(\varphi(r_{1})+'\varphi(r_{2})\Big)\star{m}
            \tag{$\varphi$ is a Homomorphism}\\
            &=\big(\varphi(r_{1})\star{m}\big)\boldsymbol{+}
                \big(\varphi(r_{2})\star{m}\big)
            \tag{Distributive Property}\\
            &=\big(r_{1}\diamond{m}\big)\boldsymbol{+}\big(r_{2}\diamond{m}\big)
            \tag{Definition of $\diamond$}
        \end{align}
        And thus $\diamond$ is distributive over module elements. If
        $r\in{R}_{1}$ and if $m_{1},m_{2}\in{M}$, then:
        \begin{align}
            r\diamond(m_{1}\boldsymbol{+}m_{2})
            &=\varphi(r)\star(m_{1}\boldsymbol{+}m_{2})
            \tag{Definition of $\diamond$}\\
            &=\big(\varphi(r)\star{m}_{1}\big)\boldsymbol{+}
                \big(\varphi(r)\star{m}_{2}\big)
            \tag{Distributive Property}\\
            &=(r\diamond{m}_{1})\boldsymbol{+}(r\diamond{m}_{2})
            \tag{Definition of $\diamond$}
        \end{align}
        Also, the identity is preserved. For let $\mathbb{I}_{1}$ and
        $\mathbb{I}_{2}$ be the unital elements of $(R_{1},\cdot)$ and
        $(R_{2},*)$, respectively. Since $\varphi$ is a ring homomorphism,
        $\varphi(\mathbb{I}_{1})=\mathbb{I}_{2}$, and therefore:
        \begin{align}
            \mathbb{I}_{1}\diamond{m}
            &=\varphi(\mathbb{I}_{1})\star{m}
            \tag{Definition of $\diamond$}\\
            &=\mathbb{I}_{2}\star{m}
            \tag{Identity Property of Homomorphisms}\\
            &=m
            \tag{Identity Property of Modules}
        \end{align}
        And thus $\diamond$ preserves the identity. Lastly, $\diamond$ is
        compatible with scalar multiplication.
        \begin{align}
            r_{1}\diamond(r_{2}\diamond{m})
            &=\varphi(r_{1})\star(\varphi\star{m})
            \tag{Definition of $\diamond$}\\
            &=\big(\varphi(r_{1})*\varphi(r_{2})\big)\star{m}
            \tag{Module Compatibility with Multiplication}\\
            &=(r_{1}\cdot{r}_{2})\diamond{m}
            \tag{Definition of $\diamond$}
        \end{align}
        Therefore, $(M,\boldsymbol{+},\diamond)$ is a module over
        $(R_{1},+,\cdot\,)$ (Def.~\ref{def:Module}).
    \end{proof}
\section{Old}
    An attempt has been made to preserve the differences
    between the various operations in a left module.
    $+$ and $\cdot$ are binary operations that act on
    elements of $R$. That is, for $a,b\in{R}$, $a+b$
    gives another element of $R$, as does $a\cdot{b}$.
    However, $+_{M}$ is a binary operation over
    $M$. If $a,b\in{R}$, $a+_{M}b$ has no meaning.
    For $a,b\in{M}$, $a+_{M}b$ is well defined, and returns
    another element of $M$. The ``function,'' $*$
    takes an ordered pair $(r,a)$, where $r\in{R}$ and
    $a\in{M}$, and returns another element in $M$. For
    convenience we write $r*a$. If $a,b\in{R}$, then
    $a*b$ has no meaning, and if $a,b\in{M}$ then
    $a*b$ also has no meaning. Usually this is very
    unimportant, and $+$ and $+_{M}$ are given the same
    symbol, as are $\cdot$ and $*$. We can then more loosely
    rewrite the definition as, for all $r,s\in{R}$, and
    all $a,b\in{M}$:
    \begin{enumerate}
        \begin{multicols}{2}
            \item $r(a+b)=ra+rb$
            \item $(rs)(a)=r(sa)$
            \item $(r+s)(a)=(ra)+(rs)$
            \item $1a=a$
        \end{multicols}
    \end{enumerate}
    This is the more natural notation one finds when defining
    vector spaces. A module is analogous to a vector space:
    In a vector space one has a set $V$ and a
    \textit{field} $K$, whereas in a module one has a set
    $M$ and a \textit{ring with identity} $R$.
    \begin{theorem}
        If $(G,+)$ is an Abelian group, then there is a
        function $*:\mathbb{Z}\times{G}\rightarrow{G}$
        such that $(G,+_{G})$ is a left module of
        $(\mathbb{Z},+,\cdot)$, where $+$ and $\cdot$ are
        the standard arithmetic operations over $\mathbb{Z}$.
    \end{theorem}
    \begin{proof}
        For define $0*a=e$ and $1*a=a$ for all $a\in{G}$,
        and for all
        $n\in\mathbb{Z}$, $n>1$ inductively define
        $(n+1)*a=n*a+_{G}a$. For $n<0$ define
        $n*a=((-n)*a)^{-1}$, where the inverse is taken
        with respect to the group $G$. Then $*$ is a function
        $*:\mathbb{Z}\times{G}\rightarrow{G}$. If $n>0$, we
        have the following:
        \begin{align*}
            (n*a)+_{G}(n*b)
            &=\underset{n}{\underbrace{(a+_{G}\cdots+_{G}a)}}
            +_{G}
            \underset{n}{\underbrace{(b+_{G}\cdots+_{G}b)}}\\
            &=\underset{n}
                {\underbrace{(a+b)+_{G}\cdots+_{G}(a+b)}}\\
            &=n*(a+b)
        \end{align*}
        If $n,m>0$, we have:
        \begin{equation*}
            n*a+_{G}m*a
            =\underset{n+m}{\underbrace{a+_{G}\cdots+_{G}a}}
            =(n+m)*a
        \end{equation*}
        And finally:
        \begin{equation*}
            (n\cdot{m})*a=
            \underset{n\cdot{m}}
                {\underbrace{a+_{G}\cdots+_{G}a}}
            =n*(\underset{m}{\underbrace{a+_{G}\cdots+_{G}a}})
            =n*(m*a)
        \end{equation*}
        Similarly for when $n,m<0$, $n<0<m$, or $m<0<n$.
    \end{proof}
    Thus, every Abelian group $(G,+_{G})$ can be seen
    as a left module over $(R,+,\cdot)$. Moreover the
    function $*$ is unique, so this correspondence is
    unique as well. As another example, every vector space
    $V$ over a field $K$ is a left module over $K$, since
    any field $K$ is also a ring with identity.
    A \textbf{Left Ideal} of a ring $(R,+,*)$ is a subset
    $I\subseteq{R}$ such that:
    \begin{enumerate}
        \item $\forall_{a,b\in{I}},a+b\in{I}$
        \item $\forall_{r\in{R}}\forall_{a\in{I}},r\cdot{a}\in{I}$
    \end{enumerate}
    This can be rephrased by saying that $(I,+)$ is a subgroup
    of $(R,+)$, and $I$ absorbs left-multiplication. A
    \textbf{Right Ideal} replaces $r\cdot{a}$ with
    $a\cdot{r}$. An \textbf{Ideal} or \textbf{Two-Sided Ideal}
    is a subset that is both a left and a right ideal.
    \begin{theorem}
        If $(R,+,\cdot)$ is a ring with identity
        and $(I,+)$ is a left ideal
        of $R$, then there is a function
        $*:R\times{I}\rightarrow{I}$ such that
        $(I,+)$ is a left module over $R$.
    \end{theorem}
    \begin{proof}
        For let $*$ be the restriction of $\cdot$ to
        $R\times{I}$. Then, for all $a\in{I}$,
        $1\cdot{a}=a$. If $a,b\in{I}$ and $r\in{R}$, then:
        \begin{equation*}
            r*(a+b)
            =r\cdot(a+b)
            =(r\cdot{a})+(r\cdot{b})
        \end{equation*}
        If $r,s\in{R}$ and $a\in{I}$, then:
        \begin{gather*}
            (r\cdot{s})*a
            =(r\cdot{s})\cdot{a}
            =r\cdot(s\cdot{a})\\
            (r+s)*(a)
            =(r+s)\cdot{a}
            =(r\cdot{a})+(s\cdot{a})
        \end{gather*}
    \end{proof}
    \begin{definition}
        The Annihilator of a Left Module $(M,+_{M})$ over
        a ring with identity $(R,+,\cdot)$ is the set:
        \begin{equation*}
            I=\{r\in{R}:\forall_{m\in{M}},r*m=0\}
        \end{equation*}
    \end{definition}
    \begin{theorem}
        If $(M,+_{M})$ is a Left Module over a ring with
        identity $(R,+,\cdot)$, and if $I$ is the
        annihilator of $M$, then $I$ is a two-sided
        ideal of $R$.
    \end{theorem}
    \begin{proof}
        For if $r,s\in{I}$, then for all $m\in{M}$,
        $r*m=0$ and $s*m=0$. But $(r+s)*m=(r*m)+_{M}(s*m)=0$.
        Therefore $r+s\in{I}$. If $r\in{R}$ and $s\in{I}$,
        then $(r\cdot{s})*m=r*(s*m)=r*0=0$, and therefore
        $r\cdot{s}\in{I}$. Furthermore,
        $(s\cdot{r})*m=s*(r*m)=0$, and thus $s\cdot{r}\in{I}$.
        Therefore $I$ is a two-sided ideal.
    \end{proof}