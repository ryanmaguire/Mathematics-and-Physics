\section{Dihedral Groups}
    Pretty pictures, presentation, etc. If $k\in\mathbb{Z}_{k}$ then
    $fr^{k}=r^{\minus{k}}f$. Every element has unique representation
    $f^{j}r^{k}$ with $j=0$ or $j=1$ and $k\in\mathbb{Z}_{n}$.
    \begin{fdefinition}{Generator of a Group}{Generator_of_a_Group}
        A generator of a group $\monoid{G}$ is a subset $S\subseteq{G}$
        such that for all $g\in{G}$ there exists an $n\in\mathbb{N}$ and
        a sequence $a:\mathbb{Z}_{n}\rightarrow{S}$ such that:
        \begin{equation*}
            g=\prod_{k\in\mathbb{Z}_{n}}a_{k}
        \end{equation*}
    \end{fdefinition}
    Note that sequences allow for repetition and have a notion of order,
    and so this respect the potential non-commutativity of a group.
    Another way of interpreting this definition is that every element of
    $G$ can be written as the finite product of elements in $S$.
    \begin{example}
        For any dihedral group $(D_{2n},\circ)$, the set
        $S=\{r,f\}$ containing the rotation and the reflection element
        is a generator for $D_{2n}$. To tell two such dihedral groups
        apart we introduce \textit{relations}. For example, $r^{n}=e$
        and $f^{2}=e$, and these are the least such positive integers
        with these properties. Moreover, $rf=fr^{\minus{1}}$.
    \end{example}
    \begin{example}
        Consider $G$ with the presentation:
        \begin{equation}
            G=\langle{a},b\;|\;a^{n}=e,b^{2}=e,ab=ba^{2}\rangle
        \end{equation}
        Let's see if we can determine what this group is. We have:
        \par\vspace{-2.5ex}
        \begin{subequations}
            \begin{minipage}[t]{0.49\textwidth}
                \begin{align}
                    a&=ae\\
                    &=ab^{2}\\
                    &=(ab)b\\
                    &=(ba^{2})b\\
                    &=(ba)(ab)
                \end{align}
            \end{minipage}
            \hfill
            \begin{minipage}[t]{0.49\textwidth}
                \begin{align}
                    &=(ba)(ba^{2})\\
                    &=(bab)a^{2}\\
                    &=b(ba^{2})a^{2}\\
                    &=b^{2}a^{4}\\
                    &=a^{4}
                \end{align}
            \end{minipage}
        \end{subequations}
        \par\vspace{2.5ex}
        and hence by the cancellation law we conclude that $a^{3}=e$.
        Hence, we may take $n$ to be either 1, 2, or 3. If $n=1$ we are
        left with the group presented by a single variable $b$ such that
        $b^{2}=e$, and this is just $\mathbb{Z}/2\mathbb{Z}$. If $n=2$
        then from $ab=ba^{2}$ we conclude $ab=b$, and hence $a=e$, again
        leading us to $\mathbb{Z}/2\mathbb{Z}$. Finally, with $n=3$ we
        have $ab=ba^{2}$ implying that $ab=ba^{\minus{1}}$, and this is
        precisely the presentation for the dihedral group $D_{6}$.
    \end{example}
    \begin{example}
        Consider $G$ defined by:
        \begin{equation}
            G=\langle{a,b}\;|\;a^{4}=e,b^{3}=e,ab=b^{2}a^{2}\rangle
        \end{equation}
        Let's show that $G$ is just the trivial group. We have:
        \begin{equation}
            bab=b(ab)=b(b^{2}a^{2})=b^{3}a^{2}=ea^{2}=a^{2}
        \end{equation}
        and hence:
        \begin{equation}
            e=a^{4}=(bab)^{2}=(bab)(bab)=bab^{2}ab=
            bab^{2}b^{2}a^{2}=bab^{3}ba^{2}=baba^{2}
        \end{equation}
        and since $b^{3}=e$, we apply the cancellation law to obtain
        $b^{2}=aba^{2}$. But then:
        \begin{equation}
            a=ae=ab^{3}=(ab)b^{2}=b^{2}a^{2}(aba^{2})
            =b^{2}a^{3}ba^{2}=b^{\minus{1}}a^{\minus{1}}ba^{2}
        \end{equation}
        and hence $aba=ba^{2}$, and thus $ab=ba$. So the operation
        commutes. But then:
        \begin{equation}
            ab=b^{2}a^{2}=a^{2}b^{2}
        \end{equation}
        and so applying cancellation we have $ab=e$ and hence
        $a=b^{\minus{1}}$. But $b^{\minus{1}}=b^{2}$ since $b^{3}=e$
        and thus $a=b^{2}$. But then:
        \begin{equation}
            a^{3}=(b^{2})^{3}=(b^{3})^{2}=e
        \end{equation}
        But $a^{4}=e$ and therefore by the cancellation law $a=e$.
        And since $b=a^{\minus{1}}$ we have that $b=e$. Hence, $G$ is
        the trivial group.
    \end{example}
    The presentation of the general dihedral group $D_{2n}$ is:
    \begin{equation}
        D_{2n}=\langle{r,f}\;|\;r^{n}=e,f^{2}=e,rf=fr^{\minus{1}}\rangle
    \end{equation}
    \begin{theorem}
        If $D_{2n}$ is the dihedral group with rotational generator $r$
        and reflectional generator $f$, and if $x\in{D}_{2n}$ is not a
        power of $r$, then $rx=xr^{\minus{1}}$.
    \end{theorem}
    \begin{proof}
        By induction. Suppose $a:\mathbb{Z}_{k}\rightarrow\{r,f\}$ is
        a least sequence such that the product is $x$. In the base case
        of $k=1$, since $x$ is not a power of $r$ we simply have that
        $x=f$. But $rf=fr^{\minus{1}}$, so we are done. Suppose it is
        true of $k\in\mathbb{N}$ and let
        $a:\mathbb{Z}_{k+1}\rightarrow\{r,f\}$ be a sequence whose
        product equals $x$. Then either $a_{k}=r$ or $a_{k}=f$. Suppose
        $a_{k}=r$. Since $x$ is not a power of $r$, there is and
        $i\in\mathbb{Z}_{k}$ such that $a_{i}\ne{r}$. Define $y$ by:
        \begin{equation}
            y=\prod_{j\in\mathbb{Z}_{n}}a_{j}
        \end{equation}
        Then since $y$ is not a power of $r$, $ry=yr^{\minus{1}}$. But
        then:
        \begin{equation}
            rx=r(yr)=(ry)r=(yr^{\minus{1}})r=y(r^{\minus{1}}r)=ye=y
        \end{equation}
        but $x=yr$, and hence $y=xr^{\minus{1}}$. Thus
        $rx=x^{\minus{1}}$. If $a_{k}=f$, let $y$ be defined similarly.
        If $y=r^{k}$, then:
        \begin{equation}
            rx=ryf=rr^{k}f=r^{k}rf=r^{k}fr^{\minus{1}}=xr^{\minus{1}}
        \end{equation}
        if $y$ is not a power of $r$, then by the induction hypothesis
        we have that $ry=yr^{\minus{1}}$. But then:
        \begin{equation}
            rx=r(yr)=(ry)r=(yr^{\minus{1}})r=y(r^{\minus{1}}r)=ye=y
        \end{equation}
        But $x=yr$ and hence $y=xr^{\minus{1}}$, so $rx=xr^{\minus{1}}$,
        as claimed.
    \end{proof}
    \begin{theorem}
        In $D_{2n}$ every element that is not a power of $r$ has order
        2.
    \end{theorem}
    \begin{proof}
        For every such element can be written $fr^{k}$ and hence
        $x^{2}=(fr^{k})(fr^{k})=(fr^{k})(r^{\minus{k}}f)=f^{2}=e$.
    \end{proof}
    \begin{theorem}
        If $n\in\mathbb{N}^{+}$, if $D_{2n}$ is the dihedral group with
        rotational element $r$, then the order of $r$ is $n$.
    \end{theorem}
    \begin{proof}
        By definition of $D_{2n}$, $r^{n}$ is the identity element.
    \end{proof}
    \begin{example}
        We can also consider the dihedral group on two points. We have
        the presentation:
        \begin{equation}
            D_{4}=\langle{r,f}\;|\;
                r^{2}=e,f^{2}=e,rf=fr^{\minus{1}}\rangle
        \end{equation}
        This is isomorphic to
        $\mathbb{Z}/2\mathbb{Z}\times\mathbb{Z}/2\mathbb{Z}$. To see
        this, note that the presentation allows us to show that $D_{4}$
        is commutative. Since $r^{2}=e$ we know that $r^{\minus{1}}=r$
        and thus the last relation shows that $rf=fr$, so $D_{4}$ is
        commutative. Let
        $\phi:\mathbb{Z}/2\mathbb{Z}\times\mathbb{Z}/2\mathbb{Z}%
         \rightarrow{D}_{4}$ be defined by:
        \begin{equation}
            \phi(x)=
            \begin{cases}
                e,&x=(0,0)\\
                r,&x=(1,0)\\
                f,&x=(0,1)\\
                rf,&x=(1,1)
            \end{cases}
        \end{equation}
        Then $\phi$ is injective. Moreover, it is surjective. For if
        $y\in{D}_{4}$ then since $D_{4}$ is commutative we know that
        $y=r^{j}f^{k}$. But we only care about $j$ and $k$ mod 2, and
        so we have four possibilies, each of which is mapped onto by
        $\phi$. Moreover, it is an isomorphism.
    \end{example}
\section{Polyhedra Groups}
    DF 1.2 Problems 9-13.
