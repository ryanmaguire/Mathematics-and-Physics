\section{Preliminaries}
    A collection of sets $\mathcal{O}$ with the finite intersection property is
    one such that for every finite subset $\Delta\subseteq\mathcal{O}$ the
    intersection is non-empty: $\bigcap\Delta\ne\emptyset$. A cofinal subset of
    a partial ordered set $(X,\leq)$ is a subset $Y\subseteq{X}$ such that for
    all $x\in{X}$ there is a $y\in{Y}$ with $x\leq{y}$. Characteristic function
    of a set $E\subseteq{X}$ is a function $\xi_{E}:X\rightarrow\{0,1\}$ defined
    by:
    \begin{equation}
        \xi_{E}(x)=
        \begin{cases}
            0,&x\notin{E}\\
            1,&x\in{E}
        \end{cases}
    \end{equation}
    The cartesian product of a non-empty collection of sets $\mathcal{O}$ is the
    set of all functions $f:\mathcal{O}\rightarrow\bigcup\mathcal{O}$ such that
    for all $\mathcal{U}\in\mathcal{O}$ it is true that
    $f(\mathcal{U})\in\mathcal{U}$.
\section{Notes from Hewitt and Ross}
For a non-empty set $X$, $\ell_{p}(X)$ is the set of all function
$\alpha:X\rightarrow\mathbb{C}$ such that:
\begin{equation}
    \sum_{x\in{X}}\norm{\alpha(x)}<\infty
\end{equation}
Inner automorphisms of a group are the function $f:G\rightarrow{G}$ defined by
$f(x)=a*x*a^{\minus{1}}$ for some fixed $a\in{G}$. The set of inner
automorphisms forms a subgroup of the automorphism group of $G$. The function
$\varphi:G\rightarrow\autgroup[\textbf{Grp}]{G}$ defined by
$\varphi(g)=f_{g}$, where $f_{g}$ is the automorphism
$f_{g}(x)=g*x*g^{\minus{1}}$, is itself a homomorphism. Given a normal subgroup
$H$, and an element $a\in{G}$, the mapping
$\varphi_{a}:G/H\rightarrow{G}/H$ defined by
$\varphi_{a}(xH)=(a*x)H$, where $xH$ is the left coset of $H$ by $x\in{G}$. This
is an isomorphism. Given $xH$ and $yH$, the mapping
$\varphi_{xy^{\minus{1}}}$ takes $yH$ to $xH$. Conjugate elements are those
such that $a=gbg^{\minus{1}}$ for some $g\in{G}$. Conjugacy classes partition a
group.
\par\hfill\par
If $G$ is a group, and $H$ is a normal subgroup, then for any subgroup $A$ of
$G$, $AH=HA$ is a subgroup, $H\subseteq{AH}$ is normal, as is $A\cap{H}$, and
$(AH)/H$ is isomorphic to $A/(H\cap{A})$. Use the mapping:
\begin{equation}
    f(aH)=(aH)\cap{A}=a(H\cap{A})
\end{equation}
with $a\in{A}$. Note the preimage of subgroups by a homomorphism is again a
subgroup, as is the forward image. If $G$ and $H$ are groups, if
$\varphi:G\rightarrow{G}$ is a homomorphism, if $B\subseteq{H}$ is a normal
subgroup, and if $A\subseteq{G}$ is defined by $A=\varphi^{\minus{1}}[B]$, then
$G/A$, $H/B$, and $(G/\ker{\varphi})/(B/\ker{\varphi})$ are isomorphic.
\par\hfill\par
The direct product of groups is defined on the Cartesian product by performing
multiplication component-wise. The \textit{weak} direct product is a subset of
the Cartesian product $\prod_{\alpha\in{A}}G_{\alpha}$ where given
$f\in\prod_{\alpha\in{A}}G_{\alpha}$, for all but finitely many $\alpha\in{A}$
it is true that $f_{\alpha}=e_{\alpha}$, the identity element of $G_{\alpha}$.
\begin{theorem}
    If $\monoid{G}$ is a group, if $A_{1},\dots,A_{n}$ are a collection of
    normal subgroups of $G$ such that $A_{1}*A_{2}*\cdots*A_{n}=G$, and if for
    all $k<n$ we have $(A_{1}*\cdots*A_{k})\cap{A}_{k+1}=\{e\}$, then $G$ is
    isomorphic to $\prod_{k\in\mathbb{Z}_{n}}A_{k}$.
\end{theorem}
This is just the recognition principle for groups but with finitely many groups
instead of just two.
\begin{theorem}
    If $\monoid{G}$ is a group, if $\Lambda\subseteq\powset{G}$ is a collection
    of normal subgroups of $G$ indexed by a set $I$ with indexing function
    $N:I\rightarrow\Lambda$, if for all $i\in{I}$, $M_{I}$ is the smallest
    subgroup of $G$ containing all $N_{\lambda}\in\Lambda$ for $i\ne\lambda$, if
    $G$ is the smallest subgroup containing all $N_{i}$, and if
    $M_{i}\cap{N}_{i}=\{e\}$ for all $i\in{I}$, then $G$ is isomorphic to the
    weak direct product of the $N_{i}$
\end{theorem}
For proof, Hewtii and Ross 2.5. If $\monoid{G}$ is a group, if $A\subseteq{G}$
is a normal subgroup, and if $B\subseteq{G}$ is a subgroup such that
$A*B=G$ and $A\cap{B}=\{e\}$. We know that if $B$ is also a normal subgroup,
then $G$ is isomorphic to the direct product of $A$ and $B$. In the case that
$B$ is not normal, we can still reconstruct $G$ is we know about the inner
automorphisms of $B$ behave on $A$. Let $a_{1},a_{2}\in{A}$ and
$b_{1},b_{2}\in{B}$. Then:
\begin{subequations}
    \begin{align}
        (a_{1}*b_{1})*(a_{2}*b_{2})
        &=x_{1}*(b_{1}*a_{2}*b_{1}^{\minus{1}})*b_{1}*b_{2}\\
        &=\big(x_{1}*\varphi_{b_{1}}(a_{2})\big)*(b_{1}*b_{2})
    \end{align}
\end{subequations}
And since $A*B=G$, we can write any arbitrary element of $G$ in the form
$(a_{1}*\varphi(a_{2}))*(b_{1}*b_{2})$ where $\varphi$ is the appropriate
inner automorphism. This is how we define our operation:
\begin{equation}
    (a_{1},b_{1})\cdot(a_{2},b_{2})
    =(a_{1}*\varphi(a_{2}),b_{1}*b_{2})
\end{equation}
We take this and abstract it for any two groups $\monoid[G]{G}$ and
$\monoid[H]{H}$ with a group homomorphism
$\varphi:H\rightarrow\autgroup[\textbf{Grp}]{\monoid{G}}$. The underlying set of
the semidirect product is just $G\times{H}$, and the operation $\semidirectprod$
is defined by:
\begin{equation}
    (g_{1},h_{1})\semidirectprod[\varphi](g_{2},h_{2})=
    \big(g_{1}*_{G}\varphi_{h_{1}}(g_{2}),\,h_{1}*_{H}h_{2}\big)
\end{equation}
$(G\times{H},\semidirectprod[\varphi])$ is indeed a group. The identity is
$(e_{G},e_{H})$, and the inverse of $(g,h)$ is the element
$(\varphi_{h^{\minus{1}}}(g^{\minus{1}}),h^{\minus{1}})$. To recover the original
construction, consider $A=G\times\{e_{H}\}$ and $B=\{e_{G}\}\times{H}$. Then
$A$ is a normal subgroup, $B$ is a subgroup, and $A\cap{B}$ is the identity of
the semidirect product group.
\par\hfill\par
The free \textit{Abelian} group can be defined as the Abelianization of the free
group of a set $X$, or as the weak direct product of the free groups $F_{x}$ for
all $x\in{X}$. For $X$ a finite set this is just $\mathbb{Z}^{n}$. The set of
all permutations of a set $X$ is a group. A transitive subset of this is one
such that for all $x,y\in{X}$ there is an $f$ with $f(x)=y$.
\par\hfill\par
Hewitt and Ross Theorem 2.9.
\par\hfill\par
\begin{theorem}
    If $\topspace{X}$ is a normal Hausdorff topological space, if
    $\mathcal{C}\subseteq{X}$ is a closed subset of $X$, and if $\mathcal{O}$ is
    a finite open cover of $F$ of cardinality $N\in\mathbb{N}$, then there is a
    partition of unity of $F$ subordinate to $\mathcal{O}$. That is, a function
    $F:\mathbb{Z}_{N}\times[0,1]$ such that for all $x\in\mathcal{C}$ we have:
    \begin{equation}
        \sum_{n\in\mathbb{Z}_{N}}F_{n}(x)=1
    \end{equation}
    and $\support{F_{n}}\subseteq\mathcal{O}_{n}$.
\end{theorem}
\begin{proof}
    We prove by induction on $N$. Suppose first that $\bigcup\mathcal{O}=X$. We
    must show there are closed sets $\mathcal{C}_{1},\dots,\mathcal{C}_{N}$ such
    that $\bigcup_{n=1}^{N}\mathcal{C}_{n}=X$ with
    $\mathcal{C}_{n}\subseteq\mathcal{U}_{n}$. In the base case of $N=1$ we have
    $\mathcal{U}_{1}=X$, and simply choosing $\mathcal{C}_{1}=X$ solves the
    problem. Suppose $\mathcal{U}_{1}\cap\mathcal{U}_{2}=X$. Then
    $X\setminus\mathcal{U}_{1}$ and $X\setminus\mathcal{U}_{2}$ are disjoint,
    for if $x\in(X\setminus\mathcal{U}_{1})\cap(X\setminus\mathcal{U}_{2})$,
    then $x\notin\mathcal{U}_{1}\cup\mathcal{U}_{2}$, a contradiction since
    $\mathcal{U}_{1}\cup\mathcal{U}_{2}=X$. But then $X\setminus\mathcal{U}_{1}$
    and $X\setminus\mathcal{U}_{2}$ are disjoint closed subsets, and hence since
    $\topspace{X}$ is normal there exists open subsets $\mathcal{V}_{1}$ and
    $\mathcal{V}_{2}$ such that
    $X\setminus\mathcal{U}_{1}\subseteq\mathcal{V}_{1}$,
    $X\setminus\mathcal{U}_{2}\subseteq\mathcal{V}_{2}$, and
    $\mathcal{V}_{1}\cap\mathcal{V}_{2}=\emptyset$. Let
    $\mathcal{C}_{1}=X\setminus\mathcal{V}_{1}$ and
    $\mathcal{V}_{2}=X\setminus\mathcal{V}_{2}$.
\end{proof}