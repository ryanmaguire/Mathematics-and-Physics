\section{Integral of Signed Functions}
    Given a function $f:\Omega\rightarrow\mathbb{R}$, we can define
    the following two functions:
    \begin{equation}
        f^{+}(\omega)=
        \begin{cases}
            f(\omega),&f(\omega)\geq{0}\\
            0,&f(\omega)<0
        \end{cases}
    \end{equation}
    \begin{equation}
        f^{+}(\omega)=
        \begin{cases}
            0,&f(\omega)\geq{0}\\
            -f(\omega),&f(\omega)<0
        \end{cases}
    \end{equation}
    From these definitions we see that:
    \begin{equation}
        f=f^{+}-f^{-}
    \end{equation}
    There are two useful formala for computed $f^{+}$ and $f^{-1}$:
    \begin{align}
        f^{+}&=\frac{|f|+f}{2}\\
        f^{-}&=\frac{|f|-f}{2}
    \end{align}
    \begin{theorem}
        If $f$ is measurable, then $f^{+}$ and $f^{-}$ are measurable.
    \end{theorem}
    \begin{ldefinition}{Integral of Signed Function}
        The integral of a measurable function $f$ such that either
        the integral of $f^{+}$ or the integral of $f^{-}$, or both,
        is finite, is the difference:
        \begin{equation}
            \int_{\Omega}f\diff{\mu}=
            \int_{\Omega}f^{+}\diff{\mu}-\int_{\Omega}f^{-}\diff{\mu}
        \end{equation}
    \end{ldefinition}
    \begin{ldefinition}{Summable Function}
        A summable function is a function $f$ such:
        \begin{equation}
            \int_{\Omega}f^{+}\diff{\mu}<\infty
        \end{equation}
        \begin{equation}
            \int_{\Omega}f^{-}\diff{\mu}<\infty
        \end{equation}
    \end{ldefinition}
    \begin{theorem}
        A function $f$ is summable if and only if:
        \begin{equation}
            \int_{\Omega}|f|\diff{\mu}<\infty
        \end{equation}
    \end{theorem}
    \begin{ltheorem}{Homogeneity of the Integral of Signed Functions}
        If $f$ is a signed integrable function, and if $c$ is a
        real number, then:
        \begin{equation}
            \int_{\Omega}(cf)\diff{\mu}=
            c\int_{\Omega}f\diff{\mu}
        \end{equation}
    \end{ltheorem}
    \begin{ltheorem}{Additivity of the Integral of Signed Functions}
        If $f$ and $g$ are summable functions, then:
        \begin{equation}
            \int_{\Omega}(f+g)\diff{\mu}
            =\int_{\Omega}f\diff{\mu}+\int_{\Omega}g\diff{\mu}
        \end{equation}
    \end{ltheorem}
    \begin{proof}
        For:
        \begin{equation}
            (f+g)^{+}=
            \frac{|f+g|+f+g}{2}\leq
            \frac{|f|+|g|+f+g}{2}=f^{+}+g^{+}
        \end{equation}
        Similarly:
        \begin{equation}
            (f+g)^{-}\leq{f}^{-}+g^{-}
        \end{equation}
        And therefore $f+g$ is summable. Evaluating the integral:
        \begin{equation}
            \int_{\Omega}(f+g)\diff{\mu}=
            \int_{\Omega}(f+g)^{+}\diff{\mu}-
            \int_{\Omega}(f+g)^{-}\diff{\mu}
        \end{equation}
        But we have:
        \begin{align}
            f+g&=(f+g)^{+}-(f+g)^{-}\\
            &=(f^{+}-f^{-})+(g^{+}-g^{-})
        \end{align}
        Rearranging, we have:
        \begin{equation}
            (f+g)^{+}+f^{-}+g^{-}=
            (f+g)^{-}+f^{+}+g^{+}
        \end{equation}
        Computing the integral, we have:
        \begin{equation}
            \begin{split}
                \int_{\Omega}(f+g)^{+}\diff{\mu}+
                \int_{\Omega}f^{-}&\diff{\mu}+
                \int_{\Omega}g^{-}\diff{\mu}\\
                &=\int_{\Omega}(f+g)^{-}\diff{\mu}+
                \int_{\Omega}f^{+}\diff{\mu}+
                \int_{\Omega}g^{+}\diff{\mu}
            \end{split}
        \end{equation}
        Rearranging this, we obtain:
        \begin{equation}
            \begin{split}
                \int_{\Omega}(f+g)^{+}\diff{\mu}-
                \int_{\Omega}&(f+g)^{-}\diff{\mu}\\
                &=\int_{\Omega}f^{+}\diff{\mu}-
                \int_{\Omega}f^{-}\diff{\mu}+
                \int_{\Omega}g^{+}\diff{\mu}-
                \int_{\Omega}g^{-}\diff{\mu}
            \end{split}
        \end{equation}
        This completes the proof.
    \end{proof}
    \begin{theorem}
        If $f$ is integrable and if $E_{1}$ and $E_{2}$ are disjoint,
        then:
        \begin{equation}
            \int_{E_{1}\cup{E}_{2}}f\diff{\mu}=
            \int_{E_{1}}f\diff{\mu}+\int_{E_{2}}f\diff{\mu}
        \end{equation}
    \end{theorem}
    \begin{theorem}
        If $f$ and $g$ are summable, and if $f\geq{g}$, then:
        \begin{equation}
            \int_{\Omega}f\diff{\mu}\geq\int_{\Omega}g\diff{\mu}
        \end{equation}
    \end{theorem}
    \begin{theorem}
        If $f=0$ $\mu$ almost everywhere, then:
        \begin{equation}
            \int_{\Omega}f\diff{\mu}=0
        \end{equation}
    \end{theorem}
    \begin{theorem}
        If $f$ is an integrable signed function such that:
        \begin{equation}
            \int_{\Omega}f\diff{\mu}=0
        \end{equation}
        Then $f=0$ $\mu$ almost everywhere.
    \end{theorem}
    \begin{ltheorem}{The Triangle Inequality for Integrals}
        If $f$ is an integrable signed function, then:
        \begin{equation}
            \Big|\int_{\Omega}f\diff{\mu}\Big|
            \leq\int_{\Omega}|f|\diff{\mu}
        \end{equation}
    \end{ltheorem}
    \begin{ltheorem}{Monotone Convergence for Signed Functions}
        If $F$ is a summable function, if $f_{n}$ is a sequence of
        measurable functions that is monotonically increasing and such
        that, for all $n\in\mathbb{N}$, $F\leq{f}_{n}$, then:
        \begin{equation}
            \underset{n\rightarrow\infty}{\lim}
            \int_{\Omega}f_{n}\diff{\mu}
            =\int_{\Omega}\underset{n\rightarrow\infty}{\lim}
            f\diff{\mu}
        \end{equation}
    \end{ltheorem}
    \begin{proof}
        For let $\tilde{f}_{n}$ be defined by:
        \begin{equation}
            \tilde{f}_{n}(\omega)=f_{n}(\omega)-F(\omega)
        \end{equation}
        Then for all $n\in\mathbb{N}$ and for all $\omega$,
        $\tilde{f}_{n}(\omega)\geq{0}$. But then by the monotone
        convergence theorem:
        \begin{equation}
            \underset{n\rightarrow\infty}{\lim}
            \int_{\Omega}(f_{n}-F)\diff{\mu}=
            \int_{\Omega}\underset{n\rightarrow\infty}{\lim}
            (f_{n}-F)\diff{\mu}
        \end{equation}
        But $F$ is summable, and thus we may cancel this from both
        sides. Therefore, etc.
    \end{proof}
    Without the requirement that there is a summable \textit{floor}
    for the sequence of functions $f_{n}$, the theorem may not
    hold. For consider the sequence defined by:
    \begin{equation}
        f_{n}(\omega)=\frac{\minus{1}}{n\omega}
    \end{equation}
    Then $f_{n}\rightarrow{0}$ on $(0,1)$, but the integral of
    $f_{n}$ is infinite for all $n$.
    There is an equivalent theorem with a summable majorant, rather
    than a summable minorant. Here we'd have a sequence of
    monotonically decreasing functions with a summable \textit{roof}.
    \begin{ltheorem}{Fatou's First Theorem for Signed Functions}
        If $f_{n}$ is a sequence of measurable functions such that
        there is a summable function $F$ such that $f_{n}\geq{F}$,
        then:
        \begin{equation}
            \int_{\Omega}
            \underset{n\rightarrow\infty}{\underline{\lim}}
            f_{n}\diff{\mu}
            \leq\underset{n\rightarrow\infty}{\underline{\lim}}
            \int_{\Omega}f_{n}\diff{\mu}
        \end{equation}
    \end{ltheorem}
    \begin{ltheorem}{Fatou's Second Theorem for Signed Functions}
        If $f_{n}$ is a sequence of measurable functions such that
        there is a summable function $F$ such that
        $f_{n}\leq{F}$, then:
        \begin{equation}
            \underset{n\rightarrow\infty}{\overline{\lim}}
            \int_{\Omega}f_{n}\diff{\mu}
            \leq\int_{\Omega}
            \underset{n\rightarrow\infty}{\overline{\lim}}
            f_{n}\diff{\mu}
        \end{equation}
        Where $\overline{\lim}$ denotes the limit superior.
    \end{ltheorem}
    \begin{ltheorem}{Dominated Convergence Theorem}
        If $f_{n}$ is a sequence of functions such that there is a
        summable minorant $F_{1}$ and a summable majorant
        $F_{2}$, that is $F_{1}\leq{f}_{n}\leq{F}_{2}$, and if
        $f_{n}\rightarrow{f}$, then:
        \begin{equation}
            \underset{n\rightarrow\infty}{\lim}
            \int_{\Omega}f_{n}\diff{\mu}
            =\int_{\Omega}
            \underset{n\rightarrow\infty}{\lim}
            f_{n}\diff{\mu}
        \end{equation}
        That is, the limit of the integrals exists.
    \end{ltheorem}
    \begin{proof}
        For if $f_{n}$ has a summable majorant and a summable minorant,
        then both of Fatou's theorem's apply. That is:
        \begin{align}
            \int_{\Omega}
            \underset{n\rightarrow\infty}{\underline{\lim}}
            f_{n}\diff{\mu}
            &\leq\underset{n\rightarrow\infty}{\underline{\lim}}
            \int_{\Omega}f_{n}\diff{\mu}\\
            \underset{n\rightarrow\infty}{\overline{\lim}}
            \int_{\Omega}f_{n}\diff{\mu}
            &\leq\int_{\Omega}
            \underset{n\rightarrow\infty}{\overline{\lim}}
            f_{n}\diff{\mu}
        \end{align}
        But the limit of $f_{n}$ exists, so we have:
        \begin{equation}
            \int_{\Omega}\underset{n\rightarrow\infty}{\lim}
            f_{n}\diff{\mu}
            \leq\underset{n\rightarrow\infty}{\underline{\lim}}
            \int_{\Omega}f_{n}\diff{\mu}
            \leq\underset{n\rightarrow\infty}{\overline{\lim}}
            \int_{\Omega}f_{n}\diff{\mu}
            \leq\int_{\Omega}\underset{n\rightarrow\infty}{\lim}
            f_{n}\diff{\mu}
        \end{equation}
        Therefore:
        \begin{equation}
            \underset{n\rightarrow\infty}{\underline{\lim}}
            \int_{\Omega}f_{n}\diff{\mu}
            =\underset{n\rightarrow\infty}{\overline{\lim}}
            \int_{\Omega}f_{n}\diff{\mu}
        \end{equation}
        Therefore the limit exists, and by the inequalities:
        \begin{equation}
            \underset{n\rightarrow\infty}{\lim}
            \int_{\Omega}f_{n}\diff{\mu}
            =\int_{\Omega}
            \underset{n\rightarrow\infty}{\lim}
            f_{n}\diff{\mu}
        \end{equation}
    \end{proof}
    We can relax the requirements of the monotone convergence theorems,
    Fatou's theorems, and the dominated convergence theorem to be
    true on all but a set of measure zero, and the results are still
    valid.
    \begin{ltheorem}{Generalized Monotone Converence Theorem}
        If $f_{n}$ is a sequence of measurable functions such that
        $f_{n}(\omega)\leq{f}_{n+1}(\omega)$ $\mu$ almost everywhere,
        and if $F$ is a summable function such that
        $F\leq{F}_{n}(\omega)$ $\mu$ almost everywhere, then:
        \begin{equation}
            \underset{n\rightarrow\infty}{\lim}
            \int_{\Omega}f_{n}\diff{\mu}
            =\int_{\Omega}f\diff{\mu}
        \end{equation}
    \end{ltheorem}
    \begin{theorem}
        For define $E_{n}$ be:
        \begin{equation}
            E_{n}=\{\omega:f_{n}(\omega)\not\leq{f}_{n+1}(\omega)\}
        \end{equation}
        And let $E$ be defined by:
        \begin{equation}
            E=\Big(\bigcup_{n=1}^{\infty}E_{n}\Big)^{C}
        \end{equation}
        Then, as the countable union of sets of measure zero has
        measure zero, $\mu(E^{C})=0$. 
    \end{theorem}