\section{Measures}
    \subsection{A Review Infinite Series}
        Given a sequence of real numbers,
        $a:\mathbb{N}\rightarrow\mathbb{R}$, the sum of this
        sequence is defined as the limit of
        finite partial sums. That is:
        \begin{equation}
            \sum_{n=1}^{\infty}a_{n}=
            \underset{N\rightarrow\infty}{\lim}
                \sum_{n=1}^{N}a_{n}
        \end{equation}
        In general, this limit may not in general exists. If it
        does, we say the series converges. If the limit does
        not exists, we do not define the sum and instead just
        have a meaningless combination of symbols. If the
        sequence is positive, then the sequence of partial sums
        will be increasing. If this sequence is bounded, then
        the limit exists. This comes from the fact that bounded
        monotonic sequences converge, a result that stems from
        the least upper bound property of $\mathbb{R}$.
        Moreover, if $a:\mathbb{N}\rightarrow\mathbb{R}$ is a
        sequence of positive real numbers, and if
        $f:\mathbb{N}\rightarrow\mathbb{N}$ is any bijective
        function, then the following is true:
        \begin{equation}
            \sum_{n=1}^{\infty}a_{n}
            =\sum_{n=1}^{\infty}a_{f(n)}
        \end{equation}
        We can also split the sequence into a grid,
        and take the
        double sum, obtaining the same result. If
        $A_{1},A_{2},\hdots$ are disjoint sets whose union is
        $\mathbb{N}$, and if $b_{nm}$ is the $n^{th}$ element
        of $A_{m}$, then:
        \begin{equation}
            \sum_{i=1}^{\infty}a_{i}=
            \sum_{n=1}^{\infty}\sum_{m=1}^{\infty}b_{nm}
        \end{equation}
        We should be precise in what we mean. The double
        sum is the \textit{limit of a limit}.
        \begin{equation}
            \sum_{n=1}^{\infty}\sum_{m=1}^{\infty}a_{nm}
            =\underset{N\rightarrow\infty}{\lim}\sum_{n=1}^{N}
            \Big(\underset{M\rightarrow\infty}{\lim}
            \sum_{m=1}^{M}a_{nm}\Big)
        \end{equation}
        We use infinite series to define \textit{measures} on
        $\sigma$-algebra.
    \subsection{Measure Functions}
        A set function on a collection of sets $\mathcal{E}$
        is a function $\mu:\mathcal{E}\rightarrow\mathbb{R}$.
        For example, if we consider the set of all
        semi-intervals of the form $[a,b)$, where
        $a,b\in\mathbb{R}$ and $a\leq{b}$, then we can define
        $\mu([a,b))=b-a$. This gives rise to the notion of
        a measure function.
        A measure function on a collection of set
        $\mathcal{E}$ is a function
        $\mu:\mathcal{E}\rightarrow\mathbb{R}$ such that:
        \begin{enumerate}
            \item If $\emptyset\in\mathcal{E}$, then
                  $\mu(\emptyset)=0$
            \item For all $A\in\mathcal{E}$, $\mu(A)\geq{0}$
            \item For any countable collection of pair-wise
                  disjoint sets whose
                  union also lies in $\mathcal{E}$,
                  $\mu(\cup_{n=1}^{\infty}A_{n})=%
                   \sum_{n=1}^{\infty}\mu(A_{n})$
        \end{enumerate}
        It helps if we don't have to consider the case where
        $\mu(\emptyset)$ is undefined, or where we don't have
        closure under countable unions, so we discuss measure
        functions on $\sigma$-algebras.
        \begin{definition}
            A measure on a $\sigma$-algebra
            $\mathcal{A}$ is a function
            $\mu:\mathcal{A}\rightarrow\mathbb{R}$ such that:
            \begin{enumerate}
                \item $\mu(\emptyset)=0$
                \item For all $A\in\mathcal{A}$,
                      $\mu(A)\geq{0}$
                \item For any countable collection of pairwise
                      disjoint elements of $\mathcal{A}$,
                      $\mu(\cup_{n=1}^{\infty}A_{n})=%
                       \sum_{n=1}^{\infty}\mu(A_{n})$
            \end{enumerate}
        \end{definition}
        \begin{example}
            Let $\Omega$ be a set, and let
            $\mathcal{A}=\mathcal{P}(\Omega)$. Then
            $\mathcal{A}$ is a $\sigma$-algebra on $\Omega$.
            If $\omega_{1},\hdots,\omega_{n}\in\Omega$ and if
            $p_{1},\hdots,p_{n}\in\mathbb{R}^{+}$, then:
            \begin{equation}
                \mu(A)=\sum_{k=1}^{n}p_{k}\chi_{A}(\omega_{k})
            \end{equation}
            Where $\xi_{A}$ is the indicator function:
            \begin{equation}
                \chi_{A}(\omega)=
                \begin{cases}
                    0,&\omega\notin{A}\\
                    1,&\omega\in{A}
                \end{cases}
            \end{equation}
            This is an example of a \textit{point measure}
            on $\mathcal{A}$. It defines a measure function.
        \end{example}
    A $\sigma\text{-Algebra}$ on a set $\Omega$ is a subset
    $\mathcal{A}$ of $\mathcal{P}(\Omega)$ such that
    $\Omega\in\mathcal{A}$ and for any countable collection of
    elements $A_{i}\in\mathcal{A}$, the union
    $\bigcup_{i=1}^{\infty}A_{i}$ is also contained in
    $\mathcal{A}$. $\mathcal{A}$ does not have to consist of
    countably many elements. The sequence of subset $A_{i}$ does
    not have to exhaust the entirety of $\mathcal{A}$, much the
    way that any sequence of real numbers will not exhaust the
    entire of $\mathbb{R}$. Going in the other direction,
    $\sigma\text{-Algebras}$ can be finite. If $\Omega$ is a
    set, and if $A\subset\Omega$ is non-empty, then
    $\mathcal{A}=\{\emptyset,A,A^{C},\Omega\}$ defines a
    $\sigma\text{-algebra}$ on $\Omega$. A measure on a
    $\sigma\text{-Algebra}$ $\mathcal{A}$ is a function
    $\mu:\mathcal{A}\rightarrow\mathbb{R}$ such that, for all
    $A\in\mathcal{A}$, $\mu(A)\geq{0}$, $\mu(\emptyset)=0$, and
    given a mutually disjoint countable collection of elements of
    $\mathcal{A}$, the following holds:
    \begin{equation}
        \mu\Big(\bigcup_{i=1}^{\infty}A_{i}\Big)
        =\sum_{n=1}^{\infty}\mu(A_{i})
    \end{equation}
    \begin{example}
        A pure point measure is a measure that assigns to a
        collection of elements $\omega_{j}\in\Omega$ a positive
        real number $p_{j}$, and then the measure of any set
        $A$ is:
        \begin{equation}
            \mu(A)=\sum_{j:\omega_{j}\in{A}}p_{j}
        \end{equation}
    \end{example}
    \subsection{Properties of Measure}
        \subsubsection{Monotonicity}
            If $A$ and $B$ are elements of a $\sigma\text{-Algebra}$
            $\mathcal{A}$, if $\mu$ is a measure on
            $\mathcal{A}$, and if $A\subseteq{B}$, then
            $\mu(A)\leq\mu(B)$. This is the monotonic property
            of measures.
            \begin{theorem}
                If $\Omega$ is a set, $\mathcal{A}$ is a
                $\sigma\text{-Algebra}$ on $\Omega$, if
                $\mu$ is a measure on $\mathcal{A}$, and if
                $A,B$ are elements of $\mathcal{A}$ such that
                $A\subseteq{B}$, then $\mu(A)\leq\mu(B)$.
            \end{theorem}
            \begin{proof}
                For as $\mathcal{A}$ is a $\sigma\text{-Algebra}$
                on $\Omega$, and as $A,B\in\mathcal{A}$,
                $B\setminus{A}\in\mathcal{A}$. But, as
                $A\subseteq{B}$, $B=(B\setminus{A})\cup{A}$.
                But then, as measures are additive and positive:
                \begin{align}
                    \mu(B)&=\mu\big((B\setminus{A})\cup{A}\big)\\
                    &=\mu(B\setminus{A})+\mu(A)\\
                    &\geq\mu(A)
                \end{align}
            \end{proof}
            \begin{theorem}
                If $\Omega$ is a set, $\mathcal{A}$ is a
                $\sigma\text{-Algebra}$ on $\Omega$, if
                $\mu$ is a measure on $\mathcal{A}$, and if
                $A,B$ are elements of $\mathcal{A}$ such that
                $A\subseteq{B}$ and $\mu(A),\mu(B)<\infty$,
                then $\mu\big(B\setminus{A}\big)=\mu(B)-\mu(A)$.
            \end{theorem}
        \subsubsection{Continuity Theorems}
            \begin{theorem}[Continuity from Below]
                If $\Omega$ is a set, $\mathcal{A}$ is a
                $\sigma\text{-Algebra}$ on $\Omega$, if
                $\mu$ is a measure on $\mathcal{A}$, and if
                $A_{i}$ is a sequence of elements in $\mathcal{A}$
                such that, for all $i\in\mathbb{N}$,
                $A_{i}\subseteq{A}_{i+1}$, then:
                \begin{equation}
                    \mu\Big(\bigcup_{i=1}^{\infty}A_{i}\Big)
                    =\underset{N\rightarrow\infty}{\lim}
                    \mu(A_{N})
                \end{equation}
            \end{theorem}
            \begin{proof}
                For let $A=\cup_{n=1}^{\infty}A_{n}$ and let
                $B_{n}=A_{n+1}\setminus{A}_{n}$. Then, as
                $A_{n}\subseteq{A}_{n+1}$, for all
                $i,j\in\mathbb{N}$, $B_{i}\cap{B}_{j}=\emptyset$.
                But $A=A_{1}\cup\Big(\cup_{n=1}^{\infty}B_{n}\Big)$
                and this is the countable union of mutually
                disjoint sets, and therefore, using the
                telescoping series:
                \begin{align}
                    \mu(A)&=\mu(A_{1})+
                    \sum_{n=1}^{\infty}\mu(B_{n})\\
                    &=\mu(A_{1})+\sum_{n=1}^{\infty}
                    \Big(\mu(A_{n+1})-\mu(A_{n})\Big)\\
                    &=\mu(A_{1})+
                    \underset{N\rightarrow\infty}{\lim}
                    \Big(\mu(A_{N})-\mu(A_{1})\Big)\\
                    &=\underset{N\rightarrow\infty}{\lim}
                    \mu(A_{N})
                \end{align}
            \end{proof}
            \begin{theorem}[Continuity from Above]
                If $\Omega$ is a set, $\mathcal{A}$ is a
                $\sigma\text{-Algebra}$ on $\Omega$, if
                $\mu$ is a measure on $\mathcal{A}$, and if
                $A_{i}$ is a sequence of elements in $\mathcal{A}$
                such that, for all $i\in\mathbb{N}$,
                $A_{i+1}\subseteq{A}_{i}$ and there exists an
                $n\in\mathbb{N}$ such that $\mu(A_{n})$ is finite,
                then:
                \begin{equation}
                    \mu\Big(\bigcap_{n=1}^{\infty}A_{n}\Big)
                    =\underset{N\rightarrow\infty}{\lim}
                    \mu(A_{N})
                \end{equation}
            \end{theorem}
            \begin{proof}
                For let $A=\cap_{n=1}^{\infty}A_{n}$ and let
                $B_{n}=A_{n}\setminus{A}_{n+1}$. Then:
                \begin{equation}
                    A_{1}=
                    A\cup\big(\bigcup_{n=1}^{\infty}B_{n}\Big)
                \end{equation}
                And this is the union of countably many disjoint
                sets. Therefore:
                \begin{align}
                    \mu(A_{1})&=
                    \mu(A)+\sum_{n=1}^{\infty}\mu(B_{n})\\
                    &=\mu(A)+\sum_{n=1}^{\infty}
                    \Big(\mu(A_{n}-\mu(A_{n+1})\Big)\\
                    &=\mu(A)+\mu(A_{1})-
                    \underset{N\rightarrow\infty}{\lim}\mu(A_{N})
                \end{align}
                Subtracting by $\mu(A_{1})$ obtains the result.
            \end{proof}
            If $\mu(A_{i})=\infty$ for all $i\in\mathbb{N}$, then
            the above theorem may not be true. For consider
            the collection of sets $A_{n}=[n,\infty)$. The
            measure of each $A_{n}$ is infinite, but the
            intersection of the entire collection is empty.
            Thus the measure of the intersection is zero.
            \begin{theorem}[Countable Sub-Additivity]
                If $\Omega$ is a set, $\mathcal{A}$ a
                $\sigma\text{-Algebra}$ on $\mathcal{A}$, and
                if $A_{i}$ is a countable collection of elements
                of $\mathcal{A}$, then:
                \begin{equation}
                    \mu\Big(\bigcup_{n=1}^{\infty}A_{n}\Big)
                    \leq\sum_{n=1}^{\infty}\mu(A_{n})
                \end{equation}
            \end{theorem}
            \begin{proof}
                For if $A_{1},A_{2}\in\mathcal{A}$, then:
                \begin{equation}
                    \mu(A_{1}\cup{A}_{2})=
                    \mu(A_{1}\setminus{A}_{2})+
                    \mu(A_{2}\setminus{A}_{1})+
                    \mu(A_{1}\cap{A}_{2})
                \end{equation}
                But also:
                \begin{align}
                    \mu(A_{1})&=
                    \mu(A_{1}\setminus{A}_{2})+
                    \mu(A_{1}\cap{A}_{2})\\
                    \mu(A_{2})&=
                    \mu(A_{2}\setminus{A}_{1})+
                    \mu(A_{1}\cap{A}_{2})
                \end{align}
                And therefore:
                \begin{equation}
                    \mu(A_{1})+\mu(A_{2})=
                    \mu(A_{1}\cup{A}_{2})+\mu(A_{1}\cap{A}_{2})
                \end{equation}
                We now prove by induction. Suppose this is true
                of a collection of $N$ elements. Given a collection
                $A_{i}$ of $N+1$ elements, let
                $B=\cup_{n=1}^{N}A_{i}$. But then:
                \begin{align}
                    \mu(A_{N+1}\cup{B})&
                    \leq\mu(A_{N+1})+\mu(B)\\
                    &\leq\mu(A_{N+1})+\sum_{n=1}^{N}A_{n}\\
                    &=\sum_{n=1}^{N+1}\mu(A_{n})
                \end{align}
            \end{proof}