\section{Equicontinuity}
    \begin{ltheorem}{Arzela-Ascoli Theorem}
          {Funct_Analysis_Arzela_Ascoli}
        If $X$ is a compact metric space, if
        $F_{n}\in{C}(X)$ is a sequence of equicontinuous
        point-wise bounded functions, then $F_{n}$ has a
        uniformly convergent subsequence.
    \end{ltheorem}
    \begin{theorem}
        If $X$ is a compact metric space and
        $\mathscr{F}\subseteq{C}(X)$ is a closed subset with
        respect to the uniform norm, and if
        $\mathscr{F}$ is equicontinuous on $X$ and point-wise
        bounded, then $\mathscr{F}$ is compact.
    \end{theorem}
    \begin{proof}
        It suffices to show that $\mathscr{F}$ is sequentially
        compact. Let $F_{n}$ be s sequence in $\mathscr{F}$.
        The by the Arzela-Ascoli theorem, there is a uniformly
        convergent subsequence $F_{k_{n}}$. But $\mathscr{F}$
        is closed, and thus the limit function is contained
        in $\mathscr{F}$. Thus, $\mathscr{F}$ is sequentially
        compact. But sequentially compact metric spaces are
        compact. Therefore, etc.
    \end{proof}
    \begin{theorem}
        If $X$ is a compact metric space and
        $\mathscr{F}\subseteq{C}(X)$ is a closed subset with
        respect to the uniform norm, and if
        $\mathscr{F}$ is equicontinuous on $X$ and point-wise
        bounded, then $\mathscr{F}$ is uniformly bounded.
    \end{theorem}
    \begin{proof}
        For $\mathscr{F}$ is compact by the previous theorem.
        But then $\mathscr{F}$ is bounded with respect to
        $\norm{\cdot}_{\infty}$. Therefore, $\mathscr{F}$ is
        uniformly bounded.
    \end{proof}
    \begin{theorem}
        If $X$ is a compact metric space and if
        $\mathscr{F}\subseteq{C}(X)$ is closed, equicontinuous,
        and uniformly bounded on $X$, then $\mathscr{F}$ is
        compact.
    \end{theorem}
    \begin{proof}
        For suppose $\mathscr{F}$ is compact. Then $\mathscr{F}$
        is closed and uniformly bounded. Thus it suffices to
        show that $\mathscr{F}$ is equicontinuous. Suppose not.
        Then there is a point $x\in{X}$ such that
        $\mathscr{F}$ is not equicontinuous at $x$. Thus,
        there exists an $\varepsilon>0$ such that, for all
        $\delta>0$, there are points $x,y$ such that
        $d(x,y)<\delta$, but $|f(x)-f(y)|\geq\varepsilon$
        for some $f\in\mathscr{F}$. Thus, for all
        $n\in\mathbb{N}$, there is an $x_{n}\in{X}$ such that
        $d(x,x_{n})<1/n$, and
        $|f_{n}(x)-f_{n}(x_{n})|\geq\varepsilon_{0}$. But if
        $\mathscr{F}$ is compact, then $f_{n}$ has a convergent
        subsequence $f_{k_{n}}$. Let $f$ be the limit.
        Since $\mathscr{F}$ is compact, $f\in\mathscr{F}$.
        But then $f_{k_{n}}(x_{k_{n}})\rightarrow{f}(x)$. But
        then there is an $N\in\mathbb{N}$ such that,
        for $k_{n}>N$,
        $\norm{f_{k_{n}}-f}_{\infty}<\varepsilon_{0}/3$.
        But then:
        \begin{align}
            |f(x_{k_{n}})-f(x)|&=
            |f(x_{k_{n}})-f_{k_{n}}(x_{k_{n}})
            +f_{k_{n}}(x_{k_{n}})-f_{k_{n}}(x)
            +f_{k_{n}}(x)-f(x)|\\
            &\geq|f_{k_{n}}(x_{k_{n}})-f_{k_{n}}(x)|+
            |f(x_{k_{n}})-f_{k_{n}}(x_{k_{n}})
            +f_{k_{n}}(x)-f(x)|\\
            &>\varepsilon
        \end{align}
        A contradiction.
    \end{proof}
    \begin{ldefinition}{Baire Space}{Baire_Space}
        A Baire space is a metric space $(X,d)$ such that, for
        countable collection of open and dense sets, the
        intersection is also dense.
    \end{ldefinition}
    This is a topological property, and so Baire spaces can
    be defined for a more general topological space. The
    interior of a set in a topological space is:
    \begin{equation}
        \Int(A)=\bigcup\{\mathcal{U}\in\tau:\mathcal{U}\subseteq{A}\}
    \end{equation}
    \begin{theorem}
    A metric space $(X,d)$ is a Baire space if and only
    if given a countable collection $F_{n}$ of closed
    sets such that the union over all of $F_{n}$ has
    non-empty interior, then at least one of the $F_{n}$
    has non-empty interior.
    \end{theorem}
    \begin{theorem}
        There exist countable Baire spaces.
    \end{theorem}
    Suppose $\mathcal{U}\subseteq{X}$ is open at
    $x_{0}\in\mathcal{U}$. There there is a $\delta>0$ such
    $B_{\delta}(x)\subseteq\mathcal{U}$. Then:
    \begin{equation}
        \overline{B_{\delta/2}(x)}\subseteq{B}_{\delta}(x)
    \end{equation}
    Thus, $\overline{B_{\delta/2}(x)}\subseteq\mathcal{U}$
    and the diameter is less than $2\delta$.
    \begin{ltheorem}{Baire Category Theorem}{Baire_Category_Theorem}
        Every complete metric space is a Baire space.
    \end{ltheorem}
    \begin{proof}
        Suppose $\mathcal{O}_{n}\subseteq{X}$ is open and
        dense for all $n\in\mathbb{N}$. Let $x_{0}\in{X}$ and
        $r_{0}>0$. It will suffice to show that:
        \begin{equation}
            B_{r_{0}}(x_{0})\cap\bigcap_{n\in\mathbb{N}}
                \mathcal{O}_{n}\ne\emptyset
        \end{equation}
        Inductively, we create a sequence of points $x_{k}$
        and real numbers $r_{k}>0$ such that $r_{k}$ is strictly
        monotonically decreasing, and thus that:
        \begin{equation}
            \overline{B_{r_{k+1}}(x_{k+1})}
            \subseteq{B}_{r_{k}}(x_{k})\cap\mathcal{O}_{k+1}
        \end{equation}
    \end{proof}
    Consider the set of all lines through the origin with
    rational slope. The complete of any given line is the
    union of two open half planes, which are open and dense
    subsets of $\mathbb{R}^{2}$. Since we have only a countable
    collection of such lines, the intersection of the complement
    is dense in $\mathbb{R}^{2}$. Baire's Category Theorem
    holds even if $(X,d)$ is not complete, but is equivalent
    to a complete metric. For example, let $X=(0,1)$ and let
    $d(x,y)=|x-y|$ be the standard metric. This is not a
    complete space, but is homeomorphic to $\mathbb{R}$,
    which is a complete metric space. Using this homeomorphism,
    we can find a metric $\tilde{d}$ on $(0,1)$ that is complete
    and which is equivalent to the original metric. Thus,
    $(0,1)$ is a Baire space.
    \begin{theorem}
        If $V$ is a non-empty open subset of a complete metric
        space $(X,d)$, then there is a metric $\tilde{d}$ such
        that $(V,\tilde{d})$ is complete.
    \end{theorem}
    Hence, $V$ is a Baire space. Then, given a set
    $F_{n}$ of closed subsets of $X$ such that:
    \begin{equation}
        V=\bigcup_{n=1}^{\infty}(V\cap{F}_{n})
    \end{equation}
    Then some $F_{n}\cap{V}$ has non-empty interior in $V$,
    and hence in $X$.
    \begin{theorem}
        If $X$ is a Baire space and if $f_{n}$ is a sequence
        of continuous function in $C(X)$ which converges
        point-wise to $f:X\rightarrow\mathbb{C}$, then
        the set:
        \begin{equation}
            \{x\in{X}:f\textrm{ is continuous at }x\}
        \end{equation}
        Is dense in $X$.
    \end{theorem}
    \begin{proof}
        Let $\varepsilon>0$ and define:
        \begin{align}
            A_{N}(\varepsilon)&=
            \{x:|f_{n}(x)-f_{m}(x)|\leq\varepsilon,
                n,m\in\mathbb{N}\}\\
            &=\bigcap_{n,m\geq{N}}
                \{x:|f_{n}(x)-f_{m}(x)|\leq\varepsilon\}
        \end{align}
        Then $A_{N}(\varepsilon)$ is closed. But also:
        \begin{equation}
            X=\bigcup_{N=1}^{\infty}A_{N}(\varepsilon)
        \end{equation}
        Thus, by the Baire category theorem, we have:
        \begin{equation}
            \mathcal{U}(\varepsilon)=
            \bigcup_{N=1}^{\infty}\Int(A_{N}(\varepsilon))
        \end{equation}
        Is non-empty and open. Moreover,
        $\mathcal{U}(\varepsilon)$ is dense. But then:
        \begin{equation}
            \mathcal{C}=\bigcap_{n=1}^{\infty}
            \mathcal{U}(\frac{1}{n})
        \end{equation}
        Is dense in $X$, and $f$ is continuous at all
        $x\in\mathcal{C}$.
    \end{proof}
    The Baire Category Theorem says that every complete metric
    space is a Baire space. The notion of Baire space is a
    topological property, and not a metric property. Thus, even
    if $(X,d)$ is not complete but is equivalent to a complete
    metric space $(X,\tilde{d})$, then $(X,d)$ is a Baire space.
    A topological space is called completely metrizable if there
    is a metric on the space that is complete and generates the
    topology. Given a complete metric space $(X,d)$, every
    non-empty open set $\mathcal{V}$ has a metric
    $d_{\mathcal{V}}$ such that $(\mathcal{V},d_{\mathcal{V}})$
    is complete, and is therefore a Baire space. Thus, if:
    \begin{equation}
        \mathcal{V}=\bigcup_{n\in\mathbb{N}}\Big(
            \mathcal{V}\cap{F}_{n}\Big)
    \end{equation}
    Where $F_{n}$ is closed for all $n\in\mathbb{N}$, then
    for some $N\in\mathbb{N}$, $\mathcal{V}\cap{F}_{N}$ has
    interior.
    \begin{theorem}
    If $X$ is a Baire space, and if
    $F_{n}$ is a sequence of continuous functions that
    converges point-wise to $f:X\rightarrow\mathbb{C}$, then
    the set $\mathcal{D}$ defined by:
    \begin{equation}
        \mathcal{D}=
            \{x\in{X}:\textrm{$f$ is continuous as $x$}\}
    \end{equation}
    Then $\mathcal{D}$ is dense in $X$.
\end{theorem}
    \begin{proof}
    For let $\varepsilon>0$, and let:
    \begin{equation}
        A_{N}(\varepsilon)=
        \{x\in{X}:|f_{n}(x)-f_{m}(x)|\leq\varepsilon,n,m>N\}
    \end{equation}
    Then, for all $N\in\mathbb{N}$, $A_{N}(\varepsilon)$ is
    closed. Let $\mathcal{U}(\varepsilon)$ be defined by:
    \begin{equation}
        \mathcal{U}=\bigcup_{n\in\mathbb{N}}
            \Int\Big(A_{N}(\varepsilon)\Big)
    \end{equation}
    Then $\mathcal{U}(\varepsilon)$ is open and dense. It
    is open for it is the union of open sets. For let
    $\mathcal{V}$ be a non-empty subset. Then:
    \begin{equation}
        \mathcal{V}=\bigcup_{n\in\mathbb{N}}
            \Big(A_{n}(\varepsilon)\cap\mathcal{V}\Big)
    \end{equation}
    Hence there exists an $N\in\mathbb{N}$ such that:
    \begin{equation}
        A_{N}(\varepsilon)\cap\mathcal{V}\ne\emptyset
    \end{equation}
    And this has interior, and therefore:
    \begin{equation}
        \Int(A_{N}(\varepsilon))\cap\mathcal{V}\ne\emptyset
    \end{equation}
    Therefore,
    $\mathcal{V}\cap\mathcal{U}(\varepsilon)\ne\emptyset$.
    Now, define:
    \begin{equation}
        \mathcal{V}=\bigcap_{n\in\mathbb{N}}
            \mathcal{U}\big(\frac{1}{n}\big)
    \end{equation}
    And therefore $\mathcal{C}$ is dense in $X$. We
    now want to show that $f$ is continuous for all
    $x\in\mathcal{C}$. For let $x_{0}\in\mathcal{C}$ and
    let $\varepsilon>0$. Let $k\in\mathbb{N}$ be such that
    $k^{\minus{1}}<\varepsilon$. Then
    $x_{0}\in\mathcal{U}(k^{\minus{1}})$ and thus there is
    an $N\in\mathbb{N}$ such that:
    \begin{equation}
        x_{0}\in\Int\big(A_{N}(k^{\minus{1}})\big)
    \end{equation}
    But $f_{N}$ is continuous, and thus there is a
    neighborhood $\omega$ of $x_{0}$ such that, for all
    $y\in\omega$:
    \begin{equation}
        |f_{N}(x_{0})-f_{N}(y)|<\varepsilon/3
    \end{equation}
    Shrink $\omega$ so that it resides inside of
    $\Int(A_{N}(k^{\minus{1}})$. Then:
    \begin{equation}
        |f_{n}(y)-f_{N}(y)|<k^{\minus{1}}
        \quad\quad
        n\geq{N}
    \end{equation}
    But then, use the Cauchy trick and you're down.
\end{proof}