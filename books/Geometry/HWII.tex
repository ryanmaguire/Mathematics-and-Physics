%------------------------------------------------------------------------------%
\documentclass{article}                                                        %
%------------------------------Preamble----------------------------------------%
\makeatletter                                                                  %
    \def\input@path{{../../}}                                                  %
\makeatother                                                                   %
%---------------------------Packages----------------------------%
\usepackage{geometry}
\geometry{b5paper, margin=1.0in}
\usepackage[T1]{fontenc}
\usepackage{graphicx, float}            % Graphics/Images.
\usepackage{natbib}                     % For bibliographies.
\bibliographystyle{agsm}                % Bibliography style.
\usepackage[french, english]{babel}     % Language typesetting.
\usepackage[dvipsnames]{xcolor}         % Color names.
\usepackage{listings, lstlinebgrd}      % Verbatim-Like Tools.
\usepackage{mathtools, esint, mathrsfs} % amsmath and integrals.
\usepackage{amsthm, amsfonts}           % Fonts and theorems.
\usepackage{tabularx}
\usepackage{tcolorbox}                  % Frames around theorems.
\usepackage{upgreek}                    % Non-Italic Greek.
\usepackage{paracol}                    % Two-column styling.
\usepackage{wrapfig}                    % Wrap text around figure.
\usepackage{fmtcount, etoolbox}         % For the \book{} command.
\usepackage[newparttoc]{titlesec}       % Formatting chapter, etc.
\usepackage{titletoc}                   % Allows \book in toc.
\usepackage[nottoc]{tocbibind}          % Bibliography in toc.
\usepackage[titles]{tocloft}            % ToC formatting.
\usepackage{multicol, enumitem}         % Multi-column/enumerate.
\usepackage{import}                     % Import external files.
\usepackage{pgfplots, tikz}             % Drawing/graphing tools.
\usetikzlibrary{
    calc,                   % Calculating right angles and more.
    angles,                 % Drawing angles within triangles.
    arrows.meta,            % Latex and Stealth arrows.
    quotes,                 % Adding labels to angles.
    positioning,            % Relative positioning of nodes.
    decorations.markings,   % Adding arrows in the middle of a line.
    patterns,
    arrows,
    shapes,
    shapes.geometric,
    cd,
    hobby,
    babel
}                                       % Libraries for tikz.
\pgfplotsset{compat=1.9}                % Version of pgfplots.
\usepackage[font=scriptsize,
            labelformat=simple,
            labelsep=colon]{subcaption} % Subfigure captions.
\usepackage[font={scriptsize},
            hypcap=true,
            labelsep=colon]{caption}    % Figure captions.
\usepackage{hyperref}                   % Allows for hyperlinks.
\hypersetup{
    colorlinks=true,
    linkcolor=blue,
    filecolor=magenta,
    urlcolor=Cerulean,
    citecolor=SkyBlue
}                           % Colors for hyperref.
\usepackage[toc,acronym,nogroupskip]{glossaries} % Glossaries and acronyms.
\usepackage[subpreambles=false]{standalone}      % Complileable sub files.

% Various font stuff from kiwi.
% Use this for Times text and Computer Modern math
%\usepackage{times}

% Quite nice
%\usepackage[charter, greekfamily=, greekuppercase=italicized]{mathdesign}
%\usepackage[utopia, greekuppercase=italicized]{mathdesign}    % Math is narrower

% Use this for Times text and math
%\usepackage{newtxtext}
%\usepackage[libertine,cmintegrals]{newtxmath}
%\usepackage{fix-cm}

%\usepackage{txfontsb}
% or
%\usepackage{mathptmx}

%\usepackage[scaled=0.92]{helvet}
%\renewcommand{\rmdefault}{ptm}

%\usepackage{mathpazo}    % add possibly `sc` and `osf` options
%\usepackage{eulervm}

%\usepackage{fourier}
%\renewcommand{\rmdefault}{ptm}
%\usepackage{mathptm}

%\usepackage{fontspec}
%\setmainfont{lmodern}

%\usepackage[varg]{txfonts}
%\usepackage{fouriernc}
%\usepackage{mathpazo}

%\usepackage{bookman}
%\usepackage[scaled]{uarial}
%\usepackage[scaled]{helvet}
%\renewcommand*\familydefault{\sfdefault}
%\usepackage[math]{anttor}

%\newcommand\fgeorgia{\fontfamily{jvn}\selectfont}
%\newcommand\ftimes{\fontfamily{ptm}\selectfont}
%\newcommand\fhelvetica{\fontfamily{phv}\selectfont}
%\newcommand\fcourier{\fontfamily{pcr}\selectfont}
%\newcommand\fbookman{\fontfamily{pbk}\selectfont}
%\newcommand\fnewcentury{\fontfamily{pnc}\selectfont}
%\newcommand\fpalatino{\fontfamily{ppl}\selectfont}
%\newcommand\favantgarde{\fontfamily{pag}\selectfont}
%\newcommand\fnormal{\normalfont}
%\newcommand\fsize[1]{\ifnum#1>0\fontsize{#1}{#1}\selectfont\else\normalsize\fi}
%------------------------Theorem Styles-------------------------%
% Define theorem style for default spacing and normal font.
\newtheoremstyle{normal}
    {\topsep}               % Amount of space above the theorem.
    {\topsep}               % Amount of space below the theorem.
    {}                      % Font used for body of theorem.
    {}                      % Measure of space to indent.
    {\bfseries}             % Font of the header of the theorem.
    {}                      % Punctuation between head and body.
    {.5em}                  % Space after theorem head.
    {}

% Define theorem style for default spacing with italicized font.
\newtheoremstyle{normalit}{\topsep}{\topsep}
                {\itshape}{}{\bfseries}{}{.5em}{}

% Italic header environment.
\newtheoremstyle{thmit}{\topsep}{\topsep}{}{}{\itshape}{}{0.5em}{}

% Define italicized environments.
\theoremstyle{normalit}
\newtheorem{theorem}{Theorem}[section]
\newtheorem{lemma}{Lemma}[section]
\newtheorem{corollary}{Corollary}[section]
\newtheorem{proposition}{Proposition}[section]
\newtheorem*{theorem*}{Theorem}

% Define environments with italic headers.
\theoremstyle{thmit}
\newtheorem*{solution}{Solution}
\newtheorem*{fsolution}{Solution}

% Define default environments.
\theoremstyle{normal}
\newtheorem{example}{Example}[section]
\newtheorem{definition}{Definition}[section]
\newtheorem{problem}{Problem}[section]
\newtheorem{question}{Question}[section]
\newtheorem{remark}{Remark}[section]
\newtheorem{properties}{Properties}[section]
\newtheorem{notation}{Notation}[section]
\newtheorem{axiom}{Axiom}[section]
\newtheorem*{properties*}{Properties}
\newtheorem*{remark*}{Remark}
\newtheorem*{definition*}{Definition}
\theoremstyle{plain}

% Define framed environment.
\tcbuselibrary{most}
\newtcbtheorem[use counter*=theorem]{ftheorem}{Theorem}%
    {colback=green!5,colframe=green!35!black,
     fonttitle=\bfseries\upshape}{th}

\newtcbtheorem[use counter*=example]{fdefinition}{Definition}%
    {fonttitle=\bfseries\upshape,
     colback=blue!5!white,colframe=blue!75!black}{def}

\newtcbtheorem[use counter*=example]{fexample}{Example}%
    {fonttitle=\bfseries\upshape,
     colback=red!5!white,colframe=red!75!black}{ex}

\newtcbtheorem[use counter*=notation]{fnotation}{Notation}%
    {fonttitle=\bfseries\upshape,
     colback=SeaGreen!5!white,colframe=SeaGreen!75!black}{ex}

\newtcbtheorem[use counter*=corollary]{fcorollary}{Corollary}%
    {fonttitle=\bfseries\upshape,
     colback=Orchid!5!white,colframe=Orchid!75!black}{ex}

\newenvironment{bproof}{\textit{Proof.}}{\hfill$\square$}
\tcolorboxenvironment{bproof}{blanker,breakable,left=5mm,
                             before skip=10pt,after skip=10pt,
                             borderline west={1mm}{0pt}{red}}
\tcolorboxenvironment{fsolution}
    {enhanced jigsaw,colframe=cyan,interior hidden,breakable}

%--------------------Declared Math Operators--------------------%
\DeclareMathOperator{\Refl}{Refl}           % Reflection operator.
\DeclareMathOperator{\Span}{Span}           % Span of a set of vectors.
\DeclareMathOperator{\Card}{Card}           % Cardinality of set.
\DeclareMathOperator{\Ord}{Ord}             % Ordinal of ordered set.
\DeclareMathOperator{\Tr}{Tr}               % Trace of matrix.
\DeclareMathOperator{\adjoint}{adj}         % Adjoint of matrix.
\DeclareMathOperator{\rk}{rk}               % Rank of operator.
\DeclareMathOperator{\nul}{nul}             % Null space of operator.
\DeclareMathOperator{\sgn}{sgn}             % Sign of a number.
\DeclareMathOperator{\multideg}{mutlideg}   % Multi-Degree (Graphs).
\DeclareMathOperator{\GCD}{GCD}             % Greatest common denominator.
\DeclareMathOperator{\LM}{LM}               % Leading monomial
\DeclareMathOperator{\LC}{LC}               % Leading coefficient.
\DeclareMathOperator{\LT}{LT}               % Leading term.
\DeclareMathOperator{\LCM}{LCM}             % Least common multiple.
\DeclareMathOperator{\Mon}{Mon}             % Monomial.
\DeclareMathOperator{\Spec}{Spec}           % Spectrum.
\DeclareMathOperator{\proj}{proj}           % Projection.
\DeclareMathOperator{\comp}{comp}           % Component.
\DeclareMathOperator{\sinc}{sinc}           % Sinc function.
\DeclareMathOperator{\Ima}{Im}              % Image of operator.
\DeclareMathOperator{\Prin}{Prin}           % Principal value.
\DeclareMathOperator{\Mod}{mod}             % Modulus.
%------------------------New Commands---------------------------%
\DeclarePairedDelimiter\norm{\lVert}{\rVert}
\DeclarePairedDelimiter\ceil{\lceil}{\rceil}
\DeclarePairedDelimiter\floor{\lfloor}{\rfloor}
\newcommand*\diff{\mathop{}\!\mathrm{d}}
\newcommand*\Diff[1]{\mathop{}\!\mathrm{d^#1}}
\renewcommand{\mod}{\ \Mod}
\renewcommand*{\glstextformat}[1]{\textcolor{RoyalBlue}{#1}}
\renewcommand{\glsnamefont}[1]{\textbf{#1}}
\renewcommand\labelitemii{$\circ$}
\renewcommand\thesubfigure{\arabic{chapter}.\arabic{figure}}
\renewcommand\thesubfigure{%
    \arabic{chapter}.\arabic{figure}.\arabic{subfigure}}
\addto\captionsenglish{\renewcommand{\figurename}{Fig.}}
%------------------------Book Command---------------------------%
\makeatletter
\renewcommand\@pnumwidth{1cm}
\newcounter{book}
\renewcommand\thebook{\@Roman\c@book}
\newcommand\book{%
    \if@openright
        \cleardoublepage
    \else
        \clearpage
    \fi
    \thispagestyle{plain}%
    \if@twocolumn
        \onecolumn
        \@tempswatrue
    \else
        \@tempswafalse
    \fi
    \null\vfil
    \secdef\@book\@sbook
}
\def\@book[#1]#2{%
    \ifnum \c@secnumdepth >-3\relax
        \refstepcounter{book}%
        \addcontentsline{toc}{book}{
            \bookname\ \thebook:\hspace{1em}#1
        }
    \else
        \addcontentsline{toc}{book}{#1}%
    \fi
    \markboth{}{}%
    {\centering
     \interlinepenalty \@M
     \normalfont
     \ifnum \c@secnumdepth >-2\relax
       \huge\bfseries \bookname\nobreakspace\thebook
       \par
       \vskip 20\p@
     \fi
     \Huge \bfseries #2\par}%
    \@endbook}
\def\@sbook#1{%
    {\centering
     \interlinepenalty \@M
     \normalfont
     \Huge \bfseries #1\par}%
    \@endbook}
\def\@endbook{
    \vfil\newpage
        \if@twoside
            \if@openright
                \null
                \thispagestyle{empty}%
                \newpage
            \fi
        \fi
        \if@tempswa
            \twocolumn
        \fi
}
\newcommand*\l@book[2]{%
    \ifnum \c@tocdepth >-2\relax
        \addpenalty{-\@highpenalty}%
        \addvspace{2.25em \@plus\p@}%
        \setlength\@tempdima{3em}%
        \begingroup
            \parindent \z@ \rightskip \@pnumwidth
            \parfillskip -\@pnumwidth
            {
                \leavevmode
                \Large \bfseries #1\hfil \hb@xt@\@pnumwidth{
                    \hss #2
                }
            }
            \par
            \nobreak
            \global\@nobreaktrue
            \everypar{\global\@nobreakfalse\everypar{}}%
        \endgroup
    \fi}
\newcommand\bookname{Book}
\renewcommand{\thebook}{\texorpdfstring{\Numberstring{book}}{book}}
\providecommand*{\toclevel@book}{-2}
\makeatother
\titlecontents{chapter}[0pt]
    {\bfseries}
    {\chaptername\ \thecontentslabel:\quad}
    {}
    {\hfill\contentspage}
\titleformat{\part}[display]
    {\Large\bfseries}
    {\partname\nobreakspace\thepart}
    {0mm}
    {\Huge\bfseries}
    \titlecontents{part}[0pt]
    {\large\bfseries}
    {\partname\ \thecontentslabel: \quad}
    {}
    {\hfill\contentspage}
\newcommand{\MarkRightAngle}[4][.3cm]
    {\coordinate (tempa) at ($(#3)!#1!(#2)$);
     \coordinate (tempb) at ($(#3)!#1!(#4)$);
     \coordinate (tempc) at ($(tempa)!0.5!(tempb)$);%midpoint
     \draw (tempa) -- ($(#3)!2!(tempc)$) -- (tempb);}
%--------------------------LENGTHS------------------------------%
% Spacings for the Table of Contents.
\addtolength{\cftsecnumwidth}{1ex}
\addtolength{\cftsubsecindent}{1ex}
\addtolength{\cftsubsecnumwidth}{1ex}
\addtolength{\cftfignumwidth}{1ex}
\addtolength{\cfttabnumwidth}{1ex}

% Spacing for multi-column and enumerate environments.
\setlength{\multicolsep}{6pt}
\setlist[enumerate]{itemsep=0pt,topsep=3pt}

% Indent and paragraph spacing.
\setlength{\parindent}{0em}
\setlength{\parskip}{0em}                                                           %
%----------------------------Main Document-------------------------------------%
\begin{document}
    \title{Differential Topology}
    \author{Ryan Maguire}
    \date{\vspace{-5ex}}
    \maketitle
    \section{Homework II}
        \begin{problem}
            Prove that if $M_{1},\dots,M_{k}$ are smooth manifolds with or
            without boundary such that at most one of them has non-empty
            boundary, if $\mathcal{M}$ is the product manifold, if $\pi_{i}$
            is the $i^{th}$ projection mapping, then $F:N\rightarrow\mathcal{M}$
            is smooth if and only if $\pi_{i}\circ{F}$ is smooth for all
            $i\in\mathbb{Z}_{k}$.
        \end{problem}
        \begin{solution}
            First we show that $\pi_{i}$ is smooth. Let
            $\mathcal{M}=M_{1}\times{M}_{2}$ and let $\pi_{1}$ be the projection
            map $(p,q)\mapsto{p}$. Let $(p,q)\in\mathcal{M}$. Since
            $M_{1}$ and $M_{2}$ are manifolds, there are charts
            $(\mathcal{U}_{1},\varphi_{1})$ and $(\mathcal{U}_{2},\varphi_{2})$
            such that $p\in\mathcal{U}_{1}$ and $q\in\mathcal{U}_{2}$ but by the
            definition of the product manifold,
            $\mathcal{U}_{1}\times\mathcal{U}_{2}$ is open in $\mathcal{M}$ and
            $(p,q)\in\mathcal{U}_{1}\times\mathcal{U}_{2}$. Moreover, the
            product function $\varphi_{1}\times\varphi_{2}$ is smooth. But then
            for all $(x,y)\in\mathcal{U}_{1}\times\mathcal{U}_{2}$ we have:
            \begin{align}
                \varphi_{1}\circ\pi_{1}\circ
                    (\varphi_{1}\times\varphi_{2})^{\minus{1}}(x,y)
                &=\varphi_{1}\circ\big(
                    \pi_{1}\circ(\varphi_{1}\times\varphi_{2})^{\minus{1}}(x,y)
                \big)\\
                &=\varphi_{1}\circ\Big(
                    \pi_{1}\big(\varphi_{1}^{\minus{1}}(x),
                                \varphi_{2}^{\minus{1}}(y)\big)\Big)\\
                &=\varphi_{1}\circ\big(\varphi_{1}^{\minus{1}}(x)\big)\\
                &=x
            \end{align}
            And thus we have the projection map from $\mathbb{R}^{d_{1}+d_{1}}$
            to $\mathbb{R}^{d_{1}}$, which is smooth. Hence for all
            $(p,q)\in\mathcal{M}$ there is a chart $(\mathcal{U},\varphi)$
            containing $(p,q)$ and a chart $(\mathcal{V},\psi)$ containg
            $\pi_{1}(p,q)$ such that $\psi\circ\pi_{1}\circ\varphi^{\minus{1}}$
            is smooth, and thus $\pi_{1}$ is smooth. Similarly, $\pi_{2}$ is
            smooth. For the general case we proceed by induction and write:
            \begin{equation}
                \mathcal{M}=\prod_{k=1}^{n+1}M_{k}=
                    \Big(\prod_{k=1}^{n}M_{k}\Big)\times{M}_{n+1}
                \equiv\widehat{\mathcal{M}}\times{M}_{n+1}
            \end{equation}
            which is the product of two manifolds, one of dimension
            $d_{1}+d_{2}+\dots+d_{n}$ and the other of dimension $d_{n+1}$, and
            thus by the previous argument the projection maps are smooth. Hence,
            $\pi_{n+1}$ is smooth. But by the induction hypothesis, all of the
            $\pi_{i}$ are smooth of $i=1,\dots,n$, and thus all $\pi_{i}$ are
            smooth. Now, suppose $f:N\rightarrow\mathcal{M}$ is smooth. Then
            $F_{i}=\pi_{i}\circ{F}$ is the composition of smooth function and
            hence by theorem 2.10 (d) in Lee's text, $F_{i}$ is smooth. In the
            other direction, we again start with the case that
            $\mathcal{M}=M_{1}\times{M}_{2}$. Suppose
            $F:N\rightarrow\mathcal{M}$ is such that $\pi_{1}\circ{F}$ and
            $\pi_{2}\circ{F}$ are smooth. Let $v\in{N}$ and let
            $(\mathcal{V},\psi)$ be a chart containing $v$ and let
            $(\widehat{\mathcal{U}},\widehat{\varphi})$ be a chart containing
            $F(v)$. Let $\mathcal{U}_{i}=\pi_{i}[\widehat{\mathcal{U}}]$ and
            let $\varphi_{i}=\pi_{i}\circ\widehat{\varphi}$. Then the
            $\varphi_{i}$ are the composition of smooth functions, and hence are
            smooth, and the $\mathcal{U}_{i}$ are the projections of an open set
            and are hence open. Moreover,
            $(\mathcal{U}_{1}\times\mathcal{U}_{2},%
             \varphi_{1}\times\varphi_{2})$ is a chart containing $F(v)$.
        \end{solution}
        \begin{problem}
            Show that $z^{n}:\nsphere[1]\rightarrow\nsphere[1]$ is smooth, the
            antipodal map $\vector{x}\mapsto\minus\vector{x}$, and the function
            $f:\nsphere[3]\rightarrow\nsphere[2]$ defined by:
            \begin{equation}
                f(w,z)=(z\overline{w}+w\overline{z},
                        iw\overline{z}-iz\overline{w},
                        z\overline{z}-w\overline{w})
            \end{equation}
        \end{problem}
        \begin{solution}
            Since $\nsphere[1]$ can be covered by two charts, we simply need to
            check that this function is smooth on these charts. Let
            $\mathcal{U}^{-}=\nsphere[1]\setminus\{(0,1)\}$ and similarly for
            $\mathcal{U}^{+}$. The stereographic projection onto $\nspace[]$
            is just:
            \twocolumneq{\varphi_{\minus}(x,y)=\frac{x}{1-y}}
                        {\varphi_{+}(x,y)=\frac{x}{1+y}}
            The inverse functions are:
            \begin{equation}
                \varphi_{\minus}^{\minus{1}}(X)
                    =\Big(\frac{X}{1+X^{2}},\frac{X^{2}-1}{X^{2}+1}\Big)
            \end{equation}
            and similarly for $\varphi_{+}$. To show that $p_{n}$ is smooth it
            suffices to show that
            $\varphi_{\minus}\circ{p}_{n}\circ\varphi_{\minus}^{\minus{1}}$ is
            smooth. We have:
            \begin{align}
                \big(\varphi_{\minus}\circ{p}_{n}\circ
                    \varphi_{\minus}^{\minus{1}}\big)(X)
                    &=\big(\varphi_{\minus}\circ{p}_{n}\big)
                    \Big(\frac{2X+i(X^{2}-1)}{1+X^{2}}\Big)\\
                    &=\varphi_{\minus}\Big(
                        \frac{\big(2X+i(X^{2}-1)^{2})^{n}}{(1+X^{2}\big)^{n}}
                    \Big)\\
            \end{align}
            Thus, the $x$ and $y$ components will both be rational functions
            in $X$, and therefore $\varphi(x,y)$ will be a rational function
            in $X$, which is smooth. In a similarly manner,
            $\varphi_{+}\circ{p}_{n}\circ\varphi_{+}^{\minus{1}}$ is smooth.
            Thus for every point $\vector{x}\in\nsphere[1]$ there is a chart
            $(\mathcal{U},\varphi)$ such that $\vector{x}\in\mathcal{U}$ and
            $\varphi\circ{p}_{n}\circ\varphi^{\minus{1}}$ is smooth. Therefore,
            $p_{n}$ is smooth. For the antipodal map on $\nsphere$ we can use
            the orthographic projections. That is,
            $(\mathcal{U}_{j}^{+},\varphi_{+})$ is the chart where
            $\mathcal{U}_{j}^{+}$ is the $j^{th}$ upper hemisphere and
            $\varphi$ is the mapping that projects
            $\nsphere\subseteq\nspace[n+1]$ down to the $\nspace$ hyper plane
            obtained by fixed all but the $j^{th}$ coordinate, and similarly
            let $(\mathcal{U}^{\minus},\varphi_{\minus})$ be the opposite
            hemisphere. Given $\vector{s}\in\nsphere$, $\vector{s}$ is contained
            in one of these, and $\minus\vector{s}$ will be contained in the
            opposite. We thus need to show that, if $f$ is the antipodal map,
            then $\varphi_{\minus}\circ{f}\circ\varphi_{+}^{\minus{1}}$ is
            smooth. We have:
            \begin{align}
                \varphi_{\minus}\circ{f}\circ\varphi_{+}^{\minus{1}}(\vector{x})
                &=\varphi_{\minus}\circ{f}(x_{0},\dots,x_{j-1},
                    \sqrt{1-\norm{\vector{x}}^{2}},x_{j+1},\dots,x_{n})\\
                &=\varphi_{\minus}(\minus{x}_{0},\dots,\minus{x}_{j-1},
                    \minus\sqrt{1-\norm{\vector{x}}^{2}},\minus{x}_{j+1},
                    \dots,\minus{x}_{n}\big)\\
                &=(\minus{x}_{0},\dots,\minus{x}_{j-1},\minus{x}_{j+1},\dots,
                    \minus{x}_{n})\\
                &=\minus\vector{x}
            \end{align}
            Hence, this is a smooth mapping since multiplying by a constant is
            a smooth mapping from $\nspace$ to itself. Lastly, we show that the
            \textit{Hopf Fibration} is smooth. Again, sticking to orthographic
            projections, let $(\mathcal{U},\varphi)$ and $(\mathcal{V},\psi)$
            be orthographic charts in $\nsphere[3]$ and $\nsphere[2]$,
            respectively (Suppose both in the $x$ axis). Let
            $f:\nsphere[3]\rightarrow\nsphere[2]$ be the Hopf fibration. Then
            for $\vector{x}\in\nball[3]$, we have:
            \begin{equation}
                \psi\circ{f}\circ\varphi^{\minus{1}}(\vector{x})
                =\psi\circ{f}\circ(\sqrt{1-\norm{\vector{x}}^{2}},\vector{x})
                =\psi\circ{f}\Big(
                    \big(\sqrt{1-\norm{\vector{x}}^{2}}+ix_{1}\big),
                    \big(x_{2}+ix_{3}\big)
                \Big)
            \end{equation}
            Using the definition of $f$ and $\psi$ (which simply projection
            down to the $yz$ plane), we have:
            \begin{equation}
                \psi\circ{f}\circ\varphi^{\minus{1}}(\vector{x})=\big(
                    2\sqrt{1-\norm{\vector{x}}^{2}}x_{1}+2x_{2}x_{3},1-x_{1}^{2}
                \big)
            \end{equation}
            which is smooth in both components, and hence is a smooth function
            from an open subset of $\nspace[3]$ to an open subset of
            $\nsphere[2]$ and is therefore smooth. Similarly for the other
            hemispheres.
        \end{solution}
        \begin{problem}
            Show that if
            $f:\nspace[n+1]\setminus\{0\}\rightarrow\nspace[k+1]\setminus\{0\}$
            is a smooth homogeneous of degree $d\in\mathbb{Z}$, then the induced
            maps $\tilde{f}:\nrealproj\rightarrow\nrealproj[k]$ are well defined
            and smooth.
        \end{problem}
        \begin{solution}
            It is indeed well defined, for if $\vector{x},\vector{y}$ have the
            the same equivalence class: $[\vector{x}]=[\vector{y}]$, then there
            is a $\lambda\in\nspace[]\setminus\{0\}$ such that
            $\vector{y}=\lambda\vector{x}$. But then:
            \begin{equation}
                \tilde{f}([\vector{y}])=[f(\vector{y})]
                    =[f(\lambda\vector{x})]
                    =[\lambda^{d}f(\vector{x})]
                    =[f(\vector{x})]
                    =\tilde{f}([\vector{x}])
            \end{equation}
            and therefore elements of the same equivalence class map to the same
            points.
        \end{solution}
        \begin{problem}
            Lee 2-10.
        \end{problem}
        \begin{solution}
            Let $\manifold[M]{M}$ and $\manifold[N]{N}$ be manifolds,
            $F:M\rightarrow{N}$ continuous. $F^{*}$ is linear. For if
            $a,b\in\mathbb{R}$, $f,g\in\Ckspace{}{M}$, then:
            \begin{subequations}
                \begin{align}
                    F^{*}(af+bg)&=(af+bg)\circ{F}\\
                    &=\big((af)\circ{F}\big)+\big((bg)\circ{F}\big)\\
                    &=a(f\circ{F})+b(g\circ{F})\\
                    &=aF^{*}(f)+bF^{*}(g)
                \end{align}
            \end{subequations}
            If $F:M\rightarrow{N}$ is smooth, then for all
            $f\in\Ckspace{\infty}{N}$ we have $F^{*}(f)=f\circ{F}$, which is the
            composition of smooth function and is hence smooth. Therefore, we
            obtain $F^{*}[\Ckspace{\infty}{N}]\subseteq\Ckspace{\infty}{M}$.
            Conversely, suppose
            $F^{*}[\Ckspace{\infty}{N}]\subseteq\Ckspace{\infty}{M}$ and suppose
            that $F$ is not smooth. Then there is a point $p\in{M}$ such tha for
            every chart $(\mathcal{U},\varphi)\in\mathcal{A}_{M}$ that contains
            $p$ and for every chart $(\mathcal{V},\psi)\in\mathcal{A}_{N}$ that
            contains $F(p)$, the function $\psi\circ{F}\circ\varphi^{\minus{1}}$
            is not smooth. But since $N$ is a manifold, there is a precompact
            chart $(\mathcal{V},\psi)$ that contains $F(p)$. But $F$ is
            continuous, and hence $F^{\minus{1}}[\mathcal{V}]$ is an open subset
            of $M$ that contains $p$. Let
            $(\tilde{\mathcal{U}},\tilde{\varphi})$ be a precompact chart
            containing $p$, and let
            $\mathcal{U}=\tilde{\mathcal{U}}\cap{F}^{\minus{1}}[\mathcal{V}]$
            and $\varphi=\tilde{\varphi}|_{\mathcal{U}}$. Then, since
            $\mathcal{U}$ is the non-empty intersection of two open subsets,
            it will open and non-empty, and hence $(\mathcal{U},\varphi)$
            is a chart in $M$. Since $\mathcal{A}_{M}$ is maximal,
            $(\mathcal{U},\varphi)\in\mathcal{A}_{M}$. By hypothesis,
            $\psi\circ{F}\circ\varphi^{\minus{1}}$ is not smooth, and hence
            there a component $k$ such that composing with the projection map
            $\pi_{k}:\nspace[m]\rightarrow\nspace[]$ is not smooth. But we may
            use the bump function to extend $\pi_{k}\circ\psi$ to all of $N$,
            obtaining a smooth function in $\Ckspace{\infty}{N}$. But then by
            hypothesis, $\pi_{k}\circ\psi\circ{F}$ is smooth. And
            $\varphi^{\minus{1}}$ is smooth, so
            $\pi_{k}\circ\psi\circ{F}\circ\varphi^{\minus{1}}$ is smooth,
            a contradiction. Hence, $F$ is smooth. Lastly, show that if $F$ is a
            homeomorphism, then it is a diffeomorphism if and only if $F^{*}$ is
            an isomorphism. Since $F$ and $F^{\minus{1}}$ are continuous, they
            are both smooth if and only if
            $F^{*}[\Ckspace{\infty}{N}]\subseteq\Ckspace{\infty}{M}$ and
            ${F^{\minus{1}}}^{*}[\Ckspace{\infty}{M}]%
             \subseteq\Ckspace{\infty}{N}$. That is, they are smooth if and only
            if $F*|_{\Ckspace{\infty}{N}}\rightarrow\Ckspace{\infty}{M}$ is a
            bijective function. Since $F^{*}$ is linear, $F$ and $F^{\minus{1}}$
            are smooth if and only if $F^{*}$ is an isomorphism.
        \end{solution}
        \begin{problem}
            Show that paracompactness and subordinate partitions of unity are
            equivalent.
        \end{problem}
        \begin{solution}
            That paracompactness implies subordinate partitions of unity was
            proved both in class and in Lee. Going the other way, let
            $\topspace{X}$ be a topological space and suppose every
            open cover has a subordinate partition of unity. Let $\mathcal{O}$
            be an open cover and let $\mathcal{F}$ be a subordinate partition of
            unity. Let $\Delta$ be defined by:
            \begin{equation}
                \Delta=\{\mathcal{U}\;|\;\exists_{f\in\mathcal{F}}
                (\mathcal{U}=f^{\minus{1}}[\nspace[]\setminus\{0\}])\}
            \end{equation}
            Then, since all $f$ are continuous, all of the $\mathcal{U}$ are the
            pre-images of open sets and are therefore open. Since $\mathcal{F}$
            is a partiion of unity, for every point $x\in{X}$ there is an
            $f\in\mathcal{F}$ such that $f(x)\ne{0}$, hence $\Delta$ is an open
            subcover of $X$. Moreover, every $\mathcal{U}$ is contained in the
            support of some function, and since $\mathcal{F}$ is partition of
            unity, these form a locally finite cover. Therefore $\Delta$ is a
            locally finite refinement of $\mathcal{O}$. Thus, $\topspace{X}$
            is paracompact.
        \end{solution}
        \begin{problem}
            Show that disjoint closed subsets can be separated by smooth
            functions.
        \end{problem}
        \begin{solution}
            For manifolds are normal, and thus if $C_{1}$ and $C_{2}$ are
            disjoint closed sets then there are disjoint open sets
            $\mathcal{U}_{1}$, $\mathcal{U}_{2}$ containing $C_{1}$ and $C_{2}$.
            Furthermore, we can shrink $\mathcal{U}_{1}$ and $\mathcal{U}_{2}$
            so that they have disjoint closures. From theorem 2.29 we can find
            functions $f,g:M\rightarrow\mathbb{R}$ such that
            $f^{\minus{1}}[\{0\}]=C_{1}$ and $g^{\minus{1}}[\{1\}]=C_{2}$.
            But since $\closure{\mathcal{U}_{1}}$ is closed, we can find a bump
            function for $\closure{\mathcal{U}_{1}}$ supported on $\mathcal{U}$
            that evaluates to $1$ on $C_{1}$, and similarly for $C_{2}$.
            Denote these bump functions $h_{1},h_{2}$. Let
            $F:M\rightarrow\mathbb{R}$ be defined by:
            \begin{equation}
                F(p)=g_{1}(p)f(p)+g_{2}(p)g(p)
            \end{equation}
        \end{solution}
\end{document}