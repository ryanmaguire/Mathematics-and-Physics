%------------------------------------------------------------------------------%
\documentclass{article}                                                        %
%------------------------------Preamble----------------------------------------%
\makeatletter                                                                  %
    \def\input@path{{../../}}                                                  %
\makeatother                                                                   %
%---------------------------Packages----------------------------%
\usepackage{geometry}
\geometry{b5paper, margin=1.0in}
\usepackage[T1]{fontenc}
\usepackage{graphicx, float}            % Graphics/Images.
\usepackage{natbib}                     % For bibliographies.
\bibliographystyle{agsm}                % Bibliography style.
\usepackage[french, english]{babel}     % Language typesetting.
\usepackage[dvipsnames]{xcolor}         % Color names.
\usepackage{listings, lstlinebgrd}      % Verbatim-Like Tools.
\usepackage{mathtools, esint, mathrsfs} % amsmath and integrals.
\usepackage{amsthm, amsfonts}           % Fonts and theorems.
\usepackage{tabularx}
\usepackage{tcolorbox}                  % Frames around theorems.
\usepackage{upgreek}                    % Non-Italic Greek.
\usepackage{paracol}                    % Two-column styling.
\usepackage{wrapfig}                    % Wrap text around figure.
\usepackage{fmtcount, etoolbox}         % For the \book{} command.
\usepackage[newparttoc]{titlesec}       % Formatting chapter, etc.
\usepackage{titletoc}                   % Allows \book in toc.
\usepackage[nottoc]{tocbibind}          % Bibliography in toc.
\usepackage[titles]{tocloft}            % ToC formatting.
\usepackage{multicol, enumitem}         % Multi-column/enumerate.
\usepackage{import}                     % Import external files.
\usepackage{pgfplots, tikz}             % Drawing/graphing tools.
\usetikzlibrary{
    calc,                   % Calculating right angles and more.
    angles,                 % Drawing angles within triangles.
    arrows.meta,            % Latex and Stealth arrows.
    quotes,                 % Adding labels to angles.
    positioning,            % Relative positioning of nodes.
    decorations.markings,   % Adding arrows in the middle of a line.
    patterns,
    arrows,
    shapes,
    shapes.geometric,
    cd,
    hobby,
    babel
}                                       % Libraries for tikz.
\pgfplotsset{compat=1.9}                % Version of pgfplots.
\usepackage[font=scriptsize,
            labelformat=simple,
            labelsep=colon]{subcaption} % Subfigure captions.
\usepackage[font={scriptsize},
            hypcap=true,
            labelsep=colon]{caption}    % Figure captions.
\usepackage{hyperref}                   % Allows for hyperlinks.
\hypersetup{
    colorlinks=true,
    linkcolor=blue,
    filecolor=magenta,
    urlcolor=Cerulean,
    citecolor=SkyBlue
}                           % Colors for hyperref.
\usepackage[toc,acronym,nogroupskip]{glossaries} % Glossaries and acronyms.
\usepackage[subpreambles=false]{standalone}      % Complileable sub files.

% Various font stuff from kiwi.
% Use this for Times text and Computer Modern math
%\usepackage{times}

% Quite nice
%\usepackage[charter, greekfamily=, greekuppercase=italicized]{mathdesign}
%\usepackage[utopia, greekuppercase=italicized]{mathdesign}    % Math is narrower

% Use this for Times text and math
%\usepackage{newtxtext}
%\usepackage[libertine,cmintegrals]{newtxmath}
%\usepackage{fix-cm}

%\usepackage{txfontsb}
% or
%\usepackage{mathptmx}

%\usepackage[scaled=0.92]{helvet}
%\renewcommand{\rmdefault}{ptm}

%\usepackage{mathpazo}    % add possibly `sc` and `osf` options
%\usepackage{eulervm}

%\usepackage{fourier}
%\renewcommand{\rmdefault}{ptm}
%\usepackage{mathptm}

%\usepackage{fontspec}
%\setmainfont{lmodern}

%\usepackage[varg]{txfonts}
%\usepackage{fouriernc}
%\usepackage{mathpazo}

%\usepackage{bookman}
%\usepackage[scaled]{uarial}
%\usepackage[scaled]{helvet}
%\renewcommand*\familydefault{\sfdefault}
%\usepackage[math]{anttor}

%\newcommand\fgeorgia{\fontfamily{jvn}\selectfont}
%\newcommand\ftimes{\fontfamily{ptm}\selectfont}
%\newcommand\fhelvetica{\fontfamily{phv}\selectfont}
%\newcommand\fcourier{\fontfamily{pcr}\selectfont}
%\newcommand\fbookman{\fontfamily{pbk}\selectfont}
%\newcommand\fnewcentury{\fontfamily{pnc}\selectfont}
%\newcommand\fpalatino{\fontfamily{ppl}\selectfont}
%\newcommand\favantgarde{\fontfamily{pag}\selectfont}
%\newcommand\fnormal{\normalfont}
%\newcommand\fsize[1]{\ifnum#1>0\fontsize{#1}{#1}\selectfont\else\normalsize\fi}
%------------------------Theorem Styles-------------------------%
% Define theorem style for default spacing and normal font.
\newtheoremstyle{normal}
    {\topsep}               % Amount of space above the theorem.
    {\topsep}               % Amount of space below the theorem.
    {}                      % Font used for body of theorem.
    {}                      % Measure of space to indent.
    {\bfseries}             % Font of the header of the theorem.
    {}                      % Punctuation between head and body.
    {.5em}                  % Space after theorem head.
    {}

% Define theorem style for default spacing with italicized font.
\newtheoremstyle{normalit}{\topsep}{\topsep}
                {\itshape}{}{\bfseries}{}{.5em}{}

% Italic header environment.
\newtheoremstyle{thmit}{\topsep}{\topsep}{}{}{\itshape}{}{0.5em}{}

% Define italicized environments.
\theoremstyle{normalit}
\newtheorem{theorem}{Theorem}[section]
\newtheorem{lemma}{Lemma}[section]
\newtheorem{corollary}{Corollary}[section]
\newtheorem{proposition}{Proposition}[section]
\newtheorem*{theorem*}{Theorem}

% Define environments with italic headers.
\theoremstyle{thmit}
\newtheorem*{solution}{Solution}
\newtheorem*{fsolution}{Solution}

% Define default environments.
\theoremstyle{normal}
\newtheorem{example}{Example}[section]
\newtheorem{definition}{Definition}[section]
\newtheorem{problem}{Problem}[section]
\newtheorem{question}{Question}[section]
\newtheorem{remark}{Remark}[section]
\newtheorem{properties}{Properties}[section]
\newtheorem{notation}{Notation}[section]
\newtheorem{axiom}{Axiom}[section]
\newtheorem*{properties*}{Properties}
\newtheorem*{remark*}{Remark}
\newtheorem*{definition*}{Definition}
\theoremstyle{plain}

% Define framed environment.
\tcbuselibrary{most}
\newtcbtheorem[use counter*=theorem]{ftheorem}{Theorem}%
    {colback=green!5,colframe=green!35!black,
     fonttitle=\bfseries\upshape}{th}

\newtcbtheorem[use counter*=example]{fdefinition}{Definition}%
    {fonttitle=\bfseries\upshape,
     colback=blue!5!white,colframe=blue!75!black}{def}

\newtcbtheorem[use counter*=example]{fexample}{Example}%
    {fonttitle=\bfseries\upshape,
     colback=red!5!white,colframe=red!75!black}{ex}

\newtcbtheorem[use counter*=notation]{fnotation}{Notation}%
    {fonttitle=\bfseries\upshape,
     colback=SeaGreen!5!white,colframe=SeaGreen!75!black}{ex}

\newtcbtheorem[use counter*=corollary]{fcorollary}{Corollary}%
    {fonttitle=\bfseries\upshape,
     colback=Orchid!5!white,colframe=Orchid!75!black}{ex}

\newenvironment{bproof}{\textit{Proof.}}{\hfill$\square$}
\tcolorboxenvironment{bproof}{blanker,breakable,left=5mm,
                             before skip=10pt,after skip=10pt,
                             borderline west={1mm}{0pt}{red}}
\tcolorboxenvironment{fsolution}
    {enhanced jigsaw,colframe=cyan,interior hidden,breakable}

%--------------------Declared Math Operators--------------------%
\DeclareMathOperator{\Refl}{Refl}           % Reflection operator.
\DeclareMathOperator{\Span}{Span}           % Span of a set of vectors.
\DeclareMathOperator{\Card}{Card}           % Cardinality of set.
\DeclareMathOperator{\Ord}{Ord}             % Ordinal of ordered set.
\DeclareMathOperator{\Tr}{Tr}               % Trace of matrix.
\DeclareMathOperator{\adjoint}{adj}         % Adjoint of matrix.
\DeclareMathOperator{\rk}{rk}               % Rank of operator.
\DeclareMathOperator{\nul}{nul}             % Null space of operator.
\DeclareMathOperator{\sgn}{sgn}             % Sign of a number.
\DeclareMathOperator{\multideg}{mutlideg}   % Multi-Degree (Graphs).
\DeclareMathOperator{\GCD}{GCD}             % Greatest common denominator.
\DeclareMathOperator{\LM}{LM}               % Leading monomial
\DeclareMathOperator{\LC}{LC}               % Leading coefficient.
\DeclareMathOperator{\LT}{LT}               % Leading term.
\DeclareMathOperator{\LCM}{LCM}             % Least common multiple.
\DeclareMathOperator{\Mon}{Mon}             % Monomial.
\DeclareMathOperator{\Spec}{Spec}           % Spectrum.
\DeclareMathOperator{\proj}{proj}           % Projection.
\DeclareMathOperator{\comp}{comp}           % Component.
\DeclareMathOperator{\sinc}{sinc}           % Sinc function.
\DeclareMathOperator{\Ima}{Im}              % Image of operator.
\DeclareMathOperator{\Prin}{Prin}           % Principal value.
\DeclareMathOperator{\Mod}{mod}             % Modulus.
%------------------------New Commands---------------------------%
\DeclarePairedDelimiter\norm{\lVert}{\rVert}
\DeclarePairedDelimiter\ceil{\lceil}{\rceil}
\DeclarePairedDelimiter\floor{\lfloor}{\rfloor}
\newcommand*\diff{\mathop{}\!\mathrm{d}}
\newcommand*\Diff[1]{\mathop{}\!\mathrm{d^#1}}
\renewcommand{\mod}{\ \Mod}
\renewcommand*{\glstextformat}[1]{\textcolor{RoyalBlue}{#1}}
\renewcommand{\glsnamefont}[1]{\textbf{#1}}
\renewcommand\labelitemii{$\circ$}
\renewcommand\thesubfigure{\arabic{chapter}.\arabic{figure}}
\renewcommand\thesubfigure{%
    \arabic{chapter}.\arabic{figure}.\arabic{subfigure}}
\addto\captionsenglish{\renewcommand{\figurename}{Fig.}}
%------------------------Book Command---------------------------%
\makeatletter
\renewcommand\@pnumwidth{1cm}
\newcounter{book}
\renewcommand\thebook{\@Roman\c@book}
\newcommand\book{%
    \if@openright
        \cleardoublepage
    \else
        \clearpage
    \fi
    \thispagestyle{plain}%
    \if@twocolumn
        \onecolumn
        \@tempswatrue
    \else
        \@tempswafalse
    \fi
    \null\vfil
    \secdef\@book\@sbook
}
\def\@book[#1]#2{%
    \ifnum \c@secnumdepth >-3\relax
        \refstepcounter{book}%
        \addcontentsline{toc}{book}{
            \bookname\ \thebook:\hspace{1em}#1
        }
    \else
        \addcontentsline{toc}{book}{#1}%
    \fi
    \markboth{}{}%
    {\centering
     \interlinepenalty \@M
     \normalfont
     \ifnum \c@secnumdepth >-2\relax
       \huge\bfseries \bookname\nobreakspace\thebook
       \par
       \vskip 20\p@
     \fi
     \Huge \bfseries #2\par}%
    \@endbook}
\def\@sbook#1{%
    {\centering
     \interlinepenalty \@M
     \normalfont
     \Huge \bfseries #1\par}%
    \@endbook}
\def\@endbook{
    \vfil\newpage
        \if@twoside
            \if@openright
                \null
                \thispagestyle{empty}%
                \newpage
            \fi
        \fi
        \if@tempswa
            \twocolumn
        \fi
}
\newcommand*\l@book[2]{%
    \ifnum \c@tocdepth >-2\relax
        \addpenalty{-\@highpenalty}%
        \addvspace{2.25em \@plus\p@}%
        \setlength\@tempdima{3em}%
        \begingroup
            \parindent \z@ \rightskip \@pnumwidth
            \parfillskip -\@pnumwidth
            {
                \leavevmode
                \Large \bfseries #1\hfil \hb@xt@\@pnumwidth{
                    \hss #2
                }
            }
            \par
            \nobreak
            \global\@nobreaktrue
            \everypar{\global\@nobreakfalse\everypar{}}%
        \endgroup
    \fi}
\newcommand\bookname{Book}
\renewcommand{\thebook}{\texorpdfstring{\Numberstring{book}}{book}}
\providecommand*{\toclevel@book}{-2}
\makeatother
\titlecontents{chapter}[0pt]
    {\bfseries}
    {\chaptername\ \thecontentslabel:\quad}
    {}
    {\hfill\contentspage}
\titleformat{\part}[display]
    {\Large\bfseries}
    {\partname\nobreakspace\thepart}
    {0mm}
    {\Huge\bfseries}
    \titlecontents{part}[0pt]
    {\large\bfseries}
    {\partname\ \thecontentslabel: \quad}
    {}
    {\hfill\contentspage}
\newcommand{\MarkRightAngle}[4][.3cm]
    {\coordinate (tempa) at ($(#3)!#1!(#2)$);
     \coordinate (tempb) at ($(#3)!#1!(#4)$);
     \coordinate (tempc) at ($(tempa)!0.5!(tempb)$);%midpoint
     \draw (tempa) -- ($(#3)!2!(tempc)$) -- (tempb);}
%--------------------------LENGTHS------------------------------%
% Spacings for the Table of Contents.
\addtolength{\cftsecnumwidth}{1ex}
\addtolength{\cftsubsecindent}{1ex}
\addtolength{\cftsubsecnumwidth}{1ex}
\addtolength{\cftfignumwidth}{1ex}
\addtolength{\cfttabnumwidth}{1ex}

% Spacing for multi-column and enumerate environments.
\setlength{\multicolsep}{6pt}
\setlist[enumerate]{itemsep=0pt,topsep=3pt}

% Indent and paragraph spacing.
\setlength{\parindent}{0em}
\setlength{\parskip}{0em}                                                           %
%----------------------------Main Document-------------------------------------%
\begin{document}
    \title{The Fundamental Theorem of Semi-Riemannian Geometry}
    \author{Ryan Maguire}
    \date{\vspace{-5ex}}
    \maketitle
    \section{Manifolds}
        \begin{definition}
            A locally Euclidean topological space is a topological space
            $(X,\tau)$ such that for all $x\in{X}$ there is an $n\in\mathbb{N}$
            such that there exists an open set $\mathcal{U}\subseteq{X}$ such
            that $x\in\mathcal{U}$ and a continuous injective open mapping
            $\varphi:\mathcal{U}\rightarrow\mathbb{R}^{n}$.
        \end{definition}
        Since $\varphi$ is an injective open mapping, it is automatically in
        homeomorphism onto it's image. Indeed, it is common to define locally
        Euclidean in terms of homeomorphisms. The dimension can very in this
        definition and we can consider the disjoint union of a sphere and a
        line. However, dimension is locally constant. This requires Brouer's
        invariance of domain.
        \begin{ftheorem}{Invariance of Domain}{Invariance of Domain}
            If $\mathcal{U}\subseteq\mathbb{R}^{n}$ is open, and if
            $f:\mathcal{U}\rightarrow\mathbb{R}^{n}$ is a continuous injective
            function, then $f(\mathcal{U})$ is open.
        \end{ftheorem}
        \begin{bproof}
            Difficult.
        \end{bproof}
        We can now prove that dimension is locally constant.
        \begin{ltheorem}{Invariance of Dimension}{Invariance_of_Dimension}
            If $\mathcal{U}\subseteq\mathbb{R}^{n}$ is open and
            $\mathcal{V}\subseteq\mathbb{R}^{m}$ is open, and if
            $f:\mathcal{U}\rightarrow\mathcal{V}$ is a homeomorphism, then
            $n=m$
        \end{ltheorem}
        \begin{proof}
            For suppose not and suppose $n>m$. Let $\tilde{\mathcal{V}}$ be the
            extension of $\mathcal{V}$ into $\mathbb{R}^{n}$ where the last
            $n-m$ coordinates are simply 0. But $f$ is a homeomorphism, and thus
            the extension $\tilde{f}:\mathcal{U}\rightarrow\mathbb{R}^{n}$ is
            an injective continuous function, and hence an open mapping.
            But $f(\mathcal{U})$ is not open since for any point and for any
            $\varepsilon>0$ the $\varepsilon$ be must leak into the last
            $n-m$ coordinates, a contradiction. Thus $n=m$.
        \end{proof}
        \begin{theorem}
            If $(X,\tau)$ is a locally Euclidean topological space, if
            $x\in{X}$, and if $n,m\in\mathbb{N}$ are such that there exists
            open sets $\mathcal{U}_{n}$ and $\mathcal{V}_{m}$ such that there
            are injective open mapping
            $\varphi_{n}:\mathcal{U}_{n}\rightarrow\mathbb{R}^{n}$ and
            $\varphi_{m}:\mathcal{V}_{m}\rightarrow\mathbb{R}^{m}$, then $m=n$.
        \end{theorem}
        \begin{proof}
            For $\mathcal{U}_{n}\cap\mathcal{V}_{m}$ is non-empty since $x$ is
            in the intersection, and the intersection of open is open. Thus the
            resection $\varphi_{n}|_{\mathcal{U}_{n}\cap\mathcal{V}_{m}}$ and
            $\varphi_{m}|_{\mathcal{U}_{n}\cap\mathcal{V}_{m}}$ are open
            mappings into $\mathbb{R}^{n}$ and $\mathbb{R}^{m}$, respectively.
            But then these are homeomorphisms onto their images. But by
            composing $\varphi_{n}\circ\varphi_{m}^{\minus{1}}$ we obtain a
            homeomorphism from an open subset of $\mathbb{R}^{m}$ to
            $\mathbb{R}^{n}$. But then $n=m$, by the previous theorem.
        \end{proof}
        We have thus shown that there is a well defined dimension function
        $\textrm{dim}:X\rightarrow\mathbb{N}$ that assigns to every $x\in{X}$
        the unique dimension of $x$. That is, the space $\mathbb{R}^{n}$ that
        $x$ locally looks like.
        \begin{definition}
            A locally constant function on from a topological space $(X,\tau)$
            to a set $Y$ is a function $f:X\rightarrow{Y}$ such that for all
            $x\in{X}$ there exists an open subset $\mathcal{U}\in\tau$ such that
            $x\in\mathcal{U}$ and $f|_{\mathcal{U}}$ is a constant mapping.
        \end{definition}
        Note that there need not be any topology on $Y$.
        \begin{theorem}
            If $(X,\tau)$ is a connected topological space, if $Y$ is a set with
            at least two distinct points, and if $f:X\rightarrow{Y}$ is a
            locally constant function, then it is a constant function.
        \end{theorem}
        \begin{proof}
            For suppose not. Then there are $c_{1},c_{2}\in{Y}$ such that
            $f^{\minus{1}}[\{c_{1}\}]$ and $f^{\minus{1}}[\{c_{2}\}]$ are
            non-empty. Let $x\in{f}^{\minus{1}}[\{c_{1}\}]$. Since $f$ is a
            locally constant function, there exists an open subset
            $\mathcal{U}_{x}\subseteq{X}$ such that $x\in\mathcal{U}$ and
            $f|_{\mathcal{U}}$ is a constant mapping. But this is true of all
            $x\in{f}^{\minus{1}}[\{c_{1}\}]$, and hence:
            \begin{equation}
                f^{\minus{1}}[\{c_{1}\}]=
                    \bigcup_{x\in{f}^{\minus{1}}[\{c_{1}\}]}\mathcal{U}_{x}
            \end{equation}
            which is the union of open, and hence open. Similarly, the
            complement can be written as:
            \begin{equation}
                X\setminus{f}^{\minus{1}}[\{c_{1}\}]=
                    \bigcup_{x\notin{f}^{\minus{1}}[\{c_{1}\}]}\mathcal{U}_{x}
            \end{equation}
            which is the union of open, and hence open. Moreover it is non-empty
            since $f^{\minus{1}}[\{c_{2}\}]$ is a subset, and this set is
            non-empty. But then $X$ is the union of two disjoint non-empty open
            subsets and is hence disconnected, a contradiction. Thus, $f$ is
            constant.
        \end{proof}
        \begin{theorem}
            If $(X,\tau)$ is a locally Euclidean topological space, then the
            dimension function $\textrm{dim}:X\rightarrow\mathbb{N}$ is locally
            constant.
        \end{theorem}
        \begin{proof}
            For let $x\in{X}$. Then there is an open subset
            $\mathcal{U}\subseteq{X}$ such that $x\in\mathcal{U}$ and
            there exists a continuous injective open mapping
            $f:\mathcal{U}\rightarrow\mathbb{R}^{n}$. But then for all
            $y\in\mathcal{U}$, $\mathcal{U}$ is an open set such that
            $y\in\mathcal{U}$ and $f:\mathcal{U}\rightarrow\mathbb{R}^{n}$ is a
            continuous injective open mapping, and hence
            $\textrm{dim}(y)=\textrm{dim}(x)$. Thus, $\textrm{dim}$ is locally
            constant.
        \end{proof}
        \begin{theorem}
            If $(X,\tau)$ is a connected locally Euclidean topological space,
            then there is a unique dimension.
        \end{theorem}
        \begin{proof}
            For $\textrm{dim}$ is locally constant, and a locally constant
            function on a connected space is constant.
        \end{proof}
        \begin{definition}
            A topological manifold is a topological space $(X,\tau)$
            that is locally Euclidean, Hausdorff, and second countable.
        \end{definition}
        \begin{example}
            Neither the Hausdorff nor the second countable claims are redundant.
            The bug eye line is a non-Hausdorff locally Euclidean space that is
            second countable, and the long line is a non-second countable space
            that is locally Euclidean and Hausdorff.
        \end{example}
        \begin{definition}
            A chart on a topological space $(X,\tau)$ is an open set
            $\mathcal{U}$ and a continuous injective open mapping
            $\varphi:\mathcal{U}\rightarrow\mathbb{R}^{n}$.
        \end{definition}
        \begin{definition}
            Compatible charts on a topological space $(X,\tau)$ are charts
            $(\mathcal{U},\varphi)$ and $(\mathcal{V},\psi)$ such that either
            $\mathcal{U}\cap\mathcal{V}$ are empty, or
            $\varphi\circ\psi^{\minus{1}}:%
             \psi(\mathcal{V})\rightarrow\mathbb{R}^{n}$ and
            $\psi\circ\varphi^{\minus{1}}:%
             \varphi(\mathcal{U})\rightarrow\mathbb{R}^{n}$ are smooth.
        \end{definition}
        \begin{definition}
            A smooth atlas on a topological manifold $(X,\tau)$ is a collection
            of compatible charts that cover $X$.
        \end{definition}
        \begin{definition}
            A maximal smooth atlas on a topological manifold $(X,\tau)$ is a
            smooth atlas $\mathcal{A}$ such that for all smooth atlases
            $\mathcal{A}'$ such that $\mathcal{A}\subseteq\mathcal{A}'$, it is
            true that $\mathcal{A}=\mathcal{A}'$.
        \end{definition}
        \begin{definition}
            Compatible atlases on a topological manifold $(X,\tau)$ are atlases
            $\mathcal{A}$ and $\mathcal{A}'$ such that for every chart in
            $\mathcal{A}$ is compatible with every chart in $\mathcal{A}'$, and
            vice-versa.
        \end{definition}
        \begin{theorem}
            If $(X,\tau)$ is a topological manifold and if $\mathcal{A}$ and
            $\mathcal{A}'$ are smooth atlases, then they are compatible if and
            only if $\mathcal{A}\cup\mathcal{A}'$ is a smooth atlas.
        \end{theorem}
        \begin{proof}
            For if $\mathcal{A}$ and $\mathcal{A}'$ are compatible, then
            every element of $\mathcal{A}\cup\mathcal{A}'$ is compatible with
            every other element, and moreover the domains cover $X$. Hence,
            the union is a smooth atlas. If the union is a smooth atlas, then
            every element of $\mathcal{A}$ is compatible with every element of
            $\mathcal{A}\cup\mathcal{A}'$, and in particular it is compatible
            with every element of $\mathcal{A}'$. Similarly, every element of
            $\mathcal{A}'$ is compatible with $\mathcal{A}$, and hence they are
            compatible atlases.
        \end{proof}
        \begin{ltheorem}{Existence and Uniqueness of Maximal Smooth Atlas}
                        {Existence and Uniqueness of Maximal Smooth Atlas}
            If $(X,\tau)$ is a topological manifold and if $\mathcal{A}$ is a
            smooth atlas, then there is a maximal smooth atlas $\mathcal{C}$
            such that $\mathcal{A}\subseteq\mathcal{C}$.
        \end{ltheorem}
        \begin{proof}
            For consider the set of all atlases on $(X,\tau)$. This forms a
            partially ordered set by inclusion. Given a chain of atlases,
            the union over the entire chain is again an atlas by the previous
            theorem. But then every chain is bounded, and thus by Zorn's lemma
            for every $\mathcal{A}$ there is a maximal element $\mathcal{C}$
            that bounds $\mathcal{A}$. If $\mathcal{C}'$ is another, then
            they are compatible since they are both compatible with
            $\mathcal{A}$, and thus $\mathcal{C}\cup\mathcal{C}'$ is a smooth
            atlas, contradicting maximality. Hence, $\mathcal{C}$ is unique.
        \end{proof}
        \begin{definition}
            A smooth manifold is a topological manifold $(X,\tau)$ with a
            maximal smooth atlas $\mathcal{A}$.
        \end{definition}
        \begin{definition}
            A smooth function from a smooth manifold
            $(M,\tau_{M},\mathcal{A}_{M})$ to a smooth manifold
            $(N,\tau_{N},\mathcal{A}_{n})$ is a function $\phi:M\rightarrow{N}$
            such that for all $p\in{M}$ there is a chart
            $(\mathcal{U},\varphi)\in\mathcal{A}_{M}$ and a chart
            $(\mathcal{V},\psi)\in\mathcal{A}_{N}$ such that $p\in\mathcal{U}$,
            $\phi(p)\in\mathcal{V}$, and the mapping
            $\psi\circ\phi\circ\varphi^{\minus{1}}:%
             \varphi(\mathcal{U}\cap\phi^{\minus{1}}(\mathcal{V}))%
             \rightarrow\mathbb{R}^{n}$ is a smooth function.
        \end{definition}
        Note that since charts in a smooth atlas overlap smoothly, this
        definition does not actually depend on the choice of charts.
        \begin{definition}
            A diffeomorphism from a smooth manifold
            $(M,\tau_{M},\mathcal{A}_{M})$ to a smooth manifold
            $(N,\tau_{N},\mathcal{A}_{N})$ is a homeomorphism
            $\phi:M\rightarrow{N}$ such that $\phi$ and $\phi^{\minus{1}}$ are
            smooth.
        \end{definition}
        In particular, we can consider the set of all smooth functions
        from the manifold $M$ into $\mathbb{R}$, where $\mathbb{R}$ has its
        usual smooth structure. This structure $C^{\infty}(M,\mathbb{R})$ forms
        a commutative ring.
        \begin{definition}
            A tangent vector at a point $p$ in a smooth manifold
            $(M,\tau,\mathcal{A})$ is a function
            $v:C^{\infty}(M,\mathbb{R})\rightarrow\mathbb{R}$ that is linear and
            Liebnezian. That is:
            \begin{align}
                v(af+bg)&=av(f)+bv(g)
                \tag{Linearity}\\
                v(fg)&=v(f)g(p)+f(p)v(g)
                \tag{Liebnezian}
            \end{align}
        \end{definition}
        \begin{definition}
            The tangent space at a point $p$ in a smooth manifold
            $(M,\tau,\mathcal{A})$ is the set $T_{p}M$ of all tangent vectors
            at $p$.
        \end{definition}
        This may seem abstract, but it is a direct generalization of the
        directional derivative one studies in vector calculus.
        \begin{theorem}
            If $\phi:M\rightarrow{N}$ is a smooth function, if $p\in{M}$, and if
            $v$ is a tangent vector at $p$, then the function
            $\diff_{p}\phi:C^{\infty}(N,\mathbb{R})\rightarrow\mathbb{R}$
            defined by $\diff_{p}\phi(v)(f)=v(f\circ\phi)$ is an element of
            $T_{\phi(p)}N$.
        \end{theorem}
        \begin{proof}
            For:
            \begin{equation}
                v\big((af+bg)\circ\phi\big)
                =v\big(a(f\circ\phi)+b(g\circ\phi)\big)
                =av(f\circ\phi)+bv(g\circ\phi)
            \end{equation}
            and:
            \begin{equation}
                v\big((fg)\circ\phi\big)
                =v\big((f\circ\phi)(g\circ\phi)\big)
                =v(f\circ\phi)(g\circ\phi)(p)+(f\circ\phi)(p)v(g\circ\phi)
            \end{equation}
        \end{proof}
        \begin{definition}
            The differential pushforward of a smooth function
            $\phi:M\rightarrow{N}$ from a smooth manifold
            $(M,\tau_{M},\mathcal{A}_{M})$ to a smooth manifold
            $(N,\tau_{N},\mathcal{A}_{M})$ at a point $p\in{M}$ is the function
            $\diff_{p}\phi:T_{p}M\rightarrow{T}_{\phi(p)}N$ defined by
            $\diff_{p}\phi(v)(f)=v(f\circ\phi)$ for all
            $f\in{C}^{\infty}(N,\mathbb{R})$.
        \end{definition}
        \begin{definition}
            An open curve in a topological space $(X,\tau)$ is a continuous
            function $\alpha:I\rightarrow{X}$, where $I$ is the open unit
            interval.
        \end{definition}
        \begin{definition}
            A smooth curve in a smooth manifold $(M,\tau,\mathcal{A})$ is a
            curve $\alpha:I\rightarrow{M}$ such that $\alpha$ is a smooth
            function with respect to the standard smooth structure on $I$.
        \end{definition}
        \begin{definition}
            The velocity of a smooth curve $\alpha:I\rightarrow{M}$ at a point
            $t\in{I}$ differential pushforward evaluated at the tangent vector
            $\diff/\diff{x}|_{x=t}$. That is:
            \begin{equation}
                \dot{\alpha}(t)=
                    \diff_{t}\alpha\Big(\frac{\diff}{\diff{x}}\big|_{x=t}\Big)
            \end{equation}
        \end{definition}
        \begin{definition}
            The tangent bundle of a smooth manifold $(M,\tau,\mathcal{A})$ is
            the set $TM=\coprod_{p\in{M}}T_{p}M$. That is, the disjoint union of
            all tangent spaces.
        \end{definition}
        There is a topology that one can place on the tangent bundle that makes
        it a $2n$ dimensional manifold (with a natural smooth atlas) and this is
        not the disjoint union topology. Indeed for any manifold of non-zero
        dimension the disjoint union topology is not second countable, and hence
        not a manifold (but it will be an $n$ dimensional locally Euclidean
        Hausdorff space).
        \begin{definition}
            A vector field is a smooth function $V:M\rightarrow{TM}$ such that
            for all $p\in{M}$, $V(p)\in{T}_{p}M$.
        \end{definition}
        The collection $\mathfrak{X}(M)$ of all smooth vector fields on a smooth
        manifold $M$ has a module structure over $C^{\infty}(M,\mathbb{R})$ is
        we define addition and scalar multiplication as follows:
        \begin{align}
            (fV)_{p}&=f(p)V_{p}\\
            (V+W)_{p}&=V_{p}+W_{p}
        \end{align}
        We can also evaluate vector fields at functions in
        $C^{\infty}(M,\mathbb{R})$ as follows:
        \begin{equation}
            (Vf)(p)=V_{p}(f)
        \end{equation}
        This is well defined since for all $p\in{M}$, $V_{p}$ is a tangent
        vector in $T_{p}M$, which is a linear functional. Thus $Vf$ is a
        function from $C^{\infty}(M,\mathbb{R})$ to itself. The structure of
        $C^{\infty}(M,\mathbb{R})$ can also be seen as an algebra over the field
        $\mathbb{R}$. As such we can define derivations.
        \begin{definition}
            A derivation on a $C^{\infty}(M,\mathbb{R})$ is a function
            $D:C^{\infty}(M,\mathbb{R})\rightarrow{C}^{\infty}(M,\mathbb{R})$
            that is $\mathbb{R}$ linear and Liebnezian:
            \begin{align}
                D(af+bg)&=aD(f)+bD(g)\\
                D(fg)&=D(f)g+fD(g)
            \end{align}
        \end{definition}
        This is slightly redundant since every derivation comes from a vector
        field. Given $V\in\mathfrak{X}(M)$, the function
        $D:C^{\infty}(M,\mathbb{R})\rightarrow{C}^{\infty}(M,\mathbb{R})$
        defined by $D(v)=Vf$ is a derivation, and moreover every derivation
        comes from a vector field. Simply let $V_{p}$ be the tanget vector such
        that $V_{p}(f)=D(f)(p)$ for all $p\in{M}$ and
        $f\in{C}^{\infty}(M,\mathbb{R})$. 
        \par\hfill\par
        We can also evaluated vector fields on other vector fields. Given
        $V,W\in\mathfrak{X}(M)$, we denote $V(W)$ to be the function
        $V(W):C^{\infty}(M,\mathbb{R})\rightarrow{C}^{\infty}(M,\mathbb{R})$
        defined by:
        \begin{equation}
            V(W)(f)=V(Wf)
        \end{equation}
        More explicitly, if $p\in{M}$, then:
        \begin{equation}
            \Big(\big(V(W)\big)(f)\Big)(p)=V_{p}(Wf)
        \end{equation}
        Remember that $Wf$ is a function from $C^{\infty}(M,\mathbb{R})$ into
        itself, where as $V_{p}$ is a tangent vector and hence a linear
        functional from $C^{\infty}(M,\mathbb{R})$ into $\mathbb{R}$. Thus,
        while strange, this definition is well posed. What's important is that
        now we can define the Lie bracket of two vector fields.
        \begin{definition}
            The Lie Bracket is the function
            $[\cdot,\cdot]:\mathfrak{X}(M)\times\mathfrak{X}(M)%
             \rightarrow\mathfrak{X}(M)$ defined by:
            \begin{equation}
                [V,W]=V(W)-W(V)
            \end{equation}
        \end{definition}
        More explicitly, given $f\in{C}^{\infty}(M,\mathbb{R})$ and $p\in{M}$,
        we have:
        \begin{equation}
            [V,W]_{p}(f)=V_{p}(Wf)-W_{p}(Vf)
        \end{equation}
        Before moving on to connections, we define integral curves.
        \begin{definition}
            An integral curve of a vector field $V\in\mathfrak{X}(M)$ is a
            curve $\alpha:I\rightarrow{M}$ such that for all $t\in{I}$,
            $\dot{\alpha}(t)=V_{\alpha(t)}$.
        \end{definition}
        \begin{definition}
            A bilinear form on a vector field $V$ over a field $F$ is a function
            $g:V\times{V}\rightarrow{F}$ such that:
            \begin{align}
                g(a\vector{x}+b\vector{y},\vector{z})
                    &=ag(\vector{x},\vector{z})+bg(\vector{y},\vector{z})\\
                g(\vector{x},a\vector{y}+b\vector{z})
                    &=ag(\vector{x},\vector{y})+bg(\vector{x},\vector{z})
            \end{align}
        \end{definition}
        \begin{definition}
            A symmetric form on a vector space $V$ over a field $F$ is a
            function $g:V\times{V}\rightarrow{F}$ such that for all
            $\vector{x},\vector{y}\in{V}$,
            $g(\vector{x},\vector{y})=g(\vector{y},\vector{x})$.
        \end{definition}
        \begin{definition}
            A non-degenerate form on a vector space $V$ over a field $F$ is a
            function $g:V\times{V}\rightarrow{F}$ such that for all
            $\vector{x}\in{V}$ such that $\vector{x}\ne\vector{0}$ there exists
            a $\vector{y}\in{V}$ such that $g(\vector{x},\vector{y})\ne{0}$.
        \end{definition}
        In other words, if $g(\vector{x},\vector{y})=0$ for all $\vector{y}$,
        then $\vector{x}=\vector{0}$. Lastly, positive-definiteness.
        \begin{definition}
            A positive-definite form on a vector space $V$ over an ordered field
            $F$ is a function $g:V\times{V}\rightarrow{F}$ such that for all
            $\vector{x}\in{V}$ it is true that $g(\vector{x},\vector{x})\geq{0}$
            and $g(\vector{x},\vector{x})=0$ implies that
            $\vector{x}=\vector{0}$.
        \end{definition}
        Semi-Riemannian and Riemannian geometry are concerned with smooth
        manifolds that have symmetric bilinear forms that are non-degenerate
        (Semi-Riemannian) or positive-definite (Riemannian). The vector spaces
        are simply the tangent spaces, and we need a function $g$ that takes in
        a point $p\in{M}$ and two elements of $T_{p}M$ and returns a real
        number. We further want this function to be a symmetric non-degenerate
        bilinear form for every $p$, and we would also like this function to
        vary smoothly between points. So in essence, we want a function
        $g:\mathfrak{X}(M)\times\mathfrak{X}(M)%
         \rightarrow{C}^{\infty}(M,\mathbb{R})$. Given a point $p\in{M}$,
        and two vector fields $V,W\in\mathfrak{X}(M)$, we want
        $g(V,W)(p)=g_{p}(V_{p},W_{p})$ to vary smoothly with $p$. We also want
        symmetry, bilinearity, and non-degenacy.
        \begin{definition}
            A metric tensor on a smooth manifold $(M,\tau,\mathcal{A})$ is a
            smooth non-degenacy symmetric bilinear form
            $g:\mathfrak{X}(M)\times\mathfrak{X}(M)%
             \rightarrow{C}^{\infty}(M,\mathbb{R})$.
        \end{definition}
        \begin{definition}
            A semi-Riemannian manifold is a smooth manifold
            $(M,\tau,\mathcal{A})$ with a metric tensor $g$.
        \end{definition}
    \section{Connections}
        We now want to talk about transporting data along a manifold in a
        parallel manner. For Euclidean space $\mathbb{R}^{n}$ there is an
        intuitive manner to do this: Simply translate your vector from point $a$
        to point $b$. On a sphere there's a slightly intuitive manner.
        Translations may leave the sphere and so we need a new method. Give two
        points we can simply rotate a tangent vector at $a$ to $b$, but the
        resultant vector depends on how one rotated the sphere. That is, along
        which path did one rotate. This is simply a consequence of the curvature
        of the sphere. For the problem of generalizing to aribtrary
        manifolds one might think coordinates suffice to perform parallel
        transport, but this is not true. Even on the sphere, using the two
        stereographic projections about the north and south pole, one runs into
        incompatibility issues. So the idea is to find a way of describing the
        rate of change of one vector field with respect to another. Given a
        vector field $V$ and a tangent vector $v\in{T}_{p}M$ for some point
        $p\in{M}$ such that $V_{p}=v$, and given another vector field $W$,
        we can think of parallel transport as transporting $v$ along an
        integral curve of $W$.
        \begin{definition}
            A connection $\nabla$ on a smooth manifold $(M,\tau,\mathcal{A})$
            is a function $\nabla:\mathfrak{X}(M)\times\mathfrak{X}(M)%
            \rightarrow\mathfrak{X}(M)$ such that:
            \begin{align}
                \nabla(fV_{1}+gV_{2},W)
                    &=f\nabla(V_{1},W)+g\nabla(V_{2},W)\\
                \nabla(V,aW_{1}+bW_{2})
                    &=a\nabla(V,W_{1})+b\nabla(V,W_{2})\\
                \nabla(V,fW)&=
                    (Vf)W+f\nabla(V,W)
            \end{align}
        \end{definition}
        \begin{definition}
            A Levi-Civita connection on a semi-Riemannian manifold
            $(M,g)$ is a connection $\nabla$ that is torsion
            free (preserves the Lie bracket):
            \begin{equation}
                \nabla(V,W)-\nabla(W,V)=[V,W]
            \end{equation}
            and preserves the metric tensor:
            \begin{equation}
                Xg(V,W)=g\big(\nabla(X,V),W\big)+g\big(V,\nabla(X,W)\big)
            \end{equation}
        \end{definition}
        \begin{ftheorem}{Fundamental Theorem of Riemannian Geometry}
                        {Fundamental_Theorem_of_Riemannian Geometry}
            If $(M,g)$ is a semi-Riemannian manifold, then there is a unique
            Levi-Civita connection $\nabla$ on $(M,g)$.
        \end{ftheorem}
\end{document}