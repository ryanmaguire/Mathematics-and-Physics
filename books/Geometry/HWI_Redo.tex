%------------------------------------------------------------------------------%
\documentclass{article}                                                        %
%------------------------------Preamble----------------------------------------%
\makeatletter                                                                  %
    \def\input@path{{../../}}                                                  %
\makeatother                                                                   %
%---------------------------Packages----------------------------%
\usepackage{geometry}
\geometry{b5paper, margin=1.0in}
\usepackage[T1]{fontenc}
\usepackage{graphicx, float}            % Graphics/Images.
\usepackage{natbib}                     % For bibliographies.
\bibliographystyle{agsm}                % Bibliography style.
\usepackage[french, english]{babel}     % Language typesetting.
\usepackage[dvipsnames]{xcolor}         % Color names.
\usepackage{listings}                   % Verbatim-Like Tools.
\usepackage{mathtools, esint, mathrsfs} % amsmath and integrals.
\usepackage{amsthm, amsfonts, amssymb}  % Fonts and theorems.
\usepackage{tcolorbox}                  % Frames around theorems.
\usepackage{upgreek}                    % Non-Italic Greek.
\usepackage{fmtcount, etoolbox}         % For the \book{} command.
\usepackage[newparttoc]{titlesec}       % Formatting chapter, etc.
\usepackage{titletoc}                   % Allows \book in toc.
\usepackage[nottoc]{tocbibind}          % Bibliography in toc.
\usepackage[titles]{tocloft}            % ToC formatting.
\usepackage{pgfplots, tikz}             % Drawing/graphing tools.
\usepackage{imakeidx}                   % Used for index.
\usetikzlibrary{
    calc,                   % Calculating right angles and more.
    angles,                 % Drawing angles within triangles.
    arrows.meta,            % Latex and Stealth arrows.
    quotes,                 % Adding labels to angles.
    positioning,            % Relative positioning of nodes.
    decorations.markings,   % Adding arrows in the middle of a line.
    patterns,
    arrows
}                                       % Libraries for tikz.
\pgfplotsset{compat=1.9}                % Version of pgfplots.
\usepackage[font=scriptsize,
            labelformat=simple,
            labelsep=colon]{subcaption} % Subfigure captions.
\usepackage[font={scriptsize},
            hypcap=true,
            labelsep=colon]{caption}    % Figure captions.
\usepackage[pdftex,
            pdfauthor={Ryan Maguire},
            pdftitle={Mathematics and Physics},
            pdfsubject={Mathematics, Physics, Science},
            pdfkeywords={Mathematics, Physics, Computer Science, Biology},
            pdfproducer={LaTeX},
            pdfcreator={pdflatex}]{hyperref}
\hypersetup{
    colorlinks=true,
    linkcolor=blue,
    filecolor=magenta,
    urlcolor=Cerulean,
    citecolor=SkyBlue
}                           % Colors for hyperref.
\usepackage[toc,acronym,nogroupskip,nopostdot]{glossaries}
\usepackage{glossary-mcols}
%------------------------Theorem Styles-------------------------%
\theoremstyle{plain}
\newtheorem{theorem}{Theorem}[section]

% Define theorem style for default spacing and normal font.
\newtheoremstyle{normal}
    {\topsep}               % Amount of space above the theorem.
    {\topsep}               % Amount of space below the theorem.
    {}                      % Font used for body of theorem.
    {}                      % Measure of space to indent.
    {\bfseries}             % Font of the header of the theorem.
    {}                      % Punctuation between head and body.
    {.5em}                  % Space after theorem head.
    {}

% Italic header environment.
\newtheoremstyle{thmit}{\topsep}{\topsep}{}{}{\itshape}{}{0.5em}{}

% Define environments with italic headers.
\theoremstyle{thmit}
\newtheorem*{solution}{Solution}

% Define default environments.
\theoremstyle{normal}
\newtheorem{example}{Example}[section]
\newtheorem{definition}{Definition}[section]
\newtheorem{problem}{Problem}[section]

% Define framed environment.
\tcbuselibrary{most}
\newtcbtheorem[use counter*=theorem]{ftheorem}{Theorem}{%
    before=\par\vspace{2ex},
    boxsep=0.5\topsep,
    after=\par\vspace{2ex},
    colback=green!5,
    colframe=green!35!black,
    fonttitle=\bfseries\upshape%
}{thm}

\newtcbtheorem[auto counter, number within=section]{faxiom}{Axiom}{%
    before=\par\vspace{2ex},
    boxsep=0.5\topsep,
    after=\par\vspace{2ex},
    colback=Apricot!5,
    colframe=Apricot!35!black,
    fonttitle=\bfseries\upshape%
}{ax}

\newtcbtheorem[use counter*=definition]{fdefinition}{Definition}{%
    before=\par\vspace{2ex},
    boxsep=0.5\topsep,
    after=\par\vspace{2ex},
    colback=blue!5!white,
    colframe=blue!75!black,
    fonttitle=\bfseries\upshape%
}{def}

\newtcbtheorem[use counter*=example]{fexample}{Example}{%
    before=\par\vspace{2ex},
    boxsep=0.5\topsep,
    after=\par\vspace{2ex},
    colback=red!5!white,
    colframe=red!75!black,
    fonttitle=\bfseries\upshape%
}{ex}

\newtcbtheorem[auto counter, number within=section]{fnotation}{Notation}{%
    before=\par\vspace{2ex},
    boxsep=0.5\topsep,
    after=\par\vspace{2ex},
    colback=SeaGreen!5!white,
    colframe=SeaGreen!75!black,
    fonttitle=\bfseries\upshape%
}{not}

\newtcbtheorem[use counter*=remark]{fremark}{Remark}{%
    fonttitle=\bfseries\upshape,
    colback=Goldenrod!5!white,
    colframe=Goldenrod!75!black}{ex}

\newenvironment{bproof}{\textit{Proof.}}{\hfill$\square$}
\tcolorboxenvironment{bproof}{%
    blanker,
    breakable,
    left=3mm,
    before skip=5pt,
    after skip=10pt,
    borderline west={0.6mm}{0pt}{green!80!black}
}

\AtEndEnvironment{lexample}{$\hfill\textcolor{red}{\blacksquare}$}
\newtcbtheorem[use counter*=example]{lexample}{Example}{%
    empty,
    title={Example~\theexample},
    boxed title style={%
        empty,
        size=minimal,
        toprule=2pt,
        top=0.5\topsep,
    },
    coltitle=red,
    fonttitle=\bfseries,
    parbox=false,
    boxsep=0pt,
    before=\par\vspace{2ex},
    left=0pt,
    right=0pt,
    top=3ex,
    bottom=1ex,
    before=\par\vspace{2ex},
    after=\par\vspace{2ex},
    breakable,
    pad at break*=0mm,
    vfill before first,
    overlay unbroken={%
        \draw[red, line width=2pt]
            ([yshift=-1.2ex]title.south-|frame.west) to
            ([yshift=-1.2ex]title.south-|frame.east);
        },
    overlay first={%
        \draw[red, line width=2pt]
            ([yshift=-1.2ex]title.south-|frame.west) to
            ([yshift=-1.2ex]title.south-|frame.east);
    },
}{ex}

\AtEndEnvironment{ldefinition}{$\hfill\textcolor{Blue}{\blacksquare}$}
\newtcbtheorem[use counter*=definition]{ldefinition}{Definition}{%
    empty,
    title={Definition~\thedefinition:~{#1}},
    boxed title style={%
        empty,
        size=minimal,
        toprule=2pt,
        top=0.5\topsep,
    },
    coltitle=Blue,
    fonttitle=\bfseries,
    parbox=false,
    boxsep=0pt,
    before=\par\vspace{2ex},
    left=0pt,
    right=0pt,
    top=3ex,
    bottom=0pt,
    before=\par\vspace{2ex},
    after=\par\vspace{1ex},
    breakable,
    pad at break*=0mm,
    vfill before first,
    overlay unbroken={%
        \draw[Blue, line width=2pt]
            ([yshift=-1.2ex]title.south-|frame.west) to
            ([yshift=-1.2ex]title.south-|frame.east);
        },
    overlay first={%
        \draw[Blue, line width=2pt]
            ([yshift=-1.2ex]title.south-|frame.west) to
            ([yshift=-1.2ex]title.south-|frame.east);
    },
}{def}

\AtEndEnvironment{ltheorem}{$\hfill\textcolor{Green}{\blacksquare}$}
\newtcbtheorem[use counter*=theorem]{ltheorem}{Theorem}{%
    empty,
    title={Theorem~\thetheorem:~{#1}},
    boxed title style={%
        empty,
        size=minimal,
        toprule=2pt,
        top=0.5\topsep,
    },
    coltitle=Green,
    fonttitle=\bfseries,
    parbox=false,
    boxsep=0pt,
    before=\par\vspace{2ex},
    left=0pt,
    right=0pt,
    top=3ex,
    bottom=-1.5ex,
    breakable,
    pad at break*=0mm,
    vfill before first,
    overlay unbroken={%
        \draw[Green, line width=2pt]
            ([yshift=-1.2ex]title.south-|frame.west) to
            ([yshift=-1.2ex]title.south-|frame.east);},
    overlay first={%
        \draw[Green, line width=2pt]
            ([yshift=-1.2ex]title.south-|frame.west) to
            ([yshift=-1.2ex]title.south-|frame.east);
    }
}{thm}

%--------------------Declared Math Operators--------------------%
\DeclareMathOperator{\adjoint}{adj}         % Adjoint.
\DeclareMathOperator{\Card}{Card}           % Cardinality.
\DeclareMathOperator{\curl}{curl}           % Curl.
\DeclareMathOperator{\diam}{diam}           % Diameter.
\DeclareMathOperator{\dist}{dist}           % Distance.
\DeclareMathOperator{\Div}{div}             % Divergence.
\DeclareMathOperator{\Erf}{Erf}             % Error Function.
\DeclareMathOperator{\Erfc}{Erfc}           % Complementary Error Function.
\DeclareMathOperator{\Ext}{Ext}             % Exterior.
\DeclareMathOperator{\GCD}{GCD}             % Greatest common denominator.
\DeclareMathOperator{\grad}{grad}           % Gradient
\DeclareMathOperator{\Ima}{Im}              % Image.
\DeclareMathOperator{\Int}{Int}             % Interior.
\DeclareMathOperator{\LC}{LC}               % Leading coefficient.
\DeclareMathOperator{\LCM}{LCM}             % Least common multiple.
\DeclareMathOperator{\LM}{LM}               % Leading monomial.
\DeclareMathOperator{\LT}{LT}               % Leading term.
\DeclareMathOperator{\Mod}{mod}             % Modulus.
\DeclareMathOperator{\Mon}{Mon}             % Monomial.
\DeclareMathOperator{\multideg}{mutlideg}   % Multi-Degree (Graphs).
\DeclareMathOperator{\nul}{nul}             % Null space of operator.
\DeclareMathOperator{\Ord}{Ord}             % Ordinal of ordered set.
\DeclareMathOperator{\Prin}{Prin}           % Principal value.
\DeclareMathOperator{\proj}{proj}           % Projection.
\DeclareMathOperator{\Refl}{Refl}           % Reflection operator.
\DeclareMathOperator{\rk}{rk}               % Rank of operator.
\DeclareMathOperator{\sgn}{sgn}             % Sign of a number.
\DeclareMathOperator{\sinc}{sinc}           % Sinc function.
\DeclareMathOperator{\Span}{Span}           % Span of a set.
\DeclareMathOperator{\Spec}{Spec}           % Spectrum.
\DeclareMathOperator{\supp}{supp}           % Support
\DeclareMathOperator{\Tr}{Tr}               % Trace of matrix.
%--------------------Declared Math Symbols--------------------%
\DeclareMathSymbol{\minus}{\mathbin}{AMSa}{"39} % Unary minus sign.
%------------------------New Commands---------------------------%
\DeclarePairedDelimiter\norm{\lVert}{\rVert}
\DeclarePairedDelimiter\ceil{\lceil}{\rceil}
\DeclarePairedDelimiter\floor{\lfloor}{\rfloor}
\newcommand*\diff{\mathop{}\!\mathrm{d}}
\newcommand*\Diff[1]{\mathop{}\!\mathrm{d^#1}}
\renewcommand*{\glstextformat}[1]{\textcolor{RoyalBlue}{#1}}
\renewcommand{\glsnamefont}[1]{\textbf{#1}}
\renewcommand\labelitemii{$\circ$}
\renewcommand\thesubfigure{%
    \arabic{chapter}.\arabic{figure}.\arabic{subfigure}}
\addto\captionsenglish{\renewcommand{\figurename}{Fig.}}
\numberwithin{equation}{section}

\renewcommand{\vector}[1]{\boldsymbol{\mathrm{#1}}}

\newcommand{\uvector}[1]{\boldsymbol{\hat{\mathrm{#1}}}}
\newcommand{\topspace}[2][]{(#2,\tau_{#1})}
\newcommand{\measurespace}[2][]{(#2,\varSigma_{#1},\mu_{#1})}
\newcommand{\measurablespace}[2][]{(#2,\varSigma_{#1})}
\newcommand{\manifold}[2][]{(#2,\tau_{#1},\mathcal{A}_{#1})}
\newcommand{\tanspace}[2]{T_{#1}{#2}}
\newcommand{\cotanspace}[2]{T_{#1}^{*}{#2}}
\newcommand{\Ckspace}[3][\mathbb{R}]{C^{#2}(#3,#1)}
\newcommand{\funcspace}[2][\mathbb{R}]{\mathcal{F}(#2,#1)}
\newcommand{\smoothvecf}[1]{\mathfrak{X}(#1)}
\newcommand{\smoothonef}[1]{\mathfrak{X}^{*}(#1)}
\newcommand{\bracket}[2]{[#1,#2]}

%------------------------Book Command---------------------------%
\makeatletter
\renewcommand\@pnumwidth{1cm}
\newcounter{book}
\renewcommand\thebook{\@Roman\c@book}
\newcommand\book{%
    \if@openright
        \cleardoublepage
    \else
        \clearpage
    \fi
    \thispagestyle{plain}%
    \if@twocolumn
        \onecolumn
        \@tempswatrue
    \else
        \@tempswafalse
    \fi
    \null\vfil
    \secdef\@book\@sbook
}
\def\@book[#1]#2{%
    \refstepcounter{book}
    \addcontentsline{toc}{book}{\bookname\ \thebook:\hspace{1em}#1}
    \markboth{}{}
    {\centering
     \interlinepenalty\@M
     \normalfont
     \huge\bfseries\bookname\nobreakspace\thebook
     \par
     \vskip 20\p@
     \Huge\bfseries#2\par}%
    \@endbook}
\def\@sbook#1{%
    {\centering
     \interlinepenalty \@M
     \normalfont
     \Huge\bfseries#1\par}%
    \@endbook}
\def\@endbook{
    \vfil\newpage
        \if@twoside
            \if@openright
                \null
                \thispagestyle{empty}%
                \newpage
            \fi
        \fi
        \if@tempswa
            \twocolumn
        \fi
}
\newcommand*\l@book[2]{%
    \ifnum\c@tocdepth >-3\relax
        \addpenalty{-\@highpenalty}%
        \addvspace{2.25em\@plus\p@}%
        \setlength\@tempdima{3em}%
        \begingroup
            \parindent\z@\rightskip\@pnumwidth
            \parfillskip -\@pnumwidth
            {
                \leavevmode
                \Large\bfseries#1\hfill\hb@xt@\@pnumwidth{\hss#2}
            }
            \par
            \nobreak
            \global\@nobreaktrue
            \everypar{\global\@nobreakfalse\everypar{}}%
        \endgroup
    \fi}
\newcommand\bookname{Book}
\renewcommand{\thebook}{\texorpdfstring{\Numberstring{book}}{book}}
\providecommand*{\toclevel@book}{-2}
\makeatother
\titleformat{\part}[display]
    {\Large\bfseries}
    {\partname\nobreakspace\thepart}
    {0mm}
    {\Huge\bfseries}
\titlecontents{part}[0pt]
    {\large\bfseries}
    {\partname\ \thecontentslabel: \quad}
    {}
    {\hfill\contentspage}
\titlecontents{chapter}[0pt]
    {\bfseries}
    {\chaptername\ \thecontentslabel:\quad}
    {}
    {\hfill\contentspage}
\newglossarystyle{longpara}{%
    \setglossarystyle{long}%
    \renewenvironment{theglossary}{%
        \begin{longtable}[l]{{p{0.25\hsize}p{0.65\hsize}}}
    }{\end{longtable}}%
    \renewcommand{\glossentry}[2]{%
        \glstarget{##1}{\glossentryname{##1}}%
        &\glossentrydesc{##1}{~##2.}
        \tabularnewline%
        \tabularnewline
    }%
}
\newglossary[not-glg]{notation}{not-gls}{not-glo}{Notation}
\newcommand*{\newnotation}[4][]{%
    \newglossaryentry{#2}{type=notation, name={\textbf{#3}, },
                          text={#4}, description={#4},#1}%
}
%--------------------------LENGTHS------------------------------%
% Spacings for the Table of Contents.
\addtolength{\cftsecnumwidth}{1ex}
\addtolength{\cftsubsecindent}{1ex}
\addtolength{\cftsubsecnumwidth}{1ex}
\addtolength{\cftfignumwidth}{1ex}
\addtolength{\cfttabnumwidth}{1ex}

% Indent and paragraph spacing.
\setlength{\parindent}{0em}
\setlength{\parskip}{0em}                                                           %
%----------------------------Main Document-------------------------------------%
\begin{document}
    \title{Differential Topology}
    \author{Ryan Maguire}
    \date{\vspace{-5ex}}
    \maketitle
    \section{Preliminary Stuff}
        \begin{fdefinition}{Sequentially Continuous}{Sequentially_Continuous}
            A sequentially continuous function from a topological space
            $(X,\tau_{X})$ to a topological space $(Y,\tau_{Y})$ is a
            function $f:X\rightarrow{Y}$ such that for every sequence
            $a:\mathbb{N}\rightarrow{X}$ such that there exists an $x\in{X}$
            such that $a_{n}\rightarrow{x}$, it is true that
            $f(a_{n})\rightarrow{f}(x)$.
        \end{fdefinition}
        \begin{fdefinition}{Sequential Space}{Sequential_Space}
            A sequential topological space is a topological space
            $\topspace[X]{X}$ such that for every topological space
            $\topspace[Y]{Y}$ and for every sequentially continuous function
            $f:X\rightarrow{Y}$, it is true that $f$ is continuous.
        \end{fdefinition}
        Continuity implies sequential continuity, the converse need not always
        hold. If $(X,\tau)$ is first countable, the result is true.
        \begin{theorem}
            If $(X,\tau)$ is a first countable topological space, if
            $\mathcal{U}\subseteq{X}$, then $\mathcal{U}$ is open if and only if
            for every sequence $a:\mathbb{N}\rightarrow{X}$ such that there is
            an $x\in\mathcal{U}$ such that $a_{n}\rightarrow{x}$, then there
            exists an $N\in\mathbb{N}$ such that for all $n>N$, it is true that
            $a_{n}\in\mathcal{U}$.
        \end{theorem}
        \begin{proof}
            One direction is the definition of convergence. In the other, since
            $(X,\tau)$ is first countable, for all $x\in\mathcal{U}$ there is a
            countable neighborhood basis $\mathcal{B}_{x}$. Let
            $\mathcal{V}_{x}:\mathbb{N}\rightarrow\mathcal{B}_{x}$ be a
            bijection. Then there exists an $N\in\mathbb{N}$ such that
            $\mathcal{V}_{x,N}\subseteq\mathcal{U}$. For suppose not. Let
            $B_{n}$ be defined by:
            \begin{equation}
                B_{n}=\bigcap_{k\in\mathbb{Z}_{n}}\mathcal{V}_{x,k}
            \end{equation}
            Since $B_{n}$ is the intersection of finitely many open sets, it is
            open. Moreover it is non-empty since $x\in{B}_{n}$ for all $n$.
            But then $B_{n}$ is an open neighborhood about $x$, and since
            $\mathcal{B}_{x}$ is a neighborhood basis there is an element
            $\mathcal{V}_{x,N}$ such that $\mathcal{V}_{x,N}\subseteq{B}_{n}$.
            But by hypothesis, for any such set there is a
            $y_{n}\in\mathcal{V}_{x,N}$ such that $y_{n}\notin\mathcal{U}$.
            Thus, for all $n$ there is a $y_{n}\in{B}_{n}$ such that
            $y_{n}\notin\mathcal{U}$. Then $y_{n}\rightarrow{x}$ since for any
            open subset about $x$ there is an $N\in\mathbb{N}$ such that
            $\mathcal{V}_{x,N}$ sits inside this open set. But for all $n>N$,
            $y_{n}\notin\mathcal{U}$. Thus $y_{n}$ is a sequence that
            converges to $x$, but is never contained inside $\mathcal{U}$,
            a contradiction. Thus, for all $x\in\mathcal{U}$ there is an open
            subset $\mathcal{V}_{x,N}$ such that $x\in\mathcal{V}_{x,N}$ and
            $\mathcal{V}_{x,N}\subseteq\mathcal{U}$. But then $\mathcal{U}$ is
            simply the union over all of these open sets, and is thus open.
        \end{proof}
        \begin{theorem}
            If $(X,\tau)$ is a first countable topological space, then it is
            a sequential space.
        \end{theorem}
        \begin{proof}
            For let $(Y,\tau_{Y})$ be a topological space, and let
            $f:X\rightarrow{Y}$ be a sequentially continuous function. Let
            $\mathcal{V}\in\tau_{Y}$ be an open subset of $Y$. If
            $f^{\minus{1}}[\mathcal{V}]=\emptyset$, we are done. If not, let
            $x\in{f}^{\minus{1}}[\mathcal{V}]$ and let
            $a:\mathbb{N}\rightarrow{X}$ be a sequence such that
            $a_{n}\rightarrow{x}$. But $f$ is sequentially continuous, and thus
            $f(a_{n})\rightarrow{f}(x)$. But since $\mathcal{V}$ is open, there
            is an $N\in\mathbb{N}$ such that for all $n>N$ it is true that
            $f(a_{n})\in\mathcal{V}$. But then for all $n>N$ it is true that
            $a_{n}\in{f}^{\minus{1}}[\mathcal{V}]$. Thus every sequence that
            converges to a point in $f^{\minus{1}}[\mathcal{V}]$ is eventually
            contained in $f^{\minus{1}}[\mathcal{V}]$, and thus by the
            previous theorem this set is open. Thus the pre-image of open is
            open, and hence $f$ is continuous.
        \end{proof}
        \begin{fdefinition}{$\sigma$ Compact}{Sigma_Compact}
            A $\sigma$ compact topological space is a topological space
            $\topspace{X}$ such that there exists a sequence
            $K:\mathbb{N}\rightarrow\powset{X}$ of compact subsets of $X$ such
            that:
            \begin{equation*}
                X=\bigcup{K}_{n}
            \end{equation*}
        \end{fdefinition}
        \begin{fdefinition}{Locally Compact Topological Space}{Locally_Compact}
            A locally compact topological space is a topological space
            $\topspace{X}$ such that for all $x\in{X}$ there exists a compact
            subset $K\subseteq{X}$ and an open set $\mathcal{U}\in\tau$ such
            that $x\in\mathcal{U}$ and $\mathcal{U}\subseteq{K}$.
        \end{fdefinition}
        \begin{fdefinition}{Lindel\"{o}f Topology Space}{Lindelof_Space}
            A Lindel\"{o}f topological space is a topological space $(X,\tau)$
            such that for every open cover $\mathcal{O}$ of $X$ there exists a
            countable subcover.
        \end{fdefinition}
        \begin{theorem}
            If $\topspace{X}$ is a $\sigma$ compact topological space, then it
            is Lindel\"{o}f.
        \end{theorem}
        \begin{proof}
            For suppose not. Then there exists an open cover $\mathcal{O}$ of
            $X$ with no countable subcover. But $X$ is $\sigma$ compact, and
            hence there is a sequence $K:\mathbb{N}\rightarrow\powset{X}$ of
            compact sets such that $X=\bigcup{X}_{n}$. But the for all
            $n\in\mathbb{N}$, $K_{n}$ is a subset of $X$, and hence
            $\mathcal{O}$ is an open cover of $K_{n}$. But by hypothesis,
            $K_{n}$ is compact and hence there is a finite subcover
            $\Delta\subseteq\mathcal{O}$ of $K_{n}$. That is, if we define the
            sequence $A_{n}:\mathbb{N}\rightarrow\powset{\mathcal{O}}$ by:
            \begin{equation}
                A_{n}=\{\mathscr{D}\in\powset{\mathcal{O}}\;|\;
                    \mathscr{D}\textrm{ is finite and }
                    K_{n}\subseteq\bigcup\mathscr{D}\,\}
            \end{equation}
            Then for all $n\in\mathbb{N}$, $A_{n}$ is non-empty. Hence by the
            axiom of choice, there is a choice function
            $\Delta:\mathbb{N}\rightarrow\powset{\mathcal{O}}$ such that for all
            $n\in\mathbb{N}$, $\Delta_{n}$ is a finite open subcover of $K_{n}$.
            But then the union of all $\Delta_{n}$ is the countable union of
            finite collections, and is hence countable. But then this collection
            covers $K_{n}$ for all $n\in\mathbb{N}$, and hence covers
            $\bigcup{K}_{n}$. But $\bigcup{K}_{n}=X$, a contradiction since
            $\mathcal{O}$ has no countable subcover. Thus, $\topspace{X}$ is
            Lindel\"{o}f.
        \end{proof}
        This theorem reverses if we add locally compact. The requirement of
        local compactness can be seen by examining the irrational numbers with
        the subspace topology. This is Lindel\"{o}f (since it is second
        countable), but it is not $\sigma$ compact.
        \begin{theorem}
            If $\topspace{X}$ is a locally compact Lindel\"{o}f space, then it
            is $\sigma$ compact.
        \end{theorem}
        \begin{proof}
            For if $\topspace{X}$ is locally compact, then for all $x\in{X}$
            there is a compact set $K\subseteq{X}$ and an open set
            $\mathcal{U}\in\tau$ such that $x\in\mathcal{U}$ and
            $\mathcal{U}\subseteq{K}$. Invoking the axiom of choice, we get a
            function $A:X\rightarrow\tau\times\powset{X}$ such that for all
            $x$, $A_{x}=(\mathcal{U},K)$ where $x\in\mathcal{U}$ and
            $\mathcal{U}\subseteq{K}$, where $K$ is compact. But then the
            collection of all such $\mathcal{U}$ is an open cover of $X$. But
            $X$ is Lindel\"{o}f and hence there is a countable subcover. But
            then the subcollection of all $K$ form a countable collection of
            compact sets that cover $X$. Hence, $\topspace{X}$ is $\sigma$
            compact.
        \end{proof}
        \begin{fdefinition}{Sequentially Compact}{Sequentially_Compact}
            A sequentially compact topological space is a topological space
            $\topspace{X}$ such that for every sequence
            $a:\mathbb{N}\rightarrow{X}$ there exists a strictly increasing
            sequence $k:\mathbb{N}\rightarrow\mathbb{N}$ such that
            $a\circ{k}:\mathbb{N}\rightarrow{X}$ converges.
        \end{fdefinition}
        \begin{theorem}
            If $\topspace{X}$ is metrizable, and if $\mathcal{C}\subseteq{X}$,
            then $\mathcal{C}$ is compact if and only if it is sequentially
            compact.
        \end{theorem}
        This is not true in general. The long ling is sequentially compact but
        not compact, whereas the space of all functions
        $f:[0,1]\rightarrow[0,1]$ is compact (by Tychonoff) but not sequentially
        compact.
        \begin{fdefinition}{$\sigma$ locally finite basis}
            A $\sigma$ locally finite basis of a topological space
            $\topspace{X}$ is a basis $\mathcal{B}$ of $X$ such that
            $\mathcal{B}$ is $\sigma$ locally finite. That is, for all $x\in{X}$
            there is an open set $\mathcal{U}\in\tau$ such that only countably
            many elements of $\mathcal{B}$ have non-empty intersection with
            $\mathcal{U}$.
        \end{fdefinition}
        \begin{theorem}
            \label{thm:Second_Countable_Implies_Sigma_Loc_Fin_Basis}%
            If $\topspace{X}$ is second countable, then it is has a $\sigma$
            locally finite.
        \end{theorem}
        \begin{proof}
            For if $\topspace{X}$ is has a countable basis $\mathcal{B}$, and
            thus every open subset meats $\mathcal{B}$ at mostly countably many
            times. Hence, given $x\in{X}$, pick the whole space $X\in\tau$.
            Then $x\in{X}$, and $X$ intersects with only countably many elements
            of $\mathcal{B}$ (it intersects with all of them, but $\mathcal{B}$
            is countable).
        \end{proof}
        \begin{ftheorem}{Nagata-Smirnov Metrization Theorem}
                        {Nagata_Smirnov_Metrization_Theorem}
            If $\topspace{X}$ is a topological space, then it is metrizable if
            and only if it is regular, Hausdorff, and has a $\sigma$ locally
            finite basis.
        \end{ftheorem}
        \begin{proof}
            Munkres.
        \end{proof}
        \begin{ftheorem}{Smirnov Metrization Theorem}
                        {Smirnov_Metrization_Theorem}
            If $\topspace{X}$ is a topological space, then it is metrizable if
            and only if it is Hausdorff, paracompact, and locally metrizable.
        \end{ftheorem}
        \begin{proof}
            Also Munkres.
        \end{proof}
        \begin{theorem}
            If $\topspace{X}$ is a topological space, if
            $\mathcal{O}\subseteq\tau$ is a countable subset of open sets such
            that for all $\mathcal{U}\in\mathcal{O}$,
            $\topspace[\mathcal{U}]{\mathcal{U}}$ is second countable, where
            $\tau_{\mathcal{U}}$ is the subspace topology, then $\topspace{X}$
            is second countable.
        \end{theorem}
        \begin{proof}
            For let $\mathcal{B}$ be the collection of all of the basis for all
            of the $\mathcal{U}\in\mathcal{O}$. Since it is the countable union
            of countable sets, it is countable. But since $\mathcal{U}$ is open
            for all $\mathcal{U}\in\mathcal{O}$, this is a countable open cover
            of $X$. It suffices to show that it is a basis. Let
            $\mathcal{V}\in\tau$ be an open subset. But then:
            \begin{align}
                \mathcal{V}&=\mathcal{V}\cap{X}\\
                &=\mathcal{V}\cap\Big(
                    \bigcup_{\mathcal{U}\in\mathcal{O}}\mathcal{U}
                \Big)\\
                &=\bigcup_{\mathcal{U}\in\mathcal{O}}
                    \big(\mathcal{V}\cap\mathcal{U}\big)
            \end{align}
            But $\mathcal{V}\cap\mathcal{U}$ is an open subset of the subspace
            $\topspace[\mathcal{U}]{\mathcal{U}}$, and hence there is a subset
            of $\Delta_{\mathcal{V}}\subseteq\mathcal{B}_{\mathcal{U}}$ such
            that $\mathcal{V}\cap\mathcal{U}=\bigcup\Delta_{\mathcal{V}}$. But
            then the entire of $\mathcal{V}$ is the union of all such
            collections, each of which is contained in $\mathcal{B}$, and hence
            $\mathcal{V}$ can be written as the union of elements of
            $\mathcal{B}$. Thus, $\mathcal{B}$ is a basis.
        \end{proof}
        The requirement that the covering collection be open subspaces is
        crucial. The quotient space $\mathbb{R}/R$, where $R$ is the equivalence
        relation generated by $nRm$ for all $n,m\in\mathbb{N}$, can be thought
        of as a countable collection of rings all glued together at the origin.
        Hence, it can be covered by countably many closed subspaces, each of
        which is homeomorphic to $\nsphere[1]$ in the subspace topology, and
        hence each of which is second countable. However, this space
        $\mathbb{R}/R$ is not even first countable, let alone second countable.
        The point $[0]\in\mathbb{R}/R$ has no countable neighborhood basis.
    \section{Problems}
        \begin{problem}
            Show that if $\topspace{X}$ is a locally Euclidean connected
            Hausdorff space, then it is a manifold if and only if it is
            paracompact.
        \end{problem}
        \begin{solution}
            If $\topspace{X}$ is a topological manifold, then it is locally
            Euclidean, Hausdorff, and second countable. But if $X$ is locally
            Euclidean, then it is locally metrizable. But then $\topspace{X}$ is
            a locally metrizable Hausdorff space that is second countable,
            and second countable spaces have a $\sigma$ locally finite basis
            (Thm.~\ref{thm:Second_Countable_Implies_Sigma_Loc_Fin_Basis}).
            But then $X$ is metrizable by the Nagata-Smirnov theorem
            (Thm.~\ref{thm:Nagata_Smirnov_Metrization_Theorem}). But if
            $X$ is metrizable, then it is paracompact by the
            Smirnov theorem (Thm.~\ref{thm:Smirnov_Metrization_Theorem}). Hence,
            $\topspace{X}$ is paracompact. Going the other way, if $X$ is
            locally Euclidean, Hausdorff, paracompact, and connected, to show
            that it is a manifold suffices to show that it is second countable.
            But if $X$ is locally Euclidean, then there is a basis of
            precompact open subsets $\mathcal{B}$. But $X$ is paracompact, and
            hence there is a locally finite refinement $\Delta$ of
            $\mathcal{B}$. Let $\mathcal{U}_{0}$ be an element of $\Delta$.
            Since $\Delta$ is a refinement of $\mathcal{B}$, then is an element
            $\mathcal{V}\in\mathcal{B}$ such that
            $\mathcal{U}_{0}\subseteq\mathcal{V}$. But then
            $\closure{\mathcal{U}_{0}}\subseteq\closure{\mathcal{V}}$, and
            and $\mathcal{V}$ is precompact, and thus $\closure{\mathcal{V}}$
            is compact. But then $\closure{\mathcal{U}_{0}}$ is a closed subset
            of a compact set, and is hence compact. For all $n\in\mathbb{N}$,
            let $\mathcal{U}_{n}$ be defined as follows:
            \begin{equation}
                \mathcal{U}_{n}=\Big\{\,x\in{X}\;|\;
                    \exists_{A:\mathbb{Z}_{n}\rightarrow\Delta}\big(
                        \mathcal{U}_{0}=A_{0},\,x\in{A}_{n-1}\textrm{ and }
                        \forall_{i<n-1}(A_{i}\cap{A}_{i+1}\ne\emptyset)
                    \big)\,\Big\}
            \end{equation}
            That is, the set of all points that are separated from
            $\mathcal{U}_{0}$ by at most $n$ consecutive elements of $\Delta$.
            Then $\mathcal{U}_{n}$ is precompact. We prove by induction. The
            base case of $\mathcal{U}_{0}$ is true from the previous paragraph.
            Suppose it is true for $n\in\mathbb{N}$. Since $X$ is locally
            metrizable, Hausdorff, and paracompact, it is metrizable
            (Thm.~\ref{thm:Smirnov_Metrization_Theorem}). But then if
            $\mathcal{C}\subseteq{X}$, then $\mathcal{C}$ is compact if and only
            if it is sequentially compact. But if $\mathcal{U}_{n+1}$ is not
            precompact, then there is an open cover such that
            $\closure{\mathcal{U}_{n+1}}$ has not finite subcover. But
            $\mathcal{U}_{n+1}$ is the union of $\closure{\mathcal{U}_{n}}$ and
            elements of $\mathcal{V}\in\Delta$ such that
            $\mathcal{V}\cap\mathcal{U}_{n}\ne\emptyset$. Since
            $\closure{\mathcal{U}_{n}}$ is not compact, there must be infinitely
            many such $\mathcal{V}$. But $\mathcal{V}\cap\mathcal{U}_{n}$ is non
            empty for all such $\mathcal{V}$, and hence by the axiom of choice
            there is a sequence $a:\mathbb{N}\rightarrow\mathcal{U}_{n}$ such
            that $a_{n}$ lies in a distinct $\mathcal{V}$ for all
            $n\in\mathbb{N}$. But $\closure{\mathcal{U}_{n}}$ is compact, and
            hence sequentially compact, and thus there is a convergent
            subsequence $a_{k}:\mathbb{N}\rightarrow\closure{\mathcal{U}_{n}}$
            with a limit $x$. But $x\in\closure{\mathcal{U}_{n}}$ and hence
            there is an open set $\mathcal{V}\in\tau$ that fits inside finitely
            many of the elements of $\Delta$, since $\Delta$ is a locally finite
            refinement. But then $a_{k_{n}}$ is eventually contained in
            $\mathcal{V}$, contradicting that each $a_{k_{n}}$ is in a different
            element of $\Delta$. Hence, $\mathcal{U}_{n+1}$ is covered by
            finitely many elements of $\Delta$ and is therefore precompact.
            Moreover, $\bigcup\closure{\mathcal{U}_{n}}=X$. For if
            $y\in{X}$, let $x\in\mathcal{U}_{0}$ be any point. Then since
            $X$ is locally path connected and connected, it is path connected.
            Let $\gamma:[0,1]\rightarrow{X}$ be a path from $x$ to $y$. But
            $[0,1]$ is compact and hence $\gamma\big[[0,1]\big]$ is a compact
            subset of $X$. But then it is covered by only finitely many elements
            of $\Delta$, and hence $y$ is contained in one of the
            $\mathcal{U}_{n}$. BUt this shows that $X$ is $\sigma$ compact. But
            if $X$ is $\sigma$ compact, then it is Lindel\"{o}f. But the
            collection of all precompact coordinate balls about every point
            forms a cover, and hence there is a countable subcover. But then
            $X$ is the union of countably many open subspaces, each of which
            is second countable, and hence $X$ is second countable. Therefore,
            $X$ is a manifold.
        \end{solution}
\end{document}