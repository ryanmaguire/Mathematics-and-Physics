%------------------------------------------------------------------------------%
\documentclass{article}                                                        %
%------------------------------Preamble----------------------------------------%
\makeatletter                                                                  %
    \def\input@path{{../../}}                                                  %
\makeatother                                                                   %
%---------------------------Packages----------------------------%
\usepackage{geometry}
\geometry{b5paper, margin=1.0in}
\usepackage[T1]{fontenc}
\usepackage{graphicx, float}            % Graphics/Images.
\usepackage{natbib}                     % For bibliographies.
\bibliographystyle{agsm}                % Bibliography style.
\usepackage[french, english]{babel}     % Language typesetting.
\usepackage[dvipsnames]{xcolor}         % Color names.
\usepackage{listings, lstlinebgrd}      % Verbatim-Like Tools.
\usepackage{mathtools, esint, mathrsfs} % amsmath and integrals.
\usepackage{amsthm, amsfonts}           % Fonts and theorems.
\usepackage{tabularx}
\usepackage{tcolorbox}                  % Frames around theorems.
\usepackage{upgreek}                    % Non-Italic Greek.
\usepackage{paracol}                    % Two-column styling.
\usepackage{wrapfig}                    % Wrap text around figure.
\usepackage{fmtcount, etoolbox}         % For the \book{} command.
\usepackage[newparttoc]{titlesec}       % Formatting chapter, etc.
\usepackage{titletoc}                   % Allows \book in toc.
\usepackage[nottoc]{tocbibind}          % Bibliography in toc.
\usepackage[titles]{tocloft}            % ToC formatting.
\usepackage{multicol, enumitem}         % Multi-column/enumerate.
\usepackage{import}                     % Import external files.
\usepackage{pgfplots, tikz}             % Drawing/graphing tools.
\usetikzlibrary{
    calc,                   % Calculating right angles and more.
    angles,                 % Drawing angles within triangles.
    arrows.meta,            % Latex and Stealth arrows.
    quotes,                 % Adding labels to angles.
    positioning,            % Relative positioning of nodes.
    decorations.markings,   % Adding arrows in the middle of a line.
    patterns,
    arrows,
    shapes,
    shapes.geometric,
    cd,
    hobby,
    babel
}                                       % Libraries for tikz.
\pgfplotsset{compat=1.9}                % Version of pgfplots.
\usepackage[font=scriptsize,
            labelformat=simple,
            labelsep=colon]{subcaption} % Subfigure captions.
\usepackage[font={scriptsize},
            hypcap=true,
            labelsep=colon]{caption}    % Figure captions.
\usepackage{hyperref}                   % Allows for hyperlinks.
\hypersetup{
    colorlinks=true,
    linkcolor=blue,
    filecolor=magenta,
    urlcolor=Cerulean,
    citecolor=SkyBlue
}                           % Colors for hyperref.
\usepackage[toc,acronym,nogroupskip]{glossaries} % Glossaries and acronyms.
\usepackage[subpreambles=false]{standalone}      % Complileable sub files.

% Various font stuff from kiwi.
% Use this for Times text and Computer Modern math
%\usepackage{times}

% Quite nice
%\usepackage[charter, greekfamily=, greekuppercase=italicized]{mathdesign}
%\usepackage[utopia, greekuppercase=italicized]{mathdesign}    % Math is narrower

% Use this for Times text and math
%\usepackage{newtxtext}
%\usepackage[libertine,cmintegrals]{newtxmath}
%\usepackage{fix-cm}

%\usepackage{txfontsb}
% or
%\usepackage{mathptmx}

%\usepackage[scaled=0.92]{helvet}
%\renewcommand{\rmdefault}{ptm}

%\usepackage{mathpazo}    % add possibly `sc` and `osf` options
%\usepackage{eulervm}

%\usepackage{fourier}
%\renewcommand{\rmdefault}{ptm}
%\usepackage{mathptm}

%\usepackage{fontspec}
%\setmainfont{lmodern}

%\usepackage[varg]{txfonts}
%\usepackage{fouriernc}
%\usepackage{mathpazo}

%\usepackage{bookman}
%\usepackage[scaled]{uarial}
%\usepackage[scaled]{helvet}
%\renewcommand*\familydefault{\sfdefault}
%\usepackage[math]{anttor}

%\newcommand\fgeorgia{\fontfamily{jvn}\selectfont}
%\newcommand\ftimes{\fontfamily{ptm}\selectfont}
%\newcommand\fhelvetica{\fontfamily{phv}\selectfont}
%\newcommand\fcourier{\fontfamily{pcr}\selectfont}
%\newcommand\fbookman{\fontfamily{pbk}\selectfont}
%\newcommand\fnewcentury{\fontfamily{pnc}\selectfont}
%\newcommand\fpalatino{\fontfamily{ppl}\selectfont}
%\newcommand\favantgarde{\fontfamily{pag}\selectfont}
%\newcommand\fnormal{\normalfont}
%\newcommand\fsize[1]{\ifnum#1>0\fontsize{#1}{#1}\selectfont\else\normalsize\fi}
%------------------------Theorem Styles-------------------------%
% Define theorem style for default spacing and normal font.
\newtheoremstyle{normal}
    {\topsep}               % Amount of space above the theorem.
    {\topsep}               % Amount of space below the theorem.
    {}                      % Font used for body of theorem.
    {}                      % Measure of space to indent.
    {\bfseries}             % Font of the header of the theorem.
    {}                      % Punctuation between head and body.
    {.5em}                  % Space after theorem head.
    {}

% Define theorem style for default spacing with italicized font.
\newtheoremstyle{normalit}{\topsep}{\topsep}
                {\itshape}{}{\bfseries}{}{.5em}{}

% Italic header environment.
\newtheoremstyle{thmit}{\topsep}{\topsep}{}{}{\itshape}{}{0.5em}{}

% Define italicized environments.
\theoremstyle{normalit}
\newtheorem{theorem}{Theorem}[section]
\newtheorem{lemma}{Lemma}[section]
\newtheorem{corollary}{Corollary}[section]
\newtheorem{proposition}{Proposition}[section]
\newtheorem*{theorem*}{Theorem}

% Define environments with italic headers.
\theoremstyle{thmit}
\newtheorem*{solution}{Solution}
\newtheorem*{fsolution}{Solution}

% Define default environments.
\theoremstyle{normal}
\newtheorem{example}{Example}[section]
\newtheorem{definition}{Definition}[section]
\newtheorem{problem}{Problem}[section]
\newtheorem{question}{Question}[section]
\newtheorem{remark}{Remark}[section]
\newtheorem{properties}{Properties}[section]
\newtheorem{notation}{Notation}[section]
\newtheorem{axiom}{Axiom}[section]
\newtheorem*{properties*}{Properties}
\newtheorem*{remark*}{Remark}
\newtheorem*{definition*}{Definition}
\theoremstyle{plain}

% Define framed environment.
\tcbuselibrary{most}
\newtcbtheorem[use counter*=theorem]{ftheorem}{Theorem}%
    {colback=green!5,colframe=green!35!black,
     fonttitle=\bfseries\upshape}{th}

\newtcbtheorem[use counter*=example]{fdefinition}{Definition}%
    {fonttitle=\bfseries\upshape,
     colback=blue!5!white,colframe=blue!75!black}{def}

\newtcbtheorem[use counter*=example]{fexample}{Example}%
    {fonttitle=\bfseries\upshape,
     colback=red!5!white,colframe=red!75!black}{ex}

\newtcbtheorem[use counter*=notation]{fnotation}{Notation}%
    {fonttitle=\bfseries\upshape,
     colback=SeaGreen!5!white,colframe=SeaGreen!75!black}{ex}

\newtcbtheorem[use counter*=corollary]{fcorollary}{Corollary}%
    {fonttitle=\bfseries\upshape,
     colback=Orchid!5!white,colframe=Orchid!75!black}{ex}

\newenvironment{bproof}{\textit{Proof.}}{\hfill$\square$}
\tcolorboxenvironment{bproof}{blanker,breakable,left=5mm,
                             before skip=10pt,after skip=10pt,
                             borderline west={1mm}{0pt}{red}}
\tcolorboxenvironment{fsolution}
    {enhanced jigsaw,colframe=cyan,interior hidden,breakable}

%--------------------Declared Math Operators--------------------%
\DeclareMathOperator{\Refl}{Refl}           % Reflection operator.
\DeclareMathOperator{\Span}{Span}           % Span of a set of vectors.
\DeclareMathOperator{\Card}{Card}           % Cardinality of set.
\DeclareMathOperator{\Ord}{Ord}             % Ordinal of ordered set.
\DeclareMathOperator{\Tr}{Tr}               % Trace of matrix.
\DeclareMathOperator{\adjoint}{adj}         % Adjoint of matrix.
\DeclareMathOperator{\rk}{rk}               % Rank of operator.
\DeclareMathOperator{\nul}{nul}             % Null space of operator.
\DeclareMathOperator{\sgn}{sgn}             % Sign of a number.
\DeclareMathOperator{\multideg}{mutlideg}   % Multi-Degree (Graphs).
\DeclareMathOperator{\GCD}{GCD}             % Greatest common denominator.
\DeclareMathOperator{\LM}{LM}               % Leading monomial
\DeclareMathOperator{\LC}{LC}               % Leading coefficient.
\DeclareMathOperator{\LT}{LT}               % Leading term.
\DeclareMathOperator{\LCM}{LCM}             % Least common multiple.
\DeclareMathOperator{\Mon}{Mon}             % Monomial.
\DeclareMathOperator{\Spec}{Spec}           % Spectrum.
\DeclareMathOperator{\proj}{proj}           % Projection.
\DeclareMathOperator{\comp}{comp}           % Component.
\DeclareMathOperator{\sinc}{sinc}           % Sinc function.
\DeclareMathOperator{\Ima}{Im}              % Image of operator.
\DeclareMathOperator{\Prin}{Prin}           % Principal value.
\DeclareMathOperator{\Mod}{mod}             % Modulus.
%------------------------New Commands---------------------------%
\DeclarePairedDelimiter\norm{\lVert}{\rVert}
\DeclarePairedDelimiter\ceil{\lceil}{\rceil}
\DeclarePairedDelimiter\floor{\lfloor}{\rfloor}
\newcommand*\diff{\mathop{}\!\mathrm{d}}
\newcommand*\Diff[1]{\mathop{}\!\mathrm{d^#1}}
\renewcommand{\mod}{\ \Mod}
\renewcommand*{\glstextformat}[1]{\textcolor{RoyalBlue}{#1}}
\renewcommand{\glsnamefont}[1]{\textbf{#1}}
\renewcommand\labelitemii{$\circ$}
\renewcommand\thesubfigure{\arabic{chapter}.\arabic{figure}}
\renewcommand\thesubfigure{%
    \arabic{chapter}.\arabic{figure}.\arabic{subfigure}}
\addto\captionsenglish{\renewcommand{\figurename}{Fig.}}
%------------------------Book Command---------------------------%
\makeatletter
\renewcommand\@pnumwidth{1cm}
\newcounter{book}
\renewcommand\thebook{\@Roman\c@book}
\newcommand\book{%
    \if@openright
        \cleardoublepage
    \else
        \clearpage
    \fi
    \thispagestyle{plain}%
    \if@twocolumn
        \onecolumn
        \@tempswatrue
    \else
        \@tempswafalse
    \fi
    \null\vfil
    \secdef\@book\@sbook
}
\def\@book[#1]#2{%
    \ifnum \c@secnumdepth >-3\relax
        \refstepcounter{book}%
        \addcontentsline{toc}{book}{
            \bookname\ \thebook:\hspace{1em}#1
        }
    \else
        \addcontentsline{toc}{book}{#1}%
    \fi
    \markboth{}{}%
    {\centering
     \interlinepenalty \@M
     \normalfont
     \ifnum \c@secnumdepth >-2\relax
       \huge\bfseries \bookname\nobreakspace\thebook
       \par
       \vskip 20\p@
     \fi
     \Huge \bfseries #2\par}%
    \@endbook}
\def\@sbook#1{%
    {\centering
     \interlinepenalty \@M
     \normalfont
     \Huge \bfseries #1\par}%
    \@endbook}
\def\@endbook{
    \vfil\newpage
        \if@twoside
            \if@openright
                \null
                \thispagestyle{empty}%
                \newpage
            \fi
        \fi
        \if@tempswa
            \twocolumn
        \fi
}
\newcommand*\l@book[2]{%
    \ifnum \c@tocdepth >-2\relax
        \addpenalty{-\@highpenalty}%
        \addvspace{2.25em \@plus\p@}%
        \setlength\@tempdima{3em}%
        \begingroup
            \parindent \z@ \rightskip \@pnumwidth
            \parfillskip -\@pnumwidth
            {
                \leavevmode
                \Large \bfseries #1\hfil \hb@xt@\@pnumwidth{
                    \hss #2
                }
            }
            \par
            \nobreak
            \global\@nobreaktrue
            \everypar{\global\@nobreakfalse\everypar{}}%
        \endgroup
    \fi}
\newcommand\bookname{Book}
\renewcommand{\thebook}{\texorpdfstring{\Numberstring{book}}{book}}
\providecommand*{\toclevel@book}{-2}
\makeatother
\titlecontents{chapter}[0pt]
    {\bfseries}
    {\chaptername\ \thecontentslabel:\quad}
    {}
    {\hfill\contentspage}
\titleformat{\part}[display]
    {\Large\bfseries}
    {\partname\nobreakspace\thepart}
    {0mm}
    {\Huge\bfseries}
    \titlecontents{part}[0pt]
    {\large\bfseries}
    {\partname\ \thecontentslabel: \quad}
    {}
    {\hfill\contentspage}
\newcommand{\MarkRightAngle}[4][.3cm]
    {\coordinate (tempa) at ($(#3)!#1!(#2)$);
     \coordinate (tempb) at ($(#3)!#1!(#4)$);
     \coordinate (tempc) at ($(tempa)!0.5!(tempb)$);%midpoint
     \draw (tempa) -- ($(#3)!2!(tempc)$) -- (tempb);}
%--------------------------LENGTHS------------------------------%
% Spacings for the Table of Contents.
\addtolength{\cftsecnumwidth}{1ex}
\addtolength{\cftsubsecindent}{1ex}
\addtolength{\cftsubsecnumwidth}{1ex}
\addtolength{\cftfignumwidth}{1ex}
\addtolength{\cfttabnumwidth}{1ex}

% Spacing for multi-column and enumerate environments.
\setlength{\multicolsep}{6pt}
\setlist[enumerate]{itemsep=0pt,topsep=3pt}

% Indent and paragraph spacing.
\setlength{\parindent}{0em}
\setlength{\parskip}{0em}                                                           %
%----------------------------Main Document-------------------------------------%
\begin{document}
    \title{Differential Topology}
    \author{Ryan Maguire}
    \date{\vspace{-5ex}}
    \maketitle
    \tableofcontents
    \section{Preliminary Stuff}
        \begin{fdefinition}{Sequentially Continuous}{Sequentially_Continuous}
            A sequentially continuous function from a topological space
            $(X,\tau_{X})$ to a topological space $(Y,\tau_{Y})$ is a
            function $f:X\rightarrow{Y}$ such that for every convergent sequence
            $a:\mathbb{N}\rightarrow{X}$ and for every $x\in{X}$ such that
            $a_{n}\rightarrow{x}$, it is true that $f(a_{n})\rightarrow{f}(x)$.
        \end{fdefinition}
        We say \textit{for every x} since limits need not be unique in spaces
        that aren't at least $T_{1}$ (an \textit{accessible} space).
        \begin{theorem}
            \label{thm:cont_implies_seq_cont}%
            If $\topspace[X]{X}$ and $\topspace[Y]{Y}$ are topological spaces,
            and if $f:X\rightarrow{Y}$ is a continuous function, then it is
            sequentially continuous.
        \end{theorem}
        \begin{proof}
            For suppose not. Then there is a sequence
            $a:\mathbb{N}\rightarrow{X}$ and a point $x\in{X}$ such that
            $a_{n}\rightarrow{x}$, but $f(a_{n})\not\rightarrow{f}(x)$
            (Def.~\ref{def:Sequentially_Continuous}). But then there exists
            an open subset $\mathcal{V}\in\tau_{Y}$ such that
            $f(x)\in\mathcal{V}$, and for all $N\in\mathbb{N}$ there exists an
            $n\in\mathbb{N}$ such that $n>N$ and
            $f(a_{n})\notin\mathcal{V}$. But $f$ is continuous, and therefore
            $f^{\minus{1}}[\mathcal{V}]$ is an open subset of $X$. And since
            $f(x)\in\mathcal{V}$ it is true that
            $x\in{f}^{\minus{1}}[\mathcal{V}]$. But $a_{n}\rightarrow{x}$, and
            thus there is an $N\in\mathbb{N}$ such that for all
            $n\in\mathbb{N}$ with $n>N$ it is true that
            $a_{n}\in{f}^{\minus{1}}[\mathcal{V}]$. But then for all $n>N$ it is
            true that $f(a_{n})\in\mathcal{V}$, a contradiction. Therefore, $f$
            is sequentially continuous.
        \end{proof}
        This theorem does not reverse.
        \begin{example}
            Let $X$ be the order topology on the ordinal $\omega_{1}+1$, where
            $\omega_{1}$ is the first uncountable ordinal. Define
            $f:X\rightarrow\nspace[]$ by:
            \begin{equation}
                f(x)=
                \begin{cases}
                    0,&x\ne\omega_{1}\\
                    1,&x=\omega_{1}
                \end{cases}
            \end{equation}
            then $f$ is sequentially continuous, but not continuous. If
            $a:\mathbb{N}\rightarrow{X}$ is any sequence, either it coverges to
            $\omega_{1}$ or it is not. In the latter case the limit of
            $f(a_{n})$ will be zero, and in the former case there is an
            $N\in\mathbb{N}$ such that for all $n>N$ it is true that
            $a_{n}=\omega_{1}$, and hence $f(a_{n})\rightarrow{1}$. In both
            scenarios, $f$ is sequentially continuous. It is not continuous
            since $\{\omega_{1}\}$ is not an isolated point of $X$, yet
            $f^{\minus{1}}[(0,\infty)]=\{\omega_{1}\}$.
        \end{example}
        The problem with this space is that it is not first countable. First
        countability implies that sequentially continuous and continuous are
        identical concepts. There's a weaker such notion, called sequential
        spaces.
        \begin{fdefinition}{Sequentially Open}{Sequentially_Open}
            A sequentially open subset of a topological space $\topspace{X}$
            is a subset $\mathcal{U}\subseteq{X}$ such that for every sequence
            $a:\mathbb{N}\rightarrow{X}$ such that $a$ converges to an element
            $x\in\mathcal{U}$ there exists an $N\in\mathbb{N}$ such that for all
            $n\in\mathbb{N}$ with $n>N$, it is true that $a_{n}\in\mathcal{U}$.
        \end{fdefinition}
        \begin{theorem}
            \label{thm:Open_is_Seq_Open}%
            If $\topspace{X}$ is a topological space, if $\mathcal{U}\in\tau$ is
            an open subset of $X$, then $\mathcal{U}$ is sequentially open.
        \end{theorem}
        \begin{proof}
            For suppose not. Then there is a sequence
            $a:\mathbb{N}\rightarrow{X}$ such that $a$ converges to a point
            $x\in\mathcal{U}$, but for all $N\in\mathbb{N}$ there is an
            $n\in\mathbb{N}$ such that $n>N$ and $a_{n}\notin\mathcal{U}$
            (Def.~\ref{def:Sequentially_Open}). But if $\mathcal{U}$ is an open
            subset containing $x$, and if $a_{n}\rightarrow{x}$, then there is
            an $N\in\mathbb{N}$ such that for all $n>N$ it is true that
            $a_{n}\in\mathcal{U}$, a contradiction.
        \end{proof}
        \begin{fdefinition}{Sequential Space}{Sequential_Space}
            A sequential topological space is a topological space
            $\topspace{X}$ such that for every sequentially open subset
            $\mathcal{U}\subseteq{X}$ it is true that $\mathcal{U}\in\tau$.
        \end{fdefinition}
        As the name suggests, the usefulness of sequential spaces stems from the
        fact that continuity and sequential continuity are equivalent. One
        direction is true of any topological space
        (Thm.~\ref{thm:cont_implies_seq_cont}). We now prove the other direction
        in the setting of a sequential space.
        \begin{theorem}
            \label{thm:seq_space_seq_cont_eqiv_cont}%
            If $\topspace[X]{X}$ is a sequential topological space, if
            $\topspace[Y]{Y}$ is a topological space, and if $f:X\rightarrow{Y}$
            is a sequentially continuous function, then $f$ is continuous.
        \end{theorem}
        \begin{proof}
            For suppose not. Then there is a open subset
            $\mathcal{V}\in\tau_{Y}$ such that $f^{\minus{1}}[\mathcal{V}]$ is
            not open. But $\topspace[X]{X}$ is sequential, and thus if
            $f^{\minus{1}}[\mathcal{V}]$ is not open, then it is not
            sequentially open (Def.~\ref{def:Sequential_Space}). But then there
            is a sequence $a:\mathbb{N}\rightarrow{X}$ and a point
            $x\in{f}^{\minus{1}}[\mathcal{V}]$ such that $a_{n}\rightarrow{x}$
            but for all $N\in\mathbb{N}$ there is an $n\in\mathbb{N}$ with
            $n>N$ such that $a_{n}\notin{f}^{\minus{1}}[\mathcal{V}]$
            (Def.~\ref{def:Sequentially_Open}). But $f$ is sequentially
            continuous, and thus if $a_{n}\rightarrow{x}$, then
            $f(a_{n})\rightarrow{f}(x)$
            (Def.~\ref{def:Sequentially_Continuous}). But $\mathcal{V}$ is open
            and $f(x)\in\mathcal{V}$. Therefore if $f(a_{n})\rightarrow{f}(x)$,
            then there is an $N\in\mathbb{N}$ such that for all $n\in\mathbb{N}$
            with $n>N$, it is true that $f(a_{n})\in\mathcal{V}$. But then for
            all $n>N$ it is true that $a_{n}\in{f}^{\minus{1}}[\mathcal{V}]$, a
            contradiction. Therefore, $f$ is continuous.
        \end{proof}
        Two amazing equivalent definitions of sequential spaces that we will not
        prove since these will not be used.
        \begin{itemize}
            \item $\topspace{X}$ is sequential if and only if it is the quotient
                  of a first countable space.
            \item $\topspace{X}$ is sequential if and only if it is the quotient
                  of a metrizable space.
        \end{itemize}
        In just about every setting we will be working with first countable
        spaces. It would be nice to know that first countable spaces are
        sequential, and indeed they are.
        \begin{theorem}
            \label{thm:First_Countable_Implies_Sequential}%
            If $\topspace{X}$ is a first countable topological space, then it
            is a sequential topological space.
        \end{theorem}
        \begin{proof}
            For suppose not. Then there is a sequentially open subset
            $\mathcal{U}\subseteq{X}$ such that $\mathcal{U}$ is not open
            (Def.~\ref{def:Sequential_Space}). But since $\topspace{X}$ is first
            countable, for all $x\in\mathcal{U}$ there is a countable
            neighborhood basis. By the axiom of choice there is a
            function $\mathcal{B}:\mathcal{U}\rightarrow\powset{\tau}$ such that
            for all $x\in\mathcal{U}$, $\mathcal{B}_{x}$ is a countable
            neighborhood basis of $x$. But if for all $x\in\mathcal{U}$ it is
            true that $\mathcal{B}_{x}$ is countable, then there is surjective
            function $\mathcal{V}_{x}:\mathbb{N}\rightarrow\mathcal{B}_{x}$.
            Then there exists an $N\in\mathbb{N}$ such that
            $\mathcal{V}_{x,N}\subseteq\mathcal{U}$. For suppose not and let
            $B:\mathbb{N}\rightarrow{X}$ be defined by:
            \begin{equation}
                B_{n}=\bigcap_{k\in\mathbb{Z}_{n}}\mathcal{V}_{x,k}
            \end{equation}
            Since $B_{n}$ is the intersection of finitely many open sets, it is
            open. Moreover it is non-empty since $x\in{B}_{n}$ for all $n$.
            But then $B_{n}$ is an open neighborhood about $x$, and since
            $\mathcal{B}_{x}$ is a neighborhood basis there is an element
            $V\in\mathcal{B}_{x}$ such that $V\subseteq{B}_{n}$. But
            $\mathcal{V}:\mathbb{N}\rightarrow\mathcal{B}_{x}$ is a surjection,
            and hence there is an $N\in\mathbb{N}$ such that
            $\mathcal{V}_{x,N}=V$. But by hypothesis,
            $\mathcal{V}_{x,N}\nsubseteq\mathcal{U}$ and hence there is an
            element $y\in\mathcal{V}_{x,N}$ such that $y\notin\mathcal{U}$.
            Therefore we have shown that the sequence $A_{n}$ defined by:
            \begin{equation}
                A_{n}=\{\,y\in{B}_{n}\;|\;y\not\in\mathcal{U}\,\}
            \end{equation}
            is non-empty for all $n\in\mathbb{N}$, and hence by the axiom of
            (countable) choice there is a sequence $y:\mathbb{N}\rightarrow{X}$
            such that $y_{n}\in{A}_{n}$ for all $n\in\mathbb{N}$. That is,
            for all $n\in\mathbb{N}$, $y_{n}\in{B}_{n}$ and
            $y_{n}\notin\mathcal{U}$. But if $y_{n}\in{B}_{n}$ for all
            $n\in\mathbb{N}$, then $y_{n}\rightarrow{x}$. For if not, then there
            is an open set $U\in\tau$ such that $x\in{U}$ and for all
            $N\in\mathbb{N}$ there exists an $n>N$ such that $y_{n}\notin{U}$.
            But $\mathcal{B}_{x}$ is a neighborhood basis of $x$, and hence
            there is a $V\in\mathcal{B}_{x}$ such that $V\subseteq{U}$. But
            $\mathcal{V}_{x}:\mathbb{N}\rightarrow\mathcal{B}_{x}$ is a
            surjection, and hence there is an $N\in\mathbb{N}$ such that
            $\mathcal{V}_{x,N}=V$. But then for all $n>N$,
            $B_{n}\subseteq\mathcal{V}_{x,N}$ and thus $B_{n}\subseteq{U}$.
            But then for all $n>N$, $y_{n}\in{U}$, a contradiction. Hence,
            $y_{n}\rightarrow{x}$. But for all $n\in\mathbb{N}$,
            $y_{n}\notin\mathcal{U}$. Thus $y_{n}$ is a sequence such that
            $y_{n}\rightarrow{x}$, but for all $N\in\mathbb{N}$ there is an
            $n\in\mathbb{N}$ with $n>N$ such that $y_{n}\notin\mathcal{U}$, a
            contradiction since $x\in\mathcal{U}$ and $\mathcal{U}$ is
            sequentially open (Def.~\ref{def:Sequentially_Open}). Thus, for all
            $x\in\mathcal{U}$ there is an open subset $\mathcal{V}_{x,N}$ such
            that $x\in\mathcal{V}_{x,N}$ and
            $\mathcal{V}_{x,N}\subseteq\mathcal{U}$. But then $\mathcal{U}$ is
            simply the union over all of these open sets, and is thus open,
            a contradiction. Hence, $\topspace{X}$ is sequential.
        \end{proof}
        And now, our useful corollary.
        \begin{theorem}
            \label{thm:First_Countable_Implies_Seq_Cont_is_Cont}%
            If $\topspace[X]{X}$ is a first countable topological space, if
            $\topspace[Y]{Y}$ is a topological space, and if $f:X\rightarrow{Y}$
            is sequential continuous, then $f$ is continuous.
        \end{theorem}
        \begin{proof}
            For since $\topspace[X]{X}$ is first countable, it is sequential
            (Thm.~\ref{thm:First_Countable_Implies_Sequential}). But if
            $\topspace[X]{X}$ is a sequential topological space and if $f$ is
            sequentially continuous, then $f$ is continuous
            (Thm.~\ref{thm:seq_space_seq_cont_eqiv_cont}). Therefore, $f$ is
            continuous.
        \end{proof}
        It's laborious to show a certain space is first countable, especially in
        the case of manifold theory, and so we rely on the following theorems.
        \begin{theorem}
            \label{thm:Locally_Metrizable_is_First_Countable}%
            If $\topspace{X}$ is a locally metrizable topological space, then it
            is first countable.
        \end{theorem}
        \begin{proof}
            For suppose not. Then there is a point $x\in{X}$ with no countable
            neighborhood basis. But if $\topspace{X}$ is locally metrizable,
            then there is an open subset $\mathcal{U}$ such that
            $x\in\mathcal{U}$ and $\topspace[\mathcal{U}]{\mathcal{U}}$ is
            metrizable, where $\tau_{\mathcal{U}}$ is the subspace topology. But
            then there is a metric
            $d:\mathcal{U}\times\mathcal{U}\rightarrow\nspace[]$ such that
            $d$ induces $\tau_{\mathcal{U}}$. Let $\mathcal{B}\subseteq\tau$ be
            defined by:
            \begin{equation}
                \mathcal{B}=\{\,\rball{n^{\minus{1}}}{\metspace{\mathcal{U}}}{x}
                    \;|\;n\in\mathbb{N}^{+}\}
            \end{equation}
            That is, the set of all open balls about $x$ of radius $1/n$.
            Suppose $\mathcal{V}\in\tau$ is such that $x\in\mathcal{V}$. But
            then $x\in\mathcal{V}\cap\mathcal{U}$. But
            $\mathcal{V}\cap\mathcal{U}$ is an open subset of $\mathcal{U}$, and
            thus there is an $r>0$ such that:
            \begin{equation}
                \rball{r}{\metspace{\mathcal{U}}}{x}\subseteq\mathcal{V}
            \end{equation}
            but by Archimede's Theorem, there is an $N\in\mathbb{N}$ such that
            $N>r$. But it is true that
            $\rball{N^{\minus{1}}}{\metspace{\mathcal{U}}}{x}\in\mathcal{B}$,
            and hence $\mathcal{B}$ is a countable neighborhood basis of $x$,
            a contradiction. Thus, $\topspace{X}$ is first countable.
        \end{proof}
        \begin{fdefinition}{$\sigma$ Compact}{Sigma_Compact}
            A $\sigma$ compact topological space is a topological space
            $\topspace{X}$ such that there exists a sequence
            $K:\mathbb{N}\rightarrow\powset{X}$ of compact subsets of $X$ such
            that:
            \begin{equation*}
                X=\bigcup_{n\in\mathbb{N}}K_{n}
            \end{equation*}
        \end{fdefinition}
        This definition gives the following triviality.
        \begin{theorem}
            \label{thm:Compact_Implies_Sigma_Compact}%
            If $\topspace{X}$ is a compact topological space, then it is
            $\sigma$ compact.
        \end{theorem}
        \begin{proof}
            For let $K:\mathbb{N}\rightarrow\powset{X}$ be defined by
            $K_{n}=X$. Then $K_{n}$ is compact for all $n$ and
            $\bigcup{K}_{n}=X$. Thus, $\topspace{X}$ is $\sigma$ compact
            (Def.~\ref{def:Sigma_Compact}).
        \end{proof}
        \begin{fdefinition}{Lindel\"{o}f Topological Space}{Lindelof_Space}
            A Lindel\"{o}f topological space is a topological space $(X,\tau)$
            such that for every open cover $\mathcal{O}$ of $X$ there exists a
            countable subcover.
        \end{fdefinition}
        \begin{theorem}
            \label{thm:Second_Countable_Implies_Lindelof}%
            If $\topspace{X}$ is a second countable topological space, then it
            is Lindel\"{o}f.
        \end{theorem}
        \begin{proof}
            For suppose not. Then there exists an open cover $\mathcal{O}$ with
            no countable subcover. But if $\topspace{X}$ is second countable,
            then there is a countable basis $\mathcal{B}$ of $\tau$. But if
            $\mathcal{B}$ is a basis, since $\mathcal{O}$ is an open cover, for
            all $\mathcal{V}\in\mathcal{O}$ it is true that $\mathcal{V}$ is
            open and thus there is an element $\mathcal{U}\in\mathcal{B}$ such
            that $\mathcal{U}\subseteq\mathcal{V}$. That is, the function
            $F:\mathcal{O}\rightarrow\powset{\mathcal{B}}$ defined by:
            \begin{equation}
                F(\mathcal{V})=\{\,\mathcal{U}\in\mathcal{B}\;|\;
                    \mathcal{U}\subseteq\mathcal{V}\,\}
            \end{equation}
            is such that $F(\mathcal{V})\ne\emptyset$ for all
            $\mathcal{V}\in\mathcal{O}$. But then
            $F[\mathcal{O}]\subseteq\powset{\mathcal{B}}$ is a non-empty subset
            of $\powset{B}$ such that $\emptyset\notin{F}[\mathcal{O}]$. Hence
            by the axiom of choice there is a function
            $G:F[\mathcal{O}]\rightarrow\mathcal{B}$ such that for all
            $\Delta\in{F}[\mathcal{O}]$, $G(\Delta)\in\Delta$. Let
            $B=G\big[F[\mathcal{O}]\big]$. But $\mathcal{B}$ is countable and
            $B\subseteq\mathcal{B}$, and thus $B$ is countable. Moreover,
            $F\circ{G}:\mathcal{O}\rightarrow{B}$ is surjective by the
            definition of $B$ and hence there exists a right inverse
            $H:B\rightarrow\mathcal{O}$. And since $B$ is countable,
            $H[B]$ is a countable subset of $\mathcal{O}$. But $\mathcal{O}$ has
            no countable subcover, and hence there is an $x\in{X}$ such that
            $x\notin\bigcup{H}[B]$. But $\mathcal{O}$ is a cover of $X$, and
            thus there is a $\mathcal{V}\in\mathcal{O}$ such that
            $x\in\mathcal{V}$. And since $\mathcal{B}$ is a basis, there is an
            element $\mathcal{U}_{1}\in\mathcal{B}$ such that
            $x\in\mathcal{U}_{1}$. But then $x\in\mathcal{U}_{1}\cap\mathcal{V}$
            and since $\mathcal{B}$ is a basis, there is an element
            $\mathcal{U}_{2}\in\mathcal{B}$ such that $x\in\mathcal{U}_{2}$ and
            $\mathcal{U}_{2}\subseteq\mathcal{U}_{1}\cap\mathcal{V}$. But then
            $H(\mathcal{U}_{2})$ is an element of $H[B]$ such that
            $\mathcal{U}_{2}\subseteq{H}(\mathcal{U}_{2})$, and hence
            $x\in{H}[\mathcal{U}_{2}]$. A contradiction, since
            $x\notin\bigcup{H}[B]$. Therefore, $\topspace{X}$ is Lindel\"{o}f.
        \end{proof}
        \begin{theorem}
            \label{thm:Sigma_Compact_Implies_Lindelof}%
            If $\topspace{X}$ is a $\sigma$ compact topological space, then it
            is Lindel\"{o}f.
        \end{theorem}
        \begin{proof}
            For suppose not. Then there exists an open cover $\mathcal{O}$ of
            $X$ with no countable subcover. But $X$ is $\sigma$ compact, and
            hence there is a sequence $K:\mathbb{N}\rightarrow\powset{X}$ of
            compact sets such that $X=\bigcup{K}_{n}$
            (Def.~\ref{def:Sigma_Compact}). But then for all $n\in\mathbb{N}$,
            $K_{n}$ is a subset of $X$, and hence $\mathcal{O}$ is an open cover
            of $K_{n}$. But by hypothesis, $K_{n}$ is compact and hence there is
            a finite subcover $\mathscr{D}\subseteq\mathcal{O}$ of $K_{n}$. That
            is, if we define the sequence
            $A_{n}:\mathbb{N}\rightarrow\powset{\mathcal{O}}$ by:
            \begin{equation}
                A_{n}=\{\mathscr{D}\in\powset{\mathcal{O}}\;|\;
                    \mathscr{D}\textrm{ is finite and }
                    K_{n}\subseteq\bigcup\mathscr{D}\,\}
            \end{equation}
            then for all $n\in\mathbb{N}$, $A_{n}$ is non-empty. Hence by the
            axiom of (countable) choice, there is a choice function
            $\Delta:\mathbb{N}\rightarrow\powset{\mathcal{O}}$ such that for all
            $n\in\mathbb{N}$, $\Delta_{n}$ is a finite open subcover of $K_{n}$.
            But then the union of all $\Delta_{n}$ is the countable union of
            finite collections, and is hence countable. But then this collection
            covers $K_{n}$ for all $n\in\mathbb{N}$, and hence covers
            $\bigcup{K}_{n}$. But $\bigcup{K}_{n}=X$, a contradiction since
            $\mathcal{O}$ has no countable subcover. Thus, $\topspace{X}$ is
            Lindel\"{o}f.
        \end{proof}
        This theorem reverses if we add locally compact. The requirement of
        local compactness can be seen by examining the irrational numbers with
        the subspace topology. This is Lindel\"{o}f (since it is second
        countable), but it is not $\sigma$ compact. For any compact subset of
        the irrationals must have empty interior, and by the Baire category
        theorem the irrationals cannot be written as the countable union of
        nowhere dense subsets. Hence they cannot possibly $\sigma$ compact.
        \begin{fdefinition}{Locally Compact Topological Space}{Locally_Compact}
            A locally compact topological space is a topological space
            $\topspace{X}$ such that for all $x\in{X}$ there exists a compact
            subset $K\subseteq{X}$ and an open set $\mathcal{U}\in\tau$ such
            that $x\in\mathcal{U}$ and $\mathcal{U}\subseteq{K}$.
        \end{fdefinition}
        \begin{theorem}
            \label{thm:Loc_Comp_and_Lindelof_Implies_Sigma_Comp}%
            If $\topspace{X}$ is a locally compact Lindel\"{o}f space, then it
            is $\sigma$ compact.
        \end{theorem}
        \begin{proof}
            For if $\topspace{X}$ is locally compact, then for all $x\in{X}$
            there is a compact set $K\subseteq{X}$ and an open set
            $\mathcal{U}\in\tau$ such that $x\in\mathcal{U}$ and
            $\mathcal{U}\subseteq{K}$ (Def.~\ref{def:Locally_Compact}). Invoking
            the axiom of choice, there is a function
            $A:X\rightarrow\tau\times\powset{X}$ such that for all $x\in{X}$,
            $A_{x}=(\mathcal{U},K)$ where $x\in\mathcal{U}$ and
            $\mathcal{U}\subseteq{K}$, where $K$ is compact. But then the
            collection of all such $\mathcal{U}$ is an open cover of $X$. But
            $X$ is Lindel\"{o}f and hence there is a countable subcover. But
            then the subcollection of all $K$ form a countable collection of
            compact sets that cover $X$. Hence, $\topspace{X}$ is $\sigma$
            compact.
        \end{proof}
        \begin{fdefinition}{Sequentially Compact}{Sequentially_Compact}
            A sequentially compact topological space is a topological space
            $\topspace{X}$ such that for every sequence
            $a:\mathbb{N}\rightarrow{X}$ there exists a strictly increasing
            sequence $k:\mathbb{N}\rightarrow\mathbb{N}$ such that
            $a\circ{k}:\mathbb{N}\rightarrow{X}$ converges.
        \end{fdefinition}
        \begin{theorem}
            \label{thm:Met_Space_Seq_Compact_iff_Compact}%
            If $\topspace{X}$ is metrizable, and if $\mathcal{C}\subseteq{X}$,
            then $\mathcal{C}$ is compact if and only if it is sequentially
            compact.
        \end{theorem}
        \begin{proof}
            For if $\topspace[X]{X}$ is metrizable, then there is a metric
            $d:X\times{X}\rightarrow\nspace[]$ that induces the topology $\tau$.
            Suppose $X$ is compact and not sequentially compact. Then there is a
            sequence $a:\mathbb{N}\rightarrow{X}$ with no convergent
            subsequence. Then for al $x\in{X}$ there is an $r>0$ such that the
            open ball of radius $r$ about $x$ is such that only finitely many
            $n\in\mathbb{N}$ imply $a_{n}$ lies inside the ball. Invoking
            choice, we get a function $r:X\rightarrow\mathbb{R}^{+}$. Let
            $\mathcal{O}$ be defined by:
            \begin{equation}
                \mathcal{O}=\{\,\rball{r_{x}}{\metspace{X}}{x}\;|\;x\in{X}\,\}
            \end{equation}
            Then $\mathcal{O}$ is an open cover of $X$. But $X$ is compact, and
            therefore there is a finite subcover $\Delta$. But for each
            $\mathcal{U}\in\Delta$, there are only finite many
            $n\in\mathbb{N}$ such that $a_{n}\in\mathcal{U}$. But every
            $n\in\mathbb{N}$ is such that $a_{n}\in\mathcal{U}$ for at least
            one $\mathcal{U}$ since these cover $X$, a contradiction since
            $\mathbb{N}$ is not finite. Thus, $\topspace{X}$ is sequentially
            compact. In the other direction, sequential compactness implies
            complete and totally bounded, which implies compact by the
            generalized Heine-Borel theorem.
        \end{proof}
        This is not true in general. The long ling is sequentially compact but
        not compact, whereas the space of all functions
        $f:[0,1]\rightarrow[0,1]$ is compact (by Tychonoff) but not sequentially
        compact.
        \begin{fdefinition}{$\sigma$ locally finite basis}
                           {Sigma_Loc_Fin_Basis}
            A $\sigma$ locally finite basis of a topological space
            $\topspace{X}$ is a basis $\mathcal{B}$ of $X$ such that
            $\mathcal{B}$ is $\sigma$ locally finite. That is, there exists
            a sequence $B:\mathbb{N}\rightarrow\powset{X}$ such that
            $\mathcal{B}=\bigcup\{B_{n}\}$, and for all $n\in\mathbb{N}$ it is
            true that $B_{n}$ is locally finite.
        \end{fdefinition}
        \begin{theorem}
            \label{thm:Second_Countable_Implies_Sigma_Loc_Fin_Basis}%
            If $\topspace{X}$ is second countable, then it is has a $\sigma$
            locally finite basis.
        \end{theorem}
        \begin{proof}
            For if $\topspace{X}$ is has a countable basis $\mathcal{B}$. But
            then there is a surjection $B:\mathbb{N}\rightarrow\mathcal{B}$.
            But then for all $n\in\mathbb{N}$, $\{B_{n}\}$ is a finite subset
            of $\powset{X}$, and is hence locally finite, and
            $\mathcal{B}=\bigcup\{B_{n}\}$. Thus, $X$ has a $\sigma$ locally
            finite basis.
        \end{proof}
        \begin{fdefinition}{Paracompact}{Paracompact}
            A paracompact topological space is a topological space
            $\topspace{X}$ such that for every open cover $\mathcal{O}$ of $X$,
            there exists a locally finite refinement $\Delta$ of $\mathcal{O}$.
        \end{fdefinition}
        \begin{ftheorem}{Nagata-Smirnov Metrization Theorem}
                        {Nagata_Smirnov_Metrization_Theorem}
            If $\topspace{X}$ is a topological space, then it is metrizable if
            and only if it is regular, Hausdorff, and has a $\sigma$ locally
            finite basis.
        \end{ftheorem}
        \begin{proof}
            Munkres.
        \end{proof}
        \begin{ftheorem}{Smirnov Metrization Theorem}
                        {Smirnov_Metrization_Theorem}
            If $\topspace{X}$ is a topological space, then it is metrizable if
            and only if it is Hausdorff, paracompact, and locally metrizable.
        \end{ftheorem}
        \begin{proof}
            Also Munkres.
        \end{proof}
        \begin{theorem}
            \label{thm:Count_Open_Cover_of_Sec_Count_Implies_Sec_Count}%
            If $\topspace{X}$ is a topological space, if
            $\mathcal{O}\subseteq\tau$ is a countable subset of open sets such
            that for all $\mathcal{U}\in\mathcal{O}$,
            $\topspace[\mathcal{U}]{\mathcal{U}}$ is second countable, where
            $\tau_{\mathcal{U}}$ is the subspace topology, then $\topspace{X}$
            is second countable.
        \end{theorem}
        \begin{proof}
            For let $\mathcal{B}$ be the collection of all of the bases for all
            of the $\mathcal{U}\in\mathcal{O}$. Since it is the countable union
            of countable sets, it is countable. But since $\mathcal{U}$ is open
            for all $\mathcal{U}\in\mathcal{O}$, this is a countable open cover
            of $X$. It suffices to show that it is a basis. Let
            $\mathcal{V}\in\tau$ be an open subset. But then:
            \begin{align}
                \mathcal{V}&=\mathcal{V}\cap{X}\\
                &=\mathcal{V}\cap\Big(
                    \bigcup_{\mathcal{U}\in\mathcal{O}}\mathcal{U}
                \Big)\\
                &=\bigcup_{\mathcal{U}\in\mathcal{O}}
                    \big(\mathcal{V}\cap\mathcal{U}\big)
            \end{align}
            But $\mathcal{V}\cap\mathcal{U}$ is an open subset of the subspace
            $\topspace[\mathcal{U}]{\mathcal{U}}$, and hence there is a subset
            of $\Delta_{\mathcal{V}}\subseteq\mathcal{B}_{\mathcal{U}}$ such
            that $\mathcal{V}\cap\mathcal{U}=\bigcup\Delta_{\mathcal{V}}$. But
            then the entire of $\mathcal{V}$ is the union of all such
            collections, each of which is contained in $\mathcal{B}$, and hence
            $\mathcal{V}$ can be written as the union of elements of
            $\mathcal{B}$. Thus, $\mathcal{B}$ is a basis.
        \end{proof}
        The requirement that the covering collection be open subspaces is
        crucial. The quotient space $\mathbb{R}/R$, where $R$ is the equivalence
        relation generated by $nRm$ for all $n,m\in\mathbb{Z}$, can be thought
        of as a countable collection of rings all glued together at the origin.
        Hence, it can be covered by countably many closed subspaces, each of
        which is homeomorphic to $\nsphere[1]$ in the subspace topology, and
        hence each of which is second countable. However, this space
        $\mathbb{R}/R$ is not even first countable, let alone second countable.
        The point $[0]\in\mathbb{R}/R$ has no countable neighborhood basis.
        \begin{theorem}
            \label{thm:loc_path_con_imply_path_comps_open}%
            If $\topspace{X}$ is locally path connected, if $x\in{X}$, and if
            $\mathcal{U}\subseteq{X}$ is a path connected component of $X$,
            then $\mathcal{U}$ is open.
        \end{theorem}
        \begin{proof}
            For if $\topspace{X}$ is locally path connected, there is a basis
            $\mathcal{B}$ of open and path connected subsets of $X$. But if
            $\mathcal{U}\subseteq{X}$ is a path connected component, then for
            all $x,y\in\mathcal{U}$ there is a path
            $\gamma:[0,1]\rightarrow\mathcal{U}$ connecting $x$ and $y$, and for
            all $z\in{X}$ such that $z\notin\mathcal{U}$, there is no path
            between $x$ and $z$. But $\mathcal{B}$ is a basis, and hence for
            all $x\in\mathcal{U}$ there is a $B\in\mathcal{B}$ such that
            $x\in{B}$. By choice, we get a function
            $B:\mathcal{U}\rightarrow\mathcal{B}$. Moreover, since
            $B\in\mathcal{B}$, it is path connected. But then for all
            $x\in\mathcal{U}$, $x\in{B}_{x}$, and since $B_{x}$ is path
            connected it is true that $B_{x}\subseteq\mathcal{U}$ since
            $\mathcal{U}$ is a path connected component. But then
            $\mathcal{U}=\bigcup{B}_{x}$, which is the union of open sets, and
            hence $\mathcal{U}$ is open.
        \end{proof}
        \begin{theorem}
            \label{thm:Loc_Path_and_Con_Imply_Path_Con}
            If $\topspace{X}$ is locally path connected and connected, then it
            is path connected.
        \end{theorem}
        \begin{proof}
            For if not then there are two points $x,y\in{X}$ with no path
            between them. But the since $\topspace{X}$ is locally path
            connected, the path connected components of $x$ and $y$ are open
            (Thm.~\ref{thm:loc_path_con_imply_path_comps_open}). But let
            $\mathcal{U}$ be the path connected component containing $x$, and
            let $\mathcal{V}$ be the union of all other path connected
            components. Then $\mathcal{V}$ is non-empty since $y\in\mathcal{V}$,
            and hence $\mathcal{U}$ and $\mathcal{V}$ are non-empty disjoint
            open subsets that cover $X$, a contradiction since $X$ is connected.
        \end{proof}
        \begin{fdefinition}{Locally Euclidean}{Locally_Euclidean}
            A locally Euclidean topological space is a topological space
            $\topspace{X}$ such that for all $x\in{X}$ there exists an open
            subset $\mathcal{U}\in\tau$ and an $n\in\mathbb{N}$ such that
            $x\in\mathcal{U}$ and $\mathcal{U}$ is homeomorphic to an open
            subset of $\nspace$.
        \end{fdefinition}
        \begin{theorem}
            \label{thm:Equiv_Def_Loc_Euclidean}%
            If $\topspace{X}$ is locally Euclidean, then for all $x\in{X}$ there
            is an open subset $\mathcal{U}\in\tau$ and an $n\in\mathbb{N}$ such
            that $\mathcal{U}$ is homeomorphic to $\nspace$.
        \end{theorem}
        \begin{proof}
            For if $\topspace{X}$ is locally Euclidean, then for all $x\in{X}$
            there is an open subset $\mathcal{V}\in\tau$ and an $n\in\mathbb{N}$
            such that $x\in\mathcal{V}$ and $\mathcal{V}$ is homeomorphic to an
            open subset of $\nspace$. But then there is an injective continuous
            open mapping $\varphi:\mathcal{V}\rightarrow\nspace$. Let
            $\vector{y}=\varphi(x)$. But since $\varphi$ is an open mapping,
            and since $\mathcal{V}$ is open, $\varphi[\mathcal{V}]$ is an open
            subset of $\nspace$. But $\vector{y}=\varphi(x)$, and hence
            $\vector{y}\in\varphi[\mathcal{V}]$. But if $\varphi[\mathcal{V}]$
            is open and $\vector{y}\in\varphi[\mathcal{V}]$, then there is an
            $r>0$ such that the open ball $B$ defined by:
            \begin{equation}
                B=\rball{r}{\metspace[\norm{\cdot}_{2}]{\nspace}}{\vector{y}}
            \end{equation}
            is contained in $\varphi[\mathcal{V}]$. But $\varphi$ is continuous,
            and open balls are open, and hence
            $\varphi^{\minus{1}}[B]$ is an open subset of $\mathcal{V}$.
            Moreover, since $\vector{y}\in{B}$,
            $x\in\varphi^{\minus{1}}[B]$.
            But $\varphi$ is an injective continuous open mapping, and thus the
            restriction of $\varphi$ to an open subset is an injective
            continuous open mapping. Hence, $\varphi|_{\varphi^{\minus{1}}[B]}$
            is a homeomorphism onto it's image, which is $B$. And since open
            balls in $\nspace$ are homeomorphic to $\nspace$,
            $\varphi^{\minus{1}}[B]$ is homeomorphic to $\nspace$. Thus, there
            is an open subset $\varphi^{\minus{1}}[B]$ containing $x$ and an
            $n\in\mathbb{N}$ such that $\varphi^{\minus{1}}[B]$ is homeomorphic
            to $\nspace$.
        \end{proof}
        \begin{theorem}
            \label{thm:Loc_Euc_Existence_of_Basis_of_nspace_Sets}%
            If $\topspace{X}$ is a locally Euclidean topological space, then
            there exists a basis $\mathcal{B}$ such that for all
            $\mathcal{U}\in\mathcal{B}$ there is an $n\in\mathbb{N}$ such that
            $\mathcal{U}$ is homeomorphic to $\nspace$.
        \end{theorem}
        \begin{proof}
            For if $\topspace{X}$ is locally Euclidean, then for all $x\in{X}$
            there is an open subset $\mathcal{U}_{x}\in\tau$ and an
            $n\in\mathbb{N}$ such that $\mathcal{U}_{x}$ is homeomorphic to
            $\nspace$ (Thm.~\ref{thm:Equiv_Def_Loc_Euclidean}). Let
            $\varphi_{x}:\mathcal{U}_{x}\rightarrow\nspace$ be such a
            homeomorphism, and let $\mathcal{B}_{x}$ be the set:
            \begin{equation}
                \mathcal{B}_{x}=\{\,
                    \varphi_{x}^{\minus{1}}\big[
                        \rball{r}{\metspace[\norm{\cdot}_{2}]{\nspace}}
                        {\vector{y}}\big]
                    \;|\;r>0,\,\vector{y}\in\nspace\,\}
            \end{equation}
            Then by construction, every element of $\mathcal{B}_{x}$ is
            homeomorphic to $\nspace$. Let $\mathcal{B}$ be the collection of
            all such sets for all $x\in{X}$. If $\mathcal{U}$, $\mathcal{V}$ are
            elements of $\mathcal{B}$, then there is an $x\in{X}$ such that
            $\mathcal{U}\in\mathcal{B}_{x}$. Let
            $y\in\mathcal{U}\cap\mathcal{V}$ and let $\vector{y}$ be the image
            of $y$ under $\varphi_{x}$. But $\mathcal{U}\cap\mathcal{V}$ is
            open, and hence there is an $r>0$ such that the ball about
            $\vector{y}$ is contained in the image of
            $\varphi_{x}[\mathcal{U}\cap\mathcal{V}]$. But this $r$ ball is
            contained in $\mathcal{B}$. Hence, $\mathcal{B}$ is a basis.
        \end{proof}
        \begin{theorem}
            \label{thm:Loc_Euc_Existence_of_Basis_of_Precompact_Balls}%
            If $\topspace{X}$ is a locally Euclidean topological space, then
            there is a basis $\mathcal{B}$ such that for all
            $\mathcal{U}\in\mathcal{B}$ it is true that $\mathcal{U}$ is
            precompact in $\tau$ and such that there exists an $n\in\mathbb{N}$
            such that $\mathcal{U}$ is homeomorphic to $\nspace$.
        \end{theorem}
        \begin{proof}
            For by Thm.~\ref{thm:Loc_Euc_Existence_of_Basis_of_nspace_Sets},
            there is a basis $\mathcal{B}$ of $\tau$ such that for all
            $\mathcal{U}\in\mathcal{B}$ there is an $n\in\mathbb{N}$ such that
            $\mathcal{U}$ is homeomorphic to $\nspace$. Let
            $\varphi_{\mathcal{U}}:\mathcal{U}\rightarrow\nspace$ be such a
            homeomorphism. Define $\mathcal{B}_{x}$ by:
            \begin{equation}
                \mathcal{B}_{x}=\{\,
                    \varphi_{\mathcal{U}}^{\minus{1}}\big[
                        \rball{r}{\metspace[\norm{\cdot}_{2}]{\nspace}}
                        {\vector{y}}\big]\;|\;r>0,\,\vector{y}\in\nspace\,\}
            \end{equation}
            But by the Heine-Borel theorem, for all
            $\mathcal{V}\in\mathcal{B}_{x}$,
            $\closure[\nspace]{\varphi_{\mathcal{U}}[\mathcal{V}]}$ is compact
            in $\nspace$, and since $\varphi_{\mathcal{U}}$ is a homeomorphism,
            $\closure{\mathcal{V}}$ is compact in $\mathcal{U}$. But then
            $\closure{\mathcal{V}}$ is compact in $X$. The collection of all
            such $\mathcal{B}_{x}$ is thus a basis of precompact subsets that
            are homeomorphic to open balls in $\nspace$, which are thus
            homeomorphic to $\nspace$.
        \end{proof}
        \begin{theorem}
            \label{thm:Loc_Euc_is_Loc_Compact}%
            If $\topspace{X}$ is locally Euclidean, then it is locally compact.
        \end{theorem}
        \begin{proof}
            For if $\topspace{X}$ is locally Euclidean, then there exists a
            basis $\mathcal{B}$ of precompact coordinate balls
            (Thm.~\ref{thm:Loc_Euc_Existence_of_Basis_of_Precompact_Balls}). But
            then for all $x\in{X}$ there is a $\mathcal{U}\in\mathcal{B}$ such
            that $x\in\mathcal{U}$ and $\closure{\mathcal{U}}$ is compact. Thus,
            $\topspace{X}$ is locally compact (Def.~\ref{def:Locally_Compact}).
        \end{proof}
        \begin{theorem}
            \label{thm:Loc_Euc_Implies_Loc_Met}%
            If $\topspace{X}$ is locally Euclidean, then it is locally
            metrizable.
        \end{theorem}
        \begin{proof}
            For if $\topspace{X}$ is locally Euclidean, then for all $x\in{X}$
            there is an $n\in\mathbb{N}$ and a $\mathcal{U}\in\tau$ such that
            $x\in\mathcal{U}$ and $\mathcal{U}$ is homeomorphic to
            $\nspace$ (Thm.~\ref{thm:Equiv_Def_Loc_Euclidean}). But
            $\nspace$ is metrizable, and thus $\mathcal{U}$ is metrizable.
            Hence, $X$ is locally metrizable.
        \end{proof}
        \begin{fdefinition}{Topological Manifold}{Topological_Manifold}
            A topological manifold is a locally Euclidean, Hausdorff, second
            countable topological space of constant dimension.
        \end{fdefinition}
        \begin{theorem}
            \label{thm:Manifolds_are_Lindelof}%
            If $\topspace{X}$ is a topological manifold, then it is
            Lindel\"{o}f.
        \end{theorem}
        \begin{proof}
            For if $X$ is a topological manifold, then it is second countable
            (Def.~\ref{def:Topological_Manifold}). But second countable
            topological spaces are Lindel\"{o}f
            (Thm.~\ref{thm:Second_Countable_Implies_Lindelof}). Thus,
            $\topspace{X}$ is Lindel\"{o}f.
        \end{proof}
        \begin{theorem}
            If $\topspace{X}$ is a topological manifold, then it is $\sigma$
            compact.
        \end{theorem}
        \begin{proof}
            For if $X$ is a topological manifold, then it is Lindel\"{o}f
            (Thm.~\ref{thm:Manifolds_are_Lindelof}). But topological manifolds
            are locally Euclidean (Def.~\ref{def:Locally_Euclidean}) and locally
            Euclidean spaces are locally compact
            (Thm.~\ref{thm:Loc_Euc_is_Loc_Compact}). But if $\topspace{X}$ is
            locally compact and Lindel\"{o}f, then it is $\sigma$ compact
            (Thm.~\ref{thm:Loc_Comp_and_Lindelof_Implies_Sigma_Comp}).
        \end{proof}
        \begin{fdefinition}{Topological Group}{Topological_Group}
            A topological group, denoted $\topgroup{G}$, is a topological space
            $\topspace{G}$ and a binary operation $*$ such that $\monoid{G}$ is
            a group, and such that $g:G\times{G}\rightarrow{G}$ defined by
            $g(a,b)=a*b$ is continuous with respect to the product topology on
            $G$, and such that $\nu:G\rightarrow{G}$ defined by
            $\nu(a)=a^{\minus{1}}$, where $a^{\minus{1}}$ is the inverse element
            of $a$ under $*$, is continuous.
        \end{fdefinition}
        \begin{fdefinition}{Continuous Group Action}{Continuous_Group_Action}
            A continuous group action of a topological group $\topgroup{G}$
            on a topological space $\topspace[X]{X}$ is a function
            $\Theta:G\times{X}\rightarrow{X}$ such that for all $x\in{X}$ and
            for all $a,b\in{G}$, the following are true:
            \begin{align*}
                \Theta(e,x)&=x\\
                \Theta\big(a,\Theta(b,x)\big)&=\Theta(a*b,x)
            \end{align*}
            where $e$ is the unital element of $G$, and such that $\Theta$ is
            continuous with respect to the product topology.
        \end{fdefinition}
        \begin{theorem}
            \label{thm:Quotient_by_Compact_T2_Group_Preserves_T2}%
            If $\topspace[X]{X}$ is a Hausdorff topological space, if
            $\topgroup{G}$ is a compact Hausdorff topological group, if
            $\Theta$ is a continuous group action of $G$ on $X$, and if
            $\topspace[q]{X/G}$ is the quotient topology formed by the orbits
            of $\Theta$, then $X/G$ is Hausdorff.
        \end{theorem}
        \begin{proof}
            Need to fill in.
        \end{proof}
        \begin{theorem}
            \label{thm:Quotient_by_Compact_T2_Group_Preserves_Sec_Count}%
            If $\topspace[X]{X}$ is a second countable topological space, if
            $\topgroup{G}$ is a compact Hausdorff topological group, if
            $\Theta$ is a continuous group action of $G$ on $X$, and if
            $\topspace[q]{X/G}$ is the quotient topology formed by the orbits
            of $\Theta$, then $X/G$ is second countable.
        \end{theorem}
        \begin{proof}
            Also need to fill in.
        \end{proof}
    \section{Problems}
        \begin{problem}
            Show that $\mathbb{RP}^{n}$ is Hausdorff and second countable.
        \end{problem}
        \begin{solution}
            For $\mathbb{RP}^{n}$ is homeomorphic to the quotient space of
            $\nsphere$ by the multiplicative group $G=\{\minus{1},1\}$ with the
            group action $\Theta:G\times\nsphere\rightarrow\nsphere$ defined by
            $\Theta(n,\vector{s})=n\cdot\vector{s}$ (this is well defined since
            $n=\pm{1}$, and hence $\norm{n\cdot\vector{s}}=1$ and thus
            $n\cdot\vector{s}$ still lies in $\nsphere$). Equipping $G$ with the
            discrete topology makes $\topgroup{G}$ a topological group, and
            $\Theta$ a continuous group action on $\nsphere$. But $G$ is finite,
            and hence compact, and moreover it is Hausdorff since the discrete
            topology is always Hausdorff (it is metrizable, in fact). Thus
            $\nsphere/G$ is the quotient of a second countable Hausdorff space
            by a compact Hausdorff group, and is therefore Hausdorff
            (Thm.~\ref{thm:Quotient_by_Compact_T2_Group_Preserves_T2}) and
            second countable
            (Thm.~\ref{thm:Quotient_by_Compact_T2_Group_Preserves_Sec_Count}).
            Thus, $\mathbb{RP}^{n}$ is Hausdorff and second countable.
        \end{solution}
        \begin{solution}
        \end{solution}
        \begin{problem}
            Show that if $\topspace{X}$ is a locally Euclidean connected
            Hausdorff space, then it is a manifold if and only if it is
            paracompact.
        \end{problem}
        \begin{solution}
            If $\topspace{X}$ is a topological manifold, then it is locally
            Euclidean, Hausdorff, and second countable
            (Def.~\ref{def:Topological_Manifold}). But if $X$ is locally
            Euclidean, then it is locally metrizable
            (Thm.~\ref{thm:Loc_Euc_Implies_Loc_Met}). But then
            $\topspace{X}$ is a locally metrizable Hausdorff space that is
            second countable, and second countable spaces have a $\sigma$
            locally finite basis
            (Thm.~\ref{thm:Second_Countable_Implies_Sigma_Loc_Fin_Basis}).
            But then $X$ is metrizable by the Nagata-Smirnov theorem
            (Thm.~\ref{thm:Nagata_Smirnov_Metrization_Theorem}). But if
            $X$ is metrizable, then it is paracompact by the
            Smirnov theorem (Thm.~\ref{thm:Smirnov_Metrization_Theorem}). Hence,
            $\topspace{X}$ is paracompact. Going the other way, if $X$ is
            locally Euclidean, Hausdorff, paracompact, and connected, to show
            that it is a manifold suffices to show that it is second countable.
            But if $X$ is locally Euclidean, then there is a basis of
            precompact open subsets $\mathcal{B}$, each of which is homeomorphic
            to $\nspace$
            (Thm.~\ref{thm:Loc_Euc_Existence_of_Basis_of_Precompact_Balls}). But
            $X$ is paracompact, and hence there is a locally finite refinement
            $\Delta$ of $\mathcal{B}$ (Def.~\ref{def:Paracompact}). Let
            $\mathcal{U}_{0}$ be an element of $\Delta$. Since $\Delta$ is a
            refinement of $\mathcal{B}$, there is an element
            $\mathcal{V}\in\mathcal{B}$ such that
            $\mathcal{U}_{0}\subseteq\mathcal{V}$. But then
            $\closure{\mathcal{U}_{0}}\subseteq\closure{\mathcal{V}}$, and
            $\mathcal{V}$ is precompact, and thus $\closure{\mathcal{V}}$
            is compact. But then $\closure{\mathcal{U}_{0}}$ is a closed subset
            of a compact set, and is hence compact. For all $n\in\mathbb{N}$,
            define $\mathcal{U}_{n}$ by:
            \begin{equation}
                \mathcal{U}_{n}=\Big\{\,x\in{X}\;|\;
                    \exists_{A:\mathbb{Z}_{n}\rightarrow\Delta}\big(
                        \mathcal{U}_{0}=A_{0},\,x\in{A}_{n-1}\textrm{ and }
                        \forall_{i<n-1}(A_{i}\cap{A}_{i+1}\ne\emptyset)
                    \big)\,\Big\}
            \end{equation}
            That is, the set of all points that are separated from
            $\mathcal{U}_{0}$ by at most $n$ consecutive elements of $\Delta$.
            Then $\mathcal{U}_{n}$ is precompact. We prove by induction. The
            base case of $\mathcal{U}_{0}$ is true from the previous paragraph.
            Suppose it is true for $n\in\mathbb{N}$. Since $X$ is locally
            Euclidean, it is locally metrizable
            (Thm.\ref{thm:Loc_Euc_Implies_Loc_Met}). But then $X$ is locally
            metrizable, Hausdorff, and paracompact, and is therefore metrizable
            by the Smirnov metrization theorem
            (Thm.~\ref{thm:Smirnov_Metrization_Theorem}). But then if
            $\mathcal{C}\subseteq{X}$, then $\mathcal{C}$ is compact if and only
            if it is sequentially compact
            (Thm.~\ref{thm:Met_Space_Seq_Compact_iff_Compact}). But
            $\mathcal{U}_{n+1}$ is the union of $\closure{\mathcal{U}_{n}}$ and
            elements of $\mathcal{V}\in\Delta$ such that
            $\mathcal{V}\cap\mathcal{U}_{n}\ne\emptyset$. But every element of
            $\Delta$ is precompact, and since $\closure{\mathcal{U}_{n+1}}$ is
            not compact, there must be infinitely many such $\mathcal{V}$. But
            $\mathcal{V}\cap\mathcal{U}_{n}$ is non
            empty for all such $\mathcal{V}$, and hence by the axiom of choice
            there is a sequence $a:\mathbb{N}\rightarrow\mathcal{U}_{n}$ such
            that $a_{j}$ lies in a distinct $\mathcal{V}$ for all
            $j\in\mathbb{N}$. But $\closure{\mathcal{U}_{n}}$ is compact, and
            hence sequentially compact, and thus there is a convergent
            subsequence $a_{k}:\mathbb{N}\rightarrow\closure{\mathcal{U}_{n}}$
            with a limit $x$. But $x\in\closure{\mathcal{U}_{n}}$ and hence
            there is an open set $V\in\tau$ that has non-empty intersection with
            only finitely many elements of $\Delta$, since $\Delta$ is a locally
            finite refinement. But since $a_{k}$ converges to an element of $V$,
            there is an $N\in\mathbb{N}$ such that for all $j>N$,
            $a_{k_{j}}\in{V}$. But each $a_{k_{j}}$ lies in a different
            $\mathcal{V}\in\Delta$, and hence infinitely many elements of
            $\Delta$ have non-empty intersection with $V$, a contradiction.
            Hence, $\mathcal{U}_{n+1}$ is covered by finitely many elements of
            $\Delta$ and is therefore precompact. Moreover,
            $\bigcup\closure{\mathcal{U}_{n}}=X$. For if $y\in{X}$, let
            $x\in\mathcal{U}_{0}$ be any point. Then since $X$ is locally path
            connected and connected, it is path connected
            (Thm.~\ref{thm:Loc_Path_and_Con_Imply_Path_Con}).
            Let $\gamma:[0,1]\rightarrow{X}$ be a path from $x$ to $y$. But
            $[0,1]$ is compact and hence $\gamma\big[[0,1]\big]$ is a compact
            subset of $X$. But then it is covered by only finitely many elements
            of $\Delta$, and hence $y$ is contained in one of the
            $\mathcal{U}_{n}$. But this shows that $X$ is $\sigma$ compact
            (Def.~\ref{def:Sigma_Compact}). But if $X$ is $\sigma$ compact, then
            it is Lindel\"{o}f (Thm.~\ref{thm:Sigma_Compact_Implies_Lindelof}).
            But if $\topspace{X}$ is locally Euclidean, then there exists a
            basis $\mathcal{B}$ of precompact balls
            (Thm.~\ref{thm:Loc_Euc_Existence_of_Basis_of_Precompact_Balls}),
            each of which is homeomorphic of $\nspace$. But then $\mathcal{B}$
            is a cover of $X$, and since $X$ is Lindel\"{o}f there exists a
            countable subcover $\Lambda$ (Def.~\ref{def:Lindelof_Space}). But
            then $\Lambda$ is a countable collection of second countable open
            subspaces that cover $X$, and therefore $X$ is second countable
            (Thm.~\ref{thm:Count_Open_Cover_of_Sec_Count_Implies_Sec_Count}).
            Thus, $X$ is a manifold (Def.~\ref{def:Topological_Manifold}).
        \end{solution}
\end{document}