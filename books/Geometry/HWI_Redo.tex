%------------------------------------------------------------------------------%
\documentclass{article}                                                        %
%------------------------------Preamble----------------------------------------%
\makeatletter                                                                  %
    \def\input@path{{../../}}                                                  %
\makeatother                                                                   %
%---------------------------Packages----------------------------%
\usepackage{geometry}
\geometry{b5paper, margin=1.0in}
\usepackage[T1]{fontenc}
\usepackage{graphicx, float}            % Graphics/Images.
\usepackage{natbib}                     % For bibliographies.
\bibliographystyle{agsm}                % Bibliography style.
\usepackage[french, english]{babel}     % Language typesetting.
\usepackage[dvipsnames]{xcolor}         % Color names.
\usepackage{listings}                   % Verbatim-Like Tools.
\usepackage{mathtools, esint, mathrsfs} % amsmath and integrals.
\usepackage{amsthm, amsfonts, amssymb}  % Fonts and theorems.
\usepackage{tcolorbox}                  % Frames around theorems.
\usepackage{upgreek}                    % Non-Italic Greek.
\usepackage{fmtcount, etoolbox}         % For the \book{} command.
\usepackage[newparttoc]{titlesec}       % Formatting chapter, etc.
\usepackage{titletoc}                   % Allows \book in toc.
\usepackage[nottoc]{tocbibind}          % Bibliography in toc.
\usepackage[titles]{tocloft}            % ToC formatting.
\usepackage{pgfplots, tikz}             % Drawing/graphing tools.
\usepackage{imakeidx}                   % Used for index.
\usetikzlibrary{
    calc,                   % Calculating right angles and more.
    angles,                 % Drawing angles within triangles.
    arrows.meta,            % Latex and Stealth arrows.
    quotes,                 % Adding labels to angles.
    positioning,            % Relative positioning of nodes.
    decorations.markings,   % Adding arrows in the middle of a line.
    patterns,
    arrows
}                                       % Libraries for tikz.
\pgfplotsset{compat=1.9}                % Version of pgfplots.
\usepackage[font=scriptsize,
            labelformat=simple,
            labelsep=colon]{subcaption} % Subfigure captions.
\usepackage[font={scriptsize},
            hypcap=true,
            labelsep=colon]{caption}    % Figure captions.
\usepackage[pdftex,
            pdfauthor={Ryan Maguire},
            pdftitle={Mathematics and Physics},
            pdfsubject={Mathematics, Physics, Science},
            pdfkeywords={Mathematics, Physics, Computer Science, Biology},
            pdfproducer={LaTeX},
            pdfcreator={pdflatex}]{hyperref}
\hypersetup{
    colorlinks=true,
    linkcolor=blue,
    filecolor=magenta,
    urlcolor=Cerulean,
    citecolor=SkyBlue
}                           % Colors for hyperref.
\usepackage[toc,acronym,nogroupskip,nopostdot]{glossaries}
\usepackage{glossary-mcols}
%------------------------Theorem Styles-------------------------%
\theoremstyle{plain}
\newtheorem{theorem}{Theorem}[section]

% Define theorem style for default spacing and normal font.
\newtheoremstyle{normal}
    {\topsep}               % Amount of space above the theorem.
    {\topsep}               % Amount of space below the theorem.
    {}                      % Font used for body of theorem.
    {}                      % Measure of space to indent.
    {\bfseries}             % Font of the header of the theorem.
    {}                      % Punctuation between head and body.
    {.5em}                  % Space after theorem head.
    {}

% Italic header environment.
\newtheoremstyle{thmit}{\topsep}{\topsep}{}{}{\itshape}{}{0.5em}{}

% Define environments with italic headers.
\theoremstyle{thmit}
\newtheorem*{solution}{Solution}

% Define default environments.
\theoremstyle{normal}
\newtheorem{example}{Example}[section]
\newtheorem{definition}{Definition}[section]
\newtheorem{problem}{Problem}[section]

% Define framed environment.
\tcbuselibrary{most}
\newtcbtheorem[use counter*=theorem]{ftheorem}{Theorem}{%
    before=\par\vspace{2ex},
    boxsep=0.5\topsep,
    after=\par\vspace{2ex},
    colback=green!5,
    colframe=green!35!black,
    fonttitle=\bfseries\upshape%
}{thm}

\newtcbtheorem[auto counter, number within=section]{faxiom}{Axiom}{%
    before=\par\vspace{2ex},
    boxsep=0.5\topsep,
    after=\par\vspace{2ex},
    colback=Apricot!5,
    colframe=Apricot!35!black,
    fonttitle=\bfseries\upshape%
}{ax}

\newtcbtheorem[use counter*=definition]{fdefinition}{Definition}{%
    before=\par\vspace{2ex},
    boxsep=0.5\topsep,
    after=\par\vspace{2ex},
    colback=blue!5!white,
    colframe=blue!75!black,
    fonttitle=\bfseries\upshape%
}{def}

\newtcbtheorem[use counter*=example]{fexample}{Example}{%
    before=\par\vspace{2ex},
    boxsep=0.5\topsep,
    after=\par\vspace{2ex},
    colback=red!5!white,
    colframe=red!75!black,
    fonttitle=\bfseries\upshape%
}{ex}

\newtcbtheorem[auto counter, number within=section]{fnotation}{Notation}{%
    before=\par\vspace{2ex},
    boxsep=0.5\topsep,
    after=\par\vspace{2ex},
    colback=SeaGreen!5!white,
    colframe=SeaGreen!75!black,
    fonttitle=\bfseries\upshape%
}{not}

\newtcbtheorem[use counter*=remark]{fremark}{Remark}{%
    fonttitle=\bfseries\upshape,
    colback=Goldenrod!5!white,
    colframe=Goldenrod!75!black}{ex}

\newenvironment{bproof}{\textit{Proof.}}{\hfill$\square$}
\tcolorboxenvironment{bproof}{%
    blanker,
    breakable,
    left=3mm,
    before skip=5pt,
    after skip=10pt,
    borderline west={0.6mm}{0pt}{green!80!black}
}

\AtEndEnvironment{lexample}{$\hfill\textcolor{red}{\blacksquare}$}
\newtcbtheorem[use counter*=example]{lexample}{Example}{%
    empty,
    title={Example~\theexample},
    boxed title style={%
        empty,
        size=minimal,
        toprule=2pt,
        top=0.5\topsep,
    },
    coltitle=red,
    fonttitle=\bfseries,
    parbox=false,
    boxsep=0pt,
    before=\par\vspace{2ex},
    left=0pt,
    right=0pt,
    top=3ex,
    bottom=1ex,
    before=\par\vspace{2ex},
    after=\par\vspace{2ex},
    breakable,
    pad at break*=0mm,
    vfill before first,
    overlay unbroken={%
        \draw[red, line width=2pt]
            ([yshift=-1.2ex]title.south-|frame.west) to
            ([yshift=-1.2ex]title.south-|frame.east);
        },
    overlay first={%
        \draw[red, line width=2pt]
            ([yshift=-1.2ex]title.south-|frame.west) to
            ([yshift=-1.2ex]title.south-|frame.east);
    },
}{ex}

\AtEndEnvironment{ldefinition}{$\hfill\textcolor{Blue}{\blacksquare}$}
\newtcbtheorem[use counter*=definition]{ldefinition}{Definition}{%
    empty,
    title={Definition~\thedefinition:~{#1}},
    boxed title style={%
        empty,
        size=minimal,
        toprule=2pt,
        top=0.5\topsep,
    },
    coltitle=Blue,
    fonttitle=\bfseries,
    parbox=false,
    boxsep=0pt,
    before=\par\vspace{2ex},
    left=0pt,
    right=0pt,
    top=3ex,
    bottom=0pt,
    before=\par\vspace{2ex},
    after=\par\vspace{1ex},
    breakable,
    pad at break*=0mm,
    vfill before first,
    overlay unbroken={%
        \draw[Blue, line width=2pt]
            ([yshift=-1.2ex]title.south-|frame.west) to
            ([yshift=-1.2ex]title.south-|frame.east);
        },
    overlay first={%
        \draw[Blue, line width=2pt]
            ([yshift=-1.2ex]title.south-|frame.west) to
            ([yshift=-1.2ex]title.south-|frame.east);
    },
}{def}

\AtEndEnvironment{ltheorem}{$\hfill\textcolor{Green}{\blacksquare}$}
\newtcbtheorem[use counter*=theorem]{ltheorem}{Theorem}{%
    empty,
    title={Theorem~\thetheorem:~{#1}},
    boxed title style={%
        empty,
        size=minimal,
        toprule=2pt,
        top=0.5\topsep,
    },
    coltitle=Green,
    fonttitle=\bfseries,
    parbox=false,
    boxsep=0pt,
    before=\par\vspace{2ex},
    left=0pt,
    right=0pt,
    top=3ex,
    bottom=-1.5ex,
    breakable,
    pad at break*=0mm,
    vfill before first,
    overlay unbroken={%
        \draw[Green, line width=2pt]
            ([yshift=-1.2ex]title.south-|frame.west) to
            ([yshift=-1.2ex]title.south-|frame.east);},
    overlay first={%
        \draw[Green, line width=2pt]
            ([yshift=-1.2ex]title.south-|frame.west) to
            ([yshift=-1.2ex]title.south-|frame.east);
    }
}{thm}

%--------------------Declared Math Operators--------------------%
\DeclareMathOperator{\adjoint}{adj}         % Adjoint.
\DeclareMathOperator{\Card}{Card}           % Cardinality.
\DeclareMathOperator{\curl}{curl}           % Curl.
\DeclareMathOperator{\diam}{diam}           % Diameter.
\DeclareMathOperator{\dist}{dist}           % Distance.
\DeclareMathOperator{\Div}{div}             % Divergence.
\DeclareMathOperator{\Erf}{Erf}             % Error Function.
\DeclareMathOperator{\Erfc}{Erfc}           % Complementary Error Function.
\DeclareMathOperator{\Ext}{Ext}             % Exterior.
\DeclareMathOperator{\GCD}{GCD}             % Greatest common denominator.
\DeclareMathOperator{\grad}{grad}           % Gradient
\DeclareMathOperator{\Ima}{Im}              % Image.
\DeclareMathOperator{\Int}{Int}             % Interior.
\DeclareMathOperator{\LC}{LC}               % Leading coefficient.
\DeclareMathOperator{\LCM}{LCM}             % Least common multiple.
\DeclareMathOperator{\LM}{LM}               % Leading monomial.
\DeclareMathOperator{\LT}{LT}               % Leading term.
\DeclareMathOperator{\Mod}{mod}             % Modulus.
\DeclareMathOperator{\Mon}{Mon}             % Monomial.
\DeclareMathOperator{\multideg}{mutlideg}   % Multi-Degree (Graphs).
\DeclareMathOperator{\nul}{nul}             % Null space of operator.
\DeclareMathOperator{\Ord}{Ord}             % Ordinal of ordered set.
\DeclareMathOperator{\Prin}{Prin}           % Principal value.
\DeclareMathOperator{\proj}{proj}           % Projection.
\DeclareMathOperator{\Refl}{Refl}           % Reflection operator.
\DeclareMathOperator{\rk}{rk}               % Rank of operator.
\DeclareMathOperator{\sgn}{sgn}             % Sign of a number.
\DeclareMathOperator{\sinc}{sinc}           % Sinc function.
\DeclareMathOperator{\Span}{Span}           % Span of a set.
\DeclareMathOperator{\Spec}{Spec}           % Spectrum.
\DeclareMathOperator{\supp}{supp}           % Support
\DeclareMathOperator{\Tr}{Tr}               % Trace of matrix.
%--------------------Declared Math Symbols--------------------%
\DeclareMathSymbol{\minus}{\mathbin}{AMSa}{"39} % Unary minus sign.
%------------------------New Commands---------------------------%
\DeclarePairedDelimiter\norm{\lVert}{\rVert}
\DeclarePairedDelimiter\ceil{\lceil}{\rceil}
\DeclarePairedDelimiter\floor{\lfloor}{\rfloor}
\newcommand*\diff{\mathop{}\!\mathrm{d}}
\newcommand*\Diff[1]{\mathop{}\!\mathrm{d^#1}}
\renewcommand*{\glstextformat}[1]{\textcolor{RoyalBlue}{#1}}
\renewcommand{\glsnamefont}[1]{\textbf{#1}}
\renewcommand\labelitemii{$\circ$}
\renewcommand\thesubfigure{%
    \arabic{chapter}.\arabic{figure}.\arabic{subfigure}}
\addto\captionsenglish{\renewcommand{\figurename}{Fig.}}
\numberwithin{equation}{section}

\renewcommand{\vector}[1]{\boldsymbol{\mathrm{#1}}}

\newcommand{\uvector}[1]{\boldsymbol{\hat{\mathrm{#1}}}}
\newcommand{\topspace}[2][]{(#2,\tau_{#1})}
\newcommand{\measurespace}[2][]{(#2,\varSigma_{#1},\mu_{#1})}
\newcommand{\measurablespace}[2][]{(#2,\varSigma_{#1})}
\newcommand{\manifold}[2][]{(#2,\tau_{#1},\mathcal{A}_{#1})}
\newcommand{\tanspace}[2]{T_{#1}{#2}}
\newcommand{\cotanspace}[2]{T_{#1}^{*}{#2}}
\newcommand{\Ckspace}[3][\mathbb{R}]{C^{#2}(#3,#1)}
\newcommand{\funcspace}[2][\mathbb{R}]{\mathcal{F}(#2,#1)}
\newcommand{\smoothvecf}[1]{\mathfrak{X}(#1)}
\newcommand{\smoothonef}[1]{\mathfrak{X}^{*}(#1)}
\newcommand{\bracket}[2]{[#1,#2]}

%------------------------Book Command---------------------------%
\makeatletter
\renewcommand\@pnumwidth{1cm}
\newcounter{book}
\renewcommand\thebook{\@Roman\c@book}
\newcommand\book{%
    \if@openright
        \cleardoublepage
    \else
        \clearpage
    \fi
    \thispagestyle{plain}%
    \if@twocolumn
        \onecolumn
        \@tempswatrue
    \else
        \@tempswafalse
    \fi
    \null\vfil
    \secdef\@book\@sbook
}
\def\@book[#1]#2{%
    \refstepcounter{book}
    \addcontentsline{toc}{book}{\bookname\ \thebook:\hspace{1em}#1}
    \markboth{}{}
    {\centering
     \interlinepenalty\@M
     \normalfont
     \huge\bfseries\bookname\nobreakspace\thebook
     \par
     \vskip 20\p@
     \Huge\bfseries#2\par}%
    \@endbook}
\def\@sbook#1{%
    {\centering
     \interlinepenalty \@M
     \normalfont
     \Huge\bfseries#1\par}%
    \@endbook}
\def\@endbook{
    \vfil\newpage
        \if@twoside
            \if@openright
                \null
                \thispagestyle{empty}%
                \newpage
            \fi
        \fi
        \if@tempswa
            \twocolumn
        \fi
}
\newcommand*\l@book[2]{%
    \ifnum\c@tocdepth >-3\relax
        \addpenalty{-\@highpenalty}%
        \addvspace{2.25em\@plus\p@}%
        \setlength\@tempdima{3em}%
        \begingroup
            \parindent\z@\rightskip\@pnumwidth
            \parfillskip -\@pnumwidth
            {
                \leavevmode
                \Large\bfseries#1\hfill\hb@xt@\@pnumwidth{\hss#2}
            }
            \par
            \nobreak
            \global\@nobreaktrue
            \everypar{\global\@nobreakfalse\everypar{}}%
        \endgroup
    \fi}
\newcommand\bookname{Book}
\renewcommand{\thebook}{\texorpdfstring{\Numberstring{book}}{book}}
\providecommand*{\toclevel@book}{-2}
\makeatother
\titleformat{\part}[display]
    {\Large\bfseries}
    {\partname\nobreakspace\thepart}
    {0mm}
    {\Huge\bfseries}
\titlecontents{part}[0pt]
    {\large\bfseries}
    {\partname\ \thecontentslabel: \quad}
    {}
    {\hfill\contentspage}
\titlecontents{chapter}[0pt]
    {\bfseries}
    {\chaptername\ \thecontentslabel:\quad}
    {}
    {\hfill\contentspage}
\newglossarystyle{longpara}{%
    \setglossarystyle{long}%
    \renewenvironment{theglossary}{%
        \begin{longtable}[l]{{p{0.25\hsize}p{0.65\hsize}}}
    }{\end{longtable}}%
    \renewcommand{\glossentry}[2]{%
        \glstarget{##1}{\glossentryname{##1}}%
        &\glossentrydesc{##1}{~##2.}
        \tabularnewline%
        \tabularnewline
    }%
}
\newglossary[not-glg]{notation}{not-gls}{not-glo}{Notation}
\newcommand*{\newnotation}[4][]{%
    \newglossaryentry{#2}{type=notation, name={\textbf{#3}, },
                          text={#4}, description={#4},#1}%
}
%--------------------------LENGTHS------------------------------%
% Spacings for the Table of Contents.
\addtolength{\cftsecnumwidth}{1ex}
\addtolength{\cftsubsecindent}{1ex}
\addtolength{\cftsubsecnumwidth}{1ex}
\addtolength{\cftfignumwidth}{1ex}
\addtolength{\cfttabnumwidth}{1ex}

% Indent and paragraph spacing.
\setlength{\parindent}{0em}
\setlength{\parskip}{0em}                                                           %
%----------------------------Main Document-------------------------------------%
\begin{document}
    \title{Differential Topology}
    \author{Ryan Maguire}
    \date{\vspace{-5ex}}
    \maketitle
    \section{Preliminary Stuff}
        \begin{fdefinition}{Sequentially Continuous}{Sequentially_Continuous}
            A sequentially continuous function from a topological space
            $(X,\tau_{X})$ to a topological space $(Y,\tau_{Y})$ is a
            function $f:X\rightarrow{Y}$ such that for every sequence
            $a:\mathbb{N}\rightarrow{X}$ such that there exists an $x\in{X}$
            such that $a_{n}\rightarrow{x}$, it is true that
            $f(a_{n})\rightarrow{f}(x)$.
        \end{fdefinition}
        \begin{fdefinition}{Sequential Space}{Sequential_Space}
            A sequential topological space is a topological space
            $\topspace[X]{X}$ such that for every topological space
            $\topspace[Y]{Y}$ and for every sequentially continuous function
            $f:X\rightarrow{Y}$, it is true that $f$ is continuous.
        \end{fdefinition}
        Continuity implies sequential continuity, the converse need not always
        hold. If $(X,\tau)$ is first countable, the result is true.
        \begin{theorem}
            \label{thm:First_Countable_Implies_Seq_Open_are_Open}%
            If $(X,\tau)$ is a first countable topological space, if
            $\mathcal{U}\subseteq{X}$, then $\mathcal{U}$ is open if and only if
            for every sequence $a:\mathbb{N}\rightarrow{X}$ that coverges to an
            element $x\in\mathcal{U}$, then there exists an $N\in\mathbb{N}$
            such that for all $n>N$ it is true that $a_{n}\in\mathcal{U}$.
        \end{theorem}
        \begin{proof}
            One direction is the definition of convergence. In the other, since
            $\topspace{X}$ is first countable, for all $x\in\mathcal{U}$ there
            is a countable neighborhood basis $\mathcal{B}_{x}$. Let
            $\mathcal{V}_{x}:\mathbb{N}\rightarrow\mathcal{B}_{x}$ be a
            surjection. Then there exists an $N\in\mathbb{N}$ such that
            $\mathcal{V}_{x,N}\subseteq\mathcal{U}$. For suppose not and let
            $B_{n}$ be defined by:
            \begin{equation}
                B_{n}=\bigcap_{k\in\mathbb{Z}_{n}}\mathcal{V}_{x,k}
            \end{equation}
            Since $B_{n}$ is the intersection of finitely many open sets, it is
            open. Moreover it is non-empty since $x\in{B}_{n}$ for all $n$.
            But then $B_{n}$ is an open neighborhood about $x$, and since
            $\mathcal{B}_{x}$ is a neighborhood basis there is an element
            $V\in\mathcal{B}_{x}$ such that $V\subseteq{B}_{n}$. But
            $\mathcal{V}:\mathbb{N}\rightarrow\mathcal{B}_{x}$ is a surjection,
            and hence there is an $N\in\mathbb{N}$ such that
            $\mathcal{V}_{x,N}=V$. But by hypothesis,
            $\mathcal{V}_{x,N}\nsubseteq\mathcal{U}$ and hence there is an
            element $y\in\mathcal{V}_{x,N}$ such that $y\notin\mathcal{U}$.
            Therefore we have shown that the sequence $A_{n}$ defined by:
            \begin{equation}
                A_{n}=\{\,y\in{B}_{n}\;|\;y\not\in\mathcal{U}\,\}
            \end{equation}
            is non-empty for all $n\in\mathbb{N}$, and hence by the axiom of
            (countable) choice there is a sequence $y:\mathbb{N}\rightarrow{X}$
            such that $y_{n}\in{A}_{n}$ for all $n\in\mathbb{N}$. That is,
            for all $n\in\mathbb{N}$, $y_{n}\in{B}_{n}$ and
            $y_{n}\notin\mathcal{U}$. But if $y_{n}\in{B}_{n}$ for all
            $n\in\mathbb{N}$, then $y_{n}\rightarrow{x}$. For if not, then there
            is an open set $U\in\tau$ such that $x\in{U}$ and for all
            $N\in\mathbb{N}$ there exists an $n>N$ such that $y_{n}\notin{U}$.
            But $\mathcal{B}_{x}$ is a neighborhood basis of $x$, and hence
            there is a $V\in\mathcal{B}_{x}$ such that $V\subseteq{U}$. But
            $\mathcal{V}_{x}:\mathbb{N}\rightarrow\mathcal{B}_{x}$ is a
            surjection, and hence there is an $N\in\mathbb{N}$ such that
            $\mathcal{V}_{x,N}=V$. But then for all $n>N$,
            $B_{n}\subseteq\mathcal{V}_{x,N}$ and thus $B_{n}\subseteq{U}$.
            But then for all $n>N$, $y_{n}\in{U}$, a contradiction. Hence,
            $y_{n}\rightarrow{x}$. But for all $n\in\mathbb{N}$,
            $y_{n}\notin\mathcal{U}$. Thus $y_{n}$ is a sequence that converges
            to $x$, but is never contained inside $\mathcal{U}$, contradicting
            our hypothesis. Thus, for all $x\in\mathcal{U}$ there is an open
            subset $\mathcal{V}_{x,N}$ such that $x\in\mathcal{V}_{x,N}$ and
            $\mathcal{V}_{x,N}\subseteq\mathcal{U}$. But then $\mathcal{U}$ is
            simply the union over all of these open sets, and is thus open.
        \end{proof}
        \begin{theorem}
            If $(X,\tau)$ is a first countable topological space, then it is
            a sequential space.
        \end{theorem}
        \begin{proof}
            For suppose not. Then there is a topological space $\topspace[Y]{Y}$
            and a sequentially continuous function $f:X\rightarrow{Y}$ that is
            not continuous. But if $f$ is not continuous, then there is an open
            subset $\mathcal{V}\in\tau_{Y}$ such that
            $f^{\minus{1}}[\mathcal{V}]$ is not open in $X$. But
            $\topspace[X]{X}$ is first countable, and thus by
            Thm.~\ref{thm:First_Countable_Implies_Seq_Open_are_Open}, if
            $f^{\minus{1}}[\mathcal{V}]$ is not open, then there is a sequence
            $a:\mathbb{N}\rightarrow{X}$ that converges to a point
            $x\in{f}^{\minus{1}}[\mathcal{V}]$ such that for all
            $N\in\mathbb{N}$ there is an $n>N$ such that
            $a_{n}\notin{f}^{\minus{1}}[\mathcal{V}]$. But $f$ is sequentially
            continuous, and thus if $a_{n}\rightarrow{x}$, then
            $f(a_{n})\rightarrow{f}(x)$
            (Def.~\ref{def:Sequentially_Continuous}). But since $\mathcal{V}$ is
            open, there is an $N\in\mathbb{N}$ such that for all $n>N$ it is
            true that $f(a_{n})\in\mathcal{V}$. But then for all $n>N$ it is
            true that $a_{n}\in{f}^{\minus{1}}[\mathcal{V}]$, a contradiction.
            Thus, $f$ is continuous.
        \end{proof}
        \begin{fdefinition}{$\sigma$ Compact}{Sigma_Compact}
            A $\sigma$ compact topological space is a topological space
            $\topspace{X}$ such that there exists a sequence
            $K:\mathbb{N}\rightarrow\powset{X}$ of compact subsets of $X$ such
            that:
            \begin{equation*}
                X=\bigcup_{n\in\mathbb{N}}K_{n}
            \end{equation*}
        \end{fdefinition}
        \begin{fdefinition}{Locally Compact Topological Space}{Locally_Compact}
            A locally compact topological space is a topological space
            $\topspace{X}$ such that for all $x\in{X}$ there exists a compact
            subset $K\subseteq{X}$ and an open set $\mathcal{U}\in\tau$ such
            that $x\in\mathcal{U}$ and $\mathcal{U}\subseteq{K}$.
        \end{fdefinition}
        \begin{fdefinition}{Lindel\"{o}f Topology Space}{Lindelof_Space}
            A Lindel\"{o}f topological space is a topological space $(X,\tau)$
            such that for every open cover $\mathcal{O}$ of $X$ there exists a
            countable subcover.
        \end{fdefinition}
        \begin{theorem}
            \label{thm:Sigma_Compact_Implies_Lindelof}%
            If $\topspace{X}$ is a $\sigma$ compact topological space, then it
            is Lindel\"{o}f.
        \end{theorem}
        \begin{proof}
            For suppose not. Then there exists an open cover $\mathcal{O}$ of
            $X$ with no countable subcover. But $X$ is $\sigma$ compact, and
            hence there is a sequence $K:\mathbb{N}\rightarrow\powset{X}$ of
            compact sets such that $X=\bigcup{K}_{n}$
            (Def.~\ref{def:Sigma_Compact}). But the for all $n\in\mathbb{N}$,
            $K_{n}$ is a subset of $X$, and hence $\mathcal{O}$ is an open cover
            of $K_{n}$. But by hypothesis, $K_{n}$ is compact and hence there is
            a finite subcover $\Delta\subseteq\mathcal{O}$ of $K_{n}$. That is,
            if we define the sequence
            $A_{n}:\mathbb{N}\rightarrow\powset{\mathcal{O}}$ by:
            \begin{equation}
                A_{n}=\{\mathscr{D}\in\powset{\mathcal{O}}\;|\;
                    \mathscr{D}\textrm{ is finite and }
                    K_{n}\subseteq\bigcup\mathscr{D}\,\}
            \end{equation}
            Then for all $n\in\mathbb{N}$, $A_{n}$ is non-empty. Hence by the
            axiom of choice, there is a choice function
            $\Delta:\mathbb{N}\rightarrow\powset{\mathcal{O}}$ such that for all
            $n\in\mathbb{N}$, $\Delta_{n}$ is a finite open subcover of $K_{n}$.
            But then the union of all $\Delta_{n}$ is the countable union of
            finite collections, and is hence countable. But then this collection
            covers $K_{n}$ for all $n\in\mathbb{N}$, and hence covers
            $\bigcup{K}_{n}$. But $\bigcup{K}_{n}=X$, a contradiction since
            $\mathcal{O}$ has no countable subcover. Thus, $\topspace{X}$ is
            Lindel\"{o}f.
        \end{proof}
        This theorem reverses if we add locally compact. The requirement of
        local compactness can be seen by examining the irrational numbers with
        the subspace topology. This is Lindel\"{o}f (since it is second
        countable), but it is not $\sigma$ compact. For any compact subset of
        the irrationals must have empty interior, and by the Baire category
        theorem, the irrationals cannot be written as the countable union of
        nowhere dense subsets. Hence they can't possibly $\sigma$ compact.
        \begin{theorem}
            If $\topspace{X}$ is a locally compact Lindel\"{o}f space, then it
            is $\sigma$ compact.
        \end{theorem}
        \begin{proof}
            For if $\topspace{X}$ is locally compact, then for all $x\in{X}$
            there is a compact set $K\subseteq{X}$ and an open set
            $\mathcal{U}\in\tau$ such that $x\in\mathcal{U}$ and
            $\mathcal{U}\subseteq{K}$ (Def.~\ref{def:Locally_Compact}). Invoking
            the axiom of choice, there is a function
            $A:X\rightarrow\tau\times\powset{X}$ such that for all $x\in{X}$,
            $A_{x}=(\mathcal{U},K)$ where $x\in\mathcal{U}$ and
            $\mathcal{U}\subseteq{K}$, where $K$ is compact. But then the
            collection of all such $\mathcal{U}$ is an open cover of $X$. But
            $X$ is Lindel\"{o}f and hence there is a countable subcover. But
            then the subcollection of all $K$ form a countable collection of
            compact sets that cover $X$. Hence, $\topspace{X}$ is $\sigma$
            compact.
        \end{proof}
        \begin{fdefinition}{Sequentially Compact}{Sequentially_Compact}
            A sequentially compact topological space is a topological space
            $\topspace{X}$ such that for every sequence
            $a:\mathbb{N}\rightarrow{X}$ there exists a strictly increasing
            sequence $k:\mathbb{N}\rightarrow\mathbb{N}$ such that
            $a\circ{k}:\mathbb{N}\rightarrow{X}$ converges.
        \end{fdefinition}
        \begin{theorem}
            \label{thm:Met_Space_Seq_Compact_iff_Compact}%
            If $\topspace{X}$ is metrizable, and if $\mathcal{C}\subseteq{X}$,
            then $\mathcal{C}$ is compact if and only if it is sequentially
            compact.
        \end{theorem}
        \begin{proof}
            For if $\topspace[X]{X}$ is metrizable, then there is a metric
            $x:X\times{X}\rightarrow\nspace[]$ that induces the topology $\tau$.
            Suppose $X$ is compact and not sequentially compact. Then there is a
            sequence $a:\mathbb{N}\rightarrow{X}$ with no convergent
            subsequence. Then for al $x\in{X}$ there is an $r>0$ such that the
            open ball $B_{r}^{(X,d)}(x)$ is such that only finitely many
            $n\in\mathbb{N}$ imply $a_{n}\in{B}_{r}^{(X,d)}(x)$. Invoking
            choice, we get a function $r:X\rightarrow\mathbb{R}^{+}$. Let
            $\mathcal{O}$ be defined by:
            \begin{equation}
                \mathcal{O}=\{\,B_{r_{x}}^{(X,d)}(x)\;|\;x\in{X}\,\}
            \end{equation}
            Then $\mathcal{O}$ is an open cover of $X$. But $X$ is compact, and
            therefore there is a finite subcover $\Delta$. But for each
            $\mathcal{U}\in\Delta$, there are only finite many
            $n\in\mathbb{N}$ such that $a_{n}\in\mathcal{U}$. But every
            $n\in\mathbb{N}$ is such that $a_{n}\in\mathcal{U}$ for at least
            one $\mathcal{U}$ since these cover $X$, a contradiction since
            $\mathbb{N}$ is not finite. Thus, $\topspace{X}$ is sequentially
            compact. In the other direction, sequential compactness implies
            complete and totally bounded, which implies compact.
        \end{proof}
        This is not true in general. The long ling is sequentially compact but
        not compact, whereas the space of all functions
        $f:[0,1]\rightarrow[0,1]$ is compact (by Tychonoff) but not sequentially
        compact.
        \begin{fdefinition}{$\sigma$ locally finite basis}
                           {Sigma_Loc_Fin_Basis}
            A $\sigma$ locally finite basis of a topological space
            $\topspace{X}$ is a basis $\mathcal{B}$ of $X$ such that
            $\mathcal{B}$ is $\sigma$ locally finite. That is, there exists
            a sequence $B:\mathbb{N}\rightarrow\powset{X}$ such that
            $\mathcal{B}=\bigcup\{B_{n}\}$, and for all $n\in\mathbb{N}$ it is
            true that $B_{n}$ is locally finite.
        \end{fdefinition}
        \begin{theorem}
            \label{thm:Second_Countable_Implies_Sigma_Loc_Fin_Basis}%
            If $\topspace{X}$ is second countable, then it is has a $\sigma$
            locally finite basis.
        \end{theorem}
        \begin{proof}
            For if $\topspace{X}$ is has a countable basis $\mathcal{B}$. But
            then there is a surjection $B:\mathbb{N}\rightarrow\mathcal{B}$.
            But then for all $n\in\mathbb{N}$, $\{B_{n}\}$ is a finite subset
            of $\powset{X}$, and is hence locally finite, and
            $\mathcal{B}=\bigcup\{B_{n}\}$. Thus, $X$ is $\sigma$ locally
            finite.
        \end{proof}
        \begin{fdefinition}{Paracompact}{Paracompact}
            A paracompact topological space is a topological space
            $\topspace{X}$ such that for every open cover $\mathcal{O}$ of $X$,
            there exists a locally finite refinement $\Delta$ of $\mathcal{O}$.
        \end{fdefinition}
        \begin{ftheorem}{Nagata-Smirnov Metrization Theorem}
                        {Nagata_Smirnov_Metrization_Theorem}
            If $\topspace{X}$ is a topological space, then it is metrizable if
            and only if it is regular, Hausdorff, and has a $\sigma$ locally
            finite basis.
        \end{ftheorem}
        \begin{proof}
            Munkres.
        \end{proof}
        \begin{ftheorem}{Smirnov Metrization Theorem}
                        {Smirnov_Metrization_Theorem}
            If $\topspace{X}$ is a topological space, then it is metrizable if
            and only if it is Hausdorff, paracompact, and locally metrizable.
        \end{ftheorem}
        \begin{proof}
            Also Munkres.
        \end{proof}
        \begin{theorem}
            \label{thm:Count_Open_Cover_of_Sec_Count_Implies_Sec_Count}%
            If $\topspace{X}$ is a topological space, if
            $\mathcal{O}\subseteq\tau$ is a countable subset of open sets such
            that for all $\mathcal{U}\in\mathcal{O}$,
            $\topspace[\mathcal{U}]{\mathcal{U}}$ is second countable, where
            $\tau_{\mathcal{U}}$ is the subspace topology, then $\topspace{X}$
            is second countable.
        \end{theorem}
        \begin{proof}
            For let $\mathcal{B}$ be the collection of all of the basis for all
            of the $\mathcal{U}\in\mathcal{O}$. Since it is the countable union
            of countable sets, it is countable. But since $\mathcal{U}$ is open
            for all $\mathcal{U}\in\mathcal{O}$, this is a countable open cover
            of $X$. It suffices to show that it is a basis. Let
            $\mathcal{V}\in\tau$ be an open subset. But then:
            \begin{align}
                \mathcal{V}&=\mathcal{V}\cap{X}\\
                &=\mathcal{V}\cap\Big(
                    \bigcup_{\mathcal{U}\in\mathcal{O}}\mathcal{U}
                \Big)\\
                &=\bigcup_{\mathcal{U}\in\mathcal{O}}
                    \big(\mathcal{V}\cap\mathcal{U}\big)
            \end{align}
            But $\mathcal{V}\cap\mathcal{U}$ is an open subset of the subspace
            $\topspace[\mathcal{U}]{\mathcal{U}}$, and hence there is a subset
            of $\Delta_{\mathcal{V}}\subseteq\mathcal{B}_{\mathcal{U}}$ such
            that $\mathcal{V}\cap\mathcal{U}=\bigcup\Delta_{\mathcal{V}}$. But
            then the entire of $\mathcal{V}$ is the union of all such
            collections, each of which is contained in $\mathcal{B}$, and hence
            $\mathcal{V}$ can be written as the union of elements of
            $\mathcal{B}$. Thus, $\mathcal{B}$ is a basis.
        \end{proof}
        The requirement that the covering collection be open subspaces is
        crucial. The quotient space $\mathbb{R}/R$, where $R$ is the equivalence
        relation generated by $nRm$ for all $n,m\in\mathbb{N}$, can be thought
        of as a countable collection of rings all glued together at the origin.
        Hence, it can be covered by countably many closed subspaces, each of
        which is homeomorphic to $\nsphere[1]$ in the subspace topology, and
        hence each of which is second countable. However, this space
        $\mathbb{R}/R$ is not even first countable, let alone second countable.
        The point $[0]\in\mathbb{R}/R$ has no countable neighborhood basis.
        \begin{theorem}
            \label{thm:loc_path_con_imply_path_comps_open}%
            If $\topspace{X}$ is locally path connected, if $x\in{X}$, and if
            $\mathcal{U}\subseteq{X}$ is a path connected component of $X$,
            then $\mathcal{U}$ is open.
        \end{theorem}
        \begin{proof}
            For if $\topspace{X}$ is locally path connected, there is a basis
            $\mathcal{B}$ of open and path connected subsets of $X$. But if
            $\mathcal{U}\subseteq{X}$ is a path connected component, then for
            all $x,y\in\mathcal{U}$ there is a path
            $\gamma:[0,1]\rightarrow\mathcal{U}$ connected $x$ and $y$, and for
            all $z\in{X}$ such that $z\notin\mathcal{U}$, there is no path
            between $x$ and $z$. But $\mathcal{B}$ is a basis, and hence for
            all $x\in\mathcal{U}$ there is a $B\in\mathcal{B}$ such that
            $x\in{B}$. By choice, we get a function
            $B:\mathcal{U}\rightarrow\mathcal{B}$. Moreover, since
            $B\in\mathcal{B}$, it is path connected. But then for all
            $x\in\mathcal{U}$, $x\in{B}_{x}$ and $B_{x}$ is path connected,
            and hence $B_{x}\subseteq\mathcal{U}$. But then
            $\mathcal{U}=\bigcup{B}_{x}$, which is the union of open sets, and
            hence $\mathcal{U}$ is open.
        \end{proof}
        \begin{theorem}
            \label{thm:Loc_Path_and_Con_Imply_Path_Con}
            If $\topspace{X}$ is locally path connected and connected, then it
            is path connected.
        \end{theorem}
        \begin{proof}
            For if not then there are two points $x,y\in{X}$ with no path
            between them. But the since $\topspace{X}$ is locally path
            connected, the path connected components of $x$ and $y$ are open
            (Thm.~\ref{thm:loc_path_con_imply_path_comps_open}). But let
            $\mathcal{U}$ be the path connected component containing $x$, and
            let $\mathcal{V}$ be the union of all other path connected
            components. Then $\mathcal{V}$ is non-empty since $y\in\mathcal{V}$,
            and hence $\mathcal{U}$ and $\mathcal{V}$ are non-empty disjoint
            open subsets that cover $X$, a contradiction since $X$ is connected.
        \end{proof}
        \begin{fdefinition}{Locally Euclidean}{Locally_Euclidean}
            A locally Euclidean topological space is a topological space
            $\topspace{X}$ such that for all $x\in{X}$ there exists an open
            subset $\mathcal{U}\in\tau$ and an $n\in\mathbb{N}$ such that
            $x\in\mathcal{U}$ and $\mathcal{U}$ is homeomorphic to an open
            subset of $\nspace$.
        \end{fdefinition}
        \begin{theorem}
            \label{thm:Equiv_Def_Loc_Euclidean}%
            If $\topspace{X}$ is locally Euclidean, then for all $x\in{X}$ there
            is an open subset $\mathcal{U}\in\tau$ and an $n\in\mathbb{N}$ such
            that $\mathcal{U}$ is homeomorphic to $\nspace$.
        \end{theorem}
        \begin{proof}
            For if $\topspace{X}$ is locally Euclidean, then for all $x\in{X}$
            there is an open subset $\mathcal{V}\in\tau$ and an $n\in\mathbb{N}$
            such that $x\in\mathcal{V}$ and $\mathcal{V}$ is homeomorphic to an
            open subset of $\nspace$. But then there is an injective continuous
            open mapping $\varphi:\mathcal{V}\rightarrow\nspace$. Let
            $\vector{y}=\varphi(x)$. But since $\varphi$ is an open mapping,
            and since $\mathcal{V}$ is open, $\varphi[\mathcal{V}]$ is an open
            subset of $\nspace$. But $\vector{y}=\varphi(x)$, and hence
            $\vector{y}\in\varphi[\mathcal{V}]$. But if $\varphi[\mathcal{V}]$
            is open and $\vector{y}\in\varphi[\mathcal{V}]$, then there is an
            $r>0$ such that the open ball $B$ defined by:
            \begin{equation}
                B=\rball{r}{\metspace[\norm{\cdot}_{2}]{\nspace}}{\vector{y}}
            \end{equation}
            is contained in $\varphi[\mathcal{V}]$. But $\varphi$ is continuous,
            and open balls are open, and hence
            $\varphi^{\minus{1}}[B]$ is an open subset of $\mathcal{V}$.
            Moreover, since $\vector{y}\in{B}$,
            $x\in\varphi^{\minus{1}}[B]$.
            But $\varphi$ is an injective continuous open mapping, and thus the
            restriction of $\varphi$ to an open subset is an injective
            continuous open mapping. Hence, $\varphi|_{\varphi^{\minus{1}}[B]}$
            is a homeomorphism onto it's image, which is $B$. And since open
            balls in $\nspace$ are homeomorphic to $\nspace$,
            $\varphi^{\minus{1}}[B]$ is homeomorphic to $\nspace$. Thus, there
            is an open subset $\varphi^{\minus{1}}[B]$ containing $x$ and an
            $n\in\mathbb{N}$ such that $\varphi^{\minus{1}}[B]$ is homeomorphic
            to $\nspace$.
        \end{proof}
        \begin{theorem}
            \label{thm:Loc_Euc_Existence_of_Basis_of_nspace_Sets}%
            If $\topspace{X}$ is a locally Euclidean topological space, then
            there exists a basis $\mathcal{B}$ such that for all
            $\mathcal{U}\in\mathcal{B}$ there is an $n\in\mathbb{N}$ such that
            $\mathcal{U}$ is homeomorphic to $\nspace$.
        \end{theorem}
        \begin{proof}
            For if $\topspace{X}$ is locally Euclidean, then for all $x\in{X}$
            there is an open subset $\mathcal{U}_{x}\in\tau$ and an
            $n\in\mathbb{N}$ such that $\mathcal{U}_{x}$ is homeomorphic to
            $\nspace$ (Thm.~\ref{thm:Equiv_Def_Loc_Euclidean}). Let
            $\varphi_{x}:\mathcal{U}_{x}\rightarrow\nspace$ be such a
            homeomorphism, and let $\mathcal{B}_{x}$ be the set:
            \begin{equation}
                \mathcal{B}_{x}=\{\,
                    \varphi_{x}^{\minus{1}}\big[
                        \rball{r}{\metspace[\norm{\cdot}_{2}]{\nspace}}
                        {\vector{y}}\big]
                    \;|\;r>0,\,\vector{y}\in\nspace\,\}
            \end{equation}
            Then by construction, every element of $\mathcal{B}_{x}$ is
            homeomorphic to $\nspace$. Let $\mathcal{B}$ be the collection of
            all such sets for all $x\in{X}$. If $\mathcal{U}$, $\mathcal{V}$ are
            elements of $\mathcal{B}$, then there is an $x\in{X}$ such that
            $\mathcal{U}\in\mathcal{B}_{x}$. Let $\vector{y}$ be the image of
            $y$ under $\varphi_{x}$. But $\mathcal{U}\cap\mathcal{V}$ is open,
            and hence there is an $r>0$ such that the ball about
            $\vector{y}$ is contained in the image of
            $\varphi_{x}[\mathcal{U}\cap\mathcal{V}]$. But this $r$ ball is
            contained in $\mathcal{B}$. Hence, $\mathcal{B}$ is a basis.
        \end{proof}
        \begin{theorem}
            \label{thm:Loc_Euc_Existence_of_Basis_of_Precompact_Balls}%
            If $\topspace{X}$ is a locally Euclidean topological space, then
            there is a basis $\mathcal{B}$ such that for all
            $\mathcal{U}\in\mathcal{B}$ it is true that $\mathcal{U}$ is
            precompact in $\tau$ and such that there exists an $n\in\mathbb{N}$
            such that $\mathcal{U}$ is homeomorphic to $\nspace$.
        \end{theorem}
        \begin{proof}
            For by Thm.~\ref{thm:Loc_Euc_Existence_of_Basis_of_nspace_Sets},
            there is a basis $\mathcal{B}$ of $\tau$ such that for all
            $\mathcal{U}\in\mathcal{B}$ there is an $n\in\mathbb{N}$ such that
            $\mathcal{U}$ is homeomorphic to $\nspace$. Let
            $\varphi_{\mathcal{U}}:\mathcal{U}\rightarrow\nspace$ be such a
            homeomorphism. Define $\mathcal{B}_{x}$ by:
            \begin{equation}
                \mathcal{B}_{x}=\{\,
                    \varphi_{\mathcal{U}}^{\minus{1}}\big[
                        \rball{r}{\metspace[\norm{\cdot}_{2}]{\nspace}}
                        {\vector{y}}\big]\;|\;r>0,\,\vector{y}\in\nspace\,\}
            \end{equation}
            But by the Heine-Borel theorem, for all
            $\mathcal{V}\in\mathcal{B}_{x}$,
            $\closure[\nspace]{\varphi_{\mathcal{U}}[\mathcal{V}]}$ is compact
            in $\nspace$, and since $\varphi_{\mathcal{U}}$ is a homeomorphism,
            $\closure{\mathcal{V}}$ is compact in $\mathcal{U}$. But then
            $\closure{\mathcal{V}}$ is compact in $X$. The collection of all
            such $\mathcal{B}_{x}$ is thus a basis of precompact subsets that
            are homeomorphic to open balls in $\nspace$, which are therefore
            homeomorphic to $\nspace$.
        \end{proof}
        \begin{theorem}
            \label{thm:Loc_Euc_Implies_Loc_Met}%
            If $\topspace{X}$ is locally Euclidean, then it is locally
            metrizable.
        \end{theorem}
        \begin{proof}
            For if $\topspace{X}$ is locally Euclidean, then for all $x\in{X}$
            there is an $n\in\mathbb{N}$ and a $\mathcal{U}\in\tau$ such that
            $x\in\mathcal{U}$ and $\mathcal{U}$ is homeomorphic to
            $\nspace$ (Thm.~\ref{thm:Equiv_Def_Loc_Euclidean}). But
            $\nspace$ is metrizable, and thus $\mathcal{U}$ is metrizable.
            Hence, $X$ is locally metrizable.
        \end{proof}
        \begin{fdefinition}{Topological Manifold}{Topological_Manifold}
            A topological manifold is a locally Euclidean, Hausdorff, second
            countable topological space of constant dimension.
        \end{fdefinition}
    \section{Problems}
        \begin{problem}
            Show that $\mathbb{RP}^{n}$ is Hausdorff and second countable.
        \end{problem}
        \begin{problem}
            Show that if $\topspace{X}$ is a locally Euclidean connected
            Hausdorff space, then it is a manifold if and only if it is
            paracompact.
        \end{problem}
        \begin{solution}
            If $\topspace{X}$ is a topological manifold, then it is locally
            Euclidean, Hausdorff, and second countable
            (Def.~\ref{def:Topological_Manifold}). But if $X$ is locally
            Euclidean, then it is locally metrizable
            (Thm.~\ref{thm:Loc_Euc_Implies_Loc_Met}). But then
            $\topspace{X}$ is a locally metrizable Hausdorff space that is
            second countable, and second countable spaces have a $\sigma$
            locally finite basis
            (Thm.~\ref{thm:Second_Countable_Implies_Sigma_Loc_Fin_Basis}).
            But then $X$ is metrizable by the Nagata-Smirnov theorem
            (Thm.~\ref{thm:Nagata_Smirnov_Metrization_Theorem}). But if
            $X$ is metrizable, then it is paracompact by the
            Smirnov theorem (Thm.~\ref{thm:Smirnov_Metrization_Theorem}). Hence,
            $\topspace{X}$ is paracompact. Going the other way, if $X$ is
            locally Euclidean, Hausdorff, paracompact, and connected, to show
            that it is a manifold suffices to show that it is second countable.
            But if $X$ is locally Euclidean, then there is a basis of
            precompact open subsets $\mathcal{B}$, each of which is homeomorphic
            to $\nspace$
            (Thm.~\ref{thm:Loc_Euc_Existence_of_Basis_of_Precompact_Balls}). But
            $X$ is paracompact, and hence there is a locally finite refinement
            $\Delta$ of $\mathcal{B}$ (Def.~\ref{def:Paracompact}). Let
            $\mathcal{U}_{0}$ be an element of $\Delta$. Since $\Delta$ is a
            refinement of $\mathcal{B}$, there is an element
            $\mathcal{V}\in\mathcal{B}$ such that
            $\mathcal{U}_{0}\subseteq\mathcal{V}$. But then
            $\closure{\mathcal{U}_{0}}\subseteq\closure{\mathcal{V}}$, and
            $\mathcal{V}$ is precompact, and thus $\closure{\mathcal{V}}$
            is compact. But then $\closure{\mathcal{U}_{0}}$ is a closed subset
            of a compact set, and is hence compact. For all $n\in\mathbb{N}$,
            let $\mathcal{U}_{n}$ be defined as follows:
            \begin{equation}
                \mathcal{U}_{n}=\Big\{\,x\in{X}\;|\;
                    \exists_{A:\mathbb{Z}_{n}\rightarrow\Delta}\big(
                        \mathcal{U}_{0}=A_{0},\,x\in{A}_{n-1}\textrm{ and }
                        \forall_{i<n-1}(A_{i}\cap{A}_{i+1}\ne\emptyset)
                    \big)\,\Big\}
            \end{equation}
            That is, the set of all points that are separated from
            $\mathcal{U}_{0}$ by at most $n$ consecutive elements of $\Delta$.
            Then $\mathcal{U}_{n}$ is precompact. We prove by induction. The
            base case of $\mathcal{U}_{0}$ is true from the previous paragraph.
            Suppose it is true for $n\in\mathbb{N}$. Since $X$ is locally
            Euclidean, it is locally metrizable
            (Thm.\ref{thm:Loc_Euc_Implies_Loc_Met}). But then $X$ is locally
            metrizable, Hausdorff, and paracompact, and is therefore metrizable
            by the Smirnov metrization theorem
            (Thm.~\ref{thm:Smirnov_Metrization_Theorem}). But then if
            $\mathcal{C}\subseteq{X}$, then $\mathcal{C}$ is compact if and only
            if it is sequentially compact
            (Thm.~\ref{thm:Met_Space_Seq_Compact_iff_Compact}). But
            $\mathcal{U}_{n+1}$ is the union of $\closure{\mathcal{U}_{n}}$ and
            elements of $\mathcal{V}\in\Delta$ such that
            $\mathcal{V}\cap\mathcal{U}_{n}\ne\emptyset$. But every element of
            $\Delta$ is precompact, and since $\closure{\mathcal{U}_{n+1}}$ is
            not compact, there must be infinitely many such $\mathcal{V}$. But
            $\mathcal{V}\cap\mathcal{U}_{n}$ is non
            empty for all such $\mathcal{V}$, and hence by the axiom of choice
            there is a sequence $a:\mathbb{N}\rightarrow\mathcal{U}_{n}$ such
            that $a_{j}$ lies in a distinct $\mathcal{V}$ for all
            $j\in\mathbb{N}$. But $\closure{\mathcal{U}_{n}}$ is compact, and
            hence sequentially compact, and thus there is a convergent
            subsequence $a_{k}:\mathbb{N}\rightarrow\closure{\mathcal{U}_{n}}$
            with a limit $x$. But $x\in\closure{\mathcal{U}_{n}}$ and hence
            there is an open set $V\in\tau$ that has non-empty intersection with
            only finitely many elements of $\Delta$, since $\Delta$ is a locally
            finite refinement. But $a_{k}$ convergent to an element of $V$,
            there is an $N\in\mathbb{N}$ such that for all $j>N$,
            $a_{k_{j}}\in{V}$. But each $a_{k_{j}}$ lies in a different
            $\mathcal{V}\in\Delta$, and hence infinitely many elements of
            $\Delta$ have non-empty intersection with $V$, a contradiction.
            Hence, $\mathcal{U}_{n+1}$ is covered by finitely many elements of
            $\Delta$ and is therefore precompact. Moreover,
            $\bigcup\closure{\mathcal{U}_{n}}=X$. For if $y\in{X}$, let
            $x\in\mathcal{U}_{0}$ be any point. Then since $X$ is locally path
            connected and connected, it is path connected
            (Thm.~\ref{thm:Loc_Path_and_Con_Imply_Path_Con}).
            Let $\gamma:[0,1]\rightarrow{X}$ be a path from $x$ to $y$. But
            $[0,1]$ is compact and hence $\gamma\big[[0,1]\big]$ is a compact
            subset of $X$. But then it is covered by only finitely many elements
            of $\Delta$, and hence $y$ is contained in one of the
            $\mathcal{U}_{n}$. But this shows that $X$ is $\sigma$ compact
            (Def.~\ref{def:Sigma_Compact}). But if $X$ is $\sigma$ compact, then
            it is Lindel\"{o}f (Thm.~\ref{thm:Sigma_Compact_Implies_Lindelof}).
            But if $\topspace{X}$ is locally Euclidean, then there exists a
            basis $\mathcal{B}$ of precompact balls
            (Thm.~\ref{thm:Loc_Euc_Existence_of_Basis_of_Precompact_Balls}),
            each of which is homeomorphic of $\nspace$. But then $\mathcal{B}$
            is a cover of $X$, and since $X$ is Lindel\"{o}f there exists a
            countable subcover $\Lambda$ (Def.~\ref{def:Lindelof_Space}). But
            then $\Lambda$ is a countable collection of second countable open
            subspaces that cover $X$, and therefore $X$ is second countable
            (Thm.~\ref{thm:Count_Open_Cover_of_Sec_Count_Implies_Sec_Count}).
            Thus, $X$ is a manifold (Def.~\ref{def:Topological_Manifold}).
        \end{solution}
        \begin{problem}
            Show that 
        \end{problem}
\end{document}