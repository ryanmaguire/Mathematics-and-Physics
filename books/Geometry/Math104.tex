%------------------------------------------------------------------------------%
\documentclass{article}                                                        %
%------------------------------Preamble----------------------------------------%
\makeatletter                                                                  %
    \def\input@path{{../../}}                                                  %
\makeatother                                                                   %
%---------------------------Packages----------------------------%
\usepackage{geometry}
\geometry{b5paper, margin=1.0in}
\usepackage[T1]{fontenc}
\usepackage{graphicx, float}            % Graphics/Images.
\usepackage{natbib}                     % For bibliographies.
\bibliographystyle{agsm}                % Bibliography style.
\usepackage[french, english]{babel}     % Language typesetting.
\usepackage[dvipsnames]{xcolor}         % Color names.
\usepackage{listings, lstlinebgrd}      % Verbatim-Like Tools.
\usepackage{mathtools, esint, mathrsfs} % amsmath and integrals.
\usepackage{amsthm, amsfonts}           % Fonts and theorems.
\usepackage{tabularx}
\usepackage{tcolorbox}                  % Frames around theorems.
\usepackage{upgreek}                    % Non-Italic Greek.
\usepackage{paracol}                    % Two-column styling.
\usepackage{wrapfig}                    % Wrap text around figure.
\usepackage{fmtcount, etoolbox}         % For the \book{} command.
\usepackage[newparttoc]{titlesec}       % Formatting chapter, etc.
\usepackage{titletoc}                   % Allows \book in toc.
\usepackage[nottoc]{tocbibind}          % Bibliography in toc.
\usepackage[titles]{tocloft}            % ToC formatting.
\usepackage{multicol, enumitem}         % Multi-column/enumerate.
\usepackage{import}                     % Import external files.
\usepackage{pgfplots, tikz}             % Drawing/graphing tools.
\usetikzlibrary{
    calc,                   % Calculating right angles and more.
    angles,                 % Drawing angles within triangles.
    arrows.meta,            % Latex and Stealth arrows.
    quotes,                 % Adding labels to angles.
    positioning,            % Relative positioning of nodes.
    decorations.markings,   % Adding arrows in the middle of a line.
    patterns,
    arrows,
    shapes,
    shapes.geometric,
    cd,
    hobby,
    babel
}                                       % Libraries for tikz.
\pgfplotsset{compat=1.9}                % Version of pgfplots.
\usepackage[font=scriptsize,
            labelformat=simple,
            labelsep=colon]{subcaption} % Subfigure captions.
\usepackage[font={scriptsize},
            hypcap=true,
            labelsep=colon]{caption}    % Figure captions.
\usepackage{hyperref}                   % Allows for hyperlinks.
\hypersetup{
    colorlinks=true,
    linkcolor=blue,
    filecolor=magenta,
    urlcolor=Cerulean,
    citecolor=SkyBlue
}                           % Colors for hyperref.
\usepackage[toc,acronym,nogroupskip]{glossaries} % Glossaries and acronyms.
\usepackage[subpreambles=false]{standalone}      % Complileable sub files.

% Various font stuff from kiwi.
% Use this for Times text and Computer Modern math
%\usepackage{times}

% Quite nice
%\usepackage[charter, greekfamily=, greekuppercase=italicized]{mathdesign}
%\usepackage[utopia, greekuppercase=italicized]{mathdesign}    % Math is narrower

% Use this for Times text and math
%\usepackage{newtxtext}
%\usepackage[libertine,cmintegrals]{newtxmath}
%\usepackage{fix-cm}

%\usepackage{txfontsb}
% or
%\usepackage{mathptmx}

%\usepackage[scaled=0.92]{helvet}
%\renewcommand{\rmdefault}{ptm}

%\usepackage{mathpazo}    % add possibly `sc` and `osf` options
%\usepackage{eulervm}

%\usepackage{fourier}
%\renewcommand{\rmdefault}{ptm}
%\usepackage{mathptm}

%\usepackage{fontspec}
%\setmainfont{lmodern}

%\usepackage[varg]{txfonts}
%\usepackage{fouriernc}
%\usepackage{mathpazo}

%\usepackage{bookman}
%\usepackage[scaled]{uarial}
%\usepackage[scaled]{helvet}
%\renewcommand*\familydefault{\sfdefault}
%\usepackage[math]{anttor}

%\newcommand\fgeorgia{\fontfamily{jvn}\selectfont}
%\newcommand\ftimes{\fontfamily{ptm}\selectfont}
%\newcommand\fhelvetica{\fontfamily{phv}\selectfont}
%\newcommand\fcourier{\fontfamily{pcr}\selectfont}
%\newcommand\fbookman{\fontfamily{pbk}\selectfont}
%\newcommand\fnewcentury{\fontfamily{pnc}\selectfont}
%\newcommand\fpalatino{\fontfamily{ppl}\selectfont}
%\newcommand\favantgarde{\fontfamily{pag}\selectfont}
%\newcommand\fnormal{\normalfont}
%\newcommand\fsize[1]{\ifnum#1>0\fontsize{#1}{#1}\selectfont\else\normalsize\fi}
%------------------------Theorem Styles-------------------------%
% Define theorem style for default spacing and normal font.
\newtheoremstyle{normal}
    {\topsep}               % Amount of space above the theorem.
    {\topsep}               % Amount of space below the theorem.
    {}                      % Font used for body of theorem.
    {}                      % Measure of space to indent.
    {\bfseries}             % Font of the header of the theorem.
    {}                      % Punctuation between head and body.
    {.5em}                  % Space after theorem head.
    {}

% Define theorem style for default spacing with italicized font.
\newtheoremstyle{normalit}{\topsep}{\topsep}
                {\itshape}{}{\bfseries}{}{.5em}{}

% Italic header environment.
\newtheoremstyle{thmit}{\topsep}{\topsep}{}{}{\itshape}{}{0.5em}{}

% Define italicized environments.
\theoremstyle{normalit}
\newtheorem{theorem}{Theorem}[section]
\newtheorem{lemma}{Lemma}[section]
\newtheorem{corollary}{Corollary}[section]
\newtheorem{proposition}{Proposition}[section]
\newtheorem*{theorem*}{Theorem}

% Define environments with italic headers.
\theoremstyle{thmit}
\newtheorem*{solution}{Solution}
\newtheorem*{fsolution}{Solution}

% Define default environments.
\theoremstyle{normal}
\newtheorem{example}{Example}[section]
\newtheorem{definition}{Definition}[section]
\newtheorem{problem}{Problem}[section]
\newtheorem{question}{Question}[section]
\newtheorem{remark}{Remark}[section]
\newtheorem{properties}{Properties}[section]
\newtheorem{notation}{Notation}[section]
\newtheorem{axiom}{Axiom}[section]
\newtheorem*{properties*}{Properties}
\newtheorem*{remark*}{Remark}
\newtheorem*{definition*}{Definition}
\theoremstyle{plain}

% Define framed environment.
\tcbuselibrary{most}
\newtcbtheorem[use counter*=theorem]{ftheorem}{Theorem}%
    {colback=green!5,colframe=green!35!black,
     fonttitle=\bfseries\upshape}{th}

\newtcbtheorem[use counter*=example]{fdefinition}{Definition}%
    {fonttitle=\bfseries\upshape,
     colback=blue!5!white,colframe=blue!75!black}{def}

\newtcbtheorem[use counter*=example]{fexample}{Example}%
    {fonttitle=\bfseries\upshape,
     colback=red!5!white,colframe=red!75!black}{ex}

\newtcbtheorem[use counter*=notation]{fnotation}{Notation}%
    {fonttitle=\bfseries\upshape,
     colback=SeaGreen!5!white,colframe=SeaGreen!75!black}{ex}

\newtcbtheorem[use counter*=corollary]{fcorollary}{Corollary}%
    {fonttitle=\bfseries\upshape,
     colback=Orchid!5!white,colframe=Orchid!75!black}{ex}

\newenvironment{bproof}{\textit{Proof.}}{\hfill$\square$}
\tcolorboxenvironment{bproof}{blanker,breakable,left=5mm,
                             before skip=10pt,after skip=10pt,
                             borderline west={1mm}{0pt}{red}}
\tcolorboxenvironment{fsolution}
    {enhanced jigsaw,colframe=cyan,interior hidden,breakable}

%--------------------Declared Math Operators--------------------%
\DeclareMathOperator{\Refl}{Refl}           % Reflection operator.
\DeclareMathOperator{\Span}{Span}           % Span of a set of vectors.
\DeclareMathOperator{\Card}{Card}           % Cardinality of set.
\DeclareMathOperator{\Ord}{Ord}             % Ordinal of ordered set.
\DeclareMathOperator{\Tr}{Tr}               % Trace of matrix.
\DeclareMathOperator{\adjoint}{adj}         % Adjoint of matrix.
\DeclareMathOperator{\rk}{rk}               % Rank of operator.
\DeclareMathOperator{\nul}{nul}             % Null space of operator.
\DeclareMathOperator{\sgn}{sgn}             % Sign of a number.
\DeclareMathOperator{\multideg}{mutlideg}   % Multi-Degree (Graphs).
\DeclareMathOperator{\GCD}{GCD}             % Greatest common denominator.
\DeclareMathOperator{\LM}{LM}               % Leading monomial
\DeclareMathOperator{\LC}{LC}               % Leading coefficient.
\DeclareMathOperator{\LT}{LT}               % Leading term.
\DeclareMathOperator{\LCM}{LCM}             % Least common multiple.
\DeclareMathOperator{\Mon}{Mon}             % Monomial.
\DeclareMathOperator{\Spec}{Spec}           % Spectrum.
\DeclareMathOperator{\proj}{proj}           % Projection.
\DeclareMathOperator{\comp}{comp}           % Component.
\DeclareMathOperator{\sinc}{sinc}           % Sinc function.
\DeclareMathOperator{\Ima}{Im}              % Image of operator.
\DeclareMathOperator{\Prin}{Prin}           % Principal value.
\DeclareMathOperator{\Mod}{mod}             % Modulus.
%------------------------New Commands---------------------------%
\DeclarePairedDelimiter\norm{\lVert}{\rVert}
\DeclarePairedDelimiter\ceil{\lceil}{\rceil}
\DeclarePairedDelimiter\floor{\lfloor}{\rfloor}
\newcommand*\diff{\mathop{}\!\mathrm{d}}
\newcommand*\Diff[1]{\mathop{}\!\mathrm{d^#1}}
\renewcommand{\mod}{\ \Mod}
\renewcommand*{\glstextformat}[1]{\textcolor{RoyalBlue}{#1}}
\renewcommand{\glsnamefont}[1]{\textbf{#1}}
\renewcommand\labelitemii{$\circ$}
\renewcommand\thesubfigure{\arabic{chapter}.\arabic{figure}}
\renewcommand\thesubfigure{%
    \arabic{chapter}.\arabic{figure}.\arabic{subfigure}}
\addto\captionsenglish{\renewcommand{\figurename}{Fig.}}
%------------------------Book Command---------------------------%
\makeatletter
\renewcommand\@pnumwidth{1cm}
\newcounter{book}
\renewcommand\thebook{\@Roman\c@book}
\newcommand\book{%
    \if@openright
        \cleardoublepage
    \else
        \clearpage
    \fi
    \thispagestyle{plain}%
    \if@twocolumn
        \onecolumn
        \@tempswatrue
    \else
        \@tempswafalse
    \fi
    \null\vfil
    \secdef\@book\@sbook
}
\def\@book[#1]#2{%
    \ifnum \c@secnumdepth >-3\relax
        \refstepcounter{book}%
        \addcontentsline{toc}{book}{
            \bookname\ \thebook:\hspace{1em}#1
        }
    \else
        \addcontentsline{toc}{book}{#1}%
    \fi
    \markboth{}{}%
    {\centering
     \interlinepenalty \@M
     \normalfont
     \ifnum \c@secnumdepth >-2\relax
       \huge\bfseries \bookname\nobreakspace\thebook
       \par
       \vskip 20\p@
     \fi
     \Huge \bfseries #2\par}%
    \@endbook}
\def\@sbook#1{%
    {\centering
     \interlinepenalty \@M
     \normalfont
     \Huge \bfseries #1\par}%
    \@endbook}
\def\@endbook{
    \vfil\newpage
        \if@twoside
            \if@openright
                \null
                \thispagestyle{empty}%
                \newpage
            \fi
        \fi
        \if@tempswa
            \twocolumn
        \fi
}
\newcommand*\l@book[2]{%
    \ifnum \c@tocdepth >-2\relax
        \addpenalty{-\@highpenalty}%
        \addvspace{2.25em \@plus\p@}%
        \setlength\@tempdima{3em}%
        \begingroup
            \parindent \z@ \rightskip \@pnumwidth
            \parfillskip -\@pnumwidth
            {
                \leavevmode
                \Large \bfseries #1\hfil \hb@xt@\@pnumwidth{
                    \hss #2
                }
            }
            \par
            \nobreak
            \global\@nobreaktrue
            \everypar{\global\@nobreakfalse\everypar{}}%
        \endgroup
    \fi}
\newcommand\bookname{Book}
\renewcommand{\thebook}{\texorpdfstring{\Numberstring{book}}{book}}
\providecommand*{\toclevel@book}{-2}
\makeatother
\titlecontents{chapter}[0pt]
    {\bfseries}
    {\chaptername\ \thecontentslabel:\quad}
    {}
    {\hfill\contentspage}
\titleformat{\part}[display]
    {\Large\bfseries}
    {\partname\nobreakspace\thepart}
    {0mm}
    {\Huge\bfseries}
    \titlecontents{part}[0pt]
    {\large\bfseries}
    {\partname\ \thecontentslabel: \quad}
    {}
    {\hfill\contentspage}
\newcommand{\MarkRightAngle}[4][.3cm]
    {\coordinate (tempa) at ($(#3)!#1!(#2)$);
     \coordinate (tempb) at ($(#3)!#1!(#4)$);
     \coordinate (tempc) at ($(tempa)!0.5!(tempb)$);%midpoint
     \draw (tempa) -- ($(#3)!2!(tempc)$) -- (tempb);}
%--------------------------LENGTHS------------------------------%
% Spacings for the Table of Contents.
\addtolength{\cftsecnumwidth}{1ex}
\addtolength{\cftsubsecindent}{1ex}
\addtolength{\cftsubsecnumwidth}{1ex}
\addtolength{\cftfignumwidth}{1ex}
\addtolength{\cfttabnumwidth}{1ex}

% Spacing for multi-column and enumerate environments.
\setlength{\multicolsep}{6pt}
\setlist[enumerate]{itemsep=0pt,topsep=3pt}

% Indent and paragraph spacing.
\setlength{\parindent}{0em}
\setlength{\parskip}{0em}                                                           %
%----------------------------Main Document-------------------------------------%
\begin{document}
    \title{Differential Topology}
    \author{Ryan Maguire}
    \date{\vspace{-5ex}}
    \maketitle
    \section{Manifolds}
        \begin{definition}
            An $n$ dimensional smooth manifold is a topological space $(M,\tau)$
            equipped with a collection $\mathcal{A}$ or ordered pairs
            $\{(\mathcal{U}_{\alpha},\varphi_{\alpha})\}$ such that
            $\mathcal{U}_{\alpha}$ are open and cover the space, and such that
            $\varphi_{\alpha}:\mathcal{U}_{\alpha}\rightarrow\mathcal{V}_{\alpha}$
            is a homeomorphism to an open subset $\mathcal{V}_{\alpha}$ of
            $\mathbb{R}^{n}$ and such that for when
            $\mathcal{U}_{\alpha}\cap\mathcal{U}_{\beta}\ne\emptyset$, the
            functions $\varphi_{\beta}\circ\varphi_{\alpha}^{\minus{1}}$
            are smooth.
        \end{definition}
        The notion of smoothness here is simply the smoothness of functions
        from $\mathbb{R}^{n}$ to itself. The collection $\mathcal{A}$ is called
        an atlas and $(\mathcal{U},\varphi)$ are called charts.
        \begin{example}
            If $\mathcal{U}$ is an open subset of $\mathbb{R}^{n}$ and
            $f:\mathcal{U}\rightarrow\mathbb{R}$ is smooth, then the
            graph of $f$ is a smooth manifold.
        \end{example}
        \begin{example}
            If $\mathcal{U}\subseteq\mathbb{R}^{n}$ is open, if
            $f:\mathcal{U}\rightarrow\mathbb{R}$ is smooth, if $c$ is in the
            range of $f$, and if $\textrm{grad}(f)$ is non-zero on all of
            $f^{\minus{1}}(c)$, then $f^{\minus{1}}(c)$ is a smooth manifold.
        \end{example}
        \begin{example}
            Define $f:\mathbb{R}^{3}\rightarrow\mathbb{R}$ by
            $f(\mathbf{x})=\norm{\mathbf{x}}_{2}^{2}$. For all
            $\mathbf{x}\in\mathbb{R}^{3}$ such that $\norm{\mathbf{x}}=1$ we
            have $\textrm{grad}(f)\ne{0}$, and thus $f^{\minus{1}}(\{1\})$ is a
            smooth manifold. This is the unit sphere.
        \end{example}
        \begin{example}
            The orthographic projection of the sphere maps:
            \begin{equation}
                (x,y)\mapsto(x,y,\sqrt{1-x^{2}-y^{2}})
            \end{equation}
            The stereographic projection also exists.
        \end{example}
        \begin{definition}
            The product manifold of manifolds $M_{1},\hdots,M_{k}$, where
            $M_{j}$ is an $n_{j}$ smooth manifold then
            $M=\prod_{j=1}^{k}M_{j}$ has a natural structure of an
            $n=n_{1}+\dots+n_{k}$ dimensional manifold.
        \end{definition}
        \begin{definition}
            The quotient manifold of a smooth manifold $M$ with an equivalence
            relation $\sim$ is a manifold structure on $M/\sim$ with the
            quotient topology. In particular when we have group actions.
        \end{definition}
        \begin{example}
            Consider the sphere $S^{2}$ with the equivalence relation $R$
            defined by $pRq$ if and only if $p=\pm{q}$. The quotient space
            $S^{2}/R$ is the real projective plane $\mathbb{RP}^{2}$. We can
            consider this by means of a group action on $\mathbb{Z}_{2}$ on
            $S^{2}$. Define $g\cdot{p}$ by $1\cdot{p}=p$ and
            $\minus{1}\cdot{p}=\minus{p}$. This is a group action of
            $\mathbb{Z}_{2}$ on $S^{2}$.
        \end{example}
        All of the examples thus far have been subsets of some Euclidean space
        $\mathbb{R}^{n}$. One natural question is whether or not there are
        smooth manifolds that do no live in some higher dimensional Euclidean
        space. That is, are there manifolds $M$ such that $M$ can not be
        embedded into some $\mathbb{R}^{m}$?
        \begin{definition}
            A function $\phi:M\rightarrow{N}$ is smooth if for all $p\in{M}$
            there is a chart $(\mathcal{U},\varphi)$ containing $p$ and a
            chart $(\mathcal{V},\psi)$ containing $\phi(p)$ such that
            $\psi\circ\phi\circ\varphi^{\minus{1}}$ is smooth.
        \end{definition}
        Studying maps between manifolds gives us new manifolds to study. In
        particular, there are submersions and embeddings, and in particular
        embedded submanifolds.
        \par\hfill\par
        Tangent spaces are another important topic in differential topology. The
        classic example is that of a sphere $S^{2}$ in $\mathbb{R}^{3}$. We draw
        the tangent plane to a point on a sphere, which is the best linear
        approximation of the sphere at that point. This notion relies on an
        ambient space (that of $\mathbb{R}^{3}$), but since we do not yet know
        if all manifolds can be embedded into such an ambient space, we need a
        new means of defining tangent spaces that agrees with our intuition.
        There are several ways of thinking of this:
        \begin{itemize}
            \item Tangent vectors are derivations on $C^{\infty}(p)$.
            \item Tangent vectors are equivalence classes of curves through $p$.
        \end{itemize}
        Given a function $\phi:M\rightarrow{N}$ that is smooth between two
        manifolds, there is a function $\diff\phi_{o}:T_{p}M\rightarrow{T}_{p}N$
        between these tangent spaces. We can further consider the collection of
        all tangent spaces at all points $p$ of $M$, forming the tangent bundle
        of $M$, denoted $TM$. Set theoretically, this is the disjoint union
        $TM=\coprod_{p}T_{p}M$, but we do \textbf{not} give this the disjoint
        union topology. There is a natural topology that can be endowed on $TM$.
        \begin{example}
            If we take $S^{1}$, the tangent bundle is
            $TM=S^{1}\times\mathbb{R}$. This is an example of a trivial bundle.
            Most tangent bundles are not of this form. That is, for an $m$
            dimensional manifold $M$, $TM$ is usual \textbf{not} equal to
            $M\times\mathbb{R}^{m}$. It will always have dimension $2m$.
        \end{example}
    \section{A Review of Topology}
        \begin{definition}
            Let $X$ be a set. A topology on $X$ is a collection $\tau$
            consisting of subsets of $X$ such that $\emptyset\in\tau$,
            $X\in\tau$, and $\tau$ is closed to arbitrary unions and finite
            intersections. The elements of $\tau$ are called the open sets of
            $X$. $(X,\tau)$ is called a topological space. The closed subsets
            of $X$ are sets of the form $X\setminus\mathcal{U}$, where
            $\mathcal{U}$ is open. That is, a set is closed if its complement is
            open.
        \end{definition}
        \begin{example}
            The discrete topology $\mathcal{P}(X)$, the power set on $X$, is
            always a topology. So is $\{\emptyset,X\}$, the trivial topology.
        \end{example}
        \begin{example}
            If $(X,d)$ is a metric space, and if $\mathcal{U}\subseteq{X}$ is
            such that for all $x\in\mathcal{U}$ there is an $\varepsilon>0$
            such that $B_{\varepsilon}(x)\subseteq\mathcal{U}$, then we call
            $\mathcal{U}$ open. The collection of all such sets forms a topology
            on $X$, called the metric topology.
        \end{example}
        \begin{definition}
            The interior of a subset $S\subseteq{X}$ of a topology space
            $(X,\tau)$ is the set $\textrm{Int}(X)$ defined to by the
            set of all $s\in{S}$ such that there is an open set
            $\mathcal{U}\subseteq{X}$ such that $s\in\mathcal{U}$ and
            $\mathcal{U}\subseteq{S}$.
        \end{definition}
        \begin{definition}
            The closure of a subset $S\subseteq{X}$ in a topological space
            $(X,\tau)$ is the intersection of all closed sets in $X$ that
            contain $S$, denoted $\textrm{Cl}(S)$.
        \end{definition}
        \begin{definition}
            A subset subset of $X$ is a set $S$ such that
            $\textrm{Cl}(S)=X$.
        \end{definition}
        \begin{definition}
            A continuous function is a function $f:X\rightarrow{Y}$ such that
            for all open $\mathcal{V}\subseteq{Y}$, the pre-image
            $f^{\minus{1}}(\mathcal{V})$ is open in $X$.
        \end{definition}
        \begin{definition}
            A homeomorphism is a bijective continuous function
            $f:X\rightarrow{Y}$ such that $f^{\minus{1}}$ is continuous.
        \end{definition}
        \begin{definition}
            A compact space is such that every open cover has a finite subcover.
        \end{definition}
        \begin{definition}
            A connected topological space is such that it can't be written as
            the non-trivial union of disjoint open subsets.
        \end{definition}
        \begin{definition}
            Path connected, for all $x,y\in{X}$ there is a continuous function
            $\gamma:[0,1]\rightarrow{X}$ such that $\gamma(0)=x$ and
            $\gamma(1)=y$.
        \end{definition}
        \begin{definition}
            A Hausdorff space, for all $x,y\in{X}$ there are disjoint open
            $\mathcal{U},\mathcal{V}$ such that $x\in\mathcal{U}$ and
            $y\in\mathcal{V}$.
        \end{definition}
        \begin{definition}
            A basis, collection $\mathscr{B}$ such that for all
            $\mathcal{U}\in\tau$, $\tau$ can be written as the union of the
            elements of $\mathscr{B}$.
        \end{definition}
        \subsection{New Spaces from Old}
            \begin{definition}
                Given a topological space $(X,\tau)$, and a subset
                $A\subseteq{X}$, the subspace topology is the topology
                $\tau_{A}$ defined by
                $\tau_{A}=\{A\cap\mathcal{U}|\mathcal{U}\in\tau\}$.
            \end{definition}
            \begin{definition}
                The product of spaces is a thing.
            \end{definition}
            \begin{definition}
                The quotient space of $(X,\tau)$ with respect to a set $Y$ and
                a surjective map $\pi:X\rightarrow{Y}$ is the topology generated
                by $\pi$ such that it is continuous.
            \end{definition}
            More generally, let $X$ be a topological space and $G$ a topological
            group with a product $G\times{X}\rightarrow{X}$ satisfying:
            \begin{align}
                (g_{1}*{g}_{2})\cdot{x}&=g_{1}\cdot(g_{2}\cdot{x})\\
                e*x&=x
            \end{align}
            We can define an equivalence relation $R$ by $pRq$ if there is a
            $g\in{G}$ such that $p=g\cdot{q}$. This gives us a quotient space.
            \begin{example}
                Consider $X=S^{2}$, the sphere, with $G=S^{1}$ as a group.
                Then $X/G$ is simply the unit interval.
            \end{example}
            Draw this out, the orbits are lattitudinal lines, plus the two
            poles.
        \subsection{Quotient Maps}
            Let $\pi:X\rightarrow{Y}$ be a quotient map (surjective and
            continuous). A set $C\subseteq{Y}$ is closed if and only if
            $\pi^{\minus{1}}(C)$ is closed in $X$. The composition of quotient
            maps is a quotient map. If $F$ is continuous, $F\circ\pi$ from $X$
            to $B$ is continuous.
    \section{Topological Manifold}
        \begin{definition}
            A topological space is said to be an $n$ dimensional topological
            manifold if the following hold:
            \begin{itemize}
                \item $M$ is Hausdorff.
                \item $M$ is second countable.
                \item $M$ is locally Euclidean of dimension $n$. For all
                      $p\in{M}$ there is an open set $\mathcal{U}$ and a
                      homeomorphism $\varphi:\mathcal{U}\rightarrow\mathcal{V}$
                      to an open subset $\mathcal{V}\subseteq\mathbb{R}^{n}$.
            \end{itemize}
            The pair $(\mathcal{U},\varphi)$ is called a chart. If the image of
            $\mathcal{U}$ by $\varphi$ is an open ball in $\mathbb{R}^{n}$ we
            call this a coordinate ball.
        \end{definition}
        \begin{example}
            Open subsets of $\mathbb{R}^{n}$ are $n$ dimensional manifolds.
            Moreover, any open subset of an $n$ dimensional manifold, given the
            subspace topology, will again be an $n$ dimensional manifold.
        \end{example}
        \begin{example}
            The graphs of continuous functions $f:X\rightarrow{Y}$ from
            topological manifolds of dimension $n$ and $m$, repsectively, are
            manifolds. That is, if $\mathcal{U}\subseteq{X}$ is open, and
            $f|_{\mathcal{U}}$ is the restriction of $f$, then
            $\{(p,f(p)|p\in\mathcal{U}\}$, equipped with the subspace topology
            induced by the product topology on $X\times{Y}$, is a topological
            manifold of dimesnion $n$. It is Hausdorff and second countable
            since it is the subspace of such spaces.
        \end{example}
\end{document}