%------------------------------------------------------------------------------%
\documentclass{article}                                                        %
%------------------------------Preamble----------------------------------------%
\makeatletter                                                                  %
    \def\input@path{{../../}}                                                  %
\makeatother                                                                   %
%---------------------------Packages----------------------------%
\usepackage{geometry}
\geometry{b5paper, margin=1.0in}
\usepackage[T1]{fontenc}
\usepackage{graphicx, float}            % Graphics/Images.
\usepackage{natbib}                     % For bibliographies.
\bibliographystyle{agsm}                % Bibliography style.
\usepackage[french, english]{babel}     % Language typesetting.
\usepackage[dvipsnames]{xcolor}         % Color names.
\usepackage{listings, lstlinebgrd}      % Verbatim-Like Tools.
\usepackage{mathtools, esint, mathrsfs} % amsmath and integrals.
\usepackage{amsthm, amsfonts}           % Fonts and theorems.
\usepackage{tabularx}
\usepackage{tcolorbox}                  % Frames around theorems.
\usepackage{upgreek}                    % Non-Italic Greek.
\usepackage{paracol}                    % Two-column styling.
\usepackage{wrapfig}                    % Wrap text around figure.
\usepackage{fmtcount, etoolbox}         % For the \book{} command.
\usepackage[newparttoc]{titlesec}       % Formatting chapter, etc.
\usepackage{titletoc}                   % Allows \book in toc.
\usepackage[nottoc]{tocbibind}          % Bibliography in toc.
\usepackage[titles]{tocloft}            % ToC formatting.
\usepackage{multicol, enumitem}         % Multi-column/enumerate.
\usepackage{import}                     % Import external files.
\usepackage{pgfplots, tikz}             % Drawing/graphing tools.
\usetikzlibrary{
    calc,                   % Calculating right angles and more.
    angles,                 % Drawing angles within triangles.
    arrows.meta,            % Latex and Stealth arrows.
    quotes,                 % Adding labels to angles.
    positioning,            % Relative positioning of nodes.
    decorations.markings,   % Adding arrows in the middle of a line.
    patterns,
    arrows,
    shapes,
    shapes.geometric,
    cd,
    hobby,
    babel
}                                       % Libraries for tikz.
\pgfplotsset{compat=1.9}                % Version of pgfplots.
\usepackage[font=scriptsize,
            labelformat=simple,
            labelsep=colon]{subcaption} % Subfigure captions.
\usepackage[font={scriptsize},
            hypcap=true,
            labelsep=colon]{caption}    % Figure captions.
\usepackage{hyperref}                   % Allows for hyperlinks.
\hypersetup{
    colorlinks=true,
    linkcolor=blue,
    filecolor=magenta,
    urlcolor=Cerulean,
    citecolor=SkyBlue
}                           % Colors for hyperref.
\usepackage[toc,acronym,nogroupskip]{glossaries} % Glossaries and acronyms.
\usepackage[subpreambles=false]{standalone}      % Complileable sub files.

% Various font stuff from kiwi.
% Use this for Times text and Computer Modern math
%\usepackage{times}

% Quite nice
%\usepackage[charter, greekfamily=, greekuppercase=italicized]{mathdesign}
%\usepackage[utopia, greekuppercase=italicized]{mathdesign}    % Math is narrower

% Use this for Times text and math
%\usepackage{newtxtext}
%\usepackage[libertine,cmintegrals]{newtxmath}
%\usepackage{fix-cm}

%\usepackage{txfontsb}
% or
%\usepackage{mathptmx}

%\usepackage[scaled=0.92]{helvet}
%\renewcommand{\rmdefault}{ptm}

%\usepackage{mathpazo}    % add possibly `sc` and `osf` options
%\usepackage{eulervm}

%\usepackage{fourier}
%\renewcommand{\rmdefault}{ptm}
%\usepackage{mathptm}

%\usepackage{fontspec}
%\setmainfont{lmodern}

%\usepackage[varg]{txfonts}
%\usepackage{fouriernc}
%\usepackage{mathpazo}

%\usepackage{bookman}
%\usepackage[scaled]{uarial}
%\usepackage[scaled]{helvet}
%\renewcommand*\familydefault{\sfdefault}
%\usepackage[math]{anttor}

%\newcommand\fgeorgia{\fontfamily{jvn}\selectfont}
%\newcommand\ftimes{\fontfamily{ptm}\selectfont}
%\newcommand\fhelvetica{\fontfamily{phv}\selectfont}
%\newcommand\fcourier{\fontfamily{pcr}\selectfont}
%\newcommand\fbookman{\fontfamily{pbk}\selectfont}
%\newcommand\fnewcentury{\fontfamily{pnc}\selectfont}
%\newcommand\fpalatino{\fontfamily{ppl}\selectfont}
%\newcommand\favantgarde{\fontfamily{pag}\selectfont}
%\newcommand\fnormal{\normalfont}
%\newcommand\fsize[1]{\ifnum#1>0\fontsize{#1}{#1}\selectfont\else\normalsize\fi}
%------------------------Theorem Styles-------------------------%
% Define theorem style for default spacing and normal font.
\newtheoremstyle{normal}
    {\topsep}               % Amount of space above the theorem.
    {\topsep}               % Amount of space below the theorem.
    {}                      % Font used for body of theorem.
    {}                      % Measure of space to indent.
    {\bfseries}             % Font of the header of the theorem.
    {}                      % Punctuation between head and body.
    {.5em}                  % Space after theorem head.
    {}

% Define theorem style for default spacing with italicized font.
\newtheoremstyle{normalit}{\topsep}{\topsep}
                {\itshape}{}{\bfseries}{}{.5em}{}

% Italic header environment.
\newtheoremstyle{thmit}{\topsep}{\topsep}{}{}{\itshape}{}{0.5em}{}

% Define italicized environments.
\theoremstyle{normalit}
\newtheorem{theorem}{Theorem}[section]
\newtheorem{lemma}{Lemma}[section]
\newtheorem{corollary}{Corollary}[section]
\newtheorem{proposition}{Proposition}[section]
\newtheorem*{theorem*}{Theorem}

% Define environments with italic headers.
\theoremstyle{thmit}
\newtheorem*{solution}{Solution}
\newtheorem*{fsolution}{Solution}

% Define default environments.
\theoremstyle{normal}
\newtheorem{example}{Example}[section]
\newtheorem{definition}{Definition}[section]
\newtheorem{problem}{Problem}[section]
\newtheorem{question}{Question}[section]
\newtheorem{remark}{Remark}[section]
\newtheorem{properties}{Properties}[section]
\newtheorem{notation}{Notation}[section]
\newtheorem{axiom}{Axiom}[section]
\newtheorem*{properties*}{Properties}
\newtheorem*{remark*}{Remark}
\newtheorem*{definition*}{Definition}
\theoremstyle{plain}

% Define framed environment.
\tcbuselibrary{most}
\newtcbtheorem[use counter*=theorem]{ftheorem}{Theorem}%
    {colback=green!5,colframe=green!35!black,
     fonttitle=\bfseries\upshape}{th}

\newtcbtheorem[use counter*=example]{fdefinition}{Definition}%
    {fonttitle=\bfseries\upshape,
     colback=blue!5!white,colframe=blue!75!black}{def}

\newtcbtheorem[use counter*=example]{fexample}{Example}%
    {fonttitle=\bfseries\upshape,
     colback=red!5!white,colframe=red!75!black}{ex}

\newtcbtheorem[use counter*=notation]{fnotation}{Notation}%
    {fonttitle=\bfseries\upshape,
     colback=SeaGreen!5!white,colframe=SeaGreen!75!black}{ex}

\newtcbtheorem[use counter*=corollary]{fcorollary}{Corollary}%
    {fonttitle=\bfseries\upshape,
     colback=Orchid!5!white,colframe=Orchid!75!black}{ex}

\newenvironment{bproof}{\textit{Proof.}}{\hfill$\square$}
\tcolorboxenvironment{bproof}{blanker,breakable,left=5mm,
                             before skip=10pt,after skip=10pt,
                             borderline west={1mm}{0pt}{red}}
\tcolorboxenvironment{fsolution}
    {enhanced jigsaw,colframe=cyan,interior hidden,breakable}

%--------------------Declared Math Operators--------------------%
\DeclareMathOperator{\Refl}{Refl}           % Reflection operator.
\DeclareMathOperator{\Span}{Span}           % Span of a set of vectors.
\DeclareMathOperator{\Card}{Card}           % Cardinality of set.
\DeclareMathOperator{\Ord}{Ord}             % Ordinal of ordered set.
\DeclareMathOperator{\Tr}{Tr}               % Trace of matrix.
\DeclareMathOperator{\adjoint}{adj}         % Adjoint of matrix.
\DeclareMathOperator{\rk}{rk}               % Rank of operator.
\DeclareMathOperator{\nul}{nul}             % Null space of operator.
\DeclareMathOperator{\sgn}{sgn}             % Sign of a number.
\DeclareMathOperator{\multideg}{mutlideg}   % Multi-Degree (Graphs).
\DeclareMathOperator{\GCD}{GCD}             % Greatest common denominator.
\DeclareMathOperator{\LM}{LM}               % Leading monomial
\DeclareMathOperator{\LC}{LC}               % Leading coefficient.
\DeclareMathOperator{\LT}{LT}               % Leading term.
\DeclareMathOperator{\LCM}{LCM}             % Least common multiple.
\DeclareMathOperator{\Mon}{Mon}             % Monomial.
\DeclareMathOperator{\Spec}{Spec}           % Spectrum.
\DeclareMathOperator{\proj}{proj}           % Projection.
\DeclareMathOperator{\comp}{comp}           % Component.
\DeclareMathOperator{\sinc}{sinc}           % Sinc function.
\DeclareMathOperator{\Ima}{Im}              % Image of operator.
\DeclareMathOperator{\Prin}{Prin}           % Principal value.
\DeclareMathOperator{\Mod}{mod}             % Modulus.
%------------------------New Commands---------------------------%
\DeclarePairedDelimiter\norm{\lVert}{\rVert}
\DeclarePairedDelimiter\ceil{\lceil}{\rceil}
\DeclarePairedDelimiter\floor{\lfloor}{\rfloor}
\newcommand*\diff{\mathop{}\!\mathrm{d}}
\newcommand*\Diff[1]{\mathop{}\!\mathrm{d^#1}}
\renewcommand{\mod}{\ \Mod}
\renewcommand*{\glstextformat}[1]{\textcolor{RoyalBlue}{#1}}
\renewcommand{\glsnamefont}[1]{\textbf{#1}}
\renewcommand\labelitemii{$\circ$}
\renewcommand\thesubfigure{\arabic{chapter}.\arabic{figure}}
\renewcommand\thesubfigure{%
    \arabic{chapter}.\arabic{figure}.\arabic{subfigure}}
\addto\captionsenglish{\renewcommand{\figurename}{Fig.}}
%------------------------Book Command---------------------------%
\makeatletter
\renewcommand\@pnumwidth{1cm}
\newcounter{book}
\renewcommand\thebook{\@Roman\c@book}
\newcommand\book{%
    \if@openright
        \cleardoublepage
    \else
        \clearpage
    \fi
    \thispagestyle{plain}%
    \if@twocolumn
        \onecolumn
        \@tempswatrue
    \else
        \@tempswafalse
    \fi
    \null\vfil
    \secdef\@book\@sbook
}
\def\@book[#1]#2{%
    \ifnum \c@secnumdepth >-3\relax
        \refstepcounter{book}%
        \addcontentsline{toc}{book}{
            \bookname\ \thebook:\hspace{1em}#1
        }
    \else
        \addcontentsline{toc}{book}{#1}%
    \fi
    \markboth{}{}%
    {\centering
     \interlinepenalty \@M
     \normalfont
     \ifnum \c@secnumdepth >-2\relax
       \huge\bfseries \bookname\nobreakspace\thebook
       \par
       \vskip 20\p@
     \fi
     \Huge \bfseries #2\par}%
    \@endbook}
\def\@sbook#1{%
    {\centering
     \interlinepenalty \@M
     \normalfont
     \Huge \bfseries #1\par}%
    \@endbook}
\def\@endbook{
    \vfil\newpage
        \if@twoside
            \if@openright
                \null
                \thispagestyle{empty}%
                \newpage
            \fi
        \fi
        \if@tempswa
            \twocolumn
        \fi
}
\newcommand*\l@book[2]{%
    \ifnum \c@tocdepth >-2\relax
        \addpenalty{-\@highpenalty}%
        \addvspace{2.25em \@plus\p@}%
        \setlength\@tempdima{3em}%
        \begingroup
            \parindent \z@ \rightskip \@pnumwidth
            \parfillskip -\@pnumwidth
            {
                \leavevmode
                \Large \bfseries #1\hfil \hb@xt@\@pnumwidth{
                    \hss #2
                }
            }
            \par
            \nobreak
            \global\@nobreaktrue
            \everypar{\global\@nobreakfalse\everypar{}}%
        \endgroup
    \fi}
\newcommand\bookname{Book}
\renewcommand{\thebook}{\texorpdfstring{\Numberstring{book}}{book}}
\providecommand*{\toclevel@book}{-2}
\makeatother
\titlecontents{chapter}[0pt]
    {\bfseries}
    {\chaptername\ \thecontentslabel:\quad}
    {}
    {\hfill\contentspage}
\titleformat{\part}[display]
    {\Large\bfseries}
    {\partname\nobreakspace\thepart}
    {0mm}
    {\Huge\bfseries}
    \titlecontents{part}[0pt]
    {\large\bfseries}
    {\partname\ \thecontentslabel: \quad}
    {}
    {\hfill\contentspage}
\newcommand{\MarkRightAngle}[4][.3cm]
    {\coordinate (tempa) at ($(#3)!#1!(#2)$);
     \coordinate (tempb) at ($(#3)!#1!(#4)$);
     \coordinate (tempc) at ($(tempa)!0.5!(tempb)$);%midpoint
     \draw (tempa) -- ($(#3)!2!(tempc)$) -- (tempb);}
%--------------------------LENGTHS------------------------------%
% Spacings for the Table of Contents.
\addtolength{\cftsecnumwidth}{1ex}
\addtolength{\cftsubsecindent}{1ex}
\addtolength{\cftsubsecnumwidth}{1ex}
\addtolength{\cftfignumwidth}{1ex}
\addtolength{\cfttabnumwidth}{1ex}

% Spacing for multi-column and enumerate environments.
\setlength{\multicolsep}{6pt}
\setlist[enumerate]{itemsep=0pt,topsep=3pt}

% Indent and paragraph spacing.
\setlength{\parindent}{0em}
\setlength{\parskip}{0em}                                                           %
%----------------------------Main Document-------------------------------------%
\begin{document}
    \title{Differential Topology}
    \author{Ryan Maguire}
    \date{\vspace{-5ex}}
    \maketitle
    \section{Preliminary Stuff}
        The following were presented as exercises in the appendix of Lee's
        textbook and can justify the use of sequences without invoking more
        complicated results pertaining to metrizability.
        \begin{definition}
            A sequentially continuous function from a topological space
            $(X,\tau_{X})$ to a topological space $(Y,\tau_{Y})$ is a
            function $f:X\rightarrow{Y}$ such that for every convergent sequence
            $a:\mathbb{N}\rightarrow{X}$, it is true that:
            \begin{equation*}
                \lim_{n\rightarrow\infty}f(a_{n})
                =f(\lim_{n\rightarrow\infty}a_{n})
            \end{equation*}
        \end{definition}
        \begin{definition}
            A sequential topological space is a topological space $(X,\tau)$
            such that for every topological space $(Y,\tau_{Y})$ and for every
            sequentially continuous function $f:X\rightarrow{Y}$, it is true
            that $f$ is continuous.
        \end{definition}
        Continuity implies sequential continuity, the converse need not always
        hold. If $(X,\tau)$ is first countable, the result is true.
        \begin{theorem}
            If $(X,\tau)$ is a first countable topological space, if
            $\mathcal{U}\subseteq{X}$, then $\mathcal{U}$ is open if and only if
            for every sequence $a:\mathbb{N}\rightarrow{X}$ such that there is
            an $x\in\mathcal{U}$ such that $a_{n}\rightarrow{x}$, then there
            exists an $N\in\mathbb{N}$ such that for all $n>N$, it is true that
            $a_{n}\in\mathcal{U}$.
        \end{theorem}
        \begin{proof}
            One direction is the definition of convergence. In the other, since
            $(X,\tau)$ is first countable, for all $x\in\mathcal{U}$ there is a
            countable neighborhood basis $\mathcal{B}_{x}$. Let
            $\mathcal{V}_{x}:\mathbb{N}\rightarrow\mathcal{B}_{x}$ be a
            bijection. Then there exists an $N\in\mathbb{N}$ such that
            $\mathcal{V}_{x,N}\subseteq\mathcal{U}$. For suppose not. Let
            $B_{n}$ be defined by:
            \begin{equation}
                B_{n}=\bigcap_{k\in\mathbb{Z}_{n}}\mathcal{V}_{x,k}
            \end{equation}
            Since $B_{n}$ is the intersection of finitely many open sets, it is
            open. Moreover it is non-empty since $x\in{B}_{n}$ for all $n$.
            But then $B_{n}$ is an open neighborhood about $x$, and since
            $\mathcal{B}_{x}$ is a neighborhood basis there is an element
            $\mathcal{V}_{x,N}$ such that $\mathcal{V}_{x,N}\subseteq{B}_{n}$.
            But by hypothesis, for any such set there is a
            $y_{n}\in\mathcal{V}_{x,N}$ such that $y_{n}\notin\mathcal{U}$.
            Thus, for all $n$ there is a $y_{n}\in{B}_{n}$ such that
            $y_{n}\notin\mathcal{U}$. Then $y_{n}\rightarrow{x}$ since for any
            open subset about $x$ there is an $N\in\mathbb{N}$ such that
            $\mathcal{V}_{x,N}$ sits inside this open set. But for all $n>N$,
            $y_{n}\in\mathcal{V}_{x,N}$. Thus $y_{n}$ is a sequence that
            converges to $x$, but is never contained inside $\mathcal{U}$,
            a contradiction. Thus, for all $x\in\mathcal{U}$ there is an open
            subset $\mathcal{V}_{x,N}$ such that $x\in\mathcal{V}_{x,N}$ and
            $\mathcal{V}_{x,N}\subseteq\mathcal{U}$. But then $\mathcal{U}$ is
            simply the union over all of these open sets, and is thus open.
        \end{proof}
        \begin{theorem}
            If $(X,\tau)$ is a first countable topological space, then it is
            a sequential space.
        \end{theorem}
        \begin{proof}
            For let $(Y,\tau_{Y})$ be a topological space, and let
            $f:X\rightarrow{Y}$ be a sequentially continuous function. Let
            $\mathcal{V}\in\tau_{Y}$ be an open subset of $Y$. If
            $f^{\minus{1}}[\mathcal{V}]=\emptyset$, we are done. If not, let
            $x\in{f}^{\minus{1}}[\mathcal{V}]$ and let
            $a:\mathbb{N}\rightarrow{X}$ be a sequence such that
            $a_{n}\rightarrow{x}$. But $f$ is sequentially continuous, and thus
            $f(a_{n})\rightarrow{f}(x)$. But since $\mathcal{V}$ is open, there
            is an $N\in\mathbb{N}$ such that for all $n>N$ it is true that
            $f(a_{n})\in\mathcal{V}$. But then for all $n>N$ it is true that
            $a_{n}\in{f}^{\minus{1}}[\mathcal{V}]$. Thus every sequence that
            converges to a point in $f^{\minus{1}}[\mathcal{V}]$ is eventually
            contained in $f^{\minus{1}}[\mathcal{V}]$, and thus by the
            previous theorem this set is open. Thus the pre-image of open is
            open, and hence $f$ is continuous.
        \end{proof}
    \section{Homework I: Part A}
    \begin{problem}
        Show that $\mathbb{RP}^{n}$ is Hausdorff and second countable.
    \end{problem}
    \begin{solution}
        $\mathbb{RP}^{n}$ can be seen as the quotient space of $S^{n}$ under the
        continuous action of $\mathbb{Z}_{2}^{\times}$ on $S^{n}$ defined by
        $1\cdot\vector{x}=\vector{x}$ and
        $\minus{1}\cdot\vector{x}=\minus\vector{x}$. Since
        $\mathbb{Z}_{2}^{\times}$ is a compact topological group (it's finite),
        and since $S^{n}$ is Hausdorff and second countable (since it's a
        subspace of $\mathbb{R}^{n}$), then $S^{n}/\mathbb{Z}_{2}^{\times}$ is
        Hausdorff and second countable.
        \par\hfill\par
        Alternatively, Let $[\mathbf{x}]$ and $[\mathbf{y}]$ be distinct
        elements in the quotient space $\mathbb{RP}^{n}$. Let $\varepsilon$ be
        defined by:
        \begin{equation}
            \varepsilon=\frac{1}{4}\cos^{\minus{1}}\Big(
                \frac{\langle\vector{x}|\vector{y}\rangle}
                     {\norm{\vector{x}}\norm{\vector{y}}}
            \Big)
        \end{equation}
        Consider the open cones $\Lambda_{1}$ and $\Lambda_{2}$ about
        the lines passing thought $\vector{x}$ and the origin, and $\vector{y}$
        and the origin, respectively, such that each line in $\Lambda_{1}$ has
        an angle of less than $\varepsilon$ with the line containing
        the origin and $\vector{x}$, and similarly for $\vector{y}$ and
        $\Lambda_{2}$. From the definition of $\varepsilon$, these cones
        intersect only at the origin. Hence, in $\mathbb{R}^{n+1}\setminus\{0\}$
        they are disjoint. Moreover, they are saturated, i.e.
        $\pi^{\minus{1}}\big(\pi(\Lambda_{i})\big)=\Lambda_{i}$. Thus the
        forward image is an open subset in the quotient space. But
        $\pi(\Lambda_{1})$ contains $[\vector{x}]$ and $\pi(\Lambda_{2})$
        contains $[\vector{y}]$, and moreover these projections are disjoint.
        Thus $[\vector{x}]$ and $[\vector{y}]$ can be separeted by open
        disjoint sets, and hence $\mathbb{RP}^{n}$ is Hausdorff.
        \par\hfill\par
        One final way, using the open sets $\mathcal{U}_{i}$ constructed in
        Lee's text, if $n>2$, then for any $[\vector{x}]$ and $[\vector{y}]$
        there is a $\mathcal{U}_{i}$ containing both of these points. Since the
        $\mathcal{U}_{i}$ are homeomorphic to an open subset of Euclidean space,
        they are Hausdorff, and hence $[\vector{x}]$ and $[\vector{y}]$ can be
        separeted by disjoint open sets. For the case of $n=2$ the only problem
        we run in to is when $[\vector{x}]$ represents the $x$ axis and
        $[\vector{y}]$ represents the $y$ axis since $[\vector{x}]$ does not lie
        in $\mathcal{U}_{x}$ and $[\vector{y}]$ is not contained in
        $\mathcal{U}_{y}$. Thus there is no $\mathcal{U}_{i}$ in this
        construction that contains both $[\vector{x}]$ and $[\vector{y}]$. Form
        a new set $\tilde{\mathcal{U}}$ defined by the complement of the line
        $y=x$. By identical arguments the projection $\mathcal{U}$ will be
        homeomorphic to an open subset of $\mathbb{R}^{2}$, however this set
        will contain $[\vector{x}]$ and $[\vector{y}]$, and hence all distinct
        points can be separeted. Thus, $\mathbb{RP}^{2}$ is Hausdorff.
        \par\hfill\par
        For second countability, we've shown that $\mathbb{RP}^{n}$ can be
        covered by $n+1$ open subset $\mathcal{U}_{i}$, each of which is
        homeomorphic to an open subset of $\mathbb{R}^{n}$. Thus each
        $\mathcal{U}_{i}$ is second countable, since second countability is
        preserved by homeomorphisms. Let $\mathcal{B}$ be the union of these
        $n+1$ bases. Let $\mathcal{V}$ be an open subset of $\mathbb{RP}^{n}$.
        Since $\mathcal{V}$ and each $\mathcal{U}_{i}$ is open,
        $\mathcal{V}\cap\mathcal{U}_{i}$ is open. But this intersection is a
        subset of $\mathcal{U}_{i}$, and hence there is a subset
        $\Delta_{i}\subseteq\mathcal{B}$ such that:
        \begin{equation}
            \mathcal{V}\cap\mathcal{U}_{i}=
                \bigcup_{\mathcal{O}\in\Delta}\mathcal{O}
        \end{equation}
        But then:
        \begin{equation}
            \mathcal{V}=
            \bigcup_{i\in\mathbb{Z}_{n+1}}\big(
                \mathcal{V}\cap\mathcal{U}_{i}
            \big)
            =\bigcup_{i\in\mathbb{Z}_{n+1}}\Big(
                \bigcup_{\mathcal{O}\in\Delta_{i}}\mathcal{O}\Big)
        \end{equation}
        Thus every open subset can be written as the union of elements of
        $\mathcal{B}$. Since $\mathcal{B}$ is the finite union of countable
        sets, it is therefore countable. Hence, $\mathcal{B}$ is a countable
        basis.
    \end{solution}
\end{document}