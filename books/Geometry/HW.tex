%------------------------------------------------------------------------------%
\documentclass{article}                                                        %
%------------------------------Preamble----------------------------------------%
\makeatletter                                                                  %
    \def\input@path{{../../}}                                                  %
\makeatother                                                                   %
%---------------------------Packages----------------------------%
\usepackage{geometry}
\geometry{b5paper, margin=1.0in}
\usepackage[T1]{fontenc}
\usepackage{graphicx, float}            % Graphics/Images.
\usepackage{natbib}                     % For bibliographies.
\bibliographystyle{agsm}                % Bibliography style.
\usepackage[french, english]{babel}     % Language typesetting.
\usepackage[dvipsnames]{xcolor}         % Color names.
\usepackage{listings}                   % Verbatim-Like Tools.
\usepackage{mathtools, esint, mathrsfs} % amsmath and integrals.
\usepackage{amsthm, amsfonts, amssymb}  % Fonts and theorems.
\usepackage{tcolorbox}                  % Frames around theorems.
\usepackage{upgreek}                    % Non-Italic Greek.
\usepackage{fmtcount, etoolbox}         % For the \book{} command.
\usepackage[newparttoc]{titlesec}       % Formatting chapter, etc.
\usepackage{titletoc}                   % Allows \book in toc.
\usepackage[nottoc]{tocbibind}          % Bibliography in toc.
\usepackage[titles]{tocloft}            % ToC formatting.
\usepackage{pgfplots, tikz}             % Drawing/graphing tools.
\usepackage{imakeidx}                   % Used for index.
\usetikzlibrary{
    calc,                   % Calculating right angles and more.
    angles,                 % Drawing angles within triangles.
    arrows.meta,            % Latex and Stealth arrows.
    quotes,                 % Adding labels to angles.
    positioning,            % Relative positioning of nodes.
    decorations.markings,   % Adding arrows in the middle of a line.
    patterns,
    arrows
}                                       % Libraries for tikz.
\pgfplotsset{compat=1.9}                % Version of pgfplots.
\usepackage[font=scriptsize,
            labelformat=simple,
            labelsep=colon]{subcaption} % Subfigure captions.
\usepackage[font={scriptsize},
            hypcap=true,
            labelsep=colon]{caption}    % Figure captions.
\usepackage[pdftex,
            pdfauthor={Ryan Maguire},
            pdftitle={Mathematics and Physics},
            pdfsubject={Mathematics, Physics, Science},
            pdfkeywords={Mathematics, Physics, Computer Science, Biology},
            pdfproducer={LaTeX},
            pdfcreator={pdflatex}]{hyperref}
\hypersetup{
    colorlinks=true,
    linkcolor=blue,
    filecolor=magenta,
    urlcolor=Cerulean,
    citecolor=SkyBlue
}                           % Colors for hyperref.
\usepackage[toc,acronym,nogroupskip,nopostdot]{glossaries}
\usepackage{glossary-mcols}
%------------------------Theorem Styles-------------------------%
\theoremstyle{plain}
\newtheorem{theorem}{Theorem}[section]

% Define theorem style for default spacing and normal font.
\newtheoremstyle{normal}
    {\topsep}               % Amount of space above the theorem.
    {\topsep}               % Amount of space below the theorem.
    {}                      % Font used for body of theorem.
    {}                      % Measure of space to indent.
    {\bfseries}             % Font of the header of the theorem.
    {}                      % Punctuation between head and body.
    {.5em}                  % Space after theorem head.
    {}

% Italic header environment.
\newtheoremstyle{thmit}{\topsep}{\topsep}{}{}{\itshape}{}{0.5em}{}

% Define environments with italic headers.
\theoremstyle{thmit}
\newtheorem*{solution}{Solution}

% Define default environments.
\theoremstyle{normal}
\newtheorem{example}{Example}[section]
\newtheorem{definition}{Definition}[section]
\newtheorem{problem}{Problem}[section]

% Define framed environment.
\tcbuselibrary{most}
\newtcbtheorem[use counter*=theorem]{ftheorem}{Theorem}{%
    before=\par\vspace{2ex},
    boxsep=0.5\topsep,
    after=\par\vspace{2ex},
    colback=green!5,
    colframe=green!35!black,
    fonttitle=\bfseries\upshape%
}{thm}

\newtcbtheorem[auto counter, number within=section]{faxiom}{Axiom}{%
    before=\par\vspace{2ex},
    boxsep=0.5\topsep,
    after=\par\vspace{2ex},
    colback=Apricot!5,
    colframe=Apricot!35!black,
    fonttitle=\bfseries\upshape%
}{ax}

\newtcbtheorem[use counter*=definition]{fdefinition}{Definition}{%
    before=\par\vspace{2ex},
    boxsep=0.5\topsep,
    after=\par\vspace{2ex},
    colback=blue!5!white,
    colframe=blue!75!black,
    fonttitle=\bfseries\upshape%
}{def}

\newtcbtheorem[use counter*=example]{fexample}{Example}{%
    before=\par\vspace{2ex},
    boxsep=0.5\topsep,
    after=\par\vspace{2ex},
    colback=red!5!white,
    colframe=red!75!black,
    fonttitle=\bfseries\upshape%
}{ex}

\newtcbtheorem[auto counter, number within=section]{fnotation}{Notation}{%
    before=\par\vspace{2ex},
    boxsep=0.5\topsep,
    after=\par\vspace{2ex},
    colback=SeaGreen!5!white,
    colframe=SeaGreen!75!black,
    fonttitle=\bfseries\upshape%
}{not}

\newtcbtheorem[use counter*=remark]{fremark}{Remark}{%
    fonttitle=\bfseries\upshape,
    colback=Goldenrod!5!white,
    colframe=Goldenrod!75!black}{ex}

\newenvironment{bproof}{\textit{Proof.}}{\hfill$\square$}
\tcolorboxenvironment{bproof}{%
    blanker,
    breakable,
    left=3mm,
    before skip=5pt,
    after skip=10pt,
    borderline west={0.6mm}{0pt}{green!80!black}
}

\AtEndEnvironment{lexample}{$\hfill\textcolor{red}{\blacksquare}$}
\newtcbtheorem[use counter*=example]{lexample}{Example}{%
    empty,
    title={Example~\theexample},
    boxed title style={%
        empty,
        size=minimal,
        toprule=2pt,
        top=0.5\topsep,
    },
    coltitle=red,
    fonttitle=\bfseries,
    parbox=false,
    boxsep=0pt,
    before=\par\vspace{2ex},
    left=0pt,
    right=0pt,
    top=3ex,
    bottom=1ex,
    before=\par\vspace{2ex},
    after=\par\vspace{2ex},
    breakable,
    pad at break*=0mm,
    vfill before first,
    overlay unbroken={%
        \draw[red, line width=2pt]
            ([yshift=-1.2ex]title.south-|frame.west) to
            ([yshift=-1.2ex]title.south-|frame.east);
        },
    overlay first={%
        \draw[red, line width=2pt]
            ([yshift=-1.2ex]title.south-|frame.west) to
            ([yshift=-1.2ex]title.south-|frame.east);
    },
}{ex}

\AtEndEnvironment{ldefinition}{$\hfill\textcolor{Blue}{\blacksquare}$}
\newtcbtheorem[use counter*=definition]{ldefinition}{Definition}{%
    empty,
    title={Definition~\thedefinition:~{#1}},
    boxed title style={%
        empty,
        size=minimal,
        toprule=2pt,
        top=0.5\topsep,
    },
    coltitle=Blue,
    fonttitle=\bfseries,
    parbox=false,
    boxsep=0pt,
    before=\par\vspace{2ex},
    left=0pt,
    right=0pt,
    top=3ex,
    bottom=0pt,
    before=\par\vspace{2ex},
    after=\par\vspace{1ex},
    breakable,
    pad at break*=0mm,
    vfill before first,
    overlay unbroken={%
        \draw[Blue, line width=2pt]
            ([yshift=-1.2ex]title.south-|frame.west) to
            ([yshift=-1.2ex]title.south-|frame.east);
        },
    overlay first={%
        \draw[Blue, line width=2pt]
            ([yshift=-1.2ex]title.south-|frame.west) to
            ([yshift=-1.2ex]title.south-|frame.east);
    },
}{def}

\AtEndEnvironment{ltheorem}{$\hfill\textcolor{Green}{\blacksquare}$}
\newtcbtheorem[use counter*=theorem]{ltheorem}{Theorem}{%
    empty,
    title={Theorem~\thetheorem:~{#1}},
    boxed title style={%
        empty,
        size=minimal,
        toprule=2pt,
        top=0.5\topsep,
    },
    coltitle=Green,
    fonttitle=\bfseries,
    parbox=false,
    boxsep=0pt,
    before=\par\vspace{2ex},
    left=0pt,
    right=0pt,
    top=3ex,
    bottom=-1.5ex,
    breakable,
    pad at break*=0mm,
    vfill before first,
    overlay unbroken={%
        \draw[Green, line width=2pt]
            ([yshift=-1.2ex]title.south-|frame.west) to
            ([yshift=-1.2ex]title.south-|frame.east);},
    overlay first={%
        \draw[Green, line width=2pt]
            ([yshift=-1.2ex]title.south-|frame.west) to
            ([yshift=-1.2ex]title.south-|frame.east);
    }
}{thm}

%--------------------Declared Math Operators--------------------%
\DeclareMathOperator{\adjoint}{adj}         % Adjoint.
\DeclareMathOperator{\Card}{Card}           % Cardinality.
\DeclareMathOperator{\curl}{curl}           % Curl.
\DeclareMathOperator{\diam}{diam}           % Diameter.
\DeclareMathOperator{\dist}{dist}           % Distance.
\DeclareMathOperator{\Div}{div}             % Divergence.
\DeclareMathOperator{\Erf}{Erf}             % Error Function.
\DeclareMathOperator{\Erfc}{Erfc}           % Complementary Error Function.
\DeclareMathOperator{\Ext}{Ext}             % Exterior.
\DeclareMathOperator{\GCD}{GCD}             % Greatest common denominator.
\DeclareMathOperator{\grad}{grad}           % Gradient
\DeclareMathOperator{\Ima}{Im}              % Image.
\DeclareMathOperator{\Int}{Int}             % Interior.
\DeclareMathOperator{\LC}{LC}               % Leading coefficient.
\DeclareMathOperator{\LCM}{LCM}             % Least common multiple.
\DeclareMathOperator{\LM}{LM}               % Leading monomial.
\DeclareMathOperator{\LT}{LT}               % Leading term.
\DeclareMathOperator{\Mod}{mod}             % Modulus.
\DeclareMathOperator{\Mon}{Mon}             % Monomial.
\DeclareMathOperator{\multideg}{mutlideg}   % Multi-Degree (Graphs).
\DeclareMathOperator{\nul}{nul}             % Null space of operator.
\DeclareMathOperator{\Ord}{Ord}             % Ordinal of ordered set.
\DeclareMathOperator{\Prin}{Prin}           % Principal value.
\DeclareMathOperator{\proj}{proj}           % Projection.
\DeclareMathOperator{\Refl}{Refl}           % Reflection operator.
\DeclareMathOperator{\rk}{rk}               % Rank of operator.
\DeclareMathOperator{\sgn}{sgn}             % Sign of a number.
\DeclareMathOperator{\sinc}{sinc}           % Sinc function.
\DeclareMathOperator{\Span}{Span}           % Span of a set.
\DeclareMathOperator{\Spec}{Spec}           % Spectrum.
\DeclareMathOperator{\supp}{supp}           % Support
\DeclareMathOperator{\Tr}{Tr}               % Trace of matrix.
%--------------------Declared Math Symbols--------------------%
\DeclareMathSymbol{\minus}{\mathbin}{AMSa}{"39} % Unary minus sign.
%------------------------New Commands---------------------------%
\DeclarePairedDelimiter\norm{\lVert}{\rVert}
\DeclarePairedDelimiter\ceil{\lceil}{\rceil}
\DeclarePairedDelimiter\floor{\lfloor}{\rfloor}
\newcommand*\diff{\mathop{}\!\mathrm{d}}
\newcommand*\Diff[1]{\mathop{}\!\mathrm{d^#1}}
\renewcommand*{\glstextformat}[1]{\textcolor{RoyalBlue}{#1}}
\renewcommand{\glsnamefont}[1]{\textbf{#1}}
\renewcommand\labelitemii{$\circ$}
\renewcommand\thesubfigure{%
    \arabic{chapter}.\arabic{figure}.\arabic{subfigure}}
\addto\captionsenglish{\renewcommand{\figurename}{Fig.}}
\numberwithin{equation}{section}

\renewcommand{\vector}[1]{\boldsymbol{\mathrm{#1}}}

\newcommand{\uvector}[1]{\boldsymbol{\hat{\mathrm{#1}}}}
\newcommand{\topspace}[2][]{(#2,\tau_{#1})}
\newcommand{\measurespace}[2][]{(#2,\varSigma_{#1},\mu_{#1})}
\newcommand{\measurablespace}[2][]{(#2,\varSigma_{#1})}
\newcommand{\manifold}[2][]{(#2,\tau_{#1},\mathcal{A}_{#1})}
\newcommand{\tanspace}[2]{T_{#1}{#2}}
\newcommand{\cotanspace}[2]{T_{#1}^{*}{#2}}
\newcommand{\Ckspace}[3][\mathbb{R}]{C^{#2}(#3,#1)}
\newcommand{\funcspace}[2][\mathbb{R}]{\mathcal{F}(#2,#1)}
\newcommand{\smoothvecf}[1]{\mathfrak{X}(#1)}
\newcommand{\smoothonef}[1]{\mathfrak{X}^{*}(#1)}
\newcommand{\bracket}[2]{[#1,#2]}

%------------------------Book Command---------------------------%
\makeatletter
\renewcommand\@pnumwidth{1cm}
\newcounter{book}
\renewcommand\thebook{\@Roman\c@book}
\newcommand\book{%
    \if@openright
        \cleardoublepage
    \else
        \clearpage
    \fi
    \thispagestyle{plain}%
    \if@twocolumn
        \onecolumn
        \@tempswatrue
    \else
        \@tempswafalse
    \fi
    \null\vfil
    \secdef\@book\@sbook
}
\def\@book[#1]#2{%
    \refstepcounter{book}
    \addcontentsline{toc}{book}{\bookname\ \thebook:\hspace{1em}#1}
    \markboth{}{}
    {\centering
     \interlinepenalty\@M
     \normalfont
     \huge\bfseries\bookname\nobreakspace\thebook
     \par
     \vskip 20\p@
     \Huge\bfseries#2\par}%
    \@endbook}
\def\@sbook#1{%
    {\centering
     \interlinepenalty \@M
     \normalfont
     \Huge\bfseries#1\par}%
    \@endbook}
\def\@endbook{
    \vfil\newpage
        \if@twoside
            \if@openright
                \null
                \thispagestyle{empty}%
                \newpage
            \fi
        \fi
        \if@tempswa
            \twocolumn
        \fi
}
\newcommand*\l@book[2]{%
    \ifnum\c@tocdepth >-3\relax
        \addpenalty{-\@highpenalty}%
        \addvspace{2.25em\@plus\p@}%
        \setlength\@tempdima{3em}%
        \begingroup
            \parindent\z@\rightskip\@pnumwidth
            \parfillskip -\@pnumwidth
            {
                \leavevmode
                \Large\bfseries#1\hfill\hb@xt@\@pnumwidth{\hss#2}
            }
            \par
            \nobreak
            \global\@nobreaktrue
            \everypar{\global\@nobreakfalse\everypar{}}%
        \endgroup
    \fi}
\newcommand\bookname{Book}
\renewcommand{\thebook}{\texorpdfstring{\Numberstring{book}}{book}}
\providecommand*{\toclevel@book}{-2}
\makeatother
\titleformat{\part}[display]
    {\Large\bfseries}
    {\partname\nobreakspace\thepart}
    {0mm}
    {\Huge\bfseries}
\titlecontents{part}[0pt]
    {\large\bfseries}
    {\partname\ \thecontentslabel: \quad}
    {}
    {\hfill\contentspage}
\titlecontents{chapter}[0pt]
    {\bfseries}
    {\chaptername\ \thecontentslabel:\quad}
    {}
    {\hfill\contentspage}
\newglossarystyle{longpara}{%
    \setglossarystyle{long}%
    \renewenvironment{theglossary}{%
        \begin{longtable}[l]{{p{0.25\hsize}p{0.65\hsize}}}
    }{\end{longtable}}%
    \renewcommand{\glossentry}[2]{%
        \glstarget{##1}{\glossentryname{##1}}%
        &\glossentrydesc{##1}{~##2.}
        \tabularnewline%
        \tabularnewline
    }%
}
\newglossary[not-glg]{notation}{not-gls}{not-glo}{Notation}
\newcommand*{\newnotation}[4][]{%
    \newglossaryentry{#2}{type=notation, name={\textbf{#3}, },
                          text={#4}, description={#4},#1}%
}
%--------------------------LENGTHS------------------------------%
% Spacings for the Table of Contents.
\addtolength{\cftsecnumwidth}{1ex}
\addtolength{\cftsubsecindent}{1ex}
\addtolength{\cftsubsecnumwidth}{1ex}
\addtolength{\cftfignumwidth}{1ex}
\addtolength{\cfttabnumwidth}{1ex}

% Indent and paragraph spacing.
\setlength{\parindent}{0em}
\setlength{\parskip}{0em}                                                           %
%----------------------------Main Document-------------------------------------%
\begin{document}
    \title{Differential Topology}
    \author{Ryan Maguire}
    \date{\vspace{-5ex}}
    \maketitle
    \tableofcontents
    \section{Preliminary Stuff}
        \begin{fdefinition}{Sequentially Continuous}{Sequentially_Continuous}
            A sequentially continuous function from a topological space
            $(X,\tau_{X})$ to a topological space $(Y,\tau_{Y})$ is a
            function $f:X\rightarrow{Y}$ such that for every convergent sequence
            $a:\mathbb{N}\rightarrow{X}$ and for every $x\in{X}$ such that
            $a_{n}\rightarrow{x}$, it is true that $f(a_{n})\rightarrow{f}(x)$.
        \end{fdefinition}
        We say \textit{for every x} since limits need not be unique in spaces
        that aren't at least $T_{1}$ (an \textit{accessible} space).
        \begin{theorem}
            \label{thm:cont_implies_seq_cont}%
            If $\topspace[X]{X}$ and $\topspace[Y]{Y}$ are topological spaces,
            and if $f:X\rightarrow{Y}$ is a continuous function, then it is
            sequentially continuous.
        \end{theorem}
        \begin{proof}
            For suppose not. Then there is a sequence
            $a:\mathbb{N}\rightarrow{X}$ and a point $x\in{X}$ such that
            $a_{n}\rightarrow{x}$, but $f(a_{n})\not\rightarrow{f}(x)$
            (Def.~\ref{def:Sequentially_Continuous}). But then there exists
            an open subset $\mathcal{V}\in\tau_{Y}$ such that
            $f(x)\in\mathcal{V}$, and for all $N\in\mathbb{N}$ there exists an
            $n\in\mathbb{N}$ such that $n>N$ and
            $f(a_{n})\notin\mathcal{V}$. But $f$ is continuous, and therefore
            $f^{\minus{1}}[\mathcal{V}]$ is an open subset of $X$. And since
            $f(x)\in\mathcal{V}$ it is true that
            $x\in{f}^{\minus{1}}[\mathcal{V}]$. But $a_{n}\rightarrow{x}$, and
            thus there is an $N\in\mathbb{N}$ such that for all
            $n\in\mathbb{N}$ with $n>N$ it is true that
            $a_{n}\in{f}^{\minus{1}}[\mathcal{V}]$. But then for all $n>N$ it is
            true that $f(a_{n})\in\mathcal{V}$, a contradiction. Therefore, $f$
            is sequentially continuous.
        \end{proof}
        This theorem does not reverse.
        \begin{example}
            Let $X$ be the order topology on the ordinal $\omega_{1}+1$, where
            $\omega_{1}$ is the first uncountable ordinal. Define
            $f:X\rightarrow\nspace[]$ by:
            \begin{equation}
                f(x)=
                \begin{cases}
                    0,&x\ne\omega_{1}\\
                    1,&x=\omega_{1}
                \end{cases}
            \end{equation}
            then $f$ is sequentially continuous, but not continuous. If
            $a:\mathbb{N}\rightarrow{X}$ is any sequence, either it coverges to
            $\omega_{1}$ or it is not. In the latter case the limit of
            $f(a_{n})$ will be zero, and in the former case there is an
            $N\in\mathbb{N}$ such that for all $n>N$ it is true that
            $a_{n}=\omega_{1}$, and hence $f(a_{n})\rightarrow{1}$. In both
            scenarios, $f$ is sequentially continuous. It is not continuous
            since $\{\omega_{1}\}$ is not an isolated point of $X$, yet
            $f^{\minus{1}}[(0,\infty)]=\{\omega_{1}\}$.
        \end{example}
        The problem with this space is that it is not first countable. First
        countability implies that sequentially continuous and continuous are
        identical concepts. There's a weaker such notion, called sequential
        spaces.
        \begin{fdefinition}{Sequentially Open}{Sequentially_Open}
            A sequentially open subset of a topological space $\topspace{X}$
            is a subset $\mathcal{U}\subseteq{X}$ such that for every sequence
            $a:\mathbb{N}\rightarrow{X}$ such that $a$ converges to an element
            $x\in\mathcal{U}$ there exists an $N\in\mathbb{N}$ such that for all
            $n\in\mathbb{N}$ with $n>N$, it is true that $a_{n}\in\mathcal{U}$.
        \end{fdefinition}
        \begin{theorem}
            \label{thm:Open_is_Seq_Open}%
            If $\topspace{X}$ is a topological space, if $\mathcal{U}\in\tau$ is
            an open subset of $X$, then $\mathcal{U}$ is sequentially open.
        \end{theorem}
        \begin{proof}
            For suppose not. Then there is a sequence
            $a:\mathbb{N}\rightarrow{X}$ such that $a$ converges to a point
            $x\in\mathcal{U}$, but for all $N\in\mathbb{N}$ there is an
            $n\in\mathbb{N}$ such that $n>N$ and $a_{n}\notin\mathcal{U}$
            (Def.~\ref{def:Sequentially_Open}). But if $\mathcal{U}$ is an open
            subset containing $x$, and if $a_{n}\rightarrow{x}$, then there is
            an $N\in\mathbb{N}$ such that for all $n>N$ it is true that
            $a_{n}\in\mathcal{U}$, a contradiction.
        \end{proof}
        \begin{fdefinition}{Sequential Space}{Sequential_Space}
            A sequential topological space is a topological space
            $\topspace{X}$ such that for every sequentially open subset
            $\mathcal{U}\subseteq{X}$ it is true that $\mathcal{U}\in\tau$.
        \end{fdefinition}
        As the name suggests, the usefulness of sequential spaces stems from the
        fact that continuity and sequential continuity are equivalent. One
        direction is true of any topological space
        (Thm.~\ref{thm:cont_implies_seq_cont}). We now prove the other direction
        in the setting of a sequential space.
        \begin{theorem}
            \label{thm:seq_space_seq_cont_eqiv_cont}%
            If $\topspace[X]{X}$ is a sequential topological space, if
            $\topspace[Y]{Y}$ is a topological space, and if $f:X\rightarrow{Y}$
            is a sequentially continuous function, then $f$ is continuous.
        \end{theorem}
        \begin{proof}
            For suppose not. Then there is a open subset
            $\mathcal{V}\in\tau_{Y}$ such that $f^{\minus{1}}[\mathcal{V}]$ is
            not open. But $\topspace[X]{X}$ is sequential, and thus if
            $f^{\minus{1}}[\mathcal{V}]$ is not open, then it is not
            sequentially open (Def.~\ref{def:Sequential_Space}). But then there
            is a sequence $a:\mathbb{N}\rightarrow{X}$ and a point
            $x\in{f}^{\minus{1}}[\mathcal{V}]$ such that $a_{n}\rightarrow{x}$
            but for all $N\in\mathbb{N}$ there is an $n\in\mathbb{N}$ with
            $n>N$ such that $a_{n}\notin{f}^{\minus{1}}[\mathcal{V}]$
            (Def.~\ref{def:Sequentially_Open}). But $f$ is sequentially
            continuous, and thus if $a_{n}\rightarrow{x}$, then
            $f(a_{n})\rightarrow{f}(x)$
            (Def.~\ref{def:Sequentially_Continuous}). But $\mathcal{V}$ is open
            and $f(x)\in\mathcal{V}$. Therefore if $f(a_{n})\rightarrow{f}(x)$,
            then there is an $N\in\mathbb{N}$ such that for all $n\in\mathbb{N}$
            with $n>N$, it is true that $f(a_{n})\in\mathcal{V}$. But then for
            all $n>N$ it is true that $a_{n}\in{f}^{\minus{1}}[\mathcal{V}]$, a
            contradiction. Therefore, $f$ is continuous.
        \end{proof}
        Two amazing equivalent definitions of sequential spaces that we will not
        prove since these will not be used.
        \begin{itemize}
            \item $\topspace{X}$ is sequential if and only if it is the quotient
                  of a first countable space.
            \item $\topspace{X}$ is sequential if and only if it is the quotient
                  of a metrizable space.
        \end{itemize}
        In just about every setting we will be working with first countable
        spaces. It would be nice to know that first countable spaces are
        sequential, and indeed they are.
        \begin{theorem}
            \label{thm:First_Countable_Implies_Sequential}%
            If $\topspace{X}$ is a first countable topological space, then it
            is a sequential topological space.
        \end{theorem}
        \begin{proof}
            For suppose not. Then there is a sequentially open subset
            $\mathcal{U}\subseteq{X}$ such that $\mathcal{U}$ is not open
            (Def.~\ref{def:Sequential_Space}). But since $\topspace{X}$ is first
            countable, for all $x\in\mathcal{U}$ there is a countable
            neighborhood basis. By the axiom of choice there is a
            function $\mathcal{B}:\mathcal{U}\rightarrow\powset{\tau}$ such that
            for all $x\in\mathcal{U}$, $\mathcal{B}_{x}$ is a countable
            neighborhood basis of $x$. But if for all $x\in\mathcal{U}$ it is
            true that $\mathcal{B}_{x}$ is countable, then there is surjective
            function $\mathcal{V}_{x}:\mathbb{N}\rightarrow\mathcal{B}_{x}$.
            Then there exists an $N\in\mathbb{N}$ such that
            $\mathcal{V}_{x,N}\subseteq\mathcal{U}$. For suppose not and let
            $B:\mathbb{N}\rightarrow{X}$ be defined by:
            \begin{equation}
                B_{n}=\bigcap_{k\in\mathbb{Z}_{n}}\mathcal{V}_{x,k}
            \end{equation}
            Since $B_{n}$ is the intersection of finitely many open sets, it is
            open. Moreover it is non-empty since $x\in{B}_{n}$ for all $n$.
            But then $B_{n}$ is an open neighborhood about $x$, and since
            $\mathcal{B}_{x}$ is a neighborhood basis there is an element
            $V\in\mathcal{B}_{x}$ such that $V\subseteq{B}_{n}$. But
            $\mathcal{V}:\mathbb{N}\rightarrow\mathcal{B}_{x}$ is a surjection,
            and hence there is an $N\in\mathbb{N}$ such that
            $\mathcal{V}_{x,N}=V$. But by hypothesis,
            $\mathcal{V}_{x,N}\nsubseteq\mathcal{U}$ and hence there is an
            element $y\in\mathcal{V}_{x,N}$ such that $y\notin\mathcal{U}$.
            Therefore we have shown that the sequence $A_{n}$ defined by:
            \begin{equation}
                A_{n}=\{\,y\in{B}_{n}\;|\;y\not\in\mathcal{U}\,\}
            \end{equation}
            is non-empty for all $n\in\mathbb{N}$, and hence by the axiom of
            (countable) choice there is a sequence $y:\mathbb{N}\rightarrow{X}$
            such that $y_{n}\in{A}_{n}$ for all $n\in\mathbb{N}$. That is,
            for all $n\in\mathbb{N}$, $y_{n}\in{B}_{n}$ and
            $y_{n}\notin\mathcal{U}$. But if $y_{n}\in{B}_{n}$ for all
            $n\in\mathbb{N}$, then $y_{n}\rightarrow{x}$. For if not, then there
            is an open set $U\in\tau$ such that $x\in{U}$ and for all
            $N\in\mathbb{N}$ there exists an $n>N$ such that $y_{n}\notin{U}$.
            But $\mathcal{B}_{x}$ is a neighborhood basis of $x$, and hence
            there is a $V\in\mathcal{B}_{x}$ such that $V\subseteq{U}$. But
            $\mathcal{V}_{x}:\mathbb{N}\rightarrow\mathcal{B}_{x}$ is a
            surjection, and hence there is an $N\in\mathbb{N}$ such that
            $\mathcal{V}_{x,N}=V$. But then for all $n>N$,
            $B_{n}\subseteq\mathcal{V}_{x,N}$ and thus $B_{n}\subseteq{U}$.
            But then for all $n>N$, $y_{n}\in{U}$, a contradiction. Hence,
            $y_{n}\rightarrow{x}$. But for all $n\in\mathbb{N}$,
            $y_{n}\notin\mathcal{U}$. Thus $y_{n}$ is a sequence such that
            $y_{n}\rightarrow{x}$, but for all $N\in\mathbb{N}$ there is an
            $n\in\mathbb{N}$ with $n>N$ such that $y_{n}\notin\mathcal{U}$, a
            contradiction since $x\in\mathcal{U}$ and $\mathcal{U}$ is
            sequentially open (Def.~\ref{def:Sequentially_Open}). Thus, for all
            $x\in\mathcal{U}$ there is an open subset $\mathcal{V}_{x,N}$ such
            that $x\in\mathcal{V}_{x,N}$ and
            $\mathcal{V}_{x,N}\subseteq\mathcal{U}$. But then $\mathcal{U}$ is
            simply the union over all of these open sets, and is thus open,
            a contradiction. Hence, $\topspace{X}$ is sequential.
        \end{proof}
        And now, our useful corollary.
        \begin{theorem}
            \label{thm:First_Countable_Implies_Seq_Cont_is_Cont}%
            If $\topspace[X]{X}$ is a first countable topological space, if
            $\topspace[Y]{Y}$ is a topological space, and if $f:X\rightarrow{Y}$
            is sequential continuous, then $f$ is continuous.
        \end{theorem}
        \begin{proof}
            For since $\topspace[X]{X}$ is first countable, it is sequential
            (Thm.~\ref{thm:First_Countable_Implies_Sequential}). But if
            $\topspace[X]{X}$ is a sequential topological space and if $f$ is
            sequentially continuous, then $f$ is continuous
            (Thm.~\ref{thm:seq_space_seq_cont_eqiv_cont}). Therefore, $f$ is
            continuous.
        \end{proof}
        It's laborious to show a certain space is first countable, especially in
        the case of manifold theory, and so we rely on the following theorems.
        \begin{theorem}
            \label{thm:Locally_Metrizable_is_First_Countable}%
            If $\topspace{X}$ is a locally metrizable topological space, then it
            is first countable.
        \end{theorem}
        \begin{proof}
            For suppose not. Then there is a point $x\in{X}$ with no countable
            neighborhood basis. But if $\topspace{X}$ is locally metrizable,
            then there is an open subset $\mathcal{U}$ such that
            $x\in\mathcal{U}$ and $\topspace[\mathcal{U}]{\mathcal{U}}$ is
            metrizable, where $\tau_{\mathcal{U}}$ is the subspace topology. But
            then there is a metric
            $d:\mathcal{U}\times\mathcal{U}\rightarrow\nspace[]$ such that
            $d$ induces $\tau_{\mathcal{U}}$. Let $\mathcal{B}\subseteq\tau$ be
            defined by:
            \begin{equation}
                \mathcal{B}=\{\,\rball{n^{\minus{1}}}{\metspace{\mathcal{U}}}{x}
                    \;|\;n\in\mathbb{N}^{+}\}
            \end{equation}
            That is, the set of all open balls about $x$ of radius $1/n$.
            Suppose $\mathcal{V}\in\tau$ is such that $x\in\mathcal{V}$. But
            then $x\in\mathcal{V}\cap\mathcal{U}$. But
            $\mathcal{V}\cap\mathcal{U}$ is an open subset of $\mathcal{U}$, and
            thus there is an $r>0$ such that:
            \begin{equation}
                \rball{r}{\metspace{\mathcal{U}}}{x}\subseteq\mathcal{V}
            \end{equation}
            but by Archimede's Theorem, there is an $N\in\mathbb{N}$ such that
            $N>r$. But it is true that
            $\rball{N^{\minus{1}}}{\metspace{\mathcal{U}}}{x}\in\mathcal{B}$,
            and hence $\mathcal{B}$ is a countable neighborhood basis of $x$,
            a contradiction. Thus, $\topspace{X}$ is first countable.
        \end{proof}
        \begin{fdefinition}{$\sigma$ Compact}{Sigma_Compact}
            A $\sigma$ compact topological space is a topological space
            $\topspace{X}$ such that there exists a sequence
            $K:\mathbb{N}\rightarrow\powset{X}$ of compact subsets of $X$ such
            that:
            \begin{equation*}
                X=\bigcup_{n\in\mathbb{N}}K_{n}
            \end{equation*}
        \end{fdefinition}
        This definition gives the following triviality.
        \begin{theorem}
            \label{thm:Compact_Implies_Sigma_Compact}%
            If $\topspace{X}$ is a compact topological space, then it is
            $\sigma$ compact.
        \end{theorem}
        \begin{proof}
            For let $K:\mathbb{N}\rightarrow\powset{X}$ be defined by
            $K_{n}=X$. Then $K_{n}$ is compact for all $n$ and
            $\bigcup{K}_{n}=X$. Thus, $\topspace{X}$ is $\sigma$ compact
            (Def.~\ref{def:Sigma_Compact}).
        \end{proof}
        \begin{fdefinition}{Lindel\"{o}f Topological Space}{Lindelof_Space}
            A Lindel\"{o}f topological space is a topological space $(X,\tau)$
            such that for every open cover $\mathcal{O}$ of $X$ there exists a
            countable subcover.
        \end{fdefinition}
        \begin{theorem}
            \label{thm:Second_Countable_Implies_Lindelof}%
            If $\topspace{X}$ is a second countable topological space, then it
            is Lindel\"{o}f.
        \end{theorem}
        \begin{proof}
            For suppose not. Then there exists an open cover $\mathcal{O}$ with
            no countable subcover. But if $\topspace{X}$ is second countable,
            then there is a countable basis $\mathcal{B}$ of $\tau$. But if
            $\mathcal{B}$ is a basis, since $\mathcal{O}$ is an open cover, for
            all $\mathcal{V}\in\mathcal{O}$ it is true that $\mathcal{V}$ is
            open and thus there is an element $\mathcal{U}\in\mathcal{B}$ such
            that $\mathcal{U}\subseteq\mathcal{V}$. That is, the function
            $F:\mathcal{O}\rightarrow\powset{\mathcal{B}}$ defined by:
            \begin{equation}
                F(\mathcal{V})=\{\,\mathcal{U}\in\mathcal{B}\;|\;
                    \mathcal{U}\subseteq\mathcal{V}\,\}
            \end{equation}
            is such that $F(\mathcal{V})\ne\emptyset$ for all
            $\mathcal{V}\in\mathcal{O}$. But then
            $F[\mathcal{O}]\subseteq\powset{\mathcal{B}}$ is a non-empty subset
            of $\powset{B}$ such that $\emptyset\notin{F}[\mathcal{O}]$. Hence
            by the axiom of choice there is a function
            $G:F[\mathcal{O}]\rightarrow\mathcal{B}$ such that for all
            $\Delta\in{F}[\mathcal{O}]$, $G(\Delta)\in\Delta$. Let
            $B=G\big[F[\mathcal{O}]\big]$. But $\mathcal{B}$ is countable and
            $B\subseteq\mathcal{B}$, and thus $B$ is countable. Moreover,
            $F\circ{G}:\mathcal{O}\rightarrow{B}$ is surjective by the
            definition of $B$ and hence there exists a right inverse
            $H:B\rightarrow\mathcal{O}$. And since $B$ is countable,
            $H[B]$ is a countable subset of $\mathcal{O}$. But $\mathcal{O}$ has
            no countable subcover, and hence there is an $x\in{X}$ such that
            $x\notin\bigcup{H}[B]$. But $\mathcal{O}$ is a cover of $X$, and
            thus there is a $\mathcal{V}\in\mathcal{O}$ such that
            $x\in\mathcal{V}$. And since $\mathcal{B}$ is a basis, there is an
            element $\mathcal{U}_{1}\in\mathcal{B}$ such that
            $x\in\mathcal{U}_{1}$. But then $x\in\mathcal{U}_{1}\cap\mathcal{V}$
            and since $\mathcal{B}$ is a basis, there is an element
            $\mathcal{U}_{2}\in\mathcal{B}$ such that $x\in\mathcal{U}_{2}$ and
            $\mathcal{U}_{2}\subseteq\mathcal{U}_{1}\cap\mathcal{V}$. But then
            $H(\mathcal{U}_{2})$ is an element of $H[B]$ such that
            $\mathcal{U}_{2}\subseteq{H}(\mathcal{U}_{2})$, and hence
            $x\in{H}[\mathcal{U}_{2}]$. A contradiction, since
            $x\notin\bigcup{H}[B]$. Therefore, $\topspace{X}$ is Lindel\"{o}f.
        \end{proof}
        \begin{theorem}
            \label{thm:Sigma_Compact_Implies_Lindelof}%
            If $\topspace{X}$ is a $\sigma$ compact topological space, then it
            is Lindel\"{o}f.
        \end{theorem}
        \begin{proof}
            For suppose not. Then there exists an open cover $\mathcal{O}$ of
            $X$ with no countable subcover. But $X$ is $\sigma$ compact, and
            hence there is a sequence $K:\mathbb{N}\rightarrow\powset{X}$ of
            compact sets such that $X=\bigcup{K}_{n}$
            (Def.~\ref{def:Sigma_Compact}). But then for all $n\in\mathbb{N}$,
            $K_{n}$ is a subset of $X$, and hence $\mathcal{O}$ is an open cover
            of $K_{n}$. But by hypothesis, $K_{n}$ is compact and hence there is
            a finite subcover $\mathscr{D}\subseteq\mathcal{O}$ of $K_{n}$. That
            is, if we define the sequence
            $A_{n}:\mathbb{N}\rightarrow\powset{\mathcal{O}}$ by:
            \begin{equation}
                A_{n}=\{\mathscr{D}\in\powset{\mathcal{O}}\;|\;
                    \mathscr{D}\textrm{ is finite and }
                    K_{n}\subseteq\bigcup\mathscr{D}\,\}
            \end{equation}
            then for all $n\in\mathbb{N}$, $A_{n}$ is non-empty. Hence by the
            axiom of (countable) choice, there is a choice function
            $\Delta:\mathbb{N}\rightarrow\powset{\mathcal{O}}$ such that for all
            $n\in\mathbb{N}$, $\Delta_{n}$ is a finite open subcover of $K_{n}$.
            But then the union of all $\Delta_{n}$ is the countable union of
            finite collections, and is hence countable. But then this collection
            covers $K_{n}$ for all $n\in\mathbb{N}$, and hence covers
            $\bigcup{K}_{n}$. But $\bigcup{K}_{n}=X$, a contradiction since
            $\mathcal{O}$ has no countable subcover. Thus, $\topspace{X}$ is
            Lindel\"{o}f.
        \end{proof}
        This theorem reverses if we add locally compact. The requirement of
        local compactness can be seen by examining the irrational numbers with
        the subspace topology. This is Lindel\"{o}f (since it is second
        countable), but it is not $\sigma$ compact. For any compact subset of
        the irrationals must have empty interior, and by the Baire category
        theorem the irrationals cannot be written as the countable union of
        nowhere dense subsets. Hence they cannot possibly $\sigma$ compact.
        \begin{fdefinition}{Locally Compact Topological Space}{Locally_Compact}
            A locally compact topological space is a topological space
            $\topspace{X}$ such that for all $x\in{X}$ there exists a compact
            subset $K\subseteq{X}$ and an open set $\mathcal{U}\in\tau$ such
            that $x\in\mathcal{U}$ and $\mathcal{U}\subseteq{K}$.
        \end{fdefinition}
        \begin{theorem}
            \label{thm:Loc_Comp_and_Lindelof_Implies_Sigma_Comp}%
            If $\topspace{X}$ is a locally compact Lindel\"{o}f space, then it
            is $\sigma$ compact.
        \end{theorem}
        \begin{proof}
            For if $\topspace{X}$ is locally compact, then for all $x\in{X}$
            there is a compact set $K\subseteq{X}$ and an open set
            $\mathcal{U}\in\tau$ such that $x\in\mathcal{U}$ and
            $\mathcal{U}\subseteq{K}$ (Def.~\ref{def:Locally_Compact}). Invoking
            the axiom of choice, there is a function
            $A:X\rightarrow\tau\times\powset{X}$ such that for all $x\in{X}$,
            $A_{x}=(\mathcal{U},K)$ where $x\in\mathcal{U}$ and
            $\mathcal{U}\subseteq{K}$, where $K$ is compact. But then the
            collection of all such $\mathcal{U}$ is an open cover of $X$. But
            $X$ is Lindel\"{o}f and hence there is a countable subcover. But
            then the subcollection of all $K$ form a countable collection of
            compact sets that cover $X$. Hence, $\topspace{X}$ is $\sigma$
            compact.
        \end{proof}
        \begin{fdefinition}{Sequentially Compact}{Sequentially_Compact}
            A sequentially compact topological space is a topological space
            $\topspace{X}$ such that for every sequence
            $a:\mathbb{N}\rightarrow{X}$ there exists a strictly increasing
            sequence $k:\mathbb{N}\rightarrow\mathbb{N}$ such that
            $a\circ{k}:\mathbb{N}\rightarrow{X}$ converges.
        \end{fdefinition}
        \begin{theorem}
            \label{thm:Met_Space_Seq_Compact_iff_Compact}%
            If $\topspace{X}$ is metrizable, and if $\mathcal{C}\subseteq{X}$,
            then $\mathcal{C}$ is compact if and only if it is sequentially
            compact.
        \end{theorem}
        \begin{proof}
            For if $\topspace[X]{X}$ is metrizable, then there is a metric
            $d:X\times{X}\rightarrow\nspace[]$ that induces the topology $\tau$.
            Suppose $X$ is compact and not sequentially compact. Then there is a
            sequence $a:\mathbb{N}\rightarrow{X}$ with no convergent
            subsequence. Then for al $x\in{X}$ there is an $r>0$ such that the
            open ball of radius $r$ about $x$ is such that only finitely many
            $n\in\mathbb{N}$ imply $a_{n}$ lies inside the ball. Invoking
            choice, we get a function $r:X\rightarrow\mathbb{R}^{+}$. Let
            $\mathcal{O}$ be defined by:
            \begin{equation}
                \mathcal{O}=\{\,\rball{r_{x}}{\metspace{X}}{x}\;|\;x\in{X}\,\}
            \end{equation}
            Then $\mathcal{O}$ is an open cover of $X$. But $X$ is compact, and
            therefore there is a finite subcover $\Delta$. But for each
            $\mathcal{U}\in\Delta$, there are only finite many
            $n\in\mathbb{N}$ such that $a_{n}\in\mathcal{U}$. But every
            $n\in\mathbb{N}$ is such that $a_{n}\in\mathcal{U}$ for at least
            one $\mathcal{U}$ since these cover $X$, a contradiction since
            $\mathbb{N}$ is not finite. Thus, $\topspace{X}$ is sequentially
            compact. In the other direction, sequential compactness implies
            complete and totally bounded, which implies compact by the
            generalized Heine-Borel theorem.
        \end{proof}
        This is not true in general. The long ling is sequentially compact but
        not compact, whereas the space of all functions
        $f:[0,1]\rightarrow[0,1]$ is compact (by Tychonoff) but not sequentially
        compact.
        \begin{fdefinition}{$\sigma$ locally finite basis}
                           {Sigma_Loc_Fin_Basis}
            A $\sigma$ locally finite basis of a topological space
            $\topspace{X}$ is a basis $\mathcal{B}$ of $X$ such that
            $\mathcal{B}$ is $\sigma$ locally finite. That is, there exists
            a sequence $B:\mathbb{N}\rightarrow\powset{X}$ such that
            $\mathcal{B}=\bigcup\{B_{n}\}$, and for all $n\in\mathbb{N}$ it is
            true that $B_{n}$ is locally finite.
        \end{fdefinition}
        \begin{theorem}
            \label{thm:Second_Countable_Implies_Sigma_Loc_Fin_Basis}%
            If $\topspace{X}$ is second countable, then it is has a $\sigma$
            locally finite basis.
        \end{theorem}
        \begin{proof}
            For if $\topspace{X}$ is has a countable basis $\mathcal{B}$. But
            then there is a surjection $B:\mathbb{N}\rightarrow\mathcal{B}$.
            But then for all $n\in\mathbb{N}$, $\{B_{n}\}$ is a finite subset
            of $\powset{X}$, and is hence locally finite, and
            $\mathcal{B}=\bigcup\{B_{n}\}$. Thus, $X$ has a $\sigma$ locally
            finite basis.
        \end{proof}
        \begin{theorem}
            \label{thm:Compact_Hausdorff_is_Regular}%
            If $\topspace{X}$ is a compact Hausdorff topological space, then it
            is regular.
        \end{theorem}
        \begin{proof}
            For suppose $C\subseteq{X}$ is closed, and $x\in{X}\setminus{C}$.
            But closed subsets of compact spaces are compact, and thus $C$ is
            compact. And since $X$ is Hausdorff, for all $y\in{C}$ there are
            open subsets $\mathcal{U}_{y},\mathcal{V}_{y}$ such that
            $x\in\mathcal{U}_{y}$, $y\in\mathcal{V}_{y}$, and
            $\mathcal{U}_{y}\cap\mathcal{V}_{y}=\emptyset$. But the
            $\mathcal{V}_{y}$ cover $C$, and hence there is a finite subcover
            $\mathcal{V}_{n}$. Let $\mathcal{U}=\bigcap\mathcal{U}_{n}$ and
            $\mathcal{V}=\bigcup\mathcal{V}_{n}$. Then since
            $\mathcal{U}$ is the finite intersection of open subsets, it is
            open. And since $\mathcal{V}$ is the union of open subsets, it is
            open. But $x\in\mathcal{U}$, $C\subseteq\mathcal{V}$, and
            $\mathcal{U}\cap\mathcal{V}=\emptyset$. Hence, $\topspace{X}$ is
            regular.
        \end{proof}
        \begin{theorem}
            \label{thm:Loc_Comp_and_Hausdorff_is_Regular}%
            If $\topspace{X}$ is locally compact and Hausdorff, then it is
            regular.
        \end{theorem}
        \begin{proof}
            For let $C\subseteq{X}$ be closed and $x\in{X}\setminus{C}$. Since
            $X$ is locally compact there is an open subset
            $\mathcal{U}\in\tau$ and a compact subset $K\subseteq{X}$ such that
            $x\in\mathcal{U}$ and $\mathcal{U}\subseteq{K}$
            (Def.~\ref{def:Locally_Compact}). But since $K$ is compact and $X$
            is Hausdorff, $K$ is closed. But $C$ is closed, and therefore
            $C\cap{K}$ is closed. If $C\cap{K}=\emptyset$, let
            $\mathcal{V}=X\setminus{K}$. Since $K$ is closed,  $\mathcal{V}$ is
            therefore open. But $C\subseteq\mathcal{V}$ and
            $\mathcal{U}\cap\mathcal{V}=\emptyset$, thus separating $x$ and $C$.
            If $C\cap{K}\ne\emptyset$, then since $X$ is Hausdorff, $\{x\}$ is
            closed. But then $(C\cap{K})\cup\{x\}$ is a closed subset of a
            compact set, and is hence compact. But compact Hausdorff spaces are
            regular (Thm.~\ref{thm:Compact_Hausdorff_is_Regular}). Thus there is
            are open subsets $\mathcal{U}_{0},\mathcal{V}_{0}$ such that
            $x\in\mathcal{U}_{0}$, $C\cap{K}\subseteq\mathcal{V}_{0}$ and
            $\mathcal{U}_{0}\cap\mathcal{V}_{0}=\emptyset$.
            Let $\mathcal{V}=\mathcal{V}_{0}\cup(X\setminus{K})$. Then
            $C\subseteq\mathcal{V}$, and
            $\mathcal{V}\cap\mathcal{U}_{0}=\emptyset$, hence separating
            $x$ and $C$.
        \end{proof}
        \begin{fdefinition}{Paracompact}{Paracompact}
            A paracompact topological space is a topological space
            $\topspace{X}$ such that for every open cover $\mathcal{O}$ of $X$,
            there exists a locally finite refinement $\Delta$ of $\mathcal{O}$.
        \end{fdefinition}
        \begin{ftheorem}{Urysohn's Metrization Theorem}
                        {Urysohn_Metrization_Theorem}
            If $\topspace{X}$ is a regular second countable Hausdorff
            topological space, then it is metrizable.
        \end{ftheorem}
        \begin{ftheorem}{Stone's Paracompactness Theorem}
                        {Stones_Paracompactness_Theorem}
            If $\topspace{X}$ is a metrizable topological space, then it is
            paracompact.
        \end{ftheorem}
        \begin{ftheorem}{Smirnov Metrization Theorem}
                        {Smirnov_Metrization_Theorem}
            If $\topspace{X}$ is locally metrizable and paracompact, then it is
            metrizbale.
        \end{ftheorem}
        \begin{theorem}
            \label{thm:Count_Open_Cover_of_Sec_Count_Implies_Sec_Count}%
            If $\topspace{X}$ is a topological space, if
            $\mathcal{O}\subseteq\tau$ is a countable subset of open sets such
            that for all $\mathcal{U}\in\mathcal{O}$,
            $\topspace[\mathcal{U}]{\mathcal{U}}$ is second countable, where
            $\tau_{\mathcal{U}}$ is the subspace topology, then $\topspace{X}$
            is second countable.
        \end{theorem}
        \begin{proof}
            For let $\mathcal{B}$ be the collection of all of the bases for all
            of the $\mathcal{U}\in\mathcal{O}$. Since it is the countable union
            of countable sets, it is countable. But since $\mathcal{U}$ is open
            for all $\mathcal{U}\in\mathcal{O}$, this is a countable open cover
            of $X$. It suffices to show that it is a basis. Let
            $\mathcal{V}\in\tau$ be an open subset. But then:
            \begin{align}
                \mathcal{V}&=\mathcal{V}\cap{X}\\
                &=\mathcal{V}\cap\Big(
                    \bigcup_{\mathcal{U}\in\mathcal{O}}\mathcal{U}
                \Big)\\
                &=\bigcup_{\mathcal{U}\in\mathcal{O}}
                    \big(\mathcal{V}\cap\mathcal{U}\big)
            \end{align}
            But $\mathcal{V}\cap\mathcal{U}$ is an open subset of the subspace
            $\topspace[\mathcal{U}]{\mathcal{U}}$, and hence there is a subset
            of $\Delta_{\mathcal{V}}\subseteq\mathcal{B}_{\mathcal{U}}$ such
            that $\mathcal{V}\cap\mathcal{U}=\bigcup\Delta_{\mathcal{V}}$. But
            then the entire of $\mathcal{V}$ is the union of all such
            collections, each of which is contained in $\mathcal{B}$, and hence
            $\mathcal{V}$ can be written as the union of elements of
            $\mathcal{B}$. Thus, $\mathcal{B}$ is a basis.
        \end{proof}
        The requirement that the covering collection be open subspaces is
        crucial. The quotient space $\mathbb{R}/R$, where $R$ is the equivalence
        relation generated by $nRm$ for all $n,m\in\mathbb{Z}$, can be thought
        of as a countable collection of rings all glued together at the origin.
        Hence, it can be covered by countably many closed subspaces, each of
        which is homeomorphic to $\nsphere[1]$ in the subspace topology, and
        hence each of which is second countable. However, this space
        $\mathbb{R}/R$ is not even first countable, let alone second countable.
        The point $[0]\in\mathbb{R}/R$ has no countable neighborhood basis.
        \begin{theorem}
            \label{thm:loc_path_con_imply_path_comps_open}%
            If $\topspace{X}$ is locally path connected, if $x\in{X}$, and if
            $\mathcal{U}\subseteq{X}$ is a path connected component of $X$,
            then $\mathcal{U}$ is open.
        \end{theorem}
        \begin{proof}
            For if $\topspace{X}$ is locally path connected, there is a basis
            $\mathcal{B}$ of open and path connected subsets of $X$. But if
            $\mathcal{U}\subseteq{X}$ is a path connected component, then for
            all $x,y\in\mathcal{U}$ there is a path
            $\gamma:[0,1]\rightarrow\mathcal{U}$ connecting $x$ and $y$, and for
            all $z\in{X}$ such that $z\notin\mathcal{U}$, there is no path
            between $x$ and $z$. But $\mathcal{B}$ is a basis, and hence for
            all $x\in\mathcal{U}$ there is a $B\in\mathcal{B}$ such that
            $x\in{B}$. By choice, we get a function
            $B:\mathcal{U}\rightarrow\mathcal{B}$. Moreover, since
            $B\in\mathcal{B}$, it is path connected. But then for all
            $x\in\mathcal{U}$, $x\in{B}_{x}$, and since $B_{x}$ is path
            connected it is true that $B_{x}\subseteq\mathcal{U}$ since
            $\mathcal{U}$ is a path connected component. But then
            $\mathcal{U}=\bigcup{B}_{x}$, which is the union of open sets, and
            hence $\mathcal{U}$ is open.
        \end{proof}
        \begin{theorem}
            \label{thm:Loc_Path_and_Con_Imply_Path_Con}
            If $\topspace{X}$ is locally path connected and connected, then it
            is path connected.
        \end{theorem}
        \begin{proof}
            For if not then there are two points $x,y\in{X}$ with no path
            between them. But the since $\topspace{X}$ is locally path
            connected, the path connected components of $x$ and $y$ are open
            (Thm.~\ref{thm:loc_path_con_imply_path_comps_open}). But let
            $\mathcal{U}$ be the path connected component containing $x$, and
            let $\mathcal{V}$ be the union of all other path connected
            components. Then $\mathcal{V}$ is non-empty since $y\in\mathcal{V}$,
            and hence $\mathcal{U}$ and $\mathcal{V}$ are non-empty disjoint
            open subsets that cover $X$, a contradiction since $X$ is connected.
        \end{proof}
        \begin{fdefinition}{Locally Euclidean}{Locally_Euclidean}
            A locally Euclidean topological space is a topological space
            $\topspace{X}$ such that for all $x\in{X}$ there exists an open
            subset $\mathcal{U}\in\tau$ and an $n\in\mathbb{N}$ such that
            $x\in\mathcal{U}$ and $\mathcal{U}$ is homeomorphic to an open
            subset of $\nspace$.
        \end{fdefinition}
        \begin{theorem}
            \label{thm:Equiv_Def_Loc_Euclidean}%
            If $\topspace{X}$ is locally Euclidean, then for all $x\in{X}$ there
            is an open subset $\mathcal{U}\in\tau$ and an $n\in\mathbb{N}$ such
            that $\mathcal{U}$ is homeomorphic to $\nspace$.
        \end{theorem}
        \begin{proof}
            For if $\topspace{X}$ is locally Euclidean, then for all $x\in{X}$
            there is an open subset $\mathcal{V}\in\tau$ and an $n\in\mathbb{N}$
            such that $x\in\mathcal{V}$ and $\mathcal{V}$ is homeomorphic to an
            open subset of $\nspace$. But then there is an injective continuous
            open mapping $\varphi:\mathcal{V}\rightarrow\nspace$. Let
            $\vector{y}=\varphi(x)$. But since $\varphi$ is an open mapping,
            and since $\mathcal{V}$ is open, $\varphi[\mathcal{V}]$ is an open
            subset of $\nspace$. But $\vector{y}=\varphi(x)$, and hence
            $\vector{y}\in\varphi[\mathcal{V}]$. But if $\varphi[\mathcal{V}]$
            is open and $\vector{y}\in\varphi[\mathcal{V}]$, then there is an
            $r>0$ such that the open ball $B$ defined by:
            \begin{equation}
                B=\rball{r}{\metspace[\norm{\cdot}_{2}]{\nspace}}{\vector{y}}
            \end{equation}
            is contained in $\varphi[\mathcal{V}]$. But $\varphi$ is continuous,
            and open balls are open, and hence
            $\varphi^{\minus{1}}[B]$ is an open subset of $\mathcal{V}$.
            Moreover, since $\vector{y}\in{B}$,
            $x\in\varphi^{\minus{1}}[B]$.
            But $\varphi$ is an injective continuous open mapping, and thus the
            restriction of $\varphi$ to an open subset is an injective
            continuous open mapping. Hence, $\varphi|_{\varphi^{\minus{1}}[B]}$
            is a homeomorphism onto it's image, which is $B$. And since open
            balls in $\nspace$ are homeomorphic to $\nspace$,
            $\varphi^{\minus{1}}[B]$ is homeomorphic to $\nspace$. Thus, there
            is an open subset $\varphi^{\minus{1}}[B]$ containing $x$ and an
            $n\in\mathbb{N}$ such that $\varphi^{\minus{1}}[B]$ is homeomorphic
            to $\nspace$.
        \end{proof}
        \begin{theorem}
            \label{thm:Loc_Euc_Existence_of_Basis_of_nspace_Sets}%
            If $\topspace{X}$ is a locally Euclidean topological space, then
            there exists a basis $\mathcal{B}$ such that for all
            $\mathcal{U}\in\mathcal{B}$ there is an $n\in\mathbb{N}$ such that
            $\mathcal{U}$ is homeomorphic to $\nspace$.
        \end{theorem}
        \begin{proof}
            For if $\topspace{X}$ is locally Euclidean, then for all $x\in{X}$
            there is an open subset $\mathcal{U}_{x}\in\tau$ and an
            $n\in\mathbb{N}$ such that $\mathcal{U}_{x}$ is homeomorphic to
            $\nspace$ (Thm.~\ref{thm:Equiv_Def_Loc_Euclidean}). Let
            $\varphi_{x}:\mathcal{U}_{x}\rightarrow\nspace$ be such a
            homeomorphism, and let $\mathcal{B}_{x}$ be the set:
            \begin{equation}
                \mathcal{B}_{x}=\{\,
                    \varphi_{x}^{\minus{1}}\big[
                        \rball{r}{\metspace[\norm{\cdot}_{2}]{\nspace}}
                        {\vector{y}}\big]
                    \;|\;r>0,\,\vector{y}\in\nspace\,\}
            \end{equation}
            Then by construction, every element of $\mathcal{B}_{x}$ is
            homeomorphic to $\nspace$. Let $\mathcal{B}$ be the collection of
            all such sets for all $x\in{X}$. If $\mathcal{U}$, $\mathcal{V}$ are
            elements of $\mathcal{B}$, then there is an $x\in{X}$ such that
            $\mathcal{U}\in\mathcal{B}_{x}$. Let
            $y\in\mathcal{U}\cap\mathcal{V}$ and let $\vector{y}$ be the image
            of $y$ under $\varphi_{x}$. But $\mathcal{U}\cap\mathcal{V}$ is
            open, and hence there is an $r>0$ such that the ball about
            $\vector{y}$ is contained in the image of
            $\varphi_{x}[\mathcal{U}\cap\mathcal{V}]$. But this $r$ ball is
            contained in $\mathcal{B}$. Hence, $\mathcal{B}$ is a basis.
        \end{proof}
        \begin{theorem}
            \label{thm:Loc_Euc_Existence_of_Basis_of_Precompact_Balls}%
            If $\topspace{X}$ is a locally Euclidean topological space, then
            there is a basis $\mathcal{B}$ such that for all
            $\mathcal{U}\in\mathcal{B}$ it is true that $\mathcal{U}$ is
            precompact in $\tau$ and such that there exists an $n\in\mathbb{N}$
            such that $\mathcal{U}$ is homeomorphic to $\nspace$.
        \end{theorem}
        \begin{proof}
            For by Thm.~\ref{thm:Loc_Euc_Existence_of_Basis_of_nspace_Sets},
            there is a basis $\mathcal{B}$ of $\tau$ such that for all
            $\mathcal{U}\in\mathcal{B}$ there is an $n\in\mathbb{N}$ such that
            $\mathcal{U}$ is homeomorphic to $\nspace$. Let
            $\varphi_{\mathcal{U}}:\mathcal{U}\rightarrow\nspace$ be such a
            homeomorphism. Define $\mathcal{B}_{x}$ by:
            \begin{equation}
                \mathcal{B}_{x}=\{\,
                    \varphi_{\mathcal{U}}^{\minus{1}}\big[
                        \rball{r}{\metspace[\norm{\cdot}_{2}]{\nspace}}
                        {\vector{y}}\big]\;|\;r>0,\,\vector{y}\in\nspace\,\}
            \end{equation}
            But by the Heine-Borel theorem, for all
            $\mathcal{V}\in\mathcal{B}_{x}$,
            $\closure[\nspace]{\varphi_{\mathcal{U}}[\mathcal{V}]}$ is compact
            in $\nspace$, and since $\varphi_{\mathcal{U}}$ is a homeomorphism,
            $\closure{\mathcal{V}}$ is compact in $\mathcal{U}$. But then
            $\closure{\mathcal{V}}$ is compact in $X$. The collection of all
            such $\mathcal{B}_{x}$ is thus a basis of precompact subsets that
            are homeomorphic to open balls in $\nspace$, which are homeomorphic
            to $\nspace$.
        \end{proof}
        \begin{theorem}
            \label{thm:Loc_Euc_is_Loc_Compact}%
            If $\topspace{X}$ is locally Euclidean, then it is locally compact.
        \end{theorem}
        \begin{proof}
            For if $\topspace{X}$ is locally Euclidean, then there exists a
            basis $\mathcal{B}$ of precompact coordinate balls
            (Thm.~\ref{thm:Loc_Euc_Existence_of_Basis_of_Precompact_Balls}). But
            then for all $x\in{X}$ there is a $\mathcal{U}\in\mathcal{B}$ such
            that $x\in\mathcal{U}$ and $\closure{\mathcal{U}}$ is compact. Thus,
            $\topspace{X}$ is locally compact (Def.~\ref{def:Locally_Compact}).
        \end{proof}
        \begin{theorem}
            \label{thm:Loc_Euc_Implies_Loc_Met}%
            If $\topspace{X}$ is locally Euclidean, then it is locally
            metrizable.
        \end{theorem}
        \begin{proof}
            For if $\topspace{X}$ is locally Euclidean, then for all $x\in{X}$
            there is an $n\in\mathbb{N}$ and a $\mathcal{U}\in\tau$ such that
            $x\in\mathcal{U}$ and $\mathcal{U}$ is homeomorphic to
            $\nspace$ (Thm.~\ref{thm:Equiv_Def_Loc_Euclidean}). But
            $\nspace$ is metrizable, and thus $\mathcal{U}$ is metrizable.
            Hence, $X$ is locally metrizable.
        \end{proof}
        \begin{theorem}
            \label{thm:Loc_Euc_Haus_is_Regular}%
            If $\topspace{X}$ is locally Euclidean and Hausdorff, then it is
            regular.
        \end{theorem}
        \begin{proof}
            For if $\topspace{X}$ is locally Euclidean, then it is locally
            compact (Thm.~\ref{thm:Loc_Euc_is_Loc_Compact}). But locally
            compact Hausdorff spaces are regular
            (Thm.~\ref{thm:Loc_Comp_and_Hausdorff_is_Regular}). Thus,
            $X$ is regular.
        \end{proof}
        \begin{fdefinition}{Topological Manifold}{Topological_Manifold}
            A topological manifold is a locally Euclidean, Hausdorff, second
            countable topological space of constant dimension.
        \end{fdefinition}
        \begin{theorem}
            \label{thm:Manifolds_are_Lindelof}%
            If $\topspace{X}$ is a topological manifold, then it is
            Lindel\"{o}f.
        \end{theorem}
        \begin{proof}
            For if $X$ is a topological manifold, then it is second countable
            (Def.~\ref{def:Topological_Manifold}). But second countable
            topological spaces are Lindel\"{o}f
            (Thm.~\ref{thm:Second_Countable_Implies_Lindelof}). Thus,
            $\topspace{X}$ is Lindel\"{o}f.
        \end{proof}
        \begin{theorem}
            If $\topspace{X}$ is a topological manifold, then it is $\sigma$
            compact.
        \end{theorem}
        \begin{proof}
            For if $X$ is a topological manifold, then it is Lindel\"{o}f
            (Thm.~\ref{thm:Manifolds_are_Lindelof}). But topological manifolds
            are locally Euclidean (Def.~\ref{def:Locally_Euclidean}) and locally
            Euclidean spaces are locally compact
            (Thm.~\ref{thm:Loc_Euc_is_Loc_Compact}). But if $\topspace{X}$ is
            locally compact and Lindel\"{o}f, then it is $\sigma$ compact
            (Thm.~\ref{thm:Loc_Comp_and_Lindelof_Implies_Sigma_Comp}).
        \end{proof}
        \begin{theorem}
            \label{thm:Loc_Euc_Haus_Sigma_Comp_is_Manifold}%
            If $\topspace{X}$ is a locally Euclidean Hausdorff topological
            space that is $\sigma$ compact, then it is a topological manifold.
        \end{theorem}
        \begin{proof}
            For if $\topspace{X}$ is locally Euclidean, then there is a basis
            $\mathcal{B}$ of precompact coordinate balls
            (Thm.~\ref{thm:Loc_Euc_Existence_of_Basis_of_Precompact_Balls}).
            But if $X$ is $\sigma$ compact, then it is Lindel\"{o}f
            (Thm.~\ref{thm:Sigma_Compact_Implies_Lindelof}). But $\mathcal{B}$
            is an open cover of $X$, and hence there is a countable subcover
            $\Delta$. But then $\Delta$ is a countable collection of open
            subspaces of $X$, each of which is homeomorphic to
            $\nspace$ and hence second countable. Hence $X$ is second countable
            (Thm.~\ref{thm:Count_Open_Cover_of_Sec_Count_Implies_Sec_Count}).
            But if $X$ is locally Euclidean, Hausdorff, and second countable,
            then it is a topological manifold
            (Def.~\ref{def:Topological_Manifold}).
        \end{proof}
        \begin{theorem}
            \label{thm:Loc_Euc_Hausdorff_Lindelof_is_Manifold}%
            If $\topspace{X}$ is a locally Euclidean Hausdorff topological space
            that is Lindel\"{o}f, then $X$ is a topological manifold.
        \end{theorem}
        \begin{proof}
            For if $\topspace{X}$ is locally Euclidean, then it is locally
            compact (Thm.~\ref{thm:Loc_Euc_is_Loc_Compact}). But locally compact
            Lindel\"{o}f spaces are $\sigma$ compact
            (Thm.~\ref{thm:Loc_Comp_and_Lindelof_Implies_Sigma_Comp}). But
            locally Euclidean Hausdorff topological spaces that are $\sigma$
            compact are topological manifolds
            (Thm.~\ref{thm:Loc_Euc_Haus_Sigma_Comp_is_Manifold}). Therefore,
            $\topspace{X}$ is a topological manifold.
        \end{proof}
        \begin{theorem}
            \label{thm:Top_Man_is_Metrizable}%
            If $\topspace{X}$ is a topological manifold, then it is metrizable.
        \end{theorem}
        \begin{proof}
            For if $\topspace{X}$ is a topological manifold, then it is locally
            Euclidean and Hausdorff (Def.~\ref{def:Topological_Manifold}). But
            locally Euclidean Hausdorff spaces are regular
            (Thm.~\ref{thm:Loc_Euc_Haus_is_Regular}). But topological manifolds
            are also second countable (Def.~\ref{def:Topological_Manifold}) and
            regular second countable Hausdorff spaces are metrizable by
            Urysohn's metrization theorem 
            Thm.~\ref{thm:Urysohn_Metrization_Theorem}). Thus, $\topspace{X}$ is
            metrizable.
        \end{proof}
        \begin{theorem}
            \label{thm:Top_Man_is_Paracompact}%
            If $\topspace{X}$ is a topological manifold, then it is paracompact.
        \end{theorem}
        \begin{proof}
            For if $\topspace{X}$ is a topological manifold, then it is
            metrizable (Thm.~\ref{thm:Top_Man_is_Metrizable}). But by Stone's
            paracompactness theorem, metrizable spaces are paracompact
            (Thm.~\ref{thm:Stones_Paracompactness_Theorem}). Hence,
            $\topspace{X}$ is paracompact.
        \end{proof}
        \begin{fdefinition}{Topological Group}{Topological_Group}
            A topological group, denoted $\topgroup{G}$, is a topological space
            $\topspace{G}$ and a binary operation $*$ such that $\monoid{G}$ is
            a group, and such that $g:G\times{G}\rightarrow{G}$ defined by
            $g(a,b)=a*b$ is continuous with respect to the product topology on
            $G$, and such that $\nu:G\rightarrow{G}$ defined by
            $\nu(a)=a^{\minus{1}}$, where $a^{\minus{1}}$ is the inverse element
            of $a$ under $*$, is continuous.
        \end{fdefinition}
        \begin{fdefinition}{Continuous Group Action}{Continuous_Group_Action}
            A continuous group action of a topological group $\topgroup{G}$
            on a topological space $\topspace[X]{X}$ is a function
            $\Theta:G\times{X}\rightarrow{X}$ such that for all $x\in{X}$ and
            for all $a,b\in{G}$, the following are true:
            \begin{align*}
                \Theta(e,x)&=x\\
                \Theta\big(a,\Theta(b,x)\big)&=\Theta(a*b,x)
            \end{align*}
            where $e$ is the unital element of $G$, and such that $\Theta$ is
            continuous with respect to the product topology.
        \end{fdefinition}
        \begin{theorem}
            \label{thm:Quotient_by_Compact_T2_Group_Preserves_T2}%
            If $\topspace[X]{X}$ is a Hausdorff topological space, if
            $\topgroup{G}$ is a compact Hausdorff topological group, if
            $\Theta$ is a continuous group action of $G$ on $X$, and if
            $\topspace[q]{X/G}$ is the quotient topology formed by the orbits
            of $\Theta$, then $X/G$ is Hausdorff.
        \end{theorem}
        \begin{proof}
            Need to fill in.
        \end{proof}
        \begin{theorem}
            \label{thm:Quotient_by_Compact_T2_Group_Preserves_Sec_Count}%
            If $\topspace[X]{X}$ is a second countable topological space, if
            $\topgroup{G}$ is a compact Hausdorff topological group, if
            $\Theta$ is a continuous group action of $G$ on $X$, and if
            $\topspace[q]{X/G}$ is the quotient topology formed by the orbits
            of $\Theta$, then $X/G$ is second countable.
        \end{theorem}
        \begin{proof}
            Also need to fill in.
        \end{proof}
    \section{Homework I: Part A}
        \begin{problem}
            Show that $\mathbb{RP}^{n}$ is Hausdorff and second countable.
        \end{problem}
        \begin{solution}
            For $\mathbb{RP}^{n}$ is homeomorphic to the quotient space of
            $\nsphere$ by the multiplicative group $G=\{\minus{1},1\}$ with the
            group action $\Theta:G\times\nsphere\rightarrow\nsphere$ defined by
            $\Theta(n,\vector{s})=n\cdot\vector{s}$ (this is well defined since
            $n=\pm{1}$, and hence $\norm{n\cdot\vector{s}}=1$ and thus
            $n\cdot\vector{s}$ still lies in $\nsphere$). Equipping $G$ with the
            discrete topology makes $\topgroup{G}$ a topological group, and
            $\Theta$ a continuous group action on $\nsphere$. But $G$ is finite,
            and hence compact, and moreover it is Hausdorff since the discrete
            topology is always Hausdorff (it is metrizable, in fact). Thus
            $\nsphere/G$ is the quotient of a second countable Hausdorff space
            by a compact Hausdorff group, and is therefore Hausdorff
            (Thm.~\ref{thm:Quotient_by_Compact_T2_Group_Preserves_T2}) and
            second countable
            (Thm.~\ref{thm:Quotient_by_Compact_T2_Group_Preserves_Sec_Count}).
            Thus, $\mathbb{RP}^{n}$ is Hausdorff and second countable.
        \end{solution}
        \begin{problem}
            Show that if $\topspace{X}$ is a locally Euclidean connected
            Hausdorff space, then it is a manifold if and only if it is
            paracompact.
        \end{problem}
        \begin{solution}
            For topological manifolds are paracompact
            (Thm.~\ref{thm:Top_Man_is_Paracompact}). Going the other way, if $X$
            is locally Euclidean, then there is a basis of precompact open
            subsets $\mathcal{B}$, each of which is homeomorphic to $\nspace$
            (Thm.~\ref{thm:Loc_Euc_Existence_of_Basis_of_Precompact_Balls}). But
            $X$ is paracompact, and hence there is a locally finite refinement
            $\Delta$ of $\mathcal{B}$ (Def.~\ref{def:Paracompact}). Let
            $\mathcal{U}_{0}$ be an element of $\Delta$. Since $\Delta$ is a
            refinement of $\mathcal{B}$, there is an element
            $\mathcal{V}\in\mathcal{B}$ such that
            $\mathcal{U}_{0}\subseteq\mathcal{V}$. But then
            $\closure{\mathcal{U}_{0}}\subseteq\closure{\mathcal{V}}$, and
            $\mathcal{V}$ is precompact, and thus $\closure{\mathcal{V}}$
            is compact. But then $\closure{\mathcal{U}_{0}}$ is a closed subset
            of a compact set, and is hence compact. For all $n\in\mathbb{N}$,
            define $\mathcal{U}_{n}$ by:
            \begin{equation}
                \mathcal{U}_{n}=\Big\{\,x\in{X}\;|\;
                    \exists_{A:\mathbb{Z}_{n}\rightarrow\Delta}\big(
                        \mathcal{U}_{0}=A_{0},\,x\in{A}_{n-1}\textrm{ and }
                        \forall_{i<n-1}(A_{i}\cap{A}_{i+1}\ne\emptyset)
                    \big)\,\Big\}
            \end{equation}
            That is, the set of all points that are separated from
            $\mathcal{U}_{0}$ by at most $n$ consecutive elements of $\Delta$.
            Then $\mathcal{U}_{n}$ is precompact. We prove by induction. The
            base case of $\mathcal{U}_{0}$ is true from the previous paragraph.
            Suppose it is true for $n\in\mathbb{N}$. Since $X$ is locally
            Euclidean, it is locally metrizable
            (Thm.\ref{thm:Loc_Euc_Implies_Loc_Met}). But then $X$ is locally
            metrizable, Hausdorff, and paracompact, and is therefore metrizable
            by the Smirnov metrization theorem
            (Thm.~\ref{thm:Smirnov_Metrization_Theorem}). But then if
            $\mathcal{C}\subseteq{X}$, then $\mathcal{C}$ is compact if and only
            if it is sequentially compact
            (Thm.~\ref{thm:Met_Space_Seq_Compact_iff_Compact}). But
            $\mathcal{U}_{n+1}$ is the union of $\closure{\mathcal{U}_{n}}$ and
            elements of $\mathcal{V}\in\Delta$ such that
            $\mathcal{V}\cap\mathcal{U}_{n}\ne\emptyset$. But every element of
            $\Delta$ is precompact, and since $\closure{\mathcal{U}_{n+1}}$ is
            not compact, there must be infinitely many such $\mathcal{V}$. But
            $\mathcal{V}\cap\mathcal{U}_{n}$ is non
            empty for all such $\mathcal{V}$, and hence by the axiom of choice
            there is a sequence $a:\mathbb{N}\rightarrow\mathcal{U}_{n}$ such
            that $a_{j}$ lies in a distinct $\mathcal{V}$ for all
            $j\in\mathbb{N}$. But $\closure{\mathcal{U}_{n}}$ is compact, and
            hence sequentially compact, and thus there is a convergent
            subsequence $a_{k}:\mathbb{N}\rightarrow\closure{\mathcal{U}_{n}}$
            with a limit $x$. But $x\in\closure{\mathcal{U}_{n}}$ and hence
            there is an open set $V\in\tau$ that has non-empty intersection with
            only finitely many elements of $\Delta$, since $\Delta$ is a locally
            finite refinement. But since $a_{k}$ converges to an element of $V$,
            there is an $N\in\mathbb{N}$ such that for all $j>N$,
            $a_{k_{j}}\in{V}$. But each $a_{k_{j}}$ lies in a different
            $\mathcal{V}\in\Delta$, and hence infinitely many elements of
            $\Delta$ have non-empty intersection with $V$, a contradiction.
            Hence, $\mathcal{U}_{n+1}$ is covered by finitely many elements of
            $\Delta$ and is therefore precompact. Moreover,
            $\bigcup\closure{\mathcal{U}_{n}}=X$. For if $y\in{X}$, let
            $x\in\mathcal{U}_{0}$ be any point. Then since $X$ is locally path
            connected and connected, it is path connected
            (Thm.~\ref{thm:Loc_Path_and_Con_Imply_Path_Con}).
            Let $\gamma:[0,1]\rightarrow{X}$ be a path from $x$ to $y$. But
            $[0,1]$ is compact and hence $\gamma\big[[0,1]\big]$ is a compact
            subset of $X$. But then it is covered by only finitely many elements
            of $\Delta$, and hence $y$ is contained in one of the
            $\mathcal{U}_{n}$. Thus $X$ is $\sigma$ compact
            (Def.~\ref{def:Sigma_Compact}). But locally Euclidean $\sigma$
            compact Hausdorff spaces are topological manifolds
            (Thm.~\ref{thm:Loc_Euc_Haus_Sigma_Comp_is_Manifold}).
        \end{solution}
        \begin{problem}
            Lee Problem 1-7.
        \end{problem}
        \begin{solution}
            Let $\vector{x}\in\nsphere\setminus{N}$, where $N$ is the north
            pole. Let $\Gamma$ be the line through $N$ and $\vector{x}$. We can
            parameterize this as follows:
            \begin{equation}
                \Gamma(t)=tN+(1-t)\vector{x}
            \end{equation}
            To find when this lies on the $\nspace$ hyperplane of $\nspace[n+1]$
            defined by $(x_{1},\dots,x_{n},0)$ we simply need to find
            $t$ such that $\Gamma(t)=(x_{1},\dots,x_{n},0)$. Solving for the
            last coordinate, we get (Since $N=(0,\dots,0,1)$):
            \begin{equation}
                t+(1-t)x_{n+1}=0
                \Longrightarrow
                t(1-x_{n+1})=\minus{x}_{n+1}
                \Longrightarrow
                t=\frac{x_{n+1}}{x_{n+1}-1}
            \end{equation}
            evaluating $\Gamma$ at this particular time obtains:
            \begin{equation}
                \Gamma\Big(\frac{x_{n+1}}{x_{n+1}-1}\Big)
                    =\frac{1}{1-x_{n+1}}(x_{1},\dots,x_{n},0)
            \end{equation}
            in agreement with $\sigma$. For the south pole we repeat the
            argument, obtaining the curve $\Lambda$ defined by:
            \begin{equation}
                \Lambda(t)=tS+(1-t)\vector{x}
            \end{equation}
            Solving for $\Lambda(t)=(x_{1},\dots,x_{n},0)$, we look at the last
            component and set this equal to zero:
            \begin{equation}
                \minus{t}+(1-t)x_{n+1}=0
                \Longrightarrow
                \minus{t}(1+x_{n+1})=\minus{x}_{n+1}
                \Longrightarrow
                t=\frac{x_{n+1}}{1+x_{n+1}}
            \end{equation}
            Evaluating this $t$ into $\Lambda$ gives:
            \begin{align*}
                \Lambda\Big(\frac{x_{n+1}}{1+x_{n+1}}\Big)
                &=\Big(0,\dots,0,\frac{\minus{x}_{n+1}}{1+x_{n+1}}\Big)
                    +\Big(1-\frac{x_{n+1}}{1+x_{n+1}}\Big)
                    (x_{1},\dots,x_{n},x_{n+1})\\
                &=\Big(0,\dots,0,\frac{\minus{x}_{n+1}}{1+x_{n+1}}\Big)+
                    \Big(\frac{1}{1+x_{n+1}}\Big)(x_{1},\dots,x_{n},x_{n+1})\\
                &=\frac{(x_{1},\dots,x_{n},0)}{1+x_{n+1}}
            \end{align*}
            To show that $\sigma$ is bijective, it suffices to show that
            $\sigma\circ\sigma^{\minus{1}}$ and $\sigma^{\minus{1}}\circ\sigma$
            are the identity mappings, where:
            \begin{equation}
                \sigma^{\minus{1}}(X_{1},\dots,X_{n})=
                    \frac{(2X_{1},\dots,2X_{n},\norm{\vector{X}}^{2}-1)}
                         {\norm{\vector{X}}^{2}+1}
            \end{equation}
            But:
            \begin{align*}
                (\sigma\circ\sigma^{\minus{1}})(\vector{X})&=
                \sigma\Big(
                    \frac{(2X_{1},\dots,2X_{n},\norm{\vector{X}}^{2}-1)}
                         {\norm{\vector{X}}^{2}+1}
                \Big)\\
                &=\frac{1}
                    {1-\frac{\norm{\vector{X}}^{2}-1}{\norm{\vector{X}}^{2}+1}}
                    \Big(\frac{2X_{1}}{\norm{\vector{X}}^{2}+1)},\dots,
                    \frac{2X_{n}}{\norm{\vector{X}}^{2}+1)}\Big)\\
                &=\frac{1}{\norm{\vector{X}^{2}}+1}\cdot
                    \frac{1}
                    {1-\frac{\norm{\vector{X}}^{2}-1}{\norm{\vector{X}}^{2}+1}}
                    (2X_{1},\dots,2X_{n})\\
                &=\frac{1}{\norm{\vector{X}}^{2}+1-\norm{\vector{X}}^{2}+1}
                    (2X_{1},\dots,2X_{n})\\
                &=\frac{1}{2}(2X_{1},\dots,2X_{n})\\
                &=(X_{1},\dots,X_{n})\\
                &=\vector{X}
            \end{align*}
            and therefore $\sigma\circ\sigma^{\minus{1}}$ is the identity. Going
            the other way, for $\vector{x}\in\nsphere\setminus\{N\}$ we have:
            \begin{align*}
                (\sigma^{\minus{1}}\circ\sigma)(\vector{x})
                &=\sigma^{\minus{1}}\Big(
                    \frac{(x_{1},\dots,x_{n})}{1-x_{n+1}}
                \Big)\\
                &=\frac{1}{\norm{\frac{(x_{1},\dots,x_{n})}{1-x_{n+1}}}^{2}+1}
                    \Big(
                        \frac{2x_{1}}{1-x_{n+1}},\dots,\frac{2x_{n}}{1-x_{n+1}},
                        \frac{\norm{(x_{1},\dots,x_{n})}^{2}}{(1-x_{n+1})^{2}}-1
                    \Big)
            \end{align*}
            But since $\vector{x}\in\nsphere{S}\setminus\{N\}$,
            $\norm{\vector{x}}=1$. And:
            \begin{align*}
                \frac{1}{\norm{\frac{(x_{1},\dots,x_{n})}{1-x_{n+1}}}^{2}}
                &=\frac{(1-x_{n+1})^{2}}
                      {\norm{(x_{1},\dots,x_{n})}^{2}+(1-x_{n+1})^{2}}\\
                &=\frac{(1-x_{n+1})^{2}}{\norm{\vector{x}}^{2}+1-2x_{n+1}}\\
                &=\frac{(1-x_{n+1})^{2}}{1+1-2x_{n+1}}\\
                &=\frac{(1-x_{n+1})^{2}}{2(1-x_{n+1})}\\
                &=\frac{1-x_{n+1}}{2}
            \end{align*}
            returning to the problem, we then get:
            \begin{align*}
                (\sigma^{\minus{1}}\circ\sigma)(\vector{x})
                &=\frac{1-x_{n+1}}{2}\Big(
                    \frac{2x_{1}}{1-x_{n+1}},\dots,\frac{2x_{n}}{1-x_{n+1}},
                    \frac{\norm{(x_{1},\dots,x_{n})}^{2}}{(1-x_{n+1})^{2}}-1
                \Big)\\
                &=\frac{1-x_{n+1}}{2}\Big(
                    \frac{2x_{1}}{1-x_{n+1}},\dots,\frac{2x_{n}}{1-x_{n+1}},
                    \frac{1-x_{n+1}^{2}}{(1-x_{n+1})^{2}}-1
                \Big)\\
                &=\frac{1-x_{n+1}}{2}\Big(
                    \frac{2x_{1}}{1-x_{n+1}},\dots,\frac{2x_{n}}{1-x_{n+1}},
                    \frac{1+x_{n+1}}{1-x_{n+1}}-1
                \Big)\\
                &=\frac{1-x_{n+1}}{2}\Big(
                    \frac{2x_{1}}{1-x_{n+1}},\dots,\frac{2x_{n}}{1-x_{n+1}},
                    \frac{2x_{n+1}}{1-x_{n+1}}
                \Big)\\
                &=(x_{1},\dots,x_{n},x_{n+1})
            \end{align*}
            and therefore, $\sigma^{\minus{1}}\sigma$ is the identity. Thus,
            $\sigma$ and $\sigma^{\minus{1}}$ are bijections and inverses of
            each other. The transition map
            $\tilde{\sigma}\circ\sigma^{\minus{1}}$, where $\tilde{\sigma}$ is
            the stereographic graphic projection from the south pole, can be
            computed as follows:
            \begin{align*}
                (\tilde{\sigma}\circ\sigma^{\minus{1}})(\vector{X})
                &=\tilde{\sigma}\Big(
                    \frac{2X_{1},\dots,2X_{n},\norm{\vector{X}}^{2}-1}
                         {\norm{\vector{X}}^{2}+1}
                    \Big)\\
                &=\frac{1}{1+\frac{\norm{\vector{X}}^{2}-1}
                                  {\norm{\vector{X}}^{2}+1}}
                    \Big(\frac{2X_{1}}{\norm{\vector{X}}^{2}+1},\dots,
                         \frac{2X_{n}}{\norm{\vector{X}}^{2}+1}\Big)\\
                &=\frac{1}{\norm{\vector{X}}^{2}}(X_{1},\dots,X_{n})\\
                &=\frac{\vector{X}}{\norm{\vector{X}}^{2}}
            \end{align*}
            Since $N$ and $S$ are not included, $\vector{0}$ is not in the
            domain of $\tilde{\sigma}\circ\sigma^{\minus{1}}$, and hence this
            is well defined everywhere and smooth. By symmetry,
            $\sigma\circ\tilde{\sigma}^{\minus{1}}$ is smooth. Hence, these two
            charts constitute a smooth atlas on $\nsphere$ since they cover
            the set and overlap smoothly. Lastly, show that the standard smooth
            structure on $\nsphere$ and the stereographic one are compatible.
            For $i\in\mathbb{Z}_{n}$, let $\varphi_{i}$ be the projection
            mapping of $\mathcal{U}_{i}^{+}$ down to $\nsphere$. We have:
            \begin{align*}
                (\varphi_{i}\circ\sigma^{\minus{1}})(\vector{X})
                &=\varphi_{i}\Big(
                    \frac{(2X_{1},\dots,2X_{n},\norm{\vector{X}}^{2}-1)}
                         {\norm{\vector{X}}^{2}+1}
                \Big)\\
                &=\frac{(2X_{1},\dots,2X_{i-1},2X_{i+1},\dots,
                        \norm{\vector{X}}^{2}-1)}{\norm{\vector{X}}^{2}+1}
            \end{align*}
            which is smooth. Similarly:
            \begin{align*}
                (\sigma\circ\varphi_{i}^{\minus{1}})(\vector{X})
                &=\sigma\Big(
                    \big(x_{1},\dots,x_{i-1},\sqrt{1-\norm{\vector{X}}^{2}},
                    x_{i},\dots,x_{n}\big)\Big)\\
                &=\frac{(x_{1},\dots,x_{i-1},\sqrt{1-\norm{\vector{X}}^{2}},
                        x_{i+1},\dots,x_{n-1})}{1-x_{n}}
            \end{align*}
            which, since the domain of $\sigma$ does not include $x_{n}=1$, this
            is well defined and smooth. Hence, the standard smooth structure on
            $\nsphere$ is compatible with the stereographic one.
        \end{solution}
        \begin{problem}
            Show that $\mathbb{CP}^{n}$ is a smooth $2n$ dimensional manifold.
        \end{problem}
        \begin{solution}
            Given a $\vector{z}$, let
            $\uvector{z}=\vector{z}/\norm{\vector{z}}$. Then $[\vector{z}]$ is
            the same as the orbit of $\uvector{z}$ by the rotation group
            $U(1)$ acting on $\nsphere[2n+1]$, and similary the quotient
            topologies are the same. But $U(1)$ is a compact topological group.
            It is compact since it is a closed and bounded subset of Euclidean
            space, since all of the elements of $U(1)$ have norm 1. But then
            $\nsphere[2n+1]/U(1)$ is Hausdorff
            (Thm.~\ref{thm:Quotient_by_Compact_T2_Group_Preserves_T2}) and
            second countable
            (Thm.~\ref{thm:Quotient_by_Compact_T2_Group_Preserves_Sec_Count}).
            Moreover, since it is the quotient of a compact topological space
            ($\nsphere[2n+1]$ is compact), $\mathbb{CP}^{n}$ is compact as well.
            Lastly, we must show that $\mathbb{CP}^{n}$ can be given a smooth
            structure compatible with this quotient topology. For all
            $i\in\mathbb{Z}_{n+1}$, let $\mathcal{U}_{i}$ be the set of all
            $\vector{z}\in\mathbb{C}^{n+1}$ such that the $i^{th}$ component
            $z_{i}$ is non-zero. Let $\mathcal{V}_{i}=q[\mathcal{U}_{i}]$, where
            $q$ is the canonical projection of $\mathbb{C}^{n+1}$ down to
            $\mathbb{CP}^{n}$ sending $\vector{z}\mapsto[\vector{z}]$. Since
            $\mathcal{U}_{i}$ is a saturated set, $\mathcal{V}_{i}$ is therefore
            open and the restriction of $q$ to $\mathcal{U}_{i}$ is a quotient
            map. Let $\varphi_{i}:\mathcal{V}_{i}\rightarrow\mathbb{C}^{n}$
            be defined by:
            \begin{equation}
                \varphi_{i}[\vector{z}]=\Big(
                    \frac{z_{1}}{z_{i}},\dots,\frac{z_{i-1}}{z_{i}},
                    \frac{z_{i+1}}{z_{i}},\dots,\frac{z_{n+1}}{z_{i}}
                )
            \end{equation}
            this is well defined, for if $\vector{w}\in[\vector{z}]$,
            then there is a $c\in\mathbb{C}$ such that
            $\vector{w}=c\vector{z}$. But then:
            \begin{align*}
                \varphi_{i}[\vector{w}]&=\Big(
                    \frac{w_{1}}{w_{i}},\dots,\frac{w_{i-1}}{w_{i}},
                    \frac{w_{i+1}}{w_{i}},\dots,\frac{w_{n+1}}{w_{i}}
                \Big)\\
                &=\Big(
                    \frac{cz_{1}}{cz_{i}},\dots,\frac{cz_{i-1}}{cz_{i}},
                    \frac{cz_{i+1}}{cz_{i}},\dots,\frac{cz_{n+1}}{cz_{i}}
                \Big)\\
                &=\Big(
                    \frac{z_{1}}{z_{i}},\dots,\frac{z_{i-1}}{z_{i}},
                    \frac{z_{i+1}}{z_{i}},\dots,\frac{z_{n+1}}{z_{i}}
                \Big)\\
                &=\varphi_{i}[\vector{z}]
            \end{align*}
            Thus, $\varphi_{i}$ is continuous. Moreover, it is a homomorphism
            onto it's image, with inverse:
            \begin{equation}
                \varphi_{i}^{\minus{1}}(z_{1},\dots,z_{n})
                [z_{1},\dots,z_{i-1},1,z_{i+1},\dots,z_{n}]
            \end{equation}
            which is continuous. Thus, $\mathcal{V}_{i}$ is homeomorphic to an
            open subset of $\mathbb{C}^{n}$. But $\mathbb{C}^{n}$ is
            homeomorphic to $\nspace[2n]$ (Indeed, set theoretically it simply
            \textit{is} $\nspace[2n]$). This shows that $\mathbb{CP}^{n}$ is
            locally Euclidean, and since it is also Hausdorff and second
            countable, it is therefore a topological manifold
            (Def.~\ref{def:Topological_Manifold}). Lastly, we must show that
            the $\varphi_{i}$ overlap smoothly. But:
            \begin{align*}
                \varphi_{i}\circ\varphi_{j}^{\minus{1}}(z_{1},\dots,z_{n})
                &=\varphi_{i}\big(
                    [z_{1},\dots,z_{j-1},1,z_{j+1},\dots,z_{n}]
                \big)\\
                &=\Big(
                    \frac{z_{1}}{z_{i}},\dots,\frac{z_{j-1}}{z_{i}},
                    \frac{1}{z_{i}},\frac{z_{j+1}}{z_{i}},\dots,
                    \frac{z_{i-1}}{z_{i}},\frac{z_{i+1}}{z_{i}},\dots,
                    \frac{z_{n}}{z_{i}}
                \Big)
            \end{align*}
            which is smooth.
        \end{solution}
    \section{Homework I: Part B}
        \begin{problem}
            Show that the group action
            $\Theta:\autgroup[]{V}\times{V}\rightarrow{V}$ given by
            $\Theta(T,v)=Tv$ is continuous, where $V$ is an $n$ dimensional
            vector space over $\nspace[]$, and $\autgroup[]{V}$ carries the
            subspace topology of $\nspace[n^{2}]$.
        \end{problem}
        \begin{solution}
            Both spaces are second countable, and hence first countable, and
            thus to check continuity it suffices to show that $\Theta$ is
            sequentially continuous
            (Thm.~\ref{thm:First_Countable_Implies_Seq_Cont_is_Cont}).
            Let $(T_{n},v_{n})$ be a sequence such that
            $T_{n}\rightarrow{T}$ and $v_{n}\rightarrow{v}$. But then:
            \begin{subequations}
                \begin{align}
                    \norm{T_{n}v_{n}-Tv}&=\norm{T_{n}v_{n}-T_{n}v+T_{n}v-Tv}\\
                    &\leq\norm{T_{n}v_{n}-T_{n}v}+\norm{T_{n}v-Tv}\\
                    &\leq\norm{T_{n}}\norm{v_{n}-v}+\norm{T_{n}-T}\norm{v}
                \end{align}
            \end{subequations}
            and since $T_{n}\rightarrow{T}$ and $v_{n}\rightarrow{v}$, this
            converges to zero. Hence, $T_{n}v_{n}\rightarrow{T}v$, and thus
            $\Theta$ is sequentially continuous, and therefore continuous.
            Moreover, it is a group action:
            \begin{equation}
                \Theta(\identity{V},v)=\identity{V}(v)=v
            \end{equation}
            and for $T,S\in\autgroup{V}$ we have:
            \begin{equation}
                \Theta\big(T,\theta(S,v)\big)=\Theta(T,Sv)
                =T(Sv)=(TS)v=\Theta(TS,v)
            \end{equation}
            and thus $\Theta$ is a continuous group action
            (Def.~\ref{def:Continuous_Group_Action}).
        \end{solution}
        \begin{problem}
            Let $G$ be a topological Lie group and $V$ a finite dimensional real
            vector space with the usual topology. A real representation of $G$
            on $V$ is a continuous group homeomorphism
            $\rho:G\rightarrow\autgroup{V}$. Any real representation $(V,\rho)$
            or $G$ on $V$ defines a group action
            $\Theta:G\times{V}\rightarrow{V}$ given by
            $\Theta(g,v)=\rho(g)v$. Show that $\rho$ is a representation if and
            only if $\Theta$ is a continuous group action such that
            $\Theta_{g}=\Theta(g,\cdot)\in\autgroup{V}$ for all $g\in{G}$.
        \end{problem}
        \begin{proof}
            Suppose $\rho$ is a real representation of $G$ on $V$. Since $G$ and
            $V$ are manifolds, so is $G\times{V}$. And manifolds are locally
            metrizable, and hence first countable
            (Thm.~\ref{thm:Locally_Metrizable_is_First_Countable}), and
            therefore sequential
            (Thm.~\ref{thm:First_Countable_Implies_Sequential}). Thus, it
            suffices to show that $\Theta$ is a group action and that it is
            sequentially continuous
            (Thm.~\ref{thm:seq_space_seq_cont_eqiv_cont}). It is indeed a group
            action, for:
            \begin{equation}
                \Theta(e,v)=\rho(e)v=\identity{v}=v
            \end{equation}
            since $\rho$ is a group homeomorphism, and thus takes the identity
            of $G$ to the identity of $\autgroup{V}$. Moreover, if
            $g_{1},g_{2}\in{G}$, $v\in{V}$, then since $\rho$ is a group
            homeomorphism we obtain::
            \begin{subequations}
                \begin{align}
                    \Theta\big(g_{1},\Theta(g_{2},v)\big)
                        &=\Theta\big(g_{1},\rho(g_{2})v\big)\\
                        &=\rho(g_{1})\big(\rho(g_{2})v\big)\\
                        &=\big(\rho(g_{1})\rho(g_{2})\big)v\\
                        &=\rho(g_{1}*g_{2})v
                \end{align}
            \end{subequations}
            and thus $\Theta$ is a group action. It is continuous, for let
            $(g_{n},v_{n})\rightarrow(g,v)$. Then:
            \begin{equation}
                \Theta(g_{n},v_{n})=\rho(g_{n})v_{n}
            \end{equation}
            But $\rho$ is continuous, and is therefore sequentially continuous
            (Thm.~\ref{thm:cont_implies_seq_cont}). Thus if
            $g_{n}\rightarrow{g}$, then $\rho(g_{n})\rightarrow\rho(g)$
            (Def.~\ref{def:Sequentially_Continuous}). But then:
            \begin{subequations}
                \begin{align}
                    \norm{\Theta(g_{n},v_{n})-\Theta(g,v)}
                    &=\norm{\rho(g_{n})v_{n}-\rho(g)v}\\
                    &=\norm{\rho(g_{n})v_{n}-\rho(g)v_{n}
                        +\rho(g)v_{n}-\rho(g)v}\\
                    &\leq\norm{\rho(g_{n})v_{n}-\rho(g)v_{n}}+
                        \norm{\rho(g)v_{n}-\rho(g)v}\\
                    &\leq\norm{\rho(g_{n})-\rho(g)}\norm{v_{n}}
                        +\norm{\rho(g)}\norm{v_{n}-v}
                \end{align}
            \end{subequations}
            But $v_{n}$ is a convergent sequence, and is therefore bounded, and
            hence $\norm{v_{n}}$ is bounded. Since $v_{n}\rightarrow{v}$ and
            $\rho(g_{n})\rightarrow\rho(g)$, this whole things tends to zero.
            Therefore, $\Theta$ is sequentially continuous and therefore
            continuous. Moreover, if $g\in{G}$, then
            $\Theta(g,\cdot)\in\autgroup{V}$. For
            $\Theta(g,\cdot)=\rho(g)(\cdot)$, and $\rho(g)\in\autgroup{V}$ by
            hypothesis. In the other direction, suppose $\Theta$ is a continuous
            group action such that $\Theta(g,\cdot)\in\autgroup{V}$. Again,
            since $G$ is a manifold, it suffices to show that $\rho$ is
            sequentially continuous. If $g_{n}\rightarrow{g}$, then for all
            $v\in{V}$, $\Theta(g_{n},v)=\rho(g_{n})v\rightarrow\rho(g)v$ since
            $\Theta$ is continuous and therefore
            $\Theta(g_{n},v)\rightarrow\Theta(g,v)$. But then:
            \begin{equation}
                \norm{\rho(g_{n})v-\rho(g)v}
                \leq\norm{\rho(g_{n})-\rho(g)}\norm{v}
            \end{equation}
            and this tends to zero, showing that
            $\rho(g_{n})\rightarrow\rho(g)$. Hence, $\rho$ is continuous.
            Moreover, it is a group homeomorphism. For if $e\in{G}$ is the
            identity, then for all $v\in{V}$ it is true that:
            \begin{equation}
                \rho(e)v=\Theta(e,v)=v
            \end{equation}
            since $\Theta$ is a group action. Hence, $\rho(e)$ is the identity
            mapping. If $g_{1},g_{2}\in{G}$, then:
            \begin{equation}
                \rho(g_{1}*g_{2})(\cdot)=\Theta(g_{1}*g_{2},\cdot)
                    =\Theta\big(g_{1},\Theta(g_{2},\cdot)\big)
                    =\big(\rho(g_{1})\rho(g_{2})\big)(\cdot)
            \end{equation}
            and thus $\rho(g_{1}*g_{2})=\rho(g_{1})\rho(g_{2})$. Therefore,
            $\rho$ is a group homeomorphism.
        \end{proof}
    \section{Homework II}
        \begin{problem}
            Prove that if $M_{1},\dots,M_{k}$ are smooth manifolds with or
            without boundary such that at most one of them has non-empty
            boundary, if $\mathcal{M}$ is the product manifold, if $\pi_{i}$
            is the $i^{th}$ projection mapping, then $F:N\rightarrow\mathcal{M}$
            is smooth if and only if $\pi_{i}\circ{F}$ is smooth for all
            $i\in\mathbb{Z}_{k}$.
        \end{problem}
        \begin{solution}
            First we show that $\pi_{i}$ is smooth. Let
            $\mathcal{M}=M_{1}\times{M}_{2}$ and let $\pi_{1}$ be the projection
            map $(p,q)\mapsto{p}$. Let $(p,q)\in\mathcal{M}$. Since
            $M_{1}$ and $M_{2}$ are manifolds, there are charts
            $(\mathcal{U}_{1},\varphi_{1})$ and $(\mathcal{U}_{2},\varphi_{2})$
            such that $p\in\mathcal{U}_{1}$ and $q\in\mathcal{U}_{2}$ but by the
            definition of the product manifold,
            $\mathcal{U}_{1}\times\mathcal{U}_{2}$ is open in $\mathcal{M}$ and
            $(p,q)\in\mathcal{U}_{1}\times\mathcal{U}_{2}$. Moreover, the
            product function $\varphi_{1}\times\varphi_{2}$ is smooth. But then
            for all $(x,y)\in\mathcal{U}_{1}\times\mathcal{U}_{2}$ we have:
            \begin{align}
                \varphi_{1}\circ\pi_{1}\circ
                    (\varphi_{1}\times\varphi_{2})^{\minus{1}}(x,y)
                &=\varphi_{1}\circ\big(
                    \pi_{1}\circ(\varphi_{1}\times\varphi_{2})^{\minus{1}}(x,y)
                \big)\\
                &=\varphi_{1}\circ\Big(
                    \pi_{1}\big(\varphi_{1}^{\minus{1}}(x),
                                \varphi_{2}^{\minus{1}}(y)\big)\Big)\\
                &=\varphi_{1}\circ\big(\varphi_{1}^{\minus{1}}(x)\big)\\
                &=x
            \end{align}
            And thus we have the projection map from $\mathbb{R}^{d_{1}+d_{1}}$
            to $\mathbb{R}^{d_{1}}$, which is smooth. Hence for all
            $(p,q)\in\mathcal{M}$ there is a chart $(\mathcal{U},\varphi)$
            containing $(p,q)$ and a chart $(\mathcal{V},\psi)$ containg
            $\pi_{1}(p,q)$ such that $\psi\circ\pi_{1}\circ\varphi^{\minus{1}}$
            is smooth, and thus $\pi_{1}$ is smooth. Similarly, $\pi_{2}$ is
            smooth. For the general case we proceed by induction and write:
            \begin{equation}
                \mathcal{M}=\prod_{k=1}^{n+1}M_{k}=
                    \Big(\prod_{k=1}^{n}M_{k}\Big)\times{M}_{n+1}
                \equiv\widehat{\mathcal{M}}\times{M}_{n+1}
            \end{equation}
            which is the product of two manifolds, one of dimension
            $d_{1}+d_{2}+\dots+d_{n}$ and the other of dimension $d_{n+1}$, and
            thus by the previous argument the projection maps are smooth. Hence,
            $\pi_{n+1}$ is smooth. But by the induction hypothesis, all of the
            $\pi_{i}$ are smooth of $i=1,\dots,n$, and thus all $\pi_{i}$ are
            smooth. Now, suppose $f:N\rightarrow\mathcal{M}$ is smooth. Then
            $F_{i}=\pi_{i}\circ{F}$ is the composition of smooth function and
            hence by theorem 2.10 (d) in Lee's text, $F_{i}$ is smooth. In the
            other direction, we again start with the case that
            $\mathcal{M}=M_{1}\times{M}_{2}$. Suppose
            $F:N\rightarrow\mathcal{M}$ is such that $\pi_{1}\circ{F}$ and
            $\pi_{2}\circ{F}$ are smooth. Let $v\in{N}$ and let
            $(\mathcal{V},\psi)$ be a chart containing $v$ and let
            $(\widehat{\mathcal{U}},\widehat{\varphi})$ be a chart containing
            $F(v)$. Let $\mathcal{U}_{i}=\pi_{i}[\widehat{\mathcal{U}}]$ and
            let $\varphi_{i}=\pi_{i}\circ\widehat{\varphi}$. Then the
            $\varphi_{i}$ are the composition of smooth functions, and hence are
            smooth, and the $\mathcal{U}_{i}$ are the projections of an open set
            and are hence open. Moreover,
            $(\mathcal{U}_{1}\times\mathcal{U}_{2},%
             \varphi_{1}\times\varphi_{2})$ is a chart containing $F(v)$.
        \end{solution}
        \begin{problem}
            Show that $z^{n}:\nsphere[1]\rightarrow\nsphere[1]$ is smooth, the
            antipodal map $\vector{x}\mapsto\minus\vector{x}$, and the function
            $f:\nsphere[3]\rightarrow\nsphere[2]$ defined by:
            \begin{equation}
                f(w,z)=(z\overline{w}+w\overline{z},
                        iw\overline{z}-iz\overline{w},
                        z\overline{z}-w\overline{w})
            \end{equation}
        \end{problem}
        \begin{solution}
            Since $\nsphere[1]$ can be covered by two charts, we simply need to
            check that this function is smooth on these charts. Let
            $\mathcal{U}^{-}=\nsphere[1]\setminus\{(0,1)\}$ and similarly for
            $\mathcal{U}^{+}$. The stereographic projection onto $\nspace[]$
            is just:
            \twocolumneq{\varphi_{\minus}(x,y)=\frac{x}{1-y}}
                        {\varphi_{+}(x,y)=\frac{x}{1+y}}
            The inverse functions are:
            \begin{equation}
                \varphi_{\minus}^{\minus{1}}(X)
                    =\Big(\frac{X}{1+X^{2}},\frac{X^{2}-1}{X^{2}+1}\Big)
            \end{equation}
            and similarly for $\varphi_{+}$. To show that $p_{n}$ is smooth it
            suffices to show that
            $\varphi_{\minus}\circ{p}_{n}\circ\varphi_{\minus}^{\minus{1}}$ is
            smooth. We have:
            \begin{align}
                \big(\varphi_{\minus}\circ{p}_{n}\circ
                    \varphi_{\minus}^{\minus{1}}\big)(X)
                    &=\big(\varphi_{\minus}\circ{p}_{n}\big)
                    \Big(\frac{2X+i(X^{2}-1)}{1+X^{2}}\Big)\\
                    &=\varphi_{\minus}\Big(
                        \frac{\big(2X+i(X^{2}-1)^{2})^{n}}{(1+X^{2}\big)^{n}}
                    \Big)\\
            \end{align}
            Thus, the $x$ and $y$ components will both be rational functions
            in $X$, and therefore $\varphi(x,y)$ will be a rational function
            in $X$, which is smooth. In a similarly manner,
            $\varphi_{+}\circ{p}_{n}\circ\varphi_{+}^{\minus{1}}$ is smooth.
            Thus for every point $\vector{x}\in\nsphere[1]$ there is a chart
            $(\mathcal{U},\varphi)$ such that $\vector{x}\in\mathcal{U}$ and
            $\varphi\circ{p}_{n}\circ\varphi^{\minus{1}}$ is smooth. Therefore,
            $p_{n}$ is smooth. For the antipodal map on $\nsphere$ we can use
            the orthographic projections. That is,
            $(\mathcal{U}_{j}^{+},\varphi_{+})$ is the chart where
            $\mathcal{U}_{j}^{+}$ is the $j^{th}$ upper hemisphere and
            $\varphi$ is the mapping that projects
            $\nsphere\subseteq\nspace[n+1]$ down to the $\nspace$ hyper plane
            obtained by fixed all but the $j^{th}$ coordinate, and similarly
            let $(\mathcal{U}^{\minus},\varphi_{\minus})$ be the opposite
            hemisphere. Given $\vector{s}\in\nsphere$, $\vector{s}$ is contained
            in one of these, and $\minus\vector{s}$ will be contained in the
            opposite. We thus need to show that, if $f$ is the antipodal map,
            then $\varphi_{\minus}\circ{f}\circ\varphi_{+}^{\minus{1}}$ is
            smooth. We have:
            \begin{align}
                \varphi_{\minus}\circ{f}\circ\varphi_{+}^{\minus{1}}(\vector{x})
                &=\varphi_{\minus}\circ{f}(x_{0},\dots,x_{j-1},
                    \sqrt{1-\norm{\vector{x}}^{2}},x_{j+1},\dots,x_{n})\\
                &=\varphi_{\minus}(\minus{x}_{0},\dots,\minus{x}_{j-1},
                    \minus\sqrt{1-\norm{\vector{x}}^{2}},\minus{x}_{j+1},
                    \dots,\minus{x}_{n}\big)\\
                &=(\minus{x}_{0},\dots,\minus{x}_{j-1},\minus{x}_{j+1},\dots,
                    \minus{x}_{n})\\
                &=\minus\vector{x}
            \end{align}
            Hence, this is a smooth mapping since multiplying by a constant is
            a smooth mapping from $\nspace$ to itself. Lastly, we show that the
            \textit{Hopf Fibration} is smooth. Again, sticking to orthographic
            projections, let $(\mathcal{U},\varphi)$ and $(\mathcal{V},\psi)$
            be orthographic charts in $\nsphere[3]$ and $\nsphere[2]$,
            respectively (Suppose both in the $x$ axis). Let
            $f:\nsphere[3]\rightarrow\nsphere[2]$ be the Hopf fibration. Then
            for $\vector{x}\in\nball[3]$, we have:
            \begin{equation}
                \psi\circ{f}\circ\varphi^{\minus{1}}(\vector{x})
                =\psi\circ{f}\circ(\sqrt{1-\norm{\vector{x}}^{2}},\vector{x})
                =\psi\circ{f}\Big(
                    \big(\sqrt{1-\norm{\vector{x}}^{2}}+ix_{1}\big),
                    \big(x_{2}+ix_{3}\big)
                \Big)
            \end{equation}
            Using the definition of $f$ and $\psi$ (which simply projection
            down to the $yz$ plane), we have:
            \begin{equation}
                \psi\circ{f}\circ\varphi^{\minus{1}}(\vector{x})=\big(
                    2\sqrt{1-\norm{\vector{x}}^{2}}x_{1}+2x_{2}x_{3},1-x_{1}^{2}
                \big)
            \end{equation}
            which is smooth in both components, and hence is a smooth function
            from an open subset of $\nspace[3]$ to an open subset of
            $\nsphere[2]$ and is therefore smooth. Similarly for the other
            hemispheres.
        \end{solution}
        \begin{problem}
            Show that if
            $f:\nspace[n+1]\setminus\{0\}\rightarrow\nspace[k+1]\setminus\{0\}$
            is a smooth homogeneous of degree $d\in\mathbb{Z}$, then the induced
            maps $\tilde{f}:\nrealproj\rightarrow\nrealproj[k]$ are well defined
            and smooth.
        \end{problem}
        \begin{solution}
            It is indeed well defined, for if $\vector{x},\vector{y}$ have the
            the same equivalence class: $[\vector{x}]=[\vector{y}]$, then there
            is a $\lambda\in\nspace[]\setminus\{0\}$ such that
            $\vector{y}=\lambda\vector{x}$. But then:
            \begin{equation}
                \tilde{f}([\vector{y}])=[f(\vector{y})]
                    =[f(\lambda\vector{x})]
                    =[\lambda^{d}f(\vector{x})]
                    =[f(\vector{x})]
                    =\tilde{f}([\vector{x}])
            \end{equation}
            and therefore elements of the same equivalence class map to the same
            points.
        \end{solution}
        \begin{problem}
            Lee 2-10.
        \end{problem}
        \begin{solution}
            Let $\manifold[M]{M}$ and $\manifold[N]{N}$ be manifolds,
            $F:M\rightarrow{N}$ continuous. $F^{*}$ is linear. For if
            $a,b\in\mathbb{R}$, $f,g\in\Ckspace{}{M}$, then:
            \begin{subequations}
                \begin{align}
                    F^{*}(af+bg)&=(af+bg)\circ{F}\\
                    &=\big((af)\circ{F}\big)+\big((bg)\circ{F}\big)\\
                    &=a(f\circ{F})+b(g\circ{F})\\
                    &=aF^{*}(f)+bF^{*}(g)
                \end{align}
            \end{subequations}
            If $F:M\rightarrow{N}$ is smooth, then for all
            $f\in\Ckspace{\infty}{N}$ we have $F^{*}(f)=f\circ{F}$, which is the
            composition of smooth function and is hence smooth. Therefore, we
            obtain $F^{*}[\Ckspace{\infty}{N}]\subseteq\Ckspace{\infty}{M}$.
            Conversely, suppose
            $F^{*}[\Ckspace{\infty}{N}]\subseteq\Ckspace{\infty}{M}$ and suppose
            that $F$ is not smooth. Then there is a point $p\in{M}$ such tha for
            every chart $(\mathcal{U},\varphi)\in\mathcal{A}_{M}$ that contains
            $p$ and for every chart $(\mathcal{V},\psi)\in\mathcal{A}_{N}$ that
            contains $F(p)$, the function $\psi\circ{F}\circ\varphi^{\minus{1}}$
            is not smooth. But since $N$ is a manifold, there is a precompact
            chart $(\mathcal{V},\psi)$ that contains $F(p)$. But $F$ is
            continuous, and hence $F^{\minus{1}}[\mathcal{V}]$ is an open subset
            of $M$ that contains $p$. Let
            $(\tilde{\mathcal{U}},\tilde{\varphi})$ be a precompact chart
            containing $p$, and let
            $\mathcal{U}=\tilde{\mathcal{U}}\cap{F}^{\minus{1}}[\mathcal{V}]$
            and $\varphi=\tilde{\varphi}|_{\mathcal{U}}$. Then, since
            $\mathcal{U}$ is the non-empty intersection of two open subsets,
            it will open and non-empty, and hence $(\mathcal{U},\varphi)$
            is a chart in $M$. Since $\mathcal{A}_{M}$ is maximal,
            $(\mathcal{U},\varphi)\in\mathcal{A}_{M}$. By hypothesis,
            $\psi\circ{F}\circ\varphi^{\minus{1}}$ is not smooth, and hence
            there a component $k$ such that composing with the projection map
            $\pi_{k}:\nspace[m]\rightarrow\nspace[]$ is not smooth. But we may
            use the bump function to extend $\pi_{k}\circ\psi$ to all of $N$,
            obtaining a smooth function in $\Ckspace{\infty}{N}$. But then by
            hypothesis, $\pi_{k}\circ\psi\circ{F}$ is smooth. And
            $\varphi^{\minus{1}}$ is smooth, so
            $\pi_{k}\circ\psi\circ{F}\circ\varphi^{\minus{1}}$ is smooth,
            a contradiction. Hence, $F$ is smooth. Lastly, show that if $F$ is a
            homeomorphism, then it is a diffeomorphism if and only if $F^{*}$ is
            an isomorphism. Since $F$ and $F^{\minus{1}}$ are continuous, they
            are both smooth if and only if
            $F^{*}[\Ckspace{\infty}{N}]\subseteq\Ckspace{\infty}{M}$ and
            ${F^{\minus{1}}}^{*}[\Ckspace{\infty}{M}]%
             \subseteq\Ckspace{\infty}{N}$. That is, they are smooth if and only
            if $F*|_{\Ckspace{\infty}{N}}\rightarrow\Ckspace{\infty}{M}$ is a
            bijective function. Since $F^{*}$ is linear, $F$ and $F^{\minus{1}}$
            are smooth if and only if $F^{*}$ is an isomorphism.
        \end{solution}
        \begin{problem}
            Show that paracompactness and subordinate partitions of unity are
            equivalent.
        \end{problem}
        \begin{solution}
            That paracompactness implies subordinate partitions of unity was
            proved both in class and in Lee. Going the other way, let
            $\topspace{X}$ be a topological space and suppose every
            open cover has a subordinate partition of unity. Let $\mathcal{O}$
            be an open cover and let $\mathcal{F}$ be a subordinate partition of
            unity. Let $\Delta$ be defined by:
            \begin{equation}
                \Delta=\{\mathcal{U}\;|\;\exists_{f\in\mathcal{F}}
                (\mathcal{U}=f^{\minus{1}}[\nspace[]\setminus\{0\}])\}
            \end{equation}
            Then, since all $f$ are continuous, all of the $\mathcal{U}$ are the
            pre-images of open sets and are therefore open. Since $\mathcal{F}$
            is a partiion of unity, for every point $x\in{X}$ there is an
            $f\in\mathcal{F}$ such that $f(x)\ne{0}$, hence $\Delta$ is an open
            subcover of $X$. Moreover, every $\mathcal{U}$ is contained in the
            support of some function, and since $\mathcal{F}$ is partition of
            unity, these form a locally finite cover. Therefore $\Delta$ is a
            locally finite refinement of $\mathcal{O}$. Thus, $\topspace{X}$
            is paracompact.
        \end{solution}
        \begin{problem}
            Show that disjoint closed subsets can be separated by smooth
            functions.
        \end{problem}
        \begin{solution}
            For manifolds are normal, and thus if $C_{1}$ and $C_{2}$ are
            disjoint closed sets then there are disjoint open sets
            $\mathcal{U}_{1}$, $\mathcal{U}_{2}$ containing $C_{1}$ and $C_{2}$.
            Furthermore, we can shrink $\mathcal{U}_{1}$ and $\mathcal{U}_{2}$
            so that they have disjoint closures. From theorem 2.29 we can find
            functions $f,g:M\rightarrow\mathbb{R}$ such that
            $f^{\minus{1}}[\{0\}]=C_{1}$ and $g^{\minus{1}}[\{1\}]=C_{2}$.
            But since $\closure{\mathcal{U}_{1}}$ is closed, we can find a bump
            function for $\closure{\mathcal{U}_{1}}$ supported on $\mathcal{U}$
            that evaluates to $1$ on $C_{1}$, and similarly for $C_{2}$.
            Denote these bump functions $h_{1},h_{2}$. Let
            $F:M\rightarrow\mathbb{R}$ be defined by:
            \begin{equation}
                F(p)=g_{1}(p)f(p)+g_{2}(p)g(p)
            \end{equation}
        \end{solution}
\end{document}