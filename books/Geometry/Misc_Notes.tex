%------------------------------------------------------------------------------%
\documentclass{article}                                                        %
%------------------------------Preamble----------------------------------------%
\makeatletter                                                                  %
    \def\input@path{{../../}}                                                  %
\makeatother                                                                   %
%---------------------------Packages----------------------------%
\usepackage{geometry}
\geometry{b5paper, margin=1.0in}
\usepackage[T1]{fontenc}
\usepackage{graphicx, float}            % Graphics/Images.
\usepackage{natbib}                     % For bibliographies.
\bibliographystyle{agsm}                % Bibliography style.
\usepackage[french, english]{babel}     % Language typesetting.
\usepackage[dvipsnames]{xcolor}         % Color names.
\usepackage{listings}                   % Verbatim-Like Tools.
\usepackage{mathtools, esint, mathrsfs} % amsmath and integrals.
\usepackage{amsthm, amsfonts, amssymb}  % Fonts and theorems.
\usepackage{tcolorbox}                  % Frames around theorems.
\usepackage{upgreek}                    % Non-Italic Greek.
\usepackage{fmtcount, etoolbox}         % For the \book{} command.
\usepackage[newparttoc]{titlesec}       % Formatting chapter, etc.
\usepackage{titletoc}                   % Allows \book in toc.
\usepackage[nottoc]{tocbibind}          % Bibliography in toc.
\usepackage[titles]{tocloft}            % ToC formatting.
\usepackage{pgfplots, tikz}             % Drawing/graphing tools.
\usepackage{imakeidx}                   % Used for index.
\usetikzlibrary{
    calc,                   % Calculating right angles and more.
    angles,                 % Drawing angles within triangles.
    arrows.meta,            % Latex and Stealth arrows.
    quotes,                 % Adding labels to angles.
    positioning,            % Relative positioning of nodes.
    decorations.markings,   % Adding arrows in the middle of a line.
    patterns,
    arrows
}                                       % Libraries for tikz.
\pgfplotsset{compat=1.9}                % Version of pgfplots.
\usepackage[font=scriptsize,
            labelformat=simple,
            labelsep=colon]{subcaption} % Subfigure captions.
\usepackage[font={scriptsize},
            hypcap=true,
            labelsep=colon]{caption}    % Figure captions.
\usepackage[pdftex,
            pdfauthor={Ryan Maguire},
            pdftitle={Mathematics and Physics},
            pdfsubject={Mathematics, Physics, Science},
            pdfkeywords={Mathematics, Physics, Computer Science, Biology},
            pdfproducer={LaTeX},
            pdfcreator={pdflatex}]{hyperref}
\hypersetup{
    colorlinks=true,
    linkcolor=blue,
    filecolor=magenta,
    urlcolor=Cerulean,
    citecolor=SkyBlue
}                           % Colors for hyperref.
\usepackage[toc,acronym,nogroupskip,nopostdot]{glossaries}
\usepackage{glossary-mcols}
%------------------------Theorem Styles-------------------------%
\theoremstyle{plain}
\newtheorem{theorem}{Theorem}[section]

% Define theorem style for default spacing and normal font.
\newtheoremstyle{normal}
    {\topsep}               % Amount of space above the theorem.
    {\topsep}               % Amount of space below the theorem.
    {}                      % Font used for body of theorem.
    {}                      % Measure of space to indent.
    {\bfseries}             % Font of the header of the theorem.
    {}                      % Punctuation between head and body.
    {.5em}                  % Space after theorem head.
    {}

% Italic header environment.
\newtheoremstyle{thmit}{\topsep}{\topsep}{}{}{\itshape}{}{0.5em}{}

% Define environments with italic headers.
\theoremstyle{thmit}
\newtheorem*{solution}{Solution}

% Define default environments.
\theoremstyle{normal}
\newtheorem{example}{Example}[section]
\newtheorem{definition}{Definition}[section]
\newtheorem{problem}{Problem}[section]

% Define framed environment.
\tcbuselibrary{most}
\newtcbtheorem[use counter*=theorem]{ftheorem}{Theorem}{%
    before=\par\vspace{2ex},
    boxsep=0.5\topsep,
    after=\par\vspace{2ex},
    colback=green!5,
    colframe=green!35!black,
    fonttitle=\bfseries\upshape%
}{thm}

\newtcbtheorem[auto counter, number within=section]{faxiom}{Axiom}{%
    before=\par\vspace{2ex},
    boxsep=0.5\topsep,
    after=\par\vspace{2ex},
    colback=Apricot!5,
    colframe=Apricot!35!black,
    fonttitle=\bfseries\upshape%
}{ax}

\newtcbtheorem[use counter*=definition]{fdefinition}{Definition}{%
    before=\par\vspace{2ex},
    boxsep=0.5\topsep,
    after=\par\vspace{2ex},
    colback=blue!5!white,
    colframe=blue!75!black,
    fonttitle=\bfseries\upshape%
}{def}

\newtcbtheorem[use counter*=example]{fexample}{Example}{%
    before=\par\vspace{2ex},
    boxsep=0.5\topsep,
    after=\par\vspace{2ex},
    colback=red!5!white,
    colframe=red!75!black,
    fonttitle=\bfseries\upshape%
}{ex}

\newtcbtheorem[auto counter, number within=section]{fnotation}{Notation}{%
    before=\par\vspace{2ex},
    boxsep=0.5\topsep,
    after=\par\vspace{2ex},
    colback=SeaGreen!5!white,
    colframe=SeaGreen!75!black,
    fonttitle=\bfseries\upshape%
}{not}

\newtcbtheorem[use counter*=remark]{fremark}{Remark}{%
    fonttitle=\bfseries\upshape,
    colback=Goldenrod!5!white,
    colframe=Goldenrod!75!black}{ex}

\newenvironment{bproof}{\textit{Proof.}}{\hfill$\square$}
\tcolorboxenvironment{bproof}{%
    blanker,
    breakable,
    left=3mm,
    before skip=5pt,
    after skip=10pt,
    borderline west={0.6mm}{0pt}{green!80!black}
}

\AtEndEnvironment{lexample}{$\hfill\textcolor{red}{\blacksquare}$}
\newtcbtheorem[use counter*=example]{lexample}{Example}{%
    empty,
    title={Example~\theexample},
    boxed title style={%
        empty,
        size=minimal,
        toprule=2pt,
        top=0.5\topsep,
    },
    coltitle=red,
    fonttitle=\bfseries,
    parbox=false,
    boxsep=0pt,
    before=\par\vspace{2ex},
    left=0pt,
    right=0pt,
    top=3ex,
    bottom=1ex,
    before=\par\vspace{2ex},
    after=\par\vspace{2ex},
    breakable,
    pad at break*=0mm,
    vfill before first,
    overlay unbroken={%
        \draw[red, line width=2pt]
            ([yshift=-1.2ex]title.south-|frame.west) to
            ([yshift=-1.2ex]title.south-|frame.east);
        },
    overlay first={%
        \draw[red, line width=2pt]
            ([yshift=-1.2ex]title.south-|frame.west) to
            ([yshift=-1.2ex]title.south-|frame.east);
    },
}{ex}

\AtEndEnvironment{ldefinition}{$\hfill\textcolor{Blue}{\blacksquare}$}
\newtcbtheorem[use counter*=definition]{ldefinition}{Definition}{%
    empty,
    title={Definition~\thedefinition:~{#1}},
    boxed title style={%
        empty,
        size=minimal,
        toprule=2pt,
        top=0.5\topsep,
    },
    coltitle=Blue,
    fonttitle=\bfseries,
    parbox=false,
    boxsep=0pt,
    before=\par\vspace{2ex},
    left=0pt,
    right=0pt,
    top=3ex,
    bottom=0pt,
    before=\par\vspace{2ex},
    after=\par\vspace{1ex},
    breakable,
    pad at break*=0mm,
    vfill before first,
    overlay unbroken={%
        \draw[Blue, line width=2pt]
            ([yshift=-1.2ex]title.south-|frame.west) to
            ([yshift=-1.2ex]title.south-|frame.east);
        },
    overlay first={%
        \draw[Blue, line width=2pt]
            ([yshift=-1.2ex]title.south-|frame.west) to
            ([yshift=-1.2ex]title.south-|frame.east);
    },
}{def}

\AtEndEnvironment{ltheorem}{$\hfill\textcolor{Green}{\blacksquare}$}
\newtcbtheorem[use counter*=theorem]{ltheorem}{Theorem}{%
    empty,
    title={Theorem~\thetheorem:~{#1}},
    boxed title style={%
        empty,
        size=minimal,
        toprule=2pt,
        top=0.5\topsep,
    },
    coltitle=Green,
    fonttitle=\bfseries,
    parbox=false,
    boxsep=0pt,
    before=\par\vspace{2ex},
    left=0pt,
    right=0pt,
    top=3ex,
    bottom=-1.5ex,
    breakable,
    pad at break*=0mm,
    vfill before first,
    overlay unbroken={%
        \draw[Green, line width=2pt]
            ([yshift=-1.2ex]title.south-|frame.west) to
            ([yshift=-1.2ex]title.south-|frame.east);},
    overlay first={%
        \draw[Green, line width=2pt]
            ([yshift=-1.2ex]title.south-|frame.west) to
            ([yshift=-1.2ex]title.south-|frame.east);
    }
}{thm}

%--------------------Declared Math Operators--------------------%
\DeclareMathOperator{\adjoint}{adj}         % Adjoint.
\DeclareMathOperator{\Card}{Card}           % Cardinality.
\DeclareMathOperator{\curl}{curl}           % Curl.
\DeclareMathOperator{\diam}{diam}           % Diameter.
\DeclareMathOperator{\dist}{dist}           % Distance.
\DeclareMathOperator{\Div}{div}             % Divergence.
\DeclareMathOperator{\Erf}{Erf}             % Error Function.
\DeclareMathOperator{\Erfc}{Erfc}           % Complementary Error Function.
\DeclareMathOperator{\Ext}{Ext}             % Exterior.
\DeclareMathOperator{\GCD}{GCD}             % Greatest common denominator.
\DeclareMathOperator{\grad}{grad}           % Gradient
\DeclareMathOperator{\Ima}{Im}              % Image.
\DeclareMathOperator{\Int}{Int}             % Interior.
\DeclareMathOperator{\LC}{LC}               % Leading coefficient.
\DeclareMathOperator{\LCM}{LCM}             % Least common multiple.
\DeclareMathOperator{\LM}{LM}               % Leading monomial.
\DeclareMathOperator{\LT}{LT}               % Leading term.
\DeclareMathOperator{\Mod}{mod}             % Modulus.
\DeclareMathOperator{\Mon}{Mon}             % Monomial.
\DeclareMathOperator{\multideg}{mutlideg}   % Multi-Degree (Graphs).
\DeclareMathOperator{\nul}{nul}             % Null space of operator.
\DeclareMathOperator{\Ord}{Ord}             % Ordinal of ordered set.
\DeclareMathOperator{\Prin}{Prin}           % Principal value.
\DeclareMathOperator{\proj}{proj}           % Projection.
\DeclareMathOperator{\Refl}{Refl}           % Reflection operator.
\DeclareMathOperator{\rk}{rk}               % Rank of operator.
\DeclareMathOperator{\sgn}{sgn}             % Sign of a number.
\DeclareMathOperator{\sinc}{sinc}           % Sinc function.
\DeclareMathOperator{\Span}{Span}           % Span of a set.
\DeclareMathOperator{\Spec}{Spec}           % Spectrum.
\DeclareMathOperator{\supp}{supp}           % Support
\DeclareMathOperator{\Tr}{Tr}               % Trace of matrix.
%--------------------Declared Math Symbols--------------------%
\DeclareMathSymbol{\minus}{\mathbin}{AMSa}{"39} % Unary minus sign.
%------------------------New Commands---------------------------%
\DeclarePairedDelimiter\norm{\lVert}{\rVert}
\DeclarePairedDelimiter\ceil{\lceil}{\rceil}
\DeclarePairedDelimiter\floor{\lfloor}{\rfloor}
\newcommand*\diff{\mathop{}\!\mathrm{d}}
\newcommand*\Diff[1]{\mathop{}\!\mathrm{d^#1}}
\renewcommand*{\glstextformat}[1]{\textcolor{RoyalBlue}{#1}}
\renewcommand{\glsnamefont}[1]{\textbf{#1}}
\renewcommand\labelitemii{$\circ$}
\renewcommand\thesubfigure{%
    \arabic{chapter}.\arabic{figure}.\arabic{subfigure}}
\addto\captionsenglish{\renewcommand{\figurename}{Fig.}}
\numberwithin{equation}{section}

\renewcommand{\vector}[1]{\boldsymbol{\mathrm{#1}}}

\newcommand{\uvector}[1]{\boldsymbol{\hat{\mathrm{#1}}}}
\newcommand{\topspace}[2][]{(#2,\tau_{#1})}
\newcommand{\measurespace}[2][]{(#2,\varSigma_{#1},\mu_{#1})}
\newcommand{\measurablespace}[2][]{(#2,\varSigma_{#1})}
\newcommand{\manifold}[2][]{(#2,\tau_{#1},\mathcal{A}_{#1})}
\newcommand{\tanspace}[2]{T_{#1}{#2}}
\newcommand{\cotanspace}[2]{T_{#1}^{*}{#2}}
\newcommand{\Ckspace}[3][\mathbb{R}]{C^{#2}(#3,#1)}
\newcommand{\funcspace}[2][\mathbb{R}]{\mathcal{F}(#2,#1)}
\newcommand{\smoothvecf}[1]{\mathfrak{X}(#1)}
\newcommand{\smoothonef}[1]{\mathfrak{X}^{*}(#1)}
\newcommand{\bracket}[2]{[#1,#2]}

%------------------------Book Command---------------------------%
\makeatletter
\renewcommand\@pnumwidth{1cm}
\newcounter{book}
\renewcommand\thebook{\@Roman\c@book}
\newcommand\book{%
    \if@openright
        \cleardoublepage
    \else
        \clearpage
    \fi
    \thispagestyle{plain}%
    \if@twocolumn
        \onecolumn
        \@tempswatrue
    \else
        \@tempswafalse
    \fi
    \null\vfil
    \secdef\@book\@sbook
}
\def\@book[#1]#2{%
    \refstepcounter{book}
    \addcontentsline{toc}{book}{\bookname\ \thebook:\hspace{1em}#1}
    \markboth{}{}
    {\centering
     \interlinepenalty\@M
     \normalfont
     \huge\bfseries\bookname\nobreakspace\thebook
     \par
     \vskip 20\p@
     \Huge\bfseries#2\par}%
    \@endbook}
\def\@sbook#1{%
    {\centering
     \interlinepenalty \@M
     \normalfont
     \Huge\bfseries#1\par}%
    \@endbook}
\def\@endbook{
    \vfil\newpage
        \if@twoside
            \if@openright
                \null
                \thispagestyle{empty}%
                \newpage
            \fi
        \fi
        \if@tempswa
            \twocolumn
        \fi
}
\newcommand*\l@book[2]{%
    \ifnum\c@tocdepth >-3\relax
        \addpenalty{-\@highpenalty}%
        \addvspace{2.25em\@plus\p@}%
        \setlength\@tempdima{3em}%
        \begingroup
            \parindent\z@\rightskip\@pnumwidth
            \parfillskip -\@pnumwidth
            {
                \leavevmode
                \Large\bfseries#1\hfill\hb@xt@\@pnumwidth{\hss#2}
            }
            \par
            \nobreak
            \global\@nobreaktrue
            \everypar{\global\@nobreakfalse\everypar{}}%
        \endgroup
    \fi}
\newcommand\bookname{Book}
\renewcommand{\thebook}{\texorpdfstring{\Numberstring{book}}{book}}
\providecommand*{\toclevel@book}{-2}
\makeatother
\titleformat{\part}[display]
    {\Large\bfseries}
    {\partname\nobreakspace\thepart}
    {0mm}
    {\Huge\bfseries}
\titlecontents{part}[0pt]
    {\large\bfseries}
    {\partname\ \thecontentslabel: \quad}
    {}
    {\hfill\contentspage}
\titlecontents{chapter}[0pt]
    {\bfseries}
    {\chaptername\ \thecontentslabel:\quad}
    {}
    {\hfill\contentspage}
\newglossarystyle{longpara}{%
    \setglossarystyle{long}%
    \renewenvironment{theglossary}{%
        \begin{longtable}[l]{{p{0.25\hsize}p{0.65\hsize}}}
    }{\end{longtable}}%
    \renewcommand{\glossentry}[2]{%
        \glstarget{##1}{\glossentryname{##1}}%
        &\glossentrydesc{##1}{~##2.}
        \tabularnewline%
        \tabularnewline
    }%
}
\newglossary[not-glg]{notation}{not-gls}{not-glo}{Notation}
\newcommand*{\newnotation}[4][]{%
    \newglossaryentry{#2}{type=notation, name={\textbf{#3}, },
                          text={#4}, description={#4},#1}%
}
%--------------------------LENGTHS------------------------------%
% Spacings for the Table of Contents.
\addtolength{\cftsecnumwidth}{1ex}
\addtolength{\cftsubsecindent}{1ex}
\addtolength{\cftsubsecnumwidth}{1ex}
\addtolength{\cftfignumwidth}{1ex}
\addtolength{\cfttabnumwidth}{1ex}

% Indent and paragraph spacing.
\setlength{\parindent}{0em}
\setlength{\parskip}{0em}                                                           %
%----------------------------Main Document-------------------------------------%
\begin{document}
    \title{Misc Notes from Semi-Remiannian Geometry}
    \author{Ryan Maguire}
    \date{\vspace{-5ex}}
    \maketitle
    \tableofcontents
    \listoffigures
    \section{Locally Euclidean Spaces}
        We wish to speak about topological spaces that are sufficiently well
        behaved in the sense that it allows one to do calculus. The first
        naive approach is to consider such spaces that can be locally
        approximated by Euclidean space.
        \begin{fdefinition}{Chart}{Chart}
            A chart of dimension $n\in\mathbb{N}$ in a topological space
            $(X,\tau)$, denoted $(\mathcal{U},\varphi)$, is an open set
            $\mathcal{U}\in\tau$ and a function
            $\varphi:\mathcal{U}\rightarrow\mathbb{R}^{n}$ such that
            $\varphi$ is injective, continuous, and an open mapping with
            respect to the standard topology on $\mathbb{R}^{n}$.
        \end{fdefinition}
        Some authors simply denote a chart by the function $\varphi$. Note
        that since the image of $\varphi$ lies in $\mathbb{R}^{n}$, if we
        compose with one of the projection mappings $\pi_{k}$ we obtain a
        function $\pi_{k}\circ\phi:\mathcal{U}\rightarrow\mathbb{R}$. These
        are called the coordinate functions of the chart
        $(\mathcal{U},\phi)$ and are often denoted $x^{k}=\pi_{k}\circ\phi$.
        Note that we may then write the image of $p\in\mathcal{U}$ as:
        \begin{equation}
            \phi(p)=\big(x^{1}(p),\dots,x^{n}(p)\big)
        \end{equation}
        and such notation is often useful. Avoiding the \textit{dot dot dot}
        notation, we can recall that $\mathbf{x}\in\mathbb{R}^{n}$ is a
        function $\mathbf{x}:\mathbb{Z}_{n}\rightarrow\mathbb{R}$. Thus if
        $p\in\mathcal{U}$ and $\mathbf{x}=\phi(p)$, we can write this
        explicitly by:
        \begin{equation}
            \big(\phi(p)\big)(k)=\mathbf{x}(k)=x_{k}=x^{k}(p)
        \end{equation}
        Thus distinguishing the difference between the notation $x^{k}$ and
        $x_{k}$. Here, $x_{k}$ is simply a real number, $x^{k}$ is a function
        $x^{k}:\mathcal{U}\rightarrow\mathbb{R}$, and $\mathbf{x}(k)$ is the
        image of $k\in\mathbb{Z}_{n}$ under $\mathbf{x}$. In other words, it is
        the $k^{th}$ component of $\mathbf{x}$.
        \begin{example}
            If $X=\mathbb{R}^{n}$ and $\tau$ is the standard topology, then
            for any open subset $\mathcal{U}\subseteq\mathbb{R}^{n}$ the
            pair $(\mathcal{U},\textrm{id}_{\mathbb{R}^{n}}|_{\mathcal{U}})$
            is a chart, where $\textrm{id}_{\mathbb{R}^{n}}|_{\mathcal{U}}$
            denotes the restriction of the identity map to $\mathcal{U}$.
        \end{example}
        \begin{example}
            If $X=S^{2}$, and if $\tau$ is the subspace topology inherited
            from $\mathbb{R}^{3}$, then we can form a chart around the
            south pole $(0,0,\minus{1})$ as follows. Let
            $\mathcal{U}=S^{2}\setminus\{(0,0,1)\}$. Since $\mathbb{R}^{3}$
            is Hausdorff, and since $S^{2}$ is a subspace of
            $\mathbb{R}^{3}$, it then follows that $S^{2}$ is Hausdorff. But
            if $S^{2}$ is Hausdorff, then the set $\{(0,0,1)\}$ is closed
            since finite sets are closed in a Hausdorff space. But then
            $X\setminus\{(0,0,1)\}$ is the complement of a closed set, and
            thus by definition is open. Hence, $\mathcal{U}$ is an open
            subset of $S^{2}$. We can construct our open mapping
            $\phi:\mathcal{U}\rightarrow\mathbb{R}^{2}$ by using the
            \textit{stereographic projection}:
            \begin{equation}
                \phi(x,y,z)=\Big(\frac{x}{1-z},\frac{y}{1-z}\Big)
                \quad\quad
                \forall_{(x,y,z)\in{S}^{2}\setminus\{(0,0,1)\}}
            \end{equation}
            This function is continuous and bijective, and has a continuous
            inverse:
            \begin{equation}
                \phi^{\minus{1}}(X,Y)=
                \Big(\frac{2X}{1+X^{2}+Y^{2}},\frac{2Y}{1+X^{2}+Y^{2}},
                    \frac{\minus{1}+X^{2}+Y^{2}}{1+X^{2}+Y^{2}}\Big)
            \end{equation}
            Therefore $\phi$ is a homeomorphism from $\mathcal{U}$ to
            $\mathbb{R}^{2}$, and is thus necessarily a continuous injective
            open mapping. This chart $(\mathcal{U},\phi)$ is actually a
            chart for any point that is not the north pole. If we flip this
            around and do the stereographic projection about the south pole,
            we'll obtain a chart that contains the north pole. Thus the
            sphere can be covered by two charts.
        \end{example}
        \begin{example}
            Another example of a chart on the sphere is the
            \textit{orthographic projection}. The stereographic projection
            is obtain by placing an observer at one of the poles, and
            drawing a straight line from the observer to a point on the
            sphere. This point then gets mapped to $\mathbb{R}^{2}$ by
            finding where this line intersects the $xy$ plane. The
            orthographic is obtained in a similar manner by placing the
            observer \textit{at infinity}. There is now the problem that
            such a straight line intersects the sphere twice: Once on the
            top and once on the bottom. That is, if $(x,y,z)$ lies on this
            line, then so does $(x,y,\minus{z})$. To form our chart we must
            consistently choose either the top or the bottom. Let
            $\mathcal{U}$ be defined as follows:
            \begin{equation}
                \mathcal{U}=\{(x,y,z)\in{S}^{2}\;|\;z>0\}
            \end{equation}
            Define the orthographic projection by:
            \begin{equation}
                \phi(x,y,z)=(x,y)
            \end{equation}
            This is a homeomorphism between $\mathcal{U}$ and the open unit
            disc $B_{1}(0)$. The inverse function is:
            \begin{equation}
                \phi^{\minus{1}}(X,Y)=\big(X,Y,\sqrt{1-X^{2}-Y^{2}}\big)
            \end{equation}
            where we unambiguously take the positive square root. This is
            a chart for any point in the upper hemisphere (the equator is
            not included). To completely cover the sphere using orthographic
            projections requires 6 charts ($\pm{x}$, $\pm{y}$, $\pm{z}$,
            our open set $\mathcal{U}$ corresponding to $+z$).
        \end{example}
        Equivalently, one could say that $\phi$ is a homeomorphism between
        $\mathcal{U}$ and it's image under $\phi$. That is,
        $\phi(\mathcal{U})\subseteq\mathbb{R}^{n}$ is homeomorphic to
        $\mathcal{U}$ and $\phi:\mathcal{U}\rightarrow\phi(\mathcal{U})$ is
        a homeomorphism.
        \begin{theorem}
            If $(X,\tau)$ is a topological space, if $n\in\mathbb{N}$, if
            $(\mathcal{U},\phi)$ is a chart in $(X,\tau)$, and if
            $\mathcal{V}=\phi(\mathcal{U})$ is the image of $\mathcal{U}$
            under $\phi$, then $\phi:\mathcal{U}\rightarrow\mathcal{V}$ is
            a homeomorphism.
        \end{theorem}
        \begin{proof}
            For since $(\mathcal{U},\phi)$ is a chart, $\phi$ is injective,
            continuous, and an open mapping. But since $\mathcal{V}$ is the
            image of $\mathcal{U}$ under $\phi$, the restriction
            $\phi:\mathcal{U}\rightarrow\mathcal{V}$ is therefore
            surjective. But then $\phi:\mathcal{U}\rightarrow\mathcal{V}$ is
            both injective and surjective, and is therefore bijective. But
            since $\phi$ is an open mapping, $\phi^{\minus{1}}$ is
            continuous. But then $\phi$ is a continuous bijection with a
            continuous inverse and is thus a homeomorphism.
        \end{proof}
        \begin{fdefinition}{Locally Euclidean Space}
                           {Locally Euclidean Space}
            A locally Euclidean space is a topological space $(X,\tau)$
            such that for all $x\in{X}$ there exists an $n\in\mathbb{N}$ and
            an $n$ dimensional chart $(\mathcal{U},\phi)$ of $(X,\tau)$ such
            that $x\in\mathcal{U}$.
        \end{fdefinition}
        This is the most general setting for one to define a manifold. The
        dimension of a locally Euclidean space need not be constant, for we
        can consider the disjoint union of a sphere with a line. Thus there
        will be 2 dimensional points and 1 dimensional points. Dimension is,
        however, a locally constant property. In particular, for any fixed
        point $x\in{X}$ there is only one $n\in\mathbb{N}$ such that
        $x$ is locally like $\mathbb{R}^{n}$. If in addition the space
        $(X,\tau)$ is connected, then there is only one unambiguous number
        $n\in\mathbb{N}$ such that every point $x\in{X}$ is locally like
        $\mathbb{R}^{n}$. This allows us to define dimension.
        \begin{theorem}
            If $X$ and $Y$ are sets, if $\mathcal{U}\subseteq{X}$, if
            $f:X\rightarrow{Y}$ is a function, if
            $f|_{\mathcal{U}}:\mathcal{U}\rightarrow{f}(\mathcal{U})$ is the
            restriction of $f$ to $\mathcal{U}$, and if
            $\mathcal{V}\subseteq{f}(\mathcal{U})$, then:
            \begin{equation}
                f|_{\mathcal{U}}^{\minus{1}}(\mathcal{V})
                =\mathcal{U}\cap{f}^{\minus{1}}(\mathcal{V})
            \end{equation}
        \end{theorem}
        \begin{proof}
            For if $x\in{f}|_{\mathcal{U}}^{\minus{1}}(\mathcal{V})$, then
            there is an $x\in\mathcal{U}$ such that
            $f|_{\mathcal{U}}(x)\in\mathcal{V}$. But for all
            $x\in\mathcal{U}$ it is true that $f|_{\mathcal{U}}(x)=f(x)$,
            and thus $f(x)\in\mathcal{V}$. But if $f(x)\in\mathcal{V}$, then
            $x\in{f}^{\minus{1}}(\mathcal{V})$. Thus $x\in\mathcal{U}$ and
            $x\in{f}^{\minus{1}}(\mathcal{V})$, and therefore
            $x\in\mathcal{U}\cap{f}^{\minus{1}}(\mathcal{V})$. So we obtain:
            \begin{equation}
                f|_{\mathcal{U}}^{\minus{1}}(\mathcal{V})
                \subseteq\mathcal{U}\cap{f}^{\minus{1}}(\mathcal{V})
            \end{equation}
            In the other direction, if
            $x\in\mathcal{U}\cap{f}^{\minus{1}}(\mathcal{V})$, then
            $x\in\mathcal{U}$ and $f(x)\in\mathcal{V}$. But if
            $x\in\mathcal{U}$, then $f(x)=f|_{\mathcal{U}}(x)$ and therefore
            $f|_{\mathcal{U}}(x)\in\mathcal{V}$. But then
            $x\in{f}|_{\mathcal{U}}^{\minus{1}}(\mathcal{V})$. That is:
            \begin{equation}
                \mathcal{U}\cap{f}^{\minus{1}}(\mathcal{V})
                \subseteq{f}|_{\mathcal{U}}^{\minus{1}}(\mathcal{V})
            \end{equation}
            From the definition of equality, we are done.
        \end{proof}
        \begin{theorem}
            If $(X,\tau_{X})$ and $(Y,\tau_{Y})$ are topological spaces,
            if $f:X\rightarrow{Y}$ is a homeomorphism, and if
            $\mathcal{U}\subseteq{X}$, then the restriction
            $f|_{\mathcal{U}}:\mathcal{U}\rightarrow{f}(\mathcal{U})$ is
            a homeomorphism between $(\mathcal{U},\tau_{X}|_{\mathcal{U}})$
            and $(f(\mathcal{U}),\tau_{Y}|_{f(\mathcal{U})})$, where
            $\tau_{X}|_{\mathcal{U}}$ and $\tau_{Y}|_{f(\mathcal{U})}$ are
            the subspace topologies.
        \end{theorem}
        \begin{proof}
            Since $f:X\rightarrow{Y}$ is a homeomorphism it is therefore
            bijective. But then
            $f|_{\mathcal{U}}:\mathcal{U}\rightarrow{f}(\mathcal{U})$ is
            bijective. For if not, then it is either not injective or not
            surjective. But if it is not injective then there exists points
            $x_{1},x_{2}\in\mathcal{U}$ such that $x_{1}\ne{x}_{2}$ and
            $f|_{\mathcal{U}}(x_{1})=f|_{\mathcal{U}}(x_{2})$. But since
            $f|_{\mathcal{U}}$ is the restriction of $f$ to $\mathcal{U}$,
            for all $x\in\mathcal{U}$ it is true that
            $f(x)=f|_{\mathcal{U}}(x)$. But $x_{1},x_{2}\in\mathcal{U}$ and
            thus $f(x_{1})=f|_{\mathcal{U}}(x_{1})$ and similary for
            $x_{2}$. Then by the transitivity of equality,
            $f(x_{1})=f(x_{2})$, a contradiction since $f$ is injective.
            Therefore $f|_{\mathcal{U}}$ is injective. It is surjective by
            the definition of the image of $\mathcal{U}$ under $f$. Thus,
            $f|_{\mathcal{U}}:\mathcal{U}\rightarrow{f}(\mathcal{U})$ is
            bijective. If $\mathcal{V}\subseteq{f}(\mathcal{U})$ is open in
            the subspace topology then there is an open set
            $\mathcal{O}\in\tau_{Y}$ such that
            $\mathcal{V}=\mathcal{O}\cap{f}(\mathcal{U})$. But then:
            \begin{subequations}
                \begin{align}
                    f|_{\mathcal{U}}^{\minus{1}}(\mathcal{V})
                    &=\mathcal{U}\cap{f}^{\minus{1}}(\mathcal{V})\\
                    &=\mathcal{U}\cap
                        f^{\minus{1}}\big(
                            f(\mathcal{U})\cap\mathcal{O}
                        \big)\\
                    &=\mathcal{U}\cap\Big(
                        f^{\minus{1}}\big(f(\mathcal{U})\big)\cap
                        f^{\minus{1}}(\mathcal{O})
                    \Big)\\
                    &=\mathcal{U}\cap\big(
                        \mathcal{U}\cap{f}^{\minus{1}}(\mathcal{O})\big)\\
                    &=\big(\mathcal{U}\cap\mathcal{U})\cap
                        f^{\minus{1}}(\mathcal{O})\\
                    &=\mathcal{U}\cap{f}^{\minus{1}}(\mathcal{O})
                \end{align}
            \end{subequations}
            But $f$ is continuous and therefore $f^{\minus{1}}(\mathcal{O})$
            is an open subset of $X$. But then
            $\mathcal{U}\cap{f}^{\minus{1}}(\mathcal{O})$ is open in the
            subspace topology $\tau_{X}|_{\mathcal{U}}$, and therefore
            $f|_{\mathcal{U}}$ is continuous. In a similar manner,
            $f|_{\mathcal{U}}^{\minus{1}}$ is continuous and therefore
            $f|_{\mathcal{U}}$ is a homeomorphism.
        \end{proof}
        While we were very liberal in our definition of a locally Euclidean
        space, allowing the set $\mathcal{V}\subseteq\mathbb{R}^{n}$ to be
        any open set, we need not be. A topological space is locally
        Euclidean if and only if every point has an open neighborhood that
        is homeomorphic to all of $\mathbb{R}^{n}$.
        \begin{theorem}
            If $(X,\tau)$ is a topological space, $x\in{X}$, if
            $n\in\mathbb{N}$, and if $(\mathcal{U},\phi)$ is an $n$
            dimensional chart in $(X,\tau)$ such that $x\in\mathcal{U}$,
            then there is an open subset $\mathcal{V}\in\tau$ such that
            $x\in\mathcal{V}$ and $\mathcal{V}$ is homeomorphic to
            $\mathbb{R}^{n}$.
        \end{theorem}
        \begin{proof}
                For since $\phi:\mathcal{U}\rightarrow\mathbb{R}^{n}$ is an open
                mapping, $\phi(\mathcal{U})$ is an open subset of
                $\mathbb{R}^{n}$. But if $x\in\mathcal{U}$, then
                $\phi(x)\in\phi(\mathcal{U})$. But if $\mathcal{U}$ is open and
                if $\phi(x)\in\phi(\mathcal{U})$, then there is an $r>0$ such
                that the open ball of radius $r$ about $\phi(x)$ is contained in
                $\phi(\mathcal{U})$. That is,
                $B_{r}(\phi(x),\mathbb{R}^{n})\subseteq\phi(\mathcal{U})$. But
                since $B_{r}(\phi(x),\mathbb{R}^{n})\subseteq\phi(\mathcal{U})$,
                we have:
                \begin{equation}
                    \phi^{\minus{1}}\big(B_{r}(\phi(x),\mathbb{R}^{n})\big)
                    \subseteq\mathcal{U}
                \end{equation}
                But then since $B_{r}(\phi(x),\mathbb{R}^{n})$ is contained in
                the image of $\mathcal{U}$ under $\phi$, we have:
                \begin{equation}
                    \phi\Big(
                        \phi^{\minus{1}}\big(B_{r}(\phi(x_,\mathbb{R}^{n})\big)
                    \Big)
                    =B_{r}(\phi(x),\mathbb{R}^{n})
                \end{equation}
                But then the restriction of $\phi$ to
                $\phi^{\minus{1}}\big(B_{r}(\phi(x),\mathbb{R}^{n})\big)$ is a
                homeomorphism from
                $\phi^{\minus{1}}\big(B_{r}(\phi(x),\mathbb{R}^{n})\big)$ to
                $B_{r}(\phi(x),\mathbb{R}^{n})$. But
                $B_{r}(\phi(x),\mathbb{R}^{n})$ is homeomorphic to
                $\mathbb{R}^{n}$, and since homeomorphic is an equivalence
                relation, we have that
                $\phi^{\minus{1}}\big(B_{r}(\phi(x),\mathbb{R}^{n})\big)$ is
                homeomorphic to $\mathbb{R}^{n}$. But then
                $\phi^{\minus{1}}\big(B_{r}(\phi(x),\mathbb{R}^{n})\big)$ is an
                open subset of $X$ that contains $x$ and is homeomorphic to
                $\mathbb{R}^{n}$, completing the proof.
        \end{proof}
        \begin{theorem}
                If $(X,\tau)$ is a topological space, if $Y$ is a set with at
                least two points, and if for all locally constant functions
                $f:X\rightarrow{Y}$ it is true that $f$ is constant, then
                $(X,\tau)$ is connected.
        \end{theorem}
        \begin{proof}
                For suppose not. Then there are non-empty disjoint open subsets
                $\mathcal{U},\mathcal{V}\in\tau$ that partition $X$. That is,
                $\mathcal{U}\cap\mathcal{V}=\emptyset$ and
                $\mathcal{U}\cup\mathcal{V}=X$. Since $Y$ has at least two
                points, there are distinct $y_{1},y_{2}\in{Y}$. Let
                $f:X\rightarrow{Y}$ be defined as follows:
                \begin{equation}
                    f(x)=
                    \begin{cases}
                        y_{1},&x\in\mathcal{U}\\
                        y_{2},&x\in\mathcal{V}
                    \end{cases}
                \end{equation}
                Then $f$ is locally constant. That is, since $\mathcal{U}$ and
                $\mathcal{V}$ partition $X$, for all $z\in{X}$ it is true that
                either $z\in\mathcal{U}$ or $z\in\mathcal{V}$, but not both.
                Thus suppose there is a $z\in{X}$ such that for all
                $\mathcal{O}\in\tau$ such that $z\in\mathcal{O}$ it is not true
                that $f$ is constant on $\mathcal{O}$. But either
                $z\in\mathcal{U}$ or $z\in\mathcal{V}$. If $z\in\mathcal{U}$,
                then for all $x\in\mathcal{U}$, $f(x)=y_{1}$ and thus $f$ is
                locally constant on $\mathcal{U}$. Similarly if
                $z\in\mathcal{V}$, and thus $f$ is locally constant. It is not
                constant since $\mathcal{U}$ and $\mathcal{V}$ are non-empty,
                and thus $f(X)=\{y_{1},y_{2}\}$, which is not a singleton. A
                contradiction since by hypothesis for all locally constant
                functions $f:X\rightarrow{Y}$ it is true that $f$ is constant.
                Therefore, $(X,\tau)$ is connected.
        \end{proof}
        \begin{theorem}
                If $(X,\tau)$ is a connected topological space, if $Y$ is a set,
                and if $f:X\rightarrow{Y}$ is locally constant, then $f$ is
                constant.
        \end{theorem}
        \begin{proof}
                For suppose not. Then there is a function $f:X\rightarrow{Y}$
                that is locally constant but not constant. But if $f$ is not
                constant, then there are distinct $y_{1},y_{2}\in{Y}$ such that
                $f^{\minus{1}}\{y_{1}\}\ne\emptyset$ and similarly for $y_{2}$.
                Define $\mathcal{U}$ as follows:
                \begin{equation}
                    \mathcal{U}=f^{\minus{1}}\big(\{y_{1}\}\big)
                \end{equation}
                Since $f$ is locally constant, for all $x\in\mathcal{U}$ there
                exists an open subset $\mathcal{V}_{x}\in\tau$ such that
                $x\in\mathcal{V}_{x}$ and $f$ is constant on $\mathcal{V}_{x}$.
                But since $x\in\mathcal{U}$, for all $z\in\mathcal{V}_{x}$ it
                must be true that $f(z)=y_{1}$. But then:
                \begin{equation}
                    \mathcal{U}=
                    \bigcup_{x\in\mathcal{U}}\mathcal{V}_{x}
                \end{equation}
                Since $\mathcal{U}$ is the union of open sets, it must be open
                itself. In a similarly manner, $X\setminus\mathcal{U}$ is open.
                But $X\setminus\mathcal{U}$ is non-empty since
                $f^{\minus{1}}(\{y_{2}\})\subseteq{X}\setminus\mathcal{U}$
                and $f^{\minus{1}}(\{y_{2}\})$ is non-empty. But then
                $\mathcal{U}$ and $X\setminus\mathcal{U}$ are non-empty disjoint
                open sets that partition $X$, and therefore $(X,\tau)$ is
                disconnected, a contradiction. Therefore, $f$ is constant.
        \end{proof}
        \begin{theorem}
                If $(X,\tau)$ is a topological space, if $x\in{X}$, if
                $m,n\in\mathbb{N}$, if $(\mathcal{U},\varphi)$ and
                $(\mathcal{V},\psi)$ are $m$ and $n$ dimensional charts,
                respectively, and if $x\in\mathcal{U}$ and $x\in\mathcal{V}$,
                then $n=m$.
        \end{theorem}
        \begin{proof}
                For suppose not. Since $\mathcal{U}$ and $\mathcal{V}$ are open,
                $\mathcal{U}\cap\mathcal{V}$ is open. But $x\in\mathcal{U}$ and
                $x\in\mathcal{V}$, and thus $\mathcal{U}\cap\mathcal{V}$ is
                non-empty. But since
                $\varphi:\mathcal{U}\rightarrow\mathbb{R}^{m}$ and
                $\psi:\mathcal{V}\rightarrow\mathbb{R}^{n}$ are open mappings,
                $\psi\circ\varphi^{\minus{1}}:%
                 \varphi(\mathcal{U}\cap\mathcal{V})\rightarrow\mathbb{R}^{n}$
                is an open mapping. But since $\mathcal{U}\cap\mathcal{V}$ is
                open, and since $\varphi$ is an open mapping,
                $\varphi(\mathcal{U}\cap\mathcal{V})$ is an open subset of
                $\mathbb{R}^{m}$. But then $\varphi(\mathcal{U}\cap\mathcal{V})$
                is an open subset of $\mathbb{R}^{m}$ that is homeomorphic to
                $\psi(\mathcal{U}\cap\mathcal{V})$, which is an open subset
                of $\mathbb{R}^{n}$. But then by invariance of domain, $n=m$.
        \end{proof}
        With this we can define a function
        $\textrm{dim}:X\rightarrow\mathbb{N}$ from any locally Euclidean
        topological space $(X,\tau)$ called the \textit{dimension} function.
        Our current goal is to show that this function is locally constant.
        \begin{theorem}
                If $(X,\tau)$ is a locally Euclidean topological space, if
                $\textrm{dim}:X\rightarrow\mathbb{N}$ is the function such that
                for all $x\in{X}$, $\textrm{dim}(x)$ is the unique
                $n\in\mathbb{N}$ such that there exists a chart
                $(\mathcal{U},\varphi)$ of dimension $n$ such that
                $x\in\mathcal{U}$, then $\textrm{dim}$ is locally constant.
        \end{theorem}
        \begin{proof}
                For suppose not. Then there is a point $x\in{X}$ such that for
                all $\mathcal{U}\in\tau$ such that $x\in\mathcal{U}$, it is not
                true that $f$ is constant on $\mathcal{U}$. But since $(X,\tau)$
                is locally Euclidean, there is an $n\in\mathbb{N}$ and a chart
                $(\mathcal{U},\phi)$ of dimension $n$ such that
                $x\in\mathcal{U}$. But for all $z\in\mathcal{U}$,
                $(\mathcal{U},\phi)$ is a chart of dimension $n$ that contains
                $z$. But the dimension is pointwise invariant, and thus the
                dimension is constant on $\mathcal{U}$, a contradiction. Thus,
                dimension is a locally constant function.
        \end{proof}
        \begin{theorem}
                If $(X,\tau)$ is a connected locally Euclidean topological
                space, then the dimension is constant.
        \end{theorem}
        \begin{proof}
                For dimension is locally constant, and locally constant
                functions on a connected topological space are constant.
        \end{proof}
    \section{Homogeneous Spaces}
        \begin{fdefinition}{Homogeneous Topological Space}
                               {Homogeneous_Topological_Space}
                A homogeneous topological space is a topological space
                $(X,\tau)$ such that for all $x,y\in{X}$ there exists a
                homeomorphism $f:X\rightarrow{X}$ such that $f(x)=y$.
        \end{fdefinition}
        Homogeneous spaces are ones where any point $x$ looks locally like
        any other point.
        \begin{example}
                For any $n\in\mathbb{N}$, the Euclidean space $\mathbb{R}^{n}$
                is homogeneous. Given $\mathbf{x},\mathbf{y}\in\mathbb{R}^{n}$,
                define $f:\mathbb{R}^{n}\rightarrow\mathbb{R}^{n}$ by:
                \begin{equation}
                    f(\mathbf{z})=\mathbf{y}+(\mathbf{x}-\mathbf{z})
                \end{equation}
                Then $f(\mathbf{x})=\mathbf{y}$, as desired. This is bijective
                and bicontinuous, and therefore a homeomorphism. Indeed, by
                invariance of domain it suffices to show that $f$ is bijective
                and continuous. The fact that $f^{\minus{1}}$ is continuous then
                comes for free.
        \end{example}
        \begin{example}
                If $X=S^{n}$, $n>1$, then $S^{n}$ is homogeneous. Given two
                points $\mathbf{x},\mathbf{y}\in{S}^{n}$, let $R$ be the
                rotation mapping that takes $\mathbf{x}$ to $\mathbf{y}$. This
                is bijective and bicontinuous, and hence a homeomorphism. For
                the case of $n=0$, $S^{0}$ is merely two points and the topology
                is the power set. Letting $f$ be the function that swaps these
                two points shows that $S^{0}$ is also homogeneous.
        \end{example}
        Not every space is homogeneous. We'll need a brief discussion on
        ordered spaces to demonstrate this.
        \begin{definition}
                The order topology from a totally ordered set $(X,\leq)$ is
                the topology $\tau$ generated by sets of the form:
                \begin{equation}
                    (a,b)=\{\,x\in{X}\;|\;a<x\textrm{ and }x<b\,\}
                \end{equation}
                together with open rays:
                \par
                \begin{subequations}
                    \begin{minipage}[b]{0.49\textwidth}
                        \centering
                        \begin{equation}
                            (\minus\infty,a)=\{\,x\in{X}\;|\;x<a\,\}
                        \end{equation}
                    \end{minipage}
                    \hfill
                    \begin{minipage}[b]{0.49\textwidth}
                        \centering
                        \begin{equation}
                            (a,\infty)=\{\,x\in{X}\;|\;a<x\,\}
                        \end{equation}
                    \end{minipage}
                \end{subequations}
                \par\vspace{2.5ex}
        \end{definition}
        \begin{example}
                The standard topology on $\mathbb{R}$ is an order topology
                induced by the standard order. The long line is another such
                topology induced by the lexicographic ordering on
                $[0,1)\times\omega_{1}$, where $\omega_{1}$ is the first
                uncountable ordinal.
        \end{example}
        \begin{definition}
                A complete totally ordered set is a totally ordered set
                $(X,\leq)$ with the least upper bound property.
        \end{definition}
        \begin{theorem}
                If $(X,\leq)$ is a totally ordered set, if $\tau$ is the order
                topology, and if $(X,\tau)$ is connected, then $(X,\leq)$ is
                a complete totally ordered set.
        \end{theorem}
        \begin{proof}
                For suppose not. Then there exists a set
                $\mathcal{U}\subseteq{X}$ that is bounded above that contains
                no least upper bound. Let $\mathcal{V}_{+}$ be defined as
                follows:
                \begin{equation}
                    \mathcal{V}_{+}=\{\,x\in{X}\;|\;
                        \forall_{y\in\mathcal{U}}(y<x)\,\}
                \end{equation}
                Since $\mathcal{U}$ is bounded above, $\mathcal{V}_{+}$ is
                non-empty. Moreover, it is open since:
                \begin{equation}
                    \mathcal{V}_{+}=
                    \bigcup_{x\in\mathcal{V}_{+}}(x,\infty)
                \end{equation}
                For if $x_{0}\in\bigcup(x,\infty)$ then there is an
                $x\in\mathcal{V}_{+}$ such that $x_{0}\in(x,\infty)$. But
                for all $y\in\mathcal{U}$ it is true that $y<x$, and thus by
                transitivity $y<x_{0}$. Thus, by definition,
                $x_{0}\in\mathcal{V}_{+}$. Now if $x_{0}\in\mathcal{V}_{+}$,
                suppose it is not in $\bigcup(x,\infty)$. Then for all
                $x\in\mathcal{V}_{+}$ it is true that $x_{0}<x$, or $x=x_{0}$,
                and thus $x_{0}$ is a greatest lower bound of $\mathcal{V}_{+}$.
                But by construction, $x_{0}$ is then a least upper bound of
                $\mathcal{U}$, but $\mathcal{U}$ has no such thing. Thus
                $x_{0}\in\bigcup(\minus\infty,x)$, and hence we have equality.
                Thus $\mathcal{V}_{+}$ is open. By a similar argument,
                ${X}\setminus\mathcal{V}_{+}$ is open, and hence $X$ is
                disconnected, a contradiction. Thus, $(X,\leq)$ is complete.
        \end{proof}
        \begin{theorem}
                If $(X,\leq)$ is a totally ordered set, if $\tau$ is the order
                topology, if $\mathcal{U}$ is a connected subset of $X$, and if
                $a,b\in\mathcal{U}$ are such that $a<b$, then
                $(a,b)\subseteq\mathcal{U}$.
        \end{theorem}
        \begin{proof}
                Suppose not. Then there is a $c\in(a,b)$ such that
                $c\notin\mathcal{U}$. But then $(c,\infty)\cap\mathcal{U}$ and
                $(\minus\infty,c)\cap\mathcal{U}$ are disjoint non-empty subsets
                that are open in $\mathcal{U}$, and hence $\mathcal{U}$ is
                disconnected, a contradiction.
        \end{proof}
        \begin{ltheorem}{Intermediate Value Theorem}
                            {Intermediate_Value_Theorem}
                If $(X,<_{X})$ and $(Y,<_{Y})$ are totally ordered sets, if
                $\tau_{X}$ and $\tau_{Y}$ are the order topologies on $X$ and
                $Y$, respectively, if $(X,\tau_{X})$ is a connected topological
                topological space, if $f:X\rightarrow{Y}$ is a continuous
                function, and if $a,b\in{X}$ are such that $a<_{X}b$ and
                $f(a)<_{Y}f(b)$, then for all $c\in(f(a),f(b))$ there is an
                $x_{0}\in(a,b)$ such that $f(x_{0})=c$.
        \end{ltheorem}
        \begin{proof}
                For the continuous image of a connected set is connected, and
                thus $f\big([a,b]\big)$ is a connected subset of $Y$. But
                $f(a)\in{f}\big([a,b]\big)$ and $f(b)\in{f}\big([a,b]\big)$ and
                since $f\big([a,b]\big)$ is connected we have that
                $\big(f(a),f(b)\big)\subseteq{f}\big([a,b]\big)$. But then for
                all $c\in\big(f(a),f(b)\big)$ it is true that
                $c\in{f}\big((a,b)\big)$, and thus there is an $x_{0}\in(a,b)$
                such that $f(x_{0})=c$.
        \end{proof}
        \begin{theorem}
                If $(X,\leq)$ is a totally ordered set, if $\tau$ is the order
                topology, if $(X,\tau)$ is connected, and if $f:X\rightarrow{X}$
                is a homeomorphism, then either $f$ is order preserving or
                order reversing.
        \end{theorem}
        \begin{proof}
                For suppose not. Then there are $a,b,c\in{X}$ such that
                $a<b$ and $b<c$, yet either $f(b)<f(a)$ and $f(b)<f(c)$, or
                $f(a)<f(b)$ and $f(c)<f(b)$. Suppose $f(a)<f(b)$ and
                $f(c)<f(b)$. Since $a<b$ and $b<c$, by transitivity $a<c$, and
                thus $a\ne{c}$. Furthermore, since $f$ is a homeomorphism it is
                bijective, and therefore injective, and thus $f(a)\ne{f}(c)$.
                But $\leq$ is a total ordering and thus by trichotomy either
                $f(a)<f(c)$ or $f(c)<f(a)$. Suppose $f(a)<f(c)$. But since
                $f(a)<f(c)$ and $f(c)<f(b)$, and thus
                $f(c)\in\big(f(a),f(b)\big)$. But then by the intermediate value
                theorem there is an $x_{0}\in(a,b)$ such that $f(x_{0})=f(c)$.
                But if $x_{0}\in(a,b)$, then $x_{0}<b$. But $b<c$, and therefore
                $x_{0}<c$ and hence $x_{0}\ne{c}$. But then $f(x_{0})=f(c)$ and
                $x_{0}\ne{c}$, contradicting the fact that $f$ is a bijection,
                a contradiction. Similarly, it is not true that $f(b)<f(a)$ and
                $f(b)<f(c)$. Thus, either $f$ is order preserving or order
                reversing.
        \end{proof}
        We can now prove that there are non-homogeneous spaces.
        \begin{example}
                Let $X=(0,1]$, with the standard subspace topology inherited
                from $\mathbb{R}$. This is the same as the order topology from
                the standard ordering. Suppose $(0,1]$ is homogeneous and let
                $x=1$, $y=1/2$. Then there is a homeomorphism
                $f:X\rightarrow{X}$ such that $f(1)=1/2$. But any homeomorphism
                must be order preserving or order reserving, and since
                $1\geq{x}$ for all $x\in(0,1]$, either $f(1)\geq{x}$ for all
                $x$, or $f(1)\leq{x}$ for all $x$. But it $f(1)\leq{x}$ for all
                $x$ then $f(1)$ is a greatest lower bound, but $(0,1]$ has no
                such bound. Hence, $f(1)\geq{x}$ for all $x\in{X}$. But then
                $f(1)=1$, a contradiction since $f(1)=1/2$. Hence, $(0,1]$ is
                not homogeneous.
        \end{example}
        The next theorem to prove is that any connected Hausdorff locally
        Euclidean topological space is homogeneous.
        \begin{theorem}
                If $(X,\tau)$ is a connected Hausdorff locally Euclidean
                topological space, then it is homogeneous.
        \end{theorem}
        If the Hausdorff property is lost we may lose the homogeneity of the
        space as well. The fact that locally Euclidean spaces can be
        non-Hausdorff is one justification for including the Hausdorff
        property in the definition of a topological manifold, which is a
        locally Euclidean Hausdorff topological space that is second
        countable. First we'll demonstrate the existence of non-Hausdorff
        locally Euclidean topological spaces.
        \begin{example}
                Let $X$ be the disjoint union of $\mathbb{R}$ with itself:
                \begin{equation}
                    X=\mathbb{R}\sqcup\mathbb{R}
                \end{equation}
                That is, $X=(\mathbb{R}\times\{0\})\cup(\mathbb{R}\times\{1\})$.
                Furthermore, consider the relation $R'\subseteq{X}\times{X}$
                defined by:
                \begin{equation}
                    R'=\Big\{\,\Big((x,a),(x,b)\Big)\;\Big|\;
                        x\in\mathbb{R}\setminus\{0\}
                        \textrm{ and }a,b\in\mathbb{Z}_{2}\,\Big\}
                \end{equation}
                That is, $(x,a)R'(y,b)$ if and only if $a,b\in\{0,1\}$, and
                $x=y$ are non-zero. From this we can obtain an equivalence
                relation $R$ as follows:
                \begin{equation}
                    R=R'\cup\big\{\big((0,0),(0,0)\big\}\cup
                        \big\{\big((0,1),(0,1)\big\}
                \end{equation}
                That is, $R$ is the reflexive closure of $R'$. This is an
                equivalence relation. Let $Y$ be the topological space
                $(X/R,\tau_{R})$, where $\tau_{R}$ is the quotient topology.
                This space is then locally Euclidean. For if $x\ne{0}$, let
                $r=|x|/2$ Let $\mathcal{U}$ be defined by:
                \begin{equation}
                    \mathcal{U}=\big\{[(y,0)]\in{X}\;|\;|y-x|<r\big\}
                \end{equation}
                Then $\mathcal{U}$ is open since for all points
                $[z]\in\mathcal{U}$, either $z=(y,0)$ or $z=(y,1)$ where
                $|x-y|<r$. By the triangle inequality we can then conclude that
                $y\ne{0}$. But then:
                \begin{equation}
                    \bigcup\mathcal{U}=\Big(B_{r}^{\mathbb{R}}(x)\times\{0\}\Big)
                        \bigcup\Big(B_{r}^{\mathbb{R}}(x)\times\{1\}\Big)
                \end{equation}
                And this is an open subset of $X$ with the disjoint union
                topology, and therefore $\mathcal{U}$ is open in the quotient
                topology. We can define a homemorphism between $\mathcal{U}$ and
                the open interval $(x-r,x+r)$ by defining:
                \begin{equation}
                    \phi\big([(y,0)]\big)=y
                \end{equation}
                Similarly, for the points $(0,0)$ and $(0,1)$ we can define
                $\mathcal{U}_{0}=X\setminus\{[(0,1)]\}$ and
                $\mathcal{U}_{1}=X\setminus\{[0,0]\}$, both of which are
                homeomorphic to all of $\mathbb{R}$. Thus this space, which is
                called the \textit{bug-eyed line}\index{Bug-Eyed Line}, or the
                \textit{line with two origins}\index{Line with Two Origins}, is
                locally Euclidean but is not Hausdorff since every open set
                containing $(0,0)$ and every open set containing $(0,1)$ have
                non-empty intersection.
        \end{example}
        The bug-eyed line\index{Bug-Eyed Line} is shown in
        Fig.~\ref{fig:Bug_Eyed_Line}. The dashed line is used to denote that
        this final sketch is approximately what the space \textit{looks}
        like. We can use this figure to show that any open set containing
        the upper origin must also contain the lower one
        (see Fig.~\ref{fig:Open_Neighborhoods_of_Origins_in_Bug_Eyed_Line}).
        \begin{example}
                The bug-eyed line is not homogeneous. For the points $(0,0)$ and
                $[(1,0)]$ can be separated by open sets, as can the points
                $(0,1)$ and $[(1,0)]$. However $(0,0)$ can not be be separated
                from $(0,1)$, and since homeomorphisms preserve the Hausdorff
                property, there can be no such function $f:X\rightarrow{X}$ such
                that $f\big((0,0)\big)=[(1,0)]$. Hence, the bug-eyed line is
                not homogeneous.
        \end{example}
        \begin{figure}[H]
                \centering
                \captionsetup{type=figure}
                \begin{tikzpicture}[>=Latex]
    % Draw axes for the upper and lower lines (First figure).
    \draw[<->, thick] (-5, 0) to (5, 0);
    \draw[<->, thick] (-5, 2) to (5, 2);

    % Connect the dots for all points except for the two origins.
    \foreach\x in {-4.7, -4.5, ..., -0.3, 0.3, 0.5, ..., 4.7}{
        \draw[<->] (\x, 2) to (\x, 0);
    }

    % Fill in circles for the two origins.
    \draw[fill=black] (0, 0) circle (0.5mm);
    \draw[fill=black] (0, 2) circle (0.5mm);

    % Draw a solid arrow to the next graphic.
    \draw [->, line width=1pt] (0, -0.5) to (0, -1.5);

    % Draw the second figure shifted 2.5cm downwards.
    \begin{scope}[yshift=-2.5cm]
        % Draw the x-axis in the quotient space.
        \draw[<->, thick] (-5.0, 0.0) to (5.0, 0.0);

        % Remove the origin.
        \draw[fill=white, draw=white] (0, 0) circle (0.5mm);

        % Draw circles symbolizing the two origins.
        \draw[fill=black] (0.0,  0.5) circle (0.5mm);
        \draw[fill=black] (0.0, -0.5) circle (0.5mm);

        % Dashed arrow to the next graphic.
        \draw [->, line width=1pt, dashed] (0.0, -1.0) to (0.0, -2.0);
    \end{scope}

    % Draw the third figure shifted 5.5cm downwards.
    \begin{scope}[yshift=-5.5cm]
        % Draw the x-axis.
        \draw[<-, thick] (-5.0, 0.0) to (-0.5, 0);
        \draw[->, thick] ( 0.5, 0.0) to ( 5.0, 0);

        % Draw a curved line connecting the x-axis to the upper origin.
        \draw (-0.5, 0.0) to[out=0,in=180] (0.0, -0.5)
                          to[out=0,in=180] (0.5,  0.0);

        % Draw a curved line connecting the x-axis to the lower origin.
        \draw (-0.5, 0.0) to[out=0,in=180] (0.0, 0.5)
                          to[out=0,in=180] (0.5, 0.0);

        % Fill in the two origins.
        \draw[fill=black] (0.0,  0.5) circle (0.5mm);
        \draw[fill=black] (0.0, -0.5) circle (0.5mm);
    \end{scope}
\end{tikzpicture}
                \caption{Construction of the Bug-Eyed Line}
                \label{fig:Bug_Eyed_Line}
        \end{figure}
        \begin{figure}[H]
                \centering
                \captionsetup{type=figure}
                \begin{tikzpicture}[>=Latex]
    % Draw the x-axis.
    \draw[<-] (-5.0, 0.0) to (-0.5, 0.0);
    \draw[->] ( 0.5, 0.0) to ( 5.0, 0.0);

    % Shade an interval around the bottom origin.
    \draw[blue] (-0.5, 0.0) to[out=0,in=180] (0.0, -0.5)
                            to[out=0,in=180] (0.5,  0.0);

    % Connect the x-axis to the top origin.
    \draw (-0.5, 0.0) to[out=0,in=180] (0.0, 0.5)
                      to[out=0,in=180] (0.5, 0.0);

    % Shade the lower origin black, indicating it is not in the set.
    \draw[fill=black] (0.0,  0.5) circle (0.5mm);

    % Shade the upper origin blue, indicating is belongs to the set.
    \draw[fill=blue]  (0.0, -0.5) circle (0.5mm);

    % Label for the set U_0.
    \node at (2, 0.5) {$\mathcal{U}_{0}$};

    \begin{scope}[yshift=-2cm]
        % Draw the x-axis.
        \draw[<-] (-5.0, 0) to (-0.5, 0);
        \draw[->] ( 0.5, 0) to ( 5.0, 0);

        % Connect the lower origin to the x-axis.
        \draw (-0.5, 0.0) to[out=0,in=180] (0.0, -0.5)
                          to[out=0,in=180] (0.5,  0.0);

        % Shade an interval around the upper origin.
        \draw[blue] (-0.5, 0.0) to[out=0,in=180] (0.0, 0.5)
                                to[out=0,in=180] (0.5, 0.0);

        % Color the two origins accordingly.
        \draw[fill=blue]  (0.0,  0.5) circle (0.5mm);
        \draw[fill=black] (0.0, -0.5) circle (0.5mm);

        % Label for the set U_1.
        \node at (2, 0.5) {$\mathcal{U}_{1}$};
    \end{scope}

    \begin{scope}[yshift=-4cm]
        % Draw the x-axis.
        \draw[<-] (-5.0, 0) to (-0.5, 0);
        \draw[->] ( 0.5, 0) to ( 5.0, 0);

        % Blue coloring for the intersection of U_0 and U_1.
        \draw[blue, thick] (-0.5, 0) to (0.5, 0);

        % Delete the origin from this line.
        \draw[fill=white, draw=white] (0, 0) circle (0.5mm);

        % Color the two origins black.
        \draw[fill=black] (0,  0.5) circle (0.5mm);
        \draw[fill=black] (0, -0.5) circle (0.5mm);

        % Label for the intersection of U_0 and U_1.
        \node at (2, 0.5) {$\mathcal{U}_{0}\cap\mathcal{U}_{1}$};
    \end{scope}
\end{tikzpicture}
                \caption{Open Subsets of the Bug-Eyed Line}
                \label{fig:Open_Neighborhoods_of_Origins_in_Bug_Eyed_Line}
        \end{figure}
        Another common example of a non-Hausdorff locally Euclidean space is
        the branching line. This is defined in a similar manner as the
        bug-eyed line, but with a slightly different equivalence relation.
        We define $R$ by:
        \begin{equation}
                R=\big\{\,\big((x,a),(x,b)\big)\;|\;
                    x<0\textrm{ and }a,b\in\mathbb{Z}_{2}\big\}
        \end{equation}
        We then look at the reflexive closure of $R$, adding
        $\big((x,0),(x,0)\big)$ and $\big((x,1),(x,1)\big)$ for all
        $x\geq{0}$. This is an equivalence relation, and looking at
        $\mathbb{R}\sqcup\mathbb{R}/R$ with the quotient topology gives us
        the branching line. The construction is shown in
        Fig.~\ref{fig:Construction_of_Branching_Line}.
        \begin{figure}[H]
                \centering
                \captionsetup{type=figure}
                %--------------------------------Dependencies----------------------------------%
%   tikz                                                                       %
%       arrows.meta                                                            %
%-------------------------------Main Document----------------------------------%
\begin{tikzpicture}[>=Latex]
    % Draw axes for the upper and lower lines. (First figure).
    \draw[<->, thick] (-5, 0) to (5, 0);
    \draw[<->, thick] (-5, 2) to (5, 2);

    % Connect the dots for all points to the left of the two origins.
    \foreach\x in {-4.7, -4.5, ..., -0.3}{
        \draw[<->] (\x, 2) to (\x, 0);
    }

    % Fill in circles for the two origins.
    \draw[fill=black] (0, 0) circle (0.5mm);
    \draw[fill=black] (0, 2) circle (0.5mm);

    % Draw a solid arrow to the next graphic.
    \draw [->, line width=1pt] (0, -0.5) to (0, -1.5);

    % Draw the second figure, shifted 2.5cm downwards.
    \begin{scope}[yshift=-2.5cm]

        % Draw the x-axis in the quotient space.
        \draw[<-, thick] (-5.0,  0.0) to (0.0,  0.0);
        \draw[->, thick] ( 0.0,  0.5) to (5.0,  0.5);
        \draw[->, thick] ( 0.0, -0.5) to (5.0, -0.5);

        % Remove the origin.
        \draw[fill=white, draw=black] (0, 0) circle (0.5mm);

        % Draw circles symbolizing the two origins.
        \draw[fill=black] (0.0,  0.5) circle (0.5mm);
        \draw[fill=black] (0.0, -0.5) circle (0.5mm);

        % Dashed arrow to the next graphic.
        \draw [->, line width=1pt, dashed] (0.0, -1.0) to (0.0, -2.0);
    \end{scope}

    % Draw the third figure (shift 5.5cm downwards).
    \begin{scope}[yshift=-5.5cm]

        % Draw the x-axis.
        \draw[<-, thick] (-5.0,  0.0) to (-0.5,  0.0);
        \draw[->, thick] ( 0.0,  0.5) to ( 5.0,  0.5);
        \draw[->, thick] ( 0.0, -0.5) to ( 5.0, -0.5);

        % Draw a curved line connecting the x-axis to the upper origin.
        \draw (-0.5, 0.0) to[out=0,in=180] (0.0, -0.5);

        % Draw a curved line connecting the x-axis to the lower origin.
        \draw (-0.5, 0.0) to[out=0,in=180] (0.0, 0.5);

        % Fill in the two origins.
        \draw[fill=black] (0.0,  0.5) circle (0.5mm);
        \draw[fill=black] (0.0, -0.5) circle (0.5mm);
    \end{scope}
\end{tikzpicture}
                \caption{Construction of the Branching Line}
                \label{fig:Construction_of_Branching_Line}
        \end{figure}
        Both the bug-eyed line and the branching line are examples of
        non-Hausdorff, non-homogeneous, locally Euclidean second countable
        topological spaces. Moreover, they are both path-connected. While
        connected Hausdorff spaces that are locally Euclidean are indeed
        homogeneous, the converse is not true. That is, the are connected
        homogeneous locally Euclidean spaces that are not Hausdorff. If one
        adds the Lindel\"{o}f requirement, then it can be proved that the
        space is Hausdorff.
        \begin{fdefinition}{The Complete Feather}{The_Complete_Feather}
                The complete feather is the set:
                \begin{equation*}
                    F=\Big(\bigcup_{n\in\mathbb{N}^{+}}
                        \big\{\,\mathbf{x}\in\mathbb{R}^{n+1}\;|\;
                            \forall_{i\in\mathbb{Z}_{n-1}}(x_{i}<x_{i+1})
                            \land(x_{n-1}\leq{x}_{n})\,\big\}\Big)
                        \bigcup\mathbb{R}
                \end{equation*}
        \end{fdefinition}
        The complete feather is a subset of $\bigcup\mathbb{R}^{n}$ with the
        additional requirements that $x_{0}<x_{1}<\dots<x_{m-1}\leq{x}_{m}$.
        No such restraint is imposed on the $\mathbb{R}^{1}$ term. We place
        a topology by considering the following partial ordering.
        \begin{theorem}
                If $(X,\tau)$ is a Lindel\"{o}f topological space that is
                locally second countable, then it is second countable.
        \end{theorem}
        \begin{proof}
                For if $(X,\tau)$ is locally second countable, for all $x\in{X}$
                there is an open subset $\mathcal{U}_{x}\in\tau$ such that
                $(\mathcal{U}_{x},\tau|_{\mathcal{U}_{x}})$ has a countable
                basis, where $\tau|_{\mathcal{U}_{x}}$ is the subspace topology.
                But then the set:
                \begin{equation}
                    \mathcal{O}=
                    \{\,\mathcal{U}_{x}\in\tau\;|\;x\in{X}\,\}
                \end{equation}
                is an open cover of $X$. But $(X,\tau)$ is Lindel\"{o}f and thus
                there is a countable subcover $\Delta$. But for all
                $\mathcal{U}\in\Delta$ there is a countable basis for
                $\mathcal{U}$, $\mathscr{B}_{\mathcal{U}}$. But then:
                \begin{equation}
                    \mathscr{O}=
                    \bigcup_{\mathcal{U}\in\Delta}
                        \{\,\mathscr{B}_{\mathcal{U}}\;|\;
                            \mathcal{U}\in\Delta\,\}
                \end{equation}
                Is countable basis for $(X,\tau)$. It is countable since it is
                the countable union of countable sets. Morever, let
                $\mathcal{V}\in\tau$. But since $\Delta$ is a cover of $X$, we
                have:
                \begin{equation}
                    \mathcal{V}
                    =\mathcal{V}\cap{X}
                    =\mathcal{V}\cap\Big(
                        \bigcup_{\mathcal{U}\in\Delta}\mathcal{U}
                    \Big)
                    =\bigcup_{\mathcal{U}\in\Delta}\big(
                        \mathcal{U}\cap\mathcal{V}
                    \big)
                \end{equation}
                But for all $\mathcal{U}\in\Delta$ it is true that
                $\mathcal{U}\cap\mathcal{V}\subseteq\mathcal{U}$. But then there
                is a subset
                $\Delta_{\mathcal{U}}\subseteq\mathscr{B}_{\mathcal{U}}$
                \begin{equation}
                    \mathcal{U}\cap\mathcal{V}=
                    \bigcup_{A\in\Delta_{\mathcal{U}}}A
                \end{equation}
                Thus:
                \begin{equation}
                    \mathcal{V}=\bigcup_{\mathcal{U}\in\Delta}\Big(
                        \bigcup_{A\in\Delta_{\mathcal{U}}}A\Big)
                \end{equation}
        \end{proof}
    \section{Topological Manifolds}
        Manifolds are locally Euclidean spaces that have sufficiently nice
        structure to exclude pathalogical examples such as the bug-eyed line
        and the long line. Manifolds are required to have a second-countable
        topology and must also be Hausdorff.
        \begin{fdefinition}{Topological Manifold}{Topological_Manifold}
            A topological manifold of dimension $n\in\mathbb{N}$ is a
            Hausdorff locally Euclidean topological space of dimension $n$
            that is second countable.
        \end{fdefinition}
        \begin{fdefinition}{Atlas}{Atlas}
            An atlas on a topological space $(X,\tau)$ is a set
            $\mathcal{A}$ such that for all $z\in\mathcal{U}$ it is true
            that $z$ is a chart in $(X,\tau)$, and such that for all
            $x\in{X}$ there exists a chart
            $(\mathcal{U},\varphi)\in\mathcal{A}$ such that
            $x\in\mathcal{U}$.
        \end{fdefinition}
        An atlas for a connected space necessarily has a constant and well
        defined dimension, whereas the by definition of a manifold, any
        atlas on a manifold must have the same dimension.
    \section{Smooth Manifolds}
        Smooth manifolds are topological manifolds with a smooth structure
        on them so that one can do calculus. Such objects first require a
        notion of smooth functions, and we define this in terms of the
        familiar notions of smoothness of functions between Euclidean spaces
        $f:\mathbb{R}^{m}\rightarrow\mathbb{R}^{n}$.
        \begin{fdefinition}{Smooth Real-Valued Functions On $\mathbb{R}$}
                           {Smooth_Real_Valued_Functions_on_R}
            A smooth real-valued function on an open subset
            $\mathcal{U}\subseteq\mathbb{R}^{n}$ is a function
            $f:\mathcal{U}\rightarrow\mathbb{R}$ such that all mixed partial
            derivatives of all orders exist and are continuous for all
            $\mathbf{x}\in\mathcal{U}$.
        \end{fdefinition}
        $\mathbb{R}^{n}$ can be defined as the set of all functions
        $\mathbf{x}:\mathbb{Z}_{n}\rightarrow\mathbb{R}$. Given an element
        $\mathbf{x}\in\mathbb{R}^{n}$ and $k\in\mathbb{Z}_{n}$ we denote
        image of $k$ as $x_{k}=\mathbf{x}(k)$. This is called the $k^{th}$
        coordinate of $\mathbf{x}$. The projection mapping
        $\pi_{k}:\mathbb{R}^{n}\rightarrow\mathbb{R}$ for
        $k\in\mathbb{Z}_{n}$ is the function defined by
        $\pi_{k}(\mathbf{x})=x_{k}$. We can use this notion to define smooth
        functions between arbitrary Euclidean spaces by requiring the
        composition of $f:\mathbb{R}^{m}\rightarrow\mathbb{R}^{n}$ with
        $\pi_{k}$ to be smooth for all $k$.
        \begin{fdefinition}{Smooth Euclidean Functions}
                           {Smooth_Euclidean_Functions}
            A smooth function on a subset
            $\mathcal{U}\subseteq\mathbb{R}^{m}$ to $\mathbb{R}^{n}$ is a
            function $f:\mathcal{U}\rightarrow\mathbb{R}^{m}$ such that, for
            all $k\in\mathbb{Z}_{n}$, the function $\pi_{k}\circ{f}$ is a
            smooth real-valued function.
        \end{fdefinition}
        We can use this definition to create a notion of smoothness on a
        manifold by considering charts that overlap \textit{smoothly}.
        \begin{fdefinition}{Smoothly Overlapping Charts}
                               {Smoothly_Overlapping_Charts}
                Smoothly overlapping charts of dimension $n\in\mathbb{N}$ are
                charts $(\mathcal{U}_{1},\varphi_{1})$ and
                $(\mathcal{U}_{2},\varphi_{2})$ of dimension $n$ on a
                topological space $(X,\tau)$ such that:
                \par
                \begin{minipage}[b]{0.49\textwidth}
                    \centering
                    \begin{equation*}
                        \varphi_{1}\circ\phi_{2}^{\minus{1}}:
                            \varphi_{2}(\mathcal{U}_{1}\cap\mathcal{U}_{2})
                            \rightarrow\mathbb{R}^{n}
                    \end{equation*}
                \end{minipage}
                \hfill
                \begin{minipage}[b]{0.49\textwidth}
                    \centering
                    \begin{equation*}
                        \varphi_{2}\circ\phi_{1}^{\minus{1}}:
                        \varphi_{1}(\mathcal{U}_{1}\cap\mathcal{U}_{2})
                        \rightarrow\mathbb{R}^{n}
                    \end{equation*}
                \end{minipage}
                \par\vspace{2.5ex}
                are smooth functions, or such that
                $\mathcal{U}_{1}\cap\mathcal{U}_{2}=\emptyset$.
        \end{fdefinition}
        \begin{figure}[H]
            \centering
            \captionsetup{type=figure}
            %--------------------------------Dependencies----------------------------------%
%   tikz                                                                       %
%       arrows.meta                                                            %
%-------------------------------Main Document----------------------------------%
\begin{tikzpicture}[>=Latex, line width=0.2mm]
    % Coordinates for the manifold X.
    \coordinate (X0) at (-5.0,  0.0);
    \coordinate (X1) at (-3.5, -2.5);
    \coordinate (X2) at ( 1.0, -2.0);
    \coordinate (X3) at ( 5.0,  0.0);
    \coordinate (X4) at ( 0.0,  1.0);

    % Coordinates for the subset U.
    \coordinate (U0) at (-4.0, -0.5);
    \coordinate (U1) at (-3.0, -2.0);
    \coordinate (U2) at ( 1.5, -0.5);
    \coordinate (U3) at (-0.6,  0.2);

    % Coordinates for the subset V.
    \coordinate (V0) at ( 4.0,  0.0);
    \coordinate (V1) at ( 3.0, -1.5);
    \coordinate (V2) at (-1.5, -0.5);
    \coordinate (V3) at ( 0.6,  0.2);

    % Draw the manifold X.
    \draw   (X0) to[out=-90, in=120]  (X1)
                 to[out=-60, in=-170] (X2)
                 to[out=10, in=-90]   (X3)
                 to[out=90, in=0]     (X4)
                 to[out=-180, in=90]  cycle;

    % Fill in U and V first and then outline with dashes.
    % This prevents the fill option from drawing over the outline.
    % Setting opacity makes the overlapping part mix colors as well.

    % Fill in the background of U blue.
    \draw[fill=blue, opacity=0.5, draw=none]
        (U0) to[out=-90, in=-180] (U1)
             to[out=0, in=-100]   (U2)
             to[out=80, in=0]     (U3)
             to[out=-180, in=90]  cycle;

    % Fill in the background of V red.
    \draw[fill=red, opacity=0.5, draw=none]
        (V0) to[out=-90, in=0]   (V1)
             to[out=180, in=-80] (V2)
             to[out=100, in=180] (V3)
             to[out=0, in=90]    cycle;

    % Draw dashed lines around U.
    \draw[densely dashed]
        (U0) to[out=-90, in=-180] (U1)
             to[out=0, in=-100]   (U2)
             to[out=80, in=0]     (U3)
             to[out=-180, in=90]  cycle;

    \draw[densely dashed]
        (V0) to[out=-90, in=0]   (V1)
             to[out=180, in=-80] (V2)
             to[out=100, in=180] (V3)
             to[out=0, in=90]    cycle;

    \begin{scope}[xshift=-5cm, yshift=3cm]

        % Coordinates for phi of U.
        \coordinate (P0) at (0.5, 0.5);
        \coordinate (P1) at (1.5, 0.2);
        \coordinate (P2) at (3.3, 0.8);
        \coordinate (P3) at (2.8, 2.1);
        \coordinate (P4) at (2.2, 3.6);
        \coordinate (P5) at (1.2, 2.8);

        % Coordinate for some midpoint inside U.
        \coordinate (PM) at (2.0, 1.5);

        \draw[->] (-0.5,  0.0) to ( 4.0,  0.0);
        \draw[->] ( 0.0, -0.5) to ( 0.0,  4.0);

        \draw[draw=none, fill=blue!20!white]
            (P0)    to[out=-30,  in=180]    (P1)
                    to[out=0,    in=-90]    (P2)
                    to[out=90,   in=-120]   (P3)
                    to[out=60,   in=30]     (P4)
                    to[out=-150, in=60]     (P5)
                    to[out=-120, in=150]    cycle;

        \draw[densely dashed]
            (P0)    to[out=-30,  in=180]    (P1)
                    to[out=0,    in=-90]    (P2)
                    to[out=90,   in=-120]   (P3)
                    to[out=60,   in=30]     (P4)
                    to[out=-150, in=60]     (P5)
                    to[out=-120, in=150]    cycle;

        \draw[densely dashed, fill=cyan]
            (P3)    to[out=180,  in=70]   (PM)
                    to[out=-110, in=180]  (P1)
                    to[out=0,    in=-90]  (P2)
                    to[out=90,   in=-120] cycle;

        \node at (2.00, 3.0) {$\phi(\mathcal{U})$};
        \node at (2.45, 0.8) {$\phi(\mathcal{U}\cap\mathcal{V})$};
        \node at (3.50, 3.5) {\large{$\mathbb{R}^{n}$}};
    \end{scope}

    \begin{scope}[xshift=2cm, yshift=3cm]

        % Coordinates for phi of U.
        \coordinate (Q0) at (3.5, 0.5);
        \coordinate (Q1) at (2.5, 0.2);
        \coordinate (Q2) at (0.5, 0.8);
        \coordinate (Q3) at (1.2, 2.1);
        \coordinate (Q4) at (1.8, 3.6);
        \coordinate (Q5) at (2.8, 2.8);

        % Coordinate for some midpoint inside U.
        \coordinate (QM) at (2.0, 1.5);

        \draw[->] (-0.5,  0.0) to ( 4.0,  0.0);
        \draw[->] ( 0.0, -0.5) to ( 0.0,  4.0);

        \draw[draw=none, fill=red!20!white]
            (Q0)    to[out=-150,    in=0]       (Q1)
                    to[out=-180,    in=-90]     (Q2)
                    to[out=90,      in=-120]    (Q3)
                    to[out=60,      in=150]     (Q4)
                    to[out=-30,     in=60]      (Q5)
                    to[out=-120,    in=30]      cycle;

        \draw[densely dashed]
            (Q0)    to[out=-150,    in=0]       (Q1)
                    to[out=-180,    in=-90]     (Q2)
                    to[out=90,      in=-120]    (Q3)
                    to[out=60,      in=150]     (Q4)
                    to[out=-30,     in=60]      (Q5)
                    to[out=-120,    in=30]      cycle;

        \draw[densely dashed, fill=red!50!white]
            (Q3)    to[out=-60,     in=70]      (QM)
                    to[out=-110,    in=0]       (Q1)
                    to[out=-180,    in=-90]     (Q2)
                    to[out=90,      in=-120]    cycle;

        \node at (2.00, 3.0) {$\xi(\mathcal{V})$};
        \node at (1.25, 0.8) {$\xi(\mathcal{U}\cap\mathcal{V})$};
        \node at (3.50, 3.5) {\large{$\mathbb{R}^{n}$}};
    \end{scope}

    \begin{scope}[line width=0.4mm, ->, font=\large]
        \draw (-2.0, 0.5) to[out=130, in=-100] node[left]  {$\phi$} (-3.0, 3);
        \draw ( 2.5, 0.7) to[out=50,  in=-80]  node[right] {$\xi$}  ( 3.5, 3);
        \draw (-1.5, 4.5) to[out=30, in=150]
            node[above] {$\xi\circ\phi^{\minus{1}}$} ( 1.5, 4.5);
        \draw ( 1.5, 3.5) to[out=-150, in=-30]
            node[below] {$\phi\circ\xi^{\minus{1}}$} (-1.5, 3.5);
    \end{scope}

    \node at (-4.0,  0.5) {$X$};
    \node at (-3.0, -1.5) {$\mathcal{U}$};
    \node at ( 3.0, -1.3) {$\mathcal{V}$};
    \node at ( 0.0, -0.5) {$\mathcal{U}\cap\mathcal{V}$};
\end{tikzpicture}
            \caption{Smoothly Overlapping Charts}
            \label{fig:Smoothly_Overlapping_Charts}
        \end{figure}
        \begin{fdefinition}{Transition Function}{Transition_Function}
            The transition function of a chart $(\mathcal{U},\varphi)$ with
            respect a chart $(\mathcal{V},\psi)$ on a topological space
            $(X,\tau)$ is the function
            $f:\varphi(\mathcal{U}\cap\mathcal{V})%
             \rightarrow\psi(\mathcal{V}\cap\mathcal{V})$:
            \begin{equation*}
                f(x)=(\psi\circ\phi^{\minus{1}})(x)
            \end{equation*}
        \end{fdefinition}
        \begin{fdefinition}{Smooth Atlas}{Smooth Atlas}
            A smooth atlas on a topological space $(X,\tau)$ is an
            atlas $\mathcal{A}$ such that for all charts
            $(\mathcal{U},\phi),(\mathcal{V},\xi)\in\mathcal{A}$, the
            transition function of $(\mathcal{U},\phi)$ with respect to
            $(\mathcal{V},\xi)$ is smooth.
        \end{fdefinition}
        \begin{fdefinition}{Maximal Smooth Atlas}{Maximal_Smooth_Atlas}
            A maximal smooth atlas on a topological space $(X,\tau)$ is a
            smooth atlas $\mathcal{A}$ on $(X,\tau)$ such that for all
            charts $(\mathcal{U},\varphi)$ of $(X,\tau)$ such that
            $(\mathcal{U},\varphi)$ overlaps smoothly with all
            $(\mathcal{V},\psi)\in\mathcal{A}$, it is true that
            $(\mathcal{U},\varphi)\in\mathcal{A}$.
        \end{fdefinition}
        \begin{theorem}
            If $(X,\tau)$ is a topological space and if $\mathcal{A}$ is a
            smooth atlas of dimension $n\in\mathbb{N}$ on $(X,\tau)$, then
            there is a unique maximal smooth atlas $\mathcal{C}$ on
            $(X,\tau)$ such that $\mathcal{A}\subseteq\mathcal{C}$.
        \end{theorem}
        \begin{proof}
            For let $\mathcal{C}$ be the set of all charts on $(X,\tau)$
            that overlap smoothly with the charts in $\mathcal{A}$. Then
            since $\mathcal{A}$ is an atlas, for all
            $(\mathcal{U},\phi)\in\mathcal{A}$ and for all
            $(\mathcal{V},\xi)\in\mathcal{A}$, we have that
            $(\mathcal{U},\phi)$ and $(\mathcal{V},\xi)$ overlap smoothly,
            and thus $(\mathcal{U},\phi)\in\mathcal{C}$. Therefore
            $\mathcal{A}\subseteq\mathcal{C}$. But $\mathcal{A}$ is an
            atlas and thus for all $x\in{X}$ there is a chart
            $(\mathcal{U},\phi)\in\mathcal{A}$ such that $x\in\mathcal{U}$.
            But $\mathcal{A}\subseteq\mathcal{C}$ and thus
            $(\mathcal{U},\phi)\in\mathcal{C}$. Thus, for all $x\in{X}$ there
            is a chart $(\mathcal{U},\phi)\in\mathcal{C}$ such that
            $x\in\mathcal{U}$. Suppose
            $(\mathcal{U}_{1},\phi_{1}),%
             (\mathcal{U}_{2},\phi_{2})\in\mathcal{C}$.
            If $\mathcal{U}_{1}$ and $\mathcal{U}_{2}$ are disjoint, then
            these two charts overlap smoothly. Suppose it is non-empty and let
            $f$ be the transition function of $(\mathcal{U}_{1},\phi_{1})$ with
            respect to $(\mathcal{U}_{2},\phi_{2})$. Let
            $p\in\phi_{1}(\mathcal{U}_{1}\cap\mathcal{U}_{2})$. But then there
            is a chart $\xi\in\mathcal{A}$ such that $\phi_{2}^{\minus{1}}(p)$
            is contained in the domain of $\xi$. From the associativity of
            composition, we have:
            \begin{equation}
                \phi_{1}\circ\phi_{2}^{\minus{1}}
                =(\phi_{1}\circ\xi^{\minus{1}})\circ
                 (\xi\circ\phi_{2}^{\minus{1}})
            \end{equation}
            But by the definition of $\mathcal{C}$, $\phi_{1}$ and $\phi_{2}$
            overlap smoothly with $\xi$, and thus this is the composition of
            smooth functions, and is therefore smooth. Therefore
            $\phi_{1}\circ\phi_{2}^{\minus{1}}$ is smooth and thus
            $(\mathcal{U}_{1},\phi_{1})$ and $(\mathcal{U}_{2},\phi_{2})$
            overlap smoothly. Thus, $\mathcal{C}$ is a smooth atlas. Moreover,
            it is complete from the construction. Given any other complete
            atlas $\mathcal{C}'$ that contains $\mathcal{A}$ we would have
            $\mathcal{C}\subseteq\mathcal{C}'$ and
            $\mathcal{C}'\subseteq\mathcal{C}$, and therefore
            $\mathcal{C}=\mathcal{C}'$. Thus, this completion is unique.
        \end{proof}
        \begin{fdefinition}{Smooth Manifold}{Smooth_Manifold}
            A smooth manifold of dimension $n\in\mathbb{N}$, denoted
            $(X,\tau,\mathcal{A})$ is topological manifold $(X,\tau)$ with
            a maximal smooth atlas $\mathcal{A}$ of dimension $n$ on
            $(X,\tau)$.
        \end{fdefinition}
        Any smooth atlas $\mathcal{A}$ on a topological space $(X,\tau)$
        defines a a smooth manifold if we let $\mathcal{C}$ be the maximal
        smooth atlas generated by $\mathcal{A}$.
        \begin{example}
            Let $(\mathbb{R}^{n},\tau_{\mathbb{R}^{n}})$ be the standard
            $n$ dimensional Euclidean space. We can define a trivial smooth
            atlas on this space by let
            $\mathcal{A}=\{(\mathbb{R}^{n},\textrm{id})\}$, where $\textrm{id}$
            is the identity function. This defines a smooth atlas. By
            considering the unique maximal smooth atlas generated by this
            we obtain the standard smooth structure on $\mathbb{R}^{n}$.
        \end{example}
        \begin{theorem}
            If $(X,\tau,\mathcal{A})$ is a smooth manifold of dimension
            $n\in\mathbb{N}$, if $(\mathcal{U},\varphi)$ is a chart in
            $\mathcal{A}$, if $\mathcal{V}\in\tau$, and if
            $\varphi_{\mathcal{V}}$ denotes the restriction mapping:
            $\varphi_{\mathcal{V}}:\mathcal{V}\rightarrow\mathbb{R}^{n}$,
            then $(\mathcal{V},\varphi_{\mathcal{V}})\in\mathcal{A}$.
        \end{theorem}
        \begin{proof}
            For since $\phi$ is a homeomorphism from $\mathcal{U}$ to
            $\phi(\mathcal{U})$, and since $\mathcal{V}\in\tau$, we have
            that $\phi_{\mathcal{V}}$ is a homeomorphism between
            $\mathcal{V}$ and $\phi_{\mathcal{V}}(\mathcal{V})$, and
            therefore $(\mathcal{V},\phi_{\mathcal{V}})$ is a chart. But
            this chart meets $(\mathcal{U},\phi)$ smoothly, and $\mathcal{A}$
            is complete. Thus,
            $(\mathcal{V},\phi_{\mathcal{A}})\in\mathcal{A}$.
        \end{proof}
        \begin{theorem}
            If $n\in\mathbb{N}$, then there is a complete atlas
            $\mathcal{A}$ on $(S^{n},\tau)$, where $\tau$ is the inherited
            topology from $\mathbb{R}^{n+1}$.
        \end{theorem}
        \begin{proof}
            For all $k\in\mathbb{Z}_{n+1}$, let $\mathcal{U}_{k}^{+}$ and
            $\mathcal{U}_{k}^{\minus}$ be defined as:
            \par\hfill\par
            \begin{minipage}[b]{0.49\textwidth}
                \begin{equation}
                    \mathcal{U}_{k}^{+}
                    =\{\,\mathbf{x}\in{S}^{n}\,:\,x_{k}>0\,\}
                \end{equation}
            \end{minipage}
            \hfill
            \begin{minipage}[b]{0.49\textwidth}
                \begin{equation}
                    \mathcal{U}_{k}^{\minus}
                    =\{\,\mathbf{x}\in{S}^{n}\,:\,x_{k}<0\,\}
                \end{equation}
            \end{minipage}
            \par\vspace{2.5ex}
            Define $\phi_{\mathcal{U}_{k}^{+}}:\mathcal{U}_{k}^{+}%
                    \rightarrow\mathbb{R}^{n}$ by:
            \begin{equation}
                \phi_{\mathcal{U}_{k}^{+}}(\mathbf{x})
                =(x_{1},\,\dots,\,x_{k-1},\,x_{k+1},\,\dots,\,x_{n+1})
            \end{equation}
            That is, the mapping that projects the point onto the plane
            defined by $x_{k}=0$. Define $\phi_{\mathcal{U}_{k}^{\minus}}$
            similarly. Then all such $\phi$ are homeomorphisms from their
            domain to their image. Let $\mathcal{A}$ be defined as follows:
            \begin{equation}
                \mathcal{A}
                =\big\{\,(\mathcal{U}_{k}^{+},\,\phi_{\mathcal{U}_{k}^{+}})
                       \,:\,k\in\mathbb{Z}_{n}\big\}\bigcup
                \big\{\,(\mathcal{U}_{k}^{-},\,\phi_{\mathcal{U}_{k}^{-}})
                       \,:\,k\in\mathbb{Z}_{n}\big\}
            \end{equation}
            Then $\mathcal{A}$ is an atlas of $(S^{n},\tau)$. For if
            $\mathbf{x}\in{S}^{n}$, then $\norm{\mathbf{x}}_{2}=1$. But then
            there is a coordinate $x_{k}$ of $\mathbf{x}$ such that
            $x_{k}\ne{0}$. But then either $x_{k}>0$ or $x_{k}<0$, and thus
            either $\mathbf{x}\in\mathcal{U}_{k}^{+}$ or
            $\mathbf{x}\in\mathcal{U}_{k}^{\minus}$. If
            $(\mathcal{V}_{2},\phi_{2})$ and $(\mathcal{V}_{2},\phi_{2})$
            are charts, then either $\phi_{1}(\mathcal{V}_{1})$
            and $\phi_{2}(\mathcal{V}_{2})$ are disjoint or they are not.
            If they are disjoint, then $\phi_{1}$ and $\phi_{2}$ overlap
            smoothly. If they are not disjoint, let $\mathbf{x}$ be contained
            in the intersection. But then, for all $k\in\mathbb{Z}_{n}$,
            $\pi_{k}\circ(\phi_{1}\circ\phi_{2}^{\minus{1}})$ is smooth,
            and thus $\phi_{1}$ and $\phi_{2}$ overlap smoothly.
        \end{proof}
        \begin{fdefinition}{Open Submanifold}{Open_Submanifold}
            An open submanifold on a manifold $(X,\tau,\mathcal{A})$ is a
            an open subset $\mathcal{U}\subseteq{X}$ and the collection
            $\mathcal{A}_{\mathcal{U}}$ defined by:
            \begin{equation}
                \mathcal{A}_{\mathcal{U}}
                =\{\,(\mathcal{V},\phi)\in\mathcal{A}\,:\,
                     \mathcal{V}\subseteq\mathcal{U}\,\}
            \end{equation}
            Together with the inherited topology $\tau_{\mathcal{U}}$.
        \end{fdefinition}
        \begin{theorem}
            If $(X,\tau,\mathcal{A})$ is a smooth manifold and if
            $(\mathcal{U},\tau_{\mathcal{U}},\mathcal{A}_{\mathcal{U}})$
            is an open submanifold, then it is a smooth manifold.
        \end{theorem}
        \begin{proof}
            For by the previous theorem, $\mathcal{A}_{\mathcal{U}}$ is a
            complete atlas. Moreover, a subspace of a Hausdorff topological
            space is also a Hausdorff topological space, and hence
            $(\mathcal{U},\tau_\mathcal{U})$ is a Hausdorff space. Thus,
            $(\mathcal{U},\tau_\mathcal{U},\mathcal{A}_{\mathcal{U}})$ is
            a smooth manifold.
        \end{proof}
        \begin{fdefinition}{Product Chart}{Product_Chart}
            The product chart of an $n$ dimensional chart $(\mathcal{U},\phi)$
            on a topological space $(X,\tau_{X})$ with an $m$ dimensional chart
            $(\mathcal{V},\xi)$ on a topological space $(Y,\tau_{Y})$ is
            the ordered pair $(\mathcal{U}\times\mathcal{V},f)$ where
            $f:\mathcal{U}\times\mathcal{V}\rightarrow\mathbb{R}^{n+m}$
            defined by:
            \begin{equation}
                f(p,q)_{k}=
                \begin{cases}
                    \phi(p)_{k},&k<n\\
                    \xi(q)_{k},&n\leq{k}<n+m
                \end{cases}
            \end{equation}
            Where $\phi(p)_{k}$ is the $k^{th}$ coordinate of
            $\phi(p)\in\mathbb{R}^{n}$ and $\xi(q)_{k}$ is the $k^{th}$
            coordinate of $\xi(q)\in\mathbb{R}^{m}$. We denote this by
            $(\mathcal{U},\phi)\times(\mathcal{V},\xi)$.
        \end{fdefinition}
        Thinking of the elements of $\mathbb{R}^{n+m}$ as tuples of length
        $n+m$, we can write:
        \begin{equation}
            f(p,q)=\big(x_{1}(p),\dots,x_{n}(p),y_{1}(q),\dots,y_{m}(q)\big)
        \end{equation}
        \begin{theorem}
            If $(X,\tau_{X},\mathcal{A}_{X})$ and $(Y,\tau_{Y},\mathcal{A}_{Y})$
            are smooth manifolds, and if $\mathcal{A}$ is the set of all
            product charts on $X\times{Y}$, then $\mathcal{A}$ is a smooth
            atlas on $(X\times{Y},\tau_{X\times{Y}})$, where
            $\tau_{X\times{Y}}$ is the product topology.
        \end{theorem}
        \begin{proof}
            For if $p\in{X}\times{Y}$ then there is an $x\in{X}$ and a
            $y\in{Y}$ such that $p=(x,y)$. But $\mathcal{A}_{X}$ is a smooth
            atlas on $(X,\tau_{X})$, and thus if $x\in{X}$ then there is a
            $(\mathcal{U},\phi)\in\mathcal{A}_{X}$ such that $x\in\mathcal{U}$.
            Similarly, there is a $(\mathcal{V},\xi)\in\mathcal{A}_{Y}$ such
            that $y\in\mathcal{V}$. But then $p\in\mathcal{U}\times\mathcal{V}$,
            and $\mathcal{U}\times\mathcal{V}\in\tau_{X\times{Y}}$. But if
            $\phi:\mathcal{U}\rightarrow\mathbb{R}^{n}$ is a homeomorphism
            between $\mathcal{U}$ and $\phi(\mathcal{U})$ and
            $\xi:\mathcal{V}\rightarrow\mathbb{R}^{m}$ is a homemorphism
            between $\mathcal{V}$ and $\xi(\mathcal{V})$, then
            $f:\mathcal{U}\times\mathcal{V}\rightarrow\mathbb{R}^{n+m}$ is
            a homeomorphism between $\mathcal{U}\times\mathcal{V}$ and
            $f(\mathcal{U}\times\mathcal{V})$, and thus the product chart
            is a chart in $(X\times{Y},\tau_{X\times{Y}})$. Moreover, all of
            the elements of $\mathcal{A}$ are smoothly overlapping. Thus,
            $\mathcal{A}$ is an atlas on $(X\times{Y},\tau_{X\times{Y}})$.
        \end{proof}
        Using the maximal smooth atlas generated by the product atlas
        $\mathcal{A}$ creates the product manifold.
        \subsection{Smooth Mappings}
            \begin{fdefinition}{Smooth Real-Valued Functions}
                               {Smooth_Real_Valued_Functions}
                A smooth real-valued function on a smooth manifold
                $(X,\tau,\mathcal{A})$ of dimension $n\in\mathbb{N}$ is a
                function $\phi:X\rightarrow\mathbb{R}$ such that for every
                chart $(\mathcal{U},\varphi)\in\mathcal{A}$, the function
                $\phi\circ\varphi^{\minus{1}}:\phi(\mathcal{U})%
                 \rightarrow\mathbb{R}$ is a smooth Euclidean function.
            \end{fdefinition}
            \begin{figure}[H]
                \centering
                \captionsetup{type=figure}
                \begin{tikzpicture}[>=Latex]
    \draw[<->] (-1,  0) to (3, 0);
    \draw[<->] ( 0, -1) to (0, 3);
    \draw[dashed, fill=red!70!white]
    (0.5, 1.5) to[out=30,  in=180]  (1.2,  1.9)
               to[out=0,   in=90]   (1.9,  1.4)
               to[out=-90, in=0]    (1.4,  0.2)
               to[out=180, in=-30]  (0.4,  0.3)
               to[out=150, in=-150] cycle;

    \node at (0.5, 3.0) {$\mathbb{R}^{n}$};
    \node at (1.1, 1.0) {$\varphi^{\minus{1}}[\mathcal{U}]$};
    \draw[<-] (1.3,2) to[out=90, in=180] node[above left] {$\varphi$} (4.7,4.7);

    \begin{scope}[xshift=3cm,yshift=5cm]
        \draw (0,0) to[out=90,  in=180] (1,  1.0)
                    to[out=0,   in=150] (2,  1.0)
                    to[out=-30, in=90]  (4,  0.0)
                    to[out=-90, in=0]   (2, -2.5)
                    to[out=180, in=-30] (0, -2.5)
                    to[out=150, in=-90] cycle;
        
        \draw (0.5, -1.7) to[in=-130, out=-50] (1.8, -1.7);
        \draw (0.6, -1.8) to[in=130,  out=50]  (1.7, -1.8);

        \draw[dashed, fill=cyan]
            (2.0, 0.0) to[out=30,  in=180]  (2.7,  0.5)
                       to[out=0,   in=90]   (3.5,  0.0)
                       to[out=-90, in=0]    (3.2, -0.9)
                       to[out=180, in=-30]  (2.2, -0.8)
                       to[out=150, in=-150] cycle;

        \node at (2.8, -0.2) {$\mathcal{U}$};
        \node at (1.0,  0.6) {$X$};
    \end{scope}
    \draw[<->] (6, 0) to (10, 0) node[above] {$\mathbb{R}$};
    \draw[->]  (6.7, 4.7) to[out=0, in=90] node[right] {$f$} (8, 0.2);
    \draw[blue, thick] (9, 0) arc (0:15:0.5);
    \draw[blue, thick] (9, 0) arc (0:-15:0.5);
    \draw[blue, thick] (7, 0) arc (180:195:0.5);
    \draw[blue, thick] (7, 0) arc (180:165:0.5);
    \draw[blue, thick] (7,0) to (9, 0);
    \draw[->] (1.2, -0.2) to[out=-30, in=-150]
        node[above] {$f\circ\varphi^{\minus{1}}$} (7.8, -0.2);
\end{tikzpicture}
                \caption{Smooth Real-Valued Function on a Manifold}
            \end{figure}
            \begin{theorem}
                If $(X,\tau,\mathcal{A})$ is a manifold and if
                $f,g:X\rightarrow\mathbb{R}$ are smooth real-valued functions,
                then $(f+g):X\rightarrow\mathbb{R}$ defined by:
                \begin{equation}
                    (f+g)(x)=f(x)+g(x)
                    \quad\quad
                    x\in{X}
                \end{equation}
                Is a smooth real-valued function.
            \end{theorem}
            \begin{theorem}
                If $(X,\tau,\mathcal{A})$ is a manifold and if
                $f,g:X\rightarrow\mathbb{R}$ are smooth real-valued functions,
                then $(f\cdot{g}):X\rightarrow\mathbb{R}$ defined by:
                \begin{equation}
                    (f\cdot{g})(x)=f(x)\cdot{g}(x)
                    \quad\quad
                    x\in{X}
                \end{equation}
                Is a smooth real-valued function.
            \end{theorem}
            \begin{fdefinition}{Smooth Functions Between Manifolds}
                               {Smooth Functions Between Manifolds}
                A smooth function from a smooth manifold
                $(X,\tau_{X},\mathcal{A}_{X})$ of dimension $m\in\mathbb{N}$
                to a smooth manifold $(Y,\tau_{Y},\mathcal{A}_{Y})$ of
                dimension $n\in\mathbb{N}$ is a function
                $\phi:X\rightarrow{Y}$ such that for every chart
                $(\mathcal{U},\varphi)\in\mathcal{A}_{X}$ and for every
                chart $(\mathcal{V},\psi)\in\mathcal{A}_{Y}$, the function
                $\psi\circ\phi\circ\varphi^{\minus{1}}:\varphi(\mathcal{V})%
                 \rightarrow\mathbb{R}^{n}$ is a smooth function.
            \end{fdefinition}
            \begin{theorem}
                If $(X,\tau_{X},\mathcal{A}_{X})$ and
                $(Y,\tau_{Y},\mathcal{A}_{Y})$ are manifolds, if
                $A_{X}\subseteq\mathcal{A}_{X}$ is an atlas on $(X,\tau_{X})$,
                if $A_{Y}\subseteq\mathcal{A}_{Y}$ is an atlas on
                $(Y,\tau_{Y})$, and if $f:X\rightarrow{Y}$ is a function such
                that, for all $(\mathcal{U},\phi)\in{A}_{X}$ and for all
                $(\mathcal{Y},\xi)\in{A}_{y}$ it is true that
                $\xi\circ{f}\circ\phi^{\minus{1}}:\xi(\mathcal{V})%
                 \rightarrow\mathbb{R}^{m}$ is a smooth Euclidean function,
                then $f$ is smooth.
            \end{theorem}
            \begin{proof}
                Since charts in $\mathcal{A}_{X}$ and $\mathcal{A}_{Y}$
                overlap smoothly with charts in $A_{X}$ and $A_{Y}$, and since
                the atlases $A_{X}$ and $A_{Y}$ cover $X$ and $Y$, respectively,
                we are done.
            \end{proof}
            \begin{theorem}
                If $(X,\tau_{X},\mathcal{A}_{X})$ is a manifold, then
                $\textrm{id}:X\rightarrow{X}$ is a smooth function.
            \end{theorem}
            \begin{theorem}
                If $(X,\tau_{X},\mathcal{A}_{X})$,
                $(Y,\tau_{Y},\mathcal{A}_{Y})$, and
                $(Z,\tau_{Z},\mathcal{A}_{Z})$ are manifolds, if
                $f:X\rightarrow{Y}$ and $g:Y\rightarrow{Z}$ are smooth, then
                $g\circ{f}:X\rightarrow{Z}$ is smooth.
            \end{theorem}
            Smoothness is a local property. A function $\phi:M\rightarrow{N}$
            is smooth at $p\in{M}$ if there is a neighborhood
            $\mathcal{U}$ of $p$ such that the restriction of $\phi$ to
            $\mathcal{U}$ is smooth. A smooth function is thus a function
            that is smooth at every point.
            \begin{theorem}
                If $(X,\mathcal{A}_{X},\tau_{X})$ and
                $(Y,\mathcal{A}_{Y},\tau_{Y})$ are manifolds and if
                $f:X\rightarrow{Y}$ is smooth, then $f$ is continuous.
            \end{theorem}
            \begin{fdefinition}{Diffeomorphism}{Diffeomorphism}
                A diffeomorphism from a manifold
                $(X,\tau_{X},\mathcal{A}_{X})$ to a manifold
                $(Y,\tau_{Y},\mathcal{A}_{Y})$ is a bijective function
                $f:X\rightarrow{Y}$ such that $f$ and $f^{\minus{1}}$ are
                smooth.
            \end{fdefinition}
            \begin{example}
                For any $a,b\in\mathbb{R}$ with $a<b$, the interval
                $(a,b)$ is diffeomorphic to the unit interval $(0,1)$. For
                let $\phi:(0,1)\rightarrow(a,b)$ be defined by:
                \begin{equation}
                    \phi(t)=(a-b)t+b
                \end{equation}
                Then $\phi$ is a smooth bijection and it's inverse is
                smooth. Moreover, the unit interval is diffeomorphic to
                $\mathbb{R}$. For let $\xi:(0,1)\rightarrow\mathbb{R}$ be
                defined by:
                \begin{equation}
                    \xi(t)=\frac{2t}{t(1-t)}
                \end{equation}
            \end{example}
            \begin{theorem}
                If $(X,\tau_{X},\mathcal{A}_{X})$ and
                $(Y,\tau_{Y},\mathcal{A}_{Y})$ are manifolds, and if
                $f:X\rightarrow{Y}$ is a diffeomorphism, then $f$ is a
                homeomorphism from $(X,\tau_{X})$ to $(Y,\tau_{Y})$.
            \end{theorem}
            \begin{proof}
                For if $f$ is a diffeomorphism, then it is a smooth bijection
                such that it's inverse is smooth. But if $f$ is smooth, then
                it is continuous and therefore it is a continuous bijection.
                But if $f^{\minus{1}}$ is smooth, then it is continuous, and
                thus $f$ is a bicontinuous bijective function, and is therefore
                a homeomorphism.
            \end{proof}
            A smooth homeomorphism need not be a diffeomorphism. The inverse
            function may not be smooth. For let
            $f:\mathbb{R}\rightarrow\mathbb{R}$ be defined by $f(x)=x^{3}$.
            Then $f$ is a homeomorphism and it's forward direction is smooth,
            but $f^{\minus{1}}$ is not smooth at the origin.
            \begin{theorem}
                If $A$ is a set, if $(X,\tau,\mathcal{A})$ is a manifold, and
                if $f:A\rightarrow{X}$ is an bijective function, then there
                exists a topology $\tau_{A}$ and an atlas $\mathcal{A}_{A}$
                on $X$ such that $f$ is a diffeomorphism.
            \end{theorem}
    \section{Manifolds}
        \begin{definition}
            An $n$ dimensional smooth manifold is a topological space $(M,\tau)$
            equipped with a collection $\mathcal{A}$ or ordered pairs
            $\{(\mathcal{U}_{\alpha},\varphi_{\alpha})\}$ such that
            $\mathcal{U}_{\alpha}$ are open and cover the space, and such that
            $\varphi_{\alpha}:\mathcal{U}_{\alpha}\rightarrow\mathcal{V}_{\alpha}$
            is a homeomorphism to an open subset $\mathcal{V}_{\alpha}$ of
            $\mathbb{R}^{n}$ and such that for when
            $\mathcal{U}_{\alpha}\cap\mathcal{U}_{\beta}\ne\emptyset$, the
            functions $\varphi_{\beta}\circ\varphi_{\alpha}^{\minus{1}}$
            are smooth.
        \end{definition}
        The notion of smoothness here is simply the smoothness of functions
        from $\mathbb{R}^{n}$ to itself. The collection $\mathcal{A}$ is called
        an atlas and $(\mathcal{U},\varphi)$ are called charts.
        \begin{example}
            If $\mathcal{U}$ is an open subset of $\mathbb{R}^{n}$ and
            $f:\mathcal{U}\rightarrow\mathbb{R}$ is smooth, then the
            graph of $f$ is a smooth manifold.
        \end{example}
        \begin{example}
            If $\mathcal{U}\subseteq\mathbb{R}^{n}$ is open, if
            $f:\mathcal{U}\rightarrow\mathbb{R}$ is smooth, if $c$ is in the
            range of $f$, and if $\textrm{grad}(f)$ is non-zero on all of
            $f^{\minus{1}}(c)$, then $f^{\minus{1}}(c)$ is a smooth manifold.
        \end{example}
        \begin{example}
            Define $f:\mathbb{R}^{3}\rightarrow\mathbb{R}$ by
            $f(\mathbf{x})=\norm{\mathbf{x}}_{2}^{2}$. For all
            $\mathbf{x}\in\mathbb{R}^{3}$ such that $\norm{\mathbf{x}}=1$ we
            have $\textrm{grad}(f)\ne{0}$, and thus $f^{\minus{1}}(\{1\})$ is a
            smooth manifold. This is the unit sphere.
        \end{example}
        \begin{example}
            The orthographic projection of the sphere maps:
            \begin{equation}
                (x,y)\mapsto(x,y,\sqrt{1-x^{2}-y^{2}})
            \end{equation}
            The stereographic projection also exists.
        \end{example}
        \begin{definition}
            The product manifold of manifolds $M_{1},\hdots,M_{k}$, where
            $M_{j}$ is an $n_{j}$ smooth manifold then
            $M=\prod_{j=1}^{k}M_{j}$ has a natural structure of an
            $n=n_{1}+\dots+n_{k}$ dimensional manifold.
        \end{definition}
        \begin{definition}
            The quotient manifold of a smooth manifold $M$ with an equivalence
            relation $\sim$ is a manifold structure on $M/\sim$ with the
            quotient topology. In particular when we have group actions.
        \end{definition}
        \begin{example}
            Consider the sphere $S^{2}$ with the equivalence relation $R$
            defined by $pRq$ if and only if $p=\pm{q}$. The quotient space
            $S^{2}/R$ is the real projective plane $\mathbb{RP}^{2}$. We can
            consider this by means of a group action on $\mathbb{Z}_{2}$ on
            $S^{2}$. Define $g\cdot{p}$ by $1\cdot{p}=p$ and
            $\minus{1}\cdot{p}=\minus{p}$. This is a group action of
            $\mathbb{Z}_{2}$ on $S^{2}$.
        \end{example}
        All of the examples thus far have been subsets of some Euclidean space
        $\mathbb{R}^{n}$. One natural question is whether or not there are
        smooth manifolds that do no live in some higher dimensional Euclidean
        space. That is, are there manifolds $M$ such that $M$ can not be
        embedded into some $\mathbb{R}^{m}$?
        \begin{definition}
            A function $\phi:M\rightarrow{N}$ is smooth if for all $p\in{M}$
            there is a chart $(\mathcal{U},\varphi)$ containing $p$ and a
            chart $(\mathcal{V},\psi)$ containing $\phi(p)$ such that
            $\psi\circ\phi\circ\varphi^{\minus{1}}$ is smooth.
        \end{definition}
        Studying maps between manifolds gives us new manifolds to study. In
        particular, there are submersions and embeddings, and in particular
        embedded submanifolds.
        \par\hfill\par
        Tangent spaces are another important topic in differential topology. The
        classic example is that of a sphere $S^{2}$ in $\mathbb{R}^{3}$. We draw
        the tangent plane to a point on a sphere, which is the best linear
        approximation of the sphere at that point. This notion relies on an
        ambient space (that of $\mathbb{R}^{3}$), but since we do not yet know
        if all manifolds can be embedded into such an ambient space, we need a
        new means of defining tangent spaces that agrees with our intuition.
        There are several ways of thinking of this:
        \begin{itemize}
            \item Tangent vectors are derivations on $C^{\infty}(p)$.
            \item Tangent vectors are equivalence classes of curves through $p$.
        \end{itemize}
        Given a function $\phi:M\rightarrow{N}$ that is smooth between two
        manifolds, there is a function $\diff\phi_{o}:T_{p}M\rightarrow{T}_{p}N$
        between these tangent spaces. We can further consider the collection of
        all tangent spaces at all points $p$ of $M$, forming the tangent bundle
        of $M$, denoted $TM$. Set theoretically, this is the disjoint union
        $TM=\coprod_{p}T_{p}M$, but we do \textbf{not} give this the disjoint
        union topology. There is a natural topology that can be endowed on $TM$.
        \begin{example}
            If we take $S^{1}$, the tangent bundle is
            $TM=S^{1}\times\mathbb{R}$. This is an example of a trivial bundle.
            Most tangent bundles are not of this form. That is, for an $m$
            dimensional manifold $M$, $TM$ is usual \textbf{not} equal to
            $M\times\mathbb{R}^{m}$. It will always have dimension $2m$.
        \end{example}
    \section{A Review of Topology}
        \begin{definition}
            Let $X$ be a set. A topology on $X$ is a collection $\tau$
            consisting of subsets of $X$ such that $\emptyset\in\tau$,
            $X\in\tau$, and $\tau$ is closed to arbitrary unions and finite
            intersections. The elements of $\tau$ are called the open sets of
            $X$. $(X,\tau)$ is called a topological space. The closed subsets
            of $X$ are sets of the form $X\setminus\mathcal{U}$, where
            $\mathcal{U}$ is open. That is, a set is closed if its complement is
            open.
        \end{definition}
        \begin{example}
            The discrete topology $\mathcal{P}(X)$, the power set on $X$, is
            always a topology. So is $\{\emptyset,X\}$, the trivial topology.
        \end{example}
        \begin{example}
            If $(X,d)$ is a metric space, and if $\mathcal{U}\subseteq{X}$ is
            such that for all $x\in\mathcal{U}$ there is an $\varepsilon>0$
            such that $B_{\varepsilon}(x)\subseteq\mathcal{U}$, then we call
            $\mathcal{U}$ open. The collection of all such sets forms a topology
            on $X$, called the metric topology.
        \end{example}
        \begin{definition}
            The interior of a subset $S\subseteq{X}$ of a topology space
            $(X,\tau)$ is the set $\textrm{Int}(X)$ defined to by the
            set of all $s\in{S}$ such that there is an open set
            $\mathcal{U}\subseteq{X}$ such that $s\in\mathcal{U}$ and
            $\mathcal{U}\subseteq{S}$.
        \end{definition}
        \begin{definition}
            The closure of a subset $S\subseteq{X}$ in a topological space
            $(X,\tau)$ is the intersection of all closed sets in $X$ that
            contain $S$, denoted $\textrm{Cl}(S)$.
        \end{definition}
        \begin{definition}
            A subset subset of $X$ is a set $S$ such that
            $\textrm{Cl}(S)=X$.
        \end{definition}
        \begin{definition}
            A continuous function is a function $f:X\rightarrow{Y}$ such that
            for all open $\mathcal{V}\subseteq{Y}$, the pre-image
            $f^{\minus{1}}(\mathcal{V})$ is open in $X$.
        \end{definition}
        \begin{definition}
            A homeomorphism is a bijective continuous function
            $f:X\rightarrow{Y}$ such that $f^{\minus{1}}$ is continuous.
        \end{definition}
        \begin{definition}
            A compact space is such that every open cover has a finite subcover.
        \end{definition}
        \begin{definition}
            A connected topological space is such that it can't be written as
            the non-trivial union of disjoint open subsets.
        \end{definition}
        \begin{definition}
            Path connected, for all $x,y\in{X}$ there is a continuous function
            $\gamma:[0,1]\rightarrow{X}$ such that $\gamma(0)=x$ and
            $\gamma(1)=y$.
        \end{definition}
        \begin{definition}
            A Hausdorff space, for all $x,y\in{X}$ there are disjoint open
            $\mathcal{U},\mathcal{V}$ such that $x\in\mathcal{U}$ and
            $y\in\mathcal{V}$.
        \end{definition}
        \begin{definition}
            A basis, collection $\mathscr{B}$ such that for all
            $\mathcal{U}\in\tau$, $\tau$ can be written as the union of the
            elements of $\mathscr{B}$.
        \end{definition}
        \subsection{New Spaces from Old}
            \begin{definition}
                Given a topological space $(X,\tau)$, and a subset
                $A\subseteq{X}$, the subspace topology is the topology
                $\tau_{A}$ defined by
                $\tau_{A}=\{A\cap\mathcal{U}|\mathcal{U}\in\tau\}$.
            \end{definition}
            \begin{definition}
                The product of spaces is a thing.
            \end{definition}
            \begin{definition}
                The quotient space of $(X,\tau)$ with respect to a set $Y$ and
                a surjective map $\pi:X\rightarrow{Y}$ is the topology generated
                by $\pi$ such that it is continuous.
            \end{definition}
            More generally, let $X$ be a topological space and $G$ a topological
            group with a product $G\times{X}\rightarrow{X}$ satisfying:
            \begin{align}
                (g_{1}*{g}_{2})\cdot{x}&=g_{1}\cdot(g_{2}\cdot{x})\\
                e*x&=x
            \end{align}
            We can define an equivalence relation $R$ by $pRq$ if there is a
            $g\in{G}$ such that $p=g\cdot{q}$. This gives us a quotient space.
            \begin{example}
                Consider $X=S^{2}$, the sphere, with $G=S^{1}$ as a group.
                Then $X/G$ is simply the unit interval.
            \end{example}
            Draw this out, the orbits are lattitudinal lines, plus the two
            poles.
        \subsection{Quotient Maps}
            Let $\pi:X\rightarrow{Y}$ be a quotient map (surjective and
            continuous). A set $C\subseteq{Y}$ is closed if and only if
            $\pi^{\minus{1}}(C)$ is closed in $X$. The composition of quotient
            maps is a quotient map. If $F$ is continuous, $F\circ\pi$ from $X$
            to $B$ is continuous.
    \section{Topological Manifold}
        \begin{definition}
            A topological space is said to be an $n$ dimensional topological
            manifold if the following hold:
            \begin{itemize}
                \item $M$ is Hausdorff.
                \item $M$ is second countable.
                \item $M$ is locally Euclidean of dimension $n$. For all
                      $p\in{M}$ there is an open set $\mathcal{U}$ and a
                      homeomorphism $\varphi:\mathcal{U}\rightarrow\mathcal{V}$
                      to an open subset $\mathcal{V}\subseteq\mathbb{R}^{n}$.
            \end{itemize}
            The pair $(\mathcal{U},\varphi)$ is called a chart. If the image of
            $\mathcal{U}$ by $\varphi$ is an open ball in $\mathbb{R}^{n}$ we
            call this a coordinate ball.
        \end{definition}
        \begin{example}
            Open subsets of $\mathbb{R}^{n}$ are $n$ dimensional manifolds.
            Moreover, any open subset of an $n$ dimensional manifold, given the
            subspace topology, will again be an $n$ dimensional manifold.
        \end{example}
        \begin{example}
            The graphs of continuous functions $f:X\rightarrow{Y}$ from
            topological manifolds of dimension $n$ and $m$, repsectively, are
            manifolds. That is, if $\mathcal{U}\subseteq{X}$ is open, and
            $f|_{\mathcal{U}}$ is the restriction of $f$, then
            $\{(p,f(p)|p\in\mathcal{U}\}$, equipped with the subspace topology
            induced by the product topology on $X\times{Y}$, is a topological
            manifold of dimesnion $n$. It is Hausdorff and second countable
            since it is the subspace of such spaces.
        \end{example}
        \subsection{Review of Previous Lecture}
            \subsubsection{Subspace Topology}
                Let $X$ be a topological space with topology $\tau$. For any
                $S\subseteq{X}$ let $\tau_{S}$ be the subspace topology:
                \begin{equation}
                    \tau_{S}=\{\mathcal{U}\cap{S}\;|\;\mathcal{U}\in\tau\}
                \end{equation}
                Then $\tau_{S}$ is a topology on $S$, called the subspace
                topology. The restriction of a continuous map
                $f:X\rightarrow{Y}$ to $S$ is continuous with respect to the
                subspace topology on $f(S)\subseteq{Y}$. If $f|_{S}$ is
                continuous, then $f:X\rightarrow{f}(X)$ is continuous with the
                subspace topology.
            \subsubsection{Basis for a Topology}
                Let $\mathscr{B}$ be a bsis for a topology on $X$. Then:
                \begin{equation}
                    \tau(\mathscr{B})=\{\mathcal{U}\subseteq{X}\;|\;
                        \forall_{x\in\mathcal{U}}\exists_B\in\mathscr{B}:
                        x\in{B}\subseteq\mathcal{U}\}
                \end{equation}
                is a topology on $X$, and this is the topology generated by
                $\mathscr{B}$. If $X_{1},\dots,X_{n}$ are topological spaces,
                $X=\prod{X}_{k}$ is the product space generated by the
                \textit{open rectangles} in the $X_{i}$.
            \subsection{Continuous Maps and Products}
                Let $X$, $Y_{1},\dots,Y_{n}$ be topological spaces, and let
                $Y=\prod_{k}Y_{k}$ be the product topological space. For each
                $k$ let $\pi_{k}:Y\rightarrow{Y}_{k}$ be the projection mapping
                sending $\mathbf{y}\in{Y}$ to $y_{k}$, then $k^{th}$ component
                of $\mathbf{y}$. Then $f:X\rightarrow{Y}$ is continuous if and
                only if $\pi_{k}\circ{f}$ is continuous for each $k$. Going the
                other way, if $X=\prod_{k}X_{k}$ and $F:X\rightarrow{Y}$,
                $F=f_{1}\times\cdots\times{f}_{n}$, then it is continuous if and
                only if $f_{k}$ is continuous for all $k$. Let
                $f:X\rightarrow{Y}$ be a continuous function. Note that from the
                definition of a function, $f\subseteq{X}\times{Y}$. Thus we can
                endow $f$ with the subspace topology on the product topology of
                $X\times{Y}$. This is occasionally called the \textit{graph} of
                $f$. If $X$ and $Y$ are Hausdorff and second countable, then
                $X\times{Y}$ is, and thus any subspace of $X\times{Y}$ is
                also Hausdorff and second countable. Hence the graph of $f$ is
                a second countable Hausdorff topological space.
                \begin{example}
                    The $n$ sphere $S^{n}$ is the subset of $\mathbb{R}^{n+1}$
                    such that $\norm{\mathbf{x}}_{2}=1$, equipped with the
                    subspace topology. We can make it into a manifold with the
                    charts $(\mathcal{U}_{k}^{\pm},\phi_{k}^{\pm})$ where
                    $\mathcal{U}_{k}^{\pm}$ is the $k^{th}$ upper (or lower)
                    hemisphere, and $\phi_{k}^{\pm}$ is simply the projection
                    mapping onto the hyperplane $\mathbb{R}^{n}$. We can also
                    cover this with two simpler coordinate charts, the
                    two stereographic projections about the north and south
                    pole.
                \end{example}
                \begin{example}
                    The real projective space $\mathbb{RP}^{n}$ is another
                    manifold. For any $x,y\in\mathbb{R}^{n+1}\setminus\{0\}$,
                    we write $xRy$ if there is a $t\in\mathbb{R}$ such that
                    $x=ty$. That is, $x$ and why are equivalent if they lie on
                    the same line through the origin. We define
                    $\mathbb{RP}^{n}$ by $\mathbb{R}^{n+1}\setminus\{0\}/R$,
                    equipped with the quotient topology. That is, a subset is
                    open if and only if $\pi^{\minus{1}}(\mathcal{U})$ is open
                    where $\pi$ is the quotient map (the natural projection).
                    For any $k=1,\dots,n$, let $\tilde{\mathcal{U}}_{j}$ be the
                    $(x_{1},\dots,x_{n+1})\in\mathbb{R}^{n+1}\setminus\{0\}$
                    such that $x_{j}\ne{0}$. This is an open subset and it is
                    saturated with respect to $\pi$. That is, for all
                    $[x]\in\mathbb{RP}^{n}$,
                    $\pi^{\minus{1}}([x])\cap\tilde{\mathcal{U}}_{j}\ne\emptyset$
                    if and only if $\pi^{\minus{1}}([x])\subseteq\tilde{\mathcal{U}}_{j}$.
                    Now let $\mathcal{U}=\pi(\tilde{\mathcal{U}}_{j})$. Then
                    since $\tilde{\mathcal{U}}_{j}$ is saturated it is equal
                    to $\pi^{\minus{1}}(\mathcal{U}_{j})$ and is therefore
                    open in the quotient topology. Define $\varphi_{j}$ by:
                    \begin{equation}
                        \varphi_{j}\big([(x_{1},\dots,x_{n+1})]\big)
                        =\big(\frac{x_{1}}{x_{j}},\dots,
                            \frac{x_{j-1}}{x_{j}},\frac{x_{j+1}}{x_{j}},\dots,
                            \frac{x_{n+1}}{x_{j}}\big)
                    \end{equation}
                    Then this map commutes with $\pi$ with $\tilde{\varphi}$:
                    $\tilde{\varphi}=\varphi\circ\pi$.
                \end{example}
    \section{Topological Groups}
        \begin{definition}
            A topological Lie group is a group $(G,*)$ with a topology $\tau$ on
            $G$ such that $\nu:G\rightarrow{G}$ defined by
            $\nu(g)=g^{\minus{1}}$ is continuous $*:G\times{G}\rightarrow{G}$ is
            continuous in the product topology, and $(G,\tau)$ is a topological
            manifold.
        \end{definition}
        \begin{theorem}
            $GL_{n}(\mathbb{R})$ is a topological Lie group of dimension
            $n^{2}$.
        \end{theorem}
        \begin{proof}
            For we have:
            \begin{equation}
                GL_{n}(\mathbb{R})=
                \textrm{det}^{\minus{1}}(\mathbb{R}\setminus\{0\})
            \end{equation}
            and since $\textrm{det}$ is continuous (it's a polynomial), it's
            thus an open subset of $\mathbb{R}^{n^{2}}$ (we can identify
            $n\times{n}$ matrices with points in $\mathbb{R}^{n^{2}}$). Moreover
            multiplication is continuous since it's continuous in every slot,
            it's a polynomial. The inverse function
        \end{proof}
    \section{More on Topological Manifolds}
        \begin{definition}
            A precompact subset of a topological space $(X,\tau)$ is a subset
            $\mathcal{U}\subseteq{X}$ such that $\textrm{Cl}(\mathcal{U})$
            is compact.
        \end{definition}
        \begin{theorem}
            Every topological $n$ manifold has a countable basis of precompact
            coordinate balls.
        \end{theorem}
        \begin{proof}
            Let $(X,\tau)$ be a topological manifold and
            $\{(\mathcal{U}_{\alpha},\varphi_{\alpha})\}$ a cover by coordinate
            charts. Since second countable spaces are Lindelof, there is a
            countable subcover. Since each $\varphi_{k}$ is an open mapping,
            $\varphi_{k}(\mathcal{U}_{k})$ is an open subset of
            $\mathbb{R}^{n}$.
        \end{proof}
        \begin{definition}
            A locally path connected topological space is a topological space
            $(X,\tau)$ such that there exists a basis of path connected open
            sets.
        \end{definition}
        \begin{definition}
            A locally compact topological space is such that for all $x\in{X}$
            there is an open $\mathcal{U}\subseteq{X}$ and a compact
            $K\subseteq{X}$ such that $x\in\mathcal{U}$ and
            $\mathcal{U}\subseteq{K}$.
        \end{definition}
        \begin{definition}
            A connected component of $X$ is a maximal connected subset of $X$.
        \end{definition}
        Connected components partition the space. If two connected sets have
        a point in common, their union is connected.
        \begin{definition}
            A path connected component s a maximal path component.
        \end{definition}
        \begin{theorem}
            If $(X,\tau)$ is a topological manifold, then $M$ is locally path
            connected, locally compact, and the path components and connected
            components are identical. Moreover, $M$ has countably many connected
            components, each of which is open.
        \end{theorem}
        \begin{proof}
            Locally path connected and connected implies path connected, hence
            connected components and path connected components are the same.
            Locally compact since compact coordinate balls about each point
            suffice. Countably many connected components since second countable.
        \end{proof}
        \begin{definition}
            A locally finite subset $\mathcal{O}\subseteq\mathcal{P}(X)$ is a
            collection such that for all $x\in{X}$ there is a neighborhood
            $\mathcal{U}\in\tau$ such that $\mathcal{U}$ intersects at most
            finitely many elements of $\mathcal{O}$.
        \end{definition}
        \begin{definition}
            A refinitement of a collection $\mathcal{O}\subseteq\mathcal{P}(X)$
            is a collection $\mathcal{D}\subseteq\mathcal{P}(X)$ such that for
            all $\mathcal{V}\in\mathcal{D}$ there is a
            $\mathcal{U}\in\mathcal{O}$ such that
            $\mathcal{V}\subseteq\mathcal{U}$.
        \end{definition}
        \begin{definition}
            A paracompact space is a space such that every open cover has a
            locally finite refinement.
        \end{definition}
        \begin{theorem}
            If $\mathcal{O}$ is a locally finite collection of subsets of $X$,
            then the collection of the closures is locally finite and the union
            of the closures is the closure of the unions.
        \end{theorem}
        \begin{theorem}
            If $(X,\tau)$ is a topological manifold, then it is paracompact.
        \end{theorem}
    \section{Smooth Manifolds}
        \begin{definition}
            A diffeomprhism from an open subset
            $\mathcal{U}\subseteq\mathbb{R}^{n}$ to an open subset
            $\mathcal{V}\subseteq\mathbb{R}^{n}$ is a homeomorphism
            $f:\mathcal{U}\rightarrow\mathcal{V}$ such that $f$ and
            $f^{\minus{1}}$ are smooth functions.
        \end{definition}
        \begin{definition}
            Let $M$ an $n$ dimensional topological manifold. Two coordinate
            charts $(\mathcal{U},\varphi)$ and $(\mathcal{V},\psi)$ are said to
            be $C^{\infty}$ compatible if $\psi\circ\phi^{\minus{1}}$,
            defined as a function from $\varphi(\mathcal{U}\cap\mathcal{V})$ to
            $\psi(\mathcal{U}\cap\mathcal{V})$, is a diffeomprhism.
        \end{definition}
        \begin{definition}
            Let $M$ be an $n$ dimensional topological manifold. A smooth atlas
            on $M$ is a collection $\mathcal{A}$ of charts
            $(\mathcal{U},\varphi)$ that cover $M$ and such that all charts
            are smoothly compatible. A smooth atlas $\mathcal{A}$ on $M$ is
            called maximal if it is not contained in any strictly larger
            smooth atlas. A smooth structure on $M$ is a maximal smooth atlas.
        \end{definition}
        \begin{definition}
            A smooth manifold, denoted $\manifold{M}$, is a topological manifold
            $\topspace{M}$ with a smooth structure $\mathcal{A}$.
        \end{definition}
        \begin{theorem}
            If $M$ is an $n$ dimensional topological manifold, and if
            $\mathcal{A}$ is a smooth atlas on $M$, then there is a unique
            maximal atlas $\mathcal{A}'$ such that
            $\mathcal{A}\subseteq\mathcal{A}'$.
        \end{theorem}
    \section{More on Smooth Manifolds}
        Given a smooth manifold $(X,\tau,\mathcal{A})$, $\mathcal{A}$ a maximal
        smooth atlas, a chart $(\mathcal{U},\varphi)\in\mathcal{A}$ is called
        smooth, $\mathcal{U}$ is called a coordinate domain, and $\varphi$ is
        called a smooth coordinate map. This chart is called a coordinate ball
        if $\varphi(\mathcal{U})$ is an open ball. Similarly, it's called a
        coordinate cube if $\varphi(\mathcal{U})$ is an open cube. A chart is
        called regular if it is a smooth coordinate ball and there is a
        coordinate ball $(\mathcal{V},\psi)$ such that
        $\varphi(\mathcal{U})\subseteq\psi(\mathcal{V})$ with the same center
        and different radii.
        \begin{example}
            A zero dimensional manifold is just a countable collection of
            isolated points.
        \end{example}
        \begin{example}
            For any $n\in\mathbb{N}$, $\mathbb{R}^{n}$ with the identity mapping
            $\textrm{id}_{\mathbb{R}^{n}}$ forms an atlas, and thus lives inside
            a unique maximal smooth atlas.
        \end{example}
        \begin{example}
            We can put another non-compatible smooth structure on
            $\mathbb{R}^{n}$. Let $\mathbb{R}$ be taken with the mapping
            $f(x)=x^{3}$. Then $(\mathbb{R},f)$ forms an atlas since $f$ is a
            homemorphism. It overlaps smoothly with every other element since
            there's only element and $f\circ{f}^{\minus{1}}$ is just the
            identity mapping, which is smooth and has smooth inverse. However,
            this atlas and the standard atlas aren't compatible since the
            composition is $x^{1/3}$, which is not differentiable at the origin.
            However, there is a diffeomprhism from the standard atlas to this
            one, and hence the two maximal atlases are essentially the same.
        \end{example}
        \begin{example}
            Consider the space of $m\times{n}$ matrices over both $\mathbb{R}$
            and $\mathbb{C}$. These are smooth manifolds of dimensions
            $m\cdot{n}$ and $2\cdot{m}\cdot{n}$, respectively. From this,
            $GL_{n}(\mathbb{R})$ is an $n^{2}$ matrix since it is an open subset
            of $\mathbb{R}^{n^{2}}$. It is the complement of the pre-image of
            $\textrm{det}(0)$. Since the determinant is continuous, and points
            are closed, the pre-image is closed. Hence the complmement is open.
            Thus $\textrm{GL}_{n}(\mathbb{R})$ is an open subset of a smooth
            manifold, and hence is a smooth manifold itself.
        \end{example}
    \section{Even More on Smooth Manifolds}
        An $n$ dimensional manifold is a topological manifold $(M,\tau)$ with a
        smooth differentiable structure $\mathcal{A}$. In particular, for charts
        $(\mathcal{U},\varphi)$ and $\mathcal{V},\psi)$ such that
        $\mathcal{U}\cap\mathcal{V}\ne\emptyset$, then
        $\varphi\circ\psi^{\minus{1}}$ is a diffeomprhism from an open subset of
        $\mathbb{R}^{n}$ to another open subset of $\mathbb{R}^{n}$. A function
        $\phi:M\rightarrow\mathbb{R}$ is smooth if for all $p\in{M}$ there is a
        chart $(\mathcal{U},\varphi)\in\mathcal{A}$ such that
        $\phi\circ\varphi^{\minus{1}}$ is a smooth function from a subset
        $\mathbb{R}^{n}$ to $\mathbb{R}$.
        \begin{example}
            Matrices of full rank form a smooth manifold. Let $m<n$ and let
            $\textrm{Mat}_{m\times{n}}^{F}$ be the set of all $m\times{n}$
            matrices $A$ such that $A$ has full rank $m$. For if $A$ has full
            rank, then there is an $m\times{m}$ submatrix $B$ such that
            $\textrm{det}(B)\ne{0}$. That is, $A=[B|C]$, where $C$ is some
            $m\times{n-m}$ matrix. Since the determinant is continuous, there
            is an open neighborhood about $B$, call it $\mathcal{U}$, such that
            none of the elements have zero determinant. Let $\mathcal{V}$ be
            the set of all $m\times(n-m)$ matrices. Then $A$ is an element of
            $\mathcal{U}\times\mathcal{V}$, which is the product of open, and
            hence open in the product topology. Thus
            $\textrm{Mat}_{n\times{M}}^{F}$ is an open subset of the entire
            space, and is hence a smooth manifold.
        \end{example}
        \begin{example}
            Given an open subset $\mathcal{U}\subseteq\mathbb{R}^{n}$ and a
            smooth function $f:\mathcal{U}\rightarrow\mathbb{R}$, the graph of
            $f$, as a subset of $\mathbb{R}^{n}\times\mathbb{R}$, is a smooth
            manifold.
        \end{example}
        \begin{theorem}
            If $\mathcal{U}\subseteq\mathbb{R}^{n}\times\mathbb{R}^{k}$ is open,
            if $\varphi:\mathcal{U}\rightarrow\mathbb{R}^{k}$ is a continuously
            differentiable function, $(a,b)\in\mathcal{U}$ and
            $c=\varphi(a,b)$, then the implicit function theorem.
        \end{theorem}
        \begin{example}
            Let $\mathcal{U}\subseteq\mathbb{R}^{n}$ be open, and
            $\varphi\mathcal{U}\rightarrow\mathbb{R}$ a smooth function.
            Level sets are manifolds when the implicit function theorem applies.
        \end{example}
        \begin{example}
            Let $V$ be an $n$ dimensional real vector space,
            $k\in\mathbb{Z}_{n}$ such that $k\ne{0}$, and let $G_{k}(V)$ be the
            set of all $k$ dimensional subspaces of $V$. Then this is a smooth
            manifold of dimension $k(n-k)$ with a certain topology and smooth
            structure.
        \end{example}
    \section{Grassmannian}
        \begin{theorem}[Smooth Manifold Chart Lemma]
            Let $M$ be a set, $\{\mathcal{U}_{\alpha}\}$ a collection of subsets
            of $M$, with a collection of maps
            $\varphi_{\alpha}:M\rightarrow\mathbb{R}^{n}$ such that for all
            $\alpha\in{J}$ there is injection and
            $\varphi_{\alpha}(\mathcal{U}_{\alpha})$ is open, for all
            $\alpha,\beta$ such that
            $\mathcal{U}_{\alpha}\cap\mathcal{U}_{\beta}\ne\emptyset$ it is true
            that
            $\varphi_{\alpha}()\mathcal{U}_{\alpha}\cap\mathcal{U}_{\beta})$ is
            open in $\mathbb{R}^{n}$ and
            $\varphi_{\beta}\circ\varphi_{\alpha}^{\minus{1}}$ is smooth,
            countably many $\mathcal{U}_{\alpha}$ cover $M$, and for all
            $p\ne{q}$ then either $p,q\in\mathcal{U}_{\alpha}$ for some $\alpha$
            or there are disjoint $\mathcal{U}_{\alpha}$, $\mathcal{U}_{\beta}$
            such that $p\in\mathcal{U}_{\alpha}$ an
            $q\in\mathcal{U}_{\beta}$.
        \end{theorem}
        \begin{example}
            Let $V$ be an $n$ dimesnional real vector space over $\mathbb{R}$,
            $1\leq{k}\leq{n}$ and integer, and let $G_{k}(X)$ be the set of
            all $k$ dimesnional subspaces of $V$. Using the smooth manifold
            chart lemma $G_{k}(V)$ can be given the structure of a $C^{\infty}$
            manifold of dimension $k(n-k)$. There is a natural collection
            $\{(\mathcal{U}_{\alpha},\varphi_{\alpha})\}$ where
            $\mathcal{U}_{\alpha}\subseteq{G}_{k}(V)$, $G_{k}(V)$ is covered by
            such sets, and $\mathcal{U}_{\alpha}$ is in bijection with
            $\mathbb{R}^{k(n-k)}$. Let $P\in{G}_{k}(V)$ and $Q$ the complemetary
            subspace of $P$ in $V$. Then $V=P\oplus{Q}$ and the dimension of $Q$
            is $n-k$. Let $\textrm{Hom}(P,Q)$ be the space of linear
            transformations from $P$ to $Q$. Then $\textrm{Hom}(P,Q)$ is equal
            to the space of real $(n-k)\times{k}$ matrices, which is
            homeomorphic as a topological space to $\mathbb{R}^{k(n-k)}$. Let
            $\mathcal{U}_{Q}=\{W\in{G}_{k}(V)|W\cap{Q}=\{0\}\}$. Then
            $P\in\mathcal{U}_{Q}$. Let $L\in\textrm{Hom}(P,Q)$, then the graph
            of $L$ is a $k$ dimensional subspace of $V$ with trivial
            intersection with $Q$ and all such subspaces of $V$ arise as some
            graph of $L\in\textrm{Hom}(P,Q)$. But since $V=P\oplus{Q}$, we have
            that $\pi_{P}:V\rightarrow{P}$ and $\pi_{Q}:V\rightarrow{Q}$. Let
            $W\in\mathcal{U}_{Q}$. Then $\pi_{P}:W\rightarrow{P}$ is an
            isomorphism and $L=\pi_{Q}\circ\pi^{\minus{1}}:P\rightarrow{Q}$.
        \end{example}
    \section{Manifolds with Boundary}
        Thus far we have been looking at universes without walls or edges. For
        example, the torus, sphere, real projective spaces, and open subsets of
        $\mathbb{R}^{n}$. However, we can consider subsets of $\mathbb{R}^{2}$
        such as the closed unit disc. The interior is homeomorphic to all of
        $\mathbb{R}^{2}$< but the bounding circle makes the entire space not
        open to any open subset since it is compact and no open subset of
        $\mathbb{R}^{n}$ can be compact. However, the points on the bounding
        circle are homeomorphic to the closed half plane in $\mathbb{R}^{2}$.
        This gives rise to the notion of a smooth manifold with boundary.
        Let $\mathbb{H}^{n}$ be the set of all $\mathbb{x}\in\mathbb{R}^{n}$
        such that $x_{n}\geq{0}$.
        \begin{definition}
            A manifold with boundary is a second countable Hausdorff topological
            space such that every point $x$ has an open neighborhood about it
            that is homeomorphic to an open subset of $\mathbb{R}^{n}$ or a
            relatively open subset of $\mathbb{H}^{n}$.
        \end{definition}
        \begin{definition}
            A boundary point is a point on the boundary.
        \end{definition}
        \begin{theorem}
            If $M$ is a topological manifold with boundary, then every point is
            either an interior point or a boundary point.
        \end{theorem}
        \begin{definition}
            A closed manifold is a compact manifold without boundary.
        \end{definition}
        \begin{theorem}
            If $M$ is a topological manifold with boundary then the interior is
            an open subset of $M$ is an open subset and the boundary is a
            closed subset.
        \end{theorem}
        \begin{theorem}
            If $M$ is a topological manifold with boundary, then $M$ has a
            countable basis of precompact coordinate balls and half balls.
            $M$ is locally compact. $M$ is locally path connected. $M$ has
            countably many components, each of which is open and connected.
            Moreover, $\pi_{1}(M)$ is countable.
        \end{theorem}
    \section{Smooth Manifolds with Boundary}
        First we defined smoothness on arbitrary subsets of $\mathbb{R}^{n}$.
        Given $A\subseteq\mathbb{R}^{n}$ and $f:A\rightarrow\mathbb{R}^{k}$ we
        say that $f$ is smooth if for all $p\in{A}$ there is an open subset
        $\mathcal{U}\subseteq\mathbb{R}^{n}$ that contains $p$ and a smooth
        function $F:\mathcal{U}\rightarrow\mathbb{R}^{k}$ such that
        $F|_{\mathcal{U}\cap{A}}=f|_{\mathcal{U}\cap{A}}$. Let $\mathcal{U}$ be
        a subset of $\mathbb{H}^{n}$ that is open in the subspace topology. Then
        $F:\mathcal{U}\rightarrow\mathbb{R}^{k}$ is smooth if there is an open
        subset $\tilde{\mathcal{U}}\subseteq\mathbb{R}^{n}$ (in the usual
        Euclidean topology) such that $p\in\tilde{\mathcal{U}}$ and a smooth
        function $\tilde{F}:\tilde{\mathcal{U}}\rightarrow\mathbb{R}^{k}$ such
        that $F$ and $f$ are identical on $\mathcal{U}\cap\tilde{\mathcal{U}}$.
        For such a smooth function $F:\mathcal{U}\rightarrow\mathbb{R}^{k}$ we
        see that for the \textit{interior} points, $F$ is smooth in the usual
        sense. By the continuity of $F$ on such points, all of the values of
        $\tilde{F}$ on the boundary points are determined uniquely. That is,
        we can take a sequence of points $a:\mathbb{N}\rightarrow\mathbb{U}$
        that converge to a point on $\partial\mathbb{H}^{n}$, and by the
        continuity of all of the partial derivatives, of all orders, of
        $\tilde{F}$, we can evaluate $\tilde{F}$ and its derivatives by
        examining $\tilde{F}(a_{n})$ (and similary for partials). Thus any two
        extensions of $F$ are the same.
        \begin{example}
            Let $B^{2}\subseteq\mathbb{R}^{2}$ be the open unit disk and
            $\mathcal{U}=B^{2}\cap\mathbb{H}^{2}$. Let
            $f:\mathcal{U}\rightarrow\mathbb{R}$ be given by
            $f(x,y)=\sqrt{1-x^{2}-y^{2}}$. Then $f$ is smooth on $\mathcal{U}$
            since $\tilde{f}:B^{2}\rightarrow\mathbb{R}$ is smooth,
            $\tilde{f}(x,y)=\sqrt{1-x^{2}-y^{2}}$. On the other hand, the
            function $g:\mathcal{U}\rightarrow\mathbb{R}$ defined by
            $g(x,y)=\sqrt{y}$ does not have a smooth extension since
            $\partial{g}/\partial{y}=1/2\sqrt{y}$, and this tends to infinity as
            $y$ tends to zero.
        \end{example}
        \begin{definition}
            A smooth structure on a topological manifold with boundary $M$ is a
            maximal atlas $\mathcal{A}$ consisting of charts
            $(\mathcal{U},\varphi)$ with smooth overlap in the sense discussed
            above. $M$ equipped with such an atlas is called a smooth manifold
            with boundary. Here we've fixed the dimension.
        \end{definition}
        We previously said that a product of smooth manifolds without boundary
        has a natural smooth structure of a manifold without boundary. It then
        becomes natural to ask about the Cartesian product of two manifolds
        with non-empty boundary.\
        \begin{theorem}
            If $M_{1},\dots,M_{k}$ are $C^{\infty}$ manifolds without boundary
            and $N$ is a $C^{\infty}$ manifold with boundary, then the product
            is a $C^{\infty}$ manifold with boundary and the boundary is
            the product of $M_{1},\dots,M_{k}$ with $\partial{N}$.
        \end{theorem}
        \begin{theorem}[Smooth Invariance of the Boundary]
            If $M$ is a smooth manifold with boundary, if $p\in{M}$, and if
            $(\mathcal{U},\varphi)$ is a smooth structure such that
            $\varphi[\mathcal{U}]\subseteq\mathbb{H}^{n}$ and
            $\varphi(p)\in\partial\mathbb{H}^{n}$, then the same is true for any
            smooth chart containing $p$.
        \end{theorem}
        \begin{proof}
            For by the inverse function theorem, suppose
            $\mathcal{U},\mathcal{V}\subseteq\mathbb{R}^{n}$ are open subsets
            and $f:\mathcal{U}\rightarrow\mathcal{V}$ is a smooth function. If
            $p\in\mathcal{U}$ is such that $DF_{p}$ is a non-singular matrix
            then there are neighborhoods $\mathcal{U}_{0}\subseteq\mathcal{U}$
            which contain $p$ and $\mathcal{V}_{0}\subseteq\mathcal{V}$ which
            contain $F(p)$ such that $F|_{\mathcal{U}_{0}}$ is a diffeomprhism
            from $\mathcal{U}_{0}$ to $\mathcal{V}_{0}$. If this is true for
            every such point in $\mathcal{U}$, then $f$ is an open mapping. If
            $f$ is injective, then it is a diffeomprhism. Now, suppose to the
            contrary that $p\in{M}$ is such that there exists an interior chart
            and a boundary chart both containing $p$ and such that the image
            of $p$ lies on the boundary of $\mathbb{H}^{n}$.
        \end{proof}
    \section{Diffeomorphisms}
        \begin{definition}
            A smooth map from a manifold $M$ to $\mathbb{R}^{k}$ is a function
            $\phi:M\rightarrow\mathbb{R}^{k}$ such that for all $p\in{M}$ there
            exists a chart $(\mathcal{U},\varphi)$ such that $p\in\mathcal{U}$
            and $f\circ\varphi^{\minus{1}}$ is smooth.
        \end{definition}
        Since the charts in a smooth manifold are required to overlap smoothly,
        this definition of smoothness does not depend on the choice of chart.
        \begin{theorem}
            If $\manifold{M}$ is a smooth manifold, if
            $f:M\rightarrow\mathbb{R}^{k}$ is smooth, if $p\in{M}$, and if
            $(\mathcal{U},\varphi)\in\mathcal{A}$ is such that $p\in{M}$, then
            $f\circ\varphi^{\minus{1}}$ is smooth.
        \end{theorem}
        If $M$ is a smooth manifold with boundary, then the definition is more
        or less the same except in the case when $(\mathcal{U},\varphi)$ is a
        boundary chart, we require that $f\circ\varphi^{\minus{1}}$ to be able
        to extend to a smooth function.
        \begin{theorem}
            If $M$ and $N$ are smooth manifolds, $f:M\rightarrow{N}$ is smooth
            if and only if for all $p\in{M}$ there is a chart
            $(\mathcal{U},\varphi)$ containing $p$ and $(\mathcal{V},\psi)$
            containing $f(p)$ such that
            $\mathcal{U}\cap{f}^{\minus{1}}(\mathcal{V})$ is open and
            $\psi\circ{f}\circ\varphi^{\minus{1}}$ is smooth.
        \end{theorem}
        \begin{theorem}
            Every constant map is smooth. The identity map is smooth. The
            inclusion map from an open subset is smooth. The composition of
            smooth functions is smooth.
        \end{theorem}
        \begin{example}
            Let $S^{1}$ be equipped with it's usual smooth structure and
            $\phi:\mathbb{R}\rightarrow{S}^{1}$ be the map
            $\phi(\theta)=(\cos(\theta),\sin(\theta))$. We can generalize this
            to the $n$ torus, $\mathbb{T}^{n}$. Defined
            $f:\mathbb{R}^{n}\rightarrow\mathbb{T}^{n}$ to be:
            \begin{equation}
                f(\vector{\theta})=\exp(i2\pi\vector{\theta})
            \end{equation}
        \end{example}
        \begin{example}
            The inclusion map $\iota:S^{n}\rightarrow\mathbb{R}^{n+1}$ is
            smooth.
        \end{example}
        \begin{example}
            Given $\mathbb{R}^{n+1}\setminus\{0\}$ and the quotient map
            $q:\mathbb{R}^{n+1}\rightarrow\mathbb{RP}^{n}$, where
            $\mathbb{RP}^{n}$ is the real projective $n$ space, then $q$ is a
            smooth mapping.
        \end{example}
        \begin{definition}
            A diffeomprhism from a smooth manifold $\manifold[M]{M}$ to a smooth
            manifold $\manifold[N]{N}$ is a smooth bijective function
            $\phi:M\rightarrow{N}$ such that $\phi^{\minus{1}}$ is smooth.
        \end{definition}
        \begin{example}
            The open $n$ ball $\mathbb{B}^{n}$ is diffeomorphic to the entirety
            of $\mathbb{R}^{n}$. Use the mapping:
            \begin{equation}
                f(\vector{x})=
                    \frac{\vector{x}}{\sqrt{1-\norm{\vector{x}}_{2}^{2}}}
            \end{equation}
            This is smooth on $\mathbb{B}^{n}$, with smooth inverse:
            \begin{equation}
                f^{\minus{1}}(\vector{x})=
                    \frac{\vector{x}}{\sqrt{1+\norm{\vector{x}}_{2}^{2}}}
            \end{equation}
        \end{example}
        \begin{example}
            On $\mathbb{R}$ there are two atlases that have different maximal
            smooth atlases. Let
            $\mathcal{A}_{1}=\{()\mathbb{R},\textrm{id}_{\mathbb{R}})\}$ and
            $\mathcal{A}_{2}=\{(\mathbb{R},x^{3})\}$. However, the mapping
            $f(x)=x^{1/3}$ is a diffeomorphism.
        \end{example}
        \begin{theorem}
            If $M$ is a topological space that is covered by countably many
            $n$ dimensional coordinate charts, then $M$ is second countable.
        \end{theorem}
        \begin{theorem}
            If $M$ is a topological space covered by $n$ dimensional coordinate
            charts such that for any $p,q\in{M}$ there exists $\alpha\in{J}$
            such that $p,q\in\mathcal{U}_{\alpha}$ or there exists
            $\alpha\ne\beta\in{J}$ such that $p\in\mathcal{U}_{\alpha}$,
            $q\in\mathcal{U}_{\beta}$, and
            $\mathcal{U}_{\alpha}\cap\mathcal{U}_{\beta}=\emptyset$, then
            $M$ is Hausdorff.
        \end{theorem}
    \section{Partitions of Unity}
        \begin{theorem}
            The function $f:\mathbb{R}\rightarrow\mathbb{R}$ defined by:
            \begin{equation}
                f(t)=
                \begin{cases}
                    \exp\big(\frac{1}{t}\big),&t>0\\
                    0,&t\leq{0}
                \end{cases}
            \end{equation}
            is smooth.
        \end{theorem}
        \begin{proof}
            $f$ is smooth at all points $t\ne{0}$ and thus the only point left
            to check is $t=0$. Apply L'H\"{o}pital here.
        \end{proof}
        \begin{theorem}
            If $r_{1},r_{2}\in\mathbb{R}$ are real numbers such that
            $0<r_{1}$ and $r_{1}<r_{2}$, then there is a function
            $h:\mathbb{R}\rightarrow\mathbb{R}$ such that for all
            $t\in\mathbb{R}$ such that $t\leq{r}_{1}$ it is true that $h(t)=1$,
            for all $t\in(r_{1},t_{2})$ it is true that $h(t)\in(0,1)$, and
            for all $t\geq{r}_{2}$ it is true that $f(t)=0$.
        \end{theorem}
        \begin{proof}
            Let $f(t)$ be defined by:
            \begin{equation}
                f(t)=
                \begin{cases}
                    \exp\big(\frac{1}{t}\big),&t>0\\
                    0,&t\leq{0}
                \end{cases}
            \end{equation}
            define $h:\mathbb{R}\rightarrow\mathbb{R}$ by:
            \begin{equation}
                \frac{f(r_{2}-t)}{f(r_{2}-t)+f(t-r_{1})}
            \end{equation}
        \end{proof}
        \begin{theorem}
            If $0<r_{1}<r_{2}$ there is a bump function
            $H:\mathbb{R}^{n}\rightarrow\mathbb{R}$ such that:
            \begin{itemize}
                \item $H=1$ on $\closure[]{B_{r_{1}}(0)}$.
                \item $H=0$ on $\mathbb{R}^{n}\setminus{B}_{r_{2}}(0)$.
                \item $H$ is strictly between 0 and 1 in between these balls.
            \end{itemize}
        \end{theorem}
        \begin{proof}
            Let $H(\vector{x})=h(\norm{\vector{x}})$.
        \end{proof}
        \begin{definition}
            If $X$ is a topological space, $f:X\rightarrow\mathbb{R}^{n}$, the
            support is $\support{f}=\closure{\{x\in{X}\;|\;f(x)\ne{0}\}}$,
            compact support of $\support{f}$ is compact.
        \end{definition}
        \begin{definition}
            Let $X$ be a topological space and $\mathcal{O}$ be an open cover
            of $X$. A partition of unity subordinate to $\mathcal{O}$ is a
            collection of continuous functions
            $f_{\alpha}:X\rightarrow\mathbb{R}$ such that:
            \begin{itemize}
                \item $f_{\alpha}[X]\in[0,1]$
                \item $\support{f_{\alpha}}\in\mathcal{U}_{\alpha}$
                \item $\{\support{f_{\alpha}}\}$ is locally finite.
                \item $\sum{f}_{\alpha}=1$ for all points.
            \end{itemize}
        \end{definition}
        The last part of this definition is well defined since the support of
        all of the functions form a locally finite collection, and hence only
        finitely many such functions contribute to the sum at any point, and
        hence the sum is well defined everywhere. Smooth partition of unity is
        a partition of unity with smooth functions.
        \begin{ftheorem}{Existence of Smooth Partitions of Unity}
                        {Existence_of_Smooth_Partitions_of_Unity}
            If $M$ is a smooth manifold with boundary and if $\mathcal{O}$ is an
            open cover, then there exists a smooth partition of unity
            subordinate to $\mathcal{O}$.
        \end{ftheorem}
        \begin{proof}
            Firstly suppose the boundary is empty. For every
            $\mathcal{U}_{\alpha}\in\mathcal{O}$ there is a basis $B_{\alpha}$
            consisting of regular coordinate balls. Let $B$ be the union of all
            the $B_{\alpha}$. Then this is a basis for the topology on $M$. But
            $M$ is a smooth manifold and is hence paracompact, and thus there is
            a locally finite refinement of $B$. Moreover, there's a locally
            finite refinement consisting of elements of $B$. But then the
            collection of the closures of the $B_{j}$ will be locally finite.
        \end{proof}
        Application: We'll use this to show every manifold admits a Riemannian
        metric.
    \section{Reading From Lee (Topology Review)}
        Define topological space, open sets, neighborhoods, closed sets.
        \begin{fdefinition}{Interior Point}{Interior_Point}
            An interior point of a subset $A\subseteq{X}$ of a topological space
            $\topspace{X}$ is a point $p\in{A}$ such that there exists an open
            set $\mathcal{U}\in\tau$ such that $\mathcal{U}\subseteq{A}$ and
            $p\in\mathcal{U}$.
        \end{fdefinition}
        \begin{fdefinition}{Interior of Set}{Interior_of_Set}
            The interior of a subset $A\subseteq{X}$ of a topological
            $\topspace{X}$, denoted $\interior[\tau]{A}$ is the set of all
            interior points of $A$. That is:
            \begin{equation*}
                \interior[\tau]{A}=\{\,p\in{X}\;|\;p
                    \textrm{ is an interior point of }A\,\}
            \end{equation*}
        \end{fdefinition}
        \begin{theorem}
            If $\topspace{X}$ is a topological space, if
            $\tau_{A}$ is the set of all subset of $\tau$ that are subsets of
            in $A$, and if $\interior[\tau]{A}$ is the interior of $A$, then:
            \begin{equation}
                \interior[\tau]{A}=
                    \bigcup_{\mathcal{U}\in\tau_{A}}\mathcal{U}
            \end{equation}
        \end{theorem}
        \begin{proof}
            For if $p\in\interior[\tau]{A}$, then $p$ is an interior point of
            $A$ (Def.~\ref{def:Interior_of_Set}) and hence there is an open
            subset $\mathcal{U}\subseteq{A}$ such that $p\in\mathcal{U}$
            (Def.~\ref{def:Interior_Point}). But then by hypothesis
            $\mathcal{U}\in\tau_{A}$, and hence $p\in\bigcup\tau_{A}$. Thus,
            $\interior[\tau]{A}\subseteq\bigcup\tau_{A}$. If
            $p\in\bigcup\tau_{A}$, then there is a $\mathcal{U}\in\tau_{A}$
            such that $p\in\mathcal{U}$. But by hypothesis if
            $\mathcal{U}\in\tau_{A}$, then $\mathcal{U}\subseteq{A}$, and hence
            $p$ is an interior point of $A$. Thus, $p\in\interior[\tau]{A}$.
            Therefore, $\interior[\tau]{A}=\bigcup\tau_{A}$.
        \end{proof}
        Thus, an equivalent formulation of the interior of a set is the union
        of all open sets that are contained in $A$. There's a similar notion for
        exterior points and the exterior of a set.
        \begin{fdefinition}{Exterior Point}{Exterior_Point}
            An exterior point of a subset $A\subseteq{X}$ of a topological
            spacec $\topspace{X}$ is a point $p\in{X}\setminus{A}$ such that
            there exists an open set $\mathcal{U}\in\tau$ such that
            $\mathcal{U}\subseteq{A}\setminus{A}$ and $p\in\mathcal{U}$.
        \end{fdefinition}
        \begin{fdefinition}{Exterior of a Set}{Exterior_of_a_Set}
            The exterior of a subset $A\subseteq{X}$ in a topological space
            $\topspace{X}$, denoted $\exterior[\tau]{A}$, is the set of all
            exterior points of $A$. That is:
            \begin{equation*}
                \exterior{A}=\{\,p\in{X}\;|\;
                    p\textrm{ is an exterior point of }X\,\}
            \end{equation*}
        \end{fdefinition}
        \begin{theorem}
            If $\topspace{X}$ is a topological space, if $A\subseteq{A}$, if
            $\tau_{A^{C}}$ is the set of all open subsets contained in
            $X\setminus{A}$, then:
            \begin{equation}
                \exterior{A}=\bigcup_{\mathcal{U}\in\tau_{A}^{C}}\mathcal{U}
            \end{equation}
        \end{theorem}
        \begin{proof}
            For if $p\in\exterior{A}$, then there is an open subset
            $\mathcal{U}\in\tau$ such that $\mathcal{U}\subseteq{X}\setminus{A}$
            and $p\in\mathcal{U}$. But then $\mathcal{U}\in\tau_{A^{C}}$, and
            hence $p\in\bigcup\tau_{A^{C}}$. Thus,
            $\exterior[\tau]{A}\subseteq\bigcup\tau_{A^{C}}$. Similarly in the
            other direction.
        \end{proof}
        \begin{fdefinition}{Boundary of a Set}{Boundary_of_Set}
            The boundary of a subset $A\subseteq{X}$ in a topological space
            $\topspace{X}$ is the set $\partial{A}$ defined by:
            \begin{equation*}
                \partial{A}=\closure{A}\setminus\interior{A}
            \end{equation*}
            Where $\closure{A}$ is the closure of $A$ and $\interior{A}$ is its
            interior.
        \end{fdefinition}
        \begin{theorem}
            If $\topspace{X}$ is a topological space, if $A\subseteq{X}$, then:
            \begin{equation}
                \partial{A}=X\setminus\big(\interior{A}\cup\exterior{A}\big)
            \end{equation}
        \end{theorem}
        \begin{fdefinition}{Isolated Point}{Isolated_Point}
            An isolated point of a subset $A\subseteq{X}$ in a topological space
            $\topspace{X}$ is a point $p\in{A}$ such that there exists an
            open set $\mathcal{U}\in\tau$ such that $\mathcal{U}\cap{A}=\{p\}$.
        \end{fdefinition}
        \begin{fdefinition}{Limit Point}{Limit_Point}
            A limit point of a subset $A\subseteq{X}$ in a topological space
            $\topspace{X}$ is a point $p\in{X}$ such that
            $p\in\closure{A}$ and $p$ is not an isolated point of $A$.
        \end{fdefinition}
        That is, limit points can be approximated arbitrarily well by points in
        $A$, other than the point $p$ itself.
        \begin{fdefinition}{Dense Subset}{Dense_Subset}
            A dense subset of a topological space $\topspace{X}$ is a subset
            $A\subseteq{X}$ such that $\closure{A}=X$
        \end{fdefinition}
        \begin{fdefinition}{Nowhere Dense}{Nowhere_Dense}
            A nowhere dense subset of a topological space $\topspace{X}$ is a
            subset $A\subseteq{A}$ such that:
            \begin{equation*}
                \interior{\closure{A}}=\emptyset
            \end{equation*}
        \end{fdefinition}
        \begin{theorem}
            If $\topspace{X}$ is a topological space, then $A\subseteq{X}$ is
            nowhere dense if and only if for every non-empty open subset
            $\mathcal{U}\in\tau$, $A$ is not dense in the subspace topology of
            $\mathcal{U}$.
        \end{theorem}
        \begin{proof}
            For if suppose not and suppose $\mathcal{U}\in\tau$ is such that
            $A$ is dense in $\mathcal{U}$ in the subspace topology. But then:
            \begin{equation}
                \mathcal{U}=\interior{\mathcal{U}}
                    \subseteq\interior{\closure{A}}
            \end{equation}
            But $A$ is nowhere dense, and hence
            $\interior{\closure{A}}=\emptyset$, a contradiction. In the other
            direction, suppose $\interior{\closure{A}}\ne\emptyset$. But the
            interior of any set is open, and hence $A$ is dense in
            $\interior{\closure{A}}$, a contradiction.
        \end{proof}
        Def continuous, def homeomorphism.
        \begin{fdefinition}{Local Homeomorphism}{Local_Homeomorphism}
            A local homeomorphism from a topological space $\topspace[X]{X}$ to
            a topological space $\topspace[Y]{Y}$ is a function
            $f:X\rightarrow{Y}$ such that for all $x\in{X}$ there exists and
            open set $\mathcal{U}\in\tau$ such that $x\in\mathcal{U}$ and
            $f|_{\mathcal{U}}$ is a continuous bijective open mapping.
        \end{fdefinition}
        Def convergent sequence.
        \begin{ftheorem}{Kuratowski's Continuity Theorem}
                        {Kuratowski_Continuity_Theorem}
            If $\topspace[X]{X}$ and $\topspace[Y]{Y}$ are topological spaces,
            then $f:X\rightarrow{Y}$ is continuous if and only if for all
            $A\subseteq{X}$ it is true that:
            \begin{equation*}
                f\big[\closure[\tau_{X}]{A}\big]\subseteq
                \closure[\tau_{Y}]{f[A]}
            \end{equation*}
        \end{ftheorem}
        \begin{bproof}
            For suppore $f:X\rightarrow{Y}$ is continuous. Since
            $\closure[\tau_{Y}]{f[A]}$ is closed and $f$ is continuous,
            $f^{\minus{1}}\big[\closure[\tau_{Y}]{f[A]}\big]$ is closed.
            But $f[A]\subseteq\closure[\tau_{Y}]{f[a]}$, and thus
            $A\subseteq{f}^{\minus{1}}\big[\closure[\tau_{Y}]{f[A]}\big]$,
            and therefore:
            \begin{equation}
                \closure[\tau_{X}]{A}\subseteq
                \closure[\tau_{X}]
                    {{f}^{\minus{1}}\big[\closure[\tau_{Y}]{f[A]}\big]}
            \end{equation}
            But the closure of a closed set is the original set, and hence:
            \begin{equation}
                \closure[\tau_{X}]{A}\subseteq
                    f^{\minus{1}}\big[\closure[\tau_{Y}]{f[A]}\big]
            \end{equation}
            But then:
            \begin{equation}
                f\big[\closure[\tau_{X}]{A}\big]\subseteq
                f\Big[f^{\minus{1}}\big[\closure[\tau_{Y}]{f[A]}\big]\Big]
                \subseteq\closure[\tau_{Y}]{f[A]}
            \end{equation}
            In other direction, let $C\subseteq{Y}$ be closed. Let
            $x\in\closure[\tau_{X}]{f^{\minus{1}}[C]}$. But then by hypothesis
            $f(x)\in\closure[\tau_{Y}]{f\big[f^{\minus{1}}[C]\big]}$ and so
            $f(x)\in\closure[\tau_{Y}]{C}$. But $C$ is closed, and hence
            $\closure[\tau_{Y}]{C}=C$, and therefore $f(x)\in{C}$. But then
            $x\in{f}^{\minus{1}}[C]$, and thus
            $\closure[\tau_{X}]{f^{\minus{1}}[c]}\subseteq{f}^{\minus{1}}[C]$,
            and thus $f^{\minus{1}}[C]$ is closed. Hence, $f$ is continuous.
        \end{bproof}
        There's a dual theorem for the interior operator.
        \begin{theorem}
            If $\topspace[X]{X}$ and $\topspace[Y]{Y}$ are topological spaces,
            then $f:X\rightarrow{Y}$ is continuous if and only if for all
            $B\subseteq{Y}$ it is true that:
            \begin{equation}
                f^{\minus{1}}\big[\interior[\tau_{Y}]{B}\big]
                \subseteq\interior[\tau_{X}]{f^{\minus{1}}[B]}
            \end{equation}
        \end{theorem}
        Identity map is continuous, constant maps are continuous, composition is
        continuous.
        \begin{fdefinition}{Sequentially Continuous}{Sequentially_Continuous}
            A sequentially continuous function from a topological space
            $\topspace[X]{X}$ to a topological space $\topspace[Y]{Y}$ is a
            function $f:X\rightarrow{Y}$ such that for every convergent sequence
            $a:\mathbb{N}\rightarrow{X}$ and for every limit $x$ of $a$ it is
            true that $f(a_{n})\rightarrow{f}(x)$.
        \end{fdefinition}
        \begin{fdefinition}{Sequential Topological Space}
                           {Sequential_Topological_Space}
            A sequential topological space is a topological space $\topspace{X}$
            such that for every topological space $\topspace{Y}$ and for every
            sequentially continuous function $f:X\rightarrow{Y}$, it is true
            that $f$ is continuous.
        \end{fdefinition}
        Without sufficient separation properties (such as being Hausdorff) we
        don't know if the limits of sequences are unique, hence the phrasing
        \textit{for every limit}.
        \begin{theorem}
            If $\topspace[X]{X}$ and $\topspace[Y]{Y}$ are topological spaces,
            if $f:X\rightarrow{Y}$ is continuous, then it is sequentially
            continuous.
        \end{theorem}
        \begin{theorem}
            If $\topspace{X}$ is a locally compact Lindel\"{o}f topological
            space, then it is $\sigma$ compact.
        \end{theorem}
        \begin{proof}
            For every $x\in{X}$ there exists compact $K_{x}$ and an open
            $\mathcal{U}_{x}$ such that $x\in\mathcal{U}_{x}$ and
            $\mathcal{U}_{x}\subseteq{K}_{x}$. But then $\{\mathcal{U}_{x}\}$ is
            a cover of $X$, and since $\topspace{X}$ is Lindel\"{o}f there
            exists a countable subcover $\mathcal{U}_{n}$. Then
            $\mathcal{U}_{n}\subseteq{K}_{n}$, and thus $K_{n}$ is a countable
            covering of $X$ by compact sets, hence $X$ is $\sigma$ compact.
        \end{proof}
        \begin{theorem}
            Limits in Hausdorff are unique.
        \end{theorem}
        \begin{proof}
            Suppose $a_{n}\rightarrow{x}$ and $a_{n}\rightarrow{y}$, $x\ne{y}$.
            Then there are disjoint non-empty $\mathcal{U}_{x}$,
            $\mathcal{U}_{y}$. But then there exists $N_{x},N_{y}$ such that
            $n>N_{x}$ implies $a_{n}\in\mathcal{U}_{x}$ and $n>N_{y}$ implies
            $a_{n}\in\mathcal{U}_{y}$. Choose $N=\max\{N_{x},N_{y}\}$, a
            contradiction.
        \end{proof}
        \begin{theorem}
            If $\topspace{X}$ is a Hausdorff topological spacec, and if
            $x\in{X}$, then $\{x\}$ is a closed subset of $X$.
        \end{theorem}
        \begin{proof}
            For all $y\in{X}$, $y\ne{x}$, there is a $\mathcal{U}_{y}\in\tau$
            such that $y\in\mathcal{U}_{y}$ and $x\notin\mathcal{U}_{y}$. But
            then $X\setminus\{x\}=\bigcup\mathcal{U}_{y}$, and $\{x\}$ is
            the complement of an open set and is therefore closed.
        \end{proof}
        \begin{theorem}
            If $\topspace{X}$ is a Hausdorff topological space, and if
            $A\subseteq{X}$ is finite, then $A$ is closed.
        \end{theorem}
        \begin{proof}
            For $A$ is the union of finitely many closed sets, and thus closed.
        \end{proof}
        \begin{fdefinition}{Neighborhood Basis}{Neighborhood_Basis}
            A neighborhood basis for a point $x$ in a topological space
            $\topspace{X}$ is a subset $\mathcal{B}_{x}\subseteq\tau$ such that
            for all $\mathcal{V}\in\mathcal{B}_{x}$ it is true that
            $x\in\mathcal{V}$, and for all $\mathcal{U}\in\tau$ such that
            $x\in\mathcal{U}$ there exists $\mathcal{V}\in\mathcal{B}_{x}$ such
            that $\mathcal{V}\subseteq\mathcal{U}$.
        \end{fdefinition}
        \begin{fdefinition}{First Countable Topological Space}
                           {First_Countable_Topological_Space}
            A first countable topological space is a topological space
            $\topspace{X}$ such that all $x\in{X}$ there exists a countable
            neighborhood basis $\mathcal{B}$ of $x$.
        \end{fdefinition}
        \begin{example}
            Any metric space.
        \end{example}
        \begin{fdefinition}{Second Countable Topological Space}
                           {Second_Countable_Topological_Space}
            A second countable topological space is a topological space
            $\topspace{X}$ such that there exists a countable basis
            $\mathcal{B}$ for $\tau$.
        \end{fdefinition}
        \begin{theorem}
            Second countable implies first countable.
        \end{theorem}
        \begin{example}
            This does not reverse, take the discrete metric on $\mathbb{R}$.
        \end{example}
        \begin{theorem}
            If $\topspace{X}$ is a first countable topological space, if
            $A\subseteq{X}$, and if $x\in{X}$, then $x\in\closure{A}$, if and
            only if there is a sequence $a:\mathbb{N}\rightarrow{X}$ such that
            $a_{n}\rightarrow{x}$ and for all $n\in\mathbb{N}$ it is true that
            $a_{n}\in{A}$.
        \end{theorem}
        \begin{proof}
            For suppose $x\in\closure{A}$. Since $\topspace{X}$ is first
            countable, there exists a countable neighborhood basis $\mathcal{B}$
            of $x$. Since $\mathcal{B}$ is countable, there exists a surjection
            $B:\mathbb{N}\rightarrow\mathcal{B}$. Let
            $\mathcal{U}:\mathbb{N}\rightarrow\tau$ be defined by:
            \begin{equation}
                \mathcal{U}_{n}=\bigcap_{k\in\mathbb{Z}_{n+1}}B_{k}
            \end{equation}
            Since $x\in\closure{A}$, for every open subset $\mathcal{U}\in\tau$
            that contains $x$ there exists a point $y\in{A}$ such that
            $y\in\mathcal{U}$. But then for all $n\in\mathbb{N}$ the set
            $A_{n}$ defined by:
            \begin{equation}
                A_{n}=\{\,y\in{A}\;|\;y\in\mathcal{U}_{n}\}
            \end{equation}
            is non-empty. Thus by the axiom of choice there is a choice function
            $a:\mathbb{N}\rightarrow{X}$ such that $a_{n}\in{A}_{n}$. From the
            definition of $\mathcal{U}_{n}$, $a_{n}\rightarrow{X}$. The other
            direction is the definition of convergence and closure
            (no first countability needed).
        \end{proof}
        \begin{theorem}
            If $\topspace{X}$ is a first countable topological space, if
            $A\subseteq{X}$, and if $x\in{X}$, then $x\in\interior{A}$ if and
            only if for every sequence $a:\mathbb{N}\rightarrow{X}$ such that
            $a_{n}\rightarrow{x}$ there exists an $N\in\mathbb{N}$ such that for
            all $n>N$, $a_{n}\in{A}$.
        \end{theorem}
        \begin{proof}
            For if $x\in\interior{A}$, and if $a:\mathbb{N}\rightarrow{X}$ is a
            sequence such that $a_{n}\rightarrow{x}$, then for every open subset
            $\mathcal{U}\in\tau$ such that $x\in\mathcal{U}$ there exists an
            $N\in\mathbb{N}$ such that $n>N$ implies $a_{n}\in\mathcal{U}$.
            But $\interior{A}$ is open and $\interior{A}\subseteq{A}$. Going the
            other way, suppose $x\in{A}$ is such that for every sequence
            $a:\mathbb{N}\rightarrow{X}$ such that $a_{n}\rightarrow{x}$ there
            exists an $N\in\mathbb{N}$ such that for all $n>N$ it is true that
            $a_{n}\in{A}$. Suppose $x\notin\interior{A}$. Since $\topspace{X}$
            is first countable, there is a countable neighborhood basis
            $\mathcal{B}_{x}$ of $x$. Let $B:\mathbb{N}\rightarrow\mathcal{B}$
            be a surjection and let $\mathcal{U}_{n}$ be defined by:
            \begin{equation}
                \mathcal{U}_{n}=\bigcap_{k\in\mathbb{Z}_{n+1}}B_{k}
            \end{equation}
            But if $xn\notin\interior{A}$ then for all $n\in\mathcal{N}$ there
            is a $y\in\mathcal{U}_{n}$ such that $y\notin{A}$. By the axiom of
            choice there is a sequence $y:\mathbb{N}\rightarrow{X}$ such that
            $y_{n}\in\mathcal{U}_{n}$ and $y_{n}\notin{A}$. But then
            $y_{n}\rightarrow{x}$ and $y_{n}$ is never in $A$, a contradiction.
        \end{proof}
        \begin{theorem}
            If $\topspace{X}$ is a first countable topological space, and if
            $A\subseteq{X}$, then $A$ is open if and only if for every sequence
            $a:\mathbb{N}\rightarrow{X}$ such that there exists an $x\in{A}$
            such that $a_{n}\rightarrow{x}$, then there is an $N\in\mathbb{N}$
            such that for all $n>N$, $a_{n}\in{A}$.
        \end{theorem}
        \begin{proof}
            One direction is the definition of convergence in a topological
            space. Suppose for every sequence $a:\mathbb{N}\rightarrow{X}$ such
            that there exists $x\in{A}$ such that $a_{n}\rightarrow{x}$, it is
            true that there exists an $N\in\mathbb{N}$ such that $n>N$ implies
            $a_{n}\in{A}$. Then for every $x\in{A}$ there is an open subset
            $\mathcal{U}_{x}\subseteq{A}$ such that $x\in\mathcal{U}_{x}$. For
            suppose not, and let $x\in{A}$ be such that there is not open
            neighborhood $\mathcal{U}_{x}\in\tau$ such that
            $x\in\mathcal{U}_{x}$ and $\mathcal{U}_{x}\subseteq{A}$. But
            $\topspace{X}$ is first countable and thus there is a countable
            neighborhood basis $\mathcal{B}_{x}$ about $x$. Let
            $B:\mathbb{N}\rightarrow\mathcal{B}$ be a surjection and let
            $\mathcal{U}_{n}$ be the intersection of $B_{0},\dots,B_{n}$. Since
            none of these are contained in $A$, by hypothesis, there exists a
            $y_{n}\in\mathcal{U}_{n}$ such that $y_{n}\notin{A}$. But then
            $y_{n}\rightarrow{x}$, a contradiction.
        \end{proof}
        \begin{theorem}
            If $\topspace{X}$ is a first countable topological space, and if
            $A\subseteq{X}$, then $A$ is closed if and only for every convergent
            sequence $a:\mathbb{N}\rightarrow{X}$ with limit $x\in{X}$ such that
            for all $n\in\mathbb{N}$ it is true that $a_{n}\in{A}$, then
            $x\in{A}$.
        \end{theorem}
        \begin{proof}
            For if $A$ is closed, and if $a:\mathbb{N}\rightarrow{A}$ is such
            that $a_{n}\rightarrow{x}$, suppose $x\notin{A}$. Then since $A$ is
            closed, $X\setminus{A}$ is open. But if $a_{n}\rightarrow{x}$ then
            for all $\mathcal{U}\in\tau$ such that $x\in\mathcal{U}$, there is
            and $N\in\mathbb{N}$ such that for all $n>N$ it is true that
            $a_{n}\in\mathcal{U}$. But $X\setminus{A}$ is open and
            $x\in{X}\setminus{A}$, a contradiction since $a_{n}\in{A}$ for all
            $n$. In the other direction, a set is closed if and only if it is
            equal to it's closure, hence apply the previous theorem.
        \end{proof}
        \begin{fdefinition}{Lindel\"{o}f Topological Space}
                           {Lindelof_Topological_Space}
            A Lindel\"{o}f topological space is a topological $\topspace{X}$
            such that for every open cover $\mathcal{O}$ of $X$ there exists a
            countable subcover $\Delta$.
        \end{fdefinition}
        This is a weaker version of compactness, but it has it's uses.
        \begin{theorem}
            If $\topspace{X}$ is a second countable topological space, then it
            is Lindel\"{o}f.
        \end{theorem}
        \begin{proof}
            For let $\mathcal{O}$ be an open cover of $X$. Since $\topspace{X}$
            is second countable, there exists a countable basis $\mathcal{B}$
            for $\tau$. Let $B:\mathbb{N}\rightarrow\mathcal{B}$ be a
            surjection. For all $\mathcal{U}\in\mathcal{O}$, define:
            \begin{equation}
                A_{\mathcal{U}}=
                    \{\,n\in\mathbb{N}\;|\;B_{n}\subseteq\mathcal{U}\}
            \end{equation}
            Since $\mathcal{B}$ is a basis, for all non-empty
            $\mathcal{U}\in\mathcal{O}$ there is an $n\in\mathbb{N}$ such that
            $B_{n}\subseteq\mathcal{U}$, and hence $A_{\mathcal{U}}$ is
            non-empty. There is thus an inverse mapping (axiom of choice)
            $f:\mathbb{N}\rightarrow\mathcal{O}$ such that for all
            $n\in\mathbb{N}$, $B_{n}\subseteq{f}_{n}$, and thus
            $\Delta=\{f_{n}\}$ is a countable subcover.
        \end{proof}
        Def subspace, def relatively open, relatively closed.
        \begin{fdefinition}{Topological Embedding}{Topological_Embedding}
            A topological embedding of a topological space $\topspace[X]{X}$
            into a topological space $\topspace[Y]{Y}$ is a continuous injective
            function $f:X\rightarrow{Y}$ such that $\topspace{X}$ is
            homeomorphic to $t\topspace{f[X]}{f[X]}$, where
            $\tau_{f[X]}$ is the subspace topology in $Y$.
        \end{fdefinition}
        \begin{theorem}
            If $\topspace[X]{X}$ is a topological space, if $A\subseteq{X}$,
            if $\tau_{A}$ is the subspace topology on $A$, if $\topspace[Y]{Y}$
            is a topological space, and if $f:Y\rightarrow{A}$ is a function,
            then $f$ is continuous if and only if
            $\inclusion{A}\circ{f}:Y\rightarrow{X}$ is continuous, where
            $\inclusion{A}\rightarrow{X}$ is the inclusion mapping.
        \end{theorem}
        \begin{theorem}
            If $\topspace[X]{X}$ is a topological space, if $A\subseteq{X}$,
            if $\tau$ is a topology on $A$ such that for every topological space
            $\topspace[Y]{Y}$ and for every continuous function
            $f:Y\rightarrow{A}$ it is true that $\inclusion{A}\circ{f}$ is
            continuous, then $\tau=\tau_{A}$ where $\tau_{A}$ is the subspace
            topology.
        \end{theorem}
        \begin{theorem}
            Closed in subspace $A$ if and only if $C_{A}=A\cap{C}_{X}$, where
            $C_{X}$ is closed.
        \end{theorem}
        Inclusion map it top embedding, restriction of continuous is continuous,
        subspace of Hausdorff is Hausdorff, same for first/second countable.
        If $f:X\rightarrow{Y}$ is continuous, then it is continuous in it's
        restriction to all subspaces of $X$. What about the converse? If
        $f$ is continuous in it's subspaces, is it continuous?
        \begin{theorem}
            If $\topspace[X]{X}$ is a topological space, if $\topspace[Y]{Y}$ is
            a topological space, and if $f:X\rightarrow{Y}$ is such that for
            every point $x\in{X}$ there exists an open subset
            $\mathcal{U}\in\tau$ such that $x\in\mathcal{U}$ and
            $f|_{\mathcal{U}}:\mathcal{U}\rightarrow{Y}$ is continuous in the
            subspace topology, then $f$ is continuous.
        \end{theorem}
        This theorem let's us make continuous functions by gluing together
        pieces of functions that are defined on open subsets which agree on the
        overlap.
        \begin{theorem}
            If $\topspace[X]{X}$ and $\topspace[Y]{Y}$ are topological spaces,
            if $\mathcal{O}$ is an open cover of $X$, if
            $\mathscr{F}:\mathcal{O}\rightarrow\funcspace[Y]{X}$ is a bijection
            such that for all $\mathcal{U},\mathcal{V}\in\mathcal{O}$ it is true
            that:
            \begin{equation}
                \mathscr{F}(\mathcal{U})|_{\mathcal{U}\cap\mathcal{V}}=
                \mathscr{F}(\mathcal{V})|_{\mathcal{U}\cap\mathcal{V}}
            \end{equation}
            and such that
            $\mathscr{F}(\mathcal{U})|_{\mathcal{U}}:\mathcal{U}\rightarrow{Y}$
            is continuous in the subspace topology, then there is a unique
            continuous function $f:X\rightarrow{Y}$ such that for all
            $\mathcal{U}\in\mathscr{F}$ it is true that
            $f|_{mathcal{U}}=\mathscr{F}(\mathcal{U})|_{\mathcal{U}}$
        \end{theorem}
        \begin{theorem}
            If $\topspace{X}$ is a topological space, if
            $\mathcal{O}$ is a countable open cover of $X$ such that for all
            $\mathcal{U}\in\mathcal{O}$ it is true that
            $\topspace[\mathcal{U}]{\mathcal{U}}$ is second countable, where
            $\tau_{\mathcal{U}}$ is the subspace topology, then $\topspace{X}$
            is second countable.
        \end{theorem}
        \begin{proof}
            For let $\mathcal{B}$ be the union of all of the bases of all of the
            $\mathcal{U}\in\mathcal{O}$. This is countable since it is the
            countable union of countable sets. Let $\mathcal{V}\in\tau$ be an
            open subset of $X$. But then $\mathcal{V}\cap\mathcal{U}$ is open
            for all $\mathcal{U}\in\mathcal{O}$ and it is relatively open in
            $\mathcal{U}$. But $\mathcal{B}_{\mathcal{U}}$ is a basis for
            $\mathcal{U}$, and hence there is a subset
            $\Delta_{\mathcal{V}}\subseteq\mathcal{B}_{\mathcal{U}}$ such that
            $\cap\mathcal{U}=\bigcup\Delta_{\mathcal{V}}$. But then:
            \begin{equation}
                \mathcal{V}=\mathcal{V}\cap{X}=
                \mathcal{V}\cap\Big(\bigcup\mathcal{U}\Big)
                =\bigcup\big(\mathcal{V}\cap\mathcal{U}\big)
                =\bigcup\Big(\bigcup\Delta_{\mathcal{V}}\Big)
            \end{equation}
            which is the union of elements of $\mathcal{B}$, and hence this is a
            basis for $\tau$.
        \end{proof}
        Product topology, projection map, continuous iff components are
        continuous. The roduct topology is unique topology with this property.
        Projections are continuious. Product maps. Product of continuous maps
        is continuous. Subspace topology is the same as product topology of
        subspaces. That is, $A_{1}\times{A}_{2}$, doesn't matter if we consider
        subspace topology of product or product topology of subspaces.
        Open rectangles form basis in finite products.
        \par\hfill\par
        Def disjoint union, canonical injective map.
        \begin{fdefinition}{Disjoint Union Topology}{Disjoint_Union_Topology}
            The disjoint union topology generated by the disjoint union of a
            set of topological spaces $\mathscr{T}$ is subset
            $\tau_{T}\subseteq\mathcal{P}(\coprod\mathscr{T})$ defined by
            declaring $\mathcal{U}$ open if and only if for all $\alpha$,
            $\pi_{\alpha}(\mathcal{U})$ is open in $X_{\alpha}$.
        \end{fdefinition}
        Function fro disjoint union to $Y$ is continuous if and only if
        $F\circ\inclusion{\alpha}$ is continuous for all $\alpha$ where
        $\inclusion{\alpha}$ is the canonical inclusion map. Disjoint union
        topology is unique topology with this trait. Canonical inclusions are
        embeddings. Disjoint union of Hausdorff is Hausdorff, similarly for
        firsct countable. Disjoint unions of countably many second countable
        spaces is second countable.
        \begin{fdefinition}{Quotient Map}{Quotient_Map}
            A quotient map form a topological space $\topspace[X]{X}$ to a
            topological $\topspace[Y]{Y}$ is a surjective continuous function
            $q:X\rightarrow{Y}$ and such that for all
            $\mathcal{V}\subseteq{Y}$ such that
            $q^{\minus{1}}[\mathcal{V}]\in\tau_{X}$, it is true that
            $\mathcal{V}\in\tau_{Y}$
        \end{fdefinition}
        That is, the quotient map $q$ is continuous, surjective, and uniquely
        defines the topology on $Y$. A common confusion is that a quotient map
        is not necessarily an open mapping. The reverse direction is a true
        statement.
        \begin{theorem}
            If $\topspace[X]{X}$ and $\topspace[Y]{Y}$ are topological spaces,
            and if $f:X\rightarrow{Y}$ is a continuous surjective open map,
            then $f$ is a quotient map.
        \end{theorem}
        \begin{proof}
            For if $\mathcal{V}\in\tau_{Y}$, then since $f$ is continuous it is
            true that $f^{\minus{1}}[\mathcal{V}]\in\tau_{X}$. Moreover, if
            $\mathcal{V}\subseteq{Y}$ is such that
            $f^{\minus{1}}[\mathcal{V}]\in\tau_{X}$, then since $f$ is
            surjective it is true that
            $f\big[f^{\minus{1}}[\mathcal{V}]\big]=\mathcal{V}$. But
            by hypothesis $f$ is an open mapping, and
            $f^{\minus{1}}[\mathcal{V}]$ and an open subset of $X$, and
            therefore $\mathcal{V}$ is an open subset of $Y$. Thus, for all
            $\mathcal{V}\subseteq{Y}$ it is true that $\mathcal{V}$ is open if
            and only if $f^{\minus{1}}[\mathcal{V}]$ is open in $X$. Thus,
            $f$ is a quotient map.
        \end{proof}
        \begin{fdefinition}{Quotient Topology}{Quotient_Topology}
            The quotient topology on a set $Y$ induced by a topological space
            $\topspace[X]{X}$ under a surjective function $q:X\rightarrow{Y}$
            is the set:
            \begin{equation*}
                \tau_{Y}=\{\,\mathcal{V}\in\powset{Y}\;|\;
                    q^{\minus{1}}[\mathcal{V}]\in\tau_{X}\,\}
            \end{equation*}
        \end{fdefinition}
        The most common way of constructing quotient spaces is by means of an
        equivalence relation.
        \begin{fdefinition}{Quotient Space of Equivalence Relation}
                           {Quotient_Space_of_Equiv_Rel}
            The quotient space induced by an equivalence relation $R$ on a set
            $X$ with respect to a topology $\tau$ on $X$ is the topological
            space $\topspace[R]{X/R}$ where $X/R$ is the quotient set of $X$
            under $R$ and $\tau_{R}$ is the quotient topology induced by the
            natural projection mapping $q:X\rightarrow{X}/R$ defined by
            $q(x)=[x]$.
        \end{fdefinition}
        \begin{fdefinition}{Adjunction Space}{Adjunction_Space}
            The adjunction space of a topological space $\topspace[X]{X}$
            with respect to a topological space $\topspace[Y]{X}$ under a
            continuous function $f:A\rightarrow{X}$ from a closed subset
            $A\subseteq{Y}$ into $X$ is the quotient space
            $\adjspace{X}{Y}$ formed by the equivalence relation $R$ on
            $X\coprod{Y}$ generated by the relation:
            \begin{equation*}
                \tilde{R}=
                    \Big\{\,
                        \big((x,0),(y,1)\big)\in\big(X\coprod{Y}\big)^{2}\;|\;
                        y\in{A}\textrm{ and }x=f(y),0\,
                    \Big\}
            \end{equation*}
        \end{fdefinition}
        Pictorially, we have a copy of $X$ and a copy of $Y$ in some ambient
        space, and we glue $A\subseteq{Y}$ along $f[A]\subseteq{X}$.
        \begin{fdefinition}{Saturated Subset}{Saturated_Subset}
            A saturated subset of a set $X$ with respect to a set $Y$ under a
            function $f:X\rightarrow{Y}$ is a subset $A\subseteq{X}$ such that:
            \begin{equation*}
                f^{\minus{1}}\big[f[A]\big]=A
            \end{equation*}
        \end{fdefinition}
        \begin{theorem}
            If $X$ and $Y$ are sets, if $f:A\rightarrow{Y}$ is a function, and
            if $A\subseteq{X}$, then $A$ is saturated in $X$ if and only if
            there is a subset $B\subseteq{Y}$ such that:
            \begin{equation}
                A=\bigcup_{y\in{B}}f^{\minus{1}}[\{y\}]
            \end{equation}
            That is, $A$ is the union of fibers.
        \end{theorem}
        \begin{proof}
            For if $A$ is the union of the fibers of $B$, then
            $f[A]=B$. But then $A=f^{\minus{1}}[b]=f^{\minus{1}}\big[f[A]\big]$,
            and hence $A$ is saturated. If $A$ is saturated, let $B=f[A]$ and
            let $y\in{B}$. Then $f^{\minus{1}}[\{b\}]\subseteq{A}$ since $A$ is
            saturated. Thus, $A$ is the union of fibers.
        \end{proof}
        \begin{theorem}
            If $\topspace[X]{X}$ and $\topspace[Y]{Y}$ are topological spaces,
            if $q:X\rightarrow{Y}$ is a quotient map, and if
            $\topspace[Z]{Z}$ is a topological space, then for any continuous
            function $f:Y\rightarrow{Z}$, $f\circ{q}:X\rightarrow{Z}$ is
            continuous.
        \end{theorem}
        The quotient topology is the unique topology with this property. Closed
        if and only if $q^{\minus{1}}[\mathcal{V}]$ is closed. Injective
        quotient map is a homeomorphism. The restriction of quotient map to
        saturated subset is a quotient map onto it's image. Composition of
        quotient maps is a quotient map.
        \begin{theorem}
            If $\topspace[X]{X}$ and $\topspace[Y]{Y}$ are topological spaces,
            if $q:X\rightarrow{Y}$ is a continuous surjective function, then
            $q$ is a quotient map if and only if for every open saturated subset
            $\mathcal{U}\subseteq{X}$, $q[\mathcal{U}]$ is open in $Y$.
        \end{theorem}
        \begin{proof}
            For if $q$ is a quotient map, then $\mathcal{V}\subseteq{Y}$ is
            open if and only if $q^{\minus{1}}[\mathcal{V}]\in\tau_{X}$. But
            if $\mathcal{U}\in\tau_{X}$ is open and saturated, then
            $q^{\minus{1}}[q[\mathcal{U}]]=\mathcal{U}$. But then
            $q[\mathcal{U}]$ is such that it's pre-image under $q$ is open, and
            thus $q[\mathcal{U}]$ is open. On the other hand, suppose $q$ maps
            open saturated sets to open sets and let $\mathcal{V}\subseteq{Y}$
            be such that $q^{\minus{1}}[\mathcal{V}]$ is open. Let
            $\mathcal{U}=q^{\minus{1}}[\mathcal{V}]$. But then $\mathcal{U}$ is
            the union of fibers and is hence saturated. But then $\mathcal{U}$
            is open and saturated, and thus $q[\mathcal{U}]$ is open. But
            $q[\mathcal{U}]=\mathcal{V}$, and thus $\mathcal{V}$ is open. Thus,
            $q$ is a quotient map.
        \end{proof}
        Same theorem for closed subsets.
        \begin{theorem}
            If $\topspace[X]{X}$ and $\topspace[Y]{Y}$ are topological spaces,
            if $q:X\rightarrow{Y}$ is a quotient map, if $\topspace[Z]{Z}$ is a
            topological space, and if $f:X\rightarrow{Z}$ is a continuous
            function such that for all $x_{1},x_{2}\in{X}$ such that
            $q(x_{1})=q(x_{2})$ it is true that $f(x_{1})=f(x_{2})$, then there
            is a unique continuous function $\tilde{f}:Y\rightarrow{Z}$ such
            that $f=\tilde{f}\circ{q}$.
        \end{theorem}
        \begin{proof}
            For since $q$ is a quotient map, it is surjective. Thus for all
            $y\in{Y}$, $q^{\minus{1}}[\{y\}]$ is non-empty. Thus, by the axiom
            of choice, there is a function $p:Y\rightarrow{X}$ such
            that $p(y)\in{q}^{\minus{1}}[\{y\}]$ for all $y\in{Y}$. That is,
            $p(y)$ is a representative for the equivalence class $[p(y)]$ of
            points in $X$ that map to $y$. Let $\tilde{f}:Y\rightarrow{Z}$ be
            defined by $\tilde{f}=f\circ{p}$. But by hypothesis for all
            $x_{1},x_{2}\in{q}^{\minus{1}}[\{y\}]$ it is true that
            $f(x_{1})=f(x_{2})$. But then:
            \begin{equation}
                f(x)=f(p(y))=\tilde{f}(y)
            \end{equation}
            and thus $f=\tilde{f}\circ{q}$. But $f$ is continuous, and thus by
            the characteristic property of quotient spaces, $\tilde{f}$ is
            continuous.
        \end{proof}
        \begin{theorem}
            If $\topspace[X]{X}$, $\topspace[Y]{Y}$, and $\topspace[Z]{Z}$ are
            topological spaces, if $q_{Y}:X\rightarrow{Y}$ and
            $q_{Z}:X\rightarrow{Z}$ are quotient maps, and if for all
            $x_{1},x_{2}\in{X}$ it is true that $q_{Y}(x_{1})=q_{Y}(x_{2})$ if
            and only if $q_{Z}(x_{1})=q_{Z}(x_{2})$, then there is a unique
            homeomorphism $\varphi:Y\rightarrow{Z}$ such that
            $\varphi\circ{q_{Y}}=q_{Z}$.
        \end{theorem}
        Projection maps are open mappings. Canonical injective mappings into
        disjoint union topology are closed and open mappings. Local hoemo is
        open map. Bijective local homeo is homeo.
        \begin{theorem}
            If $\topspace[X]{X}$ and $\topspace[Y]{Y}$ are topological spaces,
            if $q:X\rightarrow{Y}$ is a quotient map, and if
            $R\subseteq{X}\times{X}$ defined by:
            \begin{equation}
                R=\{\,(x_{1},x_{2})\in{X}\times{X}\;|\;q(x_{1})=q(x_{2})\,\}
            \end{equation}
            is a closed subset of $X\times{X}$ in the product topology, then
            $Y$ is Hausdorff.
        \end{theorem}
        \begin{theorem}
            $f:X\rightarrow{Y}$ is an open mapping if and only if for all
            $B\subseteq{Y}$ it is true that:
            \begin{equation}
                \interior[\tau_{X}]{f^{\minus{1}}[B]}\subseteq
                f^{\minus{1}}\big[\interior[\tau_{Y}]{B}\big]
            \end{equation}
        \end{theorem}
        Similarly for closed mappings.
        \begin{theorem}
            If $q:X\rightarrow{Y}$ is a continuous surjective open function,
            then it is a quotient map.
        \end{theorem}
        \begin{theorem}
            If $q:X\rightarrow{Y}$ is a continuous injective open function,
            then it is a topological embedding.
        \end{theorem}
        \begin{theorem}
            If $f:X\rightarrow{Y}$ is a continuous bijective open function,
            then it is a homeomorphism.
        \end{theorem}
        Def connected space, connected component, continuous image of connected
        is connected.
        \begin{theorem}
            If $\topspace{X}$ is a topological space, if $A\subseteq{X}$ is a
            connected subset, then there is a unique connected component
            $B$ of $X$ such that $A\subseteq{B}$.
        \end{theorem}
        \begin{proof}
            For let $(C,\subseteq)$ be the partial ordered set of connected
            subsets of $X$ ordered by inclusion and let $\mathcal{U}_{J}$ be a
            chain. Suppose $\bigcup\mathcal{U}_{j}$ is disconnected. Then there
            are disjoint non-empty open sets $\mathcal{V}_{1},\mathcal{V}_{2}$
            such that
            $\mathcal{V}_{1}\cup\mathcal{V}_{2}=\bigcup\mathcal{U}_{J}$, and
            such that $\mathcal{V}_{i}\cap\bigcup\mathcal{U}_{j}\ne\emptyset$.
            But then there is an $x_{1}\in\mathcal{V}_{1}$ such that
            $x_{1}\in\bigcup\mathcal{U}_{J}$ and an $x_{2}\in\mathcal{V}_{2}$
            such that $x_{2}\in\bigcup\mathcal{U}_{J}$. But if
            $x_{1}\in\mathcal{U}_{J}$ there is some $\alpha\in{J}$ such that
            $x_{1}\in\mathcal{U}_{\alpha}$, and similarly
            $x_{2}\in\mathcal{U}_{\beta}$. But since the $\mathcal{U}_{J}$ form
            a chain, either $\mathcal{U}_{\alpha}\subseteq\mathcal{U}_{\beta}$
            or $\mathcal{U}_{\beta}\subseteq\mathcal{U}_{\alpha}$. But then
            $\mathcal{V}_{1}$ and $\mathcal{V}_{2}$ are not disjoint, a
            contradiction, and hence $\bigcup\mathcal{U}_{J}$ is connected.
            Thus by Zorn's lemma, there is a maximal element $B\in{C}$ that is
            comporable to $A$. That is, $B$ is maximal and $A\subseteq{B}$.
            If $B'$ is a different maximal element, and
            $B\cup{B}'$ would be connected and strictly larger, contradicting
            maximality.
        \end{proof}
        \begin{theorem}
            The connected components of $X$ are closed non-empty disjoint sets
            that partition $X$.
        \end{theorem}
        \begin{theorem}
            Product of connected is connected (in product topology).
        \end{theorem}
        The product of connected need not be connected in the box topology.
        \begin{theorem}
            The quotient of connected is connected.
        \end{theorem}
        \begin{fdefinition}{Locally Path Connected}{Locally_Path_Connected}
            A locally path connected topological space is a topological space
            $\topspace{X}$ such that there exists a basis $\mathcal{B}$ of
            $\tau$ such that for all $\mathcal{U}\in\mathcal{B}$ it is true that
            $\mathcal{U}$ is path connected.
        \end{fdefinition}
        \begin{example}
            Paradoxically, path connected does not imply locally path connected.
            The Warsaw Circle, which is the topologist's sine curve wrapped
            around with an arc to form a path connected space, is not locally
            path connected.
        \end{example}
        \begin{theorem}
            If $\topspace{X}$ is a locally path connected topological space, and
            if $A\subseteq{X}$ is a connected component of $X$, then $A$ is
            open.
        \end{theorem}
        \begin{proof}
            For if $\topspace{X}$ is locally path connected, there is a basis
            $\mathcal{B}$ of $\tau$ of connected open sets. But then for all
            $x\in{A}$, there is an open set $\mathcal{U}_{x}\in\mathcal{B}$
            such that $x\in\mathcal{U}_{x}$. But by hypothesis
            $\mathcal{U}_{x}$ is open and $A$ is a connected component, and
            hence $\mathcal{U}_{x}\subseteq{A}$. But then
            $A=\bigcup\mathcal{U}_{x}$, which is open, and thus $A$ is open.
        \end{proof}
        \begin{theorem}
            If $\topspace{X}$ is a locally path connected topological space, and
            if $A\subseteq{X}$ is a connected component, then it is a path
            connected component.
        \end{theorem}
        \begin{theorem}
            If $\topspace{X}$ is a locally path connected topological space,
            and if $\topspace{X}$ is connected, then it is path connected.
        \end{theorem}
        Def compact, cont image of compact is compact, EVT, finite union of
        compact is compact, disjoint compact subsets of a Hausdorff space can be
        separated by disjoint open sets. Closed subset of compact is compact.
        Compact Hausdorff is closed. Compact in metric if and only if complete
        and totally bounded. Product of compact is compact. Quotient of compact
        is compact. Def uni cont, Lipschitz cont, locally Lipschitz cont.
        Lipschitz cont -> uni cont -> cont (metric space). Cont on compact is
        uni cont. Locally Lipschitz cont on compact is Lipschitz cont.
        Implications don't reverse: Try $\sqrt{x}$ and $x^{2}$.
        \begin{theorem}
            If $\topspace{X}$ is first countable and countably compact,
            then it is sequentially compact.
        \end{theorem}
        \begin{fdefinition}{Point Finite Cover}{Point_Finite_Cover}
            A point finite cover of a topological space $\topspace{X}$ is a
            cover $\mathcal{O}$ of $X$ such that for all $x\in{X}$ the set
            $\mathcal{O}_{x}$ defined by:
            \begin{equation*}
                \mathcal{O}_{x}=\{\,\mathcal{U}\in\mathcal{O}\;|\;
                    x\in\mathcal{U}\,\}
            \end{equation*}
            is finite.
        \end{fdefinition}
        \begin{fdefinition}{Metacompact}{Metacompact}
            A metacompact topological space is a topological space
            $\topspace{X}$ such that for every open cover $\mathcal{O}$ of $X$
            there exists a point finite refinement $\Delta$ of $\mathcal{O}$.
        \end{fdefinition}
        One useful theorem about metacompact sets is how it related sequentially
        compact spaces to compact spaces. In general, these two need not be
        equivalent. The classsic examples are the long line (which is
        sequentially compact but not compact) and the product space of the
        closed unit interval $[0,1]$ with itself uncountably many times
        (which is compact but not sequentially compact). If the space is both
        metacompact and countably compact, then it is compact. Since metacompact
        is implied by paracompactness, and since every metric space is
        paracompact, we can combine all of this to get a nice theorem: A metric
        space is compact if and only if it is sequentially compact. A more
        general version says that sequentially compact metacompact spaces are
        also compact.
        \begin{theorem}
            If $\topspace[X]{X}$ is a compact topological space, if
            $\topspace[Y]{Y}$ is a Hausdorff topological space, and if
            $f:X\rightarrow{Y}$ is a continuous function, then it is a closed
            mapping.
        \end{theorem}
        \begin{proof}
            For if $C\subseteq{X}$ is closed, then it is compact since $X$ is
            compact. But the continuous image of a compact set is compact, and
            hence $f[C]$ is a compact subset of $Y$. But $Y$ is Hausdorff, and
            thus compact subsets are closed. Thus, the image of closed sets is
            closed and hence $f$ is a closed mapping.
        \end{proof}
        \begin{theorem}
            If $\topspace[X]{X}$ is a compact topological space, if
            $\topspace[Y]{Y}$ is a Hausdorff topological space, and if
            $f:X\rightarrow{Y}$ is a continuous bijection, then it is a
            homeomorphism.
        \end{theorem}
        \begin{theorem}
            For $f$ is then a continuous bijective closed mapping, and hence
            is a homeomorphism.
        \end{theorem}
        Weakening bijective to injective or surjective obtains topological
        embeddings and quotient maps, respectively.
        \begin{fdefinition}{Proper Function}{Proper_Function}
            A proper function from a topological space $\topspace[X]{X}$ to a
            topological space $\topspace[Y]{Y}$ is a function
            $f:X\rightarrow{Y}$ such that for every compact subset
            $C\subseteq{Y}$, the pre-image $f^{\minus{1}}[C]$ is a compact
            subset of $X$
        \end{fdefinition}
        Note that the definition of a proper function does not actually require
        $f$ to be continuous. Such functions play roles in the theory of
        groupoids, in analysis, and in the study of smooth manifolds. If the
        target space $Y$ is locally compact and Hausdorff, there's a simple
        theorem that can accompony this definition.
        \begin{theorem}
            If $\topspace[X]{X}$ and $\topspace[Y]{Y}$ are topological spaces,
            if $f:X\rightarrow{Y}$ is a continuous closed function, and if for
            all $y\in{Y}$ it is true that $f^{\minus{1}}[\{y\}]$ is a compact
            subset of $X$, then $f$ is a proper function.
        \end{theorem}
        \begin{proof}
            For let $C\subseteq{Y}$ be compact and let $K=f^{\minus{1}}[C]$.
            Suppose $K$ is not compact. Then there is an open cover
            $\mathcal{O}$ of $K$ such that no finite subcover exists.
        \end{proof}
        There's a rewording of a previous theorem in terms of proper functions.
        \begin{theorem}
            If $\topspace[X]{X}$ is a compact topological space, if
            $\topspace[Y]{Y}$ is a Hausdorff topological space, and if
            $f:X\rightarrow{Y}$ is a continuous function, then it is a proper
            function.
        \end{theorem}
        \begin{proof}
            For if $C\subseteq{Y}$ is compact, then it is closed since $Y$ is
            Hausdorff. But the pre-image of a closed subset under a continuous
            function is closed. Thus, $f^{\minus{1}}[C]$ is a closed subset of
            $X$. But closed subsets of compact sets are compact, and hence
            $f^{\minus{1}}[C]$ is compact.
        \end{proof}
        \begin{theorem}
            If $\topspace[X]{X}$ is a topologicall space, if $\topspace[Y]{Y}$
            is a Hausdorff topological space, if $f:X\rightarrow{Y}$ is
            continuous, and if $g:Y\rightarrow{X}$ is a continuous left inverse
            of $f$, then $f$ is proper.
        \end{theorem}
        \begin{theorem}
            If $\topspace[X]{X}$ and $\topspace[Y]{Y}$ are topological spaces,
            if $f:X\rightarrow{Y}$ is a continuous proper function, if
            $A\subseteq{X}$ is a saturated subset of $X$ with respect to $f$,
            and if $f|_{A}:A\rightarrow{Y}$ is the restriction of $f$ to $A$,
            then $f$ is proper in the subspace topology on $A$.
        \end{theorem}
        \begin{ltheorem}{Tube Lemma}
            If $\topspace[X]{X}$ and $\topspace[Y]{Y}$ are topological spaces,
            if $A\subseteq{X}$ is compact, if $B\subseteq{Y}$ is compact,
            if $\topspace[X\times{Y}]{X\times{Y}}$ is the product topological
            space, and if $\mathscr{O}\subseteq{X}\times{Y}$ is and open subset
            in the product topology such that $A\times{B}\subseteq\mathscr{O}$,
            then there exists open subsets $\mathcal{U}\in\tau_{X}$ and
            $\mathcal{V}\in\tau_{Y}$ such that
            $A\times{B}\subseteq\mathcal{U}\times\mathcal{V}$ and
            $\mathcal{U}\times\mathcal{V}\subseteq\mathscr{O}$.
        \end{ltheorem}
        \begin{theorem}
            If $\topspace{X}$ is a locally compact Hausdorff space, and if
            $x\in{X}$, then there is a precompact subset $\mathcal{U}\in\tau$
            such that $x\in\mathcal{U}$.
        \end{theorem}
        \begin{proof}
            For if $\topspace{X}$ is locally compact and if $x\in{X}$, then
            there is an open set $\mathcal{U}\in\tau$ and a compact set
            $K\subseteq{X}$ such that $x\in\mathcal{U}$ and
            $\mathcal{U}\subseteq{K}$. But $X$ is Hausdorff, and thus if $K$ is
            compact, then $K$ is closed. But then
            $\closure{\mathcal{U}}\subseteq{K}$. But then $\closure{U}$ is a
            closed subset of a compact set, and is therefore closed. Hence,
            $\mathcal{U}\in\tau$ is a precompact open set that contains $x$.
        \end{proof}
        \begin{theorem}
            If $\topspace{X}$ is a locally compact Hausdorff space then there
            exists a basis $\mathcal{B}$ for $\tau$ such that for all
            $\mathcal{U}\in\mathcal{B}$ it is true that $\mathcal{U}$ is
            precompact.
        \end{theorem}
        \begin{proof}
            For if $\topspace{X}$ is locally compact and Hausdorff, then for all
            $x\in{X}$ there is a precompact open subset $\mathcal{U}_{x}$
            such that $x\in\mathcal{U}_{x}$. Let $\mathcal{B}$ be defined by:
            \begin{equation}
                \mathcal{B}=\{\,\mathcal{V}\in\tau\;|\;
                    \exists_{x\in{X}}(\mathcal{V}\subseteq\mathcal{U}_{x})\,\}
            \end{equation}
            Then every element of $\mathcal{V}$ is precompact since
            $\closure{\mathcal{V}}\subseteq\closure{\mathcal{U}_{x}}$, which
            is compact, and closed subsets of compact sets are compact.
            Moreover, $\mathcal{B}$ is a basis for $\tau$. For if
            $\mathcal{O}\in\tau$ is open, then $\mathcal{U}_{x}\cap\mathcal{O}$
            is open for all $x\in{X}$. And this is a subset of $\mathcal{U}_{x}$
            and hence contained in $\mathcal{B}$. But then:
            \begin{equation}
                \mathcal{O}=\mathcal{O}\cap{X}
                =\mathcal{O}\cap\Big(\bigcup_{x\in{X}}\mathcal{U}_{x}\Big)
                =\bigcup_{x\in{X}}\big(\mathcal{O}\cap\mathcal{U}_{x}\big)
            \end{equation}
            Thus, $\mathcal{B}$ is a basis.
        \end{proof}
        \begin{theorem}
            Every open or closed subspace of a locally commpact Hausdorff space
            is again locally compact Hausdorff.
        \end{theorem}
        \begin{theorem}
            If $\topspace[X]{X}$ is a topological space, if $\topspace[Y]{Y}$ is
            a locally compact Hausdorff space, if $f:X\rightarrow{Y}$ is
            continuous proper function, then $f$ is a closed map.
        \end{theorem}
        \begin{proof}
            For let $C\subseteq{X}$ be closed. Suppose $f[C]$ is not closed.
            Then there exists a convergent sequence $a:\mathbb{N}\rightarrow{X}$
            with a limit $x\in{X}\setminus{A}$ such that for all
            $n\in\mathbb{N}$ it is true that $a_{n}\in{A}$. But $f$ is
            continuous and therefore $f(a_{n})\rightarrow{f}(x)$. But
            $Y$ is locally compact and Hausdorff and thus there is a precompact
            open subset $\mathcal{V}\in\tau_{Y}$ such that $f(x)\in\mathcal{V}$.
            But $f$ is proper, and therefore
            $f^{\minus{1}}[\closure[Y]{\mathcal{V}}]$ is a compact subset of
            $X$. But then $C\cap{f}^{\minus{1}}[\mathcal{V}]$ is a closed
            subset of a compact set, and is therefore compact. But then
            $f(x)\in{f}[C]$, a contradiction. Thus, $C$ is closed.
        \end{proof}
        One of the most important theorems in the study of metric and locally
        compact Hausdorff spaces is the Baire category theorem. It comes in
        three flavors.
        \begin{ftheorem}{The First Baire Category Theorem}
                        {First_Baire_Category_Theorem}
            If $\topspace{X}$ is a completely metrizable topological space,
            then it is a Baire space.
        \end{ftheorem}
        \begin{ftheorem}{The Second Baire Category Theorem}
                        {Second_Baire_Category_Theorem}
            If $\topspace{X}$ is a locally compact Hausdorff space, then it is
            a Baire space.
        \end{ftheorem}
        \begin{ftheorem}{The Third Baire Category Theorem}
                        {Third_Baire_Category_Theorem}
            If $\topspace{X}$ is a completely metrizable topological space,
            if $C:\mathbb{N}\rightarrow\powset{X}$ is a sequence of closed sets,
            and if $\bigcup{C}=X$, then there exists an $N\in\mathbb{N}$ such
            that $\interior{C_{N}}\ne\emptyset$.
        \end{ftheorem}
        \begin{theorem}
            If $\topspace{X}$ is a Baire space, if and if $C\subseteq{X}$ is a
            countable closed subset of $X$, then there exists an isolated point
            $x\in{C}$.
        \end{theorem}
        \begin{fdefinition}{Compact Exhaustion}{Compact_Exhaustion}
            A compact exhaustion of a topological space $\topspace{X}$ is a
            sequence of compact sets $K:\mathbb{N}\rightarrow\mathcal{P}(X)$
            such that for all $n\in\mathbb{N}$ it is true that:
            \begin{equation*}
                K_{n}\subseteq\interior{{K}_{n+1}}
            \end{equation*}
            and such that $X=\bigcup{K}_{n}$.
        \end{fdefinition}
        \begin{ftheorem}{Existence of Compact Exhaustions}
                        {Existence_of_Compact_Exhaustions}
            If $\topspace{X}$ is a second countable locally compact Hausdorff
            topological space, then there exists a compact exhaustion of $X$.
        \end{ftheorem}
        \begin{bproof}
            Since $\topspace{X}$ is locally compact and Hausdorff, there is a
            basis of precompact open sets. But a second countable space is
            Lindel\"{o}f, and hence there is a countable subcover. Let
            $K_{n}=\bigcup\closure{\mathcal{U}_{k}}$.
        \end{bproof}
        \subsection{Algebraic Topology}
            Def homotopy, homotopic. Path homotopic functions in a topological
            space $X$ are paths $\gamma_{1},\gamma_{2}:[0,1]\rightarrow{X}$
            which are homotopic relative to the set $\{0,1\}$. Path homotopy
            forms an equivalence relation on the set of all paths between two
            points, $x,y\in{X}$. Given two pahts $f,g:[0,1]\rightarrow{X}$
            such that $f(1)=g(0)$, their product is the new path that
            concatenates these two travelling as twice the \textit{speed}.
            \begin{equation}
                (f\star{g})(t)=
                \begin{cases}
                    f(2t),&0\leq{t}\leq\frac{1}{2}\\
                    g(2t-1),&\frac{1}{2}<t\leq{1}
                \end{cases}
            \end{equation}
            Since $f(1)=g(0)$, $f\star{g}$ is continuous at $1/2$ and is
            therefore a path. If $f,f',g,g'$ are paths such that $f$ and $f'$
            are path homotopic, and $g$ and $g'$ are path homotopic, then
            $f\star{g}$ and $f'\star{g}'$ will be path homotopic as well.
            Multiplication of paths is \textit{not} associative
            (draw a picture). Def loop ($f:[0,1]\rightarrow{X}$ vs
            $f:\nsphere[1]\rightarrow{X}$). Def fundamental group. Identity is
            constant map, inverse of $f(t)$ is $f(1-t)$.
            \begin{theorem}
                If $\topspace{X}$ is a path connected topological space and if
                $x,y\in{X}$, then the fundamental groups $\homotopygroup{X}{x}$
                and $\homotopygroup{X}{y}$ are isomorphic.
            \end{theorem}
            \begin{fdefinition}{Simply Connected}{Simply_Connected}
                A simply connected topological space is a topological space
                $\topspace{X}$ such that for all $p\in{X}$ it is true that the
                fundamental group $\homotopygroup{X}{x}$ is isomorphic to the
                trivial group.
            \end{fdefinition}
            \begin{theorem}
                If $\topspace{X}$ is a path connected topological space, then
                it is simply connected if and only if for all $x,y\in{X}$ and
                all paths $\gamma_{1},\gamma_{2}:[0,1]\rightarrow{X}$ such that
                $\gamma_{1}(0)=x$, $\gamma_{2}(0)=x$ and such that
                $\gamma_{1}(1)=y$ and $\gamma_{2}(1)=y$, it is true that
                $\gamma_{1}$ and $\gamma_{2}$ are path homotopic.
            \end{theorem}
            \begin{proof}
                For let $h=\gamma_{1}\star\gamma_{2}^{\minus{1}}$. Since $X$ is
                simply connected, there is a homotopy $H$ between $h$ and the
                path $\gamma_{2}\star\gamma_{2}^{\minus{1}}$.
            \end{proof}
            \begin{theorem}
                If $f_{1},f_{2}:X\rightarrow{Y}$, $g_{1},g_{2}:Y\rightarrow{Z}$
                are continuous, $f_{1}$ homotopy to $f_{2}$, $g_{1}$ homotopic
                to $g_{2}$, then $g_{1}\circ{f_{1}}$ is homotopic to
                $g_{2}\circ{f}_{2}$.
            \end{theorem}
            \begin{fdefinition}{Induced Homomorphism of $\pi_{1}$}
                               {Induced_Homo_Fun_Group}
                The induced homomorphism from a the fundamental group
                of a topological space $\topspace[X]{X}$ at a point $x\in{X}$
                into the fundamental group of a topological space
                $\topspace[Y]{Y}$ about a point $y\in{Y}$ by a continuous
                function $f:X\rightarrow{Y}$ such that $f(x)=y$ is the function
                $f_{*}:\homotopygroup{X}{x}\rightarrow\homotopygroup{Y}{y}$
                defined by:
                \begin{equation*}
                    f_{*}[\gamma]=[f\circ\gamma]
                \end{equation*}
            \end{fdefinition}
            Avoiding proof by 
            \begin{theorem}
                The induced homomorphism is a homomorphism.
            \end{theorem}
            \begin{theorem}
                If $\topspace[X]{X}$, $\topspace[Y]{Y}$, and $\topspace[Z]{Z}$
                are topological spaces, if $f:X\rightarrow{Y}$ and
                $g:Y\rightarrow{Z}$ are continuous, if $x\in{X}$, if
                $y=f(x)$, and if $z=g(z)$, then:
                \begin{equation}
                    (g\circ{f})_{*}=g_{*}\circ{f}_{*}
                \end{equation}
                where $h_{*}$ denotes the induced homotopy between fundamental
                groups.
            \end{theorem}
            \begin{theorem}
                If $\topspace{X}$ is a topological space, if $x\in{X}$, and if
                $\identity{X}$ is the identity map, then
                ${\identity{X}}_{*}$ is the unital element of
                $\homotopygroup{X}{x}$.
            \end{theorem}
            Homeomorphic topological spaces have isomorphic fundamental groups.
            There's a weakening of the notion of convexity that is useful in the
            study of fundamental groups.
            \begin{fdefinition}{Star Shaped}{Star_Shaped}
                A start shaped subset of a vector space $\monoid[+]{V}$ over a
                field $\ring{F}$ is a subset $\mathcal{U}\subseteq{V}$ such that
                there exists a point $\vector{x}\in\mathcal{U}$ such that for
                all $\vector{y}\in\mathcal{U}$ the straight line between
                $\vector{x}$ and $\vector{y}$ is contained in $\mathcal{U}$.
            \end{fdefinition}
            Thus, every convex body is star shaped but the converse is false.
            Indeed, a \textit{star} is star shaped by not convex since the
            line between the outer vertices leaves the body.
            \begin{theorem}
                If $\mathcal{U}\subseteq\mathbb{R}^{n}$ is star shaped, then it
                is simply connected.
            \end{theorem}
            \begin{proof}
                Use the straight-line homotopy between the central point
                $\vector{x}$ and any other point $\vector{y}$.
            \end{proof}
            Fundamental group of $\nsphere[1]$ is $\mathbb{Z}$, all higher
            spheres are simply connected. Fundamental group of product is
            product of fundamental groups. Thus, fundamental group of torus is
            $\mathbb{Z}^{2}$.
    \section{Reading From Lee (Chapter 1)}
        Smoothness cannot be a topological property. To see this, note that the
        circle and the square are homeomorphic to each other, and thus topology
        cannot distinguish between the two. They different in the smooth sense
        since one has sharp corners (the square) and the other does not. We can
        be explicit, let
        $\varphi:\mathbb{R}^{2}\setminus\{0\}\rightarrow{S}^{1}$ be the mapping:
        \begin{equation}
            \varphi(\mathbf{x})=\frac{\mathbf{x}}{\norm{\mathbf{x}}}
        \end{equation}
        The restriction of $\varphi$ to the square will then be a homemorphism
        (Fig.~\ref{fig:Homeomorphism_Square_to_Circle}).
        \begin{figure}[H]
            \centering
            \captionsetup{type=figure}
            \begin{tikzpicture}[>=LaTeX]
    \draw[->] (-3,  0) to (3, 0) node[above] {$x$};
    \draw[->] ( 0, -3) to (0, 3) node[right] {$y$};

    \draw (0, 0) circle (2);
    \draw (2, -2) to (2, 2) to (-2, 2) to (-2, -2) to cycle;

    \begin{scope}[->, shorten <= 0.1cm, shorten >= 0.1cm]
        \draw (2.0, 2.0) to (1.414, 1.414);
        \draw (1.5, 2.0) to (1.2, 1.6);
        \draw (2.0, 1.5) to (1.6, 1.2);

        \draw (-2.0, 2.0) to (-1.414, 1.414);
        \draw (-1.5, 2.0) to (-1.200, 1.600);
        \draw (-2.0, 1.5) to (-1.600, 1.200);

        \draw (-2.0, -2.0) to (-1.414, -1.414);
        \draw (-1.5, -2.0) to (-1.200, -1.600);
        \draw (-2.0, -1.5) to (-1.600, -1.200);

        \draw (2.0, -2.0) to (1.414, -1.414);
        \draw (1.5, -2.0) to (1.200, -1.600);
        \draw (2.0, -1.5) to (1.600, -1.200);
    \end{scope}

    \node at (2.5, 2.5) {$\mathbb{R}^{2}$};
    \node at (0.4, 1.6) {$S^{1}$};
\end{tikzpicture}
            \caption{homemorphism from the Square to the Circle}
            \label{fig:Homeomorphism_Square_to_Circle}
        \end{figure}
        \begin{definition}
            A topological $n$ manifold is a topological space $(X,\tau)$ that is
            Hausdorff, second-countable, and for all $p\in{X}$ there is an open
            subset $\mathcal{U}\subseteq{X}$ such that $p\in\mathcal{U}$ and
            $\mathcal{U}$ is homeomorphic to an open subset of $\mathbb{R}^{n}$.
        \end{definition}
        \begin{theorem}
            A topological space $(X,\tau)$ is a topological manifold if and only
            if if is Hausdorff, second countable, and for all $p\in{X}$ there is
            an open subset $\mathcal{U}\subseteq{X}$ such that $p\in\mathcal{U}$
            and $\mathcal{U}$ is homeomorphic to $\mathbb{R}^{n}$.
        \end{theorem}
        \begin{proof}
            If every point in $X$ is contained in an open set $\mathcal{U}$ that
            is homeomorphic to all of $\mathbb{R}^{n}$, then it is a topological
            manifold since $\mathbb{R}^{n}$ is an open subset of
            $\mathbb{R}^{n}$. Going the other direction, suppose $(X,\tau)$ is a
            topological manifold and $p\in{X}$. Then there is an open subset
            $\mathcal{U}$ such that $p\in\mathcal{U}$ and $\mathcal{U}$ is
            homeomorphic to an open subset of $\mathbb{R}^{n}$. That is, there
            is a $\mathcal{V}\subseteq\mathbb{R}^{n}$ and a homemorphism
            $\varphi:\mathcal{U}\rightarrow\mathcal{V}$. But since
            $p\in\mathcal{U}$ and $\varphi:\mathcal{U}\rightarrow\mathcal{V}$,
            it is then true that $\varphi(p)\in\mathcal{V}$. But $\mathcal{V}$
            is open, and thus there is an $r>0$ such that
            $B_{r}^{\mathbb{R}^{n}}(\varphi(p))\subseteq\mathcal{V}$. That is,
            there is an $r>0$ such that the $r$ ball centered about $\varphi(p)$
            is contained in $\mathcal{V}$. But $\varphi$ is a homeomorphism, and
            thus $\varphi^{\minus{1}}$ is a homeomorphism. And the restriction
            of a homeomorphism to a subspace is again a homeomorphism in the
            subspace topologies. Thus,
            $\varphi^{\minus{1}}|_{B_{r}^{\mathbb{R}^{n}}(\varphi(p))}$ is a
            homeomorphism from $B_{r}^{\mathbb{R}^{n}}(\varphi(p))$ to its
            image. But it's image is simply
            $\varphi^{\minus{1}}(B_{r}^{\mathbb{R}^{n}}(\varphi(p)))$, which is
            open since $\varphi$ is continuous. Thus, there is an open subset of
            $(X,\tau)$ that contains $p$ which is homeomorphic to an open ball
            in $\mathbb{R}^{n}$. But any open ball in $\mathbb{R}^{n}$ is
            homeomorphic to all of $\mathbb{R}^{n}$, and since homeomorphic is
            an equivalence relation, there is thus an open subset containing $p$
            that is homeomorphic to all of $\mathbb{R}^{n}$.
        \end{proof}
        \begin{example}
            All of $\mathbb{R}^{n}$ is itself a topological $n$ dimensional
            manifold. It is Hausdorff since it is a metric space, given two
            points a distance $d$ away one can choose open balls of radius
            $d/4$ about each point as disjoint open neighborhoods, ensuring the
            Hausdorff property. Moreover, it is second countable since the set
            of all balls with rational radii centered about rational points
            (elements of $\mathbb{Q}^{n}$) form a countable basis for the
            topology on $\mathbb{R}^{n}$. Hence it is Hausdorff and second
            countable. That any point $\mathbf{x}\in\mathbb{R}^{n}$ is contained
            in an open subset homeomorphic to $\mathbb{R}^{n}$ can be seen by
            letting $\mathcal{U}=\mathbb{R}^{n}$. That is, take as the open
            subset the entire space. Therefore, $\mathbb{R}^{n}$ is a
            topological manifold.
        \end{example}
        \begin{example}
            In the previous example we implicitly thought of $\mathbb{R}^{n}$ as
            being equipped with the standard Euclidean topology. This is the
            topology that arises from the Pythagorean distance function:
            \begin{equation}
                d_{2}(\mathbf{x},\mathbf{y})=
                \sqrt{\sum_{k\in\mathbb{Z}_{n}}(x_{k}-y_{k})^{2}}
            \end{equation}
            What if we change the metric? Any metric on $\mathbb{R}^{n}$ will
            preserve the Hausdorff property, since all metric spaces are
            Hausdorff, however the discrete metric loses the second countability
            property, and hence cannot be considered a topological manifold
            (although it is still locally \textit{zero} dimensional). What about
            the metrics that arise from the $\norm{\cdot}_{p}$ norm:
            \begin{equation}
                d_{p}(\mathbf{x},\mathbf{y})=
                \Big(\sum_{k\in\mathbb{Z}_{n}}(x_{k}-y_{k})^{p}\Big)^{1/p}
            \end{equation}
            For $1\leq{p}\leq\infty$ this will be second countable and Hausdorff
            and since these metrics are equivalent, they will induce the same
            topology, and hence will be topological manifolds of dimension $n$.
            However the case of $n=1$ and $n=\infty$ are somewhat peculiar. An
            open ball in the $\norm{\cdot}_{1}$ norm looks like a diamond,
            whereas in $\norm{\cdot}_{\infty}$ the open balls are squares. So
            while all of these metrics give rise to the same topological
            structure, they create a different \textit{smooth} structure on the
            space. When one considers $\mathbb{R}^{n}$ it is standard to assume
            the Euclidean $\norm{\cdot}_{2}$ norm (and the structure induced by
            it) is being considered. That is, open balls
            $B_{r}^{(\mathbb{R}^{n},\norm{\cdot}_{2})}(\mathbf{x})$ are just the
            interiors of $n$ dimensional spheres.
        \end{example}
        The Hausdorff condition is to exclude bizarre spaces such as the
        bug-eyed line and the branching line. The second countability criterion
        ensures the space won't be massive. For example, the uncountable
        disjoint union of spheres will be Hausdorff and locally Euclidean, but
        not second countable (but it will be first countable). This space isn't
        too horrible since it's metrizable (in the disjoint union topology). A
        more pathalogical example is the long line, occasionally denoted
        $[0,\omega_{1})$, defined as $[0,1)\times\omega_{1}$ with the dictionary
        or lexicographic ordering. Here $\omega_{1}$ denotes the first
        uncountable ordinal. This space is not metrizable, but it is Hausdorff
        and locally Euclidean (it is locally like $\mathbb{R}$), but it is
        not second countable. One of the nice theorems from topology is the
        Smirnov metrization theorem which states that a topological space is
        metrizable if and only if it is locally metrizable, Hausdorff, and
        paracompact. Every locally Euclidean space is automatically locally
        metrizable since we can steal the Euclidean metric on small enough
        neighborhoods about every point.
        \par\hfill\par
        The second countability condition allows one to prove paracompactness,
        and thus every topological manifold is automatically metrizable by the
        Smirnov theorem. This is perhaps one justification for the second
        countability condition. Note that the metric that we can place on the
        topological manifold $(X,\tau)$ may not give us much geometrical
        information, but simply allows us to apply the results of the theory of
        metric spaces without much thought. For example, topological manifolds
        must be regular, normal, perfectly normal Hausdorff, we may apply
        Urysohn's lemma and Tietze extension theorem should the need arise, and
        more. Moreover, from an analytical point of view, we have that functions
        are continuous if and only if there are sequentially continuous (i.e.
        $a_{n}\rightarrow{x}$ implies $f(a_{n})\rightarrow{f}(x)$) and compact
        if and only if sequentially compact. So while the metric does not
        necessarily have much to do with the structure we care about, it's
        existence gives us this plethora of data for free.
        \begin{example}
            Consider the sphere $S^{2}$. We can place a natural metric on this
            by defining the distance to be the length of the shortest curve
            connecting two points. One can prove the shortest curve exists and
            that it lies on the great circle between the two points (the circle
            containing the two points with the center of the sphere as the
            origin). This function is symmetric, positive definite, and obeys
            the triangle inequality and is therefore a metric. However, if we
            were to take two disjoint spheres, how can we metrize this? There's
            one way, defining:
            \begin{equation}
                d(x,y)=
                \begin{cases}
                    d_{1}(x,y),&x,y\in{S}_{1}^{2}\\
                    d_{2}(x,y),&x,y\in{S}_{2}^{2}\\
                    4,&\textrm{Otherwise}
                \end{cases}
            \end{equation}
            where $S_{i}^{2}$ is the $i^{th}$ sphere and $d_{i}$ is the
            \textit{geodesic} metric described above on the respective spheres.
            For the case of when the points lie in separate sphere we've chosen
            the constant 4 since this is greater than the furthest two points
            can be on the unit sphere (which is $\pi$, half of the perimeter
            $2\pi$). Doing this enures $d$ is a metric. So while intuitively we
            want to thing of the disjoint union of two spheres as a two
            dimesnional topological space, the metric that can induce the
            topology seems somewhat irrelevant to the picture we have in mind. 
        \end{example}
        Before continuing it is perhaps important to note that paracompactness
        does not imply second countable. Again, the disjoint union of spheres
        serves as an example.
        \begin{definition}
            A chart of dimension $n\in\mathbb{N}$ in a topological space
            $(X,\tau)$ is an open set $\mathcal{U}\in\tau$ and a function
            $\varphi:\mathcal{U}\rightarrow\mathbb{R}^{n}$ such that $\varphi$
            is continuous, injective, and an open mapping. A chart is denoted
            $(\mathcal{U},\varphi)$.
        \end{definition}
        Equivalently, we could say that $\varphi$ is a homeomorphism onto its
        image or that $\varphi:\mathcal{U}\rightarrow\mathcal{V}$ is a
        homeomorphism where $\mathcal{V}\subseteq\mathbb{R}^{n}$ is open. The
        definition adopted here is so that all of the details about a chart
        $(\mathcal{U},\varphi)$ are embedded in the notation. That is, we need
        not add a $\mathcal{V}$ nor talk about $\varphi^{\minus{1}}$.
        \begin{figure}[H]
            \centering
            \captionsetup{type=figure}
            \begin{tikzpicture}[>=LaTeX]

    % Draw the coordinate axes for R^n.
    \draw[<->] (-1,  0) to (3, 0);
    \draw[<->] ( 0, -1) to (0, 3);

    % Red blob for the image of U.
    \draw[dashed, fill=red!70!white]
    (0.5, 1.5) to[out=30,  in=180]  (1.2,  1.9)
               to[out=0,   in=90]   (1.9,  1.4)
               to[out=-90, in=0]    (1.4,  0.2)
               to[out=180, in=-30]  (0.4,  0.3)
               to[out=150, in=-150] cycle;

    % Labels for R^n and the map phi.
    \node at (0.5, 3.0) {$\mathbb{R}^{n}$};
    \node at (1.1, 1.0) {$\varphi[\mathcal{U}]$};

    % Arrow representing the mapping phi.
    \draw[->] (-3.5, 1) to node[above left] {$\varphi$} (-1.5, 1);

    % Draw the manifolx X, shifted over to the left.
    \begin{scope}[xshift=-8cm,yshift=1.3cm]
        \draw (0,0) to[out=90,  in=180] (1,  0.8)
                    to[out=0,   in=150] (2,  0.8)
                    to[out=-30, in=90]  (4,  0.0)
                    to[out=-90, in=0]   (2, -1.5)
                    to[out=180, in=-30] (0, -1.5)
                    to[out=150, in=-90] cycle;
        
        % Add a donut hole in the manifold.
        \draw (0.5, -0.7) to[in=-130, out=-50] (1.8, -0.7);
        \draw (0.6, -0.8) to[in=130,  out=50]  (1.7, -0.8);

        % Draw a cyan blob for U.
        \draw[dashed, fill=cyan]
            (2.0, 0.0) to[out=30,  in=180]  (2.7,  0.5)
                       to[out=0,   in=90]   (3.5,  0.0)
                       to[out=-90, in=0]    (3.2, -0.9)
                       to[out=180, in=-30]  (2.2, -0.8)
                       to[out=150, in=-150] cycle;

        % Add some labels.
        \node at (2.8, -0.2) {$\mathcal{U}$};
        \node at (1.0,  0.4) {$X$};
    \end{scope}
\end{tikzpicture}
            \caption{A Chart in a Topological Space}
            \label{fig:Chart_in_Topological_Space}
        \end{figure}
        Another equivalent definition of topological manifold is that for all
        $x\in{X}$ there is a chart $(\mathcal{U},\varphi)$ such that
        $x\in\mathcal{U}$.
        \begin{definition}
            The coordinate functions of a chart $(\mathcal{U},\varphi)$ of
            dimension $n\in\mathbb{N}$ in a topological space $(X,\tau)$
            are the function $x^{i}:X\rightarrow\mathbb{R}$ defined by
            $x^{i}=\varphi\circ\pi_{i}$, where
            $\pi_{i}:\mathbb{R}^{n}\rightarrow\mathbb{R}$ is the $i^{th}$
            projection mapping.
        \end{definition}
        With this definition it is often common to write the image of a point
        $p\in\mathcal{U}$ by:
        \begin{equation}
            \varphi(p)=\big(x^{1}(p),x^{2}(p),\dots,x^{n-1}(p),x^{n}(p)\big)
        \end{equation}
        One should note that $x^{2}$ does not denote the square of $x$, and this
        may cause confusion. It is the composition of $\varphi$ with the map
        $\pi_{2}:\mathbb{R}^{n}\rightarrow\mathbb{R}$ which simply selects the
        second coordinate.
        \begin{example}
            The graph of a continuous function
            $f:\mathbb{R}^{n}\rightarrow\mathbb{R}^{m}$ is an $n$ dimensional
            manifold. That is, recalling from set theory that a function
            $f:\mathbb{R}^{n}\rightarrow\mathbb{R}^{m}$ is a subset
            $f\subseteq\mathbb{R}^{n}\times\mathbb{R}^{m}$, if we endow $f$ with
            the subspace topology of $\mathbb{R}^{n+m}$, then it will be an
            $n$ dimensional manifold. To see this, for
            $(\mathbf{x},\mathbf{y})\in{f}$, let
            $\pi_{1}:\mathbb{R}^{n+m}\rightarrow\mathbb{R}^{n}$ be the mapping
            $\pi_{1}(\mathbf{x},\mathbf{y})=\mathbf{x}$. This is continuous, and
            thus the restriction $\pi_{1}|_{f}:f\rightarrow\mathbb{R}^{n}$ is
            continuous in the subspace topology (that is, in the graph of $f$).
            Moreover, $\varphi|_{f}$ is a homeomorphism with continuous inverse
            $\varphi|_{f}^{\minus{1}}(\mathbf{x})=(\mathbf{x},f(\mathbf{x})$.
            This projection then shows that the graph of $f$ is homeomorphic to
            $\mathbb{R}^{n}$ itself. If we were to replace $\mathbb{R}^{n}$ with
            and open subset $\mathcal{U}\subseteq\mathbb{R}^{n}$, the claim
            would still hold.
        \end{example}
        \begin{example}
            The most common non-trivial example of a topological manifold is
            the sphere $S^{n}$. This is the subset of $\mathbb{R}^{n}$ such that
            $\norm{\mathbf{x}}_{2}=1$, where $\norm{\cdot}_{2}$ is the Euclidean
            norm defined by the Pythagorean formula. That $S^{n}$ is a
            topological manifold can be realized in two ways. The first is
            called the \textit{orthographic projection}. We take an observer
            standing on the north pole and then then them off to infinity along
            the line through the origin and the north pole. From the observers
            new perspective, only the top half of the sphere is visible. That
            is, this projection \textit{at infinity} is the set:
            \begin{equation}
                \mathcal{U}_{n-1}^{+}=\big\{\mathbf{x}\in{S}^{2}\;|\;x_{n-1}>0\}
            \end{equation}
            where $x_{n-1}$ is the last coordinate (the coordinate that
            corresponds to the north pole). For $S^{1}$ this is just the set:
            \begin{equation}
                \mathcal{U}_{y}^{+}=
                    \{\,(x,\sqrt{1-x^{2}})\in{S}^{2}\;|\;x\in[-1,1])\,\}
            \end{equation}
            and for $S^{2}$ this is:
            \begin{equation}
                \mathcal{U}_{z}^{+}=
                    \{\,(x,y,\sqrt{1-x^{2}-y^{2}})\in{S}^{2}\;
                    |\;x^{2}+y^{2}\leq{1}\,\}
            \end{equation}
            Since we only see the top half, this does not cover the entire
            sphere. Moreover it contains the boundary, and is therefore not
            open. To meet the requirements of a chart we must therefore remove
            the equator from this projection. Thus, if we with to cover the
            entire sphere with such sets we need $2(n+1)$ charts. That these
            charts are homeomorphic to $\mathbb{R}^{n}$ can be seen by simply
            using the projection map
            $\pi^{i}:\mathbb{R}^{n+1}\rightarrow\mathbb{R}^{n}$ which maps:
            \begin{equation}
                \pi^{i}(x_{0},x_{1},\dots,x_{j-1},x_{j},x_{j+1},\dots,x_{n-1})
                =(x_{0},x_{1},\dots,x_{j-1},x_{j+1},\dots,x_{n-1})
            \end{equation}
            This is continuous, and thus it's restriction to $S^{n}$ is
            continuous. Moreover, it's restriction is an open mapping, and hence
            the sets $\textrm{Int}(\mathcal{U}_{i}^{\pm})$ together with
            $\pi^{i}$ form charts.
        \end{example}
        \begin{example}
            The second, perhaps easier but less intuitive, method of showing
            that $S^{n}$ is a manifold is using \textit{stereographic}
            projection. Before, we lifted our observer to infinity and then
            projected the sphere from this point of view. Now, we keep the
            observer at the north pole and imagine what they'd see if the sphere
            was transparent. Drawing a line from the observer to another point
            on the sphere will then intersect the hyperplane $x_{0}=0$ once.
            We can use this to describe a chart that covers almost the entirety
            of $S^{n}$. Note the line from the North pole to itself will be
            parallel to the plane, and hence will never intersect. We define
            the stereographic projection function
            $\varphi:S^{n}\setminus\{(0,0,\dots,0,1)\}\rightarrow\mathbb{R}^{n}$
            as follows:
            \begin{equation}
                \varphi(\mathbf{x})=\frac{\mathbf{x}}{1-x_{n-1}}
            \end{equation}
            Since a point is closed in a Hausdorff space, the set $S^{n}$ minus
            the north pole is open. Hence this open set is homeomorphic to
            $\mathbb{R}^{n}$ and covers all but one point of the sphere. The
            mapping $\varphi$ is a homemorphism since its inverse is continuous:
            \begin{equation}
                \varphi^{\minus{1}}(\mathbf{X})=\Big(
                    \frac{2X_{0}}{1+\norm{\mathbf{X}}^{2}},\dots,
                    \frac{2X_{n-2}}{1+\norm{\mathbf{X}}^{2}},
                    \frac{1-\norm{\mathbf{X}}^{2}}{1+\norm{\mathbf{X}}^{2}}\Big)
            \end{equation}
            To make $S^{n}$ a topological manifold of dimension $n$ we need only
            add one more stereographic projection about any other point. The
            south pole is usually the most fitting choice, and so we have that
            $S^{n}$ is a manifold that can be covered with two charts.
        \end{example}
        This may naturally lead one to ask the question
        \textit{what's the minimum number of charts needed to cover a manifold?}
        Ostrand's theorem from topology tells us the answer is $n+1$. If the
        space is oriented and 2 dimensional, Morse theory can be used to reduce
        that to 1 or 2 (The only case where 1 is occurs is if the space is
        homeomorphic to $\mathbb{R}^{2}$).
        \par\hfill\par
        The next example to discuss are the real projective spaces, commonly
        denoted $\mathbb{RP}^{n}$.
        \begin{example}
            There are three common and equivalent ways to think of
            $\mathbb{RP}^{2}$, the real projective plane. Much the way the
            torus and the Klein bottle can be realized by and equivalence
            relation on the unit square, so can the real projective plane. The
            more common method is to consider the sphere $S^{2}$ and the
            following equivalence relation: $\mathbf{x}R\mathbf{y}$ if and only
            if $\mathbf{y}=\pm\minus\mathbf{x}$. The quotient space
            $S^{2}/R$ is then the real projective plane. Such pairing of
            antipodal points generalizes to all $n$ and so we may discuss the
            more general real projective space $\mathbb{RP}^{n}$. The last
            method is by considering the set of all lines through the origin.
            We can develop a projection map $\mathbb{R}^{n+1}\setminus\{0\}$
            into $\mathbb{RP}^{n}$ be sending $\mathbf{x}$ to the line it spans.
            That is, $\mathbf{x}\mapsto{t}\mathbf{x}$, where $t$ is a real
            variable. This mapping is indeed a projection mapping and thus we
            can endow $\mathbb{RP}^{n}$ with the quotient topology. For all
            $i\in\mathbb{Z}_{n+1}$, let $\mathcal{U}_{i}$ be the subset of
            $\mathbb{R}^{n+1}$ where $x_{i}\ne{0}$. Since lines are closed in
            $\mathbb{R}^{n+1}$, the complement is open and hence
            $\mathcal{U}_{i}$ is open. Moreover, it is saturated with respect to
            the quotient map:
            \begin{equation}
                \pi^{\minus{1}}\big(\pi(\mathcal{U}_{i})\big)=\mathcal{U}_{i}
            \end{equation}
            and hence the restriction of $\pi$ to $\mathcal{U}_{i}$ is again a
            quotient map. Let
            $\mathcal{V}_{i}=\pi|_{\mathcal{U}_{i}}(\mathcal{U}_{i})$ and define
            $\varphi:\mathcal{V}_{i}\rightarrow\mathbb{R}^{n}$ by:
            \begin{equation}
                \varphi_{i}\big([\mathbf{x}]\big)=
                    \frac{\pi^{i}(\mathbf{x})}{x_{i}}
            \end{equation}
            This mapping is well define, for if $\mathbf{y}\in[\mathbf{x}]$ then
            there is a non-zero $t\in\mathbb{R}$ such that
            $\mathbf{y}=t\mathbf{x}$. But then:
            \begin{equation}
                \varphi\big([\mathbf{y}]\big)=
                    \frac{\pi^{i}(t\mathbf{x})}{tx_{i}}
                    =\frac{t\pi^{i}(\mathbf{x})}{tx_{i}}
                    =\frac{\pi^{i}(\mathbf{x})}{x_{i}}
                    =\varphi\big([\mathbf{x}]\big)
            \end{equation}
            Morevoer, since $x_{i}$ is non-zero for all
            $\mathbf{x}\in\mathcal{U}_{i}$, we are not dividing by zero.
            But $\varphi_{i}\circ\pi$ is continuous, and thus $\varphi_{i}$ is
            continuous (because $\pi$ is a quotient mapping). Moreover, it is a
            homeomorphism since it's inverse:
            \begin{equation}
                \varphi_{i}^{\minus{1}}(\mathbf{x})=
                    [x_{1},\dots,x_{i-1},1,x_{i},\dots,x_{n}]
            \end{equation}
            is continuous. Hence, $\mathbb{RP}^{n}$ is locally Euclidean.
        \end{example}
        \begin{theorem}
            $\mathbb{RP}^{n}$ is Hausdorff.
        \end{theorem}
        \begin{proof}
            For let $[\mathbf{x}]$ and $[\mathbf{y}]$ be distinct elements of
            $\mathbb{RP}^{n}$. For the case of $n>2$ there must be a
            $\mathcal{U}_{i}$ such that $\pi(\mathcal{U}_{i})$ contains both
            points. Since $\pi(\mathcal{U}_{i})$ is locally Euclidean, it is
            Hausdorff, and therefore $[\mathbf{x}]$ and $[\mathbf{y}]$ can be
            separated by open subsets of $\pi()\mathcal{U}_{i})$. For the case
            of $n=2$, the only scenario where $[\mathbf{x}]$ and $[\mathbf{y}]$
            do not lie in the same $\mathcal{U}_{i}$ for some $i$ is where
            $[\mathbf{x}]=[(1,0)]$ and $[\mathbf{y}]=[(0,1)]$. In this case,
            let $\mathcal{U}$ be the complement of the line spanned by
            $(1,1)$. Then $\pi(\mathcal{U})$ contains both $[\mathbf{x}]$ and
            $[\mathbf{y}]$, and by the preceeding arguments will be locally
            Euclidean. Being locally Euclidean, these two points can be
            separated by disjoint open sets. Thus, $\mathbb{RP}^{n}$ is
            Hausdorff.
        \end{proof}
        \begin{theorem}
            $\mathbb{RP}^{n}$ is second countable.
        \end{theorem}
        \begin{proof}
            We have shown that $\mathbb{RP}^{n}$ can be covered by finitely many
            open subsets each of which is homeomorphic to $\mathbb{R}^{n}$.
            But $\mathbb{R}^{n}$ is second countable, and second countability is
            preserved by homeomorphisms and hence each of these open subsets is
            second countable. But then $\mathbb{RP}^{n}$ is the finite unione of
            second countable open subspaces, and is hence second countable.
        \end{proof}
        The open subspace part is crucial. One might think that if one has a
        countable collection of subspaces that cover the space, each of which
        being second countable, then the whole space is second countable, but
        this is not true. For example, consider the \textit{infinite bouquet}.
        Take $\mathbb{R}$ with the equivalence relation $R$ defined by
        $xRy$ if and only if $x=y$ or $x$ and $y$ are integers. The quotient
        space collaps all of $\mathbb{Z}$ down to a point, and the result looks
        like infinitely many rings connected at the origin. The point
        $[\mathbb{Z}]$ has no neighborhood that has a countable basis. Thus not
        only is this space not second countable, it's not even first countable
        (but it is separable).
        \begin{theorem}
            $\mathbb{RP}^{n}$ is compact.
        \end{theorem}
        \begin{proof}
            For the restriction $\pi|_{S^{n}}:S^{n}\rightarrow\mathbb{RP}^{n}$
            is surjective. Given $[\mathbf{x}]\in\mathbb{RP}^{n}$, let
            $\mathbf{y}=\mathbf{x}/\norm{\mathbf{x}}_{2}$. Then
            $\mathbf{y}\in{S}^{n}$ and $\pi(\mathbf{y})=[\mathbf{x}]$. Thus
            $\pi|_{S^{n}}$ is a continuous surjective map. But $S^{n}$ is
            compact by the Heine-Borel theorem, and thus the image of
            $S^{n}$ under $\pi|_{S^{n}}$ is compact. Therefore,
            $\mathbb{RP}^{n}$ is compact.
        \end{proof}
        \begin{theorem}
            If $M_{1},M_{2}$ are topological manifolds, then
            $M_{1}\times{M}_{2}$ is a topological manifold.
        \end{theorem}
        \begin{proof}
            For the product of Hausdorff spaces is Hausdorff, and the product
            of second countable spaces is second countable. Thus, we must show
            it is locally Euclidean. The charts of the form
            $(\mathcal{U}\times\mathcal{V},\varphi\times\psi)$ with
            $\varphi\times\psi:\mathcal{U}\times\mathcal{V}%
             \rightarrow\mathbb{R}^{n_{1}+n_{1}}$ form an atlas on
            $M_{1}\times{M}_{2}$, and hence it is a topological manifold.
        \end{proof}
        \begin{fdefinition}{Hemicompact}{Hemicompact}
            A hemicompact topological space is a topological space
            $\topspace{X}$ such that there exists a sequence
            $K:\mathbb{N}\rightarrow\mathcal{P}(X)$ such that for all
            $n\in\mathbb{N}$ it is true that $K_{n}$ is compact, and such that
            for every compact subset $\mathcal{C}\subseteq{X}$ there exists an
            $n\in\mathbb{N}$ such that $\mathcal{C}\subseteq{K}_{n}$.
        \end{fdefinition}
        \begin{example}
            The real line is hemicompact. Take $K_{n}=[\minus{n},n]$ and apply
            the Heine-Borel theorem.
        \end{example}
        \begin{theorem}
            Compact spaces are hemicompact.
        \end{theorem}
        \begin{proof}
            Duh.
        \end{proof}
        \begin{theorem}
            Hemicompact spaces are $\sigma$ compact.
        \end{theorem}
        \begin{proof}
            It suffices to show that the union of the $K_{n}$ is the entire
            space. But if not then there is an $x\in{X}$ not contained in
            $\bigcup{K}_{n}$. But $\{x\}$ is a compact set, and thus there is an
            $n\in\mathbb{N}$ such that $\{x\}\subseteq{K}_{n}$, and hence
            $x\in\bigcup{K}_{n}$. Then, the space is $\sigma$ compact.
        \end{proof}
        \begin{theorem}
            If $\topspace{X}$ is hemicompact and first countable, then it is
            locally compact.
        \end{theorem}
        \begin{proof}
            For let $\mathcal{U}_{n}$ be a countable neighborhood basis, and
            $K_{n}$ be a sequence of compact sets that contain every compact
            subset eventually. Let:
            \begin{equation}
                \mathcal{V}_{n}=\bigcap_{k\in\mathbb{Z}_{n}}\mathcal{U}_{k}
            \end{equation}
            and let:
            \begin{equation}
                C_{n}=\bigcup_{k\in\mathbb{Z}_{n}}K_{k}
            \end{equation}
            Suppose for all $n\in\mathbb{N}$ it is true that
            $\mathcal{V}_{n}\nsubseteq{C}_{n}$. Then there exists
            $a_{n}\in\mathcal{V}_{n}$ such that $a_{n}\notin{C}_{n}$. But then
            $a_{n}\rightarrow{x}$, and hence $\{a_{n}\}\cup\{x\}$ is a compact
            subset of $X$, and hence there exists $N\in\mathbb{N}$ such that
            $K_{N}$ contains this. But $K_{N}\subseteq{C}_{N}$, and thus
            all of the $a_{n}$ are contained in $C_{N}$, a contradiction. Thus,
            there exists $N\in\mathbb{N}$ such that
            $\mathcal{V}_{N}\subseteq{C}_{N}$. But the finite intersection of
            open neighborhoods of $x$ is an open neighborhood, and the finite
            union of compact sets is compact, and hence $K_{n}$ is a compact
            neighborhood of $x$. Thus, $X$ is locally compact.
        \end{proof}
        \begin{example}
            First countable and $\sigma$ compact do not imply locally compact.
            The topologists sine curve serves as a counter-example. It is
            first countable since it is metrizable, it is $\sigma$ compact for
            let:
            \begin{equation}
                K_{n}=\{(x,\sin(x^{\minus{1}}))\in\mathbb{R}^{2}\;|\;
                    \frac{1}{n}\leq{x}\leq{1}\}
            \end{equation}
            These are all compact by Heine-Borel, and the union covers all but
            the origin $(0,0)$. Since the origin is a point, it is compact,
            thus adjoing $K_{0}=\{0,0\}$ gives us a sequence of compact sets
            that covers the whole space. This is not a sequence that makes
            the space hemicompact, in indeed there is no such sequence. To see
            this, note that the points that evaluate to zero under
            $\sin(x^{\minus{1}})$, together with the origin $(0,0)$, form a
            closed bounded subset and is thus compact, but no $K_{n}$ covers all
            of this.
        \end{example}
        \begin{example}
            Not every $\sigma$ compact space is hemicompact. The rationals
            $\mathbb{Q}$ are $\sigma$ compact since $\{q_{n}\}$ forms a
            countable set of compact sets whose union is all of $\mathbb{Q}$.
            However, $\mathbb{Q}$ is not hemicompact. For suppose
            $K_{n}$ is a sequence that contains every compact subset of
            $\mathbb{Q}$, eventually. A subset $C\subseteq\mathbb{Q}$ is compact
            if and only if it is complete and bounded. Since
            $[\minus{1}/n,1/n]$ is not complete in $\mathbb{Q}$, there exists
            $a_{n}\in[\minus{1}/n,n]$ such that $a_{n}\notin{K}_{n}$. Let
            Let $C=\{a_{n}\}\cup\{0\}$. This is compact, but not contained in
            any $K_{n}$ since $a_{n}\notin{K}_{n}$. Thus, $\mathbb{Q}$ is not
            hemicompact.
        \end{example}
    \section{Notes from O'Neill (Chapter 1)}
        A smooth function $f:\mathbb{R}^{n}\rightarrow\mathbb{R}^{m}$ is one
        such that $\pi_{k}\circ{f}:\mathbb{R}^{n}\rightarrow\mathbb{R}$ is
        smooth for all $k\in\mathbb{Z}_{m}$. A chart $(\mathcal{U},\varphi)$ in
        a topological space $(X,\tau)$ is an open set $\mathcal{U}\subseteq{X}$
        and an injective continuous open mapping
        $\varphi:\mathcal{U}\rightarrow\mathbb{R}^{n}$. The coordinate functions
        of the chart $(\mathcal{U},\varphi)$ are the functions
        $x^{k}:\mathcal{U}\rightarrow\mathbb{R}$ defined by
        $x^{k}=\pi_{k}\circ\varphi$ for all $k\in\mathbb{Z}_{n}$. Smoothly
        overlapping charts are charts $(\mathcal{U},\varphi)$ and
        $(\mathcal{V},\psi)$ such that either
        $\mathcal{U}\cap\mathcal{V}=\emptyset$ or both
        $\varphi\circ\psi^{\minus{1}}$ and $\psi\circ\varphi^{\minus{1}}$ are
        smooth functions (in the Euclidean sense). The domain of
        $\varphi\circ\psi^{\minus{1}}$ is $\psi(\mathcal{V})$, which is an open
        subset of $\mathbb{R}^{n}$, and hence this composition is a function
        from an open subset of $\mathbb{R}^{n}$ to $\mathbb{R}^{n}$ and thus
        asking if it is smooth is a valid well defined question. Similarly for
        $\psi\circ\varphi^{\minus{1}}$. A smooth atlas is a collection of charts
        $(\mathcal{U}_{\alpha},\varphi_{\alpha})$ such that the
        $\mathcal{U}_{\alpha}$ cover $X$ and each chart overlaps smoothly. A
        maximal smooth atlas is a smooth atlas $\mathcal{A}$ on $(X,\tau)$ such
        that for all smooth atlases $\mathcal{A}'$ such that
        $\mathcal{A}\subseteq\mathcal{A}'$ it is true that
        $\mathcal{A}=\mathcal{A}'$.
        \begin{theorem}
            If $(X,\tau)$ is a topological space, if $\mathcal{A}$ and
            $\mathcal{A}'$ are smooth atlases on $(X,\tau)$, then they are
            compatible if and only if $\mathcal{A}\cup\mathcal{A}'$ is a smooth
            atlas on $(X,\tau)$.
        \end{theorem}
        \begin{proof}
            If they are compatible, then every chart in $\mathcal{A}$ overlaps
            smoothly with every chart in $\mathcal{A}'$, and vice versa. But
            $\mathcal{A}$ is a smooth atlas and thus every chart in
            $\mathcal{A}$ overlaps smoothly with every other chart in
            $\mathcal{A}$. Thus, every chart in $\mathcal{A}$ overlaps smoothly
            with every chart in $\mathcal{A}\cup\mathcal{A}'$, and similarly for
            $\mathcal{A}'$. Hence if $\mathcal{A}$ and $\mathcal{A}'$ are
            compatible, then $\mathcal{A}\cup\mathcal{A}'$ is a smooth atlas.
            Moreover, if $\mathcal{A}\cup\mathcal{A}'$ is a smooth atlas, then
            every element of $\mathcal{A}\cup\mathcal{A}'$ overlaps smoothly
            with every other element of $\mathcal{A}\cup\mathcal{A}'$. In
            particular, every element of $\mathcal{A}$ overlaps smoothly with
            every element of $\mathcal{A}'$, and vice-versa. Hence,
            $\mathcal{A}$ and $\mathcal{A}'$ are compatible.
        \end{proof}
        \begin{theorem}
            If $\mathcal{A}$ is a smooth atlas on $(X,\tau)$, then there is a
            unique maximal smooth atlas $\mathcal{C}$ such that
            $\mathcal{A}\subseteq\mathcal{C}$.
        \end{theorem}
        \begin{proof}
            The set of all atlases on $(X,\tau)$ forms a partially ordered set
            by inclusion. Given any chain of atlases, by the previous theorem
            the union is then again an atlas compatible with all atlases in the
            chain. That is, every chain has an upper bound. Hence by Zorn's
            lemma every Atlas has a maximal atlas. If there is another maximal
            atlas, the union will again be a smooth atlas, contradicting
            maximality, and hence there is a unique maximal smooth atlas.
        \end{proof}
        A smooth manifold is a second countable Hausdorff topological space with
        a maximal smooth atlas.
        \begin{theorem}
            If $(X,\tau,\mathcal{A})$ is a smooth manifold, if
            $(\mathcal{U},\varphi)\in\mathcal{A}$, and if $\mathcal{V}$ is an
            open subset such that $\mathcal{U}\cap\mathcal{V}\ne\emptyset$, then
            $(\mathcal{U}\cap\mathcal{V},%
             \varphi|_{\mathcal{U}\cap\mathcal{V}})\in\mathcal{A}$.
        \end{theorem}
        \begin{proof}
            For since $\mathcal{U}$ and $\mathcal{V}$ are open,
            $\mathcal{U}\cap\mathcal{V}$ is open. But then
            $\varphi|_{\mathcal{U}\cap\mathcal{V}}$ is a continuous and
            injective open mapping. If $(\mathcal{O},\psi)\in\mathcal{A}$, then
            it overlaps smoothly with $(\mathcal{U},\varphi)$. But smoothness
            is a local property, and hence it overlaps smoothly with
            $(\mathcal{U}\cap\mathcal{V},\varphi|_{\mathcal{U}\cap\mathcal{V}})$
            and since $\mathcal{A}$ is maximal, it contains this chart.
        \end{proof}
        \begin{theorem}
            If $(X,\tau,\mathcal{A})$ is a smooth manifold and $A\in\tau$ is an
            open subset of $X$, then $(A,\tau|_{A},\mathcal{A}|_{A})$ is a
            smooth manifold where $\tau|_{A}$ is the subspace topology and
            is the subatlas defined by:
            \begin{equation}
                \mathcal{A}|_{A}=
                \{\,(\mathcal{U},\varphi)\in\mathcal{A}\;|\;\,
                    \mathcal{U}\subseteq{A}\,\}
            \end{equation}
        \end{theorem}
        \begin{definition}
            The product function of a function
            $\varphi:\mathcal{U}\rightarrow{X}$ and a function
            $\psi:\mathcal{V}\rightarrow{Y}$ is the function
            $\varphi\times\psi:%
             \mathcal{U}\times\mathcal{V}\rightarrow{X}\times{Y}$ defined by:
             \begin{equation*}
                \varphi\times\psi(u,v)=\big(\varphi(u),\psi(v)\big)
             \end{equation*}
        \end{definition}
        \begin{theorem}
            If $\manifold[X]{X}$ and $\manifold[Y]{Y}$ smooth manifolds of
            dimension $n$ and $m$, respectively, then
            $\manifold[X\times{Y}]{X\times{Y}}$ is a smooth manifold of
            dimension $n+m$.
        \end{theorem}
        The natural smooth structure on $\mathbb{R}^{n}$ can be seen as the
        product manifold of $n$ copies of $\mathbb{R}$.
        \subsection{Smooth Functions}
            \begin{definition}
                A smooth function from a manifold $\manifold{M}$ into
                $\mathbb{R}$ is a function $f:M\rightarrow\mathbb{R}$ such that
                for all $(\mathcal{U},\varphi)\in\mathcal{A}$ the function
                $f\circ\varphi^{\minus{1}}$ is Euclidean smooth.
            \end{definition}
            \begin{theorem}
                A function $f:M\rightarrow\mathbb{R}$ is smooth if and only if
                there is a cover $\mathcal{C}$ or charts such that for all
                $(\mathcal{U},\varphi)\in\mathcal{C}$ the function
                $f\circ\varphi^{\minus{1}}$ is Euclidean smooth.
            \end{theorem}
            \begin{proof}
                Since the atlas of a smooth manifold is maximal, $\mathcal{C}$
                is contained in it. But charts in a smooth atlas overlap
                smoothly, and thus if $(\mathcal{V},\psi)$ is a different chart
                then we have:
                \begin{equation}
                    f\circ\psi^{\minus{1}}
                        =f\circ(\varphi^{\minus{1}}\circ\varphi)
                            \circ\psi^{\minus{1}}
                        =(f\circ\varphi^{\minus{1}})
                            \circ(\varphi\circ\psi^{\minus{1}})
                \end{equation}
                The composition of smooth functions, and hence smooth.
            \end{proof}
            \begin{definition}
                A smooth function from a manifold $M$ to $\mathbb{R}^{m}$ is a
                function $f:M\rightarrow\mathbb{R}^{m}$ such that
                $f\circ\pi_{k}$ is smooth for all $k\in\mathbb{Z}_{m}$.
            \end{definition}
            \begin{definition}
                The set of all smooth functions from a smooth manifold
                $\manifold{M}$ to $\mathbb{R}$ is denoted
                $C^{\infty}(M,\mathbb{R})$.
            \end{definition}
            \begin{theorem}
                If $f,g\in{C}^{\infty}(M,\mathbb{R})$, then
                $f+g\in{C}^{\infty}(M,\mathbb{R})$ and
                $fg\in{C}^{\infty}(M,\mathbb{R})$.
            \end{theorem}
            The space $C^{\infty}(M,\mathbb{R})$ is a vector space over
            $\mathbb{R}$. Moreover, it is an algebra over $\mathbb{R}$ since
            multiplication of vectors makes since. $fg$ is simply the function
            $(fg)(p)=f(p)g(p)$. This operation is distributive and associative,
            and hence $C^{\infty}(M,\mathbb{R})$ is an associative algebra over
            the field of real numbers $\mathbb{R}$. Moreover, there is a unital
            element since the constant mapping $f(p)=1$ is such that
            $fg=g$ for all $g\in{C}^{\infty}(M,\mathbb{R})$. Hene, this space is
            also a unital algebra. Another algebraic structure that
            $C^{\infty}(M,\mathbb{R})$ has is that of a commutative ring. We can
            now extend the notion of smoothness to functions between arbitrary
            manifolds.
            \begin{definition}
                A smooth function from a smooth manifold $\manifold[M]{M}$ to a
                smooth manifold $\manifold[N]{N}$ is a function
                $f:M\rightarrow{N}$ such that for all
                $(\mathcal{U},\varphi)\in\mathcal{A}_{M}$ and for all
                $(\mathcal{V},\psi)\in\mathcal{A}_{N}$, the function
                $\psi\circ{f}\circ\varphi^{\minus{1}}$ is Euclidean smooth.
            \end{definition}
            This function $\psi\circ{f}\circ\varphi^{\minus{1}}$ is defined on
            the overlaps of the domains in question. That is, the domain is
            $\varphi(\mathcal{U}\cap{f}^{\minus{1}}(\mathcal{V}))$, which is an
            open subset of $\mathbb{R}^{m}$ since $\varphi$ is an open mapping,
            $\mathcal{U}$ is open, and $f^{\minus{1}}(\mathcal{V})$ is open
            since $f$ is smooth, and hence continuous. The range is some subset
            of $\mathbb{R}^{n}$. Specifically, the range is the set
            $\psi[\mathcal{V}\cap\phi[\mathcal{U}]]$, so we defined the
            function:
            \begin{equation}
                \psi\circ\phi\circ\varphi^{\minus{1}}:
                    \varphi\big[\mathcal{U}\cap
                        \phi^{\minus{1}}(\mathcal{V})\big]\rightarrow
                    \psi\big[\mathcal{V}\cap\phi[\mathcal{U}]\big]
            \end{equation}
            The schematic for this can be seen in
            Fig.~\ref{fig:Smooth_Function_Between_Manifolds}.
            \begin{figure}[H]
                \centering
                \captionsetup{type=figure}
                \begin{tikzpicture}[>=Latex]
    \draw (0.0, 0.0)    to[out=90,  in=150]  (1.0,  1.0)
                        to[out=-30, in=140]  (5.0,  1.0)
                        to[out=-40, in=90]   (5.0, -1.0)
                        to[out=-90, in=0]    (3.0, -1.0)
                        to[out=180, in=-50]  (1.0, -1.0)
                        to[out=130, in=-90]  cycle;

    \draw (0.5, 0.1) to[in=-130, out=-50] (1.8, 0.1);
    \draw (0.6, 0.0) to[in=130,  out=50]  (1.7, 0.0);

    \draw[fill=blue!80!white,opacity=0.6,densely dashed]
        (2.5, -0.5) to[out=-60,  in=-110] (4.0, -0.2)
                    to[out=70,   in=-90]  (4.5,  0.3)
                    to[out=90,   in=10]   (4.0,  0.8)
                    to[out=-170, in=120]  cycle;

    \draw[fill=red!80!white,opacity=0.6,densely dashed]
        (3.0, -0.5) to[out=-60,  in=-160] (4.4, -0.8)
                    to[out=20,   in=-100] (4.8, -0.8)
                    to[out=80,   in=40]   (4.5,  0.2)
                    to[out=-140, in=120]  cycle;

    \node at (3.8,  0.4) {$\mathcal{U}$};
    \node at (4.3, -0.5) {$\phi^{\minus{1}}[\mathcal{V}]$};
    \node at (1.5, -0.8) {$M$};

    \draw[->] (5.2, 1) to[out=30, in=150] node[above] {$\phi$} (6.8, 1);

    \begin{scope}[xshift=6.4cm]
        \draw (0.0, 0.0)    to[out=90,  in=150]  (2.0,  1.0)
                            to[out=-30, in=160]  (4.0,  1.0)
                            to[out=-20, in=90]   (5.0, -0.5)
                            to[out=-90, in=0]    (3.0, -1.5)
                            to[out=180, in=-50]  (0.0, -1.2)
                            to[out=130, in=-90]  cycle;

        \draw[fill=blue!80!white,opacity=0.6,densely dashed]
            (0.8, -0.5) to[out=-60,  in=-110] (3.0, -1.0)
                        to[out=90,   in=10]   cycle;

        \draw[fill=red!80!white,opacity=0.6,densely dashed]
            (0.8, -0.5) to[out=-60,  in=-110] (3.0, -1.0)
                        to[out=70,   in=-90]  (3.5,  0.0)
                        to[out=90,   in=10]   (2.5, 0.5)
                        to[out=-170, in=120]  cycle;

        \node at (4.0, -0.8) {$N$};
        \node at (2.8, -0.2) {$\mathcal{V}$};
        \node at (2.1, -0.9) {$\phi[\mathcal{U}]\cap\mathcal{V}$};
    \end{scope}

    \draw[->] (2.0, -1.5) to node[right] {$\varphi$} (1.5, -2.5);
    \begin{scope}[yshift=-5.0cm]
        \draw[->] (-0.5,  0.0) to (3.0, 0.0);
        \draw[->] ( 0.0, -0.5) to (0.0, 3.0) node[right] {$\mathbb{R}^{m}$};

        \draw[fill=blue!80!white,opacity=0.6,densely dashed]
            (0.0, 0.0) to[out=-60,  in=-110] (2.7, 0.1)
                       to[out=70,   in=-90]  (3.1, 2.1)
                       to[out=90,   in=10]   (1.0, 2.1)
                       to[out=-170, in=120]  cycle;

        \draw[fill=red!80!white,opacity=0.6,densely dashed]
            (2.7, 0.1) to[out=70,   in=-90]  (3.1, 2.1)
                       to[out=90,   in=10]   (1.0, 2.1)
                       to[out=-170, in=-110] cycle;

        \node at (0.6, 0.5) {$\varphi[\mathcal{U}]$};
        \node at (2.0, 1.5)
            {$\varphi\big[\mathcal{U}\cap\phi^{\minus{1}}[\mathcal{V}]\big]$};
    \end{scope}
    \draw[->] (9.2, -1.7) to node[right] {$\psi$} (9.6, -2.7);
    \begin{scope}[xshift=8.0cm,yshift=-5.0cm]
        \draw[->] (-0.5,  0.0) to (3.0, 0.0);
        \draw[->] ( 0.0, -0.5) to (0.0, 3.0) node[right] {$\mathbb{R}^{n}$};

        \draw[fill=blue!80!white,opacity=0.6,densely dashed]
            (1.5, 1.9) to[out=-170, in=120]  (0.2, 0.9)
                       to[out=-60,  in=-110] (2.2, 0.9)
                       to[out=90,   in=10]   cycle;

        \draw[fill=red!80!white,opacity=0.6,densely dashed]
            (0.2, 0.9) to[out=-60,  in=-110] (2.2, 0.9)
                       to[out=70,   in=-90]  (3.0, 1.6)
                       to[out=90,   in=10]   (1.5, 1.9)
                       to[out=-170, in=120]  cycle;

        \node at (2.5, 1.5) {$\psi[\mathcal{U}]$};
        \node at (1.1, 1.1) {$\psi\big[\mathcal{V}\cap\phi[\mathcal{U}]\big]$};
    \end{scope}
    \draw[->] (3.5, -4.5) to[out=-30, in=-150] node[below]
        {$\psi\circ\phi\circ\varphi^{\minus{1}}$} (7.5, -4.5);
\end{tikzpicture}
                \caption{Smooth Function Between Manifolds}
                \label{fig:Smooth_Function_Between_Manifolds}
            \end{figure}
            \begin{theorem}
                If $\manifold{X}$ is a smooth manifold, then the identity
                mapping $\textrm{id}_{X}:X\rightarrow{X}$ is a smooth function.
            \end{theorem}
            \begin{theorem}
                If $\manifold[X]{X}$, $\manifold[Y]{Y}$, and $\manifold[Z]{Z}$
                are smooth manifolds, if $f:X\rightarrow{Y}$ and
                $g:Y\rightarrow{Z}$ are smooth functions, then $g\circ{f}$ is a
                smooth function.
            \end{theorem}
            \begin{theorem}
                If $\manifold{X}$ is a smooth manifold,
                $(\mathcal{U},\varphi)\in\mathcal{A}$, and if
                $\manifold[\mathcal{U}]{\mathcal{U}}$ is the open submanifold,
                then $\varphi:\mathcal{U}\rightarrow\mathbb{R}^{n}$ is a smooth
                function.
            \end{theorem}
            From this we have that all of the coordinate functions are smooth,
            since $\varphi\circ\pi_{k}$ is the composition of smooth functions,
            which is smooth. Smoothness is a local property. A function is
            smooth if and only if it is locally smooth at every point.
            \begin{theorem}
                If $\manifold[X]{X}$ and $\manifold[Y]{Y}$ are smooth manifolds,
                and if $f:X\rightarrow{Y}$ is smooth, then it is continuous.
            \end{theorem}
            Here, smoothness is a property of the atlases $\mathcal{A}_{X}$ and
            $\mathcal{A}_{Y}$, whereas continuity is a property of the
            topologies $\tau_{X}$ and $\tau_{Y}$. This theorem shows that
            topological information is encoded into the structure of atlases.
            \begin{theorem}
                If $\manifold{X}$ is a smooth manifold, if $\mathcal{O}$ is an
                open cover, if $\mathcal{F}$ is a collection of smooth functions
                $\varphi_{\alpha}:\mathcal{U}_{\alpha}\rightarrow{N}$ for each
                $\mathcal{U}_{\alpha}\in\mathcal{O}$, then there is a unique
                function $\varphi|_{\mathcal{U}\cap\mathcal{V}}=%
                \psi|_{\mathcal{U}\cap\mathcal{V}}$
                $\varphi:\bigcup\mathcal{O}\rightarrow{N}$ such that
                $\varphi|_{\mathcal{U}_{\alpha}}=\varphi_{\alpha}$.
            \end{theorem}
            This stems from the local property of smoothness. If we can define
            a bunch of functions on various patches of a manifold, and if these
            functions agree on the overlap, then we can define a single function
            on the union of all such sets.
            \begin{definition}
                A diffeomorphism from a manifold $\manifold[X]{X}$ to a manifold
                $\manifold[Y]{Y}$ is a smooth bijection $f:X\rightarrow{Y}$
                such that $f^{\minus{1}}:Y\rightarrow{X}$ is smooth.
            \end{definition}
            By the previous theorem, any diffeomorphism is automatically a
            homeomorphism since smooth functions are continuous.
            \begin{theorem}
                If $\manifold{X}$ is a smooth manifold, then the identity map
                $\textrm{id}_{X}:X\rightarrow{X}$ is a diffeomorphism.
            \end{theorem}
            \begin{theorem}
                The composition of diffeomorphisms is a diffeomorphism.
            \end{theorem}
            \begin{theorem}
                The inverse of a diffeomorphism is a diffeomorphism.
            \end{theorem}
            \begin{theorem}
                The restriction of a diffeomorphism to an open subset of a
                smooth manifold is a diffeomorphism onto its image.
            \end{theorem}
            \begin{example}
                The interval $(a,b)$ is diffeomorphic to $(0,1)$, which is
                diffeomorphic to all of $\mathbb{R}$.
            \end{example}
            \begin{theorem}
                If $\manifold[X]{X}$ is a manifold, if $Y$ is a set, and if
                $f:X\rightarrow{Y}$ is a bijective function, then there is a
                unique topology $\tau_{Y}$ and a unique maximal atlas
                $\mathcal{A}_{Y}$ such that $f$ is a diffeomorphism.
            \end{theorem}
            \begin{proof}
                That there is a unique topology comes from point-set topology.
                Define:
                \begin{equation}
                    \tau_{Y}=\{\,f[\mathcal{U}]\;|\;\mathcal{U}\in\tau_{X}\}
                \end{equation}
                This is a topology since $f$ is bijective. That is,
                $Y=f[X]$, $\emptyset=f[\emptyset]$, and the forward image
                preserves unions and intersections:
                \begin{align}
                    f\Big[\bigcup_{\mathcal{U}\in\mathcal{O}}\mathcal{U}\Big]
                    &=\bigcup_{\mathcal{U}\in\mathcal{O}}f[\mathcal{U}]\\
                    f[\mathcal{U}\cap\mathcal{V}]
                        &=f[\mathcal{U}]\cap{f}[\mathcal{V}]
                \end{align}
                Hence, $\tau_{Y}$ will be closed to finite intersections and
                arbitrary unions. It is an open mapping by definition, and it is
                continuous since $f$ is a bijection and hence the pre-image of
                open subsets of $Y$ will be open subsets of $X$ by definition.
                Thus, $\tau_{Y}$ is a topology that makes $f$ a continuous open
                mapping, and hence a homeomorphism. For the atlas, define:
                \begin{equation}
                    \mathcal{A}_{Y}=
                        \{(f[\mathcal{U}],f^{\minus{1}}\circ\varphi)
                            \;|\;(\mathcal{U},\varphi)\in\mathcal{A}_{X}\}
                \end{equation}
                Thus $\manifold[Y]{Y}$ a smooth manifold that is
                diffeomorphic to $\manifold[X]{X}$.
            \end{proof}
            \begin{figure}[H]
                \centering
                \captionsetup{type=figure}
                \begin{tikzpicture}[>=Latex]
    \draw (0, 0) to[out=-30, in=-120] (2, -1) to[out=60,  in=-90]  (3, 0)
                 to[out=90,  in=0]    (1,  1) to[out=180, in=150]  cycle;

    \draw (6, 0) to[out=-90, in=180]  (7, -1) to[out=0,   in=-135] (9, 0)
                 to[out=45,  in=-80]  (8,  1) to[out=100, in=90]   cycle;

    \draw[fill=red!80!white,opacity=0.6]
        (1, 0) to[out=-60, in=-150] (1.5, -0.5) to[out=50,  in=-80]  (2.0, 0.0)
               to[out=100, in=0]    (1.5,  0.5) to[out=180, in=120]  cycle;

    \draw[fill=blue!80!white, opacity=0.6]
        (7, 0) to[out=-90, in=180] (7.5, -0.5) to[out=0,   in=-135] (8.0,  0.0)
               to[out=45,  in=0]   (7.5,  0.5) to[out=180, in=90]   cycle;

    \node at (7.5, 0.0) {$\mathcal{U}$};
    \node at (1.5, 0.0) {$f[\mathcal{U}]$};

    \draw[->] (3.5, -4.0) to (6.5, -4.0);
    \draw[->] (4.0, -4.5) to (4.0, -1.5) node[right] {$\mathbb{R}^{n}$};

    \draw[->] (3.5, -0.3) to node[below] {$f$}             (5.5, -0.3);
    \draw[->] (5.5,  0.3) to node[above] {$f^{\minus{1}}$} (3.5,  0.3);

    \draw[->] (6.7, -1.2) to node[below right] {$\varphi$} (6.0, -2.0);
    \draw[->] (2.3, -1.2) to
        node[below left]  {$\varphi\circ{f}^{\minus{1}}$} (3.0, -2.0);

    \draw[fill=blue!50!red, opacity=0.6]
        (4.5, -3.0) to[out=-90, in=180]  (5.0, -3.5)
                    to[out=0,   in=-110] (5.5, -3.0)
                    to[out=70,  in=0]    (5.0, -2.5)
                    to[out=180, in=90]   cycle;

    \node at (5.0, -3.0) {$\varphi[\mathcal{U}]$};
\end{tikzpicture}
                \caption{Creating a Manifold Structure on an Arbitrary Set}
                \label{fig:Manifold_Structure_on_Arbitrary_Set}
            \end{figure}
            Since there is a unique topology such that $f$ is a homeomorphism,
            and since diffeomorphisms are automatically homeomorphisms, the
            topology $\tau_{Y}$ part is done. To find $\mathcal{A}_{Y}$ we
            simply look at the forward image of all of the charts in
            $\mathcal{A}_{X}$ by the bijection $f$.
            \begin{example}
                Much the way a continuous bijection need not be a homeomorphism,
                a smooth homeomorphism need not be a diffeomorphism. The classic
                example is the mapping $f(x)=x^{3}$. This is indeed a
                homeomorphism, and is smooth, but $f^{\minus{1}}(x)=x^{1/3}$,
                which is not differentiable at the origin (let alone smooth).
            \end{example}
            Every chart is a diffeomorphism from it's domain to an open subset
            of $\mathbb{R}^{n}$.
            \begin{definition}
                A bump function on an open subset $\mathcal{U}$ in a topological
                space $\topspace{X}$ about a point $x\in{X}$ is a function
                $f:X\rightarrow\mathbb{R}$ such that $f[X]=[0,1]$,
                $\textrm{supp}(f)\subseteq\mathcal{U}$ and such that there
                exists an open subset $\mathcal{V}\in\tau$ such that
                $f[\mathcal{V}]=\{1\}$.
            \end{definition}
            \begin{theorem}
                If $\manifold{X}$ is a smooth manifold, if $x\in{X}$, if
                $\mathcal{U}\in\tau$ is such that $x\in\mathcal{U}$, then there
                is a smooth bump function $f:X\rightarrow\mathbb{R}$ about $x$
                contained in $\mathcal{U}$.
            \end{theorem}
            \begin{proof}
                For given $\varepsilon>0$, let $f:\mathbb{R}\rightarrow[0,1]$ be
                defined by:
                \begin{equation}
                    f(t)=\begin{cases}
                        \textrm{exp}\Big(
                            \minus\frac{(2\varepsilon-x)^{2}}
                                {4\varepsilon^{2}(x-\varepsilon)^{2}}
                            \Big),&\varepsilon\leq{x}\leq{2}\varepsilon\\
                            0,&x<\varepsilon\\
                            1,&2\varepsilon<x
                    \end{cases}
                \end{equation}
                Then $f(\varepsilon)=0$, $f(2\varepsilon)=1$, and all
                derivatives evaluate to zero at both points, and hence this
                function is smooth. Since $\manifold{X}$ is a manifold, there is
                a chart $(\mathcal{V},\varphi)$ such that $x\in\mathcal{V}$.
                But $x\in\mathcal{U}$, and $\mathcal{U}\in\tau$ is open, and
                hence $\mathcal{U}\cap\mathcal{V}$ is a non-empty open subset.
                But $\mathcal{A}$ is maximal, and so
                $(\mathcal{U}\cap\mathcal{V},%
                 \varphi|_{\mathcal{U}\cap\mathcal{V}})\in\mathcal{A}$.
                Relabel this as $(\mathcal{O},\psi)$. Then
                $\psi[\mathcal{O}]$ is open and $\psi(x)\in\psi[\mathcal{O}]$
                and thus there is an $\varepsilon>0$ such that the ball of
                radius $\sqrt{\varepsilon}$ centered about $\psi(x)$ is
                contained in $\psi[\mathcal{O}]$. Define
                $h:M\rightarrow\mathbb{R}$ to be:
                \begin{equation}
                    h(p)=
                    \begin{cases}
                        f\big(\norm{\psi(p)}_{2}^{2}\big),
                            &\norm{\psi(x)-\psi(p)}_{2}^{2}<\varepsilon\\
                        0,&\textrm{otherwise}
                    \end{cases}
                \end{equation}
            \end{proof}
        \subsection{Tangent Vectors}
            We now wish to generalize the concept of the tangent plane found in
            calculus. To do this needs a generalization of the directional
            derivative that is used to define such spaces. The key properties
            are linearity and the Liebniz rule.
            \begin{definition}
                A tangent vector at a point $p$ in a manifold $\manifold{X}$ is
                a linear functional
                $v:C^{\infty}(M,\mathbb{R})\rightarrow\mathbb{R}$ that is
                Liebnizian. That is, for all $a,b\in\mathbb{R}$ and for all
                $f,g\in{C}^{\infty}(M,\mathbb{R})$:
                \begin{align}
                    v(af+bg)&=av(f)+bv(g)\tag{Linearity}\\
                    v(fg)&=v(f)g(p)+f(p)v(g)\tag{Liebnizian}
                \end{align}
            \end{definition}
            \begin{definition}
                The tangent space at a point $p$ in a manifold $\manifold{X}$
                is the set $T_{p}X$ of all tangent vectors to $p$.
            \end{definition}
            If we define function addition and scalar multiplication in the
            usual way, then $T_{p}X$ is a vector space. That is:
            \begin{align}
                (v+w)(f)&=v(f)+w(g)\\
                (av)(f)&=av(f)
            \end{align}
            \begin{definition}
                The partial derivative of a smooth function
                $f\in{C}^{\infty}(M,\mathbb{R})$ on a smooth manifold
                $\manifold{X}$ at a point $x\in{X}$ with respect to a chart
                $(\mathcal{U},\varphi)$ in the $k^{th}$ direction, with
                $k\in\mathbb{Z}_{n}$, is the value:
                \begin{equation}
                    \partial_{k}f(x)=
                        \frac{\partial}{\partial{x}_{k}}
                        \Big(f\circ\varphi^{\minus{1}}\Big)
                \end{equation}
            \end{definition}
            This is well defined since $f\circ\varphi^{\minus{1}}$ is a smooth
            function from an open subset of $\mathbb{R}^{n}$ into $\mathbb{R}$,
            and thus from analysis we may take derivatives of arbitrary order
            in any of the components.
            \begin{theorem}
                If $\manifold{X}$ is a smooth manifold, if $x\in{X}$, if
                $(\mathcal{U},\varphi)\in\mathcal{A}$, if $x\in\mathcal{U}$, and
                if $\partial_{k}|_{x}:C^{\infty}(M,\mathbb{R})%
                    \rightarrow\mathbb{R}$ is defined by
                $\partial_{k}|_{x}(f)=\partial_{k}(f)(x)$, then
                $\partial_{k}|_{x}\in{T}_{x}(X)$.
            \end{theorem}
            Tangent spaces are the best linear approximation to the manifold
            $\manifold{X}$ about the point $x$. This is made clear by the
            following thoerems.
            \begin{theorem}
                If $\manifold{X}$ is a smooth manifold, if $x\in{X}$, if
                $\vector{0}:X\rightarrow\mathbb{R}$ is the zero function, and if
                $v\in{T}_{x}(X)$, then $v(\vector{0})=0$.
            \end{theorem}
            \begin{proof}
                For:
                \begin{equation}
                    v(\vector{0})=v(\vector{0}-\vector{0})
                    =v(\vector(0))-v(\vector(0))=0
                \end{equation}
            \end{proof}
            \begin{theorem}
                If $\manifold{X}$ is a smooth manifold, if $x\in{X}$, if
                $\vector{1}:X\rightarrow\mathbb{R}$ is the function such that
                $f(p)=1$ for all $p\in{X}$, and if $v\in{T}_{x}(X)$, then
                $v(\vector{1})=0$.
            \end{theorem}
            \begin{proof}
                For:
                \begin{equation}
                    v(\vector{1})=v(\vector(1)\cdot\vector(1))=
                        v(\vector{1})\cdot\vector{1}(x)+
                        \vector{1}(x)\cdot{v}(\vector{1})
                        =v(\vector{1})+v(\vector{1})
                \end{equation}
                Thus, from the cancellation law, $v(\vector{1})=0$.
            \end{proof}
            \begin{theorem}
                If $\manifold{X}$ is a smooth manifold, if $\mathcal{U}\in\tau$,
                if $x\in\mathcal{U}$, if $f,g\in{C}^{\infty}(M,\mathbb{R})$,
                if $f|_{\mathcal{U}}=g|_{\mathcal{U}}$, and if $v\in{T}_{x}(X)$,
                then $v(f)=v(g)$.
            \end{theorem}
            \begin{proof}
                For there exists a bump function $\alpha$ about the point $x$
                in the open set $\mathcal{U}$. Let $h=f-g$. But then:
                \begin{equation}
                    v(\alpha\cdot{h})v(\alpha)\cdot{h}(x)+v(h)\cdot\alpha(x)
                \end{equation}
                But $\alpha(x)=1$ and $h(x)=0$ since $h(x)=f(x)-g(x)$, and
                $f|_{\mathcal{U}}=g|_{\mathcal{U}}$ and the point $x$ is
                contained in $\mathcal{U}$. But $\alpha\cdot{h}$ is the zero
                function. For if $p\in\mathcal{U}$< then $h(p)=0$, and if
                $p\notin\mathcal{U}$ then $\alpha(p)=0$. But then
                $v(\alpha\cdot{h})=v(\vector{0})=0$ by the previous theorem.
                Therefore $v(h)=0$, and $v(f)=v(g)$.
            \end{proof}
            \begin{theorem}
                If $\manifold{X}$ is a smooth manifold, if $x\in{X}$, if
                $\mathcal{U}\in\tau$ is an open neighborhood of $x$, if
                $f\in{C}^{\infty}(X,\mathbb{R})$ is a constant mapping, and if
                $v\in{T}_{x}(X)$, then $v(x)=0$.
            \end{theorem}
            \begin{proof}
                For if $f$ is a constant, then $f=c\cdot\vector{1}$ for some
                $c\in\mathbb{R}$. But then:
                \begin{equation}
                    v(f)=v(c\cdot\vector{1})=cv(\vector{1})=c\cdot{0}=0
                \end{equation}
            \end{proof}
            \begin{theorem}
                If $\manifold{X}$ is a smooth manifold, if $x\in{X}$, if
                $(\mathcal{U},\tau)\in\mathcal{A}$ is such that $\mathcal{U}$
                $x\in\mathcal{U}$, if $\manifold[\mathcal{U}]{\mathcal{U}}$ is
                the open submanifold structure of $\mathcal{U}$, then there is
                an isomorphism $\phi:T_{x}(\mathcal{U})\rightarrow{T}_{x}(X)$.
            \end{theorem}
            \begin{proof}
                For given $v\in{T}_{x}(\mathcal{U})$, let $\phi(v)\in{T}_{x}(X)$
                be such that $\phi(v)(f)=v(f|_{\mathcal{U}})$ for all
                $f\in{C}^{\infty}(X,\mathbb{R})$. By the previous theorems, this
                is a bijection and is also linear, and hence an isomorphism.
            \end{proof}
            \begin{ftheorem}{Tangent Space Basis Theorem}
                            {Tangent_Space_Basis_Theorem}
                If $\manifold{X}$ is a smooth manifold of dimension
                $n\in\mathbb{N}$, if $p\in{X}$, and if
                $(\mathcal{U},\varphi)\in\mathcal{A}$ is a chart such that
                $x\in\mathcal{U}$, then the set
                $\{\partial_{k}|_{x}|k\in\mathbb{Z}_{n}\}$ is a basis for
                $T_{x}(X)$, where $\partial_{k}|_{x}$ is the function that maps
                $f\in{C}^{\infty}(X,\mathbb{R})$ to
                $\partial/\partial{x}_{k}(f\circ\varphi^{\minus{1}})|_{x}$.
                Moreover, for all $v\in{T}_{x}(X)$, the following is true:
                \begin{equation}
                    v=\sum_{k\in\mathbb{Z}_{n}}v(x^{k})\partial_{k}|_{p}
                \end{equation}
            \end{ftheorem}
        \subsection{The Differential Pushforward}
            \begin{definition}
                The pushforward of a tangent vector $v$ in a smooth manifold
                $\manifold[M]{M}$ at a point $p\in{M}$ by a smooth function
                $\phi:M\rightarrow{N}$ into a smooth manifold $\manifold[N]{N}$
                is the function
                $\diff\phi_{p}(v):\Ckspace{\infty}{N}\rightarrow\mathbb{R}$
                defined by:
                \begin{equation}
                    \diff\phi_{p}(v)(f)=v(f\circ\phi)
                \end{equation}
            \end{definition}
            The most important aspect of the pushforward is that it maps tangent
            vectors in $M$ to tangent vectors in $N$.
            \begin{theorem}
                If $\manifold[M]{M}$ and $\manifold[N]{N}$ are smooth manifolds,
                if $\phi:M\rightarrow{N}$ is a smooth function, if $p\in{M}$,
                if $v\in\tanspace{p}{M}$, and if $\diff\phi_{p}$ is the
                differential pushforward of $\phi$, then
                $\diff\phi_{p}(v)\in\tanspace{\phi(p)}{N}$.
            \end{theorem}
            \begin{proof}
                For if $a,b\in\mathbb{R}$, $f,g\in\Ckspace{\infty}{M}$, then:
                \begin{subequations}
                    \begin{align}
                        \diff\phi_{p}(v)(af+bg)
                            &=v\big((af+bg)\circ\phi\big)\\
                            &=v\big(a(f\circ\phi)+b(g\circ\phi)\big)\\
                            &=av(f\circ\phi)+bv(g\circ\phi)\\
                            &=a\diff\phi_{p}(v)(f)+b\diff\phi_{p}(v)(g)
                    \end{align}
                \end{subequations}
                Finally, if $f,g\in\Ckspace{\infty}{N}$, then:
                \begin{subequations}
                    \begin{align}
                        \diff\phi_{p}(v)(fg)
                            &=v\big((fg)\circ\phi\big)\\
                            &=v\big((f\circ\phi)(g\circ\phi)\big)\\
                            &=v(f\circ\phi)g\big(\phi(p)\big)\\
                            &=v\big(\phi(p)\big)v(g\circ\phi)\\
                            &=v(f\circ\phi)g\big(\phi(p)\big)
                             +f\big(\phi(p)\big)v(g\circ\phi)\\
                            &=\diff\phi_{p}(v)(f)g\big(\phi(p)\big)
                             +f\big(\phi(p)\big)\diff\phi_{p}(v)(g)
                    \end{align}
                \end{subequations}
                Thus, $\diff\phi_{p}(v)$ is a tangent vector to $\phi(p)$.
                Hence, $\diff\phi_{p}(v)\in\tanspace{\phi(p)}{N}$.
            \end{proof}
            \begin{figure}[H]
                \centering
                \captionsetup{type=figure}
                \begin{tikzpicture}[>=Latex]
    \coordinate (p)  at (3.5,  0.4);
    \coordinate (p1) at (4.0, -0.2);
    \draw (0.0, 0.0)    to[out=90,  in=150]  (2.0,  1.0)
                        to[out=-30, in=140]  (4.0,  0.5)
                        to[out=-40, in=90]   (5.0, -0.5)
                        to[out=-90, in=0]    (3.0, -1.0)
                        to[out=180, in=-50]  (1.0, -1.0)
                        to[out=130, in=-90]  cycle;

    \draw (0.5, 0.1) to[in=-130, out=-50] (1.8, 0.1);
    \draw (0.6, 0.0) to[in=130,  out=50]  (1.7, 0.0);
    \draw[fill=white, densely dashed, opacity=0.5]
        (2, -0.5) to (3.0,  1.5) to (5.0,  1.5)
                  to (4.0, -0.5) to cycle;

    \node at (1.4, -0.9) {$M$};
    \node at (4.4,  1.2) {$T_{p}M$};
    \draw[fill=black] (3.5, 0.4) circle (0.5mm);
    \draw[->] (p) to node[above right] {$v$} (p1);
    \node at (p) [below left] {$p$};

    \draw[->] (5, 1) to[out=30, in=150] node[above] {$\phi$} (7, 1);
    \begin{scope}[xshift=7cm]

        \coordinate (fp)  at (3.0, 0.0);
        \coordinate (fp1) at (4.0, 0.2);
        \draw (0.0, 0.0)    to[out=90,  in=150]  (2.0,  1.0)
                            to[out=-30, in=140]  (4.0,  1.0)
                            to[out=-40, in=90]   (5.0, -0.5)
                            to[out=-90, in=0]    (3.0, -1.5)
                            to[out=180, in=-50]  (0.5, -1.0)
                            to[out=130, in=-90]  cycle;

        \draw[fill=white, densely dashed, opacity=0.5]
            (2, -0.5) to (2.5,  1.7) to (5.0,  1.7) to (4.5, -0.5) to cycle;
                    
        \node at (1.4, -0.9) {$N$};
        \node at (4.2,  1.4) {$T_{\phi(p)}N$};
        \draw[fill=black] (fp) circle (0.5mm);
        \draw[->] (fp) to node[above] {$\diff\phi_{p}(v)$} (fp1);
        \node at (fp) [below left] {$\phi(p)$};
    \end{scope}
\end{tikzpicture}
                \caption{The Pushforward of a Tangent Vector}
                \label{fig:Pushforward_of_Tangent_Vector}
            \end{figure}
            \begin{theorem}
                If $\manifold[M]{M}$ and $\manifold[N]{N}$ are smooth manifolds,
                if $\phi:M\rightarrow{N}$ is a smooth function, if $p\in{M}$, if
                $(\mathcal{U},\varphi)\in\mathcal{A}_{M}$ and
                $(\mathcal{V},\psi)\in\mathcal{A}_{N}$ are charts such that
                $p\in\mathcal{U}$ and $\phi(p)\in\mathcal{V}$, then:
                \begin{equation}
                    \diff\phi_{p}(\partial_{\varphi^{k}}|_{p})=
                    \sum_{k\in\mathbb{Z}_{n}}
                    \frac{\partial(\psi^{k}\circ\phi)}{\partial\varphi^{k}}(p)
                    \partial_{\psi^{k}}|_{\phi(p)}
                \end{equation}
            \end{theorem}
            \begin{proof}
                For by the basis theorem, we have:
                \begin{equation}
                    \diff\phi_{p}(\partial_{\varphi^{k}}|_{p})=
                    \sum_{k\in\mathbb{Z}_{n}}
                    \diff\phi_{p}(\partial_{\varphi^{k}}|_{p})(\psi^{k})
                    \partial_{\psi^{k}}|_{\phi(p)}
                    =\sum_{k\in\mathbb{Z}_{n}}
                    \frac{\partial(\psi^{k}\circ\phi)}{\partial\varphi^{k}}(p)
                    \partial_{\psi^{k}}|_{\phi(p)}
                \end{equation}
                Completing the proof.
            \end{proof}
            The Jacobian matrix of $\phi$ about the point $p\in{M}$ with respect
            to the charts $(\mathcal{U},\varphi)\in\mathcal{A}_{M}$ and
            $(\mathcal{V},\psi)\in\mathcal{A}_{n}$ is the matrix:
            \begin{equation}
                I_{ij}=
                \frac{\partial(\psi^{i}\circ\phi)}{\partial\varphi^{k}}(p)
            \end{equation}
            \begin{theorem}
                If $\manifold[M]{M}$, $\manifold[N]{N}$, and $\manifold[P]{P}$
                are smooth manifolds, if $\phi:M\rightarrow{N}$ and
                $\xi:N\rightarrow{P}$ are smooth functions, then:
                \begin{equation}
                    \diff(\xi\circ\phi)_{p}=
                    \diff\xi_{\phi(p)}\circ\diff\phi_{p}
                \end{equation}
            \end{theorem}
            \begin{proof}
                For if $p\in{M}$, $v\in{T}_{p}M$, and $f\in\Ckspace{\infty}{P}$,
                then:
                \begin{align}
                    \diff(\xi\circ\phi)_{p}(v)(f)
                    &=v(f\circ\xi\circ\phi)\\
                    &=v\big((f\circ\xi)\circ\phi\big)\\
                    &=\diff\phi_{p}(v)(f\circ\xi)\\
                    &=\diff\xi_{\phi(p)}\big(\diff\phi_{p}(v)(f)\big)\\
                    &=(\diff\xi_{\phi(p)}\circ\diff\phi_{p})(v)(g)
                \end{align}
                Hence,
                $\diff(\xi\circ\phi)_{p}=\diff\xi_{\phi(p)}\circ\diff\phi_{p}$.
            \end{proof}
            \begin{theorem}
                If $\manifold[M]{M}$ and $\manifold[N]{N}$ are smooth manifolds,
                if $\phi:M\rightarrow{n}$ is a smooth function, and if
                for every $p\in{M}$ the differential pushforward
                $\diff\phi_{p}$ is a isomorphism between $\tanspace{p}{M}$ and
                $\tanspace{\phi(p)}{N}$, then $\phi$ is a local diffeomorphism.
            \end{theorem}
            \begin{theorem}
                If $\manifold[M]{M}$ and $\manifold[N]{N}$ are smooth manifolds,
                and if $\phi:M\rightarrow{N}$ is a local diffeomorphism, then
                for all $p\in{M}$, the differential pushforward $\diff\phi_{p}$
                is an isomorphism between $\tanspace{p}{M}$ and
                $\tanspace{\phi(p)}{N}$.
            \end{theorem}
            \begin{proof}
                For if $\phi$ is a local diffeomorphism, for all $p\in{M}$ there
                is a chart $(\mathcal{U},\varphi)\in\mathcal{A}_{M}$ and a
                chart $(\mathcal{V},\psi)\in\mathcal{A}_{N}$ such that
                $p\in\mathcal{U}$ and $\mathcal{U}$ and $\mathcal{V}$ are
                diffeomorphic. Since $\diff\phi_{p}$ is linear, it suffices to
                show that it is bijective. Suppose $v,u\in\tanspace{p}{M}$ are
                distinct but such that $\diff\phi_{p}(v)=\diff\phi_{p}(u)$.
                Then for all $f\in\Ckspace{\infty}{N}$ we have:
                \begin{equation}
                    \diff\phi_{p}(v-u)(f)=(v-u)(f\circ\phi)
                    =v(f\circ\phi)-u(f\circ\phi)=0
                \end{equation}
                But $\phi$ is a local diffeomorphism, and thus for all
                $f\in\Ckspace{\infty}{N}$ there exists a
                $g\in\Ckspace{\infty}{M}$ such that
                $f|_{\mathcal{V}}=g|_{\mathcal{U}}\circ\phi^{\minus{1}}$. Since
                tangent vectors are local objects, if $u(g)=v(g)$ for all
                $g\in\Ckspace{\infty}{\mathcal{U}}$, then $u=v$, a
                contradiction. Hence, $\diff\phi_{p}$ is injective. By a reverse
                argument, it is surjective.
            \end{proof}
        \subsection{Curves}
            \begin{definition}
                A curve in a smooth manifold is a smooth mapping
                $\alpha:I\rightarrow{M}$, where $I$ is an open interval in
                $\mathbb{R}$ with the subspace smooth structure.
            \end{definition}
            \begin{definition}
                The velocity vector of a curve $\alpha:I\rightarrow{M}$ at a
                point $t\in{I}$ is the tangent vector:
                \begin{equation}
                    \dot{\alpha}(t)=\diff\alpha\Big(
                        \frac{\diff}{\diff{u}}\Big|_{t}
                    \Big)
                \end{equation}
            \end{definition}
            The directional derivative of a function $f\in\Ckspace{\infty}{M}$
            along a curve $\alpha:I\rightarrow{M}$ can be computed from the
            definition of the velocity vector. We have:
            \begin{equation}
                \dot{\alpha}(t)f
                =\frac{\diff}{\diff{u}}(f\circ\alpha)|_{\alpha(t)}
            \end{equation}
            We can apply the basis theorem as well, given a point $t\in{I}$ and
            a chart $(\mathcal{U},\varphi)$ that contains $\alpha(t)$, we have:
            \begin{equation}
                \dot{\alpha}(t)=\sum_{k\in\mathbb{Z}_{n}}
                    \frac{\diff}{\diff{u}}(\varphi^{k}\circ\alpha)
                    \partial_{\varphi^{k}}|_{\alpha(t)}
            \end{equation}
            This is reminiscent of the chain rule from multivariable calculus.
            If $h:J\rightarrow{I}$ is a smooth function from the open interval
            $J$ to the open interval $I$, and if $\beta:J\rightarrow{M}$ is the
            composition of $h$ and $\alpha:I\rightarrow{M}$, then:
            \begin{equation}
                \dot{\beta}(s)=\frac{\diff{h}}{\diff{u}}(s)\dot{\alpha}(h(s))
            \end{equation}
            for all $s\in{J}$. Lastly, the differential pushforward preserves
            velocities. If $\alpha:I\rightarrow{M}$ is a curve, and if
            $\phi:M\rightarrow{N}$ is smooth, then:
            \begin{equation}
                \diff\phi_{\alpha(t)}\big(\dot{\alpha}(t)\big)=
                    (\phi\circ\alpha)'(t)
            \end{equation}
            This is because:
            \begin{align}
                \diff\phi_{\alpha(t)}\big(\dot{\alpha}(t)\big)
                &=\diff\phi_{\alpha(t)}\Big(\diff\alpha_{t}
                    \big(\frac{\diff}{\diff{u}}\big)\big|_{t}\Big)\\
                    &=\big(\diff\phi_{\alpha(t)}\circ\diff\alpha_{t}\big)
                        \Big(\frac{\diff}{\diff{u}}\big|_{t}\Big)\\
                    &=\diff(\phi\circ\alpha)_{t}
                        \Big(\frac{\diff}{\diff{u}}\big|_{t}\Big)\\
                    &=(\phi\circ\alpha)'(t)
            \end{align}
            \begin{definition}
                A regular curve is a curve $\alpha:I\rightarrow{M}$ such that
                for all $t\in{I}$, $\dot{\alpha}(t)\ne{0}$.
            \end{definition}
            \begin{definition}
                A curve segment is a function $\alpha:[a,b]\rightarrow{M}$ such
                that there exists an open interval $I$ such that
                $[a,b]\subseteq{I}$ and there exists a smooth extension
                $\tilde{\alpha}:I\rightarrow{M}$.
            \end{definition}
        \subsection{Vector Fields}
            \begin{definition}
                A vector field on a smooth manifold $\manifold{M}$ is a function
                $V:M\rightarrow\funcspace{\Ckspace{\infty}{M}}$ such that for
                all $p\in{M}$, $V(p)\in{T}_{p}M$.
            \end{definition}
            That is, to every point in $M$, $V$ assigns a tangent vector to that
            point. There's a way to define a function $Vf$ given a vector field
            $V$ and a function $f\in\Ckspace{\infty}{M}$. We let:
            \begin{equation}
                (Vf)(p)=V_{p}(f)
            \end{equation}
            This is now a function in $\funcspace{M}$. To see this, note that
            for all $p\in{M}$, $V_{p}$ is a tangent vector, and thus
            $V_{p}(f)$ is a real number since by definition a tangent vector
            outputs a real number. For $Vf:M\rightarrow\mathbb{R}$ is a
            function. We can thus ask if it is continuous, differentiable, or
            smooth. This gives rise to the definition of a smooth vector field.
            \begin{definition}
                A smooth vector field on a manifold $\manifold{M}$ is a vector
                field $V$ such that for all $f\in\Ckspace{\infty}{M}$, the
                function $Vf:M\rightarrow\mathbb{R}$ defined by:
                \begin{equation}
                    (Vf)(p)=V_{p}(f)
                \end{equation}
                is a smooth function. That is, $Vf\in\Ckspace{\infty}{M}$.
            \end{definition}
            We can multiply vector fields by elements of $\Ckspace{\infty}{M}$,
            as well as add two vector fields together in the following way:
            \begin{align}
                (fV)(p)&=f(p)V_{p}\\
                (V+W)_{p}&=V_{p}+W_{p}
            \end{align}
            \begin{fnotation}{Smooth Vector Fields on a Manifold}
                             {Smooth Vector Fields on a Manifold}
                The set of all smooth vector fields on a manifold $\manifold{M}$
                is denoted $\smoothvecf{M}$.
            \end{fnotation}
            This notation, while standard and can be found in a plethora of
            textbooks on differential geometry and topology, is somewhat strange
            and the connection between it and vector fields is not obvious. An
            attempt will be made to remind the reader of this notation when
            context seems unclear. By the definition of multiplication by
            elements of $\Ckspace{\infty}{M}$ and addition, we have that
            $\smoothvecf{M}$ is a module over $\Ckspace{\infty}{M}$.
            \begin{example}
                Given a smooth manifold $\manifold{M}$ and a chart
                $(\mathcal{U},\varphi)\in\mathcal{A}_{M}$, there coordinate
                vector fields are the elements
                $\partial_{\varphi^{k}}\in\smoothvecf{\mathcal{U}}$ defined by:
                \begin{equation}
                    \partial_{\varphi^{k}}(p)=\partial_{\varphi^{k}}|_{p}
                \end{equation}
                By the basis theorem, for any vector field $V$, we have:
                \begin{equation}
                    V=\sum_{k\in\mathbb{Z}_{n}}
                        V\varphi^{k}\partial_{\varphi^{k}}
                \end{equation}
                Where $V\varphi^{k}$ is the function from $\mathcal{U}$ into
                $\mathbb{R}$ defined by:
                \begin{equation}
                    (V\varphi^{k})(p)=V_{p}(\varphi^{k})
                \end{equation}
                Where $\varphi^{k}$ is the $k^{th}$ coordinate function:
                \begin{equation}
                    \varphi^{k}=\pi^{k}\circ\varphi
                \end{equation}
                $\pi^{k}$ be the projection mapping from $\mathbb{R}^{n}$ to
                $\mathbb{R}$.
            \end{example}
            We now look at some differentiable algebra to show that vector
            fields and derivations on an algebra $\Ckspace{\infty}{M}$ over the
            field $\mathbb{R}$ are actually one in the same.
            \begin{definition}
                A derivation on an algebra $\mathscr{A}$ over a ring $R$ is a
                function $D:\mathscr{A}\rightarrow\mathscr{A}$ such
                that for all $a,b\in{R}$ and for all
                $\vector{x},\vector{y}\in\mathscr{A}$, the following is true:
                \begin{align}
                    D(a\vector{x}+b\vector{y})&=aD(\vector{x})+bD(\vector{y})\\
                    D(\vector{x}\times\vector{y})
                        &=D(\vector{x})\times\vector{y}
                        +\vector{x}\times{D}(\vector{y})
                \end{align}
            \end{definition}
            \begin{theorem}
                If $\manifold{M}$ is a smooth manifold, if $D$ is a derivation
                on the algebra $\Ckspace{\infty}{M}$ over $\mathbb{R}$,
                then there is a vector field $V\in\smoothvecf{M}$ such that,
                for all $f\in\Ckspace{\infty}{M}$, $D(f)=Vf$.
            \end{theorem}
            \begin{proof}
                For let $V:M\rightarrow\funcspace{M}$ be the function such that,
                for all $f\in\Ckspace{\infty}{M}$, we have
                \begin{equation}
                    V(p)(f)=D(f)(p)
                \end{equation}
                Then for all $p$, $V_{p}\in\tanspace{p}{M}$. For since $D$ is a
                derivation, it is linear and Liebnizian, and hence $V_{p}$ is a
                tangent vector to $p$.
            \end{proof}
            The converse is also true: Every vector field gives rise to a
            derivation.
            \begin{theorem}
                If $R$ is a ring and $\mathscr{A}$ is an algebra over $R$,
                and if $D_{1},D_{2}$ are derivativions on $\mathscr{A}$, then
                $D_{1}\circ{D}_{2}-D_{2}\circ{D}_{1}$ is a derivation on
                $\mathscr{A}$.
            \end{theorem}
            \begin{proof}
                For if $a,b\in{R}$ and $\vector{x},\vector{y}\in\mathscr{A}$,
                then:
                \begin{align}
                    \big(D_{1}\circ{D}_{2}-D_{2}\circ{D}_{1})
                        (a\vector{x}+b\vector{y})
                    &=D_{1}\big(D_{2}(a\vector{x}+b\vector{y})\big)
                        -D_{2}\big(D_{1}(a\vector{x}+b\vector{y})\big)\\
                    \nonumber
                    &=aD_{1}\big(D_{2}(\vector{x})\big)
                     +bD_{1}\big(D_{2}(\vector{y})\big)\\
                    &\quad\quad
                        -aD_{2}\big(D_{1}(\vector{x})\big)
                        -bD_{2}\big(D_{1}(\vector{y})\big)\\
                    \nonumber
                    &=a\Big(D_{1}\big(D_{2}(\vector{x})\big)
                     -D_{2}\big(D_{1}(\vector{x})\big)\Big)\\
                    &\quad\quad
                        +b\Big(D_{1}\big(D_{2}(\vector{y})\big)
                        -D_{2}\big(D_{1}(\vector{y})\big)\Big)\\
                    \nonumber
                    &=a\big(D_{1}\circ{D}_{2}-D_{2}\circ{D}_{1}\big)\vector(x)\\
                    &\quad\quad
                        +b\big(D_{1}\circ{D}_{2}-D_{2}\circ{D}_{1}\big)
                        \vector(y)
                \end{align}
                And therefore $D_{1}\circ{D}_{2}-D_{2}\circ{D}_{1}$ is linear.
                It is also Liebnizian, since we have:
                \begin{align}
                    \big(D_{1}\circ{D}_{2}-D_{2}\circ{D}_{1}\big)
                        (\vector{x}\times\vector{y})
                    &=D_{1}\big(D_{2}(\vector{x}\times\vector{y})\big)
                        -D_{2}\big(D_{1}(\vector{x}\times\vector{y})\big)\\
                    \nonumber
                    &=D_{1}\big(D_{2}(\vector{x})\times\vector{y}
                     +\vector{x}\times{D}_{2}(\vector{y})\big)\\
                    &\quad\quad
                        -D_{2}\big(D_{1}(\vector{x})\times\vector{y}
                        +\vector{x}\times{D}_{1}(\vector{y})\big)
                \end{align}
                Applying linearity and to this next sum, we can simplify:
                \begin{align}
                    \nonumber
                    D_{1}\big(&
                        D_{2}(\vector{x})\times\vector{y}+
                        \vector{x}\times{D}_{2}(\vector{y})
                    \big)
                    -D_{1}\big(
                        D_{2}(\vector{x})\times\vector{y}+
                        \vector{x}\times{D}_{2}(\vector{y})
                    \big)\\
                    \nonumber
                    &=D_{1}\big(D_{2}(\vector{x})\times\vector{y}\big)+
                        D_{1}\big(\vector{x}\times{D}_{2}(\vector{y})\big)\\
                    &\quad\quad
                \end{align}
                Fill this in later, taking too long.
            \end{proof}
            \begin{definition}
                The Lie bracket on a smooth manifold $\manifold{M}$ is the
                function
                $\bracket{\cdot}{\cdot}:\smoothvecf{M}\times\smoothvecf{M}%
                 \rightarrow\smoothvecf{M}$ defined by
                \begin{equation}
                    \bracket{V}{W}=VW-WV
                \end{equation}
                That is, $\bracket{V}{W}$ is the function
                $\bracket{V}{W}:M\rightarrow\funcspace{\Ckspace{\infty}{M}}$
                such that, for all $p\in{M}$ and for all
                $f\in\Ckspace{\infty}{M}$ we have
                \begin{equation}
                    \bracket{V}{W}_{p}=V_{p}(Wf)-W_{p}(Vf)
                \end{equation}
            \end{definition}
            By the previous theorem, the Lie bracket of two vector fields is
            again a vector fields since the bracket of two derivations is again
            a derivations, and the only derivations on a smooth manifold are
            precicesly the vector fields.
            \begin{theorem}
                The following are true of any Lie bracket:
                \begin{align}
                    \bracket{aV+bW}{X}&=a\bracket{V}{W}+b\bracket{W}{X}\\
                    \bracket{V}{W}&=\minus\bracket{W}{V}\\
                    \bracket{X}{\bracket{Y}{Z}}+
                    \bracket{Y}{\bracket{X}{Z}}+
                    \bracket{Z}{\bracket{X}{Y}}&=\vector{0}
                \end{align}
            \end{theorem}
            \begin{example}
                Consider the usual smooth manifold structure on $\mathbb{R}^{2}$
                with the vector fields $V=y\partial_{y}$ and
                $W=x\partial_{y}$. Computing the Lie Bracket, we have:
                \begin{align}
                    \bracket{V}{W}(f)
                    &=y\partial_{y}\big(x\partial_{y}(f)\big)
                        -x\partial_{y}\big(\partial_{y}(f)\big)\\
                    &=xy\partial_{y}^{2}(f)-x\partial_{y}(f)
                        -xy\partial_{y}^{2}(f)\\
                    &=\minus{x}\partial_{y}(f)\\
                    &=\minus{W}f
                \end{align}
            \end{example}
            The Lie bracket, while being linear in $\mathbb{R}$, is not linear
            over $\Ckspace{\infty}{M}$.
            \begin{theorem}
                If $\manifold{M}$ is a smooth manifold, if
                $\bracket{\cdot}{\cdot}$ is the Lie bracket on $M$, if
                $f,g\in\Ckspace{\infty}{M}$, and if $V,W\in\smoothvecf{M}$,
                then:
                \begin{equation}
                    \bracket{fV}{gW}=fg\bracket{V}{W}+f(Vg)W-g(Wf)V
                \end{equation}
            \end{theorem}
            \begin{definition}
                A vector field $V$ on a smooth manifold $\manifold[M]{M}$ that
                is related to a vector field $W$ on a smooth manifold
                $\manifold[N]{N}$ is a vector field $V\in\smoothvecf{M}$ such
                that there exists a smooth function $\phi:M\rightarrow{N}$ such
                that:
                \begin{equation}
                    \diff\phi(V_{p})=W_{\phi(p)}
                \end{equation}
            \end{definition}
            \begin{theorem}
                If $\manifold[M]{M}$ and and $\manifold[N]{N}$ are smooth
                manifolds, if $V\in\smoothvecf{M}$ and $W\in\smoothvecf{N}$,
                then $V$ is related to $W$ if and only if there exists a
                smooth function $\phi:M\rightarrow{N}$ such that for all
                $f\in\Ckspace{\infty}{N}$ the following is true:
                \begin{equation}
                    V(f\circ\phi)=Wf\circ\phi
                \end{equation}
            \end{theorem}
            \begin{proof}
                For:
                \begin{align}
                    &&\diff\phi_{p}(V_{p})(f)&=W_{\phi(p)}(f)\\
                    &\Longleftrightarrow&
                    V_{p}(f\circ\phi)&=W_{\phi(p)}(f)\\
                    &\Longleftrightarrow&
                    V(f\circ\phi)(p)&=W(f)\big(\phi(p)\big)\\
                    &\Longleftrightarrow&
                    V(f\circ\phi)&=Wf
                \end{align}
            \end{proof}
            \begin{theorem}
                If $\manifold[M]{M}$ and $\manifold[N]{N}$ are smooth manifolds,
                if the function $\phi:M\rightarrow{N}$ is smooth, if
                $V_{1},V_{2}\in\smoothvecf{M}$, if
                $W_{1},W_{2}\in\smoothvecf{N}$, if $\phi$ relates $V_{1}$ to
                $W_{1}$, and if $\phi$ relates $V_{2}$ to $W_{2}$, then
                $\phi$ relates $\bracket{V_{1}}{V_{2}}$ to
                $\bracket{W_{1}}{W_{2}}$.
            \end{theorem}
            \begin{theorem}
                If $\manifold[M]{M}$ and $\manifold[N]{N}$ are smooth manifolds,
                if the function $\phi:M\rightarrow{N}$ is a diffeomorphism, and
                if $V\in\smoothvecf{M}$, then there is a unique vector field
                $W\in\smoothvecf{N}$ such that $\phi$ relates $V$ to $W$.
            \end{theorem}
        \subsection{One Forms}
            Whenever we have the notion of linear functionals on vector spaces,
            we ultimately end up talking about the dual space. We have seen that
            tangent vectors can be consider as linear functions on
            $\Ckspace{\infty}{M}$ with the additional property that they are
            Liebnizian. A vector field is a function on a manifold $M$ that
            assigns to every point $p\in{M}$ a tangent vector $V_{p}$ at $p$.
            The dual notion of this is called a one form.
            \begin{definition}
                The cotangent space on a smooth manifold $\manifold[M]{M}$ at a
                point $p\in{M}$ is the dual space of $\tanspace{p}{M}$. That
                is, $\cotanspace{p}{M}$ is the vector space of linear
                functionals on $\tanspace{p}{M}$.
            \end{definition}
            \begin{definition}
                A one form on a smooth manifold $\manifold[M]{M}$ is a function
                $\Theta:M\rightarrow\funcspace{\Ckspace{\infty}{M}^{*}}$ such
                that for all $p\in{M}$, $\Theta_{p}\in\cotanspace{p}{M}$.
            \end{definition}
            Since dual spaces can be confusing, it is perhaps worthwhile to
            spell out the inputs and outputs of $\Theta$. Give a point
            $p\in{M}$, $\Theta_{p}$ is a cotangent vector living in the space
            $\cotanspace{p}{M}$. That is, $\Theta_{p}$ takes in tangent vectors
            in $\tanspace{p}{M}$, and outputs a real numbers. Moreover, it does
            this in a linear fashion. If $p\in{M}$, and if
            $v,u\in\tanspace{p}{M}$, then:
            \begin{equation}
                \Theta_{p}(v+u)=\Theta_{p}(v)+\Theta_{p}(u)
            \end{equation}
            Much the way $\smoothvecf{M}$ was given a module structure, we can
            do this for one forms. Given a vector field $V\in\smoothvecf{M}$ and
            a one form $\Theta$, we define $\Theta{V}:M\rightarrow\mathbb{R}$
            to be the function:
            \begin{equation}
                (\Theta{v})(p)=\Theta_{p}(V_{p})
            \end{equation}
            That is, $\Theta_{p}$ is the cotangent vector that $\Theta$ assigns
            to $p$, and $V_{p}$ is the tangent vector that $V$ assigns to $p$.
            Since cotangent vectors take in as inputs tangent vectors and output
            real number, $\Theta{V}$ is a well defined function from $M$ to
            $\mathbb{R}$. Thus we can ask if it is continuous, differentiable,
            or smooth.
            \begin{definition}
                A smooth one form on a manifold $\manifold[M]{M}$ is a one form
                $\Theta$ on $M$ such that for every smooth vector field
                $V\in\smoothvecf{M}$, $\Theta{V}:M\rightarrow\mathbb{R}$ is a
                smooth function.
            \end{definition}
            \begin{fnotation}{Smooth One Forms on a Manifold}
                             {Smooth One Forms on a Manifold}
                The set of all smooth one forms on a smooth manifold
                $\manifold[M]{M}$ is denoted $\smoothonef{M}$.
            \end{fnotation}
            Like $\smoothvecf{M}$, $\smoothonef{M}$ is also a module over
            $\Ckspace{\infty}{M}$. To see this we need to define multiplication
            and addition, and we do this in the same way as with vector fields:
            \begin{align}
                (f\Theta)(p)&=f(p)\Theta_{p}\\
                (\Theta+\Xi)(p)&=\Theta_{p}+\Xi_{p}
            \end{align}
            With this, $\smoothonef{M}$ is a module over $\Ckspace{\infty}{M}$.
            Given any function $f\in\Ckspace{\infty}{M}$ on a smooth manifold
            $\manifold[M]{M}$, we can always associate to this a smooth one
            form. This is called the \textit{differential} of $f$.
            \begin{definition}
                The differential of a smooth function $f\in\Ckspace{\infty}{M}$
                on a smooth manifold $\manifold[M]{M}$ is the one form
                $\diff{f}\in\smoothonef{M}$ defined by:
                \begin{equation}
                    \diff{f}(v)=v(f)
                \end{equation}
            \end{definition}
            \begin{theorem}
                If $\manifold{M}$ is a smooth manifold, if $\Theta$ is a one
                form on $M$, and if $(\mathcal{U},\varphi)\in\mathcal{A}$ is a
                chart, then for all $p\in\mathcal{U}$ the following is true:
                \begin{equation}
                    \Theta_{p}=\sum_{k\in\mathbb{Z}_{n}}
                        \Theta_{p}(\partial_{\varphi^{l}}|_{p})\diff\varphi^{k}
                \end{equation}
            \end{theorem}
            \begin{theorem}
                If $\manifold{M}$ is a smooth manifold, if
                $(\mathcal{U},\varphi)\in\mathcal{A}$ is a chart, and if
                $f\in\Ckspace{\infty}{M}$, then for all $p\in\mathcal{U}$:
                \begin{equation}
                    \diff{f}=\sum_{k\in\mathbb{Z}_{n}}
                        \frac{\partial{f}}{\partial\varphi^{k}}
                            \diff\varphi^{k}
                \end{equation}
            \end{theorem}
            \begin{theorem}
                If $\manifold{M}$ is a smooth manifold, if $a,b\in\mathbb{R}$,
                and if $f,g\in\Ckspace{\infty}{M}$, then:
                \begin{equation}
                    \diff(af+bg)=a\diff{f}+b\diff{g}
                \end{equation}
            \end{theorem}
            \begin{theorem}
                If $\manifold{M}$ is a smooth manifold, if
                $f,g\in\Ckspace{\infty}{M}$, then:
                \begin{equation}
                    \diff(fg)=g\diff(f)+f\diff{g}
                \end{equation}
            \end{theorem}
            \begin{theorem}
                If $\manifold{M}$ is a smooth manifold, if
                $f\in\Ckspace{\infty}{M}$, and if
                $\alpha\in\Ckspace{\infty}{\mathbb{R}}$, then:
                \begin{equation}
                    \diff(\alpha\circ{f})=\dot{\alpha}(f)\diff{f}
                \end{equation}
            \end{theorem}
        \subsection{Submanifolds}
            \begin{definition}
                A submanifold of a manifold $\manifold{M}$ is a manifold
                $\manifold[P]{P}$ such that $\topspace[P]{P}$ is a topological
                subspace of $\topspace{M}$, and such that the inclusion map
                $\iota:P\rightarrow{M}$ is smooth and the differential
                $\diff\iota$ is injective.
            \end{definition}
            \begin{theorem}
                If $\manifold[M]{M}$ and $\manifold[N]{N}$ are smooth manifolds,
                if $\phi:M\rightarrow{N}$ is a smooth function, and if
                $\manifold[P]{P}$ is a submanifold of $M$, then
                $\phi|_{P}:P\rightarrow{N}$ is a smooth function.
            \end{theorem}
            \begin{proof}
                For $\phi|_{P}=\phi\circ\iota$, where $\iota$ is the inclusion.
                Since $\manifold[P]{P}$ is a submanifold, $\iota$ is a smooth,
                and hence $\phi|_{P}$ is the composition of smooth functions,
                which is smooth.
            \end{proof}
            The fact that $\diff\iota$ is injective means that for every point
            $p\in{P}$, $\tanspace{p}{P}$ is a vector subspace of
            $\tanspace{p}{M}$.
            \begin{example}
                Any open subset of a smooth manifold $\manifold{M}$ is
                automatically a submanifold. Non-trivial examples include the
                sphere $S^{n}$ which is a submanifold of $\mathbb{R}^{n+1}$.
                Hyperplanes in $\mathbb{R}^{n}$ obtained by holding any $k<n$
                components constant are also submanifolds.
            \end{example}
            \begin{definition}
                A chart that is adapted to a subset $P$ of a manifold
                $\manifold{M}$ is a chart $(\mathcal{U},\varphi)\in\mathcal{A}$
                such that there exists an $m<n$ such that:
                \begin{equation}
                    P\cap\mathcal{U}
                        =\{p\in\mathcal{U}\;|\;\varphi^{j}(p)=0,j>n-m\}
                \end{equation}
            \end{definition}
            That is, $P$ looks roughly like a hyperplane. Since $\varphi$ might
            be a curvilinear mapping, this \textit{hyperplane} may not be a
            plane at all, but rather some $m$ dimensional submanifold of
            $\mathbb{R}^{n}$. As it turns out, all submanifolds of a manifold
            $\manifold{M}$ look like this, at least locally.
            \begin{theorem}
                If $\manifold{M}$ is a smooth manifold of dimension
                $n\in\mathbb{N}$, and if $\manifold[P]{P}$ is a submanifold of
                dimension $m\in\mathbb{N}$, then there is a chart
                $(\mathcal{U},\varphi)$ that is adapted to $P$ at each point of
                $p$.
            \end{theorem}
            \begin{proof}
                For let $(\mathcal{U},\varphi)\in\mathcal{A}$ be such that
                $p\in\mathcal{U}$ and let $(\mathcal{V},\psi)\in\mathcal{A}_{P}$
                be also such that $p\in\mathcal{U}_{P}$. But since the
                differential of the inclusion map is injective, the Jacobian
                matrix has rank $m$ and thus we may rearrange it so that the
                first $m$ rows are linearly independent. But then
                $\varphi^{k}|_{P}$ for $k\in\mathbb{Z}_{m}$ forms a chart for
                $P$ on a neighborhood $\mathcal{W}$ of $p$. I don't know, some
                exercise that is reference in O'Neill, fuck this. Come back
                later.
            \end{proof}
            \begin{theorem}
                If $P$ is a submanifold of $M$, if $\phi:N\rightarrow{M}$ is
                smooth, and if $\phi[N]\subseteq{M}$, then the induced map
                $\tilde{\phi}:N\rightarrow{P}$ is smooth.
            \end{theorem}
            \begin{theorem}
                If $\manifold{M}$ is a smooth manifold, and $P\subseteq{M}$,
                then either there is no smooth structure on $P$ that makes $P$
                a submanifold of $M$, or this is a unique smooth structure.
            \end{theorem}
            \begin{proof}
                For if there are two different smooth structures, by the
                previous theorem the identity mapping
                $\textrm{id}_{P}:P\rightarrow{P}$ is a diffeomorphism from the
                first structure to the second, a contradiction.
            \end{proof}
            By this theorem it then makes sense to ask whether or not a subset
            $P$ of a smooth manifold $\manifold{M}$ is a submanifold or not,
            since there is either a unique smooth atlas on $P$ that makes it a
            submanifold, or there is no such structure.
            \begin{theorem}
                A subset $P$ of a manifold $\manifold{M}$ is a submanifold if
                and only if for all $p\in{P}$ there exists a chart
                $(\mathcal{U},\varphi)\in\mathcal{A}$ such that
                $p\in\mathcal{U}$ and $(\mathcal{U},\varphi)$ is adapted to $P$.
            \end{theorem}
            \begin{proof}
                For if $(\mathcal{U},\varphi)$ is such a chart, then the
                restriction of $\varphi$ to $\mathcal{U}\cap{P}$ is a
                homemorphism from an open subset of $P$ (with the subspace
                topology) to a subset of $\mathbb{R}^{n}$. But by the definition
                of this chart, $n-m$ of the coordinates will be constant, and
                hence $\varphi|_{P}$ is a homemorphism from an open subset of
                $P$ to an open subset of $\mathbb{R}^{m}$. The collection of all
                such charts form an atlas for $P$. For the cover $P$, and if two
                charts $(\mathcal{U},\varphi)$ and $(\mathcal{V},\psi)$ overlap
                then they overlap smoothly because of reasons. The inclusion map
                is also smooth because reasons, and $\diff\iota$ is injective or
                whatever.
            \end{proof}
            \begin{definition}
                A tangent vector vield to a submanifold $\manifold[P]{P}$ of a
                smooth manifold $\manifold[M]{M}$ is a vector field
                $V\in\smoothonef{M}$ such that for all $p\in{P}$ it is true that
                $V_{p}\in\tanspace{p}{P}$
            \end{definition}
            \begin{theorem}
                If $\manifold[M]{M}$ is a smooth manifold, if $\manifold[P]{P}$
                is a submanifold of $M$, and if $V\in\smoothvecf{M}$ is a
                tangent vector to $P$, then $V|_{P}\in\smoothvecf{P}$.
            \end{theorem}
            \begin{theorem}
                If $\manifold[M]{M}$ is a smooth manifold, if $\manifold[P]{P}$
                is a submanifold of $M$, if $V,W\in\smoothvecf{M}$ are tangent
                vector fields to $P$, if
                $\prescript{M}{}{\bracket{\cdot}{\cdot}}$ is the Lie bracket on
                $M$, and if $\prescript{P}{}{\bracket{\cdot}{\cdot}}$ is the Lie
                bracket on $P$, then:
                \begin{equation}
                    \prescript{M}{}{\bracket{V}{W}|_{P}}
                    =\prescript{P}{}{\bracket{V|_{P}}{W|_{P}}}
                \end{equation}
            \end{theorem}
        \subsection{Immersions and Submersions}
            \begin{theorem}
                If $\manifold[M]{M}$ and $\manifold[N]{N}$ are smooth manifolds,
                if the function $\phi:M\rightarrow{N}$ is smooth, then for all
                $p\in{M}$ the function
                $\diff\phi_{p}:\tanspace{p}{M}\rightarrow\tanspace{\phi(p)}{N}$
                is injective if and only if the Jacobian matrix has rank $m$ for
                every chart $(\mathcal{U},\varphi)$ and $(\mathcal{V},\psi)$
                containing $p$ and $\phi(p)$, respectively.
            \end{theorem}
            \begin{definition}
                An immersion from a smooth manifold $\manifold[M]{M}$ to another
                smooth manifold $\manifold[N]{N}$ is a smooth function
                $\phi:M\rightarrow{N}$ such that for all $p\in{M}$ the
                differential pushforward $\diff\phi_{p}$ is injective.
            \end{definition}
            \begin{theorem}
                If $\alpha:I\rightarrow{M}$ is a regular curve
                ($\dot{\alpha}(t)\ne{0}$), then $\alpha$ is an immersion.
            \end{theorem}
            \begin{definition}
                An embedding is an injective immersion $\phi:M\rightarrow{N}$
                such that $\phi$ is a homemorphism from $M$ onto its image.
            \end{definition}
            \begin{theorem}
                If $\manifold{M}$ is a smooth manifold and if $\manifold[P]{P}$
                is a submanifold of $M$, then the inclusion map
                $\iota:P\rightarrow{M}$ is an embedding.
            \end{theorem}
            \begin{theorem}
                If $\manifold[M]{M}$ and $\manifold[N]{N}$ are smooth manifolds,
                if $\phi:M\rightarrow{N}$ is an embedding, then $\phi[M]$ is a
                submanifold of $N$.
            \end{theorem}
            \begin{definition}
                An immersed submanifold of a smooth manifold $\manifold{M}$ is
                a subset $P$ such that the inclusion map $\iota:P\rightarrow{M}$
                is an immersion.
            \end{definition}
            By the previous theorem, every submanifold is an immersed
            submanifold but the converse is not true. The standard picture of
            the Klein bottle represents an immersed submanifold that is not a
            submanifold since it's image in $\mathbb{R}^{3}$ is not homeomorphic
            to the original Klein bottle.
            \begin{theorem}
                If $\manifold[M]{M}$, $\manifold[N]{N}$ are smooth manifolds,
                $\phi:M\rightarrow{N}$ a smooth function, and if for all
                $p\in{M}$ it is true that the differential pushforward
                $\diff\phi_{p}$ is surjective, then for every chart
                $(\mathcal{U},\varphi)\in\mathcal{A}_{M}$ and
                $(\mathcal{V},\psi)\in\mathcal{A}_{N}$, the Jacobian matrix has
                rank $n$.
            \end{theorem}
            \begin{definition}
                A regular value of a smooth function $\phi:M\rightarrow{N}$ from
                a smooth manifold $\manifold[M]{M}$ to a manifold
                $\manifold[N]{N}$ is a point $q\in{N}$ such that for all
                $p\in\phi^{\minus{1}}[\{q\}]$ the differential pushforward
                $\diff\phi_{p}$ is surjective.
            \end{definition}
            \begin{theorem}
                If $\manifold[M]{M}$, $\manifold[N]{N}$ are smooth manifolds,
                $\phi:M\rightarrow{N}$ a smooth function, and if $q\in{N}$ is a
                regular point of $\phi$, then the dimension of $M$ is equal to
                the dimension of $N$ plus the dimension of
                $\phi^{\minus{1}}[\{q\}]$.
            \end{theorem}
            \begin{definition}
                The codimension of a submanifold $P$ of a smooth manifold
                $\manifold{M}$ is the dimension of $M$ minus the dimension of
                $P$.
            \end{definition}
            \begin{definition}
                A hypersurface in a smooth manifold $\manifold{M}$ is a
                submanifold of codimension 1.
            \end{definition}
            \begin{theorem}
                If $\manifold{M}$ is a smooth manifold, if
                $f\in\Ckspace{\infty}{M}$, and if $c\in\mathbb{R}$ is a regular
                point of $f$, then $f^{\minus{1}}[\{c\}]$ is a hypersurface in
                $M$.
            \end{theorem}
            \begin{example}
                This is one way of showing that the sphere $S^{n}$ is a manifold
                since it can be realized as a hypersurface in
                $\mathbb{R}^{n+1}$. Let
                $f(\vector{x})=\norm{\vector{x}}_{2}^{2}$ for all
                $\vector{x}\in\mathbb{R}^{n+1}$. Then $1\in\mathbb{R}$ is a
                regular point, and hence $f^{\minus{1}}[\{1\}]$ is a
                hypersurface (and hence a submanifold, and therefore a manifold)
                but this is simply $S^{n}$.
            \end{example}
            \begin{definition}
                A submersion from a smooth manifold $\manifold[M]{M}$ to another
                smooth manifold $\manifold[N]{N}$ is a function
                $\phi:M\rightarrow{N}$ such that for all $p\in{M}$ the
                differential pushforward $\diff\phi_{p}$ is surjective.
            \end{definition}
            Since $\phi^{\minus{1}}[\{q\}]$ is a submanifold for every $q\in{N}$
            this then forms a partition of $M$. Thus, submersions can be seen as
            ways of partitioning a manifold $M$ by the level-sets or fibers of
            another manifold.
        \subsection{Partitions of Unity}
            If $f_{\alpha}\in\Ckspace{\infty}{M}$ is a collection of functions
            such that $\textrm{supp}\{f_{\alpha}\}$ forms a locally finite
            collection, then $\sum_{\alpha}f_{\alpha}$ is well defined since
            at every point there are only finitely many functions $f_{\alpha}$
            that contribute to the sum, and hence this sum converges everywhere.
            \begin{definition}
                A smooth partition of unity is of a smooth manifold
                $\manifold{M}$ is a subset
                $\mathcal{F}\subseteq\Ckspace{\infty}{M}$ such that for all
                $f\in\mathcal{F}$, $f[M]\subseteq[0,1]$, and such that the set:
                \begin{equation}
                    \mathcal{O}=\{\,\textrm{supp}{f}\;|\;f\in\mathcal{F}\,\}
                \end{equation}
                is locally finite, and $\sum_{\alpha}f_{\alpha}=1$.
            \end{definition}
            \begin{definition}
                A subordinate partition of unity on a manifold $\manifold{M}$
                with respect to an open cover $\mathcal{O}$ of $M$ is a
                partition of unity $\mathcal{F}$ such that the supports of
                $\mathcal{F}$ form a refinement of $\mathcal{O}$.
            \end{definition}
            \begin{ftheorem}{Existence of Partitions of Unity on Manifolds}
                            {Existence of Partitions of Unity on Manifolds}
                If $\manifold{M}$ is a smooth manifold, and if $\mathcal{O}$ is
                an open cover of $M$, then there exists a partition of unity
                $\mathcal{F}$ that is subordinate to $\mathcal{O}$.
            \end{ftheorem}
            The second countability of smooth manifolds is required for this
            theorem to holds since the proof relies on paracompactness. Hence,
            large manifolds like the long line may fail to have such partitions.
            This theorem is actually a significantly weaker version of a purely
            topological result:
            \begin{ftheorem}{Partitions of Unity Theorem}
                            {Partitions of Unity Theorem}
                If $\topspace{X}$ is an accessible topological space ($T_{1}$),
                then it is paracompact and Hausdorff if and only if every open
                cover has a subordinate partition of unity.
            \end{ftheorem}
            Since, by definition, manifolds are Hausdorff, and since the second
            countability criterion can prove paracompactness, the existence of
            subordinate partitions of unity falls out immediately. Furthermore,
            if somehow we have a connected locally Euclidean space that always
            has partitions of unity, then we know it is a manifold. That is,
            since locally Euclidean spaces are always $T_{1}$, this theorem then
            proves the space is Hausdorff and paracompact. Connectedness then
            gives us that the space is second countable, and hence a manifold.
            We can trivially weaken this to having countably many connected
            components.
        \subsection{Orientability}
            \begin{definition}
                A smooth oriented manifold is a manifold $\manifold{M}$ such
                there exists an atlas $\mathcal{O}\subseteq\mathcal{A}$ such
                that for all
                $(\mathcal{U},\varphi),(\mathcal{V},\psi)\in\mathcal{O}$, the
                determinant of the Jacobian matrix is positive.
            \end{definition}
            \begin{example}
                The torii, spheres, and Euclidean spaces $\mathbb{T}^{n}$,
                $\mathbb{S}^{n}$, and $\mathbb{R}^{n}$, respectively, are all
                orientable. The M\"{o}bius strip and Klein bottle are not.
            \end{example}
            \begin{theorem}
                If $A$ is a set, $\mathcal{A}$ is a topological atlas on $A$
                such that all charts in $\mathcal{A}$ overlap smoothly, and if
                for all $x,y\in{A}$ it is true that either there exists a chart
                $(\mathcal{U},\varphi)\in\mathcal{A}$ such that
                $x,y\in\mathcal{U}$ or there exist charts
                $(\mathcal{U},\varphi),(\mathcal{V},\psi)\in\mathcal{A})$ such
                that $x\in\mathcal{U}$, $y\in\mathcal{V}$, and
                $\mathcal{U}\cap\mathcal{V}$ are disjoint, then there is a
                unique topology $\tau$ on $A$ such that $\topspace{A}$ is
                a Hausdorff topological space such that $\mathcal{A}$ is a
                smooth atlas on $A$.
            \end{theorem}
            We can see what each condition in the statement of this theorem is
            doing. the first criterion, then $\mathcal{A}$ be an atlas, simply
            requires that the domains of the charts in $\mathcal{A}$ cover the
            entire set $A$. The smoothly overlapping part is so that we can
            ensure that $\mathcal{A}$ can be a smooth atlas, and the last
            criterion gives us the Hausdorff condition. If $x,y$ lie in the same
            chart, since Euclidean space is Hausdorff we can thus separate $x$
            and $y$. If they lie in distinct disjoint charts, then again they
            can be separated. Note the the disjoint condition is crucial. The
            bug-eyed line forms a countable example. We can cover the bug-eyed
            line with charts, but the two origins cannot be separated by
            disjoint charts, and hence is a non-Hausdorff manifold.
        \subsection{Special Manifolds}
            The projection mappings $\pi:M\times{N}\rightarrow{M}$ and
            $\sigma:M\times{N}\rightarrow{N}$ are smooth mappings, and are
            submersions. A map $\phi:{P}\rightarrow{M}\times{N}$ is smooth if
            and only if $\pi\circ\phi$ and $\sigma\circ\phi$ are smooth. The
            subspace $M\times{q}$ and $p\times{N}$ are submanifolds of
            $M\times{N}$.
            \begin{theorem}
                If $\manifold[M]{M}$ and $\manifold[N]{N}$ are smooth manifolds,
                if $\manifold[M\times{N}]{M\times{N}}$ is the product manifold,
                if $(p,q)\in{M}\times{N}$, then:
                \begin{equation}
                    \tanspace{(p,q)}{(M\times{N})}
                    =\tanspace{(p,q)}{(M\times\{q\})}\oplus
                    \tanspace{(p,q)}{\{p\}\times{N}}
                \end{equation}
                Where $\oplus$ is the vector space direct sum.
            \end{theorem}
            \begin{definition}
                The lift of a smooth function $f\in\Ckspace{\infty}{M}$ on a
                smooth manifold $\manifold[M]{M}$ with respect to the product
                manifold of $M$ with $\manifold[N]{N}$ is the function
                $\tilde{f}\in\Ckspace{\infty}{M\times{N}}$ defined by
                $\tilde{f}=f\circ\pi_{M}$.
            \end{definition}
            By the previous theorem, for any point $p\in{M}$ and for any tangent
            vector $v\in\tanspace{p}{M}$ there is a unique tangent vector
            $\tilde{v}\in\tanspace{(p,q)}{(M\times{N})}$ such that
            $\diff\pi_{M}(\tilde{v})=v$ and we can similarly define this as the
            lift of $v$. Extending this further, given a smooth vector field
            $V\in\smoothvecf{M}$ we can lift this to $M\times{N}$ by defining
            $\Tilde{V}_{(p,q)}$ to be the unique lift of $V_{p}$. Lifts of
            vector fields are not invarant under multiplication by elements of
            $\Ckspace{\infty}{(M\times{N})}$. For example, in $\mathbb{R}^{2}$,
            we have that $\diff/\diff{x}$ can be lifted to $\partial_{x}$, but
            $y\partial_{x}$ is not a lift. A horizontal lift is one such that
            $\pi_{N}(V)$ is simply the zero vector field.
            \begin{theorem}
                If $V,W\in\smoothvecf{M}$, then
                $\bracket{\tilde{V}}{\tilde{W}}=\tilde{\bracket{V}{W}}$.
            \end{theorem}
        \subsection{Vector Spaces as Manifolds}
            If $V$ is an $n$ dimensional vector space over $\mathbb{R}$ and if
            $\varphi,\psi$ are linear isomorphisms from $V$ to $\mathbb{R}^{n}$,
            then $\varphi\circ\psi^{\minus{1}}$ is a linear isomorphism from
            $\mathbb{R}^{n}$ to itself, which is therefore a diffeomorphism.
            By a previous theorem there is a unique Hausdorff topology on
            $V$ and a unique maximal atlas that makes $V$ into an $n$
            dimenisonal smooth manifold such that all of the linear isomorphisms
            form charts.
            \begin{theorem}
                If $V$ is an $n$ dimensional vector space, if $p,v\in{V}$, if
                $\alpha:I\rightarrow{V}$ is the curve $\alpha(t)=p+tv$, is
                $v_{p}=\dot{\alpha}{0}$, and if $(\mathcal{U},\varphi)$ is a
                linear chart containing $p$, then:
                \begin{equation}
                    v_{p}=\sum_{k\in\mathbb{Z}_{n}}
                        \varphi^{k}(v)\partial_{\varphi^{k}}|_{p}
                \end{equation}
            \end{theorem}
            \begin{proof}
                For by linearity:
                \begin{equation}
                    \varphi^{k}(\alpha(t))=\varphi^{k}(p)+t\varphi^{k}(v)
                \end{equation}
                But then:
                \begin{equation}
                    v_{p}=\dot{\alpha}(0)
                    =\sum_{k\in\mathbb{Z}_{n}}
                        \frac{\diff}{\diff{t}}(\varphi^{k}\circ\alpha)
                        \partial_{\varphi^{k}}|_{p}
                    =\sum_{k\in\mathbb{Z}_{n}}\varphi^{k}(v)
                        \partial_{\varphi^{k}}|_{p}
                \end{equation}
            \end{proof}
            The function $v\mapsto{v}_{p}$ is a linear isomorphism from $V$ to
            $\tanspace{p}{V}$, similarly for $v_{q}\mapsto{v}_{p}$.
        \subsection{The Tangent Bundle}
            \begin{definition}
                The tangent bundle of a smooth manifold $\manifold{M}$ is the
                set:
                \begin{equation}
                    TM=\coprod_{p\in{M}}\tanspace{p}{M}
                        =\bigcup_{p\in{M}}\{p\}\times\tanspace{p}{M}
                \end{equation}
            \end{definition}
            We can topologize $TM$ in a way that makes this a manifold of
            dimension $2n$ and this is \textbf{not} the disjoint union topology.
            Indeed, for any uncountable manifold $M$, the disjoint union
            topology will not be second countable, but instead will be a
            locally Euclidean Hausdorff topological space of the same dimension
            as $M$. That is, a \textit{large} $n$ dimensional manifold. To
            topologize $TM$, we consider the projection mapping $\pi$ sending
            $(p,v)\in{TM}$ to $p\in{M}$. Given a chart $(\mathcal{U},\varphi)$
            in $M$, we look at $\pi^{\minus{1}}(\mathcal{U})$, which is a
            subset of $TM$. Since the $\partial_{\varphi^{k}}|_{p}$ uniquely
            determine any tangent vector at any point $p\in\mathcal{U}$, we can
            define a new chart $(\pi^{\minus{1}}(\mathcal{U}),\tilde{\varphi})$
            defined by:
            \begin{equation}
                \tilde{\varphi}(p,v)=\big(\varphi(p),\dot{\varphi}(v)\big)
            \end{equation}
            Where $\dot{\varphi}(p)$ is the function such that the $k^{th}$
            component is $\dot{\varphi}(v)_{k}=v(\varphi^{k})$. We do this for
            every chart, showing that we now have a $2n$ dimesnional atlas that
            gives rise to a Hausdorff topology. To complete the claim that $TM$
            is a smooth manifold we need to show that this atlas has smoothly
            overlapping charts. Suppose
            $(\pi^{\minus{1}}(\mathcal{U},\tilde{\varphi}))$ and
            $(\pi^{\minus{1}}(\mathcal{V},\tilde{\psi}))$ are two such charts
            with non-empty overlap. Then of $k\in\mathbb{Z}_{n}$, we have:
            \begin{equation}
                \pi^{k}\circ(\tilde{\varphi}\circ\tilde{\psi}^{\minus{1}})
                =(\pi^{k}\circ\tilde{\varphi})\circ\tilde{\psi}^{\minus{1}}
                =(\varphi^{k}\circ\pi)\circ\tilde{\psi}^{\minus{1}}
                =\varphi^{k}\circ\psi^{\minus{1}}
            \end{equation}
            Which is smooth. Similarly, for $k\in\mathbb{Z}_{n}$, we have:
            \begin{equation}
                \pi^{n+k}\circ(\tilde{\varphi}\circ\tilde{\psi}^{\minus{1}})
                =(\pi^{n+k}\circ\tilde{\varphi})\circ\tilde{\psi}^{\minus{1}}
            \end{equation}
            And that's nice.
            \begin{definition}
                A smooth section in the tangent bundle $TM$ of a smooth manifold
                $\manifold{M}$ is a smooth function $f:M\rightarrow{TM}$ such
                that $\pi\circ{f}=\textrm{id}_{M}$.
            \end{definition}
            Every vector field is thus a smooth section in the tangent bundle.
            This gives us the generalization to vector fields on a smooth map.
            A vector field on a smooth map $\phi:M\rightarrow{N}$ is a smooth
            function $V:M\rightarrow{TN}$ such that $\pi\circ{V}=\phi$. The
            velocity of a curve is thus an example of a smooth vector field on
            the function $\alpha{I}\rightarrow{M}$.
        \subsection{Integral Curves}
            \begin{definition}
                An integral curve of a vector field $V\in\smoothvecf{M}$ on a
                smooth manifold $\manifold{M}$ is a curve
                $\alpha:I\rightarrow{M}$ such that for all $t\in{I}$,
                $\dot{\alpha}(t)=V_{\alpha(t)}$.
            \end{definition}
    \section{Notes from Hirsch (Chapter 1)}
        Manifolds are things.
    \section{Manifolds Review}
        \begin{definition}
            A locally Euclidean topological space is a topological space
            $(X,\tau)$ such that for all $x\in{X}$ there is an $n\in\mathbb{N}$
            such that there exists an open set $\mathcal{U}\subseteq{X}$ such
            that $x\in\mathcal{U}$ and a continuous injective open mapping
            $\varphi:\mathcal{U}\rightarrow\mathbb{R}^{n}$.
        \end{definition}
        Since $\varphi$ is an injective open mapping, it is automatically a
        homeomorphism onto it's image. Indeed, it is common to define locally
        Euclidean in terms of homeomorphisms. The dimension can vary in this
        definition and we can consider the disjoint union of a sphere and a
        line. However, dimension is locally constant. This requires Brouer's
        invariance of domain.
        \begin{ftheorem}{Invariance of Domain}{Invariance of Domain}
            If $\mathcal{U}\subseteq\mathbb{R}^{n}$ is open, and if
            $f:\mathcal{U}\rightarrow\mathbb{R}^{n}$ is a continuous injective
            function, then $f(\mathcal{U})$ is open.
        \end{ftheorem}
        \begin{bproof}
            Difficult.
        \end{bproof}
        We can now prove that dimension is locally constant.
        \begin{ltheorem}{Invariance of Dimension}{Invariance_of_Dimension}
            If $\mathcal{U}\subseteq\mathbb{R}^{n}$ is open and
            $\mathcal{V}\subseteq\mathbb{R}^{m}$ is open, and if
            $f:\mathcal{U}\rightarrow\mathcal{V}$ is a homeomorphism, then
            $n=m$
        \end{ltheorem}
        \begin{proof}
            For suppose not and suppose $n>m$. Let $\tilde{\mathcal{V}}$ be the
            extension of $\mathcal{V}$ into $\mathbb{R}^{n}$ where the last
            $n-m$ coordinates are simply 0. But $f$ is a homeomorphism, and thus
            the extension $\tilde{f}:\mathcal{U}\rightarrow\mathbb{R}^{n}$ is
            an injective continuous function, and hence an open mapping.
            But $f(\mathcal{U})$ is not open since for any point and for any
            $\varepsilon>0$ the $\varepsilon$ be must leak into the last
            $n-m$ coordinates, a contradiction. Thus $n=m$.
        \end{proof}
        \begin{theorem}
            If $(X,\tau)$ is a locally Euclidean topological space, if
            $x\in{X}$, and if $n,m\in\mathbb{N}$ are such that there exists
            open sets $\mathcal{U}_{n}$ and $\mathcal{V}_{m}$ such that there
            are injective open mapping
            $\varphi_{n}:\mathcal{U}_{n}\rightarrow\mathbb{R}^{n}$ and
            $\varphi_{m}:\mathcal{V}_{m}\rightarrow\mathbb{R}^{m}$, then $m=n$.
        \end{theorem}
        \begin{proof} 
            For $\mathcal{U}_{n}\cap\mathcal{V}_{m}$ is non-empty since $x$ is
            in the intersection, and the intersection of open is open. Thus the
            resection $\varphi_{n}|_{\mathcal{U}_{n}\cap\mathcal{V}_{m}}$ and
            $\varphi_{m}|_{\mathcal{U}_{n}\cap\mathcal{V}_{m}}$ are open
            mappings into $\mathbb{R}^{n}$ and $\mathbb{R}^{m}$, respectively.
            But then these are homeomorphisms onto their images. But by
            composing $\varphi_{n}\circ\varphi_{m}^{\minus{1}}$ we obtain a
            homeomorphism from an open subset of $\mathbb{R}^{m}$ to
            $\mathbb{R}^{n}$. But then $n=m$, by the previous theorem.
        \end{proof}
        We have thus shown that there is a well defined dimension function
        $\textrm{dim}:X\rightarrow\mathbb{N}$ that assigns to every $x\in{X}$
        the unique dimension of $x$. That is, the space $\mathbb{R}^{n}$ that
        $x$ locally looks like.
        \begin{definition}
            A locally constant function on from a topological space $(X,\tau)$
            to a set $Y$ is a function $f:X\rightarrow{Y}$ such that for all
            $x\in{X}$ there exists an open subset $\mathcal{U}\in\tau$ such that
            $x\in\mathcal{U}$ and $f|_{\mathcal{U}}$ is a constant mapping.
        \end{definition}
        Note that there need not be any topology on $Y$.
        \begin{theorem}
            If $(X,\tau)$ is a connected topological space, if $Y$ is a set with
            at least two distinct points, and if $f:X\rightarrow{Y}$ is a
            locally constant function, then it is a constant function.
        \end{theorem}
        \begin{proof}
            For suppose not. Then there are $c_{1},c_{2}\in{Y}$ such that
            $f^{\minus{1}}[\{c_{1}\}]$ and $f^{\minus{1}}[\{c_{2}\}]$ are
            non-empty. Let $x\in{f}^{\minus{1}}[\{c_{1}\}]$. Since $f$ is a
            locally constant function, there exists an open subset
            $\mathcal{U}_{x}\subseteq{X}$ such that $x\in\mathcal{U}$ and
            $f|_{\mathcal{U}}$ is a constant mapping. But this is true of all
            $x\in{f}^{\minus{1}}[\{c_{1}\}]$, and hence:
            \begin{equation}
                f^{\minus{1}}[\{c_{1}\}]=
                    \bigcup_{x\in{f}^{\minus{1}}[\{c_{1}\}]}\mathcal{U}_{x}
            \end{equation}
            which is the union of open, and hence open. Similarly, the
            complement can be written as:
            \begin{equation}
                X\setminus{f}^{\minus{1}}[\{c_{1}\}]=
                    \bigcup_{x\notin{f}^{\minus{1}}[\{c_{1}\}]}\mathcal{U}_{x}
            \end{equation}
            which is the union of open, and hence open. Moreover it is non-empty
            since $f^{\minus{1}}[\{c_{2}\}]$ is a subset, and this set is
            non-empty. But then $X$ is the union of two disjoint non-empty open
            subsets and is hence disconnected, a contradiction. Thus, $f$ is
            constant.
        \end{proof}
        \begin{theorem}
            If $(X,\tau)$ is a locally Euclidean topological space, then the
            dimension function $\textrm{dim}:X\rightarrow\mathbb{N}$ is locally
            constant.
        \end{theorem}
        \begin{proof}
            For let $x\in{X}$. Then there is an open subset
            $\mathcal{U}\subseteq{X}$ such that $x\in\mathcal{U}$ and
            there exists a continuous injective open mapping
            $f:\mathcal{U}\rightarrow\mathbb{R}^{n}$. But then for all
            $y\in\mathcal{U}$, $\mathcal{U}$ is an open set such that
            $y\in\mathcal{U}$ and $f:\mathcal{U}\rightarrow\mathbb{R}^{n}$ is a
            continuous injective open mapping, and hence
            $\textrm{dim}(y)=\textrm{dim}(x)$. Thus, $\textrm{dim}$ is locally
            constant.
        \end{proof}
        \begin{theorem}
            If $(X,\tau)$ is a connected locally Euclidean topological space,
            then there is a unique dimension.
        \end{theorem}
        \begin{proof}
            For $\textrm{dim}$ is locally constant, and a locally constant
            function on a connected space is constant.
        \end{proof}
        \begin{definition}
            A topological manifold is a topological space $(X,\tau)$
            that is locally Euclidean, Hausdorff, and second countable.
        \end{definition}
        \begin{example}
            Neither the Hausdorff nor the second countable claims are redundant.
            The bug eye line is a non-Hausdorff locally Euclidean space that is
            second countable, and the long line is a non-second countable space
            that is locally Euclidean and Hausdorff.
        \end{example}
        \begin{definition}
            A chart on a topological space $(X,\tau)$ is an open set
            $\mathcal{U}$ and a continuous injective open mapping
            $\varphi:\mathcal{U}\rightarrow\mathbb{R}^{n}$.
        \end{definition}
        \begin{definition}
            Compatible charts on a topological space $(X,\tau)$ are charts
            $(\mathcal{U},\varphi)$ and $(\mathcal{V},\psi)$ such that either
            $\mathcal{U}\cap\mathcal{V}$ are empty, or
            $\varphi\circ\psi^{\minus{1}}:%
             \psi(\mathcal{V})\rightarrow\mathbb{R}^{n}$ and
            $\psi\circ\varphi^{\minus{1}}:%
             \varphi(\mathcal{U})\rightarrow\mathbb{R}^{n}$ are smooth.
        \end{definition}
        \begin{definition}
            A smooth atlas on a topological manifold $(X,\tau)$ is a collection
            of compatible charts that cover $X$.
        \end{definition}
        \begin{definition}
            A maximal smooth atlas on a topological manifold $(X,\tau)$ is a
            smooth atlas $\mathcal{A}$ such that for all smooth atlases
            $\mathcal{A}'$ such that $\mathcal{A}\subseteq\mathcal{A}'$, it is
            true that $\mathcal{A}=\mathcal{A}'$.
        \end{definition}
        \begin{definition}
            Compatible atlases on a topological manifold $(X,\tau)$ are atlases
            $\mathcal{A}$ and $\mathcal{A}'$ such that for every chart in
            $\mathcal{A}$ is compatible with every chart in $\mathcal{A}'$, and
            vice-versa.
        \end{definition}
        \begin{theorem}
            If $(X,\tau)$ is a topological manifold and if $\mathcal{A}$ and
            $\mathcal{A}'$ are smooth atlases, then they are compatible if and
            only if $\mathcal{A}\cup\mathcal{A}'$ is a smooth atlas.
        \end{theorem}
        \begin{proof}
            For if $\mathcal{A}$ and $\mathcal{A}'$ are compatible, then
            every element of $\mathcal{A}\cup\mathcal{A}'$ is compatible with
            every other element, and moreover the domains cover $X$. Hence,
            the union is a smooth atlas. If the union is a smooth atlas, then
            every element of $\mathcal{A}$ is compatible with every element of
            $\mathcal{A}\cup\mathcal{A}'$, and in particular it is compatible
            with every element of $\mathcal{A}'$. Similarly, every element of
            $\mathcal{A}'$ is compatible with $\mathcal{A}$, and hence they are
            compatible atlases.
        \end{proof}
        \begin{ltheorem}{Existence and Uniqueness of Maximal Smooth Atlas}
                        {Existence and Uniqueness of Maximal Smooth Atlas}
            If $(X,\tau)$ is a topological manifold and if $\mathcal{A}$ is a
            smooth atlas, then there is a maximal smooth atlas $\mathcal{C}$
            such that $\mathcal{A}\subseteq\mathcal{C}$.
        \end{ltheorem}
        \begin{proof}
            For consider the set of all atlases on $(X,\tau)$. This forms a
            partially ordered set by inclusion. Given a chain of atlases,
            the union over the entire chain is again an atlas by the previous
            theorem. But then every chain is bounded, and thus by Zorn's lemma
            for every $\mathcal{A}$ there is a maximal element $\mathcal{C}$
            that bounds $\mathcal{A}$. If $\mathcal{C}'$ is another, then
            they are compatible since they are both compatible with
            $\mathcal{A}$, and thus $\mathcal{C}\cup\mathcal{C}'$ is a smooth
            atlas, contradicting maximality. Hence, $\mathcal{C}$ is unique.
        \end{proof}
        \begin{definition}
            A smooth manifold is a topological manifold $(X,\tau)$ with a
            maximal smooth atlas $\mathcal{A}$.
        \end{definition}
        \begin{definition}
            A smooth function from a smooth manifold
            $(M,\tau_{M},\mathcal{A}_{M})$ to a smooth manifold
            $(N,\tau_{N},\mathcal{A}_{n})$ is a function $\phi:M\rightarrow{N}$
            such that for all $p\in{M}$ there is a chart
            $(\mathcal{U},\varphi)\in\mathcal{A}_{M}$ and a chart
            $(\mathcal{V},\psi)\in\mathcal{A}_{N}$ such that $p\in\mathcal{U}$,
            $\phi(p)\in\mathcal{V}$, and the mapping
            $\psi\circ\phi\circ\varphi^{\minus{1}}:%
             \varphi(\mathcal{U}\cap\phi^{\minus{1}}(\mathcal{V}))%
             \rightarrow\mathbb{R}^{n}$ is a smooth function.
        \end{definition}
        Note that since charts in a smooth atlas overlap smoothly, this
        definition does not actually depend on the choice of charts.
        \begin{definition}
            A diffeomorphism from a smooth manifold
            $(M,\tau_{M},\mathcal{A}_{M})$ to a smooth manifold
            $(N,\tau_{N},\mathcal{A}_{N})$ is a homeomorphism
            $\phi:M\rightarrow{N}$ such that $\phi$ and $\phi^{\minus{1}}$ are
            smooth.
        \end{definition}
        In particular, we can consider the set of all smooth functions
        from the manifold $M$ into $\mathbb{R}$, where $\mathbb{R}$ has its
        usual smooth structure. This structure $C^{\infty}(M,\mathbb{R})$ forms
        a commutative ring.
        \begin{definition}
            A tangent vector at a point $p$ in a smooth manifold
            $(M,\tau,\mathcal{A})$ is a function
            $v:C^{\infty}(M,\mathbb{R})\rightarrow\mathbb{R}$ that is linear and
            Liebnizian. That is:
            \begin{align}
                v(af+bg)&=av(f)+bv(g)\tag{Linearity}\\
                v(fg)&=v(f)g(p)+f(p)v(g)\tag{Liebnizian}
            \end{align}
        \end{definition}
        \begin{definition}
            The tangent space at a point $p$ in a smooth manifold
            $(M,\tau,\mathcal{A})$ is the set $T_{p}M$ of all tangent vectors
            at $p$.
        \end{definition}
        This may seem abstract, but it is a direct generalization of the
        directional derivative one studies in vector calculus.
        \begin{theorem}
            If $\phi:M\rightarrow{N}$ is a smooth function, if $p\in{M}$, and if
            $v$ is a tangent vector at $p$, then the function
            $\diff_{p}\phi:C^{\infty}(N,\mathbb{R})\rightarrow\mathbb{R}$
            defined by $\diff_{p}\phi(v)(f)=v(f\circ\phi)$ is an element of
            $T_{\phi(p)}N$.
        \end{theorem}
        \begin{proof}
            For:
            \begin{equation}
                v\big((af+bg)\circ\phi\big)
                =v\big(a(f\circ\phi)+b(g\circ\phi)\big)
                =av(f\circ\phi)+bv(g\circ\phi)
            \end{equation}
            and:
            \begin{equation}
                v\big((fg)\circ\phi\big)
                =v\big((f\circ\phi)(g\circ\phi)\big)
                =v(f\circ\phi)(g\circ\phi)(p)+(f\circ\phi)(p)v(g\circ\phi)
            \end{equation}
        \end{proof}
        \begin{definition}
            The differential pushforward of a smooth function
            $\phi:M\rightarrow{N}$ from a smooth manifold
            $(M,\tau_{M},\mathcal{A}_{M})$ to a smooth manifold
            $(N,\tau_{N},\mathcal{A}_{M})$ at a point $p\in{M}$ is the function
            $\diff_{p}\phi:T_{p}M\rightarrow{T}_{\phi(p)}N$ defined by
            $\diff_{p}\phi(v)(f)=v(f\circ\phi)$ for all
            $f\in{C}^{\infty}(N,\mathbb{R})$.
        \end{definition}
        \begin{definition}
            An open curve in a topological space $(X,\tau)$ is a continuous
            function $\alpha:I\rightarrow{X}$, where $I$ is the open unit
            interval.
        \end{definition}
        \begin{definition}
            A smooth curve in a smooth manifold $(M,\tau,\mathcal{A})$ is a
            curve $\alpha:I\rightarrow{M}$ such that $\alpha$ is a smooth
            function with respect to the standard smooth structure on $I$.
        \end{definition}
        \begin{definition}
            The velocity of a smooth curve $\alpha:I\rightarrow{M}$ at a point
            $t\in{I}$ differential pushforward evaluated at the tangent vector
            $\diff/\diff{x}|_{x=t}$. That is:
            \begin{equation}
                \dot{\alpha}(t)=
                    \diff_{t}\alpha\Big(\frac{\diff}{\diff{x}}\big|_{x=t}\Big)
            \end{equation}
        \end{definition}
        \begin{definition}
            The tangent bundle of a smooth manifold $(M,\tau,\mathcal{A})$ is
            the set $TM=\coprod_{p\in{M}}T_{p}M$. That is, the disjoint union of
            all tangent spaces.
        \end{definition}
        There is a topology that one can place on the tangent bundle that makes
        it a $2n$ dimensional manifold (with a natural smooth atlas) and this is
        not the disjoint union topology. Indeed for any manifold of non-zero
        dimension the disjoint union topology is not second countable, and hence
        not a manifold (but it will be an $n$ dimensional locally Euclidean
        Hausdorff space).
        \begin{definition}
            A vector field is a smooth function $V:M\rightarrow{TM}$ such that
            for all $p\in{M}$, $V(p)\in{T}_{p}M$.
        \end{definition}
        The collection $\mathfrak{X}(M)$ of all smooth vector fields on a smooth
        manifold $M$ has a module structure over $C^{\infty}(M,\mathbb{R})$ is
        we define addition and scalar multiplication as follows:
        \begin{align}
            (fV)_{p}&=f(p)V_{p}\\
            (V+W)_{p}&=V_{p}+W_{p}
        \end{align}
        We can also evaluate vector fields at functions in
        $C^{\infty}(M,\mathbb{R})$ as follows:
        \begin{equation}
            (Vf)(p)=V_{p}(f)
        \end{equation}
        This is well defined since for all $p\in{M}$, $V_{p}$ is a tangent
        vector in $T_{p}M$, which is a linear functional. Thus $Vf$ is a
        function from $C^{\infty}(M,\mathbb{R})$ to itself. The structure of
        $C^{\infty}(M,\mathbb{R})$ can also be seen as an algebra over the field
        $\mathbb{R}$. As such we can define derivations.
        \begin{definition}
            A derivation on a $C^{\infty}(M,\mathbb{R})$ is a function
            $D:C^{\infty}(M,\mathbb{R})\rightarrow{C}^{\infty}(M,\mathbb{R})$
            that is $\mathbb{R}$ linear and Liebnizian:
            \begin{align}
                D(af+bg)&=aD(f)+bD(g)\\
                D(fg)&=D(f)g+fD(g)
            \end{align}
        \end{definition}
        This is slightly redundant since every derivation comes from a vector
        field. Given $V\in\mathfrak{X}(M)$, the function
        $D:C^{\infty}(M,\mathbb{R})\rightarrow{C}^{\infty}(M,\mathbb{R})$
        defined by $D(v)=Vf$ is a derivation, and moreover every derivation
        comes from a vector field. Simply let $V_{p}$ be the tanget vector such
        that $V_{p}(f)=D(f)(p)$ for all $p\in{M}$ and
        $f\in{C}^{\infty}(M,\mathbb{R})$. 
        \par\hfill\par
        We can also evaluated vector fields on other vector fields. Given
        $V,W\in\mathfrak{X}(M)$, we denote $V(W)$ to be the function
        $V(W):C^{\infty}(M,\mathbb{R})\rightarrow{C}^{\infty}(M,\mathbb{R})$
        defined by:
        \begin{equation}
            V(W)(f)=V(Wf)
        \end{equation}
        More explicitly, if $p\in{M}$, then:
        \begin{equation}
            \Big(\big(V(W)\big)(f)\Big)(p)=V_{p}(Wf)
        \end{equation}
        Remember that $Wf$ is a function from $C^{\infty}(M,\mathbb{R})$ into
        itself, where as $V_{p}$ is a tangent vector and hence a linear
        functional from $C^{\infty}(M,\mathbb{R})$ into $\mathbb{R}$. Thus,
        while strange, this definition is well posed. What's important is that
        now we can define the Lie bracket of two vector fields.
        \begin{definition}
            The Lie Bracket is the function
            $[\cdot,\cdot]:\mathfrak{X}(M)\times\mathfrak{X}(M)%
             \rightarrow\mathfrak{X}(M)$ defined by:
            \begin{equation}
                [V,W]=V(W)-W(V)
            \end{equation}
        \end{definition}
        More explicitly, given $f\in{C}^{\infty}(M,\mathbb{R})$ and $p\in{M}$,
        we have:
        \begin{equation}
            [V,W]_{p}(f)=V_{p}(Wf)-W_{p}(Vf)
        \end{equation}
        Before moving on to connections, we define integral curves.
        \begin{definition}
            An integral curve of a vector field $V\in\mathfrak{X}(M)$ is a
            curve $\alpha:I\rightarrow{M}$ such that for all $t\in{I}$,
            $\dot{\alpha}(t)=V_{\alpha(t)}$.
        \end{definition}
        \begin{definition}
            A bilinear form on a vector field $V$ over a field $F$ is a function
            $g:V\times{V}\rightarrow{F}$ such that:
            \begin{align}
                g(a\vector{x}+b\vector{y},\vector{z})
                    &=ag(\vector{x},\vector{z})+bg(\vector{y},\vector{z})\\
                g(\vector{x},a\vector{y}+b\vector{z})
                    &=ag(\vector{x},\vector{y})+bg(\vector{x},\vector{z})
            \end{align}
        \end{definition}
        \begin{definition}
            A symmetric form on a vector space $V$ over a field $F$ is a
            function $g:V\times{V}\rightarrow{F}$ such that for all
            $\vector{x},\vector{y}\in{V}$,
            $g(\vector{x},\vector{y})=g(\vector{y},\vector{x})$.
        \end{definition}
        \begin{definition}
            A non-degenerate form on a vector space $V$ over a field $F$ is a
            function $g:V\times{V}\rightarrow{F}$ such that for all
            $\vector{x}\in{V}$ such that $\vector{x}\ne\vector{0}$ there exists
            a $\vector{y}\in{V}$ such that $g(\vector{x},\vector{y})\ne{0}$.
        \end{definition}
        In other words, if $g(\vector{x},\vector{y})=0$ for all $\vector{y}$,
        then $\vector{x}=\vector{0}$. Lastly, positive-definiteness.
        \begin{definition}
            A positive-definite form on a vector space $V$ over an ordered field
            $F$ is a function $g:V\times{V}\rightarrow{F}$ such that for all
            $\vector{x}\in{V}$ it is true that $g(\vector{x},\vector{x})\geq{0}$
            and $g(\vector{x},\vector{x})=0$ implies that
            $\vector{x}=\vector{0}$.
        \end{definition}
        Semi-Riemannian and Riemannian geometry are concerned with smooth
        manifolds that have symmetric bilinear forms that are non-degenerate
        (Semi-Riemannian) or positive-definite (Riemannian). The vector spaces
        are simply the tangent spaces, and we need a function $g$ that takes in
        a point $p\in{M}$ and two elements of $T_{p}M$ and returns a real
        number. We further want this function to be a symmetric non-degenerate
        bilinear form for every $p$, and we would also like this function to
        vary smoothly between points. So in essence, we want a function
        $g:\mathfrak{X}(M)\times\mathfrak{X}(M)%
         \rightarrow{C}^{\infty}(M,\mathbb{R})$. Given a point $p\in{M}$,
        and two vector fields $V,W\in\mathfrak{X}(M)$, we want
        $g(V,W)(p)=g_{p}(V_{p},W_{p})$ to vary smoothly with $p$. We also want
        symmetry, bilinearity, and non-degenacy.
        \begin{definition}
            A metric tensor on a smooth manifold $(M,\tau,\mathcal{A})$ is a
            smooth non-degenacy symmetric bilinear form
            $g:\mathfrak{X}(M)\times\mathfrak{X}(M)%
             \rightarrow{C}^{\infty}(M,\mathbb{R})$.
        \end{definition}
        \begin{definition}
            A semi-Riemannian manifold is a smooth manifold
            $(M,\tau,\mathcal{A})$ with a metric tensor $g$.
        \end{definition}
    \section{Connections}
        We now want to talk about transporting data along a manifold in a
        parallel manner. For Euclidean space $\mathbb{R}^{n}$ there is an
        intuitive manner to do this: Simply translate your vector from point $a$
        to point $b$. On a sphere there's a slightly intuitive manner.
        Translations may leave the sphere and so we need a new method. Give two
        points we can simply rotate a tangent vector at $a$ to $b$, but the
        resultant vector depends on how one rotated the sphere. That is, along
        which path did one rotate. This is simply a consequence of the curvature
        of the sphere. For the problem of generalizing to aribtrary
        manifolds one might think coordinates suffice to perform parallel
        transport, but this is not true. Even on the sphere, using the two
        stereographic projections about the north and south pole, one runs into
        incompatibility issues. So the idea is to find a way of describing the
        rate of change of one vector field with respect to another. Given a
        vector field $V$ and a tangent vector $v\in{T}_{p}M$ for some point
        $p\in{M}$ such that $V_{p}=v$, and given another vector field $W$,
        we can think of parallel transport as transporting $v$ along an
        integral curve of $W$.
        \begin{definition}
            A connection $\nabla$ on a smooth manifold $(M,\tau,\mathcal{A})$
            is a function $\nabla:\mathfrak{X}(M)\times\mathfrak{X}(M)%
            \rightarrow\mathfrak{X}(M)$ such that:
            \begin{align}
                \nabla(fV_{1}+gV_{2},W)
                    &=f\nabla(V_{1},W)+g\nabla(V_{2},W)\\
                \nabla(V,aW_{1}+bW_{2})
                    &=a\nabla(V,W_{1})+b\nabla(V,W_{2})\\
                \nabla(V,fW)&=
                    (Vf)W+f\nabla(V,W)
            \end{align}
        \end{definition}
        \begin{definition}
            A Levi-Civita connection on a semi-Riemannian manifold
            $(M,g)$ is a connection $\nabla$ that is torsion
            free (preserves the Lie bracket):
            \begin{equation}
                \nabla(V,W)-\nabla(W,V)=[V,W]
            \end{equation}
            and preserves the metric tensor:
            \begin{equation}
                Xg(V,W)=g\big(\nabla(X,V),W\big)+g\big(V,\nabla(X,W)\big)
            \end{equation}
        \end{definition}
        \begin{ftheorem}{Fundamental Theorem of Riemannian Geometry}
                        {Fundamental_Theorem_of_Riemannian Geometry}
            If $(M,g)$ is a semi-Riemannian manifold, then there is a unique
            Levi-Civita connection $\nabla$ on $(M,g)$.
        \end{ftheorem}
        \begin{bproof}
            Just use the Koszul Formula:
            \begin{equation}
                \begin{split}
                    2g\big(\nabla(X,V),W)=&\partial_{X}\big(g(V,W)\big)+
                        \partial_{Y}\big(g(X,W)\big)\\
                        &-\partial_{Z}\big(g(X,V)\big)+g\big([X,V],W\big)\\
                        &-g\big([X,W],V\big)-g\big([V,W],X\big)
                \end{split}
            \end{equation}
        \end{bproof}
    \section{Pendulums}
        We can apply our new found Levi-Civita connection to a classic problem.
        Consider the sphere with it's standard embedding into $\mathbb{R}^{3}$.
        There is a natural metric tensor we can associate too it, combined with
        the standard smooth structure induced by the orthographic projections,
        that make $S^{2}$ a Riemannian manifold. That is, there is a natural
        metric tensor that is also positive-definite. Take the dot product
        $\langle\cdot|\cdot\rangle:\mathbb{R}^{3}\rightarrow\mathbb{R}$ apply
        this to tangent vectors in $T_{p}S^{2}$. Since the tangent spaces are
        isometric to subspaces of $\mathbb{R}^{3}$, there's a nice way to do
        this. The dot product varies smoothly as one moves smoothly from one
        point to another, and so this creates a metric tensor $g$ and
        $(M,g)$ is a Riemannian manifold.
        \par\hfill\par
        Consider a pendulum in Paris oscillating in the direction of north to
        south. We can attach a tangent vector to this pendulum pointing towards
        the north pole. What happens as the Earth rotates around it's axis after
        a full day? The resulting parallel transport is no long pointing towards
        the north pole, but rather has tipped 270 degrees and oscillates east
        to west. This is in agreement with the parallel transport we get from
        our Levi-Civita connection on a sphere. This connection gives rotations
        as the means of parallel transport, in correspondence with our
        intuition. The change in angle is a function of the area of the curve
        enclosed.
\end{document}