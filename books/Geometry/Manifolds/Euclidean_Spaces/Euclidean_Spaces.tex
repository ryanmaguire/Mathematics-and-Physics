\section{Locally Euclidean Spaces}
    We wish to speak about topological spaces that are sufficiently well behaved
    in the sense that it allows one to do calculus. The first naive approach is
    to consider such spaces that can be locally approximated by Euclidean space,
    but subtle problems can arise with such a definition. Nevertheless, we start
    with such spaces.
    \begin{fdefinition}{Locally Euclidean Topological Space}
                       {Locally_Euclidean_Topological_Space}
        A locally Euclidean topological space is a topological space $(X,\tau)$
        such that for all $x\in{X}$ there exists an $n\in\mathbb{N}$, an open
        set $\mathcal{U}\in\tau$, and an open set
        $\mathcal{V}\in\tau_{\mathbb{R}^{n}}$, where $\tau_{\mathbb{R}^{n}}$ is
        the standard topology on $\mathbb{R}^{n}$, such that $x\in\mathcal{U}$
        and $(\mathcal{U},\tau|_{\mathcal{U}})$ is homeomorphic to
        $(\mathcal{V},\tau_{\mathbb{R}^{n}}|_{\mathcal{V}})$.
    \end{fdefinition}
    While we were very liberal in our definition, allowing the set
    $\mathcal{V}\subseteq\mathbb{R}^{n}$ to be any open set, we need not be.
    A topological space is locally Euclidean if and only if every point has an
    open neighborhood that is homeomorphic to all of $\mathbb{R}^{n}$.
    \begin{theorem}
        If $(X,\tau)$ is a locally Euclidean topological space, then for all
        $x\in{X}$ there exists an $n\in\mathbb{N}$ and an open set
        $\mathcal{N}\in\tau$ such that $x\in\mathcal{N}$ and such that
        $(\mathcal{N},\tau|_{\mathcal{N}})$ is homeomorphic to
        $(\mathbb{R}^{n},\tau_{\mathbb{R}^{n}})$, where $\tau_{\mathbb{R}^{n}}$
        is the standard topology on $\mathbb{R}^{n}$.
    \end{theorem}
    \begin{proof}
        For since $(X,\tau)$ is locally Euclidean there is an open subset
        $\mathcal{U}\in\tau$ such that $x\in\mathcal{U}$ and such that there is
        an open subset $\mathcal{V}\in\tau_{\mathbb{R}^{n}}$ such that
        $\mathcal{U}$ is homeomorphic to $\mathcal{V}$
        (Def.~\ref{def:Locally_Euclidean_Topological_Space}). But if
        $\mathcal{U}$ is homeomorphic to $\mathcal{V}$, then there exists a
        homeomorphism $\phi:\mathcal{U}\rightarrow\mathcal{V}$. But since
        $x\in\mathcal{U}$, there is a $p\in\mathcal{V}$ such that $\phi(x)=p$.
        But $\mathcal{V}$ is an open subset of $\mathbb{R}^{n}$, and thus there
        exists an $r\in\mathbb{R}^{+}$ such that:
        \begin{equation}
            B_{r}^{(\mathbb{R}^{n},d)}(p)\subseteq\mathcal{V}
        \end{equation}
        where $d$ denotes the standard metric on $\mathbb{R}^{n}$ and $B_{r}$
        denotes the open ball about $p$. But if $\phi$ is a homeomorphism, then
        $\phi^{\minus{1}}:\mathcal{V}\rightarrow\mathcal{U}$ is a homeomorphism.
        Let $\mathcal{N}=\phi^{\minus{1}}(B_{r}(p))$. 
        But the restriction of a homeomorphism is a homeomorphism, and thus
        $\phi^{\minus}|_{B_{r}(p)}\rightarrow\mathcal{N}$ is a homeomorphism.
        But since $p\in{B}_{r}(p)$, and since $x=\phi^{\minus{1}}(p)$, we have
        that $x\in\mathcal{N}$. Thus $\mathcal{N}$ is an open subset of
        $X$ such that $x\in\mathcal{N}$ and $\mathcal{N}$ is homeomorphic to
        $B_{r}(p)$. But $B_{r}(p)$ is homeomorphic to $\mathbb{R}^{n}$, and
        since homeomorphic is an equivalence relation, we are done.
    \end{proof}
    Note that in our definition of a locally Euclidean space we did not require
    the \textit{dimension} of the space to be constant. That is, we said that
    for all $x\in{X}$ there exists an $n\in\mathbb{N}$ such that $x$ is locally
    similar to $\mathbb{R}^{n}$, but this value of $n$ may vary from point to
    point. In most definitions of manifolds, such variations are excluded, but
    this is exclusion is not entirely necessary if one considers connected
    spaces. That is, if $(X,\tau)$ is connected, then there is an unambiguous
    notion of dimension. It is easy enough to think of spaces that are locally
    Euclidean in which the dimension varies: consider the disjoint union of a
    line and a sphere, or a line and a plane. Such spaces will have points that
    are locally like $\mathbb{R}$ and points that are locally $\mathbb{R}^{2}$,
    but they are also disconnected. If we tried to connect the two parts we
    would have a point that is not locally Euclidean at all.
    \begin{theorem}
        \label{thm:Equiv_Def_of_Locally_Euclidean_Top_Space}%
        If $(X,\tau)$ is a connected topological space, if $Y$ is a non-empty
        set, and if $f:X\rightarrow{Y}$ is a locally constant function, then
        $f$ is a constant function.
    \end{theorem}
    \begin{proof}
        For suppose not. Then there are two values $y_{1},y_{2}\in{Y}$ such that
        $y_{1}\ne{y_{2}}$ and $f^{\minus{1}}(\{y_{1}\})$ and
        $f^{\minus{1}}(\{y_{2}\})$ are non-empty. Since $f$ is locally constant,
        for all $x\in{f}^{\minus{1}}(\{y_{1}\})$ there exists an open subset
        $\mathcal{U}_{x}\in\tau$ such that $x\in\mathcal{U}_{x}$ and for all
        $x_{0}\in\mathcal{U}_{x}$ it is true that $f(x)=f(x_{0})$. But
        $x\in{f}^{\minus{1}}(\{y_{1}\})$ and thus $f(x)=y_{1}$
        (Def.~\ref{def:Pre_Image_of_Subset}). Thus, by the transitivity of
        equality (Thm.~\ref{thm:Transitivity_of_Equality}), for all
        $x_{0}\in\mathcal{U}_{x}$ it is true that $f(x_{0})=y_{1}$. But then
        $\mathcal{U}_{x}\subseteq{f}^{\minus{1}}(\{y_{1}\})$
        (Def.~\ref{def:Subsets}). But then:
        \begin{equation}
            \bigcup_{x\in{f}^{\minus{1}}(\{y_{1}\})}\mathcal{U}_{x}
            =f^{\minus{1}}(\{y_{1}\})
        \end{equation}
        And therefore $f^{\minus{1}}(\{y_{1}\})$ is the union of open sets and
        is therefore open. Similarly the complement is open, and thus
        $(X,\tau)$ is disconnected, a contradiction.
    \end{proof}
    \begin{theorem}
        If $(X,\tau)$ is a connected locally Euclidean topological space, then
        there is a unique $d\in\mathbb{N}$ such that for all $x\in{X}$ there
        exists an open set $\mathcal{U}\in\tau$ such that $x\in\mathcal{U}$ and
        $\mathcal{U}$ is homeomorphic to $\mathbb{R}^{d}$.
    \end{theorem}
    \begin{proof}
        For by Thm.~\ref{thm:Equiv_Def_of_Locally_Euclidean_Top_Space}, for all
        $x\in{X}$ there is an $n\in\mathbb{N}$ such that
    \end{proof}
    \begin{fdefinition}{Smooth Real-Valued Functions On $\mathbb{R}$}
                       {Smooth_Real_Valued_Functions_on_R}
        A smooth real-valued function on an open subset
        $\mathcal{U}\subseteq\mathbb{R}^{n}$ is a function
        $f:\mathcal{U}\rightarrow\mathbb{R}$ such that all mixed partial
        derivatives of all orders exist and are continuous for all
        $\mathbf{x}\in\mathcal{U}$.
    \end{fdefinition}
    $\mathbb{R}^{n}$ can be defined as the set of all functions
    $\mathbf{x}:\mathbb{Z}_{n}\rightarrow\mathbb{R}$. Given an element
    $\mathbf{x}\in\mathbb{R}^{n}$ and $k\in\mathbb{Z}_{n}$ we denote image
    of $k$ as $x_{k}=\mathbf{x}(k)$. This is called the $k^{th}$ coordinate
    of $\mathbf{x}$. The projection mapping
    $\pi_{k}:\mathbb{R}^{n}\rightarrow\mathbb{R}$ for $k\in\mathbb{Z}_{n}$ is
    the function defined by $\pi_{k}(\mathbf{x})=x_{k}$.
    \begin{ldefinition}{Smooth Euclidean Functions}{Smooth_Euclidean_Functions}
        A smooth function on a subset $\mathcal{U}\subseteq\mathbb{R}^{n}$
        to $\mathbb{R}^{m}$ is a function
        $f:\mathcal{U}\rightarrow\mathbb{R}^{m}$ such that, for all
        $k\in\mathbb{Z}_{m}$, the function $\pi_{k}\circ{f}$ is a smooth
        real-valued function.
    \end{ldefinition}
    \begin{fdefinition}{Chart}{Chart}
        A chart of dimension $n\in\mathbb{N}$ in a topological space
        $(X,\tau)$, denoted $(\mathcal{U},\mathcal{V},\phi)$, is an open set
        $\mathcal{U}\in\tau$, an open set $\mathcal{V}\in\tau_{\mathbb{R}^{n}}$,
        and a function $\phi:\mathcal{U}\rightarrow\mathcal{V}$ such that $\phi$
        is a homeomorphism between $\mathcal{U}$ and $\mathcal{V}$.
    \end{fdefinition}
    Many authors simply denote a chart by $\phi$ and others denote them
    by $(\mathcal{U},\phi)$. We've adopted the convention of including all of
    the data in our notation: The domain $\mathcal{U}$, the range $\mathcal{V}$,
    and a homemorphism $\phi$. Note that since the image of $\phi$ lies in
    $\mathbb{R}^{n}$, if we compose with one of the projection mappings
    $\pi_{k}$ we obtain a function
    $\pi_{k}\circ\phi:\mathcal{U}\rightarrow\mathbb{R}$. These are called the
    coordinate functions of the chart $(\mathcal{U},\mathcal{V},\phi)$ and are
    often denoted $x^{k}=\pi_{k}\circ\phi$. Note that we may then write the
    image of $p\in\mathcal{U}$ as:
    \begin{equation}
        \phi(p)=(x^{1}(p),\dots,x^{n}{p})
    \end{equation}
    and such notation is often useful. Avoiding the \textit{dot dot dot}
    notation, we can recall that $\mathbf{x}\in\mathbb{R}^{n}$ is a function
    $\mathbf{x}:\mathbb{Z}_{n}\rightarrow\mathbb{R}$. Thus if $p\in\mathcal{U}$
    and $\mathbf{x}=\phi(p)$, we can write this explicitly by:
    \begin{equation}
        \big(\phi(p)\big)(k)=\mathbf{x}(k)=x_{k}=x^{k}(p)
    \end{equation}
    Thus distinguishing the difference between the notation $x^{k}$ and $x_{k}$.
    Here, $x_{k}$ is simply a real number, $x^{k}$ is a function
    $x^{k}:\mathcal{U}\rightarrow\mathbb{R}$, and $\mathbf{x}(k)$ is the image
    of $k\in\mathbb{Z}_{n}$ under $\mathbf{x}$. In other words, it is the
    $k^{th}$ component of $\mathbf{x}$.
    \begin{fdefinition}{Topological Semi-Manifold}{Topological_Semi_Manifold}
        A topological semi-manifold of dimension $n\in\mathbb{N}$ is a
        topological space $(X,\tau)$ such that for all $x\in{X}$ there exists a
        chart $(\mathcal{U},\mathcal{V},\phi)$ of dimension $n\in\mathbb{N}$
        such that $x\in\mathcal{U}$.
    \end{fdefinition}
    That is to say, a topological space $(X,\tau)$ is a semi-manifold if every
    point in $X$ is \textit{locally Euclidean}. The reason this is called a
    semi-manifold is because these objects do not have all of the structure one
    would like when considering true manifolds, such as two dimensional surfaces
    or one dimensional curves. In particular, being locally Euclidean is not
    sufficient to prove that $(X,\tau)$ is Hausdorff, and indeed there are
    non-Hausdorff semi-manifolds.
    \begin{example}
        Let $X$ be the disjoint union of $\mathbb{R}$ with itself:
        \begin{equation}
            X=\mathbb{R}\sqcup\mathbb{R}
        \end{equation}
        That is, $X=(\mathbb{R}\times\{0\})\cup(\mathbb{R}\times\{1\})$.
        Furthermore, consider the relation $R'\subseteq{X}\times{X}$ defined by:
        \begin{equation}
            R'=\Big\{\,\Big((x,a),(x,b)\Big)\;\Big|\;
                x\in\mathbb{R}\setminus\{0\}
                \textrm{ and }a,b\in\mathbb{Z}_{2}\,\Big\}
        \end{equation}
        That is, $(x,a)R'(y,b)$ if and only if $a,b\in\{0,1\}$, and $x=y$ are
        non-zero. From this we can obtain an equivalence relation $R$ as
        follows:
        \begin{equation}
            R=R'\cup\big\{\big((0,0),(0,0)\big\}\cup
                \big\{\big((0,1),(0,1)\big\}
        \end{equation}
        That is, $R$ is the reflexive closure of $R'$. This is an equivalence
        relation. Let $Y$ be the topological space $(X/R,\tau_{R})$, where
        $\tau_{R}$ is the quotient topology. This space is then locally
        Euclidean. For if $x\ne{0}$, let $r=|x|/2$ Let $\mathcal{U}$ be defined
        by:
        \begin{equation}
            \mathcal{U}=\big\{[(y,0)]\;|\;|y-x|<r\big\}
        \end{equation}
        Then $\mathcal{U}$ is open since for all points $z\in\mathcal{U}$,
        either $z=(y,0)$ or $z=(y,1)$ where $|x-y|<r$. By the triangle
        inequality we can then conclude that $y\ne{0}$. But then:
        \begin{equation}
            \bigcup\mathcal{U}=\Big(B_{r}^{\mathbb{R}}(x)\times\{0\}\Big)
                \bigcup\Big(B_{r}^{\mathbb{R}}(x)\times\{1\}\Big)
        \end{equation}
        And this is an open subset of $X$ with the disjoint union topology, and
        therefore $\mathcal{U}$ is open in the quotient topology. We can define
        a homemorphism between $\mathcal{U}$ and the open interval $(x-r,x+r)$
        by defining:
        \begin{equation}
            \phi\big([(y,0)]\big)=y
        \end{equation}
        Similarly, for the points $(0,0)$ and $(0,1)$ we can define
        $\mathcal{U}_{0}=Y\setminus\{[(0,1)]\}$ and
        $\mathcal{U}_{1}=Y\setminus\{[0,0]\}$, both of which are homeomorphic
        to all of $\mathbb{R}$. Thus this space, which is called the
        \textit{bug-eyed line}\index{Bug-Eyed Line}, or the
        \textit{line with two origins}\index{Line with Two Origins}, is locally
        Euclidean but is not Hausdorff since every open set containing
        $(0,0)$ also contains $(0,1)$, and vice-versa.
    \end{example}
    \begin{figure}[H]
        \centering
        \captionsetup{type=figure}
        \begin{tikzpicture}[>=Latex]
    % Draw axes for the upper and lower lines.
    \draw[<->] (-5, 0) to (5, 0);
    \draw[<->] (-5, 2) to (5, 2);

    % Connect the dots for all points except for the two origins.
    \foreach\x in {-4.7, -4.5, ..., -0.3, 0.3, 0.5, ..., 4.7}{
        \draw[<->] (\x, 2) to (\x, 0);
    }

    % Fill in circles for the two origins.
    \draw[fill=black] (0, 0) circle (0.5mm);
    \draw[fill=black] (0, 2) circle (0.5mm);

    % Draw a solid arrow to the next graphic.
    \draw [->, line width=1pt] (0, -0.5) to (0, -1.5);

    \begin{scope}[yshift=-2.5cm]
        % Draw the x-axis in the quotient space.
        \draw[<->] (-5.0, 0.0) to (5.0, 0.0);

        % Remove the origin.
        \draw[fill=white, draw=white] (0, 0) circle (0.5mm);

        % Draw circles symbolizing the two origins.
        \draw[fill=black] (0.0,  0.5) circle (0.5mm);
        \draw[fill=black] (0.0, -0.5) circle (0.5mm);

        % Dashed arrow to the next graphic.
        \draw [->, line width=1pt, dashed] (0.0, -1.0) to (0.0, -2.0);
    \end{scope}

    \begin{scope}[yshift=-5.5cm]
        % Draw the x-axis.
        \draw[<-] (-5.0, 0.0) to (-0.5, 0);
        \draw[->] ( 0.5, 0.0) to ( 5.0, 0);

        % Draw a curved line connecting the x-axis to the upper origin.
        \draw (-0.5, 0.0) to[out=0,in=180] (0.0, -0.5)
                          to[out=0,in=180] (0.5,  0.0);

        % Draw a curved line connecting the x-axis to the lower origin.
        \draw (-0.5, 0.0) to[out=0,in=180] (0.0, 0.5)
                          to[out=0,in=180] (0.5, 0.0);

        % Fill in the two origins.
        \draw[fill=black] (0.0,  0.5) circle (0.5mm);
        \draw[fill=black] (0.0, -0.5) circle (0.5mm);
    \end{scope}
\end{tikzpicture}
        \caption{Construction of the Bug-Eyed Line}
        \label{fig:Bug_Eyed_Line}
    \end{figure}
    The bug-eyed line\index{Bug-Eyed Line} is shown in
    Fig.~\ref{fig:Bug_Eyed_Line}. The dashed line is used to denote that
    this final sketch is approximately what the manifold \textit{looks} like.
    We can use this figure to show that any open set containing the upper origin
    must also contain the lower one
    (see Fig.~\ref{fig:Open_Neighborhoods_of_Origins_in_Bug_Eyed_Line}).
    \begin{figure}[H]
        \centering
        \captionsetup{type=figure}
        \begin{tikzpicture}[>=Latex]
    % Draw the x-axis.
    \draw[<-, thick] (-5.0, 0.0) to (-0.5, 0.0);
    \draw[->, thick] ( 0.5, 0.0) to ( 5.0, 0.0);

    % Shade an interval around the bottom origin.
    \draw[blue] (-0.5, 0.0) to[out=0,in=180] (0.0, -0.5)
                            to[out=0,in=180] (0.5,  0.0);

    % Connect the x-axis to the top origin.
    \draw (-0.5, 0.0) to[out=0,in=180] (0.0, 0.5)
                      to[out=0,in=180] (0.5, 0.0);

    % Shade the lower origin black, indicating it is not in the set.
    \draw[fill=black] (0.0,  0.5) circle (0.5mm);

    % Shade the upper origin blue, indicating is belongs to the set.
    \draw[fill=blue]  (0.0, -0.5) circle (0.5mm);

    % Label for the set U_0.
    \node at (2, 0.5) {$\mathcal{U}_{0}$};

    \begin{scope}[yshift=-2cm]
        % Draw the x-axis.
        \draw[<-, thick] (-5.0, 0) to (-0.5, 0);
        \draw[->, thick] ( 0.5, 0) to ( 5.0, 0);

        % Connect the lower origin to the x-axis.
        \draw (-0.5, 0.0) to[out=0,in=180] (0.0, -0.5)
                          to[out=0,in=180] (0.5,  0.0);

        % Shade an interval around the upper origin.
        \draw[blue] (-0.5, 0.0) to[out=0,in=180] (0.0, 0.5)
                                to[out=0,in=180] (0.5, 0.0);

        % Color the two origins accordingly.
        \draw[fill=blue]  (0.0,  0.5) circle (0.5mm);
        \draw[fill=black] (0.0, -0.5) circle (0.5mm);

        % Label for the set U_1.
        \node at (2, 0.5) {$\mathcal{U}_{1}$};
    \end{scope}

    \begin{scope}[yshift=-4cm]
        % Draw the x-axis.
        \draw[<-, thick] (-5.0, 0) to (-0.5, 0);
        \draw[->, thick] ( 0.5, 0) to ( 5.0, 0);

        % Blue coloring for the intersection of U_0 and U_1.
        \draw[blue, line width=2pt] (-0.5, 0) to (0.5, 0);

        % Delete the origin from this line.
        \draw[fill=white, draw=white] (0, 0) circle (0.5mm);

        % Color the two origins black.
        \draw[fill=black] (0,  0.5) circle (0.5mm);
        \draw[fill=black] (0, -0.5) circle (0.5mm);

        % Label for the intersection of U_0 and U_1.
        \node at (2, 0.5) {$\mathcal{U}_{0}\cap\mathcal{U}_{1}$};
    \end{scope}
\end{tikzpicture}
        \caption{Open Subsets of the Bug-Eyed Line}
        \label{fig:Open_Neighborhoods_of_Origins_in_Bug_Eyed_Line}
    \end{figure}
    Manifolds are also required to have second-countable topologies to avoid
    extremely large spaces such as the long line\index{Long Line}. The long line
    is a Hausdorf semi-manifold, but it is not second-countable since any bases
    has the same cardinality as $\mathbb{R}$, which is uncountable. To avoid the
    bug-eyed line and the long line, we consider topological manifolds which are
    semi-manifolds that are both Hausdorff and second-countable.
    \begin{fdefinition}{Topological Manifold}{Topological_Manifold}
        A topological manifold of dimension $n\in\mathbb{N}$ is a semi-manifold
        $(X,\tau)$ that is Hausdorff and second countable. That is, a
        Hausdorff second-countable topological space $(X,\tau)$ such that for
        all $x\in{X}$ there exists a chart $(\mathcal{U},\mathcal{V},\phi)$ such
        that $x\in\mathcal{U}$.
    \end{fdefinition}
    \begin{fdefinition}{Smoothly Overlapping Charts}
                       {Smoothly_Overlapping_Charts}
        Smoothly overlapping charts of dimension $n\in\mathbb{N}$ are
        charts $(\mathcal{U}_{1},\phi_{1})$ and
        $(\mathcal{U}_{2},\phi_{2})$ of dimension $n$ on a topological
        space $(X,\tau)$ such that
        $\phi_{1}\circ\phi_{2}^{\minus{1}}:%
         \phi_{2}(\mathcal{U}_{1}\cap\mathcal{U}_{2})%
         \rightarrow\mathbb{R}^{n}$ and
        $\phi_{2}\circ\phi_{1}^{\minus{1}}:%
         \phi_{1}(\mathcal{U}_{1}\cap\mathcal{U}_{2})%
         \rightarrow\mathbb{R}^{n}$ are smooth functions.
    \end{fdefinition}
    Given charts $(\mathcal{U}_{1},\phi_{1})$ and
    $(\mathcal{U}_{2},\phi_{2})$ in a topological space $(X,\tau)$ such
    that $\mathcal{U}_{1}$ and $\mathcal{U}_{2}$ are disjoint, we see that
    the two charts are automatically smoothly overlapping in a
    vacuous sense.
    \begin{figure}[H]
        \centering
        \captionsetup{type=figure}
        %--------------------------------Dependencies----------------------------------%
%   tikz                                                                       %
%       arrows.meta                                                            %
%-------------------------------Main Document----------------------------------%
\begin{tikzpicture}[>=Latex, line width=0.2mm]
    % Coordinates for the manifold X.
    \coordinate (X0) at (-5.0,  0.0);
    \coordinate (X1) at (-3.5, -2.5);
    \coordinate (X2) at ( 1.0, -2.0);
    \coordinate (X3) at ( 5.0,  0.0);
    \coordinate (X4) at ( 0.0,  1.0);

    % Coordinates for the subset U.
    \coordinate (U0) at (-4.0, -0.5);
    \coordinate (U1) at (-3.0, -2.0);
    \coordinate (U2) at ( 1.5, -0.5);
    \coordinate (U3) at (-0.6,  0.2);

    % Coordinates for the subset V.
    \coordinate (V0) at ( 4.0,  0.0);
    \coordinate (V1) at ( 3.0, -1.5);
    \coordinate (V2) at (-1.5, -0.5);
    \coordinate (V3) at ( 0.6,  0.2);

    % Draw the manifold X.
    \draw   (X0) to[out=-90, in=120]  (X1)
                 to[out=-60, in=-170] (X2)
                 to[out=10, in=-90]   (X3)
                 to[out=90, in=0]     (X4)
                 to[out=-180, in=90]  cycle;

    % Fill in U and V first and then outline with dashes.
    % This prevents the fill option from drawing over the outline.
    % Setting opacity makes the overlapping part mix colors as well.

    % Fill in the background of U blue.
    \draw[fill=blue, opacity=0.5, draw=none]
        (U0) to[out=-90, in=-180] (U1)
             to[out=0, in=-100]   (U2)
             to[out=80, in=0]     (U3)
             to[out=-180, in=90]  cycle;

    % Fill in the background of V red.
    \draw[fill=red, opacity=0.5, draw=none]
        (V0) to[out=-90, in=0]   (V1)
             to[out=180, in=-80] (V2)
             to[out=100, in=180] (V3)
             to[out=0, in=90]    cycle;

    % Draw dashed lines around U.
    \draw[densely dashed]
        (U0) to[out=-90, in=-180] (U1)
             to[out=0, in=-100]   (U2)
             to[out=80, in=0]     (U3)
             to[out=-180, in=90]  cycle;

    \draw[densely dashed]
        (V0) to[out=-90, in=0]   (V1)
             to[out=180, in=-80] (V2)
             to[out=100, in=180] (V3)
             to[out=0, in=90]    cycle;

    \begin{scope}[xshift=-5cm, yshift=3cm]

        % Coordinates for phi of U.
        \coordinate (P0) at (0.5, 0.5);
        \coordinate (P1) at (1.5, 0.2);
        \coordinate (P2) at (3.3, 0.8);
        \coordinate (P3) at (2.8, 2.1);
        \coordinate (P4) at (2.2, 3.6);
        \coordinate (P5) at (1.2, 2.8);

        % Coordinate for some midpoint inside U.
        \coordinate (PM) at (2.0, 1.5);

        \draw[->] (-0.5,  0.0) to ( 4.0,  0.0);
        \draw[->] ( 0.0, -0.5) to ( 0.0,  4.0);

        \draw[draw=none, fill=blue!20!white]
            (P0)    to[out=-30,  in=180]    (P1)
                    to[out=0,    in=-90]    (P2)
                    to[out=90,   in=-120]   (P3)
                    to[out=60,   in=30]     (P4)
                    to[out=-150, in=60]     (P5)
                    to[out=-120, in=150]    cycle;

        \draw[densely dashed]
            (P0)    to[out=-30,  in=180]    (P1)
                    to[out=0,    in=-90]    (P2)
                    to[out=90,   in=-120]   (P3)
                    to[out=60,   in=30]     (P4)
                    to[out=-150, in=60]     (P5)
                    to[out=-120, in=150]    cycle;

        \draw[densely dashed, fill=cyan]
            (P3)    to[out=180,  in=70]   (PM)
                    to[out=-110, in=180]  (P1)
                    to[out=0,    in=-90]  (P2)
                    to[out=90,   in=-120] cycle;

        \node at (2.00, 3.0) {$\phi(\mathcal{U})$};
        \node at (2.45, 0.8) {$\phi(\mathcal{U}\cap\mathcal{V})$};
        \node at (3.50, 3.5) {\large{$\mathbb{R}^{n}$}};
    \end{scope}

    \begin{scope}[xshift=2cm, yshift=3cm]

        % Coordinates for phi of U.
        \coordinate (Q0) at (3.5, 0.5);
        \coordinate (Q1) at (2.5, 0.2);
        \coordinate (Q2) at (0.5, 0.8);
        \coordinate (Q3) at (1.2, 2.1);
        \coordinate (Q4) at (1.8, 3.6);
        \coordinate (Q5) at (2.8, 2.8);

        % Coordinate for some midpoint inside U.
        \coordinate (QM) at (2.0, 1.5);

        \draw[->] (-0.5,  0.0) to ( 4.0,  0.0);
        \draw[->] ( 0.0, -0.5) to ( 0.0,  4.0);

        \draw[draw=none, fill=red!20!white]
            (Q0)    to[out=-150,    in=0]       (Q1)
                    to[out=-180,    in=-90]     (Q2)
                    to[out=90,      in=-120]    (Q3)
                    to[out=60,      in=150]     (Q4)
                    to[out=-30,     in=60]      (Q5)
                    to[out=-120,    in=30]      cycle;

        \draw[densely dashed]
            (Q0)    to[out=-150,    in=0]       (Q1)
                    to[out=-180,    in=-90]     (Q2)
                    to[out=90,      in=-120]    (Q3)
                    to[out=60,      in=150]     (Q4)
                    to[out=-30,     in=60]      (Q5)
                    to[out=-120,    in=30]      cycle;

        \draw[densely dashed, fill=red!50!white]
            (Q3)    to[out=-60,     in=70]      (QM)
                    to[out=-110,    in=0]       (Q1)
                    to[out=-180,    in=-90]     (Q2)
                    to[out=90,      in=-120]    cycle;

        \node at (2.00, 3.0) {$\xi(\mathcal{V})$};
        \node at (1.25, 0.8) {$\xi(\mathcal{U}\cap\mathcal{V})$};
        \node at (3.50, 3.5) {\large{$\mathbb{R}^{n}$}};
    \end{scope}

    \begin{scope}[line width=0.4mm, ->, font=\large]
        \draw (-2.0, 0.5) to[out=130, in=-100] node[left]  {$\phi$} (-3.0, 3);
        \draw ( 2.5, 0.7) to[out=50,  in=-80]  node[right] {$\xi$}  ( 3.5, 3);
        \draw (-1.5, 4.5) to[out=30, in=150]
            node[above] {$\xi\circ\phi^{\minus{1}}$} ( 1.5, 4.5);
        \draw ( 1.5, 3.5) to[out=-150, in=-30]
            node[below] {$\phi\circ\xi^{\minus{1}}$} (-1.5, 3.5);
    \end{scope}

    \node at (-4.0,  0.5) {$X$};
    \node at (-3.0, -1.5) {$\mathcal{U}$};
    \node at ( 3.0, -1.3) {$\mathcal{V}$};
    \node at ( 0.0, -0.5) {$\mathcal{U}\cap\mathcal{V}$};
\end{tikzpicture}
        \caption{Smoothly Overlapping Charts}
        \label{fig:Smoothly_Overlapping_Charts}
    \end{figure}
    \begin{fdefinition}{Atlas}{Atlas}
        An atlas on a topological space $(X,\tau)$ is a set of charts
        $\mathcal{A}$ on $(X,\tau)$ such that, for all $x\in{X}$, there
        is a $(\mathcal{U},\phi)\in\mathcal{O}$ such that $x\in\mathcal{U}$.
    \end{fdefinition}
    \begin{ldefinition}{$n$ Dimensional Atlas}{n_Dimensional_Atlas}
        An atlas of dimension $n\in\mathbb{N}$ on a topological space
        $(X,\tau)$ is an atlas $\mathcal{A}$ on $(X,\tau)$ such that, for
        all $(\mathcal{U},\phi)\in\mathcal{A}$, $(\mathcal{U},\phi)$ is
        a chart of dimension $n$.
    \end{ldefinition}
    \begin{fdefinition}{Transition Function}{Transition_Function}
        The transition function of a chart $(\mathcal{U},\phi)$ with
        respect a chart $(\mathcal{V},\xi)$ on a topological space
        $(X,\tau)$ is the function
        $f:\phi(\mathcal{U}\cap\mathcal{V})%
         \rightarrow\xi(\mathcal{V}\cap\mathcal{V})$ defined by:
        \begin{equation}
            f(x)=(\xi\circ\phi^{\minus{1}})(x)
        \end{equation}
    \end{fdefinition}
    \begin{fdefinition}{Smooth Atlas}{Smooth Atlas}
        A smooth atlas on a topological space $(X,\tau)$ is an
        atlas $\mathcal{A}$ such that for all charts
        $(\mathcal{U},\phi),(\mathcal{V},\xi)\in\mathcal{A}$, the
        transition function of $(\mathcal{U},\phi)$ with respect to
        $(\mathcal{V},\xi)$ is smooth.
    \end{fdefinition}
    \begin{fdefinition}{Maximal Smooth Atlas}{Maximal_Smooth_Atlas}
        A complete atlas on a topological space $(X,\tau)$ is a smooth
        atlas $\mathcal{A}$ on $(X,\tau)$ such that, for all charts
        $(\mathcal{U},\phi)$ of $(X,\tau)$ such that $(\mathcal{U},\phi)$
        overlaps smoothly with all $(\mathcal{V},\xi)\in\mathcal{A}$, it
        is true that $(\mathcal{U},\phi)\in\mathcal{A}$.
    \end{fdefinition}
    \begin{theorem}
        If $(X,\tau)$ is a topological space and if $\mathcal{A}$ is a
        smooth atlas of dimension $n\in\mathbb{N}$ on a topological space
        $(X,\tau)$, then there is a unique maximal unique atlas
        $\mathcal{C}$ on $(X,\tau)$ such that
        $\mathcal{A}\subseteq\mathcal{C}$.
    \end{theorem}
    \begin{proof}
        For let $\mathcal{C}$ be the set of all charts on $(X,\tau)$ that
        overlap smoothly with the charts in $\mathcal{A}$. Then since
        $\mathcal{A}$ is an atlas, for all
        $(\mathcal{U},\phi)\in\mathcal{A}$ and for all
        $(\mathcal{V},\xi)\in\mathcal{A}$, we have that $(\mathcal{U},\phi)$
        and $(\mathcal{V},\xi)$ overlap smoothly, and thus
        $(\mathcal{U},\phi)\in\mathcal{C}$. Therefore
        $\mathcal{A}\subseteq\mathcal{C}$. But $\mathcal{A}$ is an
        atlas and thus for all $x\in{X}$ there is a chart
        $(\mathcal{U},\phi)\in\mathcal{A}$ such that $x\in\mathcal{U}$.
        But $\mathcal{A}\subseteq\mathcal{C}$ and thus
        $(\mathcal{U},\phi)\in\mathcal{C}$. Thus, for all $x\in{X}$ there
        is a chart $(\mathcal{U},\phi)\in\mathcal{C}$ such that
        $x\in\mathcal{U}$. Suppose
        $(\mathcal{U}_{1},\phi_{1}),%
         (\mathcal{U}_{2},\phi_{2})\in\mathcal{C}$.
        If $\mathcal{U}_{1}$ and $\mathcal{U}_{2}$ are disjoint, then
        these two charts overlap smoothly. Suppose it is non-empty and let
        $f$ be the transition function of $(\mathcal{U}_{1},\phi_{1})$ with
        respect to $(\mathcal{U}_{2},\phi_{2})$. Let
        $p\in\phi_{1}(\mathcal{U}_{1}\cap\mathcal{U}_{2})$. But then there
        is a chart $\xi\in\mathcal{A}$ such that $\phi_{2}^{\minus{1}}(p)$
        is contained in the domain of $\xi$. From the associativity of
        composition, we have:
        \begin{equation}
            \phi_{1}\circ\phi_{2}^{\minus{1}}
            =(\phi_{1}\circ\xi^{\minus{1}})\circ
             (\xi\circ\phi_{2}^{\minus{1}})
        \end{equation}
        But by the definition of $\mathcal{C}$, $\phi_{1}$ and $\phi_{2}$
        overlap smoothly with $\xi$, and thus this is the composition of
        smooth functions, and is therefore smooth. Therefore
        $\phi_{1}\circ\phi_{2}^{\minus{1}}$ is smooth and thus
        $(\mathcal{U}_{1},\phi_{1})$ and $(\mathcal{U}_{2},\phi_{2})$
        overlap smoothly. Thus, $\mathcal{C}$ is a smooth atlas. Moreover,
        it is complete from the construction. Given any other complete
        atlas $\mathcal{C}'$ that contains $\mathcal{A}$ we would have
        $\mathcal{C}\subseteq\mathcal{C}'$ and
        $\mathcal{C}'\subseteq\mathcal{C}$, and therefore
        $\mathcal{C}=\mathcal{C}'$. Thus, this completion is unique.
    \end{proof}
    \begin{fdefinition}{Smooth Manifold}{Smooth_Manifold}
        A smooth manifold of dimension $n\in\mathbb{N}$, denoted
        $(X,\tau,\mathcal{A})$ is a Hausdorff topological space
        $(X,\tau)$ with a maximal smooth atlas $\mathcal{A}$ of
        dimension $n$.
    \end{fdefinition}
    Any smooth atlas $\mathcal{A}$ on a topological space $(X,\tau)$
    defines a a smooth manifold if we let $\mathcal{C}$ be the maximal
    smooth atlas generated by $\mathcal{A}$.
    \begin{example}
        Let $(\mathbb{R}^{n},\tau_{\mathbb{R}^{n}})$ be the standard
        $n$ dimensional Euclidean space. We can define a trivial smooth
        atlas on this space by let
        $\mathcal{A}=\{(\mathbb{R}^{n},\textrm{id})\}$, where $\textrm{id}$
        is the identity function. This defines a smooth atlas. By
        considering the unique maximal smooth atlas generated by this
        we obtain the standard smooth structure on $\mathbb{R}^{n}$.
    \end{example}
    \begin{theorem}
        If $(X,\tau,\mathcal{A})$ is a smooth manifold of dimension
        $n\in\mathbb{N}$, $(\mathcal{U},\phi)\in\mathcal{A}$,
        $\mathcal{V}\in\tau$, and if $\phi_{\mathcal{V}}$ denotes
        the restriction mapping:
        $\phi_{\mathcal{V}}:\mathcal{V}\rightarrow\mathbb{R}^{n}$,
        then $(\mathcal{V},\phi_{\mathcal{V}})\in\mathcal{A}$.
    \end{theorem}
    \begin{proof}
        For since $\phi$ is a homeomorphism from $\mathcal{U}$ to
        $\phi(\mathcal{U})$, and since $\mathcal{V}\in\tau$, we have
        that $\phi_{\mathcal{V}}$ is a homeomorphism between
        $\mathcal{V}$ and $\phi_{\mathcal{V}}(\mathcal{V})$, and
        therefore $(\mathcal{V},\phi_{\mathcal{V}})$ is a chart. But
        this chart meets $(\mathcal{U},\phi)$ smoothly, and $\mathcal{A}$
        is complete. Thus,
        $(\mathcal{V},\phi_{\mathcal{A}})\in\mathcal{A}$.
    \end{proof}
    \begin{theorem}
        If $n\in\mathbb{N}$, then there is a complete atlas
        $\mathcal{A}$ on $(S^{n},\tau)$, where $\tau$ is the inherited
        topology from $\mathbb{R}^{n+1}$.
    \end{theorem}
    \begin{proof}
        For all $k\in\mathbb{Z}_{n+1}$, let $\mathcal{U}_{k}^{+}$ and
        $\mathcal{U}_{k}^{\minus}$ be defined as:
        \par\hfill\par
        \begin{minipage}[b]{0.49\textwidth}
            \begin{equation}
                \mathcal{U}_{k}^{+}
                =\{\,\mathbf{x}\in{S}^{n}\,:\,x_{k}>0\,\}
            \end{equation}
        \end{minipage}
        \hfill
        \begin{minipage}[b]{0.49\textwidth}
            \begin{equation}
                \mathcal{U}_{k}^{\minus}
                =\{\,\mathbf{x}\in{S}^{n}\,:\,x_{k}<0\,\}
            \end{equation}
        \end{minipage}
        \par\vspace{2.5ex}
        Define $\phi_{\mathcal{U}_{k}^{+}}:\mathcal{U}_{k}^{+}%
                \rightarrow\mathbb{R}^{n}$ by:
        \begin{equation}
            \phi_{\mathcal{U}_{k}^{+}}(\mathbf{x})
            =(x_{1},\,\dots,\,x_{k-1},\,x_{k+1},\,\dots,\,x_{n+1})
        \end{equation}
        That is, the mapping that projects the point onto the plane
        defined by $x_{k}=0$. Define $\phi_{\mathcal{U}_{k}^{\minus}}$
        similarly. Then all such $\phi$ are homeomorphisms from their
        domain to their image. Let $\mathcal{A}$ be defined as follows:
        \begin{equation}
            \mathcal{A}
            =\big\{\,(\mathcal{U}_{k}^{+},\,\phi_{\mathcal{U}_{k}^{+}})
                   \,:\,k\in\mathbb{Z}_{n}\big\}\bigcup
            \big\{\,(\mathcal{U}_{k}^{-},\,\phi_{\mathcal{U}_{k}^{-}})
                   \,:\,k\in\mathbb{Z}_{n}\big\}
        \end{equation}
        Then $\mathcal{A}$ is an atlas of $(S^{n},\tau)$. For if
        $\mathbf{x}\in{S}^{n}$, then $\norm{\mathbf{x}}_{2}=1$. But then
        there is a coordinate $x_{k}$ of $\mathbf{x}$ such that
        $x_{k}\ne{0}$. But then either $x_{k}>0$ or $x_{k}<0$, and thus
        either $\mathbf{x}\in\mathcal{U}_{k}^{+}$ or
        $\mathbf{x}\in\mathcal{U}_{k}^{\minus}$. If
        $(\mathcal{V}_{2},\phi_{2})$ and $(\mathcal{V}_{2},\phi_{2})$
        are charts, then either $\phi_{1}(\mathcal{V}_{1})$
        and $\phi_{2}(\mathcal{V}_{2})$ are disjoint or they are not.
        If they are disjoint, then $\phi_{1}$ and $\phi_{2}$ overlap
        smoothly. If they are not disjoint, let $\mathbf{x}$ be contained
        in the intersection. But then, for all $k\in\mathbb{Z}_{n}$,
        $\pi_{k}\circ(\phi_{1}\circ\phi_{2}^{\minus{1}})$ is smooth,
        and thus $\phi_{1}$ and $\phi_{2}$ overlap smoothly.
    \end{proof}
    \begin{fdefinition}{Open Submanifold}{Open_Submanifold}
        An open submanifold on a manifold $(X,\tau,\mathcal{A})$ is a
        an open subset $\mathcal{U}\subseteq{X}$ and the collection
        $\mathcal{A}_{\mathcal{U}}$ defined by:
        \begin{equation}
            \mathcal{A}_{\mathcal{U}}
            =\{\,(\mathcal{V},\phi)\in\mathcal{A}\,:\,
                 \mathcal{V}\subseteq\mathcal{U}\,\}
        \end{equation}
        Together with the inherited topology $\tau_{\mathcal{U}}$.
    \end{fdefinition}
    \begin{theorem}
        If $(X,\tau,\mathcal{A})$ is a smooth manifold and if
        $(\mathcal{U},\tau_{\mathcal{U}},\mathcal{A}_{\mathcal{U}})$
        is an open submanifold, then it is a smooth manifold.
    \end{theorem}
    \begin{proof}
        For by the previous theorem, $\mathcal{A}_{\mathcal{U}}$ is a
        complete atlas. Moreover, a subspace of a Hausdorff topological
        space is also a Hausdorff topological space, and hence
        $(\mathcal{U},\tau_\mathcal{U})$ is a Hausdorff space. Thus,
        $(\mathcal{U},\tau_\mathcal{U},\mathcal{A}_{\mathcal{U}})$ is
        a smooth manifold.
    \end{proof}
    \begin{fdefinition}{Product Chart}{Product_Chart}
        The product chart of an $n$ dimensional chart $(\mathcal{U},\phi)$
        on a topological space $(X,\tau_{X})$ with an $m$ dimensional chart
        $(\mathcal{V},\xi)$ on a topological space $(Y,\tau_{Y})$ is
        the ordered pair $(\mathcal{U}\times\mathcal{V},f)$ where
        $f:\mathcal{U}\times\mathcal{V}\rightarrow\mathbb{R}^{n+m}$
        defined by:
        \begin{equation}
            f(p,q)_{k}=
            \begin{cases}
                \phi(p)_{k},&k<n\\
                \xi(q)_{k},&n\leq{k}<n+m
            \end{cases}
        \end{equation}
        Where $\phi(p)_{k}$ is the $k^{th}$ coordinate of
        $\phi(p)\in\mathbb{R}^{n}$ and $\xi(q)_{k}$ is the $k^{th}$
        coordinate of $\xi(q)\in\mathbb{R}^{m}$. We denote this by
        $(\mathcal{U},\phi)\times(\mathcal{V},\xi)$.
    \end{fdefinition}
    Thinking of the elements of $\mathbb{R}^{n+m}$ as tuples of length
    $n+m$, we can write:
    \begin{equation}
        f(p,q)=\big(x_{1}(p),\dots,x_{n}(p),y_{1}(q),\dots,y_{m}(q)\big)
    \end{equation}
    \begin{theorem}
        If $(X,\tau_{X},\mathcal{A}_{X})$ and $(Y,\tau_{Y},\mathcal{A}_{Y})$
        are smooth manifolds, and if $\mathcal{A}$ is the set of all
        product charts on $X\times{Y}$, then $\mathcal{A}$ is a smooth
        atlas on $(X\times{Y},\tau_{X\times{Y}})$, where
        $\tau_{X\times{Y}}$ is the product topology.
    \end{theorem}
    \begin{proof}
        For if $p\in{X}\times{Y}$ then there is an $x\in{X}$ and a
        $y\in{Y}$ such that $p=(x,y)$. But $\mathcal{A}_{X}$ is a smooth
        atlas on $(X,\tau_{X})$, and thus if $x\in{X}$ then there is a
        $(\mathcal{U},\phi)\in\mathcal{A}_{X}$ such that $x\in\mathcal{U}$.
        Similarly, there is a $(\mathcal{V},\xi)\in\mathcal{A}_{Y}$ such
        that $y\in\mathcal{V}$. But then $p\in\mathcal{U}\times\mathcal{V}$,
        and $\mathcal{U}\times\mathcal{V}\in\tau_{X\times{Y}}$. But if
        $\phi:\mathcal{U}\rightarrow\mathbb{R}^{n}$ is a homeomorphism
        between $\mathcal{U}$ and $\phi(\mathcal{U})$ and
        $\xi:\mathcal{V}\rightarrow\mathbb{R}^{m}$ is a homemorphism
        between $\mathcal{V}$ and $\xi(\mathcal{V})$, then
        $f:\mathcal{U}\times\mathcal{V}\rightarrow\mathbb{R}^{n+m}$ is
        a homeomorphism between $\mathcal{U}\times\mathcal{V}$ and
        $f(\mathcal{U}\times\mathcal{V})$, and thus the product chart
        is a chart in $(X\times{Y},\tau_{X\times{Y}})$. Moreover, all of
        the elements of $\mathcal{A}$ are smoothly overlapping. Thus,
        $\mathcal{A}$ is an atlas on $(X\times{Y},\tau_{X\times{Y}})$.
    \end{proof}
    Using the maximal smooth atlas generated by the product atlas
    $\mathcal{A}$ creates the product manifold.
    \subsection{Smooth Mappings}
        \begin{fdefinition}{Smooth Real-Valued Functions}
                           {Smooth_Real_Valued_Functions}
            A smooth real-valued function on a manifold
            $(X,\tau,\mathcal{A})$ of dimension $n\in\mathbb{N}$ is a
            function $f:X\rightarrow\mathbb{R}$ such that, for every chart
            $(\mathcal{U},\phi)\in\mathcal{A}$, the function
            $f\circ\phi^{\minus{1}}:\phi(\mathcal{U})\rightarrow\mathbb{R}$
            is a smooth Euclidean function.
        \end{fdefinition}
        \begin{theorem}
            If $(X,\tau,\mathcal{A})$ is a manifold and if
            $f,g:X\rightarrow\mathbb{R}$ are smooth real-valued functions,
            then $(f+g):X\rightarrow\mathbb{R}$ defined by:
            \begin{equation}
                (f+g)(x)=f(x)+g(x)
                \quad\quad
                x\in{X}
            \end{equation}
            Is a smooth real-valued function.
        \end{theorem}
        \begin{theorem}
            If $(X,\tau,\mathcal{A})$ is a manifold and if
            $f,g:X\rightarrow\mathbb{R}$ are smooth real-valued functions,
            then $(f\cdot{g}):X\rightarrow\mathbb{R}$ defined by:
            \begin{equation}
                (f\cdot{g})(x)=f(x)\cdot{g}(x)
                \quad\quad
                x\in{X}
            \end{equation}
            Is a smooth real-valued function.
        \end{theorem}
        \begin{fdefinition}{Smooth Functions Between Manifolds}
                           {Smooth Functions Between Manifolds}
            A smooth function from a manifold $(X,\tau_{X},\mathcal{A}_{X})$
            of dimension $n\in\mathbb{N}$ to a manifold
            $(Y,\tau_{Y},\mathcal{A}_{Y})$ of dimension $m\in\mathbb{N}$ is
            a function $f:X\rightarrow{Y}$ such that, for every chart
            $(\mathcal{U},\phi)\in\mathcal{A}_{X}$ and for every chart
            $(\mathcal{V},\xi)\in\mathcal{A}_{Y}$, the function
            $\xi\circ{f}\circ\phi^{\minus{1}}:\xi(\mathcal{V})%
             \rightarrow\mathbb{R}^{m}$ is a smooth Euclidean function.
        \end{fdefinition}
        \begin{theorem}
            If $(X,\tau_{X},\mathcal{A}_{X})$ and
            $(Y,\tau_{Y},\mathcal{A}_{Y})$ are manifolds, if
            $A_{X}\subseteq\mathcal{A}_{X}$ is an atlas on $(X,\tau_{X})$,
            if $A_{Y}\subseteq\mathcal{A}_{Y}$ is an atlas on
            $(Y,\tau_{Y})$, and if $f:X\rightarrow{Y}$ is a function such
            that, for all $(\mathcal{U},\phi)\in{A}_{X}$ and for all
            $(\mathcal{Y},\xi)\in{A}_{y}$ it is true that
            $\xi\circ{f}\circ\phi^{\minus{1}}:\xi(\mathcal{V})%
             \rightarrow\mathbb{R}^{m}$ is a smooth Euclidean function,
            then $f$ is smooth.
        \end{theorem}
        \begin{proof}
            Since charts in $\mathcal{A}_{X}$ and $\mathcal{A}_{Y}$
            overlap smoothly with charts in $A_{X}$ and $A_{Y}$, and since
            the atlases $A_{X}$ and $A_{Y}$ cover $X$ and $Y$, respectively,
            we are done.
        \end{proof}
        \begin{theorem}
            If $(X,\tau_{X},\mathcal{A}_{X})$ is a manifold, then
            $\textrm{id}:X\rightarrow{X}$ is a smooth function.
        \end{theorem}
        \begin{theorem}
            If $(X,\tau_{X},\mathcal{A}_{X})$,
            $(Y,\tau_{Y},\mathcal{A}_{Y})$, and
            $(Z,\tau_{Z},\mathcal{A}_{Z})$ are manifolds, if
            $f:X\rightarrow{Y}$ and $g:Y\rightarrow{Z}$ are smooth, then
            $g\circ{f}:X\rightarrow{Z}$ is smooth.
        \end{theorem}
        Smoothness is a local property. A function $\phi:M\rightarrow{N}$
        is smooth at $p\in{M}$ if there is a neighborhood
        $\mathcal{U}$ of $p$ such that the restriction of $\phi$ to
        $\mathcal{U}$ is smooth. A smooth function is thus a function
        that is smooth at every point.
        \begin{theorem}
            If $(X,\mathcal{A}_{X},\tau_{X})$ and
            $(Y,\mathcal{A}_{Y},\tau_{Y})$ are manifolds and if
            $f:X\rightarrow{Y}$ is smooth, then $f$ is continuous.
        \end{theorem}
        \begin{fdefinition}{Diffeomorphism}{Diffeomorphism}
            A diffeomorphism from a manifold $(X,\tau_{X},\mathcal{A}_{X})$
            to a manifold $(Y,\tau_{Y},\mathcal{A}_{Y})$ is a bijective
            function $f:X\rightarrow{Y}$ such that $f$ and $f^{\minus{1}}$
            are smooth.
        \end{fdefinition}
        \begin{lexample}
            For any $a,b\in\mathbb{R}$ with $a<b$, the interval
            $(a,b)$ is diffeomorphic to the unit interval $(0,1)$. For
            let $\phi:(0,1)\rightarrow(a,b)$ be defined by:
            \begin{equation}
                \phi(t)=(a-b)t+b
            \end{equation}
            Then $\phi$ is a smooth bijection and it's inverse is smooth.
            Moreover, the unit interval is diffeomorphic to $\mathbb{R}$.
            For let $\xi:(0,1)\rightarrow\mathbb{R}$ be defined by:
            \begin{equation}
                \xi(t)=\frac{2t}{t(1-t)}
            \end{equation}
        \end{lexample}
        \begin{theorem}
            If $(X,\tau_{X},\mathcal{A}_{X})$ and
            $(Y,\tau_{Y},\mathcal{A}_{Y})$ are manifolds, and if
            $f:X\rightarrow{Y}$ is a diffeomorphism, then $f$ is a
            homeomorphism from $(X,\tau_{X})$ to $(Y,\tau_{Y})$.
        \end{theorem}
        \begin{proof}
            For if $f$ is a diffeomorphism, then it is a smooth bijection
            such that it's inverse is smooth. But if $f$ is smooth, then
            it is continuous and therefore it is a continuous bijection.
            But if $f^{\minus{1}}$ is smooth, then it is continuous, and
            thus $f$ is a bicontinuous bijective function, and is therefore
            a homeomorphism.
        \end{proof}
        A smooth homeomorphism need not be a diffeomorphism. The inverse
        function may not be smooth. For let
        $f:\mathbb{R}\rightarrow\mathbb{R}$ be defined by $f(x)=x^{3}$.
        Then $f$ is a homeomorphism and it's forward direction is smooth,
        but $f^{\minus{1}}$ is not smooth at the origin.
        \begin{ftheorem}{}{}
            If $A$ is a set, if $(X,\tau,\mathcal{A})$ is a manifold, and
            if $f:A\rightarrow{X}$ is an bijective function, then there
            exists a topology $\tau_{A}$ and an atlas $\mathcal{A}_{A}$
            on $X$ such that $f$ is a diffeomorphism.
        \end{ftheorem}