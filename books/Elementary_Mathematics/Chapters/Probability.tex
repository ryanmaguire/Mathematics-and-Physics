\documentclass[crop=false,class=book,oneside]{standalone}                      %
%----------------------------------Preamble------------------------------------%
%---------------------------Packages----------------------------%
\usepackage{geometry}
\geometry{b5paper, margin=1.0in}
\usepackage[T1]{fontenc}
\usepackage{graphicx, float}            % Graphics/Images.
\usepackage{natbib}                     % For bibliographies.
\bibliographystyle{agsm}                % Bibliography style.
\usepackage[french, english]{babel}     % Language typesetting.
\usepackage[dvipsnames]{xcolor}         % Color names.
\usepackage{listings}                   % Verbatim-Like Tools.
\usepackage{mathtools, esint, mathrsfs} % amsmath and integrals.
\usepackage{amsthm, amsfonts, amssymb}  % Fonts and theorems.
\usepackage{tcolorbox}                  % Frames around theorems.
\usepackage{upgreek}                    % Non-Italic Greek.
\usepackage{fmtcount, etoolbox}         % For the \book{} command.
\usepackage[newparttoc]{titlesec}       % Formatting chapter, etc.
\usepackage{titletoc}                   % Allows \book in toc.
\usepackage[nottoc]{tocbibind}          % Bibliography in toc.
\usepackage[titles]{tocloft}            % ToC formatting.
\usepackage{pgfplots, tikz}             % Drawing/graphing tools.
\usepackage{imakeidx}                   % Used for index.
\usetikzlibrary{
    calc,                   % Calculating right angles and more.
    angles,                 % Drawing angles within triangles.
    arrows.meta,            % Latex and Stealth arrows.
    quotes,                 % Adding labels to angles.
    positioning,            % Relative positioning of nodes.
    decorations.markings,   % Adding arrows in the middle of a line.
    patterns,
    arrows
}                                       % Libraries for tikz.
\pgfplotsset{compat=1.9}                % Version of pgfplots.
\usepackage[font=scriptsize,
            labelformat=simple,
            labelsep=colon]{subcaption} % Subfigure captions.
\usepackage[font={scriptsize},
            hypcap=true,
            labelsep=colon]{caption}    % Figure captions.
\usepackage[pdftex,
            pdfauthor={Ryan Maguire},
            pdftitle={Mathematics and Physics},
            pdfsubject={Mathematics, Physics, Science},
            pdfkeywords={Mathematics, Physics, Computer Science, Biology},
            pdfproducer={LaTeX},
            pdfcreator={pdflatex}]{hyperref}
\hypersetup{
    colorlinks=true,
    linkcolor=blue,
    filecolor=magenta,
    urlcolor=Cerulean,
    citecolor=SkyBlue
}                           % Colors for hyperref.
\usepackage[toc,acronym,nogroupskip,nopostdot]{glossaries}
\usepackage{glossary-mcols}
%------------------------Theorem Styles-------------------------%
\theoremstyle{plain}
\newtheorem{theorem}{Theorem}[section]

% Define theorem style for default spacing and normal font.
\newtheoremstyle{normal}
    {\topsep}               % Amount of space above the theorem.
    {\topsep}               % Amount of space below the theorem.
    {}                      % Font used for body of theorem.
    {}                      % Measure of space to indent.
    {\bfseries}             % Font of the header of the theorem.
    {}                      % Punctuation between head and body.
    {.5em}                  % Space after theorem head.
    {}

% Italic header environment.
\newtheoremstyle{thmit}{\topsep}{\topsep}{}{}{\itshape}{}{0.5em}{}

% Define environments with italic headers.
\theoremstyle{thmit}
\newtheorem*{solution}{Solution}

% Define default environments.
\theoremstyle{normal}
\newtheorem{example}{Example}[section]
\newtheorem{definition}{Definition}[section]
\newtheorem{problem}{Problem}[section]

% Define framed environment.
\tcbuselibrary{most}
\newtcbtheorem[use counter*=theorem]{ftheorem}{Theorem}{%
    before=\par\vspace{2ex},
    boxsep=0.5\topsep,
    after=\par\vspace{2ex},
    colback=green!5,
    colframe=green!35!black,
    fonttitle=\bfseries\upshape%
}{thm}

\newtcbtheorem[auto counter, number within=section]{faxiom}{Axiom}{%
    before=\par\vspace{2ex},
    boxsep=0.5\topsep,
    after=\par\vspace{2ex},
    colback=Apricot!5,
    colframe=Apricot!35!black,
    fonttitle=\bfseries\upshape%
}{ax}

\newtcbtheorem[use counter*=definition]{fdefinition}{Definition}{%
    before=\par\vspace{2ex},
    boxsep=0.5\topsep,
    after=\par\vspace{2ex},
    colback=blue!5!white,
    colframe=blue!75!black,
    fonttitle=\bfseries\upshape%
}{def}

\newtcbtheorem[use counter*=example]{fexample}{Example}{%
    before=\par\vspace{2ex},
    boxsep=0.5\topsep,
    after=\par\vspace{2ex},
    colback=red!5!white,
    colframe=red!75!black,
    fonttitle=\bfseries\upshape%
}{ex}

\newtcbtheorem[auto counter, number within=section]{fnotation}{Notation}{%
    before=\par\vspace{2ex},
    boxsep=0.5\topsep,
    after=\par\vspace{2ex},
    colback=SeaGreen!5!white,
    colframe=SeaGreen!75!black,
    fonttitle=\bfseries\upshape%
}{not}

\newtcbtheorem[use counter*=remark]{fremark}{Remark}{%
    fonttitle=\bfseries\upshape,
    colback=Goldenrod!5!white,
    colframe=Goldenrod!75!black}{ex}

\newenvironment{bproof}{\textit{Proof.}}{\hfill$\square$}
\tcolorboxenvironment{bproof}{%
    blanker,
    breakable,
    left=3mm,
    before skip=5pt,
    after skip=10pt,
    borderline west={0.6mm}{0pt}{green!80!black}
}

\AtEndEnvironment{lexample}{$\hfill\textcolor{red}{\blacksquare}$}
\newtcbtheorem[use counter*=example]{lexample}{Example}{%
    empty,
    title={Example~\theexample},
    boxed title style={%
        empty,
        size=minimal,
        toprule=2pt,
        top=0.5\topsep,
    },
    coltitle=red,
    fonttitle=\bfseries,
    parbox=false,
    boxsep=0pt,
    before=\par\vspace{2ex},
    left=0pt,
    right=0pt,
    top=3ex,
    bottom=1ex,
    before=\par\vspace{2ex},
    after=\par\vspace{2ex},
    breakable,
    pad at break*=0mm,
    vfill before first,
    overlay unbroken={%
        \draw[red, line width=2pt]
            ([yshift=-1.2ex]title.south-|frame.west) to
            ([yshift=-1.2ex]title.south-|frame.east);
        },
    overlay first={%
        \draw[red, line width=2pt]
            ([yshift=-1.2ex]title.south-|frame.west) to
            ([yshift=-1.2ex]title.south-|frame.east);
    },
}{ex}

\AtEndEnvironment{ldefinition}{$\hfill\textcolor{Blue}{\blacksquare}$}
\newtcbtheorem[use counter*=definition]{ldefinition}{Definition}{%
    empty,
    title={Definition~\thedefinition:~{#1}},
    boxed title style={%
        empty,
        size=minimal,
        toprule=2pt,
        top=0.5\topsep,
    },
    coltitle=Blue,
    fonttitle=\bfseries,
    parbox=false,
    boxsep=0pt,
    before=\par\vspace{2ex},
    left=0pt,
    right=0pt,
    top=3ex,
    bottom=0pt,
    before=\par\vspace{2ex},
    after=\par\vspace{1ex},
    breakable,
    pad at break*=0mm,
    vfill before first,
    overlay unbroken={%
        \draw[Blue, line width=2pt]
            ([yshift=-1.2ex]title.south-|frame.west) to
            ([yshift=-1.2ex]title.south-|frame.east);
        },
    overlay first={%
        \draw[Blue, line width=2pt]
            ([yshift=-1.2ex]title.south-|frame.west) to
            ([yshift=-1.2ex]title.south-|frame.east);
    },
}{def}

\AtEndEnvironment{ltheorem}{$\hfill\textcolor{Green}{\blacksquare}$}
\newtcbtheorem[use counter*=theorem]{ltheorem}{Theorem}{%
    empty,
    title={Theorem~\thetheorem:~{#1}},
    boxed title style={%
        empty,
        size=minimal,
        toprule=2pt,
        top=0.5\topsep,
    },
    coltitle=Green,
    fonttitle=\bfseries,
    parbox=false,
    boxsep=0pt,
    before=\par\vspace{2ex},
    left=0pt,
    right=0pt,
    top=3ex,
    bottom=-1.5ex,
    breakable,
    pad at break*=0mm,
    vfill before first,
    overlay unbroken={%
        \draw[Green, line width=2pt]
            ([yshift=-1.2ex]title.south-|frame.west) to
            ([yshift=-1.2ex]title.south-|frame.east);},
    overlay first={%
        \draw[Green, line width=2pt]
            ([yshift=-1.2ex]title.south-|frame.west) to
            ([yshift=-1.2ex]title.south-|frame.east);
    }
}{thm}

%--------------------Declared Math Operators--------------------%
\DeclareMathOperator{\adjoint}{adj}         % Adjoint.
\DeclareMathOperator{\Card}{Card}           % Cardinality.
\DeclareMathOperator{\curl}{curl}           % Curl.
\DeclareMathOperator{\diam}{diam}           % Diameter.
\DeclareMathOperator{\dist}{dist}           % Distance.
\DeclareMathOperator{\Div}{div}             % Divergence.
\DeclareMathOperator{\Erf}{Erf}             % Error Function.
\DeclareMathOperator{\Erfc}{Erfc}           % Complementary Error Function.
\DeclareMathOperator{\Ext}{Ext}             % Exterior.
\DeclareMathOperator{\GCD}{GCD}             % Greatest common denominator.
\DeclareMathOperator{\grad}{grad}           % Gradient
\DeclareMathOperator{\Ima}{Im}              % Image.
\DeclareMathOperator{\Int}{Int}             % Interior.
\DeclareMathOperator{\LC}{LC}               % Leading coefficient.
\DeclareMathOperator{\LCM}{LCM}             % Least common multiple.
\DeclareMathOperator{\LM}{LM}               % Leading monomial.
\DeclareMathOperator{\LT}{LT}               % Leading term.
\DeclareMathOperator{\Mod}{mod}             % Modulus.
\DeclareMathOperator{\Mon}{Mon}             % Monomial.
\DeclareMathOperator{\multideg}{mutlideg}   % Multi-Degree (Graphs).
\DeclareMathOperator{\nul}{nul}             % Null space of operator.
\DeclareMathOperator{\Ord}{Ord}             % Ordinal of ordered set.
\DeclareMathOperator{\Prin}{Prin}           % Principal value.
\DeclareMathOperator{\proj}{proj}           % Projection.
\DeclareMathOperator{\Refl}{Refl}           % Reflection operator.
\DeclareMathOperator{\rk}{rk}               % Rank of operator.
\DeclareMathOperator{\sgn}{sgn}             % Sign of a number.
\DeclareMathOperator{\sinc}{sinc}           % Sinc function.
\DeclareMathOperator{\Span}{Span}           % Span of a set.
\DeclareMathOperator{\Spec}{Spec}           % Spectrum.
\DeclareMathOperator{\supp}{supp}           % Support
\DeclareMathOperator{\Tr}{Tr}               % Trace of matrix.
%--------------------Declared Math Symbols--------------------%
\DeclareMathSymbol{\minus}{\mathbin}{AMSa}{"39} % Unary minus sign.
%------------------------New Commands---------------------------%
\DeclarePairedDelimiter\norm{\lVert}{\rVert}
\DeclarePairedDelimiter\ceil{\lceil}{\rceil}
\DeclarePairedDelimiter\floor{\lfloor}{\rfloor}
\newcommand*\diff{\mathop{}\!\mathrm{d}}
\newcommand*\Diff[1]{\mathop{}\!\mathrm{d^#1}}
\renewcommand*{\glstextformat}[1]{\textcolor{RoyalBlue}{#1}}
\renewcommand{\glsnamefont}[1]{\textbf{#1}}
\renewcommand\labelitemii{$\circ$}
\renewcommand\thesubfigure{%
    \arabic{chapter}.\arabic{figure}.\arabic{subfigure}}
\addto\captionsenglish{\renewcommand{\figurename}{Fig.}}
\numberwithin{equation}{section}

\renewcommand{\vector}[1]{\boldsymbol{\mathrm{#1}}}

\newcommand{\uvector}[1]{\boldsymbol{\hat{\mathrm{#1}}}}
\newcommand{\topspace}[2][]{(#2,\tau_{#1})}
\newcommand{\measurespace}[2][]{(#2,\varSigma_{#1},\mu_{#1})}
\newcommand{\measurablespace}[2][]{(#2,\varSigma_{#1})}
\newcommand{\manifold}[2][]{(#2,\tau_{#1},\mathcal{A}_{#1})}
\newcommand{\tanspace}[2]{T_{#1}{#2}}
\newcommand{\cotanspace}[2]{T_{#1}^{*}{#2}}
\newcommand{\Ckspace}[3][\mathbb{R}]{C^{#2}(#3,#1)}
\newcommand{\funcspace}[2][\mathbb{R}]{\mathcal{F}(#2,#1)}
\newcommand{\smoothvecf}[1]{\mathfrak{X}(#1)}
\newcommand{\smoothonef}[1]{\mathfrak{X}^{*}(#1)}
\newcommand{\bracket}[2]{[#1,#2]}

%------------------------Book Command---------------------------%
\makeatletter
\renewcommand\@pnumwidth{1cm}
\newcounter{book}
\renewcommand\thebook{\@Roman\c@book}
\newcommand\book{%
    \if@openright
        \cleardoublepage
    \else
        \clearpage
    \fi
    \thispagestyle{plain}%
    \if@twocolumn
        \onecolumn
        \@tempswatrue
    \else
        \@tempswafalse
    \fi
    \null\vfil
    \secdef\@book\@sbook
}
\def\@book[#1]#2{%
    \refstepcounter{book}
    \addcontentsline{toc}{book}{\bookname\ \thebook:\hspace{1em}#1}
    \markboth{}{}
    {\centering
     \interlinepenalty\@M
     \normalfont
     \huge\bfseries\bookname\nobreakspace\thebook
     \par
     \vskip 20\p@
     \Huge\bfseries#2\par}%
    \@endbook}
\def\@sbook#1{%
    {\centering
     \interlinepenalty \@M
     \normalfont
     \Huge\bfseries#1\par}%
    \@endbook}
\def\@endbook{
    \vfil\newpage
        \if@twoside
            \if@openright
                \null
                \thispagestyle{empty}%
                \newpage
            \fi
        \fi
        \if@tempswa
            \twocolumn
        \fi
}
\newcommand*\l@book[2]{%
    \ifnum\c@tocdepth >-3\relax
        \addpenalty{-\@highpenalty}%
        \addvspace{2.25em\@plus\p@}%
        \setlength\@tempdima{3em}%
        \begingroup
            \parindent\z@\rightskip\@pnumwidth
            \parfillskip -\@pnumwidth
            {
                \leavevmode
                \Large\bfseries#1\hfill\hb@xt@\@pnumwidth{\hss#2}
            }
            \par
            \nobreak
            \global\@nobreaktrue
            \everypar{\global\@nobreakfalse\everypar{}}%
        \endgroup
    \fi}
\newcommand\bookname{Book}
\renewcommand{\thebook}{\texorpdfstring{\Numberstring{book}}{book}}
\providecommand*{\toclevel@book}{-2}
\makeatother
\titleformat{\part}[display]
    {\Large\bfseries}
    {\partname\nobreakspace\thepart}
    {0mm}
    {\Huge\bfseries}
\titlecontents{part}[0pt]
    {\large\bfseries}
    {\partname\ \thecontentslabel: \quad}
    {}
    {\hfill\contentspage}
\titlecontents{chapter}[0pt]
    {\bfseries}
    {\chaptername\ \thecontentslabel:\quad}
    {}
    {\hfill\contentspage}
\newglossarystyle{longpara}{%
    \setglossarystyle{long}%
    \renewenvironment{theglossary}{%
        \begin{longtable}[l]{{p{0.25\hsize}p{0.65\hsize}}}
    }{\end{longtable}}%
    \renewcommand{\glossentry}[2]{%
        \glstarget{##1}{\glossentryname{##1}}%
        &\glossentrydesc{##1}{~##2.}
        \tabularnewline%
        \tabularnewline
    }%
}
\newglossary[not-glg]{notation}{not-gls}{not-glo}{Notation}
\newcommand*{\newnotation}[4][]{%
    \newglossaryentry{#2}{type=notation, name={\textbf{#3}, },
                          text={#4}, description={#4},#1}%
}
%--------------------------LENGTHS------------------------------%
% Spacings for the Table of Contents.
\addtolength{\cftsecnumwidth}{1ex}
\addtolength{\cftsubsecindent}{1ex}
\addtolength{\cftsubsecnumwidth}{1ex}
\addtolength{\cftfignumwidth}{1ex}
\addtolength{\cfttabnumwidth}{1ex}

% Indent and paragraph spacing.
\setlength{\parindent}{0em}
\setlength{\parskip}{0em}                                                           %
                                                                               %
% Add tikz files to the file path.                                             %
\makeatletter                                                                  %
    \def\input@path{{../../../tikz/}}                                          %
\makeatother                                                                   %
%----------------------------------GLOSSARY------------------------------------%
\makeglossaries                                                                %
\loadglsentries{glossary}                                                      %
\loadglsentries{acronym}                                                       %
%--------------------------------Main Document---------------------------------%
\begin{document}
    \ifx\ifmain\undefined
        \title{Probability}
        \author{Ryan Maguire}
        \date{\vspace{-5ex}}
        \maketitle
        \tableofcontents
        \chapter*{Probability}
        \markboth{}{PROBABILITY}
        \setcounter{chapter}{1}
    \else
        \chapter{Probability}
    \fi
    \subsection{Probability and Statistics}
        Probability is defined in sample spaces. The sample space of
        a random experiment is the set of all
        possible outcomes for the experiment.
        \begin{example}
            The sample space for tossing two coings randomly is
            $S=\{(H,H),(H,T),(T,H),(T,T)\}$.
        \end{example}
        \begin{definition}
            An event is an element of a sample space.
        \end{definition}
        \begin{definition}
            A probability function is a function
            $P:S\rightarrow[0,1]$, where $S$ is a sample
            space, such that:
            \begin{enumerate}
                \item For all $E\subset{S}$, $0\leq{P(E)}\leq{1}$.
                \item $P(S)=1$ and $P(\emptyset)=0$.
                \item If $A_{n}$ is a set of mutually disjoint
                      events in $S$, then
                      $P(\bigcup_{n=1}^{\infty}A_{n})%
                       =\sum_{n=1}^{\infty}P(A_{n})$
            \end{enumerate}
        \end{definition}
        The uniform probability model on a finite sample space with
        $n$ events gives the probability $1/n$ to every event.
        This is a useful model for many problems, such as flipping
        coins, or tossing dice. Given a collection of events, the probability
        is then the total number of times these events occur divided by
        the total number of possible events. If we take $k$ elements from
        a set of $n$, and if the ordering does not matter, the total number of
        ways to choose the $k$ elements is the number of permutations of $k$
        elements. This is:
        \begin{equation*}
            P(n,k)=\frac{n!}{(n-k)!}
        \end{equation*}
        If the ordering does matter, this is the number of combinations
        that are possible. This is:
        \begin{equation*}
            C(n,r)=\binom{n}{k}=\frac{n!}{k!(n-k)!}
        \end{equation*}
        Here $\binom{n}{k}$ is the \textit{binomial coefficient}.
        \begin{example}
            Suppose 10 men and 8 women are to be selected from to form
            a committee of 5 people. What is the probability that there
            are exactly 3 men? There are $\binom{10}{3}$ ways to
            select 3 men, and $\binom{8}{2}$ ways to select 2 women.
            There is a total of $\binom{18}{5}$ ways to select 5 people.
            Thus, we have:
            \begin{equation*}
                P=\frac{\binom{10}{3}\binom{8}{2}}{\binom{18}{5}}
                =\frac{20}{51}
            \end{equation*}
        \end{example}
        The probability point function, or probability mass function,
        is the function $Q(X)=P(X=x)$. The probability distribution
        function is given by
        $F_{X}(x)=P_{X}(X\leq{x})$. This can be expressed in
        terms of the probability mass function as
        $F_{X}(x)=\sum_{x<X}Q(x)$. For continuous random variables,
        the probability density function is defined as
        $f_{X}(x)=F_{X}'(x)$. In the study of statistics, we choose a sample
        from the total population which is intended to represent the
        entire population as closely as possible.
        \begin{definition}
            The arithmetic mean of a finite set
            $\{x_{1},\hdots,x_{N}\}$ is:
            \begin{equation*}
                \overline{x}=\frac{1}{N}\sum_{n=1}^{N}x_{n}
            \end{equation*}
        \end{definition}
        If $f_{i}$ is the probability of $x_{i}$, then the arithmetic mean is:
        \begin{equation*}
            \overline{x}=
            \frac{\sum_{n=1}^{N}x_{n}f_{n}}{\sum_{n=1}^{N}f_{n}}
        \end{equation*}
        The arithmetic mean is strongly effected by extreme values, or outliers.
        The deviation of $x_{i}$ from the mean $\overline{x}$
        is $d_{i}=x_{i}-\overline{x}$. This has the special property:
        \begin{equation*}
            \sum_{n=1}^{N}d_{n}
            =\sum_{n=1}^{N}(x_{i}-\overline{x})
            =0
        \end{equation*}
        \begin{theorem}
            If $x_{n}$ and $y_{n}$ are finite sequences of
            $N$ elements, and $z_{n}=x_{n}+y_{n}$, then
            $\overline{z}=\overline{x}+\overline{z}$.
        \end{theorem}
        \begin{definition}
            The geometric mean of a finite set
            $\{x_{1},\hdots,x_{N}\}$ is:
            \begin{equation*}
                g=\sqrt[N]{\prod_{n=1}^{N}x_{n}}
            \end{equation*}
            Where
            $\prod_{n=1}^{N}x_{n}=x_{1}\cdot{x_{2}}\cdots{x_{N}}$
        \end{definition}
        Logarithms can help calculate geometric means:
        \begin{theorem}
            Given a finite set $\{x_{1},\hdots,x_{N}\}$,
            the geometric mean is:
            \begin{equation*}
                g=\exp\Big(\frac{1}{N}\sum_{n=1}^{N}x_{n}\Big)
                =\exp(\overline{x})
            \end{equation*}
        \end{theorem}
        Given an initial deposit $A$ in a bank with yearly interest $q$,
        after $n$ years the account balance is given by the
        Compound Interest Formula:
        \begin{equation*}
            M=A(1+q)^{n}
        \end{equation*}
        \begin{definition}
            The harmonic mean of $\{x_{1},\hdots,x_{N}\}$
            is:
            \begin{equation*}
                h=\frac{1}{\frac{1}{N}\sum_{n=1}^{N}\frac{1}{x_{n}}}
                =\frac{n}{\sum_{n=1}^{N}\frac{1}{x_{n}}}
            \end{equation*}
        \end{definition}
        \begin{theorem}
            Given a set of numbers $\{x_{1},\hdots,x_{n}\}$,
            the harmonic mean $h$, and the geometric mean $g$,
            $h\leq{g}\leq\overline{x}$. Equality holds if
            and only if all of the $x_{i}$ are the same.
        \end{theorem}
        \begin{definition}
            The root-mean square (\textit{rms}), or the
            quadratic mean, of a data set $\{x_{1},\hdots,x_{N}\}$
            is:
            \begin{equation*}
                rms=\sqrt{\overline{x^{2}}}
                =\sqrt{\frac{1}{N}\sum_{n=1}^{N}x_{n}^2}
            \end{equation*}
        \end{definition}
        \begin{theorem}
            The quadratic mean of two numbers is greater
            than the geometric mean. That is:
            \begin{equation*}
                \sqrt{ab}\leq\sqrt{\frac{a^{2}+b^{2}}{2}}
            \end{equation*}
        \end{theorem}
        \begin{proof}
            For $(a-b)^{2}\geq{0}$, and thus
            $a^{2}+b^{2}-2ab\geq{0}$. Therefore, etc.
        \end{proof}
        The variation of data is a measurement of how much
        the data spreads about the average of the data.
        \begin{definition}
            The range of a finite data set is the maximum value
            minus the minimum value.
        \end{definition}
        The $n^{th}$ percentile of a data set is the value such that
        $n\%$ of the data set lies below said value, and
        $(100-n)\%$ lies above it. Percentiles are usually split into
        quartiles to better represent data sets. The interquartile range
        is the difference between the $75\%$ and the $25\%$ marks.
        The average deviation of a data set is:
        \begin{equation*}
            \overline{|\overline{x}-x_{i}|}
            =\frac{1}{N}\sum_{n=1}^{N}|\overline{x}-x_{i}|
        \end{equation*}
        Note the need for the absolute value signs. For without them,
        the average would always be zero.
        \begin{definition}
            The standard deviation of a finite data set
            $\{x_{1},\hdots,x_{N}\}$ is:
            \begin{equation*}
                \sigma=
                \sqrt{\frac{1}{N}\sum_{n=1}^{N}(\overline{x}-x_{n})^{2}}
            \end{equation*}
        \end{definition}
        \begin{definition}
            The variance of a data set is the square of the
            standard deviation. $V=\sigma^{2}$.
        \end{definition}
        \begin{theorem}
            The variance is equal to:
            \begin{equation*}
                V=\overline{x^{2}}-\overline{x}^{2}
            \end{equation*}
        \end{theorem}
        For a normal, or bell curve, the interval between
        $\overline{x}$ and $\overline{x}\pm\sigma$ contains
        roughly $68\%$ of the entire data set. $2\sigma$
        contains about $95\%$ of the data, and
        $3\sigma$ contains $99.5\%$.
        The $\chi$ square statistic is defined by:
        \begin{equation*}
            \chi^{2}=\frac{1}{\sigma^{2}}
                \sum_{n=1}^{N}(\overline{x}-x_{n})^{2}
        \end{equation*}
        The $\chi$ square distribution is:
        \begin{equation*}
            Y=Y_{0}\chi{\nu-2}e^{-\frac{1}{2}\chi^{2}}
        \end{equation*}
        Where $\nu=n-1$ is the umber of degrees of freedom,
        and $Y_{0}$ is a constant to make the area under the
        curve equal 1.
\end{document}