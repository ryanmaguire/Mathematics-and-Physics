\documentclass[crop=false,class=book,oneside]{standalone}
%----------------------------Preamble-------------------------------%
%---------------------------Packages----------------------------%
\usepackage{geometry}
\geometry{b5paper, margin=1.0in}
\usepackage[T1]{fontenc}
\usepackage{graphicx, float}            % Graphics/Images.
\usepackage{natbib}                     % For bibliographies.
\bibliographystyle{agsm}                % Bibliography style.
\usepackage[french, english]{babel}     % Language typesetting.
\usepackage[dvipsnames]{xcolor}         % Color names.
\usepackage{listings, lstlinebgrd}      % Verbatim-Like Tools.
\usepackage{mathtools, esint, mathrsfs} % amsmath and integrals.
\usepackage{amsthm, amsfonts}           % Fonts and theorems.
\usepackage{tabularx}
\usepackage{tcolorbox}                  % Frames around theorems.
\usepackage{upgreek}                    % Non-Italic Greek.
\usepackage{paracol}                    % Two-column styling.
\usepackage{wrapfig}                    % Wrap text around figure.
\usepackage{fmtcount, etoolbox}         % For the \book{} command.
\usepackage[newparttoc]{titlesec}       % Formatting chapter, etc.
\usepackage{titletoc}                   % Allows \book in toc.
\usepackage[nottoc]{tocbibind}          % Bibliography in toc.
\usepackage[titles]{tocloft}            % ToC formatting.
\usepackage{multicol, enumitem}         % Multi-column/enumerate.
\usepackage{import}                     % Import external files.
\usepackage{pgfplots, tikz}             % Drawing/graphing tools.
\usetikzlibrary{
    calc,                   % Calculating right angles and more.
    angles,                 % Drawing angles within triangles.
    arrows.meta,            % Latex and Stealth arrows.
    quotes,                 % Adding labels to angles.
    positioning,            % Relative positioning of nodes.
    decorations.markings,   % Adding arrows in the middle of a line.
    patterns,
    arrows,
    shapes,
    shapes.geometric,
    cd,
    hobby,
    babel
}                                       % Libraries for tikz.
\pgfplotsset{compat=1.9}                % Version of pgfplots.
\usepackage[font=scriptsize,
            labelformat=simple,
            labelsep=colon]{subcaption} % Subfigure captions.
\usepackage[font={scriptsize},
            hypcap=true,
            labelsep=colon]{caption}    % Figure captions.
\usepackage{hyperref}                   % Allows for hyperlinks.
\hypersetup{
    colorlinks=true,
    linkcolor=blue,
    filecolor=magenta,
    urlcolor=Cerulean,
    citecolor=SkyBlue
}                           % Colors for hyperref.
\usepackage[toc,acronym,nogroupskip]{glossaries} % Glossaries and acronyms.
\usepackage[subpreambles=false]{standalone}      % Complileable sub files.

% Various font stuff from kiwi.
% Use this for Times text and Computer Modern math
%\usepackage{times}

% Quite nice
%\usepackage[charter, greekfamily=, greekuppercase=italicized]{mathdesign}
%\usepackage[utopia, greekuppercase=italicized]{mathdesign}    % Math is narrower

% Use this for Times text and math
%\usepackage{newtxtext}
%\usepackage[libertine,cmintegrals]{newtxmath}
%\usepackage{fix-cm}

%\usepackage{txfontsb}
% or
%\usepackage{mathptmx}

%\usepackage[scaled=0.92]{helvet}
%\renewcommand{\rmdefault}{ptm}

%\usepackage{mathpazo}    % add possibly `sc` and `osf` options
%\usepackage{eulervm}

%\usepackage{fourier}
%\renewcommand{\rmdefault}{ptm}
%\usepackage{mathptm}

%\usepackage{fontspec}
%\setmainfont{lmodern}

%\usepackage[varg]{txfonts}
%\usepackage{fouriernc}
%\usepackage{mathpazo}

%\usepackage{bookman}
%\usepackage[scaled]{uarial}
%\usepackage[scaled]{helvet}
%\renewcommand*\familydefault{\sfdefault}
%\usepackage[math]{anttor}

%\newcommand\fgeorgia{\fontfamily{jvn}\selectfont}
%\newcommand\ftimes{\fontfamily{ptm}\selectfont}
%\newcommand\fhelvetica{\fontfamily{phv}\selectfont}
%\newcommand\fcourier{\fontfamily{pcr}\selectfont}
%\newcommand\fbookman{\fontfamily{pbk}\selectfont}
%\newcommand\fnewcentury{\fontfamily{pnc}\selectfont}
%\newcommand\fpalatino{\fontfamily{ppl}\selectfont}
%\newcommand\favantgarde{\fontfamily{pag}\selectfont}
%\newcommand\fnormal{\normalfont}
%\newcommand\fsize[1]{\ifnum#1>0\fontsize{#1}{#1}\selectfont\else\normalsize\fi}
%------------------------Theorem Styles-------------------------%
% Define theorem style for default spacing and normal font.
\newtheoremstyle{normal}
    {\topsep}               % Amount of space above the theorem.
    {\topsep}               % Amount of space below the theorem.
    {}                      % Font used for body of theorem.
    {}                      % Measure of space to indent.
    {\bfseries}             % Font of the header of the theorem.
    {}                      % Punctuation between head and body.
    {.5em}                  % Space after theorem head.
    {}

% Define theorem style for default spacing with italicized font.
\newtheoremstyle{normalit}{\topsep}{\topsep}
                {\itshape}{}{\bfseries}{}{.5em}{}

% Italic header environment.
\newtheoremstyle{thmit}{\topsep}{\topsep}{}{}{\itshape}{}{0.5em}{}

% Define italicized environments.
\theoremstyle{normalit}
\newtheorem{theorem}{Theorem}[section]
\newtheorem{lemma}{Lemma}[section]
\newtheorem{corollary}{Corollary}[section]
\newtheorem{proposition}{Proposition}[section]
\newtheorem*{theorem*}{Theorem}

% Define environments with italic headers.
\theoremstyle{thmit}
\newtheorem*{solution}{Solution}
\newtheorem*{fsolution}{Solution}

% Define default environments.
\theoremstyle{normal}
\newtheorem{example}{Example}[section]
\newtheorem{definition}{Definition}[section]
\newtheorem{problem}{Problem}[section]
\newtheorem{question}{Question}[section]
\newtheorem{remark}{Remark}[section]
\newtheorem{properties}{Properties}[section]
\newtheorem{notation}{Notation}[section]
\newtheorem{axiom}{Axiom}[section]
\newtheorem*{properties*}{Properties}
\newtheorem*{remark*}{Remark}
\newtheorem*{definition*}{Definition}
\theoremstyle{plain}

% Define framed environment.
\tcbuselibrary{most}
\newtcbtheorem[use counter*=theorem]{ftheorem}{Theorem}%
    {colback=green!5,colframe=green!35!black,
     fonttitle=\bfseries\upshape}{th}

\newtcbtheorem[use counter*=example]{fdefinition}{Definition}%
    {fonttitle=\bfseries\upshape,
     colback=blue!5!white,colframe=blue!75!black}{def}

\newtcbtheorem[use counter*=example]{fexample}{Example}%
    {fonttitle=\bfseries\upshape,
     colback=red!5!white,colframe=red!75!black}{ex}

\newtcbtheorem[use counter*=notation]{fnotation}{Notation}%
    {fonttitle=\bfseries\upshape,
     colback=SeaGreen!5!white,colframe=SeaGreen!75!black}{ex}

\newtcbtheorem[use counter*=corollary]{fcorollary}{Corollary}%
    {fonttitle=\bfseries\upshape,
     colback=Orchid!5!white,colframe=Orchid!75!black}{ex}

\newenvironment{bproof}{\textit{Proof.}}{\hfill$\square$}
\tcolorboxenvironment{bproof}{blanker,breakable,left=5mm,
                             before skip=10pt,after skip=10pt,
                             borderline west={1mm}{0pt}{red}}
\tcolorboxenvironment{fsolution}
    {enhanced jigsaw,colframe=cyan,interior hidden,breakable}

%--------------------Declared Math Operators--------------------%
\DeclareMathOperator{\Refl}{Refl}           % Reflection operator.
\DeclareMathOperator{\Span}{Span}           % Span of a set of vectors.
\DeclareMathOperator{\Card}{Card}           % Cardinality of set.
\DeclareMathOperator{\Ord}{Ord}             % Ordinal of ordered set.
\DeclareMathOperator{\Tr}{Tr}               % Trace of matrix.
\DeclareMathOperator{\adjoint}{adj}         % Adjoint of matrix.
\DeclareMathOperator{\rk}{rk}               % Rank of operator.
\DeclareMathOperator{\nul}{nul}             % Null space of operator.
\DeclareMathOperator{\sgn}{sgn}             % Sign of a number.
\DeclareMathOperator{\multideg}{mutlideg}   % Multi-Degree (Graphs).
\DeclareMathOperator{\GCD}{GCD}             % Greatest common denominator.
\DeclareMathOperator{\LM}{LM}               % Leading monomial
\DeclareMathOperator{\LC}{LC}               % Leading coefficient.
\DeclareMathOperator{\LT}{LT}               % Leading term.
\DeclareMathOperator{\LCM}{LCM}             % Least common multiple.
\DeclareMathOperator{\Mon}{Mon}             % Monomial.
\DeclareMathOperator{\Spec}{Spec}           % Spectrum.
\DeclareMathOperator{\proj}{proj}           % Projection.
\DeclareMathOperator{\comp}{comp}           % Component.
\DeclareMathOperator{\sinc}{sinc}           % Sinc function.
\DeclareMathOperator{\Ima}{Im}              % Image of operator.
\DeclareMathOperator{\Prin}{Prin}           % Principal value.
\DeclareMathOperator{\Mod}{mod}             % Modulus.
%------------------------New Commands---------------------------%
\DeclarePairedDelimiter\norm{\lVert}{\rVert}
\DeclarePairedDelimiter\ceil{\lceil}{\rceil}
\DeclarePairedDelimiter\floor{\lfloor}{\rfloor}
\newcommand*\diff{\mathop{}\!\mathrm{d}}
\newcommand*\Diff[1]{\mathop{}\!\mathrm{d^#1}}
\renewcommand{\mod}{\ \Mod}
\renewcommand*{\glstextformat}[1]{\textcolor{RoyalBlue}{#1}}
\renewcommand{\glsnamefont}[1]{\textbf{#1}}
\renewcommand\labelitemii{$\circ$}
\renewcommand\thesubfigure{\arabic{chapter}.\arabic{figure}}
\renewcommand\thesubfigure{%
    \arabic{chapter}.\arabic{figure}.\arabic{subfigure}}
\addto\captionsenglish{\renewcommand{\figurename}{Fig.}}
%------------------------Book Command---------------------------%
\makeatletter
\renewcommand\@pnumwidth{1cm}
\newcounter{book}
\renewcommand\thebook{\@Roman\c@book}
\newcommand\book{%
    \if@openright
        \cleardoublepage
    \else
        \clearpage
    \fi
    \thispagestyle{plain}%
    \if@twocolumn
        \onecolumn
        \@tempswatrue
    \else
        \@tempswafalse
    \fi
    \null\vfil
    \secdef\@book\@sbook
}
\def\@book[#1]#2{%
    \ifnum \c@secnumdepth >-3\relax
        \refstepcounter{book}%
        \addcontentsline{toc}{book}{
            \bookname\ \thebook:\hspace{1em}#1
        }
    \else
        \addcontentsline{toc}{book}{#1}%
    \fi
    \markboth{}{}%
    {\centering
     \interlinepenalty \@M
     \normalfont
     \ifnum \c@secnumdepth >-2\relax
       \huge\bfseries \bookname\nobreakspace\thebook
       \par
       \vskip 20\p@
     \fi
     \Huge \bfseries #2\par}%
    \@endbook}
\def\@sbook#1{%
    {\centering
     \interlinepenalty \@M
     \normalfont
     \Huge \bfseries #1\par}%
    \@endbook}
\def\@endbook{
    \vfil\newpage
        \if@twoside
            \if@openright
                \null
                \thispagestyle{empty}%
                \newpage
            \fi
        \fi
        \if@tempswa
            \twocolumn
        \fi
}
\newcommand*\l@book[2]{%
    \ifnum \c@tocdepth >-2\relax
        \addpenalty{-\@highpenalty}%
        \addvspace{2.25em \@plus\p@}%
        \setlength\@tempdima{3em}%
        \begingroup
            \parindent \z@ \rightskip \@pnumwidth
            \parfillskip -\@pnumwidth
            {
                \leavevmode
                \Large \bfseries #1\hfil \hb@xt@\@pnumwidth{
                    \hss #2
                }
            }
            \par
            \nobreak
            \global\@nobreaktrue
            \everypar{\global\@nobreakfalse\everypar{}}%
        \endgroup
    \fi}
\newcommand\bookname{Book}
\renewcommand{\thebook}{\texorpdfstring{\Numberstring{book}}{book}}
\providecommand*{\toclevel@book}{-2}
\makeatother
\titlecontents{chapter}[0pt]
    {\bfseries}
    {\chaptername\ \thecontentslabel:\quad}
    {}
    {\hfill\contentspage}
\titleformat{\part}[display]
    {\Large\bfseries}
    {\partname\nobreakspace\thepart}
    {0mm}
    {\Huge\bfseries}
    \titlecontents{part}[0pt]
    {\large\bfseries}
    {\partname\ \thecontentslabel: \quad}
    {}
    {\hfill\contentspage}
\newcommand{\MarkRightAngle}[4][.3cm]
    {\coordinate (tempa) at ($(#3)!#1!(#2)$);
     \coordinate (tempb) at ($(#3)!#1!(#4)$);
     \coordinate (tempc) at ($(tempa)!0.5!(tempb)$);%midpoint
     \draw (tempa) -- ($(#3)!2!(tempc)$) -- (tempb);}
%--------------------------LENGTHS------------------------------%
% Spacings for the Table of Contents.
\addtolength{\cftsecnumwidth}{1ex}
\addtolength{\cftsubsecindent}{1ex}
\addtolength{\cftsubsecnumwidth}{1ex}
\addtolength{\cftfignumwidth}{1ex}
\addtolength{\cfttabnumwidth}{1ex}

% Spacing for multi-column and enumerate environments.
\setlength{\multicolsep}{6pt}
\setlist[enumerate]{itemsep=0pt,topsep=3pt}

% Indent and paragraph spacing.
\setlength{\parindent}{0em}
\setlength{\parskip}{0em}
%----------------------------GLOSSARY-------------------------------%
\makeglossaries
\loadglsentries{../../glossary}
\loadglsentries{../../acronym}
%--------------------------Main Document----------------------------%
\begin{document}
    \ifx\ifmathcourses\undefined
        \pagenumbering{roman}
        \title{Differential Calculus}
        \author{Ryan Maguire}
        \date{\vspace{-5ex}}
        \maketitle
        \tableofcontents
        \chapter*{Differential Calculus}
        \addcontentsline{toc}{chapter}{Differential Calculus}
        \markboth{}{DIFFERENTIAL CALCULUS}
        \setcounter{chapter}{1}
        \pagenumbering{arabic}
    \else
        \chapter{Differential Calculus}
    \fi
    \section{Limits}
    \section{Continuity}
        \begin{lexample}
            Let $k$ be the number of solutions of $\exp(x)+x-2=0$
            in the interval $[0,1]$ and let $n$ be the number of
            solutions not in $[0,1]$ What are the values of
            $k$ and $n$? Given $y(x)=\exp(x)+x-2$, we have:
            \begin{equation}
                y'(x)=\exp(x)+1>0    
            \end{equation}
            Thus $y'(x)>0$ for all $x$ and
            $y$ is strictly increasing. Recall that
            $e\approx 2.71$. So:
            \begin{subequations}
                \begin{align}
                    y(0)&=\exp(0)+0-2=1-2=-1<0\\
                    y(1)&=\exp(1)+1-2=e-1\approx1.71>0.
                \end{align}
            \end{subequations}
            Therefore $y(0)<0<y(1)$, and thus, from the continuity
            of $y$, there is a $c\in(0,1)$
            such that $y(c)=0$. But $y$ is strictly increasing, and
            therefore for all $x>c$ we have $y(x)>y(c)=0$, and for
            all $x<c$ we have $y(x)<y(c)=0$. Thus $c$ is the only
            solution. The answer is $k=1$ and $n=0$.
        \end{lexample}
    \section{Differentiation}
        \subsection{The Definition of the Derivative}
        \subsection{Elementary Differentiation Theorems}
        \subsection{Derivatives of Trigonometric Functions}
        \subsection{The Chain Rule}
        \subsection{Implicit Differentiation}
        \subsection{Derivatives of Exponentials and Logarithms}
    \section{Applications of Derivatives}
        \subsection{Tangent Lines}
            \begin{lexample}
                Find the equation of the line tangent to:
                \begin{equation}
                    \label{eqn:DIFF_CALC_EX_TANGENT_X_PLUS_EXP_X}
                    y(x)=x+\exp(x)
                \end{equation}
                for when $x=0$.
                The tangent line of $y$ at $x_{0}$
                is a line that touches
                and lies tangent to $y$ at the point $x_{0}$. Using the
                definition we have:
                \begin{equation}
                    \label{eqn:DIFF_CALC_TANGENT_LINE_FORMULA}
                    y_{T}(x)
                    =y'\big(x_{0}\big)\big(x-x_{0}\big)+y_{0}
                \end{equation}
                We have $x_{0}=0$, and can solve for $y(0)$ and $y'(0)$
                by using
                Eqn.~\ref{eqn:DIFF_CALC_EX_TANGENT_X_PLUS_EXP_X}:
                \par
                \begin{subequations}
                    \begin{minipage}[b]{0.49\textwidth}
                        \begin{align}
                            y(x)&=x+\exp(x)\\
                            y(0)&=0+\exp(0)\\
                            &=1
                        \end{align}
                    \end{minipage}
                    \hfill
                    \begin{minipage}[b]{0.49\textwidth}
                        \begin{align}
                            y'(x)&=1+\exp(x)\\
                            y'(0)&=1+\exp(0)\\
                            &=2
                        \end{align}
                    \end{minipage}
                \end{subequations}
                Using these values, we can plug this into the formula
                given in Eqn.~\ref{eqn:DIFF_CALC_TANGENT_LINE_FORMULA}
                to obtain:
                \begin{equation}
                    y_{T}(x)=2x+1
                \end{equation}
            \end{lexample}
            \begin{lexample}
                Given a real number $b\in\mathbb{R}$ and a function
                $f:\mathbb{R}\rightarrow\mathbb{R}$ defined by:
                \begin{equation}
                    f(x)=3x^{2}+bx+12
                \end{equation}
                such that $f$ has its vertex at $x=2$, what is $f(5)$?
                If the vertex of $f$ is at $x=2$, then $f'(2)=0$,
                and so we obtain:
                \par
                \begin{minipage}[b]{0.49\textwidth}
                    \begin{align}
                        f(x)&=3x^{2}+bx+12\\
                        f'(x)&=6x+b
                    \end{align}
                \end{minipage}
                \hfill
                \begin{minipage}[b]{0.49\textwidth}
                    \begin{align}
                        f'(2)&=0\\
                        \Rightarrow
                        12+b&=0
                    \end{align}
                \end{minipage}
                \par
                From this we see that $b=-12$, and thus:
                \begin{equation}
                    f(x)=3x^{2}-12x+12
                \end{equation}
                Evaluating, we obtain $f(5)=27$.
            \end{lexample}
            \begin{lexample}
                Which of the following circles has
                the greatest number of
                points of intersections with the parabola $x^{2}=y+4$:
                \begin{enumerate}
                    \begin{multicols}{5}
                        \item[A.)] $x^{2}+y^{2}=1$
                        \item[B.)] $x^{2}+y^{2}=2$
                        \item[C.)] $x^{2}+y^{2}=9$
                        \item[D.)] $x^{2}+y^{2}=16$
                        \item[E.)] $x^{2}+y^{2}=25$
                    \end{multicols}
                \end{enumerate}
                We wish to solve the system of equations:
                \begin{align*}
                    x^{2}+y^{2}&=r^{2}\\
                    x^{2}-y&=4
                \end{align*}
                The solutions are of the form $(\pm \sqrt{r^2-y^2},y)$.
                For $(x,y)$ to be a solution
                the second equation requires
                that $y\geq -4$. Substituting $x$, we have
                $y^{2}+y+4-r^{2}=0$. Using the quadratic formula
                we obtain
                $y=\frac{-1\pm\sqrt{1+4(r^{2}-4)}}{2}$.
                For $y$ to be real
                we need $1+4(r^{2}-4)\geq 0$. Solving for $r^2$,
                we have$r^{2}\geq\frac{15}{4}$.
                Thus there are no solutions for
                $r^{2}<\frac{15}{4}$. If $r^{2}=\frac{15}{4}$,
                then there are two solutions:
                $(\pm\frac{\sqrt{3}}{2},\frac{-1}{2})$. If
                $r^{2}>\frac{15}{4}$, then we are
                also constrained by $y\geq-4$. This means 
                \begin{equation*}
                    \frac{-1\pm\sqrt{1+4(r^{2}-4)}}{2}\geq-4
                    \Rightarrow -1\pm\sqrt{1+4(r^{2}-4)}\geq-8
                    \Rightarrow 7\pm\sqrt{1+4(r^{2}-4)}\geq 0
                \end{equation*}
                For $r^2 \geq \frac{15}{4}$, $7+\sqrt{1+4(r^2-4)}>0$
                is always true. Now we consider $7-\sqrt{1+4(r^2-4)}:$
                \begin{align*}
                    7-\sqrt{1+4(r^{2}-4)}&\geq 0
                    \Rightarrow 7\geq\sqrt{1+4(r^{2}-4)}\\
                    \Rightarrow 49&\geq 1+4(r^{2}-4)\Rightarrow
                    48\geq 4(r^{2}-4)\\
                    \Rightarrow 12&\geq r^{2}-4\Rightarrow 16\geq r^{2}
                \end{align*}
                Thus if $\frac{15}{4}<r^2<16$, we have two
                $y$ values, with two $x$ values corresponding
                to each for a total of $4$ solutions.
                If $r^2 = 16$, then we have three solutions:
                $(0,-4)$, $(\pm \sqrt{7},3)$. If $r^2>16$
                then there is only one possible value for
                $y$, with two corresponding $x$ values.
                Thus, if $r^2>16$ there are only two solutions.
                The maximum number of solutions is attained
                when $\frac{15}{4} < r^2 <16$. The only value
                listed that has this property is $r^2 = 9$.
                The correct answer is C.
            \end{lexample}
        \begin{problem}
            What is the greatest area of a triangular
            region with one vertex at the center of
            a circle of radius $1$, and the other
            two on the circle?
        \end{problem}
        \begin{proof}[Solution]
        \begin{theorem*}
            Area is invariant under translation
            and rotation.
        \end{theorem*}
            We can use this theorem to place the center
            at $(0,0)$, one of the points at $(1,0)$,
            and then solve for the third point. The third
            point lies on the circle $x^2+y^2 = 1$, so
            we have $y = \sqrt{1-x^2}$. The area of
            the triangle is $\frac{1}{2}bh$. The base
            is the distance from $(0,0)$ to $(1,0)$,
            which is $1$, and the height is $\sqrt{1-x^2}$.
            So we have $A = \frac{1}{2}\sqrt{1-x^2}$.
            To find the extremum we take the derivative
            with respect to $x$ and set it to $0$.
            \begin{align*}
                A&=\frac{1}{2}\sqrt{1-x^{2}}\\
                =\frac{dA}{dx}&=0\\
                \Rightarrow\frac{1}{2}\frac{-x}{\sqrt{1-x^{2}}}&=0\\
                \Rightarrow x&=0
            \end{align*}
            So the extremum occurs when $x=0$. The area is then $\frac{1}{2}\sqrt{1-0^2} = \frac{1}{2}$. The answer is $\frac{1}{2}$.
            \end{proof}
        \begin{problem}
            Order the following equations
            from least to greatest:
            \begin{align*}
                I&=1\\
                J&=\int_{0}^{1}\sqrt{1-x^{4}}dx\\
                K&=\int_{0}^{1}\sqrt{1+x^{4}}dx\\
                L&=\int_{0}^{1}\sqrt{1-x^{8}}dx
            \end{align*}
        \end{problem}
        \begin{proof}[Solution]
            \begin{theorem*}
                If $f$ and $g$ are continuous on $(a,b)$
                and $f>g$, then $\int_{a}^{b}f>\int_{a}^{b}g$
            \end{theorem*}
            All three integrals occur over the interval
            $(0,1)$. For $0 < x < 1$, we have the following:
            \begin{equation*}
                0<x^{8}<x^{4}<x<1
            \end{equation*}
            Using this, we have have:
            \begin{equation*}
                0<\sqrt{1-x^{4}}<\sqrt{1-x^{8}}<1<\sqrt{1+x^{4}}
            \end{equation*}
            Using the theorem, we have:
            \begin{equation*}
                0<\int_{0}^{1}\sqrt{1-x^4}dx
                <\int_{0}^{1}\sqrt{1-x^8}dx<1
                <\int_{0}^{1}\sqrt{1+x^4}dx
            \end{equation*}
            So $J<L<1<K$
            \end{proof}
        \begin{problem}
            Which is the best estimate of $\sqrt{1.5}(266)^{3/2}$?
            \begin{enumerate}
                \begin{multicols}{5}
                \item[A.)] 1,0000
                \item[B.)] 2,700
                \item[C.)] 3,200
                \item[D.)] 4,100
                \item[E.)] 5,300
                \end{multicols}
            \end{enumerate}
        \end{problem}
        \begin{proof}[Solution]
            First note that $1.5 = \frac{3}{2}$ and $266 = 2\cdot 133$. So we have $\sqrt{1.5}(266)^{3/2} = \sqrt{\frac{3}{2}}\sqrt{266}\cdot 266 = \frac{\sqrt{3}}{\sqrt{2}}\cdot \sqrt{2}\cdot \sqrt{133}\cdot 266 = \sqrt{3}\cdot \sqrt{133}\cdot 266 = \sqrt{3\cdot 133}\cdot 266 = \sqrt{399}\cdot 266$. Now for the approximating. We note that $399 \approx 400$. So $\sqrt{399} \approx \sqrt{400} = 20$. So, with this approximation we have $20\cdot 266 = 5320$. The closest option is E.) $5,300$.
        \end{proof}
        \begin{problem}
            If $A$ is a $2\times 2$ matrix who's columns and rows add up to a constant $k$, which of the following must be an eigenvector:
            \begin{enumerate}
                \begin{multicols}{3}
                \item[I] $\begin{pmatrix}1 \\ 0 \end{pmatrix}$
                \item[II] $\begin{pmatrix} 0 \\ 1 \end{pmatrix}$
                \item[III] $\begin{pmatrix} 1 \\ 1 \end{pmatrix}$
                \end{multicols}
            \end{enumerate}
            \begin{enumerate}
                \begin{multicols}{5}
                \item[A.)] I only
                \item[B.)] II only
                \item[C.)] III only
                \item[D.)] I and II only
                \item[E.)] I, II, and III
                \end{multicols}
            \end{enumerate}
            \end{problem}
        \begin{proof}[Solution]
            We have $A = \begin{bmatrix} a & b \\ c & d\end{bmatrix}$ With the condition that:
            \begin{align*}
                a+b&=k\\
                a+c&=k\\
                b+d&=k\\
                c+d&=k
            \end{align*}
            From this we can see that $a = d$ and $b=c = k=a$.
            So, we have $A = \begin{bmatrix} a & k-a \\ k-a & a\end{bmatrix}$. We could solve directly for the eignvectors, or we could be more time efficient and check the possibilities we were given. $A\begin{bmatrix}1 \\ 0 \end{bmatrix} = \begin{bmatrix} a \\ k-a \end{bmatrix}$. $A\begin{bmatrix} 0 \\ 1 \end{bmatrix} = \begin{bmatrix} k-a \\ a \end{bmatrix}$. So I and II are only solutions if $a = k = 0$. $A\begin{bmatrix} 1 \\ 1 \end{bmatrix} = \begin{bmatrix} k \\ k\end{bmatrix} = k \begin{bmatrix} 1 \\ 1 \end{bmatrix}$. Thus we see that $\begin{bmatrix} 1 \\ 1 \end{bmatrix}$ is always an eigenvector. The answer is C.
            \end{proof}
        \begin{problem}
            A total of $x$ fee of fencing is to form three sides of a level rectangular yard. What is the maximum possible area in terms of $x$?
            \end{problem}
        \begin{proof}[Solution]
            Let $ax$ be the length of one side of the rectangular and $bx$ be the length of the adjacent side, so we have $ax+2bx = x$. From this we have $a+2b = 1$, or $a = 1-2b$. The area of the rectangle is the product of these lengths, so $A = ax\cdot bx = abx^2$. Substituting $a = 1-2b$, we have $A = (1-2b)bx^2$. To find the extremum we differentiate with respect to $b$ and set equal to $0$.
            \begin{align}
                A&=(1-2b)bx^2\\
                \frac{dA}{db}&=0\\
                \Rightarrow (1-4b)x^{2}&=0\\
                \Rightarrow b&=\frac{1}{4}
            \end{align}
            So we have $b = \frac{1}{4}$, and $a = 1-2b = \frac{1}{2}$. The corresponding area is $abx^2 = \frac{1}{2}\cdot \frac{1}{4} x^2 = \frac{x^2}{8}$. The answer is $\frac{x^2}{8}$.
            \end{proof}
        \begin{problem}
            What is the unit digit in the decimal expansion of $7^{25}$?
            \end{problem}
        \begin{proof}[Solution]
                Asking what is the unit digit of $7^{25}$ is equivalent to asking what is $7^{25} \mod 10$.
            \begin{align*}
                7^{25} &= 7\cdot7^{24}\\
                &= 7\cdot(7^2)^{12}\\
                &= 7\cdot(49)^{12}\\
                &\equiv 7\cdot(9)^{12}\mod 10\\
                &= 7\cdot (9^2)^{6}\\
                &= 7 \cdot (81)^{6}\\
                &\equiv 7 \cdot (1)^6 \mod 10\\
                &= 7
            \end{align*}
            So, $7^{25} \cong 7 \mod 10$. The answer is $7$.
            \end{proof}
        \begin{problem}
            Let $f:[-2,3]\rightarrow \mathbb{R}$ be continuous.
            Which of the following is NOT necessarily true?
            \begin{enumerate}
                \item[A.)] $f$ is bounded.
                \item[B.)] $\int_{-2}^{3}f$ exists.
                \item[C.)] For each $c$ between $f(-2)$ and $f(3)$
                           there is an $x\in (-2,3)$
                           such that $f(x)=c$.
                \item[D.)] There is an $M$ in $f([-2,3])$
                           such that $\int_{-2}^{3}f=5M$.
                \item[E.)] $\underset{h\rightarrow 0}{\lim}%
                            \frac{f(h)-f(0)}{h}$ exists.
            \end{enumerate}
            \end{problem}
        \begin{proof}[Solution]
            \begin{theorem*}
            A subset $\mathcal{C}\subset \mathbb{R}$ is compact if and only if it is closed and bounded.
            \end{theorem*}
            \begin{theorem*}
            If $\mathcal{C}$ is compact and $f:\mathcal{C}\rightarrow \mathbb{R}$ is continuous, then f is bounded.
            \end{theorem*}
            \begin{theorem*}
            If is continuous and bounded, then f is integrable.
            \end{theorem*}
            \begin{theorem*}
            If $f:[a,b] \rightarrow \mathbb{R}$ is continuous, then for all $c$ between $f(a)$ and $f(b)$ there is an $x\in (a,b)$ such that $f(x) = c$.
            \end{theorem*}
            \begin{theorem*}
            If $f:[a,b]\rightarrow \mathbb{R}$ is continuous then there is an $M\in f\big([a,b]\big)$ such that $\int_{a}^{b}f = M(b-a)$
            \end{theorem*}
            Since $[-2,3]$ is closed and bounded, it is compact, and if $f$ is continuous then it must be bounded so A is true. Since $f$ is continuous and bounded, it is integrable so B is true. C is simply the Intermediate Value Theorem and D is simply the Mean Value Theorem. By process of elimination, E is not necessarily true. Let us find a counterexample. Let $f=|x|$. Then $\underset{h\rightarrow 0^+}\lim \frac{|h|-|0|}{h} = 1$, but $\underset{h\rightarrow 0^{-}}\lim \frac{|h|-|0|}{h} = -1$. Thus the limit does not exist for this function. The answer is E.
            \end{proof}
        \begin{problem}
            What is the volume of the solid formed by revolving about the $x-$axis the region in the first quadrant of the $xy-$plane bounded by the coordinate axes and the graph of $x = \frac{1}{\sqrt{1+x^2}}$?
            \end{problem}
        \begin{proof}[Solution]
            Working by definition, we could simply use the formula $V = \int_{0}^{\infty}\pi r^2(x)dx$, where $r(x) = \frac{1}{\sqrt{1+x^2}}$. But memorizing formulas is usually not the best idea. Instead let's try to do this intuitively. Suppose we had a small cylinder centered at $x$. The volume of a cylinder is $\pi r^2 h$, so our tiny slab will be $dV = \pi r^2(x) dx$. Integrating over all of these infinitesimal discs gives us $V = \pi \int_{0}^{\infty}r^2(x)dx$. For our problem, $r(x) = \frac{1}{\sqrt{1+x^2}}$, so we have $V = \pi \int_{0}^{\infty} \frac{1}{1+x^2}dx = \pi \tan^{-1}(x)\big|_{0}^{\infty} = \pi \cdot \frac{\pi}{2} = \frac{\pi^2}{2}$. The answer is $\frac{\pi^2}{2}$.
            \end{proof}
        \begin{problem}
            How many real zeros does the polynomial $3x^5+8x-7$ have?
            \end{problem}
        \begin{proof}[Solution]
            \begin{theorem*}
            If $f = \sum_{k=0}^{n} a_k x^k$, then the number of positive zeroes of $f(x)$ is equal to the number of sign changes of $f$ or is an even number less than that.
            \end{theorem*}
            \begin{theorem*}
            If $f=\sum_{k=0}^{n} a_k x^k$, then the number of negative zeroes is equal to the number of sign changes of $f(-x)$ or equal to an even number less than that.
            \end{theorem*}
            We have $f(x) = 2x^5+8x - 7$. So, the set of coefficients is $\{-7,8,0,0,0,2\}$. There is only one sign change, from $-7$ to $8$, so there is one positive zero. $f(-x) = -2x^5 - 8x - 7$, which has the set of coefficients $\{-7,-8,0,0,0,0,-2\}$. This has no sign changes, so there are no negative zeroes. Also, $f(0) = -7$, so $0$ is not a root of $f$. Thus $f$ has $1$ zero. The answer is $1$.
            \end{proof}
        \begin{problem}
            If $T$ is a linear transformation of $V = \mathbb{R}^{2\times 3}$ $\textbf{onto}$ $W = \mathbb{R}^4$, what is $\dim(\{v\in V:T(v) = 0\})$?
            \end{problem}
        \begin{proof}[Solution]
            \begin{theorem*}
            If $V,W$ are vector spaces and $T:V\rightarrow W$ is a linear transformation, then $\dim(\Im(T)) + \dim(\ker(T)) = \dim(V)$.
            \end{theorem*}
            This set is called the kernel of $T$, denoted $\ker(T)$. Now, as $T$ is onto, the image of $T$ is $V$, which has dimension $4$. Also, $\dim(V) = 6$. So we have $\dim(\ker(T)) + 4 = 6$, so $\dim(\ker(T)) = 2$. The answer is $2$.
            \end{proof}
        \begin{problem}
            If $f,g:\mathbb{R}\rightarrow \mathbb{R}$ are twice differentiable, and if for all $x>0$, $f'(x)>g'(x)$, then which of the following must be true for all $x>0$?
            \begin{enumerate}
                \begin{multicols}{3}
                \item[A.)] $f(x)>g(x)$
                \item[B.)] $f''(x)>g''(x)$
                \item[C.)] $f(x)-f(0)>g(x)-g(0)$
                \item[D.)] $f'(x)-f'(0)>g'(x)-g'(0)$
                \item[E.)] $f''(x) - f''(0)>g''(x)-g''(0)$
                \end{multicols}
            \end{enumerate}
            \end{problem}
        \begin{proof}[Solution]
            \begin{theorem*}
            If $f:(a,b)\rightarrow \mathbb{R}$ is continuous and $x\in (a,b)$, then $f(x)-f(a) = \int_{a}^{x}f'$ 
            \end{theorem*}
            \begin{theorem*}
            If $f,g$ are continuous and bounded on $(a,b)$, and if $f>g$, then $\int_{a}^{b}f > \int_{a}^{b}g$
            \end{theorem*}
            \begin{theorem*}
            If $f,g$ are continuous and bounded on $(a,b)$, $f'>g'$, and if $x\in (a,b)$, then $f(x) - f(0) > g(x) - g(0)$.
            \end{theorem*}
            From this we have that C is true. There are counterexamples for the others:
            \begin{enumerate}
                \item[A.)] Let $f = x$ and $g = 2$. Then $f'(x) = 1$ and $g'(x) = 0$, but $f(x)<g(x)$ for $x\in [0,2)$.
                \item[B.)] Let $f = x$, $g(x) = \int_{0}^{x} \frac{1}{2}\sin(t^2)dt$. Then $f'(x) = 1$, $g'(x) = \frac{1}{2}\sin(x^2)$, $f''(x) = 0$, $g''(x) = x\cos(x^2)$.
                \item[D.)] Let $f(x) = x$, $g(x) = \frac{1}{2}x^2$. Then $f'(x) = 1$, $g'(x) = x$, $f(x)-f(0) = 0$, $g(x)-g(0) = x$.
                \item[E.)] Use the same example from B.
            \end{enumerate}
            \end{proof}
        \begin{problem}
            Where is the function $f(x) = \begin{cases} \frac{x}{2}, & x\in \mathbb{Q} \\ \frac{x}{3}, & x \notin \mathbb{Q}\end{cases}$ disccontinuous?
            \end{problem}
        \begin{proof}[Solution]
            Let $x\in \mathbb{Q}\setminus \{0\}$ and let $x_n$ be any sequence of irrationals such that $x_n \rightarrow x$. Then $f(x_n) = \frac{x_n}{3} \rightarrow \frac{x}{3}$, but $f(x) = \frac{x}{2}$. Therefore $f$ is not continuous in $\mathbb{Q}\setminus \{0\}$. If $x\in \mathbb{R}\setminus \mathbb{Q}$, let $x_n$ be any sequence of rationals such that $x_n \rightarrow x$. Then $f(x_n) = \frac{x_n}{2} \rightarrow \frac{x}{2}$, but $f(x) = \frac{x}{3}$. So $f$ is discontinuous on $\mathbb{R}\setminus \mathbb{Q}$. If $x= 0$, let $\varepsilon>0$ and choose $\delta = \varepsilon$. Then for $|x|<\delta$ we have $|f(0) - f(x)| = |f(x)| < \frac{|x|}{2} < \frac{\varepsilon}{2}<\varepsilon$. Thus, $f$ is continuous at $0$. $f$ is discontinuous at all non-zero real numbers.
            \end{proof}
        \begin{problem}
            Let $P_1 = \{2,3,5,7,11,\hdots\}$ and $P_n = \{2n,3n,5n,7n,11n,\hdots\}$. Which of the following is non-empty?
            \begin{enumerate}
            \begin{multicols}{5}
                \item[A.)] $P_1\cap P_{23}$
                \item[B.)] $P_7\cap P_{21}$
                \item[C.)] $P_{12}\cap P_{20}$
                \item[D.)] $P_{20}\cap P_{24}$
                \item[E.)] $P_{5}\cap P_{25}$
            \end{multicols}
            \end{enumerate}
            \end{problem}
        \begin{proof}[Solution]
            \
            \begin{enumerate}
                \item[A.)] If $x\in P_{1}\cap P_{23}$, $x = p = 23\cdot q$ for primes $p$ and $q$. But then $p = 23\cdot q$, a contradiction as $p$ is prime. So A is empty.
                \item[B.)] If $x\in P_{7}\cap P_{21}$, then $x= 7p = 21q$. So $7p = 7\cdot 3q$, and thus $p = 3q$, again a contradiction.
                \item[C.)] If $x\in P_{12}\cap P_{20}$, then $x = 12p = 20q$, so we have $4\cdot 3 p = 4\cdot 5 q$, and thus $3p = 5q$. So $p=5$ and $q = 3$. $60 \in P_{12}\cap P_{20}$.
                \item[D.)] If $x\in P_{20}\cap P_{24}$, then $20p = 24q$. So $4\cdot 5p = 4\cdot 6q$, and thus $5p = 6q$. The left-side is even, and thus the only possibility is $p=5$. But then $10 = 6p$, a contradiction as $p$ is prime.
                \item[E.)] If $x\in P_{5}\cap P_{25}$, then $x = 5p = 25q$, so $5p = 5\cdot 5q$, and therefore $p=5q$, a contradiction as $p$ is prime.
            \end{enumerate}
            C is the only non-empty set. The answer is C.
            \end{proof}
        \begin{problem}
            Let $C(\mathbb{R})$ be the set of continuous function $f:\mathbb{R}\rightarrow \mathbb{R}$. Then $C(\mathbb{R})$ is a vector space with addition defined as $(f+g)(x) = f(x)+g(x)$ for all $f,g\in C(\mathbb{R})$, and scalar multiplication defined $(rf)(x) = r\cdot f(x)$, for all $x,r\in \mathbb{R}$. Which of the following are subspaces of $C(\mathbb{R})$?
            \begin{enumerate}
                \item[I] $\{f:f''\textrm{ exists and }f'' - 2f'+3f = 0\}$
                \item[II] $\{g:g''\textrm{ exists and }g'' = 3g' \}$
                \item[III] $\{h:h''\textrm{ exists and }h'' = h+1\}$
            \end{enumerate}
            \begin{enumerate}
                \begin{multicols}{5}
                \item[A.)] I only
                \item[B.)] I and II only
                \item[C.)] I and III only
                \item[D.)] II and III only
                \item[E.)] I, II, and III
                \end{multicols}
            \end{enumerate}
            \end{problem}
        \begin{proof}[Solution]
            We have to check for closure under vector addition and scalar multiplication.
            \begin{enumerate}
                \item[I] For addition, $(f_1+f_2)'' - 2(f_1+f_2)' +3(f_1+f_2) = (f''_1 - 2f'_1 +3f_1) + (f''_2 - 2f'_2 +3f_2) = 0+0 = 0$. For scalar multiplication $(rf)'' - 2(rf)' +3(rf) = r(f'' - 2f' + 3f) = r\cdot 0 = 0$. So I is closed under vector addition and scalar multiplication, and thus I is a subspace.
                \item For addition, $(g_1+g_2)'' = g''_1 +g''_2 = 3g_1 + 3g_2 = 3(g_1+g_2)$. For scalar multiplication, $(rg)'' =rg'' = r\cdot 3g = 3(rg)$. Thus II is closed under vector addition and scalar multiplication. II is a subspace.
                \item For addition, $(h_1+h_2)'' = h''_1 + h''_2 = h_1+1 + h_2+1 = (h_1+h_2) + 2 \ne (h_1+h_2)+1$. Therefore III is not closed under vector addition. III is not a subspace.
            \end{enumerate}
            We see that I and II are subspaces, but III is not. The answer is B.
            \end{proof}
        \begin{problem}
            For what value of $b$ does the line $y_1=10x$ lie tangent to the curve $y_2=e^{bx}$?
            \end{problem}
        \begin{proof}[Solution]
            If the two curves lie tangent at $x$, then $y'_1(x) = y'_2(x)$. Thus we have $10 = be^{bx}$. But also $y_1(x) = y_2(x)$. So $e^{bx} = 10x$. So we have $10 = be^{bx} = b\cdot (10x) = 10bx$. Therefore we have $bx = 10$, and $x = \frac{1}{b}$. But $10 = b e^{bx} = b e^{b\cdot \frac{1}{b}} = be$. Therefore $b = \frac{10}{e}$. The answer is $b = \frac{10}{e}$.
            \end{proof}
        \begin{problem}
            Let $h(x)=\int_{0}{x^{2}}e^{x+t}dt$. What is $h'(1)$?
            \end{problem}
        \begin{proof}[Solution]
            \begin{theorem*}
            If $\alpha$ and $\beta$ are differentiable, $f$ is continuous, and if $F(x) = \int_{\alpha(x)}^{\beta(x)}f$, then:
            \begin{equation*}
                F'(x)=f\big(\beta(x)\big)\beta'(x)-
                    f\big(\alpha(x)\big)\alpha'(x)
            \end{equation*}
            \end{theorem*}
            We have $h(x) = \int_{0}^{x^2}e^{x+t}dt = e^x \int_{0}^{x^2}e^t dt$. Using the product rule and the above theorem we have:
            \begin{align*}
                h'(x) &= e^x \int_{0}^{x^2} e^t dt + e^{x} \big(e^{x^2}\frac{d}{dx}(x^2)\big) \\
                &= e^x\big(e^{x^2}-1\big) + 2xe^xe^{x^2} \\
                \Rightarrow h'(1) &= e(e-1) + 2e^2 = \\
                &= 3e^2 - e
            \end{align*}
            The answer is $3e^{2}-e$
            \end{proof}
        \begin{problem}
            Let $\{a_n\}_{n=1}^{\infty}$ be recursively defined by $a_1 = 1$, and for all $n\in \mathbb{N}$, $a_{n+1} = \frac{n+2}{n}a_n$. What is $a_{30}$?
            \end{problem}
        \begin{proof}[Solution]
            Let us try to find a closed form for this sequence. For $n>2$ we hav:
            \begin{align*}
                a_{n+1} &= \frac{n+2}{n}a_n \\
                &= \frac{n+2}{n}\frac{n+1}{n-1}a_{n-1}\\
                &= \frac{n+2}{n}\frac{n+1}{n-1}\frac{n}{n-2}a_{n-2}
            \end{align*}
            The pattern seems to be $a_{n+1} = \frac{(n+2)!}{(n+1-k)!}\frac{(n-1-k)!}{n!}a_{n-k}$. When $k=n-1$, we have $a_{n} = \frac{(n+1)!}{2!(n-1)!}a_1$. We can confirm this by induction. The base case of $n=1$ is $\frac{2!}{2!0!}a_1 = a_1$. Suppose it is true for $n\in \mathbb{N}$. Then $a_{n+1} = \frac{n+2}{n} a_n = \frac{n+2}{n} \frac{(n+1)!}{2!(n-1)!}a_1 = \frac{(n+2)!}{2!n!}a_1$. This proves the induction step. Therefore, $a_n = \frac{(n+1)!}{2!n!}a_1$. So $a_{30} = \frac{31!}{2!29!} = \frac{31\cdot 30}{2} = 31\cdot 15$. The answer is $(15)(31)$.
            \end{proof}
    \section{Preliminaries}
    \subsection{Functions}
        \begin{definition}
            A function, denoted $f:X\rightarrow{Y}$, is a
            correspondence between two set $X$ and $Y$.
            $X$ is called the domain and $Y$ is called
            the co-domain. For every element $x\in{X}$ there
            is exactly one element $y\in{Y}$ such that
            $f(x)=y$.
        \end{definition}
        The sum of two functions is defined as
        $(f+g)(x)=f(x)+g(x)$. Similarly, the difference is
        defined as $(f-g)(x)=f(x)-g(x)$. The product is defined
        as $(fg)(x)=f(x)g(x)$, and the quotient is defined,
        for non-zero $g(x)$, as $(f/g)(x)=f(x)/g(x)$. An important
        concept is that of the composition of functions.
        \begin{definition}
            If $f:X\rightarrow{Y}$ and $g:Y\rightarrow{Z}$ are
            functions, then the composition
            $g\circ{f}:X\rightarrow{Z}$ is defined by
            $(g\circ{f})(x)=g(f(x))$.
        \end{definition}
        \begin{definition}
            A periodic function
            $f:\mathbb{R}\rightarrow\mathbb{R}$ is a function
            such that there exists a $p\in\mathbb{R}$ such that,
            for all $x\in\mathbb{R}$,
            $f(x+p)=f(x)$. The period of a periodic function
            is the smallest such $p$.
        \end{definition}
        \begin{definition}
            A one-to-one function, or an injection, is a function
            such that $f(x_{1})=f(x_{2})$ if and only if
            $x_{1}=x_{2}$.
        \end{definition}
        \begin{definition}
            The inverse of a one-to-one function
            $f:X\rightarrow{Y}$ is the function
            $f^{-1}:Y\rightarrow{X}$ defined by
            $f^{-1}(y)=x$, where $x$ is such that
            $f(x)=y$. If is, for all
            $y\in{Y}$, $f(f^{-1}(y))=y$, and for
            all $x\in{X}$, $f^{-1}(f(x))=x$.
        \end{definition}
        \begin{definition}
            A root of a function
            $f:X\rightarrow\mathbb{R}$, or
            $f:X\rightarrow\mathbb{C}$,
            is a value $x\in{X}$ such that
            $f(x)=0$
        \end{definition}
        \begin{definition}
            A fixed point of a function
            $f:X\rightarrow{X}$ is a point
            $x\in{X}$ such that $f(x)=x$.
        \end{definition}
        \begin{example}
            Find the fixed points and the inverse
            of $f(x)=(3x-2)/(x-1)$.
            A fixed point is a value $x$ such that
            $f(x)=x$. So, we have:
            $x=(3x-2)/(x-1)$, and thus $x^{2}-x=3x-2$.
            This implies $x^{2}-4x+2=0$. Using the quadratic
            formula, we have $x=2\pm\sqrt{2}$. To find the
            inverse we solve $y=(3x-2)/(x-1)$ for $x$. So
            we have: $y=(3x-2)/(x-1)$ which implies that
            $yx-y=3x-2$, and thus
            $x(3-y)=2-y$. Therefore $x=(2-y)/(3-y)$. The
            inverse function is
            $f^{-1}(x)=(2-x)/(3-x)$.
        \end{example}
    \subsection{Limits}
        The limit of a function $f$ defined on an open interval
        as $x$ approaches some value $a$ in that interval is
        the value $f(x)$ approaches. The is,
        $\lim_{x\rightarrow{a}}f(x)$ is the value $f(x)$ gets
        near to as $x$ approaches $a$. It is possible that
        there is no such value, and the limit may not
        exists. When it does exist there are several theorems
        pertaining to the limit:
        \begin{theorem}
            If $\lim_{x\rightarrow{a}}f(x)$ exists, then the
            limit is unique.
        \end{theorem}
        \begin{theorem}
            If $\lim_{x\rightarrow{a}}f(x)$ and
            $\lim_{x\rightarrow{a}}g(x)$ exists, then
            $\lim_{x\rightarrow{a}}(f+g)(x)$ exists and:
            \begin{equation*}
                \lim_{x\rightarrow{a}}(f+g)(x)
                =\lim_{x\rightarrow{a}}f(x)
                +\lim_{x\rightarrow{a}}g(x)
            \end{equation*}
        \end{theorem}
        \begin{theorem}
            If $\lim_{x\rightarrow{a}}f(x)$ and
            $\lim_{x\rightarrow{a}}g(x)$ exists, then
            $\lim_{x\rightarrow{a}}(fg)(x)$ exists and:
            \begin{equation*}
                \lim_{x\rightarrow{a}}(fg)(x)
                =\big(\lim_{x\rightarrow{a}}f(x)\big)
                \big(\lim_{x\rightarrow{a}}g(x)\big)
            \end{equation*}
        \end{theorem}
        \begin{theorem}
            If $\lim_{x\rightarrow{a}}f(x)=L$ and $L\ne{0}$,
            then $\lim_{x\rightarrow{a}}1/f(x)=1/L$.
        \end{theorem}
        You can also factor out constants in limits, raise
        limits to positive integer powers, and raise to
        arbitrary powers if the limit is positive. There are
        also left and right sided limits, for when a general
        limit doesn't exist. This also functions with jumps
        to have well defined limits as $x$ approaches from
        the right, and for when $x$ approaches from the left.
        \begin{example}
            The following are true:
            \begin{enumerate}
                \begin{multicols}{2}
                    \item $\lim_{x\rightarrow{0}}\frac{\sin(x)}{x}=1$
                    \item $\lim_{x\rightarrow{0}}\frac{1-\cos(x)}{x}=0$
                \end{multicols}
            \end{enumerate}
        \end{example}
        \begin{example}
            Let's evaluate $\sqrt{x^{2}+2x}-x$ as $x\rightarrow\infty$.
            We have:
            \begin{align*}
                \sqrt{x^{2}+2x}-x
                &=\frac{x^{2}+2x-x^{2}}{\sqrt{x^{2}+2x}+x}\\
                &=\frac{2x}{\sqrt{x^{2}+2x}}\\
                &=\frac{2}{\sqrt{1+\frac{2}{x}}+1}\\
            \end{align*}
            So the limit is 1.
        \end{example}
    \subsection{Derivatives}
        The derivative of a function is a way of determining
        the rate of change of the function with respect to
        the independent variable. Another common way of
        writing this is that it is the slope of the tangent
        line of the graph of the function for any given
        point.
        \begin{ldefinition}{Derivative of a Real-Valued Function}
            The derivative of a function
            $f:\mathcal{U}\rightarrow\mathbb{R}$
            defined on an open interval $\mathcal{U}$ is:
            \begin{equation}
                f'(x)
                =\frac{\diff{f}}{\diff{x}}(x)
                =\lim_{h\rightarrow{0}}\frac{f(x+h)-f(x)}{h}
            \end{equation}
            Equivalently, we may write:
            \begin{equation*}
                f'(x_{0})
                =\frac{\diff{f}}{\diff{x}}(x_{0})
                =\lim_{x\rightarrow{x_{0}}}
                \frac{f(x)-f(x_{0})}{x-x_{0}}
            \end{equation*}
        \end{ldefinition}
        \begin{ltheorem}{Derivative of Constants}
            If $\mathcal{U}$ is an open interval in
            $\mathbb{R}$, and if
            $f:\mathcal{U}\rightarrow\mathbb{R}$ is
            a constant function, $f(x)=c$, then
            for all $x\in\mathcal{U}$,
            $f'(x)=0$.
        \end{ltheorem}
        \begin{proof}
            For if $h\in\mathbb{R}$ is non-zero, then:
            \begin{equation}
                \frac{f(x+h)-f(x)}{h}
                =\frac{c-c}{h}
                =\frac{0}{h}
                =0
            \end{equation}
            Thus, for all $x\in\mathcal{U}$, $f'(x)=0$.
        \end{proof}
        \begin{ltheorem}{Power Rule for Derivatives (Positive Integers)}
            If $\mathcal{U}\subseteq\mathbb{R}$ is an open
            interval, $a\in\mathbb{R}$, $n\in\mathbb{N}$,
            and if $f:\mathcal{U}\rightarrow\mathbb{R}$
            is a function defined by $f(x)=ax^{n}$, then
            for all $x\in\mathcal{U}$, $f'(x)=anx^{n-1}$.
        \end{ltheorem}
        \begin{theorem}
            If $f$ and $g$ are differentiable, then
            $(f+g)'(x)=f'(x)+g'(x)$
        \end{theorem}
        \begin{theorem}
            If $f$ is differentiable and $a$ is a constant,
            then $(af)'(x)=af'(x)$
        \end{theorem}
        \begin{theorem}
            If $f$ and $g$ are differentiable, then
            $(fg)'(x)=f'(x)g(x)+f(x)g'(x)$
        \end{theorem}
        \begin{theorem}
            If $f$ and $g$ are differentiable, and $g\ne{0}$,
            then:
            \begin{equation*}
                \Big(\frac{f}{g}\Big)'(x)
                =\frac{f'(x)g(x)-f(x)g'(x)}{g(x)^{2}}
            \end{equation*}
        \end{theorem}
        \begin{theorem}
            If $r\ne{0}$ and $f(x)=ax^{r}$, then
            $f'(x)=arx^{r-1}$.
        \end{theorem}
        \begin{theorem}[Chain Rule]
            If $f$ and $g$ are differentiable, then
            $(g\circ{f})'(x)=g'(f(x))f'(x)$.
        \end{theorem}
        \begin{theorem}
            The following is true:
            \begin{align}
                \frac{\diff}{\diff{x}}\Big(\sin(x)\Big)
                &=\cos(x)\\
                \frac{\diff}{\diff{x}}\Big(\cos(x)\Big)
                &=-\sin(x)\\
                \frac{\diff}{\diff{x}}\Big(\tan(x)\Big)
                &=\sec^{2}(x)\\
                \frac{\diff}{\diff{x}}\Big(\exp(x)\Big)
                &=\exp(x)
            \end{align}
        \end{theorem}
        \begin{theorem}
            If $f$ is differentiable, then:
            \begin{equation*}
                \frac{\diff}{\diff{x}}\Big(e^{f(x)}\Big)
                =e^{f(x)}\frac{\diff{f}}{\diff{x}}
            \end{equation*}
        \end{theorem}
        \begin{theorem}
            $\frac{\diff}{\diff{x}}(\ln(x))=\frac{1}{x}$
        \end{theorem}
        \begin{theorem}
            If $f$ is differentiable, then:
            \begin{equation*}
                \frac{\diff}{\diff{x}}\Big(\ln(f)\Big)
                =\frac{1}{f(x)}\frac{\diff{f}}{\diff{x}}
            \end{equation*}
        \end{theorem}
        \begin{theorem}
            If $x$ is a real number, and $a>0$, then
            $a^{x}=\exp(x\ln(a))$
        \end{theorem}
        \begin{theorem}
            If $a>0$ and $f$ is differentiable, then:
            \begin{equation*}
                \frac{\diff}{\diff{x}}\Big(a^{f(x)}\Big)
                =a^{f(x)}\ln(a)\frac{\diff{f}}{\diff{x}}
            \end{equation*}
        \end{theorem}
        If $f:\mathbb{R}^{2}\rightarrow\mathbb{R}$ is a function
        of two variables, we can define the
        \textit{partial derivatives} of $f$ with respect to
        the two coordinates of $f$.
        \begin{definition}
            The partial derivatives of a function
            $f:\mathbb{R}^{2}\rightarrow\mathbb{R}$ are:
            \begin{align*}
                \frac{\partial{f}}{\partial{x}}
                &=\lim_{h\rightarrow{0}}\frac{f(x+h,y)-f(x,y)}{h}
                &
                \frac{\partial{f}}{\partial{y}}
                &=\lim_{h\rightarrow{0}}\frac{f(x,y+h)-f(x,y)}{h}  
            \end{align*}
        \end{definition}
        The partial derivatives of $f$ are also commonly written as
        $f_{x}$ and $f_{y}$, respectively. Partial derivatives can be
        computed by assuming that the other variable is a constant,
        and then use the various theorems that apply to normal
        derivatives. There is a multivariate form of the chain rule.
        \begin{theorem}
            If $z=f(x,y)$ is differentiable in both $x$ and $y$, and
            if $x=x(t)$ and $y=y(t)$ are differentiable functions, then:
            \begin{equation*}
                \frac{\diff{f}}{\diff{t}}
                =\frac{\partial{f}}{\partial{x}}\frac{\diff{x}}{\diff{t}}
                +\frac{\partial{f}}{\partial{y}}\frac{\diff{y}}{\diff{t}}
            \end{equation*}
        \end{theorem}
        Because of this, we often write
        $\diff{z}=\frac{\partial{z}}{\partial{x}}\diff{x}%
         +\frac{\partial{z}}{\partial{y}}\diff{y}$.
        Similar extensions of this theorem hold for functions
        $f:\mathbb{R}^{n}\rightarrow\mathbb{R}$ of $n$ variables.
        \begin{example}
            If $\exp(xy)+\sin(xy)+1=0$, compute $dy/dx$. We compute
            the partial derivatives of $f(x,y)=\exp(xy)+\sin(xy)+1$:
            \begin{align*}
                \frac{\partial{f}}{\partial{x}}
                &=y\exp(xy)+y\cos(xy)
                &
                \frac{\partial{f}}{\partial{y}}
                &=x\exp(xy)+x\cos(xy)\\
            \end{align*}
            Using this, we have:
            \begin{equation*}
                \frac{\diff{y}}{\diff{x}}
                =-\frac{\partial{f}/\partial{y}}{\partial{f}/\partial{x}}
                =-\frac{y(\exp(xy)+\cos(xy)}{x(\exp(xy)+\cos(xy)}
                =-\frac{y}{x}
            \end{equation*}
        \end{example}
    \subsection{Application of Derivatives}
        \begin{definition}
            A function $f:\mathbb{R}\rightarrow\mathbb{R}$ is said
            to have a local minima $a$ if there is an open interval
            $\mathcal{U}$ such that $a\in\mathcal{U}$ and, for all
            $x\in\mathbb{U}$, $f(a)\leq{f(x)}$.
        \end{definition}
        \begin{definition}
            A function $f:\mathbb{R}\rightarrow\mathbb{R}$ is said
            to have a local maxima $a$ if there is an open interval
            $\mathcal{U}$ such that $a\in\mathcal{U}$ and, for all
            $x\in\mathbb{U}$, $f(a)\geq{f(x)}$.
        \end{definition}
        \begin{theorem}
            If $f$ is differentiable and has a relative maximum
            or minimum at $x=a$, then $f'(a)=0$.
        \end{theorem}
        \begin{theorem}
            If $f'(a)=0$ and $f''(a)>0$, then
            $a$ is a relative minimum.
        \end{theorem}
        \begin{theorem}
            If $f'(a)=0$ and $f''(a)<0$, then
            $a$ is a relative maximum.
        \end{theorem}
        \begin{definition}
            A point of inflection for a twice differentiable
            function $f$ is a point $a$ such that
            $f''(a)=0$.
        \end{definition}
        Points of inflection are points where the concavity of
        the function may change. This is also points curve
        crosses the tangent line.
        \begin{theorem}[Rolle's Theorem]
            If $f$ is continuous on the closed interval
            $[a,b]$ and differentiable on the open interval
            $(a,b)$, and if $f(a)=f(b)=0$, there there is
            a point $c\in(a,b)$ such that $f'(c)=0$.
        \end{theorem}
        \begin{theorem}[Mean Value Theorem]
            If $f$ is continuous on $[a,b]$ and
            differentiable on $(a,b)$, there there
            is a $c\in(a,b)$ such that:
            \begin{equation*}
                f'(c)=\frac{f(b)-f(a)}{b-a}
            \end{equation*}
        \end{theorem}
        \begin{theorem}
            If $f(x)\rightarrow{0}$ and $g(x)\rightarrow{0}$ as
            $x\rightarrow{a}$, or if
            $f(x)\rightarrow\infty$ and $g(x)\rightarrow\infty$
            as $x\rightarrow{a}$, and if $f$ and $g$ are differentiable,
            then:
            \begin{equation*}
                \lim_{x\rightarrow{a}}\frac{f(x)}{g(x)}
                =\lim_{x\rightarrow{a}}\frac{f'(x)}{g'(x)}
            \end{equation*}
        \end{theorem}
        \begin{theorem}
            If $f$ is $n$ times differentiable on in interval
            $(a,b)$ and if $x_{0}\in(a,b)$, then for all
            $x\in(a,b)$ there is a point $c\in(x_{0},x)$
            or $c\in(x,x_{0})$ such that:
            \begin{equation*}
                f(x)=
                \frac{f^{(n)}(c)(x-x_{0})^{n}}{n!}
                +\sum_{k=0}^{n-1}f^{(k)}(x_{0})\frac{(x-x_{0})^{k}}{k!}
            \end{equation*}
        \end{theorem}
        The two dimensional analog is as follows:
        \begin{align*}
            f(x,y)&=R_{T}+f(a,b)+
            \sum_{1\leq{r+s}\leq{n}}\frac{\partial^{r}}{\partial{x}^{r}}
            \frac{\partial^{s}}{\partial{y}^{s}}\Big(f(a,b)\Big)
            \frac{(x-a)^{r}}{r!}\frac{(y-b)^{s}}{s!}\\
            R_{T}&=\sum_{1\leq{r+s}\leq{n}}\frac{\partial^{r}}{\partial{x}^{r}}
            \frac{\partial^{s}}{\partial{y}^{s}}\Big(f(c,d)\Big)
            \frac{(x-a)^{r}}{r!}\frac{(y-b)^{s}}{s!}
        \end{align*}
        Where $(c,d)$ lies on the line segment between $(a,b)$ and $(x,y)$.
        \begin{definition}
            If $f:\mathbb{R}^{2}\rightarrow\mathbb{R}^{2}$ and
            $g:\mathbb{R}^{2}\rightarrow\mathbb{R}^{2}$ are
            differentiable functions, then the Jacobian is:
            \begin{equation*}
                \frac{\partial(f,g)}{\partial(u,v)}=
                \det
                \begin{pmatrix}
                    \frac{\partial{f}}{\partial{u}}&
                    \frac{\partial{f}}{\partial{v}}\\
                    \frac{\partial{g}}{\partial{u}}&
                    \frac{\partial{g}}{\partial{v}}
                \end{pmatrix}
                =
                \begin{vmatrix}
                    \frac{\partial{f}}{\partial{u}}&
                    \frac{\partial{f}}{\partial{v}}\\
                    \frac{\partial{g}}{\partial{u}}&
                    \frac{\partial{g}}{\partial{v}}
                \end{vmatrix}
            \end{equation*}
        \end{definition}
        \begin{definition}
            If $f$ and $g$ are differentiable functions,
            then the Wronskian $W(x)$ is:
            \begin{equation*}
                W(x)=
                \det
                \begin{pmatrix}
                    f(x)&g(x)\\
                    f'(x)&g'(x)
                \end{pmatrix}
                =
                \begin{vmatrix}
                    f(x)&g(x)\\
                    f'(x)&g'(x)
                \end{vmatrix}
            \end{equation*}
        \end{definition}
    \section{Exams}
    \subsection{CLEP Exam}
        \begin{problem}
            If $f(x)=-2x^{-3}$, then $f'(x)=$
            \begin{enumerate}[label=(\Alph*)]
                \begin{multicols}{4}
                    \item Bob
                    \item Bill
                    \item Alice
                    \item George
                \end{multicols}
            \end{enumerate}
        \end{problem}
    \subsection{Exams from UML (Fall 2012)}
        \subsubsection{Exam I}
            \begin{problem}
                Compute the derivatives of:
                \begin{enumerate}[label=(\alph*)]
                    \begin{multicols}{2}
                        \item $y(x)=\frac{1}{2}(x^{4}+7)$
                        \item $y(x)=\frac{x^{2}+1}{5}$
                    \end{multicols}
                \end{enumerate}
            \end{problem}
            \begin{solution}
                \par\hfill\par
                \begin{minipage}[b]{.49\textwidth}
                    \centering
                    \begin{align*}
                        \frac{\diff{y}}{\diff{x}}
                        &=\frac{\diff}{\diff{x}}\Big(
                            \frac{1}{2}\big(x^{4}+7\big)
                        \Big)\\
                        &=\frac{1}{2}\frac{\diff}{\diff{x}}
                            \big(x^{4}+7\big)\\
                        &=\frac{1}{2}\Big(
                            \frac{\diff}{\diff{x}}
                            \big(x^{4}\big)+
                            \frac{\diff}{\diff{x}}
                            \big(7\big)\Big)
                        \vphantom{\frac{\diff}{\diff}}\\
                        &=2x^{3}\vphantom{\frac{\diff}{\diff}}
                    \end{align*}
                \end{minipage}
                \hfill
                \vline
                \begin{minipage}[b]{.49\textwidth}
                    \centering
                    \begin{align*}
                        \frac{dy}{dx}
                        &=\frac{\diff}{\diff{x}}
                            \Big(\frac{x^{2}+1}{5}\Big)\\
                        &=\frac{1}{5}\frac{\diff}{\diff{x}}
                            \big(x^{2}+1\big)\\
                        &=\frac{1}{5}\Big(
                            \frac{\diff}{\diff{x}}
                            \big(x^{2}\big)+
                            \frac{\diff}{\diff{x}}
                            \big(1\big)\Big)\\
                        &=\frac{2}{5}x
                    \end{align*}
                \end{minipage}
            \end{solution}
            \begin{problem}
                Compute the derivatives of:
                \begin{enumerate}[label=(\alph*)]
                    \begin{multicols}{2}
                        \item $y(x)=-3x^{-8}+2\sqrt{x}$
                        \item $y(x)=\frac{x\sqrt{x}+1}{x}$
                    \end{multicols}
                \end{enumerate}
            \end{problem}
            \begin{proof}[Solution]
                \par\hfill\par
                \begin{minipage}[b]{.49\textwidth}
                    \centering
                    \begin{align*}
                        \frac{\diff{y}}{\diff{x}}
                        &=\frac{\diff}{\diff{x}}
                            \Big(3x^{-8}+2\sqrt{x}\Big)\\
                        &=\frac{\diff}{\diff{x}}
                            \big(3x^{-8}\big)
                        +\frac{\diff}{\diff{x}}
                            \big(2x^{\frac{1}{2}}\big)\\
                        &=3\frac{\diff}{\diff{x}}
                            \big(x^{-8}\big)+
                            2\frac{\diff}{\diff{x}}
                            \big(x^{\frac{1}{2}}\big)\\
                        &=-24x^{-9}+x^{-\frac{1}{2}}
                        \vphantom{\frac{\diff}{\diff}}
                    \end{align*}
                \end{minipage}
                \hfill
                \vline
                \begin{minipage}[b]{.49\textwidth}
                    \centering
                    \begin{align*}
                        \frac{\diff{y}}{\diff{x}}
                        &=\frac{\diff}{\diff{x}}
                            \Big(\frac{x\sqrt{x}+1}{x}\Big)\\
                        &=\frac{\diff}{\diff{x}}
                            \big(x^{\frac{1}{2}}+x^{-1}\big)\\
                        &=\frac{\diff}{\diff{x}}
                            \big(x^{\frac{1}{2}}\big)
                            +\frac{\diff}{\diff{x}}
                            \big(x^{-1}\big)\\
                        &=\frac{1}{2}x^{-\frac{1}{2}}-x^{-2}
                            \vphantom{\frac{\diff}{\diff}}
                    \end{align*}
                \end{minipage}
            \end{proof}
            \begin{problem}
                Compute the derivatives of:
                \begin{enumerate}[label=(\alph*)]
                    \begin{multicols}{2}
                        \item $y(x)=\frac{\sqrt{x}+\sqrt[3]{x}}%
                                         {\sqrt[4]{x^{3}}}$
                        \item $y(t)=\frac{t^{2}+1}{3t}$
                    \end{multicols}
                \end{enumerate}
            \end{problem}
            \begin{proof}[Solution]
                \par\hfill\par
                \begin{minipage}[b]{.49\textwidth}
                    \centering
                    \begin{align*}
                        \frac{\diff{y}}{\diff{x}}
                        &=\frac{\diff}{\diff{x}}\Big(
                            \frac{\sqrt{x}+\sqrt[3]{x}}
                                 {\sqrt[4]{x^{3}}}\Big)\\
                        &=\frac{\diff}{\diff{x}}\big(
                                x^{\frac{1}{2}-\frac{3}{4}}+
                                x^{\frac{1}{3}-\frac{3}{4}}\big)\\
                        &=\frac{\diff}{\diff{x}}\big(
                                x^{-\frac{1}{4}}+
                                x^{-\frac{5}{12}}\big)\\
                        &=\frac{\diff}{\diff{x}}\big(
                                x^{-\frac{1}{4}}\big)+
                                \frac{\diff}{\diff{x}}
                                \big(x^{-\frac{5}{12}}\big)\\
                        &=-\frac{1}{4}x^{-\frac{5}{4}}-
                            \frac{5}{12}x^{-\frac{17}{12}}
                    \end{align*}
                \end{minipage}
                \hfill
                \vline
                \begin{minipage}[b]{.49\textwidth}
                    \centering
                    \begin{align*}
                        \frac{\diff{y}}{\diff{t}}
                        &=\frac{\diff}{\diff{t}}
                            \Big(\frac{t^{2}+1}{3t}\Big)\\
                        &=\frac{1}{3}\frac{\diff}{\diff{t}}
                            \Big(\frac{t^{2}+1}{t}\Big)\\
                        &=\frac{1}{3}\frac{\diff}{\diff{t}}
                            \big(t+t^{-1}\big)\\
                        &=\frac{1}{3}\Big(\frac{\diff}{\diff{t}}
                            \big(t\big)
                        +\frac{\diff}{\diff{t}}
                            \big(t^{-1}\big)\Big)\\
                        &=\frac{1-t^{-2}}{3}
                    \end{align*}
                \end{minipage}
            \end{proof}
            \begin{problem}
                What are the horizontal tangents of
                $g(t)=e^{t}-4t+5$?
            \end{problem}
            \begin{proof}[Solution]
                Horizontal tangents occur when
                $\frac{\diff{g}}{\diff{t}}=0$.
                Computing, we have:
                \begin{align*}
                    \frac{\diff{g}}{\diff{t}}
                    &=\frac{\diff}{\diff{t}}
                        \big(e^{t}-4t+5\big)
                    &
                    \frac{\diff{g}}{\diff{t}}
                    &=0
                    \\
                    &=e^{t}-4
                    &
                    \Rightarrow
                    e^{t}-4&=0\\
                    &&
                    \Rightarrow
                    t&=2\ln(2)
                \end{align*}
                There is a horizontal tangent at $t=2\ln(2)$.
            \end{proof}
            \begin{problem}
                Compute $\frac{\Diff{2}{y}}{\diff{x}^{2}}$
                of $x\sin(x)$. Simplify.
            \end{problem}
            \begin{proof}[Solution]
                $\frac{\Diff{2}{y}}{\diff{x}^{2}}%
                 =\frac{\diff}{\diff{x}}(\frac{\diff{y}}{\diff{x}})$.
                Computing the first derivative, we have:
                \begin{equation*}
                    \frac{\diff{y}}{\diff{x}}
                    =\frac{\diff}{\diff{x}}
                        \big(x\sin(x)\big)
                    =x\frac{\diff}{\diff{x}}
                        \big(\sin(x)\big)+
                        \sin(x)\frac{\diff}{\diff{x}}
                        \big(x\big)
                    =x\cos(x)+\sin(x)    
                \end{equation*}
                Computing the second derivative:
                \begin{align*}
                    \frac{\Diff{2}{y}}{\diff{x}^{2}}
                    &=\frac{\diff}{\diff{x}}
                        \Big(\frac{\diff{y}}{\diff{x}}\Big)
                    &
                    &=\frac{\diff}{\diff{x}}
                        \big(x\cos(x)\big)+
                        \frac{\diff}{\diff{x}}
                        \big(\sin(x)\big)\\
                    &=\frac{\diff}{\diff{x}}
                        \big(x\cos(x)+\sin(x)\big)
                    &
                    &=-x\sin(x)+\cos(x)+\cos(x)\\
                    &&
                    &=2\cos(x)-x\sin(x)
                \end{align*}
            \end{proof}
            \begin{problem}
                Compute the derivatives of:
                \begin{enumerate}[label=(\alph*)]
                    \begin{multicols}{2}
                        \item $f(t)=\frac{2t-1}{t+3}$
                        \item $g(x)=\frac{x^{2}-1}{x^{2}+1}$
                    \end{multicols}
                \end{enumerate}
            \end{problem}
            \begin{proof}[Solution]
                \par\hfill\par
                \begin{minipage}[b]{.49\textwidth}
                    \centering
                    \begin{align*}
                        \frac{\diff{f}}{\diff{t}}
                        &=\frac{\diff}{\diff{t}}
                            \Big(\frac{2t-1}{t+3}\Big)\\
                        &=\frac{(t+3)\frac{\diff}{\diff{t}}
                                \big(2t-1\big)-
                                (2t-1)\frac{\diff}{\diff{t}}
                                \big(t+3\big)}
                               {(t+3)^{2}}\\
                        &=\frac{2(t+3)-(2t-1)}{(t+3)^{2}}\\
                        &=\frac{7}{(t+3)^{2}}
                    \end{align*}
                \end{minipage}
                \hfill
                \vline
                \begin{minipage}[b]{.49\textwidth}
                    \centering
                    \begin{align*}
                        \frac{\diff{g}}{\diff{x}}
                        &=\frac{\diff}{\diff{x}}
                            \Big(\frac{x^{2}-1}{x^{2}+1}\Big)\\
                        &=\frac{(x^{2}+1)\frac{\diff}{\diff{x}}
                                (x^{2}-1)-(x^{2}-1)
                                \frac{\diff}{\diff{x}}(x^{2}+1)}
                               {(x^{2}+1)^{2}}\\
                        &=\frac{2x\big((x^{2}+1)-(x^{2}-1)\big)}
                               {(x^{2}+1)^{2}}\\
                        &=\frac{4x}{(x^{2}+1)^{2}}
                    \end{align*}
                \end{minipage}
            \end{proof}
            \begin{problem}
                Compute the deratives of:
                \begin{enumerate}[label=(\alph*)]
                    \begin{multicols}{2}
                        \item $h(x)=\frac{x}{x^{2}+1}$
                        \item $y(t)=\frac{3t^{2}-5}{2t+1}$
                    \end{multicols}
                \end{enumerate}
            \end{problem}
            \begin{proof}[Solution]
                \par\hfill\par
                \begin{minipage}[b]{.49\textwidth}
                    \centering
                    \begin{align*}
                        \frac{\diff{h}}{\diff{x}}
                        &=\frac{\diff}{\diff{x}}
                            \Big(\frac{x}{x^{2}+1}\Big)\\
                        &=\frac{(x^{2}+1)\frac{\diff}{\diff{x}}
                            \big(x\big)-x\frac{\diff}{\diff{x}}
                            \big(x^{2}+1\big)}
                              {(x^{2}+1)^{2}}
                            \tag{Quotient Rule}\\
                        &=\frac{(x^{2}+1)-2x^{2}}{(x^{2}+1)^{2}}
                            \tag{Power Rule}\\
                        &=\frac{1-x^{2}}{(x^{2}+1)^{2}}
                            \tag{Simplify}
                    \end{align*}
                \end{minipage}
                \hfill
                \vline
                \begin{minipage}[b]{.49\textwidth}
                    \centering
                    \begin{align*}
                        \frac{\diff{y}}{\diff{t}}
                        &=\frac{\diff}{\diff{t}}
                            \Big(\frac{3t^{2}-5}{2t+1}\Big)\\
                        &=\frac{(2t+1)\frac{\diff}{\diff{t}}
                                \big(3t^{2}-5\big)-(3t^{2}-5)
                                \frac{\diff}{\diff{t}}\big(2t+1\big)}
                               {{(2t+1)^{2}}}
                                \tag{Quotient Rule}\\
                        &=\frac{(2t+1)(6t)-(3t^{2}-5)(2)}
                               {(2t+1)^{2}}
                            \tag{Power Rule}\\
                        &=\frac{6t^{2}+6t+10}{(2t+1)^{2}}
                            \tag{Simplify}
                    \end{align*}
                \end{minipage}
            \end{proof}
            \begin{problem}
                Find the equation of the tangent line
                to the graph of $f(x)=x-2\tan(x)$
                at $x=\frac{\pi}{4}$.
            \end{problem}
            \begin{proof}[Solution]
                The equation of the tangent line is
                $y(x)=f'(x_{0})(x-x_{0})+f(x_{0})$.
                Computing, we have:
                \begin{equation*}
                    \frac{\diff{f}}{\diff{x}}
                    =\frac{\diff}{\diff{x}}(x-2\tan(x))
                    =1-2\sec^{2}(x)    
                \end{equation*}
                So $f'(\frac{\pi}{4})=-3$ and
                $f(\frac{\pi}{4})=\frac{\pi}{4}-2$.
                Simplifying, the tangent is then:
                \begin{equation*}
                    y(x)=-3x+\pi-2
                \end{equation*}
            \end{proof}
            \begin{problem}
                Use the limit definition to compute
                the derivatives of:
                \begin{enumerate}[label=(\alph*)]
                    \begin{multicols}{2}
                        \item $f(x)=\frac{1}{x}$
                        \item $f(x)=\frac{1}{x+1}$
                    \end{multicols}
                \end{enumerate}
            \end{problem}
            \begin{proof}[Solution]
                \par\hfill\par
                \begin{minipage}[b]{.49\textwidth}
                    \centering
                    \begin{align*}
                        \frac{\diff{f}}{\diff{x}}
                            \big(x\big)
                        &=\underset{h\rightarrow{0}}{\lim}
                            \frac{f(x+h)-f(x)}{h}
                            \tag{Limit Definition}\\
                        &=\underset{h\rightarrow{0}}{\lim}
                            \frac{\frac{1}{x+h}-\frac{1}{x}}{h}
                            \tag{Substitution}\\
                        &=\underset{h\rightarrow{0}}{\lim}
                            \frac{x-(x+h)}{hx(x+h)}
                            \tag{Algebra}\\
                        &=\underset{h\rightarrow{0}}{\lim}
                            \frac{-1}{x(x+h)}
                            \tag{Simplify}\\
                        &=-\frac{1}{x^{2}}
                            \tag{Let $h\rightarrow{0}$}
                    \end{align*}
                \end{minipage}
                \hfill
                \vline
                \begin{minipage}[b]{.49\textwidth}
                    \centering
                    \begin{align*}
                        \frac{\diff{f}}{\diff{x}}\big(x\big)
                        &=\underset{h\rightarrow 0}{\lim}
                            \frac{f(x+h)-f(x)}{h}
                            \tag{Limit Definition}\\
                        &=\underset{h\rightarrow 0}{\lim}
                            \frac{\frac{1}{x+h+1}-\frac{1}{x+1}}{h}
                            \tag{Substitution}\\
                        &=\underset{h\rightarrow 0}{\lim}
                            \frac{x+1-(x+h+1)}{h(x+1)(x+h+1)}
                            \tag{Algebra}\\
                        &=\underset{h\rightarrow 0}{\lim}
                            \frac{-1}{(x+h+1)(x+1)}
                            \tag{Simplify}\\
                        &=-\frac{1}{(x+1)^{2}}
                            \tag{Let $h\rightarrow{0}$}
                    \end{align*}
                \end{minipage}
            \end{proof}
            \begin{problem}
                Use the limit definition to compute
                the derivatives of:
                \begin{enumerate}[label=(\alph*)]
                    \begin{multicols}{2}
                        \item $f(x)=\frac{1}{\sqrt{x}}$
                        \item $f(x)=\frac{1}{x^{2}}$
                    \end{multicols}
                \end{enumerate}
            \end{problem}
            \begin{proof}[Solution]
                \par\hfill\par
                \begin{minipage}[t]{.49\textwidth}
                    \centering
                    \begin{align*}
                        \frac{\diff{f}}{\diff{x}}\big(x\big)
                        &=\underset{h\rightarrow{0}}{\lim}
                            \frac{f(x+h)-f(x)}{h}
                            \tag{Limit Definition}\\
                        &=\underset{h\rightarrow{0}}{\lim}
                            \frac{\frac{1}{\sqrt{x+h}}-
                                  \frac{1}{\sqrt{x}}}
                                 {h}
                            \tag{Substitution}\\
                        &=\underset{h\rightarrow{0}}{\lim}
                            \frac{\sqrt{x}-\sqrt{x+h}}{h\sqrt{x+h}\sqrt{x}}
                            \tag{Algebra}\\
                        &=\underset{h\rightarrow{0}}{\lim}
                            \frac{x-(x+h)}
                                 {h\sqrt{x+h}\sqrt{x}
                                  (\sqrt{x+h}+\sqrt{x})}
                            \tag{Simplify Using the Conjugate}\\
                        &=\underset{h\rightarrow 0}{\lim}
                            \frac{1}{\sqrt{x+h}\sqrt{x}
                                     (\sqrt{x+h}+\sqrt{x})}
                            \tag{Simplify}\\
                        &=-\frac{1}{2x^{3/2}}
                            \tag{Let $h\rightarrow{0}$}
                    \end{align*}
                \end{minipage}
                \hfill
                \vline
                \begin{minipage}[t]{.49\textwidth}
                    \centering
                    \begin{align*}
                        \frac{\diff{f}}{\diff{x}}(x)
                        &=\underset{h\rightarrow{0}}{\lim}
                            \frac{f(x+h)-f(x)}{h}
                            \tag{Limit Definition}\\
                        &=\underset{h\rightarrow{0}}{\lim}
                            \frac{\frac{1}{(x+h)^{2}}-
                            \frac{1}{x^{2}}}{h}
                            \tag{Substitution}\\
                        &=\underset{h\rightarrow{0}}{\lim}
                            \frac{x^{2}-(x+h)^{2}}
                                 {hx^{2}(x+h)^{2}}
                            \tag{Algebra}\\
                        &=\underset{h\rightarrow 0}{\lim}
                            \frac{-2x-h^{2}}{x^{2}(x+h)^{2}}
                            \tag{Simplify}\\
                        &=-\frac{2}{x^{3}}
                            \tag{Let $h\rightarrow{0}$}
                    \end{align*}
                \end{minipage}
            \end{proof}
            \begin{problem}
                Use the limit definition to
                compute the derivative of $f(x)=\sqrt{x+1}$
            \end{problem}
            \begin{proof}[Solution]
                \begin{align*}
                    \frac{\diff{f}}{\diff{x}}\big(x\big)
                    &=\underset{h\rightarrow 0}{\lim}
                        \frac{f(x+h)-f(x)}{h}
                        \tag{Limit Definition}\\
                    &=\underset{h\rightarrow 0}{\lim}
                        \frac{\sqrt{x+h+1}-\sqrt{x+1}}{h}
                        \tag{Substitution}\\
                    &=\underset{h\rightarrow{0}}{\lim}
                        \frac{(x+h+1)-(x+1)}
                             {h(\sqrt{x+h+1}+\sqrt{x+1})}
                        \tag{Simplify Using the Conjugate}\\
                    &=\underset{h\rightarrow{0}}{\lim}
                        \frac{1}{\sqrt{x+h+1}+\sqrt{x+1}}
                        \tag{Simplify}\\
                    &=\frac{1}{2\sqrt{x+1}}
                        \tag{Let $h\rightarrow{0}$}
                \end{align*}
            \end{proof}
            \begin{problem}
                Find the horizontal and vertical
                asymptotes of $f(x)=\frac{2x^{3}-3}{x^{4}-16}$
            \end{problem}
            \begin{proof}[Solution]
                Horizontal asymptotes occur when the limit
                $\lim_{x\rightarrow\pm\infty}f(x)$ exists.
                Using L'H\^{o}pital's rule we see that
                $f(x)\rightarrow{0}$ as $x\rightarrow{\pm\infty}$.
                Vertical asymptotes occur when the denominator
                approaches 0. This occurs when $x^{4}=16$.
                The real solutions to this are $x=2$ and $x=-2$.
                From this we get:
            \begin{align*}
                \underset{x\rightarrow{2}^{+}}{\lim}f(x)
                &=+\infty
                &
                \underset{x\rightarrow{2}^{-}}{\lim}f(x)
                &=-\infty
                &
                \underset{x\rightarrow{-2}^{+}}{\lim}f(x)
                &=-\infty
                &
                \underset{x\rightarrow{-2}^{-}}{\lim}f(x)
                &=+\infty
            \end{align*}
            \end{proof}
            \begin{problem}
            Evaluate $\underset{x\rightarrow\infty}{\lim}\frac{3x^{4}+5}{2x^{4}}$
            \end{problem}
            \begin{proof}[Solution]
            From L'H\^{o}pital's Rule: $\underset{x\rightarrow\infty}{\lim}\frac{3x^{4}+5}{2x^{4}}=\underset{x\rightarrow\infty}{\lim}\frac{12x^{3}}{8x^{3}}=\underset{x\rightarrow \infty}{\lim}\frac{3}{2}=\frac{3}{2}$
            \end{proof}
        \subsubsection{Exam II}
            \begin{problem}
            Compute the derivative of $f(x)=e^{\sec(x)}$
            \end{problem}
            \begin{proof}[Solution]
            From the chain rule: $\frac{df}{dx}(x)=\frac{d}{dx}(e^{\sec(x)})=e^{\sec(x)}\frac{d}{dx}(\sec(x))=e^{\sec(x)}\sec(x)\tan(x)$
            \end{proof}
            \begin{problem}
            Compute the derivative of $g(x)=\ln(\sin^{2}(x))$
            \end{problem}
            \begin{proof}[Solution]
            Chain rule: $\frac{dg}{dx}(x)=\frac{d}{dx}(\ln(\sin^{2}(x)))=\frac{1}{\sin^{2}(x)}\frac{d}{dx}(\sin^{2}(x))=\frac{2\sin(x)\cos(x)}{\sin^{2}(x)}=2\cot(x)$
            \end{proof}
            \begin{problem}
            Compute the derivative of $s(t)=\sin(\sqrt[5]{t})$
            \end{problem}
            \begin{proof}[Solution]
            Chain rule: $\frac{ds}{dt}(t)=\frac{d}{dt}(\sin(\sqrt[5]{t}))=\cos(\sqrt[5]{t})\frac{d}{dt}(\sqrt[5]{t})=\frac{\cos(\sqrt[5]{t})}{5\sqrt[5]{t^{4}}}$
            \end{proof}
            \begin{problem}
            Compute the derivative of $f(x)=\ln(\cos^{-1}(x))$
            \end{problem}
            \begin{proof}[Solution]
            Chain rule: $\frac{df}{dx}(x)=\frac{d}{dx}(\ln(\cos^{-1}(x)))=\frac{1}{\cos^{-1}(x)}\frac{d}{dx}(\cos^{-1}(x))=-\frac{1}{\cos^{-1}(x)\sqrt{1-x^{2}}}$
            \end{proof}
            \begin{problem}
            Compute the derivative of $y(x)=(x+2)^{x+2}$
            \end{problem}
            \begin{proof}[Solution]
            Let $f(x)=\ln(y(x))$. Then $f(x)=\ln((x+2)^{x+2})=(x+2)\ln(x+2)$. Then:
            \begin{equation*}
                \frac{df}{dx}(x)=(x+2)\frac{d}{dx}(\ln(x+2))+\ln(x+2)\frac{d}{dx}(x+2)=1+\ln(x+2)
            \end{equation*}
            But $\frac{df}{dx}(x)=\frac{d}{dx}(\ln(y(x)))=\frac{1}{y(x)}\frac{dy}{dx}(x)$, so $\frac{dy}{dx}(x)=y(x)\frac{df}{dx}(x)$. Thus, $\frac{dy}{dx}(x)=(x+2)^{x+2}(1+\ln(x+2))$
            \end{proof}
            \begin{problem}
            Compute the derivative of $\sqrt{xy}=1$
            \end{problem}
            \begin{proof}[Solution]
            $\frac{d}{dx}(\sqrt{xy})=\frac{d}{dx}(1)=0$. And $\sqrt{xy}=1\Rightarrow xy=1\Rightarrow y=\frac{1}{x}$. So:
            \begin{equation*}
                \frac{1}{\sqrt{xy}}\frac{d}{dx}(xy)=0\Rightarrow \frac{y}{\sqrt{xy}}+\frac{x}{\sqrt{xy}}\frac{dy}{dx}=0\Rightarrow y+x\frac{dy}{dx}=0\Rightarrow\frac{dy}{dx}=-\frac{y}{x}=\frac{dy}{dx}=-\frac{1}{x^{2}}
            \end{equation*}
            \end{proof}
            \begin{problem}
            Compute the derivative of $y(x)=x^{2}\cos^{2}(2x^{2})$
            \end{problem}
            \begin{proof}[Solution]
            $\frac{d}{dx}(x^{2}\cos^{2}(2x^{2}))=\cos^{2}(2x^{2})\frac{d}{dx}(x^{2})+x^{2}\frac{d}{dx}(\cos^{2}(2x^{2}))=2x\cos^{2}(2x^{2})+8x^{3}\sin^{2}(2x^{2})$
            \end{proof}
            \begin{problem}
            Compute the derivative of $y(t)=t\tan^{-1}(t)-\frac{1}{2}\ln(t)$
            \end{problem}
            \begin{proof}[Solution]
            $\frac{dy}{dt}(t)=\frac{d}{dt}(t\tan^{-1}(t))-\frac{1}{2}\frac{d}{dt}(\ln(t))=\tan^{-1}(t)+t\frac{d}{dt}(\tan^{-1}(t))-\frac{1}{2t}=\tan^{-1}(t)+\frac{t}{1+t^{2}}-\frac{1}{2t}$
            \end{proof}
            \begin{problem}
            Compute the derivative of $f(x)=xe^{\sqrt[3]{x}}$
            \end{problem}
            \begin{proof}[Solution]
            $\frac{df}{dx}(x)=\frac{d}{dx}(xe^{\sqrt[3]{x}})=e^{\sqrt[3]{x}}+xe^{\sqrt[3]{x}}\frac{d}{dx}(\sqrt[3]{x})=e^{\sqrt[3]{x}}+\frac{1}{3}\sqrt[3]{x}e^{\sqrt[3]{x}}$
            \end{proof}
            \begin{problem}
            Compute the derivative of $g(x)=\ln(\csc(x^{2}))$
            \end{problem}
            \begin{proof}[Solution]
            $\frac{d}{dx}(\ln(\csc(x^{2})))=\frac{1}{\csc(x^{2})}\frac{d}{dx}(\csc(x^{2}))=-\sin(x^{2})\cot(x^{2})\csc(x^{2})\frac{d}{dx}(x^{2})=-2x\cot(x^{2})$
            \end{proof}
            \begin{problem}
            Compute the derivative of $s(t)=\sin(\tan(t))$
            \end{problem}
            \begin{proof}[Solution]
            $\frac{ds}{dt}(t)=\frac{d}{dt}(\sin(\tan(t)))=\cos(\tan(t))\frac{d}{dt}(\tan(t))=\cos(\tan(t))\sec^{2}(t)$
            \end{proof}
            \begin{problem}
            Compute the derivative of $f(x)=e^{\tan(x)}$
            \end{problem}
            \begin{proof}[Solution]
            $\frac{df}{dx}(x)=\frac{d}{dx}(e^{\tan(x)})=e^{\tan(x)}\frac{d}{dx}(\tan(x))=e^{\tan(x)}\sec^{2}(x)$
            \end{proof}
            \begin{problem}
            Compute the derivative of $g(x)=\ln(\sec(x))$
            \end{problem}
            \begin{proof}[Solution]
            $\frac{dg}{dx}(x)=\frac{d}{dx}(\ln(\sec(x)))=\frac{1}{\sec(x)}\frac{d}{dx}(\sec(x))=\tan(x)$
            \end{proof}
            \begin{problem}
            Compute the derivative of $s(t)=\cos(\sqrt{1+t^{2}})$
            \end{problem}
            \begin{proof}[Solution]
            $\frac{ds}{dt}(t)=\frac{d}{dt}(\cos(\sqrt{1+t^{2}}))=-\sin(\sqrt{1+t^{2}})\frac{d}{dt}(\sqrt{1+t^{2}})=-\frac{t}{\sqrt{1+t^{2}}}\sin(\sqrt{1+t^{2}})$
            \end{proof}
            \begin{problem}
            Compute the second derivative of $y(x)=\sin^{-1}(2x)$
            \end{problem}
            \begin{proof}[Solution]
            The second derivative is $\frac{d^{2}y}{dx^{2}}(x)=\frac{d}{dx}(\frac{dy}{dx}(x))$. So, we have:
            \begin{align*}
                \frac{d^{2}y}{dx^{2}}(x)&=\frac{d}{dx}\big(\frac{dy}{dx}\big)=\frac{d}{dx}\big(\frac{d}{dx}(\sin^{-1}(2x))\big)=\frac{d}{dx}\big(\frac{1}{\sqrt{1-4x^{2}}}\frac{d}{dx}(2x)\big)=\frac{d}{dx}\big(\frac{2}{\sqrt{1-4x^{2}}}\big)\\
                &=2\frac{d}{dx}((1-4x^{2})^{-\frac{1}{2}})=-(1-4x^{2})^{-\frac{3}{2}}\frac{d}{dx}(1-4x^{2})=8x(1-4x^{2})^{-\frac{3}{2}}=\frac{8x}{(1-4x^{2})^{\frac{3}{2}}}
            \end{align*}
            \end{proof}
            \begin{problem}
            Let $f(x)=-3x^{4}+4x^{3}$
            \begin{enumerate}
                \item Find the critical points of $f$.
                \item Find the intervals for which $f$ is increasing and decreasing.
                \item Find possible inflection points of $f$.
                \item Find the intervals on which the function is concave up and down.
            \end{enumerate}
            \end{problem}
            \begin{proof}[Solution]
            \
            \begin{enumerate}
                \item Critical points occur when $f'(x)=0$. We have $f'(x)=-12x^{3}+12x^{2}$. Solving for $f'(x)=0$, we have $-12x^{3}+12x^{2}=0 \Rightarrow x^{2}(1-x)=0$. This occurs only when $x=0$ or $x=1$. 
                \item $f$ is increasing when $f'(x)>0$. Solving for this, we have $x^{2}(1-x)>0\Rightarrow x<1$. $f$ is decreasing when $f'(x)<0$, which occurs when $x>1$.
                \item Possible inflection points occur when $f''(x)=0$. Solving for this, we have $f''(x) = -36x^{2}+24x = -12x^{2}(3x-2)$. So $f''(x)=0$ when $x=0$ and when $x=\frac{2}{3}$.
                \item $f$ is concave up when $f''(x)>0$ and concave down when $f''(x)<0$. If $x>0$, then $-12x<0$, so $-12x(3x-2)>0\Rightarrow 3x-2<0$. So $0<x<\frac{2}{3}$. If $x<0$, then $-12x>0$ and $3x-2<0$, and therefore $-12x(3x-2)<0$. If $x>\frac{2}{3}$, then $(3x-2)>0$ and $-12x<0$, so $-12x(3x-2)<0$. Thus, $f$ is concave up on $(0,\frac{2}{3})$ and concanve down on $(-\infty,0)\cup(\frac{2}{3},\infty)$
            \end{enumerate}
            \end{proof}
            \begin{problem}
            Find the equation of the line tangent to the graph of $x+\sqrt{xy}=6$ at the point $(4,1)$.
            \end{problem}
            \begin{proof}[Solution]
            The tangent line is $y=m(x-x_{0})+y_{0}$, where $m$ is the derivative of the graph at $(x_{0},y_{0})$. We have $x_{0}=4$ and $y_{0}=1$. From $\sqrt{xy}=6-x$, and $y=\frac{(6-x)^{2}}{x}$, we have:
            \begin{equation*}
                \frac{d}{dx}(x+\sqrt{xy})=\frac{d}{dx}(6)=0\Rightarrow 1+\frac{1}{2\sqrt{xy}}(y+x\frac{dy}{dx})=0\Rightarrow\frac{dy}{dx}=-\frac{2\sqrt{xy}}{x}-\frac{y}{x}=-\frac{2(6-x)}{x}-\frac{(6-x)^{2}}{x^{2}}
            \end{equation*}
            Evaluating at $x=4$, we get $m=-\frac{5}{4}$. The line is $y=-\frac{5}{4}(x-4)+1$
            \end{proof}
            \begin{problem}
            Compute the second derivative of $y(x)x^{2}\cos(-x)$.
            \end{problem}
            \begin{proof}[Solution]
            First, $\cos(-x)=\cos(x)$, so we can simplify. We have:
            \begin{equation*}
                \frac{dy}{dx}(x)=\frac{d}{dx}(x^{2}\cos(x))=2x\cos(x)-x^{2}\sin(x)\Rightarrow\frac{d^{2}y}{dx^{2}}=2\cos(x)-2x\sin(x)-2x\sin(x)-x^{2}\cos(x)
            \end{equation*}
            Simplifying, we have $\frac{d^{2}y}{dx^{2}}=(2-x^{2})\cos(x)-4x\sin(x)$
            \end{proof}
            \begin{problem}
            Find the extreme values (Global and Relative) of $f(x)=4x^{2}-4x+1$ for $0\leq x\leq 1$.
            \end{problem}
            \begin{proof}[Solution]
            Extreme values occur either at endpoints, or when $f'(x)=0$. $f'(x)=8x-4$. So $f'(x)=0$ when $x=\frac{1}{2}$. At this point, $f(\frac{1}{2})=0$. At the endpoints, $f(0)=1$ and $f(1)=1$, so $x=\frac{1}{2}$ is a global minimum. $x=0$ and $x=1$ are relative maxmima.
            \end{proof}
            \begin{problem}
            Compute the derivative of $g(t)=(t^{3}-2)^{\tan(t)}$
            \end{problem}
            \begin{proof}[Solution]
            Let $y(t)=\ln(g(t))$. Then $\frac{dy}{dt}(t)=\frac{d}{dt}(\tan(t)\ln(t^{3}-2))=\sec^{2}(t)\ln(t^{3}-2)+\tan(t)\frac{3t^{2}}{t^{3}-2}$. But $\frac{dy}{dt}(t)=\frac{d}{dt}(\ln(g(t)))=\frac{1}{g(t)}\frac{dg}{dt}(t)$. So $\frac{dg}{dt}(t)=(t^{3}-2)^{\tan(t)}(\sec^{2}(t)\ln(t^{3}-2)+\tan(t)\frac{3t^{2}}{t^{3}-2})$
            \end{proof}
\end{document}