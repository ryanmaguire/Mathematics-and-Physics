\documentclass[crop=false,class=book,oneside]{standalone}
%----------------------------Preamble-------------------------------%
%---------------------------Packages----------------------------%
\usepackage{geometry}
\geometry{b5paper, margin=1.0in}
\usepackage[T1]{fontenc}
\usepackage{graphicx, float}            % Graphics/Images.
\usepackage{natbib}                     % For bibliographies.
\bibliographystyle{agsm}                % Bibliography style.
\usepackage[french, english]{babel}     % Language typesetting.
\usepackage[dvipsnames]{xcolor}         % Color names.
\usepackage{listings}                   % Verbatim-Like Tools.
\usepackage{mathtools, esint, mathrsfs} % amsmath and integrals.
\usepackage{amsthm, amsfonts, amssymb}  % Fonts and theorems.
\usepackage{tcolorbox}                  % Frames around theorems.
\usepackage{upgreek}                    % Non-Italic Greek.
\usepackage{fmtcount, etoolbox}         % For the \book{} command.
\usepackage[newparttoc]{titlesec}       % Formatting chapter, etc.
\usepackage{titletoc}                   % Allows \book in toc.
\usepackage[nottoc]{tocbibind}          % Bibliography in toc.
\usepackage[titles]{tocloft}            % ToC formatting.
\usepackage{pgfplots, tikz}             % Drawing/graphing tools.
\usepackage{imakeidx}                   % Used for index.
\usetikzlibrary{
    calc,                   % Calculating right angles and more.
    angles,                 % Drawing angles within triangles.
    arrows.meta,            % Latex and Stealth arrows.
    quotes,                 % Adding labels to angles.
    positioning,            % Relative positioning of nodes.
    decorations.markings,   % Adding arrows in the middle of a line.
    patterns,
    arrows
}                                       % Libraries for tikz.
\pgfplotsset{compat=1.9}                % Version of pgfplots.
\usepackage[font=scriptsize,
            labelformat=simple,
            labelsep=colon]{subcaption} % Subfigure captions.
\usepackage[font={scriptsize},
            hypcap=true,
            labelsep=colon]{caption}    % Figure captions.
\usepackage[pdftex,
            pdfauthor={Ryan Maguire},
            pdftitle={Mathematics and Physics},
            pdfsubject={Mathematics, Physics, Science},
            pdfkeywords={Mathematics, Physics, Computer Science, Biology},
            pdfproducer={LaTeX},
            pdfcreator={pdflatex}]{hyperref}
\hypersetup{
    colorlinks=true,
    linkcolor=blue,
    filecolor=magenta,
    urlcolor=Cerulean,
    citecolor=SkyBlue
}                           % Colors for hyperref.
\usepackage[toc,acronym,nogroupskip,nopostdot]{glossaries}
\usepackage{glossary-mcols}
%------------------------Theorem Styles-------------------------%
\theoremstyle{plain}
\newtheorem{theorem}{Theorem}[section]

% Define theorem style for default spacing and normal font.
\newtheoremstyle{normal}
    {\topsep}               % Amount of space above the theorem.
    {\topsep}               % Amount of space below the theorem.
    {}                      % Font used for body of theorem.
    {}                      % Measure of space to indent.
    {\bfseries}             % Font of the header of the theorem.
    {}                      % Punctuation between head and body.
    {.5em}                  % Space after theorem head.
    {}

% Italic header environment.
\newtheoremstyle{thmit}{\topsep}{\topsep}{}{}{\itshape}{}{0.5em}{}

% Define environments with italic headers.
\theoremstyle{thmit}
\newtheorem*{solution}{Solution}

% Define default environments.
\theoremstyle{normal}
\newtheorem{example}{Example}[section]
\newtheorem{definition}{Definition}[section]
\newtheorem{problem}{Problem}[section]

% Define framed environment.
\tcbuselibrary{most}
\newtcbtheorem[use counter*=theorem]{ftheorem}{Theorem}{%
    before=\par\vspace{2ex},
    boxsep=0.5\topsep,
    after=\par\vspace{2ex},
    colback=green!5,
    colframe=green!35!black,
    fonttitle=\bfseries\upshape%
}{thm}

\newtcbtheorem[auto counter, number within=section]{faxiom}{Axiom}{%
    before=\par\vspace{2ex},
    boxsep=0.5\topsep,
    after=\par\vspace{2ex},
    colback=Apricot!5,
    colframe=Apricot!35!black,
    fonttitle=\bfseries\upshape%
}{ax}

\newtcbtheorem[use counter*=definition]{fdefinition}{Definition}{%
    before=\par\vspace{2ex},
    boxsep=0.5\topsep,
    after=\par\vspace{2ex},
    colback=blue!5!white,
    colframe=blue!75!black,
    fonttitle=\bfseries\upshape%
}{def}

\newtcbtheorem[use counter*=example]{fexample}{Example}{%
    before=\par\vspace{2ex},
    boxsep=0.5\topsep,
    after=\par\vspace{2ex},
    colback=red!5!white,
    colframe=red!75!black,
    fonttitle=\bfseries\upshape%
}{ex}

\newtcbtheorem[auto counter, number within=section]{fnotation}{Notation}{%
    before=\par\vspace{2ex},
    boxsep=0.5\topsep,
    after=\par\vspace{2ex},
    colback=SeaGreen!5!white,
    colframe=SeaGreen!75!black,
    fonttitle=\bfseries\upshape%
}{not}

\newtcbtheorem[use counter*=remark]{fremark}{Remark}{%
    fonttitle=\bfseries\upshape,
    colback=Goldenrod!5!white,
    colframe=Goldenrod!75!black}{ex}

\newenvironment{bproof}{\textit{Proof.}}{\hfill$\square$}
\tcolorboxenvironment{bproof}{%
    blanker,
    breakable,
    left=3mm,
    before skip=5pt,
    after skip=10pt,
    borderline west={0.6mm}{0pt}{green!80!black}
}

\AtEndEnvironment{lexample}{$\hfill\textcolor{red}{\blacksquare}$}
\newtcbtheorem[use counter*=example]{lexample}{Example}{%
    empty,
    title={Example~\theexample},
    boxed title style={%
        empty,
        size=minimal,
        toprule=2pt,
        top=0.5\topsep,
    },
    coltitle=red,
    fonttitle=\bfseries,
    parbox=false,
    boxsep=0pt,
    before=\par\vspace{2ex},
    left=0pt,
    right=0pt,
    top=3ex,
    bottom=1ex,
    before=\par\vspace{2ex},
    after=\par\vspace{2ex},
    breakable,
    pad at break*=0mm,
    vfill before first,
    overlay unbroken={%
        \draw[red, line width=2pt]
            ([yshift=-1.2ex]title.south-|frame.west) to
            ([yshift=-1.2ex]title.south-|frame.east);
        },
    overlay first={%
        \draw[red, line width=2pt]
            ([yshift=-1.2ex]title.south-|frame.west) to
            ([yshift=-1.2ex]title.south-|frame.east);
    },
}{ex}

\AtEndEnvironment{ldefinition}{$\hfill\textcolor{Blue}{\blacksquare}$}
\newtcbtheorem[use counter*=definition]{ldefinition}{Definition}{%
    empty,
    title={Definition~\thedefinition:~{#1}},
    boxed title style={%
        empty,
        size=minimal,
        toprule=2pt,
        top=0.5\topsep,
    },
    coltitle=Blue,
    fonttitle=\bfseries,
    parbox=false,
    boxsep=0pt,
    before=\par\vspace{2ex},
    left=0pt,
    right=0pt,
    top=3ex,
    bottom=0pt,
    before=\par\vspace{2ex},
    after=\par\vspace{1ex},
    breakable,
    pad at break*=0mm,
    vfill before first,
    overlay unbroken={%
        \draw[Blue, line width=2pt]
            ([yshift=-1.2ex]title.south-|frame.west) to
            ([yshift=-1.2ex]title.south-|frame.east);
        },
    overlay first={%
        \draw[Blue, line width=2pt]
            ([yshift=-1.2ex]title.south-|frame.west) to
            ([yshift=-1.2ex]title.south-|frame.east);
    },
}{def}

\AtEndEnvironment{ltheorem}{$\hfill\textcolor{Green}{\blacksquare}$}
\newtcbtheorem[use counter*=theorem]{ltheorem}{Theorem}{%
    empty,
    title={Theorem~\thetheorem:~{#1}},
    boxed title style={%
        empty,
        size=minimal,
        toprule=2pt,
        top=0.5\topsep,
    },
    coltitle=Green,
    fonttitle=\bfseries,
    parbox=false,
    boxsep=0pt,
    before=\par\vspace{2ex},
    left=0pt,
    right=0pt,
    top=3ex,
    bottom=-1.5ex,
    breakable,
    pad at break*=0mm,
    vfill before first,
    overlay unbroken={%
        \draw[Green, line width=2pt]
            ([yshift=-1.2ex]title.south-|frame.west) to
            ([yshift=-1.2ex]title.south-|frame.east);},
    overlay first={%
        \draw[Green, line width=2pt]
            ([yshift=-1.2ex]title.south-|frame.west) to
            ([yshift=-1.2ex]title.south-|frame.east);
    }
}{thm}

%--------------------Declared Math Operators--------------------%
\DeclareMathOperator{\adjoint}{adj}         % Adjoint.
\DeclareMathOperator{\Card}{Card}           % Cardinality.
\DeclareMathOperator{\curl}{curl}           % Curl.
\DeclareMathOperator{\diam}{diam}           % Diameter.
\DeclareMathOperator{\dist}{dist}           % Distance.
\DeclareMathOperator{\Div}{div}             % Divergence.
\DeclareMathOperator{\Erf}{Erf}             % Error Function.
\DeclareMathOperator{\Erfc}{Erfc}           % Complementary Error Function.
\DeclareMathOperator{\Ext}{Ext}             % Exterior.
\DeclareMathOperator{\GCD}{GCD}             % Greatest common denominator.
\DeclareMathOperator{\grad}{grad}           % Gradient
\DeclareMathOperator{\Ima}{Im}              % Image.
\DeclareMathOperator{\Int}{Int}             % Interior.
\DeclareMathOperator{\LC}{LC}               % Leading coefficient.
\DeclareMathOperator{\LCM}{LCM}             % Least common multiple.
\DeclareMathOperator{\LM}{LM}               % Leading monomial.
\DeclareMathOperator{\LT}{LT}               % Leading term.
\DeclareMathOperator{\Mod}{mod}             % Modulus.
\DeclareMathOperator{\Mon}{Mon}             % Monomial.
\DeclareMathOperator{\multideg}{mutlideg}   % Multi-Degree (Graphs).
\DeclareMathOperator{\nul}{nul}             % Null space of operator.
\DeclareMathOperator{\Ord}{Ord}             % Ordinal of ordered set.
\DeclareMathOperator{\Prin}{Prin}           % Principal value.
\DeclareMathOperator{\proj}{proj}           % Projection.
\DeclareMathOperator{\Refl}{Refl}           % Reflection operator.
\DeclareMathOperator{\rk}{rk}               % Rank of operator.
\DeclareMathOperator{\sgn}{sgn}             % Sign of a number.
\DeclareMathOperator{\sinc}{sinc}           % Sinc function.
\DeclareMathOperator{\Span}{Span}           % Span of a set.
\DeclareMathOperator{\Spec}{Spec}           % Spectrum.
\DeclareMathOperator{\supp}{supp}           % Support
\DeclareMathOperator{\Tr}{Tr}               % Trace of matrix.
%--------------------Declared Math Symbols--------------------%
\DeclareMathSymbol{\minus}{\mathbin}{AMSa}{"39} % Unary minus sign.
%------------------------New Commands---------------------------%
\DeclarePairedDelimiter\norm{\lVert}{\rVert}
\DeclarePairedDelimiter\ceil{\lceil}{\rceil}
\DeclarePairedDelimiter\floor{\lfloor}{\rfloor}
\newcommand*\diff{\mathop{}\!\mathrm{d}}
\newcommand*\Diff[1]{\mathop{}\!\mathrm{d^#1}}
\renewcommand*{\glstextformat}[1]{\textcolor{RoyalBlue}{#1}}
\renewcommand{\glsnamefont}[1]{\textbf{#1}}
\renewcommand\labelitemii{$\circ$}
\renewcommand\thesubfigure{%
    \arabic{chapter}.\arabic{figure}.\arabic{subfigure}}
\addto\captionsenglish{\renewcommand{\figurename}{Fig.}}
\numberwithin{equation}{section}

\renewcommand{\vector}[1]{\boldsymbol{\mathrm{#1}}}

\newcommand{\uvector}[1]{\boldsymbol{\hat{\mathrm{#1}}}}
\newcommand{\topspace}[2][]{(#2,\tau_{#1})}
\newcommand{\measurespace}[2][]{(#2,\varSigma_{#1},\mu_{#1})}
\newcommand{\measurablespace}[2][]{(#2,\varSigma_{#1})}
\newcommand{\manifold}[2][]{(#2,\tau_{#1},\mathcal{A}_{#1})}
\newcommand{\tanspace}[2]{T_{#1}{#2}}
\newcommand{\cotanspace}[2]{T_{#1}^{*}{#2}}
\newcommand{\Ckspace}[3][\mathbb{R}]{C^{#2}(#3,#1)}
\newcommand{\funcspace}[2][\mathbb{R}]{\mathcal{F}(#2,#1)}
\newcommand{\smoothvecf}[1]{\mathfrak{X}(#1)}
\newcommand{\smoothonef}[1]{\mathfrak{X}^{*}(#1)}
\newcommand{\bracket}[2]{[#1,#2]}

%------------------------Book Command---------------------------%
\makeatletter
\renewcommand\@pnumwidth{1cm}
\newcounter{book}
\renewcommand\thebook{\@Roman\c@book}
\newcommand\book{%
    \if@openright
        \cleardoublepage
    \else
        \clearpage
    \fi
    \thispagestyle{plain}%
    \if@twocolumn
        \onecolumn
        \@tempswatrue
    \else
        \@tempswafalse
    \fi
    \null\vfil
    \secdef\@book\@sbook
}
\def\@book[#1]#2{%
    \refstepcounter{book}
    \addcontentsline{toc}{book}{\bookname\ \thebook:\hspace{1em}#1}
    \markboth{}{}
    {\centering
     \interlinepenalty\@M
     \normalfont
     \huge\bfseries\bookname\nobreakspace\thebook
     \par
     \vskip 20\p@
     \Huge\bfseries#2\par}%
    \@endbook}
\def\@sbook#1{%
    {\centering
     \interlinepenalty \@M
     \normalfont
     \Huge\bfseries#1\par}%
    \@endbook}
\def\@endbook{
    \vfil\newpage
        \if@twoside
            \if@openright
                \null
                \thispagestyle{empty}%
                \newpage
            \fi
        \fi
        \if@tempswa
            \twocolumn
        \fi
}
\newcommand*\l@book[2]{%
    \ifnum\c@tocdepth >-3\relax
        \addpenalty{-\@highpenalty}%
        \addvspace{2.25em\@plus\p@}%
        \setlength\@tempdima{3em}%
        \begingroup
            \parindent\z@\rightskip\@pnumwidth
            \parfillskip -\@pnumwidth
            {
                \leavevmode
                \Large\bfseries#1\hfill\hb@xt@\@pnumwidth{\hss#2}
            }
            \par
            \nobreak
            \global\@nobreaktrue
            \everypar{\global\@nobreakfalse\everypar{}}%
        \endgroup
    \fi}
\newcommand\bookname{Book}
\renewcommand{\thebook}{\texorpdfstring{\Numberstring{book}}{book}}
\providecommand*{\toclevel@book}{-2}
\makeatother
\titleformat{\part}[display]
    {\Large\bfseries}
    {\partname\nobreakspace\thepart}
    {0mm}
    {\Huge\bfseries}
\titlecontents{part}[0pt]
    {\large\bfseries}
    {\partname\ \thecontentslabel: \quad}
    {}
    {\hfill\contentspage}
\titlecontents{chapter}[0pt]
    {\bfseries}
    {\chaptername\ \thecontentslabel:\quad}
    {}
    {\hfill\contentspage}
\newglossarystyle{longpara}{%
    \setglossarystyle{long}%
    \renewenvironment{theglossary}{%
        \begin{longtable}[l]{{p{0.25\hsize}p{0.65\hsize}}}
    }{\end{longtable}}%
    \renewcommand{\glossentry}[2]{%
        \glstarget{##1}{\glossentryname{##1}}%
        &\glossentrydesc{##1}{~##2.}
        \tabularnewline%
        \tabularnewline
    }%
}
\newglossary[not-glg]{notation}{not-gls}{not-glo}{Notation}
\newcommand*{\newnotation}[4][]{%
    \newglossaryentry{#2}{type=notation, name={\textbf{#3}, },
                          text={#4}, description={#4},#1}%
}
%--------------------------LENGTHS------------------------------%
% Spacings for the Table of Contents.
\addtolength{\cftsecnumwidth}{1ex}
\addtolength{\cftsubsecindent}{1ex}
\addtolength{\cftsubsecnumwidth}{1ex}
\addtolength{\cftfignumwidth}{1ex}
\addtolength{\cfttabnumwidth}{1ex}

% Indent and paragraph spacing.
\setlength{\parindent}{0em}
\setlength{\parskip}{0em}
%----------------------------GLOSSARY-------------------------------%
\makeglossaries
\loadglsentries{../../glossary}
\loadglsentries{../../acronym}
%--------------------------Main Document----------------------------%
\begin{document}
    \ifx\ifmathcourses\undefined
        \pagenumbering{roman}
        \title{Special Functions}
        \author{Ryan Maguire}
        \date{\vspace{-5ex}}
        \maketitle
        \tableofcontents
        \setcounter{chapter}{6}
        \chapter{Special Functions}
        \pagenumbering{arabic}
    \else
        \chapter{Special Functions}
    \fi
    \subsection{Coordinate Geometry}
        Coordinate geometry, or Cartesian geometry, is
        the study of geometry using basic notion from
        elementary algebra. The Cartesian plane consists of
        two perpendicular lines, the $x$ axis and the $y$
        axis. The intersection of these lines is called the
        origin. This is denoted $(0,0)$. Every point to the
        right of the origin corresponds to a positive value
        $x$, and every value to the left corresponds to a
        negative value. Similarly, every value about the
        origin corresponds to a positive value $y$ and
        every value below corresponds to a negative value
        $y$. Each point in the plane is identified by the
        ordered pair $(x,y)$.
        \begin{definition}
            The abscissa of a coordinate $(x,y)$ in the
            Cartesian plane is the value $x$.
        \end{definition}
        \begin{definition}
            The ordinate of a coordinate $(x,y)$ in the
            Cartesian plane is the value $y$.
        \end{definition}
        The plane is divided into four quadrants. The
        First Quadrant is the set of all points
        $(x,y)$ such that $x$ and $y$ are positive.
        The second, third, and fourth quadrants are then
        labelled in counter-clockwise order around the
        origin.
        \begin{definition}
            The distance between
            $(x_{1},y_{1})$ and $(x_{2},y_{2})$ is
            $\sqrt{(x_{2}-x_{1})^{2}+(y_{2}-y_{1})^{2}}$
        \end{definition}
        This distance formula comes from
        Pythagoras' Theorem. Given two points in the
        Cartesian plane, form a right triangle by
        drawing lines perpendicular to the $x$ and
        $y$ axis that contain these two points. The
        height is then $y_{2}-y_{1}$ and the length
        is $x_{2}-x_{1}$.
        \begin{theorem}
            The midpoint between $(x_{1},y_{1})$
            and $(x_{2},y_{2})$, that is the point
            whose distance to either point is equal
            and lies on the line containing these two
            points, is
            $(\frac{x_{1}+x_{2}}{2},\frac{y_{1}+y_{2}}{2})$.
        \end{theorem}
        \begin{theorem}
            If $A=(x_{0},y_{0})$ is a point in the Cartesian plane,
            and if $\ell$ is a line defined by
            $ax+by+c=0$, then the minimum distance between $P$
            and $\ell$ is:
            \begin{equation*}
                d=\Big|
                    \frac{ax_{0}+by_{0}+c}{\sqrt{a^{2}+b^{2}}}
                \Big|
            \end{equation*}
        \end{theorem}
        \begin{definition}
            A vertical line is a line $\ell$ such that
            $ay+b=0$ for all $(x,y)$ that lie on $\ell$.
        \end{definition}
        \begin{definition}
            The slope of a non-vertical line with points
            $(x_{1},y_{2})$ and $(x_{2},y_{2})$ is:
            \begin{equation*}
                m=\frac{y_{2}-y_{1}}{x_{2}-x_{1}}
            \end{equation*}
        \end{definition}
        \begin{definition}
            An intercept of a line $ax+by+c=0$ is point on the
            line such that either $x=0$ or $y=0$. If $x=0$ this
            is called an $x$ intercept, and if $y=0$ this is
            called a $y$ intercept.
        \end{definition}
        \begin{theorem}
            Two non-vertical lines are perpendicular if and only if
            $m_{1}=-1/m_{2}$.
        \end{theorem}
        \begin{theorem}
            Two non-vertical lines are parallel if and only if
            $m_{1}=m_{2}$.
        \end{theorem}
        \begin{theorem}
            If $\ell$ is a line that passes through the
            origin with slope $m$, then $y=mx$.
        \end{theorem}
        \begin{theorem}
            If $\ell$ is a line with $y$ intercept
            $b$ and slope $m$, then
            $y=mx+b$.
        \end{theorem}
        \begin{theorem}
            If $\ell$ is a line with $x$ intercept
            $a$ and $y$ intercept $b$, then
            $x/a+y/b=1$.
        \end{theorem}
        \begin{theorem}
            If $\ell$ is a line containing
            $(x_{1},y_{1})$ and $(x_{2},y_{2})$, then:
            \begin{equation*}
                y=\frac{x-x_{1}}{x_{2}-x_{1}}(y_{2}-y_{1})+y_{1}
            \end{equation*}
        \end{theorem}
        \begin{theorem}
            If $\ell$ is a line containing $(x_{0},y_{0})$
            with slope $m$, then $y=m(x-x_{0})+y_{0}$.
        \end{theorem}
        A locus is a set of points satifsying a certain contiditon.
        For example, the locus of points that are a fixed distance
        $r$ away from the point $P$ is the circle of radius $r$
        centered at $P$. The locus of points that are equidistant
        from two line that intersect at an angle is the
        angle bisector.
        \begin{definition}
            A parabola is the locus, or set of all points, such that
            the distance to a fixed point (Called the focus)
            is equal to the distance to a fixed line
            (Called the directrix).
        \end{definition}
        \begin{definition}
            An ellipse is the locus of points such that
            the sum of the distances to two other points
            (Called the foci) are equal. If the two foci
            are the same, then we have a circle.
        \end{definition}
        \begin{definition}
            A hyperbola is the locus of points such that
            the difference of the distance between two
            other points (Called the foci) is constant.
        \end{definition}
        \begin{theorem}
            The equation of a circle centered at
            $(x_{0},y_{0})$ or radius $r$ is:
            $(x-x_{0})^{2}+(y-y_{0})^{2}=r^{2}$.
        \end{theorem}
        \begin{theorem}
            The equation of a parabola with vertex
            $V=(x_{0},y_{0})$ and directrix $d$ such that
            the signed distance from $(x_{0},y_{0})$ to
            $d$ is $p$ is $(y-y_{0})^{2}=4p(x-x_{0})$.
        \end{theorem}
        \begin{theorem}
            An ellipse centered at $(x_{0},y_{0})$
            has an equation of the form:
            \begin{equation*}
                \frac{(x-x_{0})^{2}}{a^{2}}+
                \frac{(y-y_{0})^{2}}{b^{2}}
                =1
            \end{equation*}
        \end{theorem}
        \begin{definition}
            The eccentricity of an ellipse is:
            \begin{equation*}
                \varepsilon=\sqrt{1-\frac{b^{2}}{a^{2}}}
            \end{equation*}
        \end{definition}
        \begin{theorem}
            A hyperbola centered at $(x_{0},y_{0})$
            has an equation of the form:
            \begin{equation*}
                \frac{(x-x_{0})^{2}}{a^{2}}-
                \frac{(y-y_{0})^{2}}{b^{2}}
                =1
            \end{equation*}
        \end{theorem}
    \subsection{Trigonometry}
        \begin{definition}
            Given a right angle triangle with height $y$,
            width $x$, hypotenuse $r$, and given the
            angle $\theta$ which is oppositive to the heigh
            and adjacent to the width of the triangle, the
            following functions are defined:
            \begin{align*}
                \sin(\theta)&=\frac{y}{r}
                &
                \cos(\theta)&=\frac{x}{r}
                &
                \tan(\theta)&=\frac{y}{x}
            \end{align*}
        \end{definition}
        \begin{example}
            In radians, we have the following:
            \begin{enumerate}
                \begin{multicols}{4}
                    \item $\sin(0)=0$
                    \item $\cos(0)=1$
                    \item $\tan(0)=0$
                    \item $\sin(\frac{\pi}{2})=1$
                    \item $\cos(\frac{\pi}{2})=0$
                    \item $\tan(\frac{\pi}{2})=\infty$
                    \item $\sin(\frac{\pi}{6})=\frac{1}{2}$
                    \item $\cos(\frac{\pi}{6})=\frac{\sqrt{3}}{2}$
                    \item $\tan(\frac{\pi}{6})=\frac{1}{\sqrt{3}}$
                    \item $\sin(\frac{\pi}{4})=\frac{1}{\sqrt{2}}$
                    \item $\cos(\frac{\pi}{4})=\frac{1}{\sqrt{2}}$
                    \item $\tan(\frac{\pi}{4})=1$
                \end{multicols}
            \end{enumerate}
        \end{example}
        \begin{definition}
            The reciprocals of the trigonometric functions are:
            \begin{align*}
                \sec(\theta)&=\frac{1}{\cos(\theta)}
                &
                \csc(\theta)&=\frac{1}{\sin(\theta)}
                &
                \cot(\theta)&=\frac{1}{\tan(\theta)}
            \end{align*}
        \end{definition}
        \begin{theorem}
            The following are true:
            \begin{align*}
                \sin(\theta)\csc(\theta)&=1
                &
                \cos(\theta)\sec(\theta)&=1\\
                \tan(\theta)\cot(\theta)&=1
                &
                \sin^{2}(\theta)+\cos^{2}(\theta)&=1\\
                \tan(\theta)&=\frac{\sin(\theta)}{\cos(\theta)}
                &
                \cot(\theta)&=\frac{\cos(\theta)}{\sin(\theta)}\\
                \sec^{2}(\theta)&=1+\tan^{2}(\theta)
                &
                \csc^{2}(\theta)&=1+\cot^{2}(\theta)\\
                \sin(a\pm{b})&=\sin(a)\cos(b)\pm\cos(a)\sin(b)
                &
                \cos(a\pm{b})&=\cos(a)\cos(b)\mp\sin(a)\sin(b)\\
                \tan(a\pm{b})
                &=\frac{\tan(a)\pm\tan(b)}{1\mp\tan(a)\tan(b)}
                &
                \cot(a\pm{b})
                &=\frac{\cot(a)\cot(b)\mp1}{\cot(b)\pm\cot(a)}\\
                \sin(2x)&=2\sin(x)\cos(x)
                &
                \sin(3x)&=3\sin(x)-4\sin^{3}(x)\\
                \sin(4x)&=8\cos^{3}(x)\sin(x)-4\cos(x)\sin(x)
                &
                \cos(2x)&=\cos^{2}(x)-\sin^{2}(x)\\
                \cos(3x)&=4\cos^{3}(x)-3\cos(x)
                &
                \cos(4x)&=8\cos^{4}(x)-8\cos^{2}(x)+1\\
                \sin(a)\pm\sin(b)
                &=2\sin(\frac{a\pm{b}}{2})\cos(\frac{a\mp{b}}{2})
                &
                \cos(a)+\cos(b)
                &=2\cos(\frac{a+b}{2})\cos(\frac{a-b}{2})\\
                \cos(a)-\cos(b)
                &=2\sin(\frac{a+b}{2})\sin(\frac{b-a}{2})
                &
                \tan(a)\pm\tan(b)
                &=\frac{\sin(a\pm{b})}{\cos(a)\cos(b)}\\
                \cot(a)\pm\cot(b)
                &=\frac{\sin(a\pm{b})}{\sin(a)\sin(b)}
                &
                \sin^{2}\Big(\frac{x}{2}\Big)
                &=\frac{1-\cos(x)}{2}\\
                \cos^{2}\Big(\frac{x}{2}\Big)
                &=\frac{1+\cos(x)}{2}
                &
                \tan\Big(\frac{a\pm{b}}{2}\Big)
                &=\frac{\sin(a)\pm\sin(b)}{\cos(a)+\cos(b)}
            \end{align*}
        \end{theorem}
\end{document}