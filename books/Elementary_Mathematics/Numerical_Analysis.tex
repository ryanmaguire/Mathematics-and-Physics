\documentclass[crop=false,class=book,oneside]{standalone}
%----------------------------Preamble-------------------------------%
%---------------------------Packages----------------------------%
\usepackage{geometry}
\geometry{b5paper, margin=1.0in}
\usepackage[T1]{fontenc}
\usepackage{graphicx, float}            % Graphics/Images.
\usepackage{natbib}                     % For bibliographies.
\bibliographystyle{agsm}                % Bibliography style.
\usepackage[french, english]{babel}     % Language typesetting.
\usepackage[dvipsnames]{xcolor}         % Color names.
\usepackage{listings}                   % Verbatim-Like Tools.
\usepackage{mathtools, esint, mathrsfs} % amsmath and integrals.
\usepackage{amsthm, amsfonts, amssymb}  % Fonts and theorems.
\usepackage{tcolorbox}                  % Frames around theorems.
\usepackage{upgreek}                    % Non-Italic Greek.
\usepackage{fmtcount, etoolbox}         % For the \book{} command.
\usepackage[newparttoc]{titlesec}       % Formatting chapter, etc.
\usepackage{titletoc}                   % Allows \book in toc.
\usepackage[nottoc]{tocbibind}          % Bibliography in toc.
\usepackage[titles]{tocloft}            % ToC formatting.
\usepackage{pgfplots, tikz}             % Drawing/graphing tools.
\usepackage{imakeidx}                   % Used for index.
\usetikzlibrary{
    calc,                   % Calculating right angles and more.
    angles,                 % Drawing angles within triangles.
    arrows.meta,            % Latex and Stealth arrows.
    quotes,                 % Adding labels to angles.
    positioning,            % Relative positioning of nodes.
    decorations.markings,   % Adding arrows in the middle of a line.
    patterns,
    arrows
}                                       % Libraries for tikz.
\pgfplotsset{compat=1.9}                % Version of pgfplots.
\usepackage[font=scriptsize,
            labelformat=simple,
            labelsep=colon]{subcaption} % Subfigure captions.
\usepackage[font={scriptsize},
            hypcap=true,
            labelsep=colon]{caption}    % Figure captions.
\usepackage[pdftex,
            pdfauthor={Ryan Maguire},
            pdftitle={Mathematics and Physics},
            pdfsubject={Mathematics, Physics, Science},
            pdfkeywords={Mathematics, Physics, Computer Science, Biology},
            pdfproducer={LaTeX},
            pdfcreator={pdflatex}]{hyperref}
\hypersetup{
    colorlinks=true,
    linkcolor=blue,
    filecolor=magenta,
    urlcolor=Cerulean,
    citecolor=SkyBlue
}                           % Colors for hyperref.
\usepackage[toc,acronym,nogroupskip,nopostdot]{glossaries}
\usepackage{glossary-mcols}
%------------------------Theorem Styles-------------------------%
\theoremstyle{plain}
\newtheorem{theorem}{Theorem}[section]

% Define theorem style for default spacing and normal font.
\newtheoremstyle{normal}
    {\topsep}               % Amount of space above the theorem.
    {\topsep}               % Amount of space below the theorem.
    {}                      % Font used for body of theorem.
    {}                      % Measure of space to indent.
    {\bfseries}             % Font of the header of the theorem.
    {}                      % Punctuation between head and body.
    {.5em}                  % Space after theorem head.
    {}

% Italic header environment.
\newtheoremstyle{thmit}{\topsep}{\topsep}{}{}{\itshape}{}{0.5em}{}

% Define environments with italic headers.
\theoremstyle{thmit}
\newtheorem*{solution}{Solution}

% Define default environments.
\theoremstyle{normal}
\newtheorem{example}{Example}[section]
\newtheorem{definition}{Definition}[section]
\newtheorem{problem}{Problem}[section]

% Define framed environment.
\tcbuselibrary{most}
\newtcbtheorem[use counter*=theorem]{ftheorem}{Theorem}{%
    before=\par\vspace{2ex},
    boxsep=0.5\topsep,
    after=\par\vspace{2ex},
    colback=green!5,
    colframe=green!35!black,
    fonttitle=\bfseries\upshape%
}{thm}

\newtcbtheorem[auto counter, number within=section]{faxiom}{Axiom}{%
    before=\par\vspace{2ex},
    boxsep=0.5\topsep,
    after=\par\vspace{2ex},
    colback=Apricot!5,
    colframe=Apricot!35!black,
    fonttitle=\bfseries\upshape%
}{ax}

\newtcbtheorem[use counter*=definition]{fdefinition}{Definition}{%
    before=\par\vspace{2ex},
    boxsep=0.5\topsep,
    after=\par\vspace{2ex},
    colback=blue!5!white,
    colframe=blue!75!black,
    fonttitle=\bfseries\upshape%
}{def}

\newtcbtheorem[use counter*=example]{fexample}{Example}{%
    before=\par\vspace{2ex},
    boxsep=0.5\topsep,
    after=\par\vspace{2ex},
    colback=red!5!white,
    colframe=red!75!black,
    fonttitle=\bfseries\upshape%
}{ex}

\newtcbtheorem[auto counter, number within=section]{fnotation}{Notation}{%
    before=\par\vspace{2ex},
    boxsep=0.5\topsep,
    after=\par\vspace{2ex},
    colback=SeaGreen!5!white,
    colframe=SeaGreen!75!black,
    fonttitle=\bfseries\upshape%
}{not}

\newtcbtheorem[use counter*=remark]{fremark}{Remark}{%
    fonttitle=\bfseries\upshape,
    colback=Goldenrod!5!white,
    colframe=Goldenrod!75!black}{ex}

\newenvironment{bproof}{\textit{Proof.}}{\hfill$\square$}
\tcolorboxenvironment{bproof}{%
    blanker,
    breakable,
    left=3mm,
    before skip=5pt,
    after skip=10pt,
    borderline west={0.6mm}{0pt}{green!80!black}
}

\AtEndEnvironment{lexample}{$\hfill\textcolor{red}{\blacksquare}$}
\newtcbtheorem[use counter*=example]{lexample}{Example}{%
    empty,
    title={Example~\theexample},
    boxed title style={%
        empty,
        size=minimal,
        toprule=2pt,
        top=0.5\topsep,
    },
    coltitle=red,
    fonttitle=\bfseries,
    parbox=false,
    boxsep=0pt,
    before=\par\vspace{2ex},
    left=0pt,
    right=0pt,
    top=3ex,
    bottom=1ex,
    before=\par\vspace{2ex},
    after=\par\vspace{2ex},
    breakable,
    pad at break*=0mm,
    vfill before first,
    overlay unbroken={%
        \draw[red, line width=2pt]
            ([yshift=-1.2ex]title.south-|frame.west) to
            ([yshift=-1.2ex]title.south-|frame.east);
        },
    overlay first={%
        \draw[red, line width=2pt]
            ([yshift=-1.2ex]title.south-|frame.west) to
            ([yshift=-1.2ex]title.south-|frame.east);
    },
}{ex}

\AtEndEnvironment{ldefinition}{$\hfill\textcolor{Blue}{\blacksquare}$}
\newtcbtheorem[use counter*=definition]{ldefinition}{Definition}{%
    empty,
    title={Definition~\thedefinition:~{#1}},
    boxed title style={%
        empty,
        size=minimal,
        toprule=2pt,
        top=0.5\topsep,
    },
    coltitle=Blue,
    fonttitle=\bfseries,
    parbox=false,
    boxsep=0pt,
    before=\par\vspace{2ex},
    left=0pt,
    right=0pt,
    top=3ex,
    bottom=0pt,
    before=\par\vspace{2ex},
    after=\par\vspace{1ex},
    breakable,
    pad at break*=0mm,
    vfill before first,
    overlay unbroken={%
        \draw[Blue, line width=2pt]
            ([yshift=-1.2ex]title.south-|frame.west) to
            ([yshift=-1.2ex]title.south-|frame.east);
        },
    overlay first={%
        \draw[Blue, line width=2pt]
            ([yshift=-1.2ex]title.south-|frame.west) to
            ([yshift=-1.2ex]title.south-|frame.east);
    },
}{def}

\AtEndEnvironment{ltheorem}{$\hfill\textcolor{Green}{\blacksquare}$}
\newtcbtheorem[use counter*=theorem]{ltheorem}{Theorem}{%
    empty,
    title={Theorem~\thetheorem:~{#1}},
    boxed title style={%
        empty,
        size=minimal,
        toprule=2pt,
        top=0.5\topsep,
    },
    coltitle=Green,
    fonttitle=\bfseries,
    parbox=false,
    boxsep=0pt,
    before=\par\vspace{2ex},
    left=0pt,
    right=0pt,
    top=3ex,
    bottom=-1.5ex,
    breakable,
    pad at break*=0mm,
    vfill before first,
    overlay unbroken={%
        \draw[Green, line width=2pt]
            ([yshift=-1.2ex]title.south-|frame.west) to
            ([yshift=-1.2ex]title.south-|frame.east);},
    overlay first={%
        \draw[Green, line width=2pt]
            ([yshift=-1.2ex]title.south-|frame.west) to
            ([yshift=-1.2ex]title.south-|frame.east);
    }
}{thm}

%--------------------Declared Math Operators--------------------%
\DeclareMathOperator{\adjoint}{adj}         % Adjoint.
\DeclareMathOperator{\Card}{Card}           % Cardinality.
\DeclareMathOperator{\curl}{curl}           % Curl.
\DeclareMathOperator{\diam}{diam}           % Diameter.
\DeclareMathOperator{\dist}{dist}           % Distance.
\DeclareMathOperator{\Div}{div}             % Divergence.
\DeclareMathOperator{\Erf}{Erf}             % Error Function.
\DeclareMathOperator{\Erfc}{Erfc}           % Complementary Error Function.
\DeclareMathOperator{\Ext}{Ext}             % Exterior.
\DeclareMathOperator{\GCD}{GCD}             % Greatest common denominator.
\DeclareMathOperator{\grad}{grad}           % Gradient
\DeclareMathOperator{\Ima}{Im}              % Image.
\DeclareMathOperator{\Int}{Int}             % Interior.
\DeclareMathOperator{\LC}{LC}               % Leading coefficient.
\DeclareMathOperator{\LCM}{LCM}             % Least common multiple.
\DeclareMathOperator{\LM}{LM}               % Leading monomial.
\DeclareMathOperator{\LT}{LT}               % Leading term.
\DeclareMathOperator{\Mod}{mod}             % Modulus.
\DeclareMathOperator{\Mon}{Mon}             % Monomial.
\DeclareMathOperator{\multideg}{mutlideg}   % Multi-Degree (Graphs).
\DeclareMathOperator{\nul}{nul}             % Null space of operator.
\DeclareMathOperator{\Ord}{Ord}             % Ordinal of ordered set.
\DeclareMathOperator{\Prin}{Prin}           % Principal value.
\DeclareMathOperator{\proj}{proj}           % Projection.
\DeclareMathOperator{\Refl}{Refl}           % Reflection operator.
\DeclareMathOperator{\rk}{rk}               % Rank of operator.
\DeclareMathOperator{\sgn}{sgn}             % Sign of a number.
\DeclareMathOperator{\sinc}{sinc}           % Sinc function.
\DeclareMathOperator{\Span}{Span}           % Span of a set.
\DeclareMathOperator{\Spec}{Spec}           % Spectrum.
\DeclareMathOperator{\supp}{supp}           % Support
\DeclareMathOperator{\Tr}{Tr}               % Trace of matrix.
%--------------------Declared Math Symbols--------------------%
\DeclareMathSymbol{\minus}{\mathbin}{AMSa}{"39} % Unary minus sign.
%------------------------New Commands---------------------------%
\DeclarePairedDelimiter\norm{\lVert}{\rVert}
\DeclarePairedDelimiter\ceil{\lceil}{\rceil}
\DeclarePairedDelimiter\floor{\lfloor}{\rfloor}
\newcommand*\diff{\mathop{}\!\mathrm{d}}
\newcommand*\Diff[1]{\mathop{}\!\mathrm{d^#1}}
\renewcommand*{\glstextformat}[1]{\textcolor{RoyalBlue}{#1}}
\renewcommand{\glsnamefont}[1]{\textbf{#1}}
\renewcommand\labelitemii{$\circ$}
\renewcommand\thesubfigure{%
    \arabic{chapter}.\arabic{figure}.\arabic{subfigure}}
\addto\captionsenglish{\renewcommand{\figurename}{Fig.}}
\numberwithin{equation}{section}

\renewcommand{\vector}[1]{\boldsymbol{\mathrm{#1}}}

\newcommand{\uvector}[1]{\boldsymbol{\hat{\mathrm{#1}}}}
\newcommand{\topspace}[2][]{(#2,\tau_{#1})}
\newcommand{\measurespace}[2][]{(#2,\varSigma_{#1},\mu_{#1})}
\newcommand{\measurablespace}[2][]{(#2,\varSigma_{#1})}
\newcommand{\manifold}[2][]{(#2,\tau_{#1},\mathcal{A}_{#1})}
\newcommand{\tanspace}[2]{T_{#1}{#2}}
\newcommand{\cotanspace}[2]{T_{#1}^{*}{#2}}
\newcommand{\Ckspace}[3][\mathbb{R}]{C^{#2}(#3,#1)}
\newcommand{\funcspace}[2][\mathbb{R}]{\mathcal{F}(#2,#1)}
\newcommand{\smoothvecf}[1]{\mathfrak{X}(#1)}
\newcommand{\smoothonef}[1]{\mathfrak{X}^{*}(#1)}
\newcommand{\bracket}[2]{[#1,#2]}

%------------------------Book Command---------------------------%
\makeatletter
\renewcommand\@pnumwidth{1cm}
\newcounter{book}
\renewcommand\thebook{\@Roman\c@book}
\newcommand\book{%
    \if@openright
        \cleardoublepage
    \else
        \clearpage
    \fi
    \thispagestyle{plain}%
    \if@twocolumn
        \onecolumn
        \@tempswatrue
    \else
        \@tempswafalse
    \fi
    \null\vfil
    \secdef\@book\@sbook
}
\def\@book[#1]#2{%
    \refstepcounter{book}
    \addcontentsline{toc}{book}{\bookname\ \thebook:\hspace{1em}#1}
    \markboth{}{}
    {\centering
     \interlinepenalty\@M
     \normalfont
     \huge\bfseries\bookname\nobreakspace\thebook
     \par
     \vskip 20\p@
     \Huge\bfseries#2\par}%
    \@endbook}
\def\@sbook#1{%
    {\centering
     \interlinepenalty \@M
     \normalfont
     \Huge\bfseries#1\par}%
    \@endbook}
\def\@endbook{
    \vfil\newpage
        \if@twoside
            \if@openright
                \null
                \thispagestyle{empty}%
                \newpage
            \fi
        \fi
        \if@tempswa
            \twocolumn
        \fi
}
\newcommand*\l@book[2]{%
    \ifnum\c@tocdepth >-3\relax
        \addpenalty{-\@highpenalty}%
        \addvspace{2.25em\@plus\p@}%
        \setlength\@tempdima{3em}%
        \begingroup
            \parindent\z@\rightskip\@pnumwidth
            \parfillskip -\@pnumwidth
            {
                \leavevmode
                \Large\bfseries#1\hfill\hb@xt@\@pnumwidth{\hss#2}
            }
            \par
            \nobreak
            \global\@nobreaktrue
            \everypar{\global\@nobreakfalse\everypar{}}%
        \endgroup
    \fi}
\newcommand\bookname{Book}
\renewcommand{\thebook}{\texorpdfstring{\Numberstring{book}}{book}}
\providecommand*{\toclevel@book}{-2}
\makeatother
\titleformat{\part}[display]
    {\Large\bfseries}
    {\partname\nobreakspace\thepart}
    {0mm}
    {\Huge\bfseries}
\titlecontents{part}[0pt]
    {\large\bfseries}
    {\partname\ \thecontentslabel: \quad}
    {}
    {\hfill\contentspage}
\titlecontents{chapter}[0pt]
    {\bfseries}
    {\chaptername\ \thecontentslabel:\quad}
    {}
    {\hfill\contentspage}
\newglossarystyle{longpara}{%
    \setglossarystyle{long}%
    \renewenvironment{theglossary}{%
        \begin{longtable}[l]{{p{0.25\hsize}p{0.65\hsize}}}
    }{\end{longtable}}%
    \renewcommand{\glossentry}[2]{%
        \glstarget{##1}{\glossentryname{##1}}%
        &\glossentrydesc{##1}{~##2.}
        \tabularnewline%
        \tabularnewline
    }%
}
\newglossary[not-glg]{notation}{not-gls}{not-glo}{Notation}
\newcommand*{\newnotation}[4][]{%
    \newglossaryentry{#2}{type=notation, name={\textbf{#3}, },
                          text={#4}, description={#4},#1}%
}
%--------------------------LENGTHS------------------------------%
% Spacings for the Table of Contents.
\addtolength{\cftsecnumwidth}{1ex}
\addtolength{\cftsubsecindent}{1ex}
\addtolength{\cftsubsecnumwidth}{1ex}
\addtolength{\cftfignumwidth}{1ex}
\addtolength{\cfttabnumwidth}{1ex}

% Indent and paragraph spacing.
\setlength{\parindent}{0em}
\setlength{\parskip}{0em}
\graphicspath{{../../images/}}   % Path to Image Folder.
%----------------------------GLOSSARY-------------------------------%
\makeglossaries
\loadglsentries{../../glossary}
\loadglsentries{../../acronym}
%--------------------------Main Document----------------------------%
\begin{document}
    \ifx\ifmathcourses\undefined
        \pagenumbering{roman}
        \title{Numerical Analysis}
        \author{Ryan Maguire}
        \date{\vspace{-5ex}}
        \maketitle
        \tableofcontents
        \clearpage
        \chapter*{Numerical Analysis}
        \addcontentsline{toc}{chapter}{Numerical Analysis}
        \markboth{}{NUMERICAL ANALYSIS}
        \vspace{10ex}
        \setcounter{chapter}{1}
        \pagenumbering{arabic}
    \else
        \chapter{Numerical Analysis}
    \fi
    \subsection{Numerical Analysis}
        \begin{definition}
            If $a_{n}\rightarrow{a}$, then $a_{n}$ converges
            with order of converges $O(\beta_{n})$, where
            $\beta_{n}\ne{0}$, if, for large enough $N$, there
            is a $K$ such that for all $n<N$:
            \begin{equation*}
                \Big|\frac{a-a_{n}}{\beta_{n}}\Big|\leq{K}
            \end{equation*}
        \end{definition}
        Given the order of convergence $\beta_{n}$, we can
        write $a_{n}=a+O(\beta_{n})$. A similar definition
        holds for the limit of functions.
        If $f$ is continuous and $f(a)$ and $f(b)$ have
        opposite signs, then the intermediate value theorem
        says that there is a $c\in(a,b)$ such that
        $f(c)=0$. The bisection method is an algorithm for
        approximation such a value $c$. Let $p$ be the
        mean of $a$ and $b$ (The midpoint). If $f(p)$ has
        the same sign as $f(a)$, choose your next point
        to be the mean of $p$ and $b$. If $f(p)$ has
        the same sign as $f(b)$, choose your next point
        to be the midpoint of $a$ and $p$. If $f(p)=0$,
        you're done. Continuing on in this manner, we bisect
        the interval each time so that after $n$ steps we
        have an interval $(b-a)/2^{n+1}$ in width. The error
        of $p$ from the actual value is the bound by this
        number, which tends to zero very fast.
        The Newton-Raphson method is a method for computing
        the roots of various equations. If $f$ is twice
        differentiable, and $x_{0}$ is your initial guess,
        we can create a sequence that converges to the
        root (Provided $f$ has various properties) by:
        \begin{equation*}
            x_{n+1}=x_{n}-\frac{f(x_{n})}{f'(x_{n})}
        \end{equation*}
        In particular, $f'(x)$ can not be zero, or near
        zero. This can cause convergence to fail.
        If the root is a simple root, the convergence
        is quadratic. The fixed tangent method is:
        \begin{equation*}
            x_{n+1}=x_{n}-\frac{f(x_{n})}{f'(x_{0})}
        \end{equation*}
        Convergence is again subjeect to various criteria
        on $f$. For derivatives that are difficult to
        compute, we can use the secant method:
        \begin{equation*}
            x_{n+1}=x_{n}-\frac{f(x_{n})(x_{n}-x_{n-1})}
                               {f(x_{n})-f(x_{n-1})}
        \end{equation*}
        The order of convergence is approximately 1.62.
        The Regula Falsi method combines the secant
        method and the bisection method. We create the
        sequence:
        \begin{equation*}
            x_{n+1}=a_{n}-\frac{f(a_{n})(b_{n}-a_{n})}
                               {f(b_{n})-f(a_{n})}
        \end{equation*}
        \begin{theorem}
            If $P$ is a polynomial over the complex plane
            of degree $n\geq{1}$,
            then there is a $z\in\mathbb{C}$
            such that $P(z)=0$. That is,
            all non-constant polynomials have at least
            one root in the complex plane.
        \end{theorem}
        \begin{theorem}
            If $P$ is a polynomial of degree $n$,
            then there exists a unique complex number $\alpha$,
            and unique complex numbers $r_{i}$ called the roots,
            and integers $m_{i}$ called the degrees,
            such that $\sum{m_{i}}=n$, and
            $P=\alpha(x-r_{1})^{m_{1}}\cdots(x-r_{N})^{m_{N}}$
        \end{theorem}
        \begin{theorem}
            A polynomial of degree $n$ has
            $n$ roots (Though they may not be distinct).
        \end{theorem}
        \begin{theorem}
            If $P$ and $Q$ are polynomials
            of degree $n$, and if there are
            $N>n$ numbers $x_{i}$ such that
            $P(x_{i})=Q(x_{i})$, then
            $P(x)=Q(x)$.
        \end{theorem}
        \begin{theorem}
            If $P$ is a polynomial and
            $z=a+bi$ has multiplicity $m$, then
            $a-bi$ has multiplicity $m$ as well.
        \end{theorem}
        Horner's Method is used to quickly calculate
        polynomials:
        $P(x)=a_{0}+x(a_{1}+a_{2}x(a_{3}+x_{4}x(\hdots)))$
        \begin{theorem}[Weierstrass Approximation Theorem]
            If $f$ is continuous on $[a,b]$, then for all
            $\varepsilon>0$ there is a polynomial $P(x)$
            such that $|P(x)-f(x)|<\varepsilon$ for all
            $x\in[a,b]$.
        \end{theorem}
        Recalling the binomial theorem from before, we have
        several theorem pertaining to polynomials:
        \begin{theorem}
            The following are true:
            \begin{enumerate}
                \begin{multicols}{2}
                    \item $\sum_{i=0}^{n}%
                           \binom{n}{i}x^{i}(1-x)^{n-i}=1$
                    \item $\sum_{i=0}^{n}%
                           \binom{n}{i}x^{i}(1-x)^{n-i}%
                           (i-nx)=0$
                    \item $\sum_{i=0}^{n}%
                           \binom{n}{i}x^{i}(1-x)^{n-i}%
                           (i-nx)^{2}=nx(1-x)$
                \end{multicols}
            \end{enumerate}
        \end{theorem}
        \begin{definition}
            If $f$ is defined on $[0,1]$, then the Bernstein
            polynomial of $f$ of degree $N$ is:
            \begin{equation*}
                B_{N}(x)=\sum_{n=0}^{N}\binom{N}{n}
                    f\Big(\frac{n}{N}\Big)
                    x^{n}(1-x)^{N-n}
            \end{equation*}
        \end{definition}
        \begin{theorem}[Bernstein's Theorem]
            If $f[0,1]:\rightarrow\mathbb{R}$ is
            continuous, then $B_{N}(x)\rightarrow{f(x)}$
            uniformly.
        \end{theorem}
        The problem of finding polynomials that pass through
        fixed points in the plane involves the
        techniques of interpolation. Lagrange polynomials can
        be used for this problem. There are other methods such
        as Neville's method and Aitken's method.
        \begin{definition}
            Given a function $f$ defined on $[a,b]$, and
            nodes $a=x_{0}<x_{1}<\hdots<x_{n}=b$, a
            cubic spline $C$ is a function such that,
            for all $[x_{i},x_{i+1}]$, the restriction
            of $C$ to that interval, $C_{i}$,
            is a cubic polynomial and
            $C(x_{i})=f(x_{i})$ for all nodes. Also,
            $C_{i}(x_{i+1})=C_{i+1}(x_{i+1})$,
            $C_{i}'(x_{i+1})=C_{i+1}'(x_{i+1})$, and
            $C_{i}''(x_{i+1})=C_{i+1}''(x_{i+1})$
            That is, the cubic sections meet
            tangentially and share the same
            second derivative.
        \end{definition}
        Numerical integration is another technique that is
        important in numerical analysis. There are two popular
        kinds. Riemann sums approximate integrates with
        rectangles. We can instead approximate with
        lines (Forming trapezoids) of quadratics.
        The trapezoid rule has the formula:
        \begin{equation*}
            \int_{a}^{b}f(x)\diff{x}\approx
            \frac{b-a}{2n}[f_{0}+2f_{1}+2f_{2}+
            \hdots+2f_{n-2}+2f_{n-1}+f_{n}]
        \end{equation*}
        The approximation error for the trapezoid rule is:
        \begin{equation*}
            E_{n}=\Big|\frac{(b-a)^{3}f''(c)}{12n^{2}}\Big|
        \end{equation*}
        Where $c\in(a,b)$. Simpson's rule uses approximating
        quadratics between each node. This formula is:
        \begin{equation*}
            \int_{a}^{b}f(x)\diff{x}\approx
            \frac{b-a}{3n}[f(x_{0})+4f(x_{1})+2f(x_{2})
            +4f(x_{3})+\hdots+2f(x_{n-2})+4f(x_{n-1})+f(x_{n})]
        \end{equation*}
        The approximating error is:
        \begin{equation*}
            E_{n}=\Big|\frac{(b-a)^{5}f^{(4)}(c)}{180n^{4}}\Big|
        \end{equation*}
        Where $c\in(a,b)$. The trapazoidal and Simpson's rules
        can be obtained by integrating the Lagrange
        interpolating formula over
        $[a,b]$. Quadrature methods are obtained from
        integrating over the Lagrange polynomials:
        \begin{equation*}
            \int_{a}^{b}\sum_{n=0}^{N}f(x_{n})L_{n}(x)\diff{x}
            =\sum_{n=0}^{N}c_{i}f(x_{i})
        \end{equation*}
        There are many types of quadrature rules, such as
        Newton-Cotes methods, and Gaussian quadrature. Another
        type of problem that arises in numerical analysis is
        finding a continuous function that best fits a
        discrete set of data. Least squares approximations are
        one such method of finding a line of best fit.
        We wish to find a line $y=ax+b$ that minimizes the
        following: If $(x_{i},y_{i})$ are the data points,
        \begin{equation*}
            M=\sum_{n=1}^{N}\big(y_{n}-(ax_{n}+b_{n})\big)^{2}
        \end{equation*}
        Thus we have:
        \begin{align*}
            \frac{\partial{M}}{\partial{a}}&=0
            &
            \frac{\partial{M}}{\partial{b}}&=0
        \end{align*}
        Solving this set of equations gives us
        our $a$ and $b$:
        \begin{align*}
            a&=
            \frac{N\sum_{n=1}^{N}x_{n}y_{n}-
                  \Big(\sum_{n=1}^{N}x_{n}\Big)
                  \Big(\sum_{n=1}^{N}y_{n}\Big)}
                 {N\sum_{n=1}^{N}x_{n}^{2}-
                  \Big(\sum_{n=1}^{N}x_{n}\Big)^{2}}\\
            b&=
            \frac{\Big(\sum_{n=1}^{N}x_{n}^{2}\Big)
                  \Big(\sum_{n=1}^{N}y_{n}\Big)-
                  \Big(\sum_{n=1}^{N}x_{n}y_{n}\Big)
                  \Big(\sum_{n=1}^{N}x_{n}\Big)}
                 {N\sum_{n=1}^{N}x_{n}^{2}-
                  \Big(\sum_{n=1}^{N}x_{n}\Big)^{2}}\\
        \end{align*}
        This least squares method uses a line to best fit
        the data. However we may use arbitrary polynomials
        and exponential functions, as well.
        \begin{definition}
            A sequence of polynomials $P_{n}(x)$ is
            orthogonal on an interval $[a,b]$ with
            respect to a weight $w$ if:
            \begin{equation*}
                \int_{a}^{b}w(x)P_{n}(x)P_{m}(x)\diff{x}
                =\delta_{nm}
            \end{equation*}
            Where $\delta_{nm}$ is the Kronecker Delta Function.
            That is, $\delta_{nm}=1$ if $n=m$, and
            is equal to zero otherwise.
        \end{definition}
        \begin{theorem}
            If $\psi_{n}$ is a set of orthogonal
            polynomials with weight $w$ on the
            interval $[a,b]$, then the least squares
            solution for approximating a continuous
            function $f$ with $\psi_{n}$ is
            $f(x)=\sum_{n=0}^{N}c_{n}\psi_{n}(x)$,
            where:
            \begin{equation*}
                c_{n}=
                \frac{\int_{a}^{b}w(x)\psi_{n}(x)f(x)\diff{x}}
                     {\int_{a}^{b}w(x)\psi_{n}^{2}(x)\diff{x}}
            \end{equation*}
        \end{theorem}
\end{document}