\chapter{Integral Calculus}
    \section{Integration}
        \subsection{Areas}
        \subsection{The Riemann Integral}
        \subsection{The Fundamental Theorem of Calculus}
    \section{Methods of Integration}
        \subsection{Integration Rules}
            \begin{lexample}
                Let's integrate the function $f(x)=|x+1|$ on
                the interval $(\minus{3},3)$. From the definition of
                the absolute value function, we have:
                \begin{equation}
                    |x+1|=
                    \begin{cases}
                        x+1,&x\geq\minus{1}\\
                        \minus(x+1),&x<\minus{1}
                    \end{cases}
                \end{equation}
                Splitting the integral into two parts, we can compute
                this as:
                \begin{subequations}
                    \begin{align}
                        \int_{\minus{3}}^{3}|x+1|\diff{x}
                        &=\int_{\minus{3}}^{\minus{1}}|x+1|\diff{x}
                        +\int_{\minus{1}}^{3}|x+1|\diff{x}\\
                        &=\int_{\minus{3}}^{\minus{1}}\minus(x+1)\diff{x}
                        +\int_{\minus{1}}^{3}(x+1)\diff{x}
                    \end{align}
                \end{subequations}
                We can evaluate these two integrals by using the power rule
                for integrals to obtain:
                \begin{subequations}
                    \begin{align}
                        \int_{\minus{3}}^{3}|x+1|\diff{x}
                        &=\minus\int_{\minus{3}}^{\minus{1}}(x+1)\diff{x}
                        +\int_{\minus{1}}^{3}(x+1)\diff{x}\\
                        &=\minus\bigg[\frac{x^{2}}{2}+x\bigg]_{\minus{3}}^{\minus{1}}
                        +\bigg[\frac{x^{2}}{2}+x\bigg]_{\minus{1}}^{3}
                    \end{align}
                \end{subequations}
                Evaluating this last expression, we see that the answer is 10.
            \end{lexample}
        \subsection{Substitution}
        \subsection{Integration by Parts}
        \subsection{Trigonometric Substitutions}
    \section{Applications of Definite Integrals}
        \subsection{Arc Length}
            \begin{lexample}
                What is the length of the curve with
                parametric equations:
                \begin{equation}
                    \big(x(t),y(t)\big)=
                    \big(\cos(t),\sin(t)\big)
                    \quad\quad
                    0\leq{t}\leq\pi
                \end{equation}
                We have $x(t)=\cos(t)$ and $y(t)=\sin(t)$, so
                $\Gamma(t)=\big(\cos(t),\sin(t)\big)$. This is the
                parameterization of the unit circle for
                $0\leq{t}\leq{2}\pi$. Knowing this we see that the length
                of $\Gamma$ from $0$ to $\pi$ is simply half the
                circumference of the unit circle, which is $\pi$. We can
                also use the definition of length to obtain:
                \par
                \begin{subequations}
                    \begin{minipage}[b]{0.49\textwidth}
                        \centering
                        \begin{align}
                            \Gamma(t)&=\big(\cos(t),\sin(t)\big)\\
                            \Gamma'(t)&=\big(\minus\sin(t),\cos(t)\big)\\
                            \norm{\Gamma'(t)}&=\sqrt{\sin^{2}(t)+\cos^{2}(t)}\\
                            &=1
                        \end{align}
                    \end{minipage}
                    \hfill
                    \begin{minipage}[b]{0.49\textwidth}
                        \centering
                        \begin{align}
                            \int_{C}\diff{s}
                            &=\int_{0}^{\pi}\norm{\Gamma'(t)}\diff{t}\\
                            &=\int_{0}^{\pi}\diff{t}\\
                            &=\pi
                        \end{align}
                    \end{minipage}
                \end{subequations}
                So we get $\pi$, in agreement with the circumference
                formula for a half-circle.
            \end{lexample}
        \subsection{Solids of Revolution}
        \subsection{Surface Area}
        \begin{definition}
            The indefinite integral of a differentiable function
            $f$ is $\int{f(x)}\diff{x}=F(x)+C$, where $F$ is such
            that $F'(x)=f(x)$, and $C$ is an arbitrary constant.
        \end{definition}
        \begin{theorem}
            The following are true:
            \begin{enumerate}
                \begin{multicols}{2}
                    \item $\int{x^{n}}\diff{x}%
                           =\frac{x^{n+1}}{n+1}+C$
                    \item $\int\frac{\diff{x}}{x}=\ln|x|+C$
                    \item $\int\frac{\diff{x}}{x-a}=\ln|x-a|+C$
                    \item $\int\frac{\diff{x}}{x^{2}+a^{2}}%
                           =\frac{1}{a}\tan^{-1}(\frac{x}{a})+C$
                    \item $\int\frac{x\diff{x}}{x^{2}+a^{2}}%
                           =\frac{1}{2}\ln|x^{2}+a^{2}|+C$
                    \item $\int\frac{\diff{x}}{\sqrt{a^{2}-x^{2}}}%
                           =\sin^{-1}(\frac{x}{a})+C$
                    \item $\int\sin(ax)\diff{x}=-\frac{1}{a}\cos(ax)+C$
                    \item $\int\cos(ax)\diff{x}=\frac{1}{a}\sin(ax)+C$
                    \item $\int\sec^{2}(x)\diff{x}=\tan(x)+C$
                    \item $\int\exp(ax)\diff{x}=\frac{1}{a}\exp(ax)+C$
                \end{multicols}
            \end{enumerate}
        \end{theorem}
        \begin{example}
            Find $\int\frac{x}{x+1}\diff{x}$:
            \begin{align*}
                \int\frac{x}{x+1}\diff{x}
                =\int\frac{x+1-1}{x+1}\diff{x}
                &=\int\frac{x+1}{x+1}\diff{x}
                -\int\frac{1}{x+1}\diff{x}\\
                &=\int\diff{x}-\int\frac{1}{x+1}\diff{x}
                =x-\ln|x+1|+C
            \end{align*}
        \end{example}
        \begin{theorem}
            If $F(x)=f(u(x))$,
            then $\int{F'(x)}\diff{x}=f(u(x))+C$.
        \end{theorem}
        This is called $u$ substituion. If we have
        $\int{f(x)\diff{x}}$ and we can write
        $f=g(u)$, $dx=g'(u)du$, then we can rewrite the
        integral as
        $\int{f(x)}\diff{x}=\int{g(u)du}$. This comes from
        reversing the chain rule found in the study of
        differentiation.
        \begin{example}
            Compute $\int\frac{x}{x^{2}+a^{2}}\diff{x}$.
            Let $u=x^{2}+a^{2}$. Then $du=2xdx$. We have:
            \begin{equation*}
                \int\frac{x}{x^{2}+a^{2}}\diff{x}
                =\frac{1}{2}\int\frac{\diff{u}}{u}
                =\frac{1}{2}\ln|u|+C
                =\frac{1}{2}\ln|x^{2}+u^{2}|+C
            \end{equation*}
        \end{example}
        The product rule says that
        $\diff(uv)=u\diff{v}+v\diff{u}$. Integrating, we have
        $uv=\int{u}\diff{v}+\int{v}\diff{u}$. This gives rise to
        the integration by parts formula:
        \begin{equation*}
            \int{u}\diff{v}=uv-\int{v}\diff{u}
        \end{equation*}
        \begin{definition}
            The definite integral of $f$ on $(a,b)$ is
            $\int_{a}^{b}f(x)\diff{x}=F(b)-F(a)$, where
            $F$ is a function such that $F'(x)=f(x)$.
        \end{definition}
    \subsection{Sequences and Series}
        \begin{definition}
            A sequence is a function $x:\mathbb{N}\rightarrow{X}$.
            That is, a function whose domain is the natural numbers.
        \end{definition}
        \begin{definition}
            A convergent sequence $a:\mathbb{N}\rightarrow\mathbb{R}$
            is a sequence such that there is an $\ell\in\mathbb{R}$ such
            that, for all $\varepsilon>0$, there is an $N\in\mathbb{N}$
            such that, for all $n>N$, $|a_{n}-\ell|<\varepsilon$.
        \end{definition}
        \begin{theorem}
            If $a_{n}\rightarrow{a}$, and
            $b_{n}\rightarrow{b}$, then:
            \begin{enumerate}
                \begin{multicols}{2}
                    \item $a_{n}+b_{n}\rightarrow{a+b}$
                    \item $a_{n}-b_{n}\rightarrow{a-b}$
                    \item $a_{n}b_{n}\rightarrow{ab}$
                    \item If $b\ne{0}$, $a_{n}/b_{n}\rightarrow{a/b}$
                \end{multicols}
            \end{enumerate}
        \end{theorem}
        \begin{theorem}[Squeeze Theorem]
            If $a_{n}$, $b_{n}$, and $c_{n}$ are sequences such
            that $a_{n}\leq{b_{n}}\leq{c_{n}}$ for all $n\in\mathbb{N}$,
            and if $a_{n}\rightarrow\ell$ and $c_{n}\rightarrow\ell$,
            then $b_{n}\rightarrow\ell$.
        \end{theorem}
        \begin{theorem}
            If $|a_{n}|\rightarrow{0}$, then
            $a_{n}\rightarrow{0}$.
        \end{theorem}
        \begin{definition}
            A monotonically increasing sequence is a sequence
            such that, for all $n$,
            $a_{n}\leq{a_{n+1}}$.
        \end{definition}
        \begin{definition}
            A monotonically decreasing sequence is a sequence
            such that, for all $n$,
            $a_{n}\geq{a_{n+1}}$.
        \end{definition}
        \begin{definition}
            A monotonic sequence is a sequence that is either
            monotonically increasing or monotonically
            decreasing.
        \end{definition}
        \begin{definition}
            A bounded sequence is a sequence $a:\mathbb{N}\rightarrow\mathbb{R}$
            such that there exists an $M\in\mathbb{R}$ such that, for all
            $n\in\mathbb{N}$, $|a_{n}|\leq{M}$.
        \end{definition}
        \begin{definition}
            The $N^{th}$ partial sum of a sequence
            $a:\mathbb{N}\rightarrow\mathbb{R}$ is
            $s_{N}=\sum_{n=1}^{N}a_{n}$.
        \end{definition}
        \begin{definition}
            An infinite series of a sequence $a_{n}$ is the
            limit of the partial sums $s_{N}$ of $a_{n}$.
        \end{definition}
        \begin{definition}
            A convergent infinite series is a series such that
            the limit of partial sums exists.
        \end{definition}
        \begin{definition}
            A divergent infinite series is a series that
            does not converge.
        \end{definition}
        \begin{definition}
            A geometric series is a sum of the form
            $\sum_{n=0}^{N}ar^{n}$, where $a$ and $r$ are constants.
        \end{definition}
        \begin{theorem}
            A series converges if and only if the sequence of partial
            sums forms a Cauchy sequence.
        \end{theorem}
        \begin{theorem}
            If $a_{n}$ is a positive sequence $(a_{n}>0)$
            and if $S_{N}=\sum_{n=1}^{N}$ is bounded above
            by some $M\in\mathbb{R}$ for all $N\in\mathbb{N}$,
            then $\sum_{n=1}^{\infty}a_{n}$ converges. If
            $S_{N}$ is not bounded above, then the sum diverges.
        \end{theorem}
        \begin{theorem}
            If $p>1$, then
            $\sum_{n=0}^{\infty}\frac{1}{p^{n}}$ converges. If
            $p\leq{1}$, the sum diverges.
        \end{theorem}
        \begin{theorem}[Comparison Test]
            If $0\leq{a_{n}}\leq{b_{n}}$ for all $n$,
            and if $\sum_{n=1}^{\infty}b_{n}$ converges,
            then $\sum_{n=1}^{\infty}a_{n}$ converges.
            If $\sum_{n=1}^{\infty}a_{n}$ diverges,
            then $\sum_{n=1}^{\infty}b_{n}$ diverges.
        \end{theorem}
        \begin{theorem}[Alternating Series Test]
            If $a_{n}$ is a strictly decreasing sequence
            of positive real numbers, and if
            $a_{n}\rightarrow{0}$, then
            $\sum_{n=1}^{\infty}(-1)^{n}a_{n}$ converges.
        \end{theorem}
        \begin{definition}
            An absolutely convergent series
            is a series such that
            $\sum_{n=1}^{\infty}|a_{n}|$ converges.
        \end{definition}
        \begin{theorem}
            If $\sum_{n=1}^{\infty}|a_{n}|$ converges,
            then $\sum_{n=1}^{\infty}a_{n}$ converges.
        \end{theorem}
        \begin{theorem}[Ratio Test]
            If $a_{n}$ is a sequence, and
            $|\frac{a_{n+1}}{a_{n}}|\rightarrow\alpha$,
            the $\sum_{n=1}^{\infty}a_{n}$ converges absolutely
            if $\alpha<1$, and diverges if $\alpha>1$.
        \end{theorem}
        \begin{theorem}[Root Test]
            If $\lim_{n\rightarrow\infty}\sqrt[n]{|a_{n}|}=\ell$,
            then $\sum_{n=1}^{\infty}a_{n}$ is absolutely convergent
            if $\ell<1$, and diverges if $\ell>1$.
        \end{theorem}
        \begin{definition}
            A power series is a function
            $f(x)=\sum_{n=0}^{N}c_{n}(x-x_{0})^{n}$
        \end{definition}
        \begin{theorem}[Taylor's Coefficient Theorem]
            If $f(x)=\sum_{n=0}^{N}c_{n}(x-x_{0})^{n}$,
            then $c_{n}=\frac{f^{(n)}(x_{0})}{n!}$.
        \end{theorem}
        \begin{definition}
            The Taylor Series of a function $f$ about $x_{0}$ is
            $f(x)=\sum_{n=0}^{\infty}f^{(n)}(x_{0})\frac{(x-x_{0})^{n}}{n!}$
        \end{definition}
        \begin{definition}
            An analytic function is a function $f$ whose
            Taylor series converges to $f$.
        \end{definition}
        \begin{definition}
            A MacLaurin series is a series of the form:
            $f(x)=\sum_{n=0}^{\infty}f^{(n)}(0)\frac{x^{n}}{n!}$
        \end{definition}
        \begin{example}
            $\sin$ and $\cos$ have the following MacLaurin Series:
            \begin{align*}
                \sin(x)
                &=\sum_{n=0}^{\infty}(-1)^{n}\frac{x^{2n+1}}{(2n+1)!}
                &
                \cos(x)
                &=\sum_{n=0}^{\infty}(-1)^{n}\frac{x^{2n}}{(2n)!}
            \end{align*}
        \end{example}
        \begin{theorem}[Binomial Theorem]
            If $n\in\mathbb{N}$, and $\binom{n}{m}$ is the
            binomial coefficient, then:
            \begin{equation*}
                (x+y)^{n}=
                \sum_{k=0}^{n}\binom{n}{k}x^{k}y^{n-k}=
                \sum_{k=0}^{n}\frac{n!}{(m-k)!k!}x^{k}y^{n-k}
            \end{equation*}
        \end{theorem}
        \begin{theorem}[Geometric Series]
            If $N\in\mathbb{N}$ and $r$ is a real number, then
            \begin{equation*}
                \sum_{n=0}^{N}ar^{n}=a\frac{1-r^{N+1}}{1-r}
            \end{equation*}
        \end{theorem}
        \begin{theorem}
            If $|r|<1$, then
            $\sum_{n=0}^{\infty}ar^{n}$ converges and:
            \begin{equation*}
                \sum_{n=0}^{\infty}ar^{n}=\frac{a}{1-r}
            \end{equation*}
        \end{theorem}
        \begin{definition}
            The Fourier Series of a function $f$ on an interval $[-\pi,\pi]$
            is:
            \begin{equation*}
                f(x)=\sum_{n=0}^{\infty}\Big(a_{n}\cos(nx)+b_{n}\sin(nx)\Big)
            \end{equation*}
            Where:
            \begin{align*}
                a_{0}&=\frac{1}{2\pi}\int_{-\pi}^{\pi}f(x)\diff{x}
                &
                a_{n}&=\frac{1}{\pi}\int_{-\pi}^{\pi}f(x)\cos(x)\diff{x}
                &
                b_{n}&=\frac{1}{\pi}\int_{-\pi}^{\pi}f(x)\sin(x)\diff{x}
            \end{align*}
        \end{definition}
    \section{Infinite Series}
        \subsection{Definitions and Notations}   
    \section{Exams}
        \subsection{Practice Exam I}
            \begin{problem}
                Use the Midpoint Rule to estimate the
                area under the graph of $f(x)=16-x^{2}$ between
                $x=-4$ and $x=4$ using four rectangles. Estimate
                the average of $f$ on the same interval.
            \end{problem}
            \begin{solution}
                \
                \begin{table}[H]
                    \centering
                    \begin{tabular}{|c|c|c|c|}
                        \hline
                        Start&End&\# Pts&$\Delta x$\\
                        \hline
                        $a=-4$&$b=4$&$n=4$&$\frac{b-a}{n}=2$\\
                        \hline
                    \end{tabular}
                \end{table}
                The midpoints are $m_{n}=\frac{x_{n-1}+x_{n}}{2}$,
                where $x_{0}=a$, and $x_{n}=x_{n-1}+\Delta x$.
                \begin{table}[H]
                    \centering
                    \begin{tabular}{|c|c|c|c|}
                        \hline
                        $m_{1}$&$m_{2}$&$m_{3}$&$m_{4}$\\
                        \hline
                        $-3$&$-1$&$1$&$3$\\
                        \hline
                    \end{tabular}
                \end{table}
                The midpoint rule says $A_{M}=\sum_{k=1}^{n}f(m_{k})\Delta x$
                \begin{table}[H]
                    \centering
                    \begin{tabular}{|c|c|c|c|}
                        \hline
                        $f(m_{1})$&$f(m_{2})$&$f(m_{3})$&$f(m_{4})$\\
                        \hline
                        $7$&$15$&$15$&$7$\\
                        \hline
                    \end{tabular}
                \end{table}
                So, $A_{M}=2(7+15+15+7)=2(44)=88$.
                The estimated average of $f$ is
                $\frac{A_{m}}{b-a}=\frac{88}{8}=11$
            \end{solution}
            \begin{problem}
                Using the Fundamental Theorem of Calculus,
                find $\frac{dy}{dx}$:
                \begin{enumerate}
                    \begin{multicols}{2}
                        \item $y=\int_{\sin(x)}^{0}%
                               \frac{4dt}{\sqrt{1-t^{2}}}$
                        \item $y=\int_{3+x^{2}}^{6}%
                               \frac{t}{1+e^{t}}dt$
                    \end{multicols}
                \end{enumerate}
            \end{problem}
            \begin{solution}
                \
                \begin{enumerate}
                    \item
                        \begin{align*}
                            y&=\int_{\sin(x)}^{0}\frac{4dt}{\sqrt{1-t^{2}}}
                            =-\int_{0}^{\sin(x)}\frac{4dt}{\sqrt{1-t^{2}}}
                            \Rightarrow\frac{dy}{dx}
                            =-\frac{d}{dx}\int_{0}^{\sin(x)}
                            \frac{4dt}{\sqrt{1-t^{2}}}\\
                            \Rightarrow\frac{dy}{dx}
                            &=-\frac{4}{\sqrt{1-\sin^{2}(x)}}
                            \frac{d}{dx}(\sin(x))=-\frac{4\cos(x)}{|\cos(x)|}
                            =-4\sgn(x)
                        \end{align*}
                    \item
                        \begin{align*}
                            y&=\int_{3+x^{2}}^{6}\frac{t}{1+e^{t}}dt
                            =-\int_{6}^{3+x^{2}}\frac{t}{1+e^{t}}dt
                            \Rightarrow\frac{dy}{dx}
                            =-\frac{d}{dx}\int_{6}^{3+x^{2}}
                            \frac{t}{1+e^{t}}dt\\
                            &=-\frac{3+x^{2}}{1+e^{3+x^{2}}}
                            \frac{d}{dx}(3+x^{2})
                            =-\frac{2x(3+x^{2})}{1+e^{3+x^{2}}}
                        \end{align*}
                \end{enumerate}
            \end{solution}
            \newpage
            \begin{problem}
                Evaluate $\int_{-2}^{2}(x^{2}+e^{2})dx$
            \end{problem}
            \begin{solution}
                \begin{equation*}
                    \int_{-2}^{2}(x^{2}+e^{2})dx
                    =\frac{1}{3}x^{3}+e^{2}x\big|_{-2}^{2}
                    =\frac{16}{3}+4e^{2}
                \end{equation*}
            \end{solution}
            \begin{problem}
                \
                \begin{enumerate}
                    \begin{multicols}{3}
                        \item $\int_{0}^{1}%
                               \frac{4x^{3}}{(x^{4}-3)^{2}}dx$
                        \item $\int e^{5x}(e^{5x}+4)^{2}dx$
                        \item $\int\cos^{3}(3x)\sin(3x)dx$
                    \end{multicols}
                \end{enumerate}
            \end{problem}
            \begin{solution}
                \
                \begin{enumerate}
                    \item
                        \begin{align*}
                            u=x^{4}-3,du
                            &=4x^{3}dx,x=0\Rightarrow
                            u=-3,x=1\Rightarrow
                            u=-2\\
                            \int_{0}^{1}\frac{4x^{3}}{(x^{4}-3)^{2}}dx
                            &=\int_{-3}^{-2}\frac{du}{u^{2}}
                            =-\frac{1}{u}\big|_{-3}^{-2}
                            =\frac{1}{6}
                        \end{align*}
                    \item
                        \begin{align*}
                            u=e^{5x}+4,du
                            &=5e^{5x}dx\Rightarrow
                            e^{5x}dx=\frac{1}{5}du\\
                            \int e^{5x}(e^{5x}+4)^{2}dx
                            &=\frac{1}{5}\int u^{2}du
                            =\frac{1}{15}u^{3}+C
                            =\frac{1}{15}(e^{5x}+4)^{3}+C
                        \end{align*}
                    \item
                        \begin{align*}
                            u=\cos(3x),du
                            &=-3\sin(3x)dx\Rightarrow
                            -\frac{1}{3}du=\sin(3x)dx\\
                            \int\cos^{3}(3x)\sin(3x)dx
                            &=-\frac{1}{3}\int
                            u^{3}du=-\frac{1}{12}u^{4}+C
                            =-\frac{1}{12}\cos^{4}(3x)+C
                        \end{align*}
                \end{enumerate}
            \end{solution}
            \begin{problem}
                Find the area of the region bounded by
                the curves $f(x)=-x^{2}+4x-2$ and $g(x)=x^{2}-2$
            \end{problem}
            \begin{solution}
                The points of intersection are:
                $x^{2}-2=-x^{2}+4x-2\Rightarrow 2x(x-2)=0%
                 \Rightarrow x=0,x=2$.
                The area between the curves is
                $A=\int_{0}^{2}(f(x)-g(x))dx$:
                \begin{equation*}
                    A=\int_{0}^{2}(f(x)-g(x))dx
                    =\int_{0}^{2}(-2x^{2}+4x)dx
                    =-\frac{2}{3}x^{3}+2x^{2}\big|_{0}^{2}
                    =\frac{8}{3}
                \end{equation*}
            \end{solution}
            \begin{problem}
                Evaluate $\int xe^{2x}dx$
            \end{problem}
            \begin{solution}
                \
                \begin{table}[H]
                    \centering
                    \begin{tabular}{|c|c|}
                        \hline
                        $u=x$&$dv=e^{2x}dx$\\
                        \hline
                        $du=dx$&$v=\frac{1}{2}e^{2x}$\\
                        \hline
                    \end{tabular}
                \end{table}
                \begin{equation*}
                    \int{u}dv=uv-\int{v}du
                    \Rightarrow\int xe^{2x}dx
                    =\frac{1}{2}xe^{2x}-\frac{1}{2}
                    \int{e}^{2x}dx
                    =\frac{1}{2}xe^{2x}-\frac{1}{4}e^{2x}+C
                \end{equation*}
            \end{solution}
            \begin{problem}
                Evaluate
                $\int_{0}^{\pi^{2}}\cos(\sqrt{x})dx$
            \end{problem}
            \begin{solution}
                    Let $y=\sqrt{x}$. Then we have:
                    \begin{equation*}
                        dy=\frac{1}{2\sqrt{x}}dx\Rightarrow
                        dx=2\sqrt{x}dy=2ydy
                        \Rightarrow\int\cos(\sqrt{x})dx
                        =2\int{y}\cos(y)dy
                    \end{equation*}
                \begin{table}[H]
                    \centering
                    \begin{tabular}{|c|c|}
                        \hline
                        $u=y$&$dv=\cos(y)dy$\\
                        \hline
                        $du=dy$&$v=\sin(y)$\\
                        \hline
                    \end{tabular}
                \end{table}
                \begin{align*}
                    \int\cos(\sqrt{x})dx
                    &=2\int y\cos(y)dy
                    =2\big(y\sin(y)-\int\sin(y)dy\big)
                    =2y\sin(y)+2\cos(y)\\
                    \Rightarrow \int_{0}^{\pi^{2}}\cos(\sqrt{x})dx
                    &=2\sqrt{x}\cos(\sqrt{x})+2\cos(\sqrt{x})\big|_{0}^{\pi^{2}}
                    =-4
                \end{align*}
            \end{solution}
        \subsection{Exam I}
            \begin{problem}
                Using the Fundamental Theorem of Calculus,
                find $\frac{dy}{dx}$:
                \begin{enumerate}
                    \begin{multicols}{2}
                        \item $y=\int_{\sqrt{x}}^{0}\cos(t^{4})dt$
                        \item $y=\int_{5+2x^{2}}^{6}(e^{t}+2t)dt$
                    \end{multicols}
                \end{enumerate}
            \end{problem}
            \begin{solution}
                \
                \begin{enumerate}
                    \item
                        \begin{align*}
                            y&=\int_{\sqrt{x}}^{0}\cos(t^{4})dt
                            =-\int_{0}^{\sqrt{x}}\cos(t^{4})dt\Rightarrow
                            \frac{dy}{dx}=-\frac{d}{dx}
                            \int_{0}^{\sqrt{x}}\cos(t^{4})dt\\
                            &=\cos(x^{2})\frac{d}{dx}(\sqrt{x})
                            =-\frac{\cos(x^{2})}{2\sqrt{x}}
                        \end{align*}
                    \item
                        \begin{align*}
                            y&=\int_{5+2x^{2}}^{6}(e^{t}+2t)dt
                            =-\int_{6}^{5+2x^{2}}(e^{t}+2t)dt\Rightarrow
                            \frac{dy}{dx}
                            =-\frac{d}{dx}\int_{6}^{5+2x^{2}}(e^{t}+2t)dt\\
                            &=-[e^{5+2x^{2}}+2(5+2x^{2})]\frac{d}{dx}(5+2x^{2})
                            =-4x[e^{5+2x^{2}}+10+4x^{2}]
                        \end{align*}
                \end{enumerate}
            \end{solution}
            \begin{problem}
                Evaluate the integral:
                $\int(\frac{3}{x}-\frac{2}{x^{3}}-%
                 \sqrt[3]{x}+2e^{2x}+\cos(\pi x)-\pi)dx$
            \end{problem}
            \begin{solution}
                \begin{equation*}
                    \int(\frac{3}{x}-\frac{2}{x^{3}}-\sqrt[3]{x}+2e^{2x}+
                    \cos(\pi{x})-\pi)dx
                    =3\ln(|x|)+\frac{1}{x^{2}}-
                    \frac{3}{4}x^{\frac{4}{3}}+e^{2x}+
                    \frac{\sin(\pi x)}{\pi}-\pi{x}+C
                \end{equation*}
            \end{solution}
            \begin{problem}
                Evaluate the following integrals:
                \begin{enumerate}
                    \begin{multicols}{4}
                        \item $\int_{0}^{2}3x^{2}\sqrt{x^{3}+8}dx$
                        \item $\int\frac{e^{3y}}{e^{3y}-5)^{2}}dy$
                        \item $\int\frac{1}{\arctan^{2}(x)(1+x^{2})}dx$
                        \item $\int_{0}^{\pi/4}\sin^{3}(2\theta)\cos(2\theta)d\theta$
                    \end{multicols}
                \end{enumerate}
            \end{problem}
            \begin{solution}
                \
                \begin{enumerate}
                    \item
                        \begin{align*}
                            u&=x^{3}+8, du=3x^{2}dx\\
                            \int 3x^{2}\sqrt{x^{3}+8}dx
                            &=\int\sqrt{u}du
                            =\frac{3}{2}u^{\frac{3}{2}}+C
                            =\frac{3}{2}(x^{3}+8)^{\frac{3}{2}}+C\\
                            \Rightarrow\int_{0}^{2}3x^{2}\sqrt{x^{3}+8}dx
                            &=\frac{3}{2}(x^{3}+8)^{\frac{3}{2}}\big|_{0}^{2}
                            =\frac{3}{2}[(2^{2}+8)^{\frac{3}{2}}-
                            (8)^{\frac{3}{2}}]=\frac{32}{3}(4-\sqrt{2})
                        \end{align*}
                    \item
                        \begin{align*}
                            u&=e^{3y}-5,du=3e^{3y}dy
                            \Rightarrow{e}^{3y}dy
                            =\frac{1}{3}du\\
                            \int\frac{e^{3y}}{(e^{3y}-5)^{3}}dy
                            &=\frac{1}{3}\int\frac{du}{u^{3}}
                            =\frac{1}{3}\int u^{-3}du
                            =-\frac{1}{6}u^{-2}+C
                            =-\frac{1}{6}(e^{3y}-5)^{-2}+C
                        \end{align*}
                    \item
                        \begin{align*}
                            u&=\arctan(x),du=\frac{1}{1+x^{2}}dx\\
                            \int\frac{1}{\arctan^{2}(x)(1+x^{2})}dx
                            &=\int\frac{du}{u^{2}}
                            =-\frac{1}{u}+C
                            =-\frac{1}{\arctan(x)}+C
                        \end{align*}
                    \item
                        \begin{align*}
                            u=\sin(2\theta),
                            du=2\cos(2\theta)d\theta\Rightarrow
                            \cos(2\theta)d\theta&=\frac{1}{2}du,
                            \theta=0\Rightarrow{u}=0,
                            \theta=\frac{\pi}{4}\Rightarrow u=1\\
                            \int_{0}^{\frac{\pi}{4}}\sin^{3}(2\theta)
                            \cos(2\theta)d\theta
                            &=\frac{1}{2}\int_{0}^{1}u^{3}du
                            =\frac{1}{8}u^{4}\big|_{0}^{1}
                            =\frac{1}{8}
                        \end{align*}
                \end{enumerate}
            \end{solution}
            \begin{problem}
                Find the area of the region bounded
                by the curves of $y=x^{2}+4$ and $y=-x^{2}+6x+4$
            \end{problem}
            \begin{proof}[Solution]
                We first find the points of intersection
                of the two curves:
                \begin{equation*}
                    x^{2}+4=-x^{2}+6x+4\Rightarrow2x^{2}-6x=0
                    \Rightarrow{2}x(x-3)=0
                    \Rightarrow{x}=0,x=3
                \end{equation*}
                The area bounded by $f(x)=-x^{2}+6x+4$ and
                $g(x)=x^{2}+4$ is $\int_{0}^{3}(f(x)-g(x))dx$
                \begin{equation*}
                    A=\int_{0}^{3}(f(x)-g(x))dx
                    =\int_{0}^{3}(-2x^{2}+6x)dx
                    =[-\frac{2}{3}+3x^{2}]_{0}^{3}
                    =-\frac{2}{3}(3)^{3}+3(3)^{2}-0=9
                \end{equation*}
            \end{proof}
            \begin{problem}
                Evaluate $\int x^{2}e^{3x}dx$
            \end{problem}
            \begin{solution}
                \
                \begin{table}[H]
                    \centering
                    \begin{tabular}{|c|c|}
                        \hline
                        $u=x^{2}$&$dv=e^{3x}dx$\\
                        \hline
                        $du=2xdx$&$v=\frac{1}{3}e^{3x}$\\
                        \hline
                    \end{tabular}
                \end{table}
                \begin{equation*}
                    \int udv=uv-\int vdu
                    \Rightarrow\int x^{2}e^{3x}dx
                    =\frac{1}{3}x^{2}e^{3x}-\frac{2}{3}\int xe^{3x}dx
                \end{equation*}
                \begin{table}[H]
                    \centering
                    \begin{tabular}{|c|c|}
                        \hline
                        $u=x$&$dv=e^{3x}dx$\\
                        \hline
                        $du=dx$&$v=\frac{1}{3}e^{3x}$\\
                        \hline
                    \end{tabular}
                \end{table}
                \begin{align*}
                    \frac{1}{3}x^{2}e^{3x}-\frac{2}{3}\int xe^{3x}dx&=\frac{1}{3}x^{2}e^{3x}-\frac{2}{3}\big(\frac{1}{3}xe^{3x}-\frac{1}{3}\int e^{3x}dx\big)=\frac{1}{3}x^{2}e^{3x}-\frac{2}{3}\big(\frac{1}{3}xe^{3x}-\frac{1}{9}e^{3x}\big)+C\\
                    &=\frac{1}{3}e^{3x}(x^{2}-\frac{2}{3}x+\frac{2}{9})+C
                \end{align*}
            \end{solution}
            \begin{problem}
                Evaluate $\int_{0}^{4\pi^{2}}\sin(\sqrt{y})dy$
            \end{problem}
            \begin{solution}
                Let $x=\sqrt{y}$. Then
                $dx=\frac{1}{2\sqrt{y}}dy\Rightarrow dy=2\sqrt{y}dx=2xdx$.
                \begin{table}[H]
                    \centering
                    \begin{tabular}{|c|c|}
                        \hline
                        $u=x$&$dv=2\sin(x)dx$\\
                        \hline
                        $du=dx$&$v=-2\cos(x)$\\
                        \hline
                    \end{tabular}
                \end{table}
                \begin{align*}
                    \int\sin(\sqrt{y})dy
                    &=\int x\sin(y)dx
                    =-2x\cos(x)-\int(-2\cos(x))dx
                    =-2x\cos(x)+2\sin(x)\\
                    \Rightarrow \int_{0}^{4\pi^{2}}\sin(\sqrt{y})dy
                    &=[-2\sqrt{y}\cos(\sqrt{y})+
                    2\sin(\sqrt{y})]_{0}^{4\pi^{2}}
                    =-4\pi
                \end{align*}
            \end{solution}
            \begin{problem}
                Evaluate
                $\int_{0}^{\ln(\sqrt{3})}\frac{1}{\pi}\frac{e^{t}}{1+e^{2t}}dt$
            \end{problem}
            \begin{solution}
                Let $u=e^{t}$.
                Then $du=e^{t}dt$. So:
                \begin{align*}
                    \int\frac{1}{\pi}\frac{e^{t}}{1+e^{2t}}dt
                    &=\int\frac{1}{\pi}\frac{1}{1+u^{2}}du
                    =\frac{1}{\pi}\arctan(u)
                    =\frac{1}{\pi}\arctan(e^{t})\\
                    \Rightarrow\int_{0}^{\ln(\sqrt{3})}
                    \frac{1}{\pi}\frac{e^{t}}{1+e^{2t}}dt
                    &=\frac{1}{\pi}\arctan(e^{t})\big|_{0}^{\ln(\sqrt{3})}
                    =\frac{1}{\pi}(\arctan(e^{\ln(\sqrt{3})})-\arctan(e^{0}))\\
                    &=\frac{1}{\pi}(\arctan(\sqrt{3})-\arctan(1))
                    =\frac{1}{\pi}(\frac{\pi}{\pi}{3}-\frac{\pi}{4})
                    =\frac{1}{12}
                \end{align*}
            \end{solution}
            \newpage
        \subsection{Exam II}
            \begin{problem}
                Evaluate the integral $\int\sin^{5}(x)dx$
            \end{problem}
            \begin{solution}
                \begin{equation*}
                    \int\sin^{5}(x)dx
                    =\int\sin(x)\sin^{4}(x)dx
                    =\int\sin(x)(1-\cos^{2}(x))^{2}dx
                \end{equation*}
                Let $u=\cos(x)$. Then $du=-\sin(x)dx$.
                \begin{align*}
                    \int\sin^{5}(x)dx
                    &=-\int(1-u^{2})^{2}du
                    =-u+\frac{2}{3}u^{3}-\frac{1}{5}u^{5}+C\\
                    &=-\cos(x)+\frac{2}{3}\cos^{3}(x)-
                    \frac{1}{5}\cos^{5}(x)+C
                \end{align*}
            \end{solution}
            \begin{problem}
                Estimate the integral
                $\int_{0}^{4}(x^{3}+1)dx$ with $n=4$
                using the trapezoidal rule
                and Simpson's rule.
            \end{problem}
            \begin{solution}
                The integral is $f(x)=x^{3}+1$.
                We have that
                $\Delta x=\frac{b-a}{n}=\frac{4-0}{4}=1$
                \begin{enumerate}
                    \item Using the trapazoidal rule:
                    \begin{equation*}
                        T=\frac{\Delta{x}}{2}
                        [f(x_{0})+2f(x_{1})+2f(x_{2})+2f(x_{3})+f(x_{4})]
                        =\frac{1}{2}[1+2(2)+2(9)+2(28)+65]=\frac{144}{2}
                        =72
                    \end{equation*}
                    \item Using Simpson's Rule:
                    \begin{equation*}
                        T=\frac{\Delta{x}}{3}
                        [f(x_{0})+4f(x_{1})+2f(x_{2})+4f(x_{3})+f(x_{4})]
                        =\frac{1}{3}[1+4(2)+2(9)+4(28)+65]=\frac{204}{3}
                        =68
                    \end{equation*}
                \end{enumerate}
            \end{solution}
            \begin{problem}
                Evaluate the integral of
                $\int\frac{x^{2}}{x^{2}+25}dx$ using
                trigonometric substitution.
            \end{problem}