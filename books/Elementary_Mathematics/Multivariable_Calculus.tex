\chapter{Multivariable Calculus}
    \section{Divergence Free Vector Fields}
        Curl Free vector fields are differentiable functions
        $\mathbf{F}:\mathbb{R}^{3}\rightarrow\mathbb{R}^{3}$ such
        that $\curl(\mathbf{F})=\mathbf{0}$. These functions are
        also called conservative and can be represented as the
        gradient of a scalar function
        $\phi:\mathbb{R}^{3}\rightarrow\mathbb{R}$. Given such a
        potential, and a smooth curve
        $\Gamma:I\rightarrow\mathbb{R}^{3}$, the line integral
        can be computed as:
        \begin{equation}
            \int_{\Gamma}\mathbf{F}\cdot\hat{\mathbf{n}}\diff{s}
            =\int_{\Gamma}\grad(\phi)\cdot\hat{\mathbf{n}}\diff{s}
            =\phi\big(\Gamma(1)\big)-\phi\big(\Gamma(0)\big)
        \end{equation}
        A divergence free vector field is a function
        $\mathbf{F}:\mathbb{R}^{3}\rightarrow\mathbb{R}^{3}$ such
        that $\Div(\mathbf{F})=0$. Divergence free fields are also
        called solenoidal fields. Any such field can be written
        as the curl of another vector field,
        $\mathbf{F}=\curl(\mathbf{F})$. This is not unique, for given
        any differentiable scalar function
        $\phi:\mathbb{R}^{3}\rightarrow\mathbb{R}$, we have:
        \begin{equation}
            \mathbf{F}=\curl\big(\mathbf{G}+\grad(\phi)\big)
                =\curl(\mathbf{G})+\curl\big(\grad(\phi)\big)
                =\curl(\mathbf{G})
        \end{equation}
        Divergence free fields can have their surface integrals
        computed as:
        \begin{equation}
            \iint_{\Sigma}\mathbf{F}\cdot\hat{\mathbf{n}}\diff{S}=
            \oint_{\partial\Sigma}\mathbf{G}\cdot\mathbf{\diff{s}}
        \end{equation}
        This comes from Stokes' Theorem.
        \begin{lexample}
            Let $\mathbb{F}:\mathbb{R}^{3}\rightarrow\mathbb{R}^{3}$
            be the vector field defined by:
            \begin{equation}
                \mathbf{F}(x,y,z)
                    =(\minus{x}y,\frac{1}{2}y^{2}+2z,x^{2}y)
            \end{equation}
            Then $\mathbf{F}$ is a solenoidal vector field.
            Let $z=x^{2}+y^{3}$ be a surface in $\mathbb{R}^{3}$.
            We choose $\mathbf{G}$ such that:
            \begin{equation}
                \mathbf{G}=\mathbf{G}'+\grad(\phi)
            \end{equation}
            Where $\mathbf{G}=(g_{1}',g_{2}',g_{3}'$ and $\phi$ is
            the such $\grad(\phi)_{z}=\minus{g}_{3}'$. Thus, 
            $\mathbf{G}=(g_{1},g_{2},0)$. Note that we could've made
            $\mathbf{G}$ zero in any one of the three components.
            We obtain:
            \begin{equation}
                \mathbf{F}=\curl(\mathbf{G})=\Big(
                    \minus\frac{\partial{g}_{2}}{\partial{z}},
                    \frac{\partial{g}_{1}}{\partial{z}},
                    \frac{\partial{g}_{2}}{\partial{x}}-
                    \frac{\partial{g}_{1}}{\partial{y}}
                \Big)
            \end{equation}
            So, in our example we have:
            \begin{subequations}
                \begin{align}
                    g_{1}(x,y,z)
                        &=\frac{1}{2}y^{2}z+z^{2}+C_{1}(x,y)\\
                    g_{2}(x,y,z)
                        &=xyz+C_{2}(x,y)
                \end{align}
            \end{subequations}
            The functions $C_{1}$ and $C_{2}$ are constrained by:
            \begin{equation}
                \frac{\partial{C}_{2}}{\partial{x}}-
                \frac{\partial{C}_{1}}{\partial{y}}
                =x^{2}y
            \end{equation}
            By letting $C_{1}$ be the zero function, we obtain:
            \begin{equation}
                C_{2}=\frac{1}{3}x^{3}y
            \end{equation}
            Thus, $\mathbf{G}$ is:
            \begin{equation}
                \mathbf{G}(x,y,z)=\Big(
                    \frac{1}{2}y^{2}z+z^{2},
                    xyz+\frac{1}{3}x^{3}y, 0\Big)
            \end{equation}
            The flux through the surface can now be computed:
            \begin{align}
                \iint_{\Sigma}\mathbf{F}\cdot\hat{\mathbf{n}}
                    \diff{S}
                &=\int_{\partial\Sigma}
                    \mathbf{G}\cdot\mathbf{\diff{s}}\\
                &=\int_{\partial\Sigma}
                    \mathbf{G}\big(x(t),y(t),z(t)\big)\cdot
                        \mathbf{r}'(t)\diff{t}
            \end{align}
            Where:
            \begin{equation}
                \mathbf{r}(t)=\Big(
                    \big(1+\sin(t)\big)\cos(t),
                    \big(1+\sin(t)\big)\sin(t),
                    x^{2}(t)+y^{3}(t)\Big)
            \end{equation}
        \end{lexample}