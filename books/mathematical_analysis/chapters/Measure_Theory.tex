    \chapter{Measure Theory}
        \section{\texorpdfstring{$\sigma$}{Sigma}-Algebras}
            \subsection{Set Rings}
                Given a set $\Omega$, $\mathcal{P}(\Omega)$ is the
                set of all subsets of $\Omega$. Often this is too
                much, and too difficult to handle. Indeed, even
                $\mathcal{P}(\mathbb{R})$ is quite large and hard
                to get a grasp on. We wish to speak of collections
                of sets that have some structure on them.
                The first thing we will talk about is a set ring.
                \begin{ldefinition}{Set Ring}{Set_Ring}
                    A set ring of a set $\Omega$ is a nonempty subset
                    $\mathcal{R}\subseteq\mathcal{P}(\Omega)$ such
                    that for all
                    $A,B\in\mathcal{R}$, $A\cup{B}\in\mathcal{R}$,
                    and $A\setminus{B}\in\mathcal{R}$.
                \end{ldefinition}
                \begin{lexample}{}{Set Ring}
                    If $\Omega$ is a set, then
                    $\mathcal{P}(\Omega)$ is a set ring of
                    $\Omega$. So is the set $R=\{\emptyset$.
                    For any $A\subset\Omega$, the set
                    $R=\{A\}$ is also a set ring. If
                    $\Omega=\{1,2,3\}$, then
                    $R=\{\emptyset,\{1\},\{2,3\},\{1,2,3\}\}$ is
                    a set ring on $\Omega$.
                \end{lexample}
                \begin{lexample}{}{Bigger_Set_Ring}
                    If $\Omega$ is an infinite set, and if
                    $\mathcal{E}=\big\{\{x\}:x\in\Omega\big\}$, then the
                    smallest set ring that contains $\mathcal{E}$ is the set of
                    all finite subsets of $\Omega$. For the union of two finite
                    sets is finite, as is the set difference of two finite sets,
                    and thus this satisfies a set ring. Moreover, if $\mathcal{R}$
                    is a set ring that contains $\mathcal{E}$ then it contains the
                    union of any finite collection of elements in $\mathcal{E}$.
                    But $\mathcal{E}$ is the set of all of the singletons of
                    $\Omega$, and any finite subset of $\Omega$ can be written
                    as the union of finitely many singletons. Thus, $\mathcal{R}$
                    is the smallest set ring that contains $\mathcal{E}$.
                \end{lexample}
                \begin{theorem}
                    If $\Omega$ is a set, if $R$ is a set ring
                    on $\Omega$, and if $A$ is a finite subset of
                    $R$, then $\cup_{\alpha\in{A}}\alpha$ is an
                    element of $R$.
                \end{theorem}
                \begin{proof}
                    Apply induction to the closure of unions.
                \end{proof}
                \begin{theorem}
                    If $\Omega$ is a set, if $R$ is a set ring on
                    $\Omega$, and if $A,B\in{R}$, then
                    $A\cup{B}\in{R}$.
                \end{theorem}
                \begin{proof}
                    For $A\cap{B}=A\setminus(A\setminus{B})$, and
                    from the closure of set difference,
                    $A\cap{B}\in{R}$.
                \end{proof}
                \begin{theorem}
                    If $\Omega$ is a set, if $R$ is a set ring
                    on $\Omega$, and if $A$ is a finite subset of
                    $R$, then $\cap_{\alpha\in{A}}\alpha$ is an
                    element of $R$.
                \end{theorem}
                \begin{proof}
                    Apply induction to the closure of intersections.
                \end{proof}
                \begin{theorem}
                    If $\Omega$ is a set, if $R$ is a set ring on
                    $\Omega$, if $A,B\subset\Omega$, and if
                    $A\setminus{B}$, $B\setminus{A}$, and
                    $A\cap{B}$ are elements of $R$, then
                    $A,B\in{R}$.
                \end{theorem}
                Thus, the set ring generated by the set $\{A,B\}$ and
                the set ring generated by
                $\{A\setminus{B},B\setminus{A},A\cap{B}\}$ are the
                same.
                \begin{theorem}
                    If $\Omega$ is a set and $R$ is a set ring
                    of $\Omega$, then $\emptyset\in{R}$.
                \end{theorem}
                \begin{proof}
                    For as $R$ is non-empty, there is an element
                    $A\in{R}$. If $A=\emptyset$, then we are done.
                    If not, as $R$ is closed under set difference,
                    $A\setminus{A}\in{R}$. But
                    $A\setminus{A}=\emptyset$.
                \end{proof}
                From this, if we have a collection $R$ of subsets of
                $\Omega$ and we wish to check if $R$ is a set ring
                of $\Omega$, there are several redundant operations
                we don't need to check. Since, for any set $A$,
                we have:
                \begin{align}
                    A\setminus\emptyset&=A\\
                    A\setminus{A}&=\emptyset\\
                    \emptyset\setminus{A}&=\emptyset\\
                    A\cup{A}&=A\\
                    A\cup\emptyset&=A\\
                    \emptyset\cup\emptyset&=\emptyset
                \end{align}
                Using our previous example $\Omega=\{1,2,3\}$,
                we can check laboriously that
                $R=\{\emptyset,\{1\},\{2,3\},\{1,2,3\}\}$ is a
                set ring on $\Omega$. The set
                $\{\emptyset,\{1\},\{2\},\{1,2,3\}\}$ is not
                a set ring, for $\{1,2\}=\{1\}\cup\{2\}$ is not
                an element.
                \begin{theorem}
                    If $\Omega$ is a set, and if $A$ and $B$ are
                    disjoint subsets of $\Omega$, then
                    $R=\{\emptyset,A,B,A\cup{B}\}$ is a set ring
                    on $\Omega$.
                \end{theorem}
                \begin{theorem}
                    If $\Omega$ is a set, if $A$ and $B$ are
                    disjoint subsets of $\Omega$, and if
                    $R$ is a set ring such that $A,B\in{R}$,
                    then $\{emptyset,A,B,A\cup{B}\}\subset{R}$.
                \end{theorem}
                As such, the set ring $\{\emptyset,A,B,A\cup{B}\}$
                is called the set ring generated by $A$ and $B$. We
                can continue and consider the case of three mutually
                disjoint subsets.
                \begin{theorem}
                    If $\Omega$ is a set, and $A_{1},A_{2},A_{3}$ are
                    mutually disjoint subsets of $\Omega$, then:
                    \begin{equation}
                        R=\{\emptyset,A_{1},A_{2},A_{3},
                            A_{1}\cup{A}_{2},A_{1}\cup{A}_{3},
                            A_{2}\cup{A}_{3},
                            A_{1}\cup{A}_{2}\cup{A}_{3}\}
                    \end{equation}
                    is a set ring on $\Omega$.
                \end{theorem}
                Indeed, we may generalize further.
                \begin{theorem}
                    If $\Omega$ is a set and if
                    $A$ is a subset of $\mathcal{P}(\Omega)$ of
                    $n$ elements such that, for all
                    $a,b\in{A}$, $a\cap{B}=\emptyset$, then:
                    \begin{equation}
                        R=\{\cup_{i\in{I}}A_{i}:
                        I\in\mathcal{P}(\mathbb{Z}_{n})\}
                    \end{equation}
                    Is a set ring on $\Omega$.
                \end{theorem}
                \begin{theorem}
                    If $\Omega$ is a set, then the set of all
                    finite subsets of $\Omega$ is a set ring on
                    $\Omega$.
                \end{theorem}
                A left semi-interval of $\mathbb{R}$ is an interval
                of the form $[a,b)$ where $a\leq{b}$. If $a=b$, this
                is the empty set. The set of all left semi-intervals
                is not a set ring on $\mathbb{R}$ since the union
                of two semi-intervals need not be a semi-interval.
                We need to add more sets to allow this to be a
                set ring. The collection of all finite unions of
                semi-intervals of $\mathbb{R}$ is a set ring.
                First, note the following:
                \begin{equation}
                    \Big(\bigcup_{n=1}^{N}[a_{n},b_{n})\Big)
                    \setminus[c,d)=\bigcup_{n=1}^{N}
                    \Big([a_{n},b_{n})\setminus[c,d)]
                \end{equation}
                This is again the finite union of intervals. By
                induction we see that this collection is a ring on
                $\mathbb{R}$. We have seen that a set ring is
                closed to unions and set differences, and this
                implies that rings are closed under intersections and
                closed under symmetric differences. As it turns out,
                this is an equivalent definition of a set ring.
                \begin{theorem}
                    If $\Omega$ is a set and
                    $R\subset\mathcal{P}(\Omega)$, then $R$ is
                    a set ring of $\Omega$ if and only if $R$ is
                    closed under symmetric differences and
                    intersections.
                \end{theorem}
                If $R$ is a set ring on $\Omega$, and if
                $A\in{R}$, let $\chi_{A}:\Omega\rightarrow[0,1]$ be
                the indicator function defined as follows:
                \begin{equation}
                    \chi_{A}(\omega)=
                    \begin{cases}
                        0,&\omega\notin{A}\\
                        1,&\omega\in{A}
                    \end{cases}
                \end{equation}
                Then we have:
                \begin{align}
                    \chi_{A\cap{B}}(\omega)
                    &=\chi_{A}(\omega)\chi_{B}(\omega)\\
                    \chi_{A\ominus{B}}&=
                    \big(\chi_{A}(\omega)+\chi_{B}(\omega)\big)
                    \mod{2}
                \end{align}
                These two operations satisfy the axioms of a ring.
                That is, a ring in the algebraic sense of the word:
                A set with two operations that behave certain
                properties. It is because of this that set rings
                have earned their name.
            \subsection{Set Algebras}
                \begin{definition}
                    A set algebra on a set $\Omega$ is a set ring
                    on $\Omega$ such that $\Omega\in\mathcal{A}$.
                \end{definition}
                \begin{lexample}{}{Set_Algebra}
                    Let $\Omega=\{1,2,3,4\}$ and
                    $R=\{\emptyset,\{1\},\{2,3\}\}$. Then $R$
                    is a set ring, but it is not a set algebra
                    since $\Omega\notin{R}$.
                \end{lexample}
                \begin{lexample}{}{Big_Set_Algebra}
                    If $\Omega$ is an infinite set, and if
                    $\mathcal{E}=\big\{\{x\}:x\in\Omega\big\}$, then
                    the smallest set algebra that contains $\mathcal{E}$
                    is the set of all finite and co-finite subsets of
                    $\Omega$. There are a few cases to check. The finite
                    union of finite subsets is finite, the finite union of
                    co-finite subsets is co-finite, and the finite union
                    of finite and co-finite is again co-finite. For set
                    difference, the difference of finite with finite is
                    again finite, and the difference of co-finite with
                    co-finite is either co-finite or finite. The
                    difference of co-finite with finite is co-finite,
                    and the difference of finite with co-finite is finite.
                    Thus, this set is a set algebra on $\Omega$. Moreoever
                    it is the smallest set algebra that will contain $\mathcal{E}$.
                \end{lexample}
                From the definition, we see that a set algebra
                is closed under complements. indeed, this creates
                and equivalent definition for set algebras.
                \begin{theorem}
                    If $\Omega$ is a set and
                    $\mathcal{A}\subseteq\mathcal{P}(\Omega)$,
                    then $\mathcal{A}$ is a set algebra on $\Omega$
                    if and only if $\Omega\in\mathcal{A}$, and
                    $\mathcal{A}$ is closed under union and
                    complement.
                \end{theorem}
                \begin{theorem}
                    If $\Omega$ is a set and $R$ is a set ring
                    on $\Omega$, and if $\mathcal{A}$ is a set
                    algebra on $\Omega$ such that
                    $R\subset\mathcal{A}$, then for all $A\in{R}$,
                    $A\in\mathcal{A}$ and $A^{C}\in\mathcal{A}$.
                \end{theorem}
                This then defines the \textit{smallest} set algebra
                that contains a set ring.
                \begin{theorem}
                    If $\Omega$ is a set and $R$ is a set ring on
                    $\Omega$, then:
                    \begin{equation}
                        \mathcal{A}=\{A,A^{C}:A\in{R}\}
                    \end{equation}
                    Is a set algebra on $\Omega$.
                \end{theorem}
                \begin{theorem}
                    If $\Omega$ is a set and $A$ and $B$ are
                    disjoint subset of $A$, then:
                    \begin{equation}
                        \mathcal{A}=
                            \{\emptyset,A,B,A\cup{B},
                              \Omega,A^{C},B^{C},A^{C}\cap{B}^{C}\}
                    \end{equation}
                    is a set algebra on $\Omega$.
                \end{theorem}
                For non-disjoint $A$ and $B$, the smallest
                set algebra becomes more complicated. We saw that
                the collection of all finite subsets of a set is
                a set ring on the set. The collection of all finite
                subsets, and their complements, is a set algebra.
            \subsection{\texorpdfstring{$\sigma$}{Sigma}-Rings}
                If $\Omega$ is a set, then $R\subset\mathcal{P}(\Omega)$
                is called a set ring on $\Omega$ if, for all
                $A,B\in{R}$, $A\cup{B}\in{R}$ and
                $A\setminus{B}\in{R}$. From this, given a ring $R$ on
                $\Omega$, the empty set is included, that is
                $\emptyset\in{R}$, and if $A,B\in{R}$, then
                $A\cap{B}\in{R}$. By induction, for any finite collection
                of elements in $R$, the union of these subsets is also
                contained in $R$, as well as the intersection. A set
                algebra on $\Omega$ is a ring $\mathcal{A}$
                on $\Omega$ such that $\Omega\in\mathcal{A}$. That is,
                $\mathcal{A}\subset\mathcal{P}(\Omega)$, and
                $\mathcal{A}$ is closed under union, set difference, and
                $\Omega\in\mathcal{A}$. There is an equivalent definition:
                $\Omega\in\mathcal{A}$, for all $A\in\mathcal{A}$,
                $A^{C}\in\mathcal{A}$, and for all $A,B\in\mathcal{A}$,
                $A\cup{B}\in\mathcal{A}$. The complement of $A$,
                $A^{C}$, is defined as $\Omega\setminus{A}$. The
                equivalence of the two definitions comes from DeMorgan's
                laws, since
                $A\setminus{B}=A\cap{B}^{C}=(A^{C}\cup{B})^{C}$. We now
                talk about $\sigma$-Ring.
                \begin{definition}
                    A $\sigma$-Ring on a set $\Omega$ is a set
                    $\sigma\subset\mathcal{P}(\Omega)$ such that,
                    for all countable subsets of $\sigma$, the union
                    $\bigcup_{i=1}^{\infty}A_{i}\in\sigma$, and for all
                    $A,B\in\sigma$, $A\setminus{B}\in\sigma$.
                \end{definition}
                The requirement that the collections be countable is
                important to note. A \textit{topology} is a subset
                of $\mathcal{P}(\Omega)$ with the property that it is
                closed under arbitrary unions. $\sigma$-Rings need only
                be closed under countable unions.
                \begin{example}
                    Every $\sigma$-Ring is a set ring, but not every
                    set ring is a $\sigma$-ring. Let $\Omega$ be
                    uncountable, and let $R$ be the set of all finite
                    subsets of $\Omega$. Then $R$ is a ring, but it is
                    not a $\sigma$-ring. For, as $\Omega$ is uncountably
                    infinite, it has a countable subset $A$, and we
                    may subscript the elements as $a_{n}$. But
                    $\bigcup_{n=1}^{\infty}\{a_{n}\}$ is not a finite
                    subset of $\Omega$, and is therefore not contained
                    in $R$. Thus, $R$ is not closed under countable unions
                    and $R$ is not a $\sigma$-ring. However, if we let
                    $\sigma$ be the set of all \textit{countable} subsets
                    of $\Omega$, the $\sigma$ is indeed a $\sigma$-ring.
                \end{example}
                \begin{lexample}{}{Ring_From_Semi-Intervals}
                    The collection of all semi-intervals and finite
                    unions of semi-intervals defines a ring on
                    $\mathbb{R}$. It is tempting to think tha the
                    collection of all countable unions of semi-intervals
                    is a $\sigma$-ring on $\mathbb{R}$, but this is not
                    the case. The Cantor set is an example of a subset
                    that can be constructed by a countable number of
                    steps of removing intervals from a given interval,
                    but the resulting set is not the countable union of
                    semi-intervals. To construct the Cantor set, consider
                    the interval $[0,1]$. From this, remove
                    $(\frac{1}{3},\frac{2}{3})$. Continuing removing the
                    middle third from each sub-interval obtained. The
                    resulting set contains no interval as a subset, and
                    thus cannot be the union of countably many intervals,
                    or semi-intervals.
                \end{lexample}
            \subsection{Dynkin System}
                A Dynkin system on a set $\Omega$ is a subset
                $\mathcal{D}\subset\mathcal{P}(\Omega)$ such that
                $\Omega\in\mathcal{D}$, if $A,B\in\Omega$ and if
                $A\subseteq{B}$, then $A\setminus{B}\in\mathcal{D}$,
                and for all countable collections of elements of
                $\mathcal{D}$ such that
                $A_{1}\subset{A}_{2}\subset\hdots$,
                $\cup_{n=1}^{\infty}A_{n}\in\mathcal{D}$. There is
                an equivalent defintion for Dynkin Systems.
                $\Omega\in\mathcal{D}$, $A\in\mathcal{D}$ implies
                $A^{C}\in\mathcal{D}$, and for all countable disjoint
                collections of elements in $\mathcal{D}$, the union
                is also contained in $\mathcal{D}$. These requirements
                are weaker than those of a $\sigma$-algebra. Any
                $\sigma$-algebra is a Dynkin system, but not every
                Dynkin system is a $\sigma$-algebra.
                \begin{ldefinition}{Dynkin System}
                    A Dynkin System on a set $\Omega$ is a subset
                    $\mathcal{D}\subseteq\mathcal{P}(\Omega)$ such that:
                    \begin{enumerate}
                        \item $\Omega\in\mathcal{D}$.
                        \item For all $A,B\in\mathcal{D}$ such that $A\subseteq{B}$,
                              $B\setminus{A}\in\mathcal{D}$.
                        \item For any sequence $A_{n}\in\mathcal{D}$ such that
                              $A_{n}\subseteq{A}_{n+1}$,
                              $\cup_{n=1}^{\infty}A_{n}\in\mathcal{D}$
                    \end{enumerate}
                \end{ldefinition}
                \begin{theorem}
                    If $\Omega$ is a set and $\mathcal{D}\subseteq\mathcal{P}(\Omega)$
                    is such that $\Omega\in\mathcal{D}$, for all $A\in\mathcal{D}$,
                    $A^{C}\in\mathcal{D}$, and if for all sequences $A_{n}\in\mathcal{D}$
                    such that $A_{n}\cap{A}_{m}=\emptyset$ for all $n\ne{m}$,
                    $\cup_{n=1}^{\infty}A_{n}\in\mathcal{D}$, then
                    $\mathcal{D}$ is a Dynkin System on $\Omega$.
                \end{theorem}
                \begin{theorem}
                    If $\mathcal{D}$ is a Dynkin system on a set
                    $\Omega$, and if $\mathcal{D}$ is closed with
                    respect to intersections, then $\mathcal{D}$
                    is a $\sigma$-algebra on $\Omega$.
                \end{theorem}
                \begin{theorem}
                    If $\Omega$ is a set, if
                    $\mathcal{E}\subset\mathcal{P}(\Omega)$ is closed
                    to intersections, and if $\mathcal{D}$ is the
                    Dynkin System generated by $\mathcal{E}$, then
                    $\mathcal{D}$ is a $\sigma$-algebra.
                \end{theorem}
                \begin{theorem}[Dynkin's Theorem]
                    If $\Omega$ is a set, $\mathcal{C}\subseteq\mathcal{P}(\Omega)$
                    is intersection-stable, and if $\mathcal{D}$ is the smallest
                    Dynkin system that contains $\mathcal{C}$, then $\mathcal{D}$
                    is also intersection-stable.
                \end{theorem}
                \begin{proof}
                    For let:
                    \begin{equation}
                        \mathcal{D}_{1}=
                        \{D\in\mathcal{D}:\forall_{C\in\mathcal{C}},D\cap{D}\in\mathcal{D}\}
                    \end{equation}
                    Then $\mathcal{D}_{1}$ is a Dynkin system, and thus
                    $\mathcal{D}_{1}=\mathcal{D}$. Now define:
                    \begin{equation}
                        \mathcal{D}_{2}=\{
                            D\in\mathcal{D}:\forall_{A\in\mathcal{D}},D\cap{A}\in\mathcal{D}
                        \}
                    \end{equation}
                    Then $\mathcal{C}\subseteq\mathcal{D}_{2}$ and $\mathcal{D}_{2}$ is a
                    Dynkin System, and thus $\mathcal{D}_{2}=\mathcal{D}$.
                \end{proof}
                \begin{theorem}
                    If $\Omega$ is a set, $\mathcal{C}\subseteq\mathcal{P}(\Omega)$
                    is intersection-stable, and if $\mathcal{D}$ is the smallest
                    Dynkin system that contains $\mathcal{C}$, then $\mathcal{D}$
                    is a $\sigma\textrm{-Algebra}$ on $\Omega$.
                \end{theorem}
                Since semi-intervals are closed to intersections,
                the Borel $\sigma$-algebra is equivalently the
                Dynkin system generated by semi-intervals.
            \subsection{\texorpdfstring{$\sigma$}{Sigma}-Algebras}
                In an analogous manner to how set rings and set algebras
                were defined, there is something called a $\sigma$-algebra.
                This notion will be one of the central themes of measure
                theory.
                \begin{definition}
                    A $\sigma$-algebra on a set $\Omega$ is a
                    $\sigma$-ring on $\Omega$ such that
                    $\Omega\in\mathcal{A}$
                \end{definition}
                That is, given any countable collection of elements in
                $\mathcal{A}$, the union is also contained in
                $\mathcal{A}$. In addition, $\mathcal{A}$ is closed under
                set differences and $\Omega\in\mathcal{A}$.
                \begin{example}
                    The first trivial example is the power set
                    $\mathcal{P}(\Omega)$. Also the set
                    $\{\emptyset,\Omega\}$ defines a $\sigma$-algebra on
                    $\Omega$. The set of all countable subsets defines
                    a $\sigma$-ring, and the set of all countable and
                    co-countable (Sets whose complement is countable)
                    will define a $\sigma$-algebra.
                \end{example}
                We can equivalently define a $\sigma$-algebra to be a
                collection of sets $\mathcal{A}$ such that
                $\Omega\in\mathcal{A}$, for all $A\in\mathcal{A}$,
                $A^{C}\in\mathcal{A}$, and $\mathcal{A}$ is closed under
                countable unions. Being closed under countable unions
                implies that it is closed under finite unions as well.
                For let $A_{1}=A$, and for all $n>1$, let $A_{n}=B$.
                Then $\bigcup_{n=1}^{\infty}A_{n}=A\cup{B}$. By induction,
                a $\sigma$-algebra is closed under any finite union.
            \subsection{Borel \texorpdfstring{$\sigma$}{Sigma}-Algebra}
                One of the most important types of $\sigma$-algebras
                is the Borel $\sigma$-algebra. We first define the
                Borel $\sigma$-algebra on $\mathbb{R}$.
                \begin{definition}
                    The Borel $\sigma$-algebra on $\mathbb{R}$, denoted
                    $\mathcal{B}$, is the $\sigma$-algebra generated
                    by the set $\{[a,b):a,b\in\mathbb{R}\}$.
                \end{definition}
                That is, the Borel $\sigma$-algebra on $\mathbb{R}$ is
                the \textit{smallest} $\sigma$-algebra that contains
                all of the semi-intervals. We can equivalently say that
                $\mathcal{B}$ is the $\sigma$-algebra generated by all
                \textit{open} intervals. If we know that every open
                subset of $\mathbb{R}$ is the countable union of open
                subsets, than we can equivalently say that
                $\mathcal{B}$ is the $\sigma$-algebra generated by all
                \textit{open subsets} of $\mathbb{R}$. Writing $[a,b)$
                as the countable intersection of open intervals, or
                $(a,b)$ as the countable union of semi-intervals comes
                from the Archimedean property of the real numbers.
                Thus, the smallest $\sigma$-algebra that contains all
                semi-intervals is the smallest $\sigma$-algebra that
                contains all open intervals, which
                is the smallest $\sigma$-algebra that contains all open
                subsets of $\mathbb{R}$. Similarly, this will contain all
                of the \textit{closed} intervals, intervals of the form
                $[a,b]$. We say that a set $\mathcal{U}\subset\mathbb{R}$
                is open if, for all $x\in\mathcal{U}$, there is an $r>0$
                such that $(x-r,x+r)\subset\mathcal{U}$. That is, every
                point in $\mathcal{U}$ can be surrounded by an interval
                that is entirely contained in $\mathcal{U}$. Thus, any
                open set can be written as:
                \begin{equation}
                    \mathcal{U}=
                        \bigcup_{x\in\mathcal{U}}(\alpha_{x},\beta_{x})
                \end{equation}
                This union is not countable, for any open set must
                contain an interval, an intervals are uncountable large.
                This is simply because $(a,b)$ is equivalent to $(0,1)$.
                By adjusting the size of $\alpha_{x}$ and $\beta_{x}$ to
                be rational numbers, we can written $\mathcal{U}$ as the
                union of intervals with rational endpoints. But there are
                only countably many such intervals, and thus
                $\mathcal{U}$ is the union of countably many open
                intervals. Thus, any open set is the union of countably
                many open intervals. From this, the smallest
                $\sigma$-algebra that contains open intervals will contain
                all open sets, since $\sigma$-algebras are closed under
                countable unions. Borel sets are elements of the
                Borel $\sigma$-algebra $\mathcal{B}$. Since all open
                sets are Borel sets, and as $\sigma$-algebras are closed
                under complenents, all closed sets are also Borel sets.
                This is because the complement of an open set is a closed
                set, and vice versa. Thus, equivalently, $\mathcal{B}$ is
                the smallest $\sigma$-algebra containing all closed sets.
                A $G_{\delta}$ sets is a subset that is the countable
                intersection of open sets. As open sets are not
                necessarily closed under countable intersections, not
                all $G_{\delta}$ sets are open. There is another notion,
        \section{Measures}
            \subsection{A Review Infinite Series}
                Given a sequence of real numbers,
                $a:\mathbb{N}\rightarrow\mathbb{R}$, the sum of this
                sequence is defined as the limit of
                finite partial sums. That is:
                \begin{equation}
                    \sum_{n=1}^{\infty}a_{n}=
                    \underset{N\rightarrow\infty}{\lim}
                        \sum_{n=1}^{N}a_{n}
                \end{equation}
                In general, this limit may not in general exists. If it
                does, we say the series converges. If the limit does
                not exists, we do not define the sum and instead just
                have a meaningless combination of symbols. If the
                sequence is positive, then the sequence of partial sums
                will be increasing. If this sequence is bounded, then
                the limit exists. This comes from the fact that bounded
                monotonic sequences converge, a result that stems from
                the least upper bound property of $\mathbb{R}$.
                Moreover, if $a:\mathbb{N}\rightarrow\mathbb{R}$ is a
                sequence of positive real numbers, and if
                $f:\mathbb{N}\rightarrow\mathbb{N}$ is any bijective
                function, then the following is true:
                \begin{equation}
                    \sum_{n=1}^{\infty}a_{n}
                    =\sum_{n=1}^{\infty}a_{f(n)}
                \end{equation}
                We can also split the sequence into a grid,
                and take the
                double sum, obtaining the same result. If
                $A_{1},A_{2},\hdots$ are disjoint sets whose union is
                $\mathbb{N}$, and if $b_{nm}$ is the $n^{th}$ element
                of $A_{m}$, then:
                \begin{equation}
                    \sum_{i=1}^{\infty}a_{i}=
                    \sum_{n=1}^{\infty}\sum_{m=1}^{\infty}b_{nm}
                \end{equation}
                We should be precise in what we mean. The double
                sum is the \textit{limit of a limit}.
                \begin{equation}
                    \sum_{n=1}^{\infty}\sum_{m=1}^{\infty}a_{nm}
                    =\underset{N\rightarrow\infty}{\lim}\sum_{n=1}^{N}
                    \Big(\underset{M\rightarrow\infty}{\lim}
                    \sum_{m=1}^{M}a_{nm}\Big)
                \end{equation}
                We use infinite series to define \textit{measures} on
                $\sigma$-algebra.
            \subsection{Measure Functions}
                A set function on a collection of sets $\mathcal{E}$
                is a function $\mu:\mathcal{E}\rightarrow\mathbb{R}$.
                For example, if we consider the set of all
                semi-intervals of the form $[a,b)$, where
                $a,b\in\mathbb{R}$ and $a\leq{b}$, then we can define
                $\mu([a,b))=b-a$. This gives rise to the notion of
                a measure function.
                A measure function on a collection of set
                $\mathcal{E}$ is a function
                $\mu:\mathcal{E}\rightarrow\mathbb{R}$ such that:
                \begin{enumerate}
                    \item If $\emptyset\in\mathcal{E}$, then
                          $\mu(\emptyset)=0$
                    \item For all $A\in\mathcal{E}$, $\mu(A)\geq{0}$
                    \item For any countable collection of pair-wise
                          disjoint sets whose
                          union also lies in $\mathcal{E}$,
                          $\mu(\cup_{n=1}^{\infty}A_{n})=%
                           \sum_{n=1}^{\infty}\mu(A_{n})$
                \end{enumerate}
                It helps if we don't have to consider the case where
                $\mu(\emptyset)$ is undefined, or where we don't have
                closure under countable unions, so we discuss measure
                functions on $\sigma$-algebras.
                \begin{definition}
                    A measure on a $\sigma$-algebra
                    $\mathcal{A}$ is a function
                    $\mu:\mathcal{A}\rightarrow\mathbb{R}$ such that:
                    \begin{enumerate}
                        \item $\mu(\emptyset)=0$
                        \item For all $A\in\mathcal{A}$,
                              $\mu(A)\geq{0}$
                        \item For any countable collection of pairwise
                              disjoint elements of $\mathcal{A}$,
                              $\mu(\cup_{n=1}^{\infty}A_{n})=%
                               \sum_{n=1}^{\infty}\mu(A_{n})$
                    \end{enumerate}
                \end{definition}
                \begin{example}
                    Let $\Omega$ be a set, and let
                    $\mathcal{A}=\mathcal{P}(\Omega)$. Then
                    $\mathcal{A}$ is a $\sigma$-algebra on $\Omega$.
                    If $\omega_{1},\hdots,\omega_{n}\in\Omega$ and if
                    $p_{1},\hdots,p_{n}\in\mathbb{R}^{+}$, then:
                    \begin{equation}
                        \mu(A)=\sum_{k=1}^{n}p_{k}\chi_{A}(\omega_{k})
                    \end{equation}
                    Where $\xi_{A}$ is the indicator function:
                    \begin{equation}
                        \chi_{A}(\omega)=
                        \begin{cases}
                            0,&\omega\notin{A}\\
                            1,&\omega\in{A}
                        \end{cases}
                    \end{equation}
                    This is an example of a \textit{point measure}
                    on $\mathcal{A}$. It defines a measure function.
                \end{example}
            A $\sigma\text{-Algebra}$ on a set $\Omega$ is a subset
            $\mathcal{A}$ of $\mathcal{P}(\Omega)$ such that
            $\Omega\in\mathcal{A}$ and for any countable collection of
            elements $A_{i}\in\mathcal{A}$, the union
            $\bigcup_{i=1}^{\infty}A_{i}$ is also contained in
            $\mathcal{A}$. $\mathcal{A}$ does not have to consist of
            countably many elements. The sequence of subset $A_{i}$ does
            not have to exhaust the entirety of $\mathcal{A}$, much the
            way that any sequence of real numbers will not exhaust the
            entire of $\mathbb{R}$. Going in the other direction,
            $\sigma\text{-Algebras}$ can be finite. If $\Omega$ is a
            set, and if $A\subset\Omega$ is non-empty, then
            $\mathcal{A}=\{\emptyset,A,A^{C},\Omega\}$ defines a
            $\sigma\text{-algebra}$ on $\Omega$. A measure on a
            $\sigma\text{-Algebra}$ $\mathcal{A}$ is a function
            $\mu:\mathcal{A}\rightarrow\mathbb{R}$ such that, for all
            $A\in\mathcal{A}$, $\mu(A)\geq{0}$, $\mu(\emptyset)=0$, and
            given a mutually disjoint countable collection of elements of
            $\mathcal{A}$, the following holds:
            \begin{equation}
                \mu\Big(\bigcup_{i=1}^{\infty}A_{i}\Big)
                =\sum_{n=1}^{\infty}\mu(A_{i})
            \end{equation}
            \begin{example}
                A pure point measure is a measure that assigns to a
                collection of elements $\omega_{j}\in\Omega$ a positive
                real number $p_{j}$, and then the measure of any set
                $A$ is:
                \begin{equation}
                    \mu(A)=\sum_{j:\omega_{j}\in{A}}p_{j}
                \end{equation}
            \end{example}
            \subsection{Properties of Measure}
                \subsubsection{Monotonicity}
                    If $A$ and $B$ are elements of a $\sigma\text{-Algebra}$
                    $\mathcal{A}$, if $\mu$ is a measure on
                    $\mathcal{A}$, and if $A\subseteq{B}$, then
                    $\mu(A)\leq\mu(B)$. This is the monotonic property
                    of measures.
                    \begin{theorem}
                        If $\Omega$ is a set, $\mathcal{A}$ is a
                        $\sigma\text{-Algebra}$ on $\Omega$, if
                        $\mu$ is a measure on $\mathcal{A}$, and if
                        $A,B$ are elements of $\mathcal{A}$ such that
                        $A\subseteq{B}$, then $\mu(A)\leq\mu(B)$.
                    \end{theorem}
                    \begin{proof}
                        For as $\mathcal{A}$ is a $\sigma\text{-Algebra}$
                        on $\Omega$, and as $A,B\in\mathcal{A}$,
                        $B\setminus{A}\in\mathcal{A}$. But, as
                        $A\subseteq{B}$, $B=(B\setminus{A})\cup{A}$.
                        But then, as measures are additive and positive:
                        \begin{align}
                            \mu(B)&=\mu\big((B\setminus{A})\cup{A}\big)\\
                            &=\mu(B\setminus{A})+\mu(A)\\
                            &\geq\mu(A)
                        \end{align}
                    \end{proof}
                    \begin{theorem}
                        If $\Omega$ is a set, $\mathcal{A}$ is a
                        $\sigma\text{-Algebra}$ on $\Omega$, if
                        $\mu$ is a measure on $\mathcal{A}$, and if
                        $A,B$ are elements of $\mathcal{A}$ such that
                        $A\subseteq{B}$ and $\mu(A),\mu(B)<\infty$,
                        then $\mu\big(B\setminus{A}\big)=\mu(B)-\mu(A)$.
                    \end{theorem}
                \subsubsection{Continuity Theorems}
                    \begin{theorem}[Continuity from Below]
                        If $\Omega$ is a set, $\mathcal{A}$ is a
                        $\sigma\text{-Algebra}$ on $\Omega$, if
                        $\mu$ is a measure on $\mathcal{A}$, and if
                        $A_{i}$ is a sequence of elements in $\mathcal{A}$
                        such that, for all $i\in\mathbb{N}$,
                        $A_{i}\subseteq{A}_{i+1}$, then:
                        \begin{equation}
                            \mu\Big(\bigcup_{i=1}^{\infty}A_{i}\Big)
                            =\underset{N\rightarrow\infty}{\lim}
                            \mu(A_{N})
                        \end{equation}
                    \end{theorem}
                    \begin{proof}
                        For let $A=\cup_{n=1}^{\infty}A_{n}$ and let
                        $B_{n}=A_{n+1}\setminus{A}_{n}$. Then, as
                        $A_{n}\subseteq{A}_{n+1}$, for all
                        $i,j\in\mathbb{N}$, $B_{i}\cap{B}_{j}=\emptyset$.
                        But $A=A_{1}\cup\Big(\cup_{n=1}^{\infty}B_{n}\Big)$
                        and this is the countable union of mutually
                        disjoint sets, and therefore, using the
                        telescoping series:
                        \begin{align}
                            \mu(A)&=\mu(A_{1})+
                            \sum_{n=1}^{\infty}\mu(B_{n})\\
                            &=\mu(A_{1})+\sum_{n=1}^{\infty}
                            \Big(\mu(A_{n+1})-\mu(A_{n})\Big)\\
                            &=\mu(A_{1})+
                            \underset{N\rightarrow\infty}{\lim}
                            \Big(\mu(A_{N})-\mu(A_{1})\Big)\\
                            &=\underset{N\rightarrow\infty}{\lim}
                            \mu(A_{N})
                        \end{align}
                    \end{proof}
                    \begin{theorem}[Continuity from Above]
                        If $\Omega$ is a set, $\mathcal{A}$ is a
                        $\sigma\text{-Algebra}$ on $\Omega$, if
                        $\mu$ is a measure on $\mathcal{A}$, and if
                        $A_{i}$ is a sequence of elements in $\mathcal{A}$
                        such that, for all $i\in\mathbb{N}$,
                        $A_{i+1}\subseteq{A}_{i}$ and there exists an
                        $n\in\mathbb{N}$ such that $\mu(A_{n})$ is finite,
                        then:
                        \begin{equation}
                            \mu\Big(\bigcap_{n=1}^{\infty}A_{n}\Big)
                            =\underset{N\rightarrow\infty}{\lim}
                            \mu(A_{N})
                        \end{equation}
                    \end{theorem}
                    \begin{proof}
                        For let $A=\cap_{n=1}^{\infty}A_{n}$ and let
                        $B_{n}=A_{n}\setminus{A}_{n+1}$. Then:
                        \begin{equation}
                            A_{1}=
                            A\cup\big(\bigcup_{n=1}^{\infty}B_{n}\Big)
                        \end{equation}
                        And this is the union of countably many disjoint
                        sets. Therefore:
                        \begin{align}
                            \mu(A_{1})&=
                            \mu(A)+\sum_{n=1}^{\infty}\mu(B_{n})\\
                            &=\mu(A)+\sum_{n=1}^{\infty}
                            \Big(\mu(A_{n}-\mu(A_{n+1})\Big)\\
                            &=\mu(A)+\mu(A_{1})-
                            \underset{N\rightarrow\infty}{\lim}\mu(A_{N})
                        \end{align}
                        Subtracting by $\mu(A_{1})$ obtains the result.
                    \end{proof}
                    If $\mu(A_{i})=\infty$ for all $i\in\mathbb{N}$, then
                    the above theorem may not be true. For consider
                    the collection of sets $A_{n}=[n,\infty)$. The
                    measure of each $A_{n}$ is infinite, but the
                    intersection of the entire collection is empty.
                    Thus the measure of the intersection is zero.
                    \begin{theorem}[Countable Sub-Additivity]
                        If $\Omega$ is a set, $\mathcal{A}$ a
                        $\sigma\text{-Algebra}$ on $\mathcal{A}$, and
                        if $A_{i}$ is a countable collection of elements
                        of $\mathcal{A}$, then:
                        \begin{equation}
                            \mu\Big(\bigcup_{n=1}^{\infty}A_{n}\Big)
                            \leq\sum_{n=1}^{\infty}\mu(A_{n})
                        \end{equation}
                    \end{theorem}
                    \begin{proof}
                        For if $A_{1},A_{2}\in\mathcal{A}$, then:
                        \begin{equation}
                            \mu(A_{1}\cup{A}_{2})=
                            \mu(A_{1}\setminus{A}_{2})+
                            \mu(A_{2}\setminus{A}_{1})+
                            \mu(A_{1}\cap{A}_{2})
                        \end{equation}
                        But also:
                        \begin{align}
                            \mu(A_{1})&=
                            \mu(A_{1}\setminus{A}_{2})+
                            \mu(A_{1}\cap{A}_{2})\\
                            \mu(A_{2})&=
                            \mu(A_{2}\setminus{A}_{1})+
                            \mu(A_{1}\cap{A}_{2})
                        \end{align}
                        And therefore:
                        \begin{equation}
                            \mu(A_{1})+\mu(A_{2})=
                            \mu(A_{1}\cup{A}_{2})+\mu(A_{1}\cap{A}_{2})
                        \end{equation}
                        We now prove by induction. Suppose this is true
                        of a collection of $N$ elements. Given a collection
                        $A_{i}$ of $N+1$ elements, let
                        $B=\cup_{n=1}^{N}A_{i}$. But then:
                        \begin{align}
                            \mu(A_{N+1}\cup{B})&
                            \leq\mu(A_{N+1})+\mu(B)\\
                            &\leq\mu(A_{N+1})+\sum_{n=1}^{N}A_{n}\\
                            &=\sum_{n=1}^{N+1}\mu(A_{n})
                        \end{align}
                    \end{proof}
        \section{Lebesgue-Stieltjes Measures}
            A Lebesgue-Stieltjes measure is any measure on the
            Borel $\sigma\text{-Algebra}$ $\mathcal{B}$ such that,
            for any finite semi-interval $[a,b)$,
            $\mu\big(\mu[a,b)\big)<\infty$. $\mu(\mathbb{R})$ may
            be infinite. Recall that the Borel $\sigma\text{-Algebra}$
            is the smallest $\sigma\text{-Algebra}$ on $\mathbb{R}$
            that contains all semi-intervals $[a,b)$. A pure point
            measure on $\mathbb{R}$, indexing over the rational
            numbers, would be such a measure. If we have a
            Lebesgue-Stieltjes measure, we wish to find a function
            $F_{\mu}:\mathbb{R}\rightarrow\mathbb{R}$ such that, for
            all semi-intervals $[a,b)$,
            $\mu([a,b))=F_{\mu}(b)-F_{\mu}(a)$. In probability, this
            is called the cummulative probability function. For now
            we wish to show that there is indeed such a function that
            does this. Consider the case when $\mu(\mathbb{R})<\infty$.
            Let $F_{\mu}(x)=\mu(-\infty,x)$. Then:
            \begin{align}
                \mu([a,b))
                &=\mu((-\infty,b)\setminus(-\infty,a))\\
                &=\mu((-\infty,b))-\mu((-\infty,a))\\
                &=F_{\mu}(b)-F_{\mu}(a)
            \end{align}
            In the more general case when that measure of the entire
            real line is infinite we still want to find a function
            such that:
            \begin{align}
                \mu\big([0,x)\big)&=F_{\mu}(x)-F_{\mu}(0)&x>0\\
                \mu\big([x,0)\big)&=F_{\mu}(0)-F_{\mu}(x)&x<0
            \end{align}
            We can define the following:
            \begin{equation}
                F_{\mu}(x)=
                \begin{cases}
                    \mu\big([0,x)\big)+C,&x>0\\
                    -\mu\big([x,0)\big)+C,&x<0\\
                    C,&x=0
                \end{cases}
            \end{equation}
            Then $F_{\mu}$ is a function that satisfies our criterion.
            Indeed, $F_{\mu}$ is defined uniquely up to an additive
            constant. Any such function is non-decreasing since, for
            any $x<y$, $F_{\mu}(y)-F_{\mu}(x)=\mu([x,y))\geq{0}$.
            In addition, $F_{\mu}$ is left-continuous. That is,
            for all $a\in\mathbb{R}$:
            \begin{equation}
                \underset{x\rightarrow{a}^{-}}{\lim}F_{\mu}(x)
                =F_{\mu}(a)
            \end{equation}
            \begin{theorem}
                If $\mu$ is a Lebesgue-Stieltjes measure on the
                Borel $\sigma\text{-Algebra}$ of $\mathbb{R}$, and if
                $F_{\mu}$ is the function thing, then $F_{\mu}$ is
                left-continuous.
            \end{theorem}
            \begin{proof}
                For let $a\in\mathbb{R}$ and let
                $x:\mathbb{N}\rightarrow\mathbb{R}$ be a monotonic
                increasing sequence such that $x_{n}\rightarrow{a}$.
                But then, for all $n\in\mathbb{N}$,
                $[x_{n+1},a)\subset[x_{n},a)$. But then:
                \begin{equation}
                    \mu\Big(\bigcap_{n=1}^{\infty}[x_{n},a)\Big)
                    =\underset{N\rightarrow\infty}{\lim}
                    \mu\big([x_{N},a)\big)
                \end{equation}
                But $\cap_{n=1}^{\infty}[x_{n},a)=\emptyset$ as
                $x_{n}\rightarrow{a}$. Therefore:
                \begin{equation}
                    \underset{N\rightarrow\infty}{\lim}
                    \mu\big([x_{n},a)\big)=0
                \end{equation}
                But from the definition of $F_{\mu}$,
                \begin{equation}
                    \mu\big([x_{n},a)\big)=F_{\mu}(a)-F_{\mu}(x_{n})
                \end{equation}
                Thus, $F_{\mu}(x_{n})\rightarrow{F}_{\mu}(a)$.
            \end{proof}
            Such a function may not be right-continuous. The requirement
            that the sequence $x$ be increasing was necessary for the
            proof. Howver, the right-hand limit does exists.
            \begin{theorem}
                If blah blah, right hand limit exists.
            \end{theorem}
            \begin{proof}
                For:
                \begin{equation}
                    \{a\}=\bigcap_{n=1}^{\infty}[a,a+\frac{1}{n})
                \end{equation}
                And thus, as $\mu$ is a Lebesgue-Stieltjes measure,
                and thus $\mu([a,b))<\infty$ for all finite semi-intervals,
                we may apply continuity from above and obtain:
                \begin{align}
                    \mu(\{a\})&=
                    \underset{N\rightarrow\infty}{\lim}
                    \mu\big([a,a+\frac{1}{n}\big)\\
                    &=\underset{x\rightarrow{a}^{+}}{\lim}
                    F_{\mu}(x)-F_{\mu}(a)
                \end{align}
            \end{proof}
            If $\mu$ has no points of positive measure, then
            $F_{\mu}$ will be continuous.
            \begin{ftheorem}{Carath\'{e}odory Extension Theorem}{}
                If $F:\mathbb{R}\rightarrow\mathbb{R}$ is a non-decreasing
                left-continuous function then there exists a unique
                Lebesgue-Stieltjes measure $\mu$ such that, for all
                $a,b\in\mathbb{R}$, $a<b$:
                \begin{equation}
                    \mu\big([a,b)\big)=F(b)-F(a)
                \end{equation}
            \end{ftheorem}
            In particular, using $F(x)=x$, we see that there is a unique
            measure on the Borel $\sigma\text{-Algebra}$ such that
            $\mu([a,b))=b-a$. This measure is called the Lebesgue measure
            on $\mathbb{R}$. We define $\mu^{*}$ on a set
            $A\subseteq\mathbb{R}$ to be:
            \begin{equation}
                \mu^{*}(A)=
                \inf\Bigg\{\sum_{i=1}^{\infty}(b_{i}-a_{i}):
                A\subseteq\bigcup_{i=1}^{\infty}[a_{i},b_{i})\Bigg\}
            \end{equation}
            If $A$ is countable, then $\mu^{*}(A)$ is zero. For let
            $a:\mathbb{N}\rightarrow{A}$ be bijection, and let
            $\varepsilon>0$. Then:
            \begin{align}
                \mu^{*}(A)&\leq
                \sum_{n=1}^{\infty}
                \Big((a_{n}+\frac{\varepsilon}{2^{n+1}})-
                (a_{n}-\frac{\varepsilon}{2^{n+1}})\Big)\\
                &=\sum_{n=1}^{\infty}\frac{\varepsilon}{2^{n}}\\
                &=\varepsilon
            \end{align}
            Taking the infininum, we see that $\mu^{*}(A)=0$. This
            function is defined on all of $\mathcal{P}(\mathbb{R})$,
            however it is not a measure. The restriction of
            $\mu^{*}$ to the Borel $\sigma\text{-Algebra}$ is
            a measure.
        \section{Measurable Functions}
            We wish to eventually talk about what it means for a
            function to be \textit{measurable}. First we do a quick review
            of function. If $f:X\rightarrow{Y}$ is a function and if
            $A\subseteq{X}$, the imsge of $A$ under $f$ is the set
            $f(A)=\{f(x):x\in{A}\}$. The notation is a little strange, but
            it has become the standard. For a subset $B\subseteq{Y}$, the
            \textit{pre-image} of $B$ is the set
            $f^{-1}(B)=\{x\in{X}:f(x)\in{B}\}$.
            \begin{lexample}{}{Pre-Image}
                If we let $f:\mathbb{R}\rightarrow\mathbb{R}$ be defined
                by $f(x)=x^{2}$, then $f([1,2])=[1,4]$, and
                $f^{-1}([1,4])=[1,2]\cup[-2,-1]$. As another example we
                can consider $f(x)=\sin(x)$. Then
                $f^{-1}(\{0\})=\{n\pi:n\in\mathbb{N}\}$ and
                $f^{-1}([-1,1])=\mathbb{R}$.
            \end{lexample}
            \begin{theorem}
                If $X$ and $Y$ are sets, $f:X\rightarrow{Y}$,
                and if $A,B\subset{X}$, then:
                \begin{equation}
                    f(A\cup{B})=f(A)\cup{f}(B)
                \end{equation}
            \end{theorem}
            \begin{theorem}
                If $X$ and $Y$ are sets, $f:X\rightarrow{Y}$,
                and if $A,B\subset{X}$, then:
                \begin{equation}
                    f(A\cap{B})\subseteq{f(A)\cap{f}(B)}
                \end{equation}
            \end{theorem}
            For pre-images, we get equality:
            \begin{theorem}
                If $X$ and $Y$ are sets, $f:X\rightarrow{Y}$,
                and if $A,B\subset{X}$, then:
                \begin{equation}
                    f^{-1}(A\cup{B})=f^{-1}(A)\cup{f}^{-1}(B)
                \end{equation}
            \end{theorem}
            \begin{theorem}
                If $X$ and $Y$ are sets, $f:X\rightarrow{Y}$,
                and if $A,B\subset{X}$, then:
                \begin{equation}
                    f^{-1}(A\cap{B})=f^{-1}(A)\cap{f}^{-1}(B)
                \end{equation}
            \end{theorem}
            \begin{theorem}
                If $X$ and $Y$ are sets, $f:X\rightarrow{Y}$,
                and if $A\subset{X}$, then:
                \begin{equation}
                    f^{-1}(A^{C})=f^{-1}(A)^{C}
                \end{equation}
            \end{theorem}
            \begin{theorem}
                If $X$ and $Y$ are sets,if
                $f:X\rightarrow{Y}$ is a function, and if
                $A\subseteq{X}$, then:
                \begin{equation}
                    A\subseteq{f^{-1}\Big(f\big(A\big)\Big)}
                \end{equation}
            \end{theorem}
            \begin{theorem}
                If $X$ and $Y$ are sets, if
                $f:X\rightarrow{Y}$ is an injective function,
                and if $A\subseteq{X}$, then:
                \begin{equation}
                    A=f^{-1}\Big(f\big(A\big)\Big)
                \end{equation}
            \end{theorem}
        \section{Lecture 4}
            Recalling some definitions, a measure on a $\sigma\textrm{-Algebra}$
            $\mathcal{A}$ is a function
            $\mu:\mathcal{A}\rightarrow\mathbb{R}$ such that
            $\mu(A)\geq{0}$, $\mu(\emptyset)=0$, and given a countable
            collection of disjoint sets $A_{i}\in\mathcal{A}$,
            $\mu(\cup_{i=1}^{\infty}A_{i})=\sum_{n=1}^{\infty}\mu(A_{i})$.
            \begin{theorem}
                If $\Omega$ is a set and $\mathcal{A}$ is a
                $\sigma\text{-Algebra}$ on $\Omega$, and if
                $\mu:\mathcal{A}\rightarrow\mathbb{R}$ is a function such that:
                \begin{enumerate}
                    \item $\mu(A)\geq{0}$
                    \item $\mu(\emptyset)=0$
                    \item $\mu$ is finitely additive
                    \item $\mu$ is continuous from below
                \end{enumerate}
                Then $\mu$ is a measure.
            \end{theorem}
            \begin{proof}
                All that is necessary to show is countable additivity.
                Let $A_{n}$ be a countable collection of disjoint elements
                of $\mathcal{A}$, and let $B_{n}=\cup_{k=1}^{n}A_{k}$.
                Then:
                \begin{align}
                    \mu(B_{n})&=\mu(\cup_{n=1}^{n}A_{k})\\
                    &=\sum_{k=1}^{n}\mu(A_{k})
                \end{align}
                But by definition, for all $n\in\mathbb{N}$,
                $B_{n}\subseteq{B_{n+1}}$, and therefore by continuity from
                below, we have:
                \begin{equation}
                    \mu(\cup_{n=1}^{\infty}B_{k})
                    =\lim_{n\rightarrow\infty}\mu(\cup_{k=1}^{n}B_{k})
                \end{equation}
                And therefore
                \begin{equation}
                    \mu(\cup_{n=1}^{\infty}A_{k})=
                    \sum_{k=1}^{\infty}\mu(A_{k})
                \end{equation}
            \end{proof}
                The Borel $\sigma\text{-Algebra}$ on $\mathbb{R}$ is the
                smallest set that makes open sets, elements of the standard
                topology on $\mathbb{R}$, measurable. In an analogous manner
                to how continuous functions are defined for topological
                spaces, measurable functions can also be defined.
            \subsection{Measurable Functions}
                \begin{ldefinition}{Measurable Functions}
                    A measure function from a measurable space
                    $(A,\mathcal{A})$ to a measurable space
                    $(B,\mathcal{B})$ is a function $f:A\rightarrow{B}$
                    such that, for all $\mathcal{U}\in\mathcal{A}$,
                    $f^{-1}(\mathcal{U})\in\mathcal{B}$.
                \end{ldefinition}
                That is, the pre-image of measurable sets is measurable.
                This is similar to continuous functions where the pre-image
                of open sets is open. Such functions are also called
                $\mathcal{A}-\mathcal{B}$ measurable.
                \begin{lexample}{}{Measurable_Function}
                    If $\mathcal{A}$ is a $\sigma\text{-Algebra}$ on $\Omega$,
                    $\Omega$ and if $\mathcal{B}=\{\emptyset,\Omega\}$,
                    then any function $f:\omega\rightarrow\Omega$ will be
                    $\mathcal{A}-\mathcal{B}$ measurable. If
                    $\mathcal{A}=\mathcal{P}(\Omega)$ and if
                    $\mathcal{B}$ is a $\sigma\text{-Algebra}$ on $\Omega$,
                    then again any function $f:\Omega\rightarrow\Omega$ will
                    be $\mathcal{A}-\mathcal{B}$ measurable. There are similar
                    notions in topology called the discrete and chaotic
                    topologies which make all functions continuous.
                \end{lexample}
                \begin{theorem}
                    If $A$ and $B$ are sets, if $\mathcal{B}$ is a
                    $\sigma\text{-Algebra}$ on $B$, and if $f:A\rightarrow{B}$
                    is a function, then the set $\mathcal{A}$ defined by:
                    \begin{equation}
                        \mathcal{A}=
                        \{f^{-1}(\mathcal{U}):\mathcal{U}\in\mathcal{B}\}
                    \end{equation}
                    Is a $\sigma\text{-Algebra}$ on $A$.
                \end{theorem}
                \begin{proof}
                    It is true that $\emptyset\in\mathcal{A}$, since
                    $\empty\in\mathcal{B}$ and $f^{-1}(\emptyset)=\emptyset$.
                    Also, $B\in\mathcal{B}$, and $A=f^{-1}(B)$, and therefore
                    $A\in\mathcal{A}$. If $A\in\mathcal{A}$, then
                    there is a $B\in\mathcal{B}$ such that
                    $A=f^{-1}(B)$, and thus:
                    \begin{equation}
                        A^{C}=f^{-1}(B)^{C}=f^{-1}(B^{C})
                    \end{equation}
                    But if $B\in\mathcal{B}$, then $B^{C}\in\mathcal{B}$,
                    and thus $A^{C}\in\mathcal{A}$. Finally, for any
                    countable collection of sets $A_{n}\in\mathcal{A}$,
                    there is a countable collection of sets $B_{n}$ such that
                    $A_{n}=f^{-1}(B_{n})$ But then:
                    \begin{equation}
                        \cup_{n=1}^{\infty}A_{n}=
                        \cup_{n=1}^{\infty}f^{-1}(B_{n})
                        =f^{-1}(\cup_{n=1}^{\infty}B_{n})
                    \end{equation}
                    But $\mathcal{B}$ is a $\sigma\text{-Algebra}$, and
                    thus $\cup_{n=1}^{\infty}B_{n}\in\mathcal{B}$. Therefore
                    $\cup_{n=1}^{\infty}A_{n}\in\mathcal{A}$.
                \end{proof}
                It's worth noting that $\mathcal{A}$ is the smallest
                $\sigma\text{-Algebra}$ on $A$ that will make
                $f$ $\mathcal{A}-\mathcal{B}$ measurable. Removing any set
                from $\mathcal{A}$ will result in $f:A\rightarrow{B}$ being
                non-measurable with respect to $\mathcal{A}$ and
                $\mathcal{B}$.
                \begin{theorem}
                    If $A$ and $B$ are sets, $f:A\rightarrow{B}$ a function,
                    and if $\mathcal{A}$ is a $\sigma\text{-Algebra}$ on
                    $A$, then the set $\mathcal{B}$ defined by:
                    \begin{equation}
                        \mathcal{B}=
                        \{B\subset{B}:f^{-1}(B)\in\mathcal{A}\}
                    \end{equation}
                    Is a $\sigma\text{-Algebra}$ on $B$.
                \end{theorem}
                \begin{proof}
                    $\emptyset$ and $B$ are elements since
                    $f^{-1}(\emptyset)=\emptyset\in\mathcal{A}$, and
                    $f^{-1}(B)=A\in\mathcal{A}$.
                \end{proof}
                \begin{theorem}
                    If $f:\Omega\rightarrow\mathbb{R}$, if $a\in\mathbb{R}$,
                    if $\mathcal{B}$ is the Borel $\sigma\text{-Algebra}$ on
                    $\mathbb{R}$, and if $\mathcal{A}$ is defined by:
                    \begin{equation}
                        \mathcal{A}=\{\omega\in\Omega:f(\omega)<a\}
                    \end{equation}
                    then $f$ is $\mathcal{A}-\mathcal{B}$ measurable.
                \end{theorem}
                \begin{theorem}
                    If $\Omega$ is a set, and if $\mathcal{A}$ is a
                    $\sigma\text{-Algebra}$ on $\Omega$, and if
                    $f:\Omega\rightarrow\mathbb{R}$ is a function such that,
                    for all $a\in\mathbb{R}$,
                    $\{\omega\in\Omega:f(\omega)<a\}\in\mathcal{A}$, then
                    $f$ is $\mathcal{A}-\mathcal{B}$ measurable, where
                    $\mathcal{B}$ is the Borel $\sigma\text{-Algebra}$.
                \end{theorem}
                \begin{theorem}
                    If $f:\mathbb{R}\rightarrow\mathbb{R}$ is continuous,
                    then it is Borel measurable.
                \end{theorem}
                \begin{theorem}
                    If $\Omega$ is a set, $\mathcal{A}$ is a
                    $\sigma\text{-Algebra}$ on $\Omega$, and if
                    $f:\Omega\rightarrow\mathbb{R}$ is
                    $\mathcal{A}-\mathcal{B}$ measurable, where
                    $\mathcal{B}$ is the Borel $\sigma\text{-Algebra}$, and
                    if $g:\mathbb{R}\rightarrow\mathbb{R}$ is
                    $\mathcal{B}-\mathcal{B}$ measurable, then
                    $g\circ{f}:\Omega\rightarrow\mathbb{R}$ is
                    $\mathcal{A}-\mathcal{B}$ measurable.
                \end{theorem}
                \begin{proof}
                    For if $B\in\mathcal{B}$, then
                    $g^{-1}(B)\in\mathcal{B}$, for $g$ is
                    $\mathcal{B}-\mathcal{B}$ measurable. but then,
                    as $f$ is $\mathcal{A}-\mathcal{B}$ measurable,
                    $f^{-1}(g^{-1}(B))\in\mathcal{A}$. Therefore,
                    $g\circ{f}$ is $\mathcal{A}-\mathcal{B}$ measurable.
                \end{proof}
                In particular, if we have two measurable functions on
                $\mathbb{R}$, then the composition of these two functions
                is also measurable. This is analogous to the fact that the
                composition of continuous functions is continuous. The sum,
                difference, and product of measurable functions is also
                measurable. We now define the Borel $\sigma\text{-Algebra}$ for
                $\mathbb{R}^{2}$. This is denoted $\mathcal{B}_{2}$. It is
                defined similarly to $\mathcal{B}$: It is the smallest
                $\sigma\text{-Algebra}$ that contains all open subsets of
                $\mathbb{R}^{2}$. We can also limit this to all open rectangles
                in the plane, or all open discs.
                \begin{theorem}
                    If $\Omega$ is a set, $\mathcal{A}$ a $\sigma\text{-Algebra}$
                    on $\Omega$, if $f,g:\Omega\rightarrow\mathbb{R}$ are
                    $\mathcal{A}-\mathcal{B}$ measurable functions, and if
                    $\vec{h}:\Omega\rightarrow\mathbb{R}^{2}$ is defined by
                    $\vec{h}(\boldsymbol{\omega})=(f(\omega),g(\omega))$,
                    then $\vec{h}$ is $\mathcal{A}-\mathcal{B}_{2}$ measurable.
                \end{theorem}
                This theorem goes the other way as well. $\vec(h)$ is
                measurable if and only if $f$ and $g$ are measurable.
                \begin{theorem}
                    If $\Omega$ is a set, $\mathcal{A}$ a $\sigma\text{-Algebra}$
                    on $\Omega$, if $\mathcal{B}$ is the Borel
                    $\sigma\text{-Algebra}$ on $\mathbb{R}$, and if
                    $f,g:\Omega\rightarrow\mathbb{R}$ are
                    $\mathcal{A}-\mathcal{B}$ measurable functions, then
                    $f+g$ is $\mathcal{A}-\mathcal{B}$ measurable.
                \end{theorem}
                \begin{proof}
                    For let $\varphi:\mathbb{R}^{2}\rightarrow\mathbb{R}$
                    be defined by $\varphi(x,y)=x+y$. Then
                    $\varphi$ is continuous, and is therefore
                    $\mathcal{B}_{2}-\mathcal{B}$ measurable. Let
                    $h:\Omega\rightarrow\mathbb{R}^{2}$ be defined by
                    $h(\omega)=(f(\omega),g(\omega))$. Then $h$ is
                    $\mathcal{A}-\mathcal{B}_{2}$ measurable. But by taking
                    the composition, we have that
                    $f+g=\varphi\circ{h}$ is $\mathcal{A}-\mathcal{B}$ measurable.
                \end{proof}
                We can do the same thing with multiplication by defining
                $\varphi(x,y)=x\cdot{y}$.
                \begin{theorem}
                    If $f,g:\Omega\rightarrow\mathbb{R}$ are
                    $\mathcal{A}-\mathcal{B}$ measurable, and if:
                    \begin{equation}
                        A=\{\omega\in\Omega:f(\omega)<g(\omega)\}
                    \end{equation}
                    $\Pi$ open set (Half plane along diagonal. Draw this).
                    Do the same thing with the line $L$. They're measurable,
                    yadda yadda.
                \end{theorem}
            \subsection{Sequences of Measurable Functions}
                If $f_{n}:\omega\rightarrow\mathbb{R}$ is a sequence of
                $\mathcal{A}-\mathcal{B}$ measurable functions, and if
                $f_{n}\rightarrow{f}$, then $f$ is $\mathcal{A}-\mathcal{B}$
                measurable. 
                \begin{theorem}
                    Let $F(\omega)=\sup\{f_{n}(\omega):n\in\mathbb{N}\}$.
                    Then $F$ is measurable.
                \end{theorem}
                \begin{proof}
                    For let $a\in\mathbb{R}$. Then:
                    \begin{equation}
                        \{\omega:F(\omega)\leq{a}\}=
                        \bigcap_{n=1}^{\infty}\{\omega:f_{n}(\omega)\leq{a}\}
                    \end{equation}
                    Then $F(\omega)\leq{a}$ if and only if
                    $f_{n}(\omega)\leq{a}$ for all $n\in\mathbb{N}$.
                \end{proof}
                Similarly, $F(\omega)=\inf\{f_{n}(\omega)\}$ is measurable.
                \begin{theorem}
                    If $f_{n}:\Omega\rightarrow\mathbb{R}$ are
                    $\mathcal{A}-\mathcal{B}$ functions, then
                    $\underset{n\rightarrow{\infty}}{\overline{\lim}}f_{n}$ and
                    $\underset{n\rightarrow{\infty}}{\underline{\lim}}f_{n}$
                    are $\mathcal{A}-\mathcal{B}$ measurable.
                \end{theorem}
        \newpage
        \section{Convergence of Measurable Functions}
            \begin{ldefinition}{Measure Space}
                A Measure Space on a set $\Omega$, denoted
                $(\Omega,\mathcal{A},\mu)$, is a
                $\sigma\textrm{-Algebra}$ on $\Omega$ and a
                measure $\mu:\Omega\rightarrow\mathbb{R}$.
            \end{ldefinition}
            A function $f:\Omega\rightarrow\mathbb{R}$ is
            $\mathcal{A}-\mathcal{B}$ measurable, or simply
            measurable, if for all $B\in\mathcal{B}$, the
            pre-image is in $\mathcal{A}$. That is,
            $f^{-1}(B)\in\mathcal{A}$.
            \begin{theorem}
                If $f,g:\Omega\rightarrow\mathbb{R}$ are
                measurable functions, and if $A$ is defined by:
                \begin{equation}
                    A=\{\omega\in\Omega:f(\omega)\ne{g}(\omega)\}
                \end{equation}
                then $A$ is $\mathcal{A}$ measurable.
            \end{theorem}
            \begin{definition}
                Two functions are equal $\mu$-Almost everywhere
                if the following is true:
                \begin{equation}
                    \mu(\omega\in\Omega:f(\omega)\ne{g}(\omega))
                    =0
                \end{equation}
            \end{definition}
            \begin{definition}
                A sequence of measurable functions
                $f_{n}:\mathbb{R}\rightarrow\mathbb{R}$ converges
                to a function $f:\Omega\rightarrow\mathbb{R}$
                if there is a set $E$ such that $\mu(E^{C})=0$,
                and $f_{n}(\omega)\rightarrow{f}(\omega)$ for all
                $\omega\in{E}$.
            \end{definition}
            \begin{theorem}
                If $f_{n}\rightarrow{f}$ almost every and
                $f_{n}\rightarrow{g}$ almost everywhere, then
                $f=g$ almost everywher.
            \end{theorem}
            \begin{theorem}
                If $f_{n}\rightarrow{f}$ almost everywhere, and
                if $f=g$ almost everywhere, then
                $f_{n}\rightarrow{g}$ almost everywhere.
            \end{theorem}
            \begin{definition}
                A function $f:\Omega\rightarrow\mathbb{R}$
                converges to $f:\Omega\rightarrow\mathbb{R}$
                uniformly if for all $\varepsilon>0$ there is
                an $N\in\mathbb{N}$ such that, for all
                $\omega\in\Omega$,
                $|f(\omega)-f_{n}(\omega)|<\varepsilon$.
            \end{definition}
            \begin{example}
                Consider $f_{n}(\omega)=\omega^{n}$ for
                $\omega\in[0,a]$, where $a<1$. Then this
                converges uniformly to zero since, all all
                $\omega\in[0,a]$:
                \begin{equation}
                    |\omega^{n}-0|=\omega^{n}\leq{a}^{n}
                \end{equation}
                But since $0\leq{a}<1$, $a^{n}$ converges to
                zero. Thus $f_{n}\rightarrow{0}$ uniformly.
            \end{example}
            We can define non-uniform converges by considering
            the logical negation of the definition for
            uniform convergence, but we can simplify this as
            well.
            \begin{theorem}
                A sequence of functions $f_{n}$ converges
                non-uniformly to a function $f$ if
                $f_{n}\rightarrow{f}$ point-wise, and there
                exists a $\delta>0$, a strictly increasing
                sequence $n_{k}$, and a sequence $\omega_{k}$
                such that
                $|f_{n_{k}}(\omega_{k})-f(\omega_{k})|>\delta$
            \end{theorem}
            \begin{example}
                If we define $f_{n}(\omega)=\omega^{n}$ on the
                interval $[0,1]$, then the convergence is no
                longer uniform. Indeed, the limit is
                discontinuous.
            \end{example}
            \begin{definition}
                A sequence $f_{n}$ converges to $f$ almost
                uniformly if, for all $\varepsilon>0$ there is
                a set $E$ such that $\mu(E^{C})<\varepsilon$,
                and $f_{n}$ converges to $f$ uniformly on
                $E$.
            \end{definition}
            \begin{example}
                If we again let $f_{n}(\omega)=\omega^{n}$ on
                $[0,1]$, then $f_{n}\rightarrow{0}$
                almost uniformly. For let $\varepsilon>0$. Then
                $f_{n}\rightarrow{0}$ uniformly on the set
                $[0,1-\varepsilon]$, and the measure of the
                compliment of this is less than$ \varepsilon$.
            \end{example}
            There is a difference between convergence almost
            everywhere and convergence almost uniformly. For
            convergence almost everywhere, we may remove a set
            of measure zero and expect that there is point-wise
            convergence on the remaining set. For almost uniform
            convergence we may remove a set of arbitrarily small
            measure, but not necessarily measure zero, and expect
            uniform convergence on the remaining set.
            \begin{theorem}
                If $f_{n}\rightarrow{f}$ almost uniformly,
                then $f_{n}\rightarrow{f}$ almost everywhere.
            \end{theorem}
            \begin{proof}
                For all $n\in\mathbb{N}$ there is a set
                $E_{n}$ such that $\mu(E_{n}^{C})<1/n$, and
                $f_{n}\rightarrow{f}$ uniformly on $E_{n}$.
                But then $f_{n}\rightarrow{f}$ on
                $\cup_{n=1}^{\infty}E_{n}$. But the complement
                of this set has measure zero. Therefore, etc.
            \end{proof}
            The converse is not true, in general. For let
            $f_{n}(\omega)$ be defined as follows:
            \begin{equation}
                f_{n}(\omega)=
                \begin{cases}
                    0,&\omega\leq{n}\\
                    1,&\omega>n
                \end{cases}
            \end{equation}
            The $f_{n}\rightarrow{0}$ almost everywhere, and
            indeed $f_{n}\rightarrow{0}$ point-wise. But
            the convergence is not uniform, nor is it
            almost uniform. There is no way to remove a set of
            finite measure and have uniform convergence on the
            resulting set. Similar to where continuity from above
            failed, the fact that $\mu(\mathbb{R})$ is infinite
            is why this failed. If we can limit the measure on
            the set, then convergence almost everywhere implies
            almost uniform convergence.
            \begin{ftheorem}{Egorov's Theorem}
                            {Measure_Theory_Egorov_Theorem}
                If $(\Omega,\mathcal{A},\mu)$ is a measure
                space and if $\mu(\Omega)<\infty$, then
                convergence $\mu$-almost everywhere implies
                $\mu$-almost uniform convergence.
            \end{ftheorem}
            \begin{bproof}
                For if $f_{n}(\omega)\rightarrow{f(\omega)}$
                $\mu$-almost everywhere, then there is a set
                $E$ such that $\mu(E^{C})=0$ and
                $f_{n}(\omega)\rightarrow{f(\omega)}$ for all
                $\omega\in{E}$. This means that for all
                $\delta>0$ and for all $\omega\in{E}$ there is
                an $N\in\mathbb{N}$ such that for all $n>N$,
                $|f_{n}(\omega)-f(\omega)|<\delta$. Let $A_{nm}$
                be defined as:
                \begin{equation}
                    A_{Nm}=\bigcup_{n=N}^{\infty}\Big\{
                        \omega\in\Omega:
                        |f_{n}(\omega)-f(\omega)|\geq\frac{1}{m}|
                        \Big\}
                \end{equation}
                Define $B_{m}$ as:
                \begin{equation}
                    B_{m}=\bigcap_{N=1}^{\infty}A_{Nm}
                \end{equation}
                But since $\mu(\Omega)<\infty$, the measure
                $\mu$ is continuous from above. Therefore:
                \begin{equation}
                    \mu(B_{m})
                        =\underset{N\rightarrow\infty}{\lim}
                        \mu(A_{Nm})
                \end{equation}
                But $B_{m}\subseteq{E^{c}}$, and thus
                $\mu(B_{m})=0$. But then
                $\mu(A_{Nm})\rightarrow{0}$.
            \end{bproof}
            So we have shown that, even though convergence
            almost everywhere and almost uniform convergence
            are diferent concepts, on sets of finite measure
            they are equivalent. In probably the total measure
            of the entire set if 1, and so finite. Thus, in
            probabability spaces, almost everywhere convergence
            and almost uniform convergence will always be
            equivalent. Thus it is common to use the term
            convergence almost surely, and forgot the differences
            between the two properties. There is a third type of
            convergence called convergence in measure.
            \begin{ldefinition}{Convergence in Measure}
                A sequence of functions $f_{n}$ convergence
                in measure to $f$ is, for all $\delta>0$, the
                following is true:
                \begin{equation}
                    \underset{n\rightarrow\infty}{\lim}
                    \mu\Big(
                        \big\{
                            \omega:|f_{n}(\omega)-f(\omega)|
                            <\delta
                        \big\}
                    \Big)=0
                \end{equation}
            \end{ldefinition}
            \begin{example}
                Let $\Omega=[0,1]$, and let $\mathcal{B}$ be
                the Borel $\sigma-\textrm{Algebra}$ on
                $[0,1]$. Finally, let $\mu$ be the standard
                Lebesgue-Measure. Define the following:
                \begin{equation}
                    f_{1}=
                    \begin{cases}
                        1,&x<\frac{1}{2}\\
                        0,&x\geq\frac{1}{2}
                    \end{cases}
                \end{equation}
                Define $f_{2}=1-f_{1}$. The split the
                interval into fourths and define
                $f_{3}$ as 1 in $[0,1/4)$ and zero otherwise,
                and continue the pattern for $f_{4}$, $f_{5}$,
                $f_{6}$, and $f_{7}$. This sequence of functions
                converges nowhere since there will be 1's and
                0's oscillating back and forth, and thus there
                is no limit. However, $f_{n}$ converges in
                measure to 0.
            \end{example}
            \begin{theorem}
                If $f_{n}\rightarrow{f}$ in measure $\mu$,
                and if $g=f$ almost everywhere, then
                $f_{n}\rightarrow{g}$ in measure $\mu$.
            \end{theorem}
            \begin{proof}
                For all $\delta>0$,
                $\mu(\{\omega:|f_{n}(\omega)-f(\omega)|>\delta\}$
                tends to zero as $n\rightarrow\infty$. But:
                \begin{equation}
                    \begin{split}
                        \{\omega:|f_{n}(\omega)-f(\omega)|
                        &>\delta\}\\
                        &=\Big(
                            \{\omega:|f_{n}(\omega)-f(\omega)|>\delta\}
                            \bigcap
                            \{\omega:f(\omega)=g(\omega)\}
                        \Big)\\
                        &\bigcup\Big(
                            \{\omega:|f_{n}(\omega)-f(\omega)|>\delta\}
                            \bigcap
                            \{\omega:f(\omega)\ne{g}(\omega)\}
                        \Big)
                    \end{split}
                \end{equation}
            \end{proof}
            \begin{theorem}
                If $f_{n}\rightarrow{f}$ in measure $\mu$ and if
                $f_{n}\rightarrow{g}$ in measure $\mu$, and
                $f=g$ $\mu$ almost everywhre.
            \end{theorem}
            \begin{proof}
                For let $A=\{\omega:f(\omega)\ne{g}(\omega)\}$.
                Then $A=\{\omega:|f(\omega)-g(\omega)|>0\}$.
                Thus we may write:
                \begin{equation}
                    A=\bigcup_{n=1}^{\infty}
                    \Big\{\omega:|f(\omega)-g(\omega)|>\frac{1}{n}
                    \Big\}
                \end{equation}
                We now show that
                $\{\omega:|f(\omega)-g(\omega)|>\frac{1}{n}\}$
                has measure zero for all $n\in\mathbb{N}$. By
                subadditivity, this will imply $A$ has measure zero.
                From the triangle inequality:
                \begin{equation}
                    |f(\omega)-g(\omega)|\leq
                    |f(\omega)-f_{n}(\omega)|+
                    |g(\omega)-f_{n}(\omega)|
                \end{equation}
                If $|f(\omega)-g(\omega)|\geq{1/m}$, then at
                least one of the two numbers here must be greater
                than $1/2m$. Thus, either
                $|f(\omega)-f_{n}(\omega)|\geq\frac{1}{2m}$ or
                $|g(\omega)-f_{n}(\omega)|\geq\frac{1}{2m}$, or
                both. Therefore:
                \begin{equation}
                    \{\omega:|f(\omega)-g(\omega)|>\frac{1}{n}\}
                    \subseteq
                    \{\omega:|f(\omega)-f_{n}(\omega)|>\frac{1}{2n}
                    \}\bigcup
                    \{\omega:|g(\omega)-f_{n}(\omega)|>\frac{1}{2n}
                    \}
                \end{equation}
                But the two sets on the left have measures that
                tend to zero as $n\rightarrow\infty$, and thus
                the set on the left has measure zero. Thus, by
                subadditivity their union has measure zero, and
                therefore $\mu(A)=0$. Thus, $f=g$ $\mu$ almost
                everywhere.
            \end{proof}
            \begin{theorem}
                If $f_{n}\rightarrow{f}$ in measure $\mu$ and
                $g_{n}\rightarrow{g}$ in measure $\mu$, then
                $f_{n}+g_{n}\rightarrow{f+g}$ in measure $\mu$.
            \end{theorem}
            \begin{theorem}
                If $f_{n}\rightarrow{f}$ in measure $\mu$, then
                there exists a subsequence of $f_{n}$ that converges
                to $f$ almost uniformly.
            \end{theorem}
            \begin{proof}
                For all $\delta>0$ the limit of
                $\mu(\{|f_{n}-f|\geq\delta\})$ tends to zero
                as $n\rightarrow\infty$. Thus, there is an
                index $n_{1}$ such that
                $\mu(\{|f_{n_{1}}-f|\geq{1}\})<1$. Choosing
                $\delta=1/2$, we find an index $n_{2}$ such that
                $\mu(\{|f_{n_{2}}-f|\geq1/2\})<1/2$.
                Carrying on, we obtain a sequence $n_{k}$ such
                that, for all $k\in\mathbb{N}$,
                $\mu(\{|f_{n_{k}}-f|\geq1/k\})<1/2^{k}$.
                Let $E_{k}=\{|f_{n_{k}}-f|\geq1/k\}^{C}$.
                $\mu(E_{k}^{C})<1/2^{k}$, and thus for all
                $\varepsilon>0$ there is an $N\in\mathbb{N}$
                such that, for all $n>N$,
                $\mu(E_{n}^{C})<\varepsilon$.
            \end{proof}
        \section{Lecture 6}
            The $\sigma-\textrm{Algebra}$ generated by
            a set $\mathcal{E}$ is the intersection of all
            possible $\sigma-\textrm{Algebra's}$ that contain
            all elements of $\mathcal{E}$. Given a function
            $f:\Omega\rightarrow\mathbb{R}$ and a
            $\sigma-\textrm{Algebra}$ $\mathcal{A}$ on
            $\Omega$, it is often a good strategy to look at
            the set:
            \begin{equation}
                \mathcal{B}_{f,\mathcal{A}}=
                    \big\{
                        B\subseteq\mathbb{R}:
                        f^{-1}(B)\in\mathcal{A}
                    \big\}
            \end{equation}
            And then show that all intervals of the form
            $(a,b)$ are contained within
            $\mathcal{B}_{f,\mathcal{A}}$, thus implying
            that $f$ is $\mathcal{A}-\mathcal{B}$ measurable,
            where $\mathcal{B}$ is the Borel
            $\sigma-\textrm{Algebra}$ on $\mathbb{R}$.
            Recapping, we have now discussed three different
            types of convergence: Almost uniform convergence,
            convergence almost everywhere, and convergence in
            measure.
            \begin{theorem}
                If $f_{n}\rightarrow{f}$ almost uniformly,
                then $f_{n}\rightarrow{f}$ in measure.
            \end{theorem}
        \section{Integration}
            The integral of a constant function on an interval
            $[a,b]$ is defined by the signed area under the
            rectangle formed by the function. That is, if
            $f(x)=C$, we define the integral on $[a,b]$ to
            be $C(b-a)$. Given a piece-wise constant function,
            we can define the integral as the sum over the
            various regions. Given an arbitrary function, the
            only reasonable way to define the integral is to
            take a limit of approximations using piece-wise
            constant functions. This is how the Riemann integral
            is defined. The Riemann integral makes sense if
            the given function is continuous. By making the
            partition small enough, approximating a continuous
            function on a small interval by a constant can be
            reasonable. But consider the function:
            \begin{equation}
                f(x)=
                \begin{cases}
                    0,&x\notin\mathbb{Q}\\
                    1,&x\in\mathbb{Q}
                \end{cases}
            \end{equation}
            Given any interval $(a,b)$, $f$ takes on the values
            0 and 1 and thus it is not reasonable to approximate
            this function by a constant anywhere. However,
            the measure, or length, of $\mathbb{Q}$ is zero, and
            the height of the function on $\mathbb{Q}$ is 1.
            Thus it may be reasonable to define the area under
            this function as zero. While the Riemann integral
            cannot handle such function, the Lebesgue integral
            can. Given a measurable space
            $(\Omega,\mathcal{A},\mu)$, and a measurable
            function $f:\Omega\rightarrow\mathbb{R}$, where
            $f$ is $\mathcal{A}-\mathcal{B}$ measurable,
            $\mathcal{B}$ being the Borel
            $\sigma-\textrm{Algebra}$, it is possible to define
            the integral of $f$.
            \begin{ldefinition}{Support of a Real Valued Function}
                  {Support_Real_Func}
                The support of a real-valued function
                $f:\Omega\rightarrow\mathbb{R}$ is the set:
                \begin{equation}
                    \mathrm{supp}(f)=\{\omega\in\Omega:f(\omega)\ne{0}\}
                \end{equation}
            \end{ldefinition}
            \begin{ldefinition}{Simple Function}{Simple_Functions}
                A simple function from a measurable space
                $(\Omega,\mathcal{A},\mu)$ to the real line
                $\mathbb{R}$ is a function
                $f:\Omega\rightarrow\mathbb{R}$ such that
                $f$ is $\mathcal{A}-\mathcal{B}$ measurable,
                where $\mathcal{B}$ is the Borel
                $\sigma-\textrm{Algebra}$, the range of $f$,
                $f(\Omega)$, is finite, and the measure of the
                support of $f$ is finite.
            \end{ldefinition}
            \begin{lexample}{}{Measurable_Space}
                Let $(\Omega,\mathcal{A},\mu)$ be a measurable
                space, and let $E\subseteq\Omega$ be measurable
                and of finite measure. Define the following:
                \begin{equation}
                    \chi_{E}(\omega)=
                    \begin{cases}
                        0,&\omega\notin{E}\\
                        1,&\omega\in{E}
                    \end{cases}
                \end{equation}
                This is called the indicator function of
                $E$. It is a simple function on the measurable
                space $(\Omega,\mathcal{A},\mu)$ since
                $\mu(E)<\infty$. To see that it is measurable,
                note that the pre-image is either
                $\emptyset$, $E$, $E^{C}$, or $\Omega$, and thus
                $\chi_{E}$ is measurable. Finally, it takes on
                only two values and thus it's range is finite.
            \end{lexample}
            \begin{lexample}{}{Another_Measurable_Space}
                Let $(\Omega,\mathcal{A},\mu)$ be a measurable
                space, and let $B_{1},\dots,B_{n}$ be measurable
                subsets of $\Omega$. Furthermore, let
                $a_{1},\dots,a_{n}$ be real numbers. If we let
                $\chi_{B_{i}}$ denote that indicator
                function of $B_{i}$, then we see that their sum
                is also a simple function. That is, define the
                following:
                \begin{equation}
                    f(\omega)=\sum_{k=1}^{n}
                        a_{k}\chi_{B_{k}}(\omega)
                \end{equation}
                THen $f:\Omega\rightarrow\mathbb{R}$ is a simple
                function. Since the sum of measurable functions
                is measurable, we see that $f$ is measurable.
                We also have that the support is finite since:
                \begin{equation}
                    \mathrm{supp}(f)\subseteq\bigcup_{k=1}^{n}B_{k}
                \end{equation}
                Finally, there are $2^{n}$ ways, at most, to
                combine the various real numbers
                $a_{1},\dots,a_{n}$, and thus the range of
                $f$ has, at most, $2^{n}$ elements. Therefore
                $f$ is simple.
            \end{lexample}
            Now, let $(\Omega,\mathcal{A},\mu)$ be a
            measurable space, and let
            $f:\Omega\rightarrow\mathbb{R}$ be a simple function.
            If $f$ is simple than it's range is finite. Let
            $a_{1},\cdots,a_{n}$ be the distinct elements of
            $f(\Omega)$ and define the following:
            \begin{equation}
                E_{k}=f^{-1}\big(\{a_{k}\}\big)
            \end{equation}
            Since $f$ is simple, it is measurable, and thus
            $E_{k}\in\mathcal{A}$ for all $k\in\mathbb{Z}_{n}$.
            Moreover, for $i\ne{j}$,
            $E_{i}\cap{E}_{j}=\emptyset$. But, since $f$ is
            simple, the measure of its support is finite, and
            thus for all $k\in\mathbb{Z}_{n}$,
            the masure of $E_{k}$ is also finite.
            We can thus obtain the following for $f$:
            \begin{equation}
                f(\omega)=
                    \sum_{k=1}^{n}a_{k}\chi_{E_{k}}(\omega)
            \end{equation}
            This is called the \textrm{Canonical Representation}
            of $f$.
            \begin{ldefinition}{Integral of a Simple Function}
                  {Integral_Simple_Function}
                The integral of a simple function
                $f:\Omega\rightarrow\mathbb{R}$ on a measurable
                space $(\Omega,\mathcal{A},\mu)$ with canonical
                form:
                \begin{equation}
                    f(\omega)=\sum_{k=1}^{n}
                        a_{k}\chi_{E_{k}}(\omega)
                \end{equation}
                Is the real number $\int_{\Omega}f\diff{\mu}$
                define by:
                \begin{equation}
                    \int_{\Omega}f\diff{\mu}=
                        \sum_{k=1}^{n}a_{k}\mu(E_{k})
                \end{equation}
            \end{ldefinition}
            The first thing to check to ensure that this is
            a good definition of integration is that the
            sum of two integrals is the integral of the sum
            of the two functions. We first prove a result that
            will make this definition more flexible.
            \begin{theorem}
                If $(\Omega,\mathcal{A},\mu)$ is a measurable
                space, and if $B_{1},\dots,B_{n}$ are measurable
                subsets of $\Omega$ that are pairwise disjoint
                and such that, for all $k\in\mathbb{Z}_{n}$,
                $\mu(B_{k})<\infty$, if
                $\lambda_{1},\dots,\lambda_{n}$ are real
                numbers, and if $f:\Omega\rightarrow\mathbb{R}$
                is defined by:
                \begin{equation}
                    f(\omega)=\sum_{k=1}^{n}
                        \lambda_{k}\chi_{B_{k}}(\omega)
                \end{equation}
                Then the integral of $f$ is given by:
                \begin{equation}
                    \int_{\Omega}f\diff{\mu}
                        =\sum_{k=1}^{n}\lambda_{k}
                        \mu(B_{k})
                \end{equation}
            \end{theorem}
            \begin{proof}
                For every $\omega\in\Omega$ falls in only
                one of the $B_{k}$. But if $\omega\in{B}_{k}$,
                then $f(\omega)=\lambda_{k}$. Let
                $a_{1},\hdots,a_{n}$ be the distinct values of
                $f$ and defin $\mathcal{J}_{k}$ as:
                \begin{equation}
                    \mathcal{J}_{k}=\{j:\lambda_{j}=a_{k}\}
                \end{equation}
                Also, let $\mathcal{J}_{0}$ be defined as:
                \begin{equation}
                    \mathcal{J}_{0}=\{j:\lambda_{j}=0\}
                \end{equation}
                If $E_{k}=f^{-1}(\{a_{k}\}$, then:
                \begin{equation}
                    E_{k}=\bigcup_{j\in\mathcal{J}_{k}}B_{j}
                \end{equation}
                But since the $B_{j}$ are pair-wise disjoint,
                we have that:
                \begin{equation}
                    \mu(E_{k})=\sum_{j\in\mathcal{J}_{k}}
                        \mu(B_{j})
                \end{equation}
                But then:
                \begin{align}
                    \int_{\Omega}f\diff{\mu}
                    &=\sum_{k=1}^{n}a_{k}\mu(E_{k})\\
                    &=\sum_{k=1}^{n}a_{k}\Big(
                            \sum_{j\in\mathcal{J}_{k}}\mu(B_{j})
                        \Big)\\
                    &=\sum_{k=1}^{n}\sum_{j\in\mathcal{J}_{k}}
                        a_{k}\mu(B_{j})\\
                    &=\sum_{k=1}^{n}\sum_{j\in\mathcal{J}_{k}}
                        \lambda_{j}\mu(B_{j})\\
                    &=\sum_{k=1}^{m}\lambda_{k}\mu(B_{k})
                \end{align}
            \end{proof}
            This theorem will make it easier to prove the
            additive property of integrals. We are still only
            talking about the integral of simple functions.
            \begin{theorem}
                If $(\Omega,\mathcal{A},\mu)$ is a measurable
                space and $f:\Omega\rightarrow\mathbb{R}$ and
                $g:\Omega\rightarrow\mathbb{R}$ are simple
                functions, then:
                \begin{equation}
                    \int_{\Omega}(f+g)\diff{\mu}
                    =\int_{\Omega}f\diff{\mu}
                    +\int_{\Omega}g\diff{\mu}
                \end{equation}
            \end{theorem}
            \begin{proof}
                For let $f$ and $g$ have the following
                canonical representations:
                \begin{align}
                    f(\omega)&=\sum_{k=1}^{n}\alpha_{k}
                        \chi_{A_{k}}(\omega)\\
                    g(\omega)&=\sum_{k=1}^{m}\beta_{k}
                        \chi_{B_{k}}(\omega)
                \end{align}
                Define the following:
                \begin{align}
                    A_{n+1}
                    =\Big(\bigcup_{k=1}^{m}B_{k}\Big)\setminus
                        \Big(\bigcup_{j=1}^{n}A_{j}\Big)\\
                    B_{m+1}
                    =\Big(\bigcup_{k=1}^{n}A_{k}\Big)\setminus
                        \Big(\bigcup_{j=1}^{m}B_{j}\Big)
                \end{align}
                From this, we have the following:
                \begin{equation}
                    \bigcup_{k=1}^{n+1}A_{k}=
                    \bigcup_{j=1}^{m+1}B_{j}
                \end{equation}
                Let $\alpha_{n+1}=\beta_{m+1}=0$. Then we have:
                \begin{align}
                    f(\omega)&=\sum_{k=1}^{n+1}\alpha_{k}
                        \chi_{A_{k}}(\omega)\\
                    g(\omega)&=\sum_{k=1}^{m+1}\beta_{k}
                        \chi_{B_{k}}(\omega)
                \end{align}
                Then, for all $i$, we have:
                \begin{align}
                    A_{i}&=A_{i}\bigcap
                        \Big(\bigcup_{j=1}^{m+1}B_{j}\Big)\\
                    &=\bigcup_{j=1}^{m+1}
                        \Big(A_{i}\cap{B}_{j}\Big)\\
                    B_{i}&=B_{i}\bigcap
                        \Big(\bigcup_{k=1}^{n+1}A_{k}\Big)\\
                    &=\bigcup_{k=1}^{n+1}
                        \Big(A_{k}\cap{B}_{i}\Big)
                \end{align}
                And these are all pair-wise disjoint sets.
                So, we have:
                \begin{align}
                    f(\omega)&=
                    \sum_{i=1}^{n+1}\sum_{j=1}^{m+1}\alpha_{i}
                        \chi_{A_{i}\cap{B}_{j}}(\Omega)\\
                    g(\omega)&=
                    \sum_{i=1}^{n+1}\sum_{j=1}^{m+1}\beta_{j}
                        \chi_{A_{i}\cap{B}_{j}}(\Omega)
                \end{align}
                Summing these two functions, we get:
                \begin{equation}
                    f(\omega)+g(\omega)=
                    \sum_{i=1}^{n+1}\sum_{j=1}^{m+1}
                        (\alpha_{i}+\beta_{j})
                        \chi_{A_{i}\cap{B}_{j}}(\Omega)
                \end{equation}
                But by the previous theorem:
                \begin{align}
                    \int_{\Omega}(f+g)\diff{\mu}
                    &=\sum_{i=1}^{n+1}\sum_{j=1}^{m+1}
                        (\alpha_{i}+\beta_{j})
                        \mu(A_{i}\cap{B}_{j})\\
                    &=\sum_{i=1}^{n+1}\sum_{j=1}^{m+1}
                        \alpha_{i}\mu(A_{i}\cap{B}_{j})+
                        \sum_{i=1}^{n+1}\sum_{j=1}^{m+1}
                        \beta_{j}\mu(A_{i}\cap{B}_{j})\\
                    &=\sum_{i=1}^{n+1}\alpha_{i}\mu(A_{i})+
                        \sum_{j=1}^{m+1}\beta_{j}\mu(B_{j})\\
                    &=\int_{\Omega}f\diff{\mu}+
                        \int_{\Omega}g\diff{\mu}
                \end{align}
            \end{proof}
            The integrals of simple functions also have the
            property of homogeneity.
            \begin{theorem}
                If $(\Omega,\mathcal{A},\mu)$ is a measurable
                space, $f:\Omega\rightarrow\mathbb{R}$ is
                a simple function, and if $c\in\mathbb{R}$,
                then:
                \begin{equation}
                    \int_{\Omega}cf\diff{\mu}=
                    c\int_{\Omega}f\diff{\mu}
                \end{equation}
            \end{theorem}
            \begin{theorem}
                If $(\Omega,\mathcal{A},\mu)$ is a measurable
                space, $f:\omega\rightarrow\mathbb{R}$ is
                a simple function, if $A_{1},\dots,A_{n}$
                are measurable subsets of $\Omega$ with
                finite measure, and if $f$ is such that:
                \begin{equation}
                    f(\omega)=\sum_{k=1}^{n}a_{k}
                        \chi_(A_{k})(\omega)
                \end{equation}
                Then:
                \begin{equation}
                    \int_{\Omega}f\diff{\mu}=
                    \sum_{k=1}^{n}a_{k}\mu(A_{k})
                \end{equation}
            \end{theorem}
            \subsection{Further Properties of the Integral}
                So far we have defined the integral of a simple
                function over the entire of a given space
                $\Omega$. We often wish to evaluate the integral
                of a function on a subset of $\Omega$, rather
                than the entire of it. We can do this by defining
                the following:
                \begin{equation}
                    f_{E}(\omega)=
                    \begin{cases}
                        f(\omega),&\omega\in{E}\\
                        0,&\omega\notin{E}
                    \end{cases}
                \end{equation}
                We need some properties of $f_{E}$.
                $f_{E}$ is measurable, has finite range, and
                has support of finite measure, and is therefore
                simple. $f_{E}$ can have only one more value
                (That is, zero) than $f$. Finally,
                $\mathrm{supp}(f_{E})\subseteq{\mathrm{supp}(f)}$. We define
                the integral on $E\in\mathcal{A}$ as follows:
                \begin{equation}
                    \int_{E}f\diff{\mu}=
                    \int_{\Omega}f_{E}\diff{\mu}
                \end{equation}
                Since $f_{E}$ is also simple, the right hand
                side of this equation is well defined.
                \begin{theorem}
                    If $(\Omega,\mathcal{A},\mu)$ is a measurable
                    space, if $E_{1},E_{2}$ are disjoint
                    measurable subsets of $\Omega$, and if
                    $f:\Omega\rightarrow\mathbb{R}$ is a simple
                    function, then:
                    \begin{equation}
                        \int_{E_{1}\cup{E}_{2}}f\diff{\mu}=
                        \int_{E_{1}}f\diff{\mu}+
                        \int_{E_{2}}f\diff{\mu}
                    \end{equation}
                \end{theorem}
                This is similar to a notion that is found when
                studying the Riemann integral. That is:
                \begin{equation}
                    \int_{a}^{b}f(x)\diff{x}=
                    \int_{a}^{c}f(x)\diff(x)+
                    \int_{c}^{b}f(x)\diff{x}
                \end{equation}
                \begin{theorem}
                    If if $f:\omega\rightarrow\mathbb{R}$
                    is a simple function, and if
                    $f(\omega)\geq{0}$ for all $\omega\in\Omega$,
                    then:
                    \begin{equation}
                        \int_{\Omega}f\diff{\mu}\geq{0}
                    \end{equation}
                \end{theorem}
                \begin{theorem}
                    If $f:\Omega\rightarrow\mathbb{R}$ is
                    simple, then:
                    \begin{equation}
                        \int_{\Omega}f\diff{\mu}=0
                    \end{equation}
                    If and only if $f=0$ $\mu$ almost everywhere.
                \end{theorem}
                \begin{ftheorem}
                      {Triangle Inequality
                       for Simple Functions}{}
                    If $f$ is a simple function, then:
                    \begin{equation}
                        \Big|\int_{\Omega}f\diff{\mu}\Big|
                        \leq\int_{\Omega}|f|\diff{\mu}
                    \end{equation}
                \end{ftheorem}
            \subsection{Limit Theorems for Simple Functions}
                \begin{theorem}
                    If $f_{n}:\Omega\rightarrow\mathbb{R}$ is
                    a sequence of simple functions such that
                    $f_{n}\rightarrow{f}$ uniformly,
                    where $f$ is a simple function,
                    and if there is a measurable set $E$
                    such that $\mathrm{supp}(f_{n})\subseteq{E}$ and
                    $\mathrm{supp}(f)\subseteq{E}$, and if
                    $\mu(E)<\infty$, then:
                    \begin{equation}
                        \underset{n\rightarrow\infty}{\lim}
                        \int_{\Omega}f_{n}\diff{\mu}
                        =\int_{\Omega}f\diff{\mu}
                    \end{equation}
                \end{theorem}
                \begin{proof}
                    For:
                    \begin{align}
                        \Big|
                            \int_{\Omega}f_{n}\diff{\mu}-
                            \int_{\Omega}f\diff{\mu}
                        \Big|
                        &=\Big|\int_{\Omega}(f_{n}-f)\diff{\mu}
                            \Big|\\
                        &\leq\int_{\Omega}|f_{n}-f|\diff{\mu}\\
                        &=\int_{E}|f_{n}-f|\diff{\mu}\\
                        &\leq\int_{E}\varepsilon\diff{\mu}\\
                        &=\varepsilon\int_{E}\diff{\mu}\\
                        &=\varepsilon\mu(E)
                    \end{align}
                    Since $\mu(E)<\infty$, this can be made
                    arbitrarily small.
                \end{proof}
                \begin{theorem}
                    If $f_{n}\rightarrow{f}$ uniformly, and if
                    $f_{n}$ and $f$ are simple, then
                    $f_{n}$ is uniformly bounded.
                \end{theorem}
                \begin{theorem}[Bounded Convergence Theorem]
                    If $f_{n}\rightarrow{f}$, $f_{n}$ are
                    simple and $f$ is simple, if
                    $f_{n}$ are uniformly bounded, and if
                    there is a measurable set $E$ of finite
                    measure such that $\mathrm{supp}(f_{n})\subseteq{E}$
                    and $\mathrm{supp}(f)\subseteq{E}$, then:
                    \begin{equation}
                        \underset{n\rightarrow\infty}{\lim}
                        \int_{\Omega}f_{n}\diff{\mu}
                        =\int_{\Omega}f\diff{\mu}
                    \end{equation}
                \end{theorem}
                \begin{lexample}{}{Bounded_Convergence_Theorem}
                    The additional assumptions are indeed
                    necessary, and without them these results
                    may not hold. For let $f_{n}$ be defined as:
                    \begin{equation}
                        f_{n}(\omega)=
                        \begin{cases}
                            n,&0\leq\omega\leq\frac{1}{n}\\
                            0,&\textrm{Otherwise}
                        \end{cases}
                    \end{equation}
                    Then $\int_{\mathbb{R}}f_{n}\diff{\mu}=1$
                    for all $n$, but the limit function is
                    $f=0$, and this has integral zero. It is
                    also not guarenteed that the results
                    fail, for let:
                    \begin{equation}
                        f_{n}(\omega)=
                        \begin{cases}
                            n,&0\leq\omega\leq\frac{1}{n^{2}}\\
                            0,&\textrm{Otherwise}
                        \end{cases}
                    \end{equation}
                    Then the limit function is zero, and the
                    integral is $\frac{1}{n}$, which does indeed
                    converge to zero. For the requirement that
                    $\mathrm{supp}(f_{n})$ and $\mathrm{supp}(f)$ be contained in
                    one set, consider the following function:
                    \begin{equation}
                        f_{n}(\omega)=
                        \begin{cases}
                            1,&n\leq\omega\leq{n+1}\\
                            0,&\textrm{Otherwise}
                        \end{cases}
                    \end{equation}
                    Then $f_{n}$ is uniformly bounded,
                    converges to $0$, but the integral is
                    1 for all $n$.
                \end{lexample}
                \begin{theorem}[Monotone Convergence Theorem]
                    If $f_{n}$, $f$ are simple functions, if
                    $f_{n}\rightarrow{f}$, if $f_{n}\leq{f}_{n+1}$,
                    then:
                    \begin{equation}
                        \int_{\Omega}f_{n}\diff{\mu}
                        \rightarrow\int_{\Omega}f\diff{\mu}
                    \end{equation}
                \end{theorem}
                \begin{theorem}
                    \label{thm:MEASURE_THEORY_LIM_INT_MONO_SIMPLE_FUNCS}
                    If $f_{n}$ and $g_{n}$ are simple and monotonically
                    increasing, and if:
                    \begin{equation}
                        \underset{n\rightarrow\infty}{\lim}f_{n}(\omega)
                        =\underset{n\rightarrow\infty}{\lim}g_{n}(\omega)
                    \end{equation}
                    Then:
                    \begin{equation}
                        \underset{n\rightarrow\infty}{\lim}
                        \int_{\Omega}f_{n}\diff\mu
                        =\underset{n\rightarrow\infty}{\lim}
                        \int_{\Omega}g_{n}\diff\mu
                    \end{equation}
                \end{theorem}
                This theorem will allow us to extend the definition
                of the integral to a more general class of functions.
            \subsection{Integration of Non-Negative Measurable Functions}
                Consider a measure space $(\Omega,\mathcal{A},\mu)$ and
                let $f:\Omega\rightarrow\mathbb{R}$ be an
                $\mathcal{A}-\mathcal{B}$ measurable function, where
                $\mathcal{B}$ is the Borel $\sigma\textrm{-Algebra}$ on
                $\mathbb{R}$. If there exist a sequence of simple
                functions $f_{n}$ that are monotonically increasing
                and such that $f_{n}\rightarrow{f}$, then we define the
                integral of $f$ as follows:
                \begin{equation}
                    \int_{\Omega}f\diff{\mu}=
                    \underset{n\rightarrow\infty}{\lim}
                    \int_{\Omega}f_{n}\diff{\mu}
                \end{equation}
                Because of
                Thm.~\ref{thm:MEASURE_THEORY_LIM_INT_MONO_SIMPLE_FUNCS}
                this is a well defined concept, since for any
                two sequences of simples functions that are monotonically
                increasing and converge to $f$, the limit of the
                integrals is the same, thus giving a consitent definition
                to the integral of $f$. The first question that then
                arises is which measurable functions can be approximated
                arbitrarily well by a sequence of monotonically
                increasing simple functions? From the definition we will
                need that $f$ is bounded below. For now we will discuss
                measurable functions that are non-negative. Let
                $f:\Omega\rightarrow\mathbb{R}$ be any non-negative
                function, it need not be measurable. We wish to construct
                a sequence of simple functions $f_{n}$ such that
                $f_{n}$ is monotonically increasing, and for all
                $\omega\in\Omega$, $f_{n}(\omega)\rightarrow{f}(\omega)$.
                We construct such a sequence as follows, dividing
                the $[0,n)$ into $n2^{n}-1$ parts
                $[\frac{i}{2^{n}},\frac{i+1}{2^{n}})$ and define the
                following sets:
                \begin{align}
                    E_{n}^{C}
                    &=\{\omega\in\Omega:f(\omega)\geq{n}\}\\
                    E_{n,i}
                    &=\{\omega:\frac{i}{2^{n}}\leq{f}(\omega)
                        \leq\frac{i+1}{2^{n}}\}\\
                    &=f^{-1}\big([\frac{i}{2^{n}},\frac{i+1}{2^{n}})\big)
                \end{align}
                Then, for every fixed $n\in\mathbb{N}$, the sets
                $E_{n,i}$ are pairwise disjoint. Defined $f_{n}$ as
                follows:
                \begin{equation}
                    f_{n}(\omega)=
                    \begin{cases}
                        n,&f(\omega)\geq{n}\\
                        \frac{i}{2^{n}},&\frac{i}{2^{n}}
                            \leq{f}(\omega)
                            \leq\frac{i+1}{2^{n}}
                    \end{cases}
                \end{equation}
                Then, using the sets $E_{n,i}$ and $E_{n}^{C}$, we
                can rewrite $f_{n}$ as follows:
                \begin{equation}
                    f_{n}(\omega)=
                    n\chi_{E_{n}^{C}}(\omega)+
                    \sum_{i=0}^{n2^{n}-1}
                        \frac{i}{2^{n}}\chi_{E_{n,i}}(\omega)
                \end{equation}
                We now have that $f_{n}$ is monotonically increasing
                and tends to $f$. For is $f(\omega)=\infty$, then:
                \begin{equation}
                    w\in\cap_{n=1}^{\infty}E_{n}^{C}
                    \Rightarrow
                    f_{n}(\omega)=n\rightarrow\infty
                \end{equation}
                If $f(\omega)\in\mathbb{R}$, then there is an
                $N\in\mathbb{N}$ such that $f(\omega)<N$. But then, for
                all $n>N$, $f(\omega)<n$ and thus there is an
                $i\in\mathbb{Z}_{n2^{n}-1}$ such that:
                \begin{equation}
                    \frac{i}{2^{n}}\leq{f}(\omega)\leq\frac{i+1}{2^{n}}
                \end{equation}
                But $f_{n}(\omega)=\frac{i}{2^{n}}$ and thus:
                \begin{equation}
                    0\leq{f}(\omega)-f_{n}(\omega)\leq\frac{1}{2^{n}}
                \end{equation}
                And therefore $f_{n}\rightarrow{f}$. Finally,
                $f_{n}$ is monotonically increasing. Now let's see what
                we can add to this if we know that $f$ is measurable.
                Since $f$ is measurable, the pre-image
                $f^{-1}([n,\infty))\in\mathcal{A}$, since
                $[n,\infty)$ is a Borel set for all $n\in\mathbb{N}$.
                Moreover, for all $n$ and $i$, $E_{n,i}\in\mathcal{A}$.
                Then all of the indicator functions $\chi_{E_{n,i}}$ are
                measurable, and thus $f_{n}$ is measurable for all
                $n$. However, the support of the $f_{n}$ may not be
                finite. Simply take $f(\omega)=1$ for all $\omega$, and
                let $\Omega=\mathbb{R}$. If the $\mu(\Omega)<\infty$,
                then $\mu(\mathrm{supp}(f_{n}))<\infty$. This case is particularly
                important when studying probability theory.
                \begin{ldefinition}{$\sigma\textrm{-Finite}$}
                    A $\sigma\textrm{-Finite}$ measure on a
                    $\sigma\textrm{-Algebra}$ $\mathcal{A}$ of a set
                    $\Omega$ is a measure $\mu$ such that there is a
                    sequence of sets $\Omega_{n}$ such that
                    $\Omega_{n}\subseteq\Omega_{n+1}$,
                    $\Omega=\cup_{n=1}^{\infty}\Omega_{n}$, and for all
                    $n\in\mathbb{N}$, $\mu(\Omega_{n})<\infty$.
                \end{ldefinition}
                \begin{lexample}{}{Lebesgue_Stieljes}
                    Let $\Omega=\mathbb{R}$ and consider the Borel
                    $\sigma\textrm{-Algebra}$ $\mathcal{B}$ on
                    $\mathbb{R}$. Then the standard Lebesgue-Measure
                    is $\sigma\textrm{-finite}$ since we can write:
                    \begin{equation}
                        \mathbb{R}=\cup_{n=1}^{\infty}[-n,n]
                    \end{equation}
                    And $\mu([-n,n])=2n$, which is finite.
                \end{lexample}
                Define $\tilde{f}_{n}$ as follows:
                \begin{equation}
                    f_{n}(\omega)
                    =f_{n}(\omega)\cdot\chi_{\Omega_{n}}(\omega)
                \end{equation}
                Then we have that $\tilde{f}_{n}$ is simple, measurable,
                takes on finitely many values, and the measure of it's
                support is finite. Thus we have that if
                $(\Omega,\mathcal{A},\mu)$ is a measure space and if
                $\mu$ is $\sigma\textrm{-Finite}$, then for any
                non-negative measurable function
                $f:\Omega\rightarrow\mathbb{R}$, the integral of $f$ is
                well defined. We write:
                \begin{equation}
                    \int_{\Omega}f\diff{\mu}=
                    \underset{n\rightarrow\infty}{\lim}
                    \int_{\Omega}f_{n}\diff{\mu}
                \end{equation}
            \subsection{Properties of the Integral
                        of Non-Negative Functions}
            \begin{theorem}[Homogeneity of the Integral]
                If $f$ is a non-negative measurable function, and if
                $c>0$, then:
                \begin{equation}
                    \int_{\Omega}(cf)\diff{\mu}=
                    c\int_{\Omega}f\diff{\mu}
                \end{equation}
            \end{theorem}
            \begin{proof}
                For let $f_{n}$ be a sequence of simple functions such
                that $f_{n}\rightarrow{f}$ and $f_{n}$ is monotonically
                increasing. Then:
                \begin{equation}
                    \int_{\Omega}(cf)\diff{\mu}
                    =\underset{n\rightarrow\infty}{\lim}
                    \int_{\Omega}(cf_{n})\diff{\mu}
                    =c\underset{n\rightarrow\infty}{\lim}
                    \int_{\Omega}f_{n}\diff{\mu}
                    =c\int_{\Omega}f\diff{\mu}
                \end{equation}
            \end{proof}
            \begin{theorem}[Additivity of the Integral]
                If $f$ and $g$ are non-negative and measurable,
                then:
                \begin{equation}
                    \int_{\Omega}(f+g)\diff{\mu}
                    =\int_{\Omega}f\diff{\mu}
                    +\int_{\Omega}g\diff{\mu}
                \end{equation}
            \end{theorem}
            \begin{proof}
                For let $f_{n}$ and $g_{n}$ be simple functions such
                that $f_{n}\rightarrow{f}$, $g_{n}\rightarrow{g}$, and
                such that $f_{n}$ and $g_{n}$ are monotonically
                increasing. Then:
                \begin{align}
                    \int_{\Omega}(f+g)\diff{\mu}
                    &=\underset{n\rightarrow\infty}{\lim}
                    \int_{\Omega}(f_{n}+g_{n})\diff{\mu}\\
                    &=\underset{n\rightarrow\infty}{\lim}
                    \int_{\Omega}f_{n}\diff{\mu}+
                    \underset{n\rightarrow\infty}{\lim}
                    \int_{\Omega}g_{n}\diff{\mu}\\
                    &=\int_{\Omega}f\diff{\mu}+\int_{\Omega}g\diff{\mu}
                \end{align}
            \end{proof}
            \begin{theorem}
                If $f$ is non-negative and measurable, and if
                $E_{1},E_{2}$ are disjoint sets, then:
                \begin{equation}
                    \int_{E_{1}\cup{E}_{2}}f\diff{\mu}
                    =\int_{E_{1}}f\diff{\mu}+\int_{E_{2}}f\diff{\mu}
                \end{equation}
            \end{theorem}
            \begin{theorem}
                If $f$ is a non-negative measurable function, then:
                \begin{equation}
                    \int_{\Omega}f\diff{\mu}\geq{0}
                \end{equation}
            \end{theorem}
            \begin{theorem}
                If $f$ is a non-negative measurable function, then:
                \begin{equation}
                    \int_{\Omega}f\diff{\mu}=0
                    \Longleftrightarrow{f=0}
                    \quad\mu\textrm{-almost everywhere}
                \end{equation}
            \end{theorem}
            \begin{ldefinition}{Summable Functions}
                A non-negative summable function is a non-negative
                and measurable function from
                a measure space $(\Omega,\mathcal{A},\mu)$ where
                $\mu$ is $\sigma\textrm{-Finite}$ to $\mathbb{R}$
                such that:
                \begin{equation}
                    \int_{\Omega}f\diff{\mu}<\infty
                \end{equation}
            \end{ldefinition}
            \begin{ltheorem}{Chebyshev's Inequality}
                If $f$ is a non-negative measurable function, then for
                all $a\in\mathbb{R}^{+}$:
                \begin{equation}
                    \mu\Big(\{\omega:f(\omega)\geq{a}\}\Big)
                    <\frac{1}{a}\int_{\Omega}f\diff{\mu}
                \end{equation}
            \end{ltheorem}
            \begin{proof}
                For define the following:
                \begin{align}
                    E_{1}&=\{\omega:f(\omega)\geq{a}\}\\
                    E_{2}&=\{\omega:f(\omega)<a\}
                \end{align}
                Then $E_{1}$ and $E_{2}$ are disjoint, and therefore:
                \begin{equation}
                    \int_{\Omega}f\diff{\mu}=
                    \int_{E_{1}}f\diff{\mu}+
                    \int_{E_{2}}f\diff{\mu}
                    \geq\int_{E_{1}}f\diff{\mu}
                    \geq\int_{\Omega}a\diff{\mu}
                    =a\mu(E_{1})
                \end{equation}
                Dividing by $a$ completes the proof.
            \end{proof}
            \begin{theorem}
                If $f$ is a non-negative summable function, then
                for all $a\in\mathbb{R}^{+}$:
                \begin{equation}
                    \underset{a\rightarrow\infty}{\lim}
                    \mu\Big(\{\omega:f(\omega)\geq{a}\}\Big)=0
                \end{equation}
            \end{theorem}
            \begin{ltheorem}{Monotone Convergence Theorem}
                If $f$ is a non-negative measurable function and if
                $f_{n}$ is a sequence of non-negative measurable
                functions that are monotonically increasing and such
                that $f_{n}\rightarrow{f}$, then:
                \begin{equation}
                    \underset{n\rightarrow\infty}{\lim}
                    \int_{\Omega}f_{n}\diff{\mu}
                    =\int_{\Omega}f\diff{\mu}
                \end{equation}
            \end{ltheorem}
            \begin{proof}
                For all $n\in\mathbb{N}$, there is a function
                $g_{n,k}$ such that $g_{n,k}$ is simple and:
                \begin{equation}
                    \int_{\Omega}f_{n}\diff{\mu}=
                    \underset{k\rightarrow\infty}{\lim}
                    \int_{\Omega}g_{n,k}\diff{\mu}
                \end{equation}
                Define $F_{n}$ as follows:
                \begin{equation}
                    F_{n}(\omega)=
                    \max\{g_{j,n}(\omega):
                        \omega\in\Omega,0\leq{j}\leq{n}\}
                \end{equation}
                Then, for all $n\in\mathbb{N}$, $F_{n}$ is simple.
                For it is the maximum of finitely many measurable
                functions, and is therefore measurable. Moreover:
                \begin{equation}
                    \mathrm{supp}(F_{n})\subseteq
                    \bigcup_{k=1}^{n}\mathrm{supp}(f_{k,n})
                \end{equation}
                And finally, $F_{n}$ is monotonically increasing from
                it's definition. Now we must show that
                $F_{n}\rightarrow{f}$. For:
                \begin{equation}
                    g_{k,n}\leq{F}_{k}\leq{f}_{k}
                \end{equation}
                Since $g_{n,k}$ increases monotonically to $f_{n}$.
                Taking the limit as $k\rightarrow\infty$, we obtain:
                \begin{equation}
                    f_{n}\leq\underset{k\rightarrow\infty}{\lim}F_{k}
                    \leq{f}
                \end{equation}
                Then taking the limit on $n$, we see that
                $F_{n}\rightarrow{f}$. Integrating this inequality, we
                get:
                \begin{equation}
                    \int_{\Omega}g_{k,n}\diff{\mu}
                    \leq\int_{\Omega}F_{k}\diff{\mu}
                    \leq\int_{\Omega}f\diff{\mu}
                \end{equation}
                Taking the limit as $k\rightarrow\infty$, we get:
                \begin{equation}
                    \underset{k\rightarrow\infty}{\lim}
                    \int_{\Omega}g_{k,n}\diff{\mu}
                    \leq\int_{\Omega}f\diff{\mu}
                    \leq\underset{k\rightarrow\infty}{\lim}
                    \int_{\Omega}f_{k}\diff{\mu}
                \end{equation}
                Finally, taking the limit on $n$, we get:
                \begin{equation}
                    \underset{n\rightarrow\infty}{\lim}
                    \int_{\Omega}f_{n}\diff{\mu}
                    \leq\int_{\Omega}f\diff{\mu}
                    \leq\underset{k\rightarrow\infty}{\lim}
                    \int_{\Omega}f_{k}\diff{\mu}
                \end{equation}
                This completes the proof.
            \end{proof}
    \section{Lebesgue Spaces}
        \begin{ldefinition}{Lebesgue Number}
              {Funct_Analysis_Lebesgue_Number}
            A Lebesgue Number of an open cover
            $\mathcal{O}$ of a metric space $(X,d)$ is
            a non-zero number $d>0$ such that, for all
            $x\in{X}$, there exists
            a $\mathcal{U}\in\mathcal{O}$ such that:
            \begin{equation}
                B_{d}^{(X,d)}(x)\subseteq\mathcal{U}
            \end{equation}
        \end{ldefinition}
        \begin{lexample}
            Let $X=\mathbb{R}$, and let $d$ be the
            standard metric. Let
            $\mathcal{O}=\{\mathcal{U}_{i}:i=1,2,3\}$ where:
            \begin{equation}
                \mathcal{U}_{1}=(-\infty,1)
                \quad\quad
                \mathcal{U}_{2}=(0,2)
                \quad\quad
                \mathcal{U}_{3}=(1,\infty)
            \end{equation}
            Then $d=1/2$ is a Lebesgue number of this cover.
            Letting $X=(0,1)$ with the standard metric, for all
            $x\in{X}$ the is a $\delta_{x}>0$ such that:
            \begin{equation}
                B_{\delta_{x}}^{(X,d)}(x)
                \subseteq{X}
            \end{equation}
            And thus these open balls are a covering of the unit
            interval, but this covering has no Lebesgue number.
        \end{lexample}
        \begin{ltheorem}{Lebesgue Covering Lemma}
              {Funct_Analysis_Lebesgue_Covering_Lemma}
            If $(X,d)$ is a compact metric space, and if
            $\mathcal{O}$ is an open covering of $X$, then
            $\mathcal{O}$ has a Lebesgue number.
        \end{ltheorem}
        \begin{proof}
            Suppose not. Suppose $(X,d)$ is compact, and suppose
            that $\mathcal{O}$ is a covering of $X$ with no
            Lebesgue number. But then, for all $n\in\mathbb{N}$,
            there is an $a_{n}$ such that, for all
            $\mathcal{U}\in\mathcal{O}$:
            \begin{equation}
                B_{1/n}^{(X,d)}(a_{n})\not\subset\mathcal{U}
            \end{equation}
            But $X$ is compact, and thus there is a convergent
            subsequence such that $a_{k_{n}}\rightarrow{X}$.
            But then there is a $\mathcal{U}\in\mathcal{O}$ such
            that $x\in\mathcal{U}$. But $\mathcal{U}$ is open,
            and thus there is an $r>0$ such that:
            \begin{equation}
                B_{r}^{(X,d)}(x)\subseteq\mathcal{U}
            \end{equation}
            Let $N\in\mathbb{N}$ be such that, for all
            $k_{n}>N$, $d(x_{k_{n}},x)<r/2$. Let
            $n>N$ be such that $1/k_{n}<r/2$. But then:
            \begin{equation}
                B_{1/k_{n}}(a_{k_{n}})\subseteq\mathcal{U}
            \end{equation}
            A contradiction.
        \end{proof}
        \begin{theorem}
            If $(X.d)$ is a compact metric space, if $(Y,\rho)$
            is a metric space, and if $f:X\rightarrow{Y}$ is a
            continuous function, then $f$ is
            uniformly continuous.
        \end{theorem}
        \begin{proof}
            For let $\varepsilon>0$. since $f$ is
            continuous, for all $x\in{X}$ there is a
            $\delta_{x}$ such that, for all $y\in{X}$ such
            that $d(x,y)<\delta_{x}$, it is true that
            $\rho(f(x),f(y))<\varepsilon/2$. But then:
            \begin{equation}
                X\subseteq
                    \bigcup_{x\in{X}}B_{\delta_{x}}^{(X,d)}(x)
            \end{equation}
            But $X$ is compact, and thus this covering has a
            Lebesgue number. Let $\delta$ be such a
            Lebesgue number. But then if $d(x,y)<\delta$,
            then there is a $z\in{X}$ such that
            $x,y\in{B}_{\delta_{z}}(z)$. But then:
            \begin{equation}
                \rho(f(x),f(x))\leq
                \rho(f(x),f(z))+\rho(f(z),f(y))
                <\varepsilon
            \end{equation}
        \end{proof}
        \section{Lecture 8}
            Let $(\Omega,\mathcal{A},\mu)$ be a measure space and let
            $f\geq{0}$ be measurable. From before we were able to define
            the integral of $f$ is $\mu$ is $\sigma\textrm{-finite}$. We
            approximate $f$ with an increasing sequence of simple functions
            that are also non-negative. The integral of $f$ is defined as
            the limit of the integrals of the approximating
            simple functions. That is, we define the integral to be:
            \begin{equation}
                \int_{\Omega}f\diff{\mu}=
                \underset{n\rightarrow\infty}{\lim}
                \int_{\Omega}f_{n}\diff{\mu}
            \end{equation}
            We have seen from a previous theorem that the value
            of the integral is independent of the approximating
            sequence. That is, for $f_{n}$ and $g_{n}$ are a
            sequence of simple functions that are monotonically
            increasing to $f$, then:
            \begin{equation}
                \underset{n\rightarrow\infty}{\lim}
                \int_{\Omega}f_{n}\diff{\mu}=
                \underset{n\rightarrow\infty}{\lim}
                \int_{\Omega}g_{n}\diff{\mu}
            \end{equation}
            We then proved the monotone convergence theorem.
            \begin{ltheorem}{Monotone Convergence Theorem}
                If $f_{n}$ is a sequence of positive measurable functions,
                not necessarily simple, and if $f_{n}$ is monotonically
                increasing, then:
                \begin{equation}
                    \underset{n\rightarrow\infty}{\lim}
                    \int_{\Omega}f_{n}\diff{\mu}
                    =\int_{\Omega}
                    \underset{n\rightarrow\infty}{\lim}f_{n}\diff{\mu}
                \end{equation}
            \end{ltheorem}
            Note that we are still only talking about non-negative measurable
            functions. We have yet to discuss functions that are possibly
            negative.
            \begin{ltheorem}{Fatou's Theorem}
                If $f_{n}$ is a sequence of non-negative measurable functions,
                then:
                \begin{equation}
                    \int_{\Omega}
                    \underset{n\rightarrow\infty}{\underline{\lim}}
                    f_{n}\diff{\mu}
                    \leq
                    \underset{n\rightarrow\infty}{\underline{\lim}}
                    \int_{\Omega}f_{n}\diff{\mu}
                \end{equation}
                Where $\underline{\lim}$ denotes the limit-inferior.
            \end{ltheorem}
            \begin{proof}
                For:
                \begin{equation}
                    0\leq\inf_{k\geq{n}}f_{k}(\omega)
                    \leq{f}_{k}(\omega)
                \end{equation}
                And therefore:
                \begin{equation}
                    \int_{\Omega}\inf_{k\geq{n}}f_{k}\diff{\mu}
                    \leq\int_{\Omega}f_{k}\diff{\mu}
                \end{equation}
                And therefore:
                \begin{equation}
                    \int_{\Omega}\inf_{k\geq{n}}f_{k}\diff{\mu}
                    \leq\inf_{k\geq{n}}\int_{\Omega}f_{k}\diff{\mu}
                \end{equation}
                But:
                \begin{equation}
                    \underset{n\rightarrow\infty}{\lim}
                    \int_{\Omega}\inf_{k\geq{n}}f_{k}\diff{\mu}
                    \leq
                    \underset{n\rightarrow\infty}{\lim}
                    \inf_{k\geq{n}}\int_{\Omega}f_{k}\diff{\mu}
                    =\underset{n\rightarrow\infty}{\underline{\lim}}
                    \int_{\Omega}f_{n}\diff{\mu}
                \end{equation}
                Therefore, etc.
            \end{proof}
            \begin{theorem}
                If $f_{n}$ is a sequence of non-negative measurable functions,
                then the function $f$ defined by:
                \begin{equation}
                    f=\underset{N\rightarrow\infty}{\lim}
                    \sum_{n=0}^{N}f_{n}
                \end{equation}
                Is measurable.
            \end{theorem}
            \begin{theorem}
                If $f_{n}$ is a sequence of non-negative measurable functions
                and if $f$ is defined by:
                \begin{equation}
                    f=\underset{N\rightarrow\infty}{\lim}
                    \sum_{n=0}^{N}f_{n}
                \end{equation}
                Then:
                \begin{equation}
                    \int_{\Omega}f\diff{\mu}=
                    \underset{N\rightarrow\infty}{\lim}
                    \sum_{n=0}^{N}\int_{\Omega}f_{n}\diff{\mu}
                \end{equation}
            \end{theorem}
            \begin{theorem}
                If $(\Omega,\mathcal{A},\mu)$ is a measure space,
                if $f$ is measurable and non-negative, and if
                $\nu:\mathcal{A}\rightarrow\mathbb{R}$ is defined by:
                \begin{equation}
                    \nu(E)=\int_{E}f\diff{\mu}=
                    \int_{\Omega}f_{E}\diff{\mu}
                \end{equation}
                Then $\nu$ is a measure on $\mathcal{A}$.
            \end{theorem}
            \begin{proof}
                For $\mu(\emptyset)=0$ by definition. Since $f$ is positive,
                for all $E\in\mathcal{A}$:
                \begin{equation}
                    \nu(E)=\int_{\Omega}f_{E}\diff{\mu}\geq{0}
                \end{equation}
                And finally, if $E_{n}$ are pairwise disjoint then:
                \begin{equation}
                    \nu\Big(\bigcup_{n=1}^{\infty}E_{n}\Big)=
                    \int_{\bigcup_{n=1}^{\infty}E_{n}}f\diff{\mu}
                    =\sum_{n=1}^{\infty}\int_{E_{n}}f\diff{\mu}
                    =\sum_{n=1}^{\infty}\nu(E_{n})
                \end{equation}
                Therefore, etc.
            \end{proof}
            \begin{theorem}
                If $(\Omega,\mathcal{A},\mu)$ is a measure space,
                if $f$ is measurable and non-negative, if
                $\nu:\mathcal{A}\rightarrow\mathbb{R}$ is defined by:
                \begin{equation}
                    \nu(E)=\int_{E}f\diff{\mu}=
                    \int_{\Omega}f_{E}\diff{\mu}
                \end{equation}
                And if $E\in\mathcal{A}$ is such that $\mu(E)=0$, then
                $\nu(E)=0$.
            \end{theorem}
            \begin{ldefinition}{Absolute Continuity}
                An absolutely continuous measure $\nu$ with respect
                to a measure space $(\Omega,\mathcal{A},\mu)$ is a meausre
                $\nu$ on $\mathcal{A}$ such that for all $E\in\mathcal{A}$
                such that $\mu(E)=0$, it is true that $\nu(E)=0$. This is
                denoted $\nu<<\mu$.
            \end{ldefinition}
            \begin{ltheorem}{Radon-Nikodym Theorem}
                If $(\Omega,\mathcal{A},\mu)$ is a measure space and if
                $\nu$ is absolutely continuous with respect to
                $(\Omega,\mathcal{A},\nu)$. then there is a measurable
                non-negative function $f$ such that, for all $E\in\mathcal{A}$:
                \begin{equation}
                    \nu(E)=\int_{E}f\diff{\mu}
                \end{equation}
            \end{ltheorem}
            The function $f$ in the previous theorem is often called the
            density of $\nu$ against $\mu$, or the
            Radon-Nikodym derivative of $\nu$ with respect to $\mu$. The
            function $f$ is unique $\mu$ almost everywhere.
        \section{Integral of Signed Functions}
            Given a function $f:\Omega\rightarrow\mathbb{R}$, we can define
            the following two functions:
            \begin{equation}
                f^{+}(\omega)=
                \begin{cases}
                    f(\omega),&f(\omega)\geq{0}\\
                    0,&f(\omega)<0
                \end{cases}
            \end{equation}
            \begin{equation}
                f^{+}(\omega)=
                \begin{cases}
                    0,&f(\omega)\geq{0}\\
                    -f(\omega),&f(\omega)<0
                \end{cases}
            \end{equation}
            From these definitions we see that:
            \begin{equation}
                f=f^{+}-f^{-}
            \end{equation}
            There are two useful formala for computed $f^{+}$ and $f^{-1}$:
            \begin{align}
                f^{+}&=\frac{|f|+f}{2}\\
                f^{-}&=\frac{|f|-f}{2}
            \end{align}
            \begin{theorem}
                If $f$ is measurable, then $f^{+}$ and $f^{-}$ are measurable.
            \end{theorem}
            \begin{ldefinition}{Integral of Signed Function}
                The integral of a measurable function $f$ such that either
                the integral of $f^{+}$ or the integral of $f^{-}$, or both,
                is finite, is the difference:
                \begin{equation}
                    \int_{\Omega}f\diff{\mu}=
                    \int_{\Omega}f^{+}\diff{\mu}-\int_{\Omega}f^{-}\diff{\mu}
                \end{equation}
            \end{ldefinition}
            \begin{ldefinition}{Summable Function}
                A summable function is a function $f$ such:
                \begin{equation}
                    \int_{\Omega}f^{+}\diff{\mu}<\infty
                \end{equation}
                \begin{equation}
                    \int_{\Omega}f^{-}\diff{\mu}<\infty
                \end{equation}
            \end{ldefinition}
            \begin{theorem}
                A function $f$ is summable if and only if:
                \begin{equation}
                    \int_{\Omega}|f|\diff{\mu}<\infty
                \end{equation}
            \end{theorem}
            \begin{ltheorem}{Homogeneity of the Integral of Signed Functions}
                If $f$ is a signed integrable function, and if $c$ is a
                real number, then:
                \begin{equation}
                    \int_{\Omega}(cf)\diff{\mu}=
                    c\int_{\Omega}f\diff{\mu}
                \end{equation}
            \end{ltheorem}
            \begin{ltheorem}{Additivity of the Integral of Signed Functions}
                If $f$ and $g$ are summable functions, then:
                \begin{equation}
                    \int_{\Omega}(f+g)\diff{\mu}
                    =\int_{\Omega}f\diff{\mu}+\int_{\Omega}g\diff{\mu}
                \end{equation}
            \end{ltheorem}
            \begin{proof}
                For:
                \begin{equation}
                    (f+g)^{+}=
                    \frac{|f+g|+f+g}{2}\leq
                    \frac{|f|+|g|+f+g}{2}=f^{+}+g^{+}
                \end{equation}
                Similarly:
                \begin{equation}
                    (f+g)^{-}\leq{f}^{-}+g^{-}
                \end{equation}
                And therefore $f+g$ is summable. Evaluating the integral:
                \begin{equation}
                    \int_{\Omega}(f+g)\diff{\mu}=
                    \int_{\Omega}(f+g)^{+}\diff{\mu}-
                    \int_{\Omega}(f+g)^{-}\diff{\mu}
                \end{equation}
                But we have:
                \begin{align}
                    f+g&=(f+g)^{+}-(f+g)^{-}\\
                    &=(f^{+}-f^{-})+(g^{+}-g^{-})
                \end{align}
                Rearranging, we have:
                \begin{equation}
                    (f+g)^{+}+f^{-}+g^{-}=
                    (f+g)^{-}+f^{+}+g^{+}
                \end{equation}
                Computing the integral, we have:
                \begin{equation}
                    \begin{split}
                        \int_{\Omega}(f+g)^{+}\diff{\mu}+
                        \int_{\Omega}f^{-}&\diff{\mu}+
                        \int_{\Omega}g^{-}\diff{\mu}\\
                        &=\int_{\Omega}(f+g)^{-}\diff{\mu}+
                        \int_{\Omega}f^{+}\diff{\mu}+
                        \int_{\Omega}g^{+}\diff{\mu}
                    \end{split}
                \end{equation}
                Rearranging this, we obtain:
                \begin{equation}
                    \begin{split}
                        \int_{\Omega}(f+g)^{+}\diff{\mu}-
                        \int_{\Omega}&(f+g)^{-}\diff{\mu}\\
                        &=\int_{\Omega}f^{+}\diff{\mu}-
                        \int_{\Omega}f^{-}\diff{\mu}+
                        \int_{\Omega}g^{+}\diff{\mu}-
                        \int_{\Omega}g^{-}\diff{\mu}
                    \end{split}
                \end{equation}
                This completes the proof.
            \end{proof}
            \begin{theorem}
                If $f$ is integrable and if $E_{1}$ and $E_{2}$ are disjoint,
                then:
                \begin{equation}
                    \int_{E_{1}\cup{E}_{2}}f\diff{\mu}=
                    \int_{E_{1}}f\diff{\mu}+\int_{E_{2}}f\diff{\mu}
                \end{equation}
            \end{theorem}
            \begin{theorem}
                If $f$ and $g$ are summable, and if $f\geq{g}$, then:
                \begin{equation}
                    \int_{\Omega}f\diff{\mu}\geq\int_{\Omega}g\diff{\mu}
                \end{equation}
            \end{theorem}
            \begin{theorem}
                If $f=0$ $\mu$ almost everywhere, then:
                \begin{equation}
                    \int_{\Omega}f\diff{\mu}=0
                \end{equation}
            \end{theorem}
            \begin{theorem}
                If $f$ is an integrable signed function such that:
                \begin{equation}
                    \int_{\Omega}f\diff{\mu}=0
                \end{equation}
                Then $f=0$ $\mu$ almost everywhere.
            \end{theorem}
            \begin{ltheorem}{The Triangle Inequality for Integrals}
                If $f$ is an integrable signed function, then:
                \begin{equation}
                    \Big|\int_{\Omega}f\diff{\mu}\Big|
                    \leq\int_{\Omega}|f|\diff{\mu}
                \end{equation}
            \end{ltheorem}
            \begin{ltheorem}{Monotone Convergence for Signed Functions}
                If $F$ is a summable function, if $f_{n}$ is a sequence of
                measurable functions that is monotonically increasing and such
                that, for all $n\in\mathbb{N}$, $F\leq{f}_{n}$, then:
                \begin{equation}
                    \underset{n\rightarrow\infty}{\lim}
                    \int_{\Omega}f_{n}\diff{\mu}
                    =\int_{\Omega}\underset{n\rightarrow\infty}{\lim}
                    f\diff{\mu}
                \end{equation}
            \end{ltheorem}
            \begin{proof}
                For let $\tilde{f}_{n}$ be defined by:
                \begin{equation}
                    \tilde{f}_{n}(\omega)=f_{n}(\omega)-F(\omega)
                \end{equation}
                Then for all $n\in\mathbb{N}$ and for all $\omega$,
                $\tilde{f}_{n}(\omega)\geq{0}$. But then by the monotone
                convergence theorem:
                \begin{equation}
                    \underset{n\rightarrow\infty}{\lim}
                    \int_{\Omega}(f_{n}-F)\diff{\mu}=
                    \int_{\Omega}\underset{n\rightarrow\infty}{\lim}
                    (f_{n}-F)\diff{\mu}
                \end{equation}
                But $F$ is summable, and thus we may cancel this from both
                sides. Therefore, etc.
            \end{proof}
            Without the requirement that there is a summable \textit{floor}
            for the sequence of functions $f_{n}$, the theorem may not
            hold. For consider the sequence defined by:
            \begin{equation}
                f_{n}(\omega)=\frac{\minus{1}}{n\omega}
            \end{equation}
            Then $f_{n}\rightarrow{0}$ on $(0,1)$, but the integral of
            $f_{n}$ is infinite for all $n$.
            There is an equivalent theorem with a summable majorant, rather
            than a summable minorant. Here we'd have a sequence of
            monotonically decreasing functions with a summable \textit{roof}.
            \begin{ltheorem}{Fatou's First Theorem for Signed Functions}
                If $f_{n}$ is a sequence of measurable functions such that
                there is a summable function $F$ such that $f_{n}\geq{F}$,
                then:
                \begin{equation}
                    \int_{\Omega}
                    \underset{n\rightarrow\infty}{\underline{\lim}}
                    f_{n}\diff{\mu}
                    \leq\underset{n\rightarrow\infty}{\underline{\lim}}
                    \int_{\Omega}f_{n}\diff{\mu}
                \end{equation}
            \end{ltheorem}
            \begin{ltheorem}{Fatou's Second Theorem for Signed Functions}
                If $f_{n}$ is a sequence of measurable functions such that
                there is a summable function $F$ such that
                $f_{n}\leq{F}$, then:
                \begin{equation}
                    \underset{n\rightarrow\infty}{\overline{\lim}}
                    \int_{\Omega}f_{n}\diff{\mu}
                    \leq\int_{\Omega}
                    \underset{n\rightarrow\infty}{\overline{\lim}}
                    f_{n}\diff{\mu}
                \end{equation}
                Where $\overline{\lim}$ denotes the limit superior.
            \end{ltheorem}
            \begin{ltheorem}{Dominated Convergence Theorem}
                If $f_{n}$ is a sequence of functions such that there is a
                summable minorant $F_{1}$ and a summable majorant
                $F_{2}$, that is $F_{1}\leq{f}_{n}\leq{F}_{2}$, and if
                $f_{n}\rightarrow{f}$, then:
                \begin{equation}
                    \underset{n\rightarrow\infty}{\lim}
                    \int_{\Omega}f_{n}\diff{\mu}
                    =\int_{\Omega}
                    \underset{n\rightarrow\infty}{\lim}
                    f_{n}\diff{\mu}
                \end{equation}
                That is, the limit of the integrals exists.
            \end{ltheorem}
            \begin{proof}
                For if $f_{n}$ has a summable majorant and a summable minorant,
                then both of Fatou's theorem's apply. That is:
                \begin{align}
                    \int_{\Omega}
                    \underset{n\rightarrow\infty}{\underline{\lim}}
                    f_{n}\diff{\mu}
                    &\leq\underset{n\rightarrow\infty}{\underline{\lim}}
                    \int_{\Omega}f_{n}\diff{\mu}\\
                    \underset{n\rightarrow\infty}{\overline{\lim}}
                    \int_{\Omega}f_{n}\diff{\mu}
                    &\leq\int_{\Omega}
                    \underset{n\rightarrow\infty}{\overline{\lim}}
                    f_{n}\diff{\mu}
                \end{align}
                But the limit of $f_{n}$ exists, so we have:
                \begin{equation}
                    \int_{\Omega}\underset{n\rightarrow\infty}{\lim}
                    f_{n}\diff{\mu}
                    \leq\underset{n\rightarrow\infty}{\underline{\lim}}
                    \int_{\Omega}f_{n}\diff{\mu}
                    \leq\underset{n\rightarrow\infty}{\overline{\lim}}
                    \int_{\Omega}f_{n}\diff{\mu}
                    \leq\int_{\Omega}\underset{n\rightarrow\infty}{\lim}
                    f_{n}\diff{\mu}
                \end{equation}
                Therefore:
                \begin{equation}
                    \underset{n\rightarrow\infty}{\underline{\lim}}
                    \int_{\Omega}f_{n}\diff{\mu}
                    =\underset{n\rightarrow\infty}{\overline{\lim}}
                    \int_{\Omega}f_{n}\diff{\mu}
                \end{equation}
                Therefore the limit exists, and by the inequalities:
                \begin{equation}
                    \underset{n\rightarrow\infty}{\lim}
                    \int_{\Omega}f_{n}\diff{\mu}
                    =\int_{\Omega}
                    \underset{n\rightarrow\infty}{\lim}
                    f_{n}\diff{\mu}
                \end{equation}
            \end{proof}
            We can relax the requirements of the monotone convergence theorems,
            Fatou's theorems, and the dominated convergence theorem to be
            true on all but a set of measure zero, and the results are still
            valid.
            \begin{ltheorem}{Generalized Monotone Converence Theorem}
                If $f_{n}$ is a sequence of measurable functions such that
                $f_{n}(\omega)\leq{f}_{n+1}(\omega)$ $\mu$ almost everywhere,
                and if $F$ is a summable function such that
                $F\leq{F}_{n}(\omega)$ $\mu$ almost everywhere, then:
                \begin{equation}
                    \underset{n\rightarrow\infty}{\lim}
                    \int_{\Omega}f_{n}\diff{\mu}
                    =\int_{\Omega}f\diff{\mu}
                \end{equation}
            \end{ltheorem}
            \begin{theorem}
                For define $E_{n}$ be:
                \begin{equation}
                    E_{n}=\{\omega:f_{n}(\omega)\not\leq{f}_{n+1}(\omega)\}
                \end{equation}
                And let $E$ be defined by:
                \begin{equation}
                    E=\Big(\bigcup_{n=1}^{\infty}E_{n}\Big)^{C}
                \end{equation}
                Then, as the countable union of sets of measure zero has
                measure zero, $\mu(E^{C})=0$. 
            \end{theorem}
        \section{Product Measures}
            Let $(\Omega_{1},\mathcal{A}_{1},\mu_{1})$ and
            $(\Omega_{2},\mathcal{A}_{2},\mu_{2})$ be measure spaces. We
            wish to define a \textit{natural} measure space
            on the Cartesian product $\Omega_{1}\times\Omega_{2}$.
            Let $\mathcal{P}$ be defined by:
            \begin{equation}
                \mathcal{P}=
                \{A_{1}\times{A}_{2}:
                    A_{1}\in\mathcal{A}_{1},A_{2}\in\mathcal{A}_{2}\}
            \end{equation}
            Then $\mathcal{P}$ is a semi-ring, but is not a
            $\sigma\textrm{-Algebra}$ on $\Omega_{1}\times\Omega_{2}$
            This is because the union of two rectangles may not be a
            rectangle. Similarly, the difference of two rectangles may not
            be a rectangle. However, the intersection of two rectangles is
            a rectangle, and hence this is a semi-ring.
            We defined the product $\sigma\textrm{-Algebra}$ to be the
            $\sigma\textrm{-Algebra}$ that is generated by $\mathcal{P}$. 
            \begin{ltheorem}{Carath\'{e}odory Extension Theorem}
                If $(\Omega_{1},\mathcal{A},\mu_{1})$ and
                $(\Omega_{2},\mathcal{A}_{2},\mu_{2})$ are measure spaces,
                if $\mathcal{A}$ is the product $\sigma\textrm{-Algebra}$
                on $\Omega_{1}\times\Omega_{2}$, then there is a unique
                measure $\mu$ on $\mathcal{A}$ such that, for all
                $A_{1}\in\mathcal{A}_{1}$ and all
                $A_{2}\in\mathcal{A}_{2}$:
                \begin{equation}
                    \mu(A_{1}\times{A}_{2})
                    =\mu_{1}(A_{1})\cdot\mu_{2}(A_{2})
                \end{equation}
            \end{ltheorem}
            \begin{ltheorem}{Funini's Theorem}
                If $f:\Omega_{1}\times\Omega_{2}\rightarrow\mathbb{R}$
                is a non-negative function that is
                $\mathcal{A}-\mathcal{B}$ measurable, where
                $\mathcal{A}$ is the product $\sigma\textrm{-Algebra}$,
                then:
                \begin{equation}
                    \int_{\Omega_{1}\times\Omega_{2}}f\diff{\mu}=
                    \int_{\Omega_{1}}\Big(
                        \int_{\Omega_{2}}f\diff{\mu_{2}}
                    \Big)\diff{\mu}_{1}
                    =\int_{\Omega_{2}}
                        \Big(\int_{\Omega_{1}}f\diff{\mu_{1}}\Big)
                        \diff{\mu}_{2}
                \end{equation}
            \end{ltheorem}
            As a summary, when is the following true?
            \begin{equation}
                \underset{n\rightarrow\infty}{\lim}
                \int_{\Omega}f_{n}\diff{\mu}
                \overset{?}{=}\int_{\Omega}
                \underset{n\rightarrow\infty}{\lim}f_{n}\diff{\mu}
            \end{equation}
            There are two special cases when equality can be guarenteed.
            The first is monotone convergence. If
            $f_{n}\rightarrow{f}$, where $f_{n+1}(x)\leq{f}_{n}(x)$ for
            all $n$, and if $f_{n}(x)\geq{F}$, where $F$ is a
            summable minorant, or if $f_{n}\rightarrow{f}$,
            $f_{n+1}(x)\leq{f}_{n}(x)$, and if
            $f_{n}(x)\leq{F}$, where $F$ is a summable majorant, then
            equality holds. The next case is by dominated convergence.
            If the limit of $f_{n}$ exists almost everywhere, and if
            $|f_{n}|\leq{F}$, where $F$ is summable, then by Fatou's
            Lemma:
            \begin{equation}
                \underset{n\rightarrow\infty}{\underline{\lim}}
                \int_{\Omega}f_{n}\diff{\mu}
                \geq\int_{\Omega}
                \underset{n\rightarrow\infty}{\underline{\lim}}
                f_{n}\diff{\mu}
            \end{equation}
            And also:
            \begin{equation}
                \underset{n\rightarrow\infty}{\overline{\lim}}
                \int_{\Omega}f_{n}\diff{\mu}
                \leq\int_{\Omega}
                \underset{n\rightarrow\infty}{\overline{\lim}}
                f_{n}\diff{\mu}
            \end{equation}
        \section{Probablity Spaces}
            To add later:
            \begin{enumerate}
                \item Probability space
                \item Independent $\sigma\textrm{-Algebras}$.
                \item Independent sets.
                \item Infinite sequence of $\sigma\textrm{-Algebras}$.
                \item Tail $\sigma\textrm{-Algebra}$.
                \item Terminal $\sigma\textrm{-Algebra}$.
                \item Kologorov zero-one law.
                \item If $\mathcal{A}_{j}$ independent, $F$ is self-independent.
                \item $E_{1},E_{2}\in{F}$,
                      $\mu(E_{1}\cap{E}_{2})=\mu(E_{1})\mu(E_{2})$, then
                      $\mu(E_{1})=0$ or $\mu(E_{1})=1$.
                \item Uniting $\sigma\textrm{-Algebras}$ lemma.
                \item $A_{1},\dots,A_{n}$ independent, then $A_{k},F$ independent,
                      where $F$ is the tail $\sigma\textrm{-Algebra}$.
            \end{enumerate}
        \section{Random Variables}
            Let $(\Omega,\mathcal{A},\mu)$ be a probability space.
            A probability space is a measure space such that
            $\mu(\Omega)=1$. Let $f:\Omega\rightarrow\mathbb{R}$ be
            $\mathcal{A}-\mathcal{B}$ measurable, where $\mathcal{B}$ is
            the Borel $\sigma\textrm{-Algebra}$. Such functions are called
            random-variables on $\Omega$. While there's nothing random
            about this, we use such functions to model problems in
            probability theory. The probability of an event
            $A\in\mathcal{A}$ is simply $\mu(A)$. The associated
            $\sigma\textrm{-Algebra}$ is defined as:
            \begin{equation}
                \mathcal{A}_{f}=\{f^{-1}(B):B\in\mathcal{B}\}
            \end{equation}
            This is also called the $\sigma\textrm{-Algebra}$ of events
            bearing on $f$. This is a $\sigma\textrm{-Algebra}$ on
            $\Omega$.
            \begin{ldefinition}{Distribution of a Random Variable}
                The distribution of a random variable
                $f:\Omega\rightarrow\mathbb{R}$ on a probability space
                $(\Omega,\mathcal{A},\mu)$ is the image measure
                $\mu_{f}$ of $f$.
            \end{ldefinition}
            The image measure is the measure:
            \begin{equation}
                \mu_{f}(B)=\mu(f^{-1}(B))
                =\mu(\{\omega\in\Omega:f(\omega)\in{B}\})
            \end{equation}
            This is a Lebesgue-Stieljes Measure on the Borel
            $\sigma\textrm{-Algebra}$ on $\mathbb{R}$.
            \begin{equation}
                \mu_{f}(\mathbb{R})=\mu(f^{-1}(\mathbb{R}))
                =\mu(\Omega)=1
            \end{equation}
            \begin{ldefinition}{Cumulative Distribution Function}
                The Cumulative Distribution Function of a random variable
                $f:\Omega\rightarrow\mathbb{R}$ on a probability space
                $(\Omega,\mathcal{A},\mu)$ is the function
                $F:\mathbb{R}\rightarrow\mathbb{R}$ defined by:
                \begin{equation}
                    F(x)=\mu_{f}\big((-\infty,a)\big)
                \end{equation}
                Where $\mu_{f}$ is the distribution of $f$.
            \end{ldefinition}
            Some facts about the cumulative distribution function:
            It is non-decreasing on $\mathbb{R}$, left continuous, and
            $F(\minus\infty)-F(\infty)=1$. By the Caratheodory extension
            theorem, and function $F$ that satisfies these three
            conditions is the cumulative distribution function of some
            Lebesgue-Stieljes probability measure on $\mathbb{R}$. From this
            we also have that every Lebesgue-Stieljes probability measure
            on $\mathbb{R}$ is a distribution for a random variable.
            \begin{example}
                Let $\Omega=\mathbb{R}$, let $\mathcal{A}=\mathcal{B}$,
                where $\mathcal{B}$ is the Borel $\sigma\textrm{-Algebra}$,
                and let $\mu$ be a Lebesgue-Stieljes probability measure
                on $\mathbb{R}$. Define the random variable
                $f:\Omega\rightarrow\mathbb{R}$ by
                $f(\omega)=\omega$. The inverse of any Borel set is itself,
                and thus we see that the distribution and the random
                variable coincide.
            \end{example}
            \begin{example}
                Let $\Omega=[0,1]$, $\mathcal{B}$ be the Borel
                $\sigma\textrm{-Algebra}$, and define
                $f_{1},f_{2}:\Omega\rightarrow\mathbb{R}$ by:
                \begin{equation}
                    f_{1}(\omega)=\omega
                    \quad\quad
                    f_{2}(\omega)=1-\omega
                \end{equation}
                These two functions, while different, will have the same
                cumulative distribution function. For we have:
                \begin{equation}
                    F_{1}(u)=\mu_{f_{1}}\big((\minus\infty,u)\big)=
                    \mu\big(f^{-1}(\minus\infty,u)\big)
                \end{equation}
                We can evaluate this case by case to get:
                \begin{equation}
                    F_{1}(u)=
                    \begin{cases}
                        \mu(\emptyset)=0,&u\leq{0}\\
                        \mu\big([0,u)\big)]u,0<u<1\\
                        \mu([0,1])=1,1\leq{u}
                    \end{cases}
                \end{equation}
                Looking at $F_{2}$, we have:
                \begin{equation}
                    F_{2}(u)=\mu_{f_{2}}\big((\minus\infty,u)\big)
                    =\mu\big(f_{2}^{\minus{1}}(\minus\infty,u)\big)
                \end{equation}
                Again, evaluating case by case, we get:
                \begin{equation}
                    F_{2}(u)=
                    \begin{cases}
                        \mu(\emptyset)=0,&u\leq{0}\\
                        \mu\big((1-u,1]\big)]u,0<u<1\\
                        \mu([0,1])=1,1\leq{u}
                    \end{cases}
                \end{equation}
                Thus, $F_{1}=F_{2}$.
            \end{example}
            \begin{ldefinition}{Random Vector}
                A random vector on a probability space
                $(\Omega,\mathcal{A},\mu)$ is an
                $\mathcal{A}-\mathcal{B}_{n}$ measurable function
                $\mathbf{f}:\Omega\rightarrow\mathbb{R}^{n}$, where
                $\mathcal{B}_{n}$ is the Borel $\sigma\textrm{-Algebra}$
                on $\mathbb{R}^{n}$.
            \end{ldefinition}
            As a comment, if $f:\Omega\rightarrow\mathbb{R}$ is
            $\mathcal{A}-\mathcal{B}$ measurable, then
            $\mathcal{A}_{f}\subseteq\mathcal{A}$. The associated
            $\sigma\textrm{-Algebra}$ of a random vector
            $\mathbf{f}:\Omega\rightarrow\mathbb{R}^{n}$ is:
            \begin{equation}
                A_{\mathbf{f}}
                =\{\mathbf{f}^{\minus{1}}(B):B\in\mathcal{B}_{n}\}
            \end{equation}
            \begin{theorem}
                If $(\Omega,\mathcal{A},\mu)$ is a probability space,
                $\mathcal{B}_{n}$ is the Borel $\sigma\textrm{-Algebra}$
                on $\mathbb{R}^{n}$, and if
                $\mathbf{f}:\Omega\rightarrow\mathbb{R}^{n}$ is a random
                vector such that:
                \begin{equation}
                    \mathbf{f}(\omega)=(f_{1}(\omega),\dots,f_{n}(\omega))
                \end{equation}
                Then:
                \begin{equation}
                    \mathcal{A}_{\mathbf{f}}=
                    \sigma\big(
                        \mathcal{A}_{f_{1}},\dots,\mathcal{A}_{f_{n}}\big)
                \end{equation}
                Where this is the $\sigma\textrm{-Algebra}$ generated by
                these sets.
            \end{theorem}
            \begin{proof}
                For any $f_{j}$,
                $\mathcal{A}_{f_{j}}\subseteq\mathcal{A}_{\mathbf{f}}$,
                and thus the generated $\sigma\textrm{-Algebra}$ is
                contained in $\mathcal{A}_{\mathbf{f}}$. Going the other
                ways, let $\tilde{\mathcal{B}}$ be the set of subsets
                $B\subseteq\mathbb{R}^{n}$ such that:
                \begin{equation}
                    \mathbf{f}^{\minus{1}}(B)\in
                    \sigma\big(
                        \mathcal{A}_{f_{1}},\dots,\mathcal{A}_{f_{n}}\big)
                \end{equation}
                But then for any sequence $B_{1},\dots,B_{n}\in\mathcal{B}$,
                $B_{1}\times\cdots\times{B}_{n}$ is contained in
                $\tilde{\mathcal{B}}$. But $\mathcal{B}_{n}$ is the
                smallest such $\sigma\textrm{-Algebra}$ to contain such
                sets, and thus
                $\mathcal{B}_{n}\subseteq\tilde{\mathcal{B}}$.
            \end{proof}
            \begin{ldefinition}{Distribution of a Random Vector}
                The distribution of a random vector
                $\mathbf{f}:\Omega\rightarrow\mathbb{R}^{n}$ on a
                measure space $(\Omega,\mathcal{A},\mu)$ is the measure:
                \begin{equation}
                    \mu_{\mathbf{f}}(B)=
                    \mu(\mathbf{f}^{\minus{1}}(B))
                \end{equation}
                Which is the joint distribution of
                $f_{1},\dots,f_{n}$, where:
                \begin{equation}
                    \mathbf{f}(\omega)=(f_{1}(\omega),\dots,f_{n}(\omega))
                \end{equation}
            \end{ldefinition}
            The individual distributions can be computed in terms of the
            joint distribution. This is because:
            \begin{equation}
                \mu_{f_{1}}(B)=
                \mu(f_{1}^{\minus{1}}(B))=
                \mu\big(\mathbf{f}^{\minus{1}}(
                    B\times\mathbb{R}^{n-1})\big)=
                \mu_{\mathbf{f}}\big(B\times\mathbb{R}^{n-1}\big)
            \end{equation}
            The joint distribution can not, in general, be computed
            in terms of the individual distributions. There is a special
            exception to this rule, and that is when the random variables
            are independent. That is, if the associated
            $\sigma\textrm{-Algebras}$ are independent. So events that
            bear on $f_{1},\dots,f_{n}$ are independent. If
            $E_{j}\in\mathcal{A}_{f_{j}}$, then:
            \begin{equation}
                \mu\Big(\bigcap_{k=1}^{n}E_{k}\Big)=
                \prod_{k=1}^{n}\mu(E_{k})
            \end{equation}
            \begin{theorem}
                A sequence of random variables $f_{1},\dots,f_{n}$ are
                independent if and only if the joint distribution is
                the product measure of the individual distributions.
            \end{theorem}
            \begin{proof}
                For let $B_{k}\in\mathcal{B}$ and let:
                \begin{equation}
                    E_{k}=f_{k}^{\minus{1}}(B_{k})
                \end{equation}
                But then:
                \begin{subequations}
                    \begin{align}
                        \mu\Big(\bigcap_{k=1}^{n}E_{k}\Big)&=
                        \mu\Big(\bigcap_{k=1}^{n}
                            f_{k}^{\minus{1}}(B_{k})\Big)\\
                        &=\mu\big(\mathbf{f}^{\minus{1}}
                            (B_{1}\times\dots\times{B}_{n})\big)\\
                        &=\mu_{\mathbf{f}}(B_{1}\times\dots\times{B}_{n})\\
                        &=\prod_{k=1}^{n}\mu(E_{n})\\
                        &=\prod_{k=1}^{n}\mu\big(f^{\minus{1}}(B_{k})\big)\\
                        &=\prod_{k=1}^{n}\mu_{f_{k}}(B_{k})
                    \end{align}
                \end{subequations}
            \end{proof}
            Let $\mu_{1},\dots,\mu_{n}$ be probability Lebesgue-stieljes
            measures on $\mathbb{R}$, and let $\mu$ be the product
            measure. Consider the probability space
            $(\mathbb{R}^{n},\mathcal{B}_{n},\mu)$ and the projection
            mappings $\pi_{k}:\mathbb{R}^{n}\rightarrow\mathbb{R}$:
            \begin{equation}
                \pi_{k}(\omega_{1},\dots,\omega_{n})=\omega_{k}
            \end{equation}
            \begin{theorem}
                Let $f_{n}$ be an infinite sequence of random variables
                on a probability space $(\Omega,\mathcal{A},\mu)$. Let
                $\mathcal{A}_{f_{n}}$ be the associated
                $\sigma\textrm{-Algebras}$. For every $\omega\in\Omega$,
                let:
                \begin{equation}
                    F_{\inf}(\Omega)=
                    \underset{n\rightarrow\infty}{\underline{\lim}}
                    f_{n}(\omega)
                    \quad\quad
                    F_{\sup}(\Omega)=
                    \underset{n\rightarrow\infty}{\overline{\lim}}
                    f_{n}(\omega)
                \end{equation}
                Then $F_{\inf}$ and $F_{\sup}$ are measurable with
                respect to the terminal $\sigma\textrm{-Algebra}$.
            \end{theorem}
            \begin{proof}
                For $F_{\inf}$ is measurable if and only if for all
                $u\in\mathbb{R}$, we have
                $F^{\minus{1}}\big((\minus\infty,u)\big)\in\mathcal{F}$.
                But:
                \begin{subequations}
                    \begin{align}
                        F^{\minus{1}}\big((\minus\infty,u)\big)
                        &=\{\omega:F(\omega)\leq{u}\}\\
                        &=\{\omega:\underline{\lim}f_{n}(\omega)\leq{u}\}\\
                        &=\{\omega:\underset{n}{\sup}
                            \underset{k\geq{n}}{\lim}f_{k}(\omega)\}\\
                        &=\bigcap_{n=1}^{\infty}\Big\{\omega:
                            \underset{n\geq{k}}{\inf}f_{k}(\omega)\leq{u}
                        \Big\}\\
                        &=\bigcap_{n=N}^{\infty}\Big\{\omega:
                            \underset{n\geq{k}}{\inf}f_{k}(\omega)\leq{u}
                        \Big\}
                    \end{align}
                \end{subequations}
            \end{proof}
            \begin{theorem}
                If $\mathcal{F}$ is a self-independent
                $\sigma\textrm{-Algebra}$, if $F$ is measurable with
                respect to $\mathcal{F}$, then $F$ is constant almost
                everywhere.
            \end{theorem}
            \begin{proof}
                For since $\mathcal{F}$ is self independent:
                \begin{equation}
                    \mu(\{\omega:F(\omega)<u\})=0
                    \quad\textrm{or}\quad
                    \mu(\{\omega:F(\omega)<u\})=1
                \end{equation}
                Define $A$ and $B$ as follows:
                \begin{align}
                    A&=\{u\in\mathbb{F}:\mu(\{\omega:F(\omega)<u\})=0\}\\
                    B&=\{u\in\mathbb{F}:\mu(\{\omega:F(\omega)<u\})=1\}\\
                \end{align}
                This separates the real line into two parts. By
                Dedekind's Axiom there is a $c\in\mathbb{R}$ such that,
                for all $a\in{A}$, and for all $b\in{B}$,
                $a\leq{c}\leq{b}$. But then:
                \begin{equation}
                    \mu(\{u:F(u)<c+\frac{1}{n}\})=1
                \end{equation}
                From continuity from above, we're done.
            \end{proof}
            \begin{theorem}
                If $(\Omega,\mathcal{A},\mu)$ is a probability space,
                $f_{n}$ is a sequence of independent random variables,
                then the limit inferior and the limit superior are
                constants $\mu$ almost everywhere.
            \end{theorem}
            \begin{proof}
                For the limit inferior and limit superior are measurable
                with respect to the terminal $\sigma\textrm{-Algebra}$.
                By the Kolmogorov zero-one law, $\mathcal{F}$ is
                self-independent if $\mathcal{A}_{f_{n}}$ are independent.
                Thus, by the previous theorem, these functions are constants
                almost everywhere.
            \end{proof}
            Thus the limit of random-variables is entirely not random, but
            constant functions.
            \begin{theorem}
                If $f_{n}$ is a sequence of random variables, then the
                limit of $f_{n}$ almost surely exists, or almost never
                exists.
            \end{theorem}
            \begin{proof}
                For since the limit superior and limit inferior are
                constants almost everywhere, then eithere they agree,
                in which there's convergence almost surely, or they do
                not agree, in which there's convergence almost never.
            \end{proof}
            \begin{ldefinition}{Expectation Value}
                The expectation value of a summable random variable
                $f:\Omega\rightarrow\mathbb{R}$ on a measure space
                $(\Omega,\mathcal{A},\mu)$ is the real number
                $E(f)$ defined by:
                \begin{equation}
                    E(f)=\int_{\Omega}f\diff{\mu}
                \end{equation}
            \end{ldefinition}
            The expectation can be expressed in terms of the distribution
            by using the measure transformation theorem. If
            $g:\mathbb{R}\rightarrow\mathbb{R}$ is a real valued function,
            then:
            \begin{equation}
                \int_{\Omega}g\diff{\mu}=
                \int_{\mathbb{R}}g\circ{f}\diff{\mu_{f}}
            \end{equation}
            Now we apply this in the simple case when $g(u)=u$. Then:
            \begin{equation}
                E(f)=\int_{\Omega}f\diff{\mu}
                =\int_\mathbb{R}u\diff{\mu_{f}}
            \end{equation}
            Where we assume that $f$ is summable against $\mu$. Thus,
            $u$ is summable against $\mu_{f}$. So, we have that:
            \begin{equation}
                \int_{\mathbb{R}}|u|\diff{\mu_{f}}<\infty
            \end{equation}
            \begin{ldefinition}{Variance}
                The variance of a random variable
                $f:\Omega\rightarrow\mathbb{R}$ on a measure space
                $(\Omega,\mathcal{A},\mu)$, is the real number
                $Var(f)$ defined by:
                \begin{equation}
                    Var(f)=E\big(f-E(f)\big)^{2}
                    =\int_{\Omega}\big(f-E(f)\big)^{2}\diff{\mu}
                \end{equation}
            \end{ldefinition}
            \begin{theorem}
                \begin{equation}
                    Var(f)=E(f^{2})-E(f)^{2}
                \end{equation}
            \end{theorem}
        \section{Lecture 8-ish Maybe}
            If $(\Omega,\mathcal{A},\mu)$ is a measure space,
            $f:\Omega\rightarrow\mathbb{R}$ is a Borel measurable
            function, then the expectation is:
            \begin{equation}
                E(f)=\int_{\Omega}f\diff{\mu}
            \end{equation}
            The functions $f_{1},\dots,f_{n}$ are independent if
            the associated $\sigma\textrm{-Algebras}$ are independent,
            $\mathcal{A}_{f_{1}}.\dots,\mathcal{A}_{f_{n}}$, where
            the associated $\sigma\textrm{-Algebra}$ is defined
            as:
            \begin{equation}
                \mathcal{A}_{f}=\{f^{\minus{1}}(B):B\in\mathcal{B}\}
            \end{equation}
            Where $\mathcal{B}$ is the Borel
            $\sigma\textrm{-Algebra}$. A random vector is a function
            $\mathbf{f}:\Omega\rightarrow\mathbb{R}^{n}$. The
            distribution of $\mathbf{f}$ is defined as:
            \begin{equation}
                \mu_{\mathbf{f}}(B)=
                    \mu\big(\mathbf{f}^{\minus{1}}(B)\big)
            \end{equation}
            This is also called the joint distribution. We then proved
            that $f_{1},\dots,f_{n}$ are independent if and only
            if the joint distribution is the product of the
            individual distributions.
            \begin{theorem}
                If $(\Omega,\mathcal{A},\mu)$ is a probabilty space,
                and if $f_{1},\dots,f_{n}$ are independent functions,
                then:
                \begin{equation}
                    E\Big(\prod_{k}f_{k}\Big)
                    =\prod_{k}E(f_{k})
                \end{equation}
            \end{theorem}
            \begin{proof}
                For define $g:\Omega\rightarrow\mathbb{R}$ by:
                \begin{equation}
                    g(\omega)=\prod_{k=1}^{n}f_{k}(\omega)
                \end{equation}
                Let $\mathbf{f}:\Omega\rightarrow\mathbb{R}^{n}$ be
                defined by:
                \begin{equation}
                    \mathbf{f}(\omega)=
                    \big(f_{1}(\omega),\dots,f_{n}(\omega)\big)
                \end{equation}
                Then using the measure transformation, we have:
                \begin{align}
                    \int_{\Omega}\prod_{k=1}^{n}f_{k}\diff{\mu}
                    &=\int_{\Omega}g\big(\mathbf{f}(\omega)\big)
                        \diff{\mu}\\
                    &\int_{\mathbb{R}^{n}}g(u_{1},\dots,u_{n})
                        \mu_{\mathbf{f}}\\
                    &=\int_{\mathbb{R}^{n}}\prod_{k=1}^{n}u_{k}
                        \mu_{\mathbf{f}}
                \end{align}
            \end{proof}
            Suppose $n=2$. Then, since $f_{1}$ and $f_{2}$ are
            independent, $\mu_{(f_{1},f_{2})}$ is the product of
            the measures $\mu_{f_{1}}$ and $\mu_{f_{2}}$. Thus
            by Fubini's theorem:
            \begin{equation}
                \int_{\mathbb{R}^{2}}u_{1}u_{2}\mu_{(f_{1},f_{2})}
                =\int_{\mathbb{R}}\Big(
                    \int_{\mathbb{R}}u_{1}u_{2}\mu_{f_{2}}\Big)
                        \mu_{f_{1}}
                =\int_{\mathbb{R}}u_{1}\Big(
                    \int_{\mathbb{R}}u_{1}\mu_{f_{1}}\Big)
                =\int_{\mathbb{R}}u\mu_{f_{1}}
                    \int_{\mathbb{R}}u_{2}\mu_{f_{2}}
                =\int_{\Omega}f_{1}\mu\int_{\Omega}f_{2}\mu
            \end{equation}
            From a course in integral calculus, one should be very
            surprised by this result, for it says that if
            $f_{1},\dots,f_{n}$ are independent, then:
            \begin{equation}
                \int_{\Omega}\prod_{k=1}^{n}f_{k}\diff{\mu}=
                \prod_{k=1}^{n}\int_{\Omega}f_{k}\diff{\mu}
            \end{equation}
            This is almost never true for a given set of functions,
            but if they are independent then the result holds.
            \subsection{Covariance}
                The covariance of $f_{1}$ and $f_{2}$ is:
                \begin{equation}
                    E\Big(\big(f_{1}-E(f_{1})\big)
                        \big(f_{2}-E(f_{2})\big)\Big)
                    =\int_{\Omega}\big(f_{1}-E(f_{1})\big)
                        \big(f_{2}-E(f_{2})\big)\diff{\mu}
                \end{equation}
                We can simplify this down to:
                \begin{equation}
                    E(f_{1}f_{2})-E(f_{1})E(f_{2})
                \end{equation}
                If $f_{1}$ and $f_{2}$ are independent, then:
                \begin{equation}
                    Cov(f_{1},f_{2})=0
                \end{equation}
                The converse is not true. It does not imply that
                $f_{1}$ and $f_{2}$ are independent.
                For let:
                \begin{equation}
                    \Omega=\{1,2,3\}
                \end{equation}
                Let $\mathcal{A}=\mathcal{P}(\Omega)$ and let
                $\mu$ be the counting measure on $\Omega$. That is:
                \begin{equation}
                    \mu(A)=\frac{\mathrm{Card}(A)}{3}
                \end{equation}
                Then $(\Omega,\mathcal{A},\mu)$ is a probability
                measure. Define $f_{1}$ and $f_{2}$ as follows:
                \begin{align}
                    f_{1}(\omega)&=
                    \begin{cases}
                        1,&\omega=1\\
                        0,&\omega=0\\
                        1&\omega=2
                    \end{cases}\\
                    f_{1}(\omega)&=
                    \begin{cases}
                        1,&\omega=1\\
                        0,&\omega=0\\
                        \minus{1}&\omega=2
                    \end{cases}
                \end{align}
                Then we compute and get:
                \begin{equation}
                    E(f_{1})=\int_{\Omega}f_{1}\diff{\mu}=0
                \end{equation}
                And also:
                \begin{equation}
                    E(f_{2})=\frac{2}{3}
                \end{equation}
                But if we multiply, we see that
                $f_{1}f_{2}=f_{1}$, and therefore:
                \begin{equation}
                    E(f_{1}f_{2})=E(f_{1})=0
                \end{equation}
                But then:
                \begin{equation}
                    E(f_{1}f_{2})-E(f_{1})E(f_{2})=0
                \end{equation}
                And thus $f_{1}$ and $f_{2}$ are uncorrelated.
                But they are dependent. We may expect this since
                $f_{2}=f_{1}^{2}$. Let's compute the associated
                $\sigma\textrm{-Algebras}$. We have:
                \begin{align}
                    f_{1}^{\minus{1}}(\{1\})
                    &=\{1\}\\
                    f_{2}^{\minus{1}}&=\{1,3\}
                \end{align}
                But then:
                \begin{align}
                    \mu(f_{1}^{\minus{1}}(\{1\})
                    &=\frac{1}{3}\\
                    \mu(f_{2}^{\minus{1}}(\{1\})&=\frac{2}{3}
                \end{align}
                But the product measure is:
                \begin{equation}
                    \mu_{(f_{1},f_{2})}(\{1\})=\frac{1}{3}
                \end{equation}
                And this is not the product of the two measure, and
                therefore it they are not independent.
                If $Cov(f_{1},f_{2})=0$, we say that $f_{1}$ and
                $f_{2}$ are uncorrelated.
                \begin{theorem}
                    If $f_{1},\dots,f_{n}$ are random variables
                    that are pairwise uncorrelated, then:
                    \begin{equation}
                        Var(\sum_{k=1}^{n}f_{k})=
                        \sum_{k=1}^{n}Var(f_{k})
                    \end{equation}
                \end{theorem}
                \begin{proof}
                    For:
                    \begin{equation}
                        \int_{\Omega}
                            \Big(\sum_{k=1}^{n}f_{k}-
                                E(\sum_{k=1}^{n}f_{k}\Big)\diff{\mu}
                        =\sum_{i,j}\int_{\Omega}
                            (f_{i}-E(f_{i}))(f_{j}-E(f_{j}))\diff{\mu}
                    \end{equation}
                    But the $f_{i}$ are pairwise uncorrelated, and
                    thus this product is zero if $i\ne{j}$. Thus, we
                    get:
                    \begin{equation}
                        \int_{\Omega}
                            \Big(\sum_{k=1}^{n}f_{k}-
                                E(\sum_{k=1}^{n}f_{k}\Big)\diff{\mu}
                        =\sum_{k=1}^{n}Var(f_{k})
                    \end{equation}
                \end{proof}
        \section{Laws of Large Numbers}
            Consider a fair coin and toss it $n$ times. We would
            expect that, as $n$ gets large, the number of times
            heads occurs and the number of times heads occurs is
            roughly the same. That is:
            \begin{equation}
                \frac{|\textrm{Heads}|-|\textrm{Tails}|}{n^{2}}
                \rightarrow{0}
            \end{equation}
            And also:
            \begin{equation}
                \frac{|\textrm{Heads}|\times|\textrm{Tails}|}{n}
                \rightarrow\frac{1}{2}
            \end{equation}
            We want to build a more rigorous notion from this idea
            and create a mathematical model out of this. We use
            probability spaces as this model. Let
            $(\Omega,\mathcal{A},\mu)$ be a probability space and
            let $f_{j}:\Omega\rightarrow\mathbb{R}$ be random variables
            take on the values $\minus{1}$ and $1$, and such that
            they are independent. Then the associated
            $\sigma\textrm{-Algebra}$ are:
            \begin{equation}
                \mathcal{A}_{f_{j}}=
                \{\emptyset,f^{\minus{1}}(\{\minus{1}\}),
                    f^{\minus{1}}(\{1\}),\Omega\}
            \end{equation}
            The measure on the space is such that:
            \begin{equation}
                \mu\Big(f^{\minus{1}}\big(\{\minus{1}\}\big)\Big)=
                \mu\Big(f^{\minus{1}}\big(\{1\}\big)\Big)=
                \frac{1}{2}
            \end{equation}
            Define a new function by:
            \begin{equation}
                F_{N}(\omega)=\frac{1}{N}\sum_{k=1}^{N}f_{k}(\omega)
            \end{equation}
            Then $F_{N}(\omega)$ is the number of times 1 occurs
            minus the number of times -1 occurs, divided by $N$.
            It seems likely that this function should converge to
            zero for large $N$. Recall that there are three different
            types of convergence. We say
            $g_{n}\rightarrow{g}$ almost everywhere if there is a
            set of measure 0 such that $g_{n}\rightarrow{g}$ on the
            complement of this set. We say that
            $g_{n}\rightarrow{g}$ almost uniformly if there is a set
            of arbitrarily small measure $\varepsilon$ such that
            $g_{n}\rightarrow{g}$ uniformly on the complement.
            Finally, $g_{n}\rightarrow{g}$ in measure if for all
            $\delta>0$:
            \begin{equation}
                \mu\Big(\{\omega:|g_{n}(\omega)-g(\omega)|\geq\delta\}
                \Big)\rightarrow{0}
            \end{equation}
            We have seen the almost uniform convergence is the
            strongest and implies the other two. By Egorov, since
            $\mu(\Omega)=1$ in a probability space,
            convergence almost everywhere implies convergence almost
            uniformly. Lastly, convergence in measure implies there
            is a subsequence that converges almost uniformly.
            \begin{ldefinition}{Strong Law of Large Numbers}
                A sequence that obeys the Strong Law of Large Numbers
                in a probability space $(\Omega,\mathcal{A},\mu)$
                is a sequence $f_{n}$ such that:
                \begin{equation}
                    \frac{1}{N}\sum_{n=1}^{N}
                        \Big[f_{n}(\omega)-E(f_{n})\Big]\rightarrow{0}
                \end{equation}
                $\mu$ almost everywhere.
            \end{ldefinition}
            \begin{ldefinition}{Weak Law of Large Numbers}
                A sequence that obeys the Weak Law of Large Numbers
                in a probability space $(\Omega,\mathcal{A},\mu)$
                is a sequence $f_{n}$ such that:
                \begin{equation}
                    \frac{1}{N}\sum_{n=1}^{N}
                        \Big[f_{n}(\omega)-E(f_{n})\Big]\rightarrow{0}
                \end{equation}
                Where the convergence is in measure.
            \end{ldefinition}
            \begin{ltheorem}{Khinchin's Weak Law of Large Numbers}
                If $(\Omega,\mathcal{A},\mu)$ is a probability space,
                if $f_{j}$ is a sequence of random variables that
                are pair-wise uncorrelated such that:
                \begin{equation}
                    \frac{1}{n^{2}}\sum_{j=1}^{n}Var(f_{k})
                    \rightarrow{0}
                \end{equation}
                Then $f_{j}$ obeys the Weak Law of Large Numbers.
            \end{ltheorem}
            \begin{proof}
                For:
                \begin{equation}
                    \int_{\Omega}\Big(\frac{1}{n}\sum_{k=1}^{n}\big(
                        f_{k}(\omega)-E(f_{k})\big)\Big)^{2}\diff{\mu}
                    =\frac{1}{n^{2}}\sum_{k=1}^{n}Var(f_{k})
                \end{equation}
                Let:
                \begin{equation}
                    \Omega_{\delta,n}=
                    \{\omega:|\frac{1}{n}\sum_{k=1}^{n}
                        \big(f_{k}(\omega)-E(f_{k})\big)|\geq\delta\}
                \end{equation}
                But by the Chebyshev inequality, we have:
                \begin{equation}
                    \int_{\Omega}\Big(\frac{1}{n}\sum_{k=1}^{n}\big(
                        f_{k}(\omega)-E(f_{k})\big)\Big)^{2}\diff{\mu}
                    \geq\int_{\Omega_{\delta}}
                        \Big(\frac{1}{n}\sum_{k=1}^{n}\big(
                        f_{k}(\omega)-E(f_{k})\big)\Big)^{2}\diff{\mu}
                    \geq\delta^{2}\int_{\Omega_{\delta}}\diff{\mu}
                \end{equation}
                But then:
                \begin{equation}
                    \mu(\Omega_{\delta,n})\leq
                    \frac{1}{\delta^{2}}\frac{1}{n}^{2}
                    \sum_{j=1}^{n}V(f_{k})
                \end{equation}
                But this last part tends to zero. Therefore, etc.
            \end{proof}
            \begin{lexample}{}{Distribution}
                If all of the $f_{i}$ have the same distribution, or
                if they are uniformly bounded, then the theorem applies.
                This can be used to show that our model for a fair
                coin toss obeys the weak law of large numbers.
            \end{lexample}
            Suppose $g_{n}(\omega)\rightarrow{g}(\omega)$ almost
            everywhere. Then, for all $\delta>0$ there is an
            $N$ such that, for all $n>N$, we have:
            \begin{equation}
                |g_{n}(\omega)-g(\omega)|<k^{\minus{1}}
            \end{equation}
            For some $k$. Consider the negation of this claim. Then
            there exists $k\in\mathbb{N}$ such that, for all
            $N\in\mathbb{N}$ there is an $n>N$ such that:
            \begin{equation}
                |g_{n}(\omega)-g(\omega)|\geq{k}^{\minus{1}}
            \end{equation}
            COnsider the following set:
            \begin{equation}
                B=\bigcup_{n=1}^{\infty}\bigcap_{N=1}^{\infty}
                    \bigcup_{k=N}^{\infty}\big\{\omega:
                    |g_{n}(\omega)-g(\omega)|\geq{k}^{\minus{1}}\big\}
            \end{equation}
            This is the set of $\omega$ such that
            $g_{n}(\omega)\not\rightarrow{g}(\omega)$. We wish to show
            that $\mu(B)=0$. This will happen if and only if for all
            $k\in\mathbb{N}$:
            \begin{equation}
                \mu\Big(\bigcap_{N=1}^{\infty}\bigcup_{n=N}^{\infty}
                    \big\{\omega:|g_{n}(\omega)-g(\omega)|
                    \geq{k}^{\minus{1}}\big\}\Big)=0
            \end{equation}
            Consider a collection of set $A_{n}$ and define:
            \begin{equation}
                \overline{A}=\bigcap_{N=1}^{\infty}
                    \bigcup_{n=N}^{\infty}A_{n}
            \end{equation}
            If the $A_{n}$ are independent, then $\overline{A}$ is
            a terminal event, and thus by the Kormogorov zero-one law,
            eighet $\mu(\overline{A})=1$ or $\mu(\overline{A})=0$.
            \begin{theorem}
                If:
                \begin{equation}
                    \sum_{n=1}^{\infty}\mu)A_{n})<\infty
                \end{equation}
                Then:
                \begin{equation}
                    \mu\Big(\bigcap_{N=1}^{\infty}
                        \bigcup_{N=n}^{\infty}A_{n}\Big)=0
                \end{equation}
            \end{theorem}
            \begin{proof}
                For:
                \begin{equation}
                    \mu\Big(\bigcap_{N=1}^{\infty}
                        \bigcup_{N=n}^{\infty}A_{n}\Big)
                    \leq\mu\Big(\bigcup_{N=n}^{\infty}A_{n}\Big)
                    \leq\sum_{n=N}^{\infty}\mu(A_{n}
                \end{equation}
                But this sum converges, and thus the tail end can
                be made arbitrarily small.
            \end{proof}
            \begin{ltheorem}{Borel-Cantelli Lemma}
                If $A_{n}$ are pair-wise independent and are such
                that:
                \begin{equation}
                    \sum_{k=1}^{\infty}\mu(A_{n})=\infty
                \end{equation}
                Then:
                \begin{equation}
                    \mu\Big(\bigcap_{N=1}^{\infty}
                        \bigcup_{N=n}^{\infty}A_{n}\Big)=1
                \end{equation}
            \end{ltheorem}
            \begin{proof}
                For if:
                \begin{equation}
                    \mu\Big(\bigcup_{N=1}^{\infty}\bigcap_{n=N}^{\infty}
                        A_{n}^{C}\big)=0
                \end{equation}
                Then, for all $N$:
                \begin{equation}
                    \mu\Big(\bigcap_{n=N}^{\infty}A_{n}^{C}\Big)=0
                \end{equation}
                So it suffices to show that this is true. For let
                $N\in\mathbb{N}$, and define:
                \begin{equation}
                    B=\bigcap_{n=N}^{\infty}A_{n}^{C}
                \end{equation}
                Also define:
                \begin{equation}
                    B_{M}=\bigcap_{n=N}^{M}A_{n}^{C}
                \end{equation}
                It then follows from continuity from below that:
                \begin{equation}
                    \mu(B)=\underset{M\rightarrow\infty}{\lim}\mu(B_{M})
                    =\underset{M\rightarrow\infty}{\lim}
                        \mu\Big(\bigcap_{n=N}^{M}A_{n}^{C}\Big)
                \end{equation}
                But from independence, we obtain:
                \begin{equation}
                    \mu(B)=\underset{M\rightarrow\infty}{\lim}
                        \prod_{n=N}^{M}\mu\big(A_{n}^{C}\big)
                    =\underset{M\rightarrow\infty}{\lim}
                        \prod_{n=N}^{M}\mu\big(1-A_{n}\big)
                \end{equation}
                Using the exponential function, we note that
                $1-x\leq\exp(\minus{x})$, and so:
                \begin{equation}
                    \mu(B)\leq               
                    \underset{M\rightarrow\infty}{\lim}
                        \prod_{n=N}^{M}\exp\big(\minus\mu(A_{n})\big)
                    =\underset{M\rightarrow\infty}{\lim}
                        \exp\Big(\sum_{n=N}^{M}\mu(A_{n})\Big)=0
                \end{equation}
            \end{proof}
            The independence of the $A_{n}$ is indeed necessary for
            this theorem. For let $A_{n}=A_{0}$, and let
            $\mu(A_{0})=\frac{1}{2}$. Then the sum will indeed
            diverge, but the measure of final set is still
            $\frac{1}{2}$.
            The Borel-Cantelli lemma thus complements the
            Kormogrov Zero-One law by giving the precise criterion for
            when the measure is either one or zero. Given a sequence
            of random events, the terminal event has measure one
            if and only if the sum of the individual measures converges,
            and is equal to one otherwise.
            \begin{ltheorem}{Borel's Strong Law of Large Numbers}
                If $f_{n}$ is a sequence of random variables such that:
                \begin{equation}
                    \int_{\Omega}|f_{n}|^{4}\diff{\mu}\leq{M}
                \end{equation}
                For all $n\in\mathbb{N}$, then $f_{n}$ obeys the
                strong law of large numbers.
            \end{ltheorem}
            \begin{proof}
                It suffices to show that, for all $\varepsilon>0$:
                \begin{equation}
                    \mu\Big(\bigcap_{N=1}^{\infty}
                        \bigcup_{N=n}^{\infty}
                        \big\{\omega:|\frac{1}{n}\sum_{k=1}^{n}
                            f_{k}(\omega)|\geq\varepsilon\Big)=0
                \end{equation}
                Denote the sequence of centered random variables by:
                \begin{equation}
                    \overset{\circ}{f}_{n}(\omega)=
                    f_{n}(\omega)=E(f_{n}(\omega))
                \end{equation}
                To show this, we need to show that:
                \begin{equation}
                    \sum_{n=1}^{\infty}\mu\Big(
                        \big\{\omega:|\frac{1}{n}\sum_{k=1}^{n}
                            \overset{\circ}{f}_{n}(\omega)|
                            \geq\varepsilon\big\}\Big)<\infty
                \end{equation}
                Define:
                \begin{equation}
                    \Omega_{n,\varepsilon}=
                    \Big\{\omega:\big|\frac{1}{n}\sum_{k=1}^{n}
                        \overset{\circ}{f}_{n}(\omega)\big|
                        \geq\varepsilon\Big\}
                \end{equation}
                But then:
                \begin{equation}
                    \int_{\Omega}\big|\frac{1}{n}\sum_{k=1}^{n}
                        \overset{\circ}{f}_{n}\big|^{4}\diff{\mu}
                    \geq\int_{\Omega_{n,\varepsilon}}
                    \big|\frac{1}{n}\sum_{k=1}^{n}
                        \overset{\circ}{f}_{n}\big|^{4}\diff{\mu}
                    \geq\varepsilon^{4}
                        \mu\big(\Omega_{\varepsilon,n}\big)
                \end{equation}
                Combining this together, we have:
                \begin{equation}
                    \mu\big(\Omega_{\varepsilon,n}\big)
                    \leq\frac{1}{\varepsilon^{4}}\frac{1}{n^{4}}
                    \int_{\Omega}\Big(\sum_{k=1}^{n}
                    \overset{\circ}{f}_{k}(\omega)\Big)^{4}\diff{\mu}
                    =\frac{1}{\varepsilon^{4}}\frac{1}{n^{4}}
                    \sum_{i,j,k,\ell}\int_{\Omega}
                    \overset{\circ}{f}_{i}\overset{\circ}{f}_{j}
                    \overset{\circ}{f}_{k}\overset{\circ}{f}_{\ell}
                    \diff{\mu}
                \end{equation}
                But the $f_{n}$ are independent, and thus the
                $\overset{\circ}{f}_{n}$ are independent. But then
                $\mathcal{A}_{\overset{\circ}{f}_{n}}$ are independent,
                and thus $\overset{\circ}{f_{i}}$ is independent
                from the product
                $\overset{\circ}{f}_{j}\overset{\circ}{f}_{k}\overset{\circ}{f}_{\ell}$. But if they are independent, then:
                \begin{equation}
                    \int_{\Omega}
                    \overset{\circ}{f}_{i}\overset{\circ}{f}_{j}
                    \overset{\circ}{f}_{k}\overset{\circ}{f}_{\ell}
                    \diff{\mu}=
                    \int_{\Omega}
                    \overset{\circ}{f}_{i}\diff{\mu}
                    \int_{\Omega}\overset{\circ}{f}_{j}
                    \overset{\circ}{f}_{k}\overset{\circ}{f}_{\ell}
                    \diff{\mu}=0
                \end{equation}
                There are two cases left, when the indices are equal
                in pairs, and when all of the indices are equal. In
                the cases where all are equal, we have:
                \begin{equation}
                    \sum_{i=1}^{n}
                    \int_{\Omega}
                    |\overset{\circ}{f}_{i}|^{4}\diff{\mu}
                    \leq{M}n
                \end{equation}
                For the case of pairs, we have $n^{2}-n$ possibilities,
                and thus:
                \begin{equation}
                    \sum_{i,j}\int_{\Omega}
                    |\overset{\circ}{f}_{i}^{2}
                    \overset{\circ}{f}_{j}^{2}|\diff{\mu}
                    \leq{M}(n^{2}-n)
                \end{equation}
                Therefore, we have:
                \begin{equation}
                    \frac{1}{\varepsilon^{4}}\frac{1}{n^{4}}
                    \sum_{i,j,k,\ell}\int_{\Omega}
                    \overset{\circ}{f}_{i}\overset{\circ}{f}_{j}
                    \overset{\circ}{f}_{k}\overset{\circ}{f}_{\ell}
                    \diff{\mu}
                    \leq\frac{M}{\varepsilon^{4}}
                    \frac{1}{n^{2}}
                \end{equation}
                But:
                \begin{equation}
                    \sum_{n=1}^{\infty}\frac{1}{n^{2}}
                    =\frac{\pi^{2}}{6}<\infty
                \end{equation}
                Thus, the measure is zero.
            \end{proof}
            The $|f_{n}|^{4}$ are called the fourth moments of the
            $f_{n}$. There are sequences that obey the weak law but
            not the strong law. Borel's theorem shows that uniformly
            bounded sequences of random variables automatically obey
            the strong law of strong numbers, since a uniformly
            bounded sequence will have uniformly bounded fourth
            moments. To find a sequence that obeys the weak law but
            not the strong law, we will need to consider sequences
            that take on arbitrarily large values.
            \begin{lexample}{}{Random_Variables}
                Let $f_{n}$ be a sequence of random variables such that
                the following are true:
                \begin{subequations}
                    \begin{align}
                        \mu\Big(\{\omega:f_{n}(\omega)=n\}\Big)
                        &=\frac{P_{n}}{2}\\
                        \mu\Big(\{\omega:f_{n}(\omega)=\minus{n}\}\Big)
                        &=\frac{P_{n}}{2}\\
                        \mu\Big(\{\omega:f_{n}(\omega)=0\}\Big)
                        &=1-P_{n}
                    \end{align}
                \end{subequations}
                We need to find a sequence $P_{n}$ such that the
                $f_{n}$ will obey the weak law but not the strong law.
                Choosing the $P_{n}$ to be small will most likely
                result in the sequence obeying the strong law. Indeed,
                if $P_{n}=0$, then the $f_{n}$ will obey the strong
                law. In fact, if:
                \begin{equation}
                    P_{n}\leq\frac{1}{n^{2}}
                \end{equation}
                Then $f_{n}$ will obey the strong law. This is a
                consequence of the Borel-Cantelli lemma. If $P_{n}$
                is to large, it may not be true that the $f_{n}$ obeys
                the weak law. For example, suppose:
                \begin{equation}
                    P_{n}=\frac{1}{n}
                \end{equation}
                We cannot apply Khinchin's theorem, since:
                \begin{equation}
                    \frac{1}{n^{2}}\sum_{j=1}^{n}V(f_{j})=
                    \frac{n(n+1)}{2n^{2}}
                \end{equation}
                And this does not converge to zero. Let:
                \begin{equation}
                    P_{n}=\frac{1}{n\ln(n+2)}
                \end{equation}
                Let's now show that $f_{n}$ will obey the weak law.
                It does. But it does not obey the strong law. We will
                need to use the Borel-Cantelli lemma. But the sum:
                \begin{equation}
                    \sum_{n=1}^{\infty}\frac{1}{n\ln(n+2)}=\infty
                \end{equation}
                Therefore:
                \begin{equation}
                    \mu\Big(\bigcap_{N=1}^{\infty}\bigcup_{n=N}^{\infty}
                        \{\omega:|f_{n}(\omega)|\}\Big)=1
                \end{equation}
                This contradicts the strong law of large numbers. For
                suppose not. Then, for almost every $\omega$, and for
                all $N$, there is an $N>N$ such that
                $|f_{n}(\omega)|=n$. Then thre exists a sequence
                $n_{k}$ such that $|f_{n_{k}}(\omega)|=n_{k}$.
                But:
                \begin{equation}
                    \frac{1}{n_{k}}\sum_{j=0}^{n_{k}}f_{j}\rightarrow{0}
                \end{equation}
                And therefore:
                \begin{equation}
                    \frac{1}{n_{k}-1}\sum_{j=0}^{n_{k}}f_{j}
                        \rightarrow{0}
                \end{equation}
                Taking the difference, we get:
                \begin{equation}
                    \frac{1}{n_{k}}f_{n_{k}}(\omega)\rightarrow{0}
                \end{equation}
                But $|f_{n_{k}}(\omega)|=n_{k}$, a contradiction.
                So the $f_{n}$ do not obey the strong law.
            \end{lexample}
            \subsection{Borel Numbers}
                Let $0\leq{x}\leq{1}$ and suppose $x$ has the
                representation $x=0.x_{1}x_{2}\dots$ and exclude
                numbers with two representations. For example,
                $1=0.999\dots$. The measure of the set of these numbers
                is zero. Let $0\leq{a}\leq{9}$. Let $C_{n}(x)$ be
                the number of $a$ among the first $n$ digits. Then:
                \begin{equation}
                    \frac{C_{n}(x)}{n}\rightarrow\frac{1}{10}
                \end{equation}
                For almost every $x$. Let $\Omega=[0,1]$ and
                $\mathcal{B}$ be the Borel $\sigma\textrm{-Algebra}$.
                Also, let $\mu$ be the Lebesgue measure. Consider
                the functions:
                \begin{equation}
                    f_{j}(x)=
                    \begin{cases}
                        1,&x_{j}=a\\
                        0,&x_{j}\ne{a}
                    \end{cases}
                \end{equation}
                Then:
                \begin{equation}
                    \frac{C_{n}(x)}{n}=\frac{1}{n}\sum_{k=1}^{n}
                        f_{k}(x)
                \end{equation}
                If the $f_{k}$ obey the strong law of large numbers,
                then:
                \begin{equation}
                    \frac{1}{n}\sum_{k=1}^{n}\big(f_{k}-E(f_{k})\big)
                    \rightarrow{0}
                \end{equation}
                $\mu$ almost everywhere. We have that $f_{j}$ are
                bounded, and thus it suffices to show that they are
                also independent. Define:
                \begin{equation}
                    \mathcal{A}_{f_{i}}=
                    \{\emptyset,A_{j},A_{j}^{C},\Omega\}
                \end{equation}
                Where:
                \begin{equation}
                    A_{j}=\{x:f_{j}(x)=1\}
                \end{equation}
                THe $A_{j}$ are the set of elements $x$ such that
                $x=0.x_{1}x_{2}\dots{a}x_{j+1}x_{j+2}\dots$ Using this
                there are $10^{j-1}$ options for the first $j-1$
                digits. This set is covered by $10^{j-1}$ intervals,
                each of length $10^{\minus{j}}$. Thus, the Lebesgue
                measure of $A_{j}$ is $\frac{1}{10}$. We now need to
                show that, for distinct $j,k$, that the measure of the
                intersection is $\frac{1}{100}$. Suppose $j<k$. Then
                $A_{j}\cap{A}_{k}$ is the set of numbers with $a$ in
                the $j^{th}$ decimal and $a$ in the $k^{th}$ decimal.
                There are $10^{j-1}$ ways to choose the first
                $j-1$ digits, and $10^{k-j-1}$ ways to choose the
                next $k-j-1$ digits. Total, there are
                $10^{k-2}$ digits to choose. So we can cover this
                set with $10^{k-2}$ intervals, each of lenght
                $10^{\minus{k}}$. Thus, the measure of the intersection
                is $10^{\minus{2}}$. Theefore, the $f_{j}$ are
                independent. By Borel's Strong Law of Large Numbers,
                the $f_{j}$ obey the strong law of large numbers.
                \par\hfill\par
                Let $\Omega=\mathbb{Z}_{n}$, let $\mathcal{A}$ be
                the power set, and let $\mu$ be the counting
                measure on $\Omega$. Taking the product
                $\Omega^{n}$, and considering the product measure,
                we see that every point has measure $10^{\minus{n}}$.
                Thus, if we consider the infinite product, points will
                have measure zero. This isn't too strange since the
                Lebesgue measure is such that points have measure
                zero. Let $\tilde{\Omega}$ be the infinite product
                and let $B_{n}\in\Omega^{n}$. Then
                $B_{n}\times\Omega_{n+1}\times\dots$ is contained in
                $\tilde{\Omega}$. Let $\tilde{\mathcal{B}}$ be the
                smallest $\sigma\textrm{-Algebra}$ on the product
                space that contains all of these types of sets, and let
                $\tilde{\mu}$ be the extension measure. Then
                $f_{j}(\omega)=\omega_{j}$ are independent by
                construction of the product measure, and also:
                \begin{equation}
                    \mu(f_{j}=a)=\frac{1}{10}
                \end{equation}
                There is a map $\tilde{\Omega}\mapsto[0,1]$ by sending
                $(\omega_{1},\dots)$ to $0.\omega_{1}\omega_{2}\dots$.
                The image measure of the product measure $\tilde{\mu}$
                is the Lebesgue measure. So we have an equivalent
                model of $[0,1]$ with the Lebesgue measure.
        \section{Central Limit Theorem}
            We now wish to discuss convergence of measures and
            distributions. We restrict ourself to
            Lebesgue-Stieljes measures on the Borel
            $\sigma\textrm{-Algebra}$ of $\mathbb{R}$. We does it
            mean for a sequence of measures $\mu_{n}$ to converge
            to a measure $\mu$? For all $B\in\mathcal{B}$:
            \begin{equation}
                \mu_{n}(B)\rightarrow\mu(B)
            \end{equation}
            This is reminiscent of point-wise convergence of functions
            of a real variable, but turns out to be too much. What if
            we restrict ourselves to sets of the form $[a,b)$? Let:
            \begin{equation}
                f_{n}(x)=
                    \begin{cases}
                        0,&|x|\geq\frac{1}{n}\\
                        n(1-|x|),&|x|<\frac{1}{n}
                    \end{cases}
            \end{equation}
            And define:
            \begin{equation}
                \mu_{n}([a,b)]=\int_{a}^{b}\rho_{n}(x)\diff{x}
            \end{equation}
            Then by the Caratheodory extension theorem, there is a
            measure $\nu_{n}$ that agrees with $\mu_{n}$ on all
            such intervals. Then $\nu_{n}$ converges to the
            Dirac measure, which is an example of an atomic measure:
            \begin{equation}
                \delta(B)=
                    \begin{cases}
                        1,&0\in{B}\\
                        0,&0\notin{B}
                    \end{cases}
            \end{equation}
            However:
            \begin{equation}
                \mu_{n}([0,b)]\rightarrow\frac{1}{2}
            \end{equation}
            And:
            \begin{equation}
                \mu_{n}([a,0)]\rightarrow\frac{1}{2}
            \end{equation}
            However:
            \begin{align}
                \delta([0,b))&=1\\
                \delta([a,0))&=0
            \end{align}
            This leads us to the correct definition of measure:
            \begin{ldefinition}{Convergence of Measure}
                A sequence of measure $\nu_{n}$ converges to a
                measure $\nu$ if, for all measure sets $B$ such
                that $\nu(\partial{B})=0$, it is true that
                $\nu_{n}(B)\rightarrow\nu(B)$.
            \end{ldefinition}
            Given:
            \begin{equation}
                \int_{\mathbb{R}}\chi_{[a,b)}\diff{\nu_{n}}
                \rightarrow\int_{\mathbb{R}}\chi_{[a,b)}\diff{\nu}
            \end{equation}
            We have that $\nu(\{a\})=\nu(\{b\})=0$, and thus
            $\chi_{[a,b)}$ is continuous $\nu$ almost everywhere.
            Suppose $\nu_{n}$ and $\nu$ are probability
            Lebesgue-Stieltjes measure on $\mathbb{R}$. Then we
            get the equivalent form:
            \begin{theorem}
                If for every continuous bounded function $g(\omega)$,
                we have that:
                \begin{equation}
                    \int_{\mathbb{R}}g\diff{\mu}_{n}\rightarrow
                    \int_{\mathbb{R}}g\diff{\mu}
                \end{equation}
                Then $\mu_{n}\rightarrow\mu$.
            \end{theorem}
            \begin{theorem}
                If for every continuous function with bounded support,
                if:
                \begin{equation}
                    \int_{\mathbb{R}}g\diff{\mu}_{n}\rightarrow
                    \int_{\mathbb{R}}g\diff{\mu}
                \end{equation}
                Then $\nu_{n}\rightarrow\nu$.
            \end{theorem}
            \begin{theorem}
                If:
                \begin{equation}
                    \int_{\mathbb{R}}\exp(itu)\diff{\nu_{n}}\rightarrow
                    \int_{\mathbb{R}}\exp(itu)\diff{\nu}
                \end{equation}
                Then $\nu_{n}\rightarrow\nu$.
            \end{theorem}
            Suppose $(\Omega,\mathcal{A},\mu)$ is a probability space,
            and suppose $f_{n}:\Omega\rightarrow\mathbb{R}$ is a
            sequence of random variables that are
            $\mathcal{A}-\mathcal{B}$ measure. Consider the
            distributions $\mu_{g_{n}}$. If $g_{n}\rightarrow{g}$ in
            measure, then the distribuctions converge to $\mu_{g}$.
            \begin{theorem}
                If $(\Omega,\mathcal{A},\mu)$ is a probability space,
                if $h_{n}:\Omega\rightarrow\mathbb{R}$ is a
                sequence of random variables, if $\mu_{h_{n}}$ are the
                distributions of $g_{n}$, and if $h_{n}\rightarrow{h}$
                in measure, then $\mu_{h_{n}}\rightarrow\mu_{h}$.
            \end{theorem}
            \begin{proof}
                For let $g$ be a continuous function with compact
                support. Then, applying the measure transformation
                theorem, we have:
                \begin{equation}
                    \Big|\int_{\mathbb{R}}g\diff{\mu_{n}}-
                        \int_{\mathbb{R}}\diff{\mu_{h}}\Big|
                    =\Big|\int_{\Omega}g(h_{n})\diff{\mu}-
                        \int_{\Omega}g(h)\diff{\mu}\Big|
                    \leq\int_{\Omega}|g_{n}(h)-g(h)|\diff{\mu}
                \end{equation}
                But $g$ is continuous on a compact set, and is
                therefore uniformly continuous. Thus, for all
                $\varepsilon>0$ there is a $\delta>0$ such that, for
                all $|u'-u''|<\delta$, we have that
                $|g(u')-g(u'')|<\varepsilon$. Define the following:
                \begin{align}
                    E_{1,n,\varepsilon}
                    &=\{\omega:|h_{n}(\omega)-h(\omega)|\geq\delta\}\\
                    E_{2,n,\varepsilon}
                    &=\{\omega:|h_{n}(\omega)-h(\omega)|<\delta\}
                \end{align}
                Then:
                \begin{align}
                    \int_{\Omega}|g_{n}(h)-g(h)|\diff{\mu}
                    &=\int_{E_{1,n,\varepsilon}}|g_{n}(h)-g(h)|\diff{\mu}
                    +\int_{E_{2,n,\varepsilon}}
                        |g_{n}(h)-g(h)|\diff{\mu}\\
                    &\leq{2}M\mu(E_{1,n,\varepsilon})+
                        \varepsilon\mu(E_{2,n,\varepsilon})
                \end{align}
                And this converges to $\varepsilon$.
            \end{proof}
            The converse of this theorem is not true in general,
            since vastly different functions can have the same
            distributions. There is a special case, however, where the
            converse holds. Consider a function $h$ such that it's
            distribution is the Dirac distribution. That is:
            \begin{equation}
                \mu(\{\omega:h(\omega)=a\})=
                \mu_{h}(\{a\})=\delta_{a}(\{a\})=1
            \end{equation}
            Then $h(\omega)=a$ $\mu$ almost everywhere, or if we are
            in a probabilty space, almost surely.
            \begin{theorem}
                If $h_{n}$ is a sequence of random variables such that
                $\mu_{h_{n}}\rightarrow\delta_{a}$, where $\delta_{a}$
                is the Diract measure centered at $a$, then
                $h_{n}\rightarrow{a}$ almost surely.
            \end{theorem}
            \begin{proof}
                For:
                \begin{align}
                    \mu(\{\omega:|h_{n}(\omega)-a|\geq\delta\})
                    &=\mu_{h_{n}}
                        (\mathbb{R}\setminus(a-\delta,a+\delta)\})
                    &=1-\mu_{h_{n}}((a-\delta,a+\delta))\\
                    &\rightarrow{1}-\delta_{a}((a-\delta,a+\delta))\\
                    &=0
                \end{align}
            \end{proof}
            Thus, the weak law of large numbers can be restated by
            saying that, if:
            \begin{equation}
                \mu_{\frac{1}{n}\sum_{j=1}^{n}f_{j}}\rightarrow
                \delta_{0}
            \end{equation}
            Then $f_{j}$ obeys the weak law of large numbers.
            \subsection{Convergence of Distributions}
                A distribution is an arbitrary probability
                Lebesgue-Stieljes measure. That is, a Lebesgue-Stieljes
                measure such that the measure of the entire space is
                one. We say that a sequence of distribuctions
                $\nu_{n}$ converges to a measure $\nu$ if any of
                the following equivalent statements holds:
                \begin{enumerate}
                    \item $\nu_{n}([a,b))\rightarrow\nu([a,b))$
                          for all $a<b$.
                    \item $\nu_{n}((\minus\infty,c))\rightarrow%
                           \nu(\minus\infty,c))$ for all $c$ such
                           that $\nu(\{c\})=0$. This requirement
                           implies that $\nu$ is continuous at $c$.
                           That is, if $F_{\nu}$ is the cumulative
                           distribution function, then $F_{\nu}$ is
                           continuous at $c$.
                    \item For every bounded continuous function $h$,
                          $\int_{\mathbb{R}}h\diff\nu_{n}\rightarrow%
                           \int_{\mathbb{R}}h\diff{\nu}$.
                    \item For every continuous function with compact
                          support:
                          $\int_{\mathbb{R}}h\diff\nu_{n}\rightarrow%
                           \int_{\mathbb{R}}h\diff{\nu}$.
                    \item $\int_{\mathbb{R}}\exp(itu)\diff{\nu_{n}}%
                           =\int_{\mathbb{R}}\exp(itu)\diff{\nu}$
                \end{enumerate}
                \begin{theorem}
                    A sequence of random variables $f_{j}$ obeys
                    the weak law of large numbers if and only if:
                    \begin{equation}
                        \mu_{\frac{1}{n}\sum_{j=1}^{n}f_{j}}
                        \rightarrow\delta_{0}
                    \end{equation}
                \end{theorem}
                \begin{proof}
                    For:
                    \begin{equation}
                        \mu\Big(\big\{\omega:\Big|\frac{1}{n}
                            \sum_{j=1}^{n}\overset{\circ}{f}_{k}(\omega)
                            \Big|\geq\delta\big\}\Big)=
                        \mu_{\frac{1}{n}\sum_{k=1}^{n}
                             \overset{\circ}{f}_{j}}
                             \big((\minus\delta,\delta)^{C}\big)
                    \end{equation}
                \end{proof}
                \begin{theorem}
                    If $f_{j}$ is a sequence of random variables such
                    that the second moments are finite, then the
                    first moments are finite.
                \end{theorem}
                \begin{proof}
                    For:
                    \begin{equation}
                        \int_{\Omega}|f_{j}|\diff{\mu}\leq
                        \int_{\Omega}(1+|f_{j}|^{2})\diff{\mu}
                        =\int_{\Omega}\diff{\mu}+
                        \int_{\Omega}|f_{j}|^{2}\diff{\mu}=
                        1+\int_{\Omega}|f_{j}|^{2}\diff{\mu}
                    \end{equation}
                    Therefore, etc.
                \end{proof}
                \begin{ftheorem}{Central Limit Theorem}
                      {Measure_Theory_Central_Limit_Theorem}
                    If $f_{j}$ are independent and identically
                    distributed, with standard deviation $\sigma$,
                    then:
                    \begin{equation}
                        \mu_{\frac{1}{\sigma\sqrt{n}}
                            \sum_{j=1}^{n}\overset{\circ}{f}_{j}}
                        \rightarrow\nu_{0,1}
                    \end{equation}
                    Where $\nu_{0,1}$ is the Gaussian distribution:
                    \begin{equation}
                        \nu_{0,1}(B)=\frac{1}{\sqrt{2\pi}}
                        \int_{B}\exp(\minus{u}^{2}/2)\diff{u}
                    \end{equation}
                \end{ftheorem}
                \begin{proof}
                    We will use the Fourier transform to prove this.
                    We have:
                    \begin{equation}
                        \int_{\mathbb{R}}\exp(iut)
                            \diff{\nu_{0,1}}=
                        \int_{\mathbb{R}}\exp(itu)
                            \exp(\minus\frac{u^{2}}{2})\diff{u}
                            =\exp(\minus{t}^{2}/2)
                    \end{equation}
                    That is, the Fourier transform of a Gaussian
                    is itself. We will use this to make the
                    computation easier. Using the measure
                    transformation theorem, we have:
                    \begin{equation}
                        \int_{\mathbb{R}}\exp(iut)
                        \mu_{\frac{1}{\sigma\sqrt{n}}
                            \sum_{j=1}^{n}\overset{\circ}{f}_{j}}
                            \diff{\mu}
                        =\int_{\Omega}\exp\Big(
                            \frac{i}{\sigma\sqrt{n}}\sum_{j=1}^{n}
                            \overset{\circ}{f}_{j}(\omega)t\Big)
                            \diff{\mu}
                    \end{equation}
                    We invoke independence to get:
                    \begin{subequations}
                        \begin{align}
                            \int_{\Omega}\exp\Big(
                                \frac{i}{\sigma\sqrt{n}}\sum_{j=1}^{n}
                                \overset{\circ}{f}_{j}(\omega)t\Big)
                                \diff{\mu}
                            &=\int_{\Omega}\prod_{j=1}^{n}
                                \exp\Big(\frac{i}{\sigma\sqrt{n}}
                                \overset{\circ}{f}_{j}\Big)\diff{\mu}\\
                            &=\prod_{j=1}^{n}\int_{\Omega}
                                \exp\Big(\frac{i}{\sigma\sqrt{n}}
                                \overset{\circ}{f}_{j}\Big)\diff{\mu}
                        \end{align}
                    \end{subequations}
                    But the distributions are identically distributed,
                    and thus we have:
                    \begin{equation}
                        \int_{\Omega}\exp\Big(
                            \frac{i}{\sigma\sqrt{n}}\sum_{j=1}^{n}
                            \overset{\circ}{f}_{j}(\omega)t\Big)
                            \diff{\mu}=
                        \Big[\int_{\Omega}\exp\Big(
                            iu\frac{t}{\sqrt{n}}\Big)
                            \diff{\mu}\Big]^{n}
                    \end{equation}
                    We now need to prove that for an arbitrary
                    Lebesgue-Stieltjes Measure on the Borel
                    $\sigma\textrm{-Algebra}$ of $\mathbb{R}$,
                    such that:
                    \begin{equation}
                        \int_{\mathbb{R}}\diff{\mu}=0\quad\quad
                        \int_{\mathbb{R}}u\diff{\mu}=0\quad\quad
                        \int_{\mathbb{R}}u^{2}\diff{\mu}=1
                    \end{equation}
                    Then:
                    \begin{equation}
                        \Big[\int_{\mathbb{R}}\exp\Big(
                            iu\frac{t}{\sqrt{n}}\Big)\diff{\mu}\Big]^{n}
                        \rightarrow\exp\big(\minus{t}^{2}/2\big)
                    \end{equation}
                    Consider the function:
                    \begin{equation}
                        \varphi_{\nu}(t)=
                        \int_{\mathbb{R}}\exp(iut)\diff{\nu}
                    \end{equation}
                    In analysis this is the Fourier transform,
                    whereas in probability this is called the
                    characteristic function of $\nu$. We are
                    tasked with showing that:
                    \begin{equation}
                        \Big[\varphi_{\mu}
                            \big(\frac{t}{\sqrt{n}}\big)\Big]^{n}
                        \rightarrow\exp\big(\minus{t}^{2}/2\big)
                    \end{equation}
                    If $\mu$ is a Lebesgue-Stieltjes measure, and
                    if the second moment if finite, and if:
                    \begin{equation}
                        \varphi_{\nu}(t)=
                        \int_{\mathbb{R}}\exp(itu)\diff{\nu}
                    \end{equation}
                    then the first two derivatives of $\varphi_{\nu}$
                    exist and are continuous. Moreover:
                    \begin{equation}
                        \varphi_{\nu}(t)=
                        \varphi_{\nu}(0)+
                        \varphi_{\nu}'(0)t+
                        \varphi_{\nu}''(0)t^{2}+h(t)
                    \end{equation}
                    Where $h$ is such that:
                    \begin{equation}
                        \underset{t\rightarrow{0}}{\lim}
                        \frac{h(t)}{t^{2}}=0
                    \end{equation}
                    First, it is continuous. For let $t_{k}$ be
                    sequence such that $t_{k}\rightarrow{t}$ and let
                    $g_{k}=\exp(it_{k}u)$. Then $|g_{k}|=1$, and is
                    therefore summable. Moreover, $g_{k}$ tends to
                    $\exp(itu)$. Thus, by the dominated convergence
                    theorem:
                    \begin{equation}
                        \underset{n\rightarrow\infty}{\lim}
                        \varphi_{\nu}(t_{k})
                        =\underset{n\rightarrow\infty}{\lim}
                        \int_{\mathbb{R}}\exp(it_{k}u)\diff{\mu}
                        =\int_{\mathbb{R}}
                            \underset{n\rightarrow\infty}{\lim}
                            \exp(it_{k}u)\diff{\mu}
                        =\varphi_{\nu}(t)
                    \end{equation}
                    And thus we have continuity. For differentiability,
                    suppose $\Delta{t}_{k}$ is a sequence that
                    tends to zero, and consider:
                    \begin{equation}
                        \frac{\varphi_{\nu}(t+\Delta{t}_{k})-
                              \varphi_{\nu}(t)}{\Delta{t}_{k}}=
                        \int_{\mathbb{R}}
                        \frac{\exp(iu\Delta{t}_{k})-1}{\Delta{t}_{k}}
                        \exp(iut)\diff{\mu}
                    \end{equation}
                    Again, we want to apply the dominated convergence
                    theorem. Thus we need to find a summable
                    majorant. Consider $f(s)=(\exp(s)-1)/s$. On the
                    real axis, this function has finite limit at
                    zero and has zero limit at infinity, and therefore
                    $f(s)$ is bounded on the real axis by some $K$.
                    Thus, $K\exp(iut)$ serves as a summable majorant.
                    Applying the dominated convergence theorem shows
                    that the limit exists, and thus
                    $\varphi_{\nu}$ is differentiable. We obtain:
                    \begin{equation}
                        \varphi_{\nu}'(t)=
                        \int_{\mathbb{R}}iu\exp(iut)\diff{\nu}
                    \end{equation}
                    Moreover, this is differentiable and:
                    \begin{equation}
                        \varphi_{\nu}''(t)=
                        \minus\int_{\mathbb{R}}u^{2}\exp(iut)
                            \diff{\mu}
                    \end{equation}
                    From Taylor, we have:
                    \begin{equation}
                        h(t)=\varphi_{\nu}(t)-\varphi_{\nu}(0)
                            -\varphi_{\nu}'(0)t-\varphi_{\nu}''(0)
                                \frac{t^{2}}{2}
                    \end{equation}
                    Thus $h''(t)$ exists and is continuous,
                    $h(0)=0$, $h'(0)=0$, and $h''(0)=0$. By the
                    mean value theorem, we have:
                    \begin{equation}
                        h(t)=h'(t_{1})t
                    \end{equation}
                    For some $t_{1}\in(0,t)$. Moreover:
                    \begin{equation}
                        h(t)=h''(t_{2})t^{2}
                    \end{equation}
                    Where $0<t_{1}<t_{2}<t$. Thus:
                    \begin{equation}
                        \frac{h(t)}{t^{2}}=h''(t_{2})
                    \end{equation}
                    And from the continuity of $h''(t)$, this
                    converges to zero as $t$ tends to zero. Thus
                    we have that $\varphi_{\nu}(0)=1$,
                    $\varphi_{\nu}'(0)=0$, and
                    $\varphi_{\nu}''(0)=\minus{1}$. Now we need to 
                    finally justify the following limit:
                    \begin{equation}
                        \Big[\varphi_{\mu}\big(\frac{t}{\sqrt{n}}\big)
                            \Big]^{n}\rightarrow\exp(\minus{t}^{2}/2)
                    \end{equation}
                    We have:
                    \begin{equation}
                        \varphi_{\nu}(t)=1-\frac{t^{2}}{2}+h(t)
                    \end{equation}
                    Where $h(t)/t^{2}\rightarrow{0}$ as
                    $t\rightarrow{0}$. Thus:
                    \begin{equation}
                        \Big[\varphi_{\nu}\big(\frac{t}{\sqrt{n}}
                            \big)\Big]^{n}=
                        \Big[1-\frac{t^{2}}{2n}+h(\frac{t}{\sqrt{n}})
                            \Big]^{n}
                    \end{equation}
                    Define:
                    \begin{equation}
                        w_{n}(t)=h(t/\sqrt{n})-\frac{t^{2}}{2n}
                    \end{equation}
                    Then we have:
                    \begin{equation}
                        \Big[\varphi_{\nu}\big(\frac{t}{\sqrt{n}}
                            \big)\Big]^{n}
                        =\Big(\Big[1+w_{n}(t)\Big]^{w_{n}(t)}
                            \Big)^{\frac{n}{w_{n}(t)}}
                    \end{equation}
                    The inner part is the definition of $e$, so we
                    now need to show that $n/w_{n}(t)$ converges to
                    $\minus{t}^{2}/2$.
                \end{proof}