\chapter{Number Theory}
    \section{Exams from UML 92.413: Spring 2017}
        \subsection{Exam I}
            \begin{problem}
                Find an integer $n$ such that $\gcd(n,4)=2$ and
                $\gcd(n,6)=3$, or prove that no such integer exists.
            \end{problem}
            \begin{proof}[Solution 1]
                If $\gcd(n,4)=2$, then ${2}\vert{n}$, and thus
                $\exists_{k\in\mathbb{Z}}:n=2k$. But
                $\gcd(n,6)=\gcd(2k,2\cdot 3)=2\gcd(k,3)$. But
                $\gcd(n,6)=3$, and therefore $2\gcd(k,3)=3$, a
                contradiction as $3$ is odd. No such $n$ exists.
            \end{proof}
            \begin{proof}[Solution 2]
                If $\gcd(n,4)=2$, then ${2}\vert{n}$, and thus
                $\exists_{j\in\mathbb{Z}}:n=2j$. If $\gcd(n,6)=3$,
                then ${3}\vert{n}$. Therefore
                $\exists_{k\in\mathbb{Z}}:n=3k$. But then $2j=3k$.
                As $3$ is odd, $k$ must be even. Therefore,
                $\exists_{m\in\mathbb{Z}}:k=2m$. But then
                $n=3k=3(2m)=6m$. Thus, ${6}\vert{n}$. But then
                $\gcd(n,6)=6$, a contradiction as $\gcd(n,6)=3$.
            \end{proof}
            \begin{proof}[Solution 3]
                If $\gcd(n,4)=2$, then ${2}\vert{n}$, and thus
                $\exists_{k\in\mathbb{Z}}:n=2k$. But $\gcd(n,6)=3$,
                and therefore $\exists_{x,y\in\mathbb{Z}}:nx+6y=3$.
                But $nx+6y=2kx+6y=2(kx+3y)$, and $nx+6y=3$, and
                therefore $2(nx+3y)=3$, a contradiction as $3$ is
                odd. No such $n$ exists.
            \end{proof}
            \begin{problem}
                Prove or disprove the following:
                \begin{enumerate}
                    \begin{multicols}{2}
                        \item ${20}\vert{300}$
                        \item If $a>0$, then ${a}\vert{1}$
                        \item $\forall_{a,b>0}$, either
                            ${a}\vert{b}$ or ${b}\vert{a}$
                        \item $\forall_{a,b,c>0}$, if ${a}\vert{b}$
                            and ${a}\vert{(b+c)}$,
                            then ${a}\vert{(c-b)}$
                        \item $\forall_{a,b,c>0}$, if ${a}\vert{b}$
                            and ${a}\vert{c}$, then 
                            ${a}\vert{(b^{2}+c^{2})}$
                        \item $\forall_{a,b,c>0}$, if ${a}\vert{b}$
                            and $a\vert{(b^{2}+c^{2})}$, then
                            ${a}\vert{c}$
                        \item $\forall_{a,b,c>0}$, if ${a}\vert{b}$
                            and ${b}\vert{c}$, then $a\leq c$
                        \item If $a,b,c>0$, then
                            $\gcd(a,bc)\geq\gcd(a,b)$
                        \item If $a,b,c>0$, then
                            $\gcd(a,c-a)=\gcd(a+c,c)$
                        \item If $p$ is prime and
                            ${p^{3}}\vert{abc}$, then ${p}\vert{a}$
                        \item If $a+b$ is prime, then $ab$ is even.
                        \item If $a$ and $b$ are composite, then
                            $a+b$ is composite.
                        \item If $p$ is prime and ${p}\vert{a^{2}}$,
                            then $p^{2}\vert{a^{2}}$
                        \item If $0<b<a$, then $a^{2}-b^{2}$ is
                            composite.
                    \end{multicols}
                \end{enumerate}
            \end{problem}
            \begin{proof}[Solution]
                \
                \begin{enumerate}
                    \item True, for $300=20\cdot 15$
                    \item False, for $2>0$, but $2$ does not divide
                        $1$
                    \item False, for $5>0$ and $7>0$ but $5$ does
                        not divide $7$ and $7$ does not
                        divide $5$ for they are prime.
                    \item True. If ${a}\vert{b}$, then
                        $\exists_{n\in\mathbb{Z}}:b=na$. If
                        ${a}\vert{(b+c)}$, then
                        $\exists_{m\in\mathbb{Z}}:b+c=ma$. But we
                        have that $c=ma-b=ma-na=a(m-n)$,
                        and therefore ${a}\vert{c}$. But then
                        $b-c=a(2n-m)$, so ${a}\vert{(b-c)}$
                    \item True. If ${a}\vert{b}$ then
                        $\exists_{n\in\mathbb{Z}}:b=an$.
                        If ${a}\vert{c}$, then
                        $\exists_{m\in\mathbb{Z}}:c=am$. But then
                        $b^{2}+c^{2}=a^{2}n^{2}+a^{2}m^{2}%
                         =a(an^{2}+am^{2})$, and therefore
                        ${a}\vert{(b^{2}+c^{2})}$
                    \item False. Let $a=4$, $b=8$, and $c=6$.
                        Then $b=2a$, $b^{2}+c^{2}=25a$, but $4$
                        does not divide $6$.
                    \item True. If $a,b,c>0$ and ${a}\vert{b}$,
                        then $\exists_{n\in\mathbb{N}}:b=na$,
                        and therefore $a\leq b$. If
                        ${b}\vert{c}$, then
                        $\exists_{m\in\mathbb{N}}:c=mb$. But then
                        $b\leq c$. But $a\leq b$, and therefore
                        $a\leq c$
                    \item True. If ${n}\vert{a}$ and ${n}\vert{b}$,
                        then ${n}\vert{a}$ and ${n}\vert{bc}$, and
                        therefore $\gcd(a,b)\leq\gcd(a,bc)$
                    \item True. If ${n}\vert{a}$ and
                        ${n}\vert{(c-a)}$, then ${n}\vert{c}$. But
                        then ${n}\vert{(a+c)}$. If ${n}\vert{c}$
                        and ${n}\vert{(a+c)}$, then ${n}\vert{c}$.
                        But then ${n}\vert{(c-a)}$, and therefore
                        $\gcd(a,c-a)=\gcd(a+c,c)$
                    \item False. Let $a=6$ and $c=10$. Then
                        $\gcd(a,b)=\gcd(6,10)=2$, and
                        $\gcd(a+c,c-a)=\gcd(16,4)=4$.
                    \item False. Let $p=5$, $a=2$, $b=5$, and $c=25$.
                        Then $p$ is prime, ${p^{3}}\vert{abc}$, but
                        $5$ does not divide $2$
                    \item False. Let $a=b=1$. Then $a+b=2$, which
                        is prime, but $ab=1$, which is odd.
                    \item False. Let $a=9$, and $b=8$. Then $a$ and
                        $b$ are composite, but $a+b=17$,
                        which is prime.
                    \item True. If ${p}\vert{a^{2}}$, then
                        $\exists_{n\in\mathbb{Z}}:a^{2}=np$. But, as
                        $p$ is prime, $a$ does not divide $p$, and
                        therefore $a=\frac{n}{a}p$. That is,
                        ${p}\vert{a}$. Therefore, ${p}\vert{a^{2}}$
                    \item False. Let $a=9$ and $b=8$. Then
                        $9^{2}-8^{2}=81-64=17$, which is prime.
                \end{enumerate}
            \end{proof}
            \begin{problem}
                Use Euclid's Algorithm to compute $\gcd(201,62)$.
            \end{problem}
            \begin{proof}[Solution]
                \begin{align*}
                    201&=62\cdot 3+15\\
                    62&=15\cdot 5+2\\
                    15&=2\cdot 7+1\\
                    2&=1\cdot 2+0
                \end{align*}
                $\gcd(201,62)=1$
            \end{proof}
            \begin{problem}
                Find all integer solutions to $201x+62y=1$
            \end{problem}
            \begin{proof}[Solution 1]
                From the previous problem, we have:
                \begin{equation*}
                    3+\frac{1}{4+\frac{1}{7}}=\frac{94}{29}
                \end{equation*}
                So $201(29)+62(-94)=1$. The general solution
                is therefore $x=29+62k$ and $y=-94-201k$ for
                all $k\in\mathbb{Z}$.
            \end{proof}
            \begin{proof}[Solution 2]
                From the previous problem, we have:
                \begin{align*}
                    1&=15-2\cdot7&
                    &=201\cdot(1+28)+62\cdot(-3-7-84)\\
                    &=(201-63\cdot3)-(62-15\cdot4)\cdot7&
                    &=201\cdot29+62\cdot(-94)\\
                    &=(201-62\cdot3)-(62-(201-62\cdot3)\cdot4)\cdot7
                \end{align*}
                The general solution is $x=29+62k$ and $y=-94-201k$
            \end{proof}
            \begin{problem}
                Solve the following:
                \begin{enumerate}
                    \begin{multicols}{2}
                        \item ${300^{3}+400^{4}}\mod{6}$
                        \item ${300^{3}+400^{4}}\mod{5}$
                        \item ${3^{1}}\mod{10}$
                        \item Last digit of $333^{222}$
                        \item ${1212^{11}}\mod{13}$
                        \item If $m$ is odd and $66\equiv{4}\mod{m}$,
                            what is $m$?
                        \item ${(21)(34)+765}\mod{9}$
                        \item ${48^{237}}\mod{4}$
                        \item ${3+3^{3}+3^{5}+3^{7}+3^{9}}\mod{8}$
                        \item If $2x\equiv{5}\mod{21}$, what is
                            ${x}\mod{21}$?
                    \end{multicols}
                \end{enumerate}
            \end{problem}
            \begin{proof}[Solution]
                \par\hfill\par
                \begin{enumerate}
                    \item We have
                        ${6}\vert{300}\Rightarrow%
                         300^{3}\equiv{0}\mod{6}$.
                        Also
                        $400\equiv{4}\mod{6}\Rightarrow%
                         400^{4}\equiv{4^{4}}\mod{6}%
                         ={256}\mod{6}\equiv{4}$
                    \item
                        ${5}\vert{300}\Rightarrow{300^{3}}%
                         \equiv{0}\mod{5}$,
                        ${5}\vert{400}\Rightarrow{400^{4}}%
                         \equiv{0}\mod{5}$.
                        ${300^{3}+400^{4}}\equiv{0}\mod{5}$
                    \item
                        ${3}\cdot{7}={21}\equiv{1}\mod{10}%
                         \Rightarrow{3^{-1}}\equiv{7}\mod{10}$
                    \item
                        ${333}\equiv{3}\mod{10}\Rightarrow%
                         {333^{222}}\equiv{3^{222}}\mod{10}$. But
                        $3^{222}=9(3^{2})^{110}$, and
                        $9^{110}={81^{55}}\equiv{1}\mod{10}$.
                        So, ${333^{222}}\equiv{9}\mod{10}$
                    \item
                        ${1212}\equiv{3}\mod{13}$, and
                        $3^{11}=9\cdot((3^{3})^{3}={9}\cdot{27}^{3}$.
                        But ${27}\equiv{1}\mod{13}$. So
                        ${1212^{11}}\equiv{9}\mod{13}$
                    \item ${62}\equiv{0}\mod{m}$. But
                        $62={31}\cdot{2}$. $m=31$
                    \item ${21}\equiv{3}\mod{9}$,
                        ${34}\equiv{7}\mod{9}$, and
                        ${765}\equiv{0}\mod{9}$. So we have
                        ${3}\cdot{7}={21}\equiv{3}\mod{9}$
                    \item ${48}\equiv{0}\mod{4}$.
                    \item $3^{2}\equiv{1}\mod{8}$,
                        $3^{5}\equiv{{3}\cdot{3^{4}}}\mod{8}%
                         \equiv{3}\mod{8}$,
                        $3^{7}\equiv{{3}\cdot{3^{6}}}\mod{8}%
                         \equiv{3}\mod{8}$, and finally
                        ${3^{9}}\equiv{{3}\cdot{3^{8}}}\mod{8}%
                         \equiv{3}\mod{8}$. So we have
                        $3+3+3+3+3={15}\equiv{7}\mod{8}$
                    \item If ${2x}\equiv{5}\mod{21}$, then
                        $x\equiv{{5}\cdot{2^{-1}}}\mod{21}$.
                        But ${2^{-1}}\equiv{11}\mod{21}$, so
                        ${x}\equiv{{5}\cdot{11}}\mod{21}$ and
                        ${5}\cdot{11}={55}\equiv{13}\mod{21}$.
                \end{enumerate}
            \end{proof}
            \begin{problem}
                Find all integers $n,m\geq{0}$ such that
                $5^{n}-4^{m}=1$.
            \end{problem}
            \begin{proof}[Solution]
                $n=m=1$ is a solution since
                $5-4=1$. Suppose there is another solution.
                Note that $5^{0}-4^{0}=1-1=0$,
                $5^{1}-4^{0}=5-1=4$, and $5^{0}-4^{1}=1-4=-3$.
                If $m\geq{1}$ and $n\geq{2}$, we have
                $5^{n}-4^{m}>5^{n}-1\geq25-4=21>1$. If $m\geq{2}$,
                then $4^{m}$ is divisible by 8, and thus
                $4^{m}\mod{8}=0$. If $(n,m)$ is a solution, then
                $1=5^{n}-4^{n}\equiv{5^{n}}\mod{8}$, and thus
                $5^{n}\equiv{1}\mod{8}$. If $n$ is even, then
                $5^{2k}=25^{k}\equiv{1}\mod{8}$. If $n$ is odd, then
                $5^{2k+1}\equiv{5}\mod{8}$. Thus $n$ must be even if it
                is a solution. But if $5^{n}-4^{m}=1$,
                then $5^{n}-4^{m}\equiv{1}\mod{3}$. But
                $5^{n}-4^{m}\equiv{(-1)^{n}-(1)^{m}}\mod{3}$. But $n$ is
                even, and thus $5^{n}-4^{m}\equiv{0}\mod{8}$. But then
                $1\equiv{0}\mod{3}$, a contradiction. Thus, there is
                no other solution. $n=m=1$ is the only solution.
            \end{proof}