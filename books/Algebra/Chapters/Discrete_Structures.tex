\documentclass[crop=false,class=book,oneside]{standalone}                      %
%----------------------------------Preamble------------------------------------%
%---------------------------Packages----------------------------%
\usepackage{geometry}
\geometry{b5paper, margin=1.0in}
\usepackage[T1]{fontenc}
\usepackage{graphicx, float}            % Graphics/Images.
\usepackage{natbib}                     % For bibliographies.
\bibliographystyle{agsm}                % Bibliography style.
\usepackage[french, english]{babel}     % Language typesetting.
\usepackage[dvipsnames]{xcolor}         % Color names.
\usepackage{listings}                   % Verbatim-Like Tools.
\usepackage{mathtools, esint, mathrsfs} % amsmath and integrals.
\usepackage{amsthm, amsfonts, amssymb}  % Fonts and theorems.
\usepackage{tcolorbox}                  % Frames around theorems.
\usepackage{upgreek}                    % Non-Italic Greek.
\usepackage{fmtcount, etoolbox}         % For the \book{} command.
\usepackage[newparttoc]{titlesec}       % Formatting chapter, etc.
\usepackage{titletoc}                   % Allows \book in toc.
\usepackage[nottoc]{tocbibind}          % Bibliography in toc.
\usepackage[titles]{tocloft}            % ToC formatting.
\usepackage{pgfplots, tikz}             % Drawing/graphing tools.
\usepackage{imakeidx}                   % Used for index.
\usetikzlibrary{
    calc,                   % Calculating right angles and more.
    angles,                 % Drawing angles within triangles.
    arrows.meta,            % Latex and Stealth arrows.
    quotes,                 % Adding labels to angles.
    positioning,            % Relative positioning of nodes.
    decorations.markings,   % Adding arrows in the middle of a line.
    patterns,
    arrows
}                                       % Libraries for tikz.
\pgfplotsset{compat=1.9}                % Version of pgfplots.
\usepackage[font=scriptsize,
            labelformat=simple,
            labelsep=colon]{subcaption} % Subfigure captions.
\usepackage[font={scriptsize},
            hypcap=true,
            labelsep=colon]{caption}    % Figure captions.
\usepackage[pdftex,
            pdfauthor={Ryan Maguire},
            pdftitle={Mathematics and Physics},
            pdfsubject={Mathematics, Physics, Science},
            pdfkeywords={Mathematics, Physics, Computer Science, Biology},
            pdfproducer={LaTeX},
            pdfcreator={pdflatex}]{hyperref}
\hypersetup{
    colorlinks=true,
    linkcolor=blue,
    filecolor=magenta,
    urlcolor=Cerulean,
    citecolor=SkyBlue
}                           % Colors for hyperref.
\usepackage[toc,acronym,nogroupskip,nopostdot]{glossaries}
\usepackage{glossary-mcols}
%------------------------Theorem Styles-------------------------%
\theoremstyle{plain}
\newtheorem{theorem}{Theorem}[section]

% Define theorem style for default spacing and normal font.
\newtheoremstyle{normal}
    {\topsep}               % Amount of space above the theorem.
    {\topsep}               % Amount of space below the theorem.
    {}                      % Font used for body of theorem.
    {}                      % Measure of space to indent.
    {\bfseries}             % Font of the header of the theorem.
    {}                      % Punctuation between head and body.
    {.5em}                  % Space after theorem head.
    {}

% Italic header environment.
\newtheoremstyle{thmit}{\topsep}{\topsep}{}{}{\itshape}{}{0.5em}{}

% Define environments with italic headers.
\theoremstyle{thmit}
\newtheorem*{solution}{Solution}

% Define default environments.
\theoremstyle{normal}
\newtheorem{example}{Example}[section]
\newtheorem{definition}{Definition}[section]
\newtheorem{problem}{Problem}[section]

% Define framed environment.
\tcbuselibrary{most}
\newtcbtheorem[use counter*=theorem]{ftheorem}{Theorem}{%
    before=\par\vspace{2ex},
    boxsep=0.5\topsep,
    after=\par\vspace{2ex},
    colback=green!5,
    colframe=green!35!black,
    fonttitle=\bfseries\upshape%
}{thm}

\newtcbtheorem[auto counter, number within=section]{faxiom}{Axiom}{%
    before=\par\vspace{2ex},
    boxsep=0.5\topsep,
    after=\par\vspace{2ex},
    colback=Apricot!5,
    colframe=Apricot!35!black,
    fonttitle=\bfseries\upshape%
}{ax}

\newtcbtheorem[use counter*=definition]{fdefinition}{Definition}{%
    before=\par\vspace{2ex},
    boxsep=0.5\topsep,
    after=\par\vspace{2ex},
    colback=blue!5!white,
    colframe=blue!75!black,
    fonttitle=\bfseries\upshape%
}{def}

\newtcbtheorem[use counter*=example]{fexample}{Example}{%
    before=\par\vspace{2ex},
    boxsep=0.5\topsep,
    after=\par\vspace{2ex},
    colback=red!5!white,
    colframe=red!75!black,
    fonttitle=\bfseries\upshape%
}{ex}

\newtcbtheorem[auto counter, number within=section]{fnotation}{Notation}{%
    before=\par\vspace{2ex},
    boxsep=0.5\topsep,
    after=\par\vspace{2ex},
    colback=SeaGreen!5!white,
    colframe=SeaGreen!75!black,
    fonttitle=\bfseries\upshape%
}{not}

\newtcbtheorem[use counter*=remark]{fremark}{Remark}{%
    fonttitle=\bfseries\upshape,
    colback=Goldenrod!5!white,
    colframe=Goldenrod!75!black}{ex}

\newenvironment{bproof}{\textit{Proof.}}{\hfill$\square$}
\tcolorboxenvironment{bproof}{%
    blanker,
    breakable,
    left=3mm,
    before skip=5pt,
    after skip=10pt,
    borderline west={0.6mm}{0pt}{green!80!black}
}

\AtEndEnvironment{lexample}{$\hfill\textcolor{red}{\blacksquare}$}
\newtcbtheorem[use counter*=example]{lexample}{Example}{%
    empty,
    title={Example~\theexample},
    boxed title style={%
        empty,
        size=minimal,
        toprule=2pt,
        top=0.5\topsep,
    },
    coltitle=red,
    fonttitle=\bfseries,
    parbox=false,
    boxsep=0pt,
    before=\par\vspace{2ex},
    left=0pt,
    right=0pt,
    top=3ex,
    bottom=1ex,
    before=\par\vspace{2ex},
    after=\par\vspace{2ex},
    breakable,
    pad at break*=0mm,
    vfill before first,
    overlay unbroken={%
        \draw[red, line width=2pt]
            ([yshift=-1.2ex]title.south-|frame.west) to
            ([yshift=-1.2ex]title.south-|frame.east);
        },
    overlay first={%
        \draw[red, line width=2pt]
            ([yshift=-1.2ex]title.south-|frame.west) to
            ([yshift=-1.2ex]title.south-|frame.east);
    },
}{ex}

\AtEndEnvironment{ldefinition}{$\hfill\textcolor{Blue}{\blacksquare}$}
\newtcbtheorem[use counter*=definition]{ldefinition}{Definition}{%
    empty,
    title={Definition~\thedefinition:~{#1}},
    boxed title style={%
        empty,
        size=minimal,
        toprule=2pt,
        top=0.5\topsep,
    },
    coltitle=Blue,
    fonttitle=\bfseries,
    parbox=false,
    boxsep=0pt,
    before=\par\vspace{2ex},
    left=0pt,
    right=0pt,
    top=3ex,
    bottom=0pt,
    before=\par\vspace{2ex},
    after=\par\vspace{1ex},
    breakable,
    pad at break*=0mm,
    vfill before first,
    overlay unbroken={%
        \draw[Blue, line width=2pt]
            ([yshift=-1.2ex]title.south-|frame.west) to
            ([yshift=-1.2ex]title.south-|frame.east);
        },
    overlay first={%
        \draw[Blue, line width=2pt]
            ([yshift=-1.2ex]title.south-|frame.west) to
            ([yshift=-1.2ex]title.south-|frame.east);
    },
}{def}

\AtEndEnvironment{ltheorem}{$\hfill\textcolor{Green}{\blacksquare}$}
\newtcbtheorem[use counter*=theorem]{ltheorem}{Theorem}{%
    empty,
    title={Theorem~\thetheorem:~{#1}},
    boxed title style={%
        empty,
        size=minimal,
        toprule=2pt,
        top=0.5\topsep,
    },
    coltitle=Green,
    fonttitle=\bfseries,
    parbox=false,
    boxsep=0pt,
    before=\par\vspace{2ex},
    left=0pt,
    right=0pt,
    top=3ex,
    bottom=-1.5ex,
    breakable,
    pad at break*=0mm,
    vfill before first,
    overlay unbroken={%
        \draw[Green, line width=2pt]
            ([yshift=-1.2ex]title.south-|frame.west) to
            ([yshift=-1.2ex]title.south-|frame.east);},
    overlay first={%
        \draw[Green, line width=2pt]
            ([yshift=-1.2ex]title.south-|frame.west) to
            ([yshift=-1.2ex]title.south-|frame.east);
    }
}{thm}

%--------------------Declared Math Operators--------------------%
\DeclareMathOperator{\adjoint}{adj}         % Adjoint.
\DeclareMathOperator{\Card}{Card}           % Cardinality.
\DeclareMathOperator{\curl}{curl}           % Curl.
\DeclareMathOperator{\diam}{diam}           % Diameter.
\DeclareMathOperator{\dist}{dist}           % Distance.
\DeclareMathOperator{\Div}{div}             % Divergence.
\DeclareMathOperator{\Erf}{Erf}             % Error Function.
\DeclareMathOperator{\Erfc}{Erfc}           % Complementary Error Function.
\DeclareMathOperator{\Ext}{Ext}             % Exterior.
\DeclareMathOperator{\GCD}{GCD}             % Greatest common denominator.
\DeclareMathOperator{\grad}{grad}           % Gradient
\DeclareMathOperator{\Ima}{Im}              % Image.
\DeclareMathOperator{\Int}{Int}             % Interior.
\DeclareMathOperator{\LC}{LC}               % Leading coefficient.
\DeclareMathOperator{\LCM}{LCM}             % Least common multiple.
\DeclareMathOperator{\LM}{LM}               % Leading monomial.
\DeclareMathOperator{\LT}{LT}               % Leading term.
\DeclareMathOperator{\Mod}{mod}             % Modulus.
\DeclareMathOperator{\Mon}{Mon}             % Monomial.
\DeclareMathOperator{\multideg}{mutlideg}   % Multi-Degree (Graphs).
\DeclareMathOperator{\nul}{nul}             % Null space of operator.
\DeclareMathOperator{\Ord}{Ord}             % Ordinal of ordered set.
\DeclareMathOperator{\Prin}{Prin}           % Principal value.
\DeclareMathOperator{\proj}{proj}           % Projection.
\DeclareMathOperator{\Refl}{Refl}           % Reflection operator.
\DeclareMathOperator{\rk}{rk}               % Rank of operator.
\DeclareMathOperator{\sgn}{sgn}             % Sign of a number.
\DeclareMathOperator{\sinc}{sinc}           % Sinc function.
\DeclareMathOperator{\Span}{Span}           % Span of a set.
\DeclareMathOperator{\Spec}{Spec}           % Spectrum.
\DeclareMathOperator{\supp}{supp}           % Support
\DeclareMathOperator{\Tr}{Tr}               % Trace of matrix.
%--------------------Declared Math Symbols--------------------%
\DeclareMathSymbol{\minus}{\mathbin}{AMSa}{"39} % Unary minus sign.
%------------------------New Commands---------------------------%
\DeclarePairedDelimiter\norm{\lVert}{\rVert}
\DeclarePairedDelimiter\ceil{\lceil}{\rceil}
\DeclarePairedDelimiter\floor{\lfloor}{\rfloor}
\newcommand*\diff{\mathop{}\!\mathrm{d}}
\newcommand*\Diff[1]{\mathop{}\!\mathrm{d^#1}}
\renewcommand*{\glstextformat}[1]{\textcolor{RoyalBlue}{#1}}
\renewcommand{\glsnamefont}[1]{\textbf{#1}}
\renewcommand\labelitemii{$\circ$}
\renewcommand\thesubfigure{%
    \arabic{chapter}.\arabic{figure}.\arabic{subfigure}}
\addto\captionsenglish{\renewcommand{\figurename}{Fig.}}
\numberwithin{equation}{section}

\renewcommand{\vector}[1]{\boldsymbol{\mathrm{#1}}}

\newcommand{\uvector}[1]{\boldsymbol{\hat{\mathrm{#1}}}}
\newcommand{\topspace}[2][]{(#2,\tau_{#1})}
\newcommand{\measurespace}[2][]{(#2,\varSigma_{#1},\mu_{#1})}
\newcommand{\measurablespace}[2][]{(#2,\varSigma_{#1})}
\newcommand{\manifold}[2][]{(#2,\tau_{#1},\mathcal{A}_{#1})}
\newcommand{\tanspace}[2]{T_{#1}{#2}}
\newcommand{\cotanspace}[2]{T_{#1}^{*}{#2}}
\newcommand{\Ckspace}[3][\mathbb{R}]{C^{#2}(#3,#1)}
\newcommand{\funcspace}[2][\mathbb{R}]{\mathcal{F}(#2,#1)}
\newcommand{\smoothvecf}[1]{\mathfrak{X}(#1)}
\newcommand{\smoothonef}[1]{\mathfrak{X}^{*}(#1)}
\newcommand{\bracket}[2]{[#1,#2]}

%------------------------Book Command---------------------------%
\makeatletter
\renewcommand\@pnumwidth{1cm}
\newcounter{book}
\renewcommand\thebook{\@Roman\c@book}
\newcommand\book{%
    \if@openright
        \cleardoublepage
    \else
        \clearpage
    \fi
    \thispagestyle{plain}%
    \if@twocolumn
        \onecolumn
        \@tempswatrue
    \else
        \@tempswafalse
    \fi
    \null\vfil
    \secdef\@book\@sbook
}
\def\@book[#1]#2{%
    \refstepcounter{book}
    \addcontentsline{toc}{book}{\bookname\ \thebook:\hspace{1em}#1}
    \markboth{}{}
    {\centering
     \interlinepenalty\@M
     \normalfont
     \huge\bfseries\bookname\nobreakspace\thebook
     \par
     \vskip 20\p@
     \Huge\bfseries#2\par}%
    \@endbook}
\def\@sbook#1{%
    {\centering
     \interlinepenalty \@M
     \normalfont
     \Huge\bfseries#1\par}%
    \@endbook}
\def\@endbook{
    \vfil\newpage
        \if@twoside
            \if@openright
                \null
                \thispagestyle{empty}%
                \newpage
            \fi
        \fi
        \if@tempswa
            \twocolumn
        \fi
}
\newcommand*\l@book[2]{%
    \ifnum\c@tocdepth >-3\relax
        \addpenalty{-\@highpenalty}%
        \addvspace{2.25em\@plus\p@}%
        \setlength\@tempdima{3em}%
        \begingroup
            \parindent\z@\rightskip\@pnumwidth
            \parfillskip -\@pnumwidth
            {
                \leavevmode
                \Large\bfseries#1\hfill\hb@xt@\@pnumwidth{\hss#2}
            }
            \par
            \nobreak
            \global\@nobreaktrue
            \everypar{\global\@nobreakfalse\everypar{}}%
        \endgroup
    \fi}
\newcommand\bookname{Book}
\renewcommand{\thebook}{\texorpdfstring{\Numberstring{book}}{book}}
\providecommand*{\toclevel@book}{-2}
\makeatother
\titleformat{\part}[display]
    {\Large\bfseries}
    {\partname\nobreakspace\thepart}
    {0mm}
    {\Huge\bfseries}
\titlecontents{part}[0pt]
    {\large\bfseries}
    {\partname\ \thecontentslabel: \quad}
    {}
    {\hfill\contentspage}
\titlecontents{chapter}[0pt]
    {\bfseries}
    {\chaptername\ \thecontentslabel:\quad}
    {}
    {\hfill\contentspage}
\newglossarystyle{longpara}{%
    \setglossarystyle{long}%
    \renewenvironment{theglossary}{%
        \begin{longtable}[l]{{p{0.25\hsize}p{0.65\hsize}}}
    }{\end{longtable}}%
    \renewcommand{\glossentry}[2]{%
        \glstarget{##1}{\glossentryname{##1}}%
        &\glossentrydesc{##1}{~##2.}
        \tabularnewline%
        \tabularnewline
    }%
}
\newglossary[not-glg]{notation}{not-gls}{not-glo}{Notation}
\newcommand*{\newnotation}[4][]{%
    \newglossaryentry{#2}{type=notation, name={\textbf{#3}, },
                          text={#4}, description={#4},#1}%
}
%--------------------------LENGTHS------------------------------%
% Spacings for the Table of Contents.
\addtolength{\cftsecnumwidth}{1ex}
\addtolength{\cftsubsecindent}{1ex}
\addtolength{\cftsubsecnumwidth}{1ex}
\addtolength{\cftfignumwidth}{1ex}
\addtolength{\cfttabnumwidth}{1ex}

% Indent and paragraph spacing.
\setlength{\parindent}{0em}
\setlength{\parskip}{0em}                                                           %
%---------------------------------tikz Path------------------------------------%
\makeatletter                                                                  %
    \def\input@path{{../../../tikz/}}                                          %
\makeatother                                                                   %
%----------------------------------GLOSSARY------------------------------------%
\makeglossaries                                                                %
\loadglsentries{glossary}                                                      %
\loadglsentries{acronym}                                                       %
%--------------------------------Main Document---------------------------------%
\begin{document}
    \ifx\ifmain\undefined
        \pagenumbering{roman}
        \title{Discrete Structures I}
        \author{Ryan Maguire}
        \date{\vspace{-5ex}}
        \maketitle
        \tableofcontents
        \clearpage
        \chapter*{Discrete Structures I}
        \addcontentsline{toc}{chapter}{Abstract Algebra}
        \markboth{}{DISCRETE STRUCTURES I}
        \vspace{10ex}
        \setcounter{chapter}{1}
        \pagenumbering{arabic}
    \else
        \chapter{Discrete Structures I}
    \fi
    \subsection{Set Theory}
        A set is a collection of objects. The objects in a
        set are called the elements of the set. The emptyset
        $\emptyset$ is the set containing no elements. A set
        is finite if there is a bijection between it
        and $\mathbb{Z}_{n}=\{1,2,\hdots,n\}$ for some
        $n\in\mathbb{N}$. A set is infinite if it is
        not finite and non-empty.
        A set is countably infinite if there is a bijection
        between it and $\mathbb{N}$. A set is uncountably
        infinite if it is infinite and not
        countably infinite. Sets can be represented by
        Venn diagrams, which are essentially blobs in
        the plane. Two sets are equal if and only if they
        contain exactly the same elements. A subset $A$ of
        a set $B$, denoted $A\subset{B}$, is a set such that
        for all $x\in{A}$, $x\in{B}$ as well. That is, every
        element of $A$ is also an element of $B$. For any
        set $A$, $A\subset{A}$. A proper subset is a susbet
        $A\subset{B}$ such that $A\ne{B}$. That is, there is
        an $x\in{B}$ such that $x\notin{A}$. Disjoint sets
        are sets that have no common elements.
        \begin{theorem}
            If $A\subset{B}$ and $B\subset{C}$,
            then $A\subset{C}$.
        \end{theorem}
        \begin{theorem}
            If $A\subset{B}$ and $B\subset{A}$, then
            $A=B$.
        \end{theorem}
        \begin{theorem}
            If $A$ is a set, then $\emptyset\subset{A}$.
        \end{theorem}
        The universe set $U$ is the set of all objects under
        consideration. If $A$ is a subset of a universe $U$,
        then the complement of $A$, denoted $A^{C}$, is the
        set of all elements in $U$ that are NOT contained
        in $A$. This is also known as set difference:
        $A^{C}=U\setminus{A}$. The power set of a set $A$,
        denoted $\mathcal{P}(A)$ is the set of all subsets
        of $A$.
        \begin{definition}
            A partial ordering on a set $A$ is a
            relation $\leq$ such that if
            $A\leq{B}$ and $B\leq{C}$, then
            $A\leq{C}$, and if $A\leq{B}$ and
            $B\leq{A}$, then $A=B$.
        \end{definition}
        Set inclusion is a partial ordering on the power
        set of a set. The union of two sets $A$ and $B$
        is the set $A\cup{B}$ containing all of the elements
        of $A$ and all of the elements of $B$.
        The intersection of $A$ and $B$
        is the set $A\cap{B}$ containing only the elements
        that are in both $A$ and $B$.
        \begin{theorem}
            The following are true:
            \begin{enumerate}
                \begin{multicols}{3}
                    \item $A\cup{A}=A$
                    \item $A\cup{B}=B\cup{A}$
                    \item $A\cup\emptyset=A$
                    \item $(A\cup{B})\cup{C}=A\cup(B\cup{C})$
                    \item $A\cup{A}=A$
                    \item $A\subset{B}\Rightarrow{A=A\cup{B}}$
                    \item $A\subset{A\cup{B}}$
                    \item $A\cap{B}=B\cap{A}$
                    \item $A\cap\emptyset=\emptyset$
                    \item $(A\cap{B})\cap{C}=A\cap(B\cap{C})$
                    \item $A\cap{B}\subset{A}$
                    \item $A\subset{B}\Rightarrow{A=A\cap{B}}$
                \end{multicols}
            \end{enumerate}
        \end{theorem}
        \begin{theorem}
            If $U$ is a universe set, and $A\subset{U}$, then:
            \begin{enumerate}
                \begin{multicols}{5}
                    \item $\emptyset^{C}=U$
                    \item $U^{C}=\emptyset$
                    \item $(A^{C})^{C}=A$
                    \item $A\cup{A^{C}}=U$
                    \item $A\cap{A^{C}}=\emptyset$
                \end{multicols}
            \end{enumerate}
        \end{theorem}
        \begin{theorem}[DeMorgan's Theorem]
            If $U$ is a universe and $A,B\subset{U}$,
            then:
            \begin{enumerate}
                \begin{multicols}{2}
                    \item $(A\cup{B})^{C}=A^{C}\cap{B^{C}}$
                    \item $(A\cap{B})^{C}=A^{C}\cup{B^{C}}$
                \end{multicols}
            \end{enumerate}
        \end{theorem}
        \begin{theorem}[Distributive Laws]
            If $A$ and $B$ are sets, then:
            \begin{enumerate}
                \begin{multicols}{2}
                    \item $A\cup(B\cap{C})=(A\cup{B})\cap(A\cup{C})$
                    \item $A\cap(B\cup{C})=(A\cap{B})\cup(A\cap{C})$
                \end{multicols}
            \end{enumerate}
        \end{theorem}
        \begin{definition}
            The difference of two sets $A$ and $B$
            is: $A\setminus{B}=\{x\in{A}:x\notin{B}\}$
        \end{definition}
        \begin{theorem}
            If $U$ is a universe, and $A,B\subset{U}$,
            then $A\setminus{B}=A\cap{B^{C}}$
        \end{theorem}
        \begin{definition}
            The symmetric difference of two set
            $A$ and $B$ is
            $A\oplus{B}=(A\cup{B})\setminus(A\cap{B})$
        \end{definition}
        The symmetric difference is the set of all elements
        that are in either $A$ or in $B$, but not contained
        in both. A function or a mapping from a set $A$ to
        a set $B$ is a rule which assigns to every element
        $a\in{A}$ a unique element $b\in{B}$. An injective
        mapping, or a one-to-one mapping, is a function
        $f:A\rightarrow{B}$ such that
        $f(a_{1})=f(a_{2})$ if and only if $a_{1}=a_{2}$.
        A surjective mapping, or a correspondence, or an
        onto mapping, is a function $f:A\rightarrow{B}$
        such that for all $b\in{B}$, there is an
        $a\in{A}$ such that $f(a)=b$. A bijection is a
        function $f:A\rightarrow{B}$ that is both
        injective and surjective. Two sets are said to
        have the same size, or the same cardinality,
        if there is a bijection between them.
        \begin{theorem}
            The rational numbers $\mathbb{Q}$ are countable.
            That is, there is a bijection
            $f:\mathbb{N}\rightarrow\mathbb{Q}$.
        \end{theorem}
        \begin{theorem}
            The real numbers $\mathbb{R}$ are uncountable.
        \end{theorem}
    \subsection{Combinatorics}
        If one event can happen $n$ ways, and another independent
        event can happen $m$ ways, then the total number
        of possibilities is $nm$. If there are 5 entrees
        and 3 sides at a restaurant, then there are
        15 total possible meals.
        \begin{definition}
            The factor of a positive integer
            $n$, denoted $n!$, is
            $n!=n\cdot(n-1)\cdots{2}\cdot{1}$
            We define $0!=1$.
        \end{definition}
        The factorial of a number $n$ is the number
        of ways to permute $n$ objects. The number of ways
        to permute $k$ objects from a set of $n$ is
        $P(n,k)=n!/(n-k)!$, and the number of ways to
        choose $k$ objects from a set of $n$ objects
        (Taking the order into account) is the
        binomial coefficient $\binom{n}{k}$. The number
        of permutations of $n$ objects into groups
        $n_{1},n_{2},\hdots,n_{N}$, where
        $n_{1}+n_{2}+\hdots+n_{N}=n$, is
        $n!/(n_{1}!n_{2}!\hdots{n_{N}!})$
        Stirling's Approximation says that, for large
        $n$, the factorial can be approximated as follows:
        \begin{equation*}
            n!\approx\sqrt{2\pi{n}}n^{n}e^{-n}
        \end{equation*}
        \begin{example}
            What is the probability of rolling
            four 4's out six tosses of a six sided
            dice? The number of ways to choose
            two numbers that are not 4 is
            $\binom{6}{2}=15$. The probability of
            an event with four 4's and two numbers
            that aren't 4 is
            $(\frac{1}{6})^{4}(\frac{5}{6})^{2}$.
            So the probability of rolling four 4's is
            $15(\frac{1}{6})^{4}(\frac{5}{6})^{2}=0.008$.
        \end{example}
    \section{Exams}
    \subsection{Practice Exam I}
    \begin{problem}
    Let $S = \{1,2,3,4,5,6,7,8\}$.
    \begin{enumerate}
        \item How many subsets of $S$ are there with exactly three elements?
        \item How many subsets of $S$ contain exactly one even number and two odd numbers?
        \item How many subsets of $S$ contain exactly three elements, at most one of which is even?
        \item How many subsets of $S$ are there with exactly three elements satisfying the condition that the subset contains the numbers $1$ or $8$ (Or both)?
        \item How many subsets of $S$ are there with exactly three elements which satisfy the property that two of its elements sum to $9$?
    \end{enumerate}
    \end{problem}
    \begin{proof}[Solution]
    \
    \begin{enumerate}
    \begin{multicols}{3}
        \item $\binom{8}{3} = \frac{8!}{3!(8-3)!} = 56$
        \item $\binom{4}{1}\binom{4}{2} = 24$
        \item $\binom{4}{1} \binom{4}{2} + \binom{4}{0} \binom{4}{3} = 28$.
    \end{multicols}
        \item $\binom{7}{2}$ contain $1$, $\binom{7}{2}$ contain $8$, and $\binom{6}{1}$ contain both. $\binom{7}{2}+\binom{7}{2}-\binom{6}{1}=36$
        \item $8+1=7+2=6+3=5+4=9$. $4$ pairs with $\binom{6}{1}$ subsets per pair. We have $4\cdot 6 = 24$.
    \end{enumerate}
    \end{proof}
    \begin{problem}
    A club has $20$ members.
    \begin{enumerate}
        \item In how many different ways can the club select a president, vice-president, and secretary?
        \item In how many different ways can a social committee with four members be elected?
        \item The club contains 12 men and 8 women. In how many different ways can a committee of four people be selected if the committee must have two men and two women?
        \item In how many different ways can a four person committee be selected from the club members if one person is designated as the leader of the committee?
    \end{enumerate}
    \end{problem}
    \begin{proof}[Solution]
    \vspace{-\topsep}
    \
    \begin{enumerate}
    \begin{multicols}{2}
        \item $P(20,3) = \frac{20!}{(20-3)!} = 6840$
        \item $\binom{20}{4} = \frac{20!}{4!(20-4)!} = 4845$
        \item $\binom{12}{2}\binom{8}{2} = 1848$
        \item $\binom{19}{3} = 969$
    \end{multicols}
    \end{enumerate}
    \end{proof}
    \clearpage
    \begin{problem}
    Let $A = \{1,2\}$, $B = \{2,4,5\}$, and $U = \{1,2,3,4,5,6,7\}$. Compute the following:
    \begin{enumerate}
    \begin{multicols}{4}
        \item $A\times B$
        \item $A^3$
        \item $\mathcal{P}(B)$.
        \item $A\setminus B$
        \item $A \oplus B$
        \item $A\cap B^c$
        \item $|\{(x,y)\in U^2:x\ne y\}|$
        \item $A\cup B$.
    \end{multicols}
    \end{enumerate}
    \end{problem}
    \begin{proof}
    \vspace{-\topsep}
    \
    \begin{enumerate}
        \item $\{(1,2),(1,4),(1,5),(2,2),(2,4),(2,5)\}$
        \item $\{(1,1,1),(1,1,2),(1,2,1),(1,2,2),(2,1,1),(2,1,2),(2,2,1),(2,2,2)\}$
        \item $\big\{\emptyset,\{2\},\{4\},\{5\},\{2,4\},\{2,5\},\{4,5\},\{2,4,5\}\big\}$
    \begin{multicols}{5}
        \item $\{1\}$
        \item $\{1,4,5\}$
        \item $\{1\}$
        \item $7^2-7 = 42$.
        \item $\{1,2,4,5\}$
    \end{multicols}
    \end{enumerate}
    \end{proof}
    \begin{problem}
    Calculate the following:
    \begin{enumerate}
    \begin{multicols}{3}
        \item $\sum_{i=0}^{2}\sum_{j=1}^{3}(3i-j)$
        \item $\prod_{i=1}^{n} \frac{2i}{i+1}$
        \item $\cup_{i=1}^{n} \{i,i+1\}$
    \end{multicols}
    \end{enumerate}
    \end{problem}
    \begin{proof}[Solution]
    \vspace{-\topsep}
    \
    \begin{enumerate}
        \item $\sum_{i=0}^{2}\sum_{j=1}^{3}(3i-j) = \sum_{i=0}^{2}\big((3i-1)+(3i-2)+(3i-3)\big) = \sum_{i=0}^{2}\big(9i-6\big) = -6+3+12 = 9$.
        \item $\prod_{i=1}^{n} \frac{2i}{i+1}= \frac{2^n}{n+1}$. We prove by induction. The base case is trivial. Suppose it is true for $n\in \mathbb{N}$. Then $\prod_{i=1}^{n+1} \frac{2i}{i+1} = \frac{2(n+1)}{(n+1)+1}\prod_{i=1}^{n}\frac{2i}{i+1}=\frac{2n}{(n+1)+1}\frac{2^n}{n+1} = \frac{2^{n+1}}{(n+1)+n}$. Therefore $\prod_{i=1}^{n+1} \frac{2i}{i+1} = \frac{2^{n+1}}{(n+1)+1}$.
        \item $\cup_{i=1}^{n} A_i = \mathbb{Z}_{n+1}$. We prove by induction. The base case is trivial. Suppose it is true for $n\in \mathbb{N}$. Then $\cup_{i=1}^{n+1}A_i = \big(\cup_{i=1}^{n}A_{i}\big) \cup A_{n+1}\mathbb{Z}_{n+1}\cup\{n+1,n+2\}=\mathbb{Z}_{n+2}$. Thus $\cup_{i=1}^{n+1} A_{i} = \mathbb{Z}_{n+2}$.
    \end{enumerate}
    \end{proof}
    \begin{problem}
    \
    \begin{enumerate}
    \begin{multicols}{2}
        \item Compute the binary representation of $75$.
        \item Convert $1100101_2$ to decimal.
    \end{multicols}
    \end{enumerate}
    \end{problem}
    \begin{proof}[Solution]
    \vspace{-\topsep}
    \
    \begin{enumerate}
        \item $75=2^{6}+2^{3}+2^{1}+2^{0}=1001011_{2}$
        \item $1100101_{2}=2^{0}+2^{2}+2^{5}+2^{6}=1+4+32+64=101$
    \end{enumerate}
    \end{proof}
    \begin{problem}
    A woman has eight friends. Answer the following:
    \begin{enumerate}
        \item In how many different ways can she invite four of her friends to dinner?
        \item Two of her friends dislike each other. If she invites one friend, she can't invite the other. How many ways can she invite her friends?
        \item Five are her friends are men and three are women. How many ways can she invite four of her friends if she wants two men and two women.
        \item The eight friends consist of two married couples and four single people. How many ways can she invite four friends under the condition that if she invites one spouse she must invite the other?
    \end{enumerate}
    \end{problem}
    \begin{proof}[Solution]
    \vspace{-\topsep}
    \
    \begin{enumerate}
    \begin{multicols}{4}
        \item $\binom{8}{4} = 70$
        \item $2\binom{6}{3}+\binom{6}{4}=55$
        \item $\binom{5}{2}\binom{3}{2} = 30$
        \item $\binom{4}{0}+\binom{4}{2}+\binom{4}{2}+\binom{4}{4}=14$
    \end{multicols}
    \end{enumerate}
    \end{proof}
    \begin{problem}
    Expand $(2x-5y)^4$.
    \end{problem}
    \begin{proof}[Solution]
    \vspace{-0.5\topsep}
    $(2x-5y)^4=\sum_{k=0}^{4}\binom{n}{k}(2x)^{4-k}(-5y)^{k}=16x^4-160x^3y+600x^2y^2-1000xy^3+625y^4$
    \end{proof}
    \begin{problem}
    Find the coefficient of $x^3y^6$ in $(x-10y)^9$.
    \end{problem}
    \begin{proof}[Solution]
    \vspace{-0.5\topsep}
    $(x-10y)^9 = \sum_{k=0}^{9} \binom{9}{k}x^{n-k}(-10y)^{k}$. For $k=6$ we have $\binom{9}{6}(-10)^6 = 84,000,000$
    \end{proof}
    \clearpage
    \begin{problem}
    A bit string is a sequence of numbers consisting of $0's$ and $1's$. Answer the following:
    \begin{enumerate}
        \item How many strings of length $5$ are there?
        \item How many strings of length $6$ have an even number of $1$'s?
        \item How many strings of length 5 begin with $0$ or end with $1$ (Or both)?
        \item How many strings of length $6$ contain exactly three $1$'s?
        \item How many strings of length $6$ contain at least three $1$'s?
        \item How many strings of length $6$ are palindromic (Same from left to right as from right to left)?
    \end{enumerate}
    \end{problem}
    \begin{proof}[Solution]
    \vspace{-\topsep}
    \
    \begin{enumerate}
    \begin{multicols}{3}
        \item $2^{5}=32$.
        \item $\binom{6}{0}+\binom{6}{2}+\binom{6}{4}+\binom{6}{6}=32$
        \item $2^{3}+2^{3}+2^{3}=24$
        \item $\binom{6}{3} = 20$
        \item $\binom{6}{3}+\binom{6}{4}+\binom{6}{5}+\binom{6}{6} = 42$
        \item $2^{3}=8$
    \end{multicols}
    \end{enumerate}
    \end{proof}
    \begin{problem}
    Let $A$ and $B$ be be sets, $|A\cup B| = 50$, $|A| = 37$, and $|A\cap B| = 20$. Calculate:
    \begin{enumerate}
    \begin{multicols}{4}
        \item $|B|$
        \item $|A\setminus B|$
        \item $|B\setminus A|$
        \item $|A\oplus B|$
    \end{multicols}
    \end{enumerate}
    \end{problem}
    \begin{proof}[Solution]
    \vspace{-\topsep}
    \
    \begin{enumerate}
    \begin{multicols}{2}
        \item $|B| = |A\cup B|-|A|+|A\cap B| = 33$.
        \item $|A\setminus B| = |A|-|A\cap B| = 17$.
        \item $|B\setminus A| = |B| - |A\cap B| = 13$.
        \item $|A\oplus B| = |A\cup B|- |A\cap B| = 30$.
    \end{multicols}
    \end{enumerate}
    \end{proof}
    \subsection{Exam I}
    \begin{problem}
    Let $A = \{1,3,5\}$, $B = \{3,5,6,7\}$, $C = \{2,7\}$, $U = \{1,2,3,4,5,6,7,8\}$. Calculate:
    \begin{enumerate}
    \begin{multicols}{4}
        \item $\mathcal{P}(C)$
        \item $A^c \setminus C$
        \item $A\oplus B$
        \item $A\cup B$
    \end{multicols}
    \end{enumerate}
    \end{problem}
    \begin{proof}[Solution]
    \vspace{-\topsep}
    \
    \begin{enumerate}
    \begin{multicols}{4}
        \item $\big\{\emptyset, \{2\}, \{7\}, \{2,7\}\big\}$
        \item $\{4,6,8\}$
        \item $\{1,6,7\}$
        \item $\{1,3,5,6,7\}$
    \end{multicols}
    \end{enumerate}
    \end{proof}
    \begin{problem}
    A deck of cards contains $28$ cards. Each card is either red, yellow, green, or blue, and there are seven cards for each color labelled with integers $1$ to $7$. Answer the following:
    \begin{enumerate}
        \item How many ways can a hand of four cards be selected such that each card is a different color.
        \item How many ways can four cards be selected so that no cards cards are the same color or number.
        \item How many ways can a hand of four cards be selected so there is one red and three yellow cards. 
        \item How many ways can four cards be drawn such that all cards are the same color?
    \end{enumerate}
    \end{problem}
    \begin{proof}[Solution]
    \vspace{-\topsep}
    \
    \begin{enumerate}
    \begin{multicols}{4}
        \item $\frac{28 \cdot 21 \cdot 14 \cdot 7}{4!} = 2401$
        \item $\frac{28\cdot 18 \cdot 10 \cdot 4}{4!} = 840$
        \item $\binom{7}{1}\binom{7}{3} = 245$
        \item $4\binom{7}{4} = 140$
    \end{multicols}
    \end{enumerate}
    \end{proof}
    \begin{problem}
    A true false quiz has $6$ questions. Answer the following:
    \begin{enumerate}
        \item How many ways are there to fill out the quiz so that there are exactly four true answer?
        \item How many different ways are there to fill out the quiz so that there are at most two false answers?
    \end{enumerate}
    \end{problem}
    \begin{proof}[Solution]
    \vspace{-\topsep}
    \
    \begin{enumerate}
    \begin{multicols}{2}
        \item $\binom{6}{2} = \frac{6!}{2!(6-4)!}=15$
        \item $\binom{6}{4}+\binom{6}{5}+\binom{6}{6} = 22$
    \end{multicols}
    \end{enumerate}
    \end{proof}
    \begin{problem}
    Find the coefficient of $x^2y^3$ in $(7x-10y)^5$.
    \end{problem}
    \begin{proof}[Solution]
    \vspace{-0.5\topsep}
    $(7x-10y)^5 = \sum_{k=0}^{5}\binom{5}{k}(7x)^{5-k}(-10y)^{k}$. For $k = 3$, $\binom{5}{3}7^2(-10)^3=-490,000$.
    \end{proof}
    \clearpage
    \begin{problem}
    Let $A,B$ be finite sets such that $|A\cap B| =10$, $|A| = 22$, and $|B| = 15$. Calculate:
    \begin{enumerate}
    \begin{multicols}{4}
        \item $|A\cup B|$
        \item $|A\setminus B|$
        \item $|A\oplus B|$
        \item $|A\times B|$
    \end{multicols}
    \end{enumerate}
    \end{problem}
    \begin{proof}[Solution]
    \vspace{-\topsep}
    \
    \begin{enumerate}
    \begin{multicols}{2}
        \item $|A\cup B| = |A|+|B|-|A\cap B| = 27$
        \item $|A\setminus B| = |A|-|A\cap B| = 12$
        \item $|A\oplus B| = |A\cup B|-|A\cap B| = 17$
        \item $|A\times B| = |A||B| = 330$.
    \end{multicols}
    \end{enumerate}
    \end{proof}
    \begin{problem}
    Expand and simplify $\sum_{i=0}^{3}\sum_{j=1}^{2}(2x^i-x^j)$
    \end{problem}
    \begin{proof}[Solution]
    \vspace{-0.5\topsep}
    $\sum_{i=0}^{3}\sum_{j=1}^{2}(2x^i-x^j)=\sum_{i=0}^{3}(4x^{i}-x^{1}-x^{2})=4+4x^{1}+4x^{2}+4x^{3}-4x^{1}-4x^{2}=4+4x^3$
    \end{proof}
    \begin{problem}
    Find a formula for $|A\cup B\cup C|$, where $A,B,C$ are finite sets.
    \end{problem}
    \begin{proof}[Solution]
    \vspace{-0.5\topsep}
    $|A\cup B\cup C| = |A|+|B|+|C|-|A\cap B|-|A\cap C|-|B\cap C| +|A\cap B \cap C|$
    \end{proof}
    \subsection{Practice Exam II}
    \begin{problem}
    \label{problem:discrete_structures_contrapos_of_a+b_less_than_1}
    Write the contrapositive of: If $a<\frac{1}{2}$ and $b< \frac{1}{2}$, then $a+b<1$. 
    \end{problem}
    \begin{proof}[Solution]
    \vspace{-0.5\topsep}
    If $a+b \geq 1$, then $a\geq \frac{1}{2}$ or $b\geq \frac{1}{2}$.
    \end{proof}
    \begin{remark}
    The contrapositive is logically equivalent to the original statement.
    \end{remark}
    \begin{problem}
    Is the converse of the statement in problem \ref{problem:discrete_structures_contrapos_of_a+b_less_than_1} true?
    \end{problem}
    \begin{proof}[Solution]
    \vspace{-0.5\topsep}
    The converse is: If $a+b<1$, then $a<\frac{1}{2}$ and $b<\frac{1}{2}$. This is false: $2-2=0<1$, but $2\not<\frac{1}{2}$
    \end{proof}
    \begin{problem}
    \label{problem:discrete_structures_practice_exam_2_problem_3}
    Make a true table for $p,q,p\land q, p\lor q, (p\land q)\lor (\neg p\land q)$
    \end{problem}
    \begin{proof}[Solution]
    \vspace{-\topsep}
    \
    \begin{table}[H]
        \centering
        \captionsetup{type=table}
        \begin{tabular}{c c c c c} 
            \hline
            $p$ & $q$ & $p\land q$ & $p\lor q$ & $(p\land q)\lor(\neg p\land q)$ \\ [0.5ex] 
            \hline
            0 & 0 & 0 & 0 & 0\\ 
            0 & 1 & 0 & 1 & 1\\
            1 & 0 & 0 & 1 & 0\\
            1 & 1 & 1 & 1 & 1\\
            \hline
        \end{tabular}
        \caption{Truth Table for Problem \ref{problem:discrete_structures_practice_exam_2_problem_3}}
        \label{tab:discrete_structures_practice_exam_2_problem_3}
    \end{table}
    \end{proof}
    \begin{problem}
    Prove that $(p\land q)\lor(\neg p\land q)$ is equivalent to $q$.
    \end{problem}
    \begin{proof}[Solution]
    \vspace{-0.5\topsep}
    By the distributive law, $(p\land q)\lor(\neg p\land q) \Leftrightarrow (p\lor \neg p)\land q \Leftrightarrow 1\land q \Leftrightarrow q$.
    \end{proof}
    \begin{problem}
    Prove that if $n$ is an integer, then $n^2+6n$ is even if and only if $n$ is even.
    \end{problem}
    \begin{proof}[Solution]
    \vspace{-0.5\topsep}
    If $n$ is even, then $n=2k$ for some integer $k$. Thus $n^2+6n = 4k^2+12k = 2\big(2k^2+6k)$, so $n^2+6n$ is even. If $n^2+6n$ is even $n$ is odd, then $n=2k+1$ for some integer $k$. Thus $n^2+6n = 2\big(k^2+8k)+1$, which is odd. A contradiction. Therefore $n$ is even.
    \end{proof}
    \begin{problem}
    Prove by contradiction that if $7n^2+2n + 1$ is even, then $n$ odd.
    \end{problem}
    \begin{proof}
    \vspace{-0.5\topsep}
    If $n$ is even, then $n=2k$ for some integer $k$. Thus $7n^2+2n+1 = 28k^2+14k+1 = 2(14k^2+7k)+1$, which is odd, a contradiction. Therefore $n$ is odd.
    \end{proof}
    \begin{problem}
    Prove the following: $\sum_{k=1}^{n} k^2 = \frac{n(n+1)(2n+1)}{6}$
    \end{problem}
    \begin{proof}[Solution]
    \vspace{-0.5\topsep}
    We prove by induction. The base case is true. Suppose it is true for some $n\in\mathbb{N}$. Then $\sum_{k=1}^{n+1}k^2=(n+1)^{2}+\sum_{k=1}^{n}k^{2}= (n+1)^2+\frac{n(n+1)(2n+1)}{6}=\frac{(n+1)(n+2)(2n+3)}{6}=\frac{(n+1)((n+1)+1)(2(n+1)+1)}{6}$
    \end{proof}
    \begin{problem}
    Does $\neg p\land (p\rightarrow q)$ imply $\neg q$?
    \end{problem}
    \begin{proof}[Solution]
    \vspace{-0.5\topsep}
    No, it does not.
    \end{proof}
    \clearpage
    \begin{problem}
    Express the following in English. Determine whether it is true or false.
    \begin{enumerate}
    \begin{multicols}{4}
        \item $\exists_{x\in \mathbb{Z}}\scriptsize{\textrm{($x$ is an even prime)}}$
        \item $\forall_{n\in \mathbb{P}}(n^2>0)$
        \item $\forall_{x\in \mathbb{P}}\exists_{y\in \mathbb{Q}}(xy=1)$
        \item $\exists_{y\in \mathbb{Q}}\forall_{x\in \mathbb{P}}(xy=1)$
    \end{multicols}
    \end{enumerate}
    \end{problem}
    \begin{proof}[Solution]
    \vspace{-\topsep}
    \
    \begin{enumerate}
        \item There exists an integer $x$ such that $x$ is even and prime. This is true for $2$ is such an integer.
        \item For all positive integers $n$, $n^2>0$. This is true of all real numbers.
        \item For all positive integers $x$ there is a rational number $y$ such that $xy=1$. This is true for let $y=\frac{1}{x}$.
        \item There exists a rational number $y$ such that for all positive integers $x$, $xy=1$. This is false. For if $x_1 y = 1$ and $x_2 y = 1$, we have $(x_1-x_2)y = 0$. But $x_1 \ne x_2$, so $y=0$. A contradiction as $x_1y=1$.
    \end{enumerate}
    \end{proof}
    \begin{problem}
    Let $U = \mathcal{P}(\{1,2,3,4,5\})$, and consider the following propositions over $U$:
    \begin{align*}
        p(A):A\cap\{1,2,3\} &=\emptyset & q(A):A&\subset\{1,2,5\} & r(A):|A| &=2
    \end{align*}
    Find the truth sets of the following $p,r, q\land r, q\lor p$.
    \end{problem}
    \begin{proof}[Solution]
    \vspace{-\topsep}
    \
    \begin{enumerate}
        \item $T_{p} = \mathcal{P}(\{4,5\})$
        \item $T_{r} = \{\{1,2\},\{1,3\},\{1,4\},\{1,5\},\{2,3\},\{2,4\},\{2,5\},\{3,4\},\{3,5\},\{4,5\}\}$.
        \item $T_{q\land r} = \{1,3\},\{1,5\},\{3,5\}$.
        \item $T_{q\lor p} = \{\emptyset, \{1\},\{3\},\{5\},\{1,3\},\{1,5\},\{3,5\},\{4,\},\{4,5\},\{1,3,5\}$.
    \end{enumerate}
    \end{proof}
    \begin{problem}
    Consider the following propositions:
    \begin{align*}
        p(n)&:n^{2}=100 & q(n)&:2n-20=0 & r(n)&:n^{4}=10,000
    \end{align*}
    \begin{enumerate}
        \item Suppose $p,q,r$ are over $\mathbb{R}$. Which are equivalent? Which propositions imply another?
        \item Suppose $p,q,r$ are over $\mathbb{C}$. Which are equivalent? Which propositions imply another?
    \end{enumerate}
    \end{problem}
    \begin{proof}[Solution]
    \vspace{-\topsep}
    \
    \begin{enumerate}
        \item Over $\mathbb{R}$, we have $T_{p} = \{-10,10\}$, $T_{q} = \{10\}$, and $T_{r} = \{-10,10\}$. So $p$ and $r$ are equivalent since $T_{p} = T_{r}$. Also $q\Rightarrow p$ and $q\Rightarrow r$ since $T_{q} \subset T_{p}$ and $T_{q}\subset T_{r}$.
        \item Over $\mathbb{C}$, we have $T_{p} = \{-10,10\}$, $T_{q} = \{10\}$, $T_{r} = \{-10i,10i,-10,10\}$. So none of the propositions are equivalent, however $q\Rightarrow p \Rightarrow r$.
    \end{enumerate}
    \end{proof}
    \begin{problem}
    \label{problem:discrete structures_make_a_truth_table_for_p_and_1_and_more}
    Make a truth table for $p\land 1$, $p\rightarrow p\lor q$, $p\lor q \rightarrow q$, $p\lor 0 \leftrightarrow \neg p$. Determine which are tautologies, contradictions, or neither.
    \end{problem}
    \begin{proof}[Solution]
    $p\rightarrow p\lor q$ is a tautology, $p\lor 0 \leftrightarrow \neg p$ is a contradiction, both $p\land 1$ and $p\lor q \rightarrow q$ are neither. 
    \begin{table}[H]
        \centering
        \captionsetup{type=table}
        \begin{tabular}{c c c c c c c c c c c c c} 
             \hline 
             $p$ & $q$ & $\neg p$ & $\neg q$ & $0$ & $1$ & $p\lor q$ & $p \lor 0$ & $p\land 1$ & $p\rightarrow p\lor q$ & $p\lor q \rightarrow q$ & $p\lor 0 \leftrightarrow \neg p$.\\ [0.5ex] 
             \hline
             0 & 0 & 1 & 1 & 0 & 1 & 0 & 0 & 0 & 1 & 1 & 0 \\
             1 & 0 & 0 & 1 & 0 & 1 & 1 & 1 & 1 & 1 & 0 & 0 \\
             0 & 1 & 1 & 0 & 0 & 1 & 1 & 0 & 0 & 1 & 1 & 0 \\
             1 & 1 & 0 & 0 & 0 & 1 & 1 & 1 & 1 & 1 & 1 & 0 \\
             \hline
        \end{tabular}
        \caption{Truth Table for Problem \ref{problem:discrete structures_make_a_truth_table_for_p_and_1_and_more}}
    \end{table}
    \end{proof}
    \clearpage
    \begin{problem}
    Consider the following propositions over $\mathbb{Z}$:
    \begin{align*}
        p(n)&:n\textrm{ is a square.} & q(n)&:n\textrm{ is even.} & r(n)&:n\textrm{ is divisible by 4.} & s(n)&:n<0
    \end{align*}
    \begin{enumerate}
        \item Express ``Every integer that is an even square is divisible by 4" symbolically.
        \item Write $\neg(\exists_{n\in \mathbb{Z}}(p\land s))$ in English.
        \item Rewrite $2$ using the $\forall$ symbol.
    \end{enumerate}
    \end{problem}
    \begin{proof}[Solution]
    \vspace{-\topsep}
    \
    \begin{enumerate}
    \begin{multicols}{2}
        \item $\forall_{n\in \mathbb{Z}}(p\land q \rightarrow r)$
        \item There is no negative square integer.
        \item $\forall_{n\in \mathbb{R}}(\neg(p\land s))$
        \end{multicols}
    \end{enumerate}
    \end{proof}
    \subsection{Exam II}
    \begin{problem}
    Let $n$ be an integer. Prove the $n$ is odd if and only if $n^2+2n+5$ is even.
    \end{problem}
    \begin{proof}[Solution]
    \vspace{-0.5\topsep}
    Suppose $n$ is odd. Then there is a $k\in \mathbb{Z}$ such that $n = 2k+1$. So $n^2+2n+5 = (2k+1)^2+2(2k+1)+5 = 4k^2+4k+1+4k+2+5 = 4k^2+8k+8 = 2(2k^2+4k+4)$, which is even. Thus if $n$ is odd, then $n^2+2n+5$ is even. Let $n^2+2n+5$ be even and suppose $n$ is even. Then $n = 2k$. But then $n^2+2n+5 = 4k^2+4k+5 = 2(2k^2+2k+2)+1$. But this is an odd number, a contradiction. Therefore, $n$ is not an even number.
    \end{proof}
    \begin{problem}
    Prove that for all $n\in \mathbb{N}$, $\sum_{k=1}^{n} 3k(k+1) = n(n+1)(n+1)$.
    \end{problem}
    \begin{proof}[Solution]
    \vspace{-0.5\topsep}
    By induction. The base case is trivial. Suppose it is true for $n\in \mathbb{N}$. Then $\sum_{k=1}^{n+1}3k(k+1)=3(n+1)(n+2)+\sum_{k=1}^{n}3k(k+1)=3(n+1)(n+2)+n(n+1)(n+2)=(n+3)(n+1)(n+2)=(n+1)((n+1)+1)((n+1)+2)$.
    \end{proof}
    \begin{problem}
    \label{problem:discrete_structures_exam_2_p_and_q_iff_not_p_or_not_q}
    Consider $(p\land q) \leftrightarrow (\neg q \lor \neg p)$.
    \begin{enumerate}
        \item Write down the truth table for $(p\land q)\leftrightarrow (\neg q \lor \neg p)$ and $\neg(q\lor p)$
        \item Determine whether this proposition is a tautology, contradiction, or neither.
        \item Is $\neg(q\lor p)$ equivalent to $\neg q \lor \neg p$
    \end{enumerate}
    \end{problem}
    \begin{proof}[Solution]
    \vspace{-\topsep}
    The Proposition is a tautology. Also, $\neg q \lor \neg p$ is not equivalent to $\neg(q \lor p)$.
    \begin{table}[H]
        \centering
        \captionsetup{type=table}
        \begin{tabular}{c c c c c c c c c} 
             \hline
             $p$ & $q$ & $p\land q$ & $p \lor q$ & $\neg q$ & $\neg p$ & $\neg q\lor \neg p$ & $\neg(q\lor p)$ & $(p\land q)\leftrightarrow(\neg q \lor \neg p)$ \\ [0.5ex] 
             \hline
             0 & 0 & 0 & 0 & 1 & 1  &   1 & 1  & 0	\\ 
             0 & 1 & 0 & 1 & 0 & 1  &   1 & 0  & 0	\\
             1 & 0 & 0 & 1 & 1 & 0  &   1 & 0  & 0	\\
             1 & 1 & 1 & 1 & 0 & 0  &	0 & 0  & 0	\\
             \hline
        \end{tabular}
        \caption{Truth Table for Problem \ref{problem:discrete_structures_exam_2_p_and_q_iff_not_p_or_not_q}}
        \label{tab:discrete_structures_Exam_II_Problem_3}
    \end{table}
    \end{proof}
    \begin{problem}
    Let $n$ be an integer.
    \begin{enumerate}
        \item Write the contrapositive of : If $n^2 \geq 1$, then $n\geq 1$. Is this true?
        \item Write the converse of: If $n^2 \geq 1$, then $n\geq 1$. Is this true?
    \end{enumerate}
    \end{problem}
    \begin{proof}[Solution]
    \vspace{-\topsep}
    \
    \begin{enumerate}
        \item If $n < 1$, then $n^2 <1$. This is false. If $n = -5$, then $n<1$, yet $n^2 = 25 >1$.
        \item If $n\geq 1$, then $n^2 \geq 1$. This is true, for is $n\geq 1$, then squaring both sides we get $n^2 \geq 1^2 = 1$.
    \end{enumerate}
    \end{proof}
    \clearpage
    \begin{problem}
    Let $U$ be a set with $n$ elements. How many different propositions of $U$ can you list without listing two that are equivalent?
    \end{problem}
    \begin{proof}[Solution]
    \vspace{-0.5\topsep}
    Two propositions $p$ and $q$ are equivalent if $T_p = T_q$. There are $|\mathcal{P}(U)| = 2^n$ possible truth sets. So there are $2^n$ non-equivalent propositions over $U$.
    \end{proof}
    \begin{problem}
    Consider the following propositions of $\mathbb{N}$.
    \begin{equation*}
        p(n): 0\leq n \leq 5 \quad\quad\quad q(n): n\textrm{ is a prime greater than 7} \quad\quad\quad r(n): n^2\leq 9 \quad\quad\quad s(m,n): 2m+n = 4
    \end{equation*}
    \begin{enumerate}
        \item Translate the following statement into English: $\forall_{n\in \mathbb{Z}}\exists_{m\in \mathbb{Z}}(s(m,n))$
        \item Translate the following symbolically: There exists an integer $n$ so that $n^2>9$ or $0\leq n \leq 5$.
        \item The Down the truth sets for $p(n),q(n),r(n)$.
        \item Does one of the propositions imply another?
        \item Is the statement in part $1$ true?
    \end{enumerate}
    \end{problem}
    \begin{proof}[Solution]
    \vspace{-\topsep}
    \
    \begin{enumerate}
        \item For all integers $n$ there exists an integer $m$ so that $2m+n = 4$. 
        \item $\exists_{n\in \mathbb{Z}}(\neg r(n)\lor p(n))$
        \item   \begin{enumerate}
                    \item $T_{p} = \{0,1,2,3,4,5\}$
                    \item $T_{q} = \{2,3,5\}$
                    \item $\{,3,2,1,0,1,2,3\}$
                \end{enumerate}
        \item Yes, $q(n)\Rightarrow P(n)$ because $T_{q}\subset T_{p}$.
        \item False. If $n = 1$ then $2m+1$ is odd for all $n\in \mathbb{Z}$, yet $4$ is even.
    \end{enumerate}
    \end{proof}
    \subsection{Practice Exam III}
    \begin{problem}
    Let $A= \{1,2,3\}$, $B = \{3,4,5,6\}$, and $C = \{1,2,4,6\}$. Let the universe set be $U = \{1,2,3,4,5,6,7\}$. 
    \begin{enumerate}
        \item Find all minsets generated by $A,B,C$.
        \item Show that the nonempty minsets form a partition of $U$.
        \item How many different sets in $\mathcal{P}(U)$ can be generated by $A,B,C$ via any combination of union, intersection, and complement?
        \item Express $\{1,2,3,5,7\}$ and $\{5,6,7\}$ in minset normal form, if possible. If not, explain why.
    \end{enumerate}
    \end{problem}
    \begin{proof}[Solution]
    The non-empty sets form a partition of $U$ for their union is $U$, and there are no overaps. There are $5$ non-empty minsets, so $2^{5} = 32$ subsets of $\mathcal{P}(U)$ can be generated from the minsets. $\{1,2,3,5,7\} = (A^c\cap B^c \cap C^c)\cup (A^c\cap B\cap C^c)\cup (A\cap B^c\cap C)\cup (A\cap B\cap C^c)$. $\{5,6,7\}$ cannot be expressed as a union of minsets, for $6$ lies in the minset $\{4,6\}$, and thus $4$ would need to be included, but it is not.
    \begin{table}[H]
        \centering
        \captionsetup{type=table}
        \begin{tabular}{c c c c c}
            \hline
            $A$ & $B$ & $C$ & &  \\ [0.5ex] 
            \hline
            $0$ & $0$ & $0$ & $A^c\cap B^c\cap C^c$ & $\{7\}$\\
            $0$ & $0$ & $1$ & $A^c\cap B^c\cap C$ & $\emptyset$\\
            $0$ & $1$ & $0$ & $A^c\cap B\cap C^c$ & $\{5\}$\\
            $0$ & $1$ & $1$ & $A^c\cap B \cap C$ & $\{4,6\}$\\
            $1$ & $0$ & $0$ & $A\cap B^c\cap C^c$ & $\emptyset$\\
            $1$ & $0$ & $1$ & $A\cap B^c\cap C$ & $\{1,2\}$\\
            $1$ & $1$ & $0$ & $A\cap B\cap C^c$ & $\{3\}$\\
            $1$ & $1$ & $1$ & $A\cap B \cap C$ & $\emptyset$\\
            \hline
        \end{tabular}
        \caption{The Minsets of $A,B,$ and $C$.}
        \label{tab:discrete_structures_practice_exam_III_Problem_1}
    \end{table}
    \end{proof}
    \clearpage
    \begin{problem}
    Let $A$ and $B$ be subsets of $U$. 
    \begin{enumerate}
        \item List all minsets generated by $A, B$.
        \item Express the sets $A$ and $A^c\cup B$ in minset normal form.
    \end{enumerate}
    \end{problem}
    \begin{proof}[Solution]
    \begin{align*}
        A &= A\cap U = A\cap(B\cup B^c) = (A\cap B)\cup(A\cap B^c)\\
        A^{c}\cup B &= (A^{c}\cap U)\cup (B\cap U)=(A^{c}\cap (B\cup B^{c}))\cup(B\cap (A\cup A^{c}))=(A^c\cap B)\cup(A^c\cap B^c)\cup (B\cap A)
    \end{align*}
    \begin{table}[H]
        \centering
        \captionsetup{type=table}
        \begin{tabular}{c c c}
            \hline
            $A$ & $B$ & \\ [0.5ex]
            \hline
            $0$ & $0$ & $A^c \cap B^c$\\
            $0$ & $1$ & $A^c \cap B$\\
            $1$ & $0$ & $A\cap B^c$\\
            $1$ & $1$ & $A\cap B$\\
            \hline
        \end{tabular}
        \caption{Minsets of $A$ and $B$.}
        \label{tab:discrete_structures_exam_II_problem_blah}
    \end{table}
    \end{proof}
    \begin{problem}
    Let:
    \begin{align*}
        A&=\begin{bmatrix*}[r]1&4\\0&5\\2&-3\end{bmatrix*} & B&=\begin{bmatrix*}[r]2&10\\-1&0\\5&-1\end{bmatrix*} & C&=\begin{bmatrix*}[r]1&3\\-2&4\end{bmatrix*} & D&=\begin{bmatrix*}[r]3&0\\0&2\end{bmatrix*} & F&=\begin{bmatrix*}1&3\\-2&-6\end{bmatrix*}
    \end{align*}
    Compute the following, if possible.
    \begin{enumerate}
    \begin{multicols}{6}
        \item $AB$
        \item $3A-2B$
        \item $7A + 2C$
        \item $A^2$
        \item $C^2$
        \item $D^4$
        \item $AC+BC$
        \item $CA$
        \item $C^{-1}$
        \item $C^{-2}$
        \item $F^{-1}$.
        \item $XC=A$
    \end{multicols}
    \end{enumerate}
    \end{problem}
    \begin{proof}
    \vspace{-\topsep}
    \
    \begin{enumerate}
        \item $AB$ is not possible because $A$ and $B$ are both $3\times 2$ matrices. The number of columns of $A$ must be the same as the number of rows of $B$.
        \item $3A-2B = \begin{pmatrix}-1 & -8\\ 2 & 15\\ -4 & -7 \end{pmatrix}$
        \item $7A+2C$ is not possible because $A$ and $C$ are of different dimensions.
        \item $A^2$ is not possible because $A$ is not a square matrix.
    \begin{multicols}{2}
        \item $C^2 = \begin{pmatrix} -5 & 15 \\ -10 & 10\end{pmatrix}$
        \item $D^4 = \begin{pmatrix} 81 & 0 \\ 0 & 16\end{pmatrix}$
    \end{multicols}
        \item $AC+BC = (A+B)C = \begin{pmatrix} 3 & 14 \\ -1 & 5 \\ 7 & -4 \end{pmatrix} \begin{pmatrix} 1 & 3 \\ -2 & 4 \end{pmatrix} = \begin{pmatrix} -25 & 65 \\ -11 & 17 \\ 15 & 5 \end{pmatrix}$
        \item $C$ is a $2\times 2$ and $A$ is a $3\times 2$, so multiplication is undefined.
    \begin{multicols}{2}
        \item $\det(C) = 10$, so $C^{-1} = \frac{1}{10}\begin{pmatrix} 4 & -3 \\ 2 & 1 \end{pmatrix}$
        \item $C^{-2} = (C^2)^{-1} = \frac{1}{100}\begin{pmatrix}10 & -15 \\ 10 & -5\end{pmatrix}$
        \item $\det(F) = 0$, so $F^{-1}$ does not exist. 
        \item $X = AC^{-1} = \frac{1}{10}\begin{pmatrix}12 & 1\\10 & 5\\ 2 & -9\end{pmatrix}$
    \end{multicols}    
    \end{enumerate}
    \end{proof}
    \clearpage
    \begin{problem}
    Convert the following into $AX = B$, and then solve.
    \begin{align*}
        4x - y &= 10\\
        3x - 2y &= 3
    \end{align*}
    \end{problem}
    \begin{proof}[Solution]
    \vspace{-\topsep}
    \begin{equation*}
        \begin{bmatrix*}[r]4&-1\\3&-2\end{bmatrix*}\begin{bmatrix}x\\y\end{bmatrix}=\begin{bmatrix}10\\3\end{bmatrix}\Rightarrow X=A^{-1}B\Rightarrow X=-\tfrac{1}{5}\begin{bmatrix*}[r]-2&-3\\1&4\end{bmatrix*}\begin{bmatrix*}[r]29\\-22\end{bmatrix*}\Rightarrow X=\frac{1}{5}\begin{bmatrix}17\\18\end{bmatrix}
    \end{equation*}
    \end{proof}
    \begin{problem}
    Prove or disprove the following for $n\times n$ matrices $A, B$.
    \begin{enumerate}
        \item $AB = BA$
        \item $(A+B)(A-B) = A^2 - B^2$
        \item If $A^2 = AB$, and $\det(A) = 4$, then $A=B$.
        \item If $AB = 0$, then $A=0$ or $B=0$
    \end{enumerate}
    \end{problem}
    \begin{proof}[Solution]
    \vspace{-\topsep}
    \
    \begin{enumerate}
        \item False, for $\begin{bmatrix} 0 & 1 \\ 0 & 1\end{bmatrix} \begin{bmatrix} 1 & 1 \\ 0 & 0 \end{bmatrix} = \begin{bmatrix} 0 & 0 \\ 0 & 0\end{bmatrix}$, but $\begin{bmatrix} 1 & 1 \\ 0 & 0 \end{bmatrix}\begin{bmatrix} 0 & 1 \\ 0 & 1\end{bmatrix}  = \begin{bmatrix} 0 & 2 \\ 0 & 0 \end{bmatrix}$.
        \item False, for $(A+B)(A-B) = A(A-B) + B(A-B) = A^2-AB+BA - B^2 =d (A^2-B^2) + (BA-AB)$. Since $AB$ may not necessarily be equal to $BA$, $BA-AB$ may not zero.
        \item True. If $\det(A) = 4$, then $A^{-1}$ exists. Then $A^2 = AB$, and thus $A^{-1}A^2  = A^{-1}AB \Leftrightarrow A = B$.
        \item False. For $\begin{bmatrix} 1 & 0 \\ 0 & 0 \end{bmatrix} \begin{bmatrix} 0 & 0 \\ 0 & 1 \end{bmatrix} = 0$.
    \end{enumerate}
    \end{proof}
    \subsection{Exam III}
    \begin{problem}
    Prove that for sets $A,B,C$, $(A\cap B)\times C \subset B\times C$.
    \end{problem}
    \begin{proof}[Solution]
    \vspace{-0.5\topsep}
    For let $(x,y) \in (A\cap B)\times C$. Then $x\in A\cap B$ and $y\in C$. But if $x\in A\cap B$, then $x\in B$. But if $x\in B$ and $y\in C$, then $(x,y) \in B\times C$. Therefore, $(A\cap B)\times C \subset B\times C$. 
    \end{proof}
    \begin{problem}
    For set $A,B\subset U$, prove that $A\cup (A\cap B)^c = U$, and find the dual of this.
    \end{problem}
    \begin{proof}[Solution]
    \vspace{-0.5\topsep}
    For Let $x\in U$. If $x\in A$, then $x\in A\cup (A\cap B)^c$. Suppose not. Then $x\in A^c$. But if $x\in A^c$, then $x\notin A\cap B$. But if $x\notin A\cap B$, then $x\in (A\cap B)^c$. But then $x\in A\cup (A\cap B)^c$. Therefore $U\subset A\cup (A\cap B)^c$. But $A\cup (A\cap B)^c \subset U$, as $U$ is the universe set. Therefore $A\cup (A\cap B)^c = U$. The dual is $A\cap (A\cup B)^c = \emptyset$
    \end{proof}
    \begin{problem}
    Convert the following into the form $AX = B$. Find $A^{-1}$, and then solve for $X$:
    \begin{align*}
        4x-6y &= 5 \\
        3x-7y &= -7
    \end{align*}
    \end{problem}
    \begin{proof}[Solution]
    \vspace{-\topsep}
    \begin{equation*}
        \begin{bmatrix*}[r]4&-6\\3&-7\end{bmatrix*}\begin{bmatrix*}x\\y\end{bmatrix*}=\begin{bmatrix*}[r]5\\-7\end{bmatrix*}\Rightarrow X=A^{-1}B= -\tfrac{1}{10}\begin{bmatrix}-7&6\\-3&4\end{bmatrix}\begin{bmatrix*}[r]5\\-7\end{bmatrix*}=\tfrac{1}{10}\begin{bmatrix}83\\47 \end{bmatrix}
    \end{equation*}
    \end{proof}
    \clearpage
    \begin{problem}
    Let $A = \begin{pmatrix} 1 & 0 & -2 \\ 5 & 3 & 0 \end{pmatrix}, B = \begin{pmatrix} 0 & 2 \\ 4 & -10 \\ 8 & -6 \end{pmatrix}, C = \begin{pmatrix} 7 & -1 \\ -2 & 5 \\ -4 & 3 \end{pmatrix}, E = \begin{pmatrix} 1 & -3 \\ 2 & 5 \end{pmatrix}$. Compute the following, if possible:
    \begin{enumerate}
    \begin{multicols}{6}
        \item $3A+5C$
        \item $2B-3C$
        \item $CE$
        \item $EC$
        \item $E^2$
        \item $BA+2CA$
    \end{multicols}
    \end{enumerate}
    \end{problem}
    \begin{proof}[Solution]
    \vspace{-\topsep}
    \
    \begin{enumerate}
        \item $A$ and $C$ do not have the same dimensions, so this can't be done.
    \begin{multicols}{2}
        \item $2B - 3C = \begin{pmatrix} -21 & 7 \\ 14 & -35 \\ 28 & -21 \end{pmatrix}$
        \item $CE = \begin{pmatrix} 5 & -26 \\ 8 & 31 \\ 2 & 27 \end{pmatrix}$
    \end{multicols}
        \item $EC$ can't be done as $E$ is a $2\times 2$ and $C$ is a $3\times 2$.
    \begin{multicols}{2}
        \item $E^2 = \begin{pmatrix} -5 & -18 \\ 12 & 19 \end{pmatrix}$
        \item $BA + 2CA = \begin{pmatrix} 13 & 0 & -28 \\ 0 & 0 & 0 \\ 0 & 0 & 0 \end{pmatrix}$
    \end{multicols}
    \end{enumerate}
    \end{proof}
    \begin{problem}
    Let $A = \{1,3,5\}$, $B = \{2,3,4,5\}$, and $U = \{1,2,3,4,5,6\}$.
    \begin{enumerate}
        \item Find all minsets generated by $A$ and $B$.
        \item How many different sets in the power set of $U$ can be generated by $A,B$ by any combination of union, intersection, and complement?
        \item Express $\{2,4,6\}$ in minset normal form.
        \item Find the maxsets generated by $A$ and $B$.
        \item Express $\{1,3,5\}$ in maxset normal form.
    \end{enumerate}
    \end{problem}
    \begin{proof}[Solution]
    \vspace{-\topsep}
    \
    \begin{enumerate}
        \item $A^c\cap B^c = \{6\}$, $A\cap B^c = \{1\}$, $A^c \cap B = \{2,4\}$, $A\cap B = \{3,5\}$.
        \item There are $4$ non-empty minsets, so $2^4 = 16$. 
        \item $\{2,4,6\} = (A^c \cap B)\cup (A^c \cap B^c)$.
        \item $A\cup B = \{1,2,3,4,5\}, A^c \cup B = \{2,3,4,5,6\}, A\cup B^c = \{1,3,5,6\}, A^c \cup B^c = \{1,2,4,6\}$.
        \item $\{1,3,5\} = (A\cup B) \cap (A\cup B^c)$.
    \end{enumerate}
    \end{proof}
    \subsection{Final Exam}
    \begin{problem}
    Let $A = \{1,2\}$, $B = \{2,3,4,6\}$, $C = \{4,6,7\}$, $U = \{1,2,3,4,5,6,7,8\}$. Compute the following:
    \begin{enumerate}
    \begin{multicols}{4}
        \item $A^c \cap B$
        \item $A\cup C$
        \item $B\oplus C$
        \item $B\setminus C$
        \item $A\times C$
        \item $A^3$
        \item $\mathcal{P}(C)$
    \end{multicols}
    \end{enumerate}
    \end{problem}
    \begin{proof}[Solution]
    \vspace{-\topsep}
    \
    \begin{enumerate}
    \begin{multicols}{2}
        \item $A^c \cap B = \{3,4,6\}$.
        \item $A \cup C = \{1,2,4,6,7\}$
        \item $B\oplus C = \{2,3,7\}$
        \item $B\setminus C = \{2,3\}$
    \end{multicols}
        \item $A\times C = \{(1,4),(1,6),(1,7),(2,4),(2,6),(2,7)\}$
        \item $A^3 = \{(1,1,1),(1,1,2),(1,2,1),(1,2,2),(2,1,1),(2,1,2),(2,2,1),(2,2,2)\}$.
        \item $\mathcal{P}(C) = \{\emptyset, \{4\},\{6\},\{7\},\{4,6\},\{4,7\},\{6,7\},\{4,6,7\}\}$.
    \end{enumerate}
    \end{proof}
    \begin{problem}
    Prove the following is false: If $A\cap B = A \cap C$, then $B = C$.
    \end{problem}
    \begin{proof}[Solution]
    \vspace{-0.5\topsep}
    For let $A = \{1\}$, $B = \{1,2\}$, and $C = \mathbb{R}$. Then $A\cap B = \{1\}$, $A\cap C = \{1\}$, but $B \ne C$.
    \end{proof}
    \begin{problem}
    Ten students are competing for a scholarship.
    \begin{enumerate}
        \item If there are three scholarships worth $\$2000$, how many ways can they be distributed?
        \item If there are two scholarships worth $\$5000$ and three worth $\$2000$, how many ways can they be distributed?
        \item Suppose that the group of ten students consists of six freshmen and four sophomores. In how many different ways can four equal scholarships be distributed if at least two of the scholarships should be awarded to freshmen?
        \item Suppose the group of ten students consists of six freshmen and four sophomores. In how many different ways can two scholarships of $\$5000$ and two scholarships of $\$2000$ be distributed if at least three of the scholarships will be awarded to freshmen?
    \end{enumerate}
    \end{problem}
    \begin{proof}[Solution]
    \vspace{-\topsep}
    \
    \begin{enumerate}
    \begin{multicols}{2}
        \item $\binom{10}{3} = \frac{10!}{3!(10-3)!} = 120$
        \item $\binom{10}{2}\binom{8}{3} = 2520$
    \end{multicols}
        \item If $2$ scholarships are awarded to freshmen, we have $\binom{6}{2}\binom{4}{2} = 90$. If $3$ scholarships are awards to freshmen, we have $\binom{6}{3}\binom{4}{1} = 80$. If $4$ scholarships are awarded to freshmen, we have $\binom{6}{4}\binom{4}{0} = 15$. Adding them together, we get $185$.
        \item If $2$ $\$5000$ scholarships are awarded to freshmen, and $1$ $\$2000$ scholarship is awarded to a freshman, then there are $\binom{6}{3}\binom{4}{1}=80$. Simiarly if $2$ $\$2000$ scholarships are awarded to freshmen and $1$ $\$5000$ scholarship is awarded to a freshamn. Finally, there are $\binom{6}{4}=15$ ways to give all scholarships to freshmen. In total, there are $175$ total possible outcomes.
    \end{enumerate}
    \end{proof}
    \section{Quizzes}
    \subsection{Quiz I}
    \begin{problem}
    Let $A = \{1,2,3,7,8\}$, $B = \{1,3,5\}$, $C = \{2,4,8\}$, and let $U = \{1,2,3,4,5,6,7,8,9,10\}$. Evaluate the following:
    \begin{enumerate}
    \begin{multicols}{6}
        \item $A\cup C$
        \item $A\cap B$
        \item $A \oplus C$
        \item $A^c$
        \item $A\setminus B$
        \item $B\times C$
    \end{multicols}
    \end{enumerate}
    \end{problem}
    \begin{proof}[Solution]
    \vspace{-\topsep}
    \
    \begin{enumerate}
    \begin{multicols}{5}
        \item $\{1,2,3,4,7,8\}$
        \item $\{1,3\}$
        \item $\{1,3,4,7\}$
        \item $\{4,5,6,9,10\}$
        \item $A\setminus B = \{2,7,8\}$
    \end{multicols}
        \item $B \times C = \{(1,2),(1,4),(1,8),(3,2),(2,4),(3,8),(5,2),(5,4),(5,8)\}$
    \end{enumerate}
    \end{proof}
    \begin{problem}
    Let $S = \{1,3,5\}$. Find $\mathcal{P}(S)$.
    \end{problem}
    \begin{proof}[Solution]
    \vspace{-0.5\topsep}
    $\mathcal{P}(S) = \{\emptyset,\{1\},\{3\},\{5\},\{1,3\},\{1,5\},\{3,5\},\{1,3,5\}\}$
    \end{proof}
    \begin{problem}
    Let $S = \{7k-3: k \in \mathbb{N}, k < 5\}$. List all of the elements of $S$.
    \end{problem}
    \begin{proof}[Solution]
    \vspace{-0.5\topsep}
    $S = \{-3,4,11,18,25\}$
    \end{proof}
    \begin{problem}
    Express $49$ in binary.
    \end{problem}
    \begin{proof}[Solution]
    \vspace{-\topsep}
    \
    \begin{align*}
        49 &= 2\cdot24+1 & 6&=2\cdot3+0\\
        24&= 2\cdot12+0 & 3&=2\cdot1+1\\
        12&= 2\cdot6\phantom{2}+0 & 1&=2\cdot0+1
    \end{align*}
    So, $49 = 110001_{2}$
    \end{proof} 
    \subsection{Quiz II}
    \begin{problem}
    Calculate $\sum_{k=-1}^{3} (2^k+1)$.
    \end{problem}
    \begin{proof}[Solution]
    \vspace{-0.5\topsep}
    $\sum_{k=-1}^{3}(2^k+1) = (2^{-1}+1) + (2^0+1) + (2^1+1)+(2^2+1) + (2^3+1) = 20 +\frac{1}{2} = \frac{41}{2}$.
    \end{proof}
    \begin{problem}
    Three men and three women are to be seated in a row.
    \begin{enumerate}
        \item How many different ways can the six people be seated?
        \item How many different ways can the six people be seated if it is required that the genders alternate.
    \end{enumerate}
    \end{problem}
    \begin{proof}[Solution]
    \vspace{-\topsep}
    \
    \begin{enumerate}
        \item $6! = 6\cdot 5 \cdot 5 \cdot 4 \cdot 3 \cdot 2 \cdot 1 = 720$
        \item It is $MWMWMW$, so $3\cdot 3 \cdot 2 \cdot 2 \cdot 1 \cdot 1 = 72$. Or, there are $6$ ways to seat the first person, $3$ ways to seat the second person, $2$ ways to seat the third person, $2$ ways to seat the fourth person, and $1$ way to seat the last two. So, $6\cdot 3 \cdot 2 \cdot 2 \cdot 1 \cdot 1 = 72$.
    \end{enumerate}
    \end{proof}
    \begin{problem}
    Calculate $P(7;3)$
    \end{problem}
    \begin{proof}[Solution]
    \vspace{-0.5\topsep}
    $P(7;3) = \frac{7!}{(7-3)!} = 7\cdot 6 \cdot 5 = 210$.
    \end{proof}
    \begin{problem}
    Let $A$ be a set such that $|A| = n$.
    \begin{enumerate}
    \begin{multicols}{3}
        \item Calculate $|A^4|$
        \item Calculate $|\{\{a,b,c,d\}\subset A:\textrm{Each Term is Different}\}|$
    \end{multicols}
    \end{enumerate}
    \end{problem}
    \begin{proof}[Solution]
    \vspace{-\topsep}
    \
    \begin{enumerate}
    \begin{multicols}{2}
        \item $|A^4| = |A\times A \times A \times A| = n^4$
        \item $n \cdot (n-1)\cdot (n-2)\cdot (n-3) = P(n;4)$
    \end{multicols}
    \end{enumerate}
    \end{proof}
    \subsection{Quiz III}
    \begin{problem}
    Let $p,q,r$ be the following propositions:
    \begin{align*}
        p(x)&:x=1 & p(x)&:x=-1 & r(x)&:x^{2}=1
    \end{align*}
    \begin{enumerate}
        \item Express ``If $x^2 = 1$, then $x=1$ and $x=-1$," in symbolic form.
        \item Write the converse of this in English, and symbolically.
        \item Express $\neg p \land \neg r$ in English.
        \item Express $r\leftrightarrow (q\lor p)$ in English.
    \end{enumerate}
    \end{problem}
    \begin{proof}[Solution]
    \vspace{-\topsep}
    \
    \begin{enumerate}
        \item $r\rightarrow (p\land q)$
        \item $(p\land q) \rightarrow r$. If $x=1$ and $x=-1$, then $x^2 = 1$.
        \item $x\ne = 1$ and $x^2 \ne 1$>
        \item $x^2 = 1$ if and only if $x=1$ or $x=-1$.
    \end{enumerate}
    \end{proof}
    \begin{problem}
    \label{discrete_structures_quiz_3_problem_2}
    Make a truth table for $(p\lor \neg q)\land r$.
    \end{problem}
    \begin{proof}[Solution]
    \vspace{-\topsep}
    \
    \begin{table}[H]
        \centering
        \captionsetup{type=table}
        \begin{tabular}{c c c c c c}
            \hline
            $p$ & $q$ & $r$ & $\neg q$ & $p\lor \neg q$ & $(p\lor \neg q)\land r$ \\ [0.5ex]
            \hline
            $0$ & $0$ & $0$ & $1$ & $1$ & $0$\\
            $0$ & $0$ & $1$ & $1$ & $1$ & $1$\\
            $0$ & $1$ & $0$ & $0$ & $0$ & $0$\\
            $0$ & $1$ & $1$ & $0$ & $0$ & $0$\\
            $1$ & $0$ & $0$ & $1$ & $1$ & $0$\\
            $1$ & $0$ & $1$ & $1$ & $1$ & $1$\\
            $1$ & $1$ & $0$ & $0$ & $1$ & $0$\\
            $1$ & $1$ & $1$ & $0$ & $1$ & $1$\\
            \hline
        \end{tabular}
        \caption{Truth Table for Problem \ref{discrete_structures_quiz_3_problem_2}}
        \label{tab:discrete_structures_final_exam_problem}
    \end{table}
    \end{proof}
    \subsection{Quiz IV}
    \begin{problem}
    Prove directly that $a\rightarrow b, \neg c\rightarrow \neg b, \neg c \Rightarrow \neg a$.
    \end{problem}
    \begin{proof}[Solution]
    \vspace{-0.5\topsep}
    For if $a\rightarrow b$, then $\neg b \rightarrow \neg a$. But $\neg c \rightarrow \neg b$. But if $\neg c \rightarrow \neg b$ and $\neg b \rightarrow \neg a$, then $\neg c \rightarrow \neg a$. Thus $a\rightarrow b, \neg c \rightarrow \neg b, \neg c \Rightarrow \neg a$.
    \end{proof}
    \begin{problem}
    Prove indirectly that $a\rightarrow b, \neg c \rightarrow \neg b, \neg c \Rightarrow \neg a$.
    \end{problem}
    \begin{proof}[Solution]
    \vspace{-0.5\topsep}
    For if $\neg c \rightarrow \neg b$, then $b\rightarrow c$. But if $a\rightarrow b$ and $b\rightarrow c$, then $a\rightarrow c$. Therefore $a\rightarrow c$. But if $a\rightarrow c$, then $\neg c \rightarrow \neg a$. Therefore, $a\rightarrow b, \neg c \rightarrow \neg b, \neg c \Rightarrow \neg a$.
    \end{proof}
    \begin{problem}
    Let $U = \{1,2,3,4,5,6,7,8,9,10\}$. Consider the following propositions over $U$:
    \begin{align*}
        p(n)&:n\textrm{ is prime} & q(n)&:n\textrm{ is odd.} & r(n)&:n\leq 7
    \end{align*}
    \begin{enumerate}
        \item Find the truth sets for $p,q,r$.
        \item Which of these propositions implies one of the others?
        \item Find the truth set of $q\land r$.
    \end{enumerate}
    \end{problem}
    \begin{proof}[Solution]
    \vspace{-\topsep}
    \
    \begin{enumerate}
        \item $T_{p} = \{2,3,5,7\}$, $T_{q} = \{1,3,5,7,9\}$, $T_{r} = \{1,2,3,4,5,6,7\}$
    \begin{multicols}{2}
        \item $p\Rightarrow r$, since $T_{p}\subset T_{r}$.
        \item $T_{q\land r} = \{1,3,5,7\}$.
    \end{multicols}
    \end{enumerate}
    \end{proof}
    \subsection{Quiz V}
    \begin{problem}
    Let $p,q,r$ be the following propositions over $\mathbb{N}$:
    \begin{align*}
        p(n)&:n^{2}+3n=1 & q(n)&:n\textrm{ is prime.} & r(n)&:n\textrm{ is even.} & s(m,n)&:m|n
    \end{align*}
    \begin{enumerate}
        \item Express ``There exists a solution for $n^2+3n = 10$ that is prime," symbolically.
        \item Express $\forall_{n\in \mathbb{Z}}(q\rightarrow \neg r)$ in English.
        \item Express $\forall_{n\in \mathbb{N}}\exists_{m\in \mathbb{N}}(s(m,n))$ in English.
    \end{enumerate}
    \end{problem}
    \begin{proof}[Solution]
    \vspace{-\topsep}
    \
    \begin{enumerate}
        \item $\exists_{n\in T_{q\land p}}$.
        \item For every integer $n$, if $n$ is prime, then $n$ is not an even number.
        \item For every positive integer $n$, there exists a positive integer $m$ such that $m$ divides $n$. 
    \end{enumerate}
    \end{proof}
    \begin{problem}
    Prove $\sum_{k=1}^{n} 10k = 5n(n+1)$ using mathematical induction. 
    \end{problem}
    \begin{proof}[Solution]
    \vspace{-0.5\topsep}
    The base case is $n=1$, so $10 = 5\cdot 1(1+1) = 5\cdot 2 = 10$, which is true. Suppose this is true for some $n\in \mathbb{N}$. Then $\sum_{k=1}^{n+1} 10k = 10(n+1) + \sum_{k=1}^{n} 10k$. By hypothesis, $\sum_{k=1}^{n} 10k = 5n(n+1)$, so $\sum_{k=1}^{n+1}10k = 10(n+1)+5n(n+1) = (n+1)(10+5n) = 5(n+1)(n+2) = 4(n+1)((n+1)+1)$. This proves the induction step.
    \end{proof}
\end{document}