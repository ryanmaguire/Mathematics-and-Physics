\section{What's the Point?}
    From the historical perspective, we can start with numbers. There's
    the standard inclusion:
    \begin{equation}
        \mathbb{N}^{+}\subseteq\mathbb{N}\subseteq
        \mathbb{Z}\subseteq\mathbb{Q}\subseteq\mathbb{R}
        \subseteq\mathbb{C}
    \end{equation}
    Suppose we are given the equation $2+x=1$. If we only know about
    the positive integers, then we cannot solve this equation. We thus
    need to introduce negative integers. Next we could write $2x=1$,
    and we are now forced to introduce the rational numbers. In ancient
    Greece, the solution to $x^{2}=2$ was proved to be irrational, and
    thus we must go beyond $\mathbb{Q}$ and develop the real numbers
    (or at the very least, the algebraic numbers $\mathbb{A}$). Pushing
    beyong this, polynomial equations such as $x^{2}+1=0$ were studied
    in Italy during the Renaissance era. It was known that there are no
    real solutions to this, as one can see from the graph of $x^{2}+1$
    (it never crosses the $x$ axis). To solve such equations one must
    invent $\mathbb{C}$. The complex numbers are the set $\mathbb{C}$ of
    the form:
    \begin{equation}
        \mathbb{C}=\{\,x+iy\;|\;x,y\in\mathbb{R}\,\}
    \end{equation}
    where $i^{2}=\minus{1}$, by definition, which is the solution to the
    equation $z^{2}+1=0$. This equation has no solutions in $\mathbb{R}$
    and so $i$ is not a real number, and hence is called the
    \textit{imaginary} unit. We can picture complex numbers by use of
    the plane $\mathbb{R}^{2}$. But there's nothing too special about
    the equation $z^{2}+1=0$, and we can consider $z^{2}+z+1=0$ and
    again we can ask if this has real solutions. Unlike the first
    equation, it's not so obvious that this has no real solution. We
    can look at the quadratic formula, and in particular the discriment,
    obtaining:
    \begin{equation}
        \Delta=b^{2}-4ac=1-4=\minus{2}
    \end{equation}
    Since this is negative, there are no real solutions, and hence
    $z^{2}+z+1$ has no solution in $\mathbb{R}$. It does have roots in
    $\mathbb{C}$:
    \twocolumneq{\omega=\minus\frac{1}{2}+\frac{\sqrt{3}}{2}i}
                {\overline{\omega}=\minus\frac{1}{2}-\frac{\sqrt{3}}{2}i}
    We can further consider the set $\mathbb{R}[w]$ defined by:
    \begin{equation}
        \mathbb{R}[\omega]=\{\,x+y\omega\;|\;x,y\in\mathbb{R}\,\}
    \end{equation}
    This has a nice field structure, like $\mathbb{C}$, and indeed this
    is equal to $\mathbb{C}$. That is, $\mathbb{R}[\omega]=\mathbb{C}$.
    We can see this since $\mathbb{R}[\omega]$ is a subspace of
    $\mathbb{C}$ with a basis consisting of two elements:
    $\{1,\omega\}$, and thus has the same dimension as $\mathbb{C}$.
    Hence, it is equal to the whole thing. We can be even more explicit:
    \begin{equation}
        x+y\omega
            =x+y\big(\minus\frac{1}{2}+\frac{\sqrt{3}}{2}i\big)
            =\big(x-\frac{1}{2}y\big)+\big(\frac{\sqrt{3}}{2}y\big)i
    \end{equation}
    And this is of the form $x'+y'i$, where:
    \twocolumneq{x'=x-\frac{1}{2}y}{y'=\frac{\sqrt{3}}{2}y}
    Since this is always solvable for both $(x,y)$ and $(x',y')$, the
    two spaces are the same. And indeed, we can generalize. If
    $f(x)=ax^{2}+bx+c$, with $a,b,c\in\mathbb{R}$ such that
    $b^{2}-4ac<0$, then defining:
    \begin{equation}
        \alpha=\frac{\minus{b}+\sqrt{b^{2}-4ac}}{2a}
    \end{equation}
    which is a complex root of $f$, then
    $\mathbb{R}[\alpha]=\mathbb{C}$. This shows there's nothing too
    special about $i$: extending $\mathbb{R}$ with any complex root of
    a quadratic gives the entirety of $\mathbb{C}$, we need not only
    choose $z^{2}+1=0$. Even if we were to stick with this polynomial,
    we could still choose $\minus{i}$, since this too is a solution.
    Choosing $i$ over $\minus{i}$ seems to purely be an accident of
    history. Going from one choice to another is an
    $\mathbb{C}$ automorphism: $x+iy\mapsto{x}-iy$. An $\mathbb{R}$
    automorphism is a bijective ring homomoprhism
    $f:\mathbb{C}\rightarrow\mathbb{C}$. That is, an isomorphism from
    $\mathbb{C}$ to itself:
    \twocolumneq[\par]{f(z_{1}+z_{2})=f(z_{1})+f(z_{2})}
                      {f(z_{1}z_{2})=f(z_{1})f(z_{2})}
    And also requiring:
    \begin{equation}
            f(1)=1
    \end{equation}
    The automorphism $x+iy\mapsto{x}-iy$ is called complex conjugation.
    If we don't like $i$, and have a complex number such as $\omega$,
    we can still take as an $\mathbb{R}$ automorphism the function
    $\sigma:\mathbb{C}\rightarrow\mathbb{C}$ where
    $x+y\omega\mapsto{x}+y\overline{\omega}$. As it turns out, this is
    the same as the automorphism $x+iy\mapsto{x}-iy$ since we can write:
    \begin{equation}
        i=\frac{1+2\omega}{\sqrt{3}}
    \end{equation}
    This is the object we wish to stress as the important part of the
    theory of complex numbers. Neither $i$ nor $\omega$ are too
    important, but rather the notion of complex conjugation is.
    The group of $\mathbb{R}$ automorphisms of $\mathbb{C}$ is equal to:
    \begin{equation}
        \textrm{Aut}_{\mathbb{R}}(\mathbb{C})
            =\{\identity{\mathbb{R}},\sigma\}
    \end{equation}
    Where $\sigma$ is complex conjugation. That is, $\sigma$ is the
    unique non-trivial $\mathbb{C}$ automorphism that has the property
    that it exchanges the roots of any $f(x)=ax^{2}+bx+c$ with
    $b^{2}-4ac<0$. The group structure comes from function composition.
    Since function composition is associative, the identity map is an
    automorphism, and since bijections have inverse functions,
    this is indeed a group. We can summarize all of this as follows:
    The roots of any real polynomial are either real or come in complex
    conjugate pairs.
    \par\hfill\par
    Looking at the numerology of the problem, there seems to be
    something special about the number two. This is the size of the
    automorphism group $\textrm{Aut}_{\mathbb{R}}(\mathbb{C})$, and
    this is also the dimension of $\mathbb{C}$, and lastly it is the
    degree of $\mathbb{C}$ over $\mathbb{R}$: $[\mathbb{C}:\mathbb{R}]$.
    More generally, consider any field $\mathbb{F}$ with characteristic
    not equal to 2 (that is, $1+1\ne{0}$), and any function
    $f(x)=ax^{2}+bx+c$, $a,b,c\in\mathbb{F}$ such that $f(x)=0$ has no
    solutions in $\mathbb{F}$. For example, $\mathbb{R}$ with
    $f(x)=x^{2}+1$, of $\mathbb{Q}$ with $f(x)=x^{2}-2$. If we have
    such conditions, then there is a field $\mathbb{K}$ and an inclusion
    $\mathbb{F}\subseteq\mathbb{K}$ making $\mathbb{F}$ a subfield,
    such that $f(x)=a(x-\alpha)(x-\beta)$, where
    $\alpha,\beta\in\mathbb{K}$. Moreover,
    $\mathbb{K}=\mathbb{F}[\alpha]$. That is:
    \begin{equation}
        \mathbb{K}=\{\,x+y\alpha\;|\;x,y\in\mathbb{F}\}
    \end{equation}
    Similarly, $\mathbb{K}=\mathbb{F}[\beta]$. Lastly, the automorphism
    group is
    \begin{equation}
        \textrm{Aut}_{\mathbb{F}}(\mathbb{K})
        =\{\,\textrm{id}_{\mathbb{F}},\sigma\}
    \end{equation}
    where $\sigma$ is the unique automorphism such that
    $\sigma(\alpha=\beta)$. The proof is simply an application of the
    quadratic formula, where we invoke the fact that $2\ne{0}$ in a
    field whose characteristic is not 2.
\subsection{Cubic Equations and Higher}
    In the $16^{th}$ century the Italians were able to solve the cubic
    equation:
    \begin{equation}
        x^{3}+px-q=0
    \end{equation}
    This may not look like the general cubic, but since we are
    interested in roots we may always divide off by the leading
    coefficient of $x^{3}$, and the quadratic term may be
    absorbed by completing the square, and thus any cubic can be
    written in such a form. The solution is much less elegant than the
    quadratic formula:
    \begin{equation}
        x=
    \end{equation}
    By the $18^{th}$ century the Italians were able to solve the general
    quartic equation. The next natural question is the solution to the
    quintic, but this was shown not to exist. The Abel-Ruffini theorem
    shows that the general quintic equation can not be solved using
    nested radicals. Galois went to prove that a polynomial has a root
    that can be written in terms of nested radicals if and only if
    $K/F$, the splitting field, has an automorphism group
    $\textrm{Aut}_{F}(K))$ that is solveable.
\subsection{Some Reminders}
    \begin{definition}
        A field $(\mathbb{F},+,\cdot)$ is an Abelian group
        $(\mathbb{F},+)$ such that $(F^{*}\setminus\{0\},\cdot)$ is an
        Abelian group as well. This is the group of \textit{units}.
    \end{definition}
    \begin{example}
        The classic exmaples are $\mathbb{Q}$, $\mathbb{R}$, and
        $\mathbb{C}$, as well as the finite fields $\mathbb{F}_{p}$,
        also commonly denoted $\mathbb{Z}/p\mathbb{Z}$ or simply
        $\mathbb{Z}_{p}$.
    \end{example}
    \begin{definition}
        A field extension of a field $F$ is a field $K$ such that
        $F\subseteq{K}$. We may also say that $F$ is a subfield of $K$.
    \end{definition}
    We often denote that $K$ is a field extension of $F$ by writing
    $K/F$. This is not to denote a quotient or anything of that manner
    and is simply to denote that $F$ sis a subfield of $K$.
    \begin{example}
        $\mathbb{C}$ is a field extension of $\mathbb{R}$ since both are
        fields and $\mathbb{R}\subseteq\mathbb{C}$. We can go backwards,
        thinking of $\mathbb{R}$ as a field extension $\mathbb{R}$.
    \end{example}
    Also important, if $K$ is a field extension of $F$, $K/F$, then
    $K$ has the structure of an $F$ vector space. That is, $K$ can be
    seen as a vector space over $F$. One thing that we write is this
    bracket notation $[K:F]$, which again is not to be confused with
    the notation found in groups about the cardinality of certain
    things. $[K:F]$ is the simply the dimesnion of the vector space
    $K$ over $F$:
    \begin{equation}
        [K:F]=\textrm{dim}_{K}(F)
    \end{equation}
    This is also called the degree of the extension $K/F$. If the
    dimension is finite, $[K:F]<\infty$, we say that $K/F$ is a finite
    extension.
    \begin{example}
        $\mathbb{C}$ is a two dimensional vector space over $\mathbb{R}$
        and thus $[\mathbb{C},\mathbb{R}]=2$. To see this, use
        $\{1,i\}$ as a basis.
    \end{example}
    \begin{theorem}
        Any countable dimensional vector space over a countable field is
        also countable.
    \end{theorem}
    \begin{example}
        Using this theorem shows that $\mathbb{R}$, as a vector space
        over $\mathbb{Q}$, in not only an infinite dimensional vector
        space, but also has an uncountably infinite basis.
        Thus, $[\mathbb{R}:\mathbb{Q}]$ is uncountably infinite.
    \end{example}
    \begin{example}
        Consider $\mathbb{Q}[\sqrt{2}]$, defined by:
        \begin{equation}
            \mathbb{Q}[\sqrt{2}]=\{x+y\sqrt{2}\;|\;x,y\in\mathbb{Q}\,\}
        \end{equation}
        This is a subfield of $\mathbb{R}$,
        $\mathbb{Q}[\sqrt{2}]\subseteq\mathbb{R}$. Addition and
        multiplication are easy enough to see, and 0 and 1 are contained
        in there, we need only check multiplicative inverses. But:
        \begin{equation}
            (x+\sqrt{2}y)^{\minus{1}}=\frac{x-\sqrt{2}y}{x^{2}-2y^{2}}
        \end{equation}
        And $x^{2}-2y^{2}$ is only zero when $x=y=0$, since if
        $x^{2}-2y^{2}=0$, then rearrange this to obtain $q^{2}=2$. But
        by the arguments of the ancient Greeks, there is no rational
        number whose square is 2, and thus the denominator is never
        zero for non-zero rational ordered pairs.
    \end{example}
    \begin{example}
        $\mathbb{R}/\mathbb{Q}[\sqrt{2}]$ is uncountably infinite, but
        $\mathbb{Q}[\sqrt{2}]/\mathbb{Q}$ has degree 2 with a basis
        $\{1,\sqrt{2}\}$.
    \end{example}
\subsection{Polynomials}
    We use $F[x]$ to denote the ring of polynomials with coefficients in
    $F$. For example:
    \begin{equation}
        f(x)=a_{n}x^{n}+a_{n-1}x^{n-1}+\cdots+a_{1}x+x_{0}
        \quad\quad
        a_{k}\in{F}
    \end{equation}
    Then $f\in{F}[x]$. The degree of a polynomial is the largest power
    of the polynomial with non-zero coefficient. Some things can be said
    about the degree of polynomials:
    \begin{align}
        \textrm{deg}(f+g)&\leq
            \textrm{max}\{\textrm{deg}(f),\textrm{def}(g)\}\\
        \textrm{deg}(fg)&=\textrm{deg}(f)+\textrm{deg}(g)
        \quad\quad
        f,g\ne{0}
    \end{align}
    The degree of a polynomial is zero if and only if the polynomial is
    constant. Since $F[x]$ has a ring structure, $F[x]^{*}$ can be seen
    as the set of all non-zero constant polynomials.
    \begin{theorem}
        $F[x]$ is a Euclidean domain. That is, for any polynomial
        $f\in{F}[x]$ and for any non-zero $g\in{F}[x]$, there exist
        unique polynomials $r,q\in{F}[x]$ such that $f=qg+r$ where
        either $r=0$ or $\textrm{deg}(r)<\textrm{deg}(g)$.
    \end{theorem}
    \begin{theorem}
        The polynomial ring $F[x]$ is a principal ideal domain. That is,
        every ideal $I\subseteq{F}[x]$ is principal. That is, every
        ideal is generated by a single element.
    \end{theorem}
    \begin{theorem}
        Every Euclidean domain is a principle ideal.
    \end{theorem}
    Thus, there is a bijection between ideals $I\subseteq{F}[x]$ and
    monic polynomials in $F[x]$. Recall that if $R$ is a commutatie ring
    with unity, then $r\in{R}$ is called irreducible if $r\ne{0}$, $r$
    not a unit, and if $r=ab$, then either $a$ or $b$ is a unit. We take
    this definition to exclude some trivialities. For example, in
    $\mathbb{Z}$, 3 is irreducible, however
    $3=(\minus{1})\cdot(\minus{3})$. We don't care about this product,
    since $\minus{1}$ is a unit. Moreover, an element $r\in{R}$ is
    prime if $(r)\subseteq{R}$ is a prime ideal. That is if
    $r$ divides $ab$, then either $r$ divides $a$ or $r$ divides $b$.
    By divides, $r|a$, we mean that $a=r\cdot{s}$ for some $s\in{R}$.
    \begin{example}
        If $F$ is a field, $f\in{F}[x]$, then $f$ is irreducible if and
        only if $f$ is not the product of two polynomials with degrees
        strictly less than $f$. That is, if $f=gh$, then one of these
        must be a constant.
    \end{example}
    \begin{example}
        In $\mathbb{Z}$, prime if and only if irreducible.
    \end{example}
    \begin{theorem}
        If $R$ is a integral domain, and if $r$ is prime, then it is
        irreducible. That is, if there are no zero divisors then prime
        implies irreducible.
    \end{theorem}
    \begin{theorem}
        If $R$ is a principal ideal domain and if $r$ is irreducible,
        then the ideal generated by $r$ is maximal.
    \end{theorem}
    \begin{theorem}
        A maximal ideal is a prime ideal.
    \end{theorem}
    Recale that an ideal is called prime if $R/I$ is a domain. That is,
    if $ab\in{I}$, then either $a\in{I}$ or $b\in{I}$. A maximal ideal
    is and ideal that has no proper ideals between it and the entire
    ring. Another way to say this is that $R/I$ is a field. In other
    words, if $I\subseteq{J}\subseteq{R}$, then $I=J$. Using this we see
    that a maximal ideal is prime since $R/I$ will be a field, which is
    certainly an integral domain.
    \begin{theorem}
        The fourth isomorphism theorem says that if $I\subseteq{R}$ is
        an ideal, then there is a bijection between ideals containing
        $I$, $I\subseteq{J}\subseteq{R}$, and ideals of $R/I$.
    \end{theorem}
    \begin{theorem}
        $f\in{F}[x]$ is irreducible if and only if $F[x]/(f)$ is a
        field.
    \end{theorem}
    Note that $F$ can been seen as a subfield of $F[x]/(f)$ since
    $F$ can be identitified with all constant polynomials, which can
    further be seen tolive inside of $F[x]/(f)$.
    \begin{theorem}
        If $\overline{g}\in{F}[x]/(f)$ then there exists a unique
        $g_{0}\in{F}[x]$ such that$\textrm{deg}(g_{0})<\textrm{deg}(f)$,
        with $\overline{g_{0}}=\overline{g}$.
    \end{theorem}
    If $n$ is the degree of $f$, then the set
    $\{\overline{1},\overline{x},\dots,\overline{x^{n-1}}\}$ is a basis
    for $F[x]/(f)$ over $F$.
    \begin{theorem}
        If $f$ is irreducible of degree $n$, then
        $F[x]/(f)$ is a field extension of $F$ of degree $n$.
    \end{theorem}
    \begin{example}
        In $\mathbb{R}$, the polynomial $f(x)=x^{2}+1$ is irreducible
        since it cannot be factors any further. Thus
        $\mathbb{R}[x]/(x^{2}+1)$ is a field extension of $\mathbb{R}$
        of degree 2.
    \end{example}
    \begin{theorem}
        $\mathbb{R}[x]/(x^{2}+1)$ is isomorphic to $\mathbb{C}$.
    \end{theorem}
    \begin{proof}
        For since $\{\overline{1},\overline{x}\}$ is a basis, we have:
        \begin{equation}
            \mathbb{R}[x]/(x^{2}+1)=
                \{a\overline{1}+b\overline{x}\;|\;a,b\in\mathbb{R}\}
        \end{equation}
        So we trivial map $a\overline{1}+b\overline{x}$ to $a+bi$.
    \end{proof}
    \begin{example}
        Consider now $\mathbb{Q}[x]$ with $x^{2}-2$. This is irreducible
        since it cannot be factor ($\sqrt{2}$ is irrational). Then
        $\mathbb{Q}[x]/(x^{2}-2)$ is isomorphic to
        $\mathbb{Q}[\sqrt{2}]$.
    \end{example}
\subsection{Review of Previous Lecture}
    If $F$ is a field, and if $f\in{F}[x]$ is irreducible, then
    $F[x]/(f)$ is a field extension of $F$ of degree $\textrm{deg}(f)$.
    Also, $\overline{x}=x+(f)\in{F}[x]/(f)$ is a root of $f(x)$ in this
    field $F[x]/(f)$. That is, $f(\overline{x})=\overline{f(x)}$, and
    this maps to zero. The fact that $f(\overline{x})=\overline{f(x)}$
    is simply the statement that the quotient ring is well defined.