%----------------------------Group Theory--------------------------------------%
\section{Monoids}
    We now inch towards the concept of a group and add some more structure.
    Namely, we give to semigroups the additional requirement that there exists
    a unital element. Such an object is called a monoid.
    \begin{fdefinition}{Monoid}{Monoid}
        A \gls{monoid} is a \gls{semigroup} $(G,*)$ such that there exists a
        \gls{unital element} $e\in{G}$.
    \end{fdefinition}
    From the uniqueness of unital elements with respect to any binary operation
    (Thm.~\ref{thm:Unital_Elements_are_Unique}), the identity in a monoid
    $(G,*)$ is unique.
    \begin{example}
        Any monoid is necessarily a semigroup, but the converse need not hold.
        We saw in Ex.~\ref{ex:The_Positive_Integers_as_a_Semigroup} that the
        positive integers, combined with the usual notion of addition $+$, form
        a pure semigroup that is neither a monoid nor a group. That is,
        $(\mathbb{N}^{+},+)$ has neither unital elements nor inverses and is
        thus not a monoid.
    \end{example}
    \begin{example}
        The set $\mathbb{Z}_{1}$ with the operation $*$ defined by $0*0=0$ is
        a monoid. As stated before, it is also a group (and thus necessarily a
        monoid and a semigroup).
    \end{example}
    \begin{fexample}{The Natural Numbers as a Monoid}
                    {The Natural_Numbers_as_a_Monoid}
        The quintessential example of a monoid is the natural numbers
        $\mathbb{N}$ with the usual additive binary operation
        $+:\mathbb{N}\times\mathbb{N}\rightarrow\mathbb{N}$. This makes
        $(\mathbb{N},+)$ a monoid with 0 acting as the identity. Much like the
        semigroup $(\mathbb{N}^{+},+)$, the monoid $(\mathbb{N},+)$ lacks
        inverses and is therefore not a group, but a pure monoid. It does have
        the additional property that addition is commutative, thus making it a
        \gls{commutative monoid}.
    \end{fexample}
    Any semigroup can be embedded into a monoid. That is, given a semigroup
    $(G,*)$ we can find a monoid $(\tilde{G},\times)$ such that
    $G\subseteq\tilde{G}$ and such that $\times|_{G}=*$.
    \begin{ltheorem}{Embedding Theorem for Semigroups}
                    {Embedding Theorem for Semigroups}
        If $(G,*)$ is a semigroup, then there is a monoid $(\tilde{G},\times)$
        such that $G\subseteq\tilde{G}$ and such that $\times|_{G}=*$.
    \end{ltheorem}
    \begin{proof}
        For by the law of the excluded middle, either $(G,*)$ is a monoid or it
        is not. If it is a monoid, then let $\tilde{G}=G$ and $\times=*$. If not
        then since it is a semigroup, $*$ must be associative
        (Def.~\ref{def:Semigroup}). But if $(G,*)$ is not a monoid, then there
        is no unital element (Def.~\ref{def:Monoid}). By
        Thm.~\ref{thm:Existence_of_Set_Containing_Set}, $\{G\}$ is a set and by
        Thm.~\ref{thm:Cor_of_Containment_NEqual_Underlying_Set}, $G\ne\{G\}$
        and by Thm.~\ref{thm:Set_Containing_A_is_not_Element_of_A},
        $\{G\}\notin{G}$. Thus, let $\tilde{G}=G\cup\{G\}$. Then
        $G\subseteq\tilde{G}$ (Thm.~\ref{thm:Union_is_Bigger_Left}). Define
        $\times$ as follows:
        \begin{equation}
            a\times{b}=
            \begin{cases}
                a*b,&a,b\in{G}\\
                a,&a\in{G},b=G\\
                b,&a=G,b\in{G}\\
                G,&a=G,b=G
            \end{cases}
        \end{equation}
        Then since $\{G\}\notin{G}$, this is well defined, and by definition
        $\times|_{G}=*$. But $\times$ is an associative operation with a unital
        element, since by definition $G$ is the unital element of $\tilde{G}$.
        It is associative since there are only a few cases to check:
        If $a,b,c\ne{G}$, then since $a*b\in{G}$ and $b*c\in{G}$, and since
        $G\notin{G}$ (Thm.~\ref{thm:Anti_Russells_Paradox}), we conclude that:
        \begin{equation}
            a\times(b\times{c})=a\times(b*c)=a*(b*c)=(a*b)*c=
            (a*b)\times{c}=(a\times{b})\times{c}
        \end{equation}
    \end{proof}