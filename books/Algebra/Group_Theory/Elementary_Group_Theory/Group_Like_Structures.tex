One of the most fundamental structures studied in mathematics is a
\gls{group}\index{Group}. Recalling some notions from
Book~\ref{book:Foundations} we define a group to be a \gls{set}\index{Set} $G$
together with a \gls{binary operation}\index{Binary Operation} $*$ that is
\glslink{associative operation}{associative}%
\index{Binary Operation!Associative}, has an \glslink{unital element}{identity}%
\index{Unital Element}, and such that all elements are
\glslink{invertible element}{invertible}\index{Invertible Element}.%
\footnote{%
    All of these notions are developed in Chapt.~\ref{chapt:Function_Theory}.
}
Groups then seem to be simple objects, and indeed the standard arithmetic that
one is familiar with in \gls{mathbbR} has far more structure.%
\footnote{%
    $\mathbb{R}$ forms a \gls{field}\index{Field}.
}
In another sense perhaps groups have too much
structure. We can certainly strip away invertibility, leaving associativity and
identity intact, and this results in a \gls{monoid}\index{Monoid}. We can
further rid of \textit{the} identity, maintaining only associativity, and it
may be reasonable that such objects are useful. Furthermore we can take away the
requirement that we have a binary operation and replace this with a partial
function\index{Partial Function}. This latter object gives rise to the notion of
a groupoid\index{Groupoid}, which
has applications in geometry and analysis. We'll start with ordinary
binary operations, but only require associativity. These are called
\glspl{semigroup}\index{Semigroup}.
\section{Group-Like Structures}
    One motivating reason for studying semigroups is that many foundational
    theorems of groups do not require all of the structure they are endowed
    with. Indeed almost nothing was needed to prove the identity is unique
    (Thm.~\ref{thm:Unital_Elements_are_Unique}). Given an element $e$ such that
    $a*e=e*a=a$ for all $a\in{G}$, if $e'$ is also an identity, then
    $e=e*e'=e'$. Requiring associativity and the existence of inverses is
    superfluous. However, associativity is important for otherwise there is an
    ambiguity in how to combine three elements. Associativity is not the only
    resolution to this problem but is common and appears in our familiar
    arithmetic.%
    \footnote{%
        The Jacobi identity\index{Jacobi Identity} is another means. It appears
        in the \textit{cross product} of vectors.\index{Cross Product}
    }
    \subsection{Semigroups and Monoids}
        Let's briefly recall some vocabulary from
        Chapt.~\ref{chapt:Function_Theory}. Given a set $G$ with a binary
        operation $*$, a right unital element is an $e_{R}\in{G}$ such that
        $a*e_{R}=a$ for all $a\in{G}$. Left unitals are similarly defined,
        $e_{L}*a=a$. Unital elements are those which are both left and right
        unital elements. Left and right identities need not be unique, but
        two-sided ones do. If there exists both left and right unital elements
        $e_{L}$ and $e_{R}$, then they coincide $e_{L}=e_{R}$ giving us an
        identity (Thm.~\ref{thm:left_and_right_identity_implies_identity}). A
        right invertible element is an $a\in{G}$ such that there is a $b\in{G}$
        with $a*b$ a unital element. Left invertible is similarly defined, $b*a$
        being a unital element, and invertible elements are those that are both
        left and right invertible. If the binary operation is associative, then
        inverses are unique and we may write $a^{\minus{1}}$. We relax these
        conditions to \textit{weakly} left and right invertible, requiring for
        $a\in{G}$ that there is a $b\in{G}$ with $b*a$ equal to a left unital
        element or $a*b$ equal to a right unital element, respectively. This is
        \textit{usually} a strictly weaker notion. Without further ado we give
        the definition of a semigroup.
        \begin{fdefinition}{Semigroup}{Semigroup}
            A \gls{semigroup} is an \gls{ordered pair} $\monoid{G}$ where $G$ is
            a \gls{set} and $*$ is an \gls{associative operation} $*$ on $G$.
            That is, for all $a,b,c\in{G}$ we have:\index{Semigroup}
            \begin{equation}
                \label{eqn:Semigroup_Associativity}
                a*(b*c)=(a*b)*c\tag{1}
            \end{equation}
        \end{fdefinition}
        Associativity was formally defined in Chapt.~\ref{chapt:Function_Theory}
        Def.~\ref{def:Associative_Operation}. This states there is no ambiguity
        in combining three elements. Indeed, continuing inductively, there is no
        ambiguity in combining an ordered list of $n$ elements for any
        $n\in\mathbb{N}$ with $n\geq{2}$. It is important to note we may not be
        able to swap the order of our operation. That is, for distinct
        $a,b\in{G}$ it may not be true that $a*b=b*a$. The binary operation $*$
        is not required to be \glslink{commutative operation}{commutative}.
        \index{Binary Operation!Commutative}
        \par\hfill\par
        Defining semigroups as \glspl{ordered pair} allows us to distinguish a
        semigroup from the underlying set. Recalling Kuratowski's definition
        (Def.~\ref{def:Ordered_Pairs}) we have:
        \begin{equation}
            \monoid{G}=\{\,\{G\},\,\{G,\,*\}\,\}
        \end{equation}
        By regularity\index{Axiom!of Regularity}, since $\{G\}\in\monoid{G}$ and
        $G\notin{G}$ (Thm.~\ref{thm:Anti_Russells_Paradox}), we have a
        set-theoretic way of distinguishing between the semigroup structure on
        $G$ and $G$ itself. In category theory one speaks of the
        \textit{forgetful functor}\index{Functor!Forgetful}. The language is
        often presented very loosely: We map the semigroup $G$ to the underlying
        set $G$. That is, we \textit{forget} about the binary operation $*$.
        A more satisfying way of presenting this is to express
        $\mathcal{F}:\mathbf{Semigroups}\rightarrow\mathbf{Sets}$ by:
        \begin{equation}
            \mathcal{F}\big(\monoid{G}\big)=G
        \end{equation}
        We did not adopt any axioms about proper classes in
        Book~\ref{book:Foundations} and hence such things are of little concern
        to us. Without further delay we present examples.
        \begin{example}
            If $G=\emptyset$ and $*$ is the empty function, then $\monoid{G}$ is
            the \textit{empty semigroup}\index{Semigroup!Empty Semigroup}. This
            is vacuously true since there are no elements that do not satisfy
            Def.~\ref{def:Semigroup}, and thus $\monoid[][\emptyset]{\emptyset}$
            is a semigroup.
        \end{example}
        \begin{example}
            Let $G=\mathbb{Z}_{1}=\{0\}$, and let $*$ be the only possible
            binary operation on $G$: $0*0=0$. Then $\monoid{G}$ is a semigroup
            trivially. Since there is only one element, we compute:
            \begin{equation}
                0*(0*0)=0*0=(0*0)*0
            \end{equation}
            Thus $\monoid{G}$ is a set with an associative binary operation,
            making it a semigroup (Def.~\ref{def:Semigroup}). $\monoid{G}$ has
            additional structure: there is a unital element 0 and every element
            is invertible.%
            \footnote{%
                There is only one element, and unital elements are invertible
                (Thm.~\ref{thm:Unital_Elements_Are_Invertible}).
            }
            This is the structure of a group.
        \end{example}
        \begin{example}
            \label{ex:The_Positive_Integers_as_a_Semigroup}%
            The defining example of a semigroup is the positive integers
            $\mathbb{N}^{+}$ with the usual notion of addition $+$. More
            precisely, we take the standard additive binary operation
            $+:\mathbb{N}\times\mathbb{N}\rightarrow\mathbb{N}$ and restrict it
            to $\mathbb{N}^{+}$ (but rather than writing $+|_{\mathbb{N}^{+}}$
            we stick to $+$). This makes $\monoid[][+]{\mathbb{N}^{+}}$ a
            semigroup. Since the restriction of $+$ to $\mathbb{N}^{+}$ is
            closed%
            \footnote{%
                That is, the sum of positive integers is again a positive
                integer.
            }
            we have that $+$ is a binary operation on $\mathbb{N}^{+}$. And
            since the restriction of an associative operation is again
            associative, $+$ is associative on $\mathbb{N}^{+}$, and hence the
            criteria of Def.~\ref{def:Semigroup} are satisfied.
        \end{example}
        Note the semigroup $\monoid[][+]{\mathbb{N}^{+}}$ does not have any of
        the additional structure mentioned in the forward to this chapter. There
        is no unital element, nor are there inverse elements for any
        $n\in\mathbb{N}^{+}$. To see this, note $\monoid[][+]{\mathbb{Z}}$ has a
        unital element 0 and for all $n\in\mathbb{Z}$ there is an inverse
        $\minus{n}$. Since unital elements are unique
        (Thm.~\ref{thm:Unital_Elements_are_Unique}) and since inverses are
        unique for associative operations
        (Thm.~\ref{thm:Assoc_Op_Inverses_are_Unique}), we have that 0 is the
        unique unital element and $\minus{n}$ is the unique inverse of
        $n\in\mathbb{Z}$. But $0\notin\mathbb{N}^{+}$ and for all
        $n\in\mathbb{N}^{+}$ we know that $\minus{n}\notin\mathbb{N}^{+}$.
        Suppose $\monoid[][+]{\mathbb{N}^{+}}$ has a unital element $e$. That
        is, for all $n\in\mathbb{N}^{+}$ we have $e+n=n$. We must show $e=0$,
        contradicting our claim that $e\in\mathbb{N}^{+}$. But then, since
        $\mathbb{N}^{+}\subseteq\mathbb{Z}$, for all $n\in\mathbb{N}^{+}$ there
        is a $\minus{n}\in\mathbb{Z}$ such that $n+(\minus{n})=0$. Applying
        associativity, we obtain:
        \begin{equation}
            0=n+(\minus{n})=(e+n)+(\minus{n})=e+\big(n+(\minus{n})\big)=e+0=e
        \end{equation}
        So $e=0$, a contradiction, and thus $\mathbb{N}^{+}$ has no unital
        element. Similarly, there are no invertible elements. From this we see
        that $\monoid[][+]{\mathbb{N}^{+}}$ is purely a semigroup, and is
        neither a monoid nor a group.%
        \footnote{%
            See the start of this chapter for a rough definition or
            Defs.~\ref{def:Monoid} and \ref{def:Group}, respectively.
        }
        This is what was meant by the claim that $\monoid[][+]{\mathbb{N}^{+}}$
        is the defining example of a semigroup.%
        \footnote{%
            It's slightly more, since it is a \textit{commutative} semigroup.%
            \index{Commutative!Semigroup}\index{Semigroup!Commutative}
        }
        \par\hfill\par
        In proving $\mathbb{N}^{+}$ has no unital element we used the fact that
        it embeds naturally as a subset of a group, the integers
        $\monoid[][+]{\mathbb{Z}}$, and the argument presented required the use
        of invertible elements. If we have a semigroup $\monoid{G}$ with a
        unital element $e\in{G}$, it is possible that $G\setminus\{e\}$ has a
        \textit{new} unital element $e'$. Since identities are unique, $e'$ is
        not a unital element in $G$, but rather a unital element in
        $G\setminus\{e\}$. We'll see examples once
        Thm.~\ref{thm:Embedding_Theorem_for_Semigroups} has been proven.
        \begin{example}
            We can put another associative binary operation on $\mathbb{N}^{+}$,
            and may define $*$ by $m*n=n+m+nm$. Then $\monoid{\mathbb{N}}$ is a
            semigroup since $*$ is associative (Def.~\ref{def:Semigroup}). To
            see this, we compute:
            \begin{align}
                a*(b*c)&=a*(b+c+bc)\\
                    &=a+(b+c+bc)+a(b+c+bc)\\
                    &=a+b+c+bc+ab+ac+abc\\
                    &=a+b+ab+c+(a+b+ab)c\\
                    &=(a+b+ab)*c\\
                    &=(a*b)*c
            \end{align}
            and hence $*$ is indeed associative
            (Def.~\ref{def:Associative_Operation}).
        \end{example}
        \begin{example}
            Given a set $A$, $\monoid[][\cup]{\powset{A}}$ is a semigroup
            since union is a well defined binary operation and moreover is
            associative (Thm.~\ref{thm:Associative_Law_of_Unions}). Similarly,
            $\monoid[][\cap]{\powset{A}}$ is a semigroup since intersection is
            associative (Thm.~\ref{thm:Associative_Law_of_Intersections}).
        \end{example}
        Semigroups can be truly reckless. We demonstrate this with an example.
        \begin{example}
            \label{ex:Semigroup_ab_eq_b}%
            Let $X$ be any non-empty set and define $*$ by $a*b=b$. That is, we
            simply choose the right element. Then $*$ is associative since:
            \vspace{-2.5ex}
            \twocolumneq{a*(b*c)=a*c=c}{(a*b)*c=b*c=c}
            In this structure every element is weakly left invertible and also
            a left unital element but there are no right identities or weakly
            right invertible elements.
        \end{example}
        \begin{example}
            \label{ex:Semigroup_ab_eq_a}%
            Similar to Ex.~\ref{ex:Semigroup_ab_eq_b} we can define $a*b=a$
            which again gives us a semigroup. Here every element is weakly right
            invertible and everything is a right identity. That is, if
            $a\in{G}$, let $b\in{G}$ be arbitrary. Then $a*b$ is a right unital
            element since every element is right unital. Hence, $a$ is weakly
            right invertible and $b$ is a weak right inverse. There are no
            left unital or weakly left invertible elements.
        \end{example}
        There is solace in the fact that any semigroup can be extended to a
        monoid by adding a single element. That is, any semigroup can be given
        an identity. This is done in
        Thm.~\ref{thm:Embedding_Theorem_for_Semigroups}. First, we formally
        define monoids.
        \begin{fdefinition}{Monoid}{Monoid}
            A \gls{monoid} is a \gls{semigroup} $\monoid{G}$ such that there
            exists a \gls{unital element} $e\in{G}$. That is, there is an
            $e\in{G}$ and for all $a,b,c\in{G}$ we have:
            \par\vspace{-2.5ex}
            \begin{minipage}[t]{0.49\textwidth}
                \centering
                \begin{equation}
                    \label{eqn:Monoid_Associativity}
                    a*(b*c)=a*(b*c)\tag{1}
                \end{equation}
            \end{minipage}
            \hfill
            \begin{minipage}[t]{0.49\textwidth}
                \centering
                \begin{equation}
                    \label{eqn:Monoid_Identity}
                    a*e=a\quad
                    e*a=a\tag{2}
                \end{equation}
            \end{minipage}
        \end{fdefinition}
        From the uniqueness of unital elements with respect to any binary
        operation (Thm.~\ref{thm:Unital_Elements_are_Unique}), the identity in a
        monoid $\monoid{G}$ is unique.
        \begin{example}
            Unlike semigroups, there is no \textit{empty} monoid since by
            definition we assume the existence of a unital element. Hence $G$
            must be nonempty.
        \end{example}
        \begin{example}
            Any monoid is necessarily a semigroup, but the converse need not
            hold. Ex.~\ref{ex:The_Positive_Integers_as_a_Semigroup} showed
            that the positive integers with the usual notion of addition $+$
            form a pure semigroup that is neither a monoid nor a group. That is,
            since $(\mathbb{N}^{+},+)$ does not have a unital element it is not
            a monoid.
        \end{example}
        \begin{example}
            The set $\mathbb{Z}_{1}$ with the operation $*$ defined by $0*0=0$
            is a monoid. As stated before it is also a group.
        \end{example}
        \begin{example}
            The quintessential example of a monoid is the natural numbers
            $\mathbb{N}$ with the usual additive binary operation $+$. This
            makes $(\mathbb{N},+)$ a monoid with 0 acting as the identity. Much
            like the semigroup $(\mathbb{N}^{+},+)$, the monoid $(\mathbb{N},+)$
            lacks inverses and is not a group, but rather a pure monoid.%
            \footnote{%
                It also has the property that addition is commutative making it
                a \gls{commutative monoid}.
            }
        \end{example}
        We now prove any semigroup can be extened to a monoid by adding one
        point.
        \begin{ltheorem}{Embedding Theorem for Semigroups}
                        {Embedding_Theorem_for_Semigroups}
            If $(G,*)$ is a semigroup, then there is a monoid
            $(\tilde{G},\times)$ such that $G\subseteq\tilde{G}$ and such that
            $\times|_{G}=*$. That is, the restriction of $\times$ to $G$ is $*$.
        \end{ltheorem}
        \begin{proof}
            If $G$ is a set, then $\{G\}$ is a set
            (Thm.~\ref{thm:Existence_of_Set_Containing_Set}). But $G\ne\{G\}$
            (Thm.~\ref{thm:Cor_of_Containment_NEqual_Underlying_Set}) and
            $\{G\}\notin{G}$
            (Thm.~\ref{thm:Set_Containing_A_is_not_Element_of_A}). Let
            $\tilde{G}=G\cup\{G\}$. Then $G\subseteq\tilde{G}$
            (Thm.~\ref{thm:Union_is_Bigger_Left}). Define $\times$ as follows:
            \begin{equation}
                a\times{b}=
                \begin{cases}
                    a*b,&a\in{G},\,b\in{G}\\
                    a,&a\in{G},\,b=G\\
                    b,&a=G,\,b\in{G}\\
                    G,&a=G,\,b=G
                \end{cases}
            \end{equation}
            Then since $\{G\}\notin{G}$, this is well defined, and by definition
            $\times|_{G}=*$. But $\times$ is an associative operation with a
            unital element, since by definition $G$ is the unital element of
            $\tilde{G}$. It is associative since there are only a few cases to
            check: If $a,b,c\ne{G}$, then since $a*b\in{G}$ and $b*c\in{G}$, and
            since $G\notin{G}$ we have:
            \begin{equation}
                a\times(b\times{c})=a\times(b*c)=a*(b*c)=(a*b)*c=
                (a*b)\times{c}=(a\times{b})\times{c}
            \end{equation}
            If $a,b\in{G}$ and $c=G$, we have:
            \begin{equation}
                a\times(b\times{G})=a\times{b}=(a\times{b})\times{G}
            \end{equation}
            And similarly if $a=G$ or $b=G$. Hence,
            $\monoid[][\times]{\tilde{G}}$ is a monoid (Def.~\ref{def:Monoid}).
        \end{proof}
        Let's see what happens if we perform the construction in
        Thm.~\ref{thm:Embedding_Theorem_for_Semigroups} on a semigroup that is
        already a monoid. Taking $\mathbb{Z}$ as our monoid (It's actually a
        group), we pick up a new element $\{\mathbb{Z}\}$ which we may as well
        label $\infty$ with the strange operation:
        \begin{equation}
            n+\infty=n\quad\quad{n}\in\mathbb{Z}
        \end{equation}
        This new monoid contains $\mathbb{Z}$ embedded inside, although
        $\mathbb{Z}$ already contains an identity, namely zero. So what is the
        identity for our appended space $\tilde{\mathbb{Z}}$? It's $\infty$
        since we have $0+\infty=0$ and hence 0 is no longer an identity. This
        leads to the paradoxical conclusion: There are monoids
        $\monoid{G}$ with proper subsets $A\subsetneq{G}$ that do \textbf{not}
        contain the identity element, but such that $\monoid{A}$ is still a
        monoid (contains an identity).
        \par\hfill\par
        In this example we see that 0 still really wants to be an identity, and
        the only thing stopping it from being so is $\infty$. Removing this
        element allows zero to be what its always wanted to be: The additive
        identity. This problem is resolved by groups. While monoids are
        structurally simpler than groups, the notion of a group is one of the
        most basic and fundamental algebraic structures that one can consider.
        The existence of inverse elements removes this unwanted phenomenon and
        if semigroup $\monoid{G}$ embeds into a group
        $\monoid[][\times]{\tilde{G}}$, then either the unital element in $G$ is
        the same as the one in $\tilde{G}$, or $G$ has no identity.
        \begin{fdefinition}{Group}{Group}
            A group is a \gls{monoid} $(G,*)$ such that for all $g\in{G}$, $g$
            is an \gls{invertible element} with respect to $*$. That is:%
            \index{Group}
            \begin{align}
                &\label{def:Group_Assoc}
                \textrm{The binary operation }*\textrm{ is associative}\tag{1}\\
                &\label{def:Group_Unit}
                \textrm{There exists a unital element $e\in{G}$}\tag{2}\\
                &\label{def:Group_Inverse}
                \textrm{For every element }a\in{A}
                \textrm{ there is an inverse element }a^{\minus{1}}\tag{3}
            \end{align}
        \end{fdefinition}
        Note that it is not necessarily true that $a*b=b*a$. Such groups are
        called Abelian\index{Abelian Group}\index{Group!Abelian}. There are
        several examples of groups that one is likely familiar with.
        \begin{example}
            If $+$ denotes the usual addition on $\mathbb{Z}$, then
            $(\mathbb{Z},+)$ is a group. The unital element is 0, and for all
            $n\in\mathbb{Z}$, $\minus{n}$ is an inverse element of $n$. That is,
            $n+(\minus{n})=0$, which is a unital element. Moreover, addition is
            associative and therefore $(\mathbb{Z},+)$ is a group.
        \end{example}
        \begin{example}
            If $\mathbb{R}^{+}$ denotes the positive real numbers, and if
            $\cdot$ denotes the usual multiplication of real numbers, then
            $(\mathbb{R},\cdot)$ is a group. Here 1 is the unital element and
            for all $r\in\mathbb{R}^{+}$ $1/r$ is the inverse element. Lastly,
            the operation is indeed associative.
        \end{example}
        One of the ways of representing semigroups (and groups and monoids) is
        via their \textit{Cayley Table}\index{Cayley Table}, named after the
        English mathematician Arthur Cayley (1821-1895 C.E.)%
        \index{Cayley, Arthur} who was one of the early pioneers of group
        theory. In 1854 he published \textit{On the Theory of Groups} which
        contains the first occurance of these tables. The concept is quite
        simple, given a semigroup with $n$ elements we present an $n\times{n}$
        table and fill in the $(i,j)$ spot by multiplying the $i^{th}$ element
        with the $j^{th}$ element, left to right. Of course, we need to put some
        order on the elements, but this is immaterial.
        \begin{example}
            Take $\mathbb{Z}_{2}=\{0,1\}$ and bequeath it with modulo
            arithmetic: $0+0=0$, $0+1=1$, $1+0=1$, $1+1=0$. This produces the
            structure of a group. We use this to compute the Cayley table
            (Tab.~\ref{tab:Cayley_Table_Z2})
            \begin{table}[H]
                \centering
                \captionsetup{type=table}
                \begin{tabular}{c|cc}
                    $+$&0&1\\
                    \hline
                    0&0&1\\
                    1&1&0
                \end{tabular}
                \caption{Cayley Table for $\mathbb{Z}_{2}$}
                \label{tab:Cayley_Table_Z2}
            \end{table}
            We could equivalently have chosen 1 in the left column and 0 in the
            right, the order being irrelevant. All of the information about a
            group or monoid or semigroup can be encoded in its Cayley table.%
            \footnote{%
                When the underlying set is infinite this is no longer useful.
            }
        \end{example}
        \begin{example}
            Consider $\mathbb{Z}_{4}$ with it's modular addition $+$. Then
            $(\mathbb{Z}_{4},+)$ is a group. We can represent the operation $+$
            with the following table:
            \begin{table}[H]
                \centering
                \captionsetup{type=table}
                \begin{tabular}{c|cccc}
                    $+$&0&1&2&3\\
                    \hline
                    0&0&1&2&3\\
                    1&1&2&3&0\\
                    2&2&3&0&1\\
                    3&3&0&1&2
                \end{tabular}
                \caption{The Group Structure of $\mathbb{Z}_{4}$}
            \end{table}
        \end{example}
        \begin{example}
            Rotations about a point $(x,y)\in\nspace[2]$ defined by:
            \begin{equation}
                \begin{pmatrix}
                    x'\\
                    y'
                \end{pmatrix}=
                \begin{pmatrix*}[r]
                    \cos(\theta)&\minus\sin(\theta)\\
                    \sin(\theta)&\cos(\theta)
                \end{pmatrix*}
                \begin{pmatrix}
                    x\\
                    y
                \end{pmatrix}
            \end{equation}
            form a group under rotation by an angle $\theta$. $\theta=0$ is
            the identity, and given a rotation $\theta$, the rotation
            $\minus\theta$ in the reverse direction serves as inverse.
        \end{example}
        \begin{example}
            Let $X$ denote the set of all $\vector{x}\in\nspace[2]$ such that
            $x_{0}\ne{0}$. That is, the Cartesian plane with the $x$ axis
            removed. Define $*$ on $X$ by:
            \begin{equation}
                (a,\,b)*(c,\,d)=(ac,\,bc+d)
            \end{equation}
            This is a group. The element $(1,0)$ serves as identity since:
            \begin{align}
                (1,\,0)*(c,\,d)&=(1\cdot{c},\,0\cdot{c}+d)=(c,\,d)\\
                (a,\,b)*(1,\,0)&=(a\cdot{1},\,b\cdot{1}+0)=(a,\,b)
            \end{align}
            Lastly, given $(a,b)$, the inverse is $(1/a,\minus{b}/a)$ since:
            \begin{equation}
                \big(a,\,b\big)*\big(\frac{1}{a},\,\minus\frac{b}{a}\big)
                    =\big(a\cdot\frac{1}{a},\,\frac{b}{a}-\frac{b}{a}\big)
                    =(1,\,0)
            \end{equation}
            and similarly for left multiplication. Hence, this is a group.
        \end{example}
        With Cayley tables presented, it is convenient to briefly discuss Latin
        squares\index{Latin Square}. These are studied in combinatorics but have
        their uses in groups theory. A Latin square on a collection of $n$
        letters is an $n\times{n}$ gride of these letters such that each symbol
        appears exactly once in each row and each column.
        \begin{table}[H]
            \centering
            \captionsetup{type=table}
            \begin{tabular}{|c|c|c|}
                \hline
                A&B&C\\
                \hline
                B&C&A\\
                \hline
                C&A&B\\
                \hline
            \end{tabular}
            \caption{Example of a Latin Square}
            \label{tab:Example_of_Latin_Square}
        \end{table}
        We can write this out algebraically. Consider a group with it's Cayley
        table. Fix some row element $a\in{G}$ and look at $a*x$ for all
        $x\in{G}$. The Cayley table is then a Latin square if this is bijective
        for all $a\in{G}$, and if $x*b$ is bijective for all $b\in{G}$. We can
        write this more succinctly, requiring for all $a,b\in{G}$ there is a
        unique $x\in{G}$ such that $a*x=b$. This property has been well studied
        and given a name.
        \begin{fdefinition}{Quasigroup}{Quasigroup}
            A quasigroup is an \gls{ordered pair} $\monoid{G}$ where $G$ is a
            \gls{set} and $*$ is a \gls{binary operation} on $G$ where for all
            $a,b\in{G}$ there exists a unique $x\in{G}$ with:%
            \index{Quasigroup}
            \begin{equation}
                \label{eqn:Quasigroup_Equation}%
                a*x=b\tag{1}
            \end{equation}
        \end{fdefinition}
        Note associativity is \textbf{not} assumed in the definition. The
        condition being imposed is known as the \textit{Latin square property}%
        \index{Latin Square Property} and allows one to define division
        unambigously. This is quite clever since no unital elements are assumed
        to exist. In most settings we define division by inverse elements. That
        is, given $a$ we define $1/a=a^{\minus{1}}$ where $a^{\minus{1}}$ is the
        inverse of $a$. Such a definition presumes the existence of an identity
        since then $a*a^{\minus{1}}$ is a unital element by definition.
        Quasigroups allow for division without identities since we require the
        solution to $a*x=b$ to be unique allowing us to define $a/b=x$.
        \par\hfill\par
        There is an equivalent condition one can impose that gives rise to this
        structure. A binary operation is called cancellative if $a*b=a*c$ can be
        simplified to $b=c$. Hence addition in $\mathbb{Z}$ is cancellative but
        multiplication is not since $0\cdot{2}=0\cdot{3}$ but $2\ne{3}$.
        Quasigroups necessarily have a cancellative binary operation.
        \begin{theorem}
            If $\monoid{G}$ is a quasigroup, then $*$ is a cancellative binary
            operation. That is, for all $a,b,c\in{G}$, if $a*b=a*c$, then
            $b=c$.
        \end{theorem}
        \begin{proof}
            For suppose not. Then there exists $a,b,c\in{G}$ such that
            $a*b=a*c$, but $b\ne{c}$. But $\monoid{G}$ is a quasigroup and hence
            there is a unique $x\in{G}$ with
        \end{proof}
    \subsection{Subsemigroups}
        Now that we've shown that semigroups can be embedded into monoids, we
        return to the study of pure semigroups and see if there are any gems to
        be found. We introduce the notion of subsemigroup, a generalization of
        \textit{subgroup} (see Def.~\ref{def:Subgroup}). First we need to
        develop the idea of the product of subsets of a semigroup.
        \begin{fdefinition}{Semigroup Set Product}{Semigroup_Set_Product}
            The product of a subset $A\subseteq{G}$ against a subset $B\in{G}$
            in a semigroup $(G,*)$ is the set $A*B$ defined by:
            \begin{equation*}
                A*B=\{\,x\in{G}\;|\;\textrm{There exists }a\in{A}
                    \textrm{ and }b\in{B}\textrm{ such that }x=a*b\,\}
            \end{equation*}
        \end{fdefinition}
        We can write the semigroup product of $A,B\subseteq{G}$ more
        suggestively as follows:
        \begin{equation}
            A*B=\{\,a*b\;|\;a\in{A}\textrm{ and }b\in{B}\,\}
        \end{equation}
        The definition we've used is simply to stay consistent with our notation
        used for the axiom schema of separation
        (Ax.~\ref{ax:Axiom_Schema_of_Specification}) which we need to prove that
        $A*B$ exists in the first place.
        \par\hfill\par
        We've now doubly used the symbol $*$ which is perhaps poor notation, but
        hopefully will not cause confusion. If $A$ and $B$ are \textit{subsets}
        of $G$, then $A*B$ is another subset of $G$. If $a$ and $b$ are elements
        of $G$, then $a*b$ is again an element of $G$. Therein lies the
        distinction between the notations.
        \begin{example}
            Let $\monoid[][+]{\mathbb{N}^{+}}$ be the usual additive semigroup
            on the positive integers, and let $A=\{1\}$ and
            $B=\mathbb{N}_{o}$, the set of all positive odd integers. The set
            $AB_{\mathbb{N}^{+}}$ is then the set:
            \begin{equation}
                A+B=\{\,a+b\;|\;a\in{A}\textrm{ and }b\in{B}\,\}
            \end{equation}
            But $A=\{1\}$ and $B=\mathbb{N}_{o}$, the set of all odd positive
            integers, and so for all $b\in{B}$ we can write $b=2n+1$ for
            some $n\in\mathbb{N}$ (possibly $n=0$). But then:
            \begin{subequations}
                \begin{align}
                    A+B&=\{\,1+(2n+1)\;|\;n\in\mathbb{N}\,\}\\
                        &=\{\,2n+2\;|\;n\in\mathbb{N}\,\}\\
                        &=\{\,2(n+1)\;|\;n\in\mathbb{N}\,\}\\
                        &=\{\,2n\;|\;n\in\mathbb{N}^{+}\,\}
                \end{align}
            \end{subequations}
            This is precisely the even positive natural integers. That is, we
            have found that:
            \begin{equation}
                A+B=\mathbb{N}_{e}\setminus\{0\}
            \end{equation}
        \end{example}
        \begin{example}
            Let $\monoid[][\cdot\,]{\mathbb{N}}$ be the semigroup of the
            natural numbers with the usual notion of multiplication. If we let
            $A=\{2\}$ and $B=\mathbb{N}$, then we obtain:
            \begin{equation}
                A\cdot{B}=\{\,2\cdot{n}\;|\;n\in\mathbb{N}^{+}\}
                    \mathbb{N}_{e}\setminus\{0\}
            \end{equation}
            same as the previous example. It is occasionally convenient to
            simply write $2\mathbb{N}$, much the way we write $n\mathbb{Z}$ to
            denote the multiples of $\mathbb{Z}$ by $n$.
        \end{example}
        To define a subsemigroup of a semigroup $\monoid{G}$ we want a subset
        $A\subseteq{G}$ that is closed under the binary operation $*$. That is,
        for all $a,b\in{A}$ we want it to be true that $a*b\in{A}$. We can
        phrase this in terms of the semigroup product by requiring that
        $AA_{G}\subseteq{A}$. This need not always be the case, one need only
        consider $\monoid[][+]{\mathbb{N}^{+}}$ with $A=\{1\}$. Then $AA=\{2\}$,
        and thus $A$ is not closed under $+$. With this, we now define
        subsemigroups.
        \begin{fdefinition}{Subsemigroup}{Subsemigroup}
            A subsemigroup of a semigroup $\monoid{G}$ is a subset
            $A\subseteq{G}$ such that for all $a,b\in{A}$ it is true that
            $a*b\in{A}$
        \end{fdefinition}
        \begin{theorem}
            \label{thm:Equiv_Def_Subsemigroup}%
            If $\monoid{G}$ is a semigroup, then $A\subseteq{G}$ is a
            subsemigroup if and only if $AA_{G}\subseteq{A}$, where $AA_{G}$ is
            the semigroup product of $A$ with itself in $G$.
        \end{theorem}
        \begin{proof}
            For suppose $A$ is a subsemigroup and $AA_{G}\nsubseteq{A}$. Then
            there exists $x\in{AA_{G}}$ such that $a\notin{A}$
            (Def.~\ref{def:Subsets}). But by the definition of semigroup
            product, $x\in{AA}_{G}$ if and only if there exists $a,b\in{A}$
            such that $a*b=x$ (Def.~\ref{def:Semigroup_Set_Product}). But $A$ is
            a subsemigroup and thus for all $a,b\in{A}$ it is true that
            $a*b\in{A}$, a contradiction. Hence, $AA_{G}\subseteq{A}$. Now
            suppose $AA_{G}\subseteq{A}$ and suppose $A$ is not a subsemigroup.
            Then there exists $a,b\in{A}$ such that $a*b\notin{A}$
            (Def.~\ref{def:Subsemigroup}). But if $a,b\in{A}$, then
            $a*b\in{AA}_{G}$ (Def.~\ref{def:Semigroup_Set_Product}) and by
            hypothesis $AA_{G}\subseteq{A}$, and thus $a*b\in{A}$
            (Def.~\ref{def:Subsets}), a contradiction. Thus, $A$ is a
            subsemigroup if and only if $AA_{G}\subseteq{A}$.
        \end{proof}
        \begin{theorem}
            \label{thm:Whole_Semigroup_is_Subsemigroup}%
            If $\monoid{G}$ is a semigroup, then $G$ is a subsemigroup of
            $\monoid{G}$.
        \end{theorem}
        \begin{proof}
            For if $\monoid{G}$ is a semigroup, then $*$ is an associative
            binary operation (Def.~\ref{def:Semigroup}), and hence $*$ is a
            binary operation. But then if $a,b\in{G}$, then $a*b\in{G}$
            (Def.~\ref{def:Binary_Operation}). Thus, $G$ is a subsemigroup.
        \end{proof}
        \begin{theorem}
            \label{thm:Emptyset_is_Subsemigroup}%
            If $\monoid{G}$ is a semigroup, then $\emptyset$ is a subsemigroup
            of $G$.
        \end{theorem}
        \begin{proof}
            For suppose not. Then there exists $a,b\in\emptyset$ such that
            $a*b\notin\emptyset$, a contradiction since for all $x$ it is true
            that $x\notin\emptyset$ (Def.~\ref{def:Empty_Set}). Hence,
            $\emptyset$ is a subsemigroup.
        \end{proof}
        \begin{theorem}
            \label{thm:Intersection_of_Subsemigroups_is_Subsemigroup}%
            If $\monoid{G}$ is a semigroup, if
            $\mathcal{S}\subseteq\mathcal{P}(G)$ is such that for all
            $A\in\mathcal{S}$ it is true that $A$ is a subsemigroup of $G$,
            then $\bigcap\mathcal{S}$ is a subsemigroup of $G$. are
        \end{theorem}
        \begin{proof}
            For if not then there exists $a,b\in\bigcap\mathcal{S}$ such that
            $a*b\notin\bigcap\mathcal{S}$ (Def.~\ref{def:Subsemigroup}). But if
            $a,b\in\bigcap\mathcal{S}$ then for all $\mathcal{U}\in\mathcal{S}$
            it is true that $a,b\in\mathcal{U}$
            (Def.~\ref{def:Intersection_Over_a_Collection}). But by hypothesis,
            for all $\mathcal{U}\in\mathcal{S}$ it is true that $\mathcal{U}$
            is a subsemigroup of $G$, and thus $a*b\in\mathcal{U}$
            (Def.~\ref{def:Subsemigroup}). But if $a*b\in\mathcal{U}$ for all
            $\mathcal{U}\in\mathcal{S}$, then $a*b\in\bigcap\mathcal{S}$
            (Def.~\ref{def:Intersection_Over_a_Collection}), a contradiction.
            Thus, $\bigcap\mathcal{S}$ is a subsemigroup.
        \end{proof}
        We can immediately apply this to any finite collection. In particular:
        \begin{theorem}
            \label{thm:Intersection_of_Two_Subsemigroups_is_Subsemigroup}%
            If $\monoid{G}$ is a semigroup, if $A,B\subseteq{G}$ are
            subsemigroups, then $A\cap{B}$ is a subsemigroup.
        \end{theorem}
        \begin{proof}
            For by Thm.~\ref{thm:Existence_of_Set_Built_from_Two_Sets}, the set
            $\{A,B\}$ exists. Thus, by
            Thm.~\ref{thm:Intersection_of_Subsemigroups_is_Subsemigroup}
            $\bigcap\{A,B\}$ is a subsemigroup. But
            $\bigcap\{A,B\}=A\cap{B}$, hence $A\cap{B}$ is a subsemigroup.
        \end{proof}
        Thus the structure of subsemigroups of a semigroup forms a complete
        lattice.
        \begin{ltheorem}{Lattice Theorem of Semigroups}
                        {Lattice_Theorem_of_Semigroups}
            If $\monoid{G}$ is a semigroup, if
            $\mathcal{S}\subseteq\mathcal{P}(G)$ is the set of all subsemigroups
            of $G$:
            \begin{equation}
                \mathcal{S}=\{\,A\in\mathcal{P}(G)\;|\;
                    A\textrm{ is a subsemigroup of }G\,\}
            \end{equation}
            If $(\mathcal{P}(G),\subseteq)$ is the usual partial ordering by
            inclusion, then $(\mathcal{S},\subseteq|_{\mathcal{S}})$ is a
            complete lattice.
        \end{ltheorem}
        \begin{proof}
            For let $A,B\in\mathcal{S}$. By
            Thm.~\ref{thm:Intersection_of_Two_Subsemigroups_is_Subsemigroup},
            $A\cap{B}$ is a subsemigroup. But $A\cap{B}\subseteq{A}$
            (Thm.~\ref{thm:Intersection_is_Smaller_Left}) and
            $A\cap{B}\subseteq{B}$
            (Thm.~\ref{thm:Intersection_is_Smaller_Right}), and hence
            $A\cap{B}$ is a lower bound for $A$ and $B$. Suppose it is not the
            greatest lower bound. Then there is a set $C\in\mathcal{S}$ such
            that $C\subseteq{A}$, $C\subseteq{B}$, and $A\cap{B}\subsetneq{C}$.
            But then there exists $c\in{C}$ such that $c\notin{A}\cap{B}$
            (Thm.~\ref{thm:Prop_Subset_Not_Equal}). But $C\subseteq{A}$ and thus
            $c\in{A}$ (Def.~\ref{def:Subsets}). Similarly, $C\subseteq{B}$ and
            hence $c\in{B}$. But then $c\in{A}\cap{B}$
            (Def.~\ref{def:Intersection_of_Two_Sets}), a contradiction. Thus,
            $A\cap{B}$ is the greatest lower bound of $A$ and $B$. By the
            axiom schema of specification
            (Ax.~\ref{ax:Axiom_Schema_of_Specification}), there exists the set:
            \begin{equation}
                \mathcal{U}_{AB}=\{\,\mathcal{U}\in\mathcal{S}\;|\;
                    A\in\mathcal{U}\textrm{ and }B\in\mathcal{U}\}
            \end{equation}
            But $A,B\subseteq{G}$ and $G$ is a subsemigroup of $G$
            (Thm.~\ref{thm:Whole_Semigroup_is_Subsemigroup}), and thus
            $\mathcal{U}_{AB}\ne\emptyset$. But by
            Thm.~\ref{thm:Intersection_of_Subsemigroups_is_Subsemigroup},
            $\bigcap\mathcal{U}_{AB}$ is a subsemigroup. But by contruction, for
            all $\mathcal{U}\in\mathcal{U}_{AB}$, $A\in\mathcal{U}$ and
            $B\in\mathcal{U}$, and hence $A,B\in\bigcap\mathcal{U}_{AB}$.
            Thus $\bigcap\mathcal{U}_{AB}$ is an upper bound for $A$ and $B$.
            Suppose it is not the least upper bound. Then there exists
            $C\in\mathcal{S}$ such that $A\subseteq{C}$, $B\subseteq{C}$, and
            $C\subsetneq\bigcap\mathcal{U}_{AB}$. But then $C$ is a subsemigroup
            such that $A\in{C}$ and $B\in{C}$, and thus $C\in\mathcal{U}_{AB}$.
            But then $\bigcap\mathcal{U}_{AB}\subseteq{C}$, a contradiction.
            Thus, $\bigcap\mathcal{U}_{AB}$ is the least upper bound of $A$ and
            $B$. Thus, $(\mathcal{S},\subseteq)$ is a complete lattice.
        \end{proof}
        \begin{example}
            \label{ex:Example_of_a_Semigroup}%
            Let $X$ be a set with several distinct elements and let
            $\mathscr{F}$ be the set of all constant mappings
            $f:X\rightarrow{X}$. That is, there is some $c\in{X}$ such that for
            all $x\in{X}$ it is true that $f(x)=c$, so $f[X]=\{c\}$. Let $\circ$
            denote function composition. We know that function composition is
            associative (Thm.~\ref{thm:Associativity_of_Composition}). Moreover,
            $\circ$ takes elements of $\mathscr{F}$ to $\mathscr{F}$. For if
            $f,g\in\mathscr{F}$ then there are $c_{f},c_{g}\in{X}$ such that
            $f[X]=\{c_{f}\}$ and $g[X]=\{c_{g}\}$. But then, for all $x\in{X}$
            we have:
            \begin{equation}
                (g\circ{f})(x)=g(f(x))=g(c_{f})=c_{g}
            \end{equation}
            and thus $g\circ{f}$ is a constant mapping as well. Therefore
            $\circ$ is an associative binary operation on $\mathscr{F}$ and
            $\monoid[][\circ]{\mathscr{F}}$ is a semigroup
            (Def.~\ref{def:Semigroup}).
        \end{example}
        The example shown in Ex.~\ref{ex:Example_of_a_Semigroup} is missing most
        algebraic properties. Notably, there is no identity. Since we chose $X$
        to have at least several distinct elements, for any distinct functions
        $f,g\in\mathscr{F}$ we have $g\circ{f}\ne{f}\circ{g}$ and hence there
        can be no unital element. There can also be no left unital element, and
        as such there can be no left invertible elements. Every element is a
        right unital element since $g\circ{f}=g$, and as such every element is
        \textit{weakly} right invertible. Moreover, as this previous expression
        shows, the operation is not commutative. Thus
        $\monoid[][\circ]{\mathscr{F}}$ is a semigroup but can't possibly be any
        of the nicer objects like monoids or groups. It motivates an important
        object in mathematics, the transformation semigroup. Avoiding proof by
        naming, we first show this is a semigroup.
        \begin{theorem}
            If $X$ is a set, if $\funcspace[X]{X}$ is the set of all function
            from $X$ to $X$, and if $\circ$ denotes function composition, then
            $\monoid[][\circ]{\funcspace[X]{X}}$ is a semigroup.
        \end{theorem}
        \begin{proof}
            Since function composition is a binary operation on
            $\funcspace[X]{X}$, and since $\circ$ is associative
            (Thm.~\ref{thm:Associativity_of_Composition}), this forms a
            semigroup (Def.~\ref{def:Semigroup}).
        \end{proof}
        \begin{fdefinition}{Semigroup of Transformations}
                           {Semigroup_of_Transformations}
            The semigroup of transformations on a set $X$ is the semigroup
            $\monoid[][\circ]{\funcspace[X]{X}}$ of the set of all functions
            $f:X\rightarrow{X}$ $\funcspace[X]{X}$ with function composition.
        \end{fdefinition}
        This will be much more important later when we consider certain subsets
        of transformation semigroups which actually form groups. It turns out
        \textit{every} group is of this form.%
        \footnote{%
            Up to isomorphism. This is Cayley's Theorem.%
            \index{Cayley's Theorem}\index{Theorem!Cayley's}
        }
        We now prove an analogous result for semigroups, that every semigroup is
        equivalent to a subset of a semigroup of transformations. To do this we
        need a precise notion of equivalence and we need to introduce the idea
        of subsemigroups, which are the semigroup analog of subsets in set
        theory.
