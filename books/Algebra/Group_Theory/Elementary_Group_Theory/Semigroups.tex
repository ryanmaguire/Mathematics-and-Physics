\section{Semigroups}
    One of the motivating reasons for studying semigroups is that many of the
    foundational theorems of groups do not require all of the structure they are
    endowed with. Indeed, we saw in Thm.~\ref{thm:Unital_Elements_are_Unique}
    that almost no structure is required to prove that identities are unique.
    The ingredients were a binary operation $*$ and an element $e$ such that
    $a*e=e*a=a$ for all $a$ in our collection. Requiring $*$ to be associative
    and for inverse elements to exist is thus superfluous. Associativity is an
    important condition to require for otherwise there is an ambiguity in how
    one should combine three elements. While associativity is not the only way
    to give structure to the combination of three elements (the Jacobi identity
    works as well)\index{Jacobi Identity}, it is one of the most common and
    appears in our familiar arithmetics (the Jacobi identity is found when
    considering the \textit{cross product} of two vectors
    $\mathbf{v}\times\mathbf{u}$)\index{Cross Product}. We now give the
    definition of a semigroup.
    \begin{fdefinition}{Semigroup}{Semigroup}
        A \gls{semigroup} $\monoid{G}$ is a set $G$ and an
        \gls{associative operation} $*$ on $G$.\index{Semigroup}
    \end{fdefinition}
    Recalling from Chapt.~\ref{chapt:Function_Theory},
    Def.~\ref{def:Associative_Operation}, the associative property states that
    for all $a,b,c\in{G}$, the following is true:
    \begin{equation}
        a*(b*c)=(a*b)*c
    \end{equation}
    Thus there is no ambiguity in combining three elements. Indeed, continuing
    inductively, there is no ambiguity in combing an ordered list of $n$
    elements for any $n\in\mathbb{N}$ with $n\geq{2}$. It is important to note
    that we may not be able to swap the order of our operation. That is, for
    distinct $a,b\in{G}$ it may not be true that $a*b=b*a$. The binary operation
    $*$ is not required to be \glslink{commutative operation}{commutative}
    \index{Binary Operation!Commutative}.
    \par\hfill\par
    We use the notation $\monoid{G}$ to denote a semigroup, but will explicitly
    state what we mean when writing this. This is is because $\monoid{G}$ is
    also used to represent monoids and groups. Since all groups are monoids and
    since all monoids are semigroups, there is not much harm in using the same
    notation, but nevertheless an attempt will be made to distinguish between
    these notions.
    \begin{example}
        If $G=\emptyset$ and $*$ is the empty function, then $\monoid{G}$ is the
        \textit{empty semigroup}\index{Semigroup!Empty Semigroup}. This is a
        semigroup in the vacuous sense: there are no elements that do not
        satisfy the criterion to be a semigroup, and thus
        $\monoid[][\emptyset]{\emptyset}$ is a semigroup.
    \end{example}
    \begin{example}
        Let $G=\mathbb{Z}_{1}$ (equivalently $G=\{0\}$), and let
        $*$ be the binary operation on $G\times{G}$ defined in the only way
        possible: $0*0=0$. Then $\monoid{G}$ is a semigroup in a trivial manner.
        Since there is only one element, we can trivially check associativity:
        \begin{equation}
            0*(0*0)=0*0=(0*0)*0
        \end{equation}
        Thus $\monoid{G}$ is a set with an associative binary operation,
        satisfying the criterion of Def.~\ref{def:Semigroup}, making it a
        semigroup. $\monoid{G}$ has additional structure: there is a unital
        element 0, and since 0 is a unital element it is therefore invertible
        (the inverse of 0 is necessarily 0, see
        Thm.~\ref{thm:Unital_Elements_Are_Invertible}). This is the structure
        required for $\monoid{G}$ to be considered a group.
    \end{example}
    \begin{fexample}{The Positive Integers as a Semigroup}
                    {The_Positive_Integers_as_a_Semigroup}
        The defining example of a semigroup is the positive integers
        $\mathbb{N}^{+}$ combined with the usual notion of addition $+$. More
        precisely, we take the standard additive binary operation
        $+:\mathbb{N}\times\mathbb{N}\rightarrow\mathbb{N}$ and restrict it to
        $\mathbb{N}^{+}$ (but rather than writing $+|_{\mathbb{N}^{+}}$ we
        simply stick with $+$). This makes $\monoid[][+]{\mathbb{N}^{+}}$ a
        semigroup. Since the restriction of $+$ to $\mathbb{N}^{+}$ is closed
        (that is, the sum of positive integers is again a positive integer), we
        have that $+$ is a binary operation on $\mathbb{N}^{+}$. And since the
        restriction of an associative operation is again associative, we again
        have that $+$ is associative on $\mathbb{N}^{+}$, and hence the criteria
        of Def.~\ref{def:Semigroup} are satisfied.
    \end{fexample}
    Note that the semigroup $\monoid[][+]{\mathbb{N}^{+}}$ does not have any of
    the additional structure mentioned in the forward to this chapter. There is
    no unital element, nor are there inverse elements for any
    $n\in\mathbb{N}^{+}$. To see this, note that in $\monoid[][+]{\mathbb{Z},+}$
    we have a unital element 0, and for all $n\in\mathbb{Z}$, $\minus{n}$ is an
    inverse element: $n+(\minus{n})=0$. Since unital elements are unique
    (Thm.~\ref{thm:Unital_Elements_are_Unique}) and since inverses are unique
    for associative operations (Thm.~\ref{thm:Assoc_Op_Inverses_are_Unique}), we
    have that 0 is the unique unital element and $\minus{n}$ is the unique
    inverse of $n\in\mathbb{Z}$. But $0\notin\mathbb{N}^{+}$ and for all
    $n\in\mathbb{N}^{+}$ we know that $\minus{n}\notin\mathbb{N}^{+}$.
    Suppose $\monoid[][+]{\mathbb{N}^{+}}$ has a unital element $e$. We must
    show that $e=0$, contradicting our claim that $e\in\mathbb{N}^{+}$. But
    then, since $\mathbb{N}^{+}\subseteq\mathbb{Z}$, for all
    $n\in\mathbb{N}^{+}$ there is a $\minus{n}\in\mathbb{Z}$ such that
    $n+(\minus{n})=0$. Applying associativity, we obtain:
    \begin{equation}
        0=n+(\minus{n})=(e+n)+(\minus{n})=e+\big(n+(\minus{n})\big)=e+0=e
    \end{equation}
    So $e=0$, a contradiction, and thus $\mathbb{N}^{+}$ has no unital
    element. Similarly, there are no invertible elements. From this we see that
    $\monoid[][+]{\mathbb{N}^{+}}$ is purely a semigroup, and is neither a monoid
    nor a group (see the beginning of this chapter for a rough definition, or
    see Defs.~\ref{def:Monoid} and \ref{def:Group}, respectively). This is what
    was meant by the claim that $\monoid[][+]{\mathbb{N}^{+},+)}$ is the
    defining example of a semigroup.
    \begin{fexample}{Example of a Semigroup}{Example_of_a_Semigroup}
        Let $X$ be a set with several distinct elements and let $\mathscr{F}$ be
        the set of all functions $f:X\rightarrow{X}$ to such that $f$ is a
        constant mapping. That is, there is some $c\in{X}$ such that for all
        $x\in{X}$ it is true that $f(x)=c$. In other words, the image of $X$ is
        $\{c\}$: $f[X]=\{c\}$. Let $\circ$ denote function composition. We know
        that function composition is associative. Moreover, $\circ$ takes
        elements of $\mathscr{F}$ to $\mathscr{F}$. For if $f,g\in\mathscr{F}$
        then there are $c_{f},c_{g}\in{X}$ such that $f[X]=\{c_{f}\}$ and
        $g[X]=\{c_{g}\}$. But then, for all $x\in{X}$ we have:
        \begin{equation}
            (g\circ{f})(x)=g(f(x))=g(c_{f})=c_{g}
        \end{equation}
        and thus $g\circ{f}$ is a constant mapping as well. Therefore $\circ$ is
        an associative binary operation on $\mathscr{F}$ and
        $\monoid[][\circ]{\mathscr{F}}$ is a semigroup.
    \end{fexample}
    The example shown in Ex.~\ref{ex:Example_of_a_Semigroup} is missing most
    algebraic properties. Most notably, there is no identity element. For since
    we chose $X$ to have at least several distinct elements, for any two
    distinct functions $f,g\in\mathscr{F}$, we have that
    $g\circ{f}\ne{f}\circ{g}$, and thus there can be no unital element. There
    can also be no left or right unital element,
    and as such there can be no invertible elements. Moreover, as this previous
    expression shows, the operation is not commutative. Thus
    $\monoid[][\circ]{\mathscr{F}}$ is a semigroup but can't possibly be any of
    the nicer objects like monoids or groups. Now we have seen two different
    objects that give rise to the algebraic structure of a semigroup, and
    nothing more. There a many other examples that, further justifying the study
    of this weaker category.
    \subsection{Subsemigroups}
        Next we'll introduce the notion os subsemigroup, a generalization of the
        concept of a subgroup (see Def.~\ref{def:Subgroup}). First we need to
        develop the idea of the product of subsets of a semigroup.
        \begin{fdefinition}{Semigroup Product}{Semigroup_Product}
            The semigroup product in a semigroup $(G,*)$ of two subsets
            $A,B\subseteq{G}$ is the set $AB_{G}$ defined by:
            \begin{equation*}
                AB_{G}=\{\,x\in{G}\;|\;\textrm{There exists }a\in{A}
                    \textrm{ and }b\in{B}\textrm{ such that }x=a*b\,\}
            \end{equation*}
        \end{fdefinition}
        We can write the semigroup product of $A,B\subseteq{G}$ more
        suggestively as follows:
        \begin{equation}
            AB_{G}=\{\,a*b\;|\;a\in{A}\textrm{ and }b\in{B}\,\}
        \end{equation}
        The definition we've used is simply to stay consistent with our notation
        used for the axiom schema of separation
        (Ax.~\ref{ax:Axiom_Schema_of_Specification}) which we need to prove that
        $AB_{G}$ exists in the first place.
        \begin{example}
            Let $\monoid[][+]{\mathbb{N}^{+}}$ be the usual additive semigroup
            on the positive integers, and let $A=\{1\}$ and
            $B=\mathbb{N}_{o}$, the set of all positive odd integers. The set
            $AB_{\mathbb{N}^{+}}$ is then the set:
            \begin{equation}
                AB_{\mathbb{N}^{+}}=\{\,a+b\;|\;a\in{A}\textrm{ and }b\in{B}\,\}
            \end{equation}
            But $A=\{1\}$ and $B=\mathbb{N}_{o}$, the set of all odd positive
            integers, and hence we for all $b\in{B}$ we can write $b=2n+1$ for
            some $n\in\mathbb{N}$ (possibly $n=0$). But then:
            \begin{subequations}
                \begin{align}
                    AB_{\mathbb{N}^{+}}&=\{\,1+(2n+1)\;|\;n\in\mathbb{N}\,\}\\
                        &=\{\,2n+2\;|\;n\in\mathbb{N}\,\}\\
                        &=\{\,2(n+1)\;|\;n\in\mathbb{N}\,\}\\
                        &=\{\,2n\;|\;n\in\mathbb{N}^{+}\,\}
                \end{align}
            \end{subequations}
            This is precisely the even positive natural integers. That is, we
            have found that:
            \begin{equation}
                AB_{\mathbb{N}^{+}}=\mathbb{N}_{e}\setminus\{0\}
            \end{equation}
        \end{example}
        \begin{example}
            From the associative of the usual multiplicative operation $\cdot$
            placed on $\mathbb{N}$, the restriction to $\mathbb{N}^{+}$ again
            makes $\monoid[][\cdot\,]{\mathbb{N}^{+}}$ a semigroup. If we let
            $A=\{2\}$ and $B=\mathbb{N}^{+}$, then we obtain:
            \begin{equation}
                AB_{\mathbb{N}^{+}}=\{\,2\cdot{n}\;|\;n\in\mathbb{N}^{+}\}
                    \mathbb{N}_{e}\setminus\{0\}
            \end{equation}
            same as the previous example.
        \end{example}
        To define a subsemigroup of a semigroup $\monoid{G}$ we want a subset
        $A\subseteq{G}$ that is closed under the binary operation $*$. That is,
        for all $a,b\in{A}$ we want it to be true that $a*b\in{A}$. We can
        phrase this in terms of the semigroup product by requiring that
        $AA_{G}\subseteq{A}$. This need not always be the case, one need only
        consider $\monoid[][+]{\mathbb{N}^{+}}$ with $A=\{1\}$. Then $AA=\{2\}$,
        and thus $A$ is not closed under $+$. With this, we now define
        subsemigroups.
        \begin{fdefinition}{Subsemigroup}{Subsemigroup}
            A subsemigroup of a semigroup $\monoid{G}$ is a subset
            $A\subseteq{G}$ such that for all $a,b\in{A}$ it is true that
            $a*b\in{A}$
        \end{fdefinition}
        \begin{theorem}
            \label{thm:Equiv_Def_Subsemigroup}%
            If $\monoid{G}$ is a semigroup, then $A\subseteq{G}$ is a
            subsemigroup if and only if $AA_{G}\subseteq{A}$, where $AA_{G}$ is
            the semigroup product of $A$ with itself in $G$.
        \end{theorem}
        \begin{proof}
            For suppose $A$ is a subsemigroup and $AA_{G}\nsubseteq{A}$. Then
            there exists $x\in{AA_{G}}$ such that $a\notin{A}$
            (Def.~\ref{def:Subsets}). But by the definition of semigroup
            product, $x\in{AA}_{G}$ if and only if there exists $a,b\in{A}$
            such that $a*b=x$ (Def.~\ref{def:Semigroup_Product}). But $A$ is a
            subsemigroup and thus for all $a,b\in{A}$ it is true that
            $a*b\in{A}$, a contradiction. Hence, $AA_{G}\subseteq{A}$. Now
            suppose $AA_{G}\subseteq{A}$ and suppose $A$ is not a subsemigroup.
            Then there exists $a,b\in{A}$ such that $a*b\notin{A}$
            (Def.~\ref{def:Subsemigroup}). But if $a,b\in{A}$, then
            $a*b\in{AA}_{G}$ (Def.~\ref{def:Semigroup_Product}) and by
            hypothesis $AA_{G}\subseteq{A}$, and thus $a*b\in{A}$
            (Def.~\ref{def:Subsets}), a contradiction. Thus, $A$ is a
            subsemigroup if and only if $AA_{G}\subseteq{A}$.
        \end{proof}
        \begin{theorem}
            \label{thm:Whole_Semigroup_is_Subsemigroup}%
            If $\monoid{G}$ is a semigroup, then $G$ is a subsemigroup of
            $\monoid{G}$.
        \end{theorem}
        \begin{proof}
            For if $\monoid{G}$ is a semigroup, then $*$ is an associative
            binary operation (Def.~\ref{def:Semigroup}), and hence $*$ is a
            binary operation. But then if $a,b\in{G}$, then $a*b\in{G}$
            (Def.~\ref{def:Binary_Operation}). Thus, $G$ is a subsemigroup.
        \end{proof}
        \begin{theorem}
            \label{thm:Emptyset_is_Subsemigroup}%
            If $\monoid{G}$ is a semigroup, then $\emptyset$ is a subsemigroup
            of $G$.
        \end{theorem}
        \begin{proof}
            For suppose not. Then there exists $a,b\in\emptyset$ such that
            $a*b\notin\emptyset$, a contradiction since for all $x$ it is true
            that $x\notin\emptyset$ (Def.~\ref{def:Empty_Set}). Hence,
            $\emptyset$ is a subsemigroup.
        \end{proof}
        \begin{theorem}
            \label{thm:Intersection_of_Subsemigroups_is_Subsemigroup}%
            If $\monoid{G}$ is a semigroup, if
            $\mathcal{S}\subseteq\mathcal{P}(G)$ is such that for all
            $A\in\mathcal{S}$ it is true that $A$ is a subsemigroup of $G$,
            then $\bigcap\mathcal{S}$ is a subsemigroup of $G$. are
        \end{theorem}
        \begin{proof}
            For if not then there exists $a,b\in\bigcap\mathcal{S}$ such that
            $a*b\notin\bigcap\mathcal{S}$ (Def.~\ref{def:Subsemigroup}). But if
            $a,b\in\bigcap\mathcal{S}$ then for all $\mathcal{U}\in\mathcal{S}$
            it is true that $a,b\in\mathcal{U}$
            (Def.~\ref{def:Intersection_Over_a_Collection}). But by hypothesis,
            for all $\mathcal{U}\in\mathcal{S}$ it is true that $\mathcal{U}$
            is a subsemigroup of $G$, and thus $a*b\in\mathcal{U}$
            (Def.~\ref{def:Subsemigroup}). But if $a*b\in\mathcal{U}$ for all
            $\mathcal{U}\in\mathcal{S}$, then $a*b\in\bigcap\mathcal{S}$
            (Def.~\ref{def:Intersection_Over_a_Collection}), a contradiction.
            Thus, $\bigcap\mathcal{S}$ is a subsemigroup.
        \end{proof}
        We can immediately apply this to any finite collection. In particular:
        \begin{theorem}
            \label{thm:Intersection_of_Two_Subsemigroups_is_Subsemigroup}%
            If $\monoid{G}$ is a semigroup, if $A,B\subseteq{G}$ are
            subsemigroups, then $A\cap{B}$ is a subsemigroup.
        \end{theorem}
        \begin{proof}
            For by Thm.~\ref{thm:Existence_of_Set_Built_from_Two_Sets}, the set
            $\{A,B\}$ exists. Thus, by
            Thm.~\ref{thm:Intersection_of_Subsemigroups_is_Subsemigroup}
            $\bigcap\{A,B\}$ is a subsemigroup. But
            $\bigcap\{A,B\}=A\cap{B}$, hence $A\cap{B}$ is a subsemigroup.
        \end{proof}
        Thus the structure of subsemigroups of a semigroup forms a complete
        lattice.
        \begin{ltheorem}{Lattice Theorem of Semigroups}
                        {Lattice_Theorem_of_Semigroups}
            If $\monoid{G}$ is a semigroup, if
            $\mathcal{S}\subseteq\mathcal{P}(G)$ is the set of all subsemigroups
            of $G$:
            \begin{equation}
                \mathcal{S}=\{\,A\in\mathcal{P}(G)\;|\;
                    A\textrm{ is a subsemigroup of }G\,\}
            \end{equation}
            If $(\mathcal{P}(G),\subseteq)$ is the usual partial ordering by
            inclusion, then $(\mathcal{S},\subseteq|_{\mathcal{S}})$ is a
            complete lattice.
        \end{ltheorem}
        \begin{proof}
            For let $A,B\in\mathcal{S}$. By
            Thm.~\ref{thm:Intersection_of_Two_Subsemigroups_is_Subsemigroup},
            $A\cap{B}$ is a subsemigroup. But $A\cap{B}\subseteq{A}$
            (Thm.~\ref{thm:Intersection_is_Smaller_Left}) and
            $A\cap{B}\subseteq{B}$
            (Thm.~\ref{thm:Intersection_is_Smaller_Right}), and hence
            $A\cap{B}$ is a lower bound for $A$ and $B$. Suppose it is not the
            greatest lower bound. Then there is a set $C\in\mathcal{S}$ such
            that $C\subseteq{A}$, $C\subseteq{B}$, and $A\cap{B}\subsetneq{C}$.
            But then there exists $c\in{C}$ such that $c\notin{A}\cap{B}$
            (Thm.~\ref{thm:Prop_Subset_Not_Equal}). But $C\subseteq{A}$ and thus
            $c\in{A}$ (Def.~\ref{def:Subsets}). Similarly, $C\subseteq{B}$ and
            hence $c\in{B}$. But then $c\in{A}\cap{B}$
            (Def.~\ref{def:Intersection_of_Two_Sets}), a contradiction. Thus,
            $A\cap{B}$ is the greatest lower bound of $A$ and $B$. By the
            axiom schema of specification
            (Ax.~\ref{ax:Axiom_Schema_of_Specification}), there exists the set:
            \begin{equation}
                \mathcal{U}_{AB}=\{\,\mathcal{U}\in\mathcal{S}\;|\;
                    A\in\mathcal{U}\textrm{ and }B\in\mathcal{U}\}
            \end{equation}
            But $A,B\subseteq{G}$ and $G$ is a subsemigroup of $G$
            (Thm.~\ref{thm:Whole_Semigroup_is_Subsemigroup}), and thus
            $\mathcal{U}_{AB}\ne\emptyset$. But by
            Thm.~\ref{thm:Intersection_of_Subsemigroups_is_Subsemigroup},
            $\bigcap\mathcal{U}_{AB}$ is a subsemigroup. But by contruction, for
            all $\mathcal{U}\in\mathcal{U}_{AB}$, $A\in\mathcal{U}$ and
            $B\in\mathcal{U}$, and hence $A,B\in\bigcap\mathcal{U}_{AB}$.
            Thus $\bigcap\mathcal{U}_{AB}$ is an upper bound for $A$ and $B$.
            Suppose it is not the least upper bound. Then there exists
            $C\in\mathcal{S}$ such that $A\subseteq{C}$, $B\subseteq{C}$, and
            $C\subsetneq\bigcap\mathcal{U}_{AB}$. But then $C$ is a subsemigroup
            such that $A\in{C}$ and $B\in{C}$, and thus $C\in\mathcal{U}_{AB}$.
            But then $\bigcap\mathcal{U}_{AB}\subseteq{C}$, a contradiction.
            Thus, $\bigcap\mathcal{U}_{AB}$ is the least upper bound of $A$ and
            $B$. Thus, $(\mathcal{S},\subseteq)$ is a complete lattice.
        \end{proof}
