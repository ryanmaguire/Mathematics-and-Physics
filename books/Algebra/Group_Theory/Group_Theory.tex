%------------------------------------------------------------------------------%
\begingroup
    \ifcsname\PATH\endcsname
        \newcommand{\PATH}{books/Algebra/Group_Theory}
        \newcommand{\OLDPATH}{\PATH}
    \else
        \newcommand{\OLDPATH}{\PATH}
        \renewcommand{\PATH}{books/Algebra/Group_Theory}
    \fi
    \chapter{Elementary Group Theory}
        \section{Semigroups}
    The bulk of group theory should discuss groups. A group is a set with a
    binary operation\index{Binary Operation} that satisfies three particular
    properties. Many theorems that can be proved about groups do no need all
    three of these properties and thus it becomes natural to generalize groups
    to slightly weaker structures. The first two objects to discuss are
    semigroups and monoids. Developing monoids is particularly useful for when
    we wish to develop rings, which is a set with two binary operations.
    \begin{fdefinition}{Semigroup}{Semigroup}
        A \gls{semigroup} is a set $G$ and an \gls{associative operation} $*$
        on $G$.\index{Semigroup}
    \end{fdefinition}
    Associativity is usually a crucial operation to have, otherwise we've no
    idea how to combine three elements to get a fourth.
    \begin{fexample}{Example of a Semigroup}{Example_of_a_Semigroup}
        Let $X$ be a set with several distinct elements and let $\mathscr{F}$ be
        the set of all functions $f:X\rightarrow{X}$ to such that $f$ is a
        constant mapping. That is, there is some $c\in{X}$ such that for all
        $x\in{X}$ it is true that $f(x)=c$. In other words, the image of $X$ is
        $\{c\}$: $f(X)=\{c\}$. Let $\circ$ denote function composition. We know
        that function composition is associative. Moreover, $\circ$ takes
        elements of $\mathscr{F}$ to $\mathscr{F}$. For if $f,g\in\mathscr{F}$
        then there are $c_{f},c_{g}\in{X}$ such that $f(X)=\{c_{f}\}$ and
        $g(X)=\{c_{g}\}$. But then, for all $x\in{X}$ we have:
        \begin{equation}
            (g\circ{f})(x)=g(f(x))=g(c_{f})=c_{g}
        \end{equation}
        and thus $g\circ{f}$ is a constant mapping as well. Therefore $\circ$ is
        an associative binary operation on $\mathscr{F}$ and
        $(\mathscr{F},\circ)$ is a semigroup.
    \end{fexample}
    The example shown in Ex.~\ref{ex:Example_of_a_Semigroup} is missing most
    algebraic properties. Most notably, there is no identity element. For since
    we chose $X$ to have at least several distinct elements, for any two
    distinct functions $f,g\in\mathscr{F}$, we have that
    $g\circ{f}\ne{f}\circ{g}$, and thus there can be no unital element. There
    can also be no left or right unital element,
    and as such there can be no invertible elements. Moreover, as this previous
    expression shows, the operation is not commutative. Thus
    $(\mathscr{F},\circ)$ is a semigroup but can't possible be any of the nicer
    objects like monoids or groups. While such examples may be trivial, this
    does show that it may be worth while studying the structure of these weaker
    algebraic systems.
        %----------------------------Group Theory--------------------------------------%
\section{Monoids}
    We now inch towards the concept of a group and add some more structure.
    Namely, we give to semigroups the additional requirement that there exists
    a unital element. Such an object is called a monoid.
    \begin{fdefinition}{Monoid}{Monoid}
        A \gls{monoid} is a \gls{semigroup} $(G,*)$ such that there exists a
        \gls{unital element} $e\in{G}$.
    \end{fdefinition}
    From the uniqueness of unital elements with respect to any binary operation
    (Thm.~\ref{thm:Unital_Elements_are_Unique}), the identity in a monoid
    $(G,*)$ is unique.
    \begin{example}
        Any monoid is necessarily a semigroup, but the converse need not hold.
        We saw in Ex.~\ref{ex:The_Positive_Integers_as_a_Semigroup} that the
        positive integers, combined with the usual notion of addition $+$, form
        a pure semigroup that is neither a monoid nor a group. That is,
        $(\mathbb{N}^{+},+)$ has neither unital elements nor inverses and is
        thus not a monoid.
    \end{example}
    \begin{example}
        The set $\mathbb{Z}_{1}$ with the operation $*$ defined by $0*0=0$ is
        a monoid. As stated before, it is also a group (and thus necessarily a
        monoid and a semigroup).
    \end{example}
    \begin{fexample}{The Natural Numbers as a Monoid}
                    {The Natural_Numbers_as_a_Monoid}
        The quintessential example of a monoid is the natural numbers
        $\mathbb{N}$ with the usual additive binary operation
        $+:\mathbb{N}\times\mathbb{N}\rightarrow\mathbb{N}$. This makes
        $(\mathbb{N},+)$ a monoid with 0 acting as the identity. Much like the
        semigroup $(\mathbb{N}^{+},+)$, the monoid $(\mathbb{N},+)$ lacks
        inverses and is therefore not a group, but a pure monoid. It does have
        the additional property that addition is commutative, thus making it a
        \gls{commutative monoid}.
    \end{fexample}
    Any semigroup can be embedded into a monoid. That is, given a semigroup
    $(G,*)$ we can find a monoid $(\tilde{G},\times)$ such that
    $G\subseteq\tilde{G}$ and such that $\times|_{G}=*$.
    \begin{ltheorem}{Embedding Theorem for Semigroups}
                    {Embedding Theorem for Semigroups}
        If $(G,*)$ is a semigroup, then there is a monoid $(\tilde{G},\times)$
        such that $G\subseteq\tilde{G}$ and such that $\times|_{G}=*$.
    \end{ltheorem}
    \begin{proof}
        For by the law of the excluded middle, either $(G,*)$ is a monoid or it
        is not. If it is a monoid, then let $\tilde{G}=G$ and $\times=*$. If not
        then since it is a semigroup, $*$ must be associative
        (Def.~\ref{def:Semigroup}). But if $(G,*)$ is not a monoid, then there
        is no unital element (Def.~\ref{def:Monoid}). By
        Thm.~\ref{thm:Existence_of_Set_Containing_Set}, $\{G\}$ is a set and by
        Thm.~\ref{thm:Cor_of_Containment_NEqual_Underlying_Set}, $G\ne\{G\}$
        and by Thm.~\ref{thm:Set_Containing_A_is_not_Element_of_A},
        $\{G\}\notin{G}$. Thus, let $\tilde{G}=G\cup\{G\}$. Then
        $G\subseteq\tilde{G}$ (Thm.~\ref{thm:Union_is_Bigger}). Define
        $\times$ as follows:
        \begin{equation}
            a\times{b}=
            \begin{cases}
                a*b,&a,b\in{G}\\
                a,&a\in{G},b=G\\
                b,&a=G,b\in{G}\\
                G,&a=G,b=G
            \end{cases}
        \end{equation}
        Then since $\{G\}\notin{G}$, this is well defined, and by definition
        $\times|_{G}=*$. But $\times$ is an associative operation with a unital
        element, since by definition $G$ is the unital element of $\tilde{G}$.
        It is associative since there are only a few cases to check:
        If $a,b,c\ne{G}$, then since $a*b\in{G}$ and $b*c\in{G}$, and since
        $G\notin{G}$ (Thm.~\ref{thm:Anti_Russells_Paradox}), we conclude that:
        \begin{equation}
            a\times(b\times{c})=a\times(b*c)=a*(b*c)=(a*b)*c=
            (a*b)\times{c}=(a\times{b})\times{c}
        \end{equation}
    \end{proof}
        \section{Groups}
    \begin{fdefinition}{Groups}{Groups}
        A group is a set $G$ with a binary operation $*$,
        denoted $(G,*)$, such that: 
        \begin{enumerate}
            \item There exists an identity element $e$.
            \item For every element $a\in{A}$, there is an inverse element.
            \item The binary operation $*$ is associative.
        \end{enumerate}
    \end{fdefinition}
    Note that it is not necessarily true that $a*b = b*a$.
    These are special groups that are called Abelian.
    \begin{theorem}
        If $(G,*)$ is a group and $e$ is the identity, then it is unique.
    \end{theorem}
    \begin{proof}
        For if $e$ and $e'$ are identities, then:
        \begin{equation}
            e'=e'*e=e
        \end{equation}
        Therefore, etc.
    \end{proof}
    \begin{theorem}
        \label{thm:Group_Theory_Inverses_Are_Unique}
        If $(G,*)$ is a group, $a\in{G}$, and if $a^{-1}$ and
        $a'^{-1}$ are inverses of $a$, then $a^{-1}=a'^{-1}$.
    \end{theorem}
    \begin{proof}
        For if $e$ is the identity, and
        $a^{-1}$ and $a'^{-1}$ are inverses of $a$, then:
        \begin{align}
            a^{-1}&=a^{-1}*e
            \tag{Identitive Property}\\
            &=a^{-1}*(a*a'^{-1})
            \tag{Inverse Property}\\
            &=(a^{-1}*a)*a'^{-1}
            \tag{Associative Property}\\
            &=e*a'^{-1}
            \tag{Inverse Property}\\
            &=a'^{-1}
            \tag{Identitive Property}
        \end{align}
        Therefore, etc.
    \end{proof}
    \begin{theorem}
        If $(G,*)$ is a group, and if $e\in{G}$ is the identity,
        then $e^{-1}=e$.
    \end{theorem}
    \begin{proof}
        For $e=e*e$, and by Thm.~\ref{thm:Group_Theory_Inverses_Are_Unique}
        inverses are unique. Therefore, etc.
    \end{proof}
    \begin{theorem}
        \label{thm:Group_Theory_Inverse_of_Product}
        If $(G,*)$ is a group and $a,b\in G$, then:
        \begin{equation}
            (a*b)^{-1} = b^{-1}*a^{-1}
        \end{equation}
    \end{theorem}
    \begin{proof}
        For:
        \begin{align}
            (a*b)*(b^{-1}*a^{-1})&=
            a*(b*b^{-1})*a^{-1}
            \tag{Associative Property}\\
            &=a*(e)*a^{-1}
            \tag{Inverse Property}\\
            &=a*a^{-1}
            \tag{Identitive Property}\\
            &=e
            \tag{Inverse Property}
        \end{align}
        Thus $b^{-1}*a^{-1}$ is a right-inverse of $a*b$.
        But since $(G,*)$ is a group, right-inverses are
        left-inverses, and therefore $b^{-1}*a^{-1}$ is
        an inverse of $a*b$. But by
        Thm.~\ref{thm:Group_Theory_Inverses_Are_Unique},
        inverses are unique. Therefore, etc.
    \end{proof}
    \begin{theorem}
        If $(G,*)$ is a group and $a\in{G}$, then:
        \begin{equation}
            (a^{-1})^{-1}=a
        \end{equation}
    \end{theorem}
    \begin{proof}
        For:
        \begin{align}
            a^{-1}*(a^{-1})^{-1}
            &=(a^{-1}* a)^{-1}
            \tag{Thm.~\ref{thm:Group_Theory_Inverse_of_Product}}\\
            &=e
            \tag{Inverse Property}
        \end{align}
        From uniqueness, $(a^{-1})^{-1}=a$.
    \end{proof}
    \begin{definition}
        If $(G,*)$ and $(G',\circ)$ are groups and
        $f:G\rightarrow G'$ is a bijective function, then $f$ is said to
        be an isomorphism between $(G,*)$ and $(G',\circ)$ if and only if
        for all $a,b\in{G}$, $f(a*b)=f(a)\circ{f}(b)$.
    \end{definition}
    \begin{theorem}
        If $(G,*)$ and $(G',\circ)$ are isomorphic with identities $e_{*}$
        and $e_{\circ}$ are the identities, then $f(e_{*})=e_{\circ}$.
    \end{theorem}
    \begin{proof}
        $\forall a\in G,\ f(a)=f(a* e_*) = f(a)\circ f(e_*)$ as $f$ is
        an isomorphism. As identities are unique, $f(e_*)=e_{\circ}$.
    \end{proof}
    \begin{theorem}
    If $\langle G, * \rangle$ and $\langle G', \circ \rangle$ are isomorphic, with isomorphism $f$, and if $a\in G$, then $f(a^{-1}) = f(a)^{-1}$.
    \end{theorem}
    \begin{proof}
    For $e_{\circ}=f(e_*) = f(a*a^{-1}) = f(a^{-1}*a) = f(a)\circ f(a^{-1})=f(a^{-1})\circ f(a)$. As inverses are unique, $f(a^{-1})=f(a)^{-1}$.
    \end{proof}
    \begin{definition}
    A binary operation $*$ on a set $A$ is said to be commutative if and only for all $a,b\in A$, $a*b = b*a$.
    \end{definition}
    \begin{definition}
    A field is a set $F$ with two operations $+$ and $\cdot$, denoted $\langle F, +,\cdot \rangle$, with the following properties:
    \begin{enumerate}
    \item $a+b=b+a$ \hfill [Addition is Commutative]
    \item $a+(b+c)=(a+b)+c$ \hfill [Addition is Associative]
    \item $a\cdot b = b\cdot a$ \hfill [Multiplication is Commutative]
    \item $a\cdot (b\cdot c) = (a\cdot b)\cdot c$ \hfill [Multiplication is Associative]
    \item There is a $0\in F$ such that $0+a=a$ for all $a\in F$ \hfill [Existence of Additive Identity]
    \item There is a $1\in F$ such that $1\cdot a = a$ for all $a\in F$ \hfill [Existence of Multiplicative Identity]
    \item For each $a\in F$ there is a $b\in F$ such that $a+b = 0$. $b$ is denoted $-a$ \hfill [Existence of Additive Inverses]
    \item For each $a\in F$, $a\ne 0$ there is a $b\in F$ such that $a\cdot b = 1$. $b$ is denoted $a^{-1}$. \hfill [Existence of Multiplicative Inverses]
    \item $a\cdot(b+c) = a\cdot b + a\cdot c$ \hfill [Distributive Property]
    \end{enumerate}
    \end{definition}
    \begin{definition}
    A subfield of a field $\langle F,+,\cdot \rangle$ is a set $K\subset F$, such that $\langle K, +,\cdot \rangle$ is a field.
    \end{definition}
    \begin{theorem}
    In a field, $0$ and $1$ are unique.
    \end{theorem}
    \begin{proof}
    For suppose not, and let $0'$ and $1'$ be other identities. Then $1'=1'\cdot 1 = 1$ and $0'=0'+0=0$.
    \end{proof}
    \begin{theorem}
    For any field $\langle{F},+,\cdot\rangle$ and $a\in{F}$,
    $a\cdot{0}=0$.
    \end{theorem}
    \begin{proof}
    For $0=a\cdot{0}+(-a\cdot 0)=a\cdot(0+0)+(-a\cdot 0)=a\cdot{0}+a\cdot 0 + (-a\cdot 0) = a\cdot 0$. Thus, $a\cdot 0 = 0$.
    \end{proof}
    If $1=0$, then $a=a\cdot{1}=a\cdot{0}=0$, and thus every element is
    zero. A very boring field.
    \begin{theorem}
        In a field $\langle F, +,\cdot \rangle$, if $0\ne 1$, then $0$
        has no inverse.
    \end{theorem}
    \begin{proof}
    For let $a$ be such an inverse. Then $a\cdot 0 = 1$. But for any element of $F$, $a \cdot 0 = 0$. But $0\ne 1$, a contradiction.
    \end{proof}
    \begin{theorem}
    If $a+b = 0$, then $b= (-1)\cdot a$ where $(-1)$ is the solution to $1+(-1)=0$.
    \end{theorem}
    \begin{proof}
    $a+(-1)a = a(1+(-1)) = a\cdot 0 = 0$. From uniqueness, $b=(-1)a$. We may thus write additive inverses as $-a$
    \end{proof}
    \begin{definition}
    Given two fields $\langle F,+,\cdot \rangle$ and $\langle F', +',\times \rangle$, a bijection function $f:F\rightarrow F'$ is said to be a field isomorphism if and only if for allelements $a,b\in F$, $f(a+b)=f(a)+'f(b)$, and $f(a\cdot b) = f(a)\times f(b)$
    \end{definition}
    \begin{definition}
    $\langle F,+,\cdot \rangle$ and $\langle F', +',\times \rangle$, are said to be isomorphic if and only if they have an isomorphism.
    \end{definition}
    \begin{theorem}
    Given an ismorphism between two fields $\langle F,+,\cdot \rangle$ and $\langle F', +',\times \rangle$, $f(1) = 1'$ and $f(0) = 0'$.
    \end{theorem}
    \begin{proof}
    For let $x\in F$. Then $f(x)=f(x\cdot 1) = f(x)\times f(1)$, and $f(x)=f(x+0) = f(x)+'f(0)$. Therefore, etc.
    \end{proof}
    \begin{theorem}
    In a field $\langle F,+,\cdot \rangle$, $(a+ b)^2 = a^2 + 2ab + b^2$ ($2$ being the solution to $1+1$).
    \end{theorem}
    \begin{proof}
    For $(a+b)^2 = (a+b)(a+b) = a(a+b)+b(a+b) = a^2 + ab + ba + b^2 = a^2 +ab(1+1)+b^2 = a^2 + 2ab + b^2$.
    \end{proof}
    \begin{definition}
        A group is a set $G$ and a binary relation $*$
        on $G$, denoted $(G,*)$, such that:
        \begin{enumerate}
            \item For all ${a,b,c}\in{G}$, $(a*b)*c=a*(b*c)$
            \item There is an ${e}\in{G}$ such that for all
                ${a}\in{G}$, $a*e=e*a=a$.
            \item For all ${a}\in{G}$ there is a ${b}\in{G}$
                such that $a*b=e$
        \end{enumerate}
    \end{definition}
    \begin{definition}
        An Abelian group is a group $(G,*)$ such that
        $*$ is commutative.
    \end{definition}
    \begin{example}
        $G=\{1\}$ is an Abelian group under multiplication.
        This is the trivial group.
    \end{example}
    \begin{theorem}
        If $(G,*)$ is a group, then the following are true:
        \begin{enumerate}
            \item The identity ${e}\in{G}$ is unique.
            \item If $a*b=a*c$, then $b=c$.
            \item If $b*a=c*a$, then $b=c$.
            \item Inverses $a^{-1}$ are unique.
            \item $\forall_{{a,b}\in{G}}\exists_{{x}\in{G}}:a*x=b$
            \item $(a*b)^{-1}=b^{-1}*a^{-1}$
        \end{enumerate}
    \end{theorem}
    \begin{definition}
        The order of a group is number of elements in the
        group.
    \end{definition}
    \begin{definition}
        A group of finite order, or a finite group,
        is a group with finitely many elements.
    \end{definition}
    \begin{definition}
        The direct product of two groups $(G,*)$ and
        $(H,\circ)$ is the group  $({G}\times{H},\star)$
        where $\star$ is the binary operation defined by
        $(g_{1},h_{1})\star(g_{2},h_{2})%
         =(g_{1}*g_{2},{h_{1}}\circ{h_{2}})$
    \end{definition}
    \begin{definition}
        A permutation group on $n$ elements is a
        group whose elements are permutations of
        $n$ elements.
    \end{definition}
    \begin{definition}
        The symmetric group on $n$ elements,
        denoted $S_{n}$, is the group formed by
        permuting $n$ elements.
    \end{definition}
    \begin{definition}
        A homomorphism from a group $(G,*)$ to
        a group $(H,\circ)$ is a function
        $h:{G}\rightarrow{H}$ such that for all
        ${a,b}\in{G}$, $h(a*b)={h(a)}\circ{h(b)}$
    \end{definition}
    \begin{definition}
        An epimorphism from a group $(G,*)$ to
        a group $(H,\circ)$ is a homomorphism
        $h:{G}\rightarrow{H}$ such that
        $h$ is surjective.
    \end{definition}
    \begin{definition}
        A monomorphism from a group $(G,*)$ to
        a group $(H,\circ)$ is a homomorphism
        $h:{G}\rightarrow{H}$ such that
        $h$ is injective.
    \end{definition}
    \begin{definition}
        An isomorphism from a group $(G,*)$ to
        a group $(H,\circ)$ is a homomorphism
        $h:{G}\rightarrow{H}$ such that
        $h$ is bijective.
    \end{definition}
    \begin{definition}
        A ring is a set $R$ and two binary operations
        on $R$, denoted $(R,\cdot,+)$, such that:
        \begin{enumerate}
            \item $(R,+)$ is an Abelian group.
            \item $a\cdot({b}\cdot{c})=({a}\cdot{b})\cdot{c}$
            \item ${a}\cdot(b+c)={a}\cdot{b}+{a}\cdot{c}$
        \end{enumerate}
    \end{definition}
    \begin{definition}
        A ring with identity is a ring $(R,\cdot,+)$
        such that there is a ${1}\in{R}$ such that for
        all ${a}\in{R}$, ${a}\cdot{1}={1}\cdot{a}=a$.
    \end{definition}
    Left and right identities are elements such that ${e_{L}}\cdot{a}=a$
    and ${e_{R}}\cdot{a}=a$. If inverses $a_{L}^{-1}$ and $a_{R}^{-1}$
    exist for $a$, then $a_{L}^{-1}=a_{R}^{-1}$. That is, the inverse
    is the same for both right and left identities.
    \begin{definition}
        A commutative ring is a ring $(R,\cdot,+)$ such that
        $\cdot$ is commutative.
    \end{definition}
    \begin{definition}
        A commutative ring with identity is a ring with identity such
        that $\cdot$ is commutative.
    \end{definition}
    \begin{definition}
        A Field is a commutative ring with identity $(F,\cdot,+)$ such
        that for all ${a}\in{F}$ such that $a$ is not an identity with
        respect to $+$, there is a $b\in{F}$ such that ${a}\cdot{b}=1$.
    \end{definition}
    \begin{definition}
        Equivalent sets are sets $A$ and $B$ such that there exists a
        bijective function $f:{A}\rightarrow{B}$
    \end{definition}
    \begin{definition}
        A finite set is a set $A$ such that there is an
        ${n}\in{\mathbb{N}}$ such that $A$ is equivalent to
        $\mathbb{Z}_{n}$.
    \end{definition}
    \begin{definition}
        A countable set (Or a denumerable set) is a set $A$ that is
        equivalent to $\mathbb{N}$.
    \end{definition}
    \begin{definition}
        An uncountable set is a set that is neither finite nor countable.
    \end{definition}
    \begin{theorem}
        Set Equivalence is an equivalence relation.
    \end{theorem}
    This equivalence allows to classify all sets by the
    number of elements they contain or, more generally,
    by their cardinality. We say that two sets $A$ and
    $B$ have the same cardinality, denoted
    $\Card(A)$, if and only if $A$ and $B$ are equivalent.
    \begin{theorem}
        The following are true:
        \begin{enumerate}
            \item $\Card(A)=0$ if and only if $A=\emptyset$.
            \item If ${A}\sim{\mathbb{Z}_{n}}$, then $\Card(A)=n$.
        \end{enumerate}
    \end{theorem}
    \begin{definition}
        A finite cardinal number is a cardinal
        number of a finite set.
    \end{definition}
    \begin{definition}
        The standard ordering on the finite cardinal
        number is $0<1<\hdots<n<n+1<\hdots$
    \end{definition}
    Thus, if $A$ and $B$ are finite sets, then we write
    $\Card(A)<\Card(B)$ if $A$ is equivalent to a
    subset of $B$ but not equivalent to $B$.
    We take this notion and generalize to
    all sets. For $A$ and $B$, we write
    $\Card(A)<\Card(B)$ if $A$ is equivalent to a subset
    of $B$ but is not equivalent to $B$. This is the
    same as saying $A$ is equivalent to a subset of $B$,
    but $B$ is not equivalent to a subset of $A$.
    We write that
    $\Card(A)\leq\Card(B)$ is $A$ is equivalent to a
    subset of $B$.
    \begin{theorem}[Schr\"{o}der-Bernstein Theorem]
        If $A$ and $B$ are sets such that
        $\Card(A)\leq\Card(B)$ and
        $\Card(B)\leq\Card(A)$, then
        $\Card(A)=\Card(B)$.
    \end{theorem}
    \begin{theorem}
        The following are true:
        \begin{enumerate}
            \item If $\Card(A)\leq\Card(B)$ and
                  $\Card(B)\leq\Card(A)$, then
                  $\Card(A)\leq\Card(C)$.
            \item If $\Card(A)\leq\Card(B)$, then
                  $\Card(A)+\Card(C)\leq\Card(B)+\Card(C)$
        \end{enumerate}
    \end{theorem}
    \begin{theorem}
        If ${A}\subset{B}\subset{C}$, and
        $\Card(A)=\Card(C)$, then $\Card(B)=\Card(C)$.
    \end{theorem}
    \begin{theorem}
        If $f:{X}\rightarrow{Y}$ is a function,
        then $\Card(f(X))\leq\Card(X)$.
    \end{theorem}
    \begin{proof}
        Note that $f^{-1}(\{y\})$ creates a set of mutually disjoint
        subsets of $X$. By the axiom of choice there is a function
        $F:{f(X)}\rightarrow{X}$ such that for all ${y}\in{f(X)}$,
        ${F(y)}\in{f^{-1}(\{y\})}$. But since these sets are disjoint,
        $F$ is injective. Thus, $f(X)$ is equivalent to a subset of $X$.
        Therefore, $\Card(f(X))\leq\Card(X)$.
    \end{proof}
    The Schr\"{o}der-Bernstein theorem can be restated equivalently as
    ``If $A$ is equivalent to a subset of $B$ and $B$ is equivalent to a
    subset of $A$, then $A$ is equivalent to $B$.'' Addition and
    multiplication of finite cardinals follows directly from the standard
    arithmetic for the natural numbers. For cardinals of infinite sets,
    the arithmetic becomes a little more complicated.
    \begin{definition}
        The sum of two cardinal numbers is the
        cardinality of the union of two disjoint sets $A$
        and $B$. That is, if ${A}\cap{B}=\emptyset$, then
        $\Card(A)+\Card(B)=\Card({A}\cup{B})$.
    \end{definition}
    \begin{theorem}
        If $a$ and $b$ are distinct cardinal numbers, then there exists
        sets $A$ and $B$ such that ${A}\cap{B}=\emptyset$, $\Card(A)=a$,
        and $\Card(B)=b$.
    \end{theorem}
    \begin{theorem}
        If $A,B,C,$ and $D$ are sets such that $\Card(A)=\Card(C)$,
        $\Card(B)=\Card(D)$, and if ${A}\cap{B}=\emptyset$ and
        ${C}\cap{D}=\emptyset$, then
        $\Card({A}\cup{B})=\Card({C}\cup{D})$.
    \end{theorem}
    \begin{theorem}
        If $x,y,$ and $z$ are cardinal numbers, then
        $x+y=y+x$ and $x+(y+z)=(x+y)+z$.
    \end{theorem}
    The carinality of the set of natural numbers is denoted $\aleph_{0}$.
    That is, $\Card(\mathbb{N})=\aleph_{0}$
    \begin{example}
        Find the cardinal sum of $2$ and $5$. Let
        $N_{2}=\{1,2\}$ and $N_{5}=\{3,4,5,6,7\}$.
        Then $N_{2}$ and $N_{5}$ are disjoint,
        $\Card(N_{2})=2$ and $\Card(N_{5})=5$.
        Therefore $2+5=\Card(N_{2}\cup{N_{5}})$.
        But ${N_{2}}\cup{N_{5}}$ is just $\mathbb{Z}_{7}$,
        and $\Card(\mathbb{Z}_{7})=7$. Thus, $2+5=7$.
    \end{example}
    \begin{theorem}
        If $n$ and $m$ are finite cardinalities, then the cardinal sum
        of $n$ and $m$ is the integer $n+m$, where $+$ is the usual
        arithmetic addition.
    \end{theorem}
    \begin{example}
        Compute the cardinal sum $\aleph_{0}+\aleph_{0}$. Let
        $\mathbb{N}_{e}$ be the set of even natural numbers, and let
        $\mathbb{N}_{o}$ be the set of odd natural numbers. Then
        $\Card(\mathbb{N}_{e})=\aleph_{0}$,
        $\Card(\mathbb{N}_{o})=\aleph_{0}$, and
        ${\mathbb{N}_{o}}\cap{\mathbb{N}_{e}}=\emptyset$.
        Thus:
        \begin{equation}
            \aleph_{0}+\aleph_{0}
            =\Card({\mathbb{N}_{o}}\cup{\mathbb{N}_{e}})
        \end{equation}
        But ${\mathbb{N}_{o}}\cup{\mathbb{N}_{e}}=\mathbb{N}$ and
        $\Card(\mathbb{N})=\aleph_{0}$. Therefore,
        $\aleph_{0}+\aleph_{0}=\aleph_{0}$.
    \end{example}
    \begin{example}
        Find $n+\aleph_{0}$, where $n\in\mathbb{N}$.
        We have that
        $\Card(\mathbb{Z}_{n}z)=n$ and
        $\Card(\mathbb{N}\setminus\mathbb{Z}_{n})%
         =\aleph_{0}$
        But then
        $n+\aleph_{0}=%
         \Card(\mathbb{Z}_{n}\cup%
         \mathbb{N}\setminus\mathbb{Z}_{n})%
         =\Card(\mathbb{N})=\aleph_{0}$.
        Therefore, $n+\aleph_{0}=\aleph_{0}$.
    \end{example}
    \begin{definition}
        The cardinality of the continuum,
        denoted $\mathfrak{c}$, is the
        cardinality of the set of real numbers.
        That is, $\mathfrak{c}=\Card(\mathbb{R})$.
    \end{definition}
    \begin{theorem}
        $\Card([0,1])=\mathfrak{c}$.
    \end{theorem}
    \begin{theorem}
        $\Card\big((0,1)\big)=\mathfrak{c}$.
    \end{theorem}
    \begin{theorem}
        $\mathbb{R}$ is uncountable. That is,
        $\mathfrak{c}>\aleph_{0}$.
    \end{theorem}
    \begin{theorem}
        $\mathfrak{c}+\aleph_{0}=\mathfrak{c}$.
    \end{theorem}
    \begin{proof}
        We have $\Card((0,1))=\mathfrak{c}$ and
        $\Card(\mathbb{N})=\aleph_{0}$. But
        $(0,1)\cap\mathbb{N}=\emptyset$, and thus
        $\aleph_{0}+\mathfrak{c}%
         =\Card((0,1)\cup\mathbb{N})$.
        But $\mathbb{R}\sim(0,1)$ and
        $\mathbb{N}\cup(0,1)\subset\mathbb{R}$.
        By the Schr\"{o}der-Bernstein theorem,
        $\mathbb{N}\cup(0,1)\sim\mathbb{R}$.
        Therefore, etc.
    \end{proof}
    \begin{definition}
        The product of two cardinal numbers $a$ and $b$
        is the cardinality of the cartesian product
        of two set $A$ and $B$ such that
        $\Card(A)=a$ and $\Card(B)=b$. That is,
        ${a}\times{b}=\Card({A}\times{B})$.
    \end{definition}
    \begin{theorem}
        The following are true of cardinal numbers:
        \begin{enumerate}
            \item $xy=yx$
            \item $x(yz)=(xy)z$
            \item $x(y+z)=xy+xz$
        \end{enumerate}
    \end{theorem}
    \begin{proof}[Proof of Part 3]
        Let $A,B,$ and $C$ be disjoint.
        Then
        ${A}\times{({B}\cup{C})}%
         =({A}\times{B})\cup({A}\times{C})$, and thus
        $\Card({A}\times{({B}\cup{C})})%
         =\Card(({A}\times{B})\cup({A}\times{C}))$.
        But ${A}\times{B}$ and ${A}\times{C}$ are disjoint.
        Thus we have
        $\Card(({A}\times{B})\cup({A}\times{C}))%
         =\Card({A}\times{B})+\Card({A}\times{C})$.
        Therefore, etc.
    \end{proof}
    \begin{theorem}
        If $\Card(T)=x$ and $F:{T}\rightarrow{\mathcal{P}(T)}$ is a
        set-valued mapping such that for all ${t}\in{T}$ we have that
        $\Card(F(t))=y$ and for all ${t}\ne{t}$,
        ${F(t)}\cap{F(t')}=\emptyset$, then $\Card(\cup_{t=1}^{N}F(t))=xy$
    \end{theorem}
    \begin{example}
        Let $f:{\mathbb{N}^{2}}\rightarrow{\mathbb{N}}$
        be defined by $f(n,m)=2^{n}3^{m}$.
        Then $f$ is injective, since $2$ and $3$
        are coprime. Therefore,
        $\aleph_{0}\times\aleph_{0}=\aleph_{0}$.
    \end{example}
    \begin{example}
        Show that $\mathbb{R}^{2}\sim\mathbb{R}$.
        Let $f:\mathbb{R}^{2}\rightarrow\mathbb{R}$
        be the rather bizarre function defined by the image
        $f(x_{0}.x_{1}x_{2}\hdots,y_{0}.y_{1}y_{2}\hdots)%
         =x_{0}y_{0}.x_{0}y_{0}x_{1}y_{1}\hdots$ Then
        $f$ is inective. But the mapping
        $g:\mathbb{R}\rightarrow\mathbb{R}^{2}$
        defined by $g(x)=(x,0)$ is also injective.
        By Schr\"{o}der-Bernstein,
        $\mathbb{R}^{2}\sim\mathbb{R}$.
    \end{example}
    \begin{definition}
       Order isomorphic set are two sets $A$ and $B$
       with well orders $<_{A}$ and $<_{B}$ such that
       there exists a bijection $f:{B}A\rightarrow{B}$
       such that for all $a_{1},a_{2}\in{A}$ such that
       $a_{1}<_{A}a_{2}$, $f(a_{1})<_{B}f(a_{2})$.
    \end{definition}
    \begin{theorem}
       Order-Isomorphism is an equivalence relation.
    \end{theorem}
    To every well ordered set, an ordinal number is
    assigned, denoted $\Ord(A,<_{A})$. Conversely,
    for every ordinal number there is a set with a
    well order corresponding to it. Two ordinal numbers
    are equal if and only if the well-ordered sets
    corresponding to them are order isomorphic.
    That is,
    $\Ord(A,<_{A})=\Ord(B,<_{B})$ if and only if
    $(A,<_{A})$ and $(B,<_{B})$ are order isomorphic.
    \begin{theorem}
       If $(A,<_{A})$ and $(B,<_{B})$ are well ordered
       sets, and if $\Card(A)=\Card(B)$, then
       $(A,<_{A})$ and $(B,<_{B})$ are order
       isomorphic.
    \end{theorem}
    The ordinal number of the empty set is $0$. The
    ordinal number of a finite set of $n$ elements with
    a well ordering is denoted $n\in\mathbb{N}$.
    The ordinal for the natural numbers $\mathbb{N}$
    with their usual well-ordering is denoted $\omega$.
    A given well-ordered set has only one cardinal number,
    but it is possible for it to have two ordinal numbers.
    \begin{definition}
        An ordinal number $\alpha$ is less than or equal to an ordinal
        number $\beta$ if there are well-ordered sets $(A,<_{A})$ and
        $(B,<_{B})$ such that $\alpha=\Ord((A,<_{A}))$ and
        $\beta=\Ord(B,<_{B})$, and $(A,<_{B})$ is order isomorphic to
        subset of $(B,<_{B})$.
    \end{definition}
    \begin{theorem}
        The only order isomorphism from a well ordered set $(A,<_{A})$ to
        itself is the identity isomorphism.
    \end{theorem}
    \begin{theorem}
        If $\alpha$ and $\beta$ are ordinal numbers and
        ${\alpha}\leq{\beta}$ and ${\beta}\leq{\alpha}$,
        then $\alpha=\beta$.
    \end{theorem}
    \begin{theorem}
        If $\alpha$ and $\beta$ are ordinal numbers, either
        ${\alpha}\leq{\beta}$, or ${\beta}\leq{\alpha}$.
    \end{theorem}
    \begin{theorem}
        If $\alpha$ and $\beta$ are ordinal numbers, either
        $\alpha<\beta$, $\beta<\alpha$, or $\alpha=\beta$.
    \end{theorem}
    \begin{definition}
        The total ordering relation of a well-ordered set $(A,<_{A})$
        with respect
       to a well-ordered set $(B,<_{B})$ is the ordering
       on the set ${A}\cup{B}$ defined as: For all
       $a_{1},a_{2}\in{A}$ such that $a_{1}<_{A}a_{2}$,
       $a_{1}<_{*}a_{2}$, for all $b_{1},b_{2}\in{B}$
       such that $b_{1}<_{B}b_{2}$, $b_{1}<_{*}b_{2}$,
       and for all ${a}\in{A}$ and ${b}\in{B}$,
       ${a}<_{*}{b}$.
    \end{definition}
    \begin{theorem}
       The total ordering relation $<_{*}$ on the set
       ${A}\cup{B}$ is a well-ordering.
    \end{theorem}
    \begin{definition}
        The ordinal sum of two ordinal numbers $\Ord((A,<_{A}))$ and
        $\Ord((B,<_{B}))$, where $A$ and $B$ are disjoint, is the ordinal
        number $\Ord(({A}\cup{B},<_{*}))$.
    \end{definition}
    \begin{theorem}
       The following are true of ordinal numbers:
       \begin{enumerate}
            \item $\alpha<\beta\Rightarrow\alpha+\gamma<\beta+\gamma$
            \item $(\alpha+\beta)+\gamma=\alpha+(\beta+\gamma)$
            \item $\alpha+\beta=\alpha+\gamma\Rightarrow\beta=\gamma$
       \end{enumerate}
    \end{theorem}
    \begin{definition}
        The lexicographic ordering on the cartesianproduct of well
        ordered set $(A,<_{A})$ and $(B,<_{B})$ is the ordering on
        ${A}\times{B}$ defined by: If ${a}<_{A}{x}$, then
        $(a,b)<_{*}(x,y)$ for all $b,y\in{B}$, and if $a=x$ and
        $b<_{B}y$, then $(a,b)<_{*}(x,y)$.
    \end{definition}
    \begin{theorem}
        If $(A,<_{A})$ and $(B,<_{B})$ are well ordered sets, then the
        lexicographic ordering on ${A}\times{B}$ is a well ordering.
    \end{theorem}
    \begin{definition}
        The ordinal product of two ordinal numbers
        $\Ord((A,<_{A}))$ and $\Ord((B,<_{B}))$,
        is $\Ord(({A}\times{B},<_{*}))$
    \end{definition}
    \begin{theorem}
        The following are true of ordinal numbers:
        \begin{enumerate}
            \item $\alpha(\beta\gamma)=(\alpha\beta)\gamma$
            \item $\alpha(\beta+\gamma)=\alpha\beta+\alpha\gamma$
        \end{enumerate}
    \end{theorem}
    \begin{definition}
       Relatively prime integers are integers
       $a,b\in\mathbb{N}$ such that $\gcd(a,b)=1$.
    \end{definition}
    \begin{theorem}
       If $p$ is prime and $a\in\mathbb{N}$ is
       such that $p$ does not divide $a$, then $a$ and $p$
       are relatively prime.
    \end{theorem}
    \begin{theorem}
       There are infinitely many prime numbers.
    \end{theorem}
    \begin{theorem}
       If $a\in\mathbb{N}$, $a>1$, then either
       $a$ is a prime number, or $a$ is the product
       of finitely many primes.
    \end{theorem}
    \begin{theorem}
       If $a\in\mathbb{N}$, $a>1$, and if $a$ is not
       prime, then the prime expansion of $a$ is
       unique.
    \end{theorem}
    \begin{definition}
       A diophantine equation is an equation whose
       solutions are required to be integers.
    \end{definition}
    \begin{definition}
       A linear diophantine equation in two variables
       $x$ and $y$ is an equation
       $ax+by=c$, where $a,b,c\in\mathbb{Z}$.
    \end{definition}
    \begin{theorem}
       If $a,b,c\in\mathbb{Z}$ $d=\gcd(a,b)$, and if $d$ does not
       divide $c$, then $ax+by=c$ has no integral solutions.
    \end{theorem}
    \begin{theorem}
       If $a,b,c\in\mathbb{Z}$ $d=\gcd(a,b)$, and if $d$ divides $c$,
       then $ax+by=c$ has infinitely many solutions.
    \end{theorem}
        \section{Group Morphisms}
    \subsection{Homomorphisms}
        \begin{fdefinition}{Group Homomorphism}{Group_Homomorphism}
            A \gls{group homomorphism} from a \gls{group} $(G,*)$ to a group
            $(G',\circ)$ is a \gls{bijective function}
            $\varphi:G\rightarrow{G}'$ such that for all $a,b\in{G}$ it is true
            that:
            \begin{equation*}
                \varphi(a*b)=\varphi(a)\circ\varphi(b)
            \end{equation*}
            \index{Homomorphism!Group}
        \end{fdefinition}
        \begin{fdefinition}{Group Isomorphism}{Group_Isomorphism}
            A \gls{group isomorphism} from a \gls{group} $(G,*)$ to a group
            $(G',\circ)$ is a \glslink{bijective function}{bijective}
            \gls{group homomorphism} $\varphi:G\rightarrow{G}'$.
            \index{Isomorphism!Group}
        \end{fdefinition}
        \begin{theorem}
            If $(G,*)$ and $(G',\circ)$ are isomorphic with identities $e_{*}$
            and $e_{\circ}$ are the identities, then $f(e_{*})=e_{\circ}$.
        \end{theorem}
        \begin{proof}
            $\forall a\in G,\ f(a)=f(a* e_*) = f(a)\circ f(e_*)$ as $f$ is
            an isomorphism. As identities are unique, $f(e_*)=e_{\circ}$.
        \end{proof}
        \begin{theorem}
            If $(G,*)$ and $(G',\circ)$ are isomorphic, with isomorphism $f$,
            and if $a\in{G}$, then $f(a^{\minus{1}})=f(a)^{\minus{1}}$.
        \end{theorem}
        \begin{proof}
            For:
            \begin{equation}
                e_{\circ}=f(e_*)
                        =f(a*a^{-1})
                        =f(a^{-1}*a)
                        =f(a)\circ f(a^{-1})
                        =f(a^{-1})\circ f(a)
            \end{equation}
            As inverses are unique, $f(a^{-1})=f(a)^{-1}$.
        \end{proof}
        \begin{fdefinition}{Group Epimorphism}{Group_Epimorphism}
            An epimorphism from a group $(G,*)$ to a group $(H,\circ)$ is a
            homomorphism $h:{G}\rightarrow{H}$ such that $h$ is surjective.
            \index{Epimorphism!Group}
        \end{fdefinition}
        \begin{fdefinition}{Group Monomorphism}{Group_Monomorphism}
            A monomorphism from a group $(G,*)$ to a group $(H,\circ)$ is a
            homomorphism $h:{G}\rightarrow{H}$ such that $h$ is injective.
            \index{Monomorphism!Group}
        \end{fdefinition}
    \chapter{Finite Groups}
        Finite groups are of fundamental interest not only to mathematicians,
        but throughout many of the other sciences. Indeed, chemists and
        physicists make regular use of the theory of finite groups, and its
        application can be found in general relativity, quantum mechanics, and
        studying the lattice structure of organic molecules. A finite group is
        exactly what it sounds like: A group $(G,*)$ where $G$ is a finite set.
        \begin{fdefinition}{Finite Group}{Finite_Group}
            A finite group is a \gls{group} $(G,*)$ such that $G$ is a
            finite set.
        \end{fdefinition}
        One of the fundamental problems of group theory is a combinatorial one.
        Given an integer $n\in\mathbb{N}$, how many groups with $n$ elements are
        there (up to isomorphism)? This challenging problem can be aided by the
        theorems of Cayley, Cauchy, Lagrange, and Sylow, and it is our aim to
        develop this theory.
        \section{Permutation Groups}
    Recall that a permutation on a set $A$ is a bijective function 
    $f:A\rightarrow{A}$. That is, $f$ is a rearranging of $A$. Under the
    operation of function composition, given a non-empty set $A$, the set of all
    permutation on $A$ together with function composition $\circ$ have a
    group structure.
    \begin{theorem}
        \label{thm:Symmetric_Group_is_a_Group}%
        If $A$ is a non-empty set, if $S_{A}$ is the set of all permutations of
        $A$, and if $\circ$ denotes function composition, then
        $S_{A},\circ)$ is a group.
    \end{theorem}
    \begin{proof}
        For $\circ$ is indeed a binary operation on $S_{A}$. There also
        exists a unital element, since $\textrm{Id}_{A}$ is a permutation on
        $A$. Lastly, if $f\in{S}_{A}$, then it is a bijection and thus there
        exists an inverse function $g:A\rightarrow{A}$. But the inverse of a
        permutation is a permutation, and thus $g\in{S}_{A}$. Thus,
        $(S_{A},\circ)$ is closed to inverses and is therefore a group
        (Def.~\ref{def:Group}).
    \end{proof}
    We will be most interested in case when $A=\mathbb{Z}_{n}$ for some
    $n\in\mathbb{N}$. The set of all permutations on a set $A$ is called the
    \textit{symmetric group}\index{Symmetric Group} of $A$.
    \begin{fdefinition}{Symmetric Group}{Symmetric_Group}
        The symmetric group of a set $A$ is the group $(S_{A}\circ)$
        of all permutations of $A$ under function composition $\circ$.
    \end{fdefinition}
    By Thm.~\ref{thm:Symmetric_Group_is_a_Group}, the symmetric group is a
    group. The reason we required the underlying set to be non-empty is because
    the set of perumtations of the empty set is empty, and thus $S_{\emptyset}$
    cannot be a group since groups are required to have a unital element, and
    thus cannot be empty.
    \begin{lexample}{Symmetric Group $S_{3}$}{Symmetric_Group_S3}
        We've seen the symmetric group for $\mathbb{Z}_{3}$ before and noted
        that it is isomorphic to $D_{6}$. Given a permutation $f\in{S}_{3}$, we
        can describe $f$ via the following matrix:
        \begin{equation}
            f=
            \begin{pmatrix}
                0&1&2\\
                1&0&2
            \end{pmatrix}
        \end{equation}
        The first row of the matrix is the input, and the second row is the
        output. This matrix tells us that $f$ can be defined as follows:
        \begin{equation}
            f(n)=
            \begin{cases}
                1,&n=0\\
                0,&n=1\\
                2,&n=2
            \end{cases}
        \end{equation}
        We have that there are $3!=6$ permutations on $\mathbb{Z}_{3}$, and we
        can list them as follows:
        \par
        \begin{subequations}
            \begin{minipage}[b]{0.49\textwidth}
                \centering
                \begin{align}
                    \textrm{Id}_{\mathbb{Z}_{3}}&=
                    \begin{pmatrix}
                        0&1&2\\
                        0&1&2
                    \end{pmatrix}\\
                    \alpha&=
                    \begin{pmatrix}
                        0&1&2\\
                        0&2&1
                    \end{pmatrix}\\
                    \beta&=
                    \begin{pmatrix}
                        0&1&2\\
                        2&0&1
                    \end{pmatrix}
                \end{align}
            \end{minipage}
            \hfill
            \begin{minipage}[b]{0.49\textwidth}
                \centering
                \begin{align}
                    \gamma&=
                    \begin{pmatrix}
                        0&1&2\\
                        1&0&2
                    \end{pmatrix}\\
                    \delta&=
                    \begin{pmatrix}
                        0&1&2\\
                        1&2&0
                    \end{pmatrix}\\
                    \epsilon&=
                    \begin{pmatrix}
                        0&1&2\\
                        2&1&0
                    \end{pmatrix}
                \end{align}
            \end{minipage}
        \end{subequations}
        \par\vspace{2.5ex}
        We can use this to compute compositions of permutations.
        \begin{equation}
            \beta\circ\delta=
            \begin{pmatrix}
                0&1&2\\
                2&0&1
            \end{pmatrix}
            \begin{pmatrix}
                0&1&2\\
                1&2&0
            \end{pmatrix}=
            \begin{pmatrix}
                0&1&2\\
                0&1&2
            \end{pmatrix}
            =\textrm{Id}_{\mathbb{Z}_{3}}
        \end{equation}
        This should be read as \textit{0 goes to 1 and 1 goes to 0},
        \textit{so 0 goes to 0}. That is, we read the $\delta$ matrix first,
        and then feed this result to the $\beta$ matrix. Similarly,
        1 goes to 2 and 2 goes to 1, so 1 goes to 1. Lastly, 2 goes to 0 and 0
        goes to 2, so 2 goes to 0. The resulting permutation is the identity
        permutation. Note that we are \textbf{not} performing matrix
        multiplication. On the one hand, we've yet to define matrix
        multiplication at this point, and on the other matrix multiplication is
        \textit{undefined} for matrices of these sizes. That is, we cannot
        multiply a $2\times{3}$ matrix by a $2\times{3}$ matrix in the usual
        fashion. This representation is simply to aid in ones understanding of
        groups of permutations. The symmetric group is non-Abelian, as is
        $D_{6}$ We can see that it is non-Abelian by considering
        $\alpha\circ\beta$ and $\beta\circ\alpha$. In $\alpha\circ\beta$ we have
        that 0 goes to 2 and 2 goes to 1, so 0 goes to 1. However in
        $\beta\circ\alpha$ we see that 0 goes to 0, and then 0 goes to 2, so 0
        goes to 2. Thus $\alpha\circ\beta\ne\beta\circ\alpha$, so
        $(S_{3},\circ)$ is not Abelian. We can form a new representation of
        $S_{3}$ to show that it is isomorphic to $D_{6}$. In fact, there are
        only two groups with 6 elements (up to isomorphism).
    \end{lexample}
    We do not have to consider \textit{all} permutations on a given group, and
    the more general \textit{group of permutations} is formed by considering
    subgroups of a symmetric.
    \begin{fdefinition}{Permutation Group}{Permutation_Group}
        A permutation group of a set $A$ is a subgroup of the symmetric group
        $(S_{A},\circ)$ on $A$.
    \end{fdefinition}
    What's remarkable is that \textit{every} group is a permutation group for
    some set $A$. This result is known as Cayley's
    Theorem\index{Cayley's Theorem} and will be proved shortly.
    \subsection{Finite Permutations}
        When dealing with permutations on a finite set, it is convenient to
        break up a given permutation into disjoint \textit{cycles}. For example,
        suppose we have the following permutation on $\mathbb{Z}_{8}$:
        \begin{equation}
            f=
            \begin{pmatrix}
                0&1&2&3&4&5&6&7\\
                2&5&6&4&0&7&3&1
            \end{pmatrix}
        \end{equation}
        We can draw this as two disjoint cycles as follows:
        \begin{figure}[H]
            \centering
            \captionsetup{type=figure}
            \begin{tikzpicture}[%
                ->-/.style={%
                    decoration={%
                        markings,
                        mark=at position .55 with \arrow{Stealth}
                    },
                    postaction={decorate}
                }
            ]
                \coordinate (0) at (0.000:2);
                \coordinate (2) at (72.00:2);
                \coordinate (6) at (144.0:2);
                \coordinate (3) at (216.0:2);
                \coordinate (4) at (288.0:2);

                \draw[fill=black] (0) circle (0.05);
                \draw[fill=black] (2) circle (0.05);
                \draw[fill=black] (6) circle (0.05);
                \draw[fill=black] (3) circle (0.05);
                \draw[fill=black] (4) circle (0.05);

                \node at (0.000:2.3) {$0$};
                \node at (72.00:2.3) {$2$};
                \node at (144.0:2.3) {$6$};
                \node at (216.0:2.3) {$3$};
                \node at (288.0:2.3) {$4$};

                % Draw the first cycle (its a pentagon)
                \draw[->-] (0) to (2);
                \draw[->-] (2) to (6);
                \draw[->-] (6) to (3);
                \draw[->-] (3) to (4);
                \draw[->-] (4) to (0);

                \begin{scope}[xshift=6cm]
                    \coordinate (1) at (90.00:2);
                    \coordinate (5) at (210.0:2);
                    \coordinate (7) at (330.0:2);

                    \draw[fill=black] (1) circle (0.05);
                    \draw[fill=black] (5) circle (0.05);
                    \draw[fill=black] (7) circle (0.05);

                    \node at (90.00:2.3) {$1$};
                    \node at (210.0:2.3) {$5$};
                    \node at (330.0:2.3) {$7$};

                    \draw[->-] (1) to (5);
                    \draw[->-] (5) to (7);
                    \draw[->-] (7) to (1);
                \end{scope}
            \end{tikzpicture}
            \caption{Cycle Diagram for a Permutation}
            \label{fig:Cycle_Diagram_of_Permutation}
        \end{figure}
        We can write this simply as the product of two cycle of the permutation:
        \begin{equation}
            f=(02634)(157)
        \end{equation}
        First we need to rigorously define a cycle.
        \begin{fdefinition}{Cycle Permutation}{Cycle_Permutation}
            A cycle permutation on a finite set $A$ is a permutation
            $f\in{S}_{A}$ such that there exists two disjoint subsets
            $M,N\subseteq{A}$ such that $M\cup{N}=A$,
            $f|_{N}=\textrm{Id}_{A}|_{N}$, and such that there 
        \end{fdefinition}
        \begin{theorem}
            Disjoint cycles commute.
        \end{theorem}
        \begin{theorem}
            If $A$ is a set and if $f$ is a permutation on $A$, then $f$ is
            the product of finitely many disjoint cycles. The product is unique.
        \end{theorem}
        \begin{fdefinition}{Transposition}{Transposition}
            A transposition is a cycle of length 2.
        \end{fdefinition}
        \begin{theorem}
            Every cycle is the product of transpositions.
        \end{theorem}
        \begin{theorem}
            If $\alpha$ is a cycle of length $s$, and if $\alpha^{2}$ is a
            cycle, then $s$ is odd.
        \end{theorem}
        \begin{theorem}
            If $\alpha$ is a permutation, then $\alpha^{2}$ is an even
            permutation.
        \end{theorem}
        The decomposition of a cycle into transpositions need not be unique, and
        even the number of transpositions in such a decomposition can vary.
        \begin{theorem}
            The identity is always the product of an even number of
            transpositions.
        \end{theorem}
        \begin{theorem}
            The number of transpositions in the decomposition of a permutation
            is either always odd or always even.
        \end{theorem}
        This always us to define the alternating group.
        \begin{fdefinition}{Alternating Group}{Alternating_Group}
            The alternating group on a set $B$ is the subgroup of $S_{B}$
            consisting of all even permutations. It is denoted $A_{B}$.
        \end{fdefinition}

    \renewcommand{\PATH}{\OLDPATH}
\endgroup