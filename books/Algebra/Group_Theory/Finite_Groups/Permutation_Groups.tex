\section{Permutation Groups}
    Recall that a permutation on a set $A$ is a bijective function 
    $f:A\rightarrow{A}$. That is, $f$ is a rearranging of $A$. Under the
    operation of function composition, given a non-empty set $A$, the set of all
    permutation on $A$ together with function composition $\circ$ have a
    group structure.
    \begin{theorem}
        \label{thm:Symmetric_Group_is_a_Group}%
        If $A$ is a non-empty set, if $S_{A}$ is the set of all permutations of
        $A$, and if $\circ$ denotes function composition, then
        $S_{A},\circ)$ is a group.
    \end{theorem}
    \begin{proof}
        For $\circ$ is indeed a binary operation on $S_{A}$. There also
        exists a unital element, since $\textrm{Id}_{A}$ is a permutation on
        $A$. Lastly, if $f\in{S}_{A}$, then it is a bijection and thus there
        exists an inverse function $g:A\rightarrow{A}$. But the inverse of a
        permutation is a permutation, and thus $g\in{S}_{A}$. Thus,
        $(S_{A},\circ)$ is closed to inverses and is therefore a group
        (Def.~\ref{def:Group}).
    \end{proof}
    We will be most interested in case when $A=\mathbb{Z}_{n}$ for some
    $n\in\mathbb{N}$. The set of all permutations on a set $A$ is called the
    \textit{symmetric group}\index{Symmetric Group} of $A$.
    \begin{fdefinition}{Symmetric Group}{Symmetric_Group}
        The symmetric group of a set $A$ is the group $(S_{A}\circ)$
        of all permutations of $A$ under function composition $\circ$.
    \end{fdefinition}
    By Thm.~\ref{thm:Symmetric_Group_is_a_Group}, the symmetric group is a
    group. The reason we required the underlying set to be non-empty is because
    the set of perumtations of the empty set is empty, and thus $S_{\emptyset}$
    cannot be a group since groups are required to have a unital element, and
    thus cannot be empty.
    \begin{lexample}{Symmetric Group $S_{3}$}{Symmetric_Group_S3}
        We've seen the symmetric group for $\mathbb{Z}_{3}$ before and noted
        that it is isomorphic to $D_{6}$. Given a permutation $f\in{S}_{3}$, we
        can describe $f$ via the following matrix:
        \begin{equation}
            f=
            \begin{pmatrix}
                0&1&2\\
                1&0&2
            \end{pmatrix}
        \end{equation}
        The first row of the matrix is the input, and the second row is the
        output. This matrix tells us that $f$ can be defined as follows:
        \begin{equation}
            f(n)=
            \begin{cases}
                1,&n=0\\
                0,&n=1\\
                2,&n=2
            \end{cases}
        \end{equation}
        We have that there are $3!=6$ permutations on $\mathbb{Z}_{3}$, and we
        can list them as follows:
        \par
        \begin{subequations}
            \begin{minipage}[b]{0.49\textwidth}
                \centering
                \begin{align}
                    \textrm{Id}_{\mathbb{Z}_{3}}&=
                    \begin{pmatrix}
                        0&1&2\\
                        0&1&2
                    \end{pmatrix}\\
                    \alpha&=
                    \begin{pmatrix}
                        0&1&2\\
                        0&2&1
                    \end{pmatrix}\\
                    \beta&=
                    \begin{pmatrix}
                        0&1&2\\
                        2&0&1
                    \end{pmatrix}
                \end{align}
            \end{minipage}
            \hfill
            \begin{minipage}[b]{0.49\textwidth}
                \centering
                \begin{align}
                    \gamma&=
                    \begin{pmatrix}
                        0&1&2\\
                        1&0&2
                    \end{pmatrix}\\
                    \delta&=
                    \begin{pmatrix}
                        0&1&2\\
                        1&2&0
                    \end{pmatrix}\\
                    \epsilon&=
                    \begin{pmatrix}
                        0&1&2\\
                        2&1&0
                    \end{pmatrix}
                \end{align}
            \end{minipage}
        \end{subequations}
        \par\vspace{2.5ex}
        We can use this to compute compositions of permutations.
        \begin{equation}
            \beta\circ\delta=
            \begin{pmatrix}
                0&1&2\\
                2&0&1
            \end{pmatrix}
            \begin{pmatrix}
                0&1&2\\
                1&2&0
            \end{pmatrix}=
            \begin{pmatrix}
                0&1&2\\
                0&1&2
            \end{pmatrix}
            =\textrm{Id}_{\mathbb{Z}_{3}}
        \end{equation}
        This should be read as \textit{0 goes to 1 and 1 goes to 0},
        \textit{so 0 goes to 0}. That is, we read the $\delta$ matrix first,
        and then feed this result to the $\beta$ matrix. Similarly,
        1 goes to 2 and 2 goes to 1, so 1 goes to 1. Lastly, 2 goes to 0 and 0
        goes to 2, so 2 goes to 0. The resulting permutation is the identity
        permutation. Note that we are \textbf{not} performing matrix
        multiplication. On the one hand, we've yet to define matrix
        multiplication at this point, and on the other matrix multiplication is
        \textit{undefined} for matrices of these sizes. That is, we cannot
        multiply a $2\times{3}$ matrix by a $2\times{3}$ matrix in the usual
        fashion. This representation is simply to aid in ones understanding of
        groups of permutations. The symmetric group is non-Abelian, as is
        $D_{6}$ We can see that it is non-Abelian by considering
        $\alpha\circ\beta$ and $\beta\circ\alpha$. In $\alpha\circ\beta$ we have
        that 0 goes to 2 and 2 goes to 1, so 0 goes to 1. However in
        $\beta\circ\alpha$ we see that 0 goes to 0, and then 0 goes to 2, so 0
        goes to 2. Thus $\alpha\circ\beta\ne\beta\circ\alpha$, so
        $(S_{3},\circ)$ is not Abelian. We can form a new representation of
        $S_{3}$ to show that it is isomorphic to $D_{6}$. In fact, there are
        only two groups with 6 elements (up to isomorphism).
    \end{lexample}
    We do not have to consider \textit{all} permutations on a given group, and
    the more general \textit{group of permutations} is formed by considering
    subgroups of a symmetric.
    \begin{fdefinition}{Permutation Group}{Permutation_Group}
        A permutation group of a set $A$ is a subgroup of the symmetric group
        $(S_{A},\circ)$ on $A$.
    \end{fdefinition}
    What's remarkable is that \textit{every} group is a permutation group for
    some set $A$. This result is known as Cayley's
    Theorem\index{Cayley's Theorem} and will be proved shortly.
    \subsection{Finite Permutations}
        When dealing with permutations on a finite set, it is convenient to
        break up a given permutation into disjoint \textit{cycles}. For example,
        suppose we have the following permutation on $\mathbb{Z}_{8}$:
        \begin{equation}
            f=
            \begin{pmatrix}
                0&1&2&3&4&5&6&7\\
                2&5&6&4&0&7&3&1
            \end{pmatrix}
        \end{equation}
        We can draw this as two disjoint cycles as follows:
        \begin{figure}[H]
            \centering
            \captionsetup{type=figure}
            \begin{tikzpicture}[%
                ->-/.style={%
                    decoration={%
                        markings,
                        mark=at position .55 with \arrow{Stealth}
                    },
                    postaction={decorate}
                }
            ]
                \coordinate (0) at (0.000:2);
                \coordinate (2) at (72.00:2);
                \coordinate (6) at (144.0:2);
                \coordinate (3) at (216.0:2);
                \coordinate (4) at (288.0:2);

                \draw[fill=black] (0) circle (0.05);
                \draw[fill=black] (2) circle (0.05);
                \draw[fill=black] (6) circle (0.05);
                \draw[fill=black] (3) circle (0.05);
                \draw[fill=black] (4) circle (0.05);

                \node at (0.000:2.3) {$0$};
                \node at (72.00:2.3) {$2$};
                \node at (144.0:2.3) {$6$};
                \node at (216.0:2.3) {$3$};
                \node at (288.0:2.3) {$4$};

                % Draw the first cycle (its a pentagon)
                \draw[->-] (0) to (2);
                \draw[->-] (2) to (6);
                \draw[->-] (6) to (3);
                \draw[->-] (3) to (4);
                \draw[->-] (4) to (0);

                \begin{scope}[xshift=6cm]
                    \coordinate (1) at (90.00:2);
                    \coordinate (5) at (210.0:2);
                    \coordinate (7) at (330.0:2);

                    \draw[fill=black] (1) circle (0.05);
                    \draw[fill=black] (5) circle (0.05);
                    \draw[fill=black] (7) circle (0.05);

                    \node at (90.00:2.3) {$1$};
                    \node at (210.0:2.3) {$5$};
                    \node at (330.0:2.3) {$7$};

                    \draw[->-] (1) to (5);
                    \draw[->-] (5) to (7);
                    \draw[->-] (7) to (1);
                \end{scope}
            \end{tikzpicture}
            \caption{Cycle Diagram for a Permutation}
            \label{fig:Cycle_Diagram_of_Permutation}
        \end{figure}
        We can write this simply as the product of two cycle of the permutation:
        \begin{equation}
            f=(02634)(157)
        \end{equation}
        First we need to rigorously define a cycle.
        \begin{fdefinition}{Cycle Permutation}{Cycle_Permutation}
            A cycle permutation on a finite set $A$ is a permutation
            $f\in{S}_{A}$ such that there exists two disjoint subsets
            $M,N\subseteq{A}$ such that $M\cup{N}=A$,
            $f|_{N}=\textrm{Id}_{A}|_{N}$, and such that there 
        \end{fdefinition}
        \begin{theorem}
            Disjoint cycles commute.
        \end{theorem}
        \begin{theorem}
            If $A$ is a set and if $f$ is a permutation on $A$, then $f$ is
            the product of finitely many disjoint cycles. The product is unique.
        \end{theorem}
        \begin{fdefinition}{Transposition}{Transposition}
            A transposition is a cycle of length 2.
        \end{fdefinition}
        \begin{theorem}
            Every cycle is the product of transpositions.
        \end{theorem}
        \begin{theorem}
            If $\alpha$ is a cycle of length $s$, and if $\alpha^{2}$ is a
            cycle, then $s$ is odd.
        \end{theorem}
        \begin{theorem}
            If $\alpha$ is a permutation, then $\alpha^{2}$ is an even
            permutation.
        \end{theorem}
        The decomposition of a cycle into transpositions need not be unique, and
        even the number of transpositions in such a decomposition can vary.
        \begin{theorem}
            The identity is always the product of an even number of
            transpositions.
        \end{theorem}
        \begin{theorem}
            The number of transpositions in the decomposition of a permutation
            is either always odd or always even.
        \end{theorem}
        This always us to define the alternating group.
        \begin{fdefinition}{Alternating Group}{Alternating_Group}
            The alternating group on a set $B$ is the subgroup of $S_{B}$
            consisting of all even permutations. It is denoted $A_{B}$.
        \end{fdefinition}
