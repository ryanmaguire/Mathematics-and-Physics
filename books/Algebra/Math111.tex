%------------------------------------------------------------------------------%
\documentclass{article}                                                        %
%------------------------------Preamble----------------------------------------%
\makeatletter                                                                  %
    \def\input@path{{../../}}                                                  %
\makeatother                                                                   %
%---------------------------Packages----------------------------%
\usepackage{geometry}
\geometry{b5paper, margin=1.0in}
\usepackage[T1]{fontenc}
\usepackage{graphicx, float}            % Graphics/Images.
\usepackage{natbib}                     % For bibliographies.
\bibliographystyle{agsm}                % Bibliography style.
\usepackage[french, english]{babel}     % Language typesetting.
\usepackage[dvipsnames]{xcolor}         % Color names.
\usepackage{listings, lstlinebgrd}      % Verbatim-Like Tools.
\usepackage{mathtools, esint, mathrsfs} % amsmath and integrals.
\usepackage{amsthm, amsfonts}           % Fonts and theorems.
\usepackage{tabularx}
\usepackage{tcolorbox}                  % Frames around theorems.
\usepackage{upgreek}                    % Non-Italic Greek.
\usepackage{paracol}                    % Two-column styling.
\usepackage{wrapfig}                    % Wrap text around figure.
\usepackage{fmtcount, etoolbox}         % For the \book{} command.
\usepackage[newparttoc]{titlesec}       % Formatting chapter, etc.
\usepackage{titletoc}                   % Allows \book in toc.
\usepackage[nottoc]{tocbibind}          % Bibliography in toc.
\usepackage[titles]{tocloft}            % ToC formatting.
\usepackage{multicol, enumitem}         % Multi-column/enumerate.
\usepackage{import}                     % Import external files.
\usepackage{pgfplots, tikz}             % Drawing/graphing tools.
\usetikzlibrary{
    calc,                   % Calculating right angles and more.
    angles,                 % Drawing angles within triangles.
    arrows.meta,            % Latex and Stealth arrows.
    quotes,                 % Adding labels to angles.
    positioning,            % Relative positioning of nodes.
    decorations.markings,   % Adding arrows in the middle of a line.
    patterns,
    arrows,
    shapes,
    shapes.geometric,
    cd,
    hobby,
    babel
}                                       % Libraries for tikz.
\pgfplotsset{compat=1.9}                % Version of pgfplots.
\usepackage[font=scriptsize,
            labelformat=simple,
            labelsep=colon]{subcaption} % Subfigure captions.
\usepackage[font={scriptsize},
            hypcap=true,
            labelsep=colon]{caption}    % Figure captions.
\usepackage{hyperref}                   % Allows for hyperlinks.
\hypersetup{
    colorlinks=true,
    linkcolor=blue,
    filecolor=magenta,
    urlcolor=Cerulean,
    citecolor=SkyBlue
}                           % Colors for hyperref.
\usepackage[toc,acronym,nogroupskip]{glossaries} % Glossaries and acronyms.
\usepackage[subpreambles=false]{standalone}      % Complileable sub files.

% Various font stuff from kiwi.
% Use this for Times text and Computer Modern math
%\usepackage{times}

% Quite nice
%\usepackage[charter, greekfamily=, greekuppercase=italicized]{mathdesign}
%\usepackage[utopia, greekuppercase=italicized]{mathdesign}    % Math is narrower

% Use this for Times text and math
%\usepackage{newtxtext}
%\usepackage[libertine,cmintegrals]{newtxmath}
%\usepackage{fix-cm}

%\usepackage{txfontsb}
% or
%\usepackage{mathptmx}

%\usepackage[scaled=0.92]{helvet}
%\renewcommand{\rmdefault}{ptm}

%\usepackage{mathpazo}    % add possibly `sc` and `osf` options
%\usepackage{eulervm}

%\usepackage{fourier}
%\renewcommand{\rmdefault}{ptm}
%\usepackage{mathptm}

%\usepackage{fontspec}
%\setmainfont{lmodern}

%\usepackage[varg]{txfonts}
%\usepackage{fouriernc}
%\usepackage{mathpazo}

%\usepackage{bookman}
%\usepackage[scaled]{uarial}
%\usepackage[scaled]{helvet}
%\renewcommand*\familydefault{\sfdefault}
%\usepackage[math]{anttor}

%\newcommand\fgeorgia{\fontfamily{jvn}\selectfont}
%\newcommand\ftimes{\fontfamily{ptm}\selectfont}
%\newcommand\fhelvetica{\fontfamily{phv}\selectfont}
%\newcommand\fcourier{\fontfamily{pcr}\selectfont}
%\newcommand\fbookman{\fontfamily{pbk}\selectfont}
%\newcommand\fnewcentury{\fontfamily{pnc}\selectfont}
%\newcommand\fpalatino{\fontfamily{ppl}\selectfont}
%\newcommand\favantgarde{\fontfamily{pag}\selectfont}
%\newcommand\fnormal{\normalfont}
%\newcommand\fsize[1]{\ifnum#1>0\fontsize{#1}{#1}\selectfont\else\normalsize\fi}
%------------------------Theorem Styles-------------------------%
% Define theorem style for default spacing and normal font.
\newtheoremstyle{normal}
    {\topsep}               % Amount of space above the theorem.
    {\topsep}               % Amount of space below the theorem.
    {}                      % Font used for body of theorem.
    {}                      % Measure of space to indent.
    {\bfseries}             % Font of the header of the theorem.
    {}                      % Punctuation between head and body.
    {.5em}                  % Space after theorem head.
    {}

% Define theorem style for default spacing with italicized font.
\newtheoremstyle{normalit}{\topsep}{\topsep}
                {\itshape}{}{\bfseries}{}{.5em}{}

% Italic header environment.
\newtheoremstyle{thmit}{\topsep}{\topsep}{}{}{\itshape}{}{0.5em}{}

% Define italicized environments.
\theoremstyle{normalit}
\newtheorem{theorem}{Theorem}[section]
\newtheorem{lemma}{Lemma}[section]
\newtheorem{corollary}{Corollary}[section]
\newtheorem{proposition}{Proposition}[section]
\newtheorem*{theorem*}{Theorem}

% Define environments with italic headers.
\theoremstyle{thmit}
\newtheorem*{solution}{Solution}
\newtheorem*{fsolution}{Solution}

% Define default environments.
\theoremstyle{normal}
\newtheorem{example}{Example}[section]
\newtheorem{definition}{Definition}[section]
\newtheorem{problem}{Problem}[section]
\newtheorem{question}{Question}[section]
\newtheorem{remark}{Remark}[section]
\newtheorem{properties}{Properties}[section]
\newtheorem{notation}{Notation}[section]
\newtheorem{axiom}{Axiom}[section]
\newtheorem*{properties*}{Properties}
\newtheorem*{remark*}{Remark}
\newtheorem*{definition*}{Definition}
\theoremstyle{plain}

% Define framed environment.
\tcbuselibrary{most}
\newtcbtheorem[use counter*=theorem]{ftheorem}{Theorem}%
    {colback=green!5,colframe=green!35!black,
     fonttitle=\bfseries\upshape}{th}

\newtcbtheorem[use counter*=example]{fdefinition}{Definition}%
    {fonttitle=\bfseries\upshape,
     colback=blue!5!white,colframe=blue!75!black}{def}

\newtcbtheorem[use counter*=example]{fexample}{Example}%
    {fonttitle=\bfseries\upshape,
     colback=red!5!white,colframe=red!75!black}{ex}

\newtcbtheorem[use counter*=notation]{fnotation}{Notation}%
    {fonttitle=\bfseries\upshape,
     colback=SeaGreen!5!white,colframe=SeaGreen!75!black}{ex}

\newtcbtheorem[use counter*=corollary]{fcorollary}{Corollary}%
    {fonttitle=\bfseries\upshape,
     colback=Orchid!5!white,colframe=Orchid!75!black}{ex}

\newenvironment{bproof}{\textit{Proof.}}{\hfill$\square$}
\tcolorboxenvironment{bproof}{blanker,breakable,left=5mm,
                             before skip=10pt,after skip=10pt,
                             borderline west={1mm}{0pt}{red}}
\tcolorboxenvironment{fsolution}
    {enhanced jigsaw,colframe=cyan,interior hidden,breakable}

%--------------------Declared Math Operators--------------------%
\DeclareMathOperator{\Refl}{Refl}           % Reflection operator.
\DeclareMathOperator{\Span}{Span}           % Span of a set of vectors.
\DeclareMathOperator{\Card}{Card}           % Cardinality of set.
\DeclareMathOperator{\Ord}{Ord}             % Ordinal of ordered set.
\DeclareMathOperator{\Tr}{Tr}               % Trace of matrix.
\DeclareMathOperator{\adjoint}{adj}         % Adjoint of matrix.
\DeclareMathOperator{\rk}{rk}               % Rank of operator.
\DeclareMathOperator{\nul}{nul}             % Null space of operator.
\DeclareMathOperator{\sgn}{sgn}             % Sign of a number.
\DeclareMathOperator{\multideg}{mutlideg}   % Multi-Degree (Graphs).
\DeclareMathOperator{\GCD}{GCD}             % Greatest common denominator.
\DeclareMathOperator{\LM}{LM}               % Leading monomial
\DeclareMathOperator{\LC}{LC}               % Leading coefficient.
\DeclareMathOperator{\LT}{LT}               % Leading term.
\DeclareMathOperator{\LCM}{LCM}             % Least common multiple.
\DeclareMathOperator{\Mon}{Mon}             % Monomial.
\DeclareMathOperator{\Spec}{Spec}           % Spectrum.
\DeclareMathOperator{\proj}{proj}           % Projection.
\DeclareMathOperator{\comp}{comp}           % Component.
\DeclareMathOperator{\sinc}{sinc}           % Sinc function.
\DeclareMathOperator{\Ima}{Im}              % Image of operator.
\DeclareMathOperator{\Prin}{Prin}           % Principal value.
\DeclareMathOperator{\Mod}{mod}             % Modulus.
%------------------------New Commands---------------------------%
\DeclarePairedDelimiter\norm{\lVert}{\rVert}
\DeclarePairedDelimiter\ceil{\lceil}{\rceil}
\DeclarePairedDelimiter\floor{\lfloor}{\rfloor}
\newcommand*\diff{\mathop{}\!\mathrm{d}}
\newcommand*\Diff[1]{\mathop{}\!\mathrm{d^#1}}
\renewcommand{\mod}{\ \Mod}
\renewcommand*{\glstextformat}[1]{\textcolor{RoyalBlue}{#1}}
\renewcommand{\glsnamefont}[1]{\textbf{#1}}
\renewcommand\labelitemii{$\circ$}
\renewcommand\thesubfigure{\arabic{chapter}.\arabic{figure}}
\renewcommand\thesubfigure{%
    \arabic{chapter}.\arabic{figure}.\arabic{subfigure}}
\addto\captionsenglish{\renewcommand{\figurename}{Fig.}}
%------------------------Book Command---------------------------%
\makeatletter
\renewcommand\@pnumwidth{1cm}
\newcounter{book}
\renewcommand\thebook{\@Roman\c@book}
\newcommand\book{%
    \if@openright
        \cleardoublepage
    \else
        \clearpage
    \fi
    \thispagestyle{plain}%
    \if@twocolumn
        \onecolumn
        \@tempswatrue
    \else
        \@tempswafalse
    \fi
    \null\vfil
    \secdef\@book\@sbook
}
\def\@book[#1]#2{%
    \ifnum \c@secnumdepth >-3\relax
        \refstepcounter{book}%
        \addcontentsline{toc}{book}{
            \bookname\ \thebook:\hspace{1em}#1
        }
    \else
        \addcontentsline{toc}{book}{#1}%
    \fi
    \markboth{}{}%
    {\centering
     \interlinepenalty \@M
     \normalfont
     \ifnum \c@secnumdepth >-2\relax
       \huge\bfseries \bookname\nobreakspace\thebook
       \par
       \vskip 20\p@
     \fi
     \Huge \bfseries #2\par}%
    \@endbook}
\def\@sbook#1{%
    {\centering
     \interlinepenalty \@M
     \normalfont
     \Huge \bfseries #1\par}%
    \@endbook}
\def\@endbook{
    \vfil\newpage
        \if@twoside
            \if@openright
                \null
                \thispagestyle{empty}%
                \newpage
            \fi
        \fi
        \if@tempswa
            \twocolumn
        \fi
}
\newcommand*\l@book[2]{%
    \ifnum \c@tocdepth >-2\relax
        \addpenalty{-\@highpenalty}%
        \addvspace{2.25em \@plus\p@}%
        \setlength\@tempdima{3em}%
        \begingroup
            \parindent \z@ \rightskip \@pnumwidth
            \parfillskip -\@pnumwidth
            {
                \leavevmode
                \Large \bfseries #1\hfil \hb@xt@\@pnumwidth{
                    \hss #2
                }
            }
            \par
            \nobreak
            \global\@nobreaktrue
            \everypar{\global\@nobreakfalse\everypar{}}%
        \endgroup
    \fi}
\newcommand\bookname{Book}
\renewcommand{\thebook}{\texorpdfstring{\Numberstring{book}}{book}}
\providecommand*{\toclevel@book}{-2}
\makeatother
\titlecontents{chapter}[0pt]
    {\bfseries}
    {\chaptername\ \thecontentslabel:\quad}
    {}
    {\hfill\contentspage}
\titleformat{\part}[display]
    {\Large\bfseries}
    {\partname\nobreakspace\thepart}
    {0mm}
    {\Huge\bfseries}
    \titlecontents{part}[0pt]
    {\large\bfseries}
    {\partname\ \thecontentslabel: \quad}
    {}
    {\hfill\contentspage}
\newcommand{\MarkRightAngle}[4][.3cm]
    {\coordinate (tempa) at ($(#3)!#1!(#2)$);
     \coordinate (tempb) at ($(#3)!#1!(#4)$);
     \coordinate (tempc) at ($(tempa)!0.5!(tempb)$);%midpoint
     \draw (tempa) -- ($(#3)!2!(tempc)$) -- (tempb);}
%--------------------------LENGTHS------------------------------%
% Spacings for the Table of Contents.
\addtolength{\cftsecnumwidth}{1ex}
\addtolength{\cftsubsecindent}{1ex}
\addtolength{\cftsubsecnumwidth}{1ex}
\addtolength{\cftfignumwidth}{1ex}
\addtolength{\cfttabnumwidth}{1ex}

% Spacing for multi-column and enumerate environments.
\setlength{\multicolsep}{6pt}
\setlist[enumerate]{itemsep=0pt,topsep=3pt}

% Indent and paragraph spacing.
\setlength{\parindent}{0em}
\setlength{\parskip}{0em}                                                           %
%----------------------------Main Document-------------------------------------%
\begin{document}
    \title{Galois Theory}
    \author{Ryan Maguire}
    \date{\vspace{-5ex}}
    \maketitle
    \section{Fields}
        \subsection{What's the Point?}
            From the historical perspective, we can start with numbers. There's
            the standard inclusion:
            \begin{equation}
                \mathbb{N}^{+}\subseteq\mathbb{N}\subseteq
                \mathbb{Z}\subseteq\mathbb{Q}\subseteq\mathbb{R}
                \subseteq\mathbb{C}
            \end{equation}
            Suppose we are given the equation $2+x=1$. If we only know about
            the positive integers, then we cannot solve this equation. We thus
            need to introduce negative integers. Next we could write $2x=1$,
            and we are now forced to introduce the rational numbers. In ancient
            Greece, the solution to $x^{2}=2$ was proved to be irrational, and
            thus we must go beyond $\mathbb{Q}$ and develop the real numbers.
            Lastly, a part studied in Italy in the Renaissance era, are
            equations like $x^{2}=\minus{1}$. There are no real solutions to
            this, and so we must invent $\mathbb{C}$. The complex numbers are
            the set $\mathbb{C}$ of the form:
            \begin{equation}
                \mathbb{C}=\{\,x+iy\;|\;x,y\in\mathbb{R}\,\}
            \end{equation}
            where $i^{2}=\minus{1}$, by definition, which is the solution to the
            equation $z^{2}+1=0$. This equation has no solutions in $\mathbb{R}$
            and so $i$ is not a real number, and hence is called the
            \textit{imaginary} unit. We can picture complex numbers by use of
            the plane $\mathbb{R}^{2}$. But there's nothing too special about
            the equation $z^{2}+1=0$, and we can consider $z^{2}+z+1=0$ and
            again we can ask if this has real solutions. Unlike the first
            equation, it's not so obvious that this has no real solution. We
            can look at the quadratic formula, and in particular the discriment,
            obtaining:
            \begin{equation}
                \Delta=b^{2}-4ac=1-4=\minus{2}
            \end{equation}
            Since this is negative, there are no real solutions, and hence
            $z^{2}+z+1$ has no solution in $\mathbb{R}$. It does have roots in
            $\mathbb{C}$:
            \begin{subequations}
                \begin{align}
                    \omega&=\minus\frac{1}{2}+\frac{\sqrt{3}}{2}i\\
                    \overline{\omega}&=\minus\frac{1}{2}-\frac{\sqrt{3}}{2}i
                \end{align}
            \end{subequations}
            We can further consider the set $\mathbb{R}[w]$ defined by:
            \begin{equation}
                \mathbb{R}[\omega]=\{\,x+y\omega\;|\;x,y\in\mathbb{R}\,\}
            \end{equation}
            This has a nice field structure, like $\mathbb{C}$, and indeed this
            is equal to $\mathbb{C}$. That is, $\mathbb{R}[\omega]=\mathbb{C}$.
            We can see this since $\mathbb{R}[\omega]$ is a subspace of
            $\mathbb{C}$ with a basis consisting of two elements:
            $\{1,\omega\}$, and thus has the same dimension as $\mathbb{C}$.
            Hence, it is equal to the whole thing. We can be even more explicit:
            \begin{align}
                x+y\omega&=x+y\big(\minus\frac{1}{2}+\frac{\sqrt{3}}{2}i\big)\\
                &=\big(x-\frac{1}{2}y\big)+\big(\frac{\sqrt{3}}{2}\big)i
            \end{align}
            And this is of the form $x'+y'i$, where:
            \begin{align}
                x'&=x-\frac{1}{2}y\\
                y'&=\frac{\sqrt{3}}{2}y
            \end{align}
            Since this is always solvable for both $(x,y)$ and $(x',y')$, the
            two spaces are the same. And indeed, we can generalize. If
            $f(x)=ax^{2}+bx+c$, with $a,b,c\in\mathbb{R}$ are such that
            $b^{2}-4ac<0$, then defining:
            \begin{equation}
                \alpha=\frac{\minus{b}+\sqrt{b^{2}-4ac}}{2a}
            \end{equation}
            which is a complex root of $f$, then
            $\mathbb{R}[\alpha]=\mathbb{C}$. This shows there's nothing too
            special about $i$: extending $\mathbb{R}$ with any complex root of
            a quadratic gives the entirety of $\mathbb{C}$, we need not only
            choose $z^{2}+1=0$. Even if we were to stick with this polynomial,
            we could still choose $\minus{i}$, since this too is a solution.
            Choosing $i$ over $\minus{i}$ seems to purely be an accident of
            history. Going from one choice to another is an
            $\mathbb{R}$ automorphism: $x+iy\mapsto{x}-iy$. An $\mathbb{R}$
            automorphism is a bijective ring homomoprhism
            $f:\mathbb{R}\rightarrow\mathbb{R}$. That is, an isomorphism from
            $\mathbb{R}$ to itself:
            \begin{align}
                f(z_{1}+z_{2})&=f(z_{1})+f(z_{2})\\
                f(z_{1}z_{2})&=f(z_{1})f(z_{2})\\
                f(1)=1
            \end{align}
            The automorphism $x+iy\mapsto{x}-iy$ is called complex conjugation.
            If we don't like $i$, and have a complex number such as $\omega$,
            we can still take as an $\mathbb{R}$ automorphism the function
            $\sigma:\mathbb{C}\rightarrow\mathbb{C}$ where
            $x+y\omega\mapsto{x}+y\overline{\omega}$. As it turns out, this is
            the same as the automorphism $x+iy\mapsto{x}-iy$ since we can write:
            \begin{equation}
                i=\frac{1+2\omega}{\sqrt{3}}
            \end{equation}
            This is the object we wish to stress as the important part of the
            theory of complex numbers. Neither $i$ nor $\omega$ are too
            important, but rather the notion of complex conjugation is.
            The group of $\mathbb{R}$ automorphisms of $\mathbb{C}$ is equal to:
            \begin{equation}
                \textrm{Aut}_{\mathbb{R}}(\mathbb{C})
                    =\{\textrm{id}_{\mathbb{R}},\sigma\}
            \end{equation}
            Where $\sigma$ is complex conjugation. That is, $\sigma$ is the
            unique non-trivial $\mathbb{R}$ automorphism that has the property
            that it exchanges the roots of any $f(x)=ax^{2}+bx+c$ with
            $b^{2}-4ac<0$. The group structure comes from function composition.
            Since function composition is always associative, since the identity
            map is an automorphism, and since bijections have inverse elements,
            this is indeed a group. We can summarize all of this as follows:
            The roots of any real polynomial are either real or come in complex
            conjugate pairs.
            \par\hfill\par
            Looking at the numerology of the problem, there seems to be
            something special about the number two (2). This is the size of the
            automorphism group $\textrm{Aut}_{\mathbb{R}}(\mathbb{C})$, and
            this is also the dimension of $\mathbb{C}$, and lastly it is the
            degree of $\mathbb{C}$ over $\mathbb{R}$: $[\mathbb{C}:\mathbb{R}]$.
            More generally, consider any field $\mathbb{F}$ with characteristic
            not equal to 2 (that is, $1+1\ne{0}$), and any function
            $f(x)=ax^{2}+bx+c$, $a,b,c\in\mathbb{F}$ such that $f(x)=0$ has no
            solutions in $\mathbb{F}$. For example, $\mathbb{R}$ with
            $f(x)=x^{2}+1$, of $\mathbb{Q}$ with $f(x)=x^{2}-2$. If we have
            such conditions, then there is a field $\mathbb{K}$ and an inclusion
            $\mathbb{F}\subseteq\mathbb{K}$ making $\mathbb{F}$ a subfield,
            such that $f(x)=a(x-\alpha)(x-\beta)$, where
            $\alpha,\beta\in\mathbb{K}$. Moreover,
            $\mathbb{K}=\mathbb{F}[\alpha]$. That is:
            \begin{equation}
                \mathbb{K}=\{\,x+y\alpha\;|\;x,y\in\mathbb{F}\}
            \end{equation}
            Similarly, $\mathbb{K}=\mathbb{F}[\beta]$. Lastly, the automorphism
            group is
            \begin{equation}
                \textrm{Aut}_{\mathbb{F}}(\mathbb{K})
                =\{\,\textrm{id}_{\mathbb{F}},\sigma\}
            \end{equation}
            where $\sigma$ is the unique automorphism such that
            $\sigma(\alpha=\beta)$. The proof is simply an application of the
            quadratic formula, where we invoke the fact that $2\ne{0}$ in a
            field whose characteristic is not 2.
        \subsection{Cubic Equations and Higher}
            In the $16^{th}$ century the Italians were able to solve the cubic
            equation: $x^{3}+px-q=0$. This may not look like the general cubic,
            but since we are interested in roots we may always divide off by
            the leading coefficient of $x^{3}$, and the quadratic term may be
            absorbed by completing the square, and thus any cubic can be
            written in such a form. The solution is:
            \begin{equation}
                Yeah
            \end{equation}
            By the $18^{th}$ century the Italians were able to solve the general
            quartic equation. The next natural question is the solution to the
            quintic, but this was shown not to exist. The Abel-Ruffini theorem
            shows that the general quintic equation can not be solved using
            nested radicals. Galois went to prove that a polynomial has a root
            that can be written in terms of nested radicals if and only if
            $K/F$, the splitting field, has an automorphism group
            $\textrm{Aut}_{F}(K))$ that is solveable.
        \subsection{Some Reminders}
            \begin{definition}
                A field $(\mathbb{F},+,\cdot)$ is an Abelian group
                $(\mathbb{F},+)$ such that $(F^{*}\setminus\{0\},\cdot)$ is an
                Abelian group as well. This is the group of \textit{units}.
            \end{definition}
            \begin{example}
                The classic exmaples are $\mathbb{Q}$, $\mathbb{R}$, and
                $\mathbb{C}$, as well as the finite fields $\mathbb{F}_{p}$,
                also commonly denoted $\mathbb{Z}/p\mathbb{Z}$ or simply
                $\mathbb{Z}_{p}$.
            \end{example}
            \begin{definition}
                A field extension of a field $F$ is a field $K$ such that
                $F\subseteq{K}$. We may also say that $F$ is a subfield of $K$.
            \end{definition}
            We often denote that $K$ is a field extension of $F$ by writing
            $K/F$. This is not to denote a quotient or anything of that manner
            and is simply to denote that $F$ sis a subfield of $K$.
            \begin{example}
                $\mathbb{C}$ is a field extension of $\mathbb{R}$ since both are
                fields and $\mathbb{R}\subseteq\mathbb{C}$. We can go backwards,
                thinking of $\mathbb{R}$ as a field extension $\mathbb{R}$.
            \end{example}
            Also important, if $K$ is a field extension of $F$, $K/F$, then
            $K$ has the structure of an $F$ vector space. That is, $K$ can be
            seen as a vector space over $F$. One thing that we write is this
            bracket notation $[K:F]$, which again is not to be confused with
            the notation found in groups about the cardinality of certain
            things. $[K:F]$ is the simply the dimesnion of the vector space
            $K$ over $F$:
            \begin{equation}
                [K:F]=\textrm{dim}_{K}(F)
            \end{equation}
            This is also called the degree of the extension $K/F$. If the
            dimension is finite, $[K:F]<\infty$, we say that $K/F$ is a finite
            extension.
            \begin{example}
                $\mathbb{C}$ is a two dimensional vector space over $\mathbb{R}$
                and thus $[\mathbb{C},\mathbb{R}]=2$. To see this, use
                $\{1,i\}$ as a basis.
            \end{example}
            \begin{theorem}
                Any countable dimensional vector space over a countable field is
                also countable.
            \end{theorem}
            \begin{example}
                Using this theorem shows that $\mathbb{R}$, as a vector space
                over $\mathbb{Q}$, in not only an infinite dimensional vector
                space, but also has an uncountably infinite basis.
                Thus, $[\mathbb{R}:\mathbb{Q}]$ is uncountably infinite.
            \end{example}
            \begin{example}
                Consider $\mathbb{Q}[\sqrt{2}]$, defined by:
                \begin{equation}
                    \mathbb{Q}[\sqrt{2}]=\{x+y\sqrt{2}\;|\;x,y\in\mathbb{Q}\,\}
                \end{equation}
                This is a subfield of $\mathbb{R}$,
                $\mathbb{Q}[\sqrt{2}]\subseteq\mathbb{R}$. Addition and
                multiplication are easy enough to see, and 0 and 1 are contained
                in there, we need only check multiplicative inverses. But:
                \begin{equation}
                    (x+\sqrt{2}y)^{\minus{1}}=\frac{x-\sqrt{2}y}{x^{2}-2y^{2}}
                \end{equation}
                And $x^{2}-2y^{2}$ is only zero when $x=y=0$, since if
                $x^{2}-2y^{2}=0$, then rearrange this to obtain $q^{2}=2$. But
                by the arguments of the ancient Greeks, there is no rational
                number whose square is 2, and thus the denominator is never
                zero for non-zero rational ordered pairs.
            \end{example}
            \begin{example}
                $\mathbb{R}/\mathbb{Q}[\sqrt{2}]$ is uncountably infinite, but
                $\mathbb{Q}[\sqrt{2}]/\mathbb{Q}$ has degree 2 with a basis
                $\{1,\sqrt{2}\}$.
            \end{example}
        \subsection{Polynomials}
            We use $F[x]$ to denote the ring of polynomials with coefficients in
            $F$. For example:
            \begin{equation}
                f(x)=a_{n}x^{n}+a_{n-1}x^{n-1}+\cdots+a_{1}x+x_{0}
                \quad\quad
                a_{k}\in{F}
            \end{equation}
            Then $f\in{F}[x]$. The degree of a polynomial is the largest power
            of the polynomial with non-zero coefficient. Some things can be said
            about the degree of polynomials:
            \begin{align}
                \textrm{deg}(f+g)&\leq
                    \textrm{max}\{\textrm{deg}(f),\textrm{def}(g)\}\\
                \textrm{deg}(fg)&=\textrm{deg}(f)+\textrm{deg}(g)
                \quad\quad
                f,g\ne{0}
            \end{align}
            The degree of a polynomial is zero if and only if the polynomial is
            constant. Since $F[x]$ has a ring structure, $F[x]^{*}$ can be seen
            as the set of all non-zero constant polynomials.
            \begin{theorem}
                $F[x]$ is a Euclidean domain. That is, for any polynomial
                $f\in{F}[x]$ and for any non-zero $g\in{F}[x]$, there exist
                unique polynomials $r,q\in{F}[x]$ such that $f=qg+r$ where
                either $r=0$ or $\textrm{deg}(r)<\textrm{deg}(g)$.
            \end{theorem}
            \begin{theorem}
                The polynomial ring $F[x]$ is a principal ideal domain. That is,
                every ideal $I\subseteq{F}[x]$ is principal. That is, every
                ideal is generated by a single element.
            \end{theorem}
            \begin{theorem}
                Every Euclidean domain is a principle ideal.
            \end{theorem}
            Thus, there is a bijection between ideals $I\subseteq{F}[x]$ and
            monic polynomials in $F[x]$. Recall that if $R$ is a commutatie ring
            with unity, then $r\in{R}$ is called irreducible if $r\ne{0}$, $r$
            not a unit, and if $r=ab$, then either $a$ or $b$ is a unit. We take
            this definition to exclude some trivialities. For example, in
            $\mathbb{Z}$, 3 is irreducible, however
            $3=(\minus{1})\cdot(\minus{3})$. We don't care about this product,
            since $\minus{1}$ is a unit. Moreover, an element $r\in{R}$ is
            prime if $(r)\subseteq{R}$ is a prime ideal. That is if
            $r$ divides $ab$, then either $r$ divides $a$ or $r$ divides $b$.
            By divides, $r|a$, we mean that $a=r\cdot{s}$ for some $s\in{R}$.
            \begin{example}
                If $F$ is a field, $f\in{F}[x]$, then $f$ is irreducible if and
                only if $f$ is not the product of two polynomials with degrees
                strictly less than $f$. That is, if $f=gh$, then one of these
                must be a constant.
            \end{example}
            \begin{example}
                In $\mathbb{Z}$, prime if and only if irreducible.
            \end{example}
            \begin{theorem}
                If $R$ is a integral domain, and if $r$ is prime, then it is
                irreducible. That is, if there are no zero divisors then prime
                implies irreducible.
            \end{theorem}
            \begin{theorem}
                If $R$ is a principal ideal domain and if $r$ is irreducible,
                then the ideal generated by $r$ is maximal.
            \end{theorem}
            \begin{theorem}
                A maximal ideal is a prime ideal.
            \end{theorem}
            Recale that an ideal is called prime if $R/I$ is a domain. That is,
            if $ab\in{I}$, then either $a\in{I}$ or $b\in{I}$. A maximal ideal
            is and ideal that has no proper ideals between it and the entire
            ring. Another way to say this is that $R/I$ is a field. In other
            words, if $I\subseteq{J}\subseteq{R}$, then $I=J$. Using this we see
            that a maximal ideal is prime since $R/I$ will be a field, which is
            certainly an integral domain.
            \begin{theorem}
                The fourth isomorphism theorem says that if $I\subseteq{R}$ is
                an ideal, then there is a bijection between ideals containing
                $I$, $I\subseteq{J}\subseteq{R}$, and ideals of $R/I$.
            \end{theorem}
            \begin{theorem}
                $f\in{F}[x]$ is irreducible if and only if $F[x]/(f)$ is a
                field.
            \end{theorem}
            Note that $F$ can been seen as a subfield of $F[x]/(f)$ since
            $F$ can be identitified with all constant polynomials, which can
            further be seen tolive inside of $F[x]/(f)$.
            \begin{theorem}
                If $\overline{g}\in{F}[x]/(f)$ then there exists a unique
                $g_{0}\in{F}[x]$ such that$\textrm{deg}(g_{0})<\textrm{deg}(f)$,
                with $\overline{g_{0}}=\overline{g}$.
            \end{theorem}
            If $n$ is the degree of $f$, then the set
            $\{\overline{1},\overline{x},\dots,\overline{x^{n-1}}\}$ is a basis
            for $F[x]/(f)$ over $F$.
            \begin{theorem}
                If $f$ is irreducible of degree $n$, then
                $F[x]/(f)$ is a field extension of $F$ of degree $n$.
            \end{theorem}
            \begin{example}
                In $\mathbb{R}$, the polynomial $f(x)=x^{2}+1$ is irreducible
                since it cannot be factors any further. Thus
                $\mathbb{R}[x]/(x^{2}+1)$ is a field extension of $\mathbb{R}$
                of degree 2.
            \end{example}
            \begin{theorem}
                $\mathbb{R}[x]/(x^{2}+1)$ is isomorphic to $\mathbb{C}$.
            \end{theorem}
            \begin{proof}
                For since $\{\overline{1},\overline{x}\}$ is a basis, we have:
                \begin{equation}
                    \mathbb{R}[x]/(x^{2}+1)=
                        \{a\overline{1}+b\overline{x}\;|\;a,b\in\mathbb{R}\}
                \end{equation}
                So we trivial map $a\overline{1}+b\overline{x}$ to $a+bi$.
            \end{proof}
            \begin{example}
                Consider now $\mathbb{Q}[x]$ with $x^{2}-2$. This is irreducible
                since it cannot be factor ($\sqrt{2}$ is irrational). Then
                $\mathbb{Q}[x]/(x^{2}-2)$ is isomorphic to
                $\mathbb{Q}[\sqrt{2}]$.
            \end{example}
\end{document}