%------------------------------------------------------------------------------%
\documentclass{article}                                                        %
%------------------------------Preamble----------------------------------------%
\makeatletter                                                                  %
    \def\input@path{{../../}}                                                  %
\makeatother                                                                   %
%---------------------------Packages----------------------------%
\usepackage{geometry}
\geometry{b5paper, margin=1.0in}
\usepackage[T1]{fontenc}
\usepackage{graphicx, float}            % Graphics/Images.
\usepackage{natbib}                     % For bibliographies.
\bibliographystyle{agsm}                % Bibliography style.
\usepackage[french, english]{babel}     % Language typesetting.
\usepackage[dvipsnames]{xcolor}         % Color names.
\usepackage{listings}                   % Verbatim-Like Tools.
\usepackage{mathtools, esint, mathrsfs} % amsmath and integrals.
\usepackage{amsthm, amsfonts, amssymb}  % Fonts and theorems.
\usepackage{tcolorbox}                  % Frames around theorems.
\usepackage{upgreek}                    % Non-Italic Greek.
\usepackage{fmtcount, etoolbox}         % For the \book{} command.
\usepackage[newparttoc]{titlesec}       % Formatting chapter, etc.
\usepackage{titletoc}                   % Allows \book in toc.
\usepackage[nottoc]{tocbibind}          % Bibliography in toc.
\usepackage[titles]{tocloft}            % ToC formatting.
\usepackage{pgfplots, tikz}             % Drawing/graphing tools.
\usepackage{imakeidx}                   % Used for index.
\usetikzlibrary{
    calc,                   % Calculating right angles and more.
    angles,                 % Drawing angles within triangles.
    arrows.meta,            % Latex and Stealth arrows.
    quotes,                 % Adding labels to angles.
    positioning,            % Relative positioning of nodes.
    decorations.markings,   % Adding arrows in the middle of a line.
    patterns,
    arrows
}                                       % Libraries for tikz.
\pgfplotsset{compat=1.9}                % Version of pgfplots.
\usepackage[font=scriptsize,
            labelformat=simple,
            labelsep=colon]{subcaption} % Subfigure captions.
\usepackage[font={scriptsize},
            hypcap=true,
            labelsep=colon]{caption}    % Figure captions.
\usepackage[pdftex,
            pdfauthor={Ryan Maguire},
            pdftitle={Mathematics and Physics},
            pdfsubject={Mathematics, Physics, Science},
            pdfkeywords={Mathematics, Physics, Computer Science, Biology},
            pdfproducer={LaTeX},
            pdfcreator={pdflatex}]{hyperref}
\hypersetup{
    colorlinks=true,
    linkcolor=blue,
    filecolor=magenta,
    urlcolor=Cerulean,
    citecolor=SkyBlue
}                           % Colors for hyperref.
\usepackage[toc,acronym,nogroupskip,nopostdot]{glossaries}
\usepackage{glossary-mcols}
%------------------------Theorem Styles-------------------------%
\theoremstyle{plain}
\newtheorem{theorem}{Theorem}[section]

% Define theorem style for default spacing and normal font.
\newtheoremstyle{normal}
    {\topsep}               % Amount of space above the theorem.
    {\topsep}               % Amount of space below the theorem.
    {}                      % Font used for body of theorem.
    {}                      % Measure of space to indent.
    {\bfseries}             % Font of the header of the theorem.
    {}                      % Punctuation between head and body.
    {.5em}                  % Space after theorem head.
    {}

% Italic header environment.
\newtheoremstyle{thmit}{\topsep}{\topsep}{}{}{\itshape}{}{0.5em}{}

% Define environments with italic headers.
\theoremstyle{thmit}
\newtheorem*{solution}{Solution}

% Define default environments.
\theoremstyle{normal}
\newtheorem{example}{Example}[section]
\newtheorem{definition}{Definition}[section]
\newtheorem{problem}{Problem}[section]

% Define framed environment.
\tcbuselibrary{most}
\newtcbtheorem[use counter*=theorem]{ftheorem}{Theorem}{%
    before=\par\vspace{2ex},
    boxsep=0.5\topsep,
    after=\par\vspace{2ex},
    colback=green!5,
    colframe=green!35!black,
    fonttitle=\bfseries\upshape%
}{thm}

\newtcbtheorem[auto counter, number within=section]{faxiom}{Axiom}{%
    before=\par\vspace{2ex},
    boxsep=0.5\topsep,
    after=\par\vspace{2ex},
    colback=Apricot!5,
    colframe=Apricot!35!black,
    fonttitle=\bfseries\upshape%
}{ax}

\newtcbtheorem[use counter*=definition]{fdefinition}{Definition}{%
    before=\par\vspace{2ex},
    boxsep=0.5\topsep,
    after=\par\vspace{2ex},
    colback=blue!5!white,
    colframe=blue!75!black,
    fonttitle=\bfseries\upshape%
}{def}

\newtcbtheorem[use counter*=example]{fexample}{Example}{%
    before=\par\vspace{2ex},
    boxsep=0.5\topsep,
    after=\par\vspace{2ex},
    colback=red!5!white,
    colframe=red!75!black,
    fonttitle=\bfseries\upshape%
}{ex}

\newtcbtheorem[auto counter, number within=section]{fnotation}{Notation}{%
    before=\par\vspace{2ex},
    boxsep=0.5\topsep,
    after=\par\vspace{2ex},
    colback=SeaGreen!5!white,
    colframe=SeaGreen!75!black,
    fonttitle=\bfseries\upshape%
}{not}

\newtcbtheorem[use counter*=remark]{fremark}{Remark}{%
    fonttitle=\bfseries\upshape,
    colback=Goldenrod!5!white,
    colframe=Goldenrod!75!black}{ex}

\newenvironment{bproof}{\textit{Proof.}}{\hfill$\square$}
\tcolorboxenvironment{bproof}{%
    blanker,
    breakable,
    left=3mm,
    before skip=5pt,
    after skip=10pt,
    borderline west={0.6mm}{0pt}{green!80!black}
}

\AtEndEnvironment{lexample}{$\hfill\textcolor{red}{\blacksquare}$}
\newtcbtheorem[use counter*=example]{lexample}{Example}{%
    empty,
    title={Example~\theexample},
    boxed title style={%
        empty,
        size=minimal,
        toprule=2pt,
        top=0.5\topsep,
    },
    coltitle=red,
    fonttitle=\bfseries,
    parbox=false,
    boxsep=0pt,
    before=\par\vspace{2ex},
    left=0pt,
    right=0pt,
    top=3ex,
    bottom=1ex,
    before=\par\vspace{2ex},
    after=\par\vspace{2ex},
    breakable,
    pad at break*=0mm,
    vfill before first,
    overlay unbroken={%
        \draw[red, line width=2pt]
            ([yshift=-1.2ex]title.south-|frame.west) to
            ([yshift=-1.2ex]title.south-|frame.east);
        },
    overlay first={%
        \draw[red, line width=2pt]
            ([yshift=-1.2ex]title.south-|frame.west) to
            ([yshift=-1.2ex]title.south-|frame.east);
    },
}{ex}

\AtEndEnvironment{ldefinition}{$\hfill\textcolor{Blue}{\blacksquare}$}
\newtcbtheorem[use counter*=definition]{ldefinition}{Definition}{%
    empty,
    title={Definition~\thedefinition:~{#1}},
    boxed title style={%
        empty,
        size=minimal,
        toprule=2pt,
        top=0.5\topsep,
    },
    coltitle=Blue,
    fonttitle=\bfseries,
    parbox=false,
    boxsep=0pt,
    before=\par\vspace{2ex},
    left=0pt,
    right=0pt,
    top=3ex,
    bottom=0pt,
    before=\par\vspace{2ex},
    after=\par\vspace{1ex},
    breakable,
    pad at break*=0mm,
    vfill before first,
    overlay unbroken={%
        \draw[Blue, line width=2pt]
            ([yshift=-1.2ex]title.south-|frame.west) to
            ([yshift=-1.2ex]title.south-|frame.east);
        },
    overlay first={%
        \draw[Blue, line width=2pt]
            ([yshift=-1.2ex]title.south-|frame.west) to
            ([yshift=-1.2ex]title.south-|frame.east);
    },
}{def}

\AtEndEnvironment{ltheorem}{$\hfill\textcolor{Green}{\blacksquare}$}
\newtcbtheorem[use counter*=theorem]{ltheorem}{Theorem}{%
    empty,
    title={Theorem~\thetheorem:~{#1}},
    boxed title style={%
        empty,
        size=minimal,
        toprule=2pt,
        top=0.5\topsep,
    },
    coltitle=Green,
    fonttitle=\bfseries,
    parbox=false,
    boxsep=0pt,
    before=\par\vspace{2ex},
    left=0pt,
    right=0pt,
    top=3ex,
    bottom=-1.5ex,
    breakable,
    pad at break*=0mm,
    vfill before first,
    overlay unbroken={%
        \draw[Green, line width=2pt]
            ([yshift=-1.2ex]title.south-|frame.west) to
            ([yshift=-1.2ex]title.south-|frame.east);},
    overlay first={%
        \draw[Green, line width=2pt]
            ([yshift=-1.2ex]title.south-|frame.west) to
            ([yshift=-1.2ex]title.south-|frame.east);
    }
}{thm}

%--------------------Declared Math Operators--------------------%
\DeclareMathOperator{\adjoint}{adj}         % Adjoint.
\DeclareMathOperator{\Card}{Card}           % Cardinality.
\DeclareMathOperator{\curl}{curl}           % Curl.
\DeclareMathOperator{\diam}{diam}           % Diameter.
\DeclareMathOperator{\dist}{dist}           % Distance.
\DeclareMathOperator{\Div}{div}             % Divergence.
\DeclareMathOperator{\Erf}{Erf}             % Error Function.
\DeclareMathOperator{\Erfc}{Erfc}           % Complementary Error Function.
\DeclareMathOperator{\Ext}{Ext}             % Exterior.
\DeclareMathOperator{\GCD}{GCD}             % Greatest common denominator.
\DeclareMathOperator{\grad}{grad}           % Gradient
\DeclareMathOperator{\Ima}{Im}              % Image.
\DeclareMathOperator{\Int}{Int}             % Interior.
\DeclareMathOperator{\LC}{LC}               % Leading coefficient.
\DeclareMathOperator{\LCM}{LCM}             % Least common multiple.
\DeclareMathOperator{\LM}{LM}               % Leading monomial.
\DeclareMathOperator{\LT}{LT}               % Leading term.
\DeclareMathOperator{\Mod}{mod}             % Modulus.
\DeclareMathOperator{\Mon}{Mon}             % Monomial.
\DeclareMathOperator{\multideg}{mutlideg}   % Multi-Degree (Graphs).
\DeclareMathOperator{\nul}{nul}             % Null space of operator.
\DeclareMathOperator{\Ord}{Ord}             % Ordinal of ordered set.
\DeclareMathOperator{\Prin}{Prin}           % Principal value.
\DeclareMathOperator{\proj}{proj}           % Projection.
\DeclareMathOperator{\Refl}{Refl}           % Reflection operator.
\DeclareMathOperator{\rk}{rk}               % Rank of operator.
\DeclareMathOperator{\sgn}{sgn}             % Sign of a number.
\DeclareMathOperator{\sinc}{sinc}           % Sinc function.
\DeclareMathOperator{\Span}{Span}           % Span of a set.
\DeclareMathOperator{\Spec}{Spec}           % Spectrum.
\DeclareMathOperator{\supp}{supp}           % Support
\DeclareMathOperator{\Tr}{Tr}               % Trace of matrix.
%--------------------Declared Math Symbols--------------------%
\DeclareMathSymbol{\minus}{\mathbin}{AMSa}{"39} % Unary minus sign.
%------------------------New Commands---------------------------%
\DeclarePairedDelimiter\norm{\lVert}{\rVert}
\DeclarePairedDelimiter\ceil{\lceil}{\rceil}
\DeclarePairedDelimiter\floor{\lfloor}{\rfloor}
\newcommand*\diff{\mathop{}\!\mathrm{d}}
\newcommand*\Diff[1]{\mathop{}\!\mathrm{d^#1}}
\renewcommand*{\glstextformat}[1]{\textcolor{RoyalBlue}{#1}}
\renewcommand{\glsnamefont}[1]{\textbf{#1}}
\renewcommand\labelitemii{$\circ$}
\renewcommand\thesubfigure{%
    \arabic{chapter}.\arabic{figure}.\arabic{subfigure}}
\addto\captionsenglish{\renewcommand{\figurename}{Fig.}}
\numberwithin{equation}{section}

\renewcommand{\vector}[1]{\boldsymbol{\mathrm{#1}}}

\newcommand{\uvector}[1]{\boldsymbol{\hat{\mathrm{#1}}}}
\newcommand{\topspace}[2][]{(#2,\tau_{#1})}
\newcommand{\measurespace}[2][]{(#2,\varSigma_{#1},\mu_{#1})}
\newcommand{\measurablespace}[2][]{(#2,\varSigma_{#1})}
\newcommand{\manifold}[2][]{(#2,\tau_{#1},\mathcal{A}_{#1})}
\newcommand{\tanspace}[2]{T_{#1}{#2}}
\newcommand{\cotanspace}[2]{T_{#1}^{*}{#2}}
\newcommand{\Ckspace}[3][\mathbb{R}]{C^{#2}(#3,#1)}
\newcommand{\funcspace}[2][\mathbb{R}]{\mathcal{F}(#2,#1)}
\newcommand{\smoothvecf}[1]{\mathfrak{X}(#1)}
\newcommand{\smoothonef}[1]{\mathfrak{X}^{*}(#1)}
\newcommand{\bracket}[2]{[#1,#2]}

%------------------------Book Command---------------------------%
\makeatletter
\renewcommand\@pnumwidth{1cm}
\newcounter{book}
\renewcommand\thebook{\@Roman\c@book}
\newcommand\book{%
    \if@openright
        \cleardoublepage
    \else
        \clearpage
    \fi
    \thispagestyle{plain}%
    \if@twocolumn
        \onecolumn
        \@tempswatrue
    \else
        \@tempswafalse
    \fi
    \null\vfil
    \secdef\@book\@sbook
}
\def\@book[#1]#2{%
    \refstepcounter{book}
    \addcontentsline{toc}{book}{\bookname\ \thebook:\hspace{1em}#1}
    \markboth{}{}
    {\centering
     \interlinepenalty\@M
     \normalfont
     \huge\bfseries\bookname\nobreakspace\thebook
     \par
     \vskip 20\p@
     \Huge\bfseries#2\par}%
    \@endbook}
\def\@sbook#1{%
    {\centering
     \interlinepenalty \@M
     \normalfont
     \Huge\bfseries#1\par}%
    \@endbook}
\def\@endbook{
    \vfil\newpage
        \if@twoside
            \if@openright
                \null
                \thispagestyle{empty}%
                \newpage
            \fi
        \fi
        \if@tempswa
            \twocolumn
        \fi
}
\newcommand*\l@book[2]{%
    \ifnum\c@tocdepth >-3\relax
        \addpenalty{-\@highpenalty}%
        \addvspace{2.25em\@plus\p@}%
        \setlength\@tempdima{3em}%
        \begingroup
            \parindent\z@\rightskip\@pnumwidth
            \parfillskip -\@pnumwidth
            {
                \leavevmode
                \Large\bfseries#1\hfill\hb@xt@\@pnumwidth{\hss#2}
            }
            \par
            \nobreak
            \global\@nobreaktrue
            \everypar{\global\@nobreakfalse\everypar{}}%
        \endgroup
    \fi}
\newcommand\bookname{Book}
\renewcommand{\thebook}{\texorpdfstring{\Numberstring{book}}{book}}
\providecommand*{\toclevel@book}{-2}
\makeatother
\titleformat{\part}[display]
    {\Large\bfseries}
    {\partname\nobreakspace\thepart}
    {0mm}
    {\Huge\bfseries}
\titlecontents{part}[0pt]
    {\large\bfseries}
    {\partname\ \thecontentslabel: \quad}
    {}
    {\hfill\contentspage}
\titlecontents{chapter}[0pt]
    {\bfseries}
    {\chaptername\ \thecontentslabel:\quad}
    {}
    {\hfill\contentspage}
\newglossarystyle{longpara}{%
    \setglossarystyle{long}%
    \renewenvironment{theglossary}{%
        \begin{longtable}[l]{{p{0.25\hsize}p{0.65\hsize}}}
    }{\end{longtable}}%
    \renewcommand{\glossentry}[2]{%
        \glstarget{##1}{\glossentryname{##1}}%
        &\glossentrydesc{##1}{~##2.}
        \tabularnewline%
        \tabularnewline
    }%
}
\newglossary[not-glg]{notation}{not-gls}{not-glo}{Notation}
\newcommand*{\newnotation}[4][]{%
    \newglossaryentry{#2}{type=notation, name={\textbf{#3}, },
                          text={#4}, description={#4},#1}%
}
%--------------------------LENGTHS------------------------------%
% Spacings for the Table of Contents.
\addtolength{\cftsecnumwidth}{1ex}
\addtolength{\cftsubsecindent}{1ex}
\addtolength{\cftsubsecnumwidth}{1ex}
\addtolength{\cftfignumwidth}{1ex}
\addtolength{\cfttabnumwidth}{1ex}

% Indent and paragraph spacing.
\setlength{\parindent}{0em}
\setlength{\parskip}{0em}                                                           %
%----------------------------Main Document-------------------------------------%
\begin{document}
    \title{Galois Theory}
    \author{Ryan Maguire}
    \date{\vspace{-5ex}}
    \maketitle
    \section{Fields}
        \subsection{What's the Point?}
            From the historical perspective, we can start with numbers. There's
            the standard inclusion:
            \begin{equation}
                \mathbb{N}^{+}\subseteq\mathbb{N}\subseteq
                \mathbb{Z}\subseteq\mathbb{Q}\subseteq\mathbb{R}
                \subseteq\mathbb{C}
            \end{equation}
            Suppose we are given the equation $2+x=1$. If we only know about
            the positive integers, then we cannot solve this equation. We thus
            need to introduce negative integers. Next we could write $2x=1$,
            and we are now forced to introduce the rational numbers. In ancient
            Greece, the solution to $x^{2}=2$ was proved to be irrational, and
            thus we must go beyond $\mathbb{Q}$ and develop the real numbers.
            Lastly, a part studied in Italy in the Renaissance era, are
            equations like $x^{2}=\minus{1}$. There are no real solutions to
            this, and so we must invent $\mathbb{C}$. The complex numbers are
            the set $\mathbb{C}$ of the form:
            \begin{equation}
                \mathbb{C}=\{\,x+iy\;|\;x,y\in\mathbb{R}\,\}
            \end{equation}
            where $i^{2}=\minus{1}$, by definition, which is the solution to the
            equation $z^{2}+1=0$. This equation has no solutions in $\mathbb{R}$
            and so $i$ is not a real number, and hence is called the
            \textit{imaginary} unit. We can picture complex numbers by use of
            the plane $\mathbb{R}^{2}$. But there's nothing too special about
            the equation $z^{2}+1=0$, and we can consider $z^{2}+z+1=0$ and
            again we can ask if this has real solutions. Unlike the first
            equation, it's not so obvious that this has no real solution. We
            can look at the quadratic formula, and in particular the discriment,
            obtaining:
            \begin{equation}
                \Delta=b^{2}-4ac=1-4=\minus{2}
            \end{equation}
            Since this is negative, there are no real solutions, and hence
            $z^{2}+z+1$ has no solution in $\mathbb{R}$. It does have roots in
            $\mathbb{C}$:
            \begin{subequations}
                \begin{align}
                    \omega&=\minus\frac{1}{2}+\frac{\sqrt{3}}{2}i\\
                    \overline{\omega}&=\minus\frac{1}{2}-\frac{\sqrt{3}}{2}i
                \end{align}
            \end{subequations}
            We can further consider the set $\mathbb{R}[w]$ defined by:
            \begin{equation}
                \mathbb{R}[\omega]=\{\,x+y\omega\;|\;x,y\in\mathbb{R}\,\}
            \end{equation}
            This has a nice field structure, like $\mathbb{C}$, and indeed this
            is equal to $\mathbb{C}$. That is, $\mathbb{R}[\omega]=\mathbb{C}$.
            We can see this since $\mathbb{R}[\omega]$ is a subspace of
            $\mathbb{C}$ with a basis consisting of two elements:
            $\{1,\omega\}$, and thus has the same dimension as $\mathbb{C}$.
            Hence, it is equal to the whole thing. We can be even more explicit:
            \begin{align}
                x+y\omega&=x+y\big(\minus\frac{1}{2}+\frac{\sqrt{3}}{2}i\big)\\
                &=\big(x-\frac{1}{2}y\big)+\big(\frac{\sqrt{3}}{2}\big)i
            \end{align}
            And this is of the form $x'+y'i$, where:
            \begin{align}
                x'&=x-\frac{1}{2}y\\
                y'&=\frac{\sqrt{3}}{2}y
            \end{align}
            Since this is always solvable for both $(x,y)$ and $(x',y')$, the
            two spaces are the same. And indeed, we can generalize. If
            $f(x)=ax^{2}+bx+c$, with $a,b,c\in\mathbb{R}$ are such that
            $b^{2}-4ac<0$, then defining:
            \begin{equation}
                \alpha=\frac{\minus{b}+\sqrt{b^{2}-4ac}}{2a}
            \end{equation}
            which is a complex root of $f$, then
            $\mathbb{R}[\alpha]=\mathbb{C}$. This shows there's nothing too
            special about $i$: extending $\mathbb{R}$ with any complex root of
            a quadratic gives the entirety of $\mathbb{C}$, we need not only
            choose $z^{2}+1=0$. Even if we were to stick with this polynomial,
            we could still choose $\minus{i}$, since this too is a solution.
            Choosing $i$ over $\minus{i}$ seems to purely be an accident of
            history. Going from one choice to another is an
            $\mathbb{R}$ automorphism: $x+iy\mapsto{x}-iy$. An $\mathbb{R}$
            automorphism is a bijective ring homomoprhism
            $f:\mathbb{R}\rightarrow\mathbb{R}$. That is, an isomorphism from
            $\mathbb{R}$ to itself:
            \begin{align}
                f(z_{1}+z_{2})&=f(z_{1})+f(z_{2})\\
                f(z_{1}z_{2})&=f(z_{1})f(z_{2})\\
                f(1)=1
            \end{align}
            The automorphism $x+iy\mapsto{x}-iy$ is called complex conjugation.
            If we don't like $i$, and have a complex number such as $\omega$,
            we can still take as an $\mathbb{R}$ automorphism the function
            $\sigma:\mathbb{C}\rightarrow\mathbb{C}$ where
            $x+y\omega\mapsto{x}+y\overline{\omega}$. As it turns out, this is
            the same as the automorphism $x+iy\mapsto{x}-iy$ since we can write:
            \begin{equation}
                i=\frac{1+2\omega}{\sqrt{3}}
            \end{equation}
            This is the object we wish to stress as the important part of the
            theory of complex numbers. Neither $i$ nor $\omega$ are too
            important, but rather the notion of complex conjugation is.
            The group of $\mathbb{R}$ automorphisms of $\mathbb{C}$ is equal to:
            \begin{equation}
                \textrm{Aut}_{\mathbb{R}}(\mathbb{C})
                    =\{\textrm{id}_{\mathbb{R}},\sigma\}
            \end{equation}
            Where $\sigma$ is complex conjugation. That is, $\sigma$ is the
            unique non-trivial $\mathbb{R}$ automorphism that has the property
            that it exchanges the roots of any $f(x)=ax^{2}+bx+c$ with
            $b^{2}-4ac<0$. The group structure comes from function composition.
            Since function composition is always associative, since the identity
            map is an automorphism, and since bijections have inverse elements,
            this is indeed a group. We can summarize all of this as follows:
            The roots of any real polynomial are either real or come in complex
            conjugate pairs.
            \par\hfill\par
            Looking at the numerology of the problem, there seems to be
            something special about the number two (2). This is the size of the
            automorphism group $\textrm{Aut}_{\mathbb{R}}(\mathbb{C})$, and
            this is also the dimension of $\mathbb{C}$, and lastly it is the
            degree of $\mathbb{C}$ over $\mathbb{R}$: $[\mathbb{C}:\mathbb{R}]$.
            More generally, consider any field $\mathbb{F}$ with characteristic
            not equal to 2 (that is, $1+1\ne{0}$), and any function
            $f(x)=ax^{2}+bx+c$, $a,b,c\in\mathbb{F}$ such that $f(x)=0$ has no
            solutions in $\mathbb{F}$. For example, $\mathbb{R}$ with
            $f(x)=x^{2}+1$, of $\mathbb{Q}$ with $f(x)=x^{2}-2$. If we have
            such conditions, then there is a field $\mathbb{K}$ and an inclusion
            $\mathbb{F}\subseteq\mathbb{K}$ making $\mathbb{F}$ a subfield,
            such that $f(x)=a(x-\alpha)(x-\beta)$, where
            $\alpha,\beta\in\mathbb{K}$. Moreover,
            $\mathbb{K}=\mathbb{F}[\alpha]$. That is:
            \begin{equation}
                \mathbb{K}=\{\,x+y\alpha\;|\;x,y\in\mathbb{F}\}
            \end{equation}
            Similarly, $\mathbb{K}=\mathbb{F}[\beta]$. Lastly, the automorphism
            group is
            \begin{equation}
                \textrm{Aut}_{\mathbb{F}}(\mathbb{K})
                =\{\,\textrm{id}_{\mathbb{F}},\sigma\}
            \end{equation}
            where $\sigma$ is the unique automorphism such that
            $\sigma(\alpha=\beta)$. The proof is simply an application of the
            quadratic formula, where we invoke the fact that $2\ne{0}$ in a
            field whose characteristic is not 2.
        \subsection{Cubic Equations and Higher}
            In the $16^{th}$ century the Italians were able to solve the cubic
            equation: $x^{3}+px-q=0$. This may not look like the general cubic,
            but since we are interested in roots we may always divide off by
            the leading coefficient of $x^{3}$, and the quadratic term may be
            absorbed by completing the square, and thus any cubic can be
            written in such a form. The solution is:
            \begin{equation}
                Yeah
            \end{equation}
            By the $18^{th}$ century the Italians were able to solve the general
            quartic equation. The next natural question is the solution to the
            quintic, but this was shown not to exist. The Abel-Ruffini theorem
            shows that the general quintic equation can not be solved using
            nested radicals. Galois went to prove that a polynomial has a root
            that can be written in terms of nested radicals if and only if
            $K/F$, the splitting field, has an automorphism group
            $\textrm{Aut}_{F}(K))$ that is solveable.
\end{document}