%------------------------------------------------------------------------------%
\documentclass{article}                                                        %
%------------------------------Preamble----------------------------------------%
\makeatletter                                                                  %
    \def\input@path{{../../}}                                                  %
\makeatother                                                                   %
%---------------------------Packages----------------------------%
\usepackage{geometry}
\geometry{b5paper, margin=1.0in}
\usepackage[T1]{fontenc}
\usepackage{graphicx, float}            % Graphics/Images.
\usepackage{natbib}                     % For bibliographies.
\bibliographystyle{agsm}                % Bibliography style.
\usepackage[french, english]{babel}     % Language typesetting.
\usepackage[dvipsnames]{xcolor}         % Color names.
\usepackage{listings}                   % Verbatim-Like Tools.
\usepackage{mathtools, esint, mathrsfs} % amsmath and integrals.
\usepackage{amsthm, amsfonts, amssymb}  % Fonts and theorems.
\usepackage{tcolorbox}                  % Frames around theorems.
\usepackage{upgreek}                    % Non-Italic Greek.
\usepackage{fmtcount, etoolbox}         % For the \book{} command.
\usepackage[newparttoc]{titlesec}       % Formatting chapter, etc.
\usepackage{titletoc}                   % Allows \book in toc.
\usepackage[nottoc]{tocbibind}          % Bibliography in toc.
\usepackage[titles]{tocloft}            % ToC formatting.
\usepackage{pgfplots, tikz}             % Drawing/graphing tools.
\usepackage{imakeidx}                   % Used for index.
\usetikzlibrary{
    calc,                   % Calculating right angles and more.
    angles,                 % Drawing angles within triangles.
    arrows.meta,            % Latex and Stealth arrows.
    quotes,                 % Adding labels to angles.
    positioning,            % Relative positioning of nodes.
    decorations.markings,   % Adding arrows in the middle of a line.
    patterns,
    arrows
}                                       % Libraries for tikz.
\pgfplotsset{compat=1.9}                % Version of pgfplots.
\usepackage[font=scriptsize,
            labelformat=simple,
            labelsep=colon]{subcaption} % Subfigure captions.
\usepackage[font={scriptsize},
            hypcap=true,
            labelsep=colon]{caption}    % Figure captions.
\usepackage[pdftex,
            pdfauthor={Ryan Maguire},
            pdftitle={Mathematics and Physics},
            pdfsubject={Mathematics, Physics, Science},
            pdfkeywords={Mathematics, Physics, Computer Science, Biology},
            pdfproducer={LaTeX},
            pdfcreator={pdflatex}]{hyperref}
\hypersetup{
    colorlinks=true,
    linkcolor=blue,
    filecolor=magenta,
    urlcolor=Cerulean,
    citecolor=SkyBlue
}                           % Colors for hyperref.
\usepackage[toc,acronym,nogroupskip,nopostdot]{glossaries}
\usepackage{glossary-mcols}
%------------------------Theorem Styles-------------------------%
\theoremstyle{plain}
\newtheorem{theorem}{Theorem}[section]

% Define theorem style for default spacing and normal font.
\newtheoremstyle{normal}
    {\topsep}               % Amount of space above the theorem.
    {\topsep}               % Amount of space below the theorem.
    {}                      % Font used for body of theorem.
    {}                      % Measure of space to indent.
    {\bfseries}             % Font of the header of the theorem.
    {}                      % Punctuation between head and body.
    {.5em}                  % Space after theorem head.
    {}

% Italic header environment.
\newtheoremstyle{thmit}{\topsep}{\topsep}{}{}{\itshape}{}{0.5em}{}

% Define environments with italic headers.
\theoremstyle{thmit}
\newtheorem*{solution}{Solution}

% Define default environments.
\theoremstyle{normal}
\newtheorem{example}{Example}[section]
\newtheorem{definition}{Definition}[section]
\newtheorem{problem}{Problem}[section]

% Define framed environment.
\tcbuselibrary{most}
\newtcbtheorem[use counter*=theorem]{ftheorem}{Theorem}{%
    before=\par\vspace{2ex},
    boxsep=0.5\topsep,
    after=\par\vspace{2ex},
    colback=green!5,
    colframe=green!35!black,
    fonttitle=\bfseries\upshape%
}{thm}

\newtcbtheorem[auto counter, number within=section]{faxiom}{Axiom}{%
    before=\par\vspace{2ex},
    boxsep=0.5\topsep,
    after=\par\vspace{2ex},
    colback=Apricot!5,
    colframe=Apricot!35!black,
    fonttitle=\bfseries\upshape%
}{ax}

\newtcbtheorem[use counter*=definition]{fdefinition}{Definition}{%
    before=\par\vspace{2ex},
    boxsep=0.5\topsep,
    after=\par\vspace{2ex},
    colback=blue!5!white,
    colframe=blue!75!black,
    fonttitle=\bfseries\upshape%
}{def}

\newtcbtheorem[use counter*=example]{fexample}{Example}{%
    before=\par\vspace{2ex},
    boxsep=0.5\topsep,
    after=\par\vspace{2ex},
    colback=red!5!white,
    colframe=red!75!black,
    fonttitle=\bfseries\upshape%
}{ex}

\newtcbtheorem[auto counter, number within=section]{fnotation}{Notation}{%
    before=\par\vspace{2ex},
    boxsep=0.5\topsep,
    after=\par\vspace{2ex},
    colback=SeaGreen!5!white,
    colframe=SeaGreen!75!black,
    fonttitle=\bfseries\upshape%
}{not}

\newtcbtheorem[use counter*=remark]{fremark}{Remark}{%
    fonttitle=\bfseries\upshape,
    colback=Goldenrod!5!white,
    colframe=Goldenrod!75!black}{ex}

\newenvironment{bproof}{\textit{Proof.}}{\hfill$\square$}
\tcolorboxenvironment{bproof}{%
    blanker,
    breakable,
    left=3mm,
    before skip=5pt,
    after skip=10pt,
    borderline west={0.6mm}{0pt}{green!80!black}
}

\AtEndEnvironment{lexample}{$\hfill\textcolor{red}{\blacksquare}$}
\newtcbtheorem[use counter*=example]{lexample}{Example}{%
    empty,
    title={Example~\theexample},
    boxed title style={%
        empty,
        size=minimal,
        toprule=2pt,
        top=0.5\topsep,
    },
    coltitle=red,
    fonttitle=\bfseries,
    parbox=false,
    boxsep=0pt,
    before=\par\vspace{2ex},
    left=0pt,
    right=0pt,
    top=3ex,
    bottom=1ex,
    before=\par\vspace{2ex},
    after=\par\vspace{2ex},
    breakable,
    pad at break*=0mm,
    vfill before first,
    overlay unbroken={%
        \draw[red, line width=2pt]
            ([yshift=-1.2ex]title.south-|frame.west) to
            ([yshift=-1.2ex]title.south-|frame.east);
        },
    overlay first={%
        \draw[red, line width=2pt]
            ([yshift=-1.2ex]title.south-|frame.west) to
            ([yshift=-1.2ex]title.south-|frame.east);
    },
}{ex}

\AtEndEnvironment{ldefinition}{$\hfill\textcolor{Blue}{\blacksquare}$}
\newtcbtheorem[use counter*=definition]{ldefinition}{Definition}{%
    empty,
    title={Definition~\thedefinition:~{#1}},
    boxed title style={%
        empty,
        size=minimal,
        toprule=2pt,
        top=0.5\topsep,
    },
    coltitle=Blue,
    fonttitle=\bfseries,
    parbox=false,
    boxsep=0pt,
    before=\par\vspace{2ex},
    left=0pt,
    right=0pt,
    top=3ex,
    bottom=0pt,
    before=\par\vspace{2ex},
    after=\par\vspace{1ex},
    breakable,
    pad at break*=0mm,
    vfill before first,
    overlay unbroken={%
        \draw[Blue, line width=2pt]
            ([yshift=-1.2ex]title.south-|frame.west) to
            ([yshift=-1.2ex]title.south-|frame.east);
        },
    overlay first={%
        \draw[Blue, line width=2pt]
            ([yshift=-1.2ex]title.south-|frame.west) to
            ([yshift=-1.2ex]title.south-|frame.east);
    },
}{def}

\AtEndEnvironment{ltheorem}{$\hfill\textcolor{Green}{\blacksquare}$}
\newtcbtheorem[use counter*=theorem]{ltheorem}{Theorem}{%
    empty,
    title={Theorem~\thetheorem:~{#1}},
    boxed title style={%
        empty,
        size=minimal,
        toprule=2pt,
        top=0.5\topsep,
    },
    coltitle=Green,
    fonttitle=\bfseries,
    parbox=false,
    boxsep=0pt,
    before=\par\vspace{2ex},
    left=0pt,
    right=0pt,
    top=3ex,
    bottom=-1.5ex,
    breakable,
    pad at break*=0mm,
    vfill before first,
    overlay unbroken={%
        \draw[Green, line width=2pt]
            ([yshift=-1.2ex]title.south-|frame.west) to
            ([yshift=-1.2ex]title.south-|frame.east);},
    overlay first={%
        \draw[Green, line width=2pt]
            ([yshift=-1.2ex]title.south-|frame.west) to
            ([yshift=-1.2ex]title.south-|frame.east);
    }
}{thm}

%--------------------Declared Math Operators--------------------%
\DeclareMathOperator{\adjoint}{adj}         % Adjoint.
\DeclareMathOperator{\Card}{Card}           % Cardinality.
\DeclareMathOperator{\curl}{curl}           % Curl.
\DeclareMathOperator{\diam}{diam}           % Diameter.
\DeclareMathOperator{\dist}{dist}           % Distance.
\DeclareMathOperator{\Div}{div}             % Divergence.
\DeclareMathOperator{\Erf}{Erf}             % Error Function.
\DeclareMathOperator{\Erfc}{Erfc}           % Complementary Error Function.
\DeclareMathOperator{\Ext}{Ext}             % Exterior.
\DeclareMathOperator{\GCD}{GCD}             % Greatest common denominator.
\DeclareMathOperator{\grad}{grad}           % Gradient
\DeclareMathOperator{\Ima}{Im}              % Image.
\DeclareMathOperator{\Int}{Int}             % Interior.
\DeclareMathOperator{\LC}{LC}               % Leading coefficient.
\DeclareMathOperator{\LCM}{LCM}             % Least common multiple.
\DeclareMathOperator{\LM}{LM}               % Leading monomial.
\DeclareMathOperator{\LT}{LT}               % Leading term.
\DeclareMathOperator{\Mod}{mod}             % Modulus.
\DeclareMathOperator{\Mon}{Mon}             % Monomial.
\DeclareMathOperator{\multideg}{mutlideg}   % Multi-Degree (Graphs).
\DeclareMathOperator{\nul}{nul}             % Null space of operator.
\DeclareMathOperator{\Ord}{Ord}             % Ordinal of ordered set.
\DeclareMathOperator{\Prin}{Prin}           % Principal value.
\DeclareMathOperator{\proj}{proj}           % Projection.
\DeclareMathOperator{\Refl}{Refl}           % Reflection operator.
\DeclareMathOperator{\rk}{rk}               % Rank of operator.
\DeclareMathOperator{\sgn}{sgn}             % Sign of a number.
\DeclareMathOperator{\sinc}{sinc}           % Sinc function.
\DeclareMathOperator{\Span}{Span}           % Span of a set.
\DeclareMathOperator{\Spec}{Spec}           % Spectrum.
\DeclareMathOperator{\supp}{supp}           % Support
\DeclareMathOperator{\Tr}{Tr}               % Trace of matrix.
%--------------------Declared Math Symbols--------------------%
\DeclareMathSymbol{\minus}{\mathbin}{AMSa}{"39} % Unary minus sign.
%------------------------New Commands---------------------------%
\DeclarePairedDelimiter\norm{\lVert}{\rVert}
\DeclarePairedDelimiter\ceil{\lceil}{\rceil}
\DeclarePairedDelimiter\floor{\lfloor}{\rfloor}
\newcommand*\diff{\mathop{}\!\mathrm{d}}
\newcommand*\Diff[1]{\mathop{}\!\mathrm{d^#1}}
\renewcommand*{\glstextformat}[1]{\textcolor{RoyalBlue}{#1}}
\renewcommand{\glsnamefont}[1]{\textbf{#1}}
\renewcommand\labelitemii{$\circ$}
\renewcommand\thesubfigure{%
    \arabic{chapter}.\arabic{figure}.\arabic{subfigure}}
\addto\captionsenglish{\renewcommand{\figurename}{Fig.}}
\numberwithin{equation}{section}

\renewcommand{\vector}[1]{\boldsymbol{\mathrm{#1}}}

\newcommand{\uvector}[1]{\boldsymbol{\hat{\mathrm{#1}}}}
\newcommand{\topspace}[2][]{(#2,\tau_{#1})}
\newcommand{\measurespace}[2][]{(#2,\varSigma_{#1},\mu_{#1})}
\newcommand{\measurablespace}[2][]{(#2,\varSigma_{#1})}
\newcommand{\manifold}[2][]{(#2,\tau_{#1},\mathcal{A}_{#1})}
\newcommand{\tanspace}[2]{T_{#1}{#2}}
\newcommand{\cotanspace}[2]{T_{#1}^{*}{#2}}
\newcommand{\Ckspace}[3][\mathbb{R}]{C^{#2}(#3,#1)}
\newcommand{\funcspace}[2][\mathbb{R}]{\mathcal{F}(#2,#1)}
\newcommand{\smoothvecf}[1]{\mathfrak{X}(#1)}
\newcommand{\smoothonef}[1]{\mathfrak{X}^{*}(#1)}
\newcommand{\bracket}[2]{[#1,#2]}

%------------------------Book Command---------------------------%
\makeatletter
\renewcommand\@pnumwidth{1cm}
\newcounter{book}
\renewcommand\thebook{\@Roman\c@book}
\newcommand\book{%
    \if@openright
        \cleardoublepage
    \else
        \clearpage
    \fi
    \thispagestyle{plain}%
    \if@twocolumn
        \onecolumn
        \@tempswatrue
    \else
        \@tempswafalse
    \fi
    \null\vfil
    \secdef\@book\@sbook
}
\def\@book[#1]#2{%
    \refstepcounter{book}
    \addcontentsline{toc}{book}{\bookname\ \thebook:\hspace{1em}#1}
    \markboth{}{}
    {\centering
     \interlinepenalty\@M
     \normalfont
     \huge\bfseries\bookname\nobreakspace\thebook
     \par
     \vskip 20\p@
     \Huge\bfseries#2\par}%
    \@endbook}
\def\@sbook#1{%
    {\centering
     \interlinepenalty \@M
     \normalfont
     \Huge\bfseries#1\par}%
    \@endbook}
\def\@endbook{
    \vfil\newpage
        \if@twoside
            \if@openright
                \null
                \thispagestyle{empty}%
                \newpage
            \fi
        \fi
        \if@tempswa
            \twocolumn
        \fi
}
\newcommand*\l@book[2]{%
    \ifnum\c@tocdepth >-3\relax
        \addpenalty{-\@highpenalty}%
        \addvspace{2.25em\@plus\p@}%
        \setlength\@tempdima{3em}%
        \begingroup
            \parindent\z@\rightskip\@pnumwidth
            \parfillskip -\@pnumwidth
            {
                \leavevmode
                \Large\bfseries#1\hfill\hb@xt@\@pnumwidth{\hss#2}
            }
            \par
            \nobreak
            \global\@nobreaktrue
            \everypar{\global\@nobreakfalse\everypar{}}%
        \endgroup
    \fi}
\newcommand\bookname{Book}
\renewcommand{\thebook}{\texorpdfstring{\Numberstring{book}}{book}}
\providecommand*{\toclevel@book}{-2}
\makeatother
\titleformat{\part}[display]
    {\Large\bfseries}
    {\partname\nobreakspace\thepart}
    {0mm}
    {\Huge\bfseries}
\titlecontents{part}[0pt]
    {\large\bfseries}
    {\partname\ \thecontentslabel: \quad}
    {}
    {\hfill\contentspage}
\titlecontents{chapter}[0pt]
    {\bfseries}
    {\chaptername\ \thecontentslabel:\quad}
    {}
    {\hfill\contentspage}
\newglossarystyle{longpara}{%
    \setglossarystyle{long}%
    \renewenvironment{theglossary}{%
        \begin{longtable}[l]{{p{0.25\hsize}p{0.65\hsize}}}
    }{\end{longtable}}%
    \renewcommand{\glossentry}[2]{%
        \glstarget{##1}{\glossentryname{##1}}%
        &\glossentrydesc{##1}{~##2.}
        \tabularnewline%
        \tabularnewline
    }%
}
\newglossary[not-glg]{notation}{not-gls}{not-glo}{Notation}
\newcommand*{\newnotation}[4][]{%
    \newglossaryentry{#2}{type=notation, name={\textbf{#3}, },
                          text={#4}, description={#4},#1}%
}
%--------------------------LENGTHS------------------------------%
% Spacings for the Table of Contents.
\addtolength{\cftsecnumwidth}{1ex}
\addtolength{\cftsubsecindent}{1ex}
\addtolength{\cftsubsecnumwidth}{1ex}
\addtolength{\cftfignumwidth}{1ex}
\addtolength{\cfttabnumwidth}{1ex}

% Indent and paragraph spacing.
\setlength{\parindent}{0em}
\setlength{\parskip}{0em}                                                           %
%----------------------------Main Document-------------------------------------%
\begin{document}
    \title{Galois Theory}
    \author{Ryan Maguire}
    \date{\vspace{-5ex}}
    \maketitle
    \section{Fields}
        \subsection{What's the Point?}
            From the historical perspective, we can start with numbers. There's
            the standard inclusion:
            \begin{equation}
                \mathbb{N}^{+}\subseteq\mathbb{N}\subseteq
                \mathbb{Z}\subseteq\mathbb{Q}\subseteq\mathbb{R}
                \subseteq\mathbb{C}
            \end{equation}
            Suppose we are given the equation $2+x=1$. If we only know about
            the positive integers, then we cannot solve this equation. We thus
            need to introduce negative integers. Next we could write $2x=1$,
            and we are now forced to introduce the rational numbers. In ancient
            Greece, the solution to $x^{2}=2$ was proved to be irrational, and
            thus we must go beyond $\mathbb{Q}$ and develop the real numbers.
            Lastly, a part studied in Italy in the Renaissance era, are
            equations like $x^{2}=\minus{1}$. There are no real solutions to
            this, and so we must invent $\mathbb{C}$. The complex numbers are
            the set $\mathbb{C}$ of the form:
            \begin{equation}
                \mathbb{C}=\{\,x+iy\;|\;x,y\in\mathbb{R}\,\}
            \end{equation}
            where $i^{2}=\minus{1}$, by definition, which is the solution to the
            equation $z^{2}+1=0$. This equation has no solutions in $\mathbb{R}$
            and so $i$ is not a real number, and hence is called the
            \textit{imaginary} unit. We can picture complex numbers by use of
            the plane $\mathbb{R}^{2}$. But there's nothing too special about
            the equation $z^{2}+1=0$, and we can consider $z^{2}+z+1=0$ and
            again we can ask if this has real solutions. Unlike the first
            equation, it's not so obvious that this has no real solution. We
            can look at the quadratic formula, and in particular the discriment,
            obtaining:
            \begin{equation}
                \Delta=b^{2}-4ac=1-4=\minus{2}
            \end{equation}
            Since this is negative, there are no real solutions, and hence
            $z^{2}+z+1$ has no solution in $\mathbb{R}$. It does have roots in
            $\mathbb{C}$:
            \begin{subequations}
                \begin{align}
                    \omega&=\minus\frac{1}{2}+\frac{\sqrt{3}}{2}i\\
                    \overline{\omega}&=\minus\frac{1}{2}-\frac{\sqrt{3}}{2}i
                \end{align}
            \end{subequations}
            We can further consider the set $\mathbb{R}[w]$ defined by:
            \begin{equation}
                \mathbb{R}[\omega]=\{\,x+y\omega\;|\;x,y\in\mathbb{R}\,\}
            \end{equation}
            This has a nice field structure, like $\mathbb{C}$, and indeed this
            is equal to $\mathbb{C}$. That is, $\mathbb{R}[\omega]=\mathbb{C}$.
            We can see this since $\mathbb{R}[\omega]$ is a subspace of
            $\mathbb{C}$ with a basis consisting of two elements:
            $\{1,\omega\}$, and thus has the same dimension as $\mathbb{C}$.
            Hence, it is equal to the whole thing. We can be even more explicit:
            \begin{align}
                x+y\omega&=x+y\big(\minus\frac{1}{2}+\frac{\sqrt{3}}{2}i\big)\\
                &=\big(x-\frac{1}{2}y\big)+\big(\frac{\sqrt{3}}{2}\big)i
            \end{align}
            And this is of the form $x'+y'i$, where:
            \begin{align}
                x'&=x-\frac{1}{2}y\\
                y'&=\frac{\sqrt{3}}{2}y
            \end{align}
            Since this is always solvable for both $(x,y)$ and $(x',y')$, the
            two spaces are the same. And indeed, we can generalize. If
            $f(x)=ax^{2}+bx+c$, with $a,b,c\in\mathbb{R}$ are such that
            $b^{2}-4ac<0$, then defining:
            \begin{equation}
                \alpha=\frac{\minus{b}+\sqrt{b^{2}-4ac}}{2a}
            \end{equation}
            which is a complex root of $f$, then
            $\mathbb{R}[\alpha]=\mathbb{C}$. This shows there's nothing too
            special about $i$: extending $\mathbb{R}$ with any complex root of
            a quadratic gives the entirety of $\mathbb{C}$, we need not only
            choose $z^{2}+1=0$. Even if we were to stick with this polynomial,
            we could still choose $\minus{i}$, since this too is a solution.
            Choosing $i$ over $\minus{i}$ seems to purely be an accident of
            history. Going from one choice to another is an
            $\mathbb{R}$ automorphism: $x+iy\mapsto{x}-iy$. An $\mathbb{R}$
            automorphism is a bijective ring homomoprhism
            $f:\mathbb{R}\rightarrow\mathbb{R}$. That is, an isomorphism from
            $\mathbb{R}$ to itself:
            \begin{align}
                f(z_{1}+z_{2})&=f(z_{1})+f(z_{2})\\
                f(z_{1}z_{2})&=f(z_{1})f(z_{2})\\
                f(1)=1
            \end{align}
            The automorphism $x+iy\mapsto{x}-iy$ is called complex conjugation.
            If we don't like $i$, and have a complex number such as $\omega$,
            we can still take as an $\mathbb{R}$ automorphism the function
            $\sigma:\mathbb{C}\rightarrow\mathbb{C}$ where
            $x+y\omega\mapsto{x}+y\overline{\omega}$. As it turns out, this is
            the same as the automorphism $x+iy\mapsto{x}-iy$ since we can write:
            \begin{equation}
                i=\frac{1+2\omega}{\sqrt{3}}
            \end{equation}
            This is the object we wish to stress as the important part of the
            theory of complex numbers. Neither $i$ nor $\omega$ are too
            important, but rather the notion of complex conjugation is.
            The group of $\mathbb{R}$ automorphisms of $\mathbb{C}$ is equal to:
            \begin{equation}
                \textrm{Aut}_{\mathbb{R}}(\mathbb{C})
                    =\{\textrm{id}_{\mathbb{R}},\sigma\}
            \end{equation}
            Where $\sigma$ is complex conjugation. That is, $\sigma$ is the
            unique non-trivial $\mathbb{R}$ automorphism that has the property
            that it exchanges the roots of any $f(x)=ax^{2}+bx+c$ with
            $b^{2}-4ac<0$. The group structure comes from function composition.
            Since function composition is always associative, since the identity
            map is an automorphism, and since bijections have inverse elements,
            this is indeed a group. We can summarize all of this as follows:
            The roots of any real polynomial are either real or come in complex
            conjugate pairs.
            \par\hfill\par
            Looking at the numerology of the problem, there seems to be
            something special about the number two (2). This is the size of the
            automorphism group $\textrm{Aut}_{\mathbb{R}}(\mathbb{C})$, and
            this is also the dimension of $\mathbb{C}$, and lastly it is the
            degree of $\mathbb{C}$ over $\mathbb{R}$: $[\mathbb{C}:\mathbb{R}]$.
            More generally, consider any field $\mathbb{F}$ with characteristic
            not equal to 2 (that is, $1+1\ne{0}$), and any function
            $f(x)=ax^{2}+bx+c$, $a,b,c\in\mathbb{F}$ such that $f(x)=0$ has no
            solutions in $\mathbb{F}$. For example, $\mathbb{R}$ with
            $f(x)=x^{2}+1$, of $\mathbb{Q}$ with $f(x)=x^{2}-2$. If we have
            such conditions, then there is a field $\mathbb{K}$ and an inclusion
            $\mathbb{F}\subseteq\mathbb{K}$ making $\mathbb{F}$ a subfield,
            such that $f(x)=a(x-\alpha)(x-\beta)$, where
            $\alpha,\beta\in\mathbb{K}$. Moreover,
            $\mathbb{K}=\mathbb{F}[\alpha]$. That is:
            \begin{equation}
                \mathbb{K}=\{\,x+y\alpha\;|\;x,y\in\mathbb{F}\}
            \end{equation}
            Similarly, $\mathbb{K}=\mathbb{F}[\beta]$. Lastly, the automorphism
            group is
            \begin{equation}
                \textrm{Aut}_{\mathbb{F}}(\mathbb{K})
                =\{\,\textrm{id}_{\mathbb{F}},\sigma\}
            \end{equation}
            where $\sigma$ is the unique automorphism such that
            $\sigma(\alpha=\beta)$. The proof is simply an application of the
            quadratic formula, where we invoke the fact that $2\ne{0}$ in a
            field whose characteristic is not 2.
        \subsection{Cubic Equations and Higher}
            In the $16^{th}$ century the Italians were able to solve the cubic
            equation: $x^{3}+px-q=0$. This may not look like the general cubic,
            but since we are interested in roots we may always divide off by
            the leading coefficient of $x^{3}$, and the quadratic term may be
            absorbed by completing the square, and thus any cubic can be
            written in such a form. The solution is:
            \begin{equation}
                Yeah
            \end{equation}
            By the $18^{th}$ century the Italians were able to solve the general
            quartic equation. The next natural question is the solution to the
            quintic, but this was shown not to exist. The Abel-Ruffini theorem
            shows that the general quintic equation can not be solved using
            nested radicals. Galois went to prove that a polynomial has a root
            that can be written in terms of nested radicals if and only if
            $K/F$, the splitting field, has an automorphism group
            $\textrm{Aut}_{F}(K))$ that is solveable.
        \subsection{Some Reminders}
            \begin{definition}
                A field $(\mathbb{F},+,\cdot)$ is an Abelian group
                $(\mathbb{F},+)$ such that $(F^{*}\setminus\{0\},\cdot)$ is an
                Abelian group as well. This is the group of \textit{units}.
            \end{definition}
            \begin{example}
                The classic exmaples are $\mathbb{Q}$, $\mathbb{R}$, and
                $\mathbb{C}$, as well as the finite fields $\mathbb{F}_{p}$,
                also commonly denoted $\mathbb{Z}/p\mathbb{Z}$ or simply
                $\mathbb{Z}_{p}$.
            \end{example}
            \begin{definition}
                A field extension of a field $F$ is a field $K$ such that
                $F\subseteq{K}$. We may also say that $F$ is a subfield of $K$.
            \end{definition}
            We often denote that $K$ is a field extension of $F$ by writing
            $K/F$. This is not to denote a quotient or anything of that manner
            and is simply to denote that $F$ sis a subfield of $K$.
            \begin{example}
                $\mathbb{C}$ is a field extension of $\mathbb{R}$ since both are
                fields and $\mathbb{R}\subseteq\mathbb{C}$. We can go backwards,
                thinking of $\mathbb{R}$ as a field extension $\mathbb{R}$.
            \end{example}
            Also important, if $K$ is a field extension of $F$, $K/F$, then
            $K$ has the structure of an $F$ vector space. That is, $K$ can be
            seen as a vector space over $F$. One thing that we write is this
            bracket notation $[K:F]$, which again is not to be confused with
            the notation found in groups about the cardinality of certain
            things. $[K:F]$ is the simply the dimesnion of the vector space
            $K$ over $F$:
            \begin{equation}
                [K:F]=\textrm{dim}_{K}(F)
            \end{equation}
            This is also called the degree of the extension $K/F$. If the
            dimension is finite, $[K:F]<\infty$, we say that $K/F$ is a finite
            extension.
            \begin{example}
                $\mathbb{C}$ is a two dimensional vector space over $\mathbb{R}$
                and thus $[\mathbb{C},\mathbb{R}]=2$. To see this, use
                $\{1,i\}$ as a basis.
            \end{example}
            \begin{theorem}
                Any countable dimensional vector space over a countable field is
                also countable.
            \end{theorem}
            \begin{example}
                Using this theorem shows that $\mathbb{R}$, as a vector space
                over $\mathbb{Q}$, in not only an infinite dimensional vector
                space, but also has an uncountably infinite basis.
                Thus, $[\mathbb{R}:\mathbb{Q}]$ is uncountably infinite.
            \end{example}
            \begin{example}
                Consider $\mathbb{Q}[\sqrt{2}]$, defined by:
                \begin{equation}
                    \mathbb{Q}[\sqrt{2}]=\{x+y\sqrt{2}\;|\;x,y\in\mathbb{Q}\,\}
                \end{equation}
                This is a subfield of $\mathbb{R}$,
                $\mathbb{Q}[\sqrt{2}]\subseteq\mathbb{R}$. Addition and
                multiplication are easy enough to see, and 0 and 1 are contained
                in there, we need only check multiplicative inverses. But:
                \begin{equation}
                    (x+\sqrt{2}y)^{\minus{1}}=\frac{x-\sqrt{2}y}{x^{2}-2y^{2}}
                \end{equation}
                And $x^{2}-2y^{2}$ is only zero when $x=y=0$, since if
                $x^{2}-2y^{2}=0$, then rearrange this to obtain $q^{2}=2$. But
                by the arguments of the ancient Greeks, there is no rational
                number whose square is 2, and thus the denominator is never
                zero for non-zero rational ordered pairs.
            \end{example}
            \begin{example}
                $\mathbb{R}/\mathbb{Q}[\sqrt{2}]$ is uncountably infinite, but
                $\mathbb{Q}[\sqrt{2}]/\mathbb{Q}$ has degree 2 with a basis
                $\{1,\sqrt{2}\}$.
            \end{example}
        \subsection{Polynomials}
            We use $F[x]$ to denote the ring of polynomials with coefficients in
            $F$. For example:
            \begin{equation}
                f(x)=a_{n}x^{n}+a_{n-1}x^{n-1}+\cdots+a_{1}x+x_{0}
                \quad\quad
                a_{k}\in{F}
            \end{equation}
            Then $f\in{F}[x]$. The degree of a polynomial is the largest power
            of the polynomial with non-zero coefficient. Some things can be said
            about the degree of polynomials:
            \begin{align}
                \textrm{deg}(f+g)&\leq
                    \textrm{max}\{\textrm{deg}(f),\textrm{def}(g)\}\\
                \textrm{deg}(fg)&=\textrm{deg}(f)+\textrm{deg}(g)
                \quad\quad
                f,g\ne{0}
            \end{align}
            The degree of a polynomial is zero if and only if the polynomial is
            constant. Since $F[x]$ has a ring structure, $F[x]^{*}$ can be seen
            as the set of all non-zero constant polynomials.
            \begin{theorem}
                $F[x]$ is a Euclidean domain. That is, for any polynomial
                $f\in{F}[x]$ and for any non-zero $g\in{F}[x]$, there exist
                unique polynomials $r,q\in{F}[x]$ such that $f=qg+r$ where
                either $r=0$ or $\textrm{deg}(r)<\textrm{deg}(g)$.
            \end{theorem}
            \begin{theorem}
                The polynomial ring $F[x]$ is a principal ideal domain. That is,
                every ideal $I\subseteq{F}[x]$ is principal. That is, every
                ideal is generated by a single element.
            \end{theorem}
            \begin{theorem}
                Every Euclidean domain is a principle ideal.
            \end{theorem}
            Thus, there is a bijection between ideals $I\subseteq{F}[x]$ and
            monic polynomials in $F[x]$. Recall that if $R$ is a commutatie ring
            with unity, then $r\in{R}$ is called irreducible if $r\ne{0}$, $r$
            not a unit, and if $r=ab$, then either $a$ or $b$ is a unit. We take
            this definition to exclude some trivialities. For example, in
            $\mathbb{Z}$, 3 is irreducible, however
            $3=(\minus{1})\cdot(\minus{3})$. We don't care about this product,
            since $\minus{1}$ is a unit. Moreover, an element $r\in{R}$ is
            prime if $(r)\subseteq{R}$ is a prime ideal. That is if
            $r$ divides $ab$, then either $r$ divides $a$ or $r$ divides $b$.
            By divides, $r|a$, we mean that $a=r\cdot{s}$ for some $s\in{R}$.
            \begin{example}
                If $F$ is a field, $f\in{F}[x]$, then $f$ is irreducible if and
                only if $f$ is not the product of two polynomials with degrees
                strictly less than $f$. That is, if $f=gh$, then one of these
                must be a constant.
            \end{example}
            \begin{example}
                In $\mathbb{Z}$, prime if and only if irreducible.
            \end{example}
            \begin{theorem}
                If $R$ is a integral domain, and if $r$ is prime, then it is
                irreducible. That is, if there are no zero divisors then prime
                implies irreducible.
            \end{theorem}
            \begin{theorem}
                If $R$ is a principal ideal domain and if $r$ is irreducible,
                then the ideal generated by $r$ is maximal.
            \end{theorem}
            \begin{theorem}
                A maximal ideal is a prime ideal.
            \end{theorem}
            Recale that an ideal is called prime if $R/I$ is a domain. That is,
            if $ab\in{I}$, then either $a\in{I}$ or $b\in{I}$. A maximal ideal
            is and ideal that has no proper ideals between it and the entire
            ring. Another way to say this is that $R/I$ is a field. In other
            words, if $I\subseteq{J}\subseteq{R}$, then $I=J$. Using this we see
            that a maximal ideal is prime since $R/I$ will be a field, which is
            certainly an integral domain.
            \begin{theorem}
                The fourth isomorphism theorem says that if $I\subseteq{R}$ is
                an ideal, then there is a bijection between ideals containing
                $I$, $I\subseteq{J}\subseteq{R}$, and ideals of $R/I$.
            \end{theorem}
            \begin{theorem}
                $f\in{F}[x]$ is irreducible if and only if $F[x]/(f)$ is a
                field.
            \end{theorem}
            Note that $F$ can been seen as a subfield of $F[x]/(f)$ since
            $F$ can be identitified with all constant polynomials, which can
            further be seen tolive inside of $F[x]/(f)$.
            \begin{theorem}
                If $\overline{g}\in{F}[x]/(f)$ then there exists a unique
                $g_{0}\in{F}[x]$ such that$\textrm{deg}(g_{0})<\textrm{deg}(f)$,
                with $\overline{g_{0}}=\overline{g}$.
            \end{theorem}
            If $n$ is the degree of $f$, then the set
            $\{\overline{1},\overline{x},\dots,\overline{x^{n-1}}\}$ is a basis
            for $F[x]/(f)$ over $F$.
            \begin{theorem}
                If $f$ is irreducible of degree $n$, then
                $F[x]/(f)$ is a field extension of $F$ of degree $n$.
            \end{theorem}
            \begin{example}
                In $\mathbb{R}$, the polynomial $f(x)=x^{2}+1$ is irreducible
                since it cannot be factors any further. Thus
                $\mathbb{R}[x]/(x^{2}+1)$ is a field extension of $\mathbb{R}$
                of degree 2.
            \end{example}
            \begin{theorem}
                $\mathbb{R}[x]/(x^{2}+1)$ is isomorphic to $\mathbb{C}$.
            \end{theorem}
            \begin{proof}
                For since $\{\overline{1},\overline{x}\}$ is a basis, we have:
                \begin{equation}
                    \mathbb{R}[x]/(x^{2}+1)=
                        \{a\overline{1}+b\overline{x}\;|\;a,b\in\mathbb{R}\}
                \end{equation}
                So we trivial map $a\overline{1}+b\overline{x}$ to $a+bi$.
            \end{proof}
            \begin{example}
                Consider now $\mathbb{Q}[x]$ with $x^{2}-2$. This is irreducible
                since it cannot be factor ($\sqrt{2}$ is irrational). Then
                $\mathbb{Q}[x]/(x^{2}-2)$ is isomorphic to
                $\mathbb{Q}[\sqrt{2}]$.
            \end{example}
        \subsection{Review of Previous Lecture}
            If $F$ is a field, and if $f\in{F}[x]$ is irreducible, then
            $F[x]/(f)$ is a field extension of $F$ of degree $\textrm{deg}(f)$.
            Also, $\overline{x}=x+(f)\in{F}[x]/(f)$ is a root of $f(x)$ in this
            field $F[x]/(f)$. That is, $f(\overline{x})=\overline{f(x)}$, and
            this maps to zero. The fact that $f(\overline{x})=\overline{f(x)}$
            is simply the statement that the quotient ring is well defined.
    \section{A Bit More Algebra (Review)}
        \begin{definition}
            A unique factorization domain (UFD) $R$ is an integral domain
            (no zero divisors) such that for every element $r\in{R}$ with
            $r\ne{0}$ and $r$ not a unit, we have that $r=\prod_{i}r_{i}$ where
            $r_{i}$ are irreducible elements, and if $r=\prod_{j}s_{j}$ with
            $s_{j}$ irreducible, then the size of the products are the same, and
            $r_{i}=s_{i}\cdot{u}_{i}$ for some unit $u_{i}$.
        \end{definition}
        \begin{example}
            The fundamental theorem of arithmetic states the $\mathbb{Z}$ is a
            UFD.
        \end{example}
        \begin{theorem}
            If $F$ is a field, then $F[x]$ is a UFD.
        \end{theorem}
        \begin{proof}
            Any principal ideal domain is a unique factorization domain.
        \end{proof}
        \begin{example}
            There are non-unique factorization domains. For example,
            $\mathbb{Z}[\sqrt{\minus{5}}]$ since
            $6=2\cdot{3}=(1+\sqrt{\minus{5}})(1-\sqrt{\minus{5}})$.
        \end{example}
        The converse of the previous example is not true.
        \begin{theorem}
            $\mathbb{Z}[x]$ is a UFD.
        \end{theorem}
        However, $\mathbb{Z}[x]$ is not a PID since $(2,x)$ is not principal.
    \section{Roots and Irreducibility}
        \begin{definition}
            An element $\alpha\in{F}$ is a root of a polynomial $f(x)$ if
            $f(x)=0$.
        \end{definition}
        \begin{theorem}
            $\alpha\in{F}$ is a root of $f\in{F}[x]$ if and only if
            $(x-\alpha)$ divides $f$.
        \end{theorem}
        \begin{proof}
            One was is clear, if $(x-\alpha)|f$, then $f(x)=(x-\alpha)r(x)$ for
            some $r\in{F}[x]$. But then $f(\alpha)=0\cdot{r}(\alpha)=0$, and
            thus $\alpha$ is a root. In the other direction, use the division
            algorithm. Write $f(x)=(x-\alpha)q(x)+r(x)$. But $r(x)$ must have
            degree less than $x-\alpha$, and is therefor a constant. But then
            $f(\alpha)=r=0$, so $r=0$. Hence, $f(x)=(x-\alpha)q(x)$, so
            $x-\alpha$ divides $f$.
        \end{proof}
        \begin{theorem}
            If $f\in{F}[x]$ is a polynomial of degree $n$, then there are at
            most $n$ roots.
        \end{theorem}
        \begin{proof}
            By induction. If there are no roots, we are done. If not, write
            $f(x)=(x-\alpha)q(x)$. Then $q$ is a polynomial of degree $n-1$, and
            by the induction hypothesis has at most $n-1$ roots. Thus, there are
            at most $n$ roots.
        \end{proof}
        \begin{example}
            $x^{2}-1\in\mathbb{Z}_{8}[x]$ has 4 roots. That is, 1, 3, 5, 7
            are all roots. This does not contradict the previous theorem since
            $\mathbb{Z}_{8}$ is not a field.
        \end{example}
        \begin{theorem}
            The fundamental theorem of algebra states that every non-constant
            polynomial $f\in\mathbb{C}[z]$ has a root.
        \end{theorem}
        \begin{theorem}
            Any linear polynomial $ax+b\in{F}[x]$ is irreducible.
        \end{theorem}
        \begin{theorem}
            If $f\in{F}[x]$ is irreducible and $\textrm{deg}(g)\geq{2}$, then
            $f$ has no roots. Moreover, the converse holds if
            $\textrm{deg}(f)=2$ or 3.
        \end{theorem}
        \begin{example}
            The previous theorem is very special for degree 2 and 3. Let
            $f(x)=x^{2}+x+1$ where $f\in\mathbb{F}_{p}[x]$, $p$ a prime.
            For the case of $p=2$ have seen that this is a irreducible. For
            $p=3$ we have that 1 is a root: $1^{2}+1+1=3\cong{0}$ in
            $\mathbb{F}_{3}$. For $p=5$, this is once again irreducible,
            but not in $\mathbb{F}_{7}$.
        \end{example}
        \begin{example}
            Consider $x^{4}+x^{2}+1\in\mathbb{F}_{2}[x]$. This has no roots,
            since 0 and 1 both map to 1, but it is reducible since
            $x^{4}+x^{2}+1=(x^{2}+x+1)^{2}$ in $\mathbb{F}_{2}[x]$. That is,
            the so called \textit{freshman's dream} is true in this case, and
            we can distribute the power. Thus we have a reducible polynomial
            with no roots. Note, however, that the degree is not 2 or 3.
        \end{example}
        Next, we present Gauss's Lemma. Note that being irreducible over
        $\mathbb{Z}$ is stronger than being irreducible over $\mathbb{Z}$.
        Consider $f(x)=2x$. While this is linear in $\mathbb{Q}$, and since
        $\mathbb{Q}$ is a field, $f$ is irreducible in $\mathbb{Q}[x]$. However
        when viewed in $\mathbb{Z}[x]$, we have $2x=2\cdot{x}$, neither of
        which are units, and hence $f$ is reducible in $\mathbb{Z}[x]$.
        \begin{theorem}
            If $f\in\mathbb{Z}[x]$ is irreducible, then it is irreducible in
            $\mathbb{Q}[x]$.
        \end{theorem}
        \subsection{Reduction Modulo a Prime}
            Consider the projection map
            $\pi:\mathbb{Z}[x]\rightarrow\mathbf{F}_{p}[x]$, reduction mod $p$
            of the coefficients:
            \begin{equation}
                f(x)=\sum{a}_{k}x^{k}\longrightarrow
                    \overline{f}(x)\sum\overline{a}_{k}x^{k}
            \end{equation}
            The mapping $\pi$ is a ring homomoprhism.
            \begin{example}
                $x^{3}+21x+31\mapsto{x}^{3}+x+1$ in $\mathbb{F}_{2}[x]$.
            \end{example}
            \begin{theorem}
                If $f\in\mathbb{Z}[x]$, if $p$ is prime, and if $p\not|a_{n}$,
                and if $\pi(f)\in\mathbb{F}_{p}[x]$ is irreducible, then
                $f\in\mathbb{Z}[x]$ is irreducible.
            \end{theorem}
            \begin{proof}
                We prove by the contrapositive. If $f(x)=g(x)h(x)$, then
                $\overline{f}(x)=\overline{g}(x)\overline{h}(x)$ since $\pi$ is
                a homomoprhism. But $\overline{g}$ and $\overline{h}$ are not
                units since $p$ does not $a_{n}$, and thus the degree of the
                reduction is equal to the degree of the original polynomial.
                That is, $\textrm{deg}(f)=\textrm{deg}(\overline{f})$. But also
                $\textrm{deg}(g)=\textrm{deg}(\overline{g})$ and similarly for
                $h$ since:
                \begin{equation}
                    \textrm{deg}(\overline{g})+\textrm{deg}(\overline{h})
                    =\textrm{deg}(g)+\textrm{deg}(h)=\textrm{deg}(f)
                \end{equation}
                Since $p$ does not divide the leading coefficients of either
                $g$ or $h$, the degrees of $\overline{g}$ and $\overline{h}$
                remain the same, and hence are not units. Thus, $\overline{f}$
                is reducible.
            \end{proof}
            \begin{example}
                Consider $x^{3}+21x+31\in\mathbb{Z}[x]$. In $\mathbb{F}_{3}[x]$
                we have $x^{3}+x+1$, which is indeed irreducible, and hence
                the original polynomial $x^{3}+21x+31$ is irreducible.
            \end{example}
            \begin{example}
                The converse is not true. There are polynomials that are
                reducible for all $p$, yet in $\mathbb{Z}[x]$ it is irreducible.
                Consider $(2x+1)(x+1)=2x^{2}+3x+1$. This is reducible, since
                we have the factorization, however in $\mathbb{F}_{2}[x]$ this
                is simply $x+1$, which is irreducible. However, 2 divides 2 so
                the theorem does not apply.
            \end{example}
            \begin{example}
                In $\mathbb{Z}[x]$, consider $x^{2}+x+1$ which we know is
                irreducible since it is irreducible in $\mathbb{R}[x]$
                (the roots are complex). In $\mathbb{F}_{2}[x]$ we can check and
                see that there are no roots, and thus $x^{2}+x+1$ is irreducible
                by Gauss' lemma. However, in $\mathbb{F}_{3}[x]$ it is reducible
                since $x^{2}+x+1=(x-1)^{2}$. That is, 1 is a root of $x^{2}+x+1$
                in $\mathbb{F}_{2}[x]$ with multiplicity 2.
            \end{example}
            \begin{ftheorem}{Eisenstein's Criterion}{Eisensteins_Criterion}
                If $p$ is prime, if $f\in\mathbb{Z}[x]$, if $p$ does not divide
                $a-{n}$, if $p$ divides $a_{k}$ for all $k<n$, and if
                $p^{2}$ does not divide $a_{0}$, then $f$ is irreducible.
            \end{ftheorem}
            \begin{bproof}
                We prove by the contradiction. Suppose $f$ is reducible.
                If $p$ divides all of the $a_{i}$, then in the reduction map
                $\pi:\mathbb{Z}[x]\rightarrow\mathbb{F}_{p}[x]$, all of the
                $a_{i}$ map to zero. So we have:
                \begin{equation}
                    \overline{a}_{n}x^{n}=\overline{f}(x)
                    =\overline{g}(x)\overline{h}(x)
                \end{equation}
                But since $\mathbb{Z}[x]$ is a UFD, both $\overline{g}$ and
                $\overline{h}$ are monomials, $\overline{g}(x)=g_{0}x^{i}$ and
                $\overline{h}(x)=h_{0}x^{j}$. If $i,j<n$, then $i,j>0$, and so
                both constant terms must be divisible by $p$. Thus
                $g(x)h(x)$ has a constant term divisible by $p^{2}$, a
                contradiction.
            \end{bproof}
            \begin{example}
                Let $f(x)=x^{4}+22x^{2}+33x+44$. By Eisenstein, with $p=11$, we
                have that this is irreducible.
            \end{example}
        \subsection{Review of Previous Lecture}
            If $\mathbf{F}$ is a field, then $\zeta\in\mathbf{F}$ is called a
            root of unity of there is some $n\geq{1}$ such that $\zeta^{n}=1$.
            \begin{example}
                In the field $\mathbb{R}$, the roots of unity of 1 and
                $\minus{1}$. The number 1 is a first root of unity, whereas
                $\minus{1}$ is a second root of unity.
            \end{example}
            \begin{example}
                In the complex numbers $\mathbb{C}$, there is an $n^{th}$ root
                of unity for all $n\in\mathbb{N}^{+}$. Let
                $\zeta=\exp(2\pi{i}/n)$. The roots of unity are scattered along
                the unit circle.
            \end{example}
            A root of unity is some element $\zeta\in\mathbb{F}$ such that
            $\zeta$ is a root of the polynomial $f(x)=x^{n}-1$.
            \begin{example}
                Let $p$ be a prime integer, and consider the roots of
                $f(x)=x^{p}-1$. There is always a root since 1 satisfies this
                criterion, and hence we can factor this and obtain:
                \begin{equation}
                    f(x)=x^{p}-1=(x-1)(x^{p-1}+x^{p-2}+\dots+x+1)
                \end{equation}
                This latter polynomial $x^{p-1}+x^{p-1}+\dots+x+1$ is
                irreducible over $\mathbb{Q}$.
            \end{example}
            \begin{theorem}
                If $p$ is prime and $f(x)=x^{p-1}+x^{p-2}+\dots+x+1$,
                $f\in\mathbb{R}[x]$, then $f$ is irreducible over $\mathbb{Q}$.
            \end{theorem}
            \begin{proof}
                For:
                \begin{equation}
                    xf(x+1)=(x+1)^{p}-1
                \end{equation}
                by the binomial theorem we have:
                \begin{equation}
                    xf(x+1)=(x+1)^{p}-1=
                    \minus{1}+\sum_{k=0}^{p}\binom{p}{k}x^{k}
                    =\sum_{k=1}^{p}\binom{p}{k}x^{k}
                \end{equation}
                Thus, simplifying, we have:
                \begin{equation}
                    f(x+1)=\sum_{k=1}^{p}\binom{p}{k}x^{k-1}
                \end{equation}
                Thus by the Eisenstein criterion, $f(x+1)$ is irreducible over
                $\mathbb{Q}$. But if $f(x+1)$ is irreducible over $\mathbb{Q}$,
                then $f(x)$ is as well.
            \end{proof}
            The converse of the statement before is not true. If the translation
            of a polynomial is reducible, it need not mean the original
            polynomial was reducible. For let $f(x)\in\mathbb{F}_{2}[x]$ be
            defined by $f(x)=x^{2}+x+1$. Plugging in $x^{2}$ we get
            $f(x)^{2}=x^{4}+x^{2}+1=(x^{2}+x+1)^{2}$, which is reducible.
    \section{Gauss's Lemma}
        \begin{definition}
            A polynomial $f\in\mathbb{Z}[x]$ is called primitive if
            $\textrm{GCD}(a_{0},\dots,a_{n})=1$. Equivalently, for all primes
            $p$ there is an $i$ such that $p\not|a_{i}$. That is, $p$ does not
            divide $a_{i}$.
        \end{definition}
        \begin{example}
            Let $f(x)=10x^{3}+5x^{2}+2x+23$. This is a primitive polynomial.
            If $p$ divides all of the $a_{i}$, it must divide 2, but 2 is the
            only prime that divides 2, and hence $p=2$. But 5 and 23 are odd,
            and hence 2 does not divide them. So, $f$ is primitive.
        \end{example}
        \begin{ltheorem}{Gauss's Lemma}{Gauss_Lemma}
            FOr any $f\in\mathbb{Q}[x]$ there is a unique $c\in\mathbb{Q}$ and
            a unique primitive polynomial $g\in\mathbb{Z}[x]$ such that
            $f(x)=c\cdot{g}(x)$.
        \end{ltheorem}
        \begin{proof}
            Existence is straight forward. Clear out the denominators of $f$,
            let $c$ be the common factor, and we're done. Suppose
            $cg(x)=c'g'(x)$. If $c$ and $c'$ are not integers, multiply by their
            denominators and thus we may assume $c$ and $c'$ are integers. Then
            we have:
            \begin{equation}
                cg(x)=c'(b_{0}+b_{1}x+\dots+b_{n}x^{n})
            \end{equation}
            And hence:
            \begin{equation}
                c=\textrm{GCD}(c'b_{0},\dots,c'b_{n})
                 =c\textrm{GCD}((b_{0},\dots,b_{n}))
            \end{equation}
            But $g'$ is trivial, so the greatest common denominator is 1. Hence,
            $c=c'$. Also, $g=g'$.
        \end{proof}
        \begin{ltheorem}{Gauss' Lemma Version 2}{Gauss_Lemma_2}
            If $g\in\mathbb{Z}[x]$ is primitive and $f\in\mathbb{Z}[x]$, and if
            $g|f$ in $\mathbb{Q}[x]$, then $g|f$ in $\mathbb{Z}[x]$.
        \end{ltheorem}
        \begin{proof}
            For if $g|f$, then $f=gh$ with $h\in\mathbb{Q}[x]$. But by Gauss'
            lemma there is a unique $c\in\mathbb{Q}$ such that
            $h(x)=ch_{0}(x)$, where $h_{0}\in\mathbb{Z}[x]$ is primitive. So
            $f(x)=cg(x)h_{0}(x)$. But then $g(x)h_{0}(x)$ is primitive, and
            hence $c$ is an integer. Thus, $h\in\mathbb{Z}[x]$. That is, since
            the product of primitive is primitive, the numerator of $c$ is equal
            to the denominator of $c$ times the GCD of the coefficients of $f$.
            But this GCD is a positive integer, and hence the numerator divides
            it, and hence $c$ is an integer itself.
        \end{proof}
        \begin{ltheorem}{Gauss' Lemma V3}{Gauss_Lemma_3}
            If $f\in\mathbb{Z}[x]$ and if $g,h\in\mathbb{Q}[x]$ are such that
            $f=gh$, then $f=g_{0}h_{0}$.
        \end{ltheorem}
    \section{Field Extensions}
        \begin{ftheorem}{Tower Law}{Tower_Law}
            If $\mathbb{F}$, $\mathbb{K}$, and $\mathbb{L}$ are fields, if
            $\mathbb{K}$ is a field extension of $\mathbb{F}$, if $\mathbb{L}$
            is a field extension over $\mathbb{K}$, then $\mathbb{L}/\mathbb{F}$
            is a finite dimensional vector space if and only if
            $\mathbb{L}/\mathbb{K}$ and $\mathbb{K}/\mathbb{F}$ is finite.
            Moreover:
            \begin{equation*}
                [\mathbb{L}:\mathbb{F}]=
                [\mathbb{L}:\mathbb{K}][\mathbb{K}:\mathbb{F}]
            \end{equation*}
        \end{ftheorem}
        Suppose $K/F$ is a field extension. That is, $F\subseteq{K}$ and $F$ is
        a subfield of $K$. Let $X\subseteq{K}$ be any non-empty subset. We now
        define the adjoinment of $X$ to $F$.
        \begin{definition}
            The field extension generated by a subfield $F$ of a field $K$ by a
            subset $X\subseteq{K}$ is the subfield:
            \begin{equation}
                F(X)=
                \bigcap_{\overset{F\subseteq{L}\subseteq{K}}{X\subseteq{L}}}L
            \end{equation}
            Where $L$ is a subfield.
        \end{definition}
        $F(X)$ is non-empty since $K$ is in the intersection, and the
        intersection of subfields is again a subfield, so $F(X)$ is a field.
        If $X=\{a_{0},a_{1},\dots\}$, we often write $F(X)=F(a_{0},a_{1},\dots)$
        and if $X$ if finite we call $F(X)/F$ finitely generated. If $X$ is a
        single element, $F(\alpha)$ is called a simple extension. A field
        extension is simple if it can be written as a simple extension.
        \begin{example}
            $\mathbb{R}/\mathbb{Q}$ is not finitely generated. A simple
            cardinality argument works here, for if it were countably generated
            then since $\mathbb{Q}$ is countable, $\mathbb{R}$ would again be
            countable, which is false.
        \end{example}
        \begin{example}
            $\mathbb{Q}(\sqrt{2},\sqrt{3},\sqrt{5},\sqrt{7},\dots)$ is not
            finitely generated. We can show this by building a chain of
            subfields. First consider $\mathbb{Q}(\sqrt{2})/\mathbb{Q}$. This
            is a field extension of degree two. One can see this since
            $\sqrt{2}$ is not rational. The next field extension
            $\mathbb{Q}(\sqrt{2},\sqrt{3})/\mathbb{Q}(\sqrt{2})$ is also an
            extension of degree 2. For suppose not and suppose:
            \begin{equation}
                \sqrt{3}=a+b\sqrt{2}
            \end{equation}
            Squaring, we obtain:
            \begin{equation}
                3=a^{2}+2b^{2}+2ab\sqrt{2}
            \end{equation}
            Now since $\sqrt{2}$ is irrational we must conclude that $ab=0$.
            Thus either $a=0$ or $b=0$. f $b=0$, then $\sqrt{3}$ is an integer,
            which is false. If $a=0$ then $3=2b^{2}$, but $3$ is prime, a
            contradiction. Thus $\sqrt{3}$ is not contained in
            $\mathbb{Q}(\sqrt{3})$.
        \end{example}
        \begin{theorem}
            If $F$ is a field, if $K$ is a field extension, if $\alpha\in{K}$,
            then:
            \begin{equation}
                F(\alpha)=\big\{\frac{f(\alpha)}{g(\alpha)}\;|\;
                    f,g\in{F}[x],g(\alpha)\ne{0}\big\}
            \end{equation}
        \end{theorem}
        \begin{theorem}
            A field extension of a field $F$ is a field extension $K$ of $F$
            is an injective homomoprhism $\iota:F\rightarrow{K}$.
        \end{theorem}
        \begin{definition}
            A morphism of field extensions $K/F$ and $K'/F$ is a homomoprhism
            $\varphi:K\rightarrow{K}'$ such that the injective homomoprhisms
            $\iota$ and $\iota'$ commute with $\varphi$. That is,
            $\iota'=\varphi\circ\iota$.
        \end{definition}
        \begin{example}
            $\mathbb{C}$ and $\mathbb{R}[x]/(x^{2}+1)$ are isomorphic field
            extensions of $\mathbb{R}$.
            $\varphi\mathbb{R}[x]\rightarrow\mathbb{R}$ mapping
            $f\mapsto{f}(i)$ is surjective since $a+ib=a(i)+b(i)^{4}$. Thus this
            is a surjective $\mathbb{R}$ algebra homomoprhism. The kernel is
            the ideal generated by $x^{2}+1$. Then by the first isomorphism
            theorem,
            $\tilde{\varphi}:\mathbb{R}[x]/(x^{2}+1)\rightarrow\mathbb{C}$ is an
            isomorphism. Hence, these are isomorphic.
        \end{example}
    \section{Minimal Polynomial}
        If $F\subseteq{K}$ is a subfield, if $\alpha\in{K}$, then there exists
        an $F$ algebra homomoprhism $\varphi_{\alpha}:F[x]\rightarrow{K}$.
        Then $\textrm{Ker}(\varphi_{\alpha})\subseteq{F}[x]$. If
        $\varphi_{\alpha}$ is injective, then the kernel is 0, and this is
        true if and only if $f(\alpha)\ne{0}$ for all $f\in{F}[x]$. That is,
        $\alpha$ is \textit{transcendental} over $F$. On the other hand, if
        $\varphi_{\alpha}$ is not injective then
        $\textrm{Ker}(\varphi_{\alpha})=(m_{\alpha}/F(x))$ where
        $m_{\alpha}/F(x)$ is a monic polynomial in $F[x]$, and this is called
        the minimal polynomial of $\alpha$ over $F$. In this case we say that
        $\alpha$ is algebraic over $F$.
        \begin{theorem}
            If $K/F$ is a field extension, $\alpha\in{K}$ is algebraic over $F$,
            if $f\in{F}[x]$ with $f(\alpha)=0$, then
            $m_{\alpha/F}(x)|f$, hence $m_{\alpha}/F(x)$ is the unique monic
            polynomial over $F$ of minimal degree with $\alpha$ as a root.
            Moreover, $m_{\alpha}/F(x)$ is irreducible over $F$ and is the
            unique monic irreducible polynomial over $F$ with $\alpha$ as a
            root.
        \end{theorem}
        \begin{example}
            Let $d\in\mathbb{Z}$ be a non-square. Then the minimal polynomial of
            $d$ is $m_{\sqrt{d}}/\mathbb{Q}(x)=x^{2}-d$.
        \end{example}
        \begin{example}
            Let $\alpha=\sqrt{2}+\sqrt{3}$. What is the minimal polynomial of
            $\alpha$ over $\mathbb{Q}$? Well we first try to find a polynomial
            that has $\alpha$ as a root. Squaring, we have:
            \begin{equation}
                \alpha^{2}=2+2\sqrt{6}+3=5+2\sqrt{6}
            \end{equation}
            Bringing the 5 over and squaring:
            \begin{equation}
                (\alpha^{2}-5)^{2}=24=\alpha^{4}-10\alpha^{2}+25
            \end{equation}
            Thus letting $f(x)=x^{4}-10x+1$, we obtain a polynomial with
            $\alpha$ as a root.
        \end{example}
        \begin{theorem}
            If $\alpha\in{K}$ is algebraic over $F$, then
            $F[x]/(m_{\alpha}/F(x))$ is isomorphic to $F[\alpha]$, and
            $F[\alpha]=F(\alpha)$.
        \end{theorem}
        Since $F(\alpha)\subseteq{F}[\alpha]$ because the first part,
        $F[x]$ is a field since $m_{\alpha}/F(x)$ is irreducible, so
        $F[x]/(m_{\alpha}/F(x))$ is a field. Hence
        $F(\alpha)\subseteq{F}[\alpha]$ since $F(\alpha)$ is minimal field
        containing $\alpha$. GIven $\alpha$ algebraic over $F$,
        $\alpha^{\minus{1}}$ is a polynomial in $\alpha$. There exists an
        algorithm demonstrating this using Bezout's identity.
        \begin{definition}
            A field extension $K/F$ is algebraic if for all $\alpha\in{K}$,
            $\alpha$ is algebraic over $F$.
        \end{definition}
        \begin{theorem}
            $K/F$ finite if and only if $K/F$ is algebraic and finitely
            generated.
        \end{theorem}
    \section{Review of Previous Lectures}
        A field extension $K/F$ is algebraic if every element $\alpha\in{K}$ is
        algebraic over $F$.
        \begin{theorem}
            A field extension $K/F$ is finite if and only if $K/F$ is algebraic
            and finitely generated.
        \end{theorem}
        \begin{proof}
            For if $K/F$ is finite, then $K$ is a finite dimensional vector
            space over $F$. Thus, if $\alpha\in{K}$, then the set
            $\{1,\alpha,\alpha,\dots\}$ is linearly dependent. That is, there
            exists $a_{0},\dots,a_{n}\in{F}$ such that
            $a_{0}+\dots+a-{n}\alpha^{n}=0$. Let $\alpha_{1},\dots,\alpha_{m}$
            be an $F$ basis for $K$. Then $K=F(\alpha_{1},\dots,\alpha_{n})$,
            and so $K$ is finitely generated. The converse is trickier. Since
            $K$ is finitely generated, $K=F(\alpha_{1},\dots,\alpha_{n})$. But
            since $K$ is algebraic over $F$, $\alpha_{i}$ is algebraic in $F$.
            Thus we build a tower:
            \begin{equation}
                F\rightarrow{F}(\alpha_{1})\rightarrow{F}(\alpha_{1},\alpha_{2})
                    \rightarrow\dots\rightarrow{F}(\alpha_{1},\dots,\alpha_{n})
                    =K
            \end{equation}
            By the generalized tower law, $K$ is finite over $F$.
        \end{proof}
        $K/F$ is transcendental (that is, not algebraic) implies that $K/F$ is
        of infinite degree. Given $\alpha\in{K}$ transcendental over $F$, then
        $\varphi_{\alpha}:F[x]\rightarrow{F}[\alpha]\subseteq{F}(\alpha)$.
        \begin{theorem}
            If $K/L$ is algebraic and $L/F$ is algebraic, then $K/F$ is
            algebraic.
        \end{theorem}
        \begin{proof}
            For let $\alpha\in{K}$. We want to show that $\alpha$ is algebraic
            over $F$. This is equivalent to the claim that $F(\alpha)$ is finite
            over $F$. But $\alpha$ is algebraic over $L$, and hence there is a
            minimal polynomial
            $m_{\alpha}/L(x)=a_{0}+\dots+a_{n-1}x^{n-1}+x^{n}\in{L}[x]$. Since
            $L/F$ is algebraic, $a_{i}$ is algebraic over $F$.
        \end{proof}
        The converse is true as well.
    \section{Compass and Straight Edge}
        Consider some subset $S\subseteq\mathbb{C}$, equipped with some rules:
        \begin{itemize}
            \item Given two points $P,Q$ there is a line through $P$ and $Q$.
            \item Given $P,Q$, there is a circle $C(P,|Q-P|)$ centering at $P$
                  with radius $|P-Q|$.
        \end{itemize}
        Any point that is the intersection of any of the lines and circles is
        said to be constructible by compass and straightedge.
        \begin{example}
            Bisecting a line is possible. Bisecting an angle is possible. Can
            you trisect an angle? What about double a cube?
        \end{example}
        We'll define inductively some sets $P_{n}$, $L_{n}$, and $C_{n}$.
        $P_{0}=\{0,1\}\subseteq\mathbb{C}$, $L_{0},C_{0}=\emptyset$. If
        $L_{n}$ has been constructed, let $L_{n+1}$ be the set of all lines
        through all points in $P_{n}$. If $C_{n}$ has been constructed, let
        $C_{n+1}$ be the set of all circles about all points in $P_{n}$ with
        radii all of the distances $|z_{1}-z_{2}|$ for points
        $z_{1},z_{2}\in{P}_{n}$. Then $P_{n}$ is finite for all $n\in\mathbb{N}$
        and thus the union over all $P_{n}$ is at most countable. This
        cardinality argument shows that there are inconstructible numbers.
        Moreover, $P=\bigcup{P}_{n}$ is a subfield of $\mathbb{C}$. To see this
        we must show that $P$ is closed to addition, multiplication, and
        inverses.
        \begin{theorem}
            The set of constructible numbers is a subfield of $\mathbb{C}$.
        \end{theorem}
        \begin{proof}
            It suffices to show that $P\cap\mathbb{R}$ is a subfield. For let
            $a,b\in{P}\cap\mathbb{R}$. We need to show that $a+b$, $a\cdot{b}$,
            and $a/b$ are constructible. Given the length $a$ and the length
            $b$, we can translate the length $b$ to start at $a$, given us the
            point $a+b$, which will have length $a+b$. For $ab$ we construct
            two triangles, one with length 1 and the other with length $a$.
            We build a triangle on the first with a length $b$, and a similar
            triangle on the second length which will have length $ab$ by
            similiarity. Lastly, do the same triangle with $b$ on the inside to
            get $a/b$. Draw some pictures later.
        \end{proof}
        \begin{theorem}
            $\mathbb{Q}\subseteq{P}$.
        \end{theorem}
        \begin{proof}
            Since $1\in{P}$, every integer multiple of 1 is also contained in
            $P$ since we can add 1 to itself $n$ times. Thus $n/m$ is contained
            in $P$, so $\mathbb{Q}\subseteq{P}$.
        \end{proof}
        \begin{theorem}
            $P$ is closed to square roots.
        \end{theorem}
        \begin{proof}
            Draw a circle of diameter $a+1$. Do that fancy circle.
        \end{proof}
        Let $Q^{py}$ be the intersection of all subfields $K\subseteq\mathbb{C}$
        such that for all $z\in{K}$ it is true that $\sqrt{z}\in{K}$. This is
        the smallest subfield of $\mathbb{C}$ in which one can always take
        square roots. It's called the Pythagorean closure of $\mathbb{Q}$. From
        the previous theorem, $Q^{py}\subseteq{P}$ is a subfield.
        \begin{theorem}
            $\mathbb{Q}^{py}=P$.
        \end{theorem}
        \begin{proof}
            For let $z\in{P}$. It suffices to show that
            $P_{n}\subseteq\mathbb{Q}^{n}$ for all $n$, since the
            $\bigcup{P}_{n}=P\subseteq\mathbb{Q}^{py}$. We prove this by
            induction. The base case is true since $P_{0}=\{0,1\}$ and this is a
            subset of $\mathbb{Q}^{py}$. Suppose $P_{n}\subseteq\mathbb{Q}^{py}$
            and recall that $P_{n+1}$ is defined as all $z\in\mathbb{C}$ such
            that $z\in{L}\cap{L}'$, or $z\in{L}\cap{C}$, or $z\in{C}\cap{C}'$,
            where $L$, $L'$, $C$, and $C'$ are lines and circles through points
            in $P_{n}$. Case 1, $z\in{L}\cap{L}'$ Then there are four points
            $z_{1},z_{2},z_{3},z_{4}$ such that:
            \begin{equation}
                z=z_{1}\alpha+(1-\alpha)z_{2}=z_{3}\beta+(1-\beta)z_{4}
            \end{equation}
            We can solve this and note that $P_{n}$ is closed under complex
            conjugation. Thus, in case 1 we have that $z$ is contained in
            $\mathbb{Q}^{py}$. Here we have:
            \begin{equation}
                z=z_{1}\alpha+(1-\alpha)z_{2}=z_{3}+r\exp(i\theta)
            \end{equation}
            This be true as well. The final case is two circles:
            \begin{equation}
                z=z_{1}+r_{1}\exp(i\theta_{1})=z_{2}+r_{2}\exp(i\theta_{2})
            \end{equation}
            Which also be like it is.
        \end{proof}
        \begin{theorem}
            $\mathbb{Q}^{py}$ is algebraic over $\mathbb{Q}$.
        \end{theorem}
        We can build the Pythagorean closure from $\mathbb{Q}$ by considering
        $\sqrt{\mathbb{Q}}$, all numbers of the for $a+b\sqrt{c}$ with
        $a,b,c\in\mathbb{Q}$, and then $\sqrt{\sqrt{\mathbb{Q}}}$, and so on.
        \begin{theorem}
            $z\in\mathbb{Q}^{py}$ if and only if there is a tower of extensions
            $K_{0},\dots,K_{n}$ such that $\mathbb{Q}=K_{0}$ and
            $K_{n}=\mathbb{Q}[z]$ where $K_{j+1}$ has degree 2 over $K_{j}$.
        \end{theorem}
        \begin{theorem}
            If $\alpha\in\mathbb{Q}^{py}$, then
            $[\mathbb{Q}[\alpha]:\mathbb{Q}]=2^{n}$ for some $n\in\mathbb{N}$.
        \end{theorem}
        \begin{proof}
            Apply the tower law to the previous theorem.
        \end{proof}
        From this we can do all of the impossibility proofs of various
        constructions.
        \begin{example}
            It is impossible to double the volume of a cube with lengths 1. The
            minimal polynomial of $\sqrt[3]{2}$ is $x^{3}-2$ by Eisenstein. Thus
            the degree of the field extension
            $[\mathbb{Q}[\sqrt[3]{2}]:\mathbb{Q}]=3$ which is not a power of 2.
            It is also impossible to trisect angles. For let $\theta=2\pi/3$.
            Trisecting this is equivalent to constructed $2\pi/9$. The minimal
            polynomial of this is $x^{3}-\frac{3}{4}x+\frac{1}{8}$, and thus
            the degree of the extension is again 3, which is not a power of 2.
        \end{example}
    \section{Review}
        If $f\in{F}[x]$ is irreducible, then $f$ has no roots. The converse is
        false. If $f\in{F}[x]$ has no roots, it may still be reducible. This
        holds for any field if $f$ has degree 2 or 3.
        \begin{example}
            If $f(x)=x^{4}+1$, $f\in\mathbb{R}[x]$, then $f$ is reducible. For
            we have:
            \begin{equation}
                x^{4}+1=(x^{2}-\sqrt{x}x+1)(x^{2}+\sqrt{2}x+1)
            \end{equation}
            We can now complete this since $x^{2}-\sqrt{2}x+1$ and
            $x^{2}+\sqrt{2}x+1$ have no real roots by the quadratic formula,
            and hence these are irreducible. Thus, $x^{4}+1$ is reducible and
            factors as above.
        \end{example}
        \begin{example}
            There is only one irreducible quadratic polynomial over
            $\mathbb{F}_{2}[x]$ and that is $x^{2}+x+1$. Using this, is the
            polynomial $x^{4}+x^{2}+x+1$ irreducible in $\mathbb{F}_{2}[x]$?
            Since this has no roots, we know that it has no linear factors, and
            thus if it is reducible it must be the product of irreducible
            quadratics. But there is only one irreducible quadratic, so we can
            use this to check. We have:
            \begin{equation}
                (x^{2}+x+1)^{2}=
                x^{4}+2x^{3}+x^{2}+2x+1=x^{4}+x^{2}+1
            \end{equation}
            Since in $\mathbb{F}_{2}$ we have that $2=0$. But this is not equal
            to the original polynomial $x^{4}+x^{3}+x^{2}+x+1$, and hence
            this is indeed irreducible.
        \end{example}
    \section{More Stuff}
        If $\alpha\in\mathbb{C}$ is constructible, and if
        $K_{i}$ is a tower of field extensions each of degree 2 over the
        previous one, then $[\mathbb{Q}(\alpha):\mathbb{Q}]=2^{n}$.
        \begin{theorem}
            A regular $n$ gon is constructible if and only if the complex number
            $\exp(2\pi{i}/n)$ is constructible.
        \end{theorem}
        If $p$ is a prime number, then the minimal polynomial of
        $\exp(i2\pi{i}/n)$ is just $x^{p-1}+\dots+1$. Thus
        $[\mathbb{Q}(\zeta_{p}):\mathbb{Q}]=p-1$. Thus we need $p-1$ to be a
        power of 2.
        \begin{theorem}
            If $p-1$ is not a power of 2 then we can not construct a regular
            $p$ gone.
        \end{theorem}
        \begin{theorem}
            If $2^{k}+1$ is a prime, then $k=2^{n}$ for some $n$.
        \end{theorem}
        There are only 5 Fermat primes, $p=3,5,17,257,65537$. An unsolved
        problem (as of 2020) is whether or not there are more such primes, or
        are there infinitely many primes? 
    \section{Splitting Fields}
        If $f\in{F}[x]$, we have given a construction of $K/F$ in which $f$ has
        a root $\alpha\in{K}$. That is, $(x-\alpha)$ divides $f$ in$ K[x]$. For
        if $f$ splits completely (factors into a product of linear polynomials
        over $f$), then there is nothing to prove. If not, let $g(x)$ be an
        irreducible factor of $f$ and define $K=F[x]/(g)$, where $(g)$ is the
        ideal generated by $g$ in $F[x]$. Then $g(x)$ has a root $\overline{x}$,
        $K=F(\alpha)$, so $(x-\alpha)$ divides $g$, which divides $f$, so we're
        done.
        \begin{example}
            Let $f(x)=x^{3}-2$, $f\in\mathbb{Q}[x]$. This has three roots,
            $\sqrt[3]{2}$, $\sqrt[3]{2}\omega$, and
            $\sqrt[3]{2}\overline{\omega}$, where $\omega$ is a complex cubed
            root of unity, and $\overline{\omega}$ is its complex conjugate.
            Thus $\mathbb{Q}[x]/(x^{3}-2)$ is isomorphic to
            $\mathbb{Q}(\sqrt[3]{2})$.
        \end{example}
        \begin{definition}
            An extension $K/F$ is a splitting field of $f\in{F}[x]$ if
            $f$ splits completely in $K[x]$ and does not split in any proper
            subfield.
        \end{definition}
        \begin{theorem}
            Splitting fields exist.
        \end{theorem}
        \begin{proof}
            Proof by induction on the degree of $n$. If it is true for $n$,
            write $f=(x-\alpha)g$ for some $\alpha$ in $F[x]/(h)$. Then
            $g$ is of degree $n$ or less, and hence has a splitting field
            $K$. Adjoing $\alpha$ to $K$. Let $E$ be the intersection of all
            subfields of $E$ containing this $K$ adjoing $\alpha$.
        \end{proof}
    \section{Notes from Milne (Chapter 1)}
        Def of rings, subrings, ring homomorphisms, commutative rings.
        \begin{definition}
            An integral domain is a ring $\ring{R}$ such that for all
            $a,b\in{R}$ such that $a\cdot{b}=0$, it is true that either $a=0$ or
            $b=0$.
        \end{definition}
        \begin{definition}
            An ideal of a ring $\ring{R}$ is a subring $\ring[I]{I}$ such that
            for all $a,b\in{I}$ it is true that $a+b\in{I}$, and for all
            $r\in{R}$ and for all $a\in{I}$ it is true that $r\cdot{a}\in{I}$
            and $a\cdot{r}\in{I}$.
        \end{definition}
        \begin{definition}
            A field is a commutative division ring $\ring{F}$.
        \end{definition}
        \begin{example}
            Since by definition a division ring is a non-zero ring, fields are
            required to have at least two elements. The smallest field is thus
            $\ring{\mathbb{Z}_{2}}$ with modulo arithmetic.
        \end{example}
        \begin{theorem}
            If $\ring{F}$ is a commutative ring, then it is a field if and only
            if there only ideals are the zero ideal and the entire ring.
        \end{theorem}
        \begin{proof}
            For if $\ring{F}$ is a field then for all $a\in{F}$ such that
            $a\ne{0}$ there exists a multiplicative inverse
            $a^{\minus{1}}\in{F}$ such that $a\cdot{a}^{\minus{1}}=1$. But then
            if $\ring[I]{I}$ is a non-zero subring of $R$, there is a non-zero
            element $a\in{I}$, and thus $a\cdot{a}^{\minus{1}}\in{I}$, and
            therefore $1\in{I}$. But then for all $r\in{F}$, $r\cdot{1}\in{I}$,
            but $r\cdot{1}=r$. Hence, $I=R$. In the other direction, if
            the only ideals are the zero ideal and the entire ring, then for
            any non-zero $a\in{F}$ the ideal generated by $a$ must be the entire
            ring. Hence, there is a $b\in{I}$ such that $a\cdot{b}=1$, and so
            $1\in{I}$, and hence $I$ is the entire ideal.
        \end{proof}
        \begin{example}
            Fields: $\mathbb{Q}$, $\mathbb{R}$, $\mathbb{C}$, $\mathbb{Z}_{p}$
            with $p$ a prime.
        \end{example}
        Def field homomoprhism, same as rings.
        \begin{theorem}
            If $\ring[F]{F}$ and $\ring[K]{K}$ are fields, and if
            $\phi:F\rightarrow{K}$ is a field homomoprhism, then it is
            injective.
        \end{theorem}
        \begin{proof}
            For $\phi^{\minus{1}}[\{0_{K}\}]$ is an ideal in $F$, and since it
            doesn't contain 1 by the definition of a field homomoprhism, it
            must be a proper ideal. Hence, it is the zero ideal. But then if
            $\phi(a)=\phi(b)$, then $\phi(a-b)=0_{K}$, and thus $a-b=0_{F}$.
            That is, $a=b$.
        \end{proof}
        \begin{theorem}
            If $\ring[F]{F}$ is a field, if $\ring{\mathbb{Z}}$ is the standard
            ring of integers, if $\ring{\mathbb{Q}}$ is the field of
            rational numbers, and if $f:\mathbb{Z}\rightarrow{F}$ is an
            injective ring homomoprhism, then there exists a field homomoprhism
            $\tilde{f}:\mathbb{Q}\rightarrow{F}$.
        \end{theorem}
        \begin{proof}
            For any $n\in\mathbb{Z}$, we have:
            \begin{equation}
                f(n)=f\Big(\sum_{k\in\mathbb{Z}_{n}}1\Big)
                =\sum_{k\in\mathbb{Z}_{n}}f(1)
            \end{equation}
            But $f$ is injective, and hence $f(1)\ne{0}_{R}$. Thus, since
            $\ring[F]{F}$ is a field, for all $n\in\mathbb{Z}\setminus\{0\}$
            there is an $m\in\mathbb{Z}\setminus\{0\}$ such that
            $f(n)\cdot{f}(m)=f(1)$. Define $\tilde{f}$ as follows:
            \begin{equation}
                \tilde{f}\big((n,m)\big)=
                \begin{cases}
                    0,&n=0\\
                    f(n)\cdot_{F}f(m)^{\minus{1}},&\textrm{otherwise}
                \end{cases}
            \end{equation}
        \end{proof}
        \begin{definition}
            A field of characteristic zero is a field $\ring{F}$ such that there
            exists a field homomoprhism $f:\mathbb{Q}\rightarrow{F}$.
        \end{definition}
        Equivalently one could say that adding $1_{F}$ to itself $n$ times never
        results in zero.
        \begin{example}
            $\mathbb{Q}$, $\mathbb{R}$, and $\mathbb{C}$ are fields of
            characteristic zero.
        \end{example}
        \begin{definition}
            The characteristic of a field $\ring{F}$ is the smallest non-zero
            $n\in\mathbb{N}$ such that:
            \begin{equation}
                \sum_{k\in\mathbb{Z}_{n}}1_{F}=0_{F}
            \end{equation}
        \end{definition}
        \begin{theorem}
            If $n\in\mathbb{N}^{+}$ is a non-negetive integer, and if
            $\ring{F}$ is a field of characteristic $n$, then $n$ is a prime
            number.
        \end{theorem}
        \begin{proof}
            For suppose not. Then there are integers $p,q\in\mathbb{N}^{+}$ such
            that $p,q<n$ and $p\cdot{q}=n$. Let $f:\mathbb{Z}\rightarrow{F}$ be
            defined by:
            \begin{equation}
                f(n)=
                \begin{cases}
                    0_{F},&n=0\\
                    \sum_{k\in\mathbb{Z}_{n}}1_{F},&n>0\\
                    \minus\sum_{k\in\mathbb{Z}_{|n|}}1_{F},&n<0
                \end{cases}
            \end{equation}
            This is a ring homomoprhism, and hence
            $f(p\cdot{q})=f(p)\cdot{f}(q)$. But $f(n)=0$, and since all fields
            are integral domains, either $f(p)=0$ or $f(q)=0$. But then there
            exists a positive integer smaller than $n$ such that $f(p)=0$,
            a contradiction as $n$ is the characteristic of $F$. Hence, $n$ is
            a prime.
        \end{proof}
        \begin{ftheorem}{Binomial Theorem}{Binomial_Theorem}
            If $\ring{R}$ is a commutative ring, if $a,b\in{R}$, and if
            $n\in\mathbb{N}$, then:
            \begin{equation*}
                (a+b)^{n}=\sum_{k\in\mathbb{Z}_{n}}\binom{n}{k}a^{k}b^{n-k}
            \end{equation*}
        \end{ftheorem}
        \begin{theorem}
            If $p\in\mathbb{N}$ is a prime number, if $r\in\mathbb{N}$ is such
            that $1\leq{r}$ and $r\leq{p}^{n}-1$, then $p$ divides
            $\binom{p^{n}}{r}$.
        \end{theorem}
        \begin{theorem}
            If $\ring{F}$ is a field of characteristic $p\in\mathbb{N}^{+}$,
            if $n\in\mathbb{N}$, and if $a,b\in{F}$, then:
            \begin{equation}
                (a+b)^{p^{n}}=a^{p^{n}}+b^{p^{n}}
            \end{equation}
        \end{theorem}
        \begin{proof}
            Apply the binomial in combintation with the previous theorem.
        \end{proof}
        \begin{definition}
            The ring of polynomials over a field $\ring{F}$ is the set of all
            finitely supported sequences $a:\mathbb{N}\rightarrow{F}$ with the
            following addition and multiplication:
            \begin{align}
                (a+b)_{n}&=a_{n}+b_{n}\\
                (ab)_{n}&=\sum_{k=0}^{n}a_{k}b_{n-k}
            \end{align}
        \end{definition}
        This is very much mimicing polynomials. The sum rule says we simply add
        the coefficients of two polynomials, and the product is the Cauchy
        product of two polynomials. That is, we multiply
        $(a_{0}+a_{1}x+\dots+a_{n}x^{n})$ by $(b_{0}+b_{1}x+\dots+b_{n}x^{n})$
        and collect the coefficients of all terms with order $x^{k}$ and group
        them. The resulting coefficient is precisely this sum.
        \begin{theorem}
            If $\ring{F}$ is a field, then $\ring{F[x]}$ is a commutative ring.
        \end{theorem}
        \begin{theorem}
            If $\ring{F}$ is a field, then $\ring{F[x]}$ is a commutative
            algebra over $F$.
        \end{theorem}
        There is a natural embedding of $F$ into $F[x]$ by looking at the
        subspace of all sequence $a:\mathbb{N}\rightarrow{F}$ such that
        $a_{k}=0$ for all $k>0$. That is, the only possible non-zero term is
        $a_{0}$.
        \begin{theorem}
            If $\ring[F]{F}$ is a field, if $\ring[R]{R}$ is a ring, if
            $F\subseteq{R}$ is a subring, and if $r\in{R}$, then there is a
            unique homomoprhism $f:F[x]\rightarrow{R}$ such that for all
            $a\in{F}$, $f(a)=r$.
        \end{theorem}
        \begin{definition}
            The degree of a non-zero polynomial $a\in{F}[x]$ is the largest
            $n\in\mathbb{N}$ such that $a_{n}\ne{0}$. The degree of the zero
            polynomial is zero.
        \end{definition}
        Since $F[x]$ is the space of all finitely supported sequences, for any
        such $a\in{F}[x]$ there will eventually be an $N\in\mathbb{N}$ such that
        for all $n>N$ it is true that $a_{n}=0$. By the well-ordering all of the
        integers, there will be a least such element, and hence the above
        definition is well defined. There's a natural way of looking at
        $F[x]$ as a subset of $\funcspace[F]{F}$, the set of all functions from
        $F$ to itself. Given $a\in{F}[x]$ of degree $n\in\mathbb{N}$ we consider
        the function $f\in\funcspace[F]{F}$ defined by:
        \begin{equation}
            f(x)=\sum_{k\in\mathbb{Z}_{n+1}}a_{k}\cdot_{F}x^{k}
            =a_{0}+_{F}a_{1}\cdot_{k}x+_{F}a_{2}\cdot_{F}x^{2}+_{F}\dots
            +_{F}a_{n}\cdot_{F}x^{n}
        \end{equation}
        In more familiar language (dropping the subscripts and using
        concatenation to denote multiplication), we have:
        \begin{equation}
            f(x)=a_{0}+a_{1}x+\cdots+a_{n}x^{n}
        \end{equation}
        The sequence definition is good for rigor and solving theorems, since
        the Cauchy product allows one to easily manipulate expressions without
        worrying about the non-existent dummy variable $x$, whereas the function
        definition (as a subset of $\funcspace[F]{F}$) is good for intuition.
        \begin{definition}
            A monic polynomial of degree $n$ in a field $\ring{F}$ is a
            polynomial $a\in{F}[x]$ of degree $n\in\mathbb{N}$ such that
            $a_{n}=1$.
        \end{definition} 
        \begin{theorem}
            If $\ring{F}$ is a field, then $F[x]$ is a Euclidean domain.
        \end{theorem}
        \begin{theorem}
            If $\ring{F}$ is a field, if $\ring[I]{I}$ is an ideal in $F[x]$,
            and if $a\in{I}$ is of least degree, then $I=(a)$, where $(a)$ is
            the ideal generated by $a$.
        \end{theorem}
        \begin{proof}
            For given $b\in(a)$, there is an $r\in{F}[x]$ such that
            $b=r\cdot{a}$. But $I$ is an ideal, and $a\in{I}$, and hence
            $b\in{I}$. Thus, $(a)\subseteq{I}$. If $b\in{I}$, then by the
            division algorithm there are polynomials $p,q\in{F}[x]$ such that
            $b=aq+r$ where the degree of $r$ is strictly less than the degree of
            $a$. But then $r=b-aq$. And since $b\in{I}$ and $q\in{F}[x]$, it is
            true that $bq\in{I}$ since $I$ is an ideal. But if $a\in{I}$ and
            $bq\in{I}$, then $b-aq\in{I}$ since $I$ is an ideal. Thus,
            $r\in{I}$. But $r$ is a polynomial of degree strictly less than
            $a$, and $a$ is a non-zero polynomial of least degree in $I$.
            Therefore, $r=0$. But then $b=aq$, and hence $b\in(a)$. Thus,
            $I\subseteq(a)$. From the definition of equality, $I=(a)$.
        \end{proof}
        \begin{theorem}
            There exists a bijection between monic polynomials in $F[x]$ and
            the ideals of $F[x]$.
        \end{theorem}
        \begin{definition}
            A root of a polynomial $a\in{F}[x]$ of degree $n\in\mathbb{N}$ over
            a field $\ring{F}$ is an element $r\in{F}$ such that:
            \begin{equation}
                \sum_{k\in\mathbb{Z}_{n+1}}a_{k}\cdot_{F}x^{k}=0_{F}
            \end{equation}
        \end{definition}
        \begin{theorem}
            If $\ring{\mathbb{Q}}$ is the field of rational numbers, if
            $a\in\mathbb{Q}[x]$ is a polynomial of degree $n\in\mathbb{N}$,
            if $r\in\mathbb{Q}$ is a root of $a$, and if $p,q\in\mathbb{Q}$
            are such that $r=p/q$ and $\GCD(p,q)=1$, then $p$ divides $a_{0}$
            and $q$ divides $a_{n}$.
        \end{theorem}
        \begin{proof}
            For if $r$ is a root, then:
            \begin{equation}
                \sum_{k\in\mathbb{Z}_{n+1}}a_{k}\cdot{r}^{k}=0
            \end{equation}
            But $r=p/q$, and thus:
            \begin{equation}
                \sum_{k\in\mathbb{Z}_{n+1}}a_{k}\cdot\big(\frac{p}{q}\big)^{k}
                =0
            \end{equation}
            Simplifying, and multiplying both sides by $q^{k}$, we have:
            \begin{equation}
                \sum_{k\in\mathbb{Z}_{n+1}}a_{k}\cdot{p}^{k}q^{n-k}=0
            \end{equation}
            From this, $q$ divides $a_{n}p^{n}$. But $\GCD(p,q)=1$, and hence
            $q$ does not divide $p^{n}$. Thus, $q$ divides $a_{n}$. Similarly,
            $p$ divides $a_{0}$.
        \end{proof}
        \begin{example}
            Sticking with $\mathbb{Q}[x]$, the polynomial $f(x)=x^{3}-3x-1$ is
            irreducible over $\mathbb{Q}$. By the previous theorem, the only
            possible roots $p/q$ must be such that $p$ divides $\minus{1}$ and
            $q$ divides $1$. Hence, $p/q=\pm{1}$. But $f(1)=\minus{3}$ and
            $f(\minus{1})=1$, neither of which are zero. Hence, $f$ has no roots
            over $\mathbb{Q}$. Since it is a cubic, it must be irreducible.
        \end{example}
        \begin{ftheorem}{Gauss's Lemma for Polynomials}
                        {Gauss's Lemma for Polynomials}
            If $\ring{\mathbb{Z}}$ is the ring of integers, if
            $\ring{\mathbb{Q}}$ is the field of rational numbers, if
            $a\in{\mathbb{Z}}[x]$ is such that $a$ factors non-trivially in
            $\mathbb{Q}[x]$, then $a$ factors non-trivially in $\mathbb{Z}[x]$.
        \end{ftheorem}
        \begin{bproof}
            For if $b,c\in\mathbb{Q}[x]$ are such that $a=b\cdot{c}$, if
            $M,N\in\mathbb{N}$ are the products of the denominators of $b$
            and $c$, respectively, then $Ma,Nb\in\mathbb{Z}[x]$. But then
            $NMa\in\mathbb{Z}[x]$. By the fundamental theorem of arithmetic,
            there exists a prime $p\in\mathbb{N}$ such that $p$ divides $MN$.
            But then $Ma\cdot{N}b$ is the zero polynomial in $\mathbb{F}_{p}[x]$
            and since $\mathbb{F}_{p}$ is a field, this means that $p$ divides
            the coefficients of every element of either $Ma$ or $Nb$. Continuing
            we remove all of the prime factor of $MNa$ and obtain a
            factorization of $a$ in $\mathbb{Z}[x]$.
        \end{bproof}
        \begin{theorem}
            If $\ring{\mathbb{Z}}$ is the ring of integers, if
            $a\in\mathbb{Z}[x]$ is a monic polynomial of degree
            $n\in\mathbb{N}$, and if $b\in\mathbb{Q}[x]$ is a monic factor of
            $a$, then $b\in\mathbb{Z}[x]$.
        \end{theorem}
        \begin{proof}
            For if $b,c\in\mathbb{Q}[x]$ are such that $a=bc$, with $b$ a monic
            polynomial, then by the Cauchy product, since $a$ is monic, $c$
            must also be monic. Let $m$ and $n$ by the least integers such that
            $mb,nc\in\mathbb{Z}[x]$. If $p\in\mathbb{N}$ is a prime that divides
            $mn$, then it divides all of the coefficients of either $mb$ of $nc$
            and hence either $(m/p)b\in\mathbb{Z}[x]$ or
            $(n/p)c\in\mathbb{Z}[x]$, a contradiction since $m$ and $n$ are the
            least such integers with this property. Hence, $m=1$ and $n=1$.
        \end{proof}
        \begin{definition}
            An algebraic integer in $\mathbb{C}$ is a complex number
            $z\in\mathbb{C}$ such that $z$ is the root of a monic polynomial
            $a\in\mathbb{Z}[x]$.
        \end{definition}
        \begin{ftheorem}{Eisenstein's Criterion}{Eisenstein_Criterion}
            If $\ring{\mathbb{Z}}$ is the ring of integers, if
            $a\in\mathbb{Z}[x]$ is a polynomial in $\mathbb{Z}$ of degree
            $n\in\mathbb{N}$, if $p\in\mathbb{N}$ is a prime number such that
            $p$ does not divide $a_{n}$, $p^{2}$ does not divide $a_{0}$, and
            such that $p$ divides $a_{k}$ for all $k\in\mathbb{Z}_{n}$, then $a$
            is irreducible in $\mathbb{Q}[x]$.
        \end{ftheorem}
        \begin{bproof}
            For suppose $a$ factors in $\mathbb{Q}[x]$. But then it factors in
            $\mathbb{Z}[x]$. Suppose $b,c\in\mathbb{Z}[x]$ are non-trivial
            factors. By the Cauchy product, $a_{0}=b_{0}c_{0}$, and thus
            $p$ divides either $b_{0}$ or $c_{0}$. Suppose it divides $b_{0}$.
            But since $p^{2}$ does not divide $a_{0}$, $p$ does not divide
            $c_{0}$. But again by the Cauchy product,
            $a_{1}=b_{0}c_{1}+b_{1}c_{0}$. But $p$ divides $a_{1}$, and hence
            $p$ divides $b_{1}$. Continuing by induction on the Cauchy product,
            $p$ divides all of the $b_{k}$, contradicting that $p$ does not
            divide $a_{n}$.
        \end{bproof}
        It is important again to note that $\mathbb{Z}_{n}=\{0,1,\dots,n-1\}$,
        hence $\mathbb{Z}_{n}$ does not contain $n$. So we require $p$ to divide
        $a_{0},\dots,a_{n-1}$, but not $a_{n}$, and we require $p^{2}$ not to
        divide $a_{0}$. Eisenstein's criterion holds for any unique
        factorization domain.
\end{document}