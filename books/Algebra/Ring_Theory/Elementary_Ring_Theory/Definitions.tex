We now add more structure by considering a set with two operations.
Everything we've studied so far (semigroups, quasigroups, monoids,
groups) has had only one operation associated to it, but in the most
fundamental forms of arithmetic there are two. The only structure we've
encountered with two operations so far has been Boolean algebras
(see Book~\ref{book:Foundations}), but as we will see when we study
topology, there is essentialy only one type of Boolean algebra and thus
this study is, in a sense, complete. If we are going to axiomitize some
algebraic structure it is then wise to avoid recreating Boolean algebras
and so instead we try to model the arithmetic of the real numbers. The
most fundamental properties can be stated quite succintly:
$\monoid[][+]{\mathbb{R}}$ is an Abelian group and
$\monoid[][\cdot]{\mathbb{R}}$ is a monoid. We cannot just leave it
there, however, since we've no way of knowing how $+$ and $\cdot$ play
together. As presented, we have two potentially unrelated binary
operations and thus cannot procede any further. To complete our
structure, we add \glslink{distributive operation}{distributivity}.
\section{Definitions}
    The notion described in the preceding paragraph is that of a \gls{ring}. We
    can weaken this slightly and describe what is called a \textit{rng}.
    \subsection{Rngs}
        \begin{fdefinition}{Rng}{Rng}
            A \gls{rng} is an \gls{Abelian group} $(R,+)$ and a \gls{semigroup}
            $(R,\cdot\,)$ such that $\cdot$ is a \gls{distributive operation}
            over $+$. A rng is denoted $(R,+,\cdot\,)$.
        \end{fdefinition}
        The unital element of the Abelian group $(R,+)$ is often denoted 0 and
        is called the zero element. Zero has the property that multiplication by
        zero returns zero for any element.
        \begin{theorem}
            \label{thm:Mult_By_Zero_in_Rng}%
            If $(R,+,\cdot\,)$ is a rng, if 0 is the unital element of $(R,+)$,
            and if $r\in{R}$, then $r\cdot{0}=0$.
        \end{theorem}
        \begin{proof}
            For since $(R,+)$ is an Abelian group, if $r\cdot{0}\in{R}$ then
            there is an inverse element $\minus{r}\cdot{0}$
            (Def.~\ref{def:Group}). But then:
            \begin{align}
                0&=r\cdot{0}-r\cdot{0}
                \tag{Inverse Property of Groups}\\
                &=r\cdot(0-0)
                \tag{Distributive Property}\\
                &=r\cdot{0}
                \tag{Identity Property}
            \end{align}
            And therefore $r\cdot{0}=0$.
        \end{proof}
        The converse of this theorem fails in a general rng, but is true in a
        ring (see Thm.~\ref{thm:Mult_by_Zero_Always_Zero_Implies_Zero_in_Ring}).
        To see this we will show that we can construct a rng from any Abelian
        group in a trivial manner such that the converse fails.
        \begin{theorem}
            \label{thm:rng_induced_by_Abelian_group}%
            If $(R,+)$ is an Abelian group, then there is an associative
            binary operation $\cdot$ on $R$ such that $(R,+,\cdot\,)$ is a rng.
        \end{theorem}
        \begin{proof}
            For let $\cdot:R\times{R}\rightarrow{R}$ be defined by $r\cdot{s}=0$
            for all $r,s\in{R}$. Then $\cdot$ is a binary operation since for
            all $(r,s)\in{R}\times{R}$ there is a unique $x\in{R}$ such that
            $r\cdot{s}=x$, and thus $\cdot$ is a function
            (Def.~\ref{def:Function}). Moreover, it is associative. For if not
            then there are $r,s,t\in{R}$ such that
            $r\cdot(s\cdot{t})\ne(r\cdot{s})\cdot{t}$. But by definition
            $s\cdot{t}=0$, and thus $r\cdot{0}=0$. Similarly, $r\cdot{s}=0$
            and $0\cdot{t}=0$, a contradiction. Thus $\cdot$ is associative.
            Thus, $(R,\cdot)$ is a semigroup (Def.~\ref{def:Semigroup}).
            Finally, $\cdot$ is associative over $+$. For let $a,b,c\in{R}$.
            Then:
            \begin{equation}
                a\cdot(b+c)=0=0+0=(a\cdot{b})+(a\cdot{c})
            \end{equation}
            and thus by the transitivity of equality
            (Thm.~\ref{thm:Transitivity_of_Equality}),
            $a\cdot(b+c)=(a\cdot{b})+(a\cdot{c})$. Thus, $(R,+,\cdot\,)$ is a
            rng (Def.~\ref{def:Rng}).
        \end{proof}
        A rng with the property that $a\cdot{b}=0$ for all $a,b\in{R}$ is a
        occasionally called a rng of square zero. If we take an Abelian group
        with at least two elements, the rng of square zero generated from this
        will have the property that there exists a $b\in{R}$ such that $b\ne{0}$
        and for all $a\in{R}$, $a\cdot{b}=b\cdot{a}=0$.
    \subsection{Rings}
        \begin{fdefinition}{Ring}{Ring}
            A \gls{ring} is a \gls{rng} $(R,+,\cdot\,)$ such that $(R,\cdot\,)$
            is a \gls{monoid}. That is, an \gls{Abelian group} $(R,+)$ and a
            monoid $(R,\cdot\,)$ such that $\cdot$ is a
            \gls{distributive operation} over $+$.
        \end{fdefinition}
        Once $(R,\cdot)$ has a unital element we can prove that zero is the
        only element such that $0\cdot{r}=0$ for all $r\in{R}$. That is, the
        converse of Thm.~\ref{thm:Mult_By_Zero_in_Rng} is true in a ring.
        \begin{theorem}
            \label{thm:Mult_by_Zero_Always_Zero_Implies_Zero_in_Ring}%
            If $(R,+,\cdot\,)$ is a ring, if $0$ is the unital element of
            $(R,+)$, and if $r$ is such that for all $s\in{R}$ it is true that
            $r\cdot{s}=0$, then $r=0$.
        \end{theorem}
        \begin{proof}
            For if $(R,+,\cdot\,)$ is a ring, then $(R,\cdot\,)$ is a monoid
            (Def.~\ref{def:Ring}) and thus there is a unital element, 1, of
            $(R,\cdot\,)$ (Def.~\ref{def:Monoid}). But then for all $s\in{R}$
            we have:
            \begin{align}
                r+s&=(r+s)\cdot{1}
                \tag{Identity Property of 1}\\
                &=(r\cdot{1})+(s\cdot{1})
                \tag{Distributive Property}\\
                &=0+s\cdot{1}
                \tag{Hypothesis}\\
                &=s\cdot{1}
                \tag{Identity Property of 0}\\
                &=s
                \tag{Identity Property of 1}
            \end{align}
            And thus, for all $s\in{R}$, $r+s=s$. But $(R,+)$ is an Abelian
            group, and thus if $r+s=s$, then $s+r=s$
            (Def.~\ref{def:Abelian_Group}) and therefore $r$ is a unital element
            of $(R,+)$ (Def.~\ref{def:Unital_Element}). But the unital element
            of a group is unique, and therefore $r=0$.
        \end{proof}
        Given any Abelian group $(R,+)$, there is an associative binary
        operation $\cdot$ such that $(R,+,\cdot\,)$ is a rng
        (Thm.~\ref{thm:rng_induced_by_Abelian_group}). Rings do not have this
        property, as we will now demonstrate.
        \begin{theorem}
            \label{thm:Abelian_group_that_is_not_a_ring}%
            If $(R,+)$ is an Abelian group such that for all $a\in{R}$ there is
            an $n\in\mathbb{N}^{+}$ such that $na=0$, if $E$ is the set:
            \begin{equation}
                E=\{\,n\in\mathbb{N}^{+}\;|
                    \;\exists_{a\in{R}}:na=0\land
                    \forall_{k\in\mathbb{Z}_{n}\setminus\{0\}}(ka\ne{0})\,\}
            \end{equation}
            and if $E$ is infinite, then there is no binary operation $\cdot$ on
            $R$ such that $(R,+,\cdot\,)$ is a ring.
        \end{theorem}
        \begin{proof}
            For suppose not and let $\cdot$ be such a binary operation. But if
            $(R,+,\cdot\,)$ is a ring, then $(R,\cdot)$ is a monoid
            (Def.~\ref{def:Ring}) and thus there is a unital element $1$ of
            $(R,\cdot\,)$ (Def.~\ref{def:Monoid}). But by hypothesis there is an
            $n\in\mathbb{N}$ such that $n1=0$. But then for all $a\in{R}$, by
            associativity and Thm.~\ref{thm:Mult_By_Zero_in_Rng} we have:
            \begin{equation}
                ka=k(1\cdot{a})=(k1)\cdot{a}=0\cdot{a}=0
            \end{equation}
            Thus the set $E$ is bounded by $k$. But then
            $E\subseteq\mathbb{Z}_{k}$, and is therefore finite. A contradiction
            as $E$ is infinite. Thus, $(R,+,\cdot)$ is not a ring.
        \end{proof}
        \begin{example}
            Let $(\mathbb{Q},+)$ and $(\mathbb{Z},+)$ be the usual additive
            groups and consider the quotient group on $\mathbb{Q}/\mathbb{Z}$.
            This is Abelian since $\mathbb{Z}$ is a normal subgroup of
            $(\mathbb{Q},+)$, but for every element $a\in\mathbb{Q}/\mathbb{Z}$
            there is an $n\in\mathbb{N}^{+}$ such that $na=0$. That is, given an
            equivalence class $a\in\mathbb{Q}/\mathbb{Z}$, let $p/q\in[0,1)$ be
            a representative. Then $qa=[p]=[0]$, where $[p]$ is the equivalence
            class of $[p]$ in the quotient group. Thus every element is a
            torsion element. Morevoer, for all $n\in\mathbb{N}^{+}$ there is a
            $a\in\mathbb{Q}/\mathbb{Z}$ such $na=0$ and for all
            $k\in\mathbb{Z}_{n}^{+}$, $ka\ne{0}$. To see this, let
            $a=[1/n]$. So by Thm.~\ref{thm:Abelian_group_that_is_not_a_ring}
            there is no ring structure on $\mathbb{Q}/\mathbb{Z}$.
        \end{example}