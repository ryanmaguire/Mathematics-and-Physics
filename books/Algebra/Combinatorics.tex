\chapter{Combinatorics}
    \section{Introduction}
        We use the following notation:
        \begin{equation}
            [n]=\mathbb{Z}_{n}=\{1,2,\hdots,n\}
        \end{equation}
        In combinatorics we study functions of the form
        $f:[n]\rightarrow[m]$. These functions can be
        one-to-one, onto, or both. A permutation of the
        elements of $[n]$ is simply a bijection
        $f:[n]\rightarrow[n]$. A partial permutation is a
        permutation of length $k\leq{n}$ of the set
        $[n]$. That is, a permutation on some subset of
        $\mathbb{Z}_{n}$. We all study sets. In particular,
        the power set of $[n]$ and subsets from $k$ chosen
        elements of $[n]$. Another topic of study is that of
        lattice paths on $\mathbb{Z}\times\mathbb{Z}$. That
        it, paths from $(i,j)$ to $(n,m)$ using a prescribed
        set of rules. For example, how many ways can you get
        from $(0,0)$ to $(21,7)$ if you're only allowed to move
        North and East. Restricted paths impose more rules,
        for example the number of moves east must be greater
        than the number of moves north. There are also things
        called Catalan paths and Motzkin paths.
        \par\hfill\par
        Another topic in combinatorics is that of words. An
        alphabet is a set $[n]$, and we wish to study the number
        of words of length $k$ in $[n]$. Binary words are words
        when the alphabet is $\{0,1\}$. There are also words
        with a prescribed number for each letter.
        There are also circular arrangements of the elemnts
        in $[n]$, and the idea of multi-sets. Multi-sets
        are sets that allow for repetition. From elementary
        set theory, we have:
        \begin{equation}
            \{a,b,c\}=\{a,a,b,c\}
        \end{equation}
        Sets are uniquely defined by the elements they contain.
        Multi-sets allow for repetition, and this distringuishes
        two different sets. We write:
        \begin{equation}
            A=\{\{1,1,1,2,3,3\}\}
        \end{equation}
        Note that $A\ne\{\{1,2,3\}\}$. The multiplicty of
        an elements $a\in{A}$ is the number of times the
        element $a$ occurs in the multi-set $A$. To know how
        many multi-sets chosen from $[n]$ with $k$ elements,
        we wish to study the following equation:
        \begin{equation}
            \sum_{i=1}^{n}m_{i}=k
        \end{equation}
        Where $m_{i}$ is the multiplicity of the $i^{th}$
        element of $[n]$. Where wish to find integer solutions
        to this equation, in particular solutions with
        non-negative integers. We can also place restrictions
        on the multi-sets, for example by requiring that
        each element occurs at least once. Thus we'd have:
        \begin{equation}
            \sum_{i=1}^{n}m_{i}=k
            \quad\quad
            m_{i}\geq{1}
        \end{equation}
        A partition of $[n]$ is a collection of sets
        $B_{1},b_{2},\dots,B_{k}$ such that:
        \begin{equation}
            \cup_{i=1}^{k}B_{i}=[n]
            \quad\quad
            B_{i}\cap{B}_{j}=\emptyset
            \quad{i}\ne{j}
        \end{equation}
        The $B_{i}$ are called \textrm{blocks}. We also
        study partitions of numbers. Given $n\geq{0}$, we
        want $\lambda=(\lambda_{1},\dots,\lambda_{\ell})$ such
        that:
        \begin{equation}
            \lambda_{i+1}\leq\lambda_{i},
            \quad\quad
            i=1,2,\dots,\ell-1
        \end{equation}
        And such that:
        \begin{equation}
            \sum_{i=1}{\ell}\lambda_{i}=n
        \end{equation}
        Another commonly studied object is a graph. Labelled
        tress, colorings of graphs, and spanning trees. Finally
        we study tableaux's. These are fillings of arrays
        of boxes with objects, such as numbers, sets, and
        multi-sets.
    \section{Counting Techniques}
        \subsection{Basic Numbers}
            \begin{equation}
                n^{k}=\big|\{f:[k]\rightarrow[n]\}\big|=
                \textrm{The number of words over the alphabet }
                [n]
            \end{equation}
            \begin{equation}
                \binom{n}{k}=
                \frac{n!}{k!(n-k)!}=
                \textrm{The number of ways to choose }k
                \textrm{ objects from }[n]
            \end{equation}
            Stirling's number of the second kind, denoted
            $S(k,n)$, is the number of ways to partition
            $[n]$ into $n$ blocks. $P(n)$ is the number of
            partitions of the integer $n$. And lastly,
            $n!$ is the number of bijections from $[n]$ into
            $[n]$. That is, $n!$ is the number of permutations
            on $[n]$.
        \subsection{Basic Counting Principles}
            Sum Rule (Divide and Conquer): If you cannot
            count the set, divide it into pieces and count
            the pieces.
            \begin{align}
                S&=\bigcup_{i=1}^{k}S_{i}
                \quad\quad
                S_{i}\cap{S}_{j}=\emptyset,
                \quad{i}\ne{j}\\
                |S|&=\sum_{i=1}^{k}|S_{i}|
            \end{align}
            Application: Classify or partition the elements
            in $S$ according to a set of mutually disjoint
            properties $p_{1},\dots,p_{k}$ and let:
            \begin{equation}
                S_{k}=\{x\subseteq{S}:p_{k}(x)\}
            \end{equation}
            We often use such a scheme to prove recurrences.
            For example Pascal's triangle:
            \begin{equation}
                \binom{n}{k}=
                \binom{n-1}{k-1}+\binom{n-1}{k}
            \end{equation}
            We can prove this by plugging in the formula and
            simplifying, but we wish to give a combinatorial
            proof. Let $S\subseteq[n]$, where
            $\Card(S)=k\leq{n}$. Define:
            \begin{align}
                S_{1}&=\{A\subseteq{S}:n\in{A}\}\\
                S_{2}=\{A\subseteq{S}:n\notin{A}\}
            \end{align}
            Then $S_{1}$ and $S_{2}$ partition $S$, and thus
            $\Card(S)=\Card(S_{1})+\Card(S_{2})$. But:
            \begin{align}
                \Card(S_{1})&=\binom{n-1}{k-1}\\
                \Card(S_{2})&=\binom{n-1}{k}
            \end{align}
            This completes the proof. Next is the difference
            rule (Count the opposite):
            \begin{align}
                S&\subseteq\mathcal{U}\\
                \overline{S}&=\mathcal{U}\setminus{S}\\
                \Card(S)&=\Card(\mathcal{U})
                    -\Card(\overline{S})
            \end{align}
            For example, how many permutations on $[n]$ are
            there so that 1 and 2 are not next to each other?
            We count 1 and 2 being together as 12 or 21.
            It's thus easier to think of this as one element.
            So we're counting the number of permutation on
            $[n-1]$ by consider $12$ as one element, and
            again by consider $21$ as one element. This
            gives $2(n-1)!$. By taking the difference, we
            have:
            \begin{equation}
                \Card(S)=n!-2(n-1)!=(n-1)!(n-2)
            \end{equation}
    \section{Posets}
        Let $n\in\mathbb{N}$, $X=2^{[n]}=\mathcal{P}(\mathbb{Z}_{n})$, and
        let consider $(X,\subseteq)$, where $\subseteq$ is the partial ordering
        of inclusion. If $\Card(A)=\Card(B)$, then either $A=B$, or
        $A$ and $B$ are not comparable, and hence $\subseteq$ is a partial
        order.
        \begin{theorem}
            For any $n\geq{1}$, $2^{[n]}$ has a SCD.
        \end{theorem}
        \begin{proof}
            Let $x\in{2}^{[n]}$. Then $x$ has a binary representation. We
            want to create 1-0 pairs as if they were parentheses. For example,
            suppse $x=\{1,5,7\}$. Then $x=0110001$. We match this up to
            $)(()))(\leftrightarrow0(())01$. To get the chain containing
            $x$, we need to describe how to go up and how to go down
            in the chain. To go up, take the right-most unpaired zero
            and change it to a one. To go down, take the left-most one
            and change it to a zero. For the example of $x=\{1,5,6\}$, we have:
            \begin{equation*}
                0110000\rightarrow0110001\rightarrow0110011\rightarrow1110011
            \end{equation*}
            The size of the smallest set in the chain the the number of
            parenthesizations. If this number is $i$, then the largest set
            has size $(n-2i)+i=n-i$. By construction, the chain is saturated.
            At every step we only add one element. Thus, the constructions
            produce symmetric chains. Notice that we never produce new
            1-0 pairings in the algorithm. Thus, all of the sets in the chain
            have the same pairings. So two sets $X$ and $Y$ produce either
            the same chain or disjoint chains. For example, consider
            $2^{[4]}$. The chain is:
            \begin{align*}
                0000\rightarrow0001\rightarrow
                0011\rightarrow0111\rightarrow1111\\
                0010\rightarrow0110\rightarrow1110\\
                0100\rightarrow0101\rightarrow1101\\
                1000\rightarrow1001\rightarrow1011
            \end{align*}
        \end{proof}
        \begin{ltheorem}{Sperner's Theorem}
            Any anti-chain of $2^{[n]}$ elements has at most
            $\binom{n}{\floor{n/2}}$ subsets.
        \end{ltheorem}
        \begin{proof}
            Notice that in the diagram of $2^{[n]}$ each level contains
            $\binom{n}{k}$ elements. An SCD partitions $2^{[n]}$. Let
            $G$ be an $SCD$. with $m$ chains. Then the maximum number of
            incomparable elements is $m$. therwise, two imcomparable elements
            are in the same chain. Thus, $m\geq\binom{n}{\floor{n/2}}$. Also,
            each chain will intersect a level in the graph of the poset at most
            once. Therefore, $m\leq\binom{n}{\floor{n/2}}$. Thus, etc.
        \end{proof}
        The existence of a symmetric chain decomposition gives an elegant
        combinatorial proof that the sequence
        $\binom{n}{k}$, $k=0,1,\dots,n$, is unimodal. To prove unimodality, it
        would suffice to show that, for $k\leq{n/2}$, there exists an
        injection from $k$ subsets to $k+1$ subsets. If we have a SCD, map
        a $k$ subset to it's successor in the chain. This gives the injection.
    \section{Binomial Coefficients and Multi-Sets}
        Recall that a multi-set is similar to a set, except that repetitions
        are allowed. For example, if we consider $[3]$, then a multi-set of
        of this could be:
        \begin{equation}
            \{\{1,1,2,2,3\}\}
        \end{equation}
        This has 5 elements, and is a mutli-set of size 5.
        \begin{theorem}
            The number of $k$ multisets of an $n$ element set is:
            \begin{equation}
                \frac{n^{\overline{k}}}{k!}
                =\frac{n(n+1)\cdots(n+k-1)}{k!}
                =\binom{n+k-1}{k}
            \end{equation}
        \end{theorem}
        \begin{proof}
            For let $S$ be the set of multi-sets of size $k$ of elements
            of an $n$ element set, and let $T$ be subsets of size $k$ in
            $[n+k-1]$. We need to produce a map $f:S\rightarrow[T]$. Let:
            \begin{equation}
                M=\{\{1\leq{a}_{1}\leq{a}_{2}\leq\cdots\leq{a}_{k}\leq{n}\}\}
            \end{equation}
            This maps to the set:
            \begin{equation}
                A=\{1\leq{a}_{1}<a_{2}+1<a_{3}+2<\cdots<a_{k}+k-1\leq{n}+k-1\}
            \end{equation}
            This mapping is reversible. Therefore, etc.
        \end{proof}
        \begin{example}
            Let $n=3$, and $k=5$. Also, define:
            \begin{equation}
                M=\{\{1,1,2,2,3\}\}
            \end{equation}
            Then:
            \begin{equation}
                A=\{1,2,4,5,7\}\subseteq[7]
            \end{equation}
        \end{example}
        Multi-sets can be seen as binary sequences (Stars and bars). For
        example, let $M=\{3,3,4,7,7\}$. We can write this as
        $||**|*|||**$. This helps count out the repetitions of various elements
        in the multi-set. $\binom{n}{k}$ can be seen as the number of
        functions that map $k$ elements to $0$ and $n-k$ elements to $1$.
        We can generalize to functions $[n]\rightarrow[m]$. Let
        $k_{1},k_{2},\dots,x_{m}$ be such that:
        \begin{equation}
            \sum_{i=1}^{m}k_{i}=n
        \end{equation}
        And such that, for all $i$, $k_{i}\geq{0}$. Then
        $\binom{n}{k_{1},\dots,k_{m}}$ is the number of ways to map
        $[n]\rightarrow[m]$ such that $k_{i}$ elements map to $i$, where:
        \begin{align}
            \binom{n}{k_{1},k_{2},\dots,k_{m}}&=
            \binom{n}{k_{1}}\binom{n-k_{1}}{k_{2}}
            \binom{n-k_{1}-k_{2}}{k_{3}}\cdots\binom{k_{m}}{k_{m}}\\
            &=\frac{n!}{k_{1}!k_{2}!\cdots{k}_{m}!}
        \end{align}
        \begin{ltheorem}{Multinomial Theorem}{Multinomial_Theorem}
            \begin{equation}
                (x_{1}+x_{2}+\cdots+x_{m})^{n}=
                \sum_{k_{1}+\cdots+k_{m}=n}
                \binom{n}{k_{1},\dots,k_{m}}x_{1}^{k_{1}}\cdots{x}_{m}^{k_{m}}
            \end{equation}
        \end{ltheorem}
        \subsection{Lattice Paths}
            Let $\mathbb{Z}^{d}$ be an integer lattice of dimension $d$, where
            $d\in\mathbb{N}$ and $d\geq{1}$.
            \begin{ldefinition}{Lattice Path}
                A lattice path in $\mathbb{Z}^{d}$ with $k$ steps in
                $S\subseteq\mathbb{Z}^{d}$ is a subset
                $L\subseteq\mathbb{Z}^{d}$ such that $L=\{v_{1},\dots,v_{k}\}$
                such that, for all $i=1,2,\dots,k-1$, $v_{i+1}-v_{i}\in{S}$.
            \end{ldefinition}
            \begin{example}
                If $d=2$, $S=\{(0,1),(1,0)\}$, then there are 6 paths
                from $(0,0)$ to $(2,2)$.
            \end{example}
            \begin{theorem}
                if $v=(a_{1},\cdots,a_{d})\in\mathbb{Z}^{d}$ and if $e_{i}$ is
                the $i^{th}$ unit vector in $\mathbb{Z}^{d}$, then the number
                of lattice paths in $\mathbb{Z}^{d}$ from the origin to
                $v$ with steps in $\{e_{i}:i\in\mathbb{d}\}$ is given by
                the multinomial coefficient
                $\binom{\norm{v}_{1}}{a_{1},\dots,a_{d}}$.
            \end{theorem}
            \begin{proof}
                For let $v_{0},\cdots,v_{k}$ be a path. Then
                $v_{1}-v_{0},v_{2}-v_{1},\dots,v_{k}-v_{k-1}$ consist of
                the $e_{i}$. Thus there are $a_{1}$ $e_{1}$'s,
                $a_{2}$ $e_{2}$'s, and so on. The total number is thus the
                multinomial coefficient.
            \end{proof}
            \begin{theorem}
                The number of lattice paths from $(0,0)$ to $(n,m)$ with
                steps in $\{(0,1),(1,0)\}$ is $\binom{n+m}{n}$.
            \end{theorem}
        \subsection{The Involution Principle}
            \begin{theorem}
                The number of lattice paths from $(i,j)$ to $(m,n)$ using
                steps $(1,0)$ and $(0,1)$ is $\binom{m-i+n-j}{m-i}$.
            \end{theorem}
            Given a set $S$ and a partition
            $S=S^{+}\cup{S}^{-}$ into a negative part $S^{-}$ and a
            positive part $S^{+}$, then $S$ is called a signed set. We
            are interested in computing $\Card(S^{+})-\Card(S^{-})$.
            \begin{ldefinition}{Sign Reversing Involution}
                A sign reversing involution is an involution
                $\psi:S\rightarrow{S}$ such that for all $x\in{S}$
                such that $\psi(x)\ne{x}$, then
                $\psi(x)\in{S}^{+}$ for all $x\in{S}^{-}$ and
                $\psi(x)\in{S}^{-}$ for all $x\in{S}^{+}$.
            \end{ldefinition}
            \begin{theorem}
                If $\psi$ is a sign reversing involution, if
                $F^{+}$ are the fixed points of $\psi$ in $S^{+}$,
                and $F^{-}$ are the fixed points are $\psi$ in
                $S^{-}$, then:
                \begin{align}
                    \Card(S^{+}\setminus{F}^{+})&=
                    \Card(S^{-}\setminus{F}^{-})\\
                    \Card(S^{+})-\Card(S^{-})&=
                    \Card(F^{+})-\Card(F^{-})
                \end{align}
            \end{theorem}
            Suppose we are given a set $X$ and we want to compute
            $\Card(X)$. Embed $X$ into a signed set $S=S^{+}\cup{S}^{-}$
            such that for all $x\in{X}$ there is a corresponding
            $s\in{S^{+}}$.
            \begin{ldefinition}{Catalan Path}
                A Catalan path is a lattice path from $(0,0)$ to
                $(n,n)$ using steps $(0,1)$ and $(1,0)$ such that the
                path never crosses the line $x=y$.
            \end{ldefinition}
            We are interested in counting the number of Catalan paths
            for a given $n\in\mathbb{N}$. The first few numbers are
            $1,2,5,14,42,\dots$ and occur frequently in mathematics.
            Let $S^{+}$ be the set of paths from $(1,0)$ to $(n+1,n)$
            and $S^{-}$ be the set of paths from $(0,1)$ to $(n+1,n)$.
            Using the previous theorem:
            \begin{align}
                \Card(S^{+})&=\binom{2n}{n}\\
                \Card(S^{-})&=\binom{2n}{n-1}
            \end{align}
            Now we need to embed the Catalan paths into $S^{+}$.
            The embedding comes from shifting the graphs 1 unit to the
            right. Note that the image never touches the line $x=y$.
            Define a sign reversing involution $\psi:S\rightarrow{S}$
            by letting $P$ be any path in $S$ that does not touch
            $x=y$, and defining $\psi(P)=P$. If $P$ touches or crosses
            $x=y$, let $p_{0}$ be the first such crossing. Let
            $\psi(P)$ be the path from $(1,0)$
            (Respectively, from $(0,1)$), such that the points from
            $(0,0)$ to $p_{0}$ are reflected, and the points from
            $p$ to $(n+1,n)$ stay the same. Now $F^{-}$ is empty, since
            given any path from $(1,0)$ to $(n+1,n)$, it must cross
            the line $y=x$. Thus there are no fixed points in $S^{-}$.
            But then:
            \begin{equation}
                \Card(F^{+})=\Card(S^{+})-\Card(S^{-})
            \end{equation}
            But the Catalan number $C_{n}$ is equal to the size of
            $F^{+}$, and thus we have:
            \begin{equation}
                C_{n}=\binom{2n}{n}-\binom{2n}{n-1}
                =\frac{1}{n+1}\binom{2n}{n}
            \end{equation}
        \subsection{Diagonal Lattice Paths}
            \begin{ldefinition}{Diagonal Lattice Paths}
                A diagonal lattice path is a lattice path with steps
                $(1,1)$ and $(1,-1)$.
            \end{ldefinition}
            \begin{lexample}
                Consider all diagonal paths from $(0,0)$ to $(4,0)$.
                Since any step increasing the $x$ coordinate by 1,
                there must be 4 steps in the lattice path. But since
                the path must end at 0, there must be an equal number
                of steps that go up as there are steps that go down.
                So, we must have two up steps and two down steps.
                The total number of diagonal lattice paths is thus
                $\binom{4}{2}=6$. In general, the total number of
                lattice paths from $(0,0)$ to $(2n,0)$ is
                $\binom{2n}{n}$.
            \end{lexample}
            These diagonal lattice paths can be seen as binary words with
            $d=(1,-1)$ and $(u=1,1)$ such that the number of occurences
            of $d$ is equal to the number of occurences of $u$. We can
            establish a correspondence between diagonal lattice paths and
            Catalan paths by considering as the bijection a reflection
            about the $x=y$ axis, and then a rotation by $45^{\circ}$.
            \begin{ldefinition}{Dyck Paths}
                A Dyck is a diagonal lattice path that never goes below
                it's starting point.
            \end{ldefinition}
    \section{q-Analogues}
        In combinatorics, a $q$ analogue of a counting function, such
        as $n!$, is typically a polynomial in $q$ which evaluates to
        the function if we set $q=1$, and if not a polynomial we take
        the limit as $q\rightarrow{1}$. We want the q-analogue to
        preserve the same reccurence properties that the counting
        function has.
        \begin{lexample}
            A q-Analogue of a real number $x\in\mathbb{R}$ could be:
            \begin{equation}
                [x]_{q}=\frac{1-q^{x}}{1-q}
            \end{equation}
            Taking the limit as $q\rightarrow{1}$, we see that this
            expression evaluates to $x$ by using L'H\^{o}pital's Rule.
            If $x=n\in\mathbb{N}$, then:
            \begin{equation}
                \frac{1-q^{n}}{1-q}=1+q+\cdots+q^{n-1}
            \end{equation}
            This allows us to construct a q-Analogue of $n!$:
            \begin{equation}
                [n]_{q}!=[1]_{q}[2]_{q}\cdots[n]_{q}
            \end{equation}
            This can be used to put statistics on sets.
        \end{lexample}
        \begin{ldefinition}{Statistic on a Finite Set}
            A statistic on a finite set $S$ is a function
            $f:S\rightarrow\mathbb{N}_{0}$
        \end{ldefinition}
        Let $S_{n}$ denote the symmetric group, which is the set of
        permutations of $1,2,\dots,n$ under the operation of composition.
        Then:
        \begin{equation}
            \Card(S_{n})=n!
        \end{equation}
        \begin{ldefinition}{Inversion of a Word}
            An inversion of a word $\sigma$ is a pair $(i,j)$, where
            $1\leq{i}<j\leq{n}$, where $\sigma_{i}>\sigma_{j}$.
        \end{ldefinition}
        \begin{lexample}
            Let $\sigma=(132)(45)(6)(7)$. Then $(1,3)$ is an inversion,
            since $\sigma_{1}=3>\sigma_{3}=2$. 
        \end{lexample}
        \begin{ldefinition}{Inversion Statistic}
            The inversion statistic on $S_{n}$ is the number of
            inversions of $\sigma\in{S}_{n}$.
        \end{ldefinition}
        \begin{theorem}
            If $S_{n}$ is the permutation group, then:
            \begin{equation}
                \sum_{\sigma\in{S}_{n}}q^{\textrm{inv}(\sigma)}
                =[n]_{q}!
            \end{equation}
        \end{theorem}
    \section{Lecture 6}
        Last week we introduced q-Analogs, and proved the following
        identities:
        \begin{equation}
            \sum_{\sigma\in{S}_{n}}q^{inv(\sigma)}=
            \sum_{\sigma\in{S}_{n}}q^{maj(\sigma)}
            =[n]_{q}!]
        \end{equation}
        Where $inv(\sigma)$ is the number of inversions, and
        $maj(\sigma)$ is the number of descents. We now want a
        q-Analog of $\binom{n}{k}$. Recall the inversion tables:
        \begin{equation}
            \mathcal{I}_{n}=
                \{(a_{1},\dots,a_{n}):0\leq{a}_{i}\leq{i}\}
        \end{equation}
        We can write this as:
        \begin{equation}
            \mathcal{I}_{n}=
                \{0\}\times\{0,1\}\times\{0,1,2\}\times\cdots
                \times\{0,1,2,\dots,n-1\}
        \end{equation}
        From this we obtain:
        \begin{equation}
            \Card(\mathcal{I}_{n})=n!
        \end{equation}
        We define the function
        $\Psi_{1}:\mathcal{I}_{n}\rightarrow{S}_{n}$ by mapping:
        \begin{equation}
            \Psi_{1}(a_{1},\dots,a_{n})=\sigma
        \end{equation}
        Where $\sigma$ is the permutation with inversions
        $a_{1},\dots,a_{n}$, and $a_{i}$ is the number inversions
        created by $i$. We also define
        $\Psi_{2}:\mathcal{I}_{n}\rightarrow{S}_{n}$ and
        re-interpreted $a_{i}$ to be the contribution of $i$ to the
        major index. Then $\Psi=\Psi_{2}\circ\Psi_{1}^{\minus{1}}$
        is a bijection from $S_{n}$ to itself. We now want to find
        a good q-Analog for $\binom{n}{k}$ that would satisfy
        similar properties as the binomial coefficient. One nice
        property is Pascal's Identity:
        \begin{equation}
            \binom{n}{k}=\binom{n-1}{k-1}+\binom{n-1}{k}
        \end{equation}
        Perhaps the obvious choice is to choose:
        \begin{equation}
            \binom{n}{k}_{q}=
            \frac{[n]_{q}!}{[k]_{q}![n-k]_{q}!}
        \end{equation}
        These are called the Gaussian polynomials, and it seems
        surprising that these are polynomials in the first place,
        since it appears to be a rational function. However, we
        can see just by plugging in that:
        \begin{equation}
            \binom{n+1}{k}_{q}\ne
            \binom{n}{k}_{q}+\binom{n}{k-1}_{q}
        \end{equation}
        And thus this is not a good q-Analog for the binomial
        coefficients. Let:
        \begin{equation}
            \mathcal{R}(1^{k}0^{n-k})=\{
            \textrm{Set of binary words of length $n$ with $k$ 1's}
            \}
        \end{equation}
        Then:
        \begin{equation}
            \Card\Big(\mathcal{R}(1^{k}0^{n-k})\Big)=
            \binom{n}{k}
        \end{equation}
        This is equivalent to saying:
        \begin{equation}
            k!(n-k)!\Card\Big(\mathcal{R}(1^{k}0^{n-k})\Big)
            =n!
        \end{equation}
        But the left-hand side of this equation s the cardinality
        of $S_{k}\times{S}_{n-k}\times\mathcal{R}(1^{k}0^{n-k})$,
        and the right-hand side is the cardinality of
        $S_{n}$. We need to define a function:
        \begin{equation}
            f:S_{k}\times{S}_{n-k}\times\mathcal{R}(1^{k}0^{n-k})
        \end{equation}
        Add $n-k$ to the numbers in the permutation $S_{k}$.
        For example consider:
        \begin{equation}
            (132,14523,10011000)\mapsto
            (687,14523,10011000)
        \end{equation}
        Send the left-most number in order from left to right
        to the 1's in the binary word (The third entry). Send the
        second entry to the 0's in the binary word.
        So, finally we have:
        \begin{equation}
            (132,14523,10011000)\mapsto
            (61487523)
        \end{equation}
        We now want to show that:
        \begin{equation}
            \sum_{r\in\mathcal{R}(1^{k}0^{n-k})}q^{inv(r)}
            =\binom{n}{k}_{q}
            =\frac{[n]_{q}!}{[k]_{q}![n-k]_{q}!}
        \end{equation}
        We can do this in a similar manner as before. We need a
        function $f$ from
        $S_{n-k}\times{S}_{k}\times\mathcal{R}(1^{k}0^{n-k})$ that
        is bijective. Let's use the one defined previously. We
        now need to show that $f$ preserves inversions.
        \begin{theorem}
            \begin{equation}
                \binom{n+1}{k}_{q}=
                q^{k}\binom{n}{k}_{q}+
                \binom{n}{k-1}_{q}
            \end{equation}
        \end{theorem}
        \begin{ltheorem}{Foata's Theorem}{Foatas_Theorem}
            \begin{equation}
                \sum_{r\in\mathcal{R}(1^{k}0^{n-k})}q^{maj(r)}
                =\binom{n}{k}_{q}
            \end{equation}
        \end{ltheorem}
        \subsection{Lattice Paths and Gaussian Polynomials}
            \begin{ldefinition}{Partitions of Integers}
                A partition of $\mathbb{Z}_{n}$, $n\in\mathbb{N}$,
                is a weakly decreasing sequence
                $\lambda=(\lambda_{1},\dots,\lambda_{\ell})$,
                such that:
                \begin{equation}
                    |\lambda|=\sum_{k=1}^{\ell}\lambda_{k}=n
                \end{equation}
                $|\lambda|$ is called the weight of $\lambda$ and
                $\lambda_{i}$ are called the parts of $\lambda$.
                The length of $\lambda$ is the number of non-zero
                parts.
            \end{ldefinition}
            A young diagram is a graphical representation of a
            partition $\lambda=(\lambda_{1},\dots,\lambda_{\ell})$.
            The conjugate of a partition $\lambda$ is obtained by
            transposing the Young diagram of $\lambda$. For example:
            \begin{equation}
                (3,3,1)\mapsto(3,2,2)
            \end{equation}
            We denote the conjugate by $\lambda'$. Recall that
            $\binom{n+m}{n}$ is the number of lattice paths from
            $(0,0)$ to $(n,m)$ using steps $(1,0)$ and $(0,1)$.
            \begin{theorem}
                There exists a bijection between the set of lattice
                paths from $(0,0)$ to $(m,n)$ and the set of
                partitions of such that $\lambda_{1}\leq{m}$ and
                $\ell(\lambda)\leq{n}$.
            \end{theorem}
            If $p(m,n)$ is the number of partitions that fit in
            the $m\times{n}$ rectangle, that is
            $\ell(\lambda)\leq{N}$ and $\lambda_{1}\leq{m}$, then
            $p(m,n)=\binom{n+m}{n}$. A statistic on partitions is
            given by the weight of $\lambda$, $|\lambda|$.
            \begin{theorem}
                For $m,n\in\mathbb{N}$:
                \begin{equation}
                    \binom{n}{m}_{q}=
                    \sum_{\lambda\subseteq[m^{n}]}q^{|\lambda|}
                \end{equation}
            \end{theorem}
            \begin{proof}
                We can show this by proving that the sum satisfies
                the same recurrence relation and initial conditions
                as the q binomial.
            \end{proof}
    \section{Lecture 7 (I Think)}
        We're currently discussing q-analogues. We want to extend
        the q-analogue defined for the factorial function to the
        binomial coefficient. The Gaussian polynomials are one
        such attempt at this:
        \begin{equation}
            \binom{n}{k}_{q}=
            \frac{[n]_{q}!}{[k]_{q}![n-k]_{q}!}
        \end{equation}
        Another such attempt was to sum over all binary words of
        length $n$ with $k$ one's, and obtain:
        \begin{equation}
            \binom{n}{k}_{q}=
            \sum_{r\in\mathcal{R}(1^{k},0^{n-k})}
                q^{inv(r)}
        \end{equation}
        Using this definition, we obtained the following equation:
        \begin{equation}
            \binom{n+1}{k}_{q}=
            q^{k}\binom{n}{k}_{q}+
            \binom{n}{k-1}_{q}
        \end{equation}
        Then we discussed lattice paths $L(m,n)$, which are paths
        from $(0,0)$ to $(m,n)$ using steps in $(1,0)$ and $(0,1)$.
        Next we discuessed partitions of numbers.
        \begin{table}[H]
            \centering
            \captionsetup{type=table}
            \begin{tabular}{|c|c|}
                \hline
                $n$&Partitions\\
                \hline
                0&$\emptyset$\\
                \hline
                1&$(1)$\\
                \hline
                2&$(2),(1,1)$\\
                \hline
                3&$(3),(2,1),)1,1,1)$\\
                \hline
            \end{tabular}
            \caption{Caption}
            \label{tab:my_label}
        \end{table}
        Given such a partition, we assign the weight to be:
        \begin{equation}
            |\lambda|=\sum_{k=1}^{\ell}\lambda_{k}
        \end{equation}
        Then we define:
        \begin{equation}
            P(m,n)=
            \{(\lambda_{1},\dots,\lambda_{\ell}:
                \lambda_{1}\leq{m},\ell(\lambda)\leq{n}\}
        \end{equation}
        We showed that there is a bijection between
        $L(m,n)$ and $P(m,n)$. Thus, we have:
        \begin{equation}
            \Card\Big(P(n,m)\Big)=\binom{m+n}{m}
        \end{equation}
        \begin{theorem}
            If $m,n\in\mathbb{N}$, then:
            \begin{equation}
                \binom{m+n}{m}_{q}=
                \sum_{\lambda\in{P}(m,n)}q^{|\lambda|}
            \end{equation}
        \end{theorem}
        \begin{proof}
            The strategy of the proof is to show that this sum
            satisfies the same initial conditions and the same
            recurrence as the original definition. We have that
            $p(m,0)=1=q^{0}$ since there is only the empty
            partition in the rectangle $m\times{0}$, and similarly
            $p(0,n)=1=q^{0}$ since there is only the empty
            partition in the rectangle $0\times{n}$. Moreover,
            $p(m,m)=1=q^{0}$ since:
            \begin{equation}
                \binom{m}{m}_{q}=
                \frac{[m]_{q}}{[0]_{q}[m]_{q}}=
                \frac{[m]_{q}}{[m]_{q}}=1
            \end{equation}
            We now must show that the recurrence relation is
            satisfied. We want:
            \begin{equation}
                p(m,n)=q^{m}p(n-1,m)+p(n,m-1)
            \end{equation}
            We have that:
            \begin{align}
                p(m,n)=&
                \sum_{\lambda\in{P}(m,n)}q^{|\lambda|}\\
                &=\sum_{\lambda_{1}=m}q^{|\lambda|}+
                \sum_{\lambda_{1}<m}q^{|\lambda|}\\
                &=q^{m}\sum_{\lambda\in{P}(m,n-1)}q^{|\lambda|}
                +\sum_{\lambda\in{P}(m-1,n)}q^{|\lambda|}
            \end{align}
            This completes the proof.
        \end{proof}
        \begin{ldefinition}{$x$ Factorization}
            Let $w\in{X}^{*}$ be a word in the alphabet $X$.
            Let $x\in{X}$ and suppose $w=vy$, where $v$ is a word
            and $y$ is a letter in $X$. That is, $y$ is the last
            letter of $w$. Then the factorization is:
            \begin{equation}
                w=v_{1}y_{1}\dots{v}_{k}y_{k}
            \end{equation}
            Where, if $y>x$ ($X$ is totally ordered):
            \begin{equation}
                y_{i}>x,
                \quad\quad
                y_{i}\in{X}
            \end{equation}
            \begin{equation}
                v_{i}\in{L}_{x}^*
            \end{equation}
            Where:
            \begin{equation}
                L_{x}=\{a:a\leq{x}\}
            \end{equation}
            If $y\leq{x}$, then:
            \begin{equation}
                y_{i}\leq{x}\quad\quad
                y_{i}\in{X}
            \end{equation}
            and:
            \begin{equation}
                v_{i}\in{R}_{x}^{*}
            \end{equation}
            Where:
            \begin{equation}
                R_{x}=\{a:a>x\}
            \end{equation}
        \end{ldefinition}
        \begin{lexample}
            Let $x=3$ and let:
            \begin{equation}
                w=125312641237
            \end{equation}
            Then $y=7$, and thus $y>x$. Then we can write:
            \begin{equation}
                w=|12|5|312|6||4|123|7
            \end{equation}
            We allow for empty words. As another example,
            consider:
            \begin{equation}
                w=135712136412
            \end{equation}
            Then $y=2$, and thus $y<2$. We obtain:
            \begin{equation}
                w=1|3|57|2|1|3|641|2
            \end{equation}
        \end{lexample}
        \begin{ltheorem}{Foata's Theorem}
            The following is true:
            \begin{equation}
                \sum_{r\in\mathcal{R}(1^{k},0^{n-k})}q^{maj(r)}
                =\binom{n}{k}_{q}
            \end{equation}
        \end{ltheorem}
        \begin{proof}
            We want to define a bijection
            $\varphi$ from $\mathcal{R}(1^{k},0^{n-k})$ to itself
            such that, for any $r$, we have $maj(r)=inv(\varphi(r))$.
            Let $X\subseteq\mathbb{N}$. Let $X^{*}$ be the set
            of all words over $X$. For example if $X=\{0,1\}$, then
            $X^{*}$ is the set of all binary words. Define
            $\varphi:X^{*}\rightarrow{X}^{*}$ be such that
            $maj(w)=inv(\varphi(w))$ for any $w\in{X}^{*}$.
            Note that inversions and descents are defined in the
            same way as for permutations. This is why we required
            the set to be totally ordered, $\mathbb{N}$ in our
            case. Define $\gamma_{x}:X^{*}\rightarrow{X}^{*}$ by:
            \begin{equation}
                \gamma_{x}(w)=
                \begin{cases}
                    \emptyset,&w=\emptyset\\
                    y_{1}v_{1}\dots{y}_{k}v_{k},&
                    w=v_{1}y_{1}\dots{v}_{k}y_{k}
                \end{cases}
            \end{equation}
            We define $\varphi$ as follows:
            \begin{equation}
                \varphi(w)=
                \begin{cases}
                    \emptyset,&w=\emptyset\\
                    w,&w\in{X}\\
                    \gamma_{x}(\varphi(v)),&
                    w=vx,x\in{X}
                \end{cases}
            \end{equation}
            This is a recursive definition. For example, let:
            \begin{equation}
                w=121314
            \end{equation}
            Then $\varphi(1)=1$, and thus
            $\varphi(12)=\gamma_{2}(\phi(1))2=12$. The first
            interesting case is with three elements. We have:
            \begin{equation}
                \varphi(121)=
                \gamma_{1}(\varphi(12))1=
                \gamma_{1}(12)1=211
            \end{equation}
            Note that $inv(211)=2$ and $maj(121)=2$. This function
            works since, if $w\in{X}^{*}$, then let
            $r_{x}$ be the number of letters in $w$ that are
            greater than $x$, and let $\ell_{x}$ be the number
            of letters in $w$ that are less than or equal to $x$.
            Let $v\in{X}^{*}$ and $x\in{X}$. Then:
            \begin{equation}
                inv(vx)=inv(v)+r_{x}(v)
            \end{equation}
            Also, when the last letter of $v$ is less than or
            equal to $x$, we have:
            \begin{equation}
                inv(\gamma_{x}(v))=inv(v)-r_{x}(v)
            \end{equation}
            And otherwise we have:
            \begin{equation}
                inv(\gamma_{x}(v))=inv(v)+\ell_{x}(v)
            \end{equation}
            Moreover, if the last letter $v$ is less than or
            equal to $x$, then:
            \begin{equation}
                maj(vx)=maj(v)
            \end{equation}
            And otherwise:
            \begin{equation}
                maj(vx)+|v|
            \end{equation}
        \end{proof}
    \section{Lecture 9}
        As a summary, we we studying q-Analogues. We have shown:
        \begin{equation}
            \sum_{\sigma\in{S}_{n}}q^{inv(\sigma)}=
            \sum_{\sigma\in{S}_{n}}q^{maj(\sigma)}=
            [n]_{q}!
        \end{equation}
        Also:
        \begin{equation}
            \sum_{w\in\mathcal{R}(1^{k},0^{n-k})}
            q^{inv(w)}=
            \sum_{w\in\mathcal{R}(1^{k},0^{n-k})}
            q^{maj(w)}=
            \binom{n}{k}_{q}
        \end{equation}
        \begin{ltheorem}{q-Binomial Theorem}
            The following is true:
            \begin{equation}
                \prod_{i=1}^{n}
                (x+q^{i}y)=
                \sum{q}^{\binom{n-k+1}{2}}
                \binom{n}{k}_{q}x^{k}y^{n-k}
            \end{equation}
        \end{ltheorem}
        For all $n,k\in\mathbb{N}$, we have:
        \begin{equation}
            \binom{n+k}{k}_{q}=
            \sum_{\lambda\in{P}(n,k)}q^{|\lambda|}
        \end{equation}
        We can use rising factorials to define
        $\binom{x}{k}_{q}$ for all $x\in\mathbb{R}$ and
        $k\in\mathbb{N}$. That is:
        \begin{equation}
            \binom{x}{k}_{q}=
            \frac{(1-q^{x-k+1})(1-q^{x-k+1})}{(1-q)(1-q^{2})\dots}
        \end{equation}
        \subsection{q-Catalan Analogue}
            Recall that $C_{n}$ is the number of lattice paths
            from $(0,0)$ to $(n,n)$ that do not go below or
            above the line $x=y$ in the plane. We showed earlier
            that:
            \begin{equation}
                C_{n}=\frac{1}{n+1}\binom{2n}{n}
            \end{equation}
            Recall that $L(m,n)$ is the number of lattice paths
            from $(0,0)$ to $(m,n)$. We define $L^{+}(m,n)$ to be
            the set of lattice paths from $(0,0)$ to $(m,n)$ that
            do not go below the line $y=\frac{n}{m}x$. In
            particular, $L^{+}(n,n)=C_{n}$. The Catalan numbers
            satisfy the following recurrence:
            \begin{equation}
                C_{n}=\sum_{k=1}^{n}C_{k-1}C_{n-k}
            \end{equation}
            Recall that
            $\omega:L(m,n)\rightarrow\mathcal{R}(0^{n},1^{m}$,
            where $\omega$ maps $N\rightarrow{0}$ and
            $E$ to 1, north and east. Let
            $\mathcal{R}^{+}(0^{n}1^{n})$ be the elements f
            $\mathcal{R}(1^{n}0^{n})$ that correspond to
            Catalan paths. From a previous observation, the words
            corresponding to the Catalan paths are characterized
            by alwas have more 0's than 1's for any initial word.
            \begin{ltheorem}{MacMahor's Theorem}
                The following is true:
                \begin{equation}
                    \sum_{p\in{L}^{+}(n,n)}q^{maj(w(p))}=
                    \frac{1}{[n+1]_{q}}\binom{2n}{n}_{q}
                \end{equation}
            \end{ltheorem}
            \begin{proof}
                Define the following:
                \begin{align}
                    \mathcal{R}^{\minus}(0^{n}1^{n})
                        &=\mathcal{R}(0^{n}1^{n})
                        -\mathcal{R}^{+}(0^{n}1^{n})\\
                    L^{\minus}(n,n)=L(n,n)-L^{+}(n,n)
                \end{align}
                Given a path $P$ in $L^{\minus}(n,n)$, let $A$ be
                the lattice point with the smallest $x$ coordinate
                among all the lattice points $(i,j)$ with
                $i-j$ maximized, whose distance from the
                $x=y$ line in the south east direction is maximized.
                Let $B$ be the lattice point just before $A$.
                Notice that the step $B\rightarrow{A}$ must be
                an east step. Create a new path as follows. Change
                the east step to a north step, and then take the
                remaining path from $A$ to $(n,n)$ and shift it
                up one and two the left one. This path ends on
                $(n-1,n+1)$. Let $\varphi$ denote the new path.
                Then:
                \begin{equation}
                    maj(w(\varphi(p)))=maj(w(p))-1
                \end{equation}
                For suppose $B\ne(0,0)$. The the step that goes to
                $B$ must be an east step. For if not, then $A$ does
                not have the smallest $x$ coordinate with maximal
                distance to the line $y=x$. If $B=(0,0)$, then the
                first position goes away. Moreover, the algorithm
                is reversible. For let $P'$ be a lattice path from
                $(0,0)$ to $(n,m)$, and let $A'$ be the point with
                maximal $x$ coordinate such that $i-j$ is maximized.
                This point corresponds to the $B$ in the previous
                path. Therefore:
                \begin{equation}
                    \sum_{w\in\mathcal{R}(1^{n}0^{n})}q^{maj(w)}=
                    \sum_{w\in\mathcal{R}(1^{n+1}0^{n-1})}
                        q^{maj(w)+1}=
                    q\binom{2n}{n+1}
                \end{equation}
            \end{proof}
            There is another q-analogue for $C_{n}$ due to
            Carlitz and Riordan. Let $p\in{L}^{+}(n,n)$ and
            define $a_{i}(p)$ to be the number of complete
            squares between the path and the $x=y$ line in row
            $i$. The number $a_{i}(p)$ is called the length of
            the $i^{th}$ row of $p$ and the sequence
            $(a_{1}(p),\dots,a_{n}(p))$ is called the
            co-area vector of $p$. The co-area statistic on
            $p$ is defined as:
            \begin{equation}
                Coarea(p)=\sum_{i=1}^{n}a_{i}(p)
            \end{equation}
            \begin{ltheorem}{Carlitz-Riordan Theorem}
                The following is true:
                \begin{equation}
                    C_{n}(q)=\sum_{p\in{L}^{+}(n,n)}q^{Coarea(p)}
                \end{equation}
            \end{ltheorem}
        Using the Carlitz-Riordan theorem, we can show the
        following result.
        \begin{theorem}
            The following is true:
            \begin{equation}
                C_{n}(q)=\sum_{k=1}^{n}q^{k-1}
                    C_{k-1}(q)C_{n-k}(q)
            \end{equation}
        \end{theorem}
        If we set $x\mapsto{q}^{i}x$, and $y\mapsto{1}$ in the
        q-binomial theorem, then we obtain:
        \begin{equation}
            (\minus{x};q)=\prod_{k=0}^{n-1}(q^{i}x+1)=
            \sum_{k=0}^{n}q^{\binom{k}{2}}\binom{n}{k}_{q}x^{k}
        \end{equation}
        Using this q-binomial, we get the following.
        \begin{theorem}
            If $h,n,m\in\mathbb{N}$, then:
            \begin{equation}
                \sum_{k=0}^{n}q^{(n-k)(h-k)}\binom{n}{k}_{q}
                \binom{m}{n-k}_{q}=
                \binom{m+n}{h}_{q}
            \end{equation}
        \end{theorem}
    \section{Generating Functions}
        Given a q-analogue and a set $S$, we can then define a
        statistic, $\lambda$. We then have:
        \begin{equation}
            \sum_{s\in{S}}q^{\lambda(s)}=
            \sum_{i=0}^{n}a_{i}q^{i}
        \end{equation}
        Where $a_{i}$ is the number of elements in $S$ with
        statistic value $i$. Thus we can think of a
        statistic $:S\rightarrow\mathbb{N}$, called the
        value function.
        \begin{lexample}
            Let $n\in\mathbb{N}$ and consider
            $S=\mathcal{P}(\mathbb{Z}_{n})$. One easy statistic
            we can place on $S$ is the cardinality function. That is,
            we define $f:S\rightarrow\mathbb{N}$ by:
            \begin{equation}
                f(\omega)=\Card(\omega)
                \quad\quad
                \omega\in\mathcal{P}(\mathbb{Z}_{n})
            \end{equation}
            Let's compute this a different way. Given a set
            $A\subseteq\mathcal{P}(\mathbb{Z}_{n})$, either
            $1\in{A}$ or $1\notin{A}$. Similarly, either
            $2\in{A}$ or $2\notin{A}$. For all $k\in\mathbb{Z}$,
            either $k\in{A}$ or $k\notin{A}$. Thus, we have:
            \begin{equation}
                (q^{1}+q^{0})\cdots(q^{1}+q^{0})=
                \prod_{k=1}^{n}(q^{1}+q^{0})
                =[2]_{q}^{n}
                =\sum_{k=0}^{n}\binom{n}{k}q^{k}
            \end{equation}
        \end{lexample}
        Next we want to consider $S$ being infinite. To get
        a generating function we require that $a_{i}$ is equal
        to the number of elements in $S$ with value $i$ being finite.
        We obtain the following power series:
        \begin{equation}
            a_{0}q^{0}+a_{1}q^{1}+\dots
            =\sum_{i=0}^{\infty}a_{i}q^{i}
        \end{equation}
        Let $\mathbb{C}[[q]]$ denote the ring of formal power
        series. This is a ring. For let:
        \begin{subequations}
            \begin{align}
                A(q)&=\sum_{i=0}^{\infty}a_{i}q^{i}\\
                B(q)&=\sum_{i=0}^{\infty}b_{i}q^{i}
            \end{align}
        \end{subequations}
        Then the sum is well defined, and we have:
        \begin{equation}
            A(q)+B(q)=
            \sum_{i=0}^{\infty}(a_{i}+b_{i})q^{i}
        \end{equation}
        We can also define the product by using the convolution
        product, or Cauchy sums:
        \begin{equation}
            A(q)B(q)=\sum_{i=0}^{\infty}c_{i}q^{i}
        \end{equation}
        Where:
        \begin{equation}
            c_{k}=\sum_{i=0}^{k}a_{i}b_{k-i}
        \end{equation}
        Some properties of $\mathbb{C}[[q]]$ is that it is a
        commutative ring. This is because we considered the
        coefficients to be over $\mathbb{C}$. Moreover, it is an
        integral domain. That is, $\mathbb{C}[[q]]$ has no zero
        divisors. The units, or invertible elements, are formal
        power series such that $a_{o}\ne{0}$. Indeed, this is a
        necessary and sufficient condition for an element to be
        invertible. For example, consider:
        \begin{equation}
            A(q)=\sum_{i=0}^{\infty}q^{i}
        \end{equation}
        This is a geometric sum, and we can show that for
        $|q|<1$, this formal sum is a convergent sum and evaluates
        to $(1-q)^{\minus{1}}$. However, for all formal sums, this
        formal power series has an inverse, and the inverse is
        indeed $(1-q)^{\minus{1}}$. Ivan Niven has a nice article
        on $\mathbb{C}[qq]]$ in the American Mathematical Monthly,
        1969. This has applications in counting partitions of
        numbers. See Andrews Theory of Partitions. Let $p(n)$ be
        the number of partitions on $n$. Then:
        \begin{equation}
            \sum_{n=0}^{\infty}p(n)q^{n}=
            1+q+2q^{2}+3q^{3}+5q^{4}+7q^{5}+11q^{6}+15q^{7}+\dots
        \end{equation}
        The value function is thus the weight of the partion
        $|\lambda|=\lambda_{1}+\dots+\lambda_{m}$.
        \begin{ltheorem}{Euler's Thoemre}
            If $P$ is the set of partitions, then:
            \begin{equation}
                \sum_{\lambda\in{P}}q^{|\lambda|}=
                \prod_{k=1}^{\infty}\frac{1}{q-q^{i}}
            \end{equation}
        \end{ltheorem}
        Give an arbitrary partition $\lambda$, consider the parts
        of size one. $\lambda$ does not have a part of size 1 or
        $\lambda$ has a part of size one, or $\lambda$ has a part of
        size two, and so on. Now do the same for each of the parts
        of size $k$, in general.
        \begin{ltheorem}{Euler's Other Theorem}
            The number of partitions with $n$ distinct parts
            is equal to the number of partitions of $n$
            with only odd parts.
        \end{ltheorem}
        \begin{proof}
            For:
            \begin{equation}
                \sum_{n=0}^{\infty}a_{n}q^{n}=
                \prod_{i=1}^{\infty}(1+q^{i})=
                \prod_{i=1}^{\infty}\frac{(q+q^{i})(1-q^{i})}{1-q^{i}}
            \end{equation}
            We can then simplify:
        \end{proof}
        \begin{ldefinition}{Durfee Square}
            The largest square that fits into a partition
            $\lambda$ is called the Durfee square of $\lambda$.
        \end{ldefinition}
        \begin{ldefinition}{Self-Conjugate Partition}
            A self-conjugate partition is a partition
            $\lambda$ such that $\lambda=\lambda'$, where
            $\lambda'$ is the conjugate of $\lambda$.
        \end{ldefinition}
        \begin{ltheorem}{Euler's Other-Other Theorem}
            The following is true:
            \begin{equation}
                \sum_{\lambda=\lambda'}q^{|\lambda|}=
                \prod_{n=0}^{\infty}(1+q^{2n+1})
            \end{equation}
            Where $\lambda=\lambda'$ are all of the self-conjugate
            partitions.
        \end{ltheorem}
        The product is the generating function for partitions with
        distinct odd parts. Euler's theorem then says that the
        generating function for this set is the equal to the sum
        over all of the self-conjugate partitions.
    \section{Euler's Theorem}
        \begin{equation}
            \sum_{n\in\mathbb{N}}p(n)q^{n}=
            \prod_{i=1}^{\infty}\frac{1}{1-q^{i}}
        \end{equation}
        Where $p(n)$ is the number of partitations of $n$.
        We also define the Euler function, not to be confused
        with the Euler totient function, as:
        \begin{equation}
            \phi(q)=\prod_{i=1}^{\infty}(1-q^{i})
        \end{equation}
        Let's try to simplify this:
        \begin{equation}
            \phi(q)=\prod_{i=1}^{\infty}(1-q^{i})=
            \sum_{k=0}^{\infty}b_{k}q^{k}
        \end{equation}
        We want to find the $b_{k}$. Multiplying through by
        the original series from Euler's theorem, we get:
        \begin{equation}
            \Big(\sum_{i=1}^{\infty}b_{i}q^{i}\Big)
            \Big(\sum_{j=0}^{\infty}p(j)q^{j}\Big)=1
        \end{equation}
        Using the convolution product, we have for all $k\geq{1}$:
        \begin{equation}
            \sum_{j=0}^{k}b_{j}p(k-j)=0
        \end{equation}
        This gives a recursion for $p(k)$. This gives us
        Euler's Pentagonal Number Theorem.
        \begin{ltheorem}{Euler's Pentagonal Number Theorem}
              {Euler_Pentagonal_Number_Theorem}
            \begin{subequations}
                \begin{align}
                    \phi(q)&=\prod_{i=1}^{\infty}(1-q^{i})\\
                    &=1+\sum_{m=1}^{\infty}(\minus{1})^{m}
                    \Big(q^{\frac{m(3m-2)}{2}}+
                        q^{\frac{m(3m+1)}{2}}\Big)\\
                    &=\sum_{m=\minus\infty}^{\infty}
                        (\minus{1})^{m}q^{\frac{m(3m-1)}{2}}
                \end{align}
            \end{subequations}
        \end{ltheorem}
        Let $p_{e}(d,n)$ denote the number of partitions $n$ with
        distinct parts and even length. Similarly, define
        $p_{o}(d,x)$ for odd length.
        \begin{theorem}
            \begin{equation}
                p_{e}(d,n)-p_{o}(d,n)=
                \begin{cases}
                    (\minus{1})^{n},&n=\frac{m(3m\pm{1})}{2}\\
                    0,&\textrm{Otherwise}
                \end{cases}
            \end{equation}
        \end{theorem}
        After some reflection, it should be easy to see that
        the first case is the inverse of the second case. Moreover,
        cases 1, 2, and 3 cover all partitions with distinct
        parts. For any partition in case three only one of
        $a$ or $b$ is true. The bijection constructed proves
        Euler's Pentagonal Theorem. Now we can compute $p$:
        \begin{subequations}
            \begin{align}
                p(0)&=1\\
                p(1)&=1\\
                p(2)&=2\\
                p(3)&=3\\
                p(4)&=p(3)+p(2)=5\\
                p(5)&=p(4)+p(3)-p(0)=7\\
                p(6)&=p(5)+p(4)-p(1)=11\\
                p(7)&=p(6)+p(5)-p(2)-p(0)=15
            \end{align}
        \end{subequations}
        And in general:
        \begin{equation}
            p(n)=p(n-1)+p(n-2)-p(n-5)-p(n-7)+\dots
        \end{equation}
        Where we add and subtrack over the pentagonal numbers.
        Gauss then turned to the question of computing powers
        of $\phi(q)$.
        \begin{ltheorem}{Gauss's Pentagonal Theorem}
              {Gauss_Pentagonal_Theorem}
            \begin{equation}
                \phi(q)^{3}=\prod_{i=1}^{\infty}(1-q^{i})^{3}
                =\sum_{r=0}^{\infty}(\minus{1})^{r}(2r+1)
                    q^{\frac{r(r+1)}{2}}
            \end{equation}
        \end{ltheorem}
        This identity occurs in many different areas of mathematics,
        such as homological algebra, complex analysis, and
        hyperbolic geometry. The proof comes from jacobi's
        Triple Product Identity.
        \begin{ltheorem}{Jacobi's Triple Product Identity}
              {Jacobi_Triple_Product_Identity}
            \begin{equation}
                \sum_{n=\minus\infty}^{\infty}z^{n}q^{n^{2}}=
                    \prod_{n=0}^{\infty}(1-q^{2n+2})
                        (1+zq^{2n+1})(1+z^{\minus{1}}q^{2n+1})
            \end{equation}
        \end{ltheorem}
        The proof of Gauss' identity then uses Sylvester's
        bijection. To get this from Jacobi, do a shift of
        index starting from $n=0$ to $n=1$. Differentiate both
        sides with respect to $q$, and then put
        $z=\minus{q}$. Finally, map $q^{2}$ to $q$.
        \par\hfill\par
        Felix Klein computed $\phi(q)^{8}$. In the theory of
        modular forms there is something called the $\tau$
        function, due to Ramanujan. This has the property that:
        \begin{equation}
            \sum_{n=1}^{\infty}\tau(n)q^{n-1}=
            \phi(q)^{24}=\prod_{m=1}^{\infty}(1-q^{m})^{24}
        \end{equation}
        Freeman Dyson also had some contributions to this subject
        and came up with nice formula for $\phi(q)^{d}$ when
        $d=3,8,10,14,15,21,24,26,28,35,36,\dots$ With the
        exception of $26$, these are the dimensions of the Lie
        algebras. Ian McDonald, working on the same problem, saw
        this as well. He came up with the following:
        \begin{equation}
            \phi(q)^{n^{2}-1}=\sum\varepsilon(k_{1},\dots,k_{n})
                \prod_{i=1}^{n}\binom{k_{i}}{n-i}q^{k_{n}}
        \end{equation}
        Where this is summed over all tuples $(k_{1},\dots,k_{n})$
        of non-negative integers such that:
        \begin{equation}
            \sum_{i=1}^{n}k_{i}^{2}=\sum_{i=1}^{k}k_{i}+
            \sum_{i=1}^{n-1}k_{i}k_{i+1}
            +k_{n}k_{1}
        \end{equation}
        And where $\varepsilon(k_{1},\dots,k_{m})=\pm{1}$.
    \section{Generating Function for Multisets}
        Recall that a multiset is a colection with repetition.
        For example:
        \begin{equation}
            A=\{\{1,1,1,3,3,4\}\}
        \end{equation}
        This is different from the set
        $B=\{1,1,1,3,3,4\}$, since sets cannot account for
        repetition. That is, $B$ can be reduced down to
        $B=\{1,3,4\}$. Note that partitions are multisets and
        we have shown that $\binom{k+(n-1)}{k}$ is the
        number of multi-sets of size $k$ chosen from the
        set $[n]$. We want to compute the generating function
        for the number of multi-sets of size $k$:
        \begin{equation}
            f(M)=\sum_{M}q^{|M|}
        \end{equation}
        Where $M$ is a multi-set of elements in $[n]$, and
        $|M|$ denotes the number of elements in $M$, with
        repetitions included. If $M$ is an arbitrary multi-set,
        then either $1\ne{M}$, or $1\in{M}$, or $1,1\in{M}$,
        and so on. So, in general, we get:
        \begin{equation}
            \sum_{k=0}^{\infty}q^{k}=\frac{1}{1-q}
        \end{equation}
        In general:
        \begin{equation}
            \sum_{M}q^{|M|}=\frac{1}{(1-q)^{n}}
            =\sum_{k=0}^{\infty}\binom{n-1+k}{k}q^{k}
        \end{equation}
        From the binomial theorem, we get:
        \begin{equation}
            (1+q)^{n}=
            \sum_{k=0}^{n}\binom{n}{k}q^{k}
        \end{equation}
        And thus, we can define:
        \begin{equation}
            \binom{\minus{n}}{k}=
            \binom{n-1+k}{k}(\minus{1})^{k}
        \end{equation}
        Extending the binomial coefficient to all
        $n\in\mathbb{Z}$. Next, recall the q-binomial theorem.
        \begin{equation}
            \sum_{k=0}^{n}\binom{n}{k}_{q}q^{\binom{k}{2}}x^{k}
            =\prod_{k=1}^{n-1}(1+q^{k}x)
        \end{equation}
        And also:
        \begin{equation}
            \sum_{k=0}^{\infty}\binom{n+k}{k}_{q}x^{k}=
            \prod_{i=0}^{n}(1-q^{i}x)^{\minus{1}}
        \end{equation}
        See Stanley and MacDonald.
    \section{Symmetric Functions}
        \subsection{Symmetric Polynomials}
            Let $x_{1},\dots,x_{N}$ commute, and let
            $f\in\mathbb{C}[x_{1},\dots,x_{N}]$. Then $f$ is
            symmetric if, for all $\sigma\in{S}_{N}$, then:
            \begin{equation}
            f(x_{1},\dots,x_{N})
            =f(x_{\sigma(1)},\dots,x_{\sigma(N)})
            \end{equation}
            Where $S_{N}$ is the symmetric group, and
            $\sigma$ is any permutation. It is convenient to
            work with infinitely many variables. We impose the
            requirements that there are countable many variables,
            so that we may list them, and that the commute. In
            this case we have power series instead of polynomials.
            We thus get sums of the form:
            \begin{equation}
                f=\sum_{\alpha}C_{\alpha}x^{\alpha}
            \end{equation}
            Where $\alpha$ is a sequence of non-negative integers.
            We require that the sum over $\alpha$ be finite, and
            thus this implies that all of the monomials are of
            finite degree. $f$ is said to be homogeneous if all
            of the monomials have the same degree. We require that
            $f$ be invariant under any permutation
            $\sigma:\mathbb{N}\rightarrow\mathbb{N}$.
            \begin{ldefinition}{Monomial Symmetric Basis}
                A monomial symmetric basis is:
                \begin{equation}
                    m_{\lambda}=m_{\lambda}(x)+
                    \sum_{\alpha}x^{\alpha}
                \end{equation}
                Where $\alpha$ is a rearrangement of $\lambda$.
            \end{ldefinition}
            That is, $m_{\lambda}$ is the sum of all monomials
            in $x_{i}$ whose exponents are the parts of $\lambda$.
            \begin{example}
                \begin{subequations}
                    \begin{align}
                        m_{1,1}=\sum_{i<j}x_{i}x_{j}&=
                            x_{1}x_{2}+x_{1}x_{3}+\dots
                            +x_{2}x_{3}+\dots\\
                        m_{2,1,1}(x_{1},x_{2},x_{3})&=
                            x_{1}^{2}x_{2}x_{2}+x_{1}x_{2}^{2}x_{3}+
                                x_{1}x_{2}x_{3}^{2}\\
                        m_{2}(x)=\sum_{i=1}^{\infty}x_{i}^{2}
                    \end{align}
                \end{subequations}
            \end{example}
            \begin{ldefinition}{Elementary Symmetric Function}
                The elementary symmetric function is defined as:
                \begin{equation}
                    e_{k}=m_{1^{k}}=
                    \sum_{i_{1}<i_{2}<\dots<i_{k}}
                        x_{i_{1}}x_{i_{2}}\cdots{x}_{i_{k}}
                \end{equation}
                For $k\in\mathbb{N}$.
            \end{ldefinition}
            Note that:
            \begin{equation}
                \prod_{i=1}^{\infty}(1+zx_{i})=
                \sum_{n=0}^{\infty}e_{n}z^{n}
            \end{equation}
            \begin{ldefinition}{Power Symmetric Function}
                The power symmetric function is defined as:
                \begin{equation}
                    P_{k}=m_{k}=
                    \sum_{i=1}^{\infty}x_{i}^{k}
                \end{equation}
                For $k\in\mathbb{N}$. That is, $k\geq{1}$.
            \end{ldefinition}
            \begin{ldefinition}
                  {Complete Homogeneous Symmetric Function}
                The complete homogeneous symmetric function is
                defined as:
                \begin{equation}
                    h_{k}=\sum_{\lambda+k}m_{\lambda}
                \end{equation}
                That is, the sum of all monomials of degree $k$.
            \end{ldefinition}
            \begin{theorem}
                The generating function for the homogeneous
                symmetric function is:
                \begin{equation}
                    \prod_{i=1}^{\infty}\frac{1}{1-zx_{i}}=
                    \sum_{n=0}^{\infty}h_{n}z^{n}
                \end{equation}
                Where $h_{0}$ is defined as $h_{0}=1$.
            \end{theorem}
            An endomorphism is a function such that
            $w(fg)=w(f)w(g)$, $w(f+g)=w(f)+w(g)$, and
            $w(cf)=cw(f)$.
            \begin{ldefinition}{$\omega$ Involution}
                The $\omega$ involution is the endomorphism defined
                by:
                \begin{equation}
                    \omega(p_{k})=(\minus{1})^{k=1}P_{k}
                \end{equation}
                And:
                \begin{equation}
                    \omega(p_{\lambda})=
                    (\minus{1})^{n-\ell(\lambda)}P_{\lambda}
                \end{equation}
            \end{ldefinition}