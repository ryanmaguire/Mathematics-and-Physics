This work contains mathematics and physics. It starts from scratch and develops,
in a rather lengthy fashion, all of the mathematics that I have come across and
decided to write down. An attempt was made (but almost certainly failed) to
mimic the style of Euclid's text \textit{The Elements}, in which he proclaims a
few postulates and definitions, and then proceeds from there in developing over
400 propositions and theorems in a logical order. This is not a complete mimicry
since there are discussions, examples, and many figures to give intuition
whereas the elements is simply theorem-proof, with figures only drawn to show a
construction described in a proof. Unlike most textbooks there are no exercises,
but rather an abundance of worked out examples and an attempt was made to prove
every claim in a logical and consistent order. The goal is to justify every step
by a definition, axiom, or previously proved theorem. As such, there are no
logical prerequesites to read the theorems and proofs, but the examples that are
used to build intuition often presume a belief in the existence of real numbers
(in particular, the non-negative integers and rational numbers), and some of
the motivating examples also use calculus and the elementary algebra of a
polynomial in one real variable. Both of these concepts are, eventually,
formally developed, but for pedagogical reasons many examples use these notions
beforehand. Theorems and proofs do not rely on examples, and in this sense there
are no prerequesites. A reader lacking a background in more rigorous mathematics
may fail to see the point in laboriously developing logic and set theory, and
might not find any motivation for certain definitions, but nonetheless should be
able to follow along the proofs in the order presented.
\par\hfill\par
This is not a textbook (or collection thereof) in the usual sense in that, as
mentioned previously, all claims are worked out in full. There are no steps that
are \textit{left as an exercise to the reader}. This can still be used as a
textbook if the reader simply reads the claim of a theorem and tries to prove
it first before reading onwards. Since it is all too tempting to allow ones eyes
to wander all of 2 inches to the solution, many excellent textbooks for various
topics are cited in the bibliography. Thus this work can be seen as a supplement
to these, or as a standalone. The existence of such a work is to give those
eager to see mathematics presented in a single logical order a source to work
with. Knowing the troubles of G\"{o}del's Incompleteness Theorems
(Discussed in Book~\ref{book:Foundations}), this is merely an attempt at doing
so. The advanced mathematician will find that having all of the details spelled
out for them to be superfluous, and the beginner will not have the time to read
such a large volume, nonetheless I feel such a text should \textit{exist}. 
\par\hfill\par
The first book deals with logic and set theory and is perhaps the most
contentious. Althought I've tried to find consensus about what the definitions
of various primitive notions, such as \textit{set}, \textit{proposition},
\textit{predicate}, \textit{truth}, etc., there seems to be no such universally
agreed upon definitions and many arguments started to feel circular. Thus an
intuitive approach was taken, defining various things in somewhat of a
dictionary style. This may appall the logician, and corrections and advice are
more than welcome, but for most of mathematics this seems to work well.
\par\hfill\par
This project is very much a work in progress and will remain so for many years,
do in part to the sheer scope of the project. Any and all suggestions,
corrections, and improvements are welcome and the source code is hosted on
GitHub\footnote{\url{https://github.com/ryanmaguire/Mathematics-and-Physics}}
under the GNU GPL 3 license. My only wish is that this material is not
\textit{stolen} in the sense that one claims the work their own, but all of the
code is freely available and may be used by anyone. This includes all of the
tikz code for reproducing figures. Figures produced via the use of the C
programming language are compatible with the C99 standard, and the asymptote
code is not too innovative either.%
\footnote{\url{https://github.com/ryanmaguire/%
               Mathematics-and-Physics/tree/master/tikz}}
\begin{flushright}
    Ryan Maguire,\\
    Lowell, MA\\
\end{flushright}