%------------------------------------------------------------------------------%
\documentclass{article}                                                        %
%------------------------------Preamble----------------------------------------%
\makeatletter                                                                  %
    \def\input@path{{../../}}                                                  %
\makeatother                                                                   %
%---------------------------Packages----------------------------%
\usepackage{geometry}
\geometry{b5paper, margin=1.0in}
\usepackage[T1]{fontenc}
\usepackage{graphicx, float}            % Graphics/Images.
\usepackage{natbib}                     % For bibliographies.
\bibliographystyle{agsm}                % Bibliography style.
\usepackage[french, english]{babel}     % Language typesetting.
\usepackage[dvipsnames]{xcolor}         % Color names.
\usepackage{listings}                   % Verbatim-Like Tools.
\usepackage{mathtools, esint, mathrsfs} % amsmath and integrals.
\usepackage{amsthm, amsfonts, amssymb}  % Fonts and theorems.
\usepackage{tcolorbox}                  % Frames around theorems.
\usepackage{upgreek}                    % Non-Italic Greek.
\usepackage{fmtcount, etoolbox}         % For the \book{} command.
\usepackage[newparttoc]{titlesec}       % Formatting chapter, etc.
\usepackage{titletoc}                   % Allows \book in toc.
\usepackage[nottoc]{tocbibind}          % Bibliography in toc.
\usepackage[titles]{tocloft}            % ToC formatting.
\usepackage{pgfplots, tikz}             % Drawing/graphing tools.
\usepackage{imakeidx}                   % Used for index.
\usetikzlibrary{
    calc,                   % Calculating right angles and more.
    angles,                 % Drawing angles within triangles.
    arrows.meta,            % Latex and Stealth arrows.
    quotes,                 % Adding labels to angles.
    positioning,            % Relative positioning of nodes.
    decorations.markings,   % Adding arrows in the middle of a line.
    patterns,
    arrows
}                                       % Libraries for tikz.
\pgfplotsset{compat=1.9}                % Version of pgfplots.
\usepackage[font=scriptsize,
            labelformat=simple,
            labelsep=colon]{subcaption} % Subfigure captions.
\usepackage[font={scriptsize},
            hypcap=true,
            labelsep=colon]{caption}    % Figure captions.
\usepackage[pdftex,
            pdfauthor={Ryan Maguire},
            pdftitle={Mathematics and Physics},
            pdfsubject={Mathematics, Physics, Science},
            pdfkeywords={Mathematics, Physics, Computer Science, Biology},
            pdfproducer={LaTeX},
            pdfcreator={pdflatex}]{hyperref}
\hypersetup{
    colorlinks=true,
    linkcolor=blue,
    filecolor=magenta,
    urlcolor=Cerulean,
    citecolor=SkyBlue
}                           % Colors for hyperref.
\usepackage[toc,acronym,nogroupskip,nopostdot]{glossaries}
\usepackage{glossary-mcols}
%------------------------Theorem Styles-------------------------%
\theoremstyle{plain}
\newtheorem{theorem}{Theorem}[section]

% Define theorem style for default spacing and normal font.
\newtheoremstyle{normal}
    {\topsep}               % Amount of space above the theorem.
    {\topsep}               % Amount of space below the theorem.
    {}                      % Font used for body of theorem.
    {}                      % Measure of space to indent.
    {\bfseries}             % Font of the header of the theorem.
    {}                      % Punctuation between head and body.
    {.5em}                  % Space after theorem head.
    {}

% Italic header environment.
\newtheoremstyle{thmit}{\topsep}{\topsep}{}{}{\itshape}{}{0.5em}{}

% Define environments with italic headers.
\theoremstyle{thmit}
\newtheorem*{solution}{Solution}

% Define default environments.
\theoremstyle{normal}
\newtheorem{example}{Example}[section]
\newtheorem{definition}{Definition}[section]
\newtheorem{problem}{Problem}[section]

% Define framed environment.
\tcbuselibrary{most}
\newtcbtheorem[use counter*=theorem]{ftheorem}{Theorem}{%
    before=\par\vspace{2ex},
    boxsep=0.5\topsep,
    after=\par\vspace{2ex},
    colback=green!5,
    colframe=green!35!black,
    fonttitle=\bfseries\upshape%
}{thm}

\newtcbtheorem[auto counter, number within=section]{faxiom}{Axiom}{%
    before=\par\vspace{2ex},
    boxsep=0.5\topsep,
    after=\par\vspace{2ex},
    colback=Apricot!5,
    colframe=Apricot!35!black,
    fonttitle=\bfseries\upshape%
}{ax}

\newtcbtheorem[use counter*=definition]{fdefinition}{Definition}{%
    before=\par\vspace{2ex},
    boxsep=0.5\topsep,
    after=\par\vspace{2ex},
    colback=blue!5!white,
    colframe=blue!75!black,
    fonttitle=\bfseries\upshape%
}{def}

\newtcbtheorem[use counter*=example]{fexample}{Example}{%
    before=\par\vspace{2ex},
    boxsep=0.5\topsep,
    after=\par\vspace{2ex},
    colback=red!5!white,
    colframe=red!75!black,
    fonttitle=\bfseries\upshape%
}{ex}

\newtcbtheorem[auto counter, number within=section]{fnotation}{Notation}{%
    before=\par\vspace{2ex},
    boxsep=0.5\topsep,
    after=\par\vspace{2ex},
    colback=SeaGreen!5!white,
    colframe=SeaGreen!75!black,
    fonttitle=\bfseries\upshape%
}{not}

\newtcbtheorem[use counter*=remark]{fremark}{Remark}{%
    fonttitle=\bfseries\upshape,
    colback=Goldenrod!5!white,
    colframe=Goldenrod!75!black}{ex}

\newenvironment{bproof}{\textit{Proof.}}{\hfill$\square$}
\tcolorboxenvironment{bproof}{%
    blanker,
    breakable,
    left=3mm,
    before skip=5pt,
    after skip=10pt,
    borderline west={0.6mm}{0pt}{green!80!black}
}

\AtEndEnvironment{lexample}{$\hfill\textcolor{red}{\blacksquare}$}
\newtcbtheorem[use counter*=example]{lexample}{Example}{%
    empty,
    title={Example~\theexample},
    boxed title style={%
        empty,
        size=minimal,
        toprule=2pt,
        top=0.5\topsep,
    },
    coltitle=red,
    fonttitle=\bfseries,
    parbox=false,
    boxsep=0pt,
    before=\par\vspace{2ex},
    left=0pt,
    right=0pt,
    top=3ex,
    bottom=1ex,
    before=\par\vspace{2ex},
    after=\par\vspace{2ex},
    breakable,
    pad at break*=0mm,
    vfill before first,
    overlay unbroken={%
        \draw[red, line width=2pt]
            ([yshift=-1.2ex]title.south-|frame.west) to
            ([yshift=-1.2ex]title.south-|frame.east);
        },
    overlay first={%
        \draw[red, line width=2pt]
            ([yshift=-1.2ex]title.south-|frame.west) to
            ([yshift=-1.2ex]title.south-|frame.east);
    },
}{ex}

\AtEndEnvironment{ldefinition}{$\hfill\textcolor{Blue}{\blacksquare}$}
\newtcbtheorem[use counter*=definition]{ldefinition}{Definition}{%
    empty,
    title={Definition~\thedefinition:~{#1}},
    boxed title style={%
        empty,
        size=minimal,
        toprule=2pt,
        top=0.5\topsep,
    },
    coltitle=Blue,
    fonttitle=\bfseries,
    parbox=false,
    boxsep=0pt,
    before=\par\vspace{2ex},
    left=0pt,
    right=0pt,
    top=3ex,
    bottom=0pt,
    before=\par\vspace{2ex},
    after=\par\vspace{1ex},
    breakable,
    pad at break*=0mm,
    vfill before first,
    overlay unbroken={%
        \draw[Blue, line width=2pt]
            ([yshift=-1.2ex]title.south-|frame.west) to
            ([yshift=-1.2ex]title.south-|frame.east);
        },
    overlay first={%
        \draw[Blue, line width=2pt]
            ([yshift=-1.2ex]title.south-|frame.west) to
            ([yshift=-1.2ex]title.south-|frame.east);
    },
}{def}

\AtEndEnvironment{ltheorem}{$\hfill\textcolor{Green}{\blacksquare}$}
\newtcbtheorem[use counter*=theorem]{ltheorem}{Theorem}{%
    empty,
    title={Theorem~\thetheorem:~{#1}},
    boxed title style={%
        empty,
        size=minimal,
        toprule=2pt,
        top=0.5\topsep,
    },
    coltitle=Green,
    fonttitle=\bfseries,
    parbox=false,
    boxsep=0pt,
    before=\par\vspace{2ex},
    left=0pt,
    right=0pt,
    top=3ex,
    bottom=-1.5ex,
    breakable,
    pad at break*=0mm,
    vfill before first,
    overlay unbroken={%
        \draw[Green, line width=2pt]
            ([yshift=-1.2ex]title.south-|frame.west) to
            ([yshift=-1.2ex]title.south-|frame.east);},
    overlay first={%
        \draw[Green, line width=2pt]
            ([yshift=-1.2ex]title.south-|frame.west) to
            ([yshift=-1.2ex]title.south-|frame.east);
    }
}{thm}

%--------------------Declared Math Operators--------------------%
\DeclareMathOperator{\adjoint}{adj}         % Adjoint.
\DeclareMathOperator{\Card}{Card}           % Cardinality.
\DeclareMathOperator{\curl}{curl}           % Curl.
\DeclareMathOperator{\diam}{diam}           % Diameter.
\DeclareMathOperator{\dist}{dist}           % Distance.
\DeclareMathOperator{\Div}{div}             % Divergence.
\DeclareMathOperator{\Erf}{Erf}             % Error Function.
\DeclareMathOperator{\Erfc}{Erfc}           % Complementary Error Function.
\DeclareMathOperator{\Ext}{Ext}             % Exterior.
\DeclareMathOperator{\GCD}{GCD}             % Greatest common denominator.
\DeclareMathOperator{\grad}{grad}           % Gradient
\DeclareMathOperator{\Ima}{Im}              % Image.
\DeclareMathOperator{\Int}{Int}             % Interior.
\DeclareMathOperator{\LC}{LC}               % Leading coefficient.
\DeclareMathOperator{\LCM}{LCM}             % Least common multiple.
\DeclareMathOperator{\LM}{LM}               % Leading monomial.
\DeclareMathOperator{\LT}{LT}               % Leading term.
\DeclareMathOperator{\Mod}{mod}             % Modulus.
\DeclareMathOperator{\Mon}{Mon}             % Monomial.
\DeclareMathOperator{\multideg}{mutlideg}   % Multi-Degree (Graphs).
\DeclareMathOperator{\nul}{nul}             % Null space of operator.
\DeclareMathOperator{\Ord}{Ord}             % Ordinal of ordered set.
\DeclareMathOperator{\Prin}{Prin}           % Principal value.
\DeclareMathOperator{\proj}{proj}           % Projection.
\DeclareMathOperator{\Refl}{Refl}           % Reflection operator.
\DeclareMathOperator{\rk}{rk}               % Rank of operator.
\DeclareMathOperator{\sgn}{sgn}             % Sign of a number.
\DeclareMathOperator{\sinc}{sinc}           % Sinc function.
\DeclareMathOperator{\Span}{Span}           % Span of a set.
\DeclareMathOperator{\Spec}{Spec}           % Spectrum.
\DeclareMathOperator{\supp}{supp}           % Support
\DeclareMathOperator{\Tr}{Tr}               % Trace of matrix.
%--------------------Declared Math Symbols--------------------%
\DeclareMathSymbol{\minus}{\mathbin}{AMSa}{"39} % Unary minus sign.
%------------------------New Commands---------------------------%
\DeclarePairedDelimiter\norm{\lVert}{\rVert}
\DeclarePairedDelimiter\ceil{\lceil}{\rceil}
\DeclarePairedDelimiter\floor{\lfloor}{\rfloor}
\newcommand*\diff{\mathop{}\!\mathrm{d}}
\newcommand*\Diff[1]{\mathop{}\!\mathrm{d^#1}}
\renewcommand*{\glstextformat}[1]{\textcolor{RoyalBlue}{#1}}
\renewcommand{\glsnamefont}[1]{\textbf{#1}}
\renewcommand\labelitemii{$\circ$}
\renewcommand\thesubfigure{%
    \arabic{chapter}.\arabic{figure}.\arabic{subfigure}}
\addto\captionsenglish{\renewcommand{\figurename}{Fig.}}
\numberwithin{equation}{section}

\renewcommand{\vector}[1]{\boldsymbol{\mathrm{#1}}}

\newcommand{\uvector}[1]{\boldsymbol{\hat{\mathrm{#1}}}}
\newcommand{\topspace}[2][]{(#2,\tau_{#1})}
\newcommand{\measurespace}[2][]{(#2,\varSigma_{#1},\mu_{#1})}
\newcommand{\measurablespace}[2][]{(#2,\varSigma_{#1})}
\newcommand{\manifold}[2][]{(#2,\tau_{#1},\mathcal{A}_{#1})}
\newcommand{\tanspace}[2]{T_{#1}{#2}}
\newcommand{\cotanspace}[2]{T_{#1}^{*}{#2}}
\newcommand{\Ckspace}[3][\mathbb{R}]{C^{#2}(#3,#1)}
\newcommand{\funcspace}[2][\mathbb{R}]{\mathcal{F}(#2,#1)}
\newcommand{\smoothvecf}[1]{\mathfrak{X}(#1)}
\newcommand{\smoothonef}[1]{\mathfrak{X}^{*}(#1)}
\newcommand{\bracket}[2]{[#1,#2]}

%------------------------Book Command---------------------------%
\makeatletter
\renewcommand\@pnumwidth{1cm}
\newcounter{book}
\renewcommand\thebook{\@Roman\c@book}
\newcommand\book{%
    \if@openright
        \cleardoublepage
    \else
        \clearpage
    \fi
    \thispagestyle{plain}%
    \if@twocolumn
        \onecolumn
        \@tempswatrue
    \else
        \@tempswafalse
    \fi
    \null\vfil
    \secdef\@book\@sbook
}
\def\@book[#1]#2{%
    \refstepcounter{book}
    \addcontentsline{toc}{book}{\bookname\ \thebook:\hspace{1em}#1}
    \markboth{}{}
    {\centering
     \interlinepenalty\@M
     \normalfont
     \huge\bfseries\bookname\nobreakspace\thebook
     \par
     \vskip 20\p@
     \Huge\bfseries#2\par}%
    \@endbook}
\def\@sbook#1{%
    {\centering
     \interlinepenalty \@M
     \normalfont
     \Huge\bfseries#1\par}%
    \@endbook}
\def\@endbook{
    \vfil\newpage
        \if@twoside
            \if@openright
                \null
                \thispagestyle{empty}%
                \newpage
            \fi
        \fi
        \if@tempswa
            \twocolumn
        \fi
}
\newcommand*\l@book[2]{%
    \ifnum\c@tocdepth >-3\relax
        \addpenalty{-\@highpenalty}%
        \addvspace{2.25em\@plus\p@}%
        \setlength\@tempdima{3em}%
        \begingroup
            \parindent\z@\rightskip\@pnumwidth
            \parfillskip -\@pnumwidth
            {
                \leavevmode
                \Large\bfseries#1\hfill\hb@xt@\@pnumwidth{\hss#2}
            }
            \par
            \nobreak
            \global\@nobreaktrue
            \everypar{\global\@nobreakfalse\everypar{}}%
        \endgroup
    \fi}
\newcommand\bookname{Book}
\renewcommand{\thebook}{\texorpdfstring{\Numberstring{book}}{book}}
\providecommand*{\toclevel@book}{-2}
\makeatother
\titleformat{\part}[display]
    {\Large\bfseries}
    {\partname\nobreakspace\thepart}
    {0mm}
    {\Huge\bfseries}
\titlecontents{part}[0pt]
    {\large\bfseries}
    {\partname\ \thecontentslabel: \quad}
    {}
    {\hfill\contentspage}
\titlecontents{chapter}[0pt]
    {\bfseries}
    {\chaptername\ \thecontentslabel:\quad}
    {}
    {\hfill\contentspage}
\newglossarystyle{longpara}{%
    \setglossarystyle{long}%
    \renewenvironment{theglossary}{%
        \begin{longtable}[l]{{p{0.25\hsize}p{0.65\hsize}}}
    }{\end{longtable}}%
    \renewcommand{\glossentry}[2]{%
        \glstarget{##1}{\glossentryname{##1}}%
        &\glossentrydesc{##1}{~##2.}
        \tabularnewline%
        \tabularnewline
    }%
}
\newglossary[not-glg]{notation}{not-gls}{not-glo}{Notation}
\newcommand*{\newnotation}[4][]{%
    \newglossaryentry{#2}{type=notation, name={\textbf{#3}, },
                          text={#4}, description={#4},#1}%
}
%--------------------------LENGTHS------------------------------%
% Spacings for the Table of Contents.
\addtolength{\cftsecnumwidth}{1ex}
\addtolength{\cftsubsecindent}{1ex}
\addtolength{\cftsubsecnumwidth}{1ex}
\addtolength{\cftfignumwidth}{1ex}
\addtolength{\cfttabnumwidth}{1ex}

% Indent and paragraph spacing.
\setlength{\parindent}{0em}
\setlength{\parskip}{0em}                                                           %
\makeindex[intoc]                                                              %
%----------------------------Main Document-------------------------------------%
\begin{document}
    \pagenumbering{gobble}
    \title{MATH 114 Algebraic Topology - Assignments}
    \author{Ryan Maguire}
    \date{\vspace{-5ex}}
    \maketitle
    \pagenumbering{roman}
    \pagenumbering{arabic}
    \setcounter{section}{1}
    \begin{problem}
        Construct an explicit deformation retraction of
        $\nspace\setminus\{\vector{0}\}$ to $\nsphere[n-1]$.
    \end{problem}
    \begin{solution}
        Let $X=\nspace\setminus\{\vector{0}\}$ and define $r:X\rightarrow{X}$
        by:
        \begin{equation}
            r(\vector{x})=\frac{\vector{x}}{\norm{\vector{x}}_{2}}
        \end{equation}
        This is well defined since for all $\vector{x}\in{X}$ we have
        $\vector{x}\ne\vector{0}$ and hence $\norm{\vector{x}}_{2}\ne{0}$.
        Moreover, it is a retract of $X$ to $\nsphere[n-1]$. For if
        $\vector{s}\in\nsphere[n-1]$ then by definition
        $\norm{\vector{s}}_{2}=1$. But then:
        \begin{equation}
            r(\vector{s})=\frac{\vector{s}}{\norm{\vector{s}}_{2}}
                =\vector{s}
                =\identity{\nsphere[n-1]}(\vector{s})
        \end{equation}
        Hence $r|_{\nsphere[n-1]}=\identity{\nsphere[n-1]}$. Moreover, the
        image of $r$ is $\nsphere[n-1]$ since:
        \begin{equation}
            \norm{r(\vector{x})}_{2}
                =\norm[\bigg]{\frac{\vector{x}}{\norm{\vector{x}}_{2}}}_{2}
                =\frac{\norm{\vector{x}}_{2}}{\norm{\vector{x}}_{2}}
                =1
        \end{equation}
        Therefore $r$ satisfies the criterion of a retract. Let
        $H:X\times{I}\rightarrow{X}$ be the straight-line homotopy:
        \begin{equation}
            H(\vector{x},\,t)
                =(1-t)\cdot\identity{X}(\vector{x})+t\cdot{r}(\vector{x})
        \end{equation}
        This is well defined since for all $\vector{x}\in{X}$ and $t\in[0,1]$
        we have:
        \begin{subequations}
            \begin{align}
                \norm{H(\vector{x},\,t)}_{2}&=
                \norm{(1-t)\cdot\vector{x}+t\cdot{r}(\vector{x})}_{2}\\
                &\geq\min\{\norm{\vector{x}}_{2},\,\norm{r(\vector{x})}_{2}\}\\
                &>0
            \end{align}
        \end{subequations}
        Intuitively, $H(\vector{x},t)$ is the straight line from
        $\vector{x}$ to $r(\vector{x})$ and this never crosses the origin.
        Hence, $H(\vector{x},t)$ is well defined and moreover is a homotopy
        between the identity map and $r$. It is a deformation retraction since
        for all $\vector{s}\in\nspace[n-1]$ and $t\in[0,1]$ we have:
        \begin{equation}
            H(\vector{s},\,t)=(1-t)\cdot{t}+t\cdot\vector{s}=\vector{s}
        \end{equation}
    \end{solution}
    \begin{problem}
        A deformation retraction in the weak sense of a space $X$ to a subspace
        $A$ is a homotopy $H:X\times{I}\rightarrow{X}$ between $\identity{X}$
        and a function $g:X\rightarrow{X}$ such that $g[X]\subseteq{A}$ and
        such that for all $t\in[0,1]$ and $a\in{A}$, $H(a,t)\in{A}$. Show that
        if $X$ deformation retracts to $A$ in the weak sense, then the
        inclusion mapping $\iota:A\rightarrow{X}$ is a homotopy equivalence.
    \end{problem}
    \begin{solution}
        We need to find a homotopy inverse $g:X\rightarrow{A}$ of
        $\iota:A\rightarrow{X}$. That is, a function $g$ such that
        $g\circ\iota$ is homotopic to $\identity{A}$ and $\iota\circ{g}$ is
        homotopic to $\identity{X}$. Let $g:X\rightarrow{A}$ be the function
        the homotopy $H$ drags $\identity{X}$ to. That is, $g(x)=H(x,1)$.
        Then $(g\circ\iota)(a)=g(a)$ for all $a\in{A}$. But then
        $H|_{A\times{I}}$ is a homotopy between $g\circ\iota$ and
        $\identity{A}$. That is, $H|_{A\times{I}}:A\times{I}\rightarrow{A}$ is
        a continuous function, the image is in $A$ since by hypothesis for all
        $a\in{A}$ and $t\in{I}$, $H(a,t)\in{A}$, and lastly:
        \begin{equation}
            H(a,\,0)=\identity{X}(a)=a=\identity{A}(a)
        \end{equation}
        and also $H(a,\,1)=g(a)$. Similarly, since the image of $g$ lies in $A$,
        $\iota\circ{g}=g$ and hence $H$ is a homotopy between
        $\iota\circ{g}$ and $\identity{X}$. Thus, $g$ is a homotopy inverse of
        $\iota$ and hence $\iota$ is a homotopy equivalence.
    \end{solution}
    Before the next problem, we prove the following theorem to make life easier.
    \begin{theorem}
        If $\topspace{X}$ is a topological space, then it is
        contractible if and only if $\identity{X}$ is nullhomotopic.
    \end{theorem}
    \begin{proof}
        If $\topspace{X}$ is contractible, then there is a homotopy
        equivalence $f:X\rightarrow{Y}$ where $Y=\{0\}$ is the
        one point space. But if $f$ is a homotopy equivalence, then
        there is a homotopy inverse $g:Y\rightarrow{X}$. But then
        $g\circ{f}$ is homotopic to $\identity{X}$. But $Y$ has only
        one point, and hence $f(x)=0$ for all $x\in{X}$. Let
        $x_{0}=g(0)$. Then $g\circ{f}:X\rightarrow{X}$ is the
        mapping $x\mapsto{x}_{0}$. But $\identity{X}$ is homotopic
        to this, and is therefore nullhomotopic. In the other
        direction, if $\identity{X}$ is nullhomotopic, then there is
        a point $x_{0}\in{X}$ such that $\identity{X}$ is homotopic
        to the function $f:X\rightarrow{X}$ defined by $f(x)=x_{0}$.
        Let $Y=\{x_{0}\}$ be given the subspace topology. Since
        there is only one topology on a space with one point, this
        is homeomorphic to the one point space. But then $f$ is a
        homotopy equivalence since the function $g:Y\rightarrow{X}$
        given by $g(x_{0})=x_{0}$ is a homotopy inverse of $f$. That
        is, since $\identity{X}$ is homotopic to $f$,
        $g\circ{f}$ is homotopic to $\identity{X}$ since
        $g\circ{f}=f$. But also $f\circ{g}=\identity{Y}$. Hence $f$
        is an homotopy equivalence and $\topspace{X}$ is
        contractible.
    \end{proof}
    \begin{problem}
        Show that a retract of a contracible space is contracible.
    \end{problem}
    \begin{solution}
        For $\topspace{X}$ is contractible if and only if
        $\identity{X}$ is nullhomotopic. That is, there is a point
        $x_{0}\in{X}$ such that $\identity{X}$ is homotopic to the
        function $g:X\rightarrow{X}$ defined by $g(x)=x_{0}$. Let $H$ be such a
        homotopy. But if $f:X\rightarrow{A}$ is a retract, then
        $f|_{A}=\identity{A}$. Hence $f\circ{H}|_{A\times{I}}$ is a
        homotopy between $\identity{A}$ and $f(x_{0})$. That is,
        $f\circ{H}|_{A\times{I}}$ is the composition of continuous functions and
        is therefore continuous. Moreover,
        \begin{subequations}
            \begin{align}
                (f\circ{H}|_{A\times{I}})(a,0)=f\big(H|_{A\times{I}}(a,0)\big)
                    =f\big(\identity{X}(a)\big)
                    =f(a)
                    =a
            \end{align}
        \end{subequations}
        But also:
        \begin{equation}
            (f\circ{H}|_{A\times{I}})(a,1)=f\big(H|_{A\times{I}}(a,1)\big)
                =f\big(g(a)\big)
                =f(x_{0})
        \end{equation}
        Therefore $\identity{A}$ is nullhomotopic and $A$ is contracible.
    \end{solution}
    \begin{problem}
        Show that every mapping cylinder for any continuous function
        $f:\nsphere[1]\rightarrow\nsphere[1]$ is a CW complex. Construct a
        CW complex that has the annulus and M\"{o}bius band as deformation
        retracts.
    \end{problem}
    \begin{solution}
        We start with two points in our 0-cell and then attach three 1-cells.
        We wrap two copies of $(0,1)$ around in circles about the two points,
        and the final one connects the two points. Hence we have two copies of
        $\nsphere[1]$ and one copy of $[0,1]$. From here we glue the square
        $[0,1]\times[0,1]$ into this by mapping $(x,1)$ to $\exp(2\pi{i}x)$ in
        the top circle, $(0,y)$ and $(1,y)$ to the closed interval part of our
        $X^{1}$ skeleton. Lastly we attach $(x,0)$ to $f(\exp(2\pi{i}x))$ in
        the bottom circle. The result is a mapping cylinder for $f$ consisting
        of two 0-cells, three 1-cells, and one 2-cell.
        \par\hfill\par
        As noted in Hatcher's text, both the annulus and the M\"{o}bius strip
        can be deformation retracted to a circle by means of a mapping cylinder.
        We use the above assertion to construct a CW complex that contains the
        annulus and a M\"{o}bius strip as retracts
        (Fig.~\ref{fig:CW_Comp_Annulus_Mobius}). We start with an actual
        cylinder obtained from the mapping cylinder of the function
        $\identity{\nsphere[1]}$. The result is homeomorphic to an annulus. We
        then use the M\"{o}bius band function
        $f:\nsphere[1]\rightarrow\nsphere[1]$ to obtain the M\"{o}bius strip as
        the mapping cylinder $M_{f}$. We glue the bottom of the mapping cylinder
        of the identity map to the top of the mapping cylinder of the M\"{o}bius
        strip. The result is a CW complex with three 0-cells, five 1-cells, and
        two 2-cells. To deformation retract to the annulus we first use the
        mapping cylinder of the M\"{o}bius band and squeeze this down, leaving
        the annulus untouched. Easier to visualize, to obtain the M\"{o}bius
        band we simply push the cylinder part of
        Fig.~\ref{fig:CW_Comp_Annulus_Mobius} down, leaving the M\"{o}bius band
        untouched.
    \end{solution}
    \begin{figure}[H]
        \centering
        \captionsetup{type=figure}
        \includegraphics{images/Mobius_Strip_Annulus_Def_Retract.pdf}
        \caption{CW Complex Containing an Annulus and a M\"{o}bius Band}
        \label{fig:CW_Comp_Annulus_Mobius}
    \end{figure}
    \begin{problem}
        Show that a homotopy equivalence $f:X\rightarrow{Y}$ induces a bijection
        between the set of path-components of $X$ and the set of path-components
        of $Y$. Show that $f$ restricts to a homotopy equivalence between each
        path-component. Prove this statement for components.
    \end{problem}
    \begin{solution}
        For let $f:X\rightarrow{Y}$ be a homotopy equivalence and
        $\mathcal{U}\subseteq{X}$ a path connected component. Let
        $x\in\mathcal{U}$ be arbitrary, let $y=f(x)$, and let $\mathcal{V}$ be
        the path connected component of $Y$ containing $y$. Let
        $\mathcal{U}\mapsto_{f}\mathcal{V}$. We must show this is well-defined
        and bijective. Since the continuous image of a path connected space is
        path connected, and since homotopy equivalences are continuous, for any
        point $p\in\mathcal{U}$ we have $f(p)\in\mathcal{V}$. That is, since
        $\mathcal{U}$ is hypothesized to be path connected, and since $x$ and
        $p$ are contained in $\mathcal{U}$, there is a path between $p$ and $x$.
        Composing this path with $f$ results in a path between $f(x)$ and
        $f(p)$. Since $\mathcal{V}$ is a path connected component we may
        conclude $f(p)\in\mathcal{V}$. Hence $\mathcal{U}$ maps to $\mathcal{V}$
        regardless of representative, and therefore this map is well defined.
        Moreover, it is injective. If
        $\mathcal{U}_{1}$ and $\mathcal{U}_{2}$ map to the same $\mathcal{V}$
        then $\mathcal{U}_{1}=\mathcal{U}_{2}$. For if $f$ is a homotopy
        equivalence, then there is a homotopy inverse $g:Y\rightarrow{X}$ such
        that $g\circ{f}$ is homotopic to $\identity{X}$. Let
        $x_{1}\in\mathcal{U}_{1}$ and $x_{2}\in\mathcal{U}_{2}$ and define
        $y_{1}=f(x_{1})$ and $y_{2}=f(x_{2})$. Since $y_{1},y_{2}\in\mathcal{V}$
        and $\mathcal{V}$ is path connected, there is a path $\gamma$ between
        $y_{1}$ and $y_{2}$. Let $\Gamma=g\circ\gamma$. Since $g$ is continuous,
        this is a path in $X$. Moreover, it is a path between $g(y_{1})$ and
        $g(y_{2})$. But $g\circ{f}$ is homotopic to $\identity{X}$
        and hence there is a homotopy $H_{X}:X\times{I}\rightarrow{X}$ such that
        $H_{X}(x,0)=(g\circ{f})(x)$ and $H_{X}(x,1)=x$. But then
        $H_{X}(x_{1},t)$ is a path between $g(y_{1})$ and $x_{1}$ and
        $H_{X}(x_{2},t)$ is a path between $g(y_{2})$ and $x_{2}$. But there is
        a path between $g(y_{1})$ and $g(y_{2})$, namely $\Gamma$, and hence by
        concatenating there is a path between $x_{1}$ and $x_{2}$. Thus,
        $x_{1}$ and $x_{2}$ reside in the same path connected component and
        hence $\mathcal{U}_{1}=\mathcal{U}_{2}$. By an identical argument,
        $\mathcal{V}\mapsto_{g}\mathcal{U}$ is an injective function from the
        path connected components of $Y$ to the path connected components of
        $X$. But then there exist injective functions in both directions, so by
        the Cantor-Schr\"{o}eder-Bernstein theorem there is a bijection. That
        is, there exists a bijection between the path connected components
        of $X$ and the path connected components of $Y$. Now, given a path
        component $\mathcal{U}\subseteq{X}$, $f|_{\mathcal{U}}$ is a homotopy
        equivalence to it's corresponding path connected component
        $\mathcal{V}\subseteq{Y}$. Consider $g|_{\mathcal{V}}$. Given the
        homotopy $H_{X}:X\times{I}\rightarrow{X}$, let $G_{X}$ be it's
        restriction to $\mathcal{U}\times{I}$. Then $G$ is a homotopy
        equivalence between $g|_{\mathcal{V}}\circ{f}|_{\mathcal{U}}$ and
        $\identity{\mathcal{U}}$. Firstly, the image of $G$ resides in
        $\mathcal{U}$ since for all $x\in\mathcal{U}$, $G(x,t)$ is a path
        between $H(x,0)=g(f(x))$ and $H(x,1)=x$, both of which reside in
        $\mathcal{U}$ and since $\mathcal{U}$ is path connected this implies the
        entire image of $G$ resides in $\mathcal{U}$. But then
        $G(x,0)=(g|_{\mathcal{V}}\circ{f}|_{\mathcal{U}})(x)$ and $G(x,1)=x$.
        Hence $G$ is a homotopy between $g|_{\mathcal{V}}\circ{f}_{\mathcal{U}}$
        and $\identity{\mathcal{U}}$. Similarly, since $g$ is a homotopy inverse
        of $g$, $f\circ{g}$ is homotopic to $\identity{Y}$. Let
        $H_{Y}:Y\times{I}\rightarrow{Y}$ be such a homotopy and
        $G_{Y}=H_{Y}|_{\mathcal{V}\times{I}}$. By an identical argument $G_{Y}$
        is a homotopy between $f|_{\mathcal{U}}\circ{g}|_{\mathcal{V}}$ and
        $\identity{\mathcal{V}}$. Hence $g|_{\mathcal{V}}$ is a homotopy inverse
        of $f|_{\mathcal{U}}$, so $f|_{\mathcal{U}}$ is a homotopy equivalence.
        For connected components, let $\mathcal{U}\mapsto_{f}\mathcal{V}$ be a
        similar mapping. Given $x\in\mathcal{U}$, let $\mathcal{V}\subseteq{Y}$
        be the connected component containing $f(x)$. We must show this is well
        defined. But the continuous image of connected is still connected, and
        hence if $p\in\mathcal{U}$, then $f(p)\in\mathcal{V}$ and therefore
        $\mathcal{U}$ maps unambiguously to $\mathcal{V}$. We must show this
        mapping is injective. Let $\mathcal{U}_{1},\mathcal{U}_{2}\subseteq{X}$
        be connected components and suppose both map to $\mathcal{V}$. But since
        $f$ is a homotopy equivalence there is a homotopy inverse
        $g:Y\rightarrow{X}$ such that $g\circ{f}$ is homotopic to $\identity{X}$
        and $f\circ{g}$ is homotopic to $\identity{Y}$. But then there is a path
        between $x_{1}$ and $(g\circ{f})(x_{1})$ and similarly for $x_{2}$. But
        path connected components are contained within connected components, and
        hence these paths reside in $\mathcal{U}_{1}$ and $\mathcal{U}_{2}$,
        respectively. But $\mathcal{V}$ is connected and $g$ is continuous, and
        therefore $g[\mathcal{V}]$ is connected. Since
        $g(f(x_{1}))\in\mathcal{U}_{1}$ we conclude
        $g[\mathcal{V}]\subseteq\mathcal{U}_{1}$. Similarly, since
        $g(f(x_{2}))\in\mathcal{U}_{2}$, we conclude
        $g[\mathcal{V}]\subseteq\mathcal{U}_{2}$. But then $\mathcal{U}_{1}$ and
        $\mathcal{U}_{2}$ have non-empty intersection. Since they are connected
        components we thus have $\mathcal{U}_{1}=\mathcal{U}_{2}$. Similarly,
        since $f$ is a homotopy inverse of $g$, the mapping
        $\mathcal{V}\mapsto_{g}\mathcal{U}$ unambiguously and injectively maps 
        connected components of $Y$ to connected components of $X$. Invoking
        Cantor-Schr\"{o}eder-Bernstein we thus obtain a bijection. Given
        a connected component $\mathcal{U}\subseteq{X}$, $f|_{\mathcal{U}}$ is a
        homotopy equivalence into the corresponding $\mathcal{V}\subseteq{Y}$
        since $g|_{\mathcal{V}}$ is a homotopy inverse. For $g\circ{f}$ is
        homotopy equivalent to $\identity{X}$ and hence there is a homotopy
        $H_{X}:X\times{I}\rightarrow{X}$ such that
        $H(x,0)=(g\circ{f})(x)$ and $H(x,1)=x$. But then
        $G_{X}=H_{X}|_{\mathcal{U}\times{I}}$ is a homotopy between
        $g|_{\mathcal{V}}\circ{f}|_{\mathcal{U}}$ and $\identity{\mathcal{U}}$.
        Firstly, the image of $G_{X}$ resides in $\mathcal{U}$. For any
        $x\in\mathcal{U}$, $G_{X}(x,t)$ is a path between
        $(g\circ{f})(x)$ and $x$. Since $\mathcal{U}$ is connected, the entirety
        of this path lies in $\mathcal{U}$. But
        $G_{X}(x,0)=H_{X}(x,0)=(g|_{\mathcal{V}}\circ{f}|_{\mathcal{U}})(x)$
        and $G_{X}(x,1)=H_{X}(x,1)=x$. That is, $G_{X}$ is a homotopy between
        $g|_{\mathcal{V}}\circ{f}|_{\mathcal{U}}$ and $\identity{\mathcal{U}}$.
        By a similar argument, $f|_{\mathcal{U}}\circ{g}|_{\mathcal{V}}$ is
        homotopic to $\identity{\mathcal{V}}$. If $X$ is such that the path
        connected components coincide with the connected components, and
        $f:X\rightarrow{Y}$ is a homotopy equivalence, then the same is true for
        $Y$. Given a connected component $\mathcal{V}\subseteq{Y}$, let
        $\mathcal{U}$ be the corresponding connected component in $X$. But by
        hypothesis $\mathcal{U}$ is path connected. Let
        $y_{1},y_{2}\in\mathcal{V}$, $x_{1}=g(y_{1})$, and $x_{2}=g(y_{2})$.
        Since $X$ is path connected, there is path $\gamma$ between $x_{1}$ and
        $x_{2}$. But then $f\circ\gamma$ is a path between $f(x_{1})$ and
        $f(x_{2})$. But $f\circ{g}$ is homotopic to $\identity{Y}$, so let
        $H_{Y}$ be such a homotopy. But then $H_{Y}(y_{1},t)$ is a path between
        $y_{1}$ and $f(x_{1})$, and similarly for $y_{2}$. But $f\circ\gamma$
        is a path from $f(x_{1})$ to $f(x_{2})$. By concatenating paths we
        obtain a path from $y_{1}$ to $y_{2}$. Hence, $\mathcal{V}$ is path
        connected and the connected components of $Y$ coincide with the path
        connected ones.
    \end{solution}
    \begin{problem}
        Show that $\nsphere[m]*\nsphere$ is homeomorphic to $\nsphere[m+n+1]$.
    \end{problem}
    \begin{solution}
        Define
        $f:\nsphere[m]\times\nsphere[n]\rightarrow{I}\rightarrow\nsphere[n+m+1]$
        as follows:
        \begin{equation}
            f\big(\vector{a},\vector{b},t\big)
            =\big(\sqrt{1-t}\cdot\vector{a},\sqrt{t}\cdot\vector{b}\big)
        \end{equation}
        The image of $f$ lies in $\nsphere[n+m+1]$ since if
        $\vector{a}\in\nsphere[m]$ and $\vector{b}\in\nsphere[m]$, then
        $\norm{\vector{a}}=\norm{\vector{b}}=1$. From orthogonality of the
        components we then obtain:
        \begin{equation}
            \norm{f(\vector{a},\vector{b},t)}^{2}
                =(1-t)\norm{\vector{a}}^{2}+t\norm{\vector{b}}^{2}=1
        \end{equation}
        Then $f$ is surjective. We then pass $f$ to the quotient obtaining
        $\tilde{f}$. This is well defined since the following equation:
        \begin{equation}
            f(\vector{a}_{1},\vector{b}_{1},t_{1})
                =f(\vector{a}_{2},\vector{b}_{2},t_{2})
        \end{equation}
        is true if and only if both $t_{1}$ and $t_{2}$ are either zero or one
        and the corresponding $\vector{a}_{i}$ match the $\vector{b}_{j}$, or if
        all of the components match. But in this former instance the points are
        identified by the equivalence relation and hence belong to the same
        equivalence class. Hence the function obtained from
        passing to the quotient is well defined, and moreover this argument
        shows it is bijective. That is, $f$ is surjective, and hence so is
        $\tilde{f}$, and from the previous argument $\tilde{f}$ is injective.
        Since $f$ is continuous, so is $\tilde{f}$, and hence we have a
        continuous bijection from $\nsphere[m]*\nsphere$ into $\nsphere[n+m+1]$.
        But the join of compact sets is the quotient of the product of compact
        sets, which is thus the quotient of a compact space, and is therefore
        compact. Also, $\nsphere[n+m+1]$ is Hausdorff, so $\tilde{f}$ is a
        continuous bijection from a compact space to a Hausdorff one, and is
        therefore a homeomorphism.
    \end{solution}
    \begin{problem}
        Show that the space obtained from $\nsphere[2]$ by attaching any $n$
        2-cells along any collection of $n$ circles is homotopy equivalence to
        the wedge sum of $n+1$ copies of $\nsphere[2]$.
    \end{problem}
    \begin{solution}
        We prove by induction on $n$. First we note that $\nsphere[2]$ is
        simply connected. That is, any continuous function
        $f:\nsphere[1]\rightarrow\nsphere[2]$ is homotopic to a constant map.
        Since we are only dealing with circles, we need only prove this for the
        case of $f$ being injective. We use stereographic projection, and the
        fact that $\nsphere[1]$ is not homeomorphic to $\nsphere[2]$. This can
        be seen since they have different dimensions, but neglecting manifold
        theory this can also be seen since removing two points from
        $\nsphere[1]$ results in a disconnected space, whereas removing any
        finite number of points from
        $\nsphere[2]$ does not disconnect the sphere. So any continuous
        injective mapping $f:\nsphere[1]\rightarrow\nsphere[2]$ is not
        surjective, for otherwise $f$ would be a homeomorphism. That is, a
        continuous bijection from a compact space to a Hausdorff one is a
        homeomorphism, $\nsphere[1]$ is compact, and $\nsphere[2]$ is Hausdorff.
        So if $\nsphere[1]$ maps into $\nsphere[2]$ injectively and
        continuously, then the mapping is not surjective. Hence
        $\nsphere[2]\setminus{f}[\nsphere[1]]$ is non-empty. Moreover, this
        subset is open since $f[\nsphere[1]]$ is compact, and hence closed.
        Given a point in the complement we may find a small open neighborhood of
        this point such that $f[\nsphere[1]]$ does not intersect its closure.
        Removing this open subset results in a space that is homeomorphic to
        $\nspace[2]$, which is contractible. Hence any continuous function is
        homotopic to a constant. Pulling back all of this to the sphere shows
        that $\nsphere[2]$ is simply connected. Hence if we attach a 2-cell
        along a circle within the sphere, we may contract this circle to a point
        and obtain a sphere with a 2-cell attached to a point on the sphere.
        But this is precisely the wedge of $\nsphere[2]$ with $\nsphere[2]$.
        That is, our original space is homotopy equivalent to
        $\nsphere[2]\lor\nsphere[2]$. Suppose this is true of $n$ circles.
        Since the wedge of path connected spaces is equivalent up to homotopy
        regardless of choice of points, we may choose the wedge of $n$ spheres
        to all be at the same point. Given a circle in the original sphere and a
        2-cell attached to this we may then contract the circle down to our
        common point, obtaining another sphere. The result is the wedge of $n+1$
        spheres.
    \end{solution}
    \begin{problem}
        Show that if $(X,A)$ has the homotopy extension property, then
        $X\times{I}$ deformation retracts to $X\times\{0\}\cup{A}\times{I}$.
    \end{problem}
    \begin{solution}
        For if $(X,A)$ satisfies the homotopy extension property, then there is
        a retract of $X\times{I}$ to $X\times\{0\}\cup{A}\times{I}$. But then
        the inclusion map
        $\iota:X\times\{0\}\cup{A}\times{I}\rightarrow{X}\times{I}$ is a
        homotopy equivalence. For let $r$ be the retract. Then
        $r\circ\iota$ is just the identity map on $X\times\{0\}\cup{A}\times{I}$
        since $r$ is a retract, and the identity is homotopic to itself. But
        since $\iota\circ{r}$ is just the restriction of $r$ to
        $X\times\{0\}\cup{A}\times{I}$ we may use the homotopy extension
        property to extend $f_{0}=\identity{X}$ and $f_{1}=r$ to a homotopy,
        showing that $\iota\circ{r}$ is homotopic to the identity map on $X$.
        Hence, the inclusion map is a homotopy equivalence. Then by the
        corollary there is a deformation retraction of $X\times{I}$ onto
        $X\times\{0\}\cup{A}\times{I}$. For the generalization of 0.18, let
        $H:A\times{I}\rightarrow{X}_{0}$ between a homotopy from $f$ to $g$, and
        consider the attaching space $X_{0}\coprod_{H}(X_{1}\times{I})$. But
        since $(X_{1},A)$ has the homotopy extension property, there is a
        deformation retraction of $X_{1}\times{I}$ to
        $X_{1}\times\{0\}\cup{A}\times{I}$, and this induces a deformation
        retraction of $X_{0}\coprod_{H}(X_{1}\times{I})$ to
        $X_{0}\coprod_{f}X_{1}$ and similarly for $g$. Since both of these
        deformation retractions are the identity on $X_{0}$, we see that
        $X_{0}\coprod_{f}X_{1}$ is homotopy equivalent to
        $X_{0}\coprod_{g}X_{1}$ relative to $X_{0}$.
    \end{solution}
    \section{Wrong problems... Woops}
    \begin{problem}
        Show that the free product of non-trivial groups $G$ and $H$ has trivial
        center and that the only elements of finite order are the conjugates
        of finite order elements in $G$ and $H$.
    \end{problem}
    \begin{solution}
        For let $C\subseteq{G}*H$ be the center and suppose $C$ is non-trivial.
        Then there is an element $a\in{C}$ that is not the identity. That is,
        $a$ is not the empty word. Suppose $a$ is in reduced form and let
        $b_{1}$ be the first element and $b_{2}$ be the last,
        $a=b_{1}\cdots{b}_{2}$. If
        $b_{1},b_{2}\in{G}$, let $d\in{H}$ be an element other than the
        identity. Since $H$ is hypothesized to be non-trivial, such an element
        exists. But then $db_{1}$ cannot be reduced since $d\in{H}$ and
        $b_{1}\in{G}$. Similarly, $b_{2}d$ cannot be reduced. But
        $hb_{1}\cdots{b}_{2}$ and $b_{1}\cdots{b}_{2}h$ are distinct words, and
        thus $a$ is not in the center. For similar reasons, it is not possible
        that $b_{1},b_{2}\in{H}$. Thus, suppose $b_{1}\in{G}$ and $b_{2}\in{H}$
        and let $d=b_{1}$. Then $b_{2}d$ cannot be reduced, and $db_{1}$ may or
        may not be able to be reduced. If it cannot, then $da\ne{ad}$. If it can
        then we note that since $a$ is in reduced form it is an alternating
        sequence of elements in $G$ and $H$. Hence, after reducing $db_{1}$ we
        would have $db_{1}h_{1}g_{1}\cdots{b}_{2}$, which cannot be reduced
        further. Moreover, this is not equal to $ad$. Hence, the center is
        trivial. Next, given a word that is a conjugate of elements of finite
        order, it too has finite order. Since
        $(g^{\minus{1}}hg)^{n}=g^{\minus{1}}h^{n}g$, given the order $n$ of the
        element $h$, we would have $(g^{\minus{1}}hg)^{n}=g^{\minus{1}}h^{n}g$
        which simplifies to $g^{\minus{1}}e_{h}g=g^{\minus{1}}g=e_{g}$,
        which is equivalent to the empty word.
    \end{solution}
    \begin{problem}
        Show that $\nspace$ minus finitely many points is simply connected for
        $n\geq{3}$.
    \end{problem}
    \begin{solution}
        For $n\geq{2}$, $\nspace$ with finitely many points removed is a
        connected open subset of $\nspace$. Since $\nspace$ is locally path
        connected, this implies that $\nspace$ without finitely many points is
        path connected for $n\geq{2}$. Suppose first that we remove a single
        point from $\nspace$, $n\geq{3}$. Up to homeomorphism we may as well
        choose the origin, so let $X=\nspace\setminus\{\vector{0}\}$. Since
        $\gamma:\nsphere[1]\rightarrow{X}$ is a  continuous function from a
        compact set, its image is compact. But then there is a closed
        $\varepsilon>0$ ball around $\vector{0}$ that is disjoint with the image
        of $\gamma$. Using this we may project the image of $\gamma$ onto the
        $n-1$ sphere $\nsphere[n-1]$ using the following homotopy:
        \begin{equation}
            H(\theta,t)=(1-t)\cdot\gamma(\theta)
                +t\cdot\frac{\gamma(\theta)}{\norm{\gamma(\theta)}}
        \end{equation}
        Hence $\gamma$ is homotopic to a loop on $\nsphere[n-1]$. But for
        $k\geq{2}$ $\nsphere[k]$ is simply connected, and hence for
        $n\geq{3}$ we have that $\nsphere[n-1]$ is simply connected. Hence
        this loop is homotopic to a point on $\nsphere[n-1]$. Since
        $X$ is path connected, for any point $\vector{x}_{0}\in{X}$ the
        fundamental groups are isomorphic. This shows that, regardless of base
        point, the fundamental group is trivial. Therefore,
        $\nspace\setminus\{\vector{0}\}$ is simply connected. Now, for $m$
        points removed we may assume they are located at the coordinates
        $(k,0,\dots,0)$ for $k=0,\dots,m-1$. Placing spheres of radius
        $\frac{1}{2}$ around each of these we note that this is just the
        wedge of $\nsphere[n-1]$ $m$ times. Moreover, $\nspace$ with these $m$
        points removed deformation retracts down to this. Following a given
        loop $\gamma$ along this retraction gives us a loop on the wedge of
        $m$ spheres. We may deform the part of the loop in the last sphere
        (the sphere centered at $(m,0,\dots,0)$) to the point of intersection
        with the $m-1$ sphere since $\nsphere[n-1]$ is simply connected for
        $n\geq{3}$. But then the last point no longer affects our loop, and we
        essentially have a loop on the wedge of $m-1$ spheres. By induction,
        this is simply connected.
    \end{solution}
\section{Correct Problems}
    \begin{problem}
        Show that the composition of paths satisfies the following cancellation
        property. If $f_{0}\cdot{g}_{0}$ is homotopic to $f_{1}\cdot{g}_{1}$,
        and if $g_{0}$ is homotopic to $g_{1}$, then $f_{0}$ is homotopic to
        $f_{1}$.
    \end{problem}
    \begin{solution}
        This is true of any group. Given $a,b,c\in{G}$, $a*c=b*c$, we may
        multiply on the right by $c^{\minus{1}}$ obtaining
        $(a*c)*c^{\minus{1}}=(b*c)*c^{\minus{1}}$. Using associativity we have
        $a*(c*c^{\minus{1}})=a*e=a$ and similarly $b*(c*c^{\minus{1}})=b*e=b$.
        Since the equivalence class of paths is associative under concatenation
        we may adapt this argument. Given that $g_{0}$ is homotopic to $g_{1}$,
        there is a homotopy $H$. But then $H$  also gives us a homotopy between
        $g_{0}\cdot{g}_{0}^{\minus{1}}$ and $g_{1}\cdot{g}_{1}^{\minus{1}}$.
        Explicitly, let $G$ be defined by:
        \begin{equation}
            G(x,t)=
            \begin{cases}
                H(x,2t),&0\leq{t}\leq\frac{1}{2}\\
                H(x,2-2t),&\frac{1}{2}\leq{t}\leq{1}
            \end{cases}
        \end{equation}
        Since homotopic is transitive, we have
        $f_{0}\cdot(g_{0}\cdot{g_{0}}^{\minus{1}})$ is homotopic to
        $f_{1}\cdot(g_{1}\cdot{g}_{1}^{\minus{1}})$. But both of these
        concatenations on the right are homotopic to a point, and hence
        $f_{0}$ is homotopic to $f_{1}$.
    \end{solution}
    \begin{problem}
        Show that the change of basis homomorphism $\beta_{h}$ depends only on
        the homotopy class of $h$.
    \end{problem}
    \begin{solution}
        For suppose $g$ and $h$ are paths from $x_{0}$ and $x_{1}$ and that $g$
        is homotopic to $h$. Let $H$ be such a homotopy, $H(t,0)=g(t)$ and
        $H(t,1)=h(t)$. Then for any loop $f$ at $x_{1}$ we have that $g\cdot{f}$
        is homotopic to $h\cdot{f}$. Indeed, let $G(t)$ be defined by $H(2t)$
        for $t\in[0,\frac{1}{2}]$ and $f(t)$ otherwise. This is a continuous
        homotopy between $h\cdot{f}$ and
        $g\cdot{f}$. Similarly, $f\cdot\overline{h}$ and $f\cdot\overline{g}$
        are homotopic, further implying that $h\cdot{f}\cdot\overline{h}$ and
        $g\cdot{f}\cdot\overline{g}$ are homotopic. Hence:
        \begin{equation}
            \beta_{h}([f])=[h\cdot{f}\cdot\overline{h}]
                =[g\cdot{f}\cdot\overline{g}]=\beta_{g}([f])
        \end{equation}
    \end{solution}
    \begin{problem}
        Show that $\pi_{1}(X)$ is Abelian for a path connected $X$ if and only
        if every change of basepoint homomorphism $\beta_{h}$ depends only on
        the endpoint.
    \end{problem}
    \begin{solution}
        For let $x,y$ be points and $\gamma_{1},\gamma_{2}$ paths taking $x$ to
        $y$. Since $\pi_{1}(X,x)$ is Abelian, so is $\pi_{1}(X,y)$. But then
        given any loop $f$ at $y$,
        $\gamma_{1}\cdot{f}\cdot\overline{\gamma}_{1}$ is a loop at $x$ and
        similarly for $\gamma_{2}$. Moreover,
        $\gamma_{1}\cdot\overline{\gamma}_{2}$ is a loop at $x$. Thus:
        \begin{subequations}
            \begin{align}
                \beta_{\gamma_{1}}([f])
                &=[\gamma_{1}\cdot{f}\cdot\overline{\gamma}_{1}]\\
                &=[\gamma_{1}\cdot\overline{\gamma}_{2}
                    \cdot\gamma_{2}\cdot{f}\cdot\overline{\gamma}_{1}]\\
                &=[\gamma_{1}\cdot\overline{\gamma}_{2}]\cdot
                    [\gamma_{2}\cdot{f}\cdot\overline{\gamma}_{1}]\\
                &=[\gamma_{2}\cdot{f}\cdot\overline{\gamma}_{1}]\cdot
                    [\gamma_{1}\cdot\overline{\gamma}_{2}]\\
                &=[\gamma_{2}\cdot{f}\cdot\overline{\gamma}_{1}\cdot
                    \gamma_{1}\cdot\overline{\gamma}_{2}]\\
                &=[\gamma_{2}\cdot{f}\cdot\overline{\gamma}_{2}]\\
                &=\beta_{\gamma_{2}}([f])
            \end{align}
        \end{subequations}
        Hence $\beta$ only depends on the endpoints. Now, if $\beta$ only
        depends on the endpoint, then $\pi_{1}(X,x)$ is Abelian. Let
        $f,g$ be loops at $x$, and let $\gamma$ be the constant loop at $x$.
        But then:
        \begin{subequations}
            \begin{align}
                \beta_{f}([g])&=[f\cdot{g}\cdot\overline{f}]\\
                &=\beta_{\gamma}([g])\\
                &=[g]
            \end{align}
        \end{subequations}
        Hence $f\cdot{g}\cdot\overline{f}$ is homotopic to $g$. By concatenating
        both sides on the right by $f$ we have $[f]\cdot[g]=[g]\cdot[f]$ showing
        us the fundamental group is Abelian.
    \end{solution}
    \begin{problem}
        Show that the following are equivalent:
        \begin{enumerate}
            \item Every map $\nsphere[1]\rightarrow{X}$ is homotopic to a
                  constant map.
            \item Every map $\nsphere[1]\rightarrow{X}$ can be extended to a
                  continuous function $D^{2}\rightarrow{X}$.
            \item $\pi_{1}(X,x_{0})=\{e\}$ for all $x_{0}\in{X}$.
        \end{enumerate}
    \end{problem}
    \begin{solution}
        $(1\Rightarrow{2})$ If $f:\nsphere[1]\rightarrow{X}$ is homotopic to a
        constant map, then let $H:\nsphere[1]\times{I}\rightarrow{X}$ be such a
        homotopy. Then $H(x,1)=const$ for all $x\in\nsphere[1]$ so we may obtain
        a well-defined continuous map by passing $H$ to the quotient
        $(\nsphere[1]\times{I})/R$ where $R$ identifies $\nsphere[1]\times\{1\}$
        to a point. But the cylinder with it's top boundary identified is
        homeomorphic to $D^{2}$ so we may pass this function to $D^{2}$ giving
        us an extension of $f$.
        \par\hfill\par
        $(2\Rightarrow{3})$ Let $x_{0}\in{X}$ be arbitrary and let
        $f:I\rightarrow{X}$ be a loop based at $x_{0}$. We obtain a continuous
        function $\tilde{f}:\nsphere[1]\rightarrow{X}$ from $f$ since
        $f(0)=f(1)$, and since $I/R$ is homeomorphic to $\nsphere[1]$
        where $R$ is the equivalence relation identifying 0 and 1. Hence $f$ may
        be passed to the quotient. But by hypothesis any continuous function
        $\tilde{f}:\nsphere[1]\rightarrow{X}$ extends to a continuous function
        on $D^{2}$. But $D^{2}$ is contractible, and moreover there is a
        deformation retract of $D^{2}$ down to the point $(1,0)\in\nsphere[1]$
        which is the point identified with $f(0)=x_{0}$.
        But this deformation retraction gives us a homotopy of
        $f$ to the constant map $g:I\rightarrow{X}$ defined by $g(t)=x_{0}$.
        Since $x_{0}$ and $f$ are arbitrary, this shows $\pi_{1}(X,x_{0})$ is
        trivial for all $x_{0}\in{X}$.
        \par\hfill\par
        $(3\Rightarrow{1})$ This is more-or-less the definition of a trivial
        fundamental group. Let $f:\nsphere[1]\rightarrow{X}$ be a continuous
        function and let $x_{0}=f(1,0)$. Let $\tilde{f}$ be the loop defined by
        $\tilde{f}(t)=f(\exp(2\pi{i}t))$. Then $\tilde{f}$ is a loop at the
        point $x_{0}$, but $\pi_{1}(X,x_{0})$ is trivial and hence this is
        homotopic to the constant map $g:I\rightarrow{X}$ defined by
        $g(t)=x_{0}$. Lifting this homotopy to $f$ shows us that $f$ is also
        homotopic to a constant map.
        \par\hfill\par
        $X$ is simply connected if and only if all continuous maps from
        $\nsphere[1]$ into $X$ are homotopic. Firstly, this implies $X$ is
        path connected. For given $x,y\in{X}$ let $f$ be the constant map at
        $x$ and $g$ the constant map at $y$ (both having domains $\nsphere[1]$).
        By hypothesis these are homotopic so let $H$ be a homotopy. Then
        $H(0,t)$ is a path from $x$ to $y$. From the previous argument, if every
        map $\nsphere[1]\rightarrow{X}$ is homotopic to a point, then
        $\pi_{1}(X,x_{0})$ is trivial for all $x_{0}\in{X}$. Hence, $X$ is path
        connected with trivial fundamental group, and is therefore simply
        connected. In the other direction, if $X$ is simply connected then it is
        path connected and $\pi_{1}(X,x_{0})$ is trivial for all $x_{0}\in{X}$.
        Then, given any two maps $f,g:\nsphere[1]\rightarrow{X}$, let
        $x_{0}=f(0,0)$ and $x_{1}=g(0,0)$. But $\pi_{1}(X,x_{0})$ is trivial so
        $f$ is homotopic to the constant map at $x_{0}$, and similarly for
        $g$. But $X$ is path connected so the constant map at $x_{0}$ is
        homotopic to the constant map at $x_{1}$. Since homotopic is a
        transitive relation we have that $f$ is homotopic to $g$. Hence $X$ is
        simply connected if and only if every pair of continuous functions
        from $\nsphere[1]$ are homotopic.
    \end{solution}
\end{document}