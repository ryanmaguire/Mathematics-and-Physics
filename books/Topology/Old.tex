\section{Jordan Arc Theorem}
    \begin{ftheorem}{The Jordan Arc Theorem}
        If $\phi:[0,1]\rightarrow\mathbb{R}^{2}$ is a continuous injective
        function, the $\mathbb{R}^{2}\setminus\phi\big[[0,1]\big]$ is path
        connected.
    \end{ftheorem}
    \begin{bproof}
        For let $E\subseteq[0,1]$ be the set of all such values $r$ where
        $\mathbb{R}^{2}\setminus\phi\big[[0,r]\big]$ is path connected. This is
        non-empty since $0\in{E}$. But it is bounded by 1, and hence there is a
        least upper bounded $M$. Suppose $M<1$. Then for all $n\in\mathbb{N}$
        there is an $x_{n}$ and a $y_{n}$ such that there is no path between
        them in $\mathbb{R}^{2}\setminus\phi\big[[0,(1+M)/2n]\big]$. We can find
        such $x_{n}$ and $y_{n}$ inside a compact ball that contains an open
        neighborhood of the entire arc. By sequential compactness, there are
        convergent subsequences which we will simply relabel $x_{n}$ and $y_{n}$
        with limits $x$ and $y$. Connect a path from $x$ to $y$, and then
        connect $x_{n}$ to $x$ and $y_{n}$ to $y$ obtaining a path from $x_{n}$
        to $y_{n}$, a contradiction. Need to fill some steps. Injectivity
        needed, as is normality. Since a path from $x$ to $y$ will be closed,
        and $\phi\big[[0,M]\big]$ is closed, these can be separated by open sets
        and this will give us our $\varepsilon$ for the $x_{n},y_{n}$. If
        injectivity can prove that $M\in{E}$, then we're done. Need to show this
        next. New problem, the limit points $x$ and $y$ may fall in the image of
        $\phi$! To correct this, look at a nested sequence of closed sets whose
        union is the complement. This will ensure the $x$ and $y$ do not lie on
        the image of $\phi$, and the union of nested path connected sets will
        again be path connected.
        \par\hfill\par
        Ok, letting $E$ as before, first show that the least upper bound $M$ is
        contained in $E$. Suppose not. Then there are points $x,y$ with no
        path between the two in $\mathbb{R}^{2}\setminus\phi\big[[0,M]\big]$.
        Note then that $M>0$. But then for all $0<r<M$ there is a path
        connecting $x$ and $y$ in $\mathbb{R}^{2}\setminus\phi\big[[0,r]\big]$
        Let $\gamma_{1}$ be a path connecting $x$ and $y$ in
        $\mathbb{R}^{2}\setminus\phi\big[[0,M/2]\big]$. But since $\phi$ is
        injective, $\phi(M)\notin\phi\big[[0,M/2]\big]$. But from continuity
        $\phi\big[[0,M/2]\big]$ is closed, and thus by the normality of
        $\mathbb{R}^{2}$ (regularity suffices) there are disjoint open subsets
        $\mathcal{U}_{M}$ and $\mathcal{U}_{C}$ such that
        $\phi(M)\in\mathcal{U}_{M}$ and
        $\phi\big[[0,M/2]\big]\subseteq\mathcal{U}_{C}$. But then there is an
        $\varepsilon_{1}>0$ such that the $\varepsilon_{1}$ ball centered about
        $\phi(M)$ is contained in $\mathcal{U}_{M}$. Let But $f$ is continuous
        and so there is a $\delta_{1}$ such that $|M-x|<\delta_{1}$ implies
        $\norm{f(x)-f(M)}<\varepsilon_{1}$. Let
        $r_{1}=\min\{\varepsilon_{1},\varepsilon,1/4\}$. If
        $\gamma_{1}$ and $\phi\big[[0,M-r_{1}]\big]$ are disjoint, let
        $\gamma_{2}=\gamma_{1}$, otherwise let $\gamma_{2}$ be a path connecting
        $x$ to $y$ in $\mathbb{R}^{2}\setminus\phi\big[[0,M-r_{1}]\big]$.
        If $\gamma_{2}$ and the $r_{1}$ ball about $\phi(M)$ are disjoint, we
        are done since then $\gamma_{2}$ will be a path connecting
        $x$ to $y$ in $\mathbb{R}^{2}\setminus\phi\big[[0,M]\big]$, thus
        suppose not. Since $M-r_{1}<M$,
        $\mathbb{R}^{2}\setminus\phi\big[[0,M-r_{1}]\big]$ is path connected.
        Thus, let $\alpha$ be the first point when $\gamma_{2}$ enters the
        $r_{1}$ ball and let $\beta$ be the last value when it leaves
        (also choose $r_{1}$ so the $r_{1}<\norm{x-phi(M)}/4$ and
        $r_{1}<\norm{y-\phi(M)}/4$, I suppose. Let $\Gamma_{1}$ be a path
        connecting $\gamma_{1}(\alpha)$ and $\gamma_{1}(\beta)$ and define:
        \begin{equation}
            \gamma_{2}=
            \begin{cases}
                \gamma_{1}(t),&0\leq{t}\leq\alpha,\beta\leq{t}\leq{1}\\
                \Gamma_{1}(t),\alpha\leq{t}\leq\beta
            \end{cases}
        \end{equation}
        Again, there is a $\varepsilon_{2}$ ball separating $\phi(M)$ from
        $\phi\big[[0,M-r_{1}]\big]$ and a $\delta_{2}$ such that for all
        $|M-x|<\delta_{2}$, $\norm{f(x)-f(M)}<\varepsilon_{2}$. Let
        $r_{2}=\min\{\varepsilon_{2},\delta_{2},1/8\}$. Fuck it, come back
        later.
    \end{bproof}
\section{Stone's Representation Theorem}
    \subsection{Motivation}
        \begin{itemize}
            \item Set theory, algebra of sets, $\sigma$ algebras,
                  $(\mathcal{P}(A),\cap,\cup)$.
            \item Boole's axioms, redundancy of absorption and associativity.
            \item Idempotant $a*a=a$, bounded $(a*e_{\circ}=e_{\circ})$,
                  absorption.
            \item Several theorems later comes associativity.
            \item Not a field, no inverses exists other than for identities.
        \end{itemize}
    \subsection{Topology}
        \begin{itemize}
            \item Topologies, Hausdorff, compact, connected, totally
                disconnected, clopen.
            \item Stone spaces.
            \item Bases.
            \item Topology generated by a subset.
        \end{itemize}
    \subsection{Lattices}
        \begin{itemize}
            \item Partial orderings, lattices (comm, assoc, abs, idem), bounded
                  lattice, Boolean lattice.
            \item Filters and ultra filters.
            \item Stone space generated by ultra filters on a Boolean lattice.
            \item Stone's representation theorem.
        \end{itemize}
        A partially ordered set is a set $A$ with a relation $\leq$ that is
        reflexive, antisymmetric, and transitive. A filter is a subset $F$ of a
        partial ordered set $P$ such that for all $x,y\in{F}$ there is a
        $z\in{F}$ such that $z\leq{x}$ and $z\leq{y}$. That is, all $x,y\in{F}$
        have a common lower bound. Moreover, if $x\in{F}$ and $y\in{P}$ are such
        that $x\leq{y}$, then $y\in{F}$. An ultra-filter is a maximal filter.
        That is, a filter that is not properly contained in any other proper
        filter. The Stone space is the set of all ultrafilters of the Boolean
        algebra under the partial ordering $a\leq{b}$ iff $a*b=b$. The topology
        is generated from the collection of all ultrafilters that contain a
        particular point, and then union over all points.
\section{Manifold Notes}
    \subsection{Intro}
        Manifolds generalize the notions of curves and surfaces, which themselves
        may be thought of a 1 and 2 dimensional manifolds, respectively. These are
        the nicest spaces to study since most of the pathologies of point-set
        topology disappear.