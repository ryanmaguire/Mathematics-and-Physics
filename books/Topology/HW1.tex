%------------------------------------------------------------------------------%
\documentclass{article}                                                        %
%------------------------------Preamble----------------------------------------%
\makeatletter                                                                  %
    \def\input@path{{../../}}                                                  %
\makeatother                                                                   %
%---------------------------Packages----------------------------%
\usepackage{geometry}
\geometry{b5paper, margin=1.0in}
\usepackage[T1]{fontenc}
\usepackage{graphicx, float}            % Graphics/Images.
\usepackage{natbib}                     % For bibliographies.
\bibliographystyle{agsm}                % Bibliography style.
\usepackage[french, english]{babel}     % Language typesetting.
\usepackage[dvipsnames]{xcolor}         % Color names.
\usepackage{listings}                   % Verbatim-Like Tools.
\usepackage{mathtools, esint, mathrsfs} % amsmath and integrals.
\usepackage{amsthm, amsfonts, amssymb}  % Fonts and theorems.
\usepackage{tcolorbox}                  % Frames around theorems.
\usepackage{upgreek}                    % Non-Italic Greek.
\usepackage{fmtcount, etoolbox}         % For the \book{} command.
\usepackage[newparttoc]{titlesec}       % Formatting chapter, etc.
\usepackage{titletoc}                   % Allows \book in toc.
\usepackage[nottoc]{tocbibind}          % Bibliography in toc.
\usepackage[titles]{tocloft}            % ToC formatting.
\usepackage{pgfplots, tikz}             % Drawing/graphing tools.
\usepackage{imakeidx}                   % Used for index.
\usetikzlibrary{
    calc,                   % Calculating right angles and more.
    angles,                 % Drawing angles within triangles.
    arrows.meta,            % Latex and Stealth arrows.
    quotes,                 % Adding labels to angles.
    positioning,            % Relative positioning of nodes.
    decorations.markings,   % Adding arrows in the middle of a line.
    patterns,
    arrows
}                                       % Libraries for tikz.
\pgfplotsset{compat=1.9}                % Version of pgfplots.
\usepackage[font=scriptsize,
            labelformat=simple,
            labelsep=colon]{subcaption} % Subfigure captions.
\usepackage[font={scriptsize},
            hypcap=true,
            labelsep=colon]{caption}    % Figure captions.
\usepackage[pdftex,
            pdfauthor={Ryan Maguire},
            pdftitle={Mathematics and Physics},
            pdfsubject={Mathematics, Physics, Science},
            pdfkeywords={Mathematics, Physics, Computer Science, Biology},
            pdfproducer={LaTeX},
            pdfcreator={pdflatex}]{hyperref}
\hypersetup{
    colorlinks=true,
    linkcolor=blue,
    filecolor=magenta,
    urlcolor=Cerulean,
    citecolor=SkyBlue
}                           % Colors for hyperref.
\usepackage[toc,acronym,nogroupskip,nopostdot]{glossaries}
\usepackage{glossary-mcols}
%------------------------Theorem Styles-------------------------%
\theoremstyle{plain}
\newtheorem{theorem}{Theorem}[section]

% Define theorem style for default spacing and normal font.
\newtheoremstyle{normal}
    {\topsep}               % Amount of space above the theorem.
    {\topsep}               % Amount of space below the theorem.
    {}                      % Font used for body of theorem.
    {}                      % Measure of space to indent.
    {\bfseries}             % Font of the header of the theorem.
    {}                      % Punctuation between head and body.
    {.5em}                  % Space after theorem head.
    {}

% Italic header environment.
\newtheoremstyle{thmit}{\topsep}{\topsep}{}{}{\itshape}{}{0.5em}{}

% Define environments with italic headers.
\theoremstyle{thmit}
\newtheorem*{solution}{Solution}

% Define default environments.
\theoremstyle{normal}
\newtheorem{example}{Example}[section]
\newtheorem{definition}{Definition}[section]
\newtheorem{problem}{Problem}[section]

% Define framed environment.
\tcbuselibrary{most}
\newtcbtheorem[use counter*=theorem]{ftheorem}{Theorem}{%
    before=\par\vspace{2ex},
    boxsep=0.5\topsep,
    after=\par\vspace{2ex},
    colback=green!5,
    colframe=green!35!black,
    fonttitle=\bfseries\upshape%
}{thm}

\newtcbtheorem[auto counter, number within=section]{faxiom}{Axiom}{%
    before=\par\vspace{2ex},
    boxsep=0.5\topsep,
    after=\par\vspace{2ex},
    colback=Apricot!5,
    colframe=Apricot!35!black,
    fonttitle=\bfseries\upshape%
}{ax}

\newtcbtheorem[use counter*=definition]{fdefinition}{Definition}{%
    before=\par\vspace{2ex},
    boxsep=0.5\topsep,
    after=\par\vspace{2ex},
    colback=blue!5!white,
    colframe=blue!75!black,
    fonttitle=\bfseries\upshape%
}{def}

\newtcbtheorem[use counter*=example]{fexample}{Example}{%
    before=\par\vspace{2ex},
    boxsep=0.5\topsep,
    after=\par\vspace{2ex},
    colback=red!5!white,
    colframe=red!75!black,
    fonttitle=\bfseries\upshape%
}{ex}

\newtcbtheorem[auto counter, number within=section]{fnotation}{Notation}{%
    before=\par\vspace{2ex},
    boxsep=0.5\topsep,
    after=\par\vspace{2ex},
    colback=SeaGreen!5!white,
    colframe=SeaGreen!75!black,
    fonttitle=\bfseries\upshape%
}{not}

\newtcbtheorem[use counter*=remark]{fremark}{Remark}{%
    fonttitle=\bfseries\upshape,
    colback=Goldenrod!5!white,
    colframe=Goldenrod!75!black}{ex}

\newenvironment{bproof}{\textit{Proof.}}{\hfill$\square$}
\tcolorboxenvironment{bproof}{%
    blanker,
    breakable,
    left=3mm,
    before skip=5pt,
    after skip=10pt,
    borderline west={0.6mm}{0pt}{green!80!black}
}

\AtEndEnvironment{lexample}{$\hfill\textcolor{red}{\blacksquare}$}
\newtcbtheorem[use counter*=example]{lexample}{Example}{%
    empty,
    title={Example~\theexample},
    boxed title style={%
        empty,
        size=minimal,
        toprule=2pt,
        top=0.5\topsep,
    },
    coltitle=red,
    fonttitle=\bfseries,
    parbox=false,
    boxsep=0pt,
    before=\par\vspace{2ex},
    left=0pt,
    right=0pt,
    top=3ex,
    bottom=1ex,
    before=\par\vspace{2ex},
    after=\par\vspace{2ex},
    breakable,
    pad at break*=0mm,
    vfill before first,
    overlay unbroken={%
        \draw[red, line width=2pt]
            ([yshift=-1.2ex]title.south-|frame.west) to
            ([yshift=-1.2ex]title.south-|frame.east);
        },
    overlay first={%
        \draw[red, line width=2pt]
            ([yshift=-1.2ex]title.south-|frame.west) to
            ([yshift=-1.2ex]title.south-|frame.east);
    },
}{ex}

\AtEndEnvironment{ldefinition}{$\hfill\textcolor{Blue}{\blacksquare}$}
\newtcbtheorem[use counter*=definition]{ldefinition}{Definition}{%
    empty,
    title={Definition~\thedefinition:~{#1}},
    boxed title style={%
        empty,
        size=minimal,
        toprule=2pt,
        top=0.5\topsep,
    },
    coltitle=Blue,
    fonttitle=\bfseries,
    parbox=false,
    boxsep=0pt,
    before=\par\vspace{2ex},
    left=0pt,
    right=0pt,
    top=3ex,
    bottom=0pt,
    before=\par\vspace{2ex},
    after=\par\vspace{1ex},
    breakable,
    pad at break*=0mm,
    vfill before first,
    overlay unbroken={%
        \draw[Blue, line width=2pt]
            ([yshift=-1.2ex]title.south-|frame.west) to
            ([yshift=-1.2ex]title.south-|frame.east);
        },
    overlay first={%
        \draw[Blue, line width=2pt]
            ([yshift=-1.2ex]title.south-|frame.west) to
            ([yshift=-1.2ex]title.south-|frame.east);
    },
}{def}

\AtEndEnvironment{ltheorem}{$\hfill\textcolor{Green}{\blacksquare}$}
\newtcbtheorem[use counter*=theorem]{ltheorem}{Theorem}{%
    empty,
    title={Theorem~\thetheorem:~{#1}},
    boxed title style={%
        empty,
        size=minimal,
        toprule=2pt,
        top=0.5\topsep,
    },
    coltitle=Green,
    fonttitle=\bfseries,
    parbox=false,
    boxsep=0pt,
    before=\par\vspace{2ex},
    left=0pt,
    right=0pt,
    top=3ex,
    bottom=-1.5ex,
    breakable,
    pad at break*=0mm,
    vfill before first,
    overlay unbroken={%
        \draw[Green, line width=2pt]
            ([yshift=-1.2ex]title.south-|frame.west) to
            ([yshift=-1.2ex]title.south-|frame.east);},
    overlay first={%
        \draw[Green, line width=2pt]
            ([yshift=-1.2ex]title.south-|frame.west) to
            ([yshift=-1.2ex]title.south-|frame.east);
    }
}{thm}

%--------------------Declared Math Operators--------------------%
\DeclareMathOperator{\adjoint}{adj}         % Adjoint.
\DeclareMathOperator{\Card}{Card}           % Cardinality.
\DeclareMathOperator{\curl}{curl}           % Curl.
\DeclareMathOperator{\diam}{diam}           % Diameter.
\DeclareMathOperator{\dist}{dist}           % Distance.
\DeclareMathOperator{\Div}{div}             % Divergence.
\DeclareMathOperator{\Erf}{Erf}             % Error Function.
\DeclareMathOperator{\Erfc}{Erfc}           % Complementary Error Function.
\DeclareMathOperator{\Ext}{Ext}             % Exterior.
\DeclareMathOperator{\GCD}{GCD}             % Greatest common denominator.
\DeclareMathOperator{\grad}{grad}           % Gradient
\DeclareMathOperator{\Ima}{Im}              % Image.
\DeclareMathOperator{\Int}{Int}             % Interior.
\DeclareMathOperator{\LC}{LC}               % Leading coefficient.
\DeclareMathOperator{\LCM}{LCM}             % Least common multiple.
\DeclareMathOperator{\LM}{LM}               % Leading monomial.
\DeclareMathOperator{\LT}{LT}               % Leading term.
\DeclareMathOperator{\Mod}{mod}             % Modulus.
\DeclareMathOperator{\Mon}{Mon}             % Monomial.
\DeclareMathOperator{\multideg}{mutlideg}   % Multi-Degree (Graphs).
\DeclareMathOperator{\nul}{nul}             % Null space of operator.
\DeclareMathOperator{\Ord}{Ord}             % Ordinal of ordered set.
\DeclareMathOperator{\Prin}{Prin}           % Principal value.
\DeclareMathOperator{\proj}{proj}           % Projection.
\DeclareMathOperator{\Refl}{Refl}           % Reflection operator.
\DeclareMathOperator{\rk}{rk}               % Rank of operator.
\DeclareMathOperator{\sgn}{sgn}             % Sign of a number.
\DeclareMathOperator{\sinc}{sinc}           % Sinc function.
\DeclareMathOperator{\Span}{Span}           % Span of a set.
\DeclareMathOperator{\Spec}{Spec}           % Spectrum.
\DeclareMathOperator{\supp}{supp}           % Support
\DeclareMathOperator{\Tr}{Tr}               % Trace of matrix.
%--------------------Declared Math Symbols--------------------%
\DeclareMathSymbol{\minus}{\mathbin}{AMSa}{"39} % Unary minus sign.
%------------------------New Commands---------------------------%
\DeclarePairedDelimiter\norm{\lVert}{\rVert}
\DeclarePairedDelimiter\ceil{\lceil}{\rceil}
\DeclarePairedDelimiter\floor{\lfloor}{\rfloor}
\newcommand*\diff{\mathop{}\!\mathrm{d}}
\newcommand*\Diff[1]{\mathop{}\!\mathrm{d^#1}}
\renewcommand*{\glstextformat}[1]{\textcolor{RoyalBlue}{#1}}
\renewcommand{\glsnamefont}[1]{\textbf{#1}}
\renewcommand\labelitemii{$\circ$}
\renewcommand\thesubfigure{%
    \arabic{chapter}.\arabic{figure}.\arabic{subfigure}}
\addto\captionsenglish{\renewcommand{\figurename}{Fig.}}
\numberwithin{equation}{section}

\renewcommand{\vector}[1]{\boldsymbol{\mathrm{#1}}}

\newcommand{\uvector}[1]{\boldsymbol{\hat{\mathrm{#1}}}}
\newcommand{\topspace}[2][]{(#2,\tau_{#1})}
\newcommand{\measurespace}[2][]{(#2,\varSigma_{#1},\mu_{#1})}
\newcommand{\measurablespace}[2][]{(#2,\varSigma_{#1})}
\newcommand{\manifold}[2][]{(#2,\tau_{#1},\mathcal{A}_{#1})}
\newcommand{\tanspace}[2]{T_{#1}{#2}}
\newcommand{\cotanspace}[2]{T_{#1}^{*}{#2}}
\newcommand{\Ckspace}[3][\mathbb{R}]{C^{#2}(#3,#1)}
\newcommand{\funcspace}[2][\mathbb{R}]{\mathcal{F}(#2,#1)}
\newcommand{\smoothvecf}[1]{\mathfrak{X}(#1)}
\newcommand{\smoothonef}[1]{\mathfrak{X}^{*}(#1)}
\newcommand{\bracket}[2]{[#1,#2]}

%------------------------Book Command---------------------------%
\makeatletter
\renewcommand\@pnumwidth{1cm}
\newcounter{book}
\renewcommand\thebook{\@Roman\c@book}
\newcommand\book{%
    \if@openright
        \cleardoublepage
    \else
        \clearpage
    \fi
    \thispagestyle{plain}%
    \if@twocolumn
        \onecolumn
        \@tempswatrue
    \else
        \@tempswafalse
    \fi
    \null\vfil
    \secdef\@book\@sbook
}
\def\@book[#1]#2{%
    \refstepcounter{book}
    \addcontentsline{toc}{book}{\bookname\ \thebook:\hspace{1em}#1}
    \markboth{}{}
    {\centering
     \interlinepenalty\@M
     \normalfont
     \huge\bfseries\bookname\nobreakspace\thebook
     \par
     \vskip 20\p@
     \Huge\bfseries#2\par}%
    \@endbook}
\def\@sbook#1{%
    {\centering
     \interlinepenalty \@M
     \normalfont
     \Huge\bfseries#1\par}%
    \@endbook}
\def\@endbook{
    \vfil\newpage
        \if@twoside
            \if@openright
                \null
                \thispagestyle{empty}%
                \newpage
            \fi
        \fi
        \if@tempswa
            \twocolumn
        \fi
}
\newcommand*\l@book[2]{%
    \ifnum\c@tocdepth >-3\relax
        \addpenalty{-\@highpenalty}%
        \addvspace{2.25em\@plus\p@}%
        \setlength\@tempdima{3em}%
        \begingroup
            \parindent\z@\rightskip\@pnumwidth
            \parfillskip -\@pnumwidth
            {
                \leavevmode
                \Large\bfseries#1\hfill\hb@xt@\@pnumwidth{\hss#2}
            }
            \par
            \nobreak
            \global\@nobreaktrue
            \everypar{\global\@nobreakfalse\everypar{}}%
        \endgroup
    \fi}
\newcommand\bookname{Book}
\renewcommand{\thebook}{\texorpdfstring{\Numberstring{book}}{book}}
\providecommand*{\toclevel@book}{-2}
\makeatother
\titleformat{\part}[display]
    {\Large\bfseries}
    {\partname\nobreakspace\thepart}
    {0mm}
    {\Huge\bfseries}
\titlecontents{part}[0pt]
    {\large\bfseries}
    {\partname\ \thecontentslabel: \quad}
    {}
    {\hfill\contentspage}
\titlecontents{chapter}[0pt]
    {\bfseries}
    {\chaptername\ \thecontentslabel:\quad}
    {}
    {\hfill\contentspage}
\newglossarystyle{longpara}{%
    \setglossarystyle{long}%
    \renewenvironment{theglossary}{%
        \begin{longtable}[l]{{p{0.25\hsize}p{0.65\hsize}}}
    }{\end{longtable}}%
    \renewcommand{\glossentry}[2]{%
        \glstarget{##1}{\glossentryname{##1}}%
        &\glossentrydesc{##1}{~##2.}
        \tabularnewline%
        \tabularnewline
    }%
}
\newglossary[not-glg]{notation}{not-gls}{not-glo}{Notation}
\newcommand*{\newnotation}[4][]{%
    \newglossaryentry{#2}{type=notation, name={\textbf{#3}, },
                          text={#4}, description={#4},#1}%
}
%--------------------------LENGTHS------------------------------%
% Spacings for the Table of Contents.
\addtolength{\cftsecnumwidth}{1ex}
\addtolength{\cftsubsecindent}{1ex}
\addtolength{\cftsubsecnumwidth}{1ex}
\addtolength{\cftfignumwidth}{1ex}
\addtolength{\cfttabnumwidth}{1ex}

% Indent and paragraph spacing.
\setlength{\parindent}{0em}
\setlength{\parskip}{0em}                                                           %
\makeindex[intoc]                                                              %
%----------------------------Main Document-------------------------------------%
\begin{document}
    \pagenumbering{gobble}
    \title{MATH 114 Algebraic Topology - Assignment 1}
    \author{Ryan Maguire}
    \date{\vspace{-5ex}}
    \maketitle
    \pagenumbering{roman}
    \pagenumbering{arabic}
    \setcounter{section}{1}
    \begin{problem}
        Construct an explicit deformation retraction of
        $\nspace\setminus\{\vector{0}\}$ to $\nsphere[n-1]$.
    \end{problem}
    \begin{solution}
        Let $X=\nspace\setminus\{\vector{0}\}$ and define $r:X\rightarrow{X}$
        by:
        \begin{equation}
            r(\vector{x})=\frac{\vector{x}}{\norm{\vector{x}}_{2}}
        \end{equation}
        This is well defined since for all $\vector{x}\in{X}$ we have
        $\vector{x}\ne\vector{0}$ and hence $\norm{\vector{x}}_{2}\ne{0}$.
        Moreover, it is a retract of $X$ to $\nsphere[n-1]$. For if
        $\vector{s}\in\nsphere[n-1]$ then by definition
        $\norm{\vector{s}}_{2}=1$. But then:
        \begin{equation}
            r(\vector{s})=\frac{\vector{s}}{\norm{\vector{s}}_{2}}
                =\vector{s}
                =\identity{\nsphere[n-1]}(\vector{s})
        \end{equation}
        Hence $r|_{\nsphere[n-1]}=\identity{\nsphere[n-1]}$. Moreover, the
        image of $r$ is $\nsphere[n-1]$ since:
        \begin{equation}
            \norm{r(\vector{x})}_{2}
                =\norm[\bigg]{\frac{\vector{x}}{\norm{\vector{x}}_{2}}}_{2}
                =\frac{\norm{\vector{x}}_{2}}{\norm{\vector{x}}_{2}}
                =1
        \end{equation}
        Therefore $r$ satisfies the criterion of a retract. Let
        $H:X\times{I}\rightarrow{X}$ be the straight-line homotopy:
        \begin{equation}
            H(\vector{x},\,t)
                =(1-t)\cdot\identity{X}(\vector{x})+t\cdot{r}(\vector{x})
        \end{equation}
        This is well defined since for all $\vector{x}\in{X}$ and $t\in[0,1]$
        we have:
        \begin{subequations}
            \begin{align}
                \norm{H(\vector{x},\,t)}_{2}&=
                \norm{(1-t)\cdot\vector{x}+t\cdot{r}(\vector{x})}_{2}\\
                &\geq\min\{\norm{\vector{x}}_{2},\,\norm{r(\vector{x})}_{2}\}\\
                &>0
            \end{align}
        \end{subequations}
        Intuitively, $H(\vector{x},t)$ is the straight line from
        $\vector{x}$ to $r(\vector{x})$ and this never crosses the origin.
        Hence, $H(\vector{x},t)$ is well defined and moreover is a homotopy
        between the identity map and $r$. It is a deformation retraction since
        for all $\vector{s}\in\nspace[n-1]$ and $t\in[0,1]$ we have:
        \begin{equation}
            H(\vector{s},\,t)=(1-t)\cdot{t}+t\cdot\vector{s}=\vector{s}
        \end{equation}
    \end{solution}
    \begin{problem}
        A deformation retraction in the weak sense of a space $X$ to a subspace
        $A$ is a homotopy $H:X\times{I}\rightarrow{X}$ between $\identity{X}$
        and a function $g:X\rightarrow{X}$ such that $g[X]\subseteq{A}$ and
        such that for all $t\in[0,1]$ and $a\in{A}$, $H(a,t)\in{A}$. Show that
        if $X$ deformation retracts to $A$ in the weak sense, then the
        inclusion mapping $\iota:A\rightarrow{X}$ is a homotopy equivalence.
    \end{problem}
    \begin{solution}
        We need to find a homotopy inverse $g:X\rightarrow{A}$ of
        $\iota:A\rightarrow{X}$. That is, a function $g$ such that
        $g\circ\iota$ is homotopic to $\identity{A}$ and $\iota\circ{g}$ is
        homotopic to $\identity{X}$. Let $g:X\rightarrow{A}$ be the function
        the homotopy $H$ drags $\identity{X}$ to. That is, $g(x)=H(x,1)$.
        Then $(g\circ\iota)(a)=g(a)$ for all $a\in{A}$. But then
        $H|_{A\times{I}}$ is a homotopy between $g\circ\iota$ and
        $\identity{A}$. That is, $H|_{A\times{I}}:A\times{I}\rightarrow{A}$ is
        a continuous function, the image is in $A$ since by hypothesis for all
        $a\in{A}$ and $t\in{I}$, $H(a,t)\in{A}$, and lastly:
        \begin{equation}
            H(a,\,0)=\identity{X}(a)=a=\identity{A}(a)
        \end{equation}
        and also $H(a,\,1)=g(a)$. Similarly, since the image of $g$ lies in $A$,
        $\iota\circ{g}=g$ and hence $H$ is a homotopy between
        $\iota\circ{g}$ and $\identity{X}$. Thus, $g$ is a homotopy inverse of
        $\iota$ and hence $\iota$ is a homotopy equivalence.
    \end{solution}
    Before the next problem, we prove the following theorem to make life easier.
    \begin{theorem}
        If $\topspace{X}$ is a topological space, then it is
        contractible if and only if $\identity{X}$ is nullhomotopic.
    \end{theorem}
    \begin{proof}
        If $\topspace{X}$ is contractible, then there is a homotopy
        equivalence $f:X\rightarrow{Y}$ where $Y=\{0\}$ is the
        one point space. But if $f$ is a homotopy equivalence, then
        there is a homotopy inverse $g:Y\rightarrow{X}$. But then
        $g\circ{f}$ is homotopic to $\identity{X}$. But $Y$ has only
        one point, and hence $f(x)=0$ for all $x\in{X}$. Let
        $x_{0}=g(0)$. Then $g\circ{f}:X\rightarrow{X}$ is the
        mapping $x\mapsto{x}_{0}$. But $\identity{X}$ is homotopic
        to this, and is therefore nullhomotopic. In the other
        direction, if $\identity{X}$ is nullhomotopic, then there is
        a point $x_{0}\in{X}$ such that $\identity{X}$ is homotopic
        to the function $f:X\rightarrow{X}$ defined by $f(x)=x_{0}$.
        Let $Y=\{x_{0}\}$ be given the subspace topology. Since
        there is only one topology on a space with one point, this
        is homeomorphic to the one point space. But then $f$ is a
        homotopy equivalence since the function $g:Y\rightarrow{X}$
        given by $g(x_{0})=x_{0}$ is a homotopy inverse of $f$. That
        is, since $\identity{X}$ is homotopic to $f$,
        $g\circ{f}$ is homotopic to $\identity{X}$ since
        $g\circ{f}=f$. But also $f\circ{g}=\identity{Y}$. Hence $f$
        is an homotopy equivalence and $\topspace{X}$ is
        contractible.
    \end{proof}
    \begin{problem}
        Show that a retract of a contracible space is contracible.
    \end{problem}
    \begin{solution}
        For $\topspace{X}$ is contractible if and only if
        $\identity{X}$ is nullhomotopic. That is, there is a point
        $x_{0}\in{X}$ such that $\identity{X}$ is homotopic to the
        function $g:X\rightarrow{X}$ defined by $g(x)=x_{0}$. Let $H$ be such a
        homotopy. But if $f:X\rightarrow{A}$ is a retract, then
        $f|_{A}=\identity{A}$. Hence $f\circ{H}|_{A\times{I}}$ is a
        homotopy between $\identity{A}$ and $f(x_{0})$. That is,
        $f\circ{H}|_{A\times{I}}$ is the composition of continuous functions and
        is therefore continuous. Moreover,
        \begin{subequations}
            \begin{align}
                (f\circ{H}|_{A\times{I}})(a,0)=f\big(H|_{A\times{I}}(a,0)\big)
                    =f\big(\identity{X}(a)\big)
                    =f(a)
                    =a
            \end{align}
        \end{subequations}
        But also:
        \begin{equation}
            (f\circ{H}|_{A\times{I}})(a,1)=f\big(H|_{A\times{I}}(a,1)\big)
                =f\big(g(a)\big)
                =f(x_{0})
        \end{equation}
        Therefore $\identity{A}$ is nullhomotopic and $A$ is contracible.
    \end{solution}
    \begin{problem}
        Show that every mapping cylinder for any continuous function
        $f:\nsphere[1]\rightarrow\nsphere[1]$ is a CW complex. Construct a
        CW complex that has the annulus and M\"{o}bius band as deformation
        retracts.
    \end{problem}
    \begin{solution}
        We start with two points in our 0-cell and then attach three 1-cells.
        We wrap two copies of $(0,1)$ around in circles about the two points,
        and the final one connects the two points. Hence we have two copies of
        $\nsphere[1]$ and one copy of $[0,1]$. From here we glue the square
        $[0,1]\times[0,1]$ into this by mapping $(x,1)$ to $\exp(2\pi{i}x)$ in
        the top circle, $(0,y)$ and $(1,y)$ to the closed interval part of our
        $X^{1}$ skeleton. Lastly we attach $(x,0)$ to $f(\exp(2\pi{i}x))$ in
        the bottom circle. The result is a mapping cylinder for $f$ consisting
        of two 0-cells, three 1-cells, and one 2-cell.
        \par\hfill\par
        As noted in Hatcher's text, both the annulus and the M\"{o}bius strip
        can be deformation retracted to a circle by means of a mapping cylinder.
        We use the above assertion to construct a CW complex that contains the
        annulus and a M\"{o}bius strip as retracts
        (Fig.~\ref{fig:CW_Comp_Annulus_Mobius}). We start with an actual
        cylinder obtained from the mapping cylinder of the function
        $\identity{\nsphere[1]}$. The result is homeomorphic to an annulus. We
        then use the M\"{o}bius band function
        $f:\nsphere[1]\rightarrow\nsphere[1]$ to obtain the M\"{o}bius strip as
        the mapping cylinder $M_{f}$. We glue the bottom of the mapping cylinder
        of the identity map to the top of the mapping cylinder of the M\"{o}bius
        strip. The result is a CW complex with three 0-cells, five 1-cells, and
        two 2-cells. To deformation retract to the annulus we first use the
        mapping cylinder of the M\"{o}bius band and squeeze this down, leaving
        the annulus untouched. Easier to visualize, to obtain the M\"{o}bius
        band we simply push the cylinder part of
        Fig.~\ref{fig:CW_Comp_Annulus_Mobius} down, leaving the M\"{o}bius band
        untouched.
    \end{solution}
    \begin{figure}[H]
        \centering
        \captionsetup{type=figure}
        \includegraphics{images/Mobius_Strip_Annulus_Def_Retract.pdf}
        \caption{CW Complex Containing an Annulus and a M\"{o}bius Band}
        \label{fig:CW_Comp_Annulus_Mobius}
    \end{figure}
\end{document}