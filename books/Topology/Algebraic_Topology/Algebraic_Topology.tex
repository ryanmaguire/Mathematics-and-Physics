\chapter{Stuff}
\section{Lecture 7-ish, Maybe}
    We've computed the following:
    \begin{equation}
        \pi_{1}(\mathbb{Z},x_{0})=
        \begin{cases}
            \mathbb{Z},&n=1\\
            \{e\},&n\geq{2}
        \end{cases}
    \end{equation}
    Where $\{e\}$ denotes the trivial group. Let's
    now prove this.
    \begin{theorem}
        If $n\geq{2}$, and if $x_{0}\in{S}^{1}$, then
        $\pi_{1}(S^{n},x_{0})\simeq\{e\}$.
    \end{theorem}
    \begin{proof}
        What we want to do is take any loop $\gamma$,
        remove a point that
        $\gamma$ doesn't map to, and then use the fact
        that $S^{n}\setminus\{x\}$ is homeomorphic to
        $\mathbb{R}^{n}$. Since $\mathbb{R}^{n}$ is
        contractible, we are done. However, there exist
        space filling curves. So now we need to show that
        space filling curves can still be contracted.
        Pick $x\in{S]^[n}$ such that $x\ne{x}_{0}$. Then
        there is an ope ball $B$ centered about $x$ such
        that $x_{0}\notin{B}$. But $B$ is open and
        $\gamma$ is continuous Thus, $\gamma^{\minus{1}}$ is
        open in $[0,1]$. But since $x_{0}\notin{B}$,
        $0,1\notin\gamma^{\minus{1}}(B)$, and therefore
        the pre-image is an open subset of $\mathbb{R}$ as
        well. But every open subset of $\mathbb{R}$ is the
        disjoint union of open intervals. Let
        $(a,b)$ be one of these intervals. But then
        $f:(a,b)\rightarrow{B}$ is such that
        $f(a),f(b)\notin{B}$. But
        $\partial{B}\simeq{S}^{n-1}$ for $n\geq{2}$, and
        $S^{n-1}$ is path connected. Lift the image
        continuously to the bounded. This works if there
        are finitely many such intervals $(a,b)$, but there
        could be countably infinitely many. But
        $f^{\minus{1}}(x)$ is closed and is a subset of
        $S^{n}$, and is thereore closed and bounded and thus
        compact, by Heine-Borel. thus finitely many of the
        $(a_{i},b_{i})$ cover $f^{\minus{1}}(x)$. Move these
        ones. Thus, we can remove $x$ and complete the proof.
    \end{proof}
    We next move to Van Kampen's theorem. As an aside we must
    talk about free products of groups. The direct product
    of two groups $G_{1}$ and $G_{2}$ is defined by:
    \begin{equation}
        (x_{1},x_{2})*(y_{1},y_{2})=
        (x_{1}*y_{1},x_{2}*y_{2})
    \end{equation}
    There is a universal property on such groups that goes
    as follows:
    \begin{theorem}
        If $G_{1},G_{2}$, and $H$ are groups such that
        there are group homomorphisms
        $\varphi_{i}:H\rightarrow{G}_{i}$, then there is
        a unique group homomorphism
        $\psi:H\rightarrow{G}_{1}\times{G}_{2}$ such that:
        \begin{equation}
            \psi(h)=(\varphi_{1}(h),\varphi_{2}(h))
        \end{equation}
    \end{theorem}
    We want to flip this picture and define something called
    the \textit{free product} of $G_{1}$ and $G_{2}$. This
    is some group $G_{1}*G_{2}$ such that it contains,
    disjointly, all of the elements of $G_{1}$ and $G_{2}$,
    with no relations between elements of $G_{1}$ and $G_{2}$.
    This is the set of \textit{reduced words}. Here, a word
    is something in an \textit{alphabet}, or a set. We take
    as our alphabet the disjoint union of $G_{1}$ and $G_{2}$.
    A word is a finite ordered sequence from this alphabet.
    This includes the empty word, which is simply the empty
    set. A reduced word is a word such that $a_{n}$ and
    $a_{n+1}$ are not from the same group, and also none of
    the letters $a_{n}$ are the identity element in either
    $G_{1}$ or $G_{2}$. The product is then to concatenate
    two reduced words, and then further reduce the
    concatenation. Similarly, for a set of groups
    $\{G_{\alpha}:\alpha\in{I}\}$, the free product:
    \begin{equation}
        \ast_{\alpha\in{I}}G_{\alpha}
        =\{\textrm{Reduced Word in }
            \coprod_{\alpha\in{I}}G_{\alpha}\}
    \end{equation}
    Again, there is a unique group that satisfies the
    universal property mentioned earlier. As a special
    exmaple, consider $\mathbb{Z}\times\mathbb{Z}$. This has
    a Cayley group, and the Cayley actually reveals the
    group structure. That is, there are two generators,
    $a=(0,1)$ and $b=(1,0)$, and has the relation that
    $ab=ba$. The Cayley graph of $\mathbb{Z}*\mathbb{Z}$
    shows that this is the free group on 2 generators. It
    is a non-trivial question to show whether or not
    $F_{n}$ is isomorphic to $\mathbb{Z}^{n}$.
\section{Van Kampen's Theorem}
    \begin{ltheorem}{Van Kampen's Theorem}
        If $X$ is path connected, if $x_{0}\in{X}$, and if
        $X$ is such that:
        \begin{equation}
            X=\bigcup_{\alpha\in{J}}A_{\alpha}
        \end{equation}
        Where $x_{0}\in{A}_{\alpha}$ for all
        $\alpha\in{I}$, $A_{\alpha}$ is path-connected,
        $A_{a}\cap{A}_{b}$ is path-connected, and
        $A_{a}\cap{A}_{b}\cap{A}_{c}$ is path-connected, then
        for all $[f]\in\pi_{1}(X,x_{0})$, $[f]$ can be
        factored in $\pi_{1}(X,x_{0})$ as the product:
        \begin{equation}
            [f]=[f_{1}]\cdot[f_{2}]\cdots[f_{m}]
        \end{equation}
        Where $f_{j}:I\rightarrow{A}_{\alpha_{j}}$ for each
        $j\in\mathbb{Z}_{m}$.
    \end{ltheorem}
    This is intuitively clear, given a loop in the space
    $X$ we can write it as a concatenation of different
    loops contained in each of the $A_{\alpha_{j}}$.
    But this theorem relates topology to algebra by the
    use of these free groups and free products.
    \subsection{Examples}
        \begin{example}
            Let $X$ be a torus with two internal circles
            identified. This is equivalent to two spheres
            attached by two lines, which is homotopic equivalent
            to two spheres and two loops, all joined at one
            point. That is,
            \begin{equation}
                X\approx{S}^{1}\lor{S}^{1}\lor{S}^{2}\lor{S}^{2}
            \end{equation}
            The fundamental group of wedge sums has the
            following property:
            \begin{equation}
                \pi_{1}(\lor_{\alpha}X_{\alpha})=
                \star_{\alpha}\pi_{1}(X_{\alpha})
            \end{equation}
            The condition is that the base point in each
            $X_{\alpha}$ must have a neighborhood
            $\mathcal{U}_{\alpha}\in{X}_{\alpha}$ that
            deformation retracts onto the base point.
        \end{example}
        Given a graph, infinite or not, what is the fundamental
        group of the graph? First fint a maximal spanning
        tree. Tree's are contractible, and thus we have:
        \begin{equation}
            \pi_{1}(Graph)\simeq\pi_{1}(graph/tree)
        \end{equation}
        This last thing is the free group with generators of
        the edges the are not removed in the modulo.
        A know is a diffeomorphic copy of $S^{1}$ in
        $\mathbb{R}^{3}$. Links are several knots put together.
        Moving on to finite CW complex, consider $X$, which
        is $n\in\mathbb{N}$ dimensional, and has skeletons
        $X^{0},\dots,X^{n}$. Then $\pi_{1}(X^{1})$ is a
        free group since $X^{1}$ is a graph. What about
        $\pi_{1}(X^{2})$? We use Van Kampen's theorem to
        simplify this problem. $X^{2}$ is obtained from
        $X^{1}$ by attaching 2-cells $e_{\alpha}^{2}$,
        $\alpha\in{J}$, where $J$ is some index set, via
        attaching maps:
        $\varphi_{\alpha}:S^{1}\rightarrow{X}^{1}$.
        \begin{theorem}
            $\pi_{1}(X^{2})=\pi_{1}(X^{1})/N$
            Where $N$ is the normal subgroup generated by
            $[\gamma_{\alpha}\varphi_{\alpha}\gamma_{\alpha}]$
            in $\pi_{1}(X^{1})$.
        \end{theorem}
        \begin{theorem}
            $\pi_{1}(X)\simeq\pi_{1}(X^{2})$.
        \end{theorem}
        So what is $\pi_{1}$ of a two torus? We can write
        the two torus as an octogon with relations on the
        sides. There are four such relations, and we get
        that the fundamental group is the group generated by
        four elements with the condition that:
        \begin{equation}
            aba^{\minus{1}}b^{\minus{1}}
            cdc^{\minus{1}}d^{\minus{1}}=e
        \end{equation}
        \begin{theorem}
            If $X$ is a CW complex, and if
            $x_{0}$ is a vertex, or a point in the one skeleton
            of $X$, then:
            \begin{equation}
                \pi_{1}(X,x_{0})=\pi_{1}(X^{2},x_{0})
                =\pi_{1}(X,x_{0})/N
            \end{equation}
        \end{theorem}
        There are higher homotopy groups, $\pi_{k}(X,x_{0})$.
        These are homotopy classes of maps
        $f:S^{k}\rightarrow{X}$ such that $f(0)=x_{0}$.
        \begin{theorem}
            If $X$ and $Y$ are CW complexes, if
            $f:X\rightarrow{Y}$ is continuous, and if $f$ is
            a homotopy equivalence, then there are group
            homomorphisms
            $f_{*}:\pi_{k}(X,x_{0})\rightarrow{\pi}_{k}(Y,y_{0})$
            for all $k\in\mathbb{N}$.
        \end{theorem}
        Whitehead's theorem says the converse of this is true
        as well.
        \begin{ltheorem}{Whitehead's Theorem}
            The converse of the previous theorem is true.
        \end{ltheorem}
        \begin{theorem}
            If $X$ is a topological space, then there exists
            a CW complex $X'$ and a map
            $f:X'\rightarrow{X}$ such that
            $f_{*}:\pi_{k}(X')\simeq\pi_{k}(X)$.
        \end{theorem}
        This says that every topological space has a unique,
        up to homotopy equivalence, CW complex that approximates
        the space very well.
\section{Covering Spaces}
    There's a deep link between covering spaces and Galois
    theory. When studying the fundamental group of a circle
    we came across loops $f:I\rightarrow{S}^{1}$ such that
    $f(0)=f(1)$. We studied the map
    $p(t)=\exp(2\pi{i}t)$ which spiralled $\mathbb{R}$ into
    the circle, and used this to define the winding number
    and calculate the fundamental group. This notion
    generalizes to other spaces.
    \begin{ldefinition}{Covering Space}
        A covering space of $X$ is a space $\tilde{X}$ such
        that, for al $x\in{X}$, there is an open neighborhood
        $\mathcal{U}_{x}\subseteq{X}$ such that
        $p^{\minus{1}}(\mathcal{U})$ is the disjoint union
        of open sets in $\tilde{X}$ and such that
        the image under $p$ of one of these sets is a
        homeomorphism with $\mathcal{U}$..
    \end{ldefinition}
    The first example is $\mathbb{R}$ and $S^{1}$. We will prove
    that nice enough topological spaces $X$ have a covering
    space $\tilde{X}$ such that $\tilde{X}$ is contractible.
    Such covers are unique up to homotopy. These are called
    the universal covers of the space. We will also show that
    there is a one-to-one correspondence between subgroups
    of $\pi_{1}(X)$ and covering spaces. That is, there
    is a function
    $\rho_{*}:\pi_{1}(\tilde{X})\rightarrow\pi_{1}(X)$ that
    is injective.
\section{Universal Cover}
    We now discuss the existence of universal covers.
    This notion relations to Lie groups, representation theory,
    the special orthogonal group $SO(n)$, and it occurs
    in modern physics. $SO(n)$ is not simply connected, but
    it's universal cover, the spin group, is simply
    connected.
    \begin{ldefinition}{Simply Connected Space}
        A simply connected topological space is a path
        connected topological space $(X,\tau)$ such that
        for any two points $x,y\in{X}$ and any two paths
        $\gamma_{1},\gamma_{2}$ between $x$ and $y$,
        $\gamma_{1}$ and $\gamma_{2}$ are homotopic.
    \end{ldefinition}
    That is, $\pi_{0}(X)$ is trivial, there is only one
    path component, and $\pi_{1}(X,x_{0})$ is also
    trivial for all $x_{0}\in{X}$.
    \begin{example}
        $\mathbb{R}^{n}$ is simply connected for all
        $n\in\mathbb{N}$. Given two paths between the same
        two points, the straight line homotopy is a homotopy
        between such paths. For $n\geq{2}$, $S^{n}$ is
        also simply connected. However, $S^{1}$ is not
        simply connected. For $\mathbb{RP}^{2}$, we know
        that $\pi_{1}(\mathbb{RP}^{2})=\mathbb{Z}/2\mathbb{Z}$,
        which is a two element group. Thus there is, up to
        homotopy, only one non-trivial loop in
        $\mathbb{RP}^{2}$. Therefore the real projective plane
        is not simply connected. There's a clear candidate for
        the universal cover of $\mathbb{RP}^{2}$, and that is
        the sphere $S^{2}$.
    \end{example}
    \begin{example}
        We know that the torus $T^{2}=S^{1}\times{S}^{1}$ is
        not simply connected, since the fundamental group of
        $T^{2}$ is the product of the fundamental group
        of $S^{1}$ with itself, that is, $\mathbb{Z}^{2}$. The
        universal cover of $T^{2}$ is $\mathbb{R}^{2}$. If
        we had a surface with $n$ wholes, it is not
        automatically clear what the universal cover is, but
        this turns out to be $\mathbb{R}^{2}$ as well.
    \end{example}
    One question that arises is which topological spaces have
    a universal cover? It is true for all CW complexes, and
    there is the following necessary condition.
    If $X$ is a topological space, and if there is a
    universal cover $\tilde{X}$, then for all
    $x\in{X}$, if $\tilde{x}$ is such that
    $p(\tilde{x})=x$, then $\mathcal{U}\subseteq{X}$, where
    $x\in\mathcal{U}$, evenly covered and
    $\tilde{x}\in\tilde{\mathcal{U}}$, which is homeomorphic
    to $\mathcal{U}$. Given any loop
    $\gamma:I\rightarrow\mathcal{U}$, where $\gamma(0)=x$,
    this lifts to a loop $\tilde{\gamma}$ in
    $\tilde{\mathcal{U}}$. But $\tilde{\mathcal{U}}$ is
    simply connected, and thus $\tilde{\gamma}$ is null
    homotopic in $\tilde{X}$. Thus, we have the following
    definition:
    \begin{ldefinition}{Semi-Locally Simply Connected Space}
        A semi-locally connected space is a topological
        space $(X,\tau)$ such that for all $x\in{X}$ there is
        a neighborhood $\mathcal{U}$ such that for loop in
        $\mathcal{U}$ contracts in $X$.
    \end{ldefinition}
    When speaking of a local property, one usually requires that
    a certain property holds in every open set about a certain
    point. For example, locally compact spaces are topological
    spaces such that for every point $x$ and every open
    neighborhood of $x$ there is a compact subset $V$ contained
    in this neighborhood. We have the following chain:
    \begin{equation}
        \begin{split}
            \textrm{CW}\Longrightarrow
            &\textrm{Locally Contractible}\Longrightarrow
            \textrm{Locally Simply-Connected}
            \Longrightarrow\cdots\\
            &\cdots\Longrightarrow
            \textrm{Semi-Locally Simply-Connected}
        \end{split}
    \end{equation}
    There are two examples of spaces that look very similar,
    but not homeomorphic. One is the Hawaiian earrings, and
    the other is the countable $\lor{S}^{1}$. The Hawaiian
    earrings are not semi-locally simply connected, whereas
    the other set is locally contractible. Thus these two
    spaces are not homeomorphic. Now for a sufficient condition
    for $X$ to have a universal cover.
    \begin{theorem}
        If $X$ is path connected, locally path conneccted,
        and semi-locally path connected, then there exists
        a simply connected cover of $X$.
    \end{theorem}
    \begin{proof}
        Let $\tilde{X}$ be the set of homotopy equivalence
        classes of paths $\gamma:I\rightarrow{X}$ such that
        $\gamma(0)=x_{0}$. We need a function
        $p:\tilde{X}\rightarrow{X}$ such that
        $p([\gamma])=\gamma(1)$. Next we need a topology
        on $\tilde{X}$. Recall that for a basis for a topology
        of a topological space $Y$ is a collection
        $\mathcal{B}$ of open subsets of $Y$ with the property
        that for all $y\in{Y}$, $\mathcal{U}\subseteq{Y}$ is
        open and $y\in\mathcal{U}$, there is a $V\in\mathcal{B}$
        such that $y\in{V}\subseteq\mathcal{U}$. For a set
        $Z$, a collection $\mathcal{B}$ of subsets of $Z$
        can be used a the basis of a topology on $Z$ if
        for all $U,V\in\mathcal{B}$, and for all
        $x\in{U}\cap{V}$, there is a $W\in\mathcal{B}$ such
        that $x\in{W}\subseteq{U}\cap{V}$. Define
        $\mathcal{B}$ as follows:
        \begin{equation}
            \mathcal{B}=\{B\subseteq{X}:B\in\tau,
                B\textrm{ path connected and }
                \pi_{1}(B)\rightarrow\pi_{1}(X)
                \textrm{ is trivial}\}
        \end{equation}
        $\mathcal{B}$ is a basis for the topology of
        $X$. For since $X$ is semi-locally simply connected,
        for all $x\in{X}$ there is a $V\subseteq{X}$ such
        that $x\in{V}$ and
        $\pi_{1}(V)\rightarrow\pi_{1}(X)$ is trivial. But
        $X$ is locally path-connected, and thus there is an
        open $W\subseteq{U}\cap{V}$ such that $x\in{W}$ and
        $W$ is path connected. From this we need to make a basis
        for $\tilde{X}$. For $\mathcal{U}\in\mathcal{B}$,
        let $\gamma:I\rightarrow{X}$ be such that
        $\gamma(0)=x\in\mathcal{U}$ and $\gamma(0)=x_{0}$.
        Define $\mathcal{U}_{\gamma}$ by:
        \begin{equation}
            \mathcal{U}_{\gamma}=\{
                [\gamma\eta]:\eta(0)=\gamma(1)\}
        \end{equation}
        The set of all $\mathcal{U}_{\gamma}$ define a basis
        for a topology on $\tilde{X}$.
    \end{proof}
    \begin{theorem}
        If $X$ is a path connected CW complex, then
        $X$ has a simply connected cover.
    \end{theorem}
    \begin{theorem}
        If $X$ is a manifold, then $X$ has a simply
        connected cover.
    \end{theorem}
\section{Group Actions}
    Let $G$ be a discrete group, and $X$. There is a notion
    called a group action of $G$ on $X$. There are two ways
    of viewing this, as a function
    $p:G\times{X}\rightarrow{X}$, or as a function
    $p:G\rightarrow\textrm{Perm}(X)$, where $\textrm{Perm}$
    denotes the set of permutations of $X$. A third definition
    is $\rho(g):X\rightarrow{X}$ for $g\in{G}$.
    A group action obeys the group law:
    \begin{equation}
        p(g_{1},g_{2})=p(g_{1})p(g_{2})
    \end{equation}
    Given a topological space $X$, $\rho(g)$ continuous
    would imply that it is a homeomorphism, since it has
    a continuous inverse.
    \begin{example}
        $\mathbb{Z}^{2}$ acts on $\mathbb{R}^{2}$:
        \begin{equation}
            \rho:\mathbb{Z}^{2}\rightarrow
            \textrm{Homeo}(\mathbb{R}^{2})
        \end{equation}
        This is defined by
        $\rho(m,n):\mathbb{R}^{2}\rightarrow\mathbb{R}^{2}$
        by $(x,y)\mapsto(x+m,y+n)$. This translates all points
        in $\mathbb{R}^{2}$ by the vector $(m,n)$. It suffices
        to define $\rho(g)$ on the generators $g$ of $G$.
        $\mathbb{Z}^{2}$ has the generators $(1,0)$ and $(0,1)$.
        So:
        \begin{subequations}
            \begin{align}
                \rho((1,0))(x,y)&=(x+1,y)\\
                \rho((0,1))(x,y)&=(x,y+1)
            \end{align}
        \end{subequations}
        We now need to chack that
        $\rho(a)\rho(b)=\rho(b)\rho(a)$.
    \end{example}
    \begin{example}
        Let $G$ be the group defined by the presentation:
        \begin{equation}
            G=\langle{a,b|ab=ba^{\minus{1}}}\rangle
        \end{equation}
        This acts on $\mathbb{R}^{2}$ by:
        \begin{subequations}
            \begin{align}
                \rho((1,0))(x,y)&=(x+1,y)\\
                \rho((0,1))(x,y)&=(\minus{x},y+1)
            \end{align}
        \end{subequations}
        We now need to check:
        \begin{equation}
            \rho(a)\big(\rho(b)(x,y)\big)=
            \rho(a)(\minus{x},y+1)=
            (\minus{x}+1,y+1)
        \end{equation}
        And also:
        \begin{equation}
            \rho(b)\big(\rho(a^{\minus{1}})(x,y)\big)=
            \rho(b)(x-1,y)=(\minus{x}+1,y+1)
        \end{equation}
        And thus this is a valid Group action.
    \end{example}
    Given a group $G$ and a set $X$, we consider the
    \textit{orbit} space $G/X$ as a set. The elements are
    called the orbits. Given a topology on $X$, we take
    the topology on the orbit space to be the quotient
    topology.
    \begin{example}
        Taking the group action discussed earlier of
        $\mathbb{Z}^{2}$ over $\mathbb{R}^{2}$, the orbit
        space $\mathbb{R}^{2}/\mathbb{Z}^{2}$. and this is
        homeomorphic to $T^{2}$. In the other example, with
        $G=\langle{a,b|ab=ba^{\minus{1}}}\rangle$,
        we have that the orbit space is
        $\mathbb{R}^{2}/G$ is the Klein bottle.
    \end{example}
    In general, if $\pi_{1}(X,x_{0})=G$, then $G$ acts on
    the covering space
    $p:\tilde{X}\rightarrow{X}$ with $\pi_{1}(X)=0$. In
    particular, if $X$ is a CW complex.
    \begin{example}
        Let $p:\mathbb{R}\rightarrow{S}^{1}$ be defined by
        $p(t)=\exp(2\pi{i}t)$. We know that
        $\pi_{1}(S)=\mathbb{Z}$. The generator of
        $\mathbb{Z}$ acts by mapping
        $x\mapsto{x}+1$. Note that:
        \begin{equation}
            \mathbb{R}/\pi_{1}(S^{1})\simeq
            \mathbb{R}/\mathbb{Z}\simeq
            S^{1}
        \end{equation}
        This is, in general, true. COnsider the Klein bottle
        The universal cover of the Klein bottle is
        $\mathbb{R}^{2}$. 
    \end{example}
    \begin{theorem}
        If $X$ is a topological space that is path connected
        and a CW complex, then for every subgroup
        $H\subseteq\pi_{1}(X)$, there exists a cover
        $p:\tilde{X}_{H}\rightarrow{X}$ such that:
        \begin{equation}
            p_{*}(\pi_{1}(\tilde{X}_{H},\tilde{x}_{0}))=H
        \end{equation}
        For suitable choice of $\tilde{x}_{0}$.
    \end{theorem}
    \begin{proof}
        Let $\tilde{X}\rightarrow{X}$ be a simply
        connected cover. Since $\pi_{1}(X,x_{0})=G$ acts on
        $\tilde{X}$, $H$ also acts on $\tilde{X}$. Now the
        quotient space $\tilde{X}/H$ has the desired
        properties.
    \end{proof}
    \subsection{Lifting Criterion}
        Given a space $X$ and a covering space
        $\tilde{X}$ with covering
        $p:\tilde{X}\rightarrow{X}$, and given a space
        $Y$ with a map $f:Y\rightarrow{X}$, when is there a
        \textit{lift} of the map $f$ to the covering
        space $\tilde{X}$? it seems natural to impose certain
        criterion on $Y$, and we require that $Y$ is
        path connected and locally path connected.
        1.33 in Hatcher.
        \begin{theorem}
            If $p:\tilde{X}\rightarrow{X}$ is a covering space,
            if $f:Y\rightarrow{X}$ is continuous, then a
            lift $\tilde{f}$ exists if and only if
            $f_{*}:\pi_{1}(Y,y_{0})\rightarrow\pi_{1}(X,x_{0})$
            is such that
            $\textrm{Ran}(f_{*})\subseteq\textrm{Ran}(p_{*})$,
            where $\textrm{Ran}$ denotes that range of the
            functions.
        \end{theorem}
        \begin{proof}
            If $\tilde{f}$ exists, then
            $f=\tilde{f}\circ{p}$, and thus
            $f_{*}=p_{*}\circ{f}_{*}$, so the range of
            $f_{*}$ is contained within the range of
            $p_{*}$. The hard part is going the other way.
            Now we need to construct such a lift.
        \end{proof}
\section{Homology}
    \subsection{History}
        The basic root of the theory stems from the
        Euler number of a graph in the plane. That is,
        from his classic formula:
        \begin{align}
            F-E+V&=1
            \tag{Planar Graphs}\\
            F-E+V&=2
            \tag{Platonic Solids}
        \end{align}
        Similarly, if one triangulates a sphere, we again
        obtain $F-E+V=2$. So there's nothing to do with the
        nature of the platonic solids here, but rather that
        all of these objects are homeomorphic to the
        unit sphere. What about the torus? There's a formula
        for any surface with genus $g$, and we obtain:
        \begin{equation}
            F-E+V=2-2g
        \end{equation}
        So, for a torus we get zero. Riemann furthered the
        theory in the 1850's when he introduced the notion of
        connectivity number. Betti and Riemann continue this
        work around 1870 and introduced the idea of Betti
        numbers. This is the number of cuts of dimension $k$
        are needed to disconnect a space. For a genus
        two surface, we see that the first Betti number is
        four, since we need two cuts for each hole. Moving on,
        Emmy Noether added more to the theory in 1925
        when she related this to groups. For example the
        Betti number 4 somehow relates to the group
        $\mathbb{Z}^{4}$, with Torsion. The first modern
        definition of homology groups comes from Mayer and
        Vietoris in 1928. Between 1930 and 1950, the theory
        began to develop into its current form. It now extends
        beyond topology and into algebra and group theory.
        There is also the notion of cohomology that came along
        with this.
    \subsection{Singular Homology}
        \begin{ldefinition}{Simplex}
            A simplex of degree $n$ is a subset of
            $\mathbb{R}^{n}$ form by $n$ points
            that are affinely independent
            $v_{0},\dots,v_{N}$, defined by the
            convex hull:
            \begin{equation}
                [v_{0},\dots,v_{n}]=
                \{\sum_{k=0}^{n}t_{k}v_{k}:
                    \sum_{k=0}^{n}t_{k}=1,t_{k}\geq{0}\}
            \end{equation}
        \end{ldefinition}
        The standard simplex, denote $\Delta^{n}$, is the
        convex hull of $e_{1},\dots,e_{n+1}$. These are
        triangles, or tetrahedron's, etc.
        Every n-simplex with ordered vertices
        $[v_{0},v_{1},\dots,v_{n}]$ is canonically homeomorphic
        to $\Delta^{n}$. Take
        $t=(t_{0},t_{1},\dots,t_{n})$ and map this to
        $t_{0}v_{0}+\dots+t_{n}v_{n}$. In particular, each
        of the faces of an $n$ simplex comes with a map:
        \begin{equation}
            [v_{0},\dots,v_{j},\dots,v_{n}]
            \mapsto[v_{0},\dots,v_{j-1},v_{j+1},\dots,v_{n}]
        \end{equation}
        This is equivalent to the face mapping
        $\Delta^{n-1}\mapsto\Delta^{n}$.
        \subsubsection{Simplicial Homology}
            Defined for $\Delta$ complexs.
            \begin{ldefinition}{$\Delta$ Complex}
                A topological space $X$ together with the
                following structure:
                \begin{enumerate}
                    \item A collection of continuous
                          maps $\sigma:\Delta^{n}\rightarrow{X}$.
                    \item $\sigma(\Delta^{n}\setminus%
                           \partial\Delta^{n})=e_{\alpha}^{n}$
                    \item The topology of $X$ is a CW topology.
                          That is, $A\subseteq{X}$ is open
                          if and only if
                          $\sigma_{\alpha}^{\minus{1}}(A)$ is
                          open in $\Delta^{n}$ for all
                          $\alpha\in{J}$.
                    \item For $\sigma_{\alpha}:%
                          \Delta^{n}\rightarrow{X}$ in the
                          structure, so is
                          $\sigma_{\alpha}\circ{F}_{j}^{n}$,
                          where $F_{j}^{n}$ is the face map.
                \end{enumerate}
            \end{ldefinition}
            \begin{ldefinition}{Chain Complex}
                A sequence of a Abelian groups
                $C_{n}$, one for each $n\in\mathbb{Z}$,
                together with group homomorphisms
                $\partial_{n}:C_{n}\rightarrow{C}_{n-1}$.
                Such that $\partial_{n-1}\circ\partial_{n}=0$. 
                Equivalently,
                $\mathrm{Im}(\partial_{n})\subseteq%
                 \mathrm{ker}(\partial_{n-1})$.
            \end{ldefinition}
            We can also let:
            \begin{equation}
                C_{\cdot}=\oplus_{n\in\mathbb{Z}}C_{n}
            \end{equation}
            With a homomorphism:
            \begin{equation}
                \partial:C_{\cdot}\rightarrow{C}_{\cdot}
            \end{equation}
            That is a degree -1 homomorphism such that
            $\partial^{2}=0$. The homology of a chain
            complex is the collection of groups:
            \begin{align}
                Z_{n}&=\mathrm{ker}(\partial_{n})
                    \subseteq{C}_{n}\\
                B_{n}&=\mathrm{Im}(\partial_{n+1})
                    \subseteq{C}_{n}\\
                H_{n}&=Z_{n}/B_{n}
            \end{align}
        \subsubsection{Simplicial Homology}
            Let $X$ be a $\Delta$ complex. That is, the
            disjoint union of open simplices. We want to create
            a chain complex:
            \begin{equation}
                \cdots\longrightarrow\Delta_{n}(X)
                \overset{\partial_{n}}{\longrightarrow}
                \Delta_{n-1}(X)
                \overset{\partial_{n-1}}{\longrightarrow}
                \cdots
                \overset{\partial_{1}}{\longrightarrow}
                \Delta_{0}(X)
                \longrightarrow{0}
            \end{equation}
            Form a free abelian group generated by
            $\rho_{\alpha}^{n}$.
            \begin{equation}
                \partial(\sigma)=
                \sum(\minus{1})^{j}\sigma|
                [v_{0},\dots,v_{j-1},v_{j+1},\dots,v_{n}]
            \end{equation}
            The $(\minus{1})^{j}$ term gives an orientation
            to simplices.
            \begin{theorem}
                $\partial_{n-1}\circ\partial_{n}=0$.
            \end{theorem}
        \subsection{Singular Homology}
            Let $X$ be a topological space. A
            singular $n$ simplex in $X$ is a continuous map
            $\sigma:\Delta^{n}\rightarrow{X}$.
            \begin{theorem}
                If $X$ has $k$ path-connected components, then
                $H_{0}(X)=\mathbb{Z}^{k}$.
            \end{theorem}
            \begin{theorem}
                If $X$ is the union f $X_{\alpha}$, where
                $X_{\alpha}$ is path connected, thn:
                \begin{equation}
                    H_{n}(X)=\oplus{H}_{n}(X_{\alpha})
                \end{equation}
            \end{theorem}
            \begin{theorem}
                If $X$ is path connected, then
                $H_{0}(X)=\mathbb{Z}$.
            \end{theorem}
            What is an $n$ cycle? The abstract definition
            is some finite combination of singular simplices:
            $\sigma:\Delta^{n}\rightarrow{X}$, allowing
            repetitions.
\section{More Homology}
    \begin{theorem}
        \begin{equation}
            H_{n}(S^{k})=
            \begin{cases}
                \mathbb{Z},&n=k,n=0\\
                0,&\textrm{Otherwise}
            \end{cases}
        \end{equation}
    \end{theorem}
    \begin{proof}
        We'll need to use the properties (Axioms) of singular
        homology. Let $X=D^{k}$ be the closed $k$ disk in
        $\mathbb{R}^{k}$. Let $A=\partial{D}^{k}=S^{k-1}$. Then
        $X.A\simeq{S}^{k}$. For CW pairs we have a long exact sequence
        as well as strong excision:
        \begin{equation}
            \cdots\longrightarrow
            \overset{A}{H_{n}(S^{k-1})}\longrightarrow
            \overset{X}{H_{n}(D^{k})}\longrightarrow
            \overset{X/A}{\tilde{H}_{n}(S^{k})}\longrightarrow
            H_{n-1}(S^{k-1})\longrightarrow\cdots
        \end{equation}
        FOr $n\geq{2}$, we have:
        \begin{equation}
            0\longrightarrow{H}_{n}(S^{k})
            \overset{\partial}{\longrightarrow}H_{n-1}(S^{k-1})
            \longrightarrow{0}
        \end{equation}
        The end of the sequence is then:
        \begin{equation}
            H_{1}(D^{k})\longrightarrow
            H_{1}(S^{k})\overset{\partial}{\longrightarrow}
            H_{0}(S^{k-1})\longrightarrow{H}_{0}(D^{k})
            \longrightarrow\tilde{H}_{0}(S^{k})\longrightarrow{0}
        \end{equation}
        But $H_{1}(D^{k})=0$, and so this reduces. When $k\geq{2}$
        we also have that $\tilde{H}_{0}(S^{k})=0$. So, for the case
        of $k\geq{2}$ we have:
        \begin{equation}
            0\longrightarrow{H}_{1}(S^{k})\longrightarrow
            H_{0}^{k-1}\longrightarrow{H}_{0}(D^{k})\longrightarrow{0}
        \end{equation}
        And this reduces to:
        \begin{equation}
            0\longrightarrow0\longrightarrow0\longrightarrow
            \mathbb{Z}\longrightarrow\mathbb{Z}\longrightarrow0
        \end{equation}
        The next case is $k=1$. We obtain the short exact sequence:
        \begin{equation}
            0\longrightarrow{H}_{1}(S^{1})\longrightarrow
            \mathbb{Z}^{2}\longrightarrow\mathbb{Z}\longrightarrow0
        \end{equation}
        From this we obtain $H_{1}(S^{1})=\mathbb{Z}$. Using induction,
        if $n>k\geq{1}$, then $H_{n}(S^{k})\simeq{H}_{1}(S^{k-n+1})$,
        and this is the trivial group. If $k=n\geq{1}$, then
        $H_{n}(S^{k})\simeq{H}_{1}(S^{1})=\mathbb{Z}$. Similarly for
        $k<n$.
    \end{proof}
    \begin{ltheorem}{Milnor's Uniqueness Theorem}{Milnor_Unqueness}
        If:
        \begin{equation}
            H_{n}(\coprod_{\alpha}X_{\alpha})
            =\oplus_{\alpha}H_{n}(X_{\alpha})
        \end{equation}
        Then all homology theories on the category of CW complexes
        are equivalent.
    \end{ltheorem}
    \begin{theorem}
        $H_{n}^{\Delta}(X)\simeq{H}_{n}(X)$.
    \end{theorem}
    \begin{equation}
        H_{n}^{\Delta}(X^{k},X^{k-1})=
        \begin{cases}
            \textrm{Free Abelian Group Generated by n-cells},&n=k\\
            0,&\textrm{Otherwise}
        \end{cases}
    \end{equation}
    Induction on the dimension of the skeleton. If $k=0$, and if
    $n>0$, then $H_{n}^{\Delta}(X^{0})=0$ and $H_{n}(X^{0})=0$. If
    $n=0$, then we obtain $\mathbb{Z}^{\oplus{c}}$, where $c$ is the
    number of connected components. The Five Lemma handles this.
\section{Mayer-Vietoris Sequence}
    Mayer-Vietoris sequences are useful computational tools for
    computing homology. It is analogous to the van Kampen theorem
    for homotopy. Often used in proofs by induction on the
    number of dimensions.
    \begin{ltheorem}{Mayer-Vietoris Theorem}
        Given a space $X$ and two subsets $A,b\subseteq{X}$
        such that:
        \begin{equation}
            X=\mathrm{Int}(A)\cup\mathrm{Int}(B)
        \end{equation}
        Then the Mayer-Vietoris sequence:
        \begin{equation}
            \cdots\rightarrow{H}_{n}(A\cap{B})
            \overset{\varphi}{\rightarrow}{H}_{n}(A)
                \oplus{H}_{n}(B)\overset{\psi}{\rightarrow}
                H_{n}(X)\rightarrow{H}_{n-1}(A\cap{B})
                \rightarrow\cdots
        \end{equation}
        Is an exact sequence.
    \end{ltheorem}
    The maps are $\phi(x)=(\minus{x},x)$, and $\psi(x,y)=x+y$.
    $\varphi:C_{n}(A\cap{B})\rightarrow{C}_{n}(A)\oplus{C}_{n}(B)$
    and $\psi:C_{n}(A)\oplus{C}_{n}(B)\rightarrow{C}_{n}(X)$. To
    define $\partial$ connecting maps, we have a short exact
    sequence of chain complexes:
    \begin{equation}
        0\rightarrow{C}_{n}(A\cap{B})
        \overset{\varphi}{\rightarrow}C_{n}(A)\oplus{C}_{n}(B)
        \overset{\psi}{\rightarrow}C_{n}(A+B)\rightarrow{0}
    \end{equation}
    $n$ chains in $X$ such that each simplex
    $\sigma:\Delta^{n}\rightarrow{X}$ is entirely in $A$ or
    in $B$, or both. The proof is similar in technical details
    to excision.
    \begin{theorem}
        If $(X,A)$ is a CW pair, then $A$ has a neighborhood
        $A\subseteq\mathcal{U}\subseteq{X}$ such that $A$ is
        a deformation retract of $\mathcal{U}$.
    \end{theorem}
    \begin{example}
        What is $H_{n}(S^{2}\times{S}^{1})$? Write $S^{2}$ as
        $S_{+}^{2}\cup{S}_{\minus}^{2}$, where
        $S_{\pm}^{2}\simeq{D}^{2}$. Let $A=S_{+}^{2}\times{S}^{1}$
        and $B=S_{\minus}^{2}\times{S}^{1}$. Then $A$ and
        $B$ are solid torii, and $A\cap{B}=S^{1}\times{S}^{1}$,
        which is a torus. If you have a closed, smooth manifold,
        then the highest degree that has homology is the degree
        of the manifold. So, we can start the Mayer-Vietoris
        sequence at 3. But $A\cap{B}$ is a torus, and thus
        $H_{3}(A\cap{B})=0$. Similarly, $H_{3}(A)=H_{3}(B)=0$.
        So, we have:
        \begin{equation}
            0\rightarrow{H}_{3}(X)\rightarrow
            H_{2}(S^{1}\times{S}^{1})\rightarrow{H}_{2}(S^{1})
            \oplus{H}_{2}(S^{1})
        \end{equation}
        But $H_{2}(S^{1}\times{S}^{1})=\mathbb{Z}$, and
        $H_{2}(S^{1})=0$, so we obtain:
        \begin{equation}
            0\rightarrow{H}_{3}(X)\rightarrow\mathbb{Z}
            \rightarrow{0}\rightarrow{H}_{2}(X)\rightarrow\cdots
        \end{equation}
        Now $H_{1}(S^{1}\times{S}^{1})=\mathbb{Z}^{2}$, so we
        can simplify further to obtain:
        \begin{equation}
            0\rightarrow{H}_{2}(X)\rightarrow\mathbb{Z}^{2}
            \rightarrow{H}_{1}(S^{1})\oplus{H}_{1}(S^{1})
            \rightarrow{H}_{1}(X)
        \end{equation}
        Thus, $H_{0}(X)=\mathbb{Z}$. Now for $H_{1}$ and $H_{2}$:
        \begin{equation}
            0\rightarrow{H}_{2}(X)\rightarrow\mathbb{Z}^{2}
            \rightarrow\mathbb{Z}\oplus\mathbb{Z}\rightarrow
            H_{1}(X)\rightarrow{0}
        \end{equation}
        We get $H_{1}(X)=H_{2}(X)=\mathbb{Z}$, and thus
        $S^{2}\times{S}^{1}\ne{S}^{3}$.
    \end{example}
\section{Cohomology}
    Let $G$ be an Abelian group, and let $\mathrm{Hom}$ be a
    functor from Abelian groups to Abelian groups:
    $\mathrm{Hom}(A,G)$ is the set of group homomorphism.
    This is again a group, $(f_{1}+f_{2})(a)=f_{1}(a)+f_{2}(a)$.
    \begin{theorem}
        If $A$, $B$, and $G$ are Abelian groups, then:
        \begin{equation}
            \mathrm{Hom}(A\oplus{B},G)\simeq
            \mathrm{Hom}(A,G)\oplus\mathrm{Hom}(B,G)
        \end{equation}
    \end{theorem}
    \begin{theorem}
        If $G$ is an Abelian group, then
        $\mathrm{Hom}(\mathbb{Z},G)\simeq{G}$.
    \end{theorem}
    \begin{proof}
        For let $f\mapsto{f}(1)\in{G}$.
    \end{proof}
    \begin{theorem}
        If $G$ is an Abelian group, and if $m\in\mathbb{N}$,
        then:
        \begin{equation}
            \mathrm{Hom}(Z^{m},G)\simeq{G}^{m}
        \end{equation}
    \end{theorem}
    $\mathrm{Hom}$ is a functorial contravariant.
    The functorial properties are:
    \begin{subequations}
        \begin{align}
            \mathrm{id}^{*}&=\mathrm{id}\\
            (\varphi\psi)^{*}&=\psi^{*}\varphi^{*}\\
            0^{*}&=0
        \end{align}
    \end{subequations}
    Given any chain complex of Abelian groups with the boundary
    map $\partial^{2}=0$ (Simplicial, singular, cellular, etc.),
    applying the $\mathrm{Hom}$ functor, we get:
    \begin{equation}
        \cdots\leftarrow
        \mathrm{Hom}(C_{n},G)\leftarrow\mathrm{Hom}(C_{n-1},G)
        \leftarrow\mathrm{Hom}(C_{n-2},G)\leftarrow\cdots
    \end{equation}
    With maps $\delta$ called the \textit{coboundary}.
    I $\varphi\in{C}_{n}^{*}=\mathrm{Hom}(C_{n},G)$,
    $\varphi:C_{n}\rightarrow{G}$, then
    $\delta\varphi\in{C}_{n+1}^{*}$.
    $\delta\varphi=\varphi\delta$. The cohomology groups are:
    \begin{equation}
        H^{n}(X,G)=\mathrm{ker}(\delta)/\mathrm{Im}(\delta)
    \end{equation}
\section{Len's Spaces}
    Take $\mathbb{C}^{n}$, and consider the unit
    sphere $S^{2n-1}\subseteq\mathbb{C}^{n}$. Define
    the map $p$ on $\mathbb{C}^{n}$ by
    $(z_{1},\dots,z_{n})\mapsto%
     (\zeta^{j_{1}}z_{1},\dots,\zeta^{j_{n}}z_{n})$
    Where $\zeta=\exp(2\pi{i}/m$, and
    $0<j_{i}<m$ for all $i$, and furthermore
    $j$ and $m$ are coprime. The lens space is
    $S^{2n-1}/\mathbb{Z}_{m}$, and is denoted
    $L_{m}(j_{1},\dots,j_{n})$. This is the first
    example of two closed manifolds without
    boundary, of the same dimension, that are
    homotopy equivalent, but not homeomorphic.
    For $L_{7}(1,1)$ is homotopy equivalent to
    $L_{7}(1,2)$, but not homeomorphic. The
    homology groups are
    $\mathbb{Z},\mathbb{Z}_{m},0,\mathbb{Z}$, and
    then $0$ for all others. The cohomology is
    $\mathrm{Hom}(G,\mathbb{Z})$, where $G$ are
    the groups from the chain complex from homology.
    We obtain the same sequence, but in reverse.
    Next is the universal coefficient theorem. We
    can compute $H^{n}(X;G)$ from
    $H(X)$ (Singular homology).
    \begin{theorem}
        If $H_{n}(X)\simeq\mathbb{Z}^{m}\oplus{T}_{n}$
        where $T_{n}$ is a torsion group, then:
        \begin{equation}
            H^{n}(X;\mathbb{Z})\simeq
            \mathbb{Z}^{m}\oplus{T}_{n-1}
        \end{equation}
    \end{theorem}
    Taking $\mathbb{R}P^{2}$ as an example, the
    homology groups are
    $\mathbb{Z},\mathbb{Z}_{2},0,0,\dots$. Thus, using
    this theorem, the cohomology is
    $\mathbb{Z},0,\mathbb{Z}_{2},0,0,\dots$. As
    another case, if $G=\mathbb{R}$ or $\mathbb{G}$,
    then:
    \begin{equation}
        H_{n}(X;\mathbb{R})\simeq\mathbb{R}^{m}
    \end{equation}
    And also:
    \begin{equation}
        H^{n}(X;\mathbb{R})\simeq\mathbb{R}^{m}
    \end{equation}
    These are dual vector spaces. From UTC, we have:
    \begin{equation}
        H^{1}(X;G)\simeq
        \mathrm{Hom}(H_{1}(X);G)
    \end{equation}
    If $X$ is path-connected, then $H_{1}(X)$ is the
    Abelianization of $\pi_{1}(X)$.
    \begin{ltheorem}{Thom's Theorem}{Thom_Theorem}
        Every $n$ cycle $\xi\in{H}_{n}(X)$ for a
        smooth manifold $X$ is realizable as a
        triangulated oriented submanfiold $M$ of
        dimension $n$, kinda.
    \end{ltheorem}
    This theorem leads to the following idea:
    To define cocycles, we need a procedure that
    assigns elements in $G$ to every $n$ cycle in
    $X$. If $G=\mathbb{R}$, then the integer $k$ from
    Thom's theorem becomes irrelevant. Thom's theorem
    says that it suffices to assign an element in
    $\mathbb{R}$ to every $n$ dimensional submanifold.
    A differential $n$ form $\alpha$ on $X$,
    $X\mapsto{TX}$, where $TX$ is the tangent bundle,
    and $TX\mapsto{T^{*}}X$, where $T^{*}X$ is the
    tangent co-bundle. 
\section{Cohomology Rings}
    Recall that $H^{\cdot}(X;)$, where $R$ is the coefficient
    ring ($\mathbb{Z},$ $\mathbb{Z}_{n}$, $\mathbb{Q}$,
    $\mathbb{R}$, $\mathbb{C}$, etc.), is the direct product:
    \begin{equation}
        H^{\cdot}(X;R)=\underset{n\in\mathbb{Z}}{\oplus}
            H^{n(X;R)}
    \end{equation}
    This is the cup product. It is distributive and associative,
    and forms a graded ring. The unit, 1, is the class
    in $\mathrm{Hom}(C_{0}(X);R)$ such that $1(x)=1_{R}$, for
    all $x\in{X}$. There is also a contravariant functor
    from topological spaces to unital graded rings.
    \begin{theorem}
        If $f:X\rightarrow{Y}$ is continuous, then there is a
        pullback
        $f^{*}:H^{\cdot}(Y,R)\rightarrow{H}^{\cdot}(X,R)$
        which is a ring homomorphism for these unital graded
        rings.
    \end{theorem}
    The proof comes directly from the various definitions
    involved in the theorem. That is, there are no surprises in
    the proof. The implications are quite strong, however, and
    limit the possibilities.
    \begin{theorem}
        If $\alpha\in{H}^{}(X,R)$ and if
        $\beta\in{H}^{\ell}(X,R)$, if $R$ is a commutative ring,
        then the cohomology ring is graded commutative, that is
        $\alpha\cup\beta=(\minus{1})^{k\ell}\beta\cup\alpha$,
        where $\cup$ is the cup product of $\alpha$ and $\beta$.
    \end{theorem}
    \begin{example}
        Let's compute $H^{\cdot}(\mathbb{T}^{2},\mathbb{Z})$,
        where $\mathbb{T}^{2}=S^{1}\times{S}^{1}$. We'll use
        simplicial homology to do this. Using the planar
        representation, we get the following:
        \begin{subequations}
            \begin{align}
                \Delta_{0}&=\mathbb{Z}
                    =\langle{v}\rangle\\
                \Delta_{1}&=\mathbb{Z}^{3}
                    =\langle{a,b,c}\rangle\\
                \Delta_{2}&=\mathbb{Z}^{2}
                    =\langle{u,L}\rangle
            \end{align}
        \end{subequations}
        After magic, we get:
        \begin{subequations}
            \begin{align}
                H^{0}&=\mathbb{Z}
                    =\langle{v^{*}}\rangle\\
                H^{1}&=\mathbb{Z}^{2}
                    =\langle{a^{*}+c^{*},b^{*}+c^{*}}\rangle\\
                H^{2}&=\mathbb{Z}
                    =\langle{u^{*}}\rangle
            \end{align}
        \end{subequations}
        WHat is the ring structure? So,
        $H^{\cdot}(\mathbb{T}^{2},\mathbb{Z})$ is generated by
        four elements, and thus we have $\mathbb{Z}^{4}$.
    \end{example}
    \begin{example}
        Consider $H^{\cdot}(\mathbb{RP}^{n},\mathbb{Z}_{2})$.
        We can compute the groups via cellular chain complexes.
        The groups $H^{k}(\mathbb{RP}^{n},\mathbb{Z}_{2})$ are
        isomorphic to $\mathbb{Z}_{2}$ for $k=0,\dots,n$.
    \end{example}
    \begin{theorem}[Hatcher, Page 212]
        $H^{\cdot}(\mathbb{RP}^{n},\mathbb{Z}_{2})$, as a
        graded ring, has one generator $x$ and is isomorphic
        to:
        \begin{equation}
            H^{\cdot}(\mathbb{RP}^{n},\mathbb{Z}_{2})
            \simeq\mathbb{Z}_{2}[x]/\langle{x^{n+1}}\rangle
        \end{equation}
        Where $x$ has degree 1, $|x|=1$.
    \end{theorem}
\section{Cap Product}
    The cap product of a ring $R$, the coefficient ring,
    is a thing. $C_{k}(X;R)$ is the singular $k$ chains with
    coefficients in $R$. $C^{\ell}(X;R)$ is the dual complex
    $\mathrm{Hom}(C_{\ell}(X;R)$. The cross product is mapped
    to $C_{k-\ell}(X;R)$ by the cap product. Take a generator
    $\sigma:\Delta^{K}\rightarrow{X}$ and
    $\varphi:C_{\ell}\rightarrow{R}$, and cap them:
    $(\sigma\cap\varphi$. We should then get an element of
    $C_{k-\ell}(X;R)$. We define this by:
    \begin{equation}
        \sigma\cap\varphi
        =\varphi(\sigma|[v_{0},\dots,v_{\ell}])
            \sigma|[v_{\ell},\dots,v_{k}]
    \end{equation}
    The image of $\varphi$ is an element of $R$, and is simply
    the coefficient. The restriction of $\sigma$ gives another
    map $\Delta^{k-\ell}\rightarrow{X}$. The cap product is well
    defined for cohomology and homology, so the is a map:
    \begin{equation}
        H_{k}(X;R)\times{H}^{\ell}(X;R)\rightarrow
            H_{k-\ell}(X;R)
    \end{equation}
    Moreover, if $\xi$ is a homology element and $\alpha$ and
    $\beta$ are cohomology elements, then:
    \begin{equation}
        (\xi\cap\alpha)\cap\beta=\xi\cap(\alpha\cup\beta)
    \end{equation}
    In other words:
    \begin{equation}
        H_{\cdot}(X;R)=\underset{j\in\mathbb{Z}}{\oplus}
            H_{j}(X;R)
    \end{equation}
    Which is a graded module over the graded cohomology ring.
    The same is true in two other theories. The theory of
    bordism and cobordism, as well as in K-Theory
    (K-Homology and H-Theory).
    \begin{ltheorem}{Poincar\'{e} Duality}{Poincare_Duality}
        If $X$ is a closed and oriented manifold that is
        compact and without boundary,
        $[X]\in{H}_{n}(X;\mathbb{R})$ (Fundamental cycle of $X$),
        then homology is a free module over cohomology, with
        generator $[X]$.
    \end{ltheorem}
    If $R$ is a ring, a free module that is finitely generated
    is just copies of $R$: $R^{n}=R\oplus\dots\oplus{R}$. As
    modules, we have an isomorphism:
    $H_{\cdot}(X;\mathbb{R})\simeq{H}^{\cdot}(X;\mathbb{R})$,
    where have the map $\alpha\mapsto[X]\cap\alpha$. Also, from
    the universal coefficient theorem,
    $H^{k}(X;\mathbb{R})\simeq{H}_{k}(X;\mathbb{R})$, and
    $H_{n-k}\simeq{H}_{k}$, ignoring torsion. This can be
    translated to Betti numbers as follows:
    $\beta_{n-k}=\beta_{k}$.