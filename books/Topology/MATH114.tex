%------------------------------------------------------------------------------%
\documentclass[oneside]{book}                                                  %
%------------------------------Preamble----------------------------------------%
\makeatletter                                                                  %
    \def\input@path{{../../}}                                                  %
\makeatother                                                                   %
%---------------------------Packages----------------------------%
\usepackage{geometry}
\geometry{b5paper, margin=1.0in}
\usepackage[T1]{fontenc}
\usepackage{graphicx, float}            % Graphics/Images.
\usepackage{natbib}                     % For bibliographies.
\bibliographystyle{agsm}                % Bibliography style.
\usepackage[french, english]{babel}     % Language typesetting.
\usepackage[dvipsnames]{xcolor}         % Color names.
\usepackage{listings}                   % Verbatim-Like Tools.
\usepackage{mathtools, esint, mathrsfs} % amsmath and integrals.
\usepackage{amsthm, amsfonts, amssymb}  % Fonts and theorems.
\usepackage{tcolorbox}                  % Frames around theorems.
\usepackage{upgreek}                    % Non-Italic Greek.
\usepackage{fmtcount, etoolbox}         % For the \book{} command.
\usepackage[newparttoc]{titlesec}       % Formatting chapter, etc.
\usepackage{titletoc}                   % Allows \book in toc.
\usepackage[nottoc]{tocbibind}          % Bibliography in toc.
\usepackage[titles]{tocloft}            % ToC formatting.
\usepackage{pgfplots, tikz}             % Drawing/graphing tools.
\usepackage{imakeidx}                   % Used for index.
\usetikzlibrary{
    calc,                   % Calculating right angles and more.
    angles,                 % Drawing angles within triangles.
    arrows.meta,            % Latex and Stealth arrows.
    quotes,                 % Adding labels to angles.
    positioning,            % Relative positioning of nodes.
    decorations.markings,   % Adding arrows in the middle of a line.
    patterns,
    arrows
}                                       % Libraries for tikz.
\pgfplotsset{compat=1.9}                % Version of pgfplots.
\usepackage[font=scriptsize,
            labelformat=simple,
            labelsep=colon]{subcaption} % Subfigure captions.
\usepackage[font={scriptsize},
            hypcap=true,
            labelsep=colon]{caption}    % Figure captions.
\usepackage[pdftex,
            pdfauthor={Ryan Maguire},
            pdftitle={Mathematics and Physics},
            pdfsubject={Mathematics, Physics, Science},
            pdfkeywords={Mathematics, Physics, Computer Science, Biology},
            pdfproducer={LaTeX},
            pdfcreator={pdflatex}]{hyperref}
\hypersetup{
    colorlinks=true,
    linkcolor=blue,
    filecolor=magenta,
    urlcolor=Cerulean,
    citecolor=SkyBlue
}                           % Colors for hyperref.
\usepackage[toc,acronym,nogroupskip,nopostdot]{glossaries}
\usepackage{glossary-mcols}
%------------------------Theorem Styles-------------------------%
\theoremstyle{plain}
\newtheorem{theorem}{Theorem}[section]

% Define theorem style for default spacing and normal font.
\newtheoremstyle{normal}
    {\topsep}               % Amount of space above the theorem.
    {\topsep}               % Amount of space below the theorem.
    {}                      % Font used for body of theorem.
    {}                      % Measure of space to indent.
    {\bfseries}             % Font of the header of the theorem.
    {}                      % Punctuation between head and body.
    {.5em}                  % Space after theorem head.
    {}

% Italic header environment.
\newtheoremstyle{thmit}{\topsep}{\topsep}{}{}{\itshape}{}{0.5em}{}

% Define environments with italic headers.
\theoremstyle{thmit}
\newtheorem*{solution}{Solution}

% Define default environments.
\theoremstyle{normal}
\newtheorem{example}{Example}[section]
\newtheorem{definition}{Definition}[section]
\newtheorem{problem}{Problem}[section]

% Define framed environment.
\tcbuselibrary{most}
\newtcbtheorem[use counter*=theorem]{ftheorem}{Theorem}{%
    before=\par\vspace{2ex},
    boxsep=0.5\topsep,
    after=\par\vspace{2ex},
    colback=green!5,
    colframe=green!35!black,
    fonttitle=\bfseries\upshape%
}{thm}

\newtcbtheorem[auto counter, number within=section]{faxiom}{Axiom}{%
    before=\par\vspace{2ex},
    boxsep=0.5\topsep,
    after=\par\vspace{2ex},
    colback=Apricot!5,
    colframe=Apricot!35!black,
    fonttitle=\bfseries\upshape%
}{ax}

\newtcbtheorem[use counter*=definition]{fdefinition}{Definition}{%
    before=\par\vspace{2ex},
    boxsep=0.5\topsep,
    after=\par\vspace{2ex},
    colback=blue!5!white,
    colframe=blue!75!black,
    fonttitle=\bfseries\upshape%
}{def}

\newtcbtheorem[use counter*=example]{fexample}{Example}{%
    before=\par\vspace{2ex},
    boxsep=0.5\topsep,
    after=\par\vspace{2ex},
    colback=red!5!white,
    colframe=red!75!black,
    fonttitle=\bfseries\upshape%
}{ex}

\newtcbtheorem[auto counter, number within=section]{fnotation}{Notation}{%
    before=\par\vspace{2ex},
    boxsep=0.5\topsep,
    after=\par\vspace{2ex},
    colback=SeaGreen!5!white,
    colframe=SeaGreen!75!black,
    fonttitle=\bfseries\upshape%
}{not}

\newtcbtheorem[use counter*=remark]{fremark}{Remark}{%
    fonttitle=\bfseries\upshape,
    colback=Goldenrod!5!white,
    colframe=Goldenrod!75!black}{ex}

\newenvironment{bproof}{\textit{Proof.}}{\hfill$\square$}
\tcolorboxenvironment{bproof}{%
    blanker,
    breakable,
    left=3mm,
    before skip=5pt,
    after skip=10pt,
    borderline west={0.6mm}{0pt}{green!80!black}
}

\AtEndEnvironment{lexample}{$\hfill\textcolor{red}{\blacksquare}$}
\newtcbtheorem[use counter*=example]{lexample}{Example}{%
    empty,
    title={Example~\theexample},
    boxed title style={%
        empty,
        size=minimal,
        toprule=2pt,
        top=0.5\topsep,
    },
    coltitle=red,
    fonttitle=\bfseries,
    parbox=false,
    boxsep=0pt,
    before=\par\vspace{2ex},
    left=0pt,
    right=0pt,
    top=3ex,
    bottom=1ex,
    before=\par\vspace{2ex},
    after=\par\vspace{2ex},
    breakable,
    pad at break*=0mm,
    vfill before first,
    overlay unbroken={%
        \draw[red, line width=2pt]
            ([yshift=-1.2ex]title.south-|frame.west) to
            ([yshift=-1.2ex]title.south-|frame.east);
        },
    overlay first={%
        \draw[red, line width=2pt]
            ([yshift=-1.2ex]title.south-|frame.west) to
            ([yshift=-1.2ex]title.south-|frame.east);
    },
}{ex}

\AtEndEnvironment{ldefinition}{$\hfill\textcolor{Blue}{\blacksquare}$}
\newtcbtheorem[use counter*=definition]{ldefinition}{Definition}{%
    empty,
    title={Definition~\thedefinition:~{#1}},
    boxed title style={%
        empty,
        size=minimal,
        toprule=2pt,
        top=0.5\topsep,
    },
    coltitle=Blue,
    fonttitle=\bfseries,
    parbox=false,
    boxsep=0pt,
    before=\par\vspace{2ex},
    left=0pt,
    right=0pt,
    top=3ex,
    bottom=0pt,
    before=\par\vspace{2ex},
    after=\par\vspace{1ex},
    breakable,
    pad at break*=0mm,
    vfill before first,
    overlay unbroken={%
        \draw[Blue, line width=2pt]
            ([yshift=-1.2ex]title.south-|frame.west) to
            ([yshift=-1.2ex]title.south-|frame.east);
        },
    overlay first={%
        \draw[Blue, line width=2pt]
            ([yshift=-1.2ex]title.south-|frame.west) to
            ([yshift=-1.2ex]title.south-|frame.east);
    },
}{def}

\AtEndEnvironment{ltheorem}{$\hfill\textcolor{Green}{\blacksquare}$}
\newtcbtheorem[use counter*=theorem]{ltheorem}{Theorem}{%
    empty,
    title={Theorem~\thetheorem:~{#1}},
    boxed title style={%
        empty,
        size=minimal,
        toprule=2pt,
        top=0.5\topsep,
    },
    coltitle=Green,
    fonttitle=\bfseries,
    parbox=false,
    boxsep=0pt,
    before=\par\vspace{2ex},
    left=0pt,
    right=0pt,
    top=3ex,
    bottom=-1.5ex,
    breakable,
    pad at break*=0mm,
    vfill before first,
    overlay unbroken={%
        \draw[Green, line width=2pt]
            ([yshift=-1.2ex]title.south-|frame.west) to
            ([yshift=-1.2ex]title.south-|frame.east);},
    overlay first={%
        \draw[Green, line width=2pt]
            ([yshift=-1.2ex]title.south-|frame.west) to
            ([yshift=-1.2ex]title.south-|frame.east);
    }
}{thm}

%--------------------Declared Math Operators--------------------%
\DeclareMathOperator{\adjoint}{adj}         % Adjoint.
\DeclareMathOperator{\Card}{Card}           % Cardinality.
\DeclareMathOperator{\curl}{curl}           % Curl.
\DeclareMathOperator{\diam}{diam}           % Diameter.
\DeclareMathOperator{\dist}{dist}           % Distance.
\DeclareMathOperator{\Div}{div}             % Divergence.
\DeclareMathOperator{\Erf}{Erf}             % Error Function.
\DeclareMathOperator{\Erfc}{Erfc}           % Complementary Error Function.
\DeclareMathOperator{\Ext}{Ext}             % Exterior.
\DeclareMathOperator{\GCD}{GCD}             % Greatest common denominator.
\DeclareMathOperator{\grad}{grad}           % Gradient
\DeclareMathOperator{\Ima}{Im}              % Image.
\DeclareMathOperator{\Int}{Int}             % Interior.
\DeclareMathOperator{\LC}{LC}               % Leading coefficient.
\DeclareMathOperator{\LCM}{LCM}             % Least common multiple.
\DeclareMathOperator{\LM}{LM}               % Leading monomial.
\DeclareMathOperator{\LT}{LT}               % Leading term.
\DeclareMathOperator{\Mod}{mod}             % Modulus.
\DeclareMathOperator{\Mon}{Mon}             % Monomial.
\DeclareMathOperator{\multideg}{mutlideg}   % Multi-Degree (Graphs).
\DeclareMathOperator{\nul}{nul}             % Null space of operator.
\DeclareMathOperator{\Ord}{Ord}             % Ordinal of ordered set.
\DeclareMathOperator{\Prin}{Prin}           % Principal value.
\DeclareMathOperator{\proj}{proj}           % Projection.
\DeclareMathOperator{\Refl}{Refl}           % Reflection operator.
\DeclareMathOperator{\rk}{rk}               % Rank of operator.
\DeclareMathOperator{\sgn}{sgn}             % Sign of a number.
\DeclareMathOperator{\sinc}{sinc}           % Sinc function.
\DeclareMathOperator{\Span}{Span}           % Span of a set.
\DeclareMathOperator{\Spec}{Spec}           % Spectrum.
\DeclareMathOperator{\supp}{supp}           % Support
\DeclareMathOperator{\Tr}{Tr}               % Trace of matrix.
%--------------------Declared Math Symbols--------------------%
\DeclareMathSymbol{\minus}{\mathbin}{AMSa}{"39} % Unary minus sign.
%------------------------New Commands---------------------------%
\DeclarePairedDelimiter\norm{\lVert}{\rVert}
\DeclarePairedDelimiter\ceil{\lceil}{\rceil}
\DeclarePairedDelimiter\floor{\lfloor}{\rfloor}
\newcommand*\diff{\mathop{}\!\mathrm{d}}
\newcommand*\Diff[1]{\mathop{}\!\mathrm{d^#1}}
\renewcommand*{\glstextformat}[1]{\textcolor{RoyalBlue}{#1}}
\renewcommand{\glsnamefont}[1]{\textbf{#1}}
\renewcommand\labelitemii{$\circ$}
\renewcommand\thesubfigure{%
    \arabic{chapter}.\arabic{figure}.\arabic{subfigure}}
\addto\captionsenglish{\renewcommand{\figurename}{Fig.}}
\numberwithin{equation}{section}

\renewcommand{\vector}[1]{\boldsymbol{\mathrm{#1}}}

\newcommand{\uvector}[1]{\boldsymbol{\hat{\mathrm{#1}}}}
\newcommand{\topspace}[2][]{(#2,\tau_{#1})}
\newcommand{\measurespace}[2][]{(#2,\varSigma_{#1},\mu_{#1})}
\newcommand{\measurablespace}[2][]{(#2,\varSigma_{#1})}
\newcommand{\manifold}[2][]{(#2,\tau_{#1},\mathcal{A}_{#1})}
\newcommand{\tanspace}[2]{T_{#1}{#2}}
\newcommand{\cotanspace}[2]{T_{#1}^{*}{#2}}
\newcommand{\Ckspace}[3][\mathbb{R}]{C^{#2}(#3,#1)}
\newcommand{\funcspace}[2][\mathbb{R}]{\mathcal{F}(#2,#1)}
\newcommand{\smoothvecf}[1]{\mathfrak{X}(#1)}
\newcommand{\smoothonef}[1]{\mathfrak{X}^{*}(#1)}
\newcommand{\bracket}[2]{[#1,#2]}

%------------------------Book Command---------------------------%
\makeatletter
\renewcommand\@pnumwidth{1cm}
\newcounter{book}
\renewcommand\thebook{\@Roman\c@book}
\newcommand\book{%
    \if@openright
        \cleardoublepage
    \else
        \clearpage
    \fi
    \thispagestyle{plain}%
    \if@twocolumn
        \onecolumn
        \@tempswatrue
    \else
        \@tempswafalse
    \fi
    \null\vfil
    \secdef\@book\@sbook
}
\def\@book[#1]#2{%
    \refstepcounter{book}
    \addcontentsline{toc}{book}{\bookname\ \thebook:\hspace{1em}#1}
    \markboth{}{}
    {\centering
     \interlinepenalty\@M
     \normalfont
     \huge\bfseries\bookname\nobreakspace\thebook
     \par
     \vskip 20\p@
     \Huge\bfseries#2\par}%
    \@endbook}
\def\@sbook#1{%
    {\centering
     \interlinepenalty \@M
     \normalfont
     \Huge\bfseries#1\par}%
    \@endbook}
\def\@endbook{
    \vfil\newpage
        \if@twoside
            \if@openright
                \null
                \thispagestyle{empty}%
                \newpage
            \fi
        \fi
        \if@tempswa
            \twocolumn
        \fi
}
\newcommand*\l@book[2]{%
    \ifnum\c@tocdepth >-3\relax
        \addpenalty{-\@highpenalty}%
        \addvspace{2.25em\@plus\p@}%
        \setlength\@tempdima{3em}%
        \begingroup
            \parindent\z@\rightskip\@pnumwidth
            \parfillskip -\@pnumwidth
            {
                \leavevmode
                \Large\bfseries#1\hfill\hb@xt@\@pnumwidth{\hss#2}
            }
            \par
            \nobreak
            \global\@nobreaktrue
            \everypar{\global\@nobreakfalse\everypar{}}%
        \endgroup
    \fi}
\newcommand\bookname{Book}
\renewcommand{\thebook}{\texorpdfstring{\Numberstring{book}}{book}}
\providecommand*{\toclevel@book}{-2}
\makeatother
\titleformat{\part}[display]
    {\Large\bfseries}
    {\partname\nobreakspace\thepart}
    {0mm}
    {\Huge\bfseries}
\titlecontents{part}[0pt]
    {\large\bfseries}
    {\partname\ \thecontentslabel: \quad}
    {}
    {\hfill\contentspage}
\titlecontents{chapter}[0pt]
    {\bfseries}
    {\chaptername\ \thecontentslabel:\quad}
    {}
    {\hfill\contentspage}
\newglossarystyle{longpara}{%
    \setglossarystyle{long}%
    \renewenvironment{theglossary}{%
        \begin{longtable}[l]{{p{0.25\hsize}p{0.65\hsize}}}
    }{\end{longtable}}%
    \renewcommand{\glossentry}[2]{%
        \glstarget{##1}{\glossentryname{##1}}%
        &\glossentrydesc{##1}{~##2.}
        \tabularnewline%
        \tabularnewline
    }%
}
\newglossary[not-glg]{notation}{not-gls}{not-glo}{Notation}
\newcommand*{\newnotation}[4][]{%
    \newglossaryentry{#2}{type=notation, name={\textbf{#3}, },
                          text={#4}, description={#4},#1}%
}
%--------------------------LENGTHS------------------------------%
% Spacings for the Table of Contents.
\addtolength{\cftsecnumwidth}{1ex}
\addtolength{\cftsubsecindent}{1ex}
\addtolength{\cftsubsecnumwidth}{1ex}
\addtolength{\cftfignumwidth}{1ex}
\addtolength{\cfttabnumwidth}{1ex}

% Indent and paragraph spacing.
\setlength{\parindent}{0em}
\setlength{\parskip}{0em}                                                           %
%----------------------------Main Document-------------------------------------%
\begin{document}
    \pagenumbering{roman}
    \title{MATH 114 Algebraic Topology Notes}
    \author{%
        Professor: Vladimir Chernov\\
        Notes by: Ryan Maguire%
    }
    \date{\vspace{-5ex}}
    \maketitle
    \tableofcontents
    \listoffigures
    \chapter{Homotopy}
    \pagenumbering{arabic}
        \section{Lecture 1: Review}
            \subsection{Topological Spaces}
                We start with the definition of topological spaces. We wish to
                generalize the notion of \textit{openness} that one comes across
                in the study of metric spaces\index{Metric Space}, or more
                concretely in the study of $\nspace$. To do this we axiomatize
                the properties of open sets: Taking arbitrary unions of open
                sets results in an open set, and the finite intersection of open
                sets is still open. Furthermore, we define the entire space and
                the empty set $\emptyset$ to be open. This gives us our
                definition of a topological space.
                \begin{fdefinition}{Topological Space}{Topological_Space}
                    A topological space is a set $X$ with a collection
                    $\tau\subseteq\powset{X}$ of subsets of $X$, called the
                    \textit{open} subsets, such that:
                    \index{Topological Space}\index{Open Subset}
                    \begin{enumerate}
                        \item \label{def:top:Empty_and_X_Open}%
                              $\emptyset\in\tau$ and $X\in\tau$
                        \item \label{def:top:Finite_Intersections}%
                              For all $\mathcal{U},\mathcal{V}\in\tau$ it is
                              true that $\mathcal{U}\cap\mathcal{V}\in\tau$
                        \item \label{def:top:Arbitrary_Unions}%
                              For any subset $\mathcal{O}\subseteq\tau$ it is
                              true that $\bigcup\mathcal{O}\in\tau$
                    \end{enumerate}
                \end{fdefinition}
                \begin{example}
                    The standard metric topology on $\nspace[]$ is induced by
                    declaring $\mathcal{U}\subseteq\nspace[]$ to be open if and
                    only if for all $x\in\mathcal{U}$ there is an
                    $\varepsilon>0$ such that
                    $(x-\varepsilon,x+\varepsilon)\subseteq\mathcal{U}$. So an
                    open interval of the form $(a,b)$ is open
                    (see Fig.~\ref{fig:Open_Subset_of_R}).
                \end{example}
                \begin{figure}[H]
                    \centering
                    \captionsetup{type=figure}
                    \begin{tikzpicture}[>=Latex]
    \coordinate (a)  at (2.0,  0.0);
    \coordinate (b)  at (9.0,  0.0);
    \coordinate (xt) at (4.0,  0.1);
    \coordinate (xb) at (4.0, -0.1);
    \coordinate (x)  at (4.0,  0.0);
    \coordinate (xl) at (3.0,  0.0);
    \coordinate (xr) at (5.0,  0.0);

    \draw[<->]   (0, 0) to (10, 0) node[above] {$\mathbb{R}$};
    \draw[thick] (a) to (b);
    \draw[thick] (b) arc (0:15:0.5);
    \draw[thick] (b) arc (0:-15:0.5);
    \draw[thick] (a) arc (180:195:0.5);
    \draw[thick] (a) arc (180:165:0.5);

    \draw (xt) to (xb);

    \node at (a)  [below=1ex] {$a$};
    \node at (b)  [below=1ex] {$b$};
    \node at (xl) [below=1ex] {$x-\varepsilon$};
    \node at (xr) [below=1ex] {$x+\varepsilon$};
    \node at (x)  [below=1ex] {$x$};

    \draw[blue, thick] (xr) arc (0:15:0.5);
    \draw[blue, thick] (xr) arc (0:-15:0.5);
    \draw[blue, thick] (xl) arc (180:195:0.5);
    \draw[blue, thick] (xl) arc (180:165:0.5);
    \draw[blue, thick] (xr) to (xl);
\end{tikzpicture}
                    \caption{An Open Subset of $\nspace[]$}
                    \label{fig:Open_Subset_of_R}
                \end{figure}
                By examining Fig.~\ref{fig:Open_Interval_Intersect_is_Open} we
                can convince ourselves that the intersection of open intervals
                is again open so long as we declare the empty set to be open.
                That is, $(0,1)$ and $(2,3)$ are open sets, but
                $(0,1)\cap(2,3)=\emptyset$. This highlights the need in
                requiring the empty set an element of the topology.
                \begin{figure}[H]
                    \centering
                    \captionsetup{type=figure}
                    \begin{tikzpicture}[>=Latex]
    \coordinate (a)  at (2.0,  0.0);
    \coordinate (b)  at (7.0,  0.0);
    \coordinate (c)  at (3.0,  0.0);
    \coordinate (d)  at (9.0,  0.0);
    \coordinate (x)  at (5.0,  0.0);
    \coordinate (xb) at (5.0, -0.1);
    \coordinate (xt) at (5.0,  0.1);
    \coordinate (xl) at (4.2,  0.0);
    \coordinate (xr) at (5.8,  0.0);

    \draw[<->]   (0, 0) to (10, 0) node[above] {$\mathbb{R}$};

    \node at (a)  [below=1ex] {$a$};
    \node at (b)  [below=1ex] {$b$};
    \node at (c)  [below=1ex] {$c$};
    \node at (d)  [below=1ex] {$d$};
    \node at (xl) [below=1ex] {$x-\varepsilon$};
    \node at (xr) [below=1ex] {$x+\varepsilon$};
    \node at (x)  [below=1ex] {$x$};
    \node at (a)  [above=1ex]
        {$\color{blue}{(a,b)}\cap\color{red}{(c,d)}=\color{Violet}{(c,b)}$};
    \node at (b) [above=1ex]
        {$\color{cyan}{(x-\varepsilon,x+\varepsilon)}%
            \subseteq\color{Violet}{(c,b)}$};

    \draw[blue,   thick]    (a)  to (c);
    \draw[red,    thick]    (b)  to (d);
    \draw[Violet, thick]    (c)  to (xl);
    \draw[Violet, thick]    (xr) to (b);
    \draw[cyan, very thick] (xl) to (xr);

    \draw[thin] (xt) to (xb);

    \draw[blue, very thick] (b) arc (0:15:0.5);
    \draw[blue, very thick] (b) arc (0:-15:0.5);
    \draw[blue, very thick] (a) arc (180:195:0.5);
    \draw[blue, very thick] (a) arc (180:165:0.5);

    \draw[red, very thick] (d) arc (0:15:0.5);
    \draw[red, very thick] (d) arc (0:-15:0.5);
    \draw[red, very thick] (c) arc (180:195:0.5);
    \draw[red, very thick] (c) arc (180:165:0.5);

    \draw[cyan, very thick] (xr) arc (0:15:0.5);
    \draw[cyan, very thick] (xr) arc (0:-15:0.5);
    \draw[cyan, very thick] (xl) arc (180:195:0.5);
    \draw[cyan, very thick] (xl) arc (180:165:0.5);
\end{tikzpicture}
                    \caption{The Intersection of Open Intervals is Open}
                    \label{fig:Open_Interval_Intersect_is_Open}
                \end{figure}
                \begin{example}
                    Given a metric space $\metspace{X}$, the metric topology is
                    defined by stating that $\mathcal{U}\subseteq{X}$ is open if
                    and only if for all $x\in\mathcal{U}$ there is an
                    $\varepsilon>0$ such that the $\varepsilon$ ball centered
                    about $x$ is contained in $\mathcal{U}$:
                    $\rball{\varepsilon}{\metspace{X}}{x}\subseteq\mathcal{U}$
                    (see Fig.~\ref{fig:Open_Subset_Metric_Space}).
                \end{example}
                \begin{figure}[H]
                    \centering
                    \captionsetup{type=figure}
                    \includegraphics{images/Open_Set_in_a_Metric_Space.pdf}
                    \caption{Open Subset of a Metric Space}
                    \label{fig:Open_Subset_Metric_Space}
                \end{figure}
                This metric topology on $\nspace[]$ is often called the
                \textit{standard} topology, but it is not the only one we can
                place on it.
                \begin{example}
                    The chaotic topology, also called the trivial topology or
                    the indiscrete topology, is the simplest topology one can
                    define on a set. We write:
                    \begin{equation}
                        \tau=\{\,\emptyset,\,X\,\}
                    \end{equation}
                    This is a topology since it trivially satisfies the three
                    properties enumerated in Def.~\ref{def:Topological_Space}.
                \end{example}
                \begin{example}
                    The largest topology one can define is the entire power set:
                    \begin{equation}
                        \tau=\powset{X}
                    \end{equation}
                    Again, rather trivially, this is a topology on $X$. For
                    those who have studied metric spaces, it is called the
                    trivial topology since it is the topology induced by the
                    discrete metric:
                    \begin{equation}
                        d(x,y)=
                        \begin{cases}
                            0,&x=y\\
                            1,&x\ne{y}
                        \end{cases}
                    \end{equation}
                \end{example}
                The closed interval $[a,b]$ is \textit{not} open in the standard
                topology (see Fig.~\ref{fig:Closed_Interval_Not_Open}), but it
                \textit{is} open with the discrete topology since everything is
                open in that topology. Because of this there is possible
                ambiguity with saying $\mathcal{U}$ is open if one has not
                specified the topology. Thankfully with most spaces one is
                interested in there is a standard or natural topology, and we
                usually choose this one without saying so. For metric spaces we
                almost always choose the metric topology.
                \begin{figure}[H]
                    \centering
                    \captionsetup{type=figure}
                    \begin{tikzpicture}[>=Latex]
    % Coordinates for various points.
    \coordinate (a)  at (4.0,  0.0);
    \coordinate (b)  at (9.0,  0.0);
    \coordinate (xl) at (3.0,  0.0);
    \coordinate (xr) at (5.0,  0.0);

    % Draw the real line.
    \draw[<->]   (0, 0) to (10, 0) node[above] {$\mathbb{R}$};

    % Draw the closed interval [a, b].
    \draw[thick] (a) to (b);

    % Add "brackets" indicating it is a closed interval.
    \draw[thick] (8.9, 0.1) to (9.0, 0.1) to (9.0, -0.1) to (8.9, -0.1);
    \draw[thick] (4.1, 0.1) to (4.0, 0.1) to (4.0, -0.1) to (4.1, -0.1);

    % Labels for the vaious points.
    \node at (a)  [below=1ex] {$a$};
    \node at (b)  [below=1ex] {$b$};
    \node at (xl) [below=1ex] {$a-\varepsilon$};
    \node at (xr) [below=1ex] {$a+\varepsilon$};

    % Draw the part of the open interval (a-e, a+e) that is inside of [a, b].
    \draw[blue, thick] (xr) arc (0:15:0.5);
    \draw[blue, thick] (xr) arc (0:-15:0.5);
    \draw[blue, thick] (a) to (xr);

    % Draw the part that falls outside.
    \draw[red, thick]  (xl) arc (180:195:0.5);
    \draw[red, thick]  (xl) arc (180:165:0.5);
    \draw[red, thick]  (a) to (xl);
\end{tikzpicture}
                    \caption{Closed Intervals are Not Open}
                    \label{fig:Closed_Interval_Not_Open}
                \end{figure}
                With the real line the intersection of finitely many open sets
                is still open, whereas we've only required the intersection of
                two open sets to still be open. We extend this to any finite
                collection by induction.
                \begin{theorem}
                    \label{thm:Finite_Intersections_Is_Open}%
                    If $\topspace{X}$ is a topological space and
                    $\mathcal{O}\subseteq\tau$ finite, then
                    $\bigcap\mathcal{O}\in\tau$.
                \end{theorem}
                \begin{proof}
                    Apply induction to Def.~\ref{def:Topological_Space}
                    part \ref{def:top:Finite_Intersections}.
                \end{proof}
                Now that we've presented some examples, we define what it means
                for a set to be closed. In analysis we defined a closed set to
                be a set that has all of its limit points. For topology this is
                not general enough (topological spaces where sequences suffice
                to define closedness are called \textit{sequential spaces}).
                There is a standard theorem one comes across that a subset of
                $\nspace[]$ is closed if and only if its complement is open. We
                take this theorem and adopt it as the definition of what it
                means to be a closed set in a general topological space.
                \begin{fdefinition}{Closed Subset}{Closed_Subset}
                    A closed subset of a topological space $\topspace{X}$ is a
                    subset $\mathcal{C}\subseteq{X}$ such that there exists an
                    open set $\mathcal{U}\in\tau$ with:
                    $\mathcal{C}=X\setminus\mathcal{U}$.
                \end{fdefinition}
                That is, closed sets are the complements of open sets. There is
                a common misconception that closed sets are simply \textit{not}
                open sets, and vice-versa, but this is not so. In the discrete
                topology every set is open, and hence every set is closed. In
                the indiscrete topology there are no non-empty proper open
                subsets, and hence most sets are neither open nor closed. These
                examples show that openness and closedness are
                \textit{a priori} unrelated notions. If we know $A\subseteq{X}$
                is open and nothing more, we cannot conclude whether or not $A$
                is closed, and similarly if we know $B\subseteq{X}$ is closed
                and nothing more, then we cannot conclude whether or not $B$ is
                open. Using the idempotent laws of complement, we obtain the
                following:
                \begin{theorem}
                    \label{thm:Closed_Iff_Comp_is_Open}%
                    If $\topspace{X}$ is a topological space and
                    $C\subseteq{X}$, then $C$ is closed if and only if
                    $X\setminus{C}$ is open.
                \end{theorem}
                \begin{proof}
                    For if $X\setminus{C}$ is open, then
                    $X\setminus(X\setminus{C})$ is closed
                    (Def.~\ref{def:Closed_Subset}). From the idempotent law of
                    complements, $X\setminus(X\setminus{C})=C$ and hence $C$ is
                    closed. By a similar argument if $C$ is closed, then
                    $X\setminus{C}$ is open.
                \end{proof}
                Using De Morgan's laws we can define an equivalent notion of
                topological spaces using closed sets. De Morgan's laws state for
                sets $A,B,X$, with $A,B\subseteq{X}$, the following is true:
                \begin{subequations}
                    \begin{align}
                        X\setminus(A\cap{B})
                            &=(X\setminus{A})\cup(X\setminus{B})\\
                        X\setminus(A\cup{B})
                            &=(X\setminus{A})\cap(X\setminus{B})
                    \end{align}
                \end{subequations}
                We can write this more suggestively if we let
                $X\setminus{A}=A^{C}$ ($C$ for complement).
                \twocolumneq{(A\cap{B})^{C}=A^{C}\cup{B}^{C}}
                            {(A\cup{B})^{C}=A^{C}\cap{B}^{C}}
                De Morgan's laws hold for arbitrary unions and intersections:
                \twocolumneq{%
                    \Big(\bigcap\mathcal{U}\Big)^{C}=\bigcup\mathcal{U}^{C}%
                }{%
                    \Big(\bigcup\mathcal{U}\Big)^{C}=\bigcap\mathcal{U}^{C}%
                }
                With this we may prove the following.
                \begin{theorem}
                    If $\topspace{X}$ is a topological space, then $\emptyset$
                    and $X$ are closed, if $\mathcal{C},\mathcal{D}\subseteq{X}$
                    are closed, then $\mathcal{C}\cup\mathcal{D}$ is closed, and
                    if $\Lambda\subseteq\powset{X}$ is a collection of closed
                    subsets of $X$, then $\bigcap\Lambda$ is closed.
                \end{theorem}
                \begin{proof}
                    For the first part, apply the definition of closed subsets
                    (Def.~\ref{def:Closed_Subset}) to
                    Def.~\ref{def:Topological_Space} part
                    \ref{def:top:Empty_and_X_Open} and recall that
                    $X\setminus\emptyset=X$ and $X\setminus{X}=\emptyset$. For
                    the latter parts, combine De Morgan's Laws with
                    Def.~\ref{def:Topological_Space} parts
                    \ref{def:top:Finite_Intersections} and
                    \ref{def:top:Arbitrary_Unions}, respectively.
                \end{proof}
                We may expand the union of two closed sets to the unions of
                finitely many closed sets inductively as we did in
                Thm.~\ref{thm:Finite_Intersections_Is_Open}.
            \subsection{Continuity}
                In an analysis course one talks about continuous functions by
                one of two equivalent means: The $\varepsilon-\delta$ definition
                and by the limit of sequences ($f(a_{n})\rightarrow{f}(x)$ for
                any sequence $a_{n}\rightarrow{x}$). Both of these are
                pictorial, the $\epsilon-\delta$ definition saying that if we
                move at most $\delta$ in the $x$ direction, then we will have
                no more than $\varepsilon$ amount of error in the $y$ direction
                (see Fig.~\ref{fig:Eps_Delta_Def_Cont}).
                \begin{figure}[H]
                    \centering
                    \captionsetup{type=figure}
                    \includegraphics{images/Continuity_Epsilon_Delta_Def.pdf}
                    \caption{The $\varepsilon-\delta$ Definition of Continuity}
                    \label{fig:Eps_Delta_Def_Cont}
                \end{figure}
                The sequence definition says that as we get arbitrarily close to
                $x_{0}$ by any sequence $a:\mathbb{N}\rightarrow\nspace[]$, then
                $f(a_{n})$ approaches $f(x_{0})$
                (see Fig.~\ref{fig:Sequence_Def_Continuity}).
                \begin{figure}[H]
                    \centering
                    \captionsetup{type=figure}
                    \includegraphics{images/Continuity_Sequence_Definition.pdf}
                    \caption{Sequence Definition of Continuity}
                    \label{fig:Sequence_Def_Continuity}
                \end{figure}
                Both of these definitions require a metric which a general
                topological space may not have. We can define continuity more
                generally by means of open sets by recalling that a function
                $f:\nspace[]\rightarrow\nspace[]$ is continuous if and only if
                for every open subset $\mathcal{V}\subseteq\nspace[]$ it is true
                that the \textit{pre-image} under $f$ is also an open subset.
                That is, $f^{\minus{1}}[\mathcal{V}]\subseteq\nspace[]$ is open.
                Since this is an if and only if we may adopt it as our
                definition of continuity, and this only uses the notion of an
                open set which topological spaces do have.
                \begin{fdefinition}{Continuous Function}{Continuous_Function}
                    A continuous function from a topological space
                    $\topspace[X]{X}$ to a topological space $\topspace[Y]{Y}$
                    is a function $f:X\rightarrow{Y}$ such that for every open
                    subset $\mathcal{V}\in\tau_{Y}$ it is true that
                    $f^{\minus{1}}[\mathcal{V}]\in\tau_{X}$. The pre-image of
                    open sets are open.
                \end{fdefinition}
                \begin{example}
                    If $\topspace{X}$ is any topological space, if
                    $Y$ is a set, and if we choose the chaotic topology
                    $\{\emptyset,Y\}$ on $Y$, then any function
                    $f:X\rightarrow{Y}$ is automatically continuous. There are
                    only two open subsets to check and we have:
                    \twocolumneq{f^{\minus{1}}[\emptyset]=\emptyset}
                                {f^{\minus{1}}[Y]=X}
                    Both of which are open subsets, and hence $f$ is continuous.
                    This is one justification for calling this the chaotic
                    topology: Every function is continuous. There are other
                    justifications: every sequence converges to every point
                    simultaneously, no points can be separated, and so on.
                \end{example}
                \begin{example}
                    If $\topspace{Y}$ is a topological space, if $X$ is a set,
                    and if we choose the discrete topology $\powset{X}$ on $X$,
                    then any function $f:X\rightarrow{Y}$ is continuous. Given
                    any open subset $\mathcal{V}\in\tau$, the pre-image
                    $f^{\minus{1}}[\mathcal{V}]$ is a subset of $X$ by
                    definition and hence is an element of $\powset{X}$. That is,
                    $f$ is continuous.
                \end{example}
                An important concept in topology is that of a
                \textit{homeomorphism}. A homeomorphism is a continuous function
                $f:X\rightarrow{Y}$ that is bijective and such that the inverse
                function $f^{\minus{1}}:Y\rightarrow{X}$ is also continuous.
                If there exists a homeomorphism between topological spaces
                $\topspace[X]{X}$ and $\topspace[Y]{Y}$, then we call them
                \textit{homeomorphic}. Topologically, homeomorphic spaces are
                indistinguishable.
                \begin{example}
                    The open interval $(\minus\frac{\pi}{2},\frac{\pi}{2})$ is
                    homeomorphic to the entire real line via the tangent
                    function $\tan:(\minus\frac{\pi}{2},\frac{\pi}{2})%
                    \rightarrow\nspace[]$. This is continuous, bijective, and
                    has a continuous inverse $\arctan$. This example shows that
                    boundedness is a metric property and not a topological one.
                \end{example}
                One question that often arises is when can we conclude two
                spaces are homeomorphic. Many simple conditions that hold in
                $\nspace[]$ do not generalize. For example, any continuous
                bijection $f:\nspace[]\rightarrow\nspace[]$ is automatically a
                a homeomophism since the inverse will be continuous. This is a
                consequence of the intermediate value theorem which implies any
                such function is then strictly monotonic. This does not
                generalize to arbitrary topological spaces, it doesn't even
                generalize to subspaces of $\nspace$. We can continuously and
                bijectively map $[0,1)$ to the unit circle $\nsphere[1]$ by
                $f(x)=\big(\cos(2\pi{x}),\sin(2\pi{x})\big)$, but the inverse is
                not continuous. Rigorously it can't be since $\nsphere[1]$ is
                \textit{compact} but $[0,1)$ is not, and homeomophisms preserve
                such a notion. Intuitively, we have tied up the ends of $[0,1)$
                continuously, but going from the unit circle to $[0,1)$ involves
                ripping the circle somewhere, which is not continuous.
                \par\hfill\par
                Another common false proposition is that if $f:X\rightarrow{Y}$
                and $g:Y\rightarrow{X}$ are continuous bijections, then the two
                spaces are homeomorphic. This is false, and we construct a
                counterexample to demonstrate.
            \subsection{More Notions to Review}
                It is hoped that the material previously discussed is all
                review. There are other notions needed that one should quickly
                read through if they have not seen them before, namely the
                notion of \textit{product spaces} and \textit{quotient spaces}.
                For product spaces we'll need to following theorem.
                \begin{ltheorem}{The Intersections of Topologies is a Topology}
                                {Intersection_of_Topologies_is_Topology}
                    If $X$ is a set, and if $T\subseteq\powset{\powset{X}}$ is a
                    non-empty collection of sets such that for all $\tau\in{T}$
                    it is true that $\tau$ is a topology on $X$, then
                    $\bigcap{T}$ is a topology on $X$.
                \end{ltheorem}
                \begin{proof}
                    Since every element of $T$ is a topology, $\emptyset$ and
                    $X$ are contained in all $\tau\in{T}$ and hence
                    $\emptyset,X\in\bigcap{T}$. If we have a collection
                    $\mathcal{O}$ of elements of $\bigcap{T}$, then each element
                    is in every topology $\tau\in{T}$ by the definition of
                    intersection. But then since all $\tau\in{T}$ are topologies
                    we know that $\bigcup\mathcal{O}\in\tau$. Since this is true
                    of all $\tau$ we conclude $\bigcup\mathcal{O}\in\bigcap{T}$.
                    By a similar argument, $\bigcap{T}$ is closed to finite
                    intersections. Hence, $\bigcap{T}$ is a topology on $X$.
                \end{proof}
                We can use this to define the product topology generated by the
                Cartesian product of two topological spaces. If we have two
                topological spaces $\topspace[X]{X}$ and $\topspace[Y]{Y}$ we
                would like to place a topology on $X\times{Y}$. Moreover we
                would like this topology to agree with product spaces we already
                know well, such as the standard topologies on $\nspace$ or the
                topologies induced by the product of metric spaces. We would
                like to make life simple and set the topology to be something
                like $\tau_{X}\times\tau_{Y}$, but this may not have the union
                property of topologies. Indeed, if we let $X=\nspace[]$ and
                $Y=\nspace[]$ then $\tau_{X}\times\tau_{Y}$ is the set of all
                $\mathcal{U}\times\mathcal{V}$ where $\mathcal{U}$ and
                $\mathcal{V}$ are open. Since open subsets of $\nspace[]$ can be
                written as the countable union of open intervals, we can suppose
                for the sake of visualization that $\mathcal{U}=(a,b)$ and
                $\mathcal{V}=(c,d)$ with $a,b,c,d\in\nspace[]$. The product
                $\mathcal{U}\times\mathcal{V}$ is an \textit{open rectangle}
                (see Fig.~\subref{fig:Open_Rectangle_in_R2}). A blob like the
                one in Fig.~\subref{fig:Open_Subset_Not_Product} is considered
                open in the standard topology on $\nspace[2]$ but cannot be
                written in the form $\mathcal{U}\times\mathcal{V}$. We use
                Thm.~\ref{thm:Intersection_of_Topologies_is_Topology} to define
                the product topology.
                \begin{figure}[H]
                    \centering
                    \captionsetup{type=figure}
                    \begin{subfigure}[b]{0.49\textwidth}
                        \centering
                        \begin{tikzpicture}[>=Latex]
    \draw[->, thick] (-0.4, 0) to (4, 0)
        node [above] {$x$};
    \draw[->, thick] (0, -0.4) to (0, 4)
        node [right] {$y$};
    \draw (1, -0.1) to (1, 0.1);
    \node at (1, -0.4) {$a$};
    \draw (3, -0.1) to (3, 0.1);
    \node at (3, -0.4) {$b$};
    \draw (-0.1, 1) to (0.1, 1);
    \node at (-0.4, 1) {$c$};
    \draw (-0.1, 3) to (0.1, 3);
    \node at (-0.4, 3) {$d$};
    \draw[fill=cyan, opacity=0.8, draw=white]
        (1, 1) to (1, 3) to (3, 3) to (3, 1) to cycle;
    \draw[densely dashed] (0, 1) to (3, 1);
    \draw[densely dashed] (0, 3) to (3, 3);
    \draw[densely dashed] (1, 0) to (1, 3);
    \draw[densely dashed] (3, 0) to (3, 3);
\end{tikzpicture}
                        \subcaption{The Open Rectangle $(a,b)\times(c,d)$.}
                        \label{fig:Open_Rectangle_in_R2}
                    \end{subfigure}
                    \begin{subfigure}[b]{0.49\textwidth}
                        \centering
                        \begin{tikzpicture}[>=Latex]
                            \draw[->, thick] (-0.4, 0) to (4, 0)
                                node [above] {$x$};
                            \draw[->, thick] (0, -0.4) to (0, 4)
                                node [right] {$y$};
                            \draw[fill=cyan, opacity=0.8, densely dashed]
                                (1, 1) to (2, 1) to (2, 2) to (3, 2)
                                       to (3, 3.5) to (1.5, 3.5)
                                       to (1.5, 2) to (1, 2) to cycle;
                        \end{tikzpicture}
                        \subcaption{A Region That Cannot be Written as
                                    $\mathcal{U}\times\mathcal{V}$.}
                        \label{fig:Open_Subset_Not_Product}
                    \end{subfigure}
                    \caption{Examples of Open Subsets of $\mathbb{R}^{2}$.}
                    \label{fig:Point_Set_Top_Open_Subsets_R2}
                \end{figure}
                Since the power set $\powset{X\times{Y}}$ is a topology on
                $X\times{Y}$, the set of all topologies that contain
                $\tau_{X}\times\tau_{Y}$ is non-empty. We thus define the
                product topology to be the \textit{smallest} topology that
                contains $\tau_{X}\times\tau_{Y}$. To make this precise we
                define $\tau_{X\times{Y}}$ to be $\bigcap\Lambda$ where
                $\Lambda$ is the set of all topologies containing
                $\tau_{X}\times\tau_{Y}$. By
                Thm.~\ref{thm:Intersection_of_Topologies_is_Topology} this is
                indeed a topology on $X\times{Y}$.
                \begin{figure}[H]
                    \centering
                    \captionsetup{type=figure}
                    \begin{tikzpicture}[>=Latex]
    \draw[<->, thick] (-3.3, 0) to (3.3, 0) node [above] {$x$};
    \draw[<->, thick] (0, -3.3) to (0, 3.3) node [right] {$y$};
    \draw[densely dashed] (0, 0) circle (1in);

    % First Layer
    \draw[fill=cyan, opacity=0.6, densely dashed]
        (0.7071in, 0.7071in) to (-0.7071in, 0.7071in)
                             to (-0.7071in, -0.7071in)
                             to (0.7071in, -0.7071in)
                             to cycle;
    
    % Second Layer
    \draw[fill=green, opacity=0.5, densely dashed]
        (0.68in, 0.3535in) to (0.935in, 0.3535in)
                           to (0.935in, -0.3535in)
                           to (0.68in, -0.3535in)
                           to cycle;
    \draw[fill=green, opacity=0.5, densely dashed]
        (-0.68in, 0.3535in) to (-0.935in, 0.3535in)
                            to (-0.935in, -0.3535in)
                            to (-0.68in, -0.3535in)
                            to cycle;
    \draw[fill=green, opacity=0.5, densely dashed]
        (0.3535in, 0.68in) to (0.3535in, 0.935in)
                           to (-0.3535in, 0.935in)
                           to (-0.3535in, 0.68in)
                           to cycle;
    \draw[fill=green, opacity=0.5, densely dashed]
        (0.3535in, -0.68in) to (0.3535in, -0.935in)
                            to (-0.3535in, -0.935in)
                            to (-0.3535in, -0.68in)
                            to cycle;

    % Third Layer.
    \draw[fill=orange, opacity=0.6, densely dashed]
        (0.68in, 0.3535in) to (0.8212in, 0.3535in)
                           to (0.8212in, 0.5705in)
                           to (0.68in, 0.5707in)
                           to cycle;
    \draw[fill=orange, opacity=0.6, densely dashed]
        (0.68in, -0.3535in) to (0.8212in, -0.3535in)
                            to (0.8212in, -0.5705in)
                            to (0.68in, -0.5707in)
                            to cycle;
    \draw[fill=orange, opacity=0.6, densely dashed]
        (-0.68in, -0.3535in) to (-0.8212in, -0.3535in)
                             to (-0.8212in, -0.5705in)
                             to (-0.68in, -0.5707in)
                             to cycle;
    \draw[fill=orange, opacity=0.6, densely dashed]
        (-0.68in, 0.3535in) to (-0.8212in, 0.3535in)
                            to (-0.8212in, 0.5705in)
                            to (-0.68in, 0.5707in)
                            to cycle;
    \draw[fill=orange, opacity=0.6, densely dashed]
        (0.3535in, 0.68in) to (0.3535in, 0.8212in)
                           to (0.5705in, 0.8212in)
                           to (0.5707in, 0.68in)
                           to cycle;
    \draw[fill=orange, opacity=0.6, densely dashed]
        (0.3535in, -0.68in) to (0.3535in, -0.8212in)
                            to (0.5705in, -0.8212in)
                            to (0.5707in, -0.68in)
                            to cycle;
    \draw[fill=orange, opacity=0.6, densely dashed]
        (-0.3535in, 0.68in) to (-0.3535in, 0.8212in)
                            to (-0.5705in, 0.8212in)
                            to (-0.5707in, 0.68in)
                            to cycle;
    \draw[fill=orange, opacity=0.6, densely dashed]
        (-0.3535in, -0.68in) to (-0.3535in, -0.8212in)
                             to (-0.5705in, -0.8212in)
                             to (-0.5707in, -0.68in)
                             to cycle;

    % Fourth Layer
    \draw[fill=red, opacity=0.5, densely dashed]
        (0.2in, 0.93in) to (0.2in, 0.9797in)
                        to (-0.2in, 0.9797in)
                        to (-0.2in, 0.93in)
                        to cycle;
    \draw[fill=red, opacity=0.5, densely dashed]
        (0.2in, -0.93in) to (0.2in, -0.9797in)
                         to (-0.2in, -0.9797in)
                         to (-0.2in, -0.93in)
                         to cycle;
    \draw[fill=red, opacity=0.5, densely dashed]
        (0.93in, 0.2in) to (0.9797in, 0.2in)
                        to (0.9797in, -0.2in)
                        to (0.93in, -0.2in)
                        to cycle;
    \draw[fill=red, opacity=0.5, densely dashed]
        (-0.93in, 0.2in) to (-0.9797in, 0.2in)
                         to (-0.9797in, -0.2in)
                         to (-0.93in, -0.2in)
                         to cycle;

    % Fifth Layer
    \draw[fill=blue, opacity=0.6, densely dashed]
        (0.82in, 0.3535in) to (0.8781in, 0.3535in)
                           to (0.8781in, 0.4784in)
                           to (0.82in, 0.4784in)
                           to cycle;
    \draw[fill=blue, opacity=0.6, densely dashed]
        (0.82in, -0.3535in) to (0.8781in, -0.3535in)
                            to (0.8781in, -0.4784in)
                            to (0.82in, -0.4784in)
                            to cycle;
    \draw[fill=blue, opacity=0.6, densely dashed]
        (-0.82in, 0.3535in) to (-0.8781in, 0.3535in)
                            to (-0.8781in, 0.4784in)
                            to (-0.82in, 0.4784in)
                            to cycle;
    \draw[fill=blue, opacity=0.6, densely dashed]
        (-0.82in, -0.3535in) to (-0.8781in, -0.3535in)
                             to (-0.8781in, -0.4784in)
                             to (-0.82in, -0.4784in)
                             to cycle;
    \draw[fill=blue, opacity=0.6, densely dashed]
        (0.3535in, 0.82in) to (0.3535in, 0.8781in)
                           to (0.4784in, 0.8781in)
                           to (0.4784in, 0.82in)
                           to cycle;
    \draw[fill=blue, opacity=0.6, densely dashed]
        (0.3535in, -0.82in) to (0.3535in, -0.8781in)
                            to (0.4784in, -0.8781in)
                            to (0.4784in, -0.82in)
                            to cycle;
    \draw[fill=blue, opacity=0.6, densely dashed]
        (-0.3535in, -0.82in) to (-0.3535in, -0.8781in)
                             to (-0.4784in, -0.8781in)
                             to (-0.4784in, -0.82in)
                             to cycle;
    \draw[fill=blue, opacity=0.6, densely dashed]
        (-0.3535in, 0.82in) to (-0.3535in, 0.8781in)
                            to (-0.4784in, 0.8781in)
                            to (-0.4784in, 0.82in)
                            to cycle;

    % Sixth Layer
    \draw[fill=yellow, opacity=0.6, densely dashed]
        (0.68in, 0.5705in) to (0.7641in, 0.5705in)
                           to (0.7641in, 0.645in)
                           to (0.68in, 0.645in)
                           to cycle;
    \draw[fill=yellow, opacity=0.6, densely dashed]
        (0.68in, -0.5705in) to (0.7641in, -0.5705in)
                            to (0.7641in, -0.645in)
                            to (0.68in, -0.645in)
                            to cycle;
    \draw[fill=yellow, opacity=0.6, densely dashed]
        (-0.68in, -0.5705in) to (-0.7641in, -0.5705in)
                             to (-0.7641in, -0.645in)
                             to (-0.68in, -0.645in)
                             to cycle;
    \draw[fill=yellow, opacity=0.6, densely dashed]
        (-0.68in, 0.5705in) to (-0.7641in, 0.5705in)
                            to (-0.7641in, 0.645in)
                            to (-0.68in, 0.645in)
                            to cycle;
    \draw[fill=yellow, opacity=0.6, densely dashed]
        (0.5705in, 0.68in) to (0.5705in, 0.7641in)
                           to (0.645in, 0.7641in)
                           to (0.645in, 0.68in)
                           to cycle;
    \draw[fill=yellow, opacity=0.6, densely dashed]
        (0.5705in, -0.68in) to (0.5705in, -0.7641in)
                            to (0.645in, -0.7641in)
                            to (0.645in, -0.68in)
                            to cycle;
    \draw[fill=yellow, opacity=0.6, densely dashed]
        (-0.5705in, -0.68in) to (-0.5705in, -0.7641in)
                             to (-0.645in, -0.7641in)
                             to (-0.645in, -0.68in)
                             to cycle;
    \draw[fill=yellow, opacity=0.6, densely dashed]
        (-0.5705in, 0.68in) to (-0.5705in, 0.7641in)
                            to (-0.645in, 0.7641in)
                            to (-0.645in, 0.68in)
                            to cycle;
\end{tikzpicture}
                    \caption{Tiling of the Open Disc by Rectangles}
                    \label{fig:Tiling_Open_Disc_by_Rectangles}
                \end{figure}
                Another way of describing this is by saying it is the topology
                \textit{generated} by all of the elements of
                $\tau_{X}\times\tau_{Y}$. That is, we take elements of
                $\tau_{X}\times\tau_{Y}$ and then add their unions and
                intersections until we have a valid topology. In this way we
                can see that the open unit disc is an open subset of the plane
                since we can tile it with rectangles
                (Fig.~\ref{fig:Tiling_Open_Disc_by_Rectangles}).
                As an example, the torus can be viewed as the Cartesian product
                of two circles equipped with the product topology
                (see Fig.~\ref{fig:Torus_as_Prod_Space}).
                \begin{figure}[H]
                    \centering
                    \captionsetup{type=figure}
                    \includegraphics{images/Torus_Skeleton_Product_Space.pdf}
                    \caption{The Torus $\ntorus[]=\nsphere[1]\times\nsphere[1]$}
                    \label{fig:Torus_as_Prod_Space}
                \end{figure}
                There is another standard way of viewing the torus which
                involves quotient spaces. For quotients one should think of
                gluing parts of a space together. Given a topological space
                $\topspace{X}$ and an equivalence relation $R$ on $X$, we look
                at the quotient set $X/R$. We topologize $X/R$ in the opposite
                manner as the product topology: we choose the \textit{largest}
                topology that makes the projection map $q:X\rightarrow{X}/R$
                continuous. There's always a topology that makes $q$ continuous
                since the chaotic topology $\{\emptyset,X/R\}$ does the trick,
                but this is too small and often boring. The quotient topology is
                defined as follows:
                \begin{equation}
                    \tau_{X/R}=\{\,\mathcal{V}\subseteq{X}/R\;|\;
                        q^{\minus{1}}[\mathcal{V}]\in\tau\,\}
                \end{equation}
                Again, the technical details are perhaps not too important for
                now, one should simply think of gluing together the parts
                identified by the equivalence relation $R$. Again we return to
                torus $\ntorus[]$, but now we define it as the quotient space
                obtained by identifying parts of the closed unit square
                together. Explicitly, we identify $(x,0)$ with $(x,1)$ for all
                $x\in[0,1]$ and similarly $(0,y)$ with $(1,y)$ for $y\in[0,1]$.
                The result of this gluing is depicted in
                Fig.~\ref{fig:Square_to_Torus}.
                \begin{figure}[H]
                    \centering
                    \captionsetup{type=figure}
                    \includegraphics{images/Square_to_Torus.pdf}
                    \caption{Gluing a Square into a Torus}
                    \label{fig:Square_to_Torus}
                \end{figure}
        \section{Lecture 2: Deformation Retractions}
            \subsection{Deformation Retraction}
                A deformation retraction from a topological space $\topspace{X}$
                onto a subset $A\subseteq{X}$ is a family of continuous
                functions $f_{t}:X\rightarrow{A}$ indexed by the closed unit
                interval $I=[0,1]$ such that $f_{0}=\identity{X}$,
                $f_{1}[X]=A$, and such that $f_{t}|_{A}=\identity{A}$ for all
                $t\in{I}$. Moreover, the function
                $H:X\times{I}\rightarrow{X}$ defined by $H(x,t)=f_{t}(x)$ should
                by continuous with respect to the product topology
                ($I$ inherits it's topology from $\nspace[]$).
                \begin{example}
                    We can take an annulus, or a punctured plane, either will
                    do, and construct a deformation retract from this onto the
                    unit circle $\nsphere[1]$. We do this by noting the function
                    $h$ defined on $\nspace[2]\setminus\{(0,0)\}$ by:
                    \begin{equation}
                        h(\vector{x})=\frac{\vector{x}}{\norm{\vector{x}}_{2}}
                    \end{equation}
                    where $\norm{\vector{x}}_{2}$ is the \textit{Euclidean Norm}
                    of the point $\vector{x}$, function maps the punctured plane
                    $\nspace[2]\setminus\{(0,0)\}$ onto the unit circle. To
                    complete our deformation retraction we drag the point
                    $\vector{x}\ne\vector{0}$ along the straight line between
                    $\vector{x}$ and $h(\vector{x})$. We have:
                    \begin{equation}
                        H(\vector{x},t)
                        =(1-t)\cdot\vector{x}+t\cdot{h}(\vector{x})
                    \end{equation}
                    This is a deformation retract since elements of the unit
                    circle are held fixed for all $t\in[0,1]$. If
                    $\vector{s}\in\nsphere[1]$, then by definition
                    $\norm{\vector{s}}_{2}=1$ and hence
                    $h(\vector{s})=\vector{s}$. Simplifying, we have:
                    \begin{equation}
                        H(\vector{s},t)=(1-t)\cdot\vector{s}+t\cdot\vector{s}
                            =\vector{s}
                    \end{equation}
                    Thus $H$ gives us a deformation retraction of the punctured
                    plane onto the unit circle. We can do the same deformation
                    retraction with an annulus if we simply restrict $h$ to that
                    domain (see Fig.~\ref{fig:Def_Retract_Annulus_to_Circle}).
                \end{example}
                \begin{figure}[H]
                    \centering
                    \captionsetup{type=figure}
                    \includegraphics{images/Homotopy_Circle.pdf}
                    \caption{Deformation Retraction of an Annulus to a Circle}
                    \label{fig:Def_Retract_Annulus_to_Circle}
                \end{figure}
                For our second example we must first define what a M\"{o}bius
                strip is. Much the way the torus was defined by gluing together
                parts of the unit square, so is the torus but with a twist
                introduced. We take $[0,1]\times[0,1]$ and identify
                $(0,y)$ with $(1,1-y)$, leaving the $x$ coordinate alone.
                The procedure is outlined in
                Fig.~\ref{fig:Square_to_Mobius_Strip}.
                \begin{figure}[H]
                    \centering
                    \captionsetup{type=figure}
                    \includegraphics{images/Square_to_Mobius_Strip.pdf}
                    \caption{Gluing a Square into a M\"{o}bius Strip}
                    \label{fig:Square_to_Mobius_Strip}
                \end{figure}
                For those familiar with the language, the outcome is a manifold
                with boundary. There is an inner circle wrapping around the
                M\"{o}bius strip that we can retract the space onto in the same
                way that the annulus could be deformation retracted to a circle.
                \par\hfill\par
                The subspace which we retract our space onto need not be unique,
                nor need they be homeomorphic. Indeed, any space can be
                deformation retracted onto itself by the function
                $H(x,t)=\identity{X}(x)$. The annulus is not homeomorphic to the
                circle (they are different dimensional), but they can both be
                deformation retracted from the same space. There are more subtle
                exams. For example, suppose we have the plane with two holes in
                it. We can deformation retract this space on to an abundance of
                non-homeomorphic subspaces.
                \begin{fdefinition}{Deformation Retraction}
                                   {Deformation_Retraction}
                    A deformation retraction of a topological space
                    $\topspace{X}$ onto a subspace $\topspace[A]{A}$ is a
                    homotopy $F:X\times{I}\rightarrow{X}$ between a deformation
                    retract $f:X\rightarrow{A}$ and the identity function
                    $\identity{X}$.
                \end{fdefinition}
                Returning to the example of the annulus
                (Fig.~\ref{fig:Def_Retract_Annulus_to_Circle}), we can convert
                this into a homotopy by considering the straight line homotopy
                between $\vector{x}$ and $\uvector{x}$. We define:
                \begin{equation}
                    H(\vector{x},t)=(1-t)\cdot\vector{x}+
                        t\cdot\frac{\vector{x}}{\norm{\vector{x}}}
                \end{equation}
                and this gives us a deformation retraction of the annulus on
                $\nsphere[1]$. Another perhaps more difficult example is the
                plane with two holes in it, say at $(\minus{1},0)$ and $(1,0)$.
                There is a deformation retraction of this space onto a figure
                eight. For the definition of a figure eight, let's start with
                the lemniscate of Gerono, studied by the French mathematician
                Camille-Christophe Gerono in the $19^{th}$ century C.E. The
                defining implicit equation goes as follows:
                \begin{equation}
                    x^{4}-x^{2}+y^{2}=0
                \end{equation}
                Setting $x(\theta)=\cos(\theta)$, we have:
                \begin{equation}
                    y^{2}=\cos^{2}(\theta)\big(1-\cos^{2}(\theta)\big)
                         =\cos^{2}(\theta)\sin^{2}(\theta)
                \end{equation}
                so we parameterize the figure eight by:
                \begin{equation}
                    \big(x(\theta),\,y(\theta)\big)
                        =\big(\cos(\theta),\,\cos(\theta)\sin(\theta)\big)
                \end{equation}
                Now to retract the plane with two holes onto this object whilst
                leaving the lemniscate fixed. First, suppose we've retracted all
                far away points down to an oval, and points near the two holes
                we've pushed out so that the holes have been enlarged from
                points to circle. To finish the retraction we can use a bit of
                physics and differential equations
                (Fig.~\ref{fig:Deformation_Retraction_lemniscate_of_Gerono}).
                \begin{figure}
                    \centering
                    \captionsetup{type=figure}
                    \includegraphics{images/Homotopy_lemniscate_of_Gerono.pdf}
                    \caption{Deformation Retraction of the Lemniscate of Gerono}
                    \label{fig:Deformation_Retraction_lemniscate_of_Gerono}
                \end{figure}
                We put two identical electric charges at the centers of the blue
                circles. The \textit{electric field} exerted at a point
                $\vector{x}=(x_{0},x_{1})$ in the plane is given by Coulomb's
                law:
                \begin{equation}
                    F(\vector{x})=
                    \frac{\vector{x}-\vector{r}_{1}}
                         {\norm{\vector{x}-\vector{r}_{1}}_{2}^{3}}+
                    \frac{\vector{x}-\vector{r}_{2}}
                         {\norm{\vector{x}-\vector{r}_{2}}_{2}^{3}}
                \end{equation}
                where $\vector{r}_{1}$ and $\vector{r}_{2}$ are the coordinates
                of the holes, we've chosen $(\minus{1},0)$ and $(1,0)$. The
                notation $\norm{\vector{x}}_{2}^{3}$ denotes the 2-norm raised
                to the third power. If this were an actual physics problem we
                would need some scale factor $Q/4\pi\epsilon_{0}$, but for our
                purposes we simply need the directions of the
                \textit{field lines}. Given a point inside of the lemniscate, we
                drag this point continuously outward along field lines until we
                hit the figure eight, and for points on the outside we contract
                inwards. All the while we leave the lemniscate fixed. The
                outcome is a deformation retraction onto our figure eight. For
                the sake of computation, one could use something like Euler's
                method of solving differential equations to numerical
                approximate this homotopy. For a more analytical approach with
                closed form solutions, we can consider Cassini ovals. If we let
                $a$ denote the distance from the first hole to the second, we
                can study the family of curves satisfying the following
                equation:
                \begin{equation}
                    \label{eqn:Cassini_Ovals}%
                    \big((x-a)^{2}+y^{2}\big)\big((x+a)^{2}+y^{2}\big)=b^{4}
                \end{equation}
                this asks for the set of all points $P$ such that the distance
                from $P$ to the first whole multiplied by the distance from $P$
                to the second hole is equal to $b^{2}$. That is:
                \begin{equation}
                    X_{b}=\{\,\vector{x}\in\nspace[2]\;|\;
                        \norm{\vector{x}-\vector{r}_{1}}_{2}\cdot
                        \norm{\vector{x}-\vector{r}_{2}}_{2}=b^{2}\,\}
                \end{equation}
                This problem was studied by the Italian astronomer Giovanni
                Domenico Cassini in 1680 C.E. and gives us a continuous means of
                retracting the plane with two holes onto a figure eight. The
                resulting figure eight is no longer the lemniscate of Gerono,
                but rather the lemniscate of Bernoulli, studied by Jakob
                Bernoulli in 1694 C.E. shortly after Cassini's investigations.
                Taking the gradient of Eqn.~\ref{eqn:Cassini_Ovals} gives us a
                vector field that once agains allows us to flow along field
                lines until we arrive at the figure eight
                (see Fig.~\ref{fig:Deformation_Retraction_Cassini_Ovals}). The
                gradient is computed to be:
                \begin{equation}
                    \grad{f}=\big(4x(x^{2}+y^{2}-a^{2}),4y(x^{2}+y^{2}+a^2)\big)
                \end{equation}
                \begin{figure}[H]
                    \centering
                    \captionsetup{type=figure}
                    \includegraphics{images/Homotopy_Cassini_Ovals_001.pdf}
                    \caption{Deformation Retraction Using Cassini Ovals}
                    \label{fig:Deformation_Retraction_Cassini_Ovals}
                \end{figure}
                The gradient allows us to show what the paths of individual
                points will look like, and allows us to draw
                Fig.~\ref{fig:Deformation_Retraction_Cassini_Ovals}, but is
                unnecessary. Cassini's equation Eqn.~\ref{eqn:Cassini_Ovals} is
                all we need. As $b$ tends to zero we get two disconnected ovals
                closing in on our two points. When $b$ is equal to the square
                root of the distance between these two points we obtain our
                lemniscate, and as $b$ grows we get a single connected object
                that looks more and more like a circle as $b$ gets large. This
                is precisely the description of a deformation retraction of the
                plane with two points missing onto the figure eight and the
                intermediate steps are shown in
                Fig.~\ref{fig:Homotopy_Cassini_Ovals}.
                \begin{figure}[H]
                    \centering
                    \captionsetup{type=figure}
                    \includegraphics{images/Homotopy_Cassini_Ovals_002.pdf}
                    \caption{Homotopy Using Cassini Ovals}
                    \label{fig:Homotopy_Cassini_Ovals}
                \end{figure}
                Now we can use our imagination to consider other possible
                deformation retractions from the plane with two holes onto
                smaller subspaces. For one, we can take our central lemniscate
                of Bernoulli that was produced in
                Fig.~\ref{fig:Deformation_Retraction_Cassini_Ovals} using
                Cassini ovals and stretch the crossing point outwards until we
                achieve two circles attached by a straight line
                (see Fig.~\ref{fig:Homotopy_Two_Circles_and_String}). While the
                retract using electric fields and the retract using Cassini
                ovals resulted in two homeomorphic objects (the lemniscate of
                Gerono and the lemniscate of Bernoulli are homeomorphic spaces
                of $\nspace[2]$), this third object is \textit{not} homeomorphic
                to either of these.
                \begin{figure}[H]
                    \centering
                    \captionsetup{type=figure}
                    \includegraphics{images/Homotopy_Two_Circles_and_String.pdf}
                    \caption{Deformation Retraction onto Two Connnected Circles}
                    \label{fig:Homotopy_Two_Circles_and_String}
                \end{figure}
                We can see that these are not homeomorphic as follows. Suppose
                $f$ is a homeomorphism from the lemniscate of Bernoulli to two
                circles with a straight line. If we remove the center of the
                lemniscate we are left with two objects that are homeomorphic to
                the open unit interval $(0,1)$. However, no matter where the
                crossing point maps to under $f$, removing a single point from
                the latter object does not result in two homeomorphic copies of
                the unit interval, even though homeomorphism preserve this
                notion of subspace. So while these are both the result of
                deformation retractions of the same space, they are \textit{not}
                homeomorphic. Further still we can imagine stretching the
                crossing point of our lemniscate vertically rather than
                horizontally, obtaining Fig.~\ref{fig:Homotopy_Oval_with_Line}.
                \begin{figure}[H]
                    \centering
                    \captionsetup{type=figure}
                    \includegraphics{images/Homotopy_Oval_with_Lines.pdf}
                    \caption{Homotopy onto a Different Figure Eight}
                    \label{fig:Homotopy_Oval_with_Line}
                \end{figure}
                It is perhaps easiest to see that the retraction obtained in
                Fig.~\ref{fig:Homotopy_Oval_with_Line} is different then both
                Fig.~\ref{fig:Deformation_Retraction_lemniscate_of_Gerono} and
                Fig.~\ref{fig:Deformation_Retraction_Cassini_Ovals}. Removing
                any point from this final figure eight does not disconnect the
                space, however if one were to remove the central points from
                either the lemniscate of Bernoulli or the circles connected by a
                line, we disconnect these spaces into two separate parts. Since
                homeomophisms preserve such notions, this final object is not
                homeomorphic to either of the other two. Hence we see that none
                of these three things are homeomorphic to each other, though
                they are homotopy equivalent.
\end{document}