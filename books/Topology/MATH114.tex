%------------------------------------------------------------------------------%
\documentclass{book}                                                           %
%------------------------------Preamble----------------------------------------%
\makeatletter                                                                  %
    \def\input@path{{../../}}                                                  %
\makeatother                                                                   %
%---------------------------Packages----------------------------%
\usepackage{geometry}
\geometry{b5paper, margin=1.0in}
\usepackage[T1]{fontenc}
\usepackage{graphicx, float}            % Graphics/Images.
\usepackage{natbib}                     % For bibliographies.
\bibliographystyle{agsm}                % Bibliography style.
\usepackage[french, english]{babel}     % Language typesetting.
\usepackage[dvipsnames]{xcolor}         % Color names.
\usepackage{listings}                   % Verbatim-Like Tools.
\usepackage{mathtools, esint, mathrsfs} % amsmath and integrals.
\usepackage{amsthm, amsfonts, amssymb}  % Fonts and theorems.
\usepackage{tcolorbox}                  % Frames around theorems.
\usepackage{upgreek}                    % Non-Italic Greek.
\usepackage{fmtcount, etoolbox}         % For the \book{} command.
\usepackage[newparttoc]{titlesec}       % Formatting chapter, etc.
\usepackage{titletoc}                   % Allows \book in toc.
\usepackage[nottoc]{tocbibind}          % Bibliography in toc.
\usepackage[titles]{tocloft}            % ToC formatting.
\usepackage{pgfplots, tikz}             % Drawing/graphing tools.
\usepackage{imakeidx}                   % Used for index.
\usetikzlibrary{
    calc,                   % Calculating right angles and more.
    angles,                 % Drawing angles within triangles.
    arrows.meta,            % Latex and Stealth arrows.
    quotes,                 % Adding labels to angles.
    positioning,            % Relative positioning of nodes.
    decorations.markings,   % Adding arrows in the middle of a line.
    patterns,
    arrows
}                                       % Libraries for tikz.
\pgfplotsset{compat=1.9}                % Version of pgfplots.
\usepackage[font=scriptsize,
            labelformat=simple,
            labelsep=colon]{subcaption} % Subfigure captions.
\usepackage[font={scriptsize},
            hypcap=true,
            labelsep=colon]{caption}    % Figure captions.
\usepackage[pdftex,
            pdfauthor={Ryan Maguire},
            pdftitle={Mathematics and Physics},
            pdfsubject={Mathematics, Physics, Science},
            pdfkeywords={Mathematics, Physics, Computer Science, Biology},
            pdfproducer={LaTeX},
            pdfcreator={pdflatex}]{hyperref}
\hypersetup{
    colorlinks=true,
    linkcolor=blue,
    filecolor=magenta,
    urlcolor=Cerulean,
    citecolor=SkyBlue
}                           % Colors for hyperref.
\usepackage[toc,acronym,nogroupskip,nopostdot]{glossaries}
\usepackage{glossary-mcols}
%------------------------Theorem Styles-------------------------%
\theoremstyle{plain}
\newtheorem{theorem}{Theorem}[section]

% Define theorem style for default spacing and normal font.
\newtheoremstyle{normal}
    {\topsep}               % Amount of space above the theorem.
    {\topsep}               % Amount of space below the theorem.
    {}                      % Font used for body of theorem.
    {}                      % Measure of space to indent.
    {\bfseries}             % Font of the header of the theorem.
    {}                      % Punctuation between head and body.
    {.5em}                  % Space after theorem head.
    {}

% Italic header environment.
\newtheoremstyle{thmit}{\topsep}{\topsep}{}{}{\itshape}{}{0.5em}{}

% Define environments with italic headers.
\theoremstyle{thmit}
\newtheorem*{solution}{Solution}

% Define default environments.
\theoremstyle{normal}
\newtheorem{example}{Example}[section]
\newtheorem{definition}{Definition}[section]
\newtheorem{problem}{Problem}[section]

% Define framed environment.
\tcbuselibrary{most}
\newtcbtheorem[use counter*=theorem]{ftheorem}{Theorem}{%
    before=\par\vspace{2ex},
    boxsep=0.5\topsep,
    after=\par\vspace{2ex},
    colback=green!5,
    colframe=green!35!black,
    fonttitle=\bfseries\upshape%
}{thm}

\newtcbtheorem[auto counter, number within=section]{faxiom}{Axiom}{%
    before=\par\vspace{2ex},
    boxsep=0.5\topsep,
    after=\par\vspace{2ex},
    colback=Apricot!5,
    colframe=Apricot!35!black,
    fonttitle=\bfseries\upshape%
}{ax}

\newtcbtheorem[use counter*=definition]{fdefinition}{Definition}{%
    before=\par\vspace{2ex},
    boxsep=0.5\topsep,
    after=\par\vspace{2ex},
    colback=blue!5!white,
    colframe=blue!75!black,
    fonttitle=\bfseries\upshape%
}{def}

\newtcbtheorem[use counter*=example]{fexample}{Example}{%
    before=\par\vspace{2ex},
    boxsep=0.5\topsep,
    after=\par\vspace{2ex},
    colback=red!5!white,
    colframe=red!75!black,
    fonttitle=\bfseries\upshape%
}{ex}

\newtcbtheorem[auto counter, number within=section]{fnotation}{Notation}{%
    before=\par\vspace{2ex},
    boxsep=0.5\topsep,
    after=\par\vspace{2ex},
    colback=SeaGreen!5!white,
    colframe=SeaGreen!75!black,
    fonttitle=\bfseries\upshape%
}{not}

\newtcbtheorem[use counter*=remark]{fremark}{Remark}{%
    fonttitle=\bfseries\upshape,
    colback=Goldenrod!5!white,
    colframe=Goldenrod!75!black}{ex}

\newenvironment{bproof}{\textit{Proof.}}{\hfill$\square$}
\tcolorboxenvironment{bproof}{%
    blanker,
    breakable,
    left=3mm,
    before skip=5pt,
    after skip=10pt,
    borderline west={0.6mm}{0pt}{green!80!black}
}

\AtEndEnvironment{lexample}{$\hfill\textcolor{red}{\blacksquare}$}
\newtcbtheorem[use counter*=example]{lexample}{Example}{%
    empty,
    title={Example~\theexample},
    boxed title style={%
        empty,
        size=minimal,
        toprule=2pt,
        top=0.5\topsep,
    },
    coltitle=red,
    fonttitle=\bfseries,
    parbox=false,
    boxsep=0pt,
    before=\par\vspace{2ex},
    left=0pt,
    right=0pt,
    top=3ex,
    bottom=1ex,
    before=\par\vspace{2ex},
    after=\par\vspace{2ex},
    breakable,
    pad at break*=0mm,
    vfill before first,
    overlay unbroken={%
        \draw[red, line width=2pt]
            ([yshift=-1.2ex]title.south-|frame.west) to
            ([yshift=-1.2ex]title.south-|frame.east);
        },
    overlay first={%
        \draw[red, line width=2pt]
            ([yshift=-1.2ex]title.south-|frame.west) to
            ([yshift=-1.2ex]title.south-|frame.east);
    },
}{ex}

\AtEndEnvironment{ldefinition}{$\hfill\textcolor{Blue}{\blacksquare}$}
\newtcbtheorem[use counter*=definition]{ldefinition}{Definition}{%
    empty,
    title={Definition~\thedefinition:~{#1}},
    boxed title style={%
        empty,
        size=minimal,
        toprule=2pt,
        top=0.5\topsep,
    },
    coltitle=Blue,
    fonttitle=\bfseries,
    parbox=false,
    boxsep=0pt,
    before=\par\vspace{2ex},
    left=0pt,
    right=0pt,
    top=3ex,
    bottom=0pt,
    before=\par\vspace{2ex},
    after=\par\vspace{1ex},
    breakable,
    pad at break*=0mm,
    vfill before first,
    overlay unbroken={%
        \draw[Blue, line width=2pt]
            ([yshift=-1.2ex]title.south-|frame.west) to
            ([yshift=-1.2ex]title.south-|frame.east);
        },
    overlay first={%
        \draw[Blue, line width=2pt]
            ([yshift=-1.2ex]title.south-|frame.west) to
            ([yshift=-1.2ex]title.south-|frame.east);
    },
}{def}

\AtEndEnvironment{ltheorem}{$\hfill\textcolor{Green}{\blacksquare}$}
\newtcbtheorem[use counter*=theorem]{ltheorem}{Theorem}{%
    empty,
    title={Theorem~\thetheorem:~{#1}},
    boxed title style={%
        empty,
        size=minimal,
        toprule=2pt,
        top=0.5\topsep,
    },
    coltitle=Green,
    fonttitle=\bfseries,
    parbox=false,
    boxsep=0pt,
    before=\par\vspace{2ex},
    left=0pt,
    right=0pt,
    top=3ex,
    bottom=-1.5ex,
    breakable,
    pad at break*=0mm,
    vfill before first,
    overlay unbroken={%
        \draw[Green, line width=2pt]
            ([yshift=-1.2ex]title.south-|frame.west) to
            ([yshift=-1.2ex]title.south-|frame.east);},
    overlay first={%
        \draw[Green, line width=2pt]
            ([yshift=-1.2ex]title.south-|frame.west) to
            ([yshift=-1.2ex]title.south-|frame.east);
    }
}{thm}

%--------------------Declared Math Operators--------------------%
\DeclareMathOperator{\adjoint}{adj}         % Adjoint.
\DeclareMathOperator{\Card}{Card}           % Cardinality.
\DeclareMathOperator{\curl}{curl}           % Curl.
\DeclareMathOperator{\diam}{diam}           % Diameter.
\DeclareMathOperator{\dist}{dist}           % Distance.
\DeclareMathOperator{\Div}{div}             % Divergence.
\DeclareMathOperator{\Erf}{Erf}             % Error Function.
\DeclareMathOperator{\Erfc}{Erfc}           % Complementary Error Function.
\DeclareMathOperator{\Ext}{Ext}             % Exterior.
\DeclareMathOperator{\GCD}{GCD}             % Greatest common denominator.
\DeclareMathOperator{\grad}{grad}           % Gradient
\DeclareMathOperator{\Ima}{Im}              % Image.
\DeclareMathOperator{\Int}{Int}             % Interior.
\DeclareMathOperator{\LC}{LC}               % Leading coefficient.
\DeclareMathOperator{\LCM}{LCM}             % Least common multiple.
\DeclareMathOperator{\LM}{LM}               % Leading monomial.
\DeclareMathOperator{\LT}{LT}               % Leading term.
\DeclareMathOperator{\Mod}{mod}             % Modulus.
\DeclareMathOperator{\Mon}{Mon}             % Monomial.
\DeclareMathOperator{\multideg}{mutlideg}   % Multi-Degree (Graphs).
\DeclareMathOperator{\nul}{nul}             % Null space of operator.
\DeclareMathOperator{\Ord}{Ord}             % Ordinal of ordered set.
\DeclareMathOperator{\Prin}{Prin}           % Principal value.
\DeclareMathOperator{\proj}{proj}           % Projection.
\DeclareMathOperator{\Refl}{Refl}           % Reflection operator.
\DeclareMathOperator{\rk}{rk}               % Rank of operator.
\DeclareMathOperator{\sgn}{sgn}             % Sign of a number.
\DeclareMathOperator{\sinc}{sinc}           % Sinc function.
\DeclareMathOperator{\Span}{Span}           % Span of a set.
\DeclareMathOperator{\Spec}{Spec}           % Spectrum.
\DeclareMathOperator{\supp}{supp}           % Support
\DeclareMathOperator{\Tr}{Tr}               % Trace of matrix.
%--------------------Declared Math Symbols--------------------%
\DeclareMathSymbol{\minus}{\mathbin}{AMSa}{"39} % Unary minus sign.
%------------------------New Commands---------------------------%
\DeclarePairedDelimiter\norm{\lVert}{\rVert}
\DeclarePairedDelimiter\ceil{\lceil}{\rceil}
\DeclarePairedDelimiter\floor{\lfloor}{\rfloor}
\newcommand*\diff{\mathop{}\!\mathrm{d}}
\newcommand*\Diff[1]{\mathop{}\!\mathrm{d^#1}}
\renewcommand*{\glstextformat}[1]{\textcolor{RoyalBlue}{#1}}
\renewcommand{\glsnamefont}[1]{\textbf{#1}}
\renewcommand\labelitemii{$\circ$}
\renewcommand\thesubfigure{%
    \arabic{chapter}.\arabic{figure}.\arabic{subfigure}}
\addto\captionsenglish{\renewcommand{\figurename}{Fig.}}
\numberwithin{equation}{section}

\renewcommand{\vector}[1]{\boldsymbol{\mathrm{#1}}}

\newcommand{\uvector}[1]{\boldsymbol{\hat{\mathrm{#1}}}}
\newcommand{\topspace}[2][]{(#2,\tau_{#1})}
\newcommand{\measurespace}[2][]{(#2,\varSigma_{#1},\mu_{#1})}
\newcommand{\measurablespace}[2][]{(#2,\varSigma_{#1})}
\newcommand{\manifold}[2][]{(#2,\tau_{#1},\mathcal{A}_{#1})}
\newcommand{\tanspace}[2]{T_{#1}{#2}}
\newcommand{\cotanspace}[2]{T_{#1}^{*}{#2}}
\newcommand{\Ckspace}[3][\mathbb{R}]{C^{#2}(#3,#1)}
\newcommand{\funcspace}[2][\mathbb{R}]{\mathcal{F}(#2,#1)}
\newcommand{\smoothvecf}[1]{\mathfrak{X}(#1)}
\newcommand{\smoothonef}[1]{\mathfrak{X}^{*}(#1)}
\newcommand{\bracket}[2]{[#1,#2]}

%------------------------Book Command---------------------------%
\makeatletter
\renewcommand\@pnumwidth{1cm}
\newcounter{book}
\renewcommand\thebook{\@Roman\c@book}
\newcommand\book{%
    \if@openright
        \cleardoublepage
    \else
        \clearpage
    \fi
    \thispagestyle{plain}%
    \if@twocolumn
        \onecolumn
        \@tempswatrue
    \else
        \@tempswafalse
    \fi
    \null\vfil
    \secdef\@book\@sbook
}
\def\@book[#1]#2{%
    \refstepcounter{book}
    \addcontentsline{toc}{book}{\bookname\ \thebook:\hspace{1em}#1}
    \markboth{}{}
    {\centering
     \interlinepenalty\@M
     \normalfont
     \huge\bfseries\bookname\nobreakspace\thebook
     \par
     \vskip 20\p@
     \Huge\bfseries#2\par}%
    \@endbook}
\def\@sbook#1{%
    {\centering
     \interlinepenalty \@M
     \normalfont
     \Huge\bfseries#1\par}%
    \@endbook}
\def\@endbook{
    \vfil\newpage
        \if@twoside
            \if@openright
                \null
                \thispagestyle{empty}%
                \newpage
            \fi
        \fi
        \if@tempswa
            \twocolumn
        \fi
}
\newcommand*\l@book[2]{%
    \ifnum\c@tocdepth >-3\relax
        \addpenalty{-\@highpenalty}%
        \addvspace{2.25em\@plus\p@}%
        \setlength\@tempdima{3em}%
        \begingroup
            \parindent\z@\rightskip\@pnumwidth
            \parfillskip -\@pnumwidth
            {
                \leavevmode
                \Large\bfseries#1\hfill\hb@xt@\@pnumwidth{\hss#2}
            }
            \par
            \nobreak
            \global\@nobreaktrue
            \everypar{\global\@nobreakfalse\everypar{}}%
        \endgroup
    \fi}
\newcommand\bookname{Book}
\renewcommand{\thebook}{\texorpdfstring{\Numberstring{book}}{book}}
\providecommand*{\toclevel@book}{-2}
\makeatother
\titleformat{\part}[display]
    {\Large\bfseries}
    {\partname\nobreakspace\thepart}
    {0mm}
    {\Huge\bfseries}
\titlecontents{part}[0pt]
    {\large\bfseries}
    {\partname\ \thecontentslabel: \quad}
    {}
    {\hfill\contentspage}
\titlecontents{chapter}[0pt]
    {\bfseries}
    {\chaptername\ \thecontentslabel:\quad}
    {}
    {\hfill\contentspage}
\newglossarystyle{longpara}{%
    \setglossarystyle{long}%
    \renewenvironment{theglossary}{%
        \begin{longtable}[l]{{p{0.25\hsize}p{0.65\hsize}}}
    }{\end{longtable}}%
    \renewcommand{\glossentry}[2]{%
        \glstarget{##1}{\glossentryname{##1}}%
        &\glossentrydesc{##1}{~##2.}
        \tabularnewline%
        \tabularnewline
    }%
}
\newglossary[not-glg]{notation}{not-gls}{not-glo}{Notation}
\newcommand*{\newnotation}[4][]{%
    \newglossaryentry{#2}{type=notation, name={\textbf{#3}, },
                          text={#4}, description={#4},#1}%
}
%--------------------------LENGTHS------------------------------%
% Spacings for the Table of Contents.
\addtolength{\cftsecnumwidth}{1ex}
\addtolength{\cftsubsecindent}{1ex}
\addtolength{\cftsubsecnumwidth}{1ex}
\addtolength{\cftfignumwidth}{1ex}
\addtolength{\cfttabnumwidth}{1ex}

% Indent and paragraph spacing.
\setlength{\parindent}{0em}
\setlength{\parskip}{0em}                                                           %
\makeindex[intoc]                                                              %
%----------------------------Main Document-------------------------------------%
\begin{document}
    \pagenumbering{gobble}
    \title{MATH 114 Algebraic Topology Notes}
    \author{%
        Professor: Vladimir Chernov\\
        Notes by: Ryan Maguire%
    }
    \date{\vspace{-5ex}}
    \maketitle
    \tableofcontents
    \pagenumbering{roman}
    \listoffigures
    \chapter{Homotopy}
        \pagenumbering{arabic}
        \section{Review}
            This section is meant to review some concepts from point-set
            topology. In particular, basic definitions, quotient spaces, and
            products. An excellent source for a more in-depth reading is
            \cite{Munkres2000}.
            \subsection{Topological Spaces}
                We start with the definition of topological spaces. We wish to
                generalize the notion of \textit{openness} that occurs in metric
                spaces\index{Metric Space}, or more concretely in the study of
                $\nspace$. To do this we axiomatize the properties of open sets:
                Taking arbitrary unions of open sets results in an open set, and
                the finite intersection of open sets is still open. Furthermore,
                we define the entire space and the empty set $\emptyset$ to be
                open. This gives us our definition of a topological space.
                \begin{fdefinition}{Topological Space}{Topological_Space}
                    A topological space is a set $X$ with a topology
                    $\tau\subseteq\powset{X}$, which is a collection of subsets
                    of $X$ called the \textit{open} subsets, such that:
                    \index{Topological Space}\index{Open Subset}
                    \begin{enumerate}
                        \item \label{def:top:Empty_and_X_Open}%
                              $\emptyset\in\tau$ and $X\in\tau$
                        \item \label{def:top:Finite_Intersections}%
                              For all $\mathcal{U},\mathcal{V}\in\tau$ it is
                              true that $\mathcal{U}\cap\mathcal{V}\in\tau$
                        \item \label{def:top:Arbitrary_Unions}%
                              For any subset $\mathcal{O}\subseteq\tau$ it is
                              true that $\bigcup\mathcal{O}\in\tau$
                    \end{enumerate}
                \end{fdefinition}
                \begin{example}
                    The standard metric topology on $\nspace[]$ is induced by
                    declaring $\mathcal{U}\subseteq\nspace[]$ to be open if and
                    only if for all $x\in\mathcal{U}$ there is an
                    $\varepsilon>0$ such that
                    $(x-\varepsilon,x+\varepsilon)\subseteq\mathcal{U}$. So an
                    open interval of the form $(a,b)$ is open
                    (see Fig.~\ref{fig:Open_Subset_of_R}).%
                    \index{Topology!on $\mathbb{R}$}
                \end{example}
                \begin{figure}[H]
                    \centering
                    \captionsetup{type=figure}
                    \begin{tikzpicture}[>=Latex]
    \coordinate (a)  at (2.0,  0.0);
    \coordinate (b)  at (9.0,  0.0);
    \coordinate (xt) at (4.0,  0.1);
    \coordinate (xb) at (4.0, -0.1);
    \coordinate (x)  at (4.0,  0.0);
    \coordinate (xl) at (3.0,  0.0);
    \coordinate (xr) at (5.0,  0.0);

    \draw[<->]   (0, 0) to (10, 0) node[above] {$\mathbb{R}$};
    \draw[thick] (a) to (b);
    \draw[thick] (b) arc (0:15:0.5);
    \draw[thick] (b) arc (0:-15:0.5);
    \draw[thick] (a) arc (180:195:0.5);
    \draw[thick] (a) arc (180:165:0.5);

    \draw (xt) to (xb);

    \node at (a)  [below=1ex] {$a$};
    \node at (b)  [below=1ex] {$b$};
    \node at (xl) [below=1ex] {$x-\varepsilon$};
    \node at (xr) [below=1ex] {$x+\varepsilon$};
    \node at (x)  [below=1ex] {$x$};

    \draw[blue, thick] (xr) arc (0:15:0.5);
    \draw[blue, thick] (xr) arc (0:-15:0.5);
    \draw[blue, thick] (xl) arc (180:195:0.5);
    \draw[blue, thick] (xl) arc (180:165:0.5);
    \draw[blue, thick] (xr) to (xl);
\end{tikzpicture}
                    \caption{An Open Subset of $\nspace[]$}
                    \label{fig:Open_Subset_of_R}
                \end{figure}
                By examining Fig.~\ref{fig:Open_Interval_Intersect_is_Open} we
                can convince ourselves that the intersection of open intervals
                is again open so long as we declare the empty set to be open.
                That is, $(0,1)$ and $(2,3)$ are open sets, but
                $(0,1)\cap(2,3)=\emptyset$. This highlights the need in
                requiring the empty set to be an element of the topology.
                \begin{figure}[H]
                    \centering
                    \captionsetup{type=figure}
                    \begin{tikzpicture}[>=Latex]
    \coordinate (a)  at (2.0,  0.0);
    \coordinate (b)  at (7.0,  0.0);
    \coordinate (c)  at (3.0,  0.0);
    \coordinate (d)  at (9.0,  0.0);
    \coordinate (x)  at (5.0,  0.0);
    \coordinate (xb) at (5.0, -0.1);
    \coordinate (xt) at (5.0,  0.1);
    \coordinate (xl) at (4.2,  0.0);
    \coordinate (xr) at (5.8,  0.0);

    \draw[<->]   (0, 0) to (10, 0) node[above] {$\mathbb{R}$};

    \node at (a)  [below=1ex] {$a$};
    \node at (b)  [below=1ex] {$b$};
    \node at (c)  [below=1ex] {$c$};
    \node at (d)  [below=1ex] {$d$};
    \node at (xl) [below=1ex] {$x-\varepsilon$};
    \node at (xr) [below=1ex] {$x+\varepsilon$};
    \node at (x)  [below=1ex] {$x$};
    \node at (a)  [above=1ex]
        {$\color{blue}{(a,b)}\cap\color{red}{(c,d)}=\color{Violet}{(c,b)}$};
    \node at (b) [above=1ex]
        {$\color{cyan}{(x-\varepsilon,x+\varepsilon)}%
            \subseteq\color{Violet}{(c,b)}$};

    \draw[blue,   thick]    (a)  to (c);
    \draw[red,    thick]    (b)  to (d);
    \draw[Violet, thick]    (c)  to (xl);
    \draw[Violet, thick]    (xr) to (b);
    \draw[cyan, very thick] (xl) to (xr);

    \draw[thin] (xt) to (xb);

    \draw[blue, very thick] (b) arc (0:15:0.5);
    \draw[blue, very thick] (b) arc (0:-15:0.5);
    \draw[blue, very thick] (a) arc (180:195:0.5);
    \draw[blue, very thick] (a) arc (180:165:0.5);

    \draw[red, very thick] (d) arc (0:15:0.5);
    \draw[red, very thick] (d) arc (0:-15:0.5);
    \draw[red, very thick] (c) arc (180:195:0.5);
    \draw[red, very thick] (c) arc (180:165:0.5);

    \draw[cyan, very thick] (xr) arc (0:15:0.5);
    \draw[cyan, very thick] (xr) arc (0:-15:0.5);
    \draw[cyan, very thick] (xl) arc (180:195:0.5);
    \draw[cyan, very thick] (xl) arc (180:165:0.5);
\end{tikzpicture}
                    \caption{The Intersection of Open Intervals is Open}
                    \label{fig:Open_Interval_Intersect_is_Open}
                \end{figure}
                \begin{example}
                    The topology induced on a metric space $\metspace{X}$ is
                    defined by making $\mathcal{U}\subseteq{X}$ open if and only
                    if for all $x\in\mathcal{U}$ there is an $\varepsilon>0$
                    such that the $\varepsilon$ ball centered about $x$ is
                    contained in $\mathcal{U}$:
                    $\rball{\varepsilon}{\metspace{X}}{x}\subseteq\mathcal{U}$
                    (see Fig.~\ref{fig:Open_Subset_Metric_Space}).%
                    \index{Topology!on a Metric Space}
                \end{example}
                \begin{figure}[H]
                    \centering
                    \captionsetup{type=figure}
                    \includegraphics{images/Open_Set_in_a_Metric_Space.pdf}
                    \caption{Open Subset of a Metric Space}
                    \label{fig:Open_Subset_Metric_Space}
                \end{figure}
                The metric topology on $\nspace[]$ is often called the
                \textit{standard} topology, but it is not the only one we can
                place on it.
                \begin{example}
                    \index{Topology!Chaotic/Indiscrete/Trivial}%
                    The chaotic topology, also called the trivial topology or
                    the indiscrete topology, is the simplest topology one can
                    define on a set. We write:
                    \begin{equation}
                        \tau=\{\,\emptyset,\,X\,\}
                    \end{equation}
                    This trivially satisfies the three properties enumerated in
                    Def.~\ref{def:Topological_Space}.
                \end{example}
                \begin{example}
                    \index{Topology!Discrete}%
                    The largest topology one can define is the entire power set:
                    \begin{equation}
                        \tau=\powset{X}
                    \end{equation}
                    Again, rather trivially, this is a topology on $X$. For
                    those who have studied metric spaces, it is called the
                    trivial topology since it is the topology induced by the
                    discrete metric:\index{Metric!Discrete}%
                    \begin{equation}
                        d(x,y)=
                        \begin{cases}
                            0,&x=y\\
                            1,&x\ne{y}
                        \end{cases}
                    \end{equation}
                \end{example}
                The closed interval $[a,b]$ is \textit{not} open in the standard
                topology (see Fig.~\ref{fig:Closed_Interval_Not_Open}), but it
                \textit{is} open with the discrete topology since everything is
                open in that topology. Because of this there is possible
                ambiguity with saying $\mathcal{U}$ is open if one has not
                specified the topology. Thankfully with most spaces one is
                interested in there is a standard or natural topology, and we
                usually choose this one without saying so. For metric spaces we
                almost always choose the metric topology.
                \begin{figure}[H]
                    \centering
                    \captionsetup{type=figure}
                    \begin{tikzpicture}[>=Latex]
    % Coordinates for various points.
    \coordinate (a)  at (4.0,  0.0);
    \coordinate (b)  at (9.0,  0.0);
    \coordinate (xl) at (3.0,  0.0);
    \coordinate (xr) at (5.0,  0.0);

    % Draw the real line.
    \draw[<->]   (0, 0) to (10, 0) node[above] {$\mathbb{R}$};

    % Draw the closed interval [a, b].
    \draw[thick] (a) to (b);

    % Add "brackets" indicating it is a closed interval.
    \draw[thick] (8.9, 0.1) to (9.0, 0.1) to (9.0, -0.1) to (8.9, -0.1);
    \draw[thick] (4.1, 0.1) to (4.0, 0.1) to (4.0, -0.1) to (4.1, -0.1);

    % Labels for the vaious points.
    \node at (a)  [below=1ex] {$a$};
    \node at (b)  [below=1ex] {$b$};
    \node at (xl) [below=1ex] {$a-\varepsilon$};
    \node at (xr) [below=1ex] {$a+\varepsilon$};

    % Draw the part of the open interval (a-e, a+e) that is inside of [a, b].
    \draw[blue, thick] (xr) arc (0:15:0.5);
    \draw[blue, thick] (xr) arc (0:-15:0.5);
    \draw[blue, thick] (a) to (xr);

    % Draw the part that falls outside.
    \draw[red, thick]  (xl) arc (180:195:0.5);
    \draw[red, thick]  (xl) arc (180:165:0.5);
    \draw[red, thick]  (a) to (xl);
\end{tikzpicture}
                    \caption{Closed Intervals are Not Open}
                    \label{fig:Closed_Interval_Not_Open}
                \end{figure}
                With the real line the intersection of finitely many open sets
                is still open, whereas we've only required the intersection of
                two open sets to still be open. We extend this to any finite
                collection by induction.
                \begin{theorem}
                    \label{thm:Finite_Intersections_Is_Open}%
                    If $\topspace{X}$ is a topological space and if
                    $\mathcal{O}\subseteq\tau$ is a finite collection of open
                    sets, then $\bigcap\mathcal{O}\in\tau$.
                \end{theorem}
                \begin{proof}
                    Apply induction to Def.~\ref{def:Topological_Space}
                    part \ref{def:top:Finite_Intersections}.
                \end{proof}
                Now that we've presented some examples, we define what it means
                for a set to be closed\index{Closed Subset}. In analysis we
                defined a closed set to be a set that has all of its limit
                points. For topology this is not general enough (topological
                spaces where sequences suffice to define closedness are called
                \textit{sequential spaces}\index{Sequential Space}). There is a
                standard theorem one comes across that a subset of $\nspace[]$
                is closed if and only if its complement is open. We take this
                theorem and adopt it as the definition of what it means to be a
                closed set in a general topological space.
                \begin{fdefinition}{Closed Subset}{Closed_Subset}
                    A closed subset of a topological space $\topspace{X}$ is a
                    subset $\mathcal{C}\subseteq{X}$ such that there exists an
                    open set $\mathcal{U}\in\tau$ with
                    $\mathcal{C}=X\setminus\mathcal{U}$.%
                    \index{Closed Subset}
                \end{fdefinition}
                That is, closed sets are the complements of open sets. There is
                a common misconception that closed sets are simply \textit{not}
                open sets, and vice-versa, but this is not so. In the discrete
                topology every set is open, and hence every set is closed. In
                the indiscrete topology there are no non-empty proper open
                subsets, and hence most sets are neither open nor closed. These
                examples show that openness and closedness are
                \textit{a priori} unrelated notions. If we know $A\subseteq{X}$
                is open and nothing more, we cannot conclude whether or not $A$
                is closed, and similarly if we know $B\subseteq{X}$ is closed
                and nothing more, then we cannot conclude whether or not $B$ is
                open.
                \begin{example}
                    The cocountable topology on $\mathbb{R}$ is the standard
                    example of a space where sequences are insufficient to
                    describe closedness. A subset of $\mathbb{R}$ is declared
                    open if it's complement is a coutable set. In this topology
                    \textit{every} subset of $\mathbb{R}$ has it's limit points
                    and hence every set is \textit{sequentially} closed, but
                    this is \textit{not} the discrete topology. That is, not
                    every set is \textit{topologically} closed, i.e. the
                    complement of an open subset.
                \end{example}
                Using the idempotent laws of complement, we obtain the
                following:
                \begin{theorem}
                    \label{thm:Closed_Iff_Comp_is_Open}%
                    If $\topspace{X}$ is a topological space and
                    $C\subseteq{X}$, then $C$ is closed if and only if
                    $X\setminus{C}$ is open.
                \end{theorem}
                \begin{proof}
                    For if $X\setminus{C}$ is open, then
                    $X\setminus(X\setminus{C})$ is closed
                    (Def.~\ref{def:Closed_Subset}). From the idempotent law of
                    complements, $X\setminus(X\setminus{C})=C$ and hence $C$ is
                    closed. By a similar argument if $C$ is closed, then
                    $X\setminus{C}$ is open.
                \end{proof}
                Using De Morgan's laws we can define an equivalent notion of
                topological spaces using closed sets. De Morgan's laws state for
                sets $A,B,X$, with $A,B\subseteq{X}$, the following is true:%
                \index{De Morgan's Laws}
                \begin{subequations}
                    \begin{align}
                        X\setminus(A\cap{B})
                            &=(X\setminus{A})\cup(X\setminus{B})\\
                        X\setminus(A\cup{B})
                            &=(X\setminus{A})\cap(X\setminus{B})
                    \end{align}
                \end{subequations}
                We can write this more suggestively if we let
                $X\setminus{A}=A^{C}$ ($C$ for complement).
                \vspace{-5ex}
                \twocolumneq{(A\cap{B})^{C}=A^{C}\cup{B}^{C}}
                            {(A\cup{B})^{C}=A^{C}\cap{B}^{C}}
                De Morgan's laws hold for arbitrary unions and intersections:
                \twocolumneq{%
                    \Big(\bigcap\mathcal{U}\Big)^{C}=\bigcup\mathcal{U}^{C}%
                }{%
                    \Big(\bigcup\mathcal{U}\Big)^{C}=\bigcap\mathcal{U}^{C}%
                }
                With this we may prove the following.
                \begin{theorem}
                    If $\topspace{X}$ is a topological space, then $\emptyset$
                    and $X$ are closed, if $\mathcal{C},\mathcal{D}\subseteq{X}$
                    are closed, then $\mathcal{C}\cup\mathcal{D}$ is closed, and
                    if $\Lambda\subseteq\powset{X}$ is a collection of closed
                    subsets of $X$, then $\bigcap\Lambda$ is closed.
                \end{theorem}
                \begin{proof}
                    For the first part, apply the definition of closed subsets
                    (Def.~\ref{def:Closed_Subset}) to
                    Def.~\ref{def:Topological_Space} part
                    \ref{def:top:Empty_and_X_Open} and recall that
                    $X\setminus\emptyset=X$ and $X\setminus{X}=\emptyset$. For
                    the latter parts, combine De Morgan's Laws with
                    Def.~\ref{def:Topological_Space} parts
                    \ref{def:top:Finite_Intersections} and
                    \ref{def:top:Arbitrary_Unions}, respectively.
                \end{proof}
                We may expand the union of two closed sets to the unions of
                finitely many closed sets inductively as we did in
                Thm.~\ref{thm:Finite_Intersections_Is_Open}.
            \subsection{Continuity}
                In an analysis course one talks about continuous functions by
                one of two equivalent means: The $\varepsilon-\delta$ definition
                and by the limit of sequences.
                \par\hfill\par
                \begin{minipage}[t]{0.50\textwidth}
                    The $\epsilon-\delta$ definition states that if we move at
                    most $\delta$ in the $x$ direction, then we will have no
                    more than $\varepsilon$ amount of error in the $y$
                    direction (see Fig.~\ref{fig:Eps_Delta_Def_Cont}), the
                    sequence definition saying that as we get arbitrarily close
                    to $x_{0}$ by any sequence
                    $a:\mathbb{N}\rightarrow\nspace[]$, then $f(a_{n})$
                    approaches $f(x_{0})$
                    (see Fig.~\ref{fig:Sequence_Def_Continuity}). Both of these
                    definitions require a metric which a general topological
                    space may not have. While one can talk about convergence of
                    sequences in a topological space, this is insufficient to
                    fully descibe continuity for many spaces.
                \end{minipage}
                \hfill
                \fbox{%
                    \begin{minipage}[t]{0.44\textwidth}
                        \begin{figure}[H]
                            \centering
                            \captionsetup{type=figure}
                            \includegraphics{images/Continuity_Epsilon_Delta_Def.pdf}
                            \caption{The $\varepsilon-\delta$ Definition of Continuity}
                            \label{fig:Eps_Delta_Def_Cont}
                        \end{figure}
                    \end{minipage}
                }
                \par\hfill\par
                \fbox{%
                    \begin{minipage}[t]{0.44\textwidth}
                        \begin{figure}[H]
                            \centering
                            \captionsetup{type=figure}
                            \includegraphics{images/Continuity_Sequence_Definition.pdf}
                            \caption{Sequence Definition of Continuity}
                            \label{fig:Sequence_Def_Continuity}
                        \end{figure}
                    \end{minipage}
                }
                \hfill
                \begin{minipage}[t]{0.50\textwidth}
                    We seek a definition that is general enough to apply to any
                    topological space and which agrees with continuity in the
                    familiar setting of a metric space. This is obtain by
                    considering the following fundamental characterization of
                    continuous real-valued functions. A function
                    $f:\mathbb{R}\rightarrow\mathbb{R}$ is continuous
                    \textit{if and only if} for every open subset
                    $\mathcal{U}\subseteq\mathbb{R}$ it is true that the
                    \textit{pre-image} of $\mathcal{U}$ under $f$ is open. That
                    is, $f^{\minus{1}}[\mathcal{U}]\subseteq\mathbb{R}$ is open.
                    This can be worded concisely by the phrase
                    \textit{the pre-image of open is open}.
                \end{minipage}
                \par\hfill\par
                Since this is an if and only if we may adopt it as our
                definition of continuity, and this only uses the notion of an
                open set which topological spaces do have.
                \begin{fdefinition}{Continuous Function}{Continuous_Function}
                    A continuous function from a topological space
                    $\topspace[X]{X}$ to a topological space $\topspace[Y]{Y}$
                    is a function $f:X\rightarrow{Y}$ such that for every open
                    subset $\mathcal{V}\in\tau_{Y}$ it is true that
                    $f^{\minus{1}}[\mathcal{V}]\in\tau_{X}$. The pre-image of
                    open is open.\index{Continuous Function}
                \end{fdefinition}
                \begin{example}
                    If $\topspace{X}$ is any topological space, if
                    $Y$ is a set, and if we choose the chaotic topology
                    $\{\emptyset,Y\}$ on $Y$, then any function
                    $f:X\rightarrow{Y}$ is automatically continuous. There are
                    only two open subsets to check and:
                    \twocolumneq{f^{\minus{1}}[\emptyset]=\emptyset}
                                {f^{\minus{1}}[Y]=X}
                    Both of which are open subsets, and hence $f$ is continuous.
                    This is one justification for calling this the chaotic
                    topology: Every function is continuous. There are other
                    reasons, every sequence converges to every point
                    simultaneously, no points can be separated, and so on.
                \end{example}
                \begin{example}
                    If $\topspace{Y}$ is a topological space, if $X$ is a set,
                    and if we choose the discrete topology $\powset{X}$ on $X$,
                    then any function $f:X\rightarrow{Y}$ is continuous. Given
                    any open subset $\mathcal{V}\in\tau$, the pre-image
                    $f^{\minus{1}}[\mathcal{V}]$ is a subset of $X$ by
                    definition and hence is an element of $\powset{X}$. That is,
                    $f$ is continuous.
                \end{example}
                An important concept in topology is that of a
                \textit{homeomorphism}. A homeomorphism is a continuous function
                $f:X\rightarrow{Y}$ that is bijective and such that the inverse
                function $f^{\minus{1}}:Y\rightarrow{X}$ is also continuous.
                If there exists a homeomorphism between topological spaces
                $\topspace[X]{X}$ and $\topspace[Y]{Y}$, then we call them
                \textit{homeomorphic}. Topologically, homeomorphic spaces are
                indistinguishable.\index{Homeomorphism}
                \begin{example}
                    The open interval $(\minus\frac{\pi}{2},\frac{\pi}{2})$ is
                    homeomorphic to the entire real line via the tangent
                    function $\tan:(\minus\frac{\pi}{2},\frac{\pi}{2})%
                    \rightarrow\nspace[]$. This is continuous, bijective, and
                    has a continuous inverse $\arctan$. This example shows that
                    boundedness is a metric property and not a topological one.
                \end{example}
                One question that often arises is when can we conclude two
                spaces are homeomorphic. Many simple conditions that hold in
                $\nspace[]$ do not generalize. For example, any continuous
                bijection $f:\nspace[]\rightarrow\nspace[]$ is automatically a
                homeomorphism since the inverse will be continuous. This is a
                consequence of the intermediate value theorem which implies any
                such function is then strictly monotonic. This does not
                generalize to arbitrary topological spaces, it doesn't even
                generalize to subspaces of $\nspace$. We can continuously and
                bijectively map $[0,1)$ to the unit circle $\nsphere[1]$ by
                $f(x)=\big(\cos(2\pi{x}),\sin(2\pi{x})\big)$, but the inverse is
                not continuous. Rigorously it can't be since $\nsphere[1]$ is
                \textit{compact} but $[0,1)$ is not, and homeomorphisms preserve
                such a notion. Intuitively, we have tied up the ends of $[0,1)$
                continuously, but going from the unit circle to $[0,1)$ involves
                ripping the circle somewhere, which is not a continuous
                operation.
                \par\hfill\par
                Another false proposition is that if $f:X\rightarrow{Y}$ and
                $g:Y\rightarrow{X}$ are continuous bijections, then the two
                spaces are homeomorphic. This is false. Consider the following
                subspaces of $\nspace[2]$ shown in
                Fig.~\ref{fig:Non_Homeomorphic_Subspace}.
                \begin{figure}[H]
                    \centering
                    \captionsetup{type=figure}
                    \includegraphics{images/Non_Homeo_with_Cont_Bij}
                    \caption{Non-Homeomorphic Subspaces of $\nspace[2]$}
                    \label{fig:Non_Homeomorphic_Subspace}
                \end{figure}
                First, let's describe these spaces. Both spaces consist of the
                $x$ axis in the plane $\nspace[2]$. We then glue the half-open
                interval $[0,1)$ upwards in the $y$ direction starting at
                $(0,0)$ and then continuing doing this to the \textit{left}
                forever. Next, we glue circles starting at $(3,0)$ and continue
                to the \textit{right} forever. The only difference is in the
                middle. For the top space we attach a circle at $(1,0)$ and a
                half-open interval at $(2,0)$. For the bottom we simply attach
                two circles.
                \par\hfill\par
                These are \textit{not} homeomorphic. To see this we need to know
                that a homeomorphism $f:X\rightarrow{Y}$ induces a
                homeomorphism between any subspace $A\subseteq{X}$ and its image
                $f[A]\subseteq{Y}$. Thus we need to find a subspace of the first
                image that does not exist in the second, or vice versa. If we
                remove the bottom point from the interval wedged between two
                circles, then we are left with three parts: an open interval,
                a string of infinitely many circles, and a string of infinitely
                many intervals plus one circle. No matter which point we remove
                from the second image the resulting subspace will not look like
                this, and hence these two spaces are not homeomorphic. However,
                there are continuous bijections $f:X\rightarrow{Y}$ and
                $g:Y\rightarrow{X}$. For this we will avoid formulas, and rather
                describe the function pictorially. For $f:X\rightarrow{Y}$ we
                take the interval wedged between the two circles and tie it up.
                This is identical to the function from $[0,1)$ to $\nsphere[1]$
                which was both continuous and bijective. The result of this
                procedure is continuous, bijective, and gives us the second
                image. For $g:Y\rightarrow{X}$, we look at the interval two bits
                away from the neareast circle and tie this up. We then translate
                everything to the right by two, yielding the first space.
                \par\hfill\par
                Now one might ask if we needed an infinitely large space, or at
                the very least a \textit{non-compact} one, and the answer is
                yes. If $X$ is compact, and if $Y$ is \textit{Hausdorff}, then
                any continuous bijection $f:X\rightarrow{Y}$ is automatically a
                homeomorphism. To find counterexamples to the general claim we
                must then avoid such spaces.
            \subsection{More Notions to Review}
                \begin{figure}
                    \centering
                    \captionsetup{type=figure}
                    \begin{subfigure}[b]{0.49\textwidth}
                        \centering
                        \includegraphics{images/Open_Rectangle_R2.pdf}
                        \subcaption{The Open Rectangle $(a,b)\times(c,d)$.}
                        \label{fig:Open_Rectangle_in_R2}
                    \end{subfigure}
                    \begin{subfigure}[b]{0.49\textwidth}
                        \centering
                        \includegraphics{images/Open_Not_Rectangle_R2.pdf}
                        \subcaption{A Region Not of the Form
                                    $\mathcal{U}\times\mathcal{V}$}
                        \label{fig:Open_Subset_Not_Product}
                    \end{subfigure}
                    \caption{Examples of Open Subsets of $\mathbb{R}^{2}$.}
                    \label{fig:Point_Set_Top_Open_Subsets_R2}
                \end{figure}
                It is hoped that the material previously discussed is all
                review. There are other notions needed that one should quickly
                read through if they have not seen them before, namely the
                notion of \textit{product spaces} and \textit{quotient spaces}.
                For product spaces we'll need to following theorem.
                \begin{ltheorem}{The Intersections of Topologies is a Topology}
                                {Intersection_of_Topologies_is_Topology}
                    If $X$ is a set and $T\subseteq\powset{\powset{X}}$ is a
                    non-empty collection of sets such that for all $\tau\in{T}$
                    it is true that $\tau$ is a topology on $X$, then
                    $\bigcap{T}$ is a topology on $X$.
                \end{ltheorem}
                \begin{proof}
                    Since every element of $T$ is a topology, $\emptyset$ and
                    $X$ are contained in all $\tau\in{T}$ and hence
                    $\emptyset,X\in\bigcap{T}$. If we have a collection
                    $\mathcal{O}$ of elements of $\bigcap{T}$, then each element
                    is in every topology $\tau\in{T}$ by the definition of
                    intersection. But then since all $\tau\in{T}$ are topologies
                    we know that $\bigcup\mathcal{O}\in\tau$. Since this is true
                    of all $\tau$ we conclude $\bigcup\mathcal{O}\in\bigcap{T}$.
                    By a similar argument, $\bigcap{T}$ is closed to finite
                    intersections. Hence, $\bigcap{T}$ is a topology on $X$.
                \end{proof}
                We can use this to define the product topology generated by the
                Cartesian product of two topological spaces. If we have two
                topological spaces $\topspace[X]{X}$ and $\topspace[Y]{Y}$ we
                would like to place a topology on $X\times{Y}$. Moreover we
                would like this topology to agree with product spaces we already
                know well, such as the standard topologies on $\nspace$ or the
                topologies induced by the product of metric spaces. We would
                like to make life simple and set the topology to be something
                like $\tau_{X}\times\tau_{Y}$, but this may not have the union
                property of topologies. Indeed, if we let $X=\nspace[]$ and
                $Y=\nspace[]$ then $\tau_{X}\times\tau_{Y}$ is the set of all
                $\mathcal{U}\times\mathcal{V}$ where $\mathcal{U}$ and
                $\mathcal{V}$ are open. Since open subsets of $\nspace[]$ can be
                written as the countable union of open intervals, we can suppose
                for the sake of visualization that $\mathcal{U}=(a,b)$ and
                $\mathcal{V}=(c,d)$ with $a,b,c,d\in\nspace[]$. The product
                $\mathcal{U}\times\mathcal{V}$ is an \textit{open rectangle}
                (see Fig.~\subref{fig:Open_Rectangle_in_R2}). A blob like the
                one in Fig.~\subref{fig:Open_Subset_Not_Product} is considered
                open in the standard topology on $\nspace[2]$ but cannot be
                written in the form $\mathcal{U}\times\mathcal{V}$.
                \par\hfill\par
                \begin{minipage}[c]{0.50\textwidth}
                    We use
                    Thm.~\ref{thm:Intersection_of_Topologies_is_Topology} to
                    define the product topology. Since the power set
                    $\powset{X\times{Y}}$ is a topology on $X\times{Y}$, the set
                    of all topologies containing $\tau_{X}\times\tau_{Y}$ is
                    non-empty. We define the product topology to be the
                    \textit{smallest} topology that contains
                    $\tau_{X}\times\tau_{Y}$. To make this precise we define
                    $\tau_{X\times{Y}}$ to be $\bigcap\Lambda$ where $\Lambda$
                    is the set of all topologies containing
                    $\tau_{X}\times\tau_{Y}$. By
                    Thm.~\ref{thm:Intersection_of_Topologies_is_Topology} this
                    is indeed a topology on $X\times{Y}$.%
                    \index{Product Topology}\index{Topology!Product Topology}
                    Another way of describing this is by saying it is the
                    topology \textit{generated} by all of the elements of
                    $\tau_{X}\times\tau_{Y}$. We take elements of
                    $\tau_{X}\times\tau_{Y}$ and then add their unions and
                    intersections until we have a valid topology.
                \end{minipage}
                \hfill
                \fbox{%
                    \begin{minipage}[c]{0.44\textwidth}
                        \begin{figure}[H]
                            \centering
                            \captionsetup{type=figure}
                            \resizebox{!}{0.75\height}{%
                                \begin{tikzpicture}[>=Latex]
    \draw[<->, thick] (-3.3, 0) to (3.3, 0) node [above] {$x$};
    \draw[<->, thick] (0, -3.3) to (0, 3.3) node [right] {$y$};
    \draw[densely dashed] (0, 0) circle (1in);

    % First Layer
    \draw[fill=cyan, opacity=0.6, densely dashed]
        (0.7071in, 0.7071in) to (-0.7071in, 0.7071in)
                             to (-0.7071in, -0.7071in)
                             to (0.7071in, -0.7071in)
                             to cycle;
    
    % Second Layer
    \draw[fill=green, opacity=0.5, densely dashed]
        (0.68in, 0.3535in) to (0.935in, 0.3535in)
                           to (0.935in, -0.3535in)
                           to (0.68in, -0.3535in)
                           to cycle;
    \draw[fill=green, opacity=0.5, densely dashed]
        (-0.68in, 0.3535in) to (-0.935in, 0.3535in)
                            to (-0.935in, -0.3535in)
                            to (-0.68in, -0.3535in)
                            to cycle;
    \draw[fill=green, opacity=0.5, densely dashed]
        (0.3535in, 0.68in) to (0.3535in, 0.935in)
                           to (-0.3535in, 0.935in)
                           to (-0.3535in, 0.68in)
                           to cycle;
    \draw[fill=green, opacity=0.5, densely dashed]
        (0.3535in, -0.68in) to (0.3535in, -0.935in)
                            to (-0.3535in, -0.935in)
                            to (-0.3535in, -0.68in)
                            to cycle;

    % Third Layer.
    \draw[fill=orange, opacity=0.6, densely dashed]
        (0.68in, 0.3535in) to (0.8212in, 0.3535in)
                           to (0.8212in, 0.5705in)
                           to (0.68in, 0.5707in)
                           to cycle;
    \draw[fill=orange, opacity=0.6, densely dashed]
        (0.68in, -0.3535in) to (0.8212in, -0.3535in)
                            to (0.8212in, -0.5705in)
                            to (0.68in, -0.5707in)
                            to cycle;
    \draw[fill=orange, opacity=0.6, densely dashed]
        (-0.68in, -0.3535in) to (-0.8212in, -0.3535in)
                             to (-0.8212in, -0.5705in)
                             to (-0.68in, -0.5707in)
                             to cycle;
    \draw[fill=orange, opacity=0.6, densely dashed]
        (-0.68in, 0.3535in) to (-0.8212in, 0.3535in)
                            to (-0.8212in, 0.5705in)
                            to (-0.68in, 0.5707in)
                            to cycle;
    \draw[fill=orange, opacity=0.6, densely dashed]
        (0.3535in, 0.68in) to (0.3535in, 0.8212in)
                           to (0.5705in, 0.8212in)
                           to (0.5707in, 0.68in)
                           to cycle;
    \draw[fill=orange, opacity=0.6, densely dashed]
        (0.3535in, -0.68in) to (0.3535in, -0.8212in)
                            to (0.5705in, -0.8212in)
                            to (0.5707in, -0.68in)
                            to cycle;
    \draw[fill=orange, opacity=0.6, densely dashed]
        (-0.3535in, 0.68in) to (-0.3535in, 0.8212in)
                            to (-0.5705in, 0.8212in)
                            to (-0.5707in, 0.68in)
                            to cycle;
    \draw[fill=orange, opacity=0.6, densely dashed]
        (-0.3535in, -0.68in) to (-0.3535in, -0.8212in)
                             to (-0.5705in, -0.8212in)
                             to (-0.5707in, -0.68in)
                             to cycle;

    % Fourth Layer
    \draw[fill=red, opacity=0.5, densely dashed]
        (0.2in, 0.93in) to (0.2in, 0.9797in)
                        to (-0.2in, 0.9797in)
                        to (-0.2in, 0.93in)
                        to cycle;
    \draw[fill=red, opacity=0.5, densely dashed]
        (0.2in, -0.93in) to (0.2in, -0.9797in)
                         to (-0.2in, -0.9797in)
                         to (-0.2in, -0.93in)
                         to cycle;
    \draw[fill=red, opacity=0.5, densely dashed]
        (0.93in, 0.2in) to (0.9797in, 0.2in)
                        to (0.9797in, -0.2in)
                        to (0.93in, -0.2in)
                        to cycle;
    \draw[fill=red, opacity=0.5, densely dashed]
        (-0.93in, 0.2in) to (-0.9797in, 0.2in)
                         to (-0.9797in, -0.2in)
                         to (-0.93in, -0.2in)
                         to cycle;

    % Fifth Layer
    \draw[fill=blue, opacity=0.6, densely dashed]
        (0.82in, 0.3535in) to (0.8781in, 0.3535in)
                           to (0.8781in, 0.4784in)
                           to (0.82in, 0.4784in)
                           to cycle;
    \draw[fill=blue, opacity=0.6, densely dashed]
        (0.82in, -0.3535in) to (0.8781in, -0.3535in)
                            to (0.8781in, -0.4784in)
                            to (0.82in, -0.4784in)
                            to cycle;
    \draw[fill=blue, opacity=0.6, densely dashed]
        (-0.82in, 0.3535in) to (-0.8781in, 0.3535in)
                            to (-0.8781in, 0.4784in)
                            to (-0.82in, 0.4784in)
                            to cycle;
    \draw[fill=blue, opacity=0.6, densely dashed]
        (-0.82in, -0.3535in) to (-0.8781in, -0.3535in)
                             to (-0.8781in, -0.4784in)
                             to (-0.82in, -0.4784in)
                             to cycle;
    \draw[fill=blue, opacity=0.6, densely dashed]
        (0.3535in, 0.82in) to (0.3535in, 0.8781in)
                           to (0.4784in, 0.8781in)
                           to (0.4784in, 0.82in)
                           to cycle;
    \draw[fill=blue, opacity=0.6, densely dashed]
        (0.3535in, -0.82in) to (0.3535in, -0.8781in)
                            to (0.4784in, -0.8781in)
                            to (0.4784in, -0.82in)
                            to cycle;
    \draw[fill=blue, opacity=0.6, densely dashed]
        (-0.3535in, -0.82in) to (-0.3535in, -0.8781in)
                             to (-0.4784in, -0.8781in)
                             to (-0.4784in, -0.82in)
                             to cycle;
    \draw[fill=blue, opacity=0.6, densely dashed]
        (-0.3535in, 0.82in) to (-0.3535in, 0.8781in)
                            to (-0.4784in, 0.8781in)
                            to (-0.4784in, 0.82in)
                            to cycle;

    % Sixth Layer
    \draw[fill=yellow, opacity=0.6, densely dashed]
        (0.68in, 0.5705in) to (0.7641in, 0.5705in)
                           to (0.7641in, 0.645in)
                           to (0.68in, 0.645in)
                           to cycle;
    \draw[fill=yellow, opacity=0.6, densely dashed]
        (0.68in, -0.5705in) to (0.7641in, -0.5705in)
                            to (0.7641in, -0.645in)
                            to (0.68in, -0.645in)
                            to cycle;
    \draw[fill=yellow, opacity=0.6, densely dashed]
        (-0.68in, -0.5705in) to (-0.7641in, -0.5705in)
                             to (-0.7641in, -0.645in)
                             to (-0.68in, -0.645in)
                             to cycle;
    \draw[fill=yellow, opacity=0.6, densely dashed]
        (-0.68in, 0.5705in) to (-0.7641in, 0.5705in)
                            to (-0.7641in, 0.645in)
                            to (-0.68in, 0.645in)
                            to cycle;
    \draw[fill=yellow, opacity=0.6, densely dashed]
        (0.5705in, 0.68in) to (0.5705in, 0.7641in)
                           to (0.645in, 0.7641in)
                           to (0.645in, 0.68in)
                           to cycle;
    \draw[fill=yellow, opacity=0.6, densely dashed]
        (0.5705in, -0.68in) to (0.5705in, -0.7641in)
                            to (0.645in, -0.7641in)
                            to (0.645in, -0.68in)
                            to cycle;
    \draw[fill=yellow, opacity=0.6, densely dashed]
        (-0.5705in, -0.68in) to (-0.5705in, -0.7641in)
                             to (-0.645in, -0.7641in)
                             to (-0.645in, -0.68in)
                             to cycle;
    \draw[fill=yellow, opacity=0.6, densely dashed]
        (-0.5705in, 0.68in) to (-0.5705in, 0.7641in)
                            to (-0.645in, 0.7641in)
                            to (-0.645in, 0.68in)
                            to cycle;
\end{tikzpicture}%
                            }
                            \caption{Tiling of $\nball[2]$}
                            \label{fig:Tiling_Open_Disc_by_Rectangles}
                        \end{figure}
                    \end{minipage}
                }
                \par\vspace{2.5ex}
                In this way we see that an open disk is an open subset of the
                plane since we can tile it with rectangles
                (Fig.~\ref{fig:Tiling_Open_Disc_by_Rectangles}).
                \begin{example}
                    The torus can be viewed as the Cartesian product of two
                    circles equipped with the product topology
                    (see Fig.~\ref{fig:Torus_as_Prod_Space}).
                \end{example}
                \begin{figure}[H]
                    \centering
                    \captionsetup{type=figure}
                    \includegraphics{images/Torus_Skeleton_Product_Space.pdf}
                    \caption{The Torus $\ntorus[]=\nsphere[1]\times\nsphere[1]$}
                    \label{fig:Torus_as_Prod_Space}
                \end{figure}
                Another way of viewing the torus involves quotient spaces. We
                take a square piece of paper and then roll it up into a
                cylinder. From here we stretch the bounding circles around and
                glue them together. This process is shown in
                Fig.~\ref{fig:Square_to_Torus}. The result is a donut shaped
                object that is homeomorphic to $\nsphere[1]\times\nsphere[1]$.
                \begin{figure}[H]
                    \centering
                    \captionsetup{type=figure}
                    \includegraphics{images/Square_to_Torus.pdf}
                    \caption{Gluing a Square into a Torus}
                    \label{fig:Square_to_Torus}
                \end{figure}
                This is made rigorous by the quotient topology induced by an
                equivalence relation $R$ on a topological space
                $\topspace{X}$\index{Topology!Quotient}. We look at the quotient
                set $X/R$ and topologize this by choosing the \textit{largest}
                topology that makes the projection map $q:X\rightarrow{X}/R$
                continuous. There's always such a topology since the chaotic
                topology $\{\emptyset,X/R\}$ does the trick, but this is too
                small and often boring. The quotient topology is defined by:%
                \index{Topology!Quotient Topology}\index{Quotient Topology}
                \begin{equation}
                    \tau_{X/R}=\{\,\mathcal{V}\subseteq{X}/R\;|\;
                        q^{\minus{1}}[\mathcal{V}]\in\tau\,\}
                \end{equation}
                The technical details are perhaps not too important for now, one
                should simply think of gluing together the parts identified by
                the equivalence relation $R$. Returning to the torus
                $\ntorus[]$, we may define it as the quotient space obtained by
                identifying parts of the closed unit square together.
                Explicitly, we identify $(x,0)$ with $(x,1)$ for all $x\in[0,1]$
                and similarly $(0,y)$ with $(1,y)$ for $y\in[0,1]$.
        \section{Homotopy Equivalence}
            With the basics of point-set topology now reviewed, we move onto to
            algebraic topology. Historically one of the first objects studied in
            this field was the \textit{fundamental group}%
            \index{Fundamental Group}, a concept related to \textit{homotopy}.
            To motivate this we first give an informal discussion on deformation
            retractions which is a very geometric and visual concept. We then
            rigorously define homotopy and retracts, which will make it easier
            to give a formal definition of deformation retractions.
            \subsection{Deformation Retraction}
                A deformation retraction\index{Deformation Retraction} from a
                topological space $\topspace{X}$ onto a subset $A\subseteq{X}$
                is a family of continuous functions $f_{t}:X\rightarrow{X}$
                indexed by the closed unit interval $I=[0,1]$ where
                $f_{0}=\identity{X}$, $f_{1}[X]=A$, and such that
                $f_{t}|_{A}=\identity{A}$ for all $t\in{I}$. Moreover, the
                function $H:X\times{I}\rightarrow{X}$ defined by
                $H(x,t)=f_{t}(x)$ should be continuous with respect to the
                product topology, the closed unit interval $I$ inheriting it's
                topology from $\nspace[]$.
                \begin{example}
                    We can take an annulus or a punctured plane, either will
                    do, and construct a deformation retract from this onto the
                    unit circle $\nsphere[1]$. We do this by noting the function
                    $h$ defined on $\nspace[2]\setminus\{(0,0)\}$ by:
                    \begin{equation}
                        h(\vector{x})=\frac{\vector{x}}{\norm{\vector{x}}_{2}}
                    \end{equation}
                    where $\norm{\vector{x}}_{2}$ is the \textit{Euclidean Norm}%
                    \index{Euclidean Norm} of the point $\vector{x}$, maps the
                    punctured plane $\nspace[2]\setminus\{(0,0)\}$ onto the unit
                    circle. To complete our deformation retraction we drag the
                    point $\vector{x}\ne\vector{0}$ along the straight line
                    between $\vector{x}$ and $h(\vector{x})$. We have:
                    \begin{equation}
                        H(\vector{x},t)
                        =(1-t)\cdot\vector{x}+t\cdot{h}(\vector{x})
                    \end{equation}
                    This is a deformation retract since elements of the unit
                    circle are held fixed for all $t\in[0,1]$. If
                    $\vector{s}\in\nsphere[1]$, then by definition
                    $\norm{\vector{s}}_{2}=1$ and hence
                    $h(\vector{s})=\vector{s}$. Simplifying, we have:
                    \begin{equation}
                        H(\vector{s},t)=(1-t)\cdot\vector{s}+t\cdot\vector{s}
                            =\vector{s}
                    \end{equation}
                    Thus $H$ gives us a deformation retraction of the punctured
                    plane onto the unit circle. We can do the same deformation
                    retraction with an annulus if we simply restrict $h$ to that
                    domain (see Fig.~\ref{fig:Def_Retract_Annulus_to_Circle}).
                \end{example}
                \par
                \begin{minipage}[t]{0.50\textwidth}
                    This family does not give us a homeomorphism since these
                    spaces are not homeomorphic. This shows deformation
                    retractions are, in general, weaker. For our second example
                    we must first define what a M\"{o}bius strip is. Similar to
                    how th torus was defined by gluing together parts
                    of the unit square, so is the M\"{o}bius strip but with a
                    twist introduced. Explicity, we take $[0,1]\times[0,1]$ and
                    identify $(0,y)$ with $(1,1-y)$, leaving $x$ alone. This
                    processed is outlined in
                    Fig.~\ref{fig:Square_to_Mobius_Strip}.
                \end{minipage}
                \hfill
                \fbox{%
                    \begin{minipage}[t]{0.44\textwidth}
                        \begin{figure}[H]
                            \centering
                            \captionsetup{type=figure}
                            \includegraphics{images/Homotopy_Circle.pdf}
                            \caption{A Deformation Retraction}
                            \label{fig:Def_Retract_Annulus_to_Circle}
                        \end{figure}
                    \end{minipage}
                }
                \par\vspace{2.5ex}
                \begin{figure}
                    \centering
                    \captionsetup{type=figure}
                    \includegraphics{images/Square_to_Mobius_Strip.pdf}
                    \caption{Gluing a Square into a M\"{o}bius Strip}
                    \label{fig:Square_to_Mobius_Strip}
                \end{figure}
                For those familiar with the language, the outcome is a manifold
                with boundary. There is an inner circle wrapping around the
                M\"{o}bius strip that we can retract the space to similar to the
                way the annulus can be deformation retracted. It is perhaps best
                if we depict this with a 3D drawing. To do so we embed the
                M\"{o}bius strip into $\nspace[3]$ with the following map:
                \begin{equation}
                    \varphi(x,y)=
                    \Big(\big(1+\frac{y}{2}\cos(\frac{x}{2})\big)\cos(x),\,
                         \big(1+\frac{y}{2}\cos(\frac{x}{2})\big)\sin(x),\,
                         \frac{y}{2}\sin(\frac{x}{2})\Big)
                \end{equation}
                The domain is $[0,2\pi)\times[\minus{1},1)$. The inner circle is
                obtained when $y=0$. Using this we may draw our deformation
                retraction (Fig.~\ref{fig:Def_Retract_Mobius_Strip}). The
                family of maps $f_{t}(x,t)$ can be obtained by mapping $f(x,y)$
                to $f\big(x,(1-t)y\big)$.
                \begin{figure}[H]
                    \centering
                    \captionsetup{type=figure}
                    \includegraphics{%
                        images/Mobius_Strip_Def_Retract_Inner_Circle%
                    }
                    \caption{Deformation Retraction of a M\"{o}bius Strip}
                    \label{fig:Def_Retract_Mobius_Strip}
                \end{figure}
                Subspaces which we retract a space onto need not be unique, nor
                need they be homeomorphic. Indeed, any space can be deformation
                retracted to itself by the function $H(x,t)=\identity{X}(x)$.
                As noted the annulus is not homeomorphic to the circle but both
                are deformation retractions of the same space. There are more
                subtle examples which we describe now.
                \par\hfill\par
                Consider the plane $\nspace[2]$ with two holes in it,
                say at $(\minus{1},0)$ and $(1,0)$. There is a deformation
                retraction of this space onto a figure eight. For the definition
                of a figure eight let's use the lemniscate of Gerono, studied
                by the French mathematician Camille-Christophe Gerono in
                the $19^{th}$ century C.E. The defining implicit equation goes
                as follows:
                \begin{equation}
                    x^{4}-x^{2}+y^{2}=0
                \end{equation}
                Setting $x(\theta)=\cos(\theta)$, we have:
                \begin{equation}
                    y^{2}=\cos^{2}(\theta)\big(1-\cos^{2}(\theta)\big)
                         =\cos^{2}(\theta)\sin^{2}(\theta)
                \end{equation}
                so we parameterize the figure eight by:
                \begin{equation}
                    \big(x(\theta),\,y(\theta)\big)
                        =\big(\cos(\theta),\,\cos(\theta)\sin(\theta)\big)
                \end{equation}
                Now we must retract the doubly punctured plane onto this object
                whilst leaving the lemniscate fixed. First suppose we've
                retracted all far away points down to an oval, and points near
                the two holes we've pushed out so that the holes have been
                enlarged from points to discs. To finish the retraction we use a
                bit of physics and differential equations
                (Fig.~\ref{fig:Deformation_Retraction_lemniscate_of_Gerono}).
                \begin{figure}
                    \centering
                    \captionsetup{type=figure}
                    \includegraphics{images/Homotopy_lemniscate_of_Gerono.pdf}
                    \caption{Deformation Retraction of the Lemniscate of Gerono}
                    \label{fig:Deformation_Retraction_lemniscate_of_Gerono}
                \end{figure}
                We put two identical electric charges at the centers of the blue
                circles. The \textit{electric field} exerted at a point
                $\vector{x}=(x_{0},y_{0})$ is given by Coulomb's law:
                \begin{equation}
                    F(\vector{x})=
                    \frac{\vector{x}-\vector{r}_{1}}
                         {\norm{\vector{x}-\vector{r}_{1}}_{2}^{3}}+
                    \frac{\vector{x}-\vector{r}_{2}}
                         {\norm{\vector{x}-\vector{r}_{2}}_{2}^{3}}
                \end{equation}
                where $\vector{r}_{1}$ and $\vector{r}_{2}$ are the coordinates
                of the holes, we've chosen $(\minus{1},0)$ and $(1,0)$. The
                notation $\norm{\vector{x}}_{2}^{3}$ denotes the 2-norm raised
                to the third power. If this were an actual physics problem we
                would need some scale factor $Q/4\pi\epsilon_{0}$, but for our
                purposes we simply need the directions of the
                \textit{field lines}. Given a point inside of the lemniscate, we
                drag this continuously outward along field lines until we hit
                the figure eight, and for points on the outside we contract
                inwards. All the while we leave the lemniscate fixed. The
                outcome is a deformation retraction onto our figure eight. For
                the sake of computation one could use Euler's method of solving
                differential equations to numerically approximate this
                procedure.
                \begin{figure}
                    \centering
                    \captionsetup{type=figure}
                    \includegraphics{images/Homotopy_Cassini_Ovals_001.pdf}
                    \caption{Deformation Retraction Using Cassini Ovals}
                    \label{fig:Deformation_Retraction_Cassini_Ovals}
                \end{figure}
                \par\hfill\par
                For a more analytical approach with
                closed form solutions we can use Cassini ovals. If we let $a$
                denote the distance from the first hole to the second, we can
                study the family of curves satisfying the following equation:
                \begin{equation}
                    \label{eqn:Cassini_Ovals}%
                    \big((x-a)^{2}+y^{2}\big)\big((x+a)^{2}+y^{2}\big)=b^{4}
                \end{equation}
                this asks for the set of all points $P$ such that the distance
                from $P$ to the first hole multiplied by the distance from $P$
                to the second hole is equal to $b^{2}$:
                \begin{equation}
                    X_{b}=\{\,\vector{x}\in\nspace[2]\;|\;
                        \norm{\vector{x}-\vector{r}_{1}}_{2}\cdot
                        \norm{\vector{x}-\vector{r}_{2}}_{2}=b^{2}\,\}
                \end{equation}
                This was studied by the Italian astronomer Giovanni Domenico
                Cassini in 1680 C.E. and gives a continuous means of retracting
                the plane with two holes onto a figure eight. The resulting
                figure eight is no longer the lemniscate of Gerono, but rather
                the lemniscate of Bernoulli, studied by Jakob Bernoulli in
                1694 C.E. shortly after Cassini's investigations.
                Taking the gradient of Eqn.~\ref{eqn:Cassini_Ovals} gives us a
                vector field that allows one to flow along field lines until we
                arrive at the lemniscate
                (Fig.~\ref{fig:Deformation_Retraction_Cassini_Ovals}). The
                gradient is computed to be:
                \begin{equation}
                    \grad{f}=\big(4x(x^{2}+y^{2}-a^{2}),4y(x^{2}+y^{2}+a^2)\big)
                \end{equation}
                The gradient shows what the paths of individual points will look
                like, and allows us to draw
                Fig.~\ref{fig:Deformation_Retraction_Cassini_Ovals}, but is
                unnecessary. Cassini's equation Eqn.~\ref{eqn:Cassini_Ovals} is
                all we need. As $b$ tends to zero we get two disconnected ovals
                closing in on our points. When $b$ is equal to the square root
                of the distance between these points we obtain the lemniscate,
                and as $b$ grows we get a single connected object that looks
                like an ellipse as $b$ gets large. This is precisely the
                description of a deformation retraction of the plane with two
                points missing onto the figure eight and the intermediate steps
                are shown in Fig.~\ref{fig:Homotopy_Cassini_Ovals}.
                \begin{figure}
                    \centering
                    \captionsetup{type=figure}
                    \includegraphics{images/Homotopy_Cassini_Ovals_002.pdf}
                    \caption{Homotopy Using Cassini Ovals}
                    \label{fig:Homotopy_Cassini_Ovals}
                \end{figure}
                \par\hfill\par
                Now we can use our imagination to consider other possible
                deformation retractions from the plane with two holes onto
                smaller subspaces. For one, we can take our central lemniscate
                of Bernoulli that was produced in
                Fig.~\ref{fig:Deformation_Retraction_Cassini_Ovals} using
                Cassini ovals and stretch the crossing point outwards until we
                achieve two circles attached by a straight line
                (see Fig.~\ref{fig:Homotopy_Two_Circles_and_String}). While the
                retracts using electric fields and Cassini ovals result in
                homeomorphic objects (the lemniscates of Gerono and  Bernoulli
                are homeomorphic subspaces of $\nspace[2]$), this third object
                is \textit{not} homeomorphic to either of these. Suppose $f$ is
                a homeomorphism from the lemniscate of Bernoulli to two circles
                with a straight line. If we remove the center of the lemniscate
                we are left with two subspaces that are homeomorphic to the open
                unit interval $(0,1)$. However, no matter where the crossing
                point maps to under $f$ removing a single point from the latter
                object does not result in two homeomorphic copies of the unit
                interval, even though homeomorphisms preserve this notion of
                subspace. So while these are both the result of deformation
                retractions of the same space, they are not homeomorphic.
                \begin{figure}
                    \centering
                    \captionsetup{type=figure}
                    \includegraphics{images/Homotopy_Two_Circles_and_String.pdf}
                    \caption{Deformation Retraction onto Two Connnected Circles}
                    \label{fig:Homotopy_Two_Circles_and_String}
                \end{figure}
                \par\hfill\par
                Further still we can imagine stretching the
                crossing point of our lemniscate vertically rather than
                horizontally, obtaining Fig.~\ref{fig:Homotopy_Oval_with_Line}.
                It is perhaps easiest to see that this retraction is different
                then Fig.~\ref{fig:Deformation_Retraction_Cassini_Ovals} and
                Fig.~\ref{fig:Homotopy_Two_Circles_and_String}. Removing
                any point from this final figure eight does not disconnect the
                space, however if we remove the central points from either the
                lemniscate of Bernoulli or the circles connected by a line we
                would disconnect these spaces into two separate parts. Since
                homeomorphisms preserve subspaces, this final object is not
                homeomorphic to either of the other two. Hence we see that none
                of these three things are homeomorphic to each other, though
                they are \textit{homotopy equivalent}, a notion we will soon
                desribe.
                \begin{figure}[H]
                    \centering
                    \captionsetup{type=figure}
                    \includegraphics{images/Homotopy_Oval_with_Lines.pdf}
                    \caption{Homotopy onto another Figure Eight}
                    \label{fig:Homotopy_Oval_with_Line}
                \end{figure}
                The previous examples involving the doubly punctured plane give
                rise to another notion, the \textit{mapping cylinder} of a
                function $f:X\rightarrow{Y}$. We take our domain $X$ and cross
                it with the closed unit interval $I=[0,1]$. If $X$ happens to be
                the circle $\nsphere[1]$ then $X\times{I}$ is precisely a
                cylinder. We take $X\times{I}$ with the product topology and
                consider the \textit{disjoint union} of this space with $Y$,
                denoted $(X\times{I})\coprod{Y}$. Set theoretically this is
                written as follows:
                \begin{equation}
                    \label{eqn:Def_Disjoint_Union}%
                    A\coprod{B}=(A\times\{0\})\cup(B\times\{1\})
                \end{equation}
                That is, we take a copy of $A$ and a copy of $B$ and form a
                new set that contains $A$ and $B$ as separate entities. For
                example, the disjoint union of $\nsphere[1]$ with $\nsphere[2]$
                can be thought of as a circle with a sphere placed far away.
                This is contrasted with the usual union
                $\nsphere[1]\cup\nsphere[2]$ which is just the sphere
                $\nsphere[2]$ since the unit circle is a subset of the unit
                sphere (it's the equator). The disjoint union topology mimics
                this newer notion. An open set is $A\coprod{B}$ is just an open
                set in $A$ plus an open set in $B$. The mapping cylinder is
                obtained by equipping the space $(X\times{I})\coprod{Y}$ with
                the equivalence relation $R$ identifying $(x,1)\in{X}\times{I}$
                and $f(x)\in{Y}$. We \textit{glue} the bottom of our cylinder
                onto the image of $X$ in $Y$ by $f$.
                \begin{fdefinition}{Mapping Cylinder}{Mapping_Cylinder}
                    The mapping cylinder of a continuous function $f$ from a
                    topological space $\topspace[X]{X}$ into a topological space
                    $\topspace[Y]{Y}$ is the quotient space induced on
                    $(X\times{I})\coprod{Y}$ by the equivalence relation $R$
                    obtained by identifying $\big((x,1),\,0\big)$ with
                    $\big(f(x),\,1\big)$ for all $x\in{X}$.
                \end{fdefinition}
                A short explaination on the notation in
                Def.~\ref{def:Mapping_Cylinder}. By
                Eqn.~\ref{eqn:Def_Disjoint_Union} an element of $A\coprod{B}$ is
                of the form $(z,i)$ where $i=0$ or $i=1$, and either $z\in{A}$
                or $z\in{B}$, depending on the value of $i$. Since we're taking
                the disjoint union of a product $(X\times{I})\coprod{Y}$ the
                elements are of the form $(z,i)$ where either
                $z=(x,t)\in{X}\times{I}$ or $z=y\in{Y}$. Technicalities aside,
                we present an example. If $X=\nsphere[1]$ then $X\times{I}$ is
                an actual cylinder. Let's map the circle into the plane and see
                what the resulting mapping cylinder looks like. Let $f$ be
                defined by:
                \begin{equation}
                    \label{eqn:Trefoil_Mapping_Cylinder}%
                    f(\theta)=\Big(
                        \frac{1+\cos^{2}\big(\frac{3\theta}{2}\big)}{2},\,
                        \theta
                    \Big)
                \end{equation}
                using polar coordinates. The result is a trefoil. First, we draw
                the space $(X\times{I})\coprod{Y}$. Since this is
                the disjoint union we think of $\nsphere[1]\times{I}$ and
                $\nspace[2]$ as separate entities
                (see Fig.~\ref{fig:Ex_Mapping_Cylinder_Disj_Union}).
                \begin{figure}
                    \centering
                    \captionsetup{type=figure}
                    \includegraphics{images/Mapping_Cylinder_Disj_Union.pdf}
                    \caption{The Disjoint Union
                             $(\nsphere[1]\times{I})\coprod\nspace[2]$}
                    \label{fig:Ex_Mapping_Cylinder_Disj_Union}
                \end{figure}
                We've drawn the image of $\nsphere[1]$ under the mapping $f$
                defined in Eqn.~\ref{eqn:Trefoil_Mapping_Cylinder} in blue to
                help transition to the next part of the process. In creating the
                mapping cylinder we now glue the \textit{bottom} of the cylinder
                $\nsphere[1]\times{I}$ (the bottom being a copy of
                $\nsphere[1]$, i.e. the \textit{boundary}) onto this blue piece
                using the function $f$ to guide us in the gluing. We can be very
                explicit if we imagine this in $\nspace[3]$. We draw the top
                circle in the $z=1$ plane and then draw a straight line between
                $\big(\cos(\theta),\sin(\theta),1\big)$ to the point $f(\theta)$
                which lies in the $xy$ plane. After drawing a straight line for
                every $\theta\in[0,2\pi)$ we arrive at our mapping cylinder.
                This is depicted in Fig.~\ref{fig:Ex_Mapping_Cylinder}.
                \begin{figure}[H]
                    \centering
                    \captionsetup{type=figure}
                    \includegraphics{images/Mapping_Cylinder.pdf}
                    \caption{Example of a Mapping Cylinder}
                    \label{fig:Ex_Mapping_Cylinder}
                \end{figure}
                The red circle at the top represents $\nsphere[1]$, the blue
                closed curve in the plane is $f[\nsphere[1]]$ and the gray
                surface is used to form our mapping cylinder. The black mesh
                lines are drawn to emphasize the construction. It's important to
                note per definition, the mapping cylinder includes the entire
                plane $\nspace[2]$ as well. This straight-line construction
                generalizes. Given a mapping cylinder induced by
                $f:X\rightarrow{Y}$ we can shrink the cylindrical portion
                downwards giving us a deformation retraction onto $Y$.
                \par\hfill\par
                Not all deformation retractions come from mapping cylinders.
                Suppose we take a thick $\textbf{X}$ and shrink it down to a
                skinny X. We then shrink this X down to the central crossing
                point. Combining these two deformation retractions gives us a
                single deformation retraction $\textbf{X}$ to $\cdot$ but this
                cannot arise from a mapping cylinder since paths must merge
                together halfway through the deformation.
            \subsection{Homotopy}
                Deformation retractions and mapping cylinders motivate a far
                more general concept, though the real motivation stems from the
                desire to define a weaker notion than homeomorphism that is
                easily computable and preserves many topological properties.
                This is called \textit{homotopy}.
                \par\hfill\par
                \begin{minipage}{0.50\textwidth}
                    Homeomorphism is an ideal definition of sameness, but many
                    spaces that are not homeomorphic share many properties. We
                    expand our study of topology by defining
                    \textit{homotopy equivalence}. We will show homeomorphisms
                    are stronger (i.e. homeomorphic spaces are homotopy
                    equivalent) and demonstrate the converse fails.
                \end{minipage}
                \hfill
                \fbox{%
                    \begin{minipage}{0.44\textwidth}
                        \centering
                        \begin{figure}[H]
                            \centering
                            \captionsetup{type=figure}
                            \includegraphics{images/Homotopy_Basic_Example.pdf}
                            \caption{Homotopy Between Functions}
                            \label{fig:Homotopy_Diagram_for_Depicting_Homotopy}
                        \end{figure}
                    \end{minipage}
                }
                \par\vspace{2.5ex}
                \begin{fdefinition}{Homotopy}{Homotopy}
                    A homotopy from a continuous function $f:X\rightarrow{Y}$ to
                    a continuous function $g:X\rightarrow{Y}$ with respect to
                    topological spaces $\topspace[X]{X}$ and $\topspace[Y]{Y}$
                    is a continuous function $H:{X}\times{I}\rightarrow{Y}$ such
                    that $H(x,0)=f(x)$ and $H(x,1)=g(x)$, where $I=[0,1]$ is the
                    closed unit interval and $X\times{I}$ carries the product
                    topology $\tau_{X}\times\tau_{\nspace[]}|_{I}$.
                \end{fdefinition}
                To clarify, $\tau_{\nspace[]}$ is the standard Euclidean
                topology on $\nspace[]$ and $\tau_{\nspace[]}|_{I}$ is the
                subspace topology induced on the closed unit interval $I$. Given
                $f,g:X\rightarrow{Y}$ we drag $f$ along a continuous path in
                $X\times{I}$ until we obtain the function $g$
                (see Fig.~\ref{fig:Homotopy_Diagram_for_Depicting_Homotopy}).
                Homotopy is very weak since, as we will see, \textit{every} pair
                of continuous functions $f,g:\nspace[m]\rightarrow\nspace$ has a
                homotopy between them. Indeed, we can be more general and
                prove this for \textit{topological vector spaces} over the real
                numbers. This is just a vector space with a topology that makes
                vector addition and scalar multiplication \textit{continuous}
                operations. Any topological vector space over $\mathbb{R}$ is
                path connected since $\gamma:[0,1]\rightarrow{X}$ defined by
                $\gamma(t)=(1-t)\cdot{x}+t\cdot{y}$ is a continuous path from
                $x$ to $y$. We use this map to define our homotopy.
                \begin{theorem}
                    If $\topspace[X]{X}$ is a topological space, if
                    $\topvecspace[Y]{Y}$ is a topological vector space over the
                    real numbers, and if $f,g:X\rightarrow{Y}$ are continuous
                    functions, then there is a homotopy
                    $H:X\times{I}\rightarrow{Y}$ from $f$ to $g$.
                \end{theorem}
                \begin{proof}
                    For let $H$ be defined by:
                    \begin{equation}
                        \label{eqn:Straight_Line_Homotopy}%
                        H(x,t)=(1-t)\cdot{f}(x)\vecadd[Y]t\cdot{g}(x)
                    \end{equation}
                    Since $Y$ is a topological vector space, $H$ is well defined
                    for all $x\in{X}$ and $t\in[0,1]$. Moreover, $H(x,0)=f(x)$,
                    $H(x,1)=g(x)$ and $H$ is continuous. Hence, $H$ is a
                    homotopy taking $f$ to $g$.
                \end{proof}
                \begin{example}
                    \label{ex:Straight_Line_Homotopy_Euc_Spaces}%
                    The simplest homotopy involves convex subsets of Euclidean
                    spaces where we may again define the \textit{straight line}
                    homotopy. Given two continuous functions
                    $f,g:X\rightarrow\mathcal{U}$ from a topological space $X$
                    to a convex subset of $\nspace$ we write
                    $H(x,t)=f(x)(1-t)+g(x)t$. This is well defined since by
                    hypothesis the space is convex and $H(x,t)$ represents a
                    straight line between $f(x)$ and $g(x)$ for each point
                    $x\in{X}$. Thus, $H$ is a homotopy from $f$ to $g$. It is
                    too demanding to ask one to only consider convex spaces, but
                    any topological space that is homeomorphic to a convex
                    subset of $\nspace$ can inherit this homotopy by using
                    function composition with the hypothesized homeomorphism. A
                    more general notion is that the space be
                    \textit{contractible} meaning it can be deformed
                    continuously to a single point.
                \end{example}
                With homotopy now defined, we should redundantly define
                \textit{homotopic} functions (functions with a homotopy between
                them). To avoid abuse of language we must first show that if
                there is a homotopy from $f$ to $g$, then there is a homotopy
                from $g$ to $f$. That is, we need not specify which function is
                the \textit{first} one.
                \begin{ltheorem}{Symmetry of Homotopy}{Symmetry_of_Homotopy}
                    If $\topspace[X]{X}$ and $\topspace[Y]{Y}$ are topological
                    spaces, if $f,g:X\rightarrow{Y}$ are continuous, and if $H$
                    is a homotopy from $f$ to $g$, then there is a homotopy
                    from $g$ to $f$.
                \end{ltheorem}
                \begin{proof}
                    For let $G:X\times{I}\rightarrow{Y}$ be defined by:
                    \begin{equation}
                        G(x,t)=H(x,1-t)
                    \end{equation}
                    for all $x\in{X}$ and $t\in[0,1]$. Then $G$ is the
                    composition of continuous functions and is therefore
                    continuous. By definition $G(x,0)=H(x,1)=g(x)$ and
                    $G(x,1)=H(x,0)=f(x)$. Hence $G$ is a homotopy taking $g$ to
                    $f$.
                \end{proof}
                We may now define \textit{homotopic}.
                \begin{fdefinition}{Homotopic Functions}{Homotopic_Functions}
                    Homotopic functions from a topological space
                    $\topspace[X]{X}$ to a topological space $\topspace[Y]{Y}$
                    are continuous functions $f,g:{X}\rightarrow{Y}$ such that
                    there is a homotopy $H$ between them.
                \end{fdefinition}
                Homotopic is an equivalence relation. We have shown that it is
                symmetric, next we show reflexivity.
                \begin{ltheorem}{Reflexivity of Homotopy}
                                {Reflexivity_of_Homotopy}
                    If $\topspace[X]{X}$ and $\topspace[Y]{Y}$ are topological
                    spaces, and if $f:X\rightarrow{Y}$ is a continuous function,
                    then $f$ is homotopic to $f$.
                \end{ltheorem}
                \begin{proof}
                    For let $H:X\times{I}\rightarrow{Y}$ be defined by
                    $H(x,t)=f(x)$. Then $H$ is continuous, $H(x,0)=f(x)$ and
                    $H(x,1)=f(x)$. Thus $f$ is homotopic to $f$.
                \end{proof}
                We complete our claim that homotopic is an equivalence relation
                by proving it is transitive. We'll need the
                \textit{pasting lemma} from point-set topology.
                \begin{ltheorem}{Transitivity of Homotopy}
                                {Transitivity_of_Homotopy}
                    If $\topspace[X]{X}$ and $\topspace[Y]{Y}$ are topological
                    spaces, if $f,g,h:X\rightarrow{Y}$ are continuous, if $f$ is
                    homotopic to $g$, and if $g$ is homotopic to $h$, then $f$
                    is homotopic to $h$.
                \end{ltheorem}
                \begin{proof}
                    If $f$ is homotopic to $g$ and $g$ is homotopic to $h$, then
                    there exist continuous functions
                    $H_{1},H_{2}:X\times{I}\rightarrow{Y}$ such that $H_{1}$ is
                    a homotopy from $f$ to $g$ and $H_{2}$ is a homotopy from
                    $g$ to $h$. Let $H:X\times{I}\rightarrow{Y}$ be defined by:
                    \begin{equation}
                        H(x,t)=
                        \begin{cases}
                            H_{1}(x,2t),&{0}\leq{t}\leq\frac{1}{2}\\
                            H_{2}(x,2t-1),&\frac{1}{2}<{t}\leq{1}
                        \end{cases}
                    \end{equation}
                    By the pasting lemma, $H$ is continuous. But from the
                    definition of $H_{1}$ and $H_{2}$, $H(x,0)=f(x)$ and
                    $H(x,1)=h(x)$. Thus, $f$ is homotopic to $h$.
                \end{proof}
                \begin{figure}
                    \centering
                    \captionsetup{type=figure}
                    \includegraphics{images/Homotopy_Straight_Line_Cubed_to_Exp_2D}
                    \caption{Paths Representing Homotopy}
                    \label{fig:Paths_Representing_Homotopy}
                \end{figure}
                \begin{lexample}{Straight Line Homotopy}{Straight_Line_Homotopy}
                    We saw in Ex.~\ref{ex:Straight_Line_Homotopy_Euc_Spaces}
                    that for any pair of continuous functions
                    $f,g:\nspace[m]\rightarrow\nspace$ there is a homotopy
                    between them, and hence all continuous functions between
                    Euclidean spaces are homotopic. For a more concrete example,
                    let's define two functions
                    $f,g:\nspace[]\rightarrow\nspace[2]$ by:
                    \twocolumneq{f(x)=\big(x,\,x^{3}\big)}
                                {g(x)=\big(x,\,\exp(x)\big)}
                    Both of these are continuous since they are continuous in
                    each coordinate. We can use the straight line homotopy
                    between them:
                    \begin{subequations}
                        \begin{align}
                            H(x,t)
                            &=(1-t)\big(x,\,x^{3}\big)+t\big(x,\,\exp(x)\big)\\
                            &=\big(x,\,(1-t)x^{3}+t\exp(x)\big)
                        \end{align}
                    \end{subequations}
                    The latter equation gives a parameterization of curves in
                    the plane that we can draw. The paths are shown in
                    Fig.~\ref{fig:Paths_Representing_Homotopy}. Note
                    $g(x)=constant$ is possible. Any continuous function
                    $f:\nspace[m]\rightarrow\nspace$ is homotopic to a point.
                \end{lexample}
                \begin{minipage}{0.50\textwidth}
                    Lastly, can visualize homotopy by letting $X=I$ and
                    $Y\subset\nspace[2]$ be a nice blob, like the one in
                    Fig.~\ref{fig:straight_line_homotopy}. Continuous functions
                    $f,g:[0,1]\rightarrow{Y}$ are just curves within the
                    blob. The homotopy defined in
                    Eqn.~\ref{eqn:Straight_Line_Homotopy} is the map taking
                    $f(x)$ to $g(x)$ via the straight line connecting the two
                    points. We may use homotopy to precisely define deformation
                    retractions. First, we define \textit{retracts}.
                \end{minipage}
                \hfill
                \fbox{%
                    \begin{minipage}{0.44\textwidth}
                        \centering
                        \begin{figure}[H]
                            \centering
                            \captionsetup{type=figure}
                            \includegraphics{%
                                images/Homotopy_on_Unit_Interval.pdf%
                            }
                            \caption{Straight-Line Homotopy}
                            \label{fig:straight_line_homotopy}
                        \end{figure}
                    \end{minipage}
                }
                \par\hfill\par
                \begin{fdefinition}{Retract}{Retract}
                    A retract of a topological space $\topspace{X}$ to a
                    subspace $A\subseteq{X}$ is a continuous function
                    $f:X\rightarrow{A}$ such that $f|_{A}=\identity{A}$.
                \end{fdefinition}
                If $f:X\rightarrow{A}$ is a retract, then composing twice simply
                yields $f$ again. That is, $f^{2}=f$. In a way retractions mimic
                projections that occur in linear algebra.
                \begin{example}
                    Let's look at the example of the annulus deformation
                    retracting down to a circle. The key mapping was the unit
                    vector function:
                    \begin{equation}
                        h(\vector{x})=\frac{\vector{x}}{\norm{\vector{x}}_{2}}
                    \end{equation}
                    This function is a \textit{retract} of the punctured plane
                    $\nspace[2]\setminus\{(0,0)\}$ to the circle $\nsphere[1]$.
                    The deformation retract was then defined by:
                    \begin{equation}
                        H(\vector{x},t)
                            =(1-t)\cdot\vector{x}+t\cdot{h}(\vector{x})
                    \end{equation}
                    But this is just the straight-line homotopy between the
                    retract $h$ and the identity function
                    $\identity{X}$, where $X=\nspace[2]\setminus\{(0,0)\}$ is
                    the punctured plane:
                    \begin{equation}
                        H(\vector{x},t)=(1-t)\cdot\identity{X}(\vector{x})+
                            t\cdot{h}(\vector{x})
                    \end{equation}
                \end{example}
                The above example spells out how we can define a deformation
                retraction.
                \begin{fdefinition}{Deformation Retraction}
                                   {Deformation_Retraction}
                    A deformation retraction of a topological space
                    $\topspace{X}$ onto a subspace $A\subseteq{X}$ is a
                    homotopy $H:X\times{I}\rightarrow{X}$ between the identity
                    function $\identity{X}$ and a retract $f:X\rightarrow{A}$.
                \end{fdefinition}
                We've omitted the requirement that $H(a,t)=a$ for all $a\in{A}$
                and $t\in{I}$. This more general concept is a \textit{weak}
                deformation contraction where the set $A$ is
                allowed to move throughout the homotopy. The notion previously
                discussed is then called a \textit{strong} deformation
                retraction. There are bizarre spaces with weak deformation
                retractions that cannot be obtained by strong ones. One such
                example is the comb space. We take the closed unit
                interval $[0,1]$ sitting in $\nspace[2]$ and attached
                $\{\frac{1}{n}\}\times[0,1]$ for all $n\in\mathbb{N}^{+}$.
                Lastly we attach $\{0\}\times[0,1]$.
                \par\hfill\par
                \begin{minipage}{0.50\textwidth}
                    We give this the usual subspace topology of $\nspace[2]$
                    which makes it very nice from a point-set topology view
                    (it's compact, path connected, etc.). It is shown in
                    Fig.~\ref{fig:Comb_Space}. Thinking about homotopy theory
                    there are a plethora of points which this space deformation
                    retracts to, for example the point in the bottom right
                    corner. We can push down all of the teeth of the comb and
                    then compress the resulting line to the right. Similarly,
                    just about every other point can be deformation retracted
                    to. One point that \textit{cannot} is the top left corner.
                \end{minipage}
                \hfill
                \fbox{%
                    \begin{minipage}{0.44\textwidth}
                        \centering
                        \begin{figure}[H]
                            \centering
                            \captionsetup{type=figure}
                            \includegraphics{images/Comb_Space.pdf}
                            \caption{The Comb Space}
                            \label{fig:Comb_Space}
                        \end{figure}
                    \end{minipage}
                }
                \par\hfill\par\vspace{1ex}
                Any deformation retract must be a continuous homotopy. If we
                look at the sequence $a_{n}=(\frac{1}{n},1)$ this converges to
                $(0,1)$.  Hence $H(a_{n},t)$ must converge to $(0,1)$ for all
                $t$ since $H$ does not move the point $(0,1)$. But this sequence
                must be pressed downwards to the $x$ axis in order to retract to
                this point, a contradiction. We can \textit{weakly} deformation
                retract to this point. First we push all of the teeth down to
                the $x$ axis and then squish the resulting line segment to the
                origin. We conclude by dragging this point back to $(0,1)$. This
                is a weak deformation retraction since we allowed the point of
                retraction to move.
                \par\hfill\par
                It is worthwhile pointing out some subtle differences between
                \textit{retracts} and \textit{deformation retractions}. For one,
                any space retracts to a single point.
                \begin{theorem}
                    If $\topspace{X}$ is a topological space, and if
                    $x_{0}\in{X}$, then there is a retract
                    $f:X\rightarrow\{x_{0}\}$.
                \end{theorem}
                \begin{proof}
                    For let $f:X\rightarrow\{x_{0}\}$ be defined in the only way
                    possible: $f(x)=x_{0}$. Then since $f$ is a constant
                    function it is continuous. But
                    $f|_{\{x_{0}\}}=\identity{\{x_{0}\}}$ and hence $f$ is a
                    retract of $X$ to $\{x_{0}\}$.
                \end{proof}
                It is \textit{not} true that every space can be deformation
                retracted to a point. As we will see, it is necessary that
                such spaces be connected, and moreover path connected. Hence any
                disconnected space can be retracted down to a point but cannot
                be deformation retracted. Path connectedness is not a sufficient
                condition since the punctured plane is path connected but has
                no deformation retraction to a point. The intuitive reason
                being there is a \textit{hole} in the space.
                \par\hfill\par
                Strong deformation retracts can be easily defined and motivated
                if we introduce \textit{relative} homotopy.
                \begin{fdefinition}{Relative Homotopy}{Relative_Homotopy}
                    A homotopy relative to a subspace $A\subseteq{X}$ between
                    continuous functions $f,g:X\rightarrow{Y}$ with respect to
                    topological spaces $\topspace[X]{X}$ and $\topspace[Y]{Y}$
                    is a homotopy $H:X\times{I}\rightarrow{Y}$ between $f$ and
                    $g$ such that for all $t\in[0,1]$ and for all $a\in{A}$ it
                    is true that $H(a,t)=f(a)$ and $H(a,t)=g(a)$.
                \end{fdefinition}
                For a relative homotopy to exist between $f$ and $g$ it is
                necessary that $f|_{A}=g|_{A}$, otherwise this definition does
                not make sense. There is now a very concise way to define
                \textit{strong} deformation retraction, it is simply a relative
                homotopy $H:X\times{I}\rightarrow{X}$ with respect to a subspace
                $A\subseteq{X}$ between the identity function $\identity{X}$ and
                a retract $f:X\rightarrow{A}$.
                \par\hfill\par
                Various examples have shown us that homotopic functions can look
                very different. Indeed, \textit{every} pair of continuous
                functions $f,g:\nspace[m]\rightarrow\nspace$ are homotopic. The
                definition of homeomorphism states that $f:X\rightarrow{Y}$ is
                a homeomorphism if and only if $f$ is a continuous bijection
                with a continuous inverse. We also saw that it was not enough to
                say that there exists a continuous bijection $f:X\rightarrow{Y}$
                and another continuous bijection $g:Y\rightarrow{X}$. However,
                it is equivalent to say that there are continuous bijections
                $f:X\rightarrow{Y}$ and $g:Y\rightarrow{X}$ such that
                $f\circ{g}=\identity{Y}$ and $g\circ{f}=\identity{X}$. This last
                condition merely stating that $g=f^{\minus{1}}$. A new form of
                sameness for topological spaces is given if we relax the
                \textit{equalities} $f\circ{g}=\identity{Y}$ and
                $g\circ{f}=\identity{X}$ and instead only require
                \textit{homotopic}. This condition has a name:
                \begin{fdefinition}{Homotopy Inverse}{Homotopy_Inverse}
                    A homotopy inverse of a continuous function
                    $f:X\rightarrow{Y}$ from a topological space
                    $\topspace[X]{X}$ to $\topspace[Y]{Y}$ is a continuous
                    function $g:Y\rightarrow{X}$ such that $g\circ{f}$ is
                    homotopic to $\identity{X}$ and $f\circ{g}$ is homotopic to
                    $\identity{Y}$.
                \end{fdefinition}
                There's often a notion of uniqueness when it comes to inverses,
                and indeed a homotopy inverse of a function $f:X\rightarrow{Y}$
                is unique \textit{up to homotopy}.
                \begin{theorem}
                    If $\topspace[X]{X}$ and $\topspace[Y]{Y}$ are topological
                    spaces, if $f:X\rightarrow{Y}$ is continuous, and if
                    $g_{1},g_{2}:Y\rightarrow{X}$ are homotopy inverses
                    of $f$, then $g_{1}$ is homotopic to ${g}_{2}$.
                \end{theorem}
                \begin{proof}
                    Since $g_{2}$ is a homotopy inverse of $f$,
                    $f\circ{g}_{2}$ is homotopic to $\identity{Y}$. But homotopy
                    is reflexive and hence $g_{1}$ is homotopic to itself, and
                    $g_{1}=g_{1}\circ\identity{Y}$. But homotopy is transitive
                    so if $g_{1}$ is homotopic $g_{1}\circ\identity{Y}$ and
                    $g_{1}\circ\identity{Y}$ is homotopic to
                    $g_{1}\circ(f\circ{g}_{2})$, then $g_{1}$ is homotopic to
                    $g_{1}\circ(f\circ{g}_{2})$. But function composition is
                    associative, thus
                    $g_{1}\circ(f\circ{g}_{2})=(g_{1}\circ{f})\circ{g}_{2}$.
                    But $g_{1}$ is a homotopy inverse of $f$, and therefore
                    $g_{1}\circ{f}$ is homotopic to $\identity{X}$. Hence
                    $(g_{1}\circ{f})\circ{g}_{2}$ is homotopic to ${g}_{2}$ and
                    since homotopic is a transitive relation, $g_{1}$ is
                    homotopic to ${g}_{2}$.
                \end{proof}
                Often in mathematics we wish to be as lazy as possible. If
                $f:X\rightarrow{Y}$ and $g:Y\rightarrow{X}$ are continuous
                bijections, and if $g\circ{f}=\identity{X}$, then there is no
                need to check if $f\circ{g}=\identity{Y}$. This comes for free
                from set theory. This is not so for homotopy inverses as the
                following example demonstrates.
                \begin{example}
                    Let $X=\{0\}$ be the one point space and $Y=\mathbb{Q}$.
                    Define $f:X\rightarrow{Y}$ by $f(0)=0$ and
                    $g:Y\rightarrow{X}$ by $g(x)=0$. Then both $f$ and $g$ are
                    continuous function, and $g\circ{f}=\identity{X}$, which is
                    certainly homotopic to $\identity{X}$. However, $f\circ{g}$
                    is \textit{not} homotopic to $\identity{\mathbb{Q}}$ for if
                    it were then $\mathbb{Q}$ would be \textit{contractible}. As
                    stated one of the requirements for a space to be contracible
                    is that it is path connected, whereas $\mathbb{Q}$ is
                    totally disconnected. Hence $f$ and $g$ are both continuous,
                    $g\circ{f}$ is homotopic to $\identity{X}$, but
                    $f\circ{g}$ is not homotopic to $\identity{Y}$. Another
                    point of disagreement between homeomorphism and homotopy.
                \end{example}
                Homotopy inverses make it easy to define homotopy equivalence.
                \begin{fdefinition}{Homotopy Equivalence}{Homotopy_Equivalence}
                    A homotopy equivalence from a topological space
                    $\topspace[X]{X}$ to a topological space $\topspace[Y]{Y}$
                    is a continuous function $f:X\rightarrow{Y}$ such that there
                    exists a homotopy inverse $g:Y\rightarrow{X}$ of $f$.
                \end{fdefinition}
                From the definition of homotopy equivalent spaces it is
                important to note that it is not required that
                $g\circ{f}=id_{X}$, but rather that $g\circ{f}$ is
                \textit{homotopic} to the identity map $\identity{X}$.
                Similarly, $f\circ{g}$ need only be homotopic to $\identity{Y}$.
                Let's look at an example of what type of operations are allowed
                by homotopy equivalences. We'll start with a torus $\ntorus[]$
                with a disk inside (Fig.~\ref{fig:Torus_with_Disc_Inside}).
                \begin{figure}[H]
                    \centering
                    \captionsetup{type=figure}
                    \includegraphics{images/Torus_Wireframe_Gradient.pdf}
                    \caption{Torus with a Disk Inside}
                    \label{fig:Torus_with_Disc_Inside}
                \end{figure}
                Next we'll shrink the disk down to a point, leaving the rest of
                the torus alone (Fig.~\ref{fig:Squished_Torus}). Such moves are
                not permitted by homeomorphisms since we are squeezing
                infinitely many elements together and identifying them with a
                single point, whereas homeomorphisms must be bijective per
                definition.
                \textit{Homotopy} does allow us to perform such an operation.
                \begin{figure}[H]
                    \centering
                    \captionsetup{type=figure}
                    \includegraphics{images/Squished_Torus.pdf}
                    \caption{A Squished Torus}
                    \label{fig:Squished_Torus}
                \end{figure}
                This can be made more concrete with equations since
                parametrizations of the torus are well known. If we let $R$ and
                $r$ denote the outer and inner radii, respectively, we have:
                \begin{subequations}
                    \begin{align}
                        x(\theta,\varphi)
                            &=\big(R+r\cos(\varphi)\big)\cos(\theta)\\
                        y(\theta,\varphi)
                            &=\big(R+r\cos(\varphi)\big)\sin(\theta)\\
                        z(\theta,\varphi)&=r\sin(\varphi)
                    \end{align}
                \end{subequations}
                The squished torus can be obtain by shrinking the $x$ and $z$
                coordinates as $y$ varies. The code used to generate
                Fig.~\ref{fig:Squished_Torus} adopted the following:
                \begin{subequations}
                    \begin{align}
                        x(\theta,\varphi)
                            &=\big(R+r\cos(\varphi)\sin(\tfrac{\theta}{2})\big)
                                \cos(\theta)\\
                        y(\theta,\varphi)
                            &=\big(R+r\cos(\varphi)\big)\sin(\theta)\\
                        z(\theta,\varphi)
                            &=r\sin(\varphi)\sin(\tfrac{\theta}{2})
                    \end{align}
                \end{subequations}
                This shrinks the torus to a point when $\theta=0$ and has no
                effect at $\theta=\pi$, varying continuously in between which is
                precisely what we want. It is important to realize that we
                needed the disk inside the torus in order to perform this move.
                We are using the disk to do our squishing.
                \par\hfill\par
                \begin{minipage}{0.50\textwidth}
                    Continuing, one can imagine taking our croissant and
                    stretching out the collapsed point into a line obtaining a
                    crescent-moon with a line connecting the poles. We take the
                    body of this and morph it into a sphere, obtaining
                    Fig.~\ref{fig:Sphere_with_String_Attached}. From here we can
                    proceed and drag the endpoints of the string down towards
                    the equator and obtain Fig.~\ref{fig:Kettle_Bell}. This
                    example demonstrates how massively different spaces with
                    homotopy equivalences can be. Even the last two objects,
                    which perhaps look similar enough, are not homeomorphic.
                \end{minipage}
                \hfill
                \fbox{%
                    \begin{minipage}{0.44\textwidth}
                        \centering
                        \begin{figure}[H]
                            \centering
                            \captionsetup{type=figure}
                            \includegraphics{%
                                images/Sphere_with_String_at_Poles.pdf%
                            }
                            \caption{A Sphere with String Attached}
                            \label{fig:Sphere_with_String_Attached}
                        \end{figure}
                    \end{minipage}
                }
                \par\vspace{2.5ex}
                \fbox{%
                    \begin{minipage}{0.40\textwidth}
                        \centering
                        \begin{figure}[H]
                            \centering
                            \captionsetup{type=figure}
                            \includegraphics{images/Kettle_Bell.pdf}
                            \caption{A Kettle Bell}
                            \label{fig:Kettle_Bell}
                        \end{figure}
                    \end{minipage}
                }
                \hfill
                \begin{minipage}{0.54\textwidth}
                    Let's demonstrate this. Homeomorphisms preserve subspaces.
                    Using this, if we remove the attaching point of our kettle
                    bell we have a disconnected space. No matter which point we
                    remove from our sphere with string attached at the poles the
                    space remains connected, so these are not homeomorphic.
                    Just like homeomorphism gives rise to the notion of
                    homeomorphic, homotopy equivalence redundantly allows us to
                    define homotopy equivalent.
                \end{minipage}
                \par\hfill\par
                \begin{fdefinition}{Homotopy Equivalent}{Homotopy_Equivalent}
                    Homotopy equivalent topological spaces are topological
                    spaces $\topspace[X]{X}$ and $\topspace[Y]{Y}$ such that
                    there exists a homotopy equivalence $f:X\rightarrow{Y}$.
                \end{fdefinition}
                As stated before, this is a weaker notion of sameness. It is
                important in algebraic topology since many properties are
                invariant under homotopy equivalence. It should be noted that it
                is a strictly weaker notion than homeomorphism. Homeomorphic
                implies homotopy equivalent, but not the other way around.
                Indeed, one of the great conjectures of topology,
                Poincar\'{e}'s Conjecture (now a theorem), asks when does
                homotopy equivalent imply homeomorphic?
                \begin{ltheorem}{Homeomorphic Implies Homotopy Equivalent}
                                {Homeomorphic_Implies_Homotopy_Equivalent}
                    If $\topspace[X]{X}$ and $\topspace[Y]{Y}$ are homeomorphic
                    topological spaces, then they are homotopy equivalent.
                \end{ltheorem}
                \begin{proof}
                    For if $X$ and $Y$ are homeomorphic, then there exists a
                    homeomorphism $f:X\rightarrow{Y}$. But then $f$ is a
                    continuous map from $X$ to $Y$, and $f^{-1}$ is a continuous
                    map from $Y$ to $X$. Moreover,
                    ${f}\circ{f^{-1}}=\identity{Y}$, and
                    ${f^{-1}}\circ{f}=\identity{X}$ since $f$ is a bijection.
                    But $\identity{X}$ is homotopic to $\identity{X}$, and
                    $\identity{Y}$ is homotopic to $\identity{Y}$. Thus,
                    $f^{\minus{1}}$ is a homotopy inverse of $f$ so $f$ is a
                    homotopy equivalence.
                \end{proof}
                It does not take much to show the converse fails.
                \begin{theorem}
                    \label{thm:homotopic_does_not_imply_homeomorphic}%
                    There exist topological spaces that are homotopy equivalent
                    but not homeomorphic.
                \end{theorem}
                \begin{proof}
                    Let $X=\nspace[2]$ and $Y=\{(0,0)\}$ be equipped with their
                    usual topologies. Let $f:X\rightarrow{Y}$ be defined by
                    $f(x,y)=(0,0)$ and $g=\identity{Y}$. Then
                    $g\circ{f}=\identity{Y}$. Let
                    $H\big((x,y),t\big)=(1-t)(x,y)$. Then $H$ is continuous,
                    $H\big((x,y),0\big)=(x,y)$, and $H\big((x,y),1\big)=(0,0)$.
                    Thus, $H$ is a homotopy between ${g}\circ{f}$ and
                    $\identity{X}$, and therefore ${g}\circ{f}$ is homotopic to
                    $\identity{X}$. But ${f}\circ{g}=\identity{Y}$ and
                    $\identity{Y}$ is homotopic to $\identity{Y}$. Thus $X$ and
                    $Y$ are homotopy equivalent. If $h:{X}\rightarrow{Y}$ is a
                    homeomorphism, then it is a bijection. But then
                    $\Card(X)=\Card(Y)$ where $\Card$ denotes the cardinality
                    of the space. But $\mathbb{R}^{2}$ is uncountable, and
                    $\Card(Y)=1$, a contradiction. Therefore $X$ and $Y$ are not
                    homeomorphic.
                \end{proof}
                Fig.~\ref{fig:homotopy_equivalence_of_plane_with_point}
                shows the mapping $f$ between $\mathbb{R}^{2}$ and $\{(0,0)\}$.
                This is a deformation retraction of $\nspace[2]$ to a single
                point. Such spaces are given a name.
                \begin{fdefinition}{Contractible}{Contractible}
                    A contractible topological space is a space $\topspace{X}$
                    that is homotopy equivalent to a one point space
                    $(\{0\},\{\emptyset,\{0\}\})$.
                \end{fdefinition}
                \par
                \begin{minipage}{0.54\textwidth}
                    A one point space $X=\{0\}$ has only one topology on it,
                    $\tau=\{\emptyset,\{0\}\}$. This is the extreme case where
                    the discrete and indiscrete topologies are identical. The
                    fact that $\nspace[2]$ is contractible generalizes to all
                    dimensions, which we will show later.
                    Thm.~\ref{thm:homotopic_does_not_imply_homeomorphic} relies
                    on the fact that $\mathbb{R}^{2}$ and $\{(0,0)\}$ are of
                    different \textit{cardinality}. Even if $X$ and $Y$ are
                    homotopy equivalent and of the same cardinality, it is still
                    possible that they are not homeomorphic. Recall
                    homeomorphisms preserve compactness. Homotopy equivalences
                    need not.
                \end{minipage}
                \hfill
                \fbox{%
                    \begin{minipage}{0.40\textwidth}
                        \centering
                        \begin{figure}[H]
                            \captionsetup{type=figure}
                            \centering
                            \includegraphics{%
                                images/Retraction_Plane_to_Point.pdf%
                            }
                            \caption{Contraction of $\nspace[2]$}
                            \label{fig:homotopy_equivalence_of_plane_with_point}
                        \end{figure}
                    \end{minipage}
                }
                \par\hfill\par
                \begin{theorem}
                    \label{thm:HE_of_Punc_Plane_and_Circle_Not_Homeo}%
                    There exists topological spaces that have the same
                    cardinality, are homotopy equivalent, but not homeomorphic.
                \end{theorem}
                \begin{proof}
                    For let $X=\nspace[2]\setminus\{(0,0)\}$ and
                    $Y=\nsphere[1]$ have their usual topologies. Then
                    $\Card(X)=\Card(Y)=\Card(\mathbb{R})$, and thus both sets
                    are of the same cardinality. They are also homotopy
                    equivalent. Define $f:X\rightarrow{Y}$ and
                    $g:Y\rightarrow{X}$ by:
                    \twocolumneq{f(x,y)=\frac{(x,y)}{\norm{(x,y)}}}
                                {g(x,y)=(x,y)}
                    $f$ is well defined since $\norm{(x,y)}\ne{0}$ for all
                    $(x,y)\in{X}$. Moreover, both $f$ and $g$ are continuous.
                    Define $H:X\times{I}\rightarrow{X}$ by:
                    \begin{equation}
                        H\big((x,y),t\big)
                        =(1-t)\cdot{f}(x,y)+t\cdot\identity{X}(x,y)
                    \end{equation}
                    $H$ is well defined since if $(x,y)\in{X}$, $t\in[0,1]$,
                    then by the triangle inequality:
                    \begin{subequations}
                        \begin{align}
                            \norm{H\big((x,y),t\big)}
                            &\geq\min\{\,\norm{\identity{X}(x,y)},\,
                                \norm{f(x,y)}\,\}\\
                            &=\min\{\,\norm{(x,y)},\,1\,\}
                            >0
                        \end{align}
                    \end{subequations}
                    That is, the image of $H$ lies in
                    $\nspace[2]\setminus\{(0,0)\}$. But then
                    $H\big((x,y),0\big)=f(x,y)$, and
                    $H\big((x,y),1\big)=\identity{X}(x,y)$. Moreover, $H$ is
                    continuous and is therefore a homotopy between ${g}\circ{f}$
                    and $\identity{X}$. But also $({f}\circ{g})(x,y)=(x,y)$ for
                    all $(x,y)\in\nsphere[1]$. Thus ${f}\circ{g}=\identity{Y}$,
                    and $\identity{Y}$ is homotopic to itself. Hence, $X$ and
                    $Y$ are homotopy equivalent. But $X$ is unbounded, and is
                    therefore not compact, and $Y$ is closed and bounded, and is
                    thus compact by the Heine-Borel theorem. But homeomorphisms
                    preserve compactness. Hence $X$ and $Y$ are not
                    homeomorphic.
                \end{proof}
                \begin{figure}
                    \centering
                    \captionsetup{type=figure}
                    \includegraphics{images/Retraction_Punc_Plane_to_Circle.pdf}
                    \caption{%
                        Homotopy Equivalence of the Punctured Plane and
                        $\nsphere[1]$%
                    }
                    \label{fig:HE_punc_plane_and_circle}
                \end{figure}
                Ahah, you might say, you're exploiting compactness. What if we
                add this condition? The answer is still no, as the following
                demonstrates.
                \begin{theorem}
                    There exists compact connected topological spaces of the
                    same cardinality that are homotopy equivalent but not
                    homeomorphic.
                \end{theorem}
                \begin{proof}
                    Let $X=[\minus{1},1]$, $Y=[\minus{1},1]\times[\minus{1},1]$
                    have their usual subspace topologies from $\mathbb{R}$ and
                    $\nspace[2]$, respectively, and define $f:X\rightarrow{Y}$
                    and $g:Y\rightarrow{X}$ by:
                    \twocolumneq{f(x)=(x,0)}{g(x,y)=x}
                    That is, $f$ is the inclusion mapping and $g$ is the
                    projection in the $y$ axis. Then $g\circ{f}=\identity{X}$.
                    But $H\big((x,y),t\big)=(1-t)(x,0)+t(x,y)$ is a homotopy
                    between $f\circ{g}$ and $\identity{Y}$, and thus
                    $f\circ{g}$ is homotopic to $\identity{Y}$. Therefore $X$
                    and $Y$ are homotopy equivalent. Moreover, they are both
                    compact, (path) connected, and have the cardinality
                    of the continuum. Suppose $h:X\rightarrow{Y}$ is a
                    homeomorphism and let $h(0)=\vector{y}\in{Y}$. If $h$
                    is a homeomorphism between $X$ and $Y$, then the restriction
                    of $h$ to $X\setminus\{0\}$ is a homeomorphism between
                    $[-1,0)\cup(0,1]$ and $[-1,1]^{2}\setminus\{\vector{y}\}$.
                    But $[-1,1]^{2}\setminus\{\vector{y}\}$ is connected and
                    $[-1,0)\cup(0,1]$ is not. But homeomorphisms preserve
                    connectedness. Therefore, $X$ and $Y$ are not homeomorphic.
                \end{proof}
                Now one might point out that these two spaces have different
                dimensions. If we require our spaces to be manifolds of the
                same dimension, is it still possible to be homotopy equivalent
                but not homeomorphic? The answer is yes, and to prove this we
                must first know that a sphere and a torus are \textit{not}
                homeomorphic. This is obvious at first glance, a torus has a
                hole whereas a sphere does not. To prove this rigorously we
                must recall the Jordan curve theorem. Any continuous injective
                closed curve $\gamma:\nsphere[1]\rightarrow\nsphere[2]$ cuts the
                sphere into two parts. Hence if we look at the space
                $\nsphere[2]\setminus\gamma[\nsphere[1]]$ we are left with a
                disconnected space. However, we can cut out a circle from the
                torus and still leave the space connected. If we cut out either
                of the defining circles of the torus
                $\ntorus[2]=\nsphere[1]\times\nsphere[1]$ we are left with
                something homeomorphic to a cylinder, which is connected.
                \par\hfill\par
                This is not entirely useful since the torus is \textit{not}
                homotopy equivalent to the sphere. To make them equivalent we
                puncture the torus in one spot and put three holes into the
                sphere. The resulting spaces are still not homeomorphic, but
                are they homotopy equivalent? The answer is yes.
                \par\hfill\par
                \begin{minipage}[t]{0.52\textwidth}
                    First we prove the sphere with a hole is homeomorphic to the
                    plane $\nspace[2]$, and then we conclude by showing
                    $\nball[2]$ and $\nspace[2]$ are also homeomorphic. The
                    first problem can be solved using the
                    \textit{stereographic projection} depicted in
                    Fig.~\ref{fig:stereographic_projection}. The construction
                    goes as follows: Given a point
                    $\vector{s}=(x,y,z)\in\nsphere[2]\setminus\{(0,0,1)\}$ we
                    draw the straight line between the north pole and
                    $\vector{s}$ and mark where this intersects the $xy$ plane.
                    This unambiguously defines a mapping
                    $\nsphere[2]\setminus\{(0,0,1)\}\rightarrow\nspace[2]$. To
                    conclude we must show it is bijective, continuous, and has
                    a continuous inverse.
                \end{minipage}
                \hfill
                \fbox{%
                    \begin{minipage}[t]{0.42\textwidth}
                        \begin{figure}[H]
                            \captionsetup{type=figure}
                            \centering
                            \includegraphics{%
                                images/Sphere_Stereographic_Projection.pdf%
                            }
                            \caption{Stereographic Projection of $\nsphere[2]$}
                            \label{fig:stereographic_projection}
                        \end{figure}
                    \end{minipage}
                }
                \par\hfill\par
                \begin{theorem}
                    \label{thm:sphere_without_point_homeomorphic_to_plane}%
                    $\nsphere[2]\setminus\{(0,0,1)\}$ is homeomorphic to
                    $\nspace[2]$
                \end{theorem}
                \begin{proof}
                    For let $f:\nsphere[2]\setminus\{(0,0,1)\}\rightarrow\nspace[2]$
                    be the stereographic projection mapping:
                    \begin{equation}
                        f(x,y,z)=\Big(\frac{x}{1-z},\,\frac{y}{1-z}\Big)
                    \end{equation}
                    Then $f$ is surjective. For if $(X,Y)\in\mathbb{R}^{2}$,
                    let:
                    \par\hfill\par
                    \begin{subequations}
                        \begin{minipage}[b]{0.49\textwidth}
                            \begin{equation}
                                x=\frac{2X}{\norm{(X,Y)}^{2}+1}
                            \end{equation}
                        \end{minipage}
                        \hfill
                        \begin{minipage}[b]{0.49\textwidth}
                            \begin{equation}
                                y=\frac{2Y}{\norm{(X,Y)}^{2}+1}
                            \end{equation}
                        \end{minipage}
                    \end{subequations}
                    \par\vspace{2.5ex}
                    and further define:
                    \begin{equation}
                        z=\frac{\norm{(X,Y)}^{2}-1}{\norm{(X,Y)}^{2}+1}
                    \end{equation}
                    Then $z\ne{1}$. We must now show that this maps the plane
                    into the punctured sphere,
                    $(x,y,z)\in\nsphere[2]\setminus\{(0,0,1)\}$. That is, we
                    must show $\norm{(x,y,z)}=1$. We compute:
                    \begin{subequations}
                        \begin{align}
                            \norm{(x,y,z)}
                                &=\sqrt{%
                                    \frac{%
                                        4X^{2}+4Y^{2}+(\norm{(X,Y)}^{2}-1)^{2}%
                                    }{%
                                        (\norm{(X,Y)}+1)^{2}%
                                    }
                                }\\[2ex]
                                &=\sqrt{%
                                \frac{%
                                    4\norm{(X,Y)}^{2}+
                                    \norm{(X,Y)}^{4}-
                                    2\norm{(X,Y)}^{2}+1%
                                }{%
                                    (\norm{(X,Y)}+1)^{2}%
                                }%
                            }\\[2ex]
                            &=\sqrt{%
                                \frac{%
                                    \norm{(X,Y)}^{4}+2\norm{(X,Y)}^{2}+1%
                                }{%
                                    (\norm{(X,Y)}^{2}+1)^{2}%
                                }%
                            }\\[2ex]
                            &=\sqrt{%
                                \frac{(\norm{(X,Y)}^{2}+1)^{2}}
                                     {(\norm{(X,Y)}^{2}+1)^{2}}%
                            }
                        \end{align}
                    \end{subequations}
                    and hence the norm of $(x,y,z)$ is 1. But also
                    $f(x,y,z)=(X,Y)$ since:
                    \begin{subequations}
                        \begin{align}
                            \Big(\frac{x}{1-z},\frac{y}{1-z}\Big)
                                &=\frac{1}{1-z}\Big(%
                                    \frac{2X}{\norm{(X,Y)}^{2}+1},\,
                                    \frac{2Y}{\norm{(X,Y)}^{2}+1}%
                                \Big)\\[2ex]
                            &=\frac{\norm{(X,Y)}^{2}+1}{2}
                                \Big(%
                                    \frac{2X}{\norm{(X,Y)}^{2}+1},\,
                                    \frac{2Y}{\norm{(X,Y)}^{2}+1}%
                                \Big)\\[2ex]
                            &=(X,Y)\vphantom{\Big(\frac{X}{Y})}
                        \end{align}
                    \end{subequations}
                    and therefore $f$ is surjective. But it is also injective
                    for suppose there are points
                    $(x_{1},y_{1},z_{1}),(x_{2},x_{2},z_{2})\in%
                    \nsphere[2]\setminus\{(0,0,1)\}$ with
                    $f(x_{1},y_{1},z_{1})=f(x_{2},y_{2},z_{2})$. Since
                    $(x_{1},y_{1},z_{1})\in\nsphere[2]\setminus\{(0,0,1)\}$ we
                    have $x_{1}^{2}+y_{1}^{2}=1-z_{1}^{2}$. But then:
                    \begin{equation}
                        \norm{(X,Y)}^{2}=\frac{x_{1}^2+y_{1}^2}{(1-z_{1})^{2}}
                            =\frac{1-z_{1}^{2}}{(1-z_{1})^{2}}
                            =\frac{x_{2}^{2}+y_{2}^{2}}{(1-z_{2})^{2}}
                            =\frac{1-z_{2}^{2}}{(1-z_{2})^{2}}
                    \end{equation}
                    So we have:
                    \begin{equation}
                        \frac{1-z_{1}^{2}}{(1-z_{1})^{2}}
                        =\frac{1-z_{2}^{2}}{(1-z_{2})^{2}}
                    \end{equation}
                    and this further implies:
                    \begin{equation}
                        \frac{1+z_{1}}{1-z_{1}}=\frac{1+z_{2}}{1-z_{2}}
                    \end{equation}
                    But the function $g(x)=\frac{1+x}{1-x}$ is injective and
                    therefore $z_{1}=z_{2}$. From this $x_{1}=x_{2}$ and
                    $y_{1}=y_{2}$. Thus, $f$ is a bijection. Moreoever, $f$ is
                    continuous and:
                    \begin{equation}
                        f^{\minus{1}}(X,Y)=\Big(%
                            \frac{2X}{\norm{(X,Y)}^{2}+1},\,
                            \frac{2Y}{\norm{(X,Y)}^{2}+1},\,
                            \frac{\norm{(X,Y)}^{2}-1}{\norm{(X,Y)}^{2}+1}%
                        \Big)
                    \end{equation}
                    which is continuous. Therefore, $f$ is a homeomorphism.
                \end{proof}
                \begin{theorem}
                    $\nsphere[2]$ is homeomorphic to $\nball[2]$.
                \end{theorem}
                \begin{proof}
                    Define $f:\nball[2]\rightarrow\nspace[2]$ by:
                    \begin{equation}
                        f(\vector{x})=\frac{\vector{x}}{1-\norm{\vector{x}}}
                    \end{equation}
                    Then $f$ is surjective. For if $\vector{y}\in\nspace[2]$,
                    let $\vector{x}=\frac{\vector{y}}{1+\norm{\vector{y}}}$.
                    Then:
                    \begin{equation}
                        \norm{\vector{x}}
                        =\frac{\norm{\vector{y}}}{1+\norm{\vector{y}}}<1
                    \end{equation}
                    and thus $\vector{x}\in\nball[2]$. Evaluating with $f$ we
                    obtain:
                    \begin{subequations}
                        \begin{align}
                            f(\vector{x})
                            &=\frac{\vector{y}}{1+\norm{\vector{y}}}\Big(%
                                1-\frac{\norm{\vector{y}}}{1+\norm{\vector{y}}}
                            \Big)^{\minus{1}}\\[1ex]
                            &=\frac{\vector{y}}{1+\norm{\vector{y}}}\Big(%
                                \frac{%
                                    1+\norm{\vector{y}}-
                                    \norm{\vector{y}}%
                                }{%
                                    1+\norm{\vector{y}}%
                                }
                            \Big)^{\minus{1}}\\[1ex]
                            &=\frac{\vector{y}}{1+\norm{\vector{y}}}\Big(%
                                \frac{1}{1+\norm{\vector{y}}}
                            \Big)^{\minus{1}}\\[1ex]
                            &=\vector{y}\vphantom{\Big(\frac{X}{Y}\Big)}
                        \end{align}
                    \end{subequations}
                    Moreover, $f$ is injective. For if
                    $f(\vector{x}_{1})=f(\vector{x}_{2})$, then:
                    \begin{equation}
                        \frac{\norm{\vector{x}_{1}}}{1+\norm{\vector{x}_{1}}}
                        =\norm{f(\vector{x}_{1})}
                        =\norm{f(\vector{x}_{2})}
                        =\frac{\norm{\vector{x}_{2}}}{1+\norm{\vector{x}}_{2}}
                    \end{equation}
                    and therefore
                    $\norm{\vector{x}}_{1}=\norm{\vector{x}_{2}}$.
                    But from the definition of $f$:
                    \begin{equation}
                        \frac{\vector{x}_{1}}{1+\norm{\vector{x}_{1}}}
                        =\frac{\vector{x}_{2}}{1+\norm{\vector{x}_{2}}}
                    \end{equation}
                    and therefore $\vector{x}_{1}=\vector{x}_{2}$. Lastly,
                    $f$ is continuous and:
                    \begin{equation}
                        f^{\minus{1}}(\vector{y})
                        =\frac{\vector{y}}{1+\norm{\vector{y}}}                    
                    \end{equation}
                    which is continuous. Therefore, $f$ is a homeomorphism.
                \end{proof}
                \begin{theorem}
                    $\nsphere[2]\setminus\{(0,0,1)\}$ is homeomorphic to
                    $\nball[2]$.
                \end{theorem}
                \begin{proof}
                    For $\nsphere[2]\setminus\{(0,0,1)\}$ is homeomorphic to
                    $\nspace[2]$, and $\nspace[2]$ is homeomorphic to
                    $\nball[2]$, and homeomorphic is transitive.
                \end{proof}
                It's easier to visual this homeomorphism in one step. This is
                shown in Fig.~\ref{fig:homeo_Punc_S2_and_Plane}.
                \begin{figure}[H]
                    \centering
                    \captionsetup{type=figure}
                    \includegraphics{images/Sphere_to_Disk_Homeo.pdf}
                    \caption{Homeomorphism of a Punctured Sphere to a Disk}
                    \label{fig:homeo_Punc_S2_and_Plane}
                \end{figure}
                We continue our endeavor of finding topological manifolds of the
                same dimension that are homotopy equivalent but not homeomorphic
                by defining Jordan curves. These are continuous injective
                functions
                $f:\nsphere[1]\rightarrow\nspace[2]$. With this we now state,
                but do not prove, the Jordan Curve Theorem.
                \begin{ftheorem}{Jordan Curve Theorem}{Jordan_Curve_Theorem}
                    If $\Gamma$ is a Jordan Curve, then there exists two unique
                    disjoint open connected sets $\interior[]{\Gamma}$ and
                    $\exterior[]{\Gamma}$, called the interior and exterior of
                    $\Gamma$, such that:
                    \begin{equation*}
                        \nspace[2]\setminus\Gamma
                            =\interior[]{\Gamma}\cup\exterior[]{\Gamma}
                    \end{equation*}
                    And such that $\partial\interior[]{\Gamma}=\Gamma$ and
                    $\partial\exterior[]{\Gamma}=\Gamma$. $\interior[](\Gamma)$
                    is bounded and $\exterior[]{\Gamma}$ is unbounded.
                \end{ftheorem}
                The Jordan Curve Theorem implies that the injective continuous
                image of $\nsphere[1]$ into the $\nsphere[2]$ will cut the
                sphere into two parts.
                \begin{theorem}
                    \label{thm:Sphere_Without_Circle_Is_Disconnected}
                    If $f:\nsphere[1]\rightarrow\nsphere[2]$ is a continuous
                    injective function, then there exists two disjoint open
                    connected sets $\mathcal{U}_{1}$ and $\mathcal{U}_{2}$ with
                    $\nsphere[2]\setminus{f}[\nsphere[1]]%
                    =\mathcal{U}_{1}\cup\mathcal{U}_{2}$
                \end{theorem}
                \begin{proof}
                    First note that $f$ is not surjective. Since $\nsphere[1]$
                    is compact, any bijection
                    $f:\nsphere[1]\rightarrow\nsphere[2]$ would be a
                    homeomorphism but $\nsphere[1]$ and $\nsphere[2]$ are not
                    homeomorphic. Removing two points from $\nsphere[1]$
                    disconnects it, but removing two points from $\nsphere[2]$
                    leaves the sphere in one piece. Hence there is a point
                    $\vector{s}\in\nsphere[2]$ that $f$ does not map too.
                    Moreover, since the image of $\nsphere[1]$ must be closed
                    since $\nsphere[1]$ is compact, there is an open ball around
                    $\vector{s}$ such that $f$ does not map into this set. Hence
                    we may remove this point without affecting the image of $f$.
                    But $\nsphere[2]\setminus\{\vector{s}\}$ is homeomorphic to
                    $\nspace[2]$ so there is a homeomorphism
                    $g:\nsphere[2]\setminus\{\vector{s}\}\rightarrow\nspace[2]$.
                    But then $\Gamma=g\circ{f}$ is a Jordan Curve. By the Jordan
                    Curve Theorem there exist disjoint open connected sets
                    $\interior[]{\Gamma}$ and $\exterior[]{\Gamma}$ with
                    $\interior[]{\Gamma}\cup\exterior[]{\Gamma}%
                    =\nspace[2]\setminus\Gamma$. Define
                    $\mathcal{U}_{1}=g^{\minus{1}}\big[\interior[]{\Gamma}\big]$
                    and $\mathcal{U}_{2}=g^{\minus{1}}%
                    \big[\exterior[]{\Gamma}\big]\cup\{\vector{s}\}$.
                    Then $\mathcal{U}_{1}$ and $\mathcal{U}_{2}$ are open,
                    disjoint, connected, and cover
                    $\nsphere[2]\setminus{f}[\nsphere[1]]$.
                \end{proof}
                We can now prove that the sphere and the torus are not
                homeomorphic.
                \begin{ltheorem}{$\nsphere[2]$ is not Homeomorphic to $\ntorus[]$}
                                {Sphere_Not_Homeo_to_Torus}
                    There is no homeomorphism between the sphere $\nsphere[2]$
                    and the torus $\ntorus[]$.
                \end{ltheorem}
                \begin{proof}
                    The Torus is the Cartesian product of two circles,
                    $\ntorus[]=\nsphere[1]\times\nsphere[1]$, and by removing
                    the inner circle we are left with a connected set. Suppose
                    the two are homeomorphic and let
                    $f:\ntorus[]\rightarrow\nsphere[2]$ be a homeomorphism. Then
                    the restriction of $f$ to $\ntorus[]\setminus\nsphere[1]$ is
                    a homeomorphism with $\nsphere[2]\setminus{f}[\nsphere[1]]$.
                    But by Thm.~\ref{thm:Sphere_Without_Circle_Is_Disconnected}
                    $\nsphere[2]\setminus{f}[\nsphere[1]]$ is disconnected, a
                    contradiction.
                \end{proof}
                It should be intuitively clear that both the torus and the
                sphere are two dimensional topological manifolds. If we remove a
                finite number of points from either of these we are still left
                with two dimensional manifolds.
                \begin{theorem}
                    There exist topological manifolds $\topspace[X]{X}$ and
                    $\topspace[Y]{Y}$ of the same dimension that are homotopy
                    equivalent but not homeomorphic.
                \end{theorem}
                We'll prove this with pictures. We know the sphere with three
                holes is homeomorphic to a plane with 2 holes. But we've seen
                that this is homotopy equivalent to a figure eight since there
                is a deformation retraction of the doubly punctured plane onto
                a lemniscate. Similarly, if we use our square representation of
                a torus we can show that a torus with one hole is homotopy
                equivalent to a figure eight, see
                Fig.~\ref{fig:HE_Torus_Sphere_Fig_8}.
                \begin{figure}
                        \centering
                        \captionsetup{type=figure}
                        \resizebox{\textwidth}{!}{%
                            \begin{tikzpicture}[%
    every edge/.style={draw=black},
    scale=1.7,
    >=latex
]
    \draw[ball color=gray!40, opacity=0.4] (0,0) circle (1cm);
    \draw (-1,0) arc (180:360:1 and 0.3);
    \draw[dashed] (1,0) arc (0:180:1 and 0.3);
    \draw[fill=white] (0.55,0.55) circle (0.75pt);
    \draw[fill=white] (0.65,0.65) circle (0.75pt);
    \draw[fill=white] (0.5,0.72)  circle (0.75pt);
    \draw[->] (1.3,0) to (2.3,0);
    \draw[fill=gray, shading angle=215]
        (2.5,-0.5)--(3.2,0.5)--(5.5,0.5)--(4.8,-0.5)--cycle;
    \draw[densely dashed] (3.7,0) circle (0.295);
    \draw[densely dashed] (4.3,0) circle (0.295);
    \draw[fill=white] (3.7,0) circle (0.75pt);
    \draw[fill=white] (4.3,0) circle (0.75pt);
    \draw[->] (5.7,0) to [in=90,out=0] (7,-1);
    \draw[->] (1.2,-2.3)--(2,-2.3);
    \draw[%
        postaction={decorate},
        decoration={%
            markings,
            mark=at position .145 with \arrow{latex},
            mark=at position .375 with \arrow{latex},
            mark=at position .395 with \arrow{latex},
            mark=at position .615 with \arrowreversed{latex},
            mark=at position .855 with \arrowreversed{latex},
            mark=at position .875 with \arrowreversed{latex}
        },
        fill=gray,
        shading angle=215
    ] (-0.75,-3)--(0.75,-3)--(0.75,-1.5)--(-0.75,-1.5)--cycle;
    \draw[fill=white] (0,-2.3) circle (0.75pt);
    \draw[%
        postaction={decorate},
        decoration={%
            markings,
            mark=at position .5 with \arrow{latex},
            mark=at position .55 with \arrow{latex}
        }
    ]   (3.1,-2.3) arc[%
                start angle=0,
                delta angle=-360,
                x radius=.35,
                y radius=.75
        ];
    \draw (2.75,-1.55) -- (4.75,-1.55);
    \draw[%
        postaction={decorate},
        decoration={%
        markings,
        mark=at position .5 with \arrow{latex},
        mark=at position .55 with \arrow{latex}
        }
    ]   (5.1cm,-2.3cm) arc[%
            start angle=0,
            delta angle=-360,
            x radius=.35,
            y radius=.75
        ];
    \draw[->] (5.4,-2.3)--(6.3,-2.3);
    \draw (7,-1.8) circle (0.5);
    \draw (7,-2.8) circle (0.5);
\end{tikzpicture}%
                        }
                        \caption{Equivalence of a Three Holed Sphere
                                 and a Punctured Torus.}
                        \label{fig:HE_Torus_Sphere_Fig_8}
                \end{figure}
                Now, you say, hold on! You lost compactness! This is precisely
                getting down to Poincar\'{e}'s conjectures. When one classifies
                all two dimensional manifolds, they come across the fact that
                if $\topspace{X}$ is a compact two dimensional topological
                manifold that is homotopy equivalent to $\nsphere[2]$, then it
                is homeomorphic to it. Quite an impressive theorem, but
                Poincar\'{e} asks for the next dimension up:
                \begin{center}
                    \textit{If} $\topspace{X}$ \textit{is a three dimensional}
                    \textit{compact manifold that is homotopy equivalent to}
                    $\nsphere[3]$, \textit{is it homeomorphic?}
                \end{center}
                The answer is yes, but the proof is \textit{hard}.
                \par\hfill\par
                Now, while we've called homotopy equivalence an
                \textit{equivalence}, it would be nice to know that it actually
                is. We prove this now.
                \begin{ltheorem}{Reflexivity of Homotopy Equivalence}
                                {Reflexivity_of_Homotopy_Equivalence}
                    If $\topspace{X}$ is a topological space, then it is
                    homotopy equivalent to itself.
                \end{ltheorem}
                \begin{proof}
                    For any topological space is homeomorphic to itself, and
                    homeomorphic topological spaces are homotopy equivalent.
                \end{proof}
                \begin{ltheorem}{Symmetry of Homotopy Equivalence}
                                {Symmetry_of_Homotopy_Equivalence}
                    If $\topspace[X]{X}$ and $\topspace[Y]{Y}$ are topological
                    spaces, and if $X$ is homotopy equivalent to $Y$, then
                    $Y$ is homotopy equivalent to $X$.
                \end{ltheorem}
                \begin{proof}
                    For if $X$ is homotopy equivalent to $Y$, then there is a
                    homotopy equivalence $f:X\rightarrow{Y}$. But if $f$ is a
                    homotopy equivalence, then there is a homotopy inverse
                    $g:Y\rightarrow{X}$. But if $g$ is a homotopy inverse of
                    $f$, then $f\circ{g}$ is homotopic to $\identity{Y}$ and
                    $g\circ{f}$ is homotopic to $\identity{X}$. But then $f$ is
                    a homotopy inverse of $g$, and hence $g$ is a homotopy
                    equivalence. Therefore, $Y$ is homotopic to $X$.
                \end{proof}
                \begin{ltheorem}{Transitivity of Homotopy Equivalence}
                                {Transitivity_of_Homotopy_Equivalence}
                    If $\topspace[X]{X}$, $\topspace[Y]{Y}$, and
                    $\topspace[Z]{Z}$ are topological spaces, if $X$ is homotopy
                    equivalent to $Y$, and if $Y$ is homotopy equivalent to $Z$,
                    then $X$ is homotopy equivalent to $Z$.
                \end{ltheorem}
                \begin{proof}
                    For if $X$ is homotopy equivalent to $Y$, then there is a
                    homotopy equivalence $f_{1}:X\rightarrow{Y}$. But if $f_{1}$
                    is a homotopy equivalence, then there is a homotopy inverse
                    $g_{1}:Y\rightarrow{X}$. But if $Y$ is homotopy equivalent
                    to $Z$, then there is a homotopy equivalence
                    $f_{2}:Y\rightarrow{Z}$. And if $f_{2}$ is a homotopy
                    equivalence, then there is a homotopy inverse
                    $g_{2}:Y\rightarrow{Z}$. But then
                    $f_{2}\circ{f}_{1}:X\rightarrow{Z}$ is a homotopy
                    equivalence since $g_{1}\circ{g}_{2}:Z\rightarrow{X}$ is a
                    homotopy inverse. That is,
                    $f_{2}\circ{f}_{1}\circ{g}_{1}\circ{g}_{2}$ is homotopic to
                    $f_{2}\circ{g}_{2}$ since $f_{1}\circ{g}_{1}$ is homotopic
                    to $\identity{Y}$. But $f_{2}\circ{g}_{2}$ is homotopic to
                    $\identity{Z}$, and homotopic is transitive
                    (Thm.~\ref{thm:Transitivity_of_Homotopy}). Similarly,
                    $g_{1}\circ{g}_{2}\circ{f}_{2}\circ{f}_{1}$ is homotopic to
                    $\identity{X}$. Hence, $X$ is homotopic to $Z$.
                \end{proof}
                Before moving on to cell complexes, we should finally discuss
                some of the subtleties of contractible spaces. First, there's an
                equivalent definition based purely on the existence of certain
                types of homotopies.
                \begin{fdefinition}{Nullhomotopic}{Nullhomotopic}
                    A nullhomotopic function from a topological
                    $\topspace[X]{X}$ to $\topspace[Y]{Y}$ is a continuous
                    function $f:X\rightarrow{Y}$ such that there exists a
                    homotopy $H:X\times{I}\rightarrow{Y}$ between $f$ and a
                    constant mapping $g:X\rightarrow{Y}$.
                \end{fdefinition}
                \begin{theorem}
                    If $\topspace{X}$ is a topological space, then it is
                    contractible if and only if $\identity{X}$ is nullhomotopic.
                \end{theorem}
                \begin{proof}
                    If $\topspace{X}$ is contractible, then there is a homotopy
                    equivalence $f:X\rightarrow{Y}$ where $Y=\{0\}$ is the
                    one point space. But if $f$ is a homotopy equivalence, then
                    there is a homotopy inverse $g:Y\rightarrow{X}$. But then
                    $g\circ{f}$ is homotopic to $\identity{X}$. But $Y$ has only
                    one point, and hence $f(x)=0$ for all $x\in{X}$. Let
                    $x_{0}=g(0)$. Then $g\circ{f}:X\rightarrow{X}$ is the
                    mapping $x\mapsto{x}_{0}$. But $\identity{X}$ is homotopic
                    to this, and is therefore nullhomotopic. In the other
                    direction, if $\identity{X}$ is nullhomotopic, then there is
                    a point $x_{0}\in{X}$ such that $\identity{X}$ is homotopic
                    to the function $f:X\rightarrow{X}$ defined by $f(x)=x_{0}$.
                    Let $Y=\{x_{0}\}$ be given the subspace topology. Since
                    there is only one topology on a space with one point, this
                    is homeomorphic to the one point space. But then $f$ is a
                    homotopy equivalence since the function $g:Y\rightarrow{X}$
                    given by $g(x_{0})=x_{0}$ is a homotopy inverse of $f$. That
                    is, since $\identity{X}$ is homotopic to $f$,
                    $g\circ{f}$ is homotopic to $\identity{X}$ since
                    $g\circ{f}=f$. But also $f\circ{g}=\identity{Y}$. Hence $f$
                    is an homotopy equivalence and $\topspace{X}$ is
                    contractible.
                \end{proof}
                \begin{theorem}
                    If $n\in\mathbb{N}$, and if $\tau_{\nspace}$ is the standard
                    topology on $\nspace$, then $\topspace[\nspace]{\nspace}$ is
                    contractible.
                \end{theorem}
                \begin{proof}
                    For let $f:\nspace\rightarrow\nspace$ be the zero mapping,
                    $f(\vector{x})=\vector{0}$. But any continuous function
                    $g:\nspace\rightarrow\nspace$ is homotopic to $f$, and hence
                    $\identity{\nspace}$ is homotopic to $f$. Hence, the
                    identity map is nullhomotopic and therefore $\nspace$ is
                    contractible.
                \end{proof}
                \begin{theorem}
                    If $\topspace{X}$ is a contractible topological space, then
                    it is path connected.
                \end{theorem}
                \begin{proof}
                    For if $X$ is contracible, then $\identity{X}$ is homotopic
                    to a constant map. Let $g:X\rightarrow{X}$ defined by
                    $g(x)=c$ be such a constant mapping, $H$ the homotopy, and
                    let $x_{1},x_{2}\in{X}$. Then $H(x_{1},t)$ is a path between
                    $x_{1}$ and $c$, and $H(x_{2},t)$ is a path between
                    $c$ and $x_{2}$. By concatenation there is a path between
                    $x_{1}$ and $x_{2}$. Hence, $X$ is path connected.
                \end{proof}
                Contractible and deformation retractions to a point aren't quite
                identical notions. There are spaces with \textit{weak}
                deformation retracts to a single point but no \textit{strong}
                deformation retracts. The concept of contractible doesn't care
                how the contraction is done. That is, the point being contracted
                to may move during the homotopy. This does give us the following
                theorem.
                \begin{theorem}
                    If $\topspace{X}$ is a topological space, then $X$ is
                    contractible if and only if there is a (weak) deformation
                    retraction of $X$ to a point.
                \end{theorem}
                \begin{proof}
                    For $X$ is contracible if and only if $\identity{X}$ is
                    nullhomotopic. Let $H$ be the homotopy. Then $H$ is a
                    weak deformation retraction of $X$ to a single point.
                    Similarly, if such a weak deformation retraction exists,
                    then $\identity{X}$ is nullhomotopic.
                \end{proof}
                We showed that any pair of functions into a topological vector
                space are homotopic, and we used the straight line homotopy.
                This generalizes to all contracible spaces.
                \begin{theorem}
                    If $\topspace[Y]{Y}$ is a contracible topological space, if
                    $\topspace[X]{X}$ is a topological space, and if
                    $f,g:X\rightarrow{Y}$ are continuous function, then they are
                    homotopic.
                \end{theorem}
                \begin{proof}
                    For if $Y$ is contracible, then there is a homotopy $H$
                    between $\identity{Y}$ and a constant map. But then
                    $G:X\times{I}\rightarrow{Y}$ defined by
                    $G(x,t)=H(f(x),t)$ is a homotopy between $f(x)$ and the
                    constant point $c\in{Y}$. Hence, $f$ is homotopic to the
                    constant map $h(x)=c$. Similarly, $g$ is homotopic to $h$.
                    But homotopic is transitive, and hence $f$ is homotopic to
                    $g$.
                \end{proof}
                \begin{figure}[H]
                    \centering
                    \captionsetup{type=figure}
                    \includegraphics{images/Dunce_Cap_001.pdf}
                    \caption{The Dunce Cap}
                    \label{fig:Dunce_Cap_001}
                \end{figure}
                Many spaces that appear to be non-contractible can actually be
                contracted. The converse usually isn't true, if a space looks
                contractible it \textit{probably} is. An example of a space that
                doesn't look contracible, but is, is the \textit{dunce cap}. We
                use the method of gluing the sides of a polygon together the
                same way we did with the torus and M\"{o}bius strip, but instead
                perform this operation on a triangle
                (Fig.~\ref{fig:Dunce_Cap_001}). After gluing the final two
                circles together, we have a beanie hat that has been wrapped
                on itself. The portion of the hat where the head goes is still
                intact and we may pull the cap back and around to the gluing
                point shown at the far right of Fig.~\ref{fig:Dunce_Cap_001}.
                The result shows this space is contractible. Later with the
                help of van Kampen's theorem we will make this rigorous.
                \par\hfill\par
                Perhaps even more exotic and less obvious is Bing's
                \textit{house with two rooms}, named after the American
                mathematician R. H. Bing.
                \begin{figure}[H]
                    \centering
                    \captionsetup{type=figure}
                    \includegraphics{images/House_with_Two_Rooms_002.pdf}
                    \caption{The House with Two Rooms}
                    \label{fig:House_with_Two_Rooms_002}
                \end{figure}
                The house has two rooms, an upstairs and a downstairs. To get to
                the basement you need to climb onto the roof and drop down the
                tunnel on the left. Similarly, to get to the attic one would
                need to dig underneath the house and climb through the tunnel on
                the right. In addition, each tunnel has a wall connecting it to
                the exterior. The result is shown in
                Fig.~\ref{fig:House_with_Two_Rooms_001}. It's hard to see why
                this is contractible, but what's even stranger is that a
                thickened house with two rooms (a house made with actual wood
                that has thickness) is \textit{homeomorphic} to the 3-ball
                $\nball[3]$. That is homeomorphic, not just homotopy equivalent,
                a much stronger notion. If one could show this then the house
                with two rooms is just a retract of a contractible space. If one
                knows that all retracts of contractible spaces are themselves
                contractible, then we're done. First, let's show that the
                thickened house is homeomorphic to $\nball[3]$. To do this we'll
                first create something that is homotopy equivalent to the house
                by taking the supports of the tunnels and squeezing them to the
                wall. We then push the tunnels into the corners. What remains is
                shown in Fig.~\ref{fig:House_with_Two_Rooms_002} (the front
                wall being removed for the sake of visualization).
                \par\hfill\par
                We take a thickened version of this and show it is homeomorphic
                to $\nball[3]$. The 3-ball is homeomorphic to a fat cube
                $(0,1)^{3}$. We take this and imagine it's made of clay and push
                into one of the corners. We then push outwards and start to
                hollow out the bottom part, creating our basement. We do the
                same procedure with the other tunnel, hollowing out the top to
                create our attic. This homotopy is actually a family of
                homeomorphisms. We are not collapsing or squeezing together any
                parts, the process is continuous and has a continuous inverse,
                and hence the end result is homeomorphic to what we started
                with. This shows the thickened house with two rooms is
                homeomorphic to $\nball[3]$. We finish our demonstration that
                Bing's house is contractible by showing that retracts of
                contractible spaces are contractible.
                \begin{figure}
                    \centering
                    \captionsetup{type=figure}
                    \includegraphics{images/House_with_Two_Rooms_001.pdf}
                    \caption{Homotopy Equivalent of the House with Two Rooms}
                    \label{fig:House_with_Two_Rooms_001}
                \end{figure}
                \begin{theorem}
                    If $\topspace{X}$ is a contractible topological space, and
                    if there is a retract $f:X\rightarrow{A}$ onto a subset
                    $A\subseteq{X}$, then $A$ is contractible.
                \end{theorem}
                \begin{proof}
                    For $\topspace{X}$ is contractible if and only if
                    $\identity{X}$ is nullhomotopic. That is, there is a point
                    $x_{0}\in{X}$ such that $\identity{X}$ is homotopic to the
                    function $g:X\rightarrow{X}$ defined by $g(x)=x_{0}$. But if
                    $f:X\rightarrow{A}$ is a retract, then
                    $f|_{A}=\identity{A}$. Hence $f\circ{H}|_{A\times{I}}$ is a
                    homotopy between $\identity{A}$ and $f(x_{0})$. Therefore,
                    $A$ is contracible.
                \end{proof}
        \section{Cell Complexes}
            Topological manifolds are spaces that look locally like some
            $\nspace$. Neither our our sphere with string attached to the poles
            nor our kettle bell (Figs.~\ref{fig:Sphere_with_String_Attached} and
            \ref{fig:Kettle_Bell}, respectively) can be considered topological
            manifolds since there are these strange points where the space
            locally looks like $\nspace[2]$ plus a pike. It is perhaps
            worthwhile to attempt to generalize manifolds to spaces that have
            various regions that are of different dimension. The approach that
            is best for homotopy theory was developed by the British
            mathematician J. H. C. Whitehead (1904-1960 C.E.) and is called a
            CW complex. It is very combinatorial and computable, but has one
            subtle problem: it is not known if every manifold is homeomorphic to
            a CW complex. Indeed, in dimension 4 it's not even known in the
            compact case (the result is true for compact manifolds of high
            enough dimension). It is perhaps a poor use of language to call this
            a generalization, but there is a saving grace: Every manifold is
            homotopy equivalent to a CW complex. So it is in the setting of
            homotopy theory where this new concept works best. Without further
            adieu, we describe the construction.
            \par\hfill\par
            We start with a collection of points, a discrete space $X^{0}$. The
            elements are called the 0 cells. We then inductively define the $n$
            skeleton $X^{n}$ from $X^{n-1}$ by attaching open balls $\nball$,
            called $n$ cells, via \textit{attaching} maps
            $\varphi:\nsphere[n-1]\rightarrow{X}^{n-1}$. That is, we attached
            the boundary of $\nball$ (which is the $n-1$ sphere) to $X^{n-1}$.
            We topologize this in same way one does an \textit{adjunction} space
            in point-set topology. We look at the disjoint union
            $X^{n-1}\coprod_{\alpha}\closure[]{\nball}_{\alpha}$ and give this
            the quotient topology by gluing elements of the boundary
            $x\in\partial\closure[]{\nball}_{\alpha}=\nsphere[n-1]$ with
            $\varphi_{\alpha}(x)$.
            \par\hfill\par
            Some explaination of notation to avoid confusion: $\nball$ is the
            \textit{open} unit ball in $\nspace$, and $\closure[]{A}$ denotes
            the closure of a subset $A\subseteq\nspace$ with respect to the
            usual topology. Hence, $\closure[]{\nball}$ is simply the
            \textit{closed} ball. From point-set we adopt the notation
            $\partial{A}$ which means the topological \textit{boundary} of $X$.
            More generally, if $\topspace{X}$ is a topological space and
            $A\subseteq{X}$, the boundary of $A$ is:
            \begin{equation}
                \partial{A}=\closure{A}\setminus\interior{A}
            \end{equation}
            where $\closure{A}$ and $\interior{A}$ denote the closures and
            interiors of $A$ with respect to the topology $\tau$. From this,
            $\partial\closure[]{\nball}$ is just the $n-1$ sphere
            $\nsphere[n-1]$.
            \par\hfill\par
            Now, if the process stops after some $n\in\mathbb{N}$ steps, then we
            can topologize this space with the quotient topology. If it does not
            then we need a way of placing a topology on the set
            $X=\bigcup_{n}X^{n}$. We say that $\mathcal{U}\subseteq{X}$ is open
            if and only if for all $n\in\mathbb{N}$ it is true that
            $\mathcal{U}\cap{X}^{n}$ is an open subset of $X^{n}$ with its
            quotient topology. The dimension of $X$ is the largest dimension of
            cells in $X$, if such a maximum exist.
            \begin{example}
                The closed unit interval $I=[0,1]$ is a CW complex by attaching
                the open unit interval $(0,1)$ to the 0-skeleton
                $X^{0}=\{0,1\}$. The open unit interval is homeomorphic to
                $\nball[1]$, which is just the set $(\minus{1},1)$, and this
                gluing is indeed a valid attaching map.
            \end{example}
            \begin{example}
                The torus $\ntorus[]=\nsphere[1]\times\nsphere[1]$ can be
                obtained using our quotienting of the square
                (Fig.~\ref{fig:Square_to_Torus}). We Start with a point and then
                attach two 1-cells. The result is a figure-eight. From here we
                attach the square $[0,1]\times[0,1]$, which is homeomorphic to
                $\closure[]{\nball[2]}$, along these circles in a manner similar
                to how we glued up the square to get a torus. The result is
                shown in Fig.~\ref{fig:CW_Complex_Torus}.
            \end{example}
            \begin{figure}
                \centering
                \captionsetup{type=figure}
                \includegraphics{images/Torus_CW_Complex.pdf}
                \caption{CW Complex for a Torus}
                \label{fig:CW_Complex_Torus}
            \end{figure}
            \begin{example}
                A two-holed surface, also known as a genus 2 surface, can be
                realized as a CW complex in a similar manner as the torus. The
                torus can be obtained by identifying parts on a square, whereas
                a genus 2 surface can be obtained by identifying parts of an
                octagon. An explicit representation is shown in
                Fig.~\ref{fig:Octagon_to_Genus_Two}. This also shows that a
                genus two surface can be represented as a CW complex with one
                0-cell, four 1-cells, and one 2-cell.
            \end{example}
            \begin{figure}[H]
                \centering
                \captionsetup{type=figure}
                \includegraphics{images/Octagon_to_Two_Genus.pdf}
                \caption{Turning an Octagon into a Genus Two Surface}
                \label{fig:Octagon_to_Genus_Two}
            \end{figure}
            \begin{example}
                The sphere can be seen as a CW complex if we take a single point
                for our 0 skeleton $X^{0}$, set $X^{1}=X^{0}$, and then attach
                the boundary of a single 2-cell to this point. The result is the
                sphere in Fig.~\subref{fig:Sphere_CW_Complex}.
            \end{example}
            \begin{figure}
                \centering
                \captionsetup{type=figure}
                \begin{subfigure}[b]{0.49\textwidth}
                    \centering
                    \captionsetup{type=figure}
                    \includegraphics{images/Sphere_CW_Complex.pdf}
                    \subcaption{Standard CW Complex}
                    \label{fig:Sphere_CW_Complex}
                \end{subfigure}
                \begin{subfigure}[b]{0.49\textwidth}
                    \centering
                    \captionsetup{type=figure}
                    \includegraphics{images/Sphere_Alt_CW_Complex.pdf}
                    \subcaption{Alternative CW Complex}
                    \label{fig:Sphere_Alt_CW_Complex}
                \end{subfigure}
                \caption{CW Complexes on a Sphere}
                \label{fig:CW_Complex_on_Sphere}
            \end{figure}
            \begin{example}
                The representation of a topological space by means of a CW
                complex is not unique. For example, if we place a single point
                and then attach the equator, given us a circle, we may then glue
                the northern and southern hemispheres. The result is a CW
                complex built from one 0-cell, one 1-cell, and two 2-cells.
                This is depicted in Fig.~\subref{fig:Sphere_Alt_CW_Complex}.
            \end{example}
            \begin{example}
                Similar to how the sphere has more than one representation as a
                CW complex, so does the genus two surface. We can represent it
                by six 0-cells, twelve 1-cells, and four 2-cells. This is shown
                in Fig.~\ref{fig:Genus_Two_CW_Complex}.
            \end{example}
            \begin{figure}[H]
                \centering
                \captionsetup{type=figure}
                \includegraphics{images/Torus_Genus_Two_CW_Complex.pdf}
                \caption{Genus Two Surface as a CW Complex}
                \label{fig:Genus_Two_CW_Complex}
            \end{figure}
            \begin{example}
                All of the examples discussed so far have been oriented smooth
                manifolds. The familiar non-oriented manifolds can also be given
                CW complex. The real projective plane can be defined several
                ways, the most common of which being by identifying all lines
                through the origin in $\nspace[3]$. We equip
                $\nspace[3]\setminus\{\vector{0}\}$ with the equivalence
                relation $R$ such that $\vector{x}R\vector{y}$ if and only if
                there exists a non-zero $\lambda\in\mathbb{R}$ with
                $\vector{y}=\lambda\vector{x}$. The quotient space
                $\nspace[3]/R$ is the real projective plane. Better yet, if we
                think of only unit vectors this becomes a quotient on
                $\nspace[2]$ which identifies all antepodal points. This also
                shows the real projective plane $\mathbb{RP}^{2}$ is compact.
                This construction generalizes to all $n\in\mathbb{N}$. A final
                way of viewing $\mathbb{RP}^{2}$ is via a
                \textit{fundamental polygon}, just like how the torus,
                M\"{o}bius strip, and two genus surface have been described. The
                fundamental polygon for $\mathbb{RP}^{2}$ is shown in
                Fig.~\ref{fig:Square_to_Real_Proj_Plane}. This shows that
                $\mathbb{RP}^{2}$ consists of one 0-cell, one 1-cell, and one
                2-cell.
            \end{example}
            \clearpage
            \begin{minipage}[b]{0.42\textwidth}
                \centering
                \begin{figure}[H]
                    \centering
                    \captionsetup{type=figure}
                    \includegraphics{images/Square_to_Real_Proj_Plane.pdf}
                    \caption{Fundamental Polygon of $\mathbb{RP}^{2}$}
                    \label{fig:Square_to_Real_Proj_Plane}
                \end{figure}
            \end{minipage}
            \hfill
            \begin{minipage}[b]{0.56\textwidth}
                It would be useless to try and draw the contortions the square
                must undergo in order to convert from the fundamental polygon
                into the real project plane, but the process does admit a
                parameterization into $\nspace[3]$ that we may draw. It is
                \textit{not} an embedding since there is no embedding of
                $\mathbb{RP}^{2}$ into $\nspace[2]$. Moreover, this
                parameterization is not even an \textit{immersion}, a weaker
                form of embedding studied in differential topology.
            \end{minipage}
            \par
            \begin{minipage}[b]{0.56\textwidth}
                The resulting object is called the \textit{cross cap}
                (Fig.~\ref{fig:Cross_Cap}). It must intersects itself since it
                is impossible to embed $\mathbb{RP}^{2}$ into $\nspace[2]$. It
                is difficult to see so a wireframe representation is given in
                Fig.~\ref{fig:Inside_Cross_Cap}. The pinch point present in this
                figure means this is not an \textit{immersion}. David Hilbert
                posed the question to Werner Boy asking him to prove no such
                immersion exists. Instead, Boy came up with a way to do it, the
                resulting surface now known as the \textit{Boy surface}.
                Topologically, these are all just $\mathbb{RP}^{2}$. Higher
                dimensional analogs of this result in the real projective space
                $\mathbb{RP}^{n}$.
            \end{minipage}
            \hfill
            \begin{minipage}[b]{0.42\textwidth}
                \centering
                \begin{figure}[H]
                    \centering
                    \captionsetup{type=figure}
                    \includegraphics{images/Real_Proj_Plane_Cross_Cap_001.pdf}
                    \caption{The Cross Cap}
                    \label{fig:Cross_Cap}
                \end{figure}
            \end{minipage}
            \par\hfill\par
            \begin{minipage}[b]{0.42\textwidth}
                \centering
                \begin{figure}[H]
                    \centering
                    \captionsetup{type=figure}
                    \includegraphics{%
                        images/Real_Proj_Plane_Cross_Cap_Wireframe.pdf%
                    }
                    \caption{Inside of the Cross Cap}
                    \label{fig:Inside_Cross_Cap}
                \end{figure}
            \end{minipage}
            \hfill
            \begin{minipage}[b]{0.56\textwidth}
                Before going \textit{up} in dimension, it is worthwhile going
                down and seeing what $\mathbb{RP}^{0}$ and $\mathbb{RP}^{1}$
                look like. The zero dimensional real projective plane is just a
                point. That is, we take $\mathbb{R}\setminus\{0\}$ and give it
                the equivalence relation $xRy$ if and only if there is a
                $\lambda\in\mathbb{R}\setminus\{0\}$ with $y=\lambda{x}$. But
                this is true of every non-zero element since
                $\mathbb{R}\setminus\{0\}$ is a group under multiplication.
                Explicitly, let $\lambda=y/x$. For $n=1$ we can note that there
                are only two connected topological manifolds of dimension 1,
                $\mathbb{R}$ and $\nsphere[1]$, and all of the real projective
                spaces are manifolds. Moreover,$\mathbb{RP}^{n}$ is compact for
                all $n$ so this rules out $\mathbb{R}$. 
            \end{minipage}
            \par\hfill\par
            From this we conclude $\mathbb{RP}^{1}$ is homeomorphic to
            $\nsphere[1]$. All of the interesting stuff finally kicks in with
            $n=2$. Higher dimensional real projective spaces can be realized by
            attaching a closed ball $\closure[]{\nball}$ to $\mathbb{RP}^{n-1}$.
            Hence by induction every real projective space is a CW complex with
            one $k$ cell for all $k=0,1,\dots,n$.
            \begin{example}
                Our first infinite dimensional example is obtained from
                the real projective spaces $\mathbb{RP}^{n}$. We define
                $\mathbb{RP}^{\infty}=\bigcup_{n}\mathbb{RP}^{n}$ and give
                this the topology of an infinite dimensional CW complex with
                $n$ skeleton $X^{n}=\mathbb{RP}^{n}$.
            \end{example}
            \begin{example}
                We can similarly obtain $\nsphere[\infty]$ from
                $\bigcup_{n}\nsphere$ and give this the structure of a CW
                complex. This can be phrased as the space of all finitely
                supported sequences $a:\mathbb{N}\rightarrow\mathbb{R}$ such
                that $\sum_{n}a_{n}^{2}=1$. Finitely supported simply means
                there is an $N\in\mathbb{N}$ such that for all $n>N$ it is
                true that $a_{n}=0$.
            \end{example}
            \par\hfill\par
            \begin{minipage}[b]{0.42\textwidth}
                \centering
                \begin{figure}[H]
                    \centering
                    \captionsetup{type=figure}
                    \includegraphics{images/Klein_Bottle.pdf}
                    \caption{The Klein Bottle}
                    \label{fig:Klein_Bottle}
                \end{figure}
            \end{minipage}
            \hfill
            \begin{minipage}[b]{0.56\textwidth}
                Since we're given most of the simple structures one can obtain
                from a square, we may as well describe the last one. The torus
                is obtained by identifying opposite sides of a square, the
                M\"{o}bius strip identified only two sides but with a twist, and
                the real projective plane had lots of twists. The last space is
                half-way between a torus and $\mathbb{RP}^{2}$. That is, we
                twist one side but not the other. The result is called a Klein
                bottle.
            \end{minipage}
    \chapter{The Fundamental Group}
    \section{Notes from Dartmouth S2019}
    \begin{example}
        $SO(3)$, $3\times{3}$ matrices that are orthogonal and have determinant
        1, and $SU(2)$. Both of these are differentiable manifolds. How can we
        determine if they are homeomorphic or not? To every continuous map
        between topological spaces $X$ and $Y$ we can correspond groups $G(X)$
        and $G(Y)$ and a group homomorphism $\varphi:G(X)\rightarrow{G}(Y)$ that
        corresponds somehow to the continuous function. By showing that certain
        properties are preserved by this correspondence we can show that $SO(3)$
        and $SU(2)$ are not homeomorphic.
    \end{example}
    \begin{align}
        &f:X\rightarrow{Y}\rightarrow{Z}\\
        &\varphi:G(X)\rightarrow{G}(Y)\rightarrow{G}(Z)
    \end{align}
    The maps between topological spaces and groups are called \textit{Functors}.
    The category of topological spaces is somewhat to large. From point-set
    topology there are many pathelogical sets that are difficult to manage. In
    differential topology we study smooth manifolds which are locally like
    Euclidean space and have a lot of structure on them. Even nicer are
    \textit{Affine Varieties}, which are solution sets to polynomials that are
    studied in algebraic geometry. In the late 1940s Whitehead came up with the
    following category, that of CW Complexes. Not every topological space is a
    CW Complex, but in some respect every topological space is like a CW
    complex. All manifolds are CW complexes.
    \subsection{CW Complexes}
        There's no intrinsic metric on the space.
        The open unit square is the a cell, since
        it is homeomorphic to the open unit disk.
        A zero dimensional cell is a point.
        One-cells are edges, two-cells are
        bubbles, and so forth.
        \begin{ldefinition}{CW Complexes}
            A CW Complex is a topological space $X$ that is the disjoint union
            of cells.
        \end{ldefinition}
        Typically CW Complexes are metric spaces.
        \begin{example}
            A zero dimensional CW complex is the disjoint union of a bunch of
            points. So $X=X^{0}$, and this has the discrete topology on it. That
            is, $\tau=\mathcal{P}(X)$, every set is open. The only defining
            characteristic is the cardinality of the space.
        \end{example}
        \begin{example}
            One dimensional CW complexes are a combination of edges and points.
            For each edge the endpoints must go to a point in the 0-skeleton
            $X^{0}$. We can send the two endpoints to the same point, which
            creates a loop, or to distinct points, which creates an edge.
        \end{example}
        \begin{example}
            Hawaiian ear-rings. A subset of $\mathbb{R}^{2}$ that looks like a
            CW complex, but is not.
        \end{example}
    \subsection{Homotopy Equivalence}
        \begin{theorem}
            $X$ is a topological space, $A\subseteq{X}$, and
            if there is a deformation retract of $X$ onto
            $A$, then $X$ is homotopy equivalent to $A$.
        \end{theorem}
        Two steps or criteria that are common in algebraic
        topology.
        \begin{ldefinition}{CW Pair}
            A CW Pair, $(X,A)$, is a CW Complex $X$ and a
            subset $A\subseteq{X}$ such that $A$ is closed,
            and $A$ is the union of cells of $X$.
        \end{ldefinition}
        The set $A$ can be considered as a sub-complex
        of $X$.
        \begin{theorem}
            If $(X,A)$ is a CW pair and if $A$ is
            contracible, then $q:X\rightarrow{X}/A$
            is a homotopy equivalence.
        \end{theorem}
        So 1-Skeletons and 1 dimensional CW Complexes can
        be contracted down to a bunch of points with loops
        on the points. That is, homotopy equivalence is
        preserved. Circles are not contractible, as well
        we eventually see, so we cannot get rid of loops.
    \subsection{Homotopy Extension Property}
        Let $X(,\tau)$ be a topological space,
        $A\subseteq{X}$ a subspace, and let
        $f:A\rightarrow{Y}$ be continuous. Does
        $f$ extend to a map $\tilde{f}:X\rightarrow{X}$
        such that $\tilde{f}|_{A}=f$? Not always, let
        $A=S^{1}$ and $X$ be the torus. For which pairs
        $(X,A)$ is this always possible? Well, suppose
        it is true. Then $id_{A}:A\rightarrow{A}$ has
        an extension $\tilde{id}_{A}:X\rightarrow{A}$.
        But then $A$ is a retract of $X$. Thus, it
        is a necessary condition that $A$ is a retract
        of $X$ for it to be true. As it turns out, it is
        also a sufficient condition. If
        $r:X\rightarrow{A}$ is a retract
        \begin{ldefinition}{Homotopy Extension Property}
            A subset $A\subseteq{X}$ with the homotopy
            extension property is a set such that for all
            maps $f:X\rightarrow{Y}$ and for every
            homotopy $f_{t}:A\rightarrow{Y}$,
            $f_{0}=f|_{A}$, extends to a homotopy of
            $\tilde{f}_{t}:X\rightarrow{Y}$ with
            $\tilde{f}_{0}=f_{0}$.
        \end{ldefinition}
        \begin{theorem}
            Every CW Pair $(X,A)$ has the homotopy
            extension property.
        \end{theorem}
        \begin{proof}
            We will product a retract:
            \begin{equation}
                X\times{I}\overset{r}{\longrightarrow}
                    X\times{0}\cup{A}\times{I}=Z
            \end{equation}
            First, $(D^{n},\partial{D}^{n})$, which is
            $(D^{n},S^{n-1})$, and this has the
            homotopy extension property.
        \end{proof}
        Recall that if $f:X\rightarrow{Y}$ is a map such that,
        for a given $A\subseteq{X}$, and for all $a\in{A}$,
        there is a fixed $y_{0}\in{Y}$ such that $f(a)=y$, then
        the quotient map $q:X\rightarrow{X/A}$ has a unique
        liften map $\overline{f}$ such that
        $\overline{f}\circ{q}=f$. Moreover, $\overline{f}$ is
        continuous with respect to the quotient topology.
        \begin{theorem}
            If $(X,A)$ is a CW pair, and if $A$ is
            homotopy equivalent to a point
            (Contractible), then the quotient map
            $q:X\rightarrow{X/A}$ is a homotopy
            equivalence.
        \end{theorem}
        \begin{proof}
            For let $(X,A)$ be a CW pair. Then it has the
            homotopy extension property. If $A$ is contractible
            then there is a homotopy of maps
            $f_{t}:A\rightarrow{A}$ such that $f_{0}=id_{A}$ and
            $f_{1}=a_{0}$, for some $a_{0}\in{A}$. We can
            extend this to $X$ by defined
            $f_{0}:X\rightarrow{X}$ as $f_{0}=id_{X}$. By
            the homotopy extension property, there's an
            extension $F_{t}:X\rightarrow{X}$ such that
            $F_{t}|_{A}=f_{t}$. Let $q$ be the quotient
            mapping and let $qf_{t}$ be the lifting mapping.
            Square $X-X$ on top, $X/A$ $X/A$ on bottom. Maps are
            $q$ and $qf_{t}$. Bottom map is
            $\overline{f}_{t}q$. Let $k$ be the mapping
            of $X/A$ to $X$. Then $q$ and $k$ are homotopy
            equivalences. Thus, $X$ and homotopy equivalent
            to $X/A$. Since $qk=\overline{f}_{1}\simeq{id}_{X/A}$
            and $kq=f_{1}\simeq{f}_{0}=id_{X}$.
        \end{proof}
        \begin{theorem}
            If $(X,A)$ is CW pair, $f,g:A\rightarrow{X_{0}}$
            are homotopic attaching maps, then:
            \begin{equation}
                X_{0}\cup_{f}X\simeq{X}_{0}\cup_{g}X
            \end{equation}
        \end{theorem}
        \begin{theorem}
            $X\times{I}$ deformation retracts to
            $(X\times\{0\})\cup(A\times\{1\})$, then
        \end{theorem}
        This determines a deformation retract of:
        \begin{equation}
            Z=X_{0}\cup_{F}(X_{1}\times{I})
        \end{equation}
        As an aside, there are two things, a wedge product
        $\lor$ and a smash product $\land$. Given two pointed
        space $(X,x_{0})$, $(Y,y_{0})$, the wedge product is:
        \begin{equation}
            (X,x_{0})\lor(Y,y_{0})
            =X\coprod{Y}/\{x_{0}\sim{y}_{0}\}
        \end{equation}
        The smash product is:
        \begin{equation}
            X\land{Y}=
            X\times{Y}/\Big(
                (\{0\}\times{Y})\cup(X\times\{y_{0}\})
            \Big)
        \end{equation}
        \begin{example}
            $S^{1}\land{S}^{1}$ is simply $S^{2}$.
            $S^{1}\times{S}^{1}$ is a torus, but we've collapsed
            the entire boundary down to a point. So we have a
            2-cell with no boundary, which is a sphere.
        \end{example}
    \subsection{Winding Number}
        \begin{theorem}
            If $f:[0,1]\rightarrow{S}^{1}$ is a continuous
            function, then there is a unique continuous lift
            $\tilde{f}:[0,1]\rightarrow\mathbb{R}$ such that
            $f=q\circ\tilde{f}$, where $q$ is the quotient map
            of $\mathbb{R}$ onto $S^{1}$.
        \end{theorem}
        \begin{proof}
            For let $\mathcal{U}_{1}=S^{1}\setminus\{(1,0)\}$ and
            $\mathcal{U}_{2}=S^{1}\setminus\{(0,1)\}$. This is an
            open cover of $S^{1}$. If $f:I\rightarrow{S}^{1}$ is
            continuous, then $f^{-1}(\mathcal{U}_{1})$ and
            $f^{-1}(\mathcal{U}_{2})$ form an open cover of
            $I$. But any open subset of $\mathbb{R}$ is the
            countable union of disjoint open intervals. Thus
            $f^{-1}(\mathcal{U}_{1})$ and
            $f^{-1}(\mathcal{U}_{2})$ are the countable union of
            disjoint open intervals. But $I$ is compact, and
            thus any open cover has a finite subcover. Therefore
            there are finitely many intervals $I_{1},\dots,I_{k}$
            such that $f(I_{k})\subseteq\mathcal{U}_{1}$ or
            $f(I_{k})\subseteq\mathcal{U}_{2}$.
        \end{proof}
        \begin{ldefinition}{Winding Number}
            The winding number of a loop
            $f:S^{1}\rightarrow{S}^{1}$ is:
            \begin{equation}
                wind(f)+\tilde{f}(1)-\tilde{f}(0)
            \end{equation}
            Where $\tilde{f}$ is a lifting map of $f$ to
            $\mathbb{R}$.
        \end{ldefinition}
\section{Stuff}
    HW Stuff. Chapter 0: 2, 3, 9, 16, 20. Bonus 6.
    \par\hfill\par
    A loop is a continuous function $f:S^{1}\rightarrow{X}$. Given
    a loop $f:S^{1}\rightarrow{S}^{1}$, there is a lifting map
    $\tilde{f}:S^{1}\rightarrow\mathbb{R}$. The winding number
    of the loop $f$ is defined as $w(f)=\tilde{f}(1)-\tilde{f}(0)$.
    \begin{theorem}
        If $f,g:S^{1}\rightarrow{S}^{1}$ are loops such that
        $f(0)=g(0)$, then $f\simeq{g}$ if and only if
        $w(f)=w(g)$.
    \end{theorem}
    \begin{proof}
        For let $\tilde{f},\tilde{g}$ be two lifts of $f$ and $g$,
        respectively, such that $\tilde{f}(0)=\tilde{g}(0)$.
        This can be done since given such lifts, they will differ
        by at most an integer. But if $w(f)=w(g)$, then
        $\tilde{f}(1)=\tilde{g}(1)$. Then let $H$ be the straight
        line homotopy between $\tilde{f}$ and $\tilde{g}$:
        \begin{equation}
            H(x,t)=\tilde{f}(x)t+(1-t)\tilde{g}(x)
        \end{equation}
        Let $h_{t}=qH$, where $q$ is the projection mapping. This
        is a homotopy between $f$ and $g$.
    \end{proof}
    Thus we have that the winding number is a homotopy invariant.
    Let $[X,y]$ be the set of homotopy equivalence classes of
    continuous maps $X\rightarrow{Y}$. Then $[S^{1},S^{1}]$ can
    be put into a one-to-one correspondence with $\mathbb{Z}$
    by mapping $f\mapsto{w}(f)$.
    \begin{theorem}
        There is no retract from $D^{2}$ onto it's boundary
        $\partial{D}^{2}=S^{1}$.
    \end{theorem}
    \begin{proof}
        Consider the identity map $id_{S^{1}}$. Then
        $w(id_{S^{1}})=1$. This is a loop in $D^{2}$ that is
        contractible to a point, and is thus null-homotopic.
        Suppose $r:D^{2}\rightarrow{S}^{1}$ is a retract. 
        Then $rf_{0}\simeq{r}f_{1}$, and thus
        $w(rf_{0})=w(rf_{1})$. But $w(rf_{0})=w(id_{S^{1}})=0$, and
        $rf_{1}$ is a point, so $w(rf_{1})=0$, a contradiction. Thus
        there is no retract.
    \end{proof}
    We now consider functors that take topological spaces to
    groups, mapping continuous functions to group homomorphisms.
    Recall that a path is a continuous function $f:I\rightarrow{X}$.
    A homotopy of paths $f,g$ is a function $F_{t}$ such that
    $F_{0}=f$ and $F_{1}=g$. The concatenation operation is:
    \begin{equation}
        (f*g)(t)=
        \begin{cases}
            f(2t),&0\leq{t}\leq\frac{1}{2}\\
            g(2t-1),&\frac{1}{2}<t\leq{1}
        \end{cases}
    \end{equation}
    \begin{theorem}
        If $U,V\subseteq{X}$ are closed sets, and if
        $U\cup{V}=X$, and if $k:X\rightarrow{Y}$ is such that
        $k|_{U}$ and $k|_{V}$ is continuous, then $k$ is continuous.
    \end{theorem}
    \begin{theorem}
        If $f_{0}\simeq{f}_{1}$, $g_{0}\simeq{g}_{1}$, then
        $f_{0}*g_{0}\simeq{f}_{1}*g_{1}$.
    \end{theorem}
    \begin{proof}
        Concatenate homotopies: $h_{t}=f_{t}*g_{t}$. This is a
        homotopy.
    \end{proof}
    A loop is a path $f:I\rightarrow{X}$ such that $f(0)=f(1)$, and
    this is called the base-point. We use this notion to define the
    fundamental group $\pi_{1}(X,x_{0})$.
    \begin{ldefinition}{Fundamental Group}
        The fundemtanl group of a topological space $X$ with a
        base point $x_{0}$ is the set:
        \begin{equation}
            \pi_{1}(X,x_{0})=\{[f]:f\in{C}(S^{1},X)\}
        \end{equation}
        Where equivalent functions are functions that are homotopy
        equivalent.
    \end{ldefinition}
    \begin{theorem}
        If $f:I\rightarrow{X}$, $\phi:I\rightarrow{I}$ is such that
        $\phi(0)=0$ and $\phi(1)=1$, then $f\circ\phi\simeq{f}$.
    \end{theorem}
    \begin{proof}
        Let $\phi_{t}(s)=(1-t)\phi(s)+ts$. This is a homotopy between
        $\phi$ and $id_{I}$. Let $f_{t}=f\phi_{t}$. This is a
        homotopy.
    \end{proof}
    \begin{theorem}
        If $X$ is a topological space, $x_{0}\in{X}$, and if
        $*$ is the concatenation operation, then
        $(\pi_{1}(X,x_{0}),*)$ is a group.
    \end{theorem}
    \begin{example}
        We have seen that $\pi_{1}(S^{1},x_{0})$ can be put into
        a correspondence with $\mathbb{Z}$, but moreover the
        group structure is preserved. That is,
        $(\pi_{1}(S^{1},x_{0}),*)$ is isomorphic to
        $(\mathbb{Z},+)$. While this group is Abelian,
        the fundamental of a topological space need not be, and
        in general it isn't.
    \end{example}
    \subsection{Lecture 6}
        Let $X$ be a space, and $x_{0}\in{X}$ a point in $X$.
        The fundamental group of $X$ with base point $x_{0}$
        is $\pi_{1}(X,x_{0})$, which is the set of loops with
        base point $x_{0}$, modulo homotopy, equipped with the
        concatenation operation.
        \begin{theorem}
            If $X$ and $Y$ are topological space, if
            $X\times{Y}$ has the product topology, and if
            $(x_{0},y_{0})\in{X}\times{Y}$, then:
            \begin{equation}
                \pi_{1}(X\times{Y},(x_{0},y_{0}))
                \simeq\pi_{1}(X,x_{0})\otimes\pi_{1}(Y,y_{0})
            \end{equation}
        \end{theorem}
        \begin{example}
            We have computed $\pi_{1}(S^{1},x_{0})$ for any
            point $x_{0}\in{S}^{1}$ by using the notion of
            winding number. We saw that:
            \begin{equation}
                \pi_{1}(S^{1},x_{0})\simeq\mathbb{Z}
            \end{equation}
            But the torus $T^{2}$ can be seen as
            $S^{1}\times{S}^{1}$. Thus, for any point
            $x_{0}\in{T}^{2}$, we have:
            \begin{equation}
                \pi_{1}(T^{2})\simeq
                \mathbb{Z}\times\mathbb{Z}
                =\mathbb{Z}^{2}
            \end{equation}
            We've seen that the torus can be identified by
            an equivalence relation on the square, which then
            creates two loops. These are the two inner circles
            of the torus. It turns out that any loop in the
            torus is going to be the product of these two loops.
            That is, the fundamental group is generated by
            two elements. By considering a trivial example,
            we see that this group has the relation:
            \begin{equation}
                \pi_{1}(S^{1},x_{0})=
                \langle{a,b}|aba^{-1}b^{-1}=e\rangle
            \end{equation}
            This relation says that:
            \begin{equation}
                ab=ba
            \end{equation}
            And thus the group is Abelian. So we have that
            the fundamental group is an Abelian group
            generated by two elements, and thus we see the
            isomorphism with $\mathbb{Z}^{2}$.
        \end{example}
        Recall that a loop is a map $f:I\rightarrow{X}$
        such that $f(0)=f(1)$. This is equivalent to a
        function $f:S^{1}\rightarrow{X}$. We can use this to
        define the higher homotopy groups, $\pi_{n}(X,x_{0})$,
        by considering functions $f:S^{n}\rightarrow{X}$,
        modulo homotopy. For $n\geq{2}$,
        $\pi_{n}(X,x_{0})$ is Abelian. There's also
        $\pi_{0}(X,x_{0})$, which is not a group in general,
        but counts the number of path connected components.
        Moving on to the base point, this is mostly annoying.
        It seems that most of the time it doesn't matter which
        one we pick, provided that the space has one connected
        component. Recall that the fundamental group can be
        seen as a functor that maps pointed topological
        spaces to a group such that continuous maps
        $\varphi:(X,x_{0})\rightarrow(Y,y_{0})$ are mapped
        to group homomorphisms
        $\varphi_{*}:\pi_{1}(X,x_{0})\rightarrow\pi_{1}(Y,y_{0})$.
        Given a loop $f:S^{1}\rightarrow{X}$ and a continuous
        function $\varphi:X\rightarrow{Y}$, we take the
        composition $\varphi\circ{f}$ and map the equivalence
        class of $f$ to the equivalence class of
        $\varphi\circ{f}$. That is:
        \begin{equation}
            \varphi_{*}([f])=[\varphi\circ{f}]
        \end{equation}
        Then $\varphi$ is a group homomorphism between the
        two fundamental groups. Functors have the requirement
        that:
        \begin{align}
            (X,x_{0})\overset{f}{\longrightarrow}
            (Y,y_{0})\overset{g}{\longrightarrow}
            (Z,z_{0})\\
            \pi_{1}(X,x_{0})
                \underset{\varphi_{*}}{\longrightarrow}
            \pi_{1}(Y,y_{0})
                \underset{\psi_{*}}{\longrightarrow}
            \pi_{1}(Z,z_{0})
        \end{align}
        With the constraint that:
        \begin{equation}
            (\psi\varphi)_{*}=\psi_{1}\circ\varphi_{*}
        \end{equation}
        Given a homotopy of loops:
        \begin{equation}
            \varphi_{t}:(X,x_{0})\rightarrow(Y,y_{0})
        \end{equation}
        such that, for all $t\in{I}$, $\varphi_{t}(x_{0})=y_{0}$.
        Then:
        \begin{equation}
            (\varphi_{1})_{*}=(\varphi_{2})_{*}
        \end{equation}
        \begin{theorem}
            If $X$ is path connected, and if $x_{0},x_{1}\in{X}$,
            then:
            \begin{equation}
                \pi_{1}(X,x_{0})\simeq\pi_{1}(X,x_{1})
            \end{equation}
        \end{theorem}
        \begin{proof}
            For let $h:I\rightarrow{X}$ be such that
            $h(0)=x_{0}$ and $h_{1}=x_{1}$. This is possible
            since $X$ is path connected. Define:
            \begin{equation}
                \beta_{h}:\pi_{1}(X,x_{0})\rightarrow
                \pi_{1}(X,x_{1})\quad\quad
                \beta_{h}([f])=[h^{\minus{1}}\circ{f}\circ{h}]
            \end{equation}
            Then this is well defined and is a group homomorphism.
            Moreover it is an isomorphism since there's an
            inverse. Therefore, etc.
        \end{proof}
        \begin{theorem}
            If $\varphi_{t}:X\rightarrow{Y}$ is a homotopy,
            if $y_{0}=\varphi_{0}(x_{0})$, and if
            $y_{0}=\varphi_{1}(x_{0})$, then
            $\pi_{1}(Y,y_{0})$ is isomorphic to
            $\pi_{1}(Y,y_{1})$.
        \end{theorem}
        \begin{theorem}
            If $\varphi:X\rightarrow{Y}$ is a homotopy
            equivalence and if $\varphi(x_{0})=y_{0}$, then
            $\pi_{1}(X,x_{0})$ is isomorphic to
            $\pi_{1}(Y,y_{0})$.
        \end{theorem}
        \begin{proof}
            For if $\varphi:X\rightarrow{Y}$ is a homotopy
            equivalence then there is an
            $\psi:Y\rightarrow{X}$ such that
            $\varphi\psi\simeq id_{Y}$ and
            $\psi\varphi\simeq id_{X}$. Then
            $\psi_{*}\varphi_{*}=(\psi\varphi){*}$.
        \end{proof}
        That is, homotopy equivalent spaces have the same
        fundamental groups.

            \par\hfill\par
    % Print bibliographies from all texts.
    \clearpage
    \bibliographystyle{alpha}
    \bibliography{../../biblio.bib}

    % Print the index.
    \clearpage
    \printindex
\end{document}