%------------------------------------------------------------------------------%
\documentclass[oneside]{book}                                                  %
%------------------------------Preamble----------------------------------------%
\makeatletter                                                                  %
    \def\input@path{{../../}}                                                  %
\makeatother                                                                   %
%---------------------------Packages----------------------------%
\usepackage{geometry}
\geometry{b5paper, margin=1.0in}
\usepackage[T1]{fontenc}
\usepackage{graphicx, float}            % Graphics/Images.
\usepackage{natbib}                     % For bibliographies.
\bibliographystyle{agsm}                % Bibliography style.
\usepackage[french, english]{babel}     % Language typesetting.
\usepackage[dvipsnames]{xcolor}         % Color names.
\usepackage{listings}                   % Verbatim-Like Tools.
\usepackage{mathtools, esint, mathrsfs} % amsmath and integrals.
\usepackage{amsthm, amsfonts, amssymb}  % Fonts and theorems.
\usepackage{tcolorbox}                  % Frames around theorems.
\usepackage{upgreek}                    % Non-Italic Greek.
\usepackage{fmtcount, etoolbox}         % For the \book{} command.
\usepackage[newparttoc]{titlesec}       % Formatting chapter, etc.
\usepackage{titletoc}                   % Allows \book in toc.
\usepackage[nottoc]{tocbibind}          % Bibliography in toc.
\usepackage[titles]{tocloft}            % ToC formatting.
\usepackage{pgfplots, tikz}             % Drawing/graphing tools.
\usepackage{imakeidx}                   % Used for index.
\usetikzlibrary{
    calc,                   % Calculating right angles and more.
    angles,                 % Drawing angles within triangles.
    arrows.meta,            % Latex and Stealth arrows.
    quotes,                 % Adding labels to angles.
    positioning,            % Relative positioning of nodes.
    decorations.markings,   % Adding arrows in the middle of a line.
    patterns,
    arrows
}                                       % Libraries for tikz.
\pgfplotsset{compat=1.9}                % Version of pgfplots.
\usepackage[font=scriptsize,
            labelformat=simple,
            labelsep=colon]{subcaption} % Subfigure captions.
\usepackage[font={scriptsize},
            hypcap=true,
            labelsep=colon]{caption}    % Figure captions.
\usepackage[pdftex,
            pdfauthor={Ryan Maguire},
            pdftitle={Mathematics and Physics},
            pdfsubject={Mathematics, Physics, Science},
            pdfkeywords={Mathematics, Physics, Computer Science, Biology},
            pdfproducer={LaTeX},
            pdfcreator={pdflatex}]{hyperref}
\hypersetup{
    colorlinks=true,
    linkcolor=blue,
    filecolor=magenta,
    urlcolor=Cerulean,
    citecolor=SkyBlue
}                           % Colors for hyperref.
\usepackage[toc,acronym,nogroupskip,nopostdot]{glossaries}
\usepackage{glossary-mcols}
%------------------------Theorem Styles-------------------------%
\theoremstyle{plain}
\newtheorem{theorem}{Theorem}[section]

% Define theorem style for default spacing and normal font.
\newtheoremstyle{normal}
    {\topsep}               % Amount of space above the theorem.
    {\topsep}               % Amount of space below the theorem.
    {}                      % Font used for body of theorem.
    {}                      % Measure of space to indent.
    {\bfseries}             % Font of the header of the theorem.
    {}                      % Punctuation between head and body.
    {.5em}                  % Space after theorem head.
    {}

% Italic header environment.
\newtheoremstyle{thmit}{\topsep}{\topsep}{}{}{\itshape}{}{0.5em}{}

% Define environments with italic headers.
\theoremstyle{thmit}
\newtheorem*{solution}{Solution}

% Define default environments.
\theoremstyle{normal}
\newtheorem{example}{Example}[section]
\newtheorem{definition}{Definition}[section]
\newtheorem{problem}{Problem}[section]

% Define framed environment.
\tcbuselibrary{most}
\newtcbtheorem[use counter*=theorem]{ftheorem}{Theorem}{%
    before=\par\vspace{2ex},
    boxsep=0.5\topsep,
    after=\par\vspace{2ex},
    colback=green!5,
    colframe=green!35!black,
    fonttitle=\bfseries\upshape%
}{thm}

\newtcbtheorem[auto counter, number within=section]{faxiom}{Axiom}{%
    before=\par\vspace{2ex},
    boxsep=0.5\topsep,
    after=\par\vspace{2ex},
    colback=Apricot!5,
    colframe=Apricot!35!black,
    fonttitle=\bfseries\upshape%
}{ax}

\newtcbtheorem[use counter*=definition]{fdefinition}{Definition}{%
    before=\par\vspace{2ex},
    boxsep=0.5\topsep,
    after=\par\vspace{2ex},
    colback=blue!5!white,
    colframe=blue!75!black,
    fonttitle=\bfseries\upshape%
}{def}

\newtcbtheorem[use counter*=example]{fexample}{Example}{%
    before=\par\vspace{2ex},
    boxsep=0.5\topsep,
    after=\par\vspace{2ex},
    colback=red!5!white,
    colframe=red!75!black,
    fonttitle=\bfseries\upshape%
}{ex}

\newtcbtheorem[auto counter, number within=section]{fnotation}{Notation}{%
    before=\par\vspace{2ex},
    boxsep=0.5\topsep,
    after=\par\vspace{2ex},
    colback=SeaGreen!5!white,
    colframe=SeaGreen!75!black,
    fonttitle=\bfseries\upshape%
}{not}

\newtcbtheorem[use counter*=remark]{fremark}{Remark}{%
    fonttitle=\bfseries\upshape,
    colback=Goldenrod!5!white,
    colframe=Goldenrod!75!black}{ex}

\newenvironment{bproof}{\textit{Proof.}}{\hfill$\square$}
\tcolorboxenvironment{bproof}{%
    blanker,
    breakable,
    left=3mm,
    before skip=5pt,
    after skip=10pt,
    borderline west={0.6mm}{0pt}{green!80!black}
}

\AtEndEnvironment{lexample}{$\hfill\textcolor{red}{\blacksquare}$}
\newtcbtheorem[use counter*=example]{lexample}{Example}{%
    empty,
    title={Example~\theexample},
    boxed title style={%
        empty,
        size=minimal,
        toprule=2pt,
        top=0.5\topsep,
    },
    coltitle=red,
    fonttitle=\bfseries,
    parbox=false,
    boxsep=0pt,
    before=\par\vspace{2ex},
    left=0pt,
    right=0pt,
    top=3ex,
    bottom=1ex,
    before=\par\vspace{2ex},
    after=\par\vspace{2ex},
    breakable,
    pad at break*=0mm,
    vfill before first,
    overlay unbroken={%
        \draw[red, line width=2pt]
            ([yshift=-1.2ex]title.south-|frame.west) to
            ([yshift=-1.2ex]title.south-|frame.east);
        },
    overlay first={%
        \draw[red, line width=2pt]
            ([yshift=-1.2ex]title.south-|frame.west) to
            ([yshift=-1.2ex]title.south-|frame.east);
    },
}{ex}

\AtEndEnvironment{ldefinition}{$\hfill\textcolor{Blue}{\blacksquare}$}
\newtcbtheorem[use counter*=definition]{ldefinition}{Definition}{%
    empty,
    title={Definition~\thedefinition:~{#1}},
    boxed title style={%
        empty,
        size=minimal,
        toprule=2pt,
        top=0.5\topsep,
    },
    coltitle=Blue,
    fonttitle=\bfseries,
    parbox=false,
    boxsep=0pt,
    before=\par\vspace{2ex},
    left=0pt,
    right=0pt,
    top=3ex,
    bottom=0pt,
    before=\par\vspace{2ex},
    after=\par\vspace{1ex},
    breakable,
    pad at break*=0mm,
    vfill before first,
    overlay unbroken={%
        \draw[Blue, line width=2pt]
            ([yshift=-1.2ex]title.south-|frame.west) to
            ([yshift=-1.2ex]title.south-|frame.east);
        },
    overlay first={%
        \draw[Blue, line width=2pt]
            ([yshift=-1.2ex]title.south-|frame.west) to
            ([yshift=-1.2ex]title.south-|frame.east);
    },
}{def}

\AtEndEnvironment{ltheorem}{$\hfill\textcolor{Green}{\blacksquare}$}
\newtcbtheorem[use counter*=theorem]{ltheorem}{Theorem}{%
    empty,
    title={Theorem~\thetheorem:~{#1}},
    boxed title style={%
        empty,
        size=minimal,
        toprule=2pt,
        top=0.5\topsep,
    },
    coltitle=Green,
    fonttitle=\bfseries,
    parbox=false,
    boxsep=0pt,
    before=\par\vspace{2ex},
    left=0pt,
    right=0pt,
    top=3ex,
    bottom=-1.5ex,
    breakable,
    pad at break*=0mm,
    vfill before first,
    overlay unbroken={%
        \draw[Green, line width=2pt]
            ([yshift=-1.2ex]title.south-|frame.west) to
            ([yshift=-1.2ex]title.south-|frame.east);},
    overlay first={%
        \draw[Green, line width=2pt]
            ([yshift=-1.2ex]title.south-|frame.west) to
            ([yshift=-1.2ex]title.south-|frame.east);
    }
}{thm}

%--------------------Declared Math Operators--------------------%
\DeclareMathOperator{\adjoint}{adj}         % Adjoint.
\DeclareMathOperator{\Card}{Card}           % Cardinality.
\DeclareMathOperator{\curl}{curl}           % Curl.
\DeclareMathOperator{\diam}{diam}           % Diameter.
\DeclareMathOperator{\dist}{dist}           % Distance.
\DeclareMathOperator{\Div}{div}             % Divergence.
\DeclareMathOperator{\Erf}{Erf}             % Error Function.
\DeclareMathOperator{\Erfc}{Erfc}           % Complementary Error Function.
\DeclareMathOperator{\Ext}{Ext}             % Exterior.
\DeclareMathOperator{\GCD}{GCD}             % Greatest common denominator.
\DeclareMathOperator{\grad}{grad}           % Gradient
\DeclareMathOperator{\Ima}{Im}              % Image.
\DeclareMathOperator{\Int}{Int}             % Interior.
\DeclareMathOperator{\LC}{LC}               % Leading coefficient.
\DeclareMathOperator{\LCM}{LCM}             % Least common multiple.
\DeclareMathOperator{\LM}{LM}               % Leading monomial.
\DeclareMathOperator{\LT}{LT}               % Leading term.
\DeclareMathOperator{\Mod}{mod}             % Modulus.
\DeclareMathOperator{\Mon}{Mon}             % Monomial.
\DeclareMathOperator{\multideg}{mutlideg}   % Multi-Degree (Graphs).
\DeclareMathOperator{\nul}{nul}             % Null space of operator.
\DeclareMathOperator{\Ord}{Ord}             % Ordinal of ordered set.
\DeclareMathOperator{\Prin}{Prin}           % Principal value.
\DeclareMathOperator{\proj}{proj}           % Projection.
\DeclareMathOperator{\Refl}{Refl}           % Reflection operator.
\DeclareMathOperator{\rk}{rk}               % Rank of operator.
\DeclareMathOperator{\sgn}{sgn}             % Sign of a number.
\DeclareMathOperator{\sinc}{sinc}           % Sinc function.
\DeclareMathOperator{\Span}{Span}           % Span of a set.
\DeclareMathOperator{\Spec}{Spec}           % Spectrum.
\DeclareMathOperator{\supp}{supp}           % Support
\DeclareMathOperator{\Tr}{Tr}               % Trace of matrix.
%--------------------Declared Math Symbols--------------------%
\DeclareMathSymbol{\minus}{\mathbin}{AMSa}{"39} % Unary minus sign.
%------------------------New Commands---------------------------%
\DeclarePairedDelimiter\norm{\lVert}{\rVert}
\DeclarePairedDelimiter\ceil{\lceil}{\rceil}
\DeclarePairedDelimiter\floor{\lfloor}{\rfloor}
\newcommand*\diff{\mathop{}\!\mathrm{d}}
\newcommand*\Diff[1]{\mathop{}\!\mathrm{d^#1}}
\renewcommand*{\glstextformat}[1]{\textcolor{RoyalBlue}{#1}}
\renewcommand{\glsnamefont}[1]{\textbf{#1}}
\renewcommand\labelitemii{$\circ$}
\renewcommand\thesubfigure{%
    \arabic{chapter}.\arabic{figure}.\arabic{subfigure}}
\addto\captionsenglish{\renewcommand{\figurename}{Fig.}}
\numberwithin{equation}{section}

\renewcommand{\vector}[1]{\boldsymbol{\mathrm{#1}}}

\newcommand{\uvector}[1]{\boldsymbol{\hat{\mathrm{#1}}}}
\newcommand{\topspace}[2][]{(#2,\tau_{#1})}
\newcommand{\measurespace}[2][]{(#2,\varSigma_{#1},\mu_{#1})}
\newcommand{\measurablespace}[2][]{(#2,\varSigma_{#1})}
\newcommand{\manifold}[2][]{(#2,\tau_{#1},\mathcal{A}_{#1})}
\newcommand{\tanspace}[2]{T_{#1}{#2}}
\newcommand{\cotanspace}[2]{T_{#1}^{*}{#2}}
\newcommand{\Ckspace}[3][\mathbb{R}]{C^{#2}(#3,#1)}
\newcommand{\funcspace}[2][\mathbb{R}]{\mathcal{F}(#2,#1)}
\newcommand{\smoothvecf}[1]{\mathfrak{X}(#1)}
\newcommand{\smoothonef}[1]{\mathfrak{X}^{*}(#1)}
\newcommand{\bracket}[2]{[#1,#2]}

%------------------------Book Command---------------------------%
\makeatletter
\renewcommand\@pnumwidth{1cm}
\newcounter{book}
\renewcommand\thebook{\@Roman\c@book}
\newcommand\book{%
    \if@openright
        \cleardoublepage
    \else
        \clearpage
    \fi
    \thispagestyle{plain}%
    \if@twocolumn
        \onecolumn
        \@tempswatrue
    \else
        \@tempswafalse
    \fi
    \null\vfil
    \secdef\@book\@sbook
}
\def\@book[#1]#2{%
    \refstepcounter{book}
    \addcontentsline{toc}{book}{\bookname\ \thebook:\hspace{1em}#1}
    \markboth{}{}
    {\centering
     \interlinepenalty\@M
     \normalfont
     \huge\bfseries\bookname\nobreakspace\thebook
     \par
     \vskip 20\p@
     \Huge\bfseries#2\par}%
    \@endbook}
\def\@sbook#1{%
    {\centering
     \interlinepenalty \@M
     \normalfont
     \Huge\bfseries#1\par}%
    \@endbook}
\def\@endbook{
    \vfil\newpage
        \if@twoside
            \if@openright
                \null
                \thispagestyle{empty}%
                \newpage
            \fi
        \fi
        \if@tempswa
            \twocolumn
        \fi
}
\newcommand*\l@book[2]{%
    \ifnum\c@tocdepth >-3\relax
        \addpenalty{-\@highpenalty}%
        \addvspace{2.25em\@plus\p@}%
        \setlength\@tempdima{3em}%
        \begingroup
            \parindent\z@\rightskip\@pnumwidth
            \parfillskip -\@pnumwidth
            {
                \leavevmode
                \Large\bfseries#1\hfill\hb@xt@\@pnumwidth{\hss#2}
            }
            \par
            \nobreak
            \global\@nobreaktrue
            \everypar{\global\@nobreakfalse\everypar{}}%
        \endgroup
    \fi}
\newcommand\bookname{Book}
\renewcommand{\thebook}{\texorpdfstring{\Numberstring{book}}{book}}
\providecommand*{\toclevel@book}{-2}
\makeatother
\titleformat{\part}[display]
    {\Large\bfseries}
    {\partname\nobreakspace\thepart}
    {0mm}
    {\Huge\bfseries}
\titlecontents{part}[0pt]
    {\large\bfseries}
    {\partname\ \thecontentslabel: \quad}
    {}
    {\hfill\contentspage}
\titlecontents{chapter}[0pt]
    {\bfseries}
    {\chaptername\ \thecontentslabel:\quad}
    {}
    {\hfill\contentspage}
\newglossarystyle{longpara}{%
    \setglossarystyle{long}%
    \renewenvironment{theglossary}{%
        \begin{longtable}[l]{{p{0.25\hsize}p{0.65\hsize}}}
    }{\end{longtable}}%
    \renewcommand{\glossentry}[2]{%
        \glstarget{##1}{\glossentryname{##1}}%
        &\glossentrydesc{##1}{~##2.}
        \tabularnewline%
        \tabularnewline
    }%
}
\newglossary[not-glg]{notation}{not-gls}{not-glo}{Notation}
\newcommand*{\newnotation}[4][]{%
    \newglossaryentry{#2}{type=notation, name={\textbf{#3}, },
                          text={#4}, description={#4},#1}%
}
%--------------------------LENGTHS------------------------------%
% Spacings for the Table of Contents.
\addtolength{\cftsecnumwidth}{1ex}
\addtolength{\cftsubsecindent}{1ex}
\addtolength{\cftsubsecnumwidth}{1ex}
\addtolength{\cftfignumwidth}{1ex}
\addtolength{\cfttabnumwidth}{1ex}

% Indent and paragraph spacing.
\setlength{\parindent}{0em}
\setlength{\parskip}{0em}                                                           %
%----------------------------Main Document-------------------------------------%
\begin{document}
    \pagenumbering{roman}
    \title{MATH 114 Algebraic Topology Notes}
    \author{%
        Professor: Vladimir Chernov\\
        Notes by: Ryan Maguire%
    }
    \date{\vspace{-5ex}}
    \maketitle
    \tableofcontents
    \listoffigures
    \chapter{Homotopy}
    \pagenumbering{arabic}
        \section{Lecture 1: Review}
            \subsection{Topological Spaces}
                We start with the definition of topological spaces. We wish to
                generalize the notion of \textit{openness} that one comes across
                in the study of metric spaces\index{Metric Space}, or more
                concretely in the study of $\nspace$. To do this we axiomatize
                the properties of open sets: Taking arbitrary unions of open
                sets results in an open set, and the finite intersection of open
                sets is still open. Furthermore, we define the entire space and
                the empty set $\emptyset$ to be open. This gives us our
                definition of a topological space.
                \begin{fdefinition}{Topological Space}{Topological_Space}
                    A topological space is a set $X$ with a topology
                    $\tau\subseteq\powset{X}$, which is a colelction of subsets
                    of $X$ called the \textit{open} subsets, such that:
                    \index{Topological Space}\index{Open Subset}
                    \begin{enumerate}
                        \item \label{def:top:Empty_and_X_Open}%
                              $\emptyset\in\tau$ and $X\in\tau$
                        \item \label{def:top:Finite_Intersections}%
                              For all $\mathcal{U},\mathcal{V}\in\tau$ it is
                              true that $\mathcal{U}\cap\mathcal{V}\in\tau$
                        \item \label{def:top:Arbitrary_Unions}%
                              For any subset $\mathcal{O}\subseteq\tau$ it is
                              true that $\bigcup\mathcal{O}\in\tau$
                    \end{enumerate}
                \end{fdefinition}
                \begin{example}
                    The standard metric topology on $\nspace[]$ is induced by
                    declaring $\mathcal{U}\subseteq\nspace[]$ to be open if and
                    only if for all $x\in\mathcal{U}$ there is an
                    $\varepsilon>0$ such that
                    $(x-\varepsilon,x+\varepsilon)\subseteq\mathcal{U}$. So an
                    open interval of the form $(a,b)$ is open
                    (see Fig.~\ref{fig:Open_Subset_of_R}).
                \end{example}
                \begin{figure}[H]
                    \centering
                    \captionsetup{type=figure}
                    \begin{tikzpicture}[>=Latex]
    \coordinate (a)  at (2.0,  0.0);
    \coordinate (b)  at (9.0,  0.0);
    \coordinate (xt) at (4.0,  0.1);
    \coordinate (xb) at (4.0, -0.1);
    \coordinate (x)  at (4.0,  0.0);
    \coordinate (xl) at (3.0,  0.0);
    \coordinate (xr) at (5.0,  0.0);

    \draw[<->]   (0, 0) to (10, 0) node[above] {$\mathbb{R}$};
    \draw[thick] (a) to (b);
    \draw[thick] (b) arc (0:15:0.5);
    \draw[thick] (b) arc (0:-15:0.5);
    \draw[thick] (a) arc (180:195:0.5);
    \draw[thick] (a) arc (180:165:0.5);

    \draw (xt) to (xb);

    \node at (a)  [below=1ex] {$a$};
    \node at (b)  [below=1ex] {$b$};
    \node at (xl) [below=1ex] {$x-\varepsilon$};
    \node at (xr) [below=1ex] {$x+\varepsilon$};
    \node at (x)  [below=1ex] {$x$};

    \draw[blue, thick] (xr) arc (0:15:0.5);
    \draw[blue, thick] (xr) arc (0:-15:0.5);
    \draw[blue, thick] (xl) arc (180:195:0.5);
    \draw[blue, thick] (xl) arc (180:165:0.5);
    \draw[blue, thick] (xr) to (xl);
\end{tikzpicture}
                    \caption{An Open Subset of $\nspace[]$}
                    \label{fig:Open_Subset_of_R}
                \end{figure}
                By examining Fig.~\ref{fig:Open_Interval_Intersect_is_Open} we
                can convince ourselves that the intersection of open intervals
                is again open so long as we declare the empty set to be open.
                That is, $(0,1)$ and $(2,3)$ are open sets, but
                $(0,1)\cap(2,3)=\emptyset$. This highlights the need in
                requiring the empty set an element of the topology.
                \begin{figure}[H]
                    \centering
                    \captionsetup{type=figure}
                    \begin{tikzpicture}[>=Latex]
    \coordinate (a)  at (2.0,  0.0);
    \coordinate (b)  at (7.0,  0.0);
    \coordinate (c)  at (3.0,  0.0);
    \coordinate (d)  at (9.0,  0.0);
    \coordinate (x)  at (5.0,  0.0);
    \coordinate (xb) at (5.0, -0.1);
    \coordinate (xt) at (5.0,  0.1);
    \coordinate (xl) at (4.2,  0.0);
    \coordinate (xr) at (5.8,  0.0);

    \draw[<->]   (0, 0) to (10, 0) node[above] {$\mathbb{R}$};

    \node at (a)  [below=1ex] {$a$};
    \node at (b)  [below=1ex] {$b$};
    \node at (c)  [below=1ex] {$c$};
    \node at (d)  [below=1ex] {$d$};
    \node at (xl) [below=1ex] {$x-\varepsilon$};
    \node at (xr) [below=1ex] {$x+\varepsilon$};
    \node at (x)  [below=1ex] {$x$};
    \node at (a)  [above=1ex]
        {$\color{blue}{(a,b)}\cap\color{red}{(c,d)}=\color{Violet}{(c,b)}$};
    \node at (b) [above=1ex]
        {$\color{cyan}{(x-\varepsilon,x+\varepsilon)}%
            \subseteq\color{Violet}{(c,b)}$};

    \draw[blue,   thick]    (a)  to (c);
    \draw[red,    thick]    (b)  to (d);
    \draw[Violet, thick]    (c)  to (xl);
    \draw[Violet, thick]    (xr) to (b);
    \draw[cyan, very thick] (xl) to (xr);

    \draw[thin] (xt) to (xb);

    \draw[blue, very thick] (b) arc (0:15:0.5);
    \draw[blue, very thick] (b) arc (0:-15:0.5);
    \draw[blue, very thick] (a) arc (180:195:0.5);
    \draw[blue, very thick] (a) arc (180:165:0.5);

    \draw[red, very thick] (d) arc (0:15:0.5);
    \draw[red, very thick] (d) arc (0:-15:0.5);
    \draw[red, very thick] (c) arc (180:195:0.5);
    \draw[red, very thick] (c) arc (180:165:0.5);

    \draw[cyan, very thick] (xr) arc (0:15:0.5);
    \draw[cyan, very thick] (xr) arc (0:-15:0.5);
    \draw[cyan, very thick] (xl) arc (180:195:0.5);
    \draw[cyan, very thick] (xl) arc (180:165:0.5);
\end{tikzpicture}
                    \caption{The Intersection of Open Intervals is Open}
                    \label{fig:Open_Interval_Intersect_is_Open}
                \end{figure}
                \begin{example}
                    Given a metric space $\metspace{X}$, the metric topology is
                    defined by stating that $\mathcal{U}\subseteq{X}$ is open if
                    and only if for all $x\in\mathcal{U}$ there is an
                    $\varepsilon>0$ such that the $\varepsilon$ ball centered
                    about $x$ is contained in $\mathcal{U}$:
                    $\rball{\varepsilon}{\metspace{X}}{x}\subseteq\mathcal{U}$
                    (see Fig.~\ref{fig:Open_Subset_Metric_Space}).
                \end{example}
                \begin{figure}[H]
                    \centering
                    \captionsetup{type=figure}
                    \includegraphics{images/Open_Set_in_a_Metric_Space.pdf}
                    \caption{Open Subset of a Metric Space}
                    \label{fig:Open_Subset_Metric_Space}
                \end{figure}
                This metric topology on $\nspace[]$ is often called the
                \textit{standard} topology, but it is not the only one we can
                place on it.
                \begin{example}
                    The chaotic topology, also called the trivial topology or
                    the indiscrete topology, is the simplest topology one can
                    define on a set. We write:
                    \begin{equation}
                        \tau=\{\,\emptyset,\,X\,\}
                    \end{equation}
                    This is a topology since it trivially satisfies the three
                    properties enumerated in Def.~\ref{def:Topological_Space}.
                \end{example}
                \begin{example}
                    The largest topology one can define is the entire power set:
                    \begin{equation}
                        \tau=\powset{X}
                    \end{equation}
                    Again, rather trivially, this is a topology on $X$. For
                    those who have studied metric spaces, it is called the
                    trivial topology since it is the topology induced by the
                    discrete metric:
                    \begin{equation}
                        d(x,y)=
                        \begin{cases}
                            0,&x=y\\
                            1,&x\ne{y}
                        \end{cases}
                    \end{equation}
                \end{example}
                The closed interval $[a,b]$ is \textit{not} open in the standard
                topology (see Fig.~\ref{fig:Closed_Interval_Not_Open}), but it
                \textit{is} open with the discrete topology since everything is
                open in that topology. Because of this there is possible
                ambiguity with saying $\mathcal{U}$ is open if one has not
                specified the topology. Thankfully with most spaces one is
                interested in there is a standard or natural topology, and we
                usually choose this one without saying so. For metric spaces we
                almost always choose the metric topology.
                \begin{figure}[H]
                    \centering
                    \captionsetup{type=figure}
                    \begin{tikzpicture}[>=Latex]
    % Coordinates for various points.
    \coordinate (a)  at (4.0,  0.0);
    \coordinate (b)  at (9.0,  0.0);
    \coordinate (xl) at (3.0,  0.0);
    \coordinate (xr) at (5.0,  0.0);

    % Draw the real line.
    \draw[<->]   (0, 0) to (10, 0) node[above] {$\mathbb{R}$};

    % Draw the closed interval [a, b].
    \draw[thick] (a) to (b);

    % Add "brackets" indicating it is a closed interval.
    \draw[thick] (8.9, 0.1) to (9.0, 0.1) to (9.0, -0.1) to (8.9, -0.1);
    \draw[thick] (4.1, 0.1) to (4.0, 0.1) to (4.0, -0.1) to (4.1, -0.1);

    % Labels for the vaious points.
    \node at (a)  [below=1ex] {$a$};
    \node at (b)  [below=1ex] {$b$};
    \node at (xl) [below=1ex] {$a-\varepsilon$};
    \node at (xr) [below=1ex] {$a+\varepsilon$};

    % Draw the part of the open interval (a-e, a+e) that is inside of [a, b].
    \draw[blue, thick] (xr) arc (0:15:0.5);
    \draw[blue, thick] (xr) arc (0:-15:0.5);
    \draw[blue, thick] (a) to (xr);

    % Draw the part that falls outside.
    \draw[red, thick]  (xl) arc (180:195:0.5);
    \draw[red, thick]  (xl) arc (180:165:0.5);
    \draw[red, thick]  (a) to (xl);
\end{tikzpicture}
                    \caption{Closed Intervals are Not Open}
                    \label{fig:Closed_Interval_Not_Open}
                \end{figure}
                With the real line the intersection of finitely many open sets
                is still open, whereas we've only required the intersection of
                two open sets to still be open. We extend this to any finite
                collection by induction.
                \begin{theorem}
                    \label{thm:Finite_Intersections_Is_Open}%
                    If $\topspace{X}$ is a topological space and
                    $\mathcal{O}\subseteq\tau$ finite, then
                    $\bigcap\mathcal{O}\in\tau$.
                \end{theorem}
                \begin{proof}
                    Apply induction to Def.~\ref{def:Topological_Space}
                    part \ref{def:top:Finite_Intersections}.
                \end{proof}
                Now that we've presented some examples, we define what it means
                for a set to be closed. In analysis we defined a closed set to
                be a set that has all of its limit points. For topology this is
                not general enough (topological spaces where sequences suffice
                to define closedness are called \textit{sequential spaces}).
                There is a standard theorem one comes across that a subset of
                $\nspace[]$ is closed if and only if its complement is open. We
                take this theorem and adopt it as the definition of what it
                means to be a closed set in a general topological space.
                \begin{fdefinition}{Closed Subset}{Closed_Subset}
                    A closed subset of a topological space $\topspace{X}$ is a
                    subset $\mathcal{C}\subseteq{X}$ such that there exists an
                    open set $\mathcal{U}\in\tau$ with
                    $\mathcal{C}=X\setminus\mathcal{U}$.
                \end{fdefinition}
                That is, closed sets are the complements of open sets. There is
                a common misconception that closed sets are simply \textit{not}
                open sets, and vice-versa, but this is not so. In the discrete
                topology every set is open, and hence every set is closed. In
                the indiscrete topology there are no non-empty proper open
                subsets, and hence most sets are neither open nor closed. These
                examples show that openness and closedness are
                \textit{a priori} unrelated notions. If we know $A\subseteq{X}$
                is open and nothing more, we cannot conclude whether or not $A$
                is closed, and similarly if we know $B\subseteq{X}$ is closed
                and nothing more, then we cannot conclude whether or not $B$ is
                open. Using the idempotent laws of complement, we obtain the
                following:
                \begin{theorem}
                    \label{thm:Closed_Iff_Comp_is_Open}%
                    If $\topspace{X}$ is a topological space and
                    $C\subseteq{X}$, then $C$ is closed if and only if
                    $X\setminus{C}$ is open.
                \end{theorem}
                \begin{proof}
                    For if $X\setminus{C}$ is open, then
                    $X\setminus(X\setminus{C})$ is closed
                    (Def.~\ref{def:Closed_Subset}). From the idempotent law of
                    complements, $X\setminus(X\setminus{C})=C$ and hence $C$ is
                    closed. By a similar argument if $C$ is closed, then
                    $X\setminus{C}$ is open.
                \end{proof}
                Using De Morgan's laws we can define an equivalent notion of
                topological spaces using closed sets. De Morgan's laws state for
                sets $A,B,X$, with $A,B\subseteq{X}$, the following is true:
                \begin{subequations}
                    \begin{align}
                        X\setminus(A\cap{B})
                            &=(X\setminus{A})\cup(X\setminus{B})\\
                        X\setminus(A\cup{B})
                            &=(X\setminus{A})\cap(X\setminus{B})
                    \end{align}
                \end{subequations}
                We can write this more suggestively if we let
                $X\setminus{A}=A^{C}$ ($C$ for complement).
                \twocolumneq{(A\cap{B})^{C}=A^{C}\cup{B}^{C}}
                            {(A\cup{B})^{C}=A^{C}\cap{B}^{C}}
                De Morgan's laws hold for arbitrary unions and intersections:
                \twocolumneq{%
                    \Big(\bigcap\mathcal{U}\Big)^{C}=\bigcup\mathcal{U}^{C}%
                }{%
                    \Big(\bigcup\mathcal{U}\Big)^{C}=\bigcap\mathcal{U}^{C}%
                }
                With this we may prove the following.
                \begin{theorem}
                    If $\topspace{X}$ is a topological space, then $\emptyset$
                    and $X$ are closed, if $\mathcal{C},\mathcal{D}\subseteq{X}$
                    are closed, then $\mathcal{C}\cup\mathcal{D}$ is closed, and
                    if $\Lambda\subseteq\powset{X}$ is a collection of closed
                    subsets of $X$, then $\bigcap\Lambda$ is closed.
                \end{theorem}
                \begin{proof}
                    For the first part, apply the definition of closed subsets
                    (Def.~\ref{def:Closed_Subset}) to
                    Def.~\ref{def:Topological_Space} part
                    \ref{def:top:Empty_and_X_Open} and recall that
                    $X\setminus\emptyset=X$ and $X\setminus{X}=\emptyset$. For
                    the latter parts, combine De Morgan's Laws with
                    Def.~\ref{def:Topological_Space} parts
                    \ref{def:top:Finite_Intersections} and
                    \ref{def:top:Arbitrary_Unions}, respectively.
                \end{proof}
                We may expand the union of two closed sets to the unions of
                finitely many closed sets inductively as we did in
                Thm.~\ref{thm:Finite_Intersections_Is_Open}.
            \subsection{Continuity}
                In an analysis course one talks about continuous functions by
                one of two equivalent means: The $\varepsilon-\delta$ definition
                and by the limit of sequences ($f(a_{n})\rightarrow{f}(x)$ for
                any sequence $a_{n}\rightarrow{x}$). Both of these are
                pictorial, the $\epsilon-\delta$ definition saying that if we
                move at most $\delta$ in the $x$ direction, then we will have
                no more than $\varepsilon$ amount of error in the $y$ direction
                (see Fig.~\ref{fig:Eps_Delta_Def_Cont}).
                \begin{figure}[H]
                    \centering
                    \captionsetup{type=figure}
                    \includegraphics{images/Continuity_Epsilon_Delta_Def.pdf}
                    \caption{The $\varepsilon-\delta$ Definition of Continuity}
                    \label{fig:Eps_Delta_Def_Cont}
                \end{figure}
                The sequence definition says that as we get arbitrarily close to
                $x_{0}$ by any sequence $a:\mathbb{N}\rightarrow\nspace[]$, then
                $f(a_{n})$ approaches $f(x_{0})$
                (see Fig.~\ref{fig:Sequence_Def_Continuity}).
                \begin{figure}[H]
                    \centering
                    \captionsetup{type=figure}
                    \includegraphics{images/Continuity_Sequence_Definition.pdf}
                    \caption{Sequence Definition of Continuity}
                    \label{fig:Sequence_Def_Continuity}
                \end{figure}
                Both of these definitions require a metric which a general
                topological space may not have. We can define continuity more
                generally by means of open sets by recalling that a function
                $f:\nspace[]\rightarrow\nspace[]$ is continuous if and only if
                for every open subset $\mathcal{V}\subseteq\nspace[]$ it is true
                that the \textit{pre-image} under $f$ is also an open subset.
                That is, $f^{\minus{1}}[\mathcal{V}]\subseteq\nspace[]$ is open.
                Since this is an if and only if we may adopt it as our
                definition of continuity, and this only uses the notion of an
                open set which topological spaces do have.
                \begin{fdefinition}{Continuous Function}{Continuous_Function}
                    A continuous function from a topological space
                    $\topspace[X]{X}$ to a topological space $\topspace[Y]{Y}$
                    is a function $f:X\rightarrow{Y}$ such that for every open
                    subset $\mathcal{V}\in\tau_{Y}$ it is true that
                    $f^{\minus{1}}[\mathcal{V}]\in\tau_{X}$. The pre-image of
                    open sets are open.
                \end{fdefinition}
                \begin{example}
                    If $\topspace{X}$ is any topological space, if
                    $Y$ is a set, and if we choose the chaotic topology
                    $\{\emptyset,Y\}$ on $Y$, then any function
                    $f:X\rightarrow{Y}$ is automatically continuous. There are
                    only two open subsets to check and we have:
                    \twocolumneq{f^{\minus{1}}[\emptyset]=\emptyset}
                                {f^{\minus{1}}[Y]=X}
                    Both of which are open subsets, and hence $f$ is continuous.
                    This is one justification for calling this the chaotic
                    topology: Every function is continuous. There are other
                    justifications: every sequence converges to every point
                    simultaneously, no points can be separated, and so on.
                \end{example}
                \begin{example}
                    If $\topspace{Y}$ is a topological space, if $X$ is a set,
                    and if we choose the discrete topology $\powset{X}$ on $X$,
                    then any function $f:X\rightarrow{Y}$ is continuous. Given
                    any open subset $\mathcal{V}\in\tau$, the pre-image
                    $f^{\minus{1}}[\mathcal{V}]$ is a subset of $X$ by
                    definition and hence is an element of $\powset{X}$. That is,
                    $f$ is continuous.
                \end{example}
                An important concept in topology is that of a
                \textit{homeomorphism}. A homeomorphism is a continuous function
                $f:X\rightarrow{Y}$ that is bijective and such that the inverse
                function $f^{\minus{1}}:Y\rightarrow{X}$ is also continuous.
                If there exists a homeomorphism between topological spaces
                $\topspace[X]{X}$ and $\topspace[Y]{Y}$, then we call them
                \textit{homeomorphic}. Topologically, homeomorphic spaces are
                indistinguishable.
                \begin{example}
                    The open interval $(\minus\frac{\pi}{2},\frac{\pi}{2})$ is
                    homeomorphic to the entire real line via the tangent
                    function $\tan:(\minus\frac{\pi}{2},\frac{\pi}{2})%
                    \rightarrow\nspace[]$. This is continuous, bijective, and
                    has a continuous inverse $\arctan$. This example shows that
                    boundedness is a metric property and not a topological one.
                \end{example}
                One question that often arises is when can we conclude two
                spaces are homeomorphic. Many simple conditions that hold in
                $\nspace[]$ do not generalize. For example, any continuous
                bijection $f:\nspace[]\rightarrow\nspace[]$ is automatically a
                homeomorphism since the inverse will be continuous. This is a
                consequence of the intermediate value theorem which implies any
                such function is then strictly monotonic. This does not
                generalize to arbitrary topological spaces, it doesn't even
                generalize to subspaces of $\nspace$. We can continuously and
                bijectively map $[0,1)$ to the unit circle $\nsphere[1]$ by
                $f(x)=\big(\cos(2\pi{x}),\sin(2\pi{x})\big)$, but the inverse is
                not continuous. Rigorously it can't be since $\nsphere[1]$ is
                \textit{compact} but $[0,1)$ is not, and homeomorphisms preserve
                such a notion. Intuitively, we have tied up the ends of $[0,1)$
                continuously, but going from the unit circle to $[0,1)$ involves
                ripping the circle somewhere, which is not a continuous
                operation.
                \par\hfill\par
                Another false proposition is that if $f:X\rightarrow{Y}$ and
                $g:Y\rightarrow{X}$ are continuous bijections, then the two
                spaces are homeomorphic. This is false, and we construct a
                counterexample to demonstrate. Consider the following subspaces
                of $\nspace[2]$:
                \begin{figure}[H]
                    \centering
                    \captionsetup{type=figure}
                    \includegraphics{images/Non_Homeo_with_Cont_Bij}
                    \caption{Non-Homeomorphic Subspaces of $\nspace[2]$}
                    \label{fig:Non_Homeomorphic_Subspace}
                \end{figure}
                First, let's describe these spaces. Both spaces consist of the
                $x$ axis in the plane $\nspace[2]$. We then glue the half-open
                interval $[0,1)$ upwards in the $y$ direction starting at
                $(0,0)$ and then continuing doing this to the \textit{left}
                forever. Next, we glue circles starting at $(3,0)$ and continue
                to the \textit{right} forever. The only difference is in the
                middle. For the top space we attach a circle at $(1,0)$ and a
                half-open interval at $(2,0)$. For the bottom we simply attach
                two circles.
                \par\hfill\par
                These spaces are \textit{not} homeomorphic. To see this we need
                to know that a homeomorphism $f:X\rightarrow{Y}$ induces a
                homeomorphism between any subspace $A\subseteq{X}$ and its image
                $f[A]\subseteq{Y}$. Thus we need to find a subspace of the first
                image that does not exist in the second, or vice versa. If we
                remove the bottom point from the interval wedged between two
                circles, then we are left with three parts: an open interval,
                a string of infinitely many circles, and a string of infinitely
                many intervals plus one circle. No matter which space we remove
                from the second image the resulting subspace will not look like
                this, and hence these two spaces are not homeomorphic. However,
                there are continuous bijections $f:X\rightarrow{Y}$ and
                $g:Y\rightarrow{X}$. For this we will avoid formulas, and rather
                describe the function pictorially. For $f:X\rightarrow{Y}$ we
                take the interval wedged between the two circles and tie it up.
                This is identical to the function from $[0,1)$ to $\nsphere[1]$
                which was both continuous and bijective. The result of this
                procedure is continuous, bijective, and results in the second
                image. For $g:Y\rightarrow{X}$, we look at the interval that is
                two bits away from the neareast circle and tie this up. After,
                we translate the entire space over to the right by two, giving
                us the first space.
                \par\hfill\par
                Now one might ask if we needed an infinitely large space, or at
                the very least a \textit{non-compact} one, and the answer is
                yes. If $X$ is compact, and if $Y$ is \textit{Hausdorff}, then
                any continuous bijection $f:X\rightarrow{Y}$ is automatically a
                homeomorphism. To find counterexamples to the general claim we
                must then avoid such spaces.
            \subsection{More Notions to Review}
                It is hoped that the material previously discussed is all
                review. There are other notions needed that one should quickly
                read through if they have not seen them before, namely the
                notion of \textit{product spaces} and \textit{quotient spaces}.
                For product spaces we'll need to following theorem.
                \begin{ltheorem}{The Intersections of Topologies is a Topology}
                                {Intersection_of_Topologies_is_Topology}
                    If $X$ is a set, and if $T\subseteq\powset{\powset{X}}$ is a
                    non-empty collection of sets such that for all $\tau\in{T}$
                    it is true that $\tau$ is a topology on $X$, then
                    $\bigcap{T}$ is a topology on $X$.
                \end{ltheorem}
                \begin{proof}
                    Since every element of $T$ is a topology, $\emptyset$ and
                    $X$ are contained in all $\tau\in{T}$ and hence
                    $\emptyset,X\in\bigcap{T}$. If we have a collection
                    $\mathcal{O}$ of elements of $\bigcap{T}$, then each element
                    is in every topology $\tau\in{T}$ by the definition of
                    intersection. But then since all $\tau\in{T}$ are topologies
                    we know that $\bigcup\mathcal{O}\in\tau$. Since this is true
                    of all $\tau$ we conclude $\bigcup\mathcal{O}\in\bigcap{T}$.
                    By a similar argument, $\bigcap{T}$ is closed to finite
                    intersections. Hence, $\bigcap{T}$ is a topology on $X$.
                \end{proof}
                We can use this to define the product topology generated by the
                Cartesian product of two topological spaces. If we have two
                topological spaces $\topspace[X]{X}$ and $\topspace[Y]{Y}$ we
                would like to place a topology on $X\times{Y}$. Moreover we
                would like this topology to agree with product spaces we already
                know well, such as the standard topologies on $\nspace$ or the
                topologies induced by the product of metric spaces. We would
                like to make life simple and set the topology to be something
                like $\tau_{X}\times\tau_{Y}$, but this may not have the union
                property of topologies. Indeed, if we let $X=\nspace[]$ and
                $Y=\nspace[]$ then $\tau_{X}\times\tau_{Y}$ is the set of all
                $\mathcal{U}\times\mathcal{V}$ where $\mathcal{U}$ and
                $\mathcal{V}$ are open. Since open subsets of $\nspace[]$ can be
                written as the countable union of open intervals, we can suppose
                for the sake of visualization that $\mathcal{U}=(a,b)$ and
                $\mathcal{V}=(c,d)$ with $a,b,c,d\in\nspace[]$. The product
                $\mathcal{U}\times\mathcal{V}$ is an \textit{open rectangle}
                (see Fig.~\subref{fig:Open_Rectangle_in_R2}). A blob like the
                one in Fig.~\subref{fig:Open_Subset_Not_Product} is considered
                open in the standard topology on $\nspace[2]$ but cannot be
                written in the form $\mathcal{U}\times\mathcal{V}$. We use
                Thm.~\ref{thm:Intersection_of_Topologies_is_Topology} to define
                the product topology.
                \begin{figure}[H]
                    \centering
                    \captionsetup{type=figure}
                    \begin{subfigure}[b]{0.49\textwidth}
                        \centering
                        \begin{tikzpicture}[>=Latex]
    \draw[->, thick] (-0.4, 0) to (4, 0)
        node [above] {$x$};
    \draw[->, thick] (0, -0.4) to (0, 4)
        node [right] {$y$};
    \draw (1, -0.1) to (1, 0.1);
    \node at (1, -0.4) {$a$};
    \draw (3, -0.1) to (3, 0.1);
    \node at (3, -0.4) {$b$};
    \draw (-0.1, 1) to (0.1, 1);
    \node at (-0.4, 1) {$c$};
    \draw (-0.1, 3) to (0.1, 3);
    \node at (-0.4, 3) {$d$};
    \draw[fill=cyan, opacity=0.8, draw=white]
        (1, 1) to (1, 3) to (3, 3) to (3, 1) to cycle;
    \draw[densely dashed] (0, 1) to (3, 1);
    \draw[densely dashed] (0, 3) to (3, 3);
    \draw[densely dashed] (1, 0) to (1, 3);
    \draw[densely dashed] (3, 0) to (3, 3);
\end{tikzpicture}
                        \subcaption{The Open Rectangle $(a,b)\times(c,d)$.}
                        \label{fig:Open_Rectangle_in_R2}
                    \end{subfigure}
                    \begin{subfigure}[b]{0.49\textwidth}
                        \centering
                        \begin{tikzpicture}[>=Latex]
                            \draw[->, thick] (-0.4, 0) to (4, 0)
                                node [above] {$x$};
                            \draw[->, thick] (0, -0.4) to (0, 4)
                                node [right] {$y$};
                            \draw[fill=cyan, opacity=0.8, densely dashed]
                                (1, 1) to (2, 1) to (2, 2) to (3, 2)
                                       to (3, 3.5) to (1.5, 3.5)
                                       to (1.5, 2) to (1, 2) to cycle;
                        \end{tikzpicture}
                        \subcaption{A Region That Cannot be Written as
                                    $\mathcal{U}\times\mathcal{V}$.}
                        \label{fig:Open_Subset_Not_Product}
                    \end{subfigure}
                    \caption{Examples of Open Subsets of $\mathbb{R}^{2}$.}
                    \label{fig:Point_Set_Top_Open_Subsets_R2}
                \end{figure}
                Since the power set $\powset{X\times{Y}}$ is a topology on
                $X\times{Y}$, the set of all topologies that contain
                $\tau_{X}\times\tau_{Y}$ is non-empty. We thus define the
                product topology to be the \textit{smallest} topology that
                contains $\tau_{X}\times\tau_{Y}$. To make this precise we
                define $\tau_{X\times{Y}}$ to be $\bigcap\Lambda$ where
                $\Lambda$ is the set of all topologies containing
                $\tau_{X}\times\tau_{Y}$. By
                Thm.~\ref{thm:Intersection_of_Topologies_is_Topology} this is
                indeed a topology on $X\times{Y}$.
                \begin{figure}[H]
                    \centering
                    \captionsetup{type=figure}
                    \begin{tikzpicture}[>=Latex]
    \draw[<->, thick] (-3.3, 0) to (3.3, 0) node [above] {$x$};
    \draw[<->, thick] (0, -3.3) to (0, 3.3) node [right] {$y$};
    \draw[densely dashed] (0, 0) circle (1in);

    % First Layer
    \draw[fill=cyan, opacity=0.6, densely dashed]
        (0.7071in, 0.7071in) to (-0.7071in, 0.7071in)
                             to (-0.7071in, -0.7071in)
                             to (0.7071in, -0.7071in)
                             to cycle;
    
    % Second Layer
    \draw[fill=green, opacity=0.5, densely dashed]
        (0.68in, 0.3535in) to (0.935in, 0.3535in)
                           to (0.935in, -0.3535in)
                           to (0.68in, -0.3535in)
                           to cycle;
    \draw[fill=green, opacity=0.5, densely dashed]
        (-0.68in, 0.3535in) to (-0.935in, 0.3535in)
                            to (-0.935in, -0.3535in)
                            to (-0.68in, -0.3535in)
                            to cycle;
    \draw[fill=green, opacity=0.5, densely dashed]
        (0.3535in, 0.68in) to (0.3535in, 0.935in)
                           to (-0.3535in, 0.935in)
                           to (-0.3535in, 0.68in)
                           to cycle;
    \draw[fill=green, opacity=0.5, densely dashed]
        (0.3535in, -0.68in) to (0.3535in, -0.935in)
                            to (-0.3535in, -0.935in)
                            to (-0.3535in, -0.68in)
                            to cycle;

    % Third Layer.
    \draw[fill=orange, opacity=0.6, densely dashed]
        (0.68in, 0.3535in) to (0.8212in, 0.3535in)
                           to (0.8212in, 0.5705in)
                           to (0.68in, 0.5707in)
                           to cycle;
    \draw[fill=orange, opacity=0.6, densely dashed]
        (0.68in, -0.3535in) to (0.8212in, -0.3535in)
                            to (0.8212in, -0.5705in)
                            to (0.68in, -0.5707in)
                            to cycle;
    \draw[fill=orange, opacity=0.6, densely dashed]
        (-0.68in, -0.3535in) to (-0.8212in, -0.3535in)
                             to (-0.8212in, -0.5705in)
                             to (-0.68in, -0.5707in)
                             to cycle;
    \draw[fill=orange, opacity=0.6, densely dashed]
        (-0.68in, 0.3535in) to (-0.8212in, 0.3535in)
                            to (-0.8212in, 0.5705in)
                            to (-0.68in, 0.5707in)
                            to cycle;
    \draw[fill=orange, opacity=0.6, densely dashed]
        (0.3535in, 0.68in) to (0.3535in, 0.8212in)
                           to (0.5705in, 0.8212in)
                           to (0.5707in, 0.68in)
                           to cycle;
    \draw[fill=orange, opacity=0.6, densely dashed]
        (0.3535in, -0.68in) to (0.3535in, -0.8212in)
                            to (0.5705in, -0.8212in)
                            to (0.5707in, -0.68in)
                            to cycle;
    \draw[fill=orange, opacity=0.6, densely dashed]
        (-0.3535in, 0.68in) to (-0.3535in, 0.8212in)
                            to (-0.5705in, 0.8212in)
                            to (-0.5707in, 0.68in)
                            to cycle;
    \draw[fill=orange, opacity=0.6, densely dashed]
        (-0.3535in, -0.68in) to (-0.3535in, -0.8212in)
                             to (-0.5705in, -0.8212in)
                             to (-0.5707in, -0.68in)
                             to cycle;

    % Fourth Layer
    \draw[fill=red, opacity=0.5, densely dashed]
        (0.2in, 0.93in) to (0.2in, 0.9797in)
                        to (-0.2in, 0.9797in)
                        to (-0.2in, 0.93in)
                        to cycle;
    \draw[fill=red, opacity=0.5, densely dashed]
        (0.2in, -0.93in) to (0.2in, -0.9797in)
                         to (-0.2in, -0.9797in)
                         to (-0.2in, -0.93in)
                         to cycle;
    \draw[fill=red, opacity=0.5, densely dashed]
        (0.93in, 0.2in) to (0.9797in, 0.2in)
                        to (0.9797in, -0.2in)
                        to (0.93in, -0.2in)
                        to cycle;
    \draw[fill=red, opacity=0.5, densely dashed]
        (-0.93in, 0.2in) to (-0.9797in, 0.2in)
                         to (-0.9797in, -0.2in)
                         to (-0.93in, -0.2in)
                         to cycle;

    % Fifth Layer
    \draw[fill=blue, opacity=0.6, densely dashed]
        (0.82in, 0.3535in) to (0.8781in, 0.3535in)
                           to (0.8781in, 0.4784in)
                           to (0.82in, 0.4784in)
                           to cycle;
    \draw[fill=blue, opacity=0.6, densely dashed]
        (0.82in, -0.3535in) to (0.8781in, -0.3535in)
                            to (0.8781in, -0.4784in)
                            to (0.82in, -0.4784in)
                            to cycle;
    \draw[fill=blue, opacity=0.6, densely dashed]
        (-0.82in, 0.3535in) to (-0.8781in, 0.3535in)
                            to (-0.8781in, 0.4784in)
                            to (-0.82in, 0.4784in)
                            to cycle;
    \draw[fill=blue, opacity=0.6, densely dashed]
        (-0.82in, -0.3535in) to (-0.8781in, -0.3535in)
                             to (-0.8781in, -0.4784in)
                             to (-0.82in, -0.4784in)
                             to cycle;
    \draw[fill=blue, opacity=0.6, densely dashed]
        (0.3535in, 0.82in) to (0.3535in, 0.8781in)
                           to (0.4784in, 0.8781in)
                           to (0.4784in, 0.82in)
                           to cycle;
    \draw[fill=blue, opacity=0.6, densely dashed]
        (0.3535in, -0.82in) to (0.3535in, -0.8781in)
                            to (0.4784in, -0.8781in)
                            to (0.4784in, -0.82in)
                            to cycle;
    \draw[fill=blue, opacity=0.6, densely dashed]
        (-0.3535in, -0.82in) to (-0.3535in, -0.8781in)
                             to (-0.4784in, -0.8781in)
                             to (-0.4784in, -0.82in)
                             to cycle;
    \draw[fill=blue, opacity=0.6, densely dashed]
        (-0.3535in, 0.82in) to (-0.3535in, 0.8781in)
                            to (-0.4784in, 0.8781in)
                            to (-0.4784in, 0.82in)
                            to cycle;

    % Sixth Layer
    \draw[fill=yellow, opacity=0.6, densely dashed]
        (0.68in, 0.5705in) to (0.7641in, 0.5705in)
                           to (0.7641in, 0.645in)
                           to (0.68in, 0.645in)
                           to cycle;
    \draw[fill=yellow, opacity=0.6, densely dashed]
        (0.68in, -0.5705in) to (0.7641in, -0.5705in)
                            to (0.7641in, -0.645in)
                            to (0.68in, -0.645in)
                            to cycle;
    \draw[fill=yellow, opacity=0.6, densely dashed]
        (-0.68in, -0.5705in) to (-0.7641in, -0.5705in)
                             to (-0.7641in, -0.645in)
                             to (-0.68in, -0.645in)
                             to cycle;
    \draw[fill=yellow, opacity=0.6, densely dashed]
        (-0.68in, 0.5705in) to (-0.7641in, 0.5705in)
                            to (-0.7641in, 0.645in)
                            to (-0.68in, 0.645in)
                            to cycle;
    \draw[fill=yellow, opacity=0.6, densely dashed]
        (0.5705in, 0.68in) to (0.5705in, 0.7641in)
                           to (0.645in, 0.7641in)
                           to (0.645in, 0.68in)
                           to cycle;
    \draw[fill=yellow, opacity=0.6, densely dashed]
        (0.5705in, -0.68in) to (0.5705in, -0.7641in)
                            to (0.645in, -0.7641in)
                            to (0.645in, -0.68in)
                            to cycle;
    \draw[fill=yellow, opacity=0.6, densely dashed]
        (-0.5705in, -0.68in) to (-0.5705in, -0.7641in)
                             to (-0.645in, -0.7641in)
                             to (-0.645in, -0.68in)
                             to cycle;
    \draw[fill=yellow, opacity=0.6, densely dashed]
        (-0.5705in, 0.68in) to (-0.5705in, 0.7641in)
                            to (-0.645in, 0.7641in)
                            to (-0.645in, 0.68in)
                            to cycle;
\end{tikzpicture}
                    \caption{Tiling of the Open Disc by Rectangles}
                    \label{fig:Tiling_Open_Disc_by_Rectangles}
                \end{figure}
                Another way of describing this is by saying it is the topology
                \textit{generated} by all of the elements of
                $\tau_{X}\times\tau_{Y}$. That is, we take elements of
                $\tau_{X}\times\tau_{Y}$ and then add their unions and
                intersections until we have a valid topology. In this way we
                can see that the open unit disc is an open subset of the plane
                since we can tile it with rectangles
                (Fig.~\ref{fig:Tiling_Open_Disc_by_Rectangles}).
                As an example, the torus can be viewed as the Cartesian product
                of two circles equipped with the product topology
                (see Fig.~\ref{fig:Torus_as_Prod_Space}).
                \begin{figure}[H]
                    \centering
                    \captionsetup{type=figure}
                    \includegraphics{images/Torus_Skeleton_Product_Space.pdf}
                    \caption{The Torus $\ntorus[]=\nsphere[1]\times\nsphere[1]$}
                    \label{fig:Torus_as_Prod_Space}
                \end{figure}
                There is another standard way of viewing the torus which
                involves quotient spaces. For quotients one should think of
                gluing parts of a space together. Given a topological space
                $\topspace{X}$ and an equivalence relation $R$ on $X$, we look
                at the quotient set $X/R$. We topologize $X/R$ in the opposite
                manner as the product topology: we choose the \textit{largest}
                topology that makes the projection map $q:X\rightarrow{X}/R$
                continuous. There's always a topology that makes $q$ continuous
                since the chaotic topology $\{\emptyset,X/R\}$ does the trick,
                but this is too small and often boring. The quotient topology is
                defined as follows:
                \begin{equation}
                    \tau_{X/R}=\{\,\mathcal{V}\subseteq{X}/R\;|\;
                        q^{\minus{1}}[\mathcal{V}]\in\tau\,\}
                \end{equation}
                Again, the technical details are perhaps not too important for
                now, one should simply think of gluing together the parts
                identified by the equivalence relation $R$. Again we return to
                torus $\ntorus[]$, but now we define it as the quotient space
                obtained by identifying parts of the closed unit square
                together. Explicitly, we identify $(x,0)$ with $(x,1)$ for all
                $x\in[0,1]$ and similarly $(0,y)$ with $(1,y)$ for $y\in[0,1]$.
                The result of this gluing is depicted in
                Fig.~\ref{fig:Square_to_Torus}.
                \begin{figure}[H]
                    \centering
                    \captionsetup{type=figure}
                    \includegraphics{images/Square_to_Torus.pdf}
                    \caption{Gluing a Square into a Torus}
                    \label{fig:Square_to_Torus}
                \end{figure}
        \section{Lecture 2: Homotopy Equivalence}
            \subsection{Deformation Retraction}
                A deformation retraction from a topological space $\topspace{X}$
                onto a subset $A\subseteq{X}$ is a family of continuous
                functions $f_{t}:X\rightarrow{A}$ indexed by the closed unit
                interval $I=[0,1]$ such that $f_{0}=\identity{X}$,
                $f_{1}[X]=A$, and such that $f_{t}|_{A}=\identity{A}$ for all
                $t\in{I}$. Moreover, the function
                $H:X\times{I}\rightarrow{X}$ defined by $H(x,t)=f_{t}(x)$ should
                by continuous with respect to the product topology
                ($I$ inherits it's topology from $\nspace[]$).
                \begin{example}
                    We can take an annulus, or a punctured plane, either will
                    do, and construct a deformation retract from this onto the
                    unit circle $\nsphere[1]$. We do this by noting the function
                    $h$ defined on $\nspace[2]\setminus\{(0,0)\}$ by:
                    \begin{equation}
                        h(\vector{x})=\frac{\vector{x}}{\norm{\vector{x}}_{2}}
                    \end{equation}
                    where $\norm{\vector{x}}_{2}$ is the \textit{Euclidean Norm}
                    of the point $\vector{x}$, maps the punctured plane
                    $\nspace[2]\setminus\{(0,0)\}$ onto the unit circle. To
                    complete our deformation retraction we drag the point
                    $\vector{x}\ne\vector{0}$ along the straight line between
                    $\vector{x}$ and $h(\vector{x})$. We have:
                    \begin{equation}
                        H(\vector{x},t)
                        =(1-t)\cdot\vector{x}+t\cdot{h}(\vector{x})
                    \end{equation}
                    This is a deformation retract since elements of the unit
                    circle are held fixed for all $t\in[0,1]$. If
                    $\vector{s}\in\nsphere[1]$, then by definition
                    $\norm{\vector{s}}_{2}=1$ and hence
                    $h(\vector{s})=\vector{s}$. Simplifying, we have:
                    \begin{equation}
                        H(\vector{s},t)=(1-t)\cdot\vector{s}+t\cdot\vector{s}
                            =\vector{s}
                    \end{equation}
                    Thus $H$ gives us a deformation retraction of the punctured
                    plane onto the unit circle. We can do the same deformation
                    retraction with an annulus if we simply restrict $h$ to that
                    domain (see Fig.~\ref{fig:Def_Retract_Annulus_to_Circle}).
                \end{example}
                \begin{figure}[H]
                    \centering
                    \captionsetup{type=figure}
                    \includegraphics{images/Homotopy_Circle.pdf}
                    \caption{Deformation Retraction of an Annulus to a Circle}
                    \label{fig:Def_Retract_Annulus_to_Circle}
                \end{figure}
                For our second example we must first define what a M\"{o}bius
                strip is. Much the way the torus was defined by gluing together
                parts of the unit square, so is the M\"{o}bius strip but with a
                twist introduced. We take $[0,1]\times[0,1]$ and identify
                $(0,y)$ with $(1,1-y)$, leaving the $x$ coordinate alone.
                The procedure is outlined in
                Fig.~\ref{fig:Square_to_Mobius_Strip}.
                \begin{figure}[H]
                    \centering
                    \captionsetup{type=figure}
                    \includegraphics{images/Square_to_Mobius_Strip.pdf}
                    \caption{Gluing a Square into a M\"{o}bius Strip}
                    \label{fig:Square_to_Mobius_Strip}
                \end{figure}
                For those familiar with the language, the outcome is a manifold
                with boundary. There is an inner circle wrapping around the
                M\"{o}bius strip that we can retract the space onto in the same
                way that the annulus could be deformation retracted to a circle.
                It is perhaps best if we depict this with a 3D drawing. To do
                so, we embed the M\"{o}bius strip into $\nspace[3]$ with the
                following map:
                \begin{equation}
                    \varphi(x,y)=
                    \Big(\big(1+\frac{y}{2}\cos(\frac{x}{2})\big)\cos(x),\,
                         \big(1+\frac{y}{2}\cos(\frac{x}{2})\big)\sin(x),\,
                         \frac{y}{2}\sin(\frac{x}{2})\Big)
                \end{equation}
                The domain is $[0,2\pi)\times[\minus{1},1)$ The inner circle is
                obtained by setting $y=0$. Using this information, we can now
                draw our deformation retraction of a M\"{o}bius strip onto its
                inner circle (Fig.~\ref{fig:Def_Retract_Mobius_Strip}).
                \begin{figure}[H]
                    \centering
                    \captionsetup{type=figure}
                    \includegraphics{%
                        images/Mobius_Strip_Def_Retract_Inner_Circle%
                    }
                    \caption{Deformation Retraction of a M\"{o}bius Strip}
                    \label{fig:Def_Retract_Mobius_Strip}
                \end{figure}
                The subspace which we retract our space onto need not be unique,
                nor need they be homeomorphic. Indeed, any space can be
                deformation retracted onto itself by the function
                $H(x,t)=\identity{X}(x)$. The annulus is not homeomorphic to the
                circle (they have different dimensions), but they can both be
                deformation retracted from the same space. There are more subtle
                examples which we describe now.
                \par\hfill\par
                We continue by describing a plane with two holes in it,
                say at $(\minus{1},0)$ and $(1,0)$. There is a deformation
                retraction of this space onto a figure eight. For the definition
                of a figure eight, let's start with the lemniscate of Gerono,
                studied by the French mathematician Camille-Christophe Gerono in
                the $19^{th}$ century C.E. The defining implicit equation goes
                as follows:
                \begin{equation}
                    x^{4}-x^{2}+y^{2}=0
                \end{equation}
                Setting $x(\theta)=\cos(\theta)$, we have:
                \begin{equation}
                    y^{2}=\cos^{2}(\theta)\big(1-\cos^{2}(\theta)\big)
                         =\cos^{2}(\theta)\sin^{2}(\theta)
                \end{equation}
                so we parameterize the figure eight by:
                \begin{equation}
                    \big(x(\theta),\,y(\theta)\big)
                        =\big(\cos(\theta),\,\cos(\theta)\sin(\theta)\big)
                \end{equation}
                Now to retract the plane with two holes onto this object whilst
                leaving the lemniscate fixed. First, suppose we've retracted all
                far away points down to an oval, and points near the two holes
                we've pushed out so that the holes have been enlarged from
                points to circle. To finish the retraction we can use a bit of
                physics and differential equations
                (Fig.~\ref{fig:Deformation_Retraction_lemniscate_of_Gerono}).
                \begin{figure}
                    \centering
                    \captionsetup{type=figure}
                    \includegraphics{images/Homotopy_lemniscate_of_Gerono.pdf}
                    \caption{Deformation Retraction of the Lemniscate of Gerono}
                    \label{fig:Deformation_Retraction_lemniscate_of_Gerono}
                \end{figure}
                We put two identical electric charges at the centers of the blue
                circles. The \textit{electric field} exerted at a point
                $\vector{x}=(x_{0},x_{1})$ in the plane is given by Coulomb's
                law:
                \begin{equation}
                    F(\vector{x})=
                    \frac{\vector{x}-\vector{r}_{1}}
                         {\norm{\vector{x}-\vector{r}_{1}}_{2}^{3}}+
                    \frac{\vector{x}-\vector{r}_{2}}
                         {\norm{\vector{x}-\vector{r}_{2}}_{2}^{3}}
                \end{equation}
                where $\vector{r}_{1}$ and $\vector{r}_{2}$ are the coordinates
                of the holes, we've chosen $(\minus{1},0)$ and $(1,0)$. The
                notation $\norm{\vector{x}}_{2}^{3}$ denotes the 2-norm raised
                to the third power. If this were an actual physics problem we
                would need some scale factor $Q/4\pi\epsilon_{0}$, but for our
                purposes we simply need the directions of the
                \textit{field lines}. Given a point inside of the lemniscate, we
                drag this point continuously outward along field lines until we
                hit the figure eight, and for points on the outside we contract
                inwards. All the while we leave the lemniscate fixed. The
                outcome is a deformation retraction onto our figure eight. For
                the sake of computation, one could use something like Euler's
                method of solving differential equations to numerical
                approximate this procedure, and this is precisely what was done
                to draw the figure. For a more analytical approach with
                closed form solutions, we can consider Cassini ovals. If we let
                $a$ denote the distance from the first hole to the second, we
                can study the family of curves satisfying the following
                equation:
                \begin{equation}
                    \label{eqn:Cassini_Ovals}%
                    \big((x-a)^{2}+y^{2}\big)\big((x+a)^{2}+y^{2}\big)=b^{4}
                \end{equation}
                this asks for the set of all points $P$ such that the distance
                from $P$ to the first whole multiplied by the distance from $P$
                to the second hole is equal to $b^{2}$. That is:
                \begin{equation}
                    X_{b}=\{\,\vector{x}\in\nspace[2]\;|\;
                        \norm{\vector{x}-\vector{r}_{1}}_{2}\cdot
                        \norm{\vector{x}-\vector{r}_{2}}_{2}=b^{2}\,\}
                \end{equation}
                This problem was studied by the Italian astronomer Giovanni
                Domenico Cassini in 1680 C.E. and gives us a continuous means of
                retracting the plane with two holes onto a figure eight. The
                resulting figure eight is no longer the lemniscate of Gerono,
                but rather the lemniscate of Bernoulli, studied by Jakob
                Bernoulli in 1694 C.E. shortly after Cassini's investigations.
                Taking the gradient of Eqn.~\ref{eqn:Cassini_Ovals} gives us a
                vector field that once agains allows us to flow along field
                lines until we arrive at the figure eight
                (see Fig.~\ref{fig:Deformation_Retraction_Cassini_Ovals}). The
                gradient is computed to be:
                \begin{equation}
                    \grad{f}=\big(4x(x^{2}+y^{2}-a^{2}),4y(x^{2}+y^{2}+a^2)\big)
                \end{equation}
                \begin{figure}[H]
                    \centering
                    \captionsetup{type=figure}
                    \includegraphics{images/Homotopy_Cassini_Ovals_001.pdf}
                    \caption{Deformation Retraction Using Cassini Ovals}
                    \label{fig:Deformation_Retraction_Cassini_Ovals}
                \end{figure}
                The gradient allows us to show what the paths of individual
                points will look like, and allows us to draw
                Fig.~\ref{fig:Deformation_Retraction_Cassini_Ovals}, but is
                unnecessary. Cassini's equation Eqn.~\ref{eqn:Cassini_Ovals} is
                all we need. As $b$ tends to zero we get two disconnected ovals
                closing in on our two points. When $b$ is equal to the square
                root of the distance between these two points we obtain our
                lemniscate, and as $b$ grows we get a single connected object
                that looks more and more like a circle as $b$ gets large. This
                is precisely the description of a deformation retraction of the
                plane with two points missing onto the figure eight and the
                intermediate steps are shown in
                Fig.~\ref{fig:Homotopy_Cassini_Ovals}.
                \begin{figure}[H]
                    \centering
                    \captionsetup{type=figure}
                    \includegraphics{images/Homotopy_Cassini_Ovals_002.pdf}
                    \caption{Homotopy Using Cassini Ovals}
                    \label{fig:Homotopy_Cassini_Ovals}
                \end{figure}
                Now we can use our imagination to consider other possible
                deformation retractions from the plane with two holes onto
                smaller subspaces. For one, we can take our central lemniscate
                of Bernoulli that was produced in
                Fig.~\ref{fig:Deformation_Retraction_Cassini_Ovals} using
                Cassini ovals and stretch the crossing point outwards until we
                achieve two circles attached by a straight line
                (see Fig.~\ref{fig:Homotopy_Two_Circles_and_String}). While the
                retract using electric fields and the retract using Cassini
                ovals resulted in two homeomorphic objects (the lemniscate of
                Gerono and the lemniscate of Bernoulli are homeomorphic
                subspaces of $\nspace[2]$), this third object is \textit{not}
                homeomorphic to either of these.
                \begin{figure}[H]
                    \centering
                    \captionsetup{type=figure}
                    \includegraphics{images/Homotopy_Two_Circles_and_String.pdf}
                    \caption{Deformation Retraction onto Two Connnected Circles}
                    \label{fig:Homotopy_Two_Circles_and_String}
                \end{figure}
                We can see that these are not homeomorphic as follows. Suppose
                $f$ is a homeomorphism from the lemniscate of Bernoulli to two
                circles with a straight line. If we remove the center of the
                lemniscate we are left with two objects that are homeomorphic to
                the open unit interval $(0,1)$. However, no matter where the
                crossing point maps to under $f$, removing a single point from
                the latter object does not result in two homeomorphic copies of
                the unit interval, even though homeomorphism preserve this
                notion of subspace. So while these are both the result of
                deformation retractions of the same space, they are \textit{not}
                homeomorphic. Further still we can imagine stretching the
                crossing point of our lemniscate vertically rather than
                horizontally, obtaining Fig.~\ref{fig:Homotopy_Oval_with_Line}.
                \begin{figure}[H]
                    \centering
                    \captionsetup{type=figure}
                    \includegraphics{images/Homotopy_Oval_with_Lines.pdf}
                    \caption{Homotopy onto a Different Figure Eight}
                    \label{fig:Homotopy_Oval_with_Line}
                \end{figure}
                It is perhaps easiest to see that the retraction obtained in
                Fig.~\ref{fig:Homotopy_Oval_with_Line} is different then both
                Fig.~\ref{fig:Deformation_Retraction_lemniscate_of_Gerono} and
                Fig.~\ref{fig:Deformation_Retraction_Cassini_Ovals}. Removing
                any point from this final figure eight does not disconnect the
                space, however if one were to remove the central points from
                either the lemniscate of Bernoulli or the circles connected by a
                line, we disconnect these spaces into two separate parts. Since
                homeomorphisms preserve such notions, this final object is not
                homeomorphic to either of the other two. Hence we see that none
                of these three things are homeomorphic to each other, though
                they are \textit{homotopy equivalent}, a notion we will soon
                desribe.
                \par\hfill\par
                The previous examples involving the doubly punctured plane give
                rise to another notion, the \textit{mapping cylinder} of a
                function $f:X\rightarrow{Y}$. We take our domain $X$ and cross
                it with the closed unit interval $I=[0,1]$. If $X$ happens to be
                the circle $\nsphere[1]$ then $X\times{I}$ is precisely a
                cylinder. We take $X\times{I}$ with the product topology and
                consider the \textit{disjoint union} of this space with $Y$,
                denoted $(X\times{I})\coprod{Y}$. Set theoretically the
                disjoint union of $A$ and $B$ can be written as follows:
                \begin{equation}
                    \label{eqn:Def_Disjoint_Union}%
                    A\coprod{B}=(A\times\{0\})\cup(B\times\{1\})
                \end{equation}
                That is, we take a copy of $A$ and a copy of $B$ and form a
                new set that contains $A$ and $B$ as separated entities. For
                example, the disjoint union of $\nsphere[1]$ with $\nsphere[2]$
                can be thought of as a circle with a sphere placed far away. The
                disjoint union topology mimics this notion. An open set in
                $A\coprod{B}$ is just an open set in $A$ plus an open set in
                $B$. To get the mapping cylinder we consider the space
                $(X\times{I})\coprod{Y}$ and equip this with the equivalence
                relation $R$ identifying $(x,1)\in{X}\times{I}$ with
                $f(x)\in{Y}$. That is, we \textit{glue} the bottom of our
                cylinder onto the image of $X$ in $Y$ by $f$.
                \begin{fdefinition}{Mapping Cylinder}{Mapping_Cylinder}
                    The mapping cylinder of a continuous function $f$ from a
                    topological space $\topspace[X]{X}$ into a topological space
                    $\topspace[Y]{Y}$ is the quotient space induced on
                    $(X\times{I})\coprod{Y}$ by the equivalence relation $R$
                    obtained by identifying $\big((x,1),\,0\big)$ with
                    $\big(f(x),\,1\big)$ for all $x\in{X}$.
                \end{fdefinition}
                A short explaination on the notation in
                Def.~\ref{def:Mapping_Cylinder}. By
                Eqn.~\ref{eqn:Def_Disjoint_Union} an element of $A\coprod{B}$ is
                of the form $(z,i)$ where $i=0$ or $i=1$, and either $z\in{A}$
                or $z\in{B}$, depending on the value of $i$. Since we're taking
                the disjoint union of a product $(X\times{I})\coprod{Y}$ the
                elements are of the form $(z,i)$ where either
                $z=(x,t)\in{X}\times{I}$ or $z=y\in{Y}$. Technicalities aside,
                we present a visual example. As noted before if $X=\nsphere[1]$
                then $X\times{I}$ is an actual cylinder. Let's map the circle
                into the plane and see what the resulting mapping cylinder looks
                like. We'll take the function:
                \begin{equation}
                    \label{eqn:Trefoil_Mapping_Cylinder}%
                    f(\theta)=\Big(
                        \frac{1+\cos^{2}\big(\frac{3\theta}{2}\big)}{2},\,
                        \theta
                    \Big)
                \end{equation}
                using polar coordinates. The result is a trefoil in the plane.
                First, let us draw the space $(X\times{I})\coprod{Y}$. Since
                this is the disjoint union, we only need think of
                $\nsphere[1]\times{I}$ and $\nspace[2]$ as separate entities
                (see Fig.~\ref{fig:Ex_Mapping_Cylinder_Disj_Union}).
                \begin{figure}[H]
                    \centering
                    \captionsetup{type=figure}
                    \includegraphics{images/Mapping_Cylinder_Disj_Union.pdf}
                    \caption{The Disjoint Union
                             $(\nsphere[1]\times{I})\coprod\nspace[2]$}
                    \label{fig:Ex_Mapping_Cylinder_Disj_Union}
                \end{figure}
                We've drawn the image of $\nsphere[1]$ under the mapping $f$
                defined in Eqn.~\ref{eqn:Trefoil_Mapping_Cylinder} in blue to
                help transition to the next part of the process. In creating the
                mapping cylinder we now glue the \textit{bottom} of the cylinder
                $\nsphere[1]\times{I}$ (the bottom being a copy of
                $\nsphere[1]$, i.e. the \textit{boundary}) onto this blue piece
                using the function $f$ to guide us in the gluing. We can be very
                explicit if we imagine this in $\nspace[3]$. We draw the top
                circle in the $z=1$ plane and then draw a straight line between
                $\big(\cos(\theta),\sin(\theta),1\big)$ to the point $f(\theta)$
                which lies in the $xy$ plane. After drawing a straight line for
                every $\theta\in[0,2\pi)$ we arrive at our mapping cylinder.
                This is depicted in Fig.~\ref{fig:Ex_Mapping_Cylinder}. The red
                circle at the top represents $\nsphere[1]$, the blue closed
                curve in the plane is $f[\nsphere[1]]$ and the gray surface used
                used to form our mapping cylinder. The black mesh lines are
                drawn to emphasize the construction. It's important to note that
                as per the definition, the mapping cylinder includes all of the
                plane $\nspace[2]$ as well.
                \begin{figure}[H]
                    \centering
                    \captionsetup{type=figure}
                    \includegraphics{images/Mapping_Cylinder.pdf}
                    \caption{Example of a Mapping Cylinder}
                    \label{fig:Ex_Mapping_Cylinder}
                \end{figure}
                This straight-line construction generalizes. Given a mapping
                cylinder induced by $f:X\rightarrow{Y}$ we can shrink the
                cylindrical portion downwards giving us a deformation retraction
                onto the subspace $Y$.
                \par\hfill\par
                Not all deformation retractions come from mapping cylinders.
                Suppose we take a thick $\textbf{X}$ and shrink it down to a
                skinny X. We then shrink this X down to the central crossing
                point. Combining these two deformation retractions gives us a
                single deformation retraction $\textbf{X}$ to $\cdot$ but this
                cannot arise from a mapping cylinder since paths must merge
                together halfway through the deformation.
            \subsection{Homotopy}
                Deformation retractions and mapping cylinders motivate a far
                more general concept, though the real motivation stems from the
                desire to define a concept that is weaker than homeomorphism but
                easily computable and preserves many topological properties.
                Homeomorphism is an ideal definition of sameness, but many
                spaces that are not homeomorphic still share many properties. We
                wish to expand our study of topology by defining
                \textit{homotopy equivalence}. We will show that homeomorphisms
                are stronger (i.e. homeomorphic spaces are homotopy equivalent)
                and also demonstrate that the converse fails. First, we define
                \textit{homotopy}.
                \begin{fdefinition}{Homotopy}{Homotopy}
                    A homotopy from a continuous function $f:X\rightarrow{Y}$ to
                    a continuous function $g:X\rightarrow{Y}$ with respect to
                    topological spaces $\topspace[X]{X}$ and $\topspace[Y]{Y}$
                    is a continuous function $H:{X}\times{I}\rightarrow{Y}$ such
                    that $H(x,0)=f(x)$ and $H(x,1)=g(x)$, where $I=[0,1]$ is the
                    closed unit interval and $X\times{I}$ carries the product
                    topology $\tau_{X}\times\tau_{\nspace[]}|_{I}$.
                \end{fdefinition}
                To clarify this definition, $\tau_{\nspace[]}$ is the standard
                Euclidean topology on $\nspace[]$ and $\tau_{\nspace[]}|_{I}$
                is the subspace topology induced on the closed unit interval
                $I$. The picture goes as follows: Given two functions
                $f,g:X\rightarrow{Y}$ we drag $f$ along a continuous path in
                $X\times{I}$ until we obtain the function $g$ (see
                Fig.~\ref{fig:Homotopy_Diagram_for_Depicting_Homotopy}).
                Homotopy is a very weak notion since, as we will soon see, for
                \textit{every} pair of continuous functions
                $f,g:\nspace[m]\rightarrow\nspace$ there exists a homotopy
                $H:\nspace[m]\times{I}\rightarrow\nspace$ which drags $f$ to
                $g$.
                \begin{figure}[H]
                    \centering
                    \captionsetup{type=figure}
                    \includegraphics{images/Homotopy_Basic_Example.pdf}
                    \caption{Homotopy Between Two Functions}
                    \label{fig:Homotopy_Diagram_for_Depicting_Homotopy}
                \end{figure}
                \begin{theorem}
                    If $\topspace[X]{X}$ is a topological space, if
                    $\topvecspace[Y]{Y}$ is a topological vector space over the
                    real numbers, and if $f,g:X\rightarrow{Y}$ are continuous
                    functions, then there is a homotopy
                    $H:X\times{I}\rightarrow{Y}$ from $f$ to $g$.
                \end{theorem}
                \begin{proof}
                    For let $H$ be defined by:
                    \begin{equation}
                        \label{eqn:Straight_Line_Homotopy}%
                        H(x,t)=(1-t)\cdot{f}(x)\vecadd[Y]t\cdot{g}(x)
                    \end{equation}
                    Since $Y$ is a topological vector space, $H$ is well defined
                    for all $x\in{X}$ and $t\in[0,1]$. Moreover, $H(x,0)=f(x)$,
                    $H(x,1)=g(x)$ and $H$ is continuous. Hence, $H$ is a
                    homotopy taking $f$ to $g$.
                \end{proof}
                \begin{example}
                    \label{ex:Straight_Line_Homotopy_Euc_Spaces}%
                    The simplest homotopy involves convex subsets of Euclidean
                    spaces where we may again define the \textit{straight line}
                    homotopy. Given two continuous functions
                    $f,g:X\rightarrow\mathcal{U}$ from a topological space $X$
                    to a convex subset of $\nspace$ we again write
                    $H(x,t)=f(x)(1-t)+g(x)t$. This is well defined since by
                    hypothesis the space is convex and $H(x,t)$ represents a
                    straight line between $f(x)$ and $g(x)$ for each point
                    $x\in{X}$. Thus, $H$ is a homotopy from $f$ to $g$. It is
                    too demanding to ask one to only consider convex spaces, but
                    any topological space that is homeomorphic to a convex
                    subset of $\nspace$ can inherit this homotopy by using
                    function composition with the hypothesized homeomorphism. A
                    more general notion is that the space be
                    \textit{contractible} meaning it can be deformed
                    continuously to a single point.
                \end{example}
                With homotopy now defined, we should redundantly define
                \textit{homotopic} functions (functions with a homotopy between
                them). To avoid abuse of language we must first show that if there
                is a homotopy from $f$ to $g$, then there is a homotopy from $g$ to
                $f$. That is, we need not specify which function is the
                \textit{first} one.
                \begin{ltheorem}{Symmetry of Homotopy}
                                {Symmetry_of_Homotopy}
                    If $\topspace[X]{X}$ and $\topspace[Y]{Y}$ are topological
                    spaces, if $f,g:X\rightarrow{Y}$ are continuous, and if $H$
                    is a homotopy taking $f$ to $g$, then there exists a
                    homotopy taking $g$ to $f$.
                \end{ltheorem}
                \begin{proof}
                    For let $G:X\times{I}\rightarrow{Y}$ be defined by:
                    \begin{equation}
                        G(x,t)=H(x,1-t)
                    \end{equation}
                    for all $x\in{X}$ and $t\in[0,1]$. Then $G$ is the
                    composition of continuous functions and is therefore
                    continuous. But from the definition we have
                    $G(x,0)=H(x,1)=g(x)$ and $G(x,1)=H(x,0)=f(x)$. Hence $G$ is
                    a homotopy taking $g$ to $f$.
                \end{proof}
                We now define what it means for continuous functions $f$ and $g$
                from a topological space $\topspace[X]{X}$ to a space
                $\topspace[Y]{Y}$ to be \textit{homotopic}.
                \begin{fdefinition}{Homotopic Functions}{Homotopic_Functions}
                    Homotopic functions from a topological space
                    $\topspace[X]{X}$ to a topological space $\topspace[Y]{Y}$
                    are continuous functions $f,g:{X}\rightarrow{Y}$ such that
                    there is a homotopy $H$ between them.
                \end{fdefinition}
                Homotopic is an equivalence relation. We have shown that it is
                symmetric, next we show reflexivity.
                \begin{ltheorem}{Reflexivity of Homotopy}
                                {Reflexivity_of_Homotopy}
                    If $\topspace[X]{X}$ and $\topspace[Y]{Y}$ are topological
                    spaces, and if $f:X\rightarrow{Y}$ is a continuous function,
                    then $f$ is homotopic to $f$.
                \end{ltheorem}
                \begin{proof}
                    For let $H:X\times{I}\rightarrow{Y}$ be defined by
                    $H(x,t)=f(x)$. Then $H$ is continuous, $H(x,0)=f(x)$ and
                    $H(x,1)=f(x)$. Hence, $H$ is a homotopy between $f$ and $f$.
                \end{proof}
                We complete our claim that homotopic is an equivalence relation
                by proving it is transitive. We'll need the
                \textit{pasting lemma} from point-set topology.
                \begin{ltheorem}{Transitivity of Homotopy}
                                {Transitivity_of_Homotopy}
                    If $\topspace[X]{X}$ and $\topspace[Y]{Y}$ are topological
                    spaces, if $f,g,h:X\rightarrow{Y}$ are continuous, if $f$ is
                    homotopic to $g$, and if $g$ is homotopic to $h$, then $f$
                    is homotopic to $h$.
                \end{ltheorem}
                \begin{proof}
                    If $f$ is homotopic to $g$ and $g$ is homotopic to $h$, then
                    there exist continuous functions
                    $H_{1},H_{2}:X\times{I}\rightarrow{Y}$ such that $H_{1}$ is
                    a homotopy from $f$ to $g$ and $H_{2}$ is a homotopy from
                    $g$ to $h$. Let $H:X\times{I}\rightarrow{Y}$ be defined by:
                    \begin{equation}
                        H(x,t)=
                        \begin{cases}
                            H_{1}(x,2t),&{0}\leq{t}\leq\frac{1}{2}\\
                            H_{2}(x,2t-1),&\frac{1}{2}<{t}\leq{1}
                        \end{cases}
                    \end{equation}
                    By the pasting lemma, $H$ is continuous. But from the
                    definition of $H_{1}$ and $H_{2}$, $H(x,0)=f(x)$ and
                    $H(x,1)=h(x)$. Thus, $f$ is homotopic to $h$.
                \end{proof}
                Thms.~\ref{thm:Symmetry_of_Homotopy},
                \ref{thm:Reflexivity_of_Homotopy}, and
                \ref{thm:Transitivity_of_Homotopy} give precisely the definition
                of an equivalence relation.
                \begin{figure}
                    \centering
                    \captionsetup{type=figure}
                    \includegraphics{%
                        images/Homotopy_Straight_Line_Cubed_to_Exp_2D%
                    }
                    \caption{Paths Representing Homotopy}
                    \label{fig:Paths_Representing_Homotopy}
                \end{figure}
                \begin{lexample}{Straight Line Homotopy}{Straight_Line_Homotopy}
                    We saw in Ex.~\ref{ex:Straight_Line_Homotopy_Euc_Spaces}
                    that for any pair of continuous functions
                    $f,g:\nspace[m]\rightarrow\nspace$ there is a homotopy
                    between them, and hence all continuous functions between
                    Euclidean spaces are homotopic. For a more concrete example,
                    let's define two functions
                    $f,g:\nspace[]\rightarrow\nspace[2]$ by:
                    \twocolumneq{f(x)=\big(x,\,x^{3}\big)}
                                {g(x)=\big(x,\,\exp(x)\big)}
                    Both of these are continuous since they are continuous in
                    each coordinate. We can use the straight line homotopy
                    between them:
                    \begin{subequations}
                        \begin{align}
                            H(x,t)
                            &=(1-t)\big(x,\,x^{3}\big)+t\big(x,\,\exp(x)\big)\\
                            &=\big(x,\,(1-t)x^{3}+t\exp(x)\big)
                        \end{align}
                    \end{subequations}
                    The latter equation gives a parameterization of curves in
                    the plane that we can draw. The paths given by this are
                    shown in Fig.~\ref{fig:Paths_Representing_Homotopy}.
                    Note that $g(x)=constant$ is possible. Any continuous
                    function $f:\nspace[m]\rightarrow\nspace$ is homotopic to a
                    point.
                \end{lexample}
                We can further visualize homotopy by letting $X=I$ and
                $Y\subset\nspace[2]$ be a nice blob, like the one shown in
                Fig.~\ref{fig:straight_line_homotopy}. Then continuous functions
                $f,g:[0,1]\rightarrow{Y}$ are just curves within the blob. The
                homotopy defined in Eqn.~\ref{eqn:Straight_Line_Homotopy} is the
                map that drags $f(x)$ to $g(x)$ via the straight line connecting
                the two points.
                \begin{figure}[H]
                    \centering
                    \captionsetup{type=figure}
                    \documentclass[crop,class=article]{standalone}
%----------------------------Preamble-------------------------------%
\usepackage{tikz}                       % Drawing/graphing tools.
\usetikzlibrary{arrows.meta}            % Latex and Stealth arrows.
%--------------------------Main Document----------------------------%
\begin{document}
    \begin{tikzpicture}[%
        line width=1pt,
        line cap=round,
        >={Stealth[black]},
        every edge/.style={draw=black,very thick},
        smalldot/.style={
            circle,
            fill=black,
            inner sep=0pt,
            outer sep=0
        },
        dashcurve/.style={
            draw=black,
            dashed,
            <->
        }
    ]
        \begin{scope}[every node/.style=smalldot]
            % Set points for upper curve.
            \node at (2.5,1.3) (a0) {};
            \node at (2.7,1.4) (b0) {};
            \node at (2.9,1.7) (c0) {};
            \node at (3.2,1.8) (d0) {};
            \node at (3.4,1.9) (e0) {};

            % Set points for lower curve.
            \node at (2.8,0.6) (a1) {};
            \node at (3,0.6) (b1) {};
            \node at (3.3,0.5) (c1) {};
            \node at (3.5,0.9) (d1) {};
            \node at (3.7,1.2) (e1) {};

            % Points for the outer blob.
            \node at (3,0) (a2) {};
            \node at (2.5,0.4) (b2) {};
            \node at (2,1.2) (c2) {};
            \node at (2.6,2.1) (d2) {};
            \node at (4.2, 2.4) (e2) {};
            \node at (4.5, 2) (f2) {};
            \node at (4,1) (g2) {};
            \node at (3.7,0.4) (h2) {};
        \end{scope}

        % Nodes labelling the domain and co-domain.
        \node at (0,1) (i) {$[0,1]$};
        \node at (4,1.9) (i1) {$Y$};

        % Draw upper curve.
        \draw (a0) to [out=15,in=-135] (b0)
                   to [out=45,in=-130] (c0)
                   to [out=50,in=-170] (d0)
                   to [out=10,in=-135] (e0);

        % Draw lower curve.
        \draw (a1) to [out=0,in=170] (b1)
                   to [out=-10,in=170] (c1)
                   to [out=-10,in=-100] (d1)
                   to [out=80,in=-160] (e1);

        \begin{scope}[%
            every path/.style=dashcurve,
            every edge/.style=semithick
        ]
            % Draw dashed lines connecting curves.
            \path (a0) edge (a1);
            \path (b0) edge (b1);
            \path (c0) edge
                node[inner sep=0pt,outer sep=0pt,fill=white]
                {\scriptsize{\textit{H}}} (c1);
            \path (d0) edge (d1);
            \path (e0) edge (e1);
        \end{scope}

        % Draw curve defining the blob.
        \draw (a2) to [out=170,in=-45] (b2)
                   to [out=135,in=-85] (c2)
                   to [out=95,in=-150] (d2)
                   to [out=30,in=150] (e2)
                   to [out=-30,in=100] (f2)
                   to [out=-80,in=60] (g2)
                   to [out=-120,in=60] (h2)
                   to [out=-120,in=-10] cycle;


        % Draw curves representing maps f and g.
        \path[shorten >=0.2cm,shorten <=0.2cm,->]
            (i) edge[bend left=40]
            node[above] {$f$} (c0);
        \path[shorten >=0.2cm,shorten <=0.2cm,->]
            (i) edge[bend right=40]
            node[below] {$g$} (c1);
    \end{tikzpicture}
\end{document}
                    \caption{Straight-Line Homotopy}
                    \label{fig:straight_line_homotopy}
                \end{figure}
                We now use the notion of homotopy to precisely define
                deformation retractions. First, we define what a
                \textit{retract} is.
                \begin{fdefinition}{Retract}{Retract}
                    A retract of a topological space $\topspace{X}$ to a
                    subspace $A\subseteq{X}$ is a continuous function
                    $f:X\rightarrow{A}$ such that $f|_{A}=\identity{A}$.
                \end{fdefinition}
                If $f:X\rightarrow{A}$ is a retract, then composing twice simply
                yields $f$ again. That is, $f^{2}=f$. In a way retractions mimic
                projections that occur in linear algebra.
                \begin{example}
                    Let's look at the example of the annulus deformation
                    retracting down to a circle. The key mapping was the unit
                    vector function:
                    \begin{equation}
                        h(\vector{x})=\frac{\vector{x}}{\norm{\vector{x}}_{2}}
                    \end{equation}
                    This function is a \textit{retract} of the punctured plane
                    $\nspace[2]\setminus\{(0,0)\}$ to the circle $\nsphere[1]$.
                    The deformation retract was then defined by:
                    \begin{equation}
                        H(\vector{x},t)
                            =(1-t)\cdot\vector{x}+t\cdot{h}(\vector{x})
                    \end{equation}
                    But this is just the straight-line homotopy between the
                    retract $h$ and the identity function
                    $\identity{X}$, where $X=\nspace[2]\setminus\{(0,0)\}$ is
                    the punctured plane. We may write:
                    \begin{equation}
                        H(\vector{x},t)=(1-t)\cdot\identity{X}(\vector{x})+
                            t\cdot{h}(\vector{x})
                    \end{equation}
                \end{example}
                The above example spells out how we can define a deformation
                retraction.
                \begin{fdefinition}{Deformation Retraction}
                                   {Deformation_Retraction}
                    A deformation retraction of a topological space
                    $\topspace{X}$ onto a subspace $A\subseteq{X}$ is a
                    homotopy $H:X\times{I}\rightarrow{X}$ between the identity
                    function $\identity{X}$ and a retract $f:X\rightarrow{A}$
                    such that for all $t\in{I}$ and $a\in{A}$, $H(a,t)=a$.
                \end{fdefinition}
                This last condition is occasionally omitted which gives the
                definition of a \textit{weak} deformation retraction. The
                definition we've given is then called a \textit{strong}
                deformation retraction. There are bizarre spaces with weak
                deformation retractions that cannot be obtained by strong
                deformation retractions. One such example is the comb space.
                We take the closed unit interval $[0,1]$ sitting in $\nspace[2]$
                and attached $\{\frac{1}{n}\}\times[0,1]$ for all
                $n\in\mathbb{N}^{+}$. Lastly we attach $\{0\}\times[0,1]$. We
                give this the usual subspace topology of $\nspace[2]$ which
                makes it very nice from a point-set topology view (it's compact,
                path connected, etc.). It is shown in Fig.~\ref{fig:Comb_Space}.
                \begin{figure}[H]
                    \centering
                    \captionsetup{type=figure}
                    \includegraphics{images/Comb_Space.pdf}
                    \caption{The Comb Space}
                    \label{fig:Comb_Space}
                \end{figure}
                From the point of view of homotopy theory there are a plethora
                of points which this space deformation retracts down to. For
                example, the point in the bottom right corner $(1,0)$ can be
                deformation retracted onto by pushing down all of the teeth of
                the comb and then compressing the resulting line to the right.
                Similarly, just about every other point can be deformation
                retracted down to. One point that \textit{cannot} is the point
                in the top left corner $(0,1)$. Any deformation retract must
                be a continuous homotopy. If we look at the sequence of points
                $a_{n}=(\frac{1}{n},1)$ this converges to $(0,1)$. Hence
                $H(a_{n},t)$ must converge to $(0,1)$ for all $t$ since $H$ does
                not move the point $(0,1)$ for all $t\in[0,1]$. But this
                sequence must be pressed downwards to the $x$ axis in order to
                retract to this point, a contradiction.
                \par\hfill\par
                With this we now try to motivate homotopy equivalence.
                \begin{fdefinition}{Relative Homotopy}{Relative_Homotopy}
                    A homotopy relative to a subspace $A\subseteq{X}$ between
                    continuous functions $f:X\rightarrow{Y}$ with respect to
                    topological spaces $\topspace[X]{X}$ and $\topspace[Y]{Y}$
                    is a homotopy $H:X\times{I}\rightarrow{Y}$ between $f$ and
                    $g$ such that for all $t\in[0,1]$ and for all $a\in{A}$ it
                    is true that $H(a,t)=f(a)$ and $H(a,t)=g(a)$.
                \end{fdefinition}
                For a relative homotopy to exist between $f$ and $g$ it is
                necessary that $f|_{A}=g|_{A}$, otherwise this definition does
                not make sense. A deformation retraction can again be redefined
                to be a relative homotopy of a subspace $A\subseteq{X}$ with
                respect to $\identity{X}$ and a retract $f:X\rightarrow{A}$.
                \par\hfill\par
                Various examples have shown us that homotopic functions can look
                very different. Indeed, \textit{every} pair of continuous
                functions $f,g:\nspace[m]\rightarrow\nspace$ are homotopic. The
                definition of homeomorphism states that $f:X\rightarrow{Y}$ is
                a homeomorphism if and only if $f$ is a continuous bijection
                with a continuous inverse. We also saw that it was not enough to
                say that there exists a continuous bijection $f:X\rightarrow{Y}$
                and another continuous bijection $g:Y\rightarrow{X}$. However,
                it is equivalent to say that there are continuous bijections
                $f:X\rightarrow{Y}$ and $g:Y\rightarrow{X}$ such that
                $f\circ{g}=\identity{Y}$ and $g\circ{f}=\identity{X}$. This last
                condition merely stating that $g=f^{\minus{1}}$. A new form of
                sameness for topological spaces is given if we relax the
                \textit{equalities} $f\circ{g}=\identity{Y}$ and
                $g\circ{f}=\identity{X}$ and instead only require
                \textit{homotopic}. This condition has a name:
                \begin{fdefinition}{Homotopy Inverse}{Homotopy_Inverse}
                    A homotopy inverse of a continuous function
                    $f:X\rightarrow{Y}$ between topological spaces
                    $\topspace[X]{X}$ and $\topspace[Y]{Y}$ is a continuous
                    function $g:Y\rightarrow{X}$ such that $g\circ{f}$ is
                    homotopic to $\identity{X}$ and $f\circ{g}$ is homotopic to
                    $\identity{Y}$.
                \end{fdefinition}
                There's often a notion of uniqueness when it comes to inverses,
                and indeed a homotopy inverse of a function $f:X\rightarrow{Y}$
                is unique \textit{up to homotopy}.
                \begin{theorem}
                    If $\topspace[X]{X}$ and $\topspace[Y]{Y}$ are topological
                    spaces, if $f:X\rightarrow{Y}$ is continuous, and if
                    $g_{1},g_{2}:Y\rightarrow{X}$ are homotopy inverses
                    of $f$, then $g_{1}$ is homotopic to ${g}_{2}$.
                \end{theorem}
                \begin{proof}
                    Since $g_{2}$ is a homotopy inverse of $f$,
                    $f\circ{g}_{2}$ is homotopic to $\identity{Y}$. But homotopy
                    is reflexive and hence $g_{1}$ is homotopic to itself, and
                    $g_{1}=g_{1}\circ\identity{Y}$. But homotopy is transitive
                    so if $g_{1}$ is homotopic $g_{1}\circ\identity{Y}$ and
                    $g_{1}\circ\identity{Y}$ is homotopic to
                    $g_{1}\circ(f\circ{g}_{2})$, then $g_{1}$ is homotopic to
                    $g_{1}\circ(f\circ{g}_{2})$. But function composition is
                    associative, and so:
                    \begin{equation}
                        g_{1}\circ(f\circ{g}_{2})=(g_{1}\circ{f})\circ{g}_{2}
                    \end{equation}
                    But $g_{1}$ is a homotopy inverse of $f$, and thus
                    $g_{1}\circ{f}$ is homotopic to $\identity{X}$. Therefore
                    $(g_{1}\circ{f})\circ{g}_{2}$ is homotopic to ${g}_{2}$ and
                    since homotopic is a transitive relation, $g_{1}$ is
                    homotopic to ${g}_{2}$.
                \end{proof}
                Homotopy inverses make it easy to define homotopy equivalence.
                \begin{fdefinition}{Homotopy Equivalence}{Homotopy_Equivalence}
                    A homotopy equivalence from a topological space
                    $\topspace[X]{X}$ to a topological space $\topspace[Y]{Y}$
                    is a continuous function $f:X\rightarrow{Y}$ such that there
                    exists a homotopy inverse $g:Y\rightarrow{X}$ of $f$.
                \end{fdefinition}
                From the definition of homotopy equivalent spaces it is
                important to note that it is not required that
                $g\circ{f}=id_{X}$, but rather that $g\circ{f}$ is
                \textit{homotopic} to the identity map $\identity{X}$.
                Similarly, $f\circ{g}$ need only be homotopic to $\identity{Y}$,
                not necessarily equal.
                \par\hfill\par
                Let's look at an example of what type of operations are allowed
                by homotopy equivalences. We'll start with a torus $\ntorus[]$
                the has a disk inside of it
                (Fig.~\ref{fig:Torus_with_Disc_Inside}).
                \begin{figure}[H]
                    \centering
                    \captionsetup{type=figure}
                    \includegraphics{images/Torus_Wireframe_Gradient.pdf}
                    \caption{Torus with a Disc Inside}
                    \label{fig:Torus_with_Disc_Inside}
                \end{figure}
                The next step we'll do is shrink the entire disc down to a
                point, leaving the rest of the torus alone
                (Fig.~\ref{fig:Squished_Torus}). Such moves are not permitted by
                homeomorphisms since we are squeezing infinitely many elements
                together and identifying them with a single point, whereas
                homeomorphisms must be bijective per definition.
                \textit{Homotopy} does allow us to perform such an operation.
                \begin{figure}[H]
                    \centering
                    \captionsetup{type=figure}
                    \includegraphics{images/Squished_Torus.pdf}
                    \caption{A Squished Torus}
                    \label{fig:Squished_Torus}
                \end{figure}
                This can be made more concrete with equations since
                parametrizations of the torus are well known. We have:
                \begin{equation}
                    F(\theta,\varphi)=\Big(
                        \big(R+r\cos(\varphi)\big)\cos(\theta),\,
                        \big(R+r\cos(\varphi)\big)\sin(\theta),\,
                        r\sin(\varphi)\Big)
                \end{equation}
                Where $R$ and $r$ are the outer and inner radii, respectively.
                The squished torus can be obtain by shrinking the $x$ and $z$
                coordinates as $y$ varies. The code used to generate
                Fig.~\ref{fig:Squished_Torus} adopted the following
                parametrization (setting $g(\theta)=\sin(\theta/2)$):
                \begin{equation}
                        G(\theta,\varphi)=\Big(
                            \big(R+r\cos(\varphi)g(\theta)\big)\cos(\theta),\,
                            \big(R+r\cos(\varphi)\big)\sin(\theta),\,
                            r\sin(\varphi)g(\theta)\Big)
                \end{equation}
                \begin{figure}[H]
                    \centering
                    \captionsetup{type=figure}
                    \includegraphics{images/Sphere_with_String_at_Poles.pdf}
                    \caption{A Sphere with String Attached}
                    \label{fig:Sphere_with_String_Attached}
                \end{figure}
                This shrinks the torus to a point when $\theta=0$ and has no
                effect at $\theta=\pi$, varying continuously in between which is
                precisely what we want. Now, one can imagine taking our
                croissant and stretching out the collapsed point into a line,
                obtaining a half-moon with a line connecting the poles. We can
                then deform the crescent part of our object into a sphere,
                obtaining Fig.~\ref{fig:Sphere_with_String_Attached}. From here
                we can proceed and drag the endpoints of the string down towards
                the equator and obtain Fig.~\ref{fig:Kettle_Bell}.
                \begin{figure}[H]
                    \centering
                    \captionsetup{type=figure}
                    \includegraphics{images/Kettle_Bell.pdf}
                    \caption{A Kettle Bell}
                    \label{fig:Kettle_Bell}
                \end{figure}
                This example demonstrates how massively different spaces with
                homotopy equivalences can be. Just like homeomorphism gives rise
                to the notion of homeomorphic, homotopy equivalence redundantly
                allows us to define homotopy equivalent.
                \begin{fdefinition}{Homotopy Equivalent}{Homotopy_Equivalent}
                    Homotopy equivalent topological spaces are topological
                    spaces $\topspace[X]{X}$ and $\topspace[Y]{Y}$ such that
                    there exists a homotopy equivalence $f:X\rightarrow{Y}$.
                \end{fdefinition}
                As stated before, this is a weaker notion of sameness. It is
                important in algebraic topology since many properties are
                invariant under homotopy equivalence. It should be noted that it
                is a strictly weaker notion than homeomorphism. Homeomorphic
                implies homotopy equivalent, but not the other way around.
                Indeed, one of the great conjectures of topology,
                Poincar\'{e}'s Conjecture (now a theorem), asks when does
                homotopy equivalent imply homeomorphic?
                \begin{ltheorem}{Homeomorphic Implies Homotopy Equivalent}
                                {Homeomorphic_Implies_Homotopy_Equivalent}
                    If $\topspace[X]{X}$ and $\topspace[Y]{Y}$ are homeomorphic
                    topological spaces, then they are homotopy equivalent.
                \end{ltheorem}
                \begin{proof}
                    For if $X$ and $Y$ are homeomorphic, then there exists a
                    homeomorphism $f:X\rightarrow{Y}$. But then $f$ is a
                    continuous map from $X$ to $Y$, and $f^{-1}$ is a continuous
                    map from $Y$ to $X$. Moreover,
                    ${f}\circ{f^{-1}}=\identity{Y}$, and
                    ${f^{-1}}\circ{f}=\identity{X}$ since $f$ is a bijection.
                    But $\identity{X}$ is homotopic to $\identity{X}$, and
                    $\identity{Y}$ is homotopic to $\identity{Y}$. Thus,
                    $f^{\minus{1}}$ is a homotopy inverse of $f$ and $f$ is
                    therefore a homotopy equivalence.
                \end{proof}
                It does not take make to show that the converse fails.
                \begin{theorem}
                    \label{thm:homotopic_does_not_imply_homeomorphic}%
                    There exist topological spaces that are homotopy equivalent
                    but not homeomorphic.
                \end{theorem}
                \begin{proof}
                    Let $X=\nspace[2]$ and $Y=\{(0,0)\}$ be equipped with their
                    usual topologies. Let $f:X\rightarrow{Y}$ be defined by
                    $f(x,y)=(0,0)$ and $g=\identity{Y}$. Then
                    $g\circ{f}=\identity{Y}$. Let
                    $H\big((x,y),t\big)=(1-t)(x,y)$. Then $H$ is continuous,
                    $H\big((x,y),0\big)=(x,y)$, and $H\big((x,y),1\big)=(0,0)$.
                    Thus, $H$ is a homotopy between ${g}\circ{f}$ and
                    $\identity{X}$, and therefore ${g}\circ{f}$ is homotopic to
                    $\identity{X}$. But ${f}\circ{g}=\identity{Y}$ and
                    $\identity{Y}$ is homotopic to $\identity{Y}$. Thus $X$ and
                    $Y$ are homotopy equivalent. If $h:{X}\rightarrow{Y}$ is a
                    homeomorphism, then it is a bijection. But then
                    $\Card(X)=\Card(Y)$ where $\Card$ denotes the cardinality
                    of the space. But $\mathbb{R}^{2}$ is uncountable, and
                    $\Card(Y)=1$, a contradiction. Therefore $X$ and $Y$ are not
                    homeomorphic.
                \end{proof}
                \begin{figure}[H]
                    \captionsetup{type=figure}
                    \centering
                    \begin{tikzpicture}[%
    scale=0.9,
    line width=1pt,
    line cap=round,
    >={Stealth[black]},
    every edge/.style={%
        draw=black,
        very thick
    },
    grayarrow/.style={%
        >=stealth,
        fill=gray,
        draw=gray,
        line width=0.7mm,
        ->
    }
]
    \filldraw[%
        even odd rule,
        inner color=gray,
        outer color=white,
        draw=white
    ] (0,0) circle (2);
    \draw[thick, <->] (-1.5,0)--(1.5,0);
    \draw[thick, <->] (0,-1.5)--(0,1.5);
    \draw[grayarrow] (1,1)--(0.5,0.5);
    \draw[grayarrow] (-1,-1)--(-0.5,-0.5);
    \draw[grayarrow] (1,-1)--(0.5,-0.5);
    \draw[grayarrow] (-1,1)--(-0.5,0.5);
    \node[%
        fill=black,
        circle,
        thick,
        draw,
        inner sep=2pt,
        outer sep=3pt
    ]
        at (0,0) (O) {};
    \node at (1.4,0) [below] {$x$};
    \node at (0,1.4) [right] {$y$};
\end{tikzpicture}
                    \caption{Retraction of $\nspace[2]$ to $(0,0)$}
                    \label{fig:homotopy_equivalence_of_plane_with_point}
                \end{figure}
                Fig.~\ref{fig:homotopy_equivalence_of_plane_with_point}
                shows the mapping $f$ between $\mathbb{R}^{2}$ and $\{(0,0)\}$.
                Thm.~\ref{thm:homotopic_does_not_imply_homeomorphic} relies on
                the fact that $\mathbb{R}^{2}$ and $\{(0,0)\}$ are of different
                \textit{cardinality}. However, even if the topological spaces
                $X$ and $Y$ are homotopy equivalent, and are of the same
                cardinality, it is still possible that they are not
                homeomorphic. Recall that homeomorphisms preserve compactness.
                Homotopy equivalence need not.
                \begin{theorem}
                    \label{thm:HE_of_Punc_Plane_and_Circle_Not_Homeo}%
                    There exists topological spaces that have the same
                    cardinality, are homotopy equivalent, but not homeomorphic.
                \end{theorem}
                \begin{proof}
                    For let $X=\nspace[2]\setminus\{(0,0)\}$ and
                    $Y=\nsphere[1]$ have their usual topologies. Then
                    $\Card(X)=\Card(Y)=\Card(\mathbb{R})$, and thus both sets
                    are of the same cardinality. They are also homotopy
                    equivalent. For define $f:X\rightarrow{Y}$ and
                    $g:Y\rightarrow{X}$ by:
                    \twocolumneq{f(x,y)=\frac{(x,y)}{\norm{(x,y)}}}
                                {g(x,y)=(x,y)}
                    Define the function $H:X\times{I}\rightarrow{Y}$ by:
                    \begin{equation}
                        H\big((x,y),t\big)=(1-t)f(x,y)+tg(x,y)
                    \end{equation}
                    But then $H\big((x,y),0\big)=f(x,y)$, and
                    $H\big((x,y),1\big)=g(x,y)$. Thus $H$ is a homotopy between
                    ${g}\circ{f}$ and $\identity{X}$. But also
                    $({f}\circ{g})(x,y)=(x,y)$, for all $(x,y)\in S^{1}$. Thus
                    ${f}\circ{g}=\identity{Y}$, and $\identity{Y}$ is homotopic
                    to itself. Hence, $X$ and $Y$ are homotopy equivalent. But
                    $X$ is unbounded, and is therefore not compact, and $Y$ is
                    closed and bounded, and is thus compact by the Heine-Borel
                    theorem. But homeomorphisms preserve compactness. Therefore
                    $X$ and $Y$ are not homeomorphic.
                \end{proof}
                \begin{figure}[H]
                    \captionsetup{type=figure}
                    \centering
                    \documentclass[crop,class=article]{standalone}
%----------------------------Preamble-------------------------------%
\usepackage{amsfonts}                   % Blackboard Bold R.
\usepackage{tikz}                       % Drawing/graphing tools.
\usetikzlibrary{arrows.meta}            % Latex and Stealth arrows.
%--------------------------Main Document----------------------------%
\begin{document}
    \begin{tikzpicture}[%
        scale=0.9,
        line width=1pt,
        line cap=round,
        >={Stealth[black]},
        every edge/.style={%
            draw=black,
            very thick
        },
        grayarrow/.style={%
            >=stealth,
            fill=gray,
            draw=gray,
            line width=0.7mm,
            ->
        }
    ]
        \filldraw[%
            even odd rule,
            inner color=gray,
            outer color=white,
            draw=white
        ]
            (0,0) circle (2);

        % Draw axes.
        \draw[<->] (-1.5,0) -- (1.5,0);
        \draw[<->] (0,-1.5) -- (0,1.5);
        \draw[<->] (4,0) -- (7,0);
        \draw[<->] (5.5,-1.5) -- (5.5,1.5);

        % Draw gray arrows indicating homotopy equivalence.
        \begin{scope}[every edge/.style=grayarrow]
            \draw(1.1,1.1) edge (0.75,0.75);
            \draw(-1.1,-1.1) edge (-0.75,-0.75);
            \draw(1.1,-1.1) edge (0.75,-0.75);
            \draw(-1.1,1.1) edge (-0.75,0.75);
            \draw(0.2,0.2) edge (0.6,0.6);
            \draw(-0.2,-0.2) edge (-0.6,-0.6);
            \draw(0.2,-0.2) edge (0.6,-0.6);
            \draw(-0.2,0.2) edge (-0.6,0.6);
        \end{scope}

        \draw[dashed,draw=black,semithick] (0,0) circle (1);
        \node[%
            fill=white,
            circle,
            thick,
            draw,
            inner sep=2pt,
            outer sep=3pt
        ]
            at (0,0) (O) {};
        \node at (1.4,0) [below] {$x$};
        \node at (0,1.4) [right] {$y$};
        \node at (1.5,1.5) {$\mathbb{R}^{2}\setminus\{(0,0)\}$};
        \draw[>=Latex,draw=blue,->] (2.4,0) -- (3.6,0);
        \draw[draw=black,semithick] (5.5,0) circle (1);
        \node at (6.9,0) [below] {$x$};
        \node at (5.5,1.4) [right] {$y$};
        \node at (6.5,1.2) {$S^{1}$};
    \end{tikzpicture}
\end{document}
                    \caption{%
                        Homotopy Equivalence of the Punctured Plane and
                        $\nsphere[1]$%
                    }
                    \label{fig:HE_punc_plane_and_circle}
                \end{figure}
                Ahah, you might say, you're exploiting compactness. What if we
                add this condition? The answer is still no, as the following
                demonstrates.
                \begin{theorem}
                    There exists compact connected topological spaces of the
                    same cardinality that are homotopy equivalent but not
                    homeomorphic.
                \end{theorem}
                \begin{proof}
                    Let $X=[-1,1]$, $Y=[-1,1]^{2}$ and define
                    $f:X\rightarrow{Y}$ and $g:Y\rightarrow{X}$ by:
                    \twocolumneq{f(x)=(x,0)}{g(x,y)=x}
                    That is, $f$ is the inclusion mapping and $g$ is the
                    projection in the $y$ axis. Then $g\circ{f}=\identity{X}$.
                    But $H\big((x,y),t\big)=(1-t)(x,0)+t(x,y)$ is a homotopy
                    between $f\circ{g}$ and $\identity{Y}$, and thus
                    $f\circ{g}$ is homotopy equivalence $\identity{Y}$.
                    Therefore $X$ and $Y$ are homotopy equivalent. Moreover,
                    they are both compact, connected, and have the cardinality
                    of the continuum. Suppose $h$ is a homeomorphism
                    $h:X\rightarrow{Y}$ and let $h(0)=\mathbf{x}\in{Y}$. If $h$
                    is a homeomorphism between $X$ and $Y$, then the restriction
                    of $h$ to $X\setminus\{0\}$ is a homeomorphism between
                    $[-1,0)\cup(0,1]$ and $[-1,1]^{2}\setminus\{\mathbf{x}\}$.
                    But $[-1,1]^{2}\setminus\{\mathbf{x}\}$ is connected, and
                    $[-1,0)\cup(0,1]$ is not. But homeomorphisms preserve
                    connectedness. Therefore, $X$ and $Y$ are not homeomorphic.
                \end{proof}
                Now one might point out that these two spaces have different
                dimensions. If we require our spaces to be manifolds of the
                same dimension, is it still possible to be homotopy equivalent
                but not homeomorphic? The answer is yes, and to prove this we
                must first know that a sphere and a torus are \textit{not}
                homeomorphic. This is obvious at first glane, a torus has a hole
                in it whereas a sphere does not. To prove this rigorously we
                must recall the Jordan curve theorem. Any continuous injective
                closed curve $\gamma:\nsphere[1]\rightarrow\nsphere[2]$ cuts the
                sphere into two parts. Hence if we look at the space
                $\nsphere[2]\setminus\gamma[\nsphere[1]]$ we are left with a
                disconnected space. However, we can cut out a circle from the
                torus and still leave the space connected. If we cut out either
                of the defining circles of the torus
                $\ntorus[2]=\nsphere[1]\times\nsphere[1]$ we are left with
                something homeomorphic to a cylinder, which is connected.
                \par\hfill\par
                This is not entirely useful since the torus is \textit{not}
                homotopy equivalent to the sphere. To make them equivalent we
                puncture the torus in one spot, and put three holes into the
                sphere. The resulting spaces are still not homeomorphic, but
                are they homotopy equivalent? The answer is yes. First note that
                a sphere with one hole is homeomorphic to a disc. We can use
                stereographic projection to see this, or just look at
                Fig.~\ref{fig:homeo_Punc_S2_and_Plane}.
                \begin{figure}[H]
                    \centering
                    \captionsetup{type=figure}
                    \includegraphics{images/Sphere_to_Disk_Homeo.pdf}
                    \caption{Homeomorphism of a Punctured Sphere to a Disk}
                    \label{fig:homeo_Punc_S2_and_Plane}
                \end{figure}
                It should be intuitively clear that both the torus and the
                sphere are two dimensional topological manifolds. If we remove a
                finite number of points from either of these we are still left
                with two dimensional manifolds.
                \begin{theorem}
                    There exist topological manifolds $\topspace[X]{X}$ and
                    $\topspace[Y]{Y}$ of the same dimension that are homotopy
                    equivalence but not homeomorphic.
                \end{theorem}
                We'll prove this with pictures. We know the sphere with three
                holes is homeomorphic to a plane with 2 holes. But we've seen
                that this is homotopy equivalent to a figure eight since there
                is a deformation retraction of $\nspace[2]\setminus\{(0,0)\}$
                onto a lemniscate. Similarly, if we use our square
                representation of a torus we can show that a torus with one hole
                is homotopy equivalent to a figure eight, see
                Fig.~\ref{fig:HE_Torus_Sphere_Fig_8}.
                \begin{figure}[H]
                        \centering
                        \captionsetup{type=figure}
                        \resizebox{\textwidth}{!}{%
                            \begin{tikzpicture}[%
    every edge/.style={draw=black},
    scale=1.7,
    >=latex
]
    \draw[ball color=gray!40, opacity=0.4] (0,0) circle (1cm);
    \draw (-1,0) arc (180:360:1 and 0.3);
    \draw[dashed] (1,0) arc (0:180:1 and 0.3);
    \draw[fill=white] (0.55,0.55) circle (0.75pt);
    \draw[fill=white] (0.65,0.65) circle (0.75pt);
    \draw[fill=white] (0.5,0.72)  circle (0.75pt);
    \draw[->] (1.3,0) to (2.3,0);
    \draw[fill=gray, shading angle=215]
        (2.5,-0.5)--(3.2,0.5)--(5.5,0.5)--(4.8,-0.5)--cycle;
    \draw[densely dashed] (3.7,0) circle (0.295);
    \draw[densely dashed] (4.3,0) circle (0.295);
    \draw[fill=white] (3.7,0) circle (0.75pt);
    \draw[fill=white] (4.3,0) circle (0.75pt);
    \draw[->] (5.7,0) to [in=90,out=0] (7,-1);
    \draw[->] (1.2,-2.3)--(2,-2.3);
    \draw[%
        postaction={decorate},
        decoration={%
            markings,
            mark=at position .145 with \arrow{latex},
            mark=at position .375 with \arrow{latex},
            mark=at position .395 with \arrow{latex},
            mark=at position .615 with \arrowreversed{latex},
            mark=at position .855 with \arrowreversed{latex},
            mark=at position .875 with \arrowreversed{latex}
        },
        fill=gray,
        shading angle=215
    ] (-0.75,-3)--(0.75,-3)--(0.75,-1.5)--(-0.75,-1.5)--cycle;
    \draw[fill=white] (0,-2.3) circle (0.75pt);
    \draw[%
        postaction={decorate},
        decoration={%
            markings,
            mark=at position .5 with \arrow{latex},
            mark=at position .55 with \arrow{latex}
        }
    ]   (3.1,-2.3) arc[%
                start angle=0,
                delta angle=-360,
                x radius=.35,
                y radius=.75
        ];
    \draw (2.75,-1.55) -- (4.75,-1.55);
    \draw[%
        postaction={decorate},
        decoration={%
        markings,
        mark=at position .5 with \arrow{latex},
        mark=at position .55 with \arrow{latex}
        }
    ]   (5.1cm,-2.3cm) arc[%
            start angle=0,
            delta angle=-360,
            x radius=.35,
            y radius=.75
        ];
    \draw[->] (5.4,-2.3)--(6.3,-2.3);
    \draw (7,-1.8) circle (0.5);
    \draw (7,-2.8) circle (0.5);
\end{tikzpicture}%
                        }
                        \caption{%
                            Equivalence of a Sphere with Three Holes and a
                            Punctured Torus.%
                        }
                        \label{fig:HE_Torus_Sphere_Fig_8}
                \end{figure}
                Now, you say, hold on! You lost compactness! This is precisely
                getting down to Poincar\'{e}'s conjectures. When one classifies
                all two dimensional manifolds, they come across that fact that
                if $\topspace{X}$ is a compact two dimensional topological
                manifold that is homotopy equivalent to $\nsphere[2]$, then it
                is homeomorphic to it. Quite an impressive theorem, but
                Poincar\'{e} asks for the next dimension up:
                \begin{center}
                    \textit{If} $\topspace{X}$ \textit{is a three dimensional}
                    \textit{compact manifold that is homotopy equivalent to}
                    $\nsphere[3]$, \textit{is it homeomorphic?}
                \end{center}
                The answer is yes, but the proof is \textit{hard}.
        \section{Lecture 3: Cell Complexes}
\end{document}