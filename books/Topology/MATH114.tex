%------------------------------------------------------------------------------%
\documentclass[oneside]{book}                                                  %
%------------------------------Preamble----------------------------------------%
\makeatletter                                                                  %
    \def\input@path{{../../}}                                                  %
\makeatother                                                                   %
%---------------------------Packages----------------------------%
\usepackage{geometry}
\geometry{b5paper, margin=1.0in}
\usepackage[T1]{fontenc}
\usepackage{graphicx, float}            % Graphics/Images.
\usepackage{natbib}                     % For bibliographies.
\bibliographystyle{agsm}                % Bibliography style.
\usepackage[french, english]{babel}     % Language typesetting.
\usepackage[dvipsnames]{xcolor}         % Color names.
\usepackage{listings}                   % Verbatim-Like Tools.
\usepackage{mathtools, esint, mathrsfs} % amsmath and integrals.
\usepackage{amsthm, amsfonts, amssymb}  % Fonts and theorems.
\usepackage{tcolorbox}                  % Frames around theorems.
\usepackage{upgreek}                    % Non-Italic Greek.
\usepackage{fmtcount, etoolbox}         % For the \book{} command.
\usepackage[newparttoc]{titlesec}       % Formatting chapter, etc.
\usepackage{titletoc}                   % Allows \book in toc.
\usepackage[nottoc]{tocbibind}          % Bibliography in toc.
\usepackage[titles]{tocloft}            % ToC formatting.
\usepackage{pgfplots, tikz}             % Drawing/graphing tools.
\usepackage{imakeidx}                   % Used for index.
\usetikzlibrary{
    calc,                   % Calculating right angles and more.
    angles,                 % Drawing angles within triangles.
    arrows.meta,            % Latex and Stealth arrows.
    quotes,                 % Adding labels to angles.
    positioning,            % Relative positioning of nodes.
    decorations.markings,   % Adding arrows in the middle of a line.
    patterns,
    arrows
}                                       % Libraries for tikz.
\pgfplotsset{compat=1.9}                % Version of pgfplots.
\usepackage[font=scriptsize,
            labelformat=simple,
            labelsep=colon]{subcaption} % Subfigure captions.
\usepackage[font={scriptsize},
            hypcap=true,
            labelsep=colon]{caption}    % Figure captions.
\usepackage[pdftex,
            pdfauthor={Ryan Maguire},
            pdftitle={Mathematics and Physics},
            pdfsubject={Mathematics, Physics, Science},
            pdfkeywords={Mathematics, Physics, Computer Science, Biology},
            pdfproducer={LaTeX},
            pdfcreator={pdflatex}]{hyperref}
\hypersetup{
    colorlinks=true,
    linkcolor=blue,
    filecolor=magenta,
    urlcolor=Cerulean,
    citecolor=SkyBlue
}                           % Colors for hyperref.
\usepackage[toc,acronym,nogroupskip,nopostdot]{glossaries}
\usepackage{glossary-mcols}
%------------------------Theorem Styles-------------------------%
\theoremstyle{plain}
\newtheorem{theorem}{Theorem}[section]

% Define theorem style for default spacing and normal font.
\newtheoremstyle{normal}
    {\topsep}               % Amount of space above the theorem.
    {\topsep}               % Amount of space below the theorem.
    {}                      % Font used for body of theorem.
    {}                      % Measure of space to indent.
    {\bfseries}             % Font of the header of the theorem.
    {}                      % Punctuation between head and body.
    {.5em}                  % Space after theorem head.
    {}

% Italic header environment.
\newtheoremstyle{thmit}{\topsep}{\topsep}{}{}{\itshape}{}{0.5em}{}

% Define environments with italic headers.
\theoremstyle{thmit}
\newtheorem*{solution}{Solution}

% Define default environments.
\theoremstyle{normal}
\newtheorem{example}{Example}[section]
\newtheorem{definition}{Definition}[section]
\newtheorem{problem}{Problem}[section]

% Define framed environment.
\tcbuselibrary{most}
\newtcbtheorem[use counter*=theorem]{ftheorem}{Theorem}{%
    before=\par\vspace{2ex},
    boxsep=0.5\topsep,
    after=\par\vspace{2ex},
    colback=green!5,
    colframe=green!35!black,
    fonttitle=\bfseries\upshape%
}{thm}

\newtcbtheorem[auto counter, number within=section]{faxiom}{Axiom}{%
    before=\par\vspace{2ex},
    boxsep=0.5\topsep,
    after=\par\vspace{2ex},
    colback=Apricot!5,
    colframe=Apricot!35!black,
    fonttitle=\bfseries\upshape%
}{ax}

\newtcbtheorem[use counter*=definition]{fdefinition}{Definition}{%
    before=\par\vspace{2ex},
    boxsep=0.5\topsep,
    after=\par\vspace{2ex},
    colback=blue!5!white,
    colframe=blue!75!black,
    fonttitle=\bfseries\upshape%
}{def}

\newtcbtheorem[use counter*=example]{fexample}{Example}{%
    before=\par\vspace{2ex},
    boxsep=0.5\topsep,
    after=\par\vspace{2ex},
    colback=red!5!white,
    colframe=red!75!black,
    fonttitle=\bfseries\upshape%
}{ex}

\newtcbtheorem[auto counter, number within=section]{fnotation}{Notation}{%
    before=\par\vspace{2ex},
    boxsep=0.5\topsep,
    after=\par\vspace{2ex},
    colback=SeaGreen!5!white,
    colframe=SeaGreen!75!black,
    fonttitle=\bfseries\upshape%
}{not}

\newtcbtheorem[use counter*=remark]{fremark}{Remark}{%
    fonttitle=\bfseries\upshape,
    colback=Goldenrod!5!white,
    colframe=Goldenrod!75!black}{ex}

\newenvironment{bproof}{\textit{Proof.}}{\hfill$\square$}
\tcolorboxenvironment{bproof}{%
    blanker,
    breakable,
    left=3mm,
    before skip=5pt,
    after skip=10pt,
    borderline west={0.6mm}{0pt}{green!80!black}
}

\AtEndEnvironment{lexample}{$\hfill\textcolor{red}{\blacksquare}$}
\newtcbtheorem[use counter*=example]{lexample}{Example}{%
    empty,
    title={Example~\theexample},
    boxed title style={%
        empty,
        size=minimal,
        toprule=2pt,
        top=0.5\topsep,
    },
    coltitle=red,
    fonttitle=\bfseries,
    parbox=false,
    boxsep=0pt,
    before=\par\vspace{2ex},
    left=0pt,
    right=0pt,
    top=3ex,
    bottom=1ex,
    before=\par\vspace{2ex},
    after=\par\vspace{2ex},
    breakable,
    pad at break*=0mm,
    vfill before first,
    overlay unbroken={%
        \draw[red, line width=2pt]
            ([yshift=-1.2ex]title.south-|frame.west) to
            ([yshift=-1.2ex]title.south-|frame.east);
        },
    overlay first={%
        \draw[red, line width=2pt]
            ([yshift=-1.2ex]title.south-|frame.west) to
            ([yshift=-1.2ex]title.south-|frame.east);
    },
}{ex}

\AtEndEnvironment{ldefinition}{$\hfill\textcolor{Blue}{\blacksquare}$}
\newtcbtheorem[use counter*=definition]{ldefinition}{Definition}{%
    empty,
    title={Definition~\thedefinition:~{#1}},
    boxed title style={%
        empty,
        size=minimal,
        toprule=2pt,
        top=0.5\topsep,
    },
    coltitle=Blue,
    fonttitle=\bfseries,
    parbox=false,
    boxsep=0pt,
    before=\par\vspace{2ex},
    left=0pt,
    right=0pt,
    top=3ex,
    bottom=0pt,
    before=\par\vspace{2ex},
    after=\par\vspace{1ex},
    breakable,
    pad at break*=0mm,
    vfill before first,
    overlay unbroken={%
        \draw[Blue, line width=2pt]
            ([yshift=-1.2ex]title.south-|frame.west) to
            ([yshift=-1.2ex]title.south-|frame.east);
        },
    overlay first={%
        \draw[Blue, line width=2pt]
            ([yshift=-1.2ex]title.south-|frame.west) to
            ([yshift=-1.2ex]title.south-|frame.east);
    },
}{def}

\AtEndEnvironment{ltheorem}{$\hfill\textcolor{Green}{\blacksquare}$}
\newtcbtheorem[use counter*=theorem]{ltheorem}{Theorem}{%
    empty,
    title={Theorem~\thetheorem:~{#1}},
    boxed title style={%
        empty,
        size=minimal,
        toprule=2pt,
        top=0.5\topsep,
    },
    coltitle=Green,
    fonttitle=\bfseries,
    parbox=false,
    boxsep=0pt,
    before=\par\vspace{2ex},
    left=0pt,
    right=0pt,
    top=3ex,
    bottom=-1.5ex,
    breakable,
    pad at break*=0mm,
    vfill before first,
    overlay unbroken={%
        \draw[Green, line width=2pt]
            ([yshift=-1.2ex]title.south-|frame.west) to
            ([yshift=-1.2ex]title.south-|frame.east);},
    overlay first={%
        \draw[Green, line width=2pt]
            ([yshift=-1.2ex]title.south-|frame.west) to
            ([yshift=-1.2ex]title.south-|frame.east);
    }
}{thm}

%--------------------Declared Math Operators--------------------%
\DeclareMathOperator{\adjoint}{adj}         % Adjoint.
\DeclareMathOperator{\Card}{Card}           % Cardinality.
\DeclareMathOperator{\curl}{curl}           % Curl.
\DeclareMathOperator{\diam}{diam}           % Diameter.
\DeclareMathOperator{\dist}{dist}           % Distance.
\DeclareMathOperator{\Div}{div}             % Divergence.
\DeclareMathOperator{\Erf}{Erf}             % Error Function.
\DeclareMathOperator{\Erfc}{Erfc}           % Complementary Error Function.
\DeclareMathOperator{\Ext}{Ext}             % Exterior.
\DeclareMathOperator{\GCD}{GCD}             % Greatest common denominator.
\DeclareMathOperator{\grad}{grad}           % Gradient
\DeclareMathOperator{\Ima}{Im}              % Image.
\DeclareMathOperator{\Int}{Int}             % Interior.
\DeclareMathOperator{\LC}{LC}               % Leading coefficient.
\DeclareMathOperator{\LCM}{LCM}             % Least common multiple.
\DeclareMathOperator{\LM}{LM}               % Leading monomial.
\DeclareMathOperator{\LT}{LT}               % Leading term.
\DeclareMathOperator{\Mod}{mod}             % Modulus.
\DeclareMathOperator{\Mon}{Mon}             % Monomial.
\DeclareMathOperator{\multideg}{mutlideg}   % Multi-Degree (Graphs).
\DeclareMathOperator{\nul}{nul}             % Null space of operator.
\DeclareMathOperator{\Ord}{Ord}             % Ordinal of ordered set.
\DeclareMathOperator{\Prin}{Prin}           % Principal value.
\DeclareMathOperator{\proj}{proj}           % Projection.
\DeclareMathOperator{\Refl}{Refl}           % Reflection operator.
\DeclareMathOperator{\rk}{rk}               % Rank of operator.
\DeclareMathOperator{\sgn}{sgn}             % Sign of a number.
\DeclareMathOperator{\sinc}{sinc}           % Sinc function.
\DeclareMathOperator{\Span}{Span}           % Span of a set.
\DeclareMathOperator{\Spec}{Spec}           % Spectrum.
\DeclareMathOperator{\supp}{supp}           % Support
\DeclareMathOperator{\Tr}{Tr}               % Trace of matrix.
%--------------------Declared Math Symbols--------------------%
\DeclareMathSymbol{\minus}{\mathbin}{AMSa}{"39} % Unary minus sign.
%------------------------New Commands---------------------------%
\DeclarePairedDelimiter\norm{\lVert}{\rVert}
\DeclarePairedDelimiter\ceil{\lceil}{\rceil}
\DeclarePairedDelimiter\floor{\lfloor}{\rfloor}
\newcommand*\diff{\mathop{}\!\mathrm{d}}
\newcommand*\Diff[1]{\mathop{}\!\mathrm{d^#1}}
\renewcommand*{\glstextformat}[1]{\textcolor{RoyalBlue}{#1}}
\renewcommand{\glsnamefont}[1]{\textbf{#1}}
\renewcommand\labelitemii{$\circ$}
\renewcommand\thesubfigure{%
    \arabic{chapter}.\arabic{figure}.\arabic{subfigure}}
\addto\captionsenglish{\renewcommand{\figurename}{Fig.}}
\numberwithin{equation}{section}

\renewcommand{\vector}[1]{\boldsymbol{\mathrm{#1}}}

\newcommand{\uvector}[1]{\boldsymbol{\hat{\mathrm{#1}}}}
\newcommand{\topspace}[2][]{(#2,\tau_{#1})}
\newcommand{\measurespace}[2][]{(#2,\varSigma_{#1},\mu_{#1})}
\newcommand{\measurablespace}[2][]{(#2,\varSigma_{#1})}
\newcommand{\manifold}[2][]{(#2,\tau_{#1},\mathcal{A}_{#1})}
\newcommand{\tanspace}[2]{T_{#1}{#2}}
\newcommand{\cotanspace}[2]{T_{#1}^{*}{#2}}
\newcommand{\Ckspace}[3][\mathbb{R}]{C^{#2}(#3,#1)}
\newcommand{\funcspace}[2][\mathbb{R}]{\mathcal{F}(#2,#1)}
\newcommand{\smoothvecf}[1]{\mathfrak{X}(#1)}
\newcommand{\smoothonef}[1]{\mathfrak{X}^{*}(#1)}
\newcommand{\bracket}[2]{[#1,#2]}

%------------------------Book Command---------------------------%
\makeatletter
\renewcommand\@pnumwidth{1cm}
\newcounter{book}
\renewcommand\thebook{\@Roman\c@book}
\newcommand\book{%
    \if@openright
        \cleardoublepage
    \else
        \clearpage
    \fi
    \thispagestyle{plain}%
    \if@twocolumn
        \onecolumn
        \@tempswatrue
    \else
        \@tempswafalse
    \fi
    \null\vfil
    \secdef\@book\@sbook
}
\def\@book[#1]#2{%
    \refstepcounter{book}
    \addcontentsline{toc}{book}{\bookname\ \thebook:\hspace{1em}#1}
    \markboth{}{}
    {\centering
     \interlinepenalty\@M
     \normalfont
     \huge\bfseries\bookname\nobreakspace\thebook
     \par
     \vskip 20\p@
     \Huge\bfseries#2\par}%
    \@endbook}
\def\@sbook#1{%
    {\centering
     \interlinepenalty \@M
     \normalfont
     \Huge\bfseries#1\par}%
    \@endbook}
\def\@endbook{
    \vfil\newpage
        \if@twoside
            \if@openright
                \null
                \thispagestyle{empty}%
                \newpage
            \fi
        \fi
        \if@tempswa
            \twocolumn
        \fi
}
\newcommand*\l@book[2]{%
    \ifnum\c@tocdepth >-3\relax
        \addpenalty{-\@highpenalty}%
        \addvspace{2.25em\@plus\p@}%
        \setlength\@tempdima{3em}%
        \begingroup
            \parindent\z@\rightskip\@pnumwidth
            \parfillskip -\@pnumwidth
            {
                \leavevmode
                \Large\bfseries#1\hfill\hb@xt@\@pnumwidth{\hss#2}
            }
            \par
            \nobreak
            \global\@nobreaktrue
            \everypar{\global\@nobreakfalse\everypar{}}%
        \endgroup
    \fi}
\newcommand\bookname{Book}
\renewcommand{\thebook}{\texorpdfstring{\Numberstring{book}}{book}}
\providecommand*{\toclevel@book}{-2}
\makeatother
\titleformat{\part}[display]
    {\Large\bfseries}
    {\partname\nobreakspace\thepart}
    {0mm}
    {\Huge\bfseries}
\titlecontents{part}[0pt]
    {\large\bfseries}
    {\partname\ \thecontentslabel: \quad}
    {}
    {\hfill\contentspage}
\titlecontents{chapter}[0pt]
    {\bfseries}
    {\chaptername\ \thecontentslabel:\quad}
    {}
    {\hfill\contentspage}
\newglossarystyle{longpara}{%
    \setglossarystyle{long}%
    \renewenvironment{theglossary}{%
        \begin{longtable}[l]{{p{0.25\hsize}p{0.65\hsize}}}
    }{\end{longtable}}%
    \renewcommand{\glossentry}[2]{%
        \glstarget{##1}{\glossentryname{##1}}%
        &\glossentrydesc{##1}{~##2.}
        \tabularnewline%
        \tabularnewline
    }%
}
\newglossary[not-glg]{notation}{not-gls}{not-glo}{Notation}
\newcommand*{\newnotation}[4][]{%
    \newglossaryentry{#2}{type=notation, name={\textbf{#3}, },
                          text={#4}, description={#4},#1}%
}
%--------------------------LENGTHS------------------------------%
% Spacings for the Table of Contents.
\addtolength{\cftsecnumwidth}{1ex}
\addtolength{\cftsubsecindent}{1ex}
\addtolength{\cftsubsecnumwidth}{1ex}
\addtolength{\cftfignumwidth}{1ex}
\addtolength{\cfttabnumwidth}{1ex}

% Indent and paragraph spacing.
\setlength{\parindent}{0em}
\setlength{\parskip}{0em}                                                           %
%----------------------------Main Document-------------------------------------%
\begin{document}
    \title{MATH 114 Algebraic Topology Notes}
    \author{%
        Professor: Vladimir Chernov\\
        Notes by: Ryan Maguire%
    }
    \date{\vspace{-5ex}}
    \maketitle
    \tableofcontents
    \listoffigures
    \chapter{Homotopy}
        \section{Lecture 1: Review}
            \subsection{Topological Spaces}
                We start with the definition of topological spaces. We wish to
                generalize the notion of \textit{openness} that one comes across
                in the study of metric spaces\index{Metric Space}, or more
                concretely in the study of $\nspace$. To do this we axiomatize
                the properties of open sets: Taking arbitrary unions of open
                sets results in an open set, and the finite intersection of open
                sets is still open. Furthermore, we define the entire space and
                the empty set $\emptyset$ to be open. This gives us our
                definition of a topological space.
                \begin{fdefinition}{Topological Space}{Topological_Space}
                    A topological space is a set $X$ with a collection
                    $\tau\subseteq\powset{X}$ of subsets of $X$, called the
                    \textit{open} subsets, such that:
                    \index{Topological Space}\index{Open Subset}
                    \begin{enumerate}
                        \item \label{def:top:Empty_and_X_Open}%
                              $\emptyset\in\tau$ and $X\in\tau$
                        \item \label{def:top:Finite_Intersections}%
                              For all $\mathcal{U},\mathcal{U}\in\tau$ it is
                              true that $\mathcal{U}\cap\mathcal{V}\in\tau$
                        \item \label{def:top:Arbitrary_Unions}%
                              For any subset $\mathcal{O}\subseteq\tau$ it is
                              true that $\bigcup\mathcal{O}\in\tau$
                    \end{enumerate}
                \end{fdefinition}
                \begin{example}
                    The standard metric topology on $\nspace[]$ is induced by
                    declaring $\mathcal{U}\subseteq\nspace[]$ to be open if and
                    only if for all $x\in\mathcal{U}$ there is an
                    $\varepsilon>0$ such that
                    $(x-\varepsilon,x+\varepsilon)\subseteq\mathcal{U}$. So an
                    open interval of the form $(a,b)$ is open
                    (see Fig.~\ref{fig:Open_Subset_of_R}).
                \end{example}
                \begin{figure}[H]
                    \centering
                    \captionsetup{type=figure}
                    \begin{tikzpicture}[>=Latex]
    \coordinate (a)  at (2.0,  0.0);
    \coordinate (b)  at (9.0,  0.0);
    \coordinate (xt) at (4.0,  0.1);
    \coordinate (xb) at (4.0, -0.1);
    \coordinate (x)  at (4.0,  0.0);
    \coordinate (xl) at (3.0,  0.0);
    \coordinate (xr) at (5.0,  0.0);

    \draw[<->]   (0, 0) to (10, 0) node[above] {$\mathbb{R}$};
    \draw[thick] (a) to (b);
    \draw[thick] (b) arc (0:15:0.5);
    \draw[thick] (b) arc (0:-15:0.5);
    \draw[thick] (a) arc (180:195:0.5);
    \draw[thick] (a) arc (180:165:0.5);

    \draw (xt) to (xb);

    \node at (a)  [below=1ex] {$a$};
    \node at (b)  [below=1ex] {$b$};
    \node at (xl) [below=1ex] {$x-\varepsilon$};
    \node at (xr) [below=1ex] {$x+\varepsilon$};
    \node at (x)  [below=1ex] {$x$};

    \draw[blue, thick] (xr) arc (0:15:0.5);
    \draw[blue, thick] (xr) arc (0:-15:0.5);
    \draw[blue, thick] (xl) arc (180:195:0.5);
    \draw[blue, thick] (xl) arc (180:165:0.5);
    \draw[blue, thick] (xr) to (xl);
\end{tikzpicture}
                    \caption{An Open Subset of $\nspace[]$}
                    \label{fig:Open_Subset_of_R}
                \end{figure}
                By examining Fig.~\ref{fig:Open_Interval_Intersect_is_Open} we
                can convince ourselves that the intersection of open intervals
                is again open so long as we declare the empty set to be open.
                That is, $(0,1)$ and $(2,3)$ are open sets, but
                $(0,1)\cap(2,3)=\emptyset$. This highlights the need in
                declaring the empty set an element of the topology.
                \begin{figure}[H]
                    \centering
                    \captionsetup{type=figure}
                    \begin{tikzpicture}[>=Latex]
    \coordinate (a)  at (2.0,  0.0);
    \coordinate (b)  at (7.0,  0.0);
    \coordinate (c)  at (3.0,  0.0);
    \coordinate (d)  at (9.0,  0.0);
    \coordinate (x)  at (5.0,  0.0);
    \coordinate (xb) at (5.0, -0.1);
    \coordinate (xt) at (5.0,  0.1);
    \coordinate (xl) at (4.2,  0.0);
    \coordinate (xr) at (5.8,  0.0);

    \draw[<->]   (0, 0) to (10, 0) node[above] {$\mathbb{R}$};

    \node at (a)  [below=1ex] {$a$};
    \node at (b)  [below=1ex] {$b$};
    \node at (c)  [below=1ex] {$c$};
    \node at (d)  [below=1ex] {$d$};
    \node at (xl) [below=1ex] {$x-\varepsilon$};
    \node at (xr) [below=1ex] {$x+\varepsilon$};
    \node at (x)  [below=1ex] {$x$};
    \node at (a)  [above=1ex]
        {$\color{blue}{(a,b)}\cap\color{red}{(c,d)}=\color{Violet}{(c,b)}$};
    \node at (b) [above=1ex]
        {$\color{cyan}{(x-\varepsilon,x+\varepsilon)}%
            \subseteq\color{Violet}{(c,b)}$};

    \draw[blue,   thick]    (a)  to (c);
    \draw[red,    thick]    (b)  to (d);
    \draw[Violet, thick]    (c)  to (xl);
    \draw[Violet, thick]    (xr) to (b);
    \draw[cyan, very thick] (xl) to (xr);

    \draw[thin] (xt) to (xb);

    \draw[blue, very thick] (b) arc (0:15:0.5);
    \draw[blue, very thick] (b) arc (0:-15:0.5);
    \draw[blue, very thick] (a) arc (180:195:0.5);
    \draw[blue, very thick] (a) arc (180:165:0.5);

    \draw[red, very thick] (d) arc (0:15:0.5);
    \draw[red, very thick] (d) arc (0:-15:0.5);
    \draw[red, very thick] (c) arc (180:195:0.5);
    \draw[red, very thick] (c) arc (180:165:0.5);

    \draw[cyan, very thick] (xr) arc (0:15:0.5);
    \draw[cyan, very thick] (xr) arc (0:-15:0.5);
    \draw[cyan, very thick] (xl) arc (180:195:0.5);
    \draw[cyan, very thick] (xl) arc (180:165:0.5);
\end{tikzpicture}
                    \caption{The Intersection of Open Intervals is Open}
                    \label{fig:Open_Interval_Intersect_is_Open}
                \end{figure}
                \begin{example}
                    Given a metric space $\metspace{X}$, the metric topology is
                    defined by stating that $\mathcal{U}\subseteq{X}$ is open if
                    and only if for all $x\in\mathcal{U}$ there is an
                    $\varepsilon>0$ such that the $\varepsilon$ ball centered
                    about $x$ is contained in $\mathcal{U}$:
                    $\rball{\varepsilon}{\metspace{X}}{x}\subseteq\mathcal{U}$
                    (see Fig.~\ref{fig:Open_Subset_Metric_Space}).
                \end{example}
                \begin{figure}[H]
                    \centering
                    \captionsetup{type=figure}
                    \includegraphics{images/Open_Set_in_a_Metric_Space.pdf}
                    \caption{Open Subset of a Metric Space}
                    \label{fig:Open_Subset_Metric_Space}
                \end{figure}
                This metric topology on $\nspace[]$ is often called the
                \textit{standard} topology, but it is not the only one we can
                place on it.
                \begin{example}
                    The chaotic topology, also called the trivial topology or
                    the indiscrete topology, is the simplest topology one can
                    define on a set. We write:
                    \begin{equation}
                        \tau=\{\,\emptyset,\,X\,\}
                    \end{equation}
                    This is a topology since it trivially satisfies the three
                    properties enumerated in Def.~\ref{def:Topological_Space}.
                \end{example}
                \begin{example}
                    The largest topology one can define is the entire power set:
                    \begin{equation}
                        \tau=\powset{X}
                    \end{equation}
                    Again, rather trivial, this is a topology on $\nspace[]$.
                    For those who have studied metric spaces, it is called the
                    trivial topology since it is the topology induced by the
                    discrete metric:
                    \begin{equation}
                        d(x,y)=
                        \begin{cases}
                            0,&x=y\\
                            1,&x\ne{y}
                        \end{cases}
                    \end{equation}
                \end{example}
                The closed interval $[a,b]$ is \textit{not} open in the standard
                topology (see Fig.~\ref{fig:Closed_Interval_Not_Open}), but it
                \textit{is} open with the discrete topology since everything is
                open in that topology. Because of this there is possible
                ambiguity with saying $\mathcal{U}$ is open if one has not
                specified the topology. Thankfully with most spaces one is
                interested in there is a standard or natural topology, and we
                usually choose this one without saying so. For metric spaces we
                almost always choose the metric topology.
                \begin{figure}[H]
                    \centering
                    \captionsetup{type=figure}
                    \begin{tikzpicture}[>=Latex]
    % Coordinates for various points.
    \coordinate (a)  at (4.0,  0.0);
    \coordinate (b)  at (9.0,  0.0);
    \coordinate (xl) at (3.0,  0.0);
    \coordinate (xr) at (5.0,  0.0);

    % Draw the real line.
    \draw[<->]   (0, 0) to (10, 0) node[above] {$\mathbb{R}$};

    % Draw the closed interval [a, b].
    \draw[thick] (a) to (b);

    % Add "brackets" indicating it is a closed interval.
    \draw[thick] (8.9, 0.1) to (9.0, 0.1) to (9.0, -0.1) to (8.9, -0.1);
    \draw[thick] (4.1, 0.1) to (4.0, 0.1) to (4.0, -0.1) to (4.1, -0.1);

    % Labels for the vaious points.
    \node at (a)  [below=1ex] {$a$};
    \node at (b)  [below=1ex] {$b$};
    \node at (xl) [below=1ex] {$a-\varepsilon$};
    \node at (xr) [below=1ex] {$a+\varepsilon$};

    % Draw the part of the open interval (a-e, a+e) that is inside of [a, b].
    \draw[blue, thick] (xr) arc (0:15:0.5);
    \draw[blue, thick] (xr) arc (0:-15:0.5);
    \draw[blue, thick] (a) to (xr);

    % Draw the part that falls outside.
    \draw[red, thick]  (xl) arc (180:195:0.5);
    \draw[red, thick]  (xl) arc (180:165:0.5);
    \draw[red, thick]  (a) to (xl);
\end{tikzpicture}
                    \caption{Closed Intervals are Not Open}
                    \label{fig:Closed_Interval_Not_Open}
                \end{figure}
                With the real line the intersection of finitely many open sets
                is still open, whereas we've only required the intersection of
                two open sets to still be open. We extend this to any finite
                collection by induction.
                \begin{theorem}
                    \label{thm:Finite_Intersections_Is_Open}%
                    If $\topspace{X}$ is a topological space and
                    $\mathcal{O}\subseteq\tau$ finite, then
                    $\bigcap\mathcal{O}\in\tau$.
                \end{theorem}
                \begin{proof}
                    Apply induction to Def.~\ref{def:Topological_Space}
                    part \ref{def:top:Finite_Intersections}.
                \end{proof}
                Now that we've presented some examples, we define what it means
                for a set to be closed. In analysis we defined a closed set to
                be a set that has all of its limit points. For topology this is
                not general enough (topological spaces where sequences suffice
                to define closedness are called \textit{sequential spaces}).
                There is a standard theorem one comes across that a subset of
                $\nspace[]$ is closed if and only if its complement is open. We
                take this theorem and adopt it as the definition of what it
                means to be a closed set in a general topological space.
                \begin{fdefinition}{Closed Subset}{Closed_Subset}
                    A closed subset of a topological space $\topspace{X}$ is a
                    subset $\mathcal{C}\subseteq{X}$ such that there exists an
                    open set $\mathcal{U}\in\tau$ with:
                    \begin{equation*}
                        \mathcal{C}=X\setminus\mathcal{U}
                    \end{equation*}
                \end{fdefinition}
                That is, closed sets are the complements of open sets. There is
                a common misconception that closed sets are simply \textit{not}
                open sets, and vice-versa, but this is not so. In the discrete
                topology every set is open, and hence every set is closed. In
                the indiscrete topology there are no non-empty proper open
                subsets, and hence most sets are neither open nor closed. These
                examples show that openness and closedness are
                \textit{a priori} unrelated notions. If we know $A\subseteq{X}$
                is open and nothing more, we cannot conclude whether or not $A$
                is closed, and similarly if we know $B\subseteq{X}$ is closed
                and nothing more, then we cannot conclude whether or not $B$ is
                open. Using the idempotent laws of complement, we obtain the
                following:
                \begin{theorem}
                    \label{thm:Closed_Iff_Comp_is_Open}%
                    If $\topspace{X}$ is a topological space and
                    $C\subseteq{X}$, then $C$ is closed if and only if
                    $X\setminus{C}$ is open.
                \end{theorem}
                \begin{proof}
                    For if $X\setminus{C}$ is open, then
                    $X\setminus(X\setminus{C})$ is closed
                    (Def.~\ref{def:Closed_Subset}). But
                    $X\setminus(X\setminus{C})=C$ and hence $C$ is closed.
                \end{proof}
                \par\hfill\par
                Using De Morgan's laws we can define an equivalent notion of
                topological spaces using closed sets. De Morgan's laws state for
                sets $A,B,X$, with $A,B\subseteq{X}$, the following is true:
                \twocolumneq{%
                    X\setminus(A\cap{B})=(X\setminus{A})\cup(X\setminus{B})%
                }{%
                    X\setminus(A\cup{B})=(X\setminus{A})\cap(X\setminus{B})%
                }
                We can write this more suggestively if we let
                $X\setminus{A}=A^{C}$ ($C$ for complement).
                \twocolumneq{(A\cap{B})^{C}=A^{C}\cup{B}^{C}}
                            {(A\cup{B})^{C}=A^{C}\cap{B}^{C}}
                De Morgan's laws hold for arbitrary unions and intersections:
                \twocolumneq{%
                    \Big(\bigcap\mathcal{U}\Big)^{C}=\bigcup\mathcal{U}^{C}%
                }{%
                    \Big(\bigcup\mathcal{U}\Big)^{C}=\bigcap\mathcal{U}^{C}%
                }
                With this we may prove the following.
                \begin{theorem}
                    If $\topspace{X}$ is a topological space, then $\emptyset$
                    and $X$ are closed, if $\mathcal{C},\mathcal{D}\subseteq{X}$
                    are closed, then $\mathcal{C}\cup\mathcal{D}$ is closed, and
                    if $\Lambda\subseteq\powset{X}$ is a collection of closed
                    subsets of $X$, then $\bigcap\Lambda$ is closed.
                \end{theorem}
                \begin{proof}
                    For the first part, apply the definition of closed subsets
                    (Def.~\ref{def:Closed_Subset}) to
                    Def.~\ref{def:Topological_Space} part
                    \ref{def:top:Empty_and_X_Open} and recall that
                    $X\setminus\emptyset=X$ and $X\setminus{X}=\emptyset$. For
                    the later parts, combine De Morgan's Laws with
                    Def.~\ref{def:Topological_Space} parts
                    \ref{def:top:Finite_Intersections} and
                    \ref{def:top:Arbitrary_Unions}, respectively.
                \end{proof}
                We may expand the union of two closed sets to the unions of
                finitely many closed sets inductively as we did in
                Thm.~\ref{thm:Finite_Intersections_Is_Open}.
            \subsection{Continuity}
                In an analysis course one talks about continuous functions by
                one of two equivalent means: The $\varepsilon-\delta$ definition
                and by the limit of sequences ($f(x_{n})\rightarrow{f}(x)$ for
                any sequence $x_{n}\rightarrow{x}$). We can also define
                continuity more generally by means of open sets. A function
                $f:X\rightarrow{Y}$ between topological spaces is called
                continuous if and only if for every open subset
                $\mathcal{V}\in{Y}$, the pre-image $f^{\minus{1}}[\mathcal{U}]$
                is open in $X$. That is, if $\topspace[X]{X}$ and
                $\topspace[Y]{Y}$ are our topological spaces with topologies
                $\tau_{X}$ and $\tau_{Y}$, respectively, then
                $f:X\rightarrow{Y}$ is continuous if and only if for all
                $\mathcal{V}\in\tau_{Y}$, it is true that
                $f^{\minus{1}}[\mathcal{V}]\in\tau_{X}$.
            \subsection{Deformation Retract}
                A deformation retraction from a topological space $\topspace{X}$
                onto a subset $A\subseteq{}$ is a family of continuous functions
                $f_{t}:X\rightarrow{A}$ such that
                \begin{example}
                    We can take an annulus, or a punctured plane, either will
                    do, and construct a deformation retract from this onto the
                    unit circle $\nsphere[1]$. We do this by noting the function
                    $h$ defined on $\nspace[2]\setminus\{(0,0)\}$ by:
                    \begin{equation}
                        h(\vector{x})=\frac{\vector{x}}{\norm{\vector{x}}_{2}}
                    \end{equation}
                    where $\norm{\vector{x}}_{2}$ is the \textit{Euclidean Norm}
                    of the point $\vector{x}$, that this function maps the
                    punctured plane $\nspace[2]\setminus\{(0,0)\}$ onto the
                    unit sphere. To complete our deformation retraction we
                    drag the point $\vector{x}\ne\vector{0}$ along the straight
                    line between $\vector{x}$ and $h(\vector{x})$. We have:
                    \begin{equation}
                        H(\vector{x},t)
                        =(1-t)\cdot\vector{x}+t\cdot{h}(\vector{x})
                    \end{equation}
                    This is a deformation retract since elements of the unit
                    circle are held fixed for all $t\in[0,1]$. If
                    $\vector{s}\in\nsphere[2]$, then by definition
                    $\norm{\vector{s}}_{2}=1$ and hence
                    $h(\vector{s})=\vector{s}$. Simplifying, we have:
                    \begin{equation}
                        H(\vector{s},s)=(1-t)\cdot\vector{s}+t\cdot\vector{s}
                            =\vector{s}
                    \end{equation}
                    Thus $H$ gives us a deformation retraction of the punctured
                    plane onto the unit circle. We can do the same deformation
                    retraction with an annulus if we simply restrict $h$ to that
                    domain (see Fig.~\ref{fig:Def_Retract_Annulus_to_Circle}).
                \end{example}
                \begin{figure}[H]
                    \centering
                    \captionsetup{type=figure}
                    \includegraphics{images/Homotopy_Circle.pdf}
                    \caption{Deformation Retraction of an Annulus to a Circle}
                    \label{fig:Def_Retract_Annulus_to_Circle}
                \end{figure}
\end{document}