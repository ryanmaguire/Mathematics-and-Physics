%------------------------------------------------------------------------------%
\documentclass{article}                                                        %
%------------------------------Preamble----------------------------------------%
\makeatletter                                                                  %
    \def\input@path{{../../}}                                                  %
\makeatother                                                                   %
%---------------------------Packages----------------------------%
\usepackage{geometry}
\geometry{b5paper, margin=1.0in}
\usepackage[T1]{fontenc}
\usepackage{graphicx, float}            % Graphics/Images.
\usepackage{natbib}                     % For bibliographies.
\bibliographystyle{agsm}                % Bibliography style.
\usepackage[french, english]{babel}     % Language typesetting.
\usepackage[dvipsnames]{xcolor}         % Color names.
\usepackage{listings, lstlinebgrd}      % Verbatim-Like Tools.
\usepackage{mathtools, esint, mathrsfs} % amsmath and integrals.
\usepackage{amsthm, amsfonts}           % Fonts and theorems.
\usepackage{tabularx}
\usepackage{tcolorbox}                  % Frames around theorems.
\usepackage{upgreek}                    % Non-Italic Greek.
\usepackage{paracol}                    % Two-column styling.
\usepackage{wrapfig}                    % Wrap text around figure.
\usepackage{fmtcount, etoolbox}         % For the \book{} command.
\usepackage[newparttoc]{titlesec}       % Formatting chapter, etc.
\usepackage{titletoc}                   % Allows \book in toc.
\usepackage[nottoc]{tocbibind}          % Bibliography in toc.
\usepackage[titles]{tocloft}            % ToC formatting.
\usepackage{multicol, enumitem}         % Multi-column/enumerate.
\usepackage{import}                     % Import external files.
\usepackage{pgfplots, tikz}             % Drawing/graphing tools.
\usetikzlibrary{
    calc,                   % Calculating right angles and more.
    angles,                 % Drawing angles within triangles.
    arrows.meta,            % Latex and Stealth arrows.
    quotes,                 % Adding labels to angles.
    positioning,            % Relative positioning of nodes.
    decorations.markings,   % Adding arrows in the middle of a line.
    patterns,
    arrows,
    shapes,
    shapes.geometric,
    cd,
    hobby,
    babel
}                                       % Libraries for tikz.
\pgfplotsset{compat=1.9}                % Version of pgfplots.
\usepackage[font=scriptsize,
            labelformat=simple,
            labelsep=colon]{subcaption} % Subfigure captions.
\usepackage[font={scriptsize},
            hypcap=true,
            labelsep=colon]{caption}    % Figure captions.
\usepackage{hyperref}                   % Allows for hyperlinks.
\hypersetup{
    colorlinks=true,
    linkcolor=blue,
    filecolor=magenta,
    urlcolor=Cerulean,
    citecolor=SkyBlue
}                           % Colors for hyperref.
\usepackage[toc,acronym,nogroupskip]{glossaries} % Glossaries and acronyms.
\usepackage[subpreambles=false]{standalone}      % Complileable sub files.

% Various font stuff from kiwi.
% Use this for Times text and Computer Modern math
%\usepackage{times}

% Quite nice
%\usepackage[charter, greekfamily=, greekuppercase=italicized]{mathdesign}
%\usepackage[utopia, greekuppercase=italicized]{mathdesign}    % Math is narrower

% Use this for Times text and math
%\usepackage{newtxtext}
%\usepackage[libertine,cmintegrals]{newtxmath}
%\usepackage{fix-cm}

%\usepackage{txfontsb}
% or
%\usepackage{mathptmx}

%\usepackage[scaled=0.92]{helvet}
%\renewcommand{\rmdefault}{ptm}

%\usepackage{mathpazo}    % add possibly `sc` and `osf` options
%\usepackage{eulervm}

%\usepackage{fourier}
%\renewcommand{\rmdefault}{ptm}
%\usepackage{mathptm}

%\usepackage{fontspec}
%\setmainfont{lmodern}

%\usepackage[varg]{txfonts}
%\usepackage{fouriernc}
%\usepackage{mathpazo}

%\usepackage{bookman}
%\usepackage[scaled]{uarial}
%\usepackage[scaled]{helvet}
%\renewcommand*\familydefault{\sfdefault}
%\usepackage[math]{anttor}

%\newcommand\fgeorgia{\fontfamily{jvn}\selectfont}
%\newcommand\ftimes{\fontfamily{ptm}\selectfont}
%\newcommand\fhelvetica{\fontfamily{phv}\selectfont}
%\newcommand\fcourier{\fontfamily{pcr}\selectfont}
%\newcommand\fbookman{\fontfamily{pbk}\selectfont}
%\newcommand\fnewcentury{\fontfamily{pnc}\selectfont}
%\newcommand\fpalatino{\fontfamily{ppl}\selectfont}
%\newcommand\favantgarde{\fontfamily{pag}\selectfont}
%\newcommand\fnormal{\normalfont}
%\newcommand\fsize[1]{\ifnum#1>0\fontsize{#1}{#1}\selectfont\else\normalsize\fi}
%------------------------Theorem Styles-------------------------%
% Define theorem style for default spacing and normal font.
\newtheoremstyle{normal}
    {\topsep}               % Amount of space above the theorem.
    {\topsep}               % Amount of space below the theorem.
    {}                      % Font used for body of theorem.
    {}                      % Measure of space to indent.
    {\bfseries}             % Font of the header of the theorem.
    {}                      % Punctuation between head and body.
    {.5em}                  % Space after theorem head.
    {}

% Define theorem style for default spacing with italicized font.
\newtheoremstyle{normalit}{\topsep}{\topsep}
                {\itshape}{}{\bfseries}{}{.5em}{}

% Italic header environment.
\newtheoremstyle{thmit}{\topsep}{\topsep}{}{}{\itshape}{}{0.5em}{}

% Define italicized environments.
\theoremstyle{normalit}
\newtheorem{theorem}{Theorem}[section]
\newtheorem{lemma}{Lemma}[section]
\newtheorem{corollary}{Corollary}[section]
\newtheorem{proposition}{Proposition}[section]
\newtheorem*{theorem*}{Theorem}

% Define environments with italic headers.
\theoremstyle{thmit}
\newtheorem*{solution}{Solution}
\newtheorem*{fsolution}{Solution}

% Define default environments.
\theoremstyle{normal}
\newtheorem{example}{Example}[section]
\newtheorem{definition}{Definition}[section]
\newtheorem{problem}{Problem}[section]
\newtheorem{question}{Question}[section]
\newtheorem{remark}{Remark}[section]
\newtheorem{properties}{Properties}[section]
\newtheorem{notation}{Notation}[section]
\newtheorem{axiom}{Axiom}[section]
\newtheorem*{properties*}{Properties}
\newtheorem*{remark*}{Remark}
\newtheorem*{definition*}{Definition}
\theoremstyle{plain}

% Define framed environment.
\tcbuselibrary{most}
\newtcbtheorem[use counter*=theorem]{ftheorem}{Theorem}%
    {colback=green!5,colframe=green!35!black,
     fonttitle=\bfseries\upshape}{th}

\newtcbtheorem[use counter*=example]{fdefinition}{Definition}%
    {fonttitle=\bfseries\upshape,
     colback=blue!5!white,colframe=blue!75!black}{def}

\newtcbtheorem[use counter*=example]{fexample}{Example}%
    {fonttitle=\bfseries\upshape,
     colback=red!5!white,colframe=red!75!black}{ex}

\newtcbtheorem[use counter*=notation]{fnotation}{Notation}%
    {fonttitle=\bfseries\upshape,
     colback=SeaGreen!5!white,colframe=SeaGreen!75!black}{ex}

\newtcbtheorem[use counter*=corollary]{fcorollary}{Corollary}%
    {fonttitle=\bfseries\upshape,
     colback=Orchid!5!white,colframe=Orchid!75!black}{ex}

\newenvironment{bproof}{\textit{Proof.}}{\hfill$\square$}
\tcolorboxenvironment{bproof}{blanker,breakable,left=5mm,
                             before skip=10pt,after skip=10pt,
                             borderline west={1mm}{0pt}{red}}
\tcolorboxenvironment{fsolution}
    {enhanced jigsaw,colframe=cyan,interior hidden,breakable}

%--------------------Declared Math Operators--------------------%
\DeclareMathOperator{\Refl}{Refl}           % Reflection operator.
\DeclareMathOperator{\Span}{Span}           % Span of a set of vectors.
\DeclareMathOperator{\Card}{Card}           % Cardinality of set.
\DeclareMathOperator{\Ord}{Ord}             % Ordinal of ordered set.
\DeclareMathOperator{\Tr}{Tr}               % Trace of matrix.
\DeclareMathOperator{\adjoint}{adj}         % Adjoint of matrix.
\DeclareMathOperator{\rk}{rk}               % Rank of operator.
\DeclareMathOperator{\nul}{nul}             % Null space of operator.
\DeclareMathOperator{\sgn}{sgn}             % Sign of a number.
\DeclareMathOperator{\multideg}{mutlideg}   % Multi-Degree (Graphs).
\DeclareMathOperator{\GCD}{GCD}             % Greatest common denominator.
\DeclareMathOperator{\LM}{LM}               % Leading monomial
\DeclareMathOperator{\LC}{LC}               % Leading coefficient.
\DeclareMathOperator{\LT}{LT}               % Leading term.
\DeclareMathOperator{\LCM}{LCM}             % Least common multiple.
\DeclareMathOperator{\Mon}{Mon}             % Monomial.
\DeclareMathOperator{\Spec}{Spec}           % Spectrum.
\DeclareMathOperator{\proj}{proj}           % Projection.
\DeclareMathOperator{\comp}{comp}           % Component.
\DeclareMathOperator{\sinc}{sinc}           % Sinc function.
\DeclareMathOperator{\Ima}{Im}              % Image of operator.
\DeclareMathOperator{\Prin}{Prin}           % Principal value.
\DeclareMathOperator{\Mod}{mod}             % Modulus.
%------------------------New Commands---------------------------%
\DeclarePairedDelimiter\norm{\lVert}{\rVert}
\DeclarePairedDelimiter\ceil{\lceil}{\rceil}
\DeclarePairedDelimiter\floor{\lfloor}{\rfloor}
\newcommand*\diff{\mathop{}\!\mathrm{d}}
\newcommand*\Diff[1]{\mathop{}\!\mathrm{d^#1}}
\renewcommand{\mod}{\ \Mod}
\renewcommand*{\glstextformat}[1]{\textcolor{RoyalBlue}{#1}}
\renewcommand{\glsnamefont}[1]{\textbf{#1}}
\renewcommand\labelitemii{$\circ$}
\renewcommand\thesubfigure{\arabic{chapter}.\arabic{figure}}
\renewcommand\thesubfigure{%
    \arabic{chapter}.\arabic{figure}.\arabic{subfigure}}
\addto\captionsenglish{\renewcommand{\figurename}{Fig.}}
%------------------------Book Command---------------------------%
\makeatletter
\renewcommand\@pnumwidth{1cm}
\newcounter{book}
\renewcommand\thebook{\@Roman\c@book}
\newcommand\book{%
    \if@openright
        \cleardoublepage
    \else
        \clearpage
    \fi
    \thispagestyle{plain}%
    \if@twocolumn
        \onecolumn
        \@tempswatrue
    \else
        \@tempswafalse
    \fi
    \null\vfil
    \secdef\@book\@sbook
}
\def\@book[#1]#2{%
    \ifnum \c@secnumdepth >-3\relax
        \refstepcounter{book}%
        \addcontentsline{toc}{book}{
            \bookname\ \thebook:\hspace{1em}#1
        }
    \else
        \addcontentsline{toc}{book}{#1}%
    \fi
    \markboth{}{}%
    {\centering
     \interlinepenalty \@M
     \normalfont
     \ifnum \c@secnumdepth >-2\relax
       \huge\bfseries \bookname\nobreakspace\thebook
       \par
       \vskip 20\p@
     \fi
     \Huge \bfseries #2\par}%
    \@endbook}
\def\@sbook#1{%
    {\centering
     \interlinepenalty \@M
     \normalfont
     \Huge \bfseries #1\par}%
    \@endbook}
\def\@endbook{
    \vfil\newpage
        \if@twoside
            \if@openright
                \null
                \thispagestyle{empty}%
                \newpage
            \fi
        \fi
        \if@tempswa
            \twocolumn
        \fi
}
\newcommand*\l@book[2]{%
    \ifnum \c@tocdepth >-2\relax
        \addpenalty{-\@highpenalty}%
        \addvspace{2.25em \@plus\p@}%
        \setlength\@tempdima{3em}%
        \begingroup
            \parindent \z@ \rightskip \@pnumwidth
            \parfillskip -\@pnumwidth
            {
                \leavevmode
                \Large \bfseries #1\hfil \hb@xt@\@pnumwidth{
                    \hss #2
                }
            }
            \par
            \nobreak
            \global\@nobreaktrue
            \everypar{\global\@nobreakfalse\everypar{}}%
        \endgroup
    \fi}
\newcommand\bookname{Book}
\renewcommand{\thebook}{\texorpdfstring{\Numberstring{book}}{book}}
\providecommand*{\toclevel@book}{-2}
\makeatother
\titlecontents{chapter}[0pt]
    {\bfseries}
    {\chaptername\ \thecontentslabel:\quad}
    {}
    {\hfill\contentspage}
\titleformat{\part}[display]
    {\Large\bfseries}
    {\partname\nobreakspace\thepart}
    {0mm}
    {\Huge\bfseries}
    \titlecontents{part}[0pt]
    {\large\bfseries}
    {\partname\ \thecontentslabel: \quad}
    {}
    {\hfill\contentspage}
\newcommand{\MarkRightAngle}[4][.3cm]
    {\coordinate (tempa) at ($(#3)!#1!(#2)$);
     \coordinate (tempb) at ($(#3)!#1!(#4)$);
     \coordinate (tempc) at ($(tempa)!0.5!(tempb)$);%midpoint
     \draw (tempa) -- ($(#3)!2!(tempc)$) -- (tempb);}
%--------------------------LENGTHS------------------------------%
% Spacings for the Table of Contents.
\addtolength{\cftsecnumwidth}{1ex}
\addtolength{\cftsubsecindent}{1ex}
\addtolength{\cftsubsecnumwidth}{1ex}
\addtolength{\cftfignumwidth}{1ex}
\addtolength{\cfttabnumwidth}{1ex}

% Spacing for multi-column and enumerate environments.
\setlength{\multicolsep}{6pt}
\setlist[enumerate]{itemsep=0pt,topsep=3pt}

% Indent and paragraph spacing.
\setlength{\parindent}{0em}
\setlength{\parskip}{0em}                                                           %
%----------------------------Main Document-------------------------------------%
\begin{document}
    \title{CW Complexes are Sequential}
    \author{Ryan Maguire}
    \date{\vspace{-5ex}}
    \maketitle
    \begin{definition}
        A sequential topological space is a topological space $\topspace{X}$
        such that $\mathcal{U}\subseteq{X}$ is open if and only if $\mathcal{U}$
        is sequentially open. That is, for every sequence
        $a:\mathbb{N}\rightarrow{X}$ with a limit $x\in\mathcal{U}$ there exists
        $N\in\mathbb{N}$ such that for all $n>N$ it is true that
        $a_{n}\in\mathcal{U}$.
    \end{definition}
    We say \textit{a} limit since no separation axioms are assumed, hence limits
    need not be unique. Since CW complexes are Hausdorff this does not cause
    concern.
    \begin{theorem}
        Open implies sequentially open.
    \end{theorem}
    \begin{proof}
        By the definition of convergence. If $a_{n}\rightarrow{x}$, then for
        every open $\mathcal{V}$ containing $x$ there is an $N\in\mathbb{N}$
        such that for all $n>N$ it is true that $a_{n}\in\mathcal{V}$. Setting
        $\mathcal{V}=\mathcal{U}$ gives the result.
    \end{proof}
    \begin{theorem}
        Continuous implies sequentially continuous.
    \end{theorem}
    \begin{proof}
        Standard topological result.
    \end{proof}
    \begin{theorem}
        First countable implies sequential.
    \end{theorem}
    \begin{proof}
        Another standard theorem.
    \end{proof}
    \begin{theorem}
        Metrizable spaces are sequential.
    \end{theorem}
    \begin{proof}
        For metrizable implies first countable, which implies sequential.
    \end{proof}
    \begin{theorem}
        If $\topspace[\alpha]{X_{\alpha}}$ are a collection of sequential
        spaces, if $Y=\coprod_{\alpha}X_{\alpha}$, and if $\tau$ is the
        disjoint union topology, then $\topspace{Y}$ is sequential.
    \end{theorem}
    \begin{proof}
        For suppose $\mathcal{V}\subseteq{Y}$ is sequentially open but not open.
        But $\mathcal{V}$ is open if and only if for all $\alpha$, the pre-image
        under the canonical inclusion mapping
        $\iota_{\alpha}:X_{\alpha}\rightarrow{Y}$ is open. That is,
        $\iota_{\alpha}^{\minus{1}}[\mathcal{V}]$ is open. Hence if
        $\mathcal{V}$ is \textit{not} open, then there exists an $\alpha$ such
        such that $\iota_{\alpha}^{\minus{1}}[\mathcal{V}]\subseteq{X}_{\alpha}$
        is not open. Let $\mathcal{U}=\iota_{\alpha}^{\minus{1}}[\mathcal{V}]$.
        But $X_{\alpha}$ is sequential and hence if $\mathcal{U}$ is not open,
        then it is not sequentially open. But then there is a sequence
        $a:\mathbb{N}\rightarrow{X}_{\alpha}$ with a limit $x\in\mathcal{U}$
        such that for all $N\in\mathbb{N}$ there is an $n>N$ with
        $a_{n}\notin\mathcal{U}$. But $\iota_{\alpha}$ is continuous and hence
        $\iota_{\alpha}(a_{n})\rightarrow\iota_{\alpha}(x)$. But
        $\iota_{\alpha}(x)\in\mathcal{V}$ and $\mathcal{V}$ is sequentially
        continuous. Therefore there is an $N\in\mathbb{N}$ such that for all
        $n>N$ it is true that $\iota_{\alpha}(a_{n})\in\mathcal{V}$. But then
        for all $n>N$, $a_{n}\in\mathcal{U}$, a contradiction. Hence,
        $\mathcal{U}$ is open and $\mathcal{V}$ is also open. That is, $Y$ is
        sequential.
    \end{proof}
    \begin{theorem}[Exercise 2.4G of Engelking's Topology]
        If $\topspace[X]{X}$ and $\topspace[Y]{Y}$ are topological spaces, if
        $X$ is sequential, and if $q:X\rightarrow{Y}$ is a quotient map, then
        $Y$ is sequential.
    \end{theorem}
    \begin{proof}
        For let $\mathcal{V}\subseteq{Y}$ be sequentially open. It suffices to
        prove $\mathcal{V}$ is open. Let
        $\mathcal{U}=q^{\minus{1}}[\mathcal{V}]$ and suppose this is not open.
        But $X$ is sequential and hence $\mathcal{U}$ is not sequentially open.
        But then there is a sequence $a:\mathbb{N}\rightarrow{X}$ with a limit
        $x\in\mathcal{U}$ such that for all $N\in\mathbb{N}$ there is an
        $n\in\mathbb{N}$ such that $n>N$ and $a_{n}\notin\mathcal{U}$. But $q$
        is continuous, and hence sequentially continuous, and therefore
        $q(a_{n})\rightarrow{q}(x)$. But $q(x)\in\mathcal{V}$ and $\mathcal{V}$
        is sequentially open, hence there is an $N\in\mathbb{N}$ such that
        for all $n>N$ it is true that $q(a_{n})\in\mathcal{V}$. But then for
        all $n>N$ we have $a_{n}\in\mathcal{U}$, a contradiction. Hence,
        $\mathcal{U}$ is open. But $\mathcal{V}$ is open if and only if
        $q^{\minus{1}}[\mathcal{V}]$ is open, and therefore $\mathcal{V}$ is
        open. That is, $Y$ is sequential.
    \end{proof}
    \begin{theorem}
        If $\topspace{X}$ is a finite dimensional CW complex, then it is
        sequential.
    \end{theorem}
    \begin{proof}
        We prove by induction on the dimension. The base case $n=0$ is trivially
        true since the disjoint union of points is a discrete space which is
        metrizable, and therefore sequential. Suppose it is true for
        $n\in\mathbb{N}$. If $X$ is an $n+1$ dimensional CW complex then it is
        of the form $X^{n}\coprod{X}^{n+1}/\sim$ where $X^{n}$ is an $n$
        dimensional CW complex, $X^{n+1}$ is the disjoint union of $n+1$
        dimensional closed balls, and $\sim$ is the equivalence relation induced
        by the attaching maps of the $n+1$ skeleton into $X^{n}$. But $X^{n}$ is
        sequential by hypothesis, and $X^{n+1}$ is
        the disjoint union of $n+1$ balls, which is metrizable, and hence
        sequential. But the disjoint union of sequential spaces is sequential,
        and hence $X^{n}\coprod{X}^{n+1}$ is sequential. But then
        $X^{n}\coprod{X}^{n+1}/\sim$ is the quotient of a sequential space, and
        is therefore sequential. Therefore, finite dimensional CW complexes are
        sequential.
    \end{proof}
    \begin{theorem}
        If $X$ is a CW complex, then it is sequential.
    \end{theorem}
    \begin{proof}
        If $X$ is a finite dimensional CW complex, then we are done. So suppose
        not and let $\mathcal{U}\subseteq{X}$ be sequentially open. To show that
        it is open suffices to check that $\mathcal{U}\cap{X}^{n}$ is an open
        subset of $X^{n}$ for all $n\in\mathbb{N}$. But $X^{n}$ is sequential,
        and hence if $\mathcal{U}\cap{X}^{n}$ is not open, then it is not
        sequentially open. But then there is a sequence
        $a:\mathbb{N}\rightarrow{X}^{n}$ with limit $x\in\mathcal{U}\cap{X}^{n}$
        such that for all $K\in\mathbb{N}$ there is an $k>K$ such that
        $a_{k}\notin\mathcal{U}\cap{X}^{n}$. But $\mathcal{U}$ is sequential,
        hence if $a_{k}\rightarrow{x}$ and $x\in\mathcal{U}$, then there is a
        $K\in\mathbb{N}$ such that for all $k>K$ it is true that
        $a_{k}\in\mathcal{U}$. But for all $k\in\mathbb{N}$ we have
        $a_{k}\in{X}^{k}$ by definition since $a$ is a sequence in $X^{n}$, and
        hence for all $k>K$ we have $a_{k}\in\mathcal{U}\cap{X}^{n}$, a
        contradiction. Therefore $\mathcal{U}\cap{X}^{n}$ is sequentially open,
        and thus open in $X^{n}$. But this is true of all $n\in\mathbb{N}$, and
        hence $\mathcal{U}$ is open.
    \end{proof}
\end{document}