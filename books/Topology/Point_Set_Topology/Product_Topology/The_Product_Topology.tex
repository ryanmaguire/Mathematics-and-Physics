\section{Product Topology}
    It is common in the literature of mathematics to drop this
    ordered pair notation, and simply call $X$ a topological space.
    To prevent confusion, we will distinguish between the two:
    $X$ is a set, $(X,\tau)$ is a topological space.
    \par\hfill\par
    The notion of a topological space is a generalization of that
    of a \textit{metric space}. We discard all properties of
    metric spaces, with the exception of the fact that open sets
    are closed under arbitrary unions and finite intersections.
    As such, we call the elements of a topology $\tau$ on a set
    $X$ the \textit{open subsets} of $X$. We wish to talk about
    the \textit{product space} formed by the Cartesian product of
    two sets and their respective topologies. We'll need to define
    the \textit{generated} topology, so we prove the following:
    \begin{theorem}
        \label{thm:Intersec_of_Tops_is_Top}%
        If $X$ is a set, and if $T$ is a set of topologies
        on $X$, then:
        \begin{equation}
            \tau=\bigcap_{t\in{T}}t
        \end{equation}
        Is a topology on $X$.
    \end{theorem}
    \begin{proof}
        First note that, since for all $t\in{T}$, $t$ is a topology,
        it is true that $\emptyset\in{t}$, and thus $\emptyset$ is
        contained in the intersection. Therefore $\emptyset\in\tau$.
        Similarly, $X\in\tau$. Given a subset
        $\mathcal{O}\subseteq\tau$, it is true that
        $\mathcal{O}\subseteq{t}$ for all $t\in{T}$. But for all
        $t\in{T}$, $t$ is a topology on $X$, and therefore the union
        of the elements of $\mathcal{O}$ are contained in $t$, and
        thus this union is contained in $\tau$. Thus, $\tau$ is
        closed to arbitrary unions. Similarly for finite
        intersections. Thus, $\tau$ is a topology on $X$.
    \end{proof}
    \begin{ldefinition}{Generated Topology}{Generated_Topology}
        The topology generated by a subset $S\subseteq\mathcal{P}(X)$
        is the set:
        \begin{equation}
            \tau=\bigcap\{\tau_{S}:\tau_{S}
                \textrm{ is a topology on $X$ and }
                S\subseteq\tau_{S}\}
        \end{equation}
        That is, the smallest topology such that the elements of $S$
        are open.
    \end{ldefinition}
    By Thm.~\ref{thm:Intersec_of_Tops_is_Top} we see
    that the topology generated by some collection of subsets of
    $X$ is indeed a topology on $X$. This notion is similar to
    the one found when one studies measure theory. For
    arbitrary topologies it is often difficult, perhaps
    even impossible, to describe explicitly the elements of the
    topology. Analogously, consider the Borel
    $\sigma\textrm{-Algebra}$ on $\mathbb{R}$. We describe
    this as the $\sigma\textrm{-Algebra}$ generated by
    semi-intervals $[a,b)$. An explicit description of the elements
    of the Borel $\sigma\textrm{-Algebra}$ is almost certainly
    impossible. With this, we can move to product spaces.
    \begin{ldefinition}{Product of Two Topologies}
          {Product_of_Two_Topologies}
        The product of two topological spaces $(X,\tau_{X})$ and
        $(Y,\tau_{Y})$ is the topological space $(X\times{Y},\tau)$,
        where $X\times{Y}$ is the Cartesian product of $X$ and $Y$,
        and where $\tau$ is the topology generated by the sets:
        \begin{equation}
            \mathcal{O}=\big\{\mathcal{U}\times\mathcal{V}:
                \mathcal{U}\in\tau_{X},\mathcal{V}\in\tau_{Y}\big\}
        \end{equation}
    \end{ldefinition}
    It is important to note that we cannot simply set the topology
    $\tau$ to be the set of all sets of the form
    $\mathcal{O}=\big\{\mathcal{U}\times\mathcal{V}\big\}$, where
    $\mathcal{U}\in\tau_{X}$ and $\mathcal{V}\in\tau_{Y}$, for
    this will most likely \textbf{not} be a topology.
    The reason being that it may fail to be closed to unions.
    \begin{figure}[H]
        \centering
        \captionsetup{type=figure}
        \begin{subfigure}[b]{0.49\textwidth}
            \centering
            \begin{tikzpicture}[>=Latex]
                \draw[->, thick] (-0.4, 0) to (4, 0)
                    node [above] {$x$};
                \draw[->, thick] (0, -0.4) to (0, 4)
                    node [right] {$y$};
                \draw (1, -0.1) to (1, 0.1);
                \node at (1, -0.4) {$a$};
                \draw (3, -0.1) to (3, 0.1);
                \node at (3, -0.4) {$b$};
                \draw (-0.1, 1) to (0.1, 1);
                \node at (-0.4, 1) {$c$};
                \draw (-0.1, 3) to (0.1, 3);
                \node at (-0.4, 3) {$d$};
                \draw[fill=cyan, opacity=0.8, draw=white]
                    (1, 1) to (1, 3) to (3, 3) to (3, 1) to cycle;
                \draw[densely dashed] (0, 1) to (3, 1);
                \draw[densely dashed] (0, 3) to (3, 3);
                \draw[densely dashed] (1, 0) to (1, 3);
                \draw[densely dashed] (3, 0) to (3, 3);
            \end{tikzpicture}
            \subcaption{The Open Rectangle $(a,b)\times(c,d)$.}
        \end{subfigure}
        \begin{subfigure}[b]{0.49\textwidth}
            \centering
            \begin{tikzpicture}[>=Latex]
                \draw[->, thick] (-0.4, 0) to (4, 0)
                    node [above] {$x$};
                \draw[->, thick] (0, -0.4) to (0, 4)
                    node [right] {$y$};
                \draw[fill=cyan, opacity=0.8, densely dashed]
                    (1, 1) to (2, 1) to (2, 2) to (3, 2)
                           to (3, 3.5) to (1.5, 3.5)
                           to (1.5, 2) to (1, 2) to cycle;
            \end{tikzpicture}
            \subcaption{A Region That Cannot be Written as
                        $\mathcal{U}\times\mathcal{V}$.}
        \end{subfigure}
        \caption{Examples of Open Subsets of $\mathbb{R}^{2}$.}
        \label{fig:Point_Set_Top_Open_Subsets_R2}
    \end{figure}
    For consider $\mathbb{R}$. The standard topology
    on $\mathbb{R}^{2}$ is constructed by considering the
    collection of all open \textit{rectangles},
    $(a,b)\times(c,d)$. However, the set of open rectangles
    will not, by itself, be a topology on $\mathbb{R}^{2}$.
    For one, the union of two rectangles may not even be
    connected: Consider two disjoint non-empty open rectangles.
    This union will \textbf{not} be a rectangle.
    But even if two open rectangles are not disjoint,
    their union may not be a rectangle. See
    Fig.~\ref{fig:Point_Set_Top_Open_Subsets_R2} for examples.
    As a final example, consider the open unit disc in
    $\mathbb{R}^{2}$. This is the set:
    \begin{equation}
        D^{2}=\{(x,y)\in\mathbb{R}^{2}:x^{2}+y^{2}<1\}
    \end{equation}
    This is not of the form $\mathcal{U}\times\mathcal{V}$ for
    some pair of sets $\mathcal{U},\mathcal{V}\subseteq\mathbb{R}$.
    However, seeing as we've called it the open unit disc, we
    would certainly like it to be open. And indeed it is, for
    it lies in the topology that is \textit{generated}
    by open rectangles.
    \begin{figure}[H]
        \centering
        \captionsetup{type=figure}
        \begin{tikzpicture}[>=Latex]
            \draw[<->, thick] (-3.3, 0) to (3.3, 0) node [above] {$x$};
            \draw[<->, thick] (0, -3.3) to (0, 3.3) node [right] {$y$};
            \draw[densely dashed] (0, 0) circle (1in);
        
            % First Layer
            \draw[fill=cyan, opacity=0.6, densely dashed]
                (0.7071in, 0.7071in) to (-0.7071in, 0.7071in)
                                     to (-0.7071in, -0.7071in)
                                     to (0.7071in, -0.7071in)
                                     to cycle;
            
            % Second Layer
            \draw[fill=green, opacity=0.5, densely dashed]
                (0.68in, 0.3535in) to (0.935in, 0.3535in)
                                   to (0.935in, -0.3535in)
                                   to (0.68in, -0.3535in)
                                   to cycle;
            \draw[fill=green, opacity=0.5, densely dashed]
                (-0.68in, 0.3535in) to (-0.935in, 0.3535in)
                                    to (-0.935in, -0.3535in)
                                    to (-0.68in, -0.3535in)
                                    to cycle;
            \draw[fill=green, opacity=0.5, densely dashed]
                (0.3535in, 0.68in) to (0.3535in, 0.935in)
                                   to (-0.3535in, 0.935in)
                                   to (-0.3535in, 0.68in)
                                   to cycle;
            \draw[fill=green, opacity=0.5, densely dashed]
                (0.3535in, -0.68in) to (0.3535in, -0.935in)
                                    to (-0.3535in, -0.935in)
                                    to (-0.3535in, -0.68in)
                                    to cycle;
        
            % Third Layer.
            \draw[fill=orange, opacity=0.6, densely dashed]
                (0.68in, 0.3535in) to (0.8212in, 0.3535in)
                                   to (0.8212in, 0.5705in)
                                   to (0.68in, 0.5707in)
                                   to cycle;
            \draw[fill=orange, opacity=0.6, densely dashed]
                (0.68in, -0.3535in) to (0.8212in, -0.3535in)
                                    to (0.8212in, -0.5705in)
                                    to (0.68in, -0.5707in)
                                    to cycle;
            \draw[fill=orange, opacity=0.6, densely dashed]
                (-0.68in, -0.3535in) to (-0.8212in, -0.3535in)
                                     to (-0.8212in, -0.5705in)
                                     to (-0.68in, -0.5707in)
                                     to cycle;
            \draw[fill=orange, opacity=0.6, densely dashed]
                (-0.68in, 0.3535in) to (-0.8212in, 0.3535in)
                                    to (-0.8212in, 0.5705in)
                                    to (-0.68in, 0.5707in)
                                    to cycle;
            \draw[fill=orange, opacity=0.6, densely dashed]
                (0.3535in, 0.68in) to (0.3535in, 0.8212in)
                                   to (0.5705in, 0.8212in)
                                   to (0.5707in, 0.68in)
                                   to cycle;
            \draw[fill=orange, opacity=0.6, densely dashed]
                (0.3535in, -0.68in) to (0.3535in, -0.8212in)
                                    to (0.5705in, -0.8212in)
                                    to (0.5707in, -0.68in)
                                    to cycle;
            \draw[fill=orange, opacity=0.6, densely dashed]
                (-0.3535in, 0.68in) to (-0.3535in, 0.8212in)
                                    to (-0.5705in, 0.8212in)
                                    to (-0.5707in, 0.68in)
                                    to cycle;
            \draw[fill=orange, opacity=0.6, densely dashed]
                (-0.3535in, -0.68in) to (-0.3535in, -0.8212in)
                                     to (-0.5705in, -0.8212in)
                                     to (-0.5707in, -0.68in)
                                     to cycle;
        
            % Fourth Layer
            \draw[fill=red, opacity=0.5, densely dashed]
                (0.2in, 0.93in) to (0.2in, 0.9797in)
                                to (-0.2in, 0.9797in)
                                to (-0.2in, 0.93in)
                                to cycle;
            \draw[fill=red, opacity=0.5, densely dashed]
                (0.2in, -0.93in) to (0.2in, -0.9797in)
                                 to (-0.2in, -0.9797in)
                                 to (-0.2in, -0.93in)
                                 to cycle;
            \draw[fill=red, opacity=0.5, densely dashed]
                (0.93in, 0.2in) to (0.9797in, 0.2in)
                                to (0.9797in, -0.2in)
                                to (0.93in, -0.2in)
                                to cycle;
            \draw[fill=red, opacity=0.5, densely dashed]
                (-0.93in, 0.2in) to (-0.9797in, 0.2in)
                                 to (-0.9797in, -0.2in)
                                 to (-0.93in, -0.2in)
                                 to cycle;
        
            % Fifth Layer
            \draw[fill=blue, opacity=0.6, densely dashed]
                (0.82in, 0.3535in) to (0.8781in, 0.3535in)
                                   to (0.8781in, 0.4784in)
                                   to (0.82in, 0.4784in)
                                   to cycle;
            \draw[fill=blue, opacity=0.6, densely dashed]
                (0.82in, -0.3535in) to (0.8781in, -0.3535in)
                                    to (0.8781in, -0.4784in)
                                    to (0.82in, -0.4784in)
                                    to cycle;
            \draw[fill=blue, opacity=0.6, densely dashed]
                (-0.82in, 0.3535in) to (-0.8781in, 0.3535in)
                                    to (-0.8781in, 0.4784in)
                                    to (-0.82in, 0.4784in)
                                    to cycle;
            \draw[fill=blue, opacity=0.6, densely dashed]
                (-0.82in, -0.3535in) to (-0.8781in, -0.3535in)
                                     to (-0.8781in, -0.4784in)
                                     to (-0.82in, -0.4784in)
                                     to cycle;
            \draw[fill=blue, opacity=0.6, densely dashed]
                (0.3535in, 0.82in) to (0.3535in, 0.8781in)
                                   to (0.4784in, 0.8781in)
                                   to (0.4784in, 0.82in)
                                   to cycle;
            \draw[fill=blue, opacity=0.6, densely dashed]
                (0.3535in, -0.82in) to (0.3535in, -0.8781in)
                                    to (0.4784in, -0.8781in)
                                    to (0.4784in, -0.82in)
                                    to cycle;
            \draw[fill=blue, opacity=0.6, densely dashed]
                (-0.3535in, -0.82in) to (-0.3535in, -0.8781in)
                                     to (-0.4784in, -0.8781in)
                                     to (-0.4784in, -0.82in)
                                     to cycle;
            \draw[fill=blue, opacity=0.6, densely dashed]
                (-0.3535in, 0.82in) to (-0.3535in, 0.8781in)
                                    to (-0.4784in, 0.8781in)
                                    to (-0.4784in, 0.82in)
                                    to cycle;
        
            % Sixth Layer
            \draw[fill=yellow, opacity=0.6, densely dashed]
                (0.68in, 0.5705in) to (0.7641in, 0.5705in)
                                   to (0.7641in, 0.645in)
                                   to (0.68in, 0.645in)
                                   to cycle;
            \draw[fill=yellow, opacity=0.6, densely dashed]
                (0.68in, -0.5705in) to (0.7641in, -0.5705in)
                                    to (0.7641in, -0.645in)
                                    to (0.68in, -0.645in)
                                    to cycle;
            \draw[fill=yellow, opacity=0.6, densely dashed]
                (-0.68in, -0.5705in) to (-0.7641in, -0.5705in)
                                     to (-0.7641in, -0.645in)
                                     to (-0.68in, -0.645in)
                                     to cycle;
            \draw[fill=yellow, opacity=0.6, densely dashed]
                (-0.68in, 0.5705in) to (-0.7641in, 0.5705in)
                                    to (-0.7641in, 0.645in)
                                    to (-0.68in, 0.645in)
                                    to cycle;
            \draw[fill=yellow, opacity=0.6, densely dashed]
                (0.5705in, 0.68in) to (0.5705in, 0.7641in)
                                   to (0.645in, 0.7641in)
                                   to (0.645in, 0.68in)
                                   to cycle;
            \draw[fill=yellow, opacity=0.6, densely dashed]
                (0.5705in, -0.68in) to (0.5705in, -0.7641in)
                                    to (0.645in, -0.7641in)
                                    to (0.645in, -0.68in)
                                    to cycle;
            \draw[fill=yellow, opacity=0.6, densely dashed]
                (-0.5705in, -0.68in) to (-0.5705in, -0.7641in)
                                     to (-0.645in, -0.7641in)
                                     to (-0.645in, -0.68in)
                                     to cycle;
            \draw[fill=yellow, opacity=0.6, densely dashed]
                (-0.5705in, 0.68in) to (-0.5705in, 0.7641in)
                                    to (-0.645in, 0.7641in)
                                    to (-0.645in, 0.68in)
                                    to cycle;
        \end{tikzpicture}
        \caption{Tiling of the Open Unit Disc by Rectangles.}
        \label{fig:Point_Set_Top_Unit_Disc_Rect_Tiling}
    \end{figure}
    Note that, in the tiling of the unit disc shown in
    Fig.~\ref{fig:Point_Set_Top_Unit_Disc_Rect_Tiling}, many of
    the rectangles overlap. This is to avoid excluding any points
    within the circle, and to give a clear picture. Such a tiling
    is allowed in the topology generated by open rectangles, since
    \textit{arbitrary} unions are allowed. With this figure we have
    some evidence that the topology generated by open rectangles is
    most likely the same as the standard topology on $\mathbb{R}^{2}$.
    That is: The set of all sets $\mathcal{U}\subseteq\mathbb{R}^{2}$
    such that, for all $\mathbf{x}\in\mathcal{U}$, there is an $r>0$
    such that, for all $\mathbf{y}\in\mathbb{R}^{2}$ such that
    $\norm{\mathbf{x}-\mathbf{y}}_{2}<r$, it is true that
    $\mathbf{y}\in\mathcal{U}$. Here $\norm{\cdot}_{2}$ denotes the
    standard Euclidean norm, where we compute length by
    invoking the Pythagorean formula. Rather than carrying out
    complicated computations, we can simply note that open
    balls in the $\norm{\cdot}_{2}$ norm are of the form:
    \begin{equation}
        B_{r}^{(\mathbb{R}^{2},\norm{\cdot}_{2})}(\mathbf{x})
            =\big\{\mathbf{y}\in\mathbb{R}^{2}:
                \norm{\mathbf{x}-\mathbf{y}}_{2}<r\big\}
    \end{equation}
    These are circles centered at $\mathbf{x}$.
    Contrast that with open balls in the $\norm{\cdot}_{\infty}$
    metric:
    \begin{equation}
        B_{r}^{(\mathbb{R}^{2},\norm{\cdot}_{\infty})}(\mathbf{x})
            =\big\{\mathbf{y}\in\mathbb{R}^{2}:
                \max\{|x_{1}-y_{1}|,|x_{2}-y_{2}|\}<r\big\}
    \end{equation}
    These are just squares centered at $\mathbf{x}$. And we know
    that the $\norm{\cdot}_{2}$ and $\norm{\cdot}_{\infty}$ metrics
    are equivalent, so these topologies must be the same. Thus
    we've found a slightly more inconvenient way of describing
    the topology on $\mathbb{R}^{2}$. The plus side is that
    this alternative notion generalizes to $X\times{Y}$
    when $(X,\tau_{X})$ and $(Y,\tau_{Y})$ are more general
    topological spaces.
    \par\hfill\par
    In defining the product topology of two topological spaces,
    we used the familiar notion of a Cartesian product. Elements
    of the Cartesian product $X\times{Y}$ are ordered pairs
    $(x,y)$, where $x\in{X}$ and $y\in{Y}$. We can continue to
    ordered triples $(x,y,z)$ and the general $n$ tuple
    $(x_{1},\dots,x_{n})$ and similarly define the product
    topology of $n$ topological spaces
    $(X_{1},\tau_{1}),\dots(X_{n},\tau_{n})$.
    But what if we wanted to define an \textit{infinite} product
    of infinitely many spaces? If the product is countable, we
    have some intuition for we can think of \textit{infinite} tuples
    $(x_{1},\dots,x_{n},\dots)$, but this lacks clarity.
    Rather, let's go back to the product of two topological
    spaces and redefine it. Let $\mathbb{Z}_{2}=\{1,2\}$
    and define:
    \begin{equation}
        \prod_{i=1}^{2}X_{i}=\{f:\mathbb{Z}_{2}\rightarrow
            \bigcup_{k=1}^{2}X_{k}:
            f(1)\in{X}_{1},f(2)\in{X}_{2}\}
    \end{equation}
    That is, the set of all functions from $\mathbb{Z}_{2}$ into
    $X_{1}\cup{X}_{2}$ with the property that 1 maps into $X_{1}$
    and $2$ maps into $X_{2}$. There is a clear bijection between
    this new thing and $X_{1}\times{X}_{2}$, simply map
    $(x,y)$ to the function $f$, where $f(1)=x$ and $f(2)=y$.
    But now we have a definition that really didn't depend
    on how many products we were making. Let
    $\mathbb{Z}_{n}=\{1,\dots,n\}$, and let $X_{1},\dots,X_{n}$
    be sets. We can then define:
    \begin{equation}
        \prod_{i\in\mathbb{Z}_{n}}X_{i}=
            \{f:\mathbb{Z}_{n}\rightarrow
                \bigcup_{k\in\mathbb{Z}_{n}}X_{i}:
                \forall_{i\in\mathbb{Z}_{n}},f(i)\in{X}_{i}\}
    \end{equation}
    And we can go further, defining the product for any collection
    of sets. Let's first introduce some notation. An indexing set
    for a collection of sets is some set $I$ such that we can
    write all of the sets in our collection as $X_{i}$, for
    $i\in{I}$. To improve rigor, let's say that an indexing set
    for a collection of sets $\mathcal{O}$ is some set $I$ such
    that there is a surjective function $X:I\rightarrow\mathcal{O}$,
    and let's write $X(i)=X_{i}$, for all $i\in{I}$. That is,
    for all $i\in{I}$, $X_{i}$ is a set in $\mathcal{O}$.
    We can now define the general product of sets.
    \begin{ldefinition}{Product of Sets}{Product_Set}
        The product of a collection of sets indexed by a set $I$
        is the set:
        \begin{equation}
            \prod_{i\in{I}}X_{i}=
            \{f:I\rightarrow\bigcup_{i\in{I}}X_{i}:
                \forall_{i\in{I}},f(i)\in{X}_{i}\}
        \end{equation}
    \end{ldefinition}
    This notion is well-defined for arbitrary products, countable
    or not. It is important to note that the elements of the
    product space are \textit{functions}.
    \begin{lexample}
        Nothing in the definition of an indexing set requires
        $\mathcal{O}$ to contain many sets, so let
        $\mathcal{O}=\{\mathbb{R}\}$ and let $I=\mathbb{N}$.
        Then the product is simply:
        \begin{equation}
            \prod_{n\in\mathbb{N}}\mathbb{R}=
                \{a:\mathbb{N}\rightarrow\mathbb{R}\}
        \end{equation}
        That is, the set of all sequences of real numbers.
        Thus, the countable product of $\mathbb{R}$ can be
        thought of in two ways: The set of all
        \textit{infinite} tuples
        $(x_{1},\dots,x_{n},\dots)$, or the set of all
        \textit{sequences} of real numbers.
    \end{lexample}
    All of this has been purely set theoretic: There is no topology
    yet. Given a collection of topological spaces
    $(X_{i},\tau_{i})$, is there a good topology to place on the
    product? That is, can we form a nice product topological space?
    There are two well established ways to do this:
    The \textit{obvious} way, and the \textit{correct} way.
    We first let intuition lead us astray, and define the obvious
    answer: The Box Topology.
    \begin{ldefinition}{Box Topology}{Box_Topology}
        The box topology on a collection of topological spaces
        $(X_{i},\tau_{i})$ indexed by a set $I$ is the topology
        $\tau$ on the set $X$ where:
        \begin{equation}
            X=\prod_{i\in{I}}X_{i}
        \end{equation}
        And where $\tau$ is the topology generated by the sets:
        \begin{equation}
            \mathcal{U}=
                \big\{\prod_{i\in{I}}\mathcal{U}_{i}:
                    \mathcal{U}_{i}\in\tau_{i}\big\}
        \end{equation}
        That is, $\tau$ is generated by all of the open sets
        in all of the $X_{i}$.
    \end{ldefinition}
    This is precisely what we did for $\mathbb{R}^{2}$.
    We took the topology to be the one generated by all of the open
    rectangles in the plane. Unfortunately, when the product is
    infinite, the box topology is horrible. Some problems with
    the box topology:
    \begin{enumerate}
        \item The product of compact spaces need not be compact.
        \item The product of connected spaces need not be connected.
        \item The product of metric spaces need not be metrizable.
    \end{enumerate}
    Moreover, some functions that \textit{look} continuous, and that
    we would obviously want to be continuous, are not.
    For example, let $X$ be the set of sequences in $\mathbb{R}$,
    and define $f:\mathbb{R}\rightarrow{X}$ by mapping
    $x$ to the sequence $a_{n}=x$, $n\in\mathbb{N}$. That is:
    \begin{equation}
        f(x)=x,x,x,x,\dots,x,x,\dots
    \end{equation}
    This function is \textit{nowhere} continuous in the box topology.
    So now we devise a plan to make a \textit{better} topology
    with the following property:
    Suppose $g:\mathbb{R}\rightarrow{X}$,
    where $X$ is again the space of real-valued sequences, and
    suppose $g$ is of the form:
    \begin{equation}
        g(x)=g_{1}(x),g_{2}(x),\dots,g_{n}(x),\dots
    \end{equation}
    Where $g_{k}$ is continuous for all $k\in\mathbb{N}$. We
    \textbf{require} that $g$ be continuous in the product space.
    We could simply make $\tau$ be the chaotic topology,
    $\tau=\{\emptyset,X\}$, but then \textit{every} function
    $f:A\rightarrow{X}$ is continuous, for \textit{any}
    topological space $(A,\tau_{A})$, and this is rather boring.
    So we try another approach. Given a collection of topological
    spaces $(X_{i},\tau_{i})$, we require that the product space
    $(X,\tau)$ is such that all of the projection mappings
    $p_{i}:X\rightarrow{X}_{i}$ are continuous. The projection
    mappings can be defined set theoretically using the notation
    we've developed. Given a product $X$ of sets $X_{i}$,
    the projection mapping $p_{i}:X\rightarrow{X}_{i}$ is simply:
    \begin{equation}
        p_{i}(x)=x(i)
    \end{equation}
    This looks strange, but remember that
    we've defined the product space to be a set of
    \textit{functions}, and therefore $x\in{X}$ is a function.
    Thus, the $i^{th}$ projection mapping simply
    evaluates these functions in the $i^{th}$ coordinate.
    \par\hfill\par
    In the search for a topology on the product set $X$ that makes
    all of the projection mappings $p_{i}$ continuous, we could
    simply take $\tau=\mathcal{P}(X)$. Then, for \textit{any}
    topological space $(A,\tau_{A})$, and for \textit{every}
    function $f:X\rightarrow{A}$, $f$ is continuous.
    This is overkill and we see that this is larger than the
    box topology. So all of the problems with the box topology still
    exist! So, we require that $\tau$ is the \textit{smallest}
    such topology. We now define the initial topology.
    \begin{ldefinition}{Initial Topology}{Initial_Topology}
        The initial topology on a set $X$ generated by
        a set of functions $f_{i}$ from $X$ to topological
        spaces $(X_{i},\tau_{i})$ is the set:
        \begin{equation}
            \tau=\bigcap\big\{\tau_{X}:
                \tau_{X}\textrm{ is a topology on $X$ and }
                \forall_{i\in{I}},f_{i}
                \textrm{ is continuous.}\big\}
        \end{equation}
    \end{ldefinition}
    This collection is non-empty, since $\mathcal{P}(X)$ is contained
    in it, and by Thm.~\ref{thm:Intersec_of_Tops_is_Top},
    $\tau$ is a topology on $X$. We now define the product topology.
    \begin{ldefinition}{Product Topology}
        The product topology on a set $X$ defined as the
        product of topological spaces $(X_{i},\tau_{i})$
        indexed over $I$:
        \begin{equation}
            X=\prod_{i\in{I}}X_{i}
        \end{equation}
        Is the initial topology defined by the set:
        \begin{equation}
            \mathscr{F}=
            \{p_{i}:X\rightarrow{X}_{i},p_{i}(x)=x(i)\}
        \end{equation}
        That is, the set of projection mappings.
    \end{ldefinition}
    In this construction one might have noted that the
    projection mappings are continuous in the box topology.
    Thus one might very reasonably ask if the product topology
    and the box topology are the same thing. And indeed, for
    a \textit{finite} product, they are! This makes sense, for in
    $\mathbb{R}^{2}$ we can think of the topology generated by
    rectangles, or the topology generated by the projection mappings,
    and they are the same. What's crucial is that they differ for
    infinite products.
    \begin{ltheorem}{Product Topology Basis Theorem}
        If $(X,\tau)$ is the product topological space formed by
        the topological spaces $(X_{i},\tau_{i})$, indexed by
        a set $I$, then:
        \begin{equation}
            \tau=\bigcup\prod_{i\in{I}}
                \big\{\mathcal{U}_{i}\in\tau_{i}:
                    \mathcal{U}_{i}=X_{i}
                    \textrm{ for all but finitely many sets.}
                \big\}
        \end{equation}
        That is, $\tau$ is the set of all products of open sets
        $\mathcal{U}_{i}$, such that all but finitely many of
        the $\mathcal{U}_{i}$ are the entire space $X_{i}$.
    \end{ltheorem}
    \par\hfill\par
    This seems confusing, so we illustrate with some pictures. What's
    important to note is that, if the product is infinite, then
    the box topology and the product topology differ. To see this,
    in the box topology we allowed \textit{all} products of open
    sets, whereas now we only allow the product of open sets
    $\mathcal{U}_{i}$ where, for all but finitely many $i$, we
    have $\mathcal{U}_{i}=X_{i}$. It should then be clear that,
    if $\tau_{B}$ is the box topology, and $\tau_{P}$ is the
    product topology, then $\tau_{P}\subseteq\tau_{B}$, and for
    infinite products $\tau_{P}$ is a proper subset.
    \par\hfill\par
    Let's dumb down the theorem a bit, and imagine again a world
    where $X=Y=\mathbb{R}$. Let's consider the topology generated
    by sets $\mathcal{U}\times\mathcal{V}$, where $\mathcal{U}$
    is an open subset of $\mathbb{R}$, and
    $\mathcal{V}=\mathbb{R}$. That is, rather than allowing
    the product to be over finitely many arbitrary open sets,
    we allow it to be over one, and it must be in the $x$ axis.
    In doing this we can get a sense of what the product topology
    might look like.
    \begin{figure}[H]
        \centering
        \captionsetup{type=figure}
        \begin{subfigure}[b]{0.49\textwidth}
            \centering
            \begin{tikzpicture}[>=Latex]
                \draw[->, thick] (-0.2, 0) to (5, 0) node [above] {$x$};
                \draw[->, thick] (0, -1) to (0, 5) node [right] {$y$};
                \draw[fill=cyan, opacity=0.5, draw=white]
                    (2, -1) to (2, 5) to (4, 5) to (4, -1) to cycle;
                \draw (2, -0.1) to (2, 0.1);
                \node at (2, -0.4) [left] {$a$};
                \draw (4, -0.1) to (4, 0.1);
                \node at (4, -0.4) [right] {$b$};
                \draw[densely dashed] (2, -1) to (2, 5);
                \draw[densely dashed] (4, -1) to (4, 5);
            \end{tikzpicture}
            \subcaption{Sets of the Form $\mathcal{U}\times\mathbb{R}$.}
        \end{subfigure}
        \begin{subfigure}[b]{0.49\textwidth}
            \begin{tikzpicture}[>=Latex]
                \draw[->, thick] (-1, 0) to (4, 0) node [above] {$x$};
                \draw[->, thick] (0, -1) to (0, 5) node [right] {$y$};
                \draw[fill=cyan, opacity=0.5, draw=white]
                    (-1, 2) to (4, 2) to (4, 4) to (-1, 4) to cycle;
                \draw (-0.2, 2) to (0.1, 2);
                \node at (-0.4, 2) [above] {$c$};
                \draw (-0.1, 4) to (0.1, 4);
                \node at (-0.4, 4) [below] {$d$};
                \draw[densely dashed] (-1, 2) to (4, 2);
                \draw[densely dashed] (-1, 4) to (4, 4);
            \end{tikzpicture}
            \subcaption{Sets of the Form $\mathbb{R}\times\mathcal{V}$.}
        \end{subfigure}
        \caption{Strips in the Plane.}
        \label{fig:Point_Set_Topology_Strips_in_R2}
    \end{figure}
    We can do the same thing and consider sets of the form
    $\mathbb{R}\times\mathcal{V}$, where $\mathcal{V}$ is open
    in $\mathbb{R}$. Recall that open sets in $\mathbb{R}$ are
    intervals and arbitrary collections of intervals. Using this, we
    see that the topology generated by $\mathcal{U}\times\mathbb{R}$
    is the collection of all open vertical \textit{strips},
    and $\mathbb{R}\times\mathcal{V}$ form the horizontal strips.
    See Fig.~\ref{fig:Point_Set_Topology_Strips_in_R2}.
    We expand this game to $\mathbb{R}^{3}$, and think of sets
    of the form $\mathcal{U}\times\mathcal{V}\times\mathbb{R}$,
    or any permutation of the three coordinates.
    See Fig.~\ref{fig:Point_Set_Top_Blocks_in_R3}
    \begin{figure}[H]
        \centering
        \captionsetup{type=figure}
        \begin{tikzpicture}[>=Latex]
            \draw[->, thick] (0, 0, 0) to (4, 0, 0) node [above] {$x$};
            \draw[->, thick] (0, 0, 0) to (0, 4, 0) node [right] {$y$};
            \draw[->, thick] (0, 0, 0) to (0, 0, 8) node [left] {$z$};
            \node at (1.4, -0.15, 0) {$a$};
            \node at (3.6, -0.15, 0) {$b$};
            \node at (0, 1, 0.4) {$c$};
            \node at (0, 3, 0.4) {$d$};
            \draw[densely dashed] (1.5, 0, 0) to (1.5, 3, 0);
            \draw[densely dashed] (3.5, 0, 0) to (3.5, 3, 0);
            \draw[densely dashed] (0, 1, 0) to (3.5, 1, 0);
            \draw[densely dashed] (0, 3, 0) to (3.5, 3, 0);
            \draw[densely dashed] (1.5, 3, 0) to (1.5, 3, 6);
            \draw[densely dashed] (3.5, 3, 0) to (3.5, 3, 6);
            \draw[densely dashed] (1.5, 1, 0) to (1.5, 1, 6);
            \draw[densely dashed] (3.5, 1, 0) to (3.5, 1, 6);
            \draw[densely dashed] (1.5, 3, 6) to (1.5, 0, 6);
            \draw[densely dashed] (3.5, 3, 6) to (3.5, 0, 6);
            \draw[densely dashed] (0, 1, 6) to (3.5, 1, 6);
            \draw[densely dashed] (0, 3, 6) to (3.5, 3, 6);
            \draw[densely dashed] (0, 0, 6) to (4, 0, 6);
            \draw[densely dashed] (3.5, 0, 0) to (3.5, 0, 8);
            \draw[densely dashed] (0, 0, 6) to (0, 4, 6);
            \draw[fill=cyan, opacity=0.5, draw=none]
                (1.5, 1, 6) to (3.5, 1, 6)
                            to (3.5, 3, 6)
                            to (1.5, 3, 6)
                            to cycle;
            \draw[fill=cyan, opacity=0.5, draw=none]
                (3.5, 1, 6) to (3.5, 3, 6)
                            to (3.5, 3, 0)
                            to (3.5, 1, 0)
                            to cycle;
            \draw[fill=cyan, opacity=0.5, draw=none]
                (1.5, 3, 6) to (1.5, 3, 0)
                            to (3.5, 3, 0)
                            to (3.5, 3, 6)
                            to cycle;
        \end{tikzpicture}
        \caption{Blocks in Space.}
        \label{fig:Point_Set_Top_Blocks_in_R3}
    \end{figure}
    The product topology has the following wonderful features:
    \begin{enumerate}
        \item The product of compact topological spaces is compact.
        \item The product of connected spaces is connected.
        \item The product of metric spaces is metrizable.
    \end{enumerate}