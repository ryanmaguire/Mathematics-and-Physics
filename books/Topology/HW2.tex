%------------------------------------------------------------------------------%
\documentclass{article}                                                        %
%------------------------------Preamble----------------------------------------%
\makeatletter                                                                  %
    \def\input@path{{../../}}                                                  %
\makeatother                                                                   %
%---------------------------Packages----------------------------%
\usepackage{geometry}
\geometry{b5paper, margin=1.0in}
\usepackage[T1]{fontenc}
\usepackage{graphicx, float}            % Graphics/Images.
\usepackage{natbib}                     % For bibliographies.
\bibliographystyle{agsm}                % Bibliography style.
\usepackage[french, english]{babel}     % Language typesetting.
\usepackage[dvipsnames]{xcolor}         % Color names.
\usepackage{listings, lstlinebgrd}      % Verbatim-Like Tools.
\usepackage{mathtools, esint, mathrsfs} % amsmath and integrals.
\usepackage{amsthm, amsfonts}           % Fonts and theorems.
\usepackage{tabularx}
\usepackage{tcolorbox}                  % Frames around theorems.
\usepackage{upgreek}                    % Non-Italic Greek.
\usepackage{paracol}                    % Two-column styling.
\usepackage{wrapfig}                    % Wrap text around figure.
\usepackage{fmtcount, etoolbox}         % For the \book{} command.
\usepackage[newparttoc]{titlesec}       % Formatting chapter, etc.
\usepackage{titletoc}                   % Allows \book in toc.
\usepackage[nottoc]{tocbibind}          % Bibliography in toc.
\usepackage[titles]{tocloft}            % ToC formatting.
\usepackage{multicol, enumitem}         % Multi-column/enumerate.
\usepackage{import}                     % Import external files.
\usepackage{pgfplots, tikz}             % Drawing/graphing tools.
\usetikzlibrary{
    calc,                   % Calculating right angles and more.
    angles,                 % Drawing angles within triangles.
    arrows.meta,            % Latex and Stealth arrows.
    quotes,                 % Adding labels to angles.
    positioning,            % Relative positioning of nodes.
    decorations.markings,   % Adding arrows in the middle of a line.
    patterns,
    arrows,
    shapes,
    shapes.geometric,
    cd,
    hobby,
    babel
}                                       % Libraries for tikz.
\pgfplotsset{compat=1.9}                % Version of pgfplots.
\usepackage[font=scriptsize,
            labelformat=simple,
            labelsep=colon]{subcaption} % Subfigure captions.
\usepackage[font={scriptsize},
            hypcap=true,
            labelsep=colon]{caption}    % Figure captions.
\usepackage{hyperref}                   % Allows for hyperlinks.
\hypersetup{
    colorlinks=true,
    linkcolor=blue,
    filecolor=magenta,
    urlcolor=Cerulean,
    citecolor=SkyBlue
}                           % Colors for hyperref.
\usepackage[toc,acronym,nogroupskip]{glossaries} % Glossaries and acronyms.
\usepackage[subpreambles=false]{standalone}      % Complileable sub files.

% Various font stuff from kiwi.
% Use this for Times text and Computer Modern math
%\usepackage{times}

% Quite nice
%\usepackage[charter, greekfamily=, greekuppercase=italicized]{mathdesign}
%\usepackage[utopia, greekuppercase=italicized]{mathdesign}    % Math is narrower

% Use this for Times text and math
%\usepackage{newtxtext}
%\usepackage[libertine,cmintegrals]{newtxmath}
%\usepackage{fix-cm}

%\usepackage{txfontsb}
% or
%\usepackage{mathptmx}

%\usepackage[scaled=0.92]{helvet}
%\renewcommand{\rmdefault}{ptm}

%\usepackage{mathpazo}    % add possibly `sc` and `osf` options
%\usepackage{eulervm}

%\usepackage{fourier}
%\renewcommand{\rmdefault}{ptm}
%\usepackage{mathptm}

%\usepackage{fontspec}
%\setmainfont{lmodern}

%\usepackage[varg]{txfonts}
%\usepackage{fouriernc}
%\usepackage{mathpazo}

%\usepackage{bookman}
%\usepackage[scaled]{uarial}
%\usepackage[scaled]{helvet}
%\renewcommand*\familydefault{\sfdefault}
%\usepackage[math]{anttor}

%\newcommand\fgeorgia{\fontfamily{jvn}\selectfont}
%\newcommand\ftimes{\fontfamily{ptm}\selectfont}
%\newcommand\fhelvetica{\fontfamily{phv}\selectfont}
%\newcommand\fcourier{\fontfamily{pcr}\selectfont}
%\newcommand\fbookman{\fontfamily{pbk}\selectfont}
%\newcommand\fnewcentury{\fontfamily{pnc}\selectfont}
%\newcommand\fpalatino{\fontfamily{ppl}\selectfont}
%\newcommand\favantgarde{\fontfamily{pag}\selectfont}
%\newcommand\fnormal{\normalfont}
%\newcommand\fsize[1]{\ifnum#1>0\fontsize{#1}{#1}\selectfont\else\normalsize\fi}
%------------------------Theorem Styles-------------------------%
% Define theorem style for default spacing and normal font.
\newtheoremstyle{normal}
    {\topsep}               % Amount of space above the theorem.
    {\topsep}               % Amount of space below the theorem.
    {}                      % Font used for body of theorem.
    {}                      % Measure of space to indent.
    {\bfseries}             % Font of the header of the theorem.
    {}                      % Punctuation between head and body.
    {.5em}                  % Space after theorem head.
    {}

% Define theorem style for default spacing with italicized font.
\newtheoremstyle{normalit}{\topsep}{\topsep}
                {\itshape}{}{\bfseries}{}{.5em}{}

% Italic header environment.
\newtheoremstyle{thmit}{\topsep}{\topsep}{}{}{\itshape}{}{0.5em}{}

% Define italicized environments.
\theoremstyle{normalit}
\newtheorem{theorem}{Theorem}[section]
\newtheorem{lemma}{Lemma}[section]
\newtheorem{corollary}{Corollary}[section]
\newtheorem{proposition}{Proposition}[section]
\newtheorem*{theorem*}{Theorem}

% Define environments with italic headers.
\theoremstyle{thmit}
\newtheorem*{solution}{Solution}
\newtheorem*{fsolution}{Solution}

% Define default environments.
\theoremstyle{normal}
\newtheorem{example}{Example}[section]
\newtheorem{definition}{Definition}[section]
\newtheorem{problem}{Problem}[section]
\newtheorem{question}{Question}[section]
\newtheorem{remark}{Remark}[section]
\newtheorem{properties}{Properties}[section]
\newtheorem{notation}{Notation}[section]
\newtheorem{axiom}{Axiom}[section]
\newtheorem*{properties*}{Properties}
\newtheorem*{remark*}{Remark}
\newtheorem*{definition*}{Definition}
\theoremstyle{plain}

% Define framed environment.
\tcbuselibrary{most}
\newtcbtheorem[use counter*=theorem]{ftheorem}{Theorem}%
    {colback=green!5,colframe=green!35!black,
     fonttitle=\bfseries\upshape}{th}

\newtcbtheorem[use counter*=example]{fdefinition}{Definition}%
    {fonttitle=\bfseries\upshape,
     colback=blue!5!white,colframe=blue!75!black}{def}

\newtcbtheorem[use counter*=example]{fexample}{Example}%
    {fonttitle=\bfseries\upshape,
     colback=red!5!white,colframe=red!75!black}{ex}

\newtcbtheorem[use counter*=notation]{fnotation}{Notation}%
    {fonttitle=\bfseries\upshape,
     colback=SeaGreen!5!white,colframe=SeaGreen!75!black}{ex}

\newtcbtheorem[use counter*=corollary]{fcorollary}{Corollary}%
    {fonttitle=\bfseries\upshape,
     colback=Orchid!5!white,colframe=Orchid!75!black}{ex}

\newenvironment{bproof}{\textit{Proof.}}{\hfill$\square$}
\tcolorboxenvironment{bproof}{blanker,breakable,left=5mm,
                             before skip=10pt,after skip=10pt,
                             borderline west={1mm}{0pt}{red}}
\tcolorboxenvironment{fsolution}
    {enhanced jigsaw,colframe=cyan,interior hidden,breakable}

%--------------------Declared Math Operators--------------------%
\DeclareMathOperator{\Refl}{Refl}           % Reflection operator.
\DeclareMathOperator{\Span}{Span}           % Span of a set of vectors.
\DeclareMathOperator{\Card}{Card}           % Cardinality of set.
\DeclareMathOperator{\Ord}{Ord}             % Ordinal of ordered set.
\DeclareMathOperator{\Tr}{Tr}               % Trace of matrix.
\DeclareMathOperator{\adjoint}{adj}         % Adjoint of matrix.
\DeclareMathOperator{\rk}{rk}               % Rank of operator.
\DeclareMathOperator{\nul}{nul}             % Null space of operator.
\DeclareMathOperator{\sgn}{sgn}             % Sign of a number.
\DeclareMathOperator{\multideg}{mutlideg}   % Multi-Degree (Graphs).
\DeclareMathOperator{\GCD}{GCD}             % Greatest common denominator.
\DeclareMathOperator{\LM}{LM}               % Leading monomial
\DeclareMathOperator{\LC}{LC}               % Leading coefficient.
\DeclareMathOperator{\LT}{LT}               % Leading term.
\DeclareMathOperator{\LCM}{LCM}             % Least common multiple.
\DeclareMathOperator{\Mon}{Mon}             % Monomial.
\DeclareMathOperator{\Spec}{Spec}           % Spectrum.
\DeclareMathOperator{\proj}{proj}           % Projection.
\DeclareMathOperator{\comp}{comp}           % Component.
\DeclareMathOperator{\sinc}{sinc}           % Sinc function.
\DeclareMathOperator{\Ima}{Im}              % Image of operator.
\DeclareMathOperator{\Prin}{Prin}           % Principal value.
\DeclareMathOperator{\Mod}{mod}             % Modulus.
%------------------------New Commands---------------------------%
\DeclarePairedDelimiter\norm{\lVert}{\rVert}
\DeclarePairedDelimiter\ceil{\lceil}{\rceil}
\DeclarePairedDelimiter\floor{\lfloor}{\rfloor}
\newcommand*\diff{\mathop{}\!\mathrm{d}}
\newcommand*\Diff[1]{\mathop{}\!\mathrm{d^#1}}
\renewcommand{\mod}{\ \Mod}
\renewcommand*{\glstextformat}[1]{\textcolor{RoyalBlue}{#1}}
\renewcommand{\glsnamefont}[1]{\textbf{#1}}
\renewcommand\labelitemii{$\circ$}
\renewcommand\thesubfigure{\arabic{chapter}.\arabic{figure}}
\renewcommand\thesubfigure{%
    \arabic{chapter}.\arabic{figure}.\arabic{subfigure}}
\addto\captionsenglish{\renewcommand{\figurename}{Fig.}}
%------------------------Book Command---------------------------%
\makeatletter
\renewcommand\@pnumwidth{1cm}
\newcounter{book}
\renewcommand\thebook{\@Roman\c@book}
\newcommand\book{%
    \if@openright
        \cleardoublepage
    \else
        \clearpage
    \fi
    \thispagestyle{plain}%
    \if@twocolumn
        \onecolumn
        \@tempswatrue
    \else
        \@tempswafalse
    \fi
    \null\vfil
    \secdef\@book\@sbook
}
\def\@book[#1]#2{%
    \ifnum \c@secnumdepth >-3\relax
        \refstepcounter{book}%
        \addcontentsline{toc}{book}{
            \bookname\ \thebook:\hspace{1em}#1
        }
    \else
        \addcontentsline{toc}{book}{#1}%
    \fi
    \markboth{}{}%
    {\centering
     \interlinepenalty \@M
     \normalfont
     \ifnum \c@secnumdepth >-2\relax
       \huge\bfseries \bookname\nobreakspace\thebook
       \par
       \vskip 20\p@
     \fi
     \Huge \bfseries #2\par}%
    \@endbook}
\def\@sbook#1{%
    {\centering
     \interlinepenalty \@M
     \normalfont
     \Huge \bfseries #1\par}%
    \@endbook}
\def\@endbook{
    \vfil\newpage
        \if@twoside
            \if@openright
                \null
                \thispagestyle{empty}%
                \newpage
            \fi
        \fi
        \if@tempswa
            \twocolumn
        \fi
}
\newcommand*\l@book[2]{%
    \ifnum \c@tocdepth >-2\relax
        \addpenalty{-\@highpenalty}%
        \addvspace{2.25em \@plus\p@}%
        \setlength\@tempdima{3em}%
        \begingroup
            \parindent \z@ \rightskip \@pnumwidth
            \parfillskip -\@pnumwidth
            {
                \leavevmode
                \Large \bfseries #1\hfil \hb@xt@\@pnumwidth{
                    \hss #2
                }
            }
            \par
            \nobreak
            \global\@nobreaktrue
            \everypar{\global\@nobreakfalse\everypar{}}%
        \endgroup
    \fi}
\newcommand\bookname{Book}
\renewcommand{\thebook}{\texorpdfstring{\Numberstring{book}}{book}}
\providecommand*{\toclevel@book}{-2}
\makeatother
\titlecontents{chapter}[0pt]
    {\bfseries}
    {\chaptername\ \thecontentslabel:\quad}
    {}
    {\hfill\contentspage}
\titleformat{\part}[display]
    {\Large\bfseries}
    {\partname\nobreakspace\thepart}
    {0mm}
    {\Huge\bfseries}
    \titlecontents{part}[0pt]
    {\large\bfseries}
    {\partname\ \thecontentslabel: \quad}
    {}
    {\hfill\contentspage}
\newcommand{\MarkRightAngle}[4][.3cm]
    {\coordinate (tempa) at ($(#3)!#1!(#2)$);
     \coordinate (tempb) at ($(#3)!#1!(#4)$);
     \coordinate (tempc) at ($(tempa)!0.5!(tempb)$);%midpoint
     \draw (tempa) -- ($(#3)!2!(tempc)$) -- (tempb);}
%--------------------------LENGTHS------------------------------%
% Spacings for the Table of Contents.
\addtolength{\cftsecnumwidth}{1ex}
\addtolength{\cftsubsecindent}{1ex}
\addtolength{\cftsubsecnumwidth}{1ex}
\addtolength{\cftfignumwidth}{1ex}
\addtolength{\cfttabnumwidth}{1ex}

% Spacing for multi-column and enumerate environments.
\setlength{\multicolsep}{6pt}
\setlist[enumerate]{itemsep=0pt,topsep=3pt}

% Indent and paragraph spacing.
\setlength{\parindent}{0em}
\setlength{\parskip}{0em}                                                           %
\makeindex[intoc]                                                              %
%----------------------------Main Document-------------------------------------%
\begin{document}
    \title{MATH 114 Algebraic Topology - Assignment 2}
    \author{Ryan Maguire}
    \date{\vspace{-5ex}}
    \maketitle
    \pagenumbering{roman}
    \pagenumbering{arabic}
    \setcounter{section}{1}
    \begin{problem}
        Show that a homotopy equivalence $f:X\rightarrow{Y}$ induces a bijection
        between the set of path-components of $X$ and the set of path-components
        of $Y$. Show that $f$ restricts to a homotopy equivalence between each
        path-component. Prove this statement for components.
    \end{problem}
    \begin{solution}
        For let $f:X\rightarrow{Y}$ be a homotopy equivalence and
        $\mathcal{U}\subseteq{X}$ a path connected component. Let
        $x\in\mathcal{U}$ be arbitrary, let $y=f(x)$, and let $\mathcal{V}$ be
        the path connected component of $Y$ containing $y$. Let
        $\mathcal{U}\mapsto_{f}\mathcal{V}$. We must show this is well-defined
        and bijective. Since the continuous image of a path connected space is
        path connected, and since homotopy equivalences are continuous, for any
        point $p\in\mathcal{U}$ we have $f(p)\in\mathcal{V}$. That is, since
        $\mathcal{U}$ is hypothesized to be path connected, and since $x$ and
        $p$ are contained in $\mathcal{U}$, there is a path between $p$ and $x$.
        Composing this path with $f$ results in a path between $f(x)$ and
        $f(p)$. Since $\mathcal{V}$ is a path connected component we may
        conclude $f(p)\in\mathcal{V}$. Hence $\mathcal{U}$ maps to $\mathcal{V}$
        regardless of representative, and therefore this map is well defined.
        Moreover, it is injective. If
        $\mathcal{U}_{1}$ and $\mathcal{U}_{2}$ map to the same $\mathcal{V}$
        then $\mathcal{U}_{1}=\mathcal{U}_{2}$. For if $f$ is a homotopy
        equivalence, then there is a homotopy inverse $g:Y\rightarrow{X}$ such
        that $g\circ{f}$ is homotopic to $\identity{X}$. Let
        $x_{1}\in\mathcal{U}_{1}$ and $x_{2}\in\mathcal{U}_{2}$ and define
        $y_{1}=f(x_{1})$ and $y_{2}=f(x_{2})$. Since $y_{1},y_{2}\in\mathcal{V}$
        and $\mathcal{V}$ is path connected, there is a path $\gamma$ between
        $y_{1}$ and $y_{2}$. Let $\Gamma=g\circ\gamma$. Since $g$ is continuous,
        this is a path in $X$. Moreover, it is a path between $g(y_{1})$ and
        $g(y_{2})$. But $g\circ{f}$ is homotopic to $\identity{X}$
        and hence there is a homotopy $H_{X}:X\times{I}\rightarrow{X}$ such that
        $H_{X}(x,0)=(g\circ{f})(x)$ and $H_{X}(x,1)=x$. But then
        $H_{X}(x_{1},t)$ is a path between $g(y_{1})$ and $x_{1}$ and
        $H_{X}(x_{2},t)$ is a path between $g(y_{2})$ and $x_{2}$. But there is
        a path between $g(y_{1})$ and $g(y_{2})$, namely $\Gamma$, and hence by
        concatenating there is a path between $x_{1}$ and $x_{2}$. Thus,
        $x_{1}$ and $x_{2}$ reside in the same path connected component and
        hence $\mathcal{U}_{1}=\mathcal{U}_{2}$. By an identical argument,
        $\mathcal{V}\mapsto_{g}\mathcal{U}$ is an injective function from the
        path connected components of $Y$ to the path connected components of
        $X$. But then there exist injective functions in both directions, so by
        the Cantor-Schr\"{o}eder-Bernstein theorem there is a bijection. That
        is, there exists a bijection between the path connected components
        of $X$ and the path connected components of $Y$. Now, given a path
        component $\mathcal{U}\subseteq{X}$, $f|_{\mathcal{U}}$ is a homotopy
        equivalence to it's corresponding path connected component
        $\mathcal{V}\subseteq{Y}$. Consider $g|_{\mathcal{V}}$. Given the
        homotopy $H_{X}:X\times{I}\rightarrow{X}$, let $G_{X}$ be it's
        restriction to $\mathcal{U}\times{I}$. Then $G$ is a homotopy
        equivalence between $g|_{\mathcal{V}}\circ{f}|_{\mathcal{U}}$ and
        $\identity{\mathcal{U}}$. Firstly, the image of $G$ resides in
        $\mathcal{U}$ since for all $x\in\mathcal{U}$, $G(x,t)$ is a path
        between $H(x,0)=g(f(x))$ and $H(x,1)=x$, both of which reside in
        $\mathcal{U}$ and since $\mathcal{U}$ is path connected this implies the
        entire image of $G$ resides in $\mathcal{U}$. But then
        $G(x,0)=(g|_{\mathcal{V}}\circ{f}|_{\mathcal{U}})(x)$ and $G(x,1)=x$.
        Hence $G$ is a homotopy between $g|_{\mathcal{V}}\circ{f}_{\mathcal{U}}$
        and $\identity{\mathcal{U}}$. Similarly, since $g$ is a homotopy inverse
        of $g$, $f\circ{g}$ is homotopic to $\identity{Y}$. Let
        $H_{Y}:Y\times{I}\rightarrow{Y}$ be such a homotopy and
        $G_{Y}=H_{Y}|_{\mathcal{V}\times{I}}$. By an identical argument $G_{Y}$
        is a homotopy between $f|_{\mathcal{U}}\circ{g}|_{\mathcal{V}}$ and
        $\identity{\mathcal{V}}$. Hence $g|_{\mathcal{V}}$ is a homotopy inverse
        of $f|_{\mathcal{U}}$, so $f|_{\mathcal{U}}$ is a homotopy equivalence.
        For connected components, let $\mathcal{U}\mapsto_{f}\mathcal{V}$ be a
        similar mapping. Given $x\in\mathcal{U}$, let $\mathcal{V}\subseteq{Y}$
        be the connected component containing $f(x)$. We must show this is well
        defined. But the continuous image of connected is still connected, and
        hence if $p\in\mathcal{U}$, then $f(p)\in\mathcal{V}$ and therefore
        $\mathcal{U}$ maps unambiguously to $\mathcal{V}$. We must show this
        mapping is injective. Let $\mathcal{U}_{1},\mathcal{U}_{2}\subseteq{X}$
        be connected components and suppose both map to $\mathcal{V}$. But since
        $f$ is a homotopy equivalence there is a homotopy inverse
        $g:Y\rightarrow{X}$ such that $g\circ{f}$ is homotopic to $\identity{X}$
        and $f\circ{g}$ is homotopic to $\identity{Y}$. But then there is a path
        between $x_{1}$ and $(g\circ{f})(x_{1})$ and similarly for $x_{2}$. But
        path connected components are contained within connected components, and
        hence these paths reside in $\mathcal{U}_{1}$ and $\mathcal{U}_{2}$,
        respectively. But $\mathcal{V}$ is connected and $g$ is continuous, and
        therefore $g[\mathcal{V}]$ is connected. Since
        $g(f(x_{1}))\in\mathcal{U}_{1}$ we conclude
        $g[\mathcal{V}]\subseteq\mathcal{U}_{1}$. Similarly, since
        $g(f(x_{2}))\in\mathcal{U}_{2}$, we conclude
        $g[\mathcal{V}]\subseteq\mathcal{U}_{2}$. But then $\mathcal{U}_{1}$ and
        $\mathcal{U}_{2}$ have non-empty intersection. Since they are connected
        components we thus have $\mathcal{U}_{1}=\mathcal{U}_{2}$. Similarly,
        since $f$ is a homotopy inverse of $g$, the mapping
        $\mathcal{V}\mapsto_{g}\mathcal{U}$ unambiguously and injectively maps 
        connected components of $Y$ to connected components of $X$. Invoking
        Cantor-Schr\"{o}eder-Bernstein we thus obtain a bijection. Given
        a connected component $\mathcal{U}\subseteq{X}$, $f|_{\mathcal{U}}$ is a
        homotopy equivalence into the corresponding $\mathcal{V}\subseteq{Y}$
        since $g|_{\mathcal{V}}$ is a homotopy inverse. For $g\circ{f}$ is
        homotopy equivalent to $\identity{X}$ and hence there is a homotopy
        $H_{X}:X\times{I}\rightarrow{X}$ such that
        $H(x,0)=(g\circ{f})(x)$ and $H(x,1)=x$. But then
        $G_{X}=H_{X}|_{\mathcal{U}\times{I}}$ is a homotopy between
        $g|_{\mathcal{V}}\circ{f}|_{\mathcal{U}}$ and $\identity{\mathcal{U}}$.
        Firstly, the image of $G_{X}$ resides in $\mathcal{U}$. For any
        $x\in\mathcal{U}$, $G_{X}(x,t)$ is a path between
        $(g\circ{f})(x)$ and $x$. Since $\mathcal{U}$ is connected, the entirety
        of this path lies in $\mathcal{U}$. But
        $G_{X}(x,0)=H_{X}(x,0)=(g|_{\mathcal{V}}\circ{f}|_{\mathcal{U}})(x)$
        and $G_{X}(x,1)=H_{X}(x,1)=x$. That is, $G_{X}$ is a homotopy between
        $g|_{\mathcal{V}}\circ{f}|_{\mathcal{U}}$ and $\identity{\mathcal{U}}$.
        By a similar argument, $f|_{\mathcal{U}}\circ{g}|_{\mathcal{V}}$ is
        homotopic to $\identity{\mathcal{V}}$. If $X$ is such that the path
        connected components coincide with the connected components, and
        $f:X\rightarrow{Y}$ is a homotopy equivalence, then the same is true for
        $Y$. Given a connected component $\mathcal{V}\subseteq{Y}$, let
        $\mathcal{U}$ be the corresponding connected component in $X$. But by
        hypothesis $\mathcal{U}$ is path connected. Let
        $y_{1},y_{2}\in\mathcal{V}$, $x_{1}=g(y_{1})$, and $x_{2}=g(y_{2})$.
        Since $X$ is path connected, there is path $\gamma$ between $x_{1}$ and
        $x_{2}$. But then $f\circ\gamma$ is a path between $f(x_{1})$ and
        $f(x_{2})$. But $f\circ{g}$ is homotopic to $\identity{Y}$, so let
        $H_{Y}$ be such a homotopy. But then $H_{Y}(y_{1},t)$ is a path between
        $y_{1}$ and $f(x_{1})$, and similarly for $y_{2}$. But $f\circ\gamma$
        is a path from $f(x_{1})$ to $f(x_{2})$. By concatenating paths we
        obtain a path from $y_{1}$ to $y_{2}$. Hence, $\mathcal{V}$ is path
        connected and the connected components of $Y$ coincide with the path
        connected ones.
    \end{solution}
    \begin{problem}
        Show that $\nsphere[m]*\nsphere$ is homeomorphic to $\nsphere[m+n+1]$.
    \end{problem}
    \begin{solution}
        Define
        $f:\nsphere[m]\times\nsphere[n]\rightarrow{I}\rightarrow\nsphere[n+m+1]$
        as follows:
        \begin{equation}
            f\big(\vector{a},\vector{b},t\big)
            =\big(\sqrt{1-t}\cdot\vector{a},\sqrt{t}\cdot\vector{b}\big)
        \end{equation}
        The image of $f$ lies in $\nsphere[n+m+1]$ since if
        $\vector{a}\in\nsphere[m]$ and $\vector{b}\in\nsphere[m]$, then
        $\norm{\vector{a}}=\norm{\vector{b}}=1$. From orthogonality of the
        components we then obtain:
        \begin{equation}
            \norm{f(\vector{a},\vector{b},t)}^{2}
                =(1-t)\norm{\vector{a}}^{2}+t\norm{\vector{b}}^{2}=1
        \end{equation}
        Then $f$ is surjective. We then pass $f$ to the quotient obtaining
        $\tilde{f}$. This is well defined since the following equation:
        \begin{equation}
            f(\vector{a}_{1},\vector{b}_{1},t_{1})
                =f(\vector{a}_{2},\vector{b}_{2},t_{2})
        \end{equation}
        is true if and only if both $t_{1}$ and $t_{2}$ are either zero or one
        and the corresponding $\vector{a}_{i}$ match the $\vector{b}_{j}$, or if
        all of the components match. But in this former instance the points are
        identified by the equivalence relation and hence belong to the same
        equivalence class. Hence the function obtained from
        passing to the quotient is well defined, and moreover this argument
        shows it is bijective. That is, $f$ is surjective, and hence so is
        $\tilde{f}$, and from the previous argument $\tilde{f}$ is injective.
        Since $f$ is continuous, so is $\tilde{f}$, and hence we have a
        continuous bijection from $\nsphere[m]*\nsphere$ into $\nsphere[n+m+1]$.
        But the join of compact sets is the quotient of the product of compact
        sets, which is thus the quotient of a compact space, and is therefore
        compact. Also, $\nsphere[n+m+1]$ is Hausdorff, so $\tilde{f}$ is a
        continuous bijection from a compact space to a Hausdorff one, and is
        therefore a homeomorphism.
    \end{solution}
    \begin{problem}
        Show that the space obtained from $\nsphere[2]$ by attaching any $n$
        2-cells along any collection of $n$ circles is homotopy equivalence to
        the wedge sum of $n+1$ copies of $\nsphere[2]$.
    \end{problem}
    \begin{solution}
        We prove by induction on $n$. First we note that $\nsphere[2]$ is
        simply connected. That is, any continuous function
        $f:\nsphere[1]\rightarrow\nsphere[2]$ is homotopic to a constant map.
        Since we are only dealing with circles, we need only prove this for the
        case of $f$ being injective. We use stereographic projection, and the
        fact that $\nsphere[1]$ is not homeomorphic to $\nsphere[2]$. This can
        be seen since they have different dimensions, but neglecting manifold
        theory this can also be seen since removing two points from
        $\nsphere[1]$ results in a disconnected space, whereas removing any
        finite number of points from
        $\nsphere[2]$ does not disconnect the sphere. So any continuous
        injective mapping $f:\nsphere[1]\rightarrow\nsphere[2]$ is not
        surjective, for otherwise $f$ would be a homeomorphism. That is, a
        continuous bijection from a compact space to a Hausdorff one is a
        homeomorphism, $\nsphere[1]$ is compact, and $\nsphere[2]$ is Hausdorff.
        So if $\nsphere[1]$ maps into $\nsphere[2]$ injectively and
        continuously, then the mapping is not surjective. Hence
        $\nsphere[2]\setminus{f}[\nsphere[1]]$ is non-empty. Moreover, this
        subset is open since $f[\nsphere[1]]$ is compact, and hence closed.
        Given a point in the complement we may find a small open neighborhood of
        this point such that $f[\nsphere[1]]$ does not intersect its closure.
        Removing this open subset results in a space that is homeomorphic to
        $\nspace[2]$, which is contractible. Hence any continuous function is
        homotopic to a constant. Pulling back all of this to the sphere shows
        that $\nsphere[2]$ is simply connected. Hence if we attach a 2-cell
        along a circle within the sphere, we may contract this circle to a point
        and obtain a sphere with a 2-cell attached to a point on the sphere.
        But this is precisely the wedge of $\nsphere[2]$ with $\nsphere[2]$.
        That is, our original space is homotopy equivalent to
        $\nsphere[2]\lor\nsphere[2]$. Suppose this is true of $n$ circles.
        Since the wedge of path connected spaces is equivalent up to homotopy
        regardless of choice of points, we may choose the wedge of $n$ spheres
        to all be at the same point. Given a circle in the original sphere and a
        2-cell attached to this we may then contract the circle down to our
        common point, obtaining another sphere. The result is the wedge of $n+1$
        spheres.
    \end{solution}
    \begin{problem}
        Show that if $(X,A)$ has the homotopy extension property, then
        $X\times{I}$ deformation retracts to $X\times\{0\}\cup{A}\times{I}$.
    \end{problem}
    \begin{solution}
        For if $(X,A)$ satisfies the homotopy extension property, then there is
        a retract of $X\times{I}$ to $X\times\{0\}\cup{A}\times{I}$. But then
        the inclusion map
        $\iota:X\times\{0\}\cup{A}\times{I}\rightarrow{X}\times{I}$ is a
        homotopy equivalence. For let $r$ be the retract. Then
        $r\circ\iota$ is just the identity map on $X\times\{0\}\cup{A}\times{I}$
        since $r$ is a retract, and the identity is homotopic to itself. But
        since $\iota\circ{r}$ is just the restriction of $r$ to
        $X\times\{0\}\cup{A}\times{I}$ we may use the homotopy extension
        property to extend $f_{0}=\identity{X}$ and $f_{1}=r$ to a homotopy,
        showing that $\iota\circ{r}$ is homotopic to the identity map on $X$.
        Hence, the inclusion map is a homotopy equivalence. Then by the
        corollary there is a deformation retraction of $X\times{I}$ onto
        $X\times\{0\}\cup{A}\times{I}$. For the generalization of 0.18, let
        $H:A\times{I}\rightarrow{X}_{0}$ between a homotopy from $f$ to $g$, and
        consider the attaching space $X_{0}\coprod_{H}(X_{1}\times{I})$. But
        since $(X_{1},A)$ has the homotopy extension property, there is a
        deformation retraction of $X_{1}\times{I}$ to
        $X_{1}\times\{0\}\cup{A}\times{I}$, and this induces a deformation
        retraction of $X_{0}\coprod_{H}(X_{1}\times{I})$ to
        $X_{0}\coprod_{f}X_{1}$ and similarly for $g$. Since both of these
        deformation retractions are the identity on $X_{0}$, we see that
        $X_{0}\coprod_{f}X_{1}$ is homotopy equivalent to
        $X_{0}\coprod_{g}X_{1}$ relative to $X_{0}$.
    \end{solution}
\end{document}