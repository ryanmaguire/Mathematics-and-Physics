%------------------------------------------------------------------------------%
\documentclass{article}                                                        %
%------------------------------Preamble----------------------------------------%
\makeatletter                                                                  %
    \def\input@path{{../../}}                                                  %
\makeatother                                                                   %
%---------------------------Packages----------------------------%
\usepackage{geometry}
\geometry{b5paper, margin=1.0in}
\usepackage[T1]{fontenc}
\usepackage{graphicx, float}            % Graphics/Images.
\usepackage{natbib}                     % For bibliographies.
\bibliographystyle{agsm}                % Bibliography style.
\usepackage[french, english]{babel}     % Language typesetting.
\usepackage[dvipsnames]{xcolor}         % Color names.
\usepackage{listings}                   % Verbatim-Like Tools.
\usepackage{mathtools, esint, mathrsfs} % amsmath and integrals.
\usepackage{amsthm, amsfonts, amssymb}  % Fonts and theorems.
\usepackage{tcolorbox}                  % Frames around theorems.
\usepackage{upgreek}                    % Non-Italic Greek.
\usepackage{fmtcount, etoolbox}         % For the \book{} command.
\usepackage[newparttoc]{titlesec}       % Formatting chapter, etc.
\usepackage{titletoc}                   % Allows \book in toc.
\usepackage[nottoc]{tocbibind}          % Bibliography in toc.
\usepackage[titles]{tocloft}            % ToC formatting.
\usepackage{pgfplots, tikz}             % Drawing/graphing tools.
\usepackage{imakeidx}                   % Used for index.
\usetikzlibrary{
    calc,                   % Calculating right angles and more.
    angles,                 % Drawing angles within triangles.
    arrows.meta,            % Latex and Stealth arrows.
    quotes,                 % Adding labels to angles.
    positioning,            % Relative positioning of nodes.
    decorations.markings,   % Adding arrows in the middle of a line.
    patterns,
    arrows
}                                       % Libraries for tikz.
\pgfplotsset{compat=1.9}                % Version of pgfplots.
\usepackage[font=scriptsize,
            labelformat=simple,
            labelsep=colon]{subcaption} % Subfigure captions.
\usepackage[font={scriptsize},
            hypcap=true,
            labelsep=colon]{caption}    % Figure captions.
\usepackage[pdftex,
            pdfauthor={Ryan Maguire},
            pdftitle={Mathematics and Physics},
            pdfsubject={Mathematics, Physics, Science},
            pdfkeywords={Mathematics, Physics, Computer Science, Biology},
            pdfproducer={LaTeX},
            pdfcreator={pdflatex}]{hyperref}
\hypersetup{
    colorlinks=true,
    linkcolor=blue,
    filecolor=magenta,
    urlcolor=Cerulean,
    citecolor=SkyBlue
}                           % Colors for hyperref.
\usepackage[toc,acronym,nogroupskip,nopostdot]{glossaries}
\usepackage{glossary-mcols}
%------------------------Theorem Styles-------------------------%
\theoremstyle{plain}
\newtheorem{theorem}{Theorem}[section]

% Define theorem style for default spacing and normal font.
\newtheoremstyle{normal}
    {\topsep}               % Amount of space above the theorem.
    {\topsep}               % Amount of space below the theorem.
    {}                      % Font used for body of theorem.
    {}                      % Measure of space to indent.
    {\bfseries}             % Font of the header of the theorem.
    {}                      % Punctuation between head and body.
    {.5em}                  % Space after theorem head.
    {}

% Italic header environment.
\newtheoremstyle{thmit}{\topsep}{\topsep}{}{}{\itshape}{}{0.5em}{}

% Define environments with italic headers.
\theoremstyle{thmit}
\newtheorem*{solution}{Solution}

% Define default environments.
\theoremstyle{normal}
\newtheorem{example}{Example}[section]
\newtheorem{definition}{Definition}[section]
\newtheorem{problem}{Problem}[section]

% Define framed environment.
\tcbuselibrary{most}
\newtcbtheorem[use counter*=theorem]{ftheorem}{Theorem}{%
    before=\par\vspace{2ex},
    boxsep=0.5\topsep,
    after=\par\vspace{2ex},
    colback=green!5,
    colframe=green!35!black,
    fonttitle=\bfseries\upshape%
}{thm}

\newtcbtheorem[auto counter, number within=section]{faxiom}{Axiom}{%
    before=\par\vspace{2ex},
    boxsep=0.5\topsep,
    after=\par\vspace{2ex},
    colback=Apricot!5,
    colframe=Apricot!35!black,
    fonttitle=\bfseries\upshape%
}{ax}

\newtcbtheorem[use counter*=definition]{fdefinition}{Definition}{%
    before=\par\vspace{2ex},
    boxsep=0.5\topsep,
    after=\par\vspace{2ex},
    colback=blue!5!white,
    colframe=blue!75!black,
    fonttitle=\bfseries\upshape%
}{def}

\newtcbtheorem[use counter*=example]{fexample}{Example}{%
    before=\par\vspace{2ex},
    boxsep=0.5\topsep,
    after=\par\vspace{2ex},
    colback=red!5!white,
    colframe=red!75!black,
    fonttitle=\bfseries\upshape%
}{ex}

\newtcbtheorem[auto counter, number within=section]{fnotation}{Notation}{%
    before=\par\vspace{2ex},
    boxsep=0.5\topsep,
    after=\par\vspace{2ex},
    colback=SeaGreen!5!white,
    colframe=SeaGreen!75!black,
    fonttitle=\bfseries\upshape%
}{not}

\newtcbtheorem[use counter*=remark]{fremark}{Remark}{%
    fonttitle=\bfseries\upshape,
    colback=Goldenrod!5!white,
    colframe=Goldenrod!75!black}{ex}

\newenvironment{bproof}{\textit{Proof.}}{\hfill$\square$}
\tcolorboxenvironment{bproof}{%
    blanker,
    breakable,
    left=3mm,
    before skip=5pt,
    after skip=10pt,
    borderline west={0.6mm}{0pt}{green!80!black}
}

\AtEndEnvironment{lexample}{$\hfill\textcolor{red}{\blacksquare}$}
\newtcbtheorem[use counter*=example]{lexample}{Example}{%
    empty,
    title={Example~\theexample},
    boxed title style={%
        empty,
        size=minimal,
        toprule=2pt,
        top=0.5\topsep,
    },
    coltitle=red,
    fonttitle=\bfseries,
    parbox=false,
    boxsep=0pt,
    before=\par\vspace{2ex},
    left=0pt,
    right=0pt,
    top=3ex,
    bottom=1ex,
    before=\par\vspace{2ex},
    after=\par\vspace{2ex},
    breakable,
    pad at break*=0mm,
    vfill before first,
    overlay unbroken={%
        \draw[red, line width=2pt]
            ([yshift=-1.2ex]title.south-|frame.west) to
            ([yshift=-1.2ex]title.south-|frame.east);
        },
    overlay first={%
        \draw[red, line width=2pt]
            ([yshift=-1.2ex]title.south-|frame.west) to
            ([yshift=-1.2ex]title.south-|frame.east);
    },
}{ex}

\AtEndEnvironment{ldefinition}{$\hfill\textcolor{Blue}{\blacksquare}$}
\newtcbtheorem[use counter*=definition]{ldefinition}{Definition}{%
    empty,
    title={Definition~\thedefinition:~{#1}},
    boxed title style={%
        empty,
        size=minimal,
        toprule=2pt,
        top=0.5\topsep,
    },
    coltitle=Blue,
    fonttitle=\bfseries,
    parbox=false,
    boxsep=0pt,
    before=\par\vspace{2ex},
    left=0pt,
    right=0pt,
    top=3ex,
    bottom=0pt,
    before=\par\vspace{2ex},
    after=\par\vspace{1ex},
    breakable,
    pad at break*=0mm,
    vfill before first,
    overlay unbroken={%
        \draw[Blue, line width=2pt]
            ([yshift=-1.2ex]title.south-|frame.west) to
            ([yshift=-1.2ex]title.south-|frame.east);
        },
    overlay first={%
        \draw[Blue, line width=2pt]
            ([yshift=-1.2ex]title.south-|frame.west) to
            ([yshift=-1.2ex]title.south-|frame.east);
    },
}{def}

\AtEndEnvironment{ltheorem}{$\hfill\textcolor{Green}{\blacksquare}$}
\newtcbtheorem[use counter*=theorem]{ltheorem}{Theorem}{%
    empty,
    title={Theorem~\thetheorem:~{#1}},
    boxed title style={%
        empty,
        size=minimal,
        toprule=2pt,
        top=0.5\topsep,
    },
    coltitle=Green,
    fonttitle=\bfseries,
    parbox=false,
    boxsep=0pt,
    before=\par\vspace{2ex},
    left=0pt,
    right=0pt,
    top=3ex,
    bottom=-1.5ex,
    breakable,
    pad at break*=0mm,
    vfill before first,
    overlay unbroken={%
        \draw[Green, line width=2pt]
            ([yshift=-1.2ex]title.south-|frame.west) to
            ([yshift=-1.2ex]title.south-|frame.east);},
    overlay first={%
        \draw[Green, line width=2pt]
            ([yshift=-1.2ex]title.south-|frame.west) to
            ([yshift=-1.2ex]title.south-|frame.east);
    }
}{thm}

%--------------------Declared Math Operators--------------------%
\DeclareMathOperator{\adjoint}{adj}         % Adjoint.
\DeclareMathOperator{\Card}{Card}           % Cardinality.
\DeclareMathOperator{\curl}{curl}           % Curl.
\DeclareMathOperator{\diam}{diam}           % Diameter.
\DeclareMathOperator{\dist}{dist}           % Distance.
\DeclareMathOperator{\Div}{div}             % Divergence.
\DeclareMathOperator{\Erf}{Erf}             % Error Function.
\DeclareMathOperator{\Erfc}{Erfc}           % Complementary Error Function.
\DeclareMathOperator{\Ext}{Ext}             % Exterior.
\DeclareMathOperator{\GCD}{GCD}             % Greatest common denominator.
\DeclareMathOperator{\grad}{grad}           % Gradient
\DeclareMathOperator{\Ima}{Im}              % Image.
\DeclareMathOperator{\Int}{Int}             % Interior.
\DeclareMathOperator{\LC}{LC}               % Leading coefficient.
\DeclareMathOperator{\LCM}{LCM}             % Least common multiple.
\DeclareMathOperator{\LM}{LM}               % Leading monomial.
\DeclareMathOperator{\LT}{LT}               % Leading term.
\DeclareMathOperator{\Mod}{mod}             % Modulus.
\DeclareMathOperator{\Mon}{Mon}             % Monomial.
\DeclareMathOperator{\multideg}{mutlideg}   % Multi-Degree (Graphs).
\DeclareMathOperator{\nul}{nul}             % Null space of operator.
\DeclareMathOperator{\Ord}{Ord}             % Ordinal of ordered set.
\DeclareMathOperator{\Prin}{Prin}           % Principal value.
\DeclareMathOperator{\proj}{proj}           % Projection.
\DeclareMathOperator{\Refl}{Refl}           % Reflection operator.
\DeclareMathOperator{\rk}{rk}               % Rank of operator.
\DeclareMathOperator{\sgn}{sgn}             % Sign of a number.
\DeclareMathOperator{\sinc}{sinc}           % Sinc function.
\DeclareMathOperator{\Span}{Span}           % Span of a set.
\DeclareMathOperator{\Spec}{Spec}           % Spectrum.
\DeclareMathOperator{\supp}{supp}           % Support
\DeclareMathOperator{\Tr}{Tr}               % Trace of matrix.
%--------------------Declared Math Symbols--------------------%
\DeclareMathSymbol{\minus}{\mathbin}{AMSa}{"39} % Unary minus sign.
%------------------------New Commands---------------------------%
\DeclarePairedDelimiter\norm{\lVert}{\rVert}
\DeclarePairedDelimiter\ceil{\lceil}{\rceil}
\DeclarePairedDelimiter\floor{\lfloor}{\rfloor}
\newcommand*\diff{\mathop{}\!\mathrm{d}}
\newcommand*\Diff[1]{\mathop{}\!\mathrm{d^#1}}
\renewcommand*{\glstextformat}[1]{\textcolor{RoyalBlue}{#1}}
\renewcommand{\glsnamefont}[1]{\textbf{#1}}
\renewcommand\labelitemii{$\circ$}
\renewcommand\thesubfigure{%
    \arabic{chapter}.\arabic{figure}.\arabic{subfigure}}
\addto\captionsenglish{\renewcommand{\figurename}{Fig.}}
\numberwithin{equation}{section}

\renewcommand{\vector}[1]{\boldsymbol{\mathrm{#1}}}

\newcommand{\uvector}[1]{\boldsymbol{\hat{\mathrm{#1}}}}
\newcommand{\topspace}[2][]{(#2,\tau_{#1})}
\newcommand{\measurespace}[2][]{(#2,\varSigma_{#1},\mu_{#1})}
\newcommand{\measurablespace}[2][]{(#2,\varSigma_{#1})}
\newcommand{\manifold}[2][]{(#2,\tau_{#1},\mathcal{A}_{#1})}
\newcommand{\tanspace}[2]{T_{#1}{#2}}
\newcommand{\cotanspace}[2]{T_{#1}^{*}{#2}}
\newcommand{\Ckspace}[3][\mathbb{R}]{C^{#2}(#3,#1)}
\newcommand{\funcspace}[2][\mathbb{R}]{\mathcal{F}(#2,#1)}
\newcommand{\smoothvecf}[1]{\mathfrak{X}(#1)}
\newcommand{\smoothonef}[1]{\mathfrak{X}^{*}(#1)}
\newcommand{\bracket}[2]{[#1,#2]}

%------------------------Book Command---------------------------%
\makeatletter
\renewcommand\@pnumwidth{1cm}
\newcounter{book}
\renewcommand\thebook{\@Roman\c@book}
\newcommand\book{%
    \if@openright
        \cleardoublepage
    \else
        \clearpage
    \fi
    \thispagestyle{plain}%
    \if@twocolumn
        \onecolumn
        \@tempswatrue
    \else
        \@tempswafalse
    \fi
    \null\vfil
    \secdef\@book\@sbook
}
\def\@book[#1]#2{%
    \refstepcounter{book}
    \addcontentsline{toc}{book}{\bookname\ \thebook:\hspace{1em}#1}
    \markboth{}{}
    {\centering
     \interlinepenalty\@M
     \normalfont
     \huge\bfseries\bookname\nobreakspace\thebook
     \par
     \vskip 20\p@
     \Huge\bfseries#2\par}%
    \@endbook}
\def\@sbook#1{%
    {\centering
     \interlinepenalty \@M
     \normalfont
     \Huge\bfseries#1\par}%
    \@endbook}
\def\@endbook{
    \vfil\newpage
        \if@twoside
            \if@openright
                \null
                \thispagestyle{empty}%
                \newpage
            \fi
        \fi
        \if@tempswa
            \twocolumn
        \fi
}
\newcommand*\l@book[2]{%
    \ifnum\c@tocdepth >-3\relax
        \addpenalty{-\@highpenalty}%
        \addvspace{2.25em\@plus\p@}%
        \setlength\@tempdima{3em}%
        \begingroup
            \parindent\z@\rightskip\@pnumwidth
            \parfillskip -\@pnumwidth
            {
                \leavevmode
                \Large\bfseries#1\hfill\hb@xt@\@pnumwidth{\hss#2}
            }
            \par
            \nobreak
            \global\@nobreaktrue
            \everypar{\global\@nobreakfalse\everypar{}}%
        \endgroup
    \fi}
\newcommand\bookname{Book}
\renewcommand{\thebook}{\texorpdfstring{\Numberstring{book}}{book}}
\providecommand*{\toclevel@book}{-2}
\makeatother
\titleformat{\part}[display]
    {\Large\bfseries}
    {\partname\nobreakspace\thepart}
    {0mm}
    {\Huge\bfseries}
\titlecontents{part}[0pt]
    {\large\bfseries}
    {\partname\ \thecontentslabel: \quad}
    {}
    {\hfill\contentspage}
\titlecontents{chapter}[0pt]
    {\bfseries}
    {\chaptername\ \thecontentslabel:\quad}
    {}
    {\hfill\contentspage}
\newglossarystyle{longpara}{%
    \setglossarystyle{long}%
    \renewenvironment{theglossary}{%
        \begin{longtable}[l]{{p{0.25\hsize}p{0.65\hsize}}}
    }{\end{longtable}}%
    \renewcommand{\glossentry}[2]{%
        \glstarget{##1}{\glossentryname{##1}}%
        &\glossentrydesc{##1}{~##2.}
        \tabularnewline%
        \tabularnewline
    }%
}
\newglossary[not-glg]{notation}{not-gls}{not-glo}{Notation}
\newcommand*{\newnotation}[4][]{%
    \newglossaryentry{#2}{type=notation, name={\textbf{#3}, },
                          text={#4}, description={#4},#1}%
}
%--------------------------LENGTHS------------------------------%
% Spacings for the Table of Contents.
\addtolength{\cftsecnumwidth}{1ex}
\addtolength{\cftsubsecindent}{1ex}
\addtolength{\cftsubsecnumwidth}{1ex}
\addtolength{\cftfignumwidth}{1ex}
\addtolength{\cfttabnumwidth}{1ex}

% Indent and paragraph spacing.
\setlength{\parindent}{0em}
\setlength{\parskip}{0em}                                                           %
%----------------------------Main Document-------------------------------------%
\begin{document}
    \title{Topology Notes}
    \author{Ryan Maguire}
    \date{\vspace{-5ex}}
    \maketitle
    \section{Hocking and Young (Chapter 1)}
        \begin{example}
            Let $\mathbb{R}$ be the real numbers with the usual topology and
            consider the subset $A\subseteq\mathbb{R}$ defined by:
            \begin{equation}
                A=\Big\{\,\frac{1}{n}\;|\;n\in\mathbb{N}\,\Big\}
            \end{equation}
            $A$ has only one limit point and this is zero. Note that
            $0\notin{A}$, and hence the derived set $\derived{A}$ is disjoint
            from $A$ itself. That is:
            \begin{equation}
                \derived[]{A}=\{0\}
            \end{equation}
            And thence $\derived[]{A}\cap{A}=\emptyset$.
        \end{example}
        \begin{example}
            Again consider $\mathbb{R}$ and define $B$ by:
            \begin{equation}
                B=\Big\{\,\frac{n+1}{n}+(\minus{1})^{n}\frac{n-1}{n}\;|\;
                    n\in\mathbb{N}\,\Big\}
            \end{equation}
            This oscillates between values that are asymptotically approaching
            zero and one. The derived set is then:
            \begin{equation}
                \derived[]{B}=\{0,\,1\}
            \end{equation}
            Once again, the derived set is disjoint from the original set.
        \end{example}
        \begin{example}
            In the usual topology, the derived set of $\mathbb{Q}$ is the
            entirety of $\mathbb{R}$. That is:
            \begin{equation}
                \derived[]{\mathbb{Q}}=\mathbb{R}
            \end{equation}
            So it is possible for the derived set to be strictly larger than the
            original set.
        \end{example}
        \begin{example}
            If $\topspace{X}$ is the chaotic topological space on some set $X$
            with the trivial topology $\tau=\{\emptyset,X\}$, then for any
            subset $A\subseteq{X}$, it is true that $\derived{A}=X$. That is,
            in the chaotic topology every point is a limit point of every other
            point.
        \end{example}
        \begin{example}
            If $\topspace{X}$ is the discrete topology, $\tau=\powset{X}$, then
            for any subset $A\subseteq{X}$, the derived set is empty:
            $\derived{A}=\emptyset$. That is, no point is a limit point of any
            subset. This is because every element in the discrete topology is
            isolated since $\{x\}$ is an open subset for every $x\in{X}$.
        \end{example}
        \begin{theorem}
            If $\topspace{X}$ is a topological space, if $A,B\subseteq{X}$, and
            if $A\subseteq{B}$, then $\derived{A}\subseteq\derived{B}$, where
            $\derived{Y}$ denotes the derived set of $Y$ in $\tau$.
        \end{theorem}
        \begin{theorem}
            If $\topspace{X}$ is a topological space, if $C\subseteq{X}$, then
            $C$ is closed if and only if $\closure{C}=C$.
        \end{theorem}
        \begin{theorem}
            If $\topspace{X}$ is a topological space, if
            $\mathcal{C}\subseteq\powset{X}$ is such that for all
            $C\in\mathcal{C}$ it is true that $C$ is closed, then
            $\bigcap\mathcal{C}$ is closed.
        \end{theorem}
        \begin{theorem}
            If $\topspace{X}$ is a topological space, if $C,D\subseteq{X}$ are
            closed in $X$, then $C\cup{D}$ is closed in $X$.
        \end{theorem}
        \begin{definition}
            A basis for a topology $\tau$ on a set $X$ is a subset
            $\mathcal{B}\subseteq\tau$ such that $\bigcup\mathcal{B}=X$ and such
            that for all $B_{1},B_{2}\in\mathcal{B}$ and for all
            $p\in{B}_{1}\cap{B}_{2}$, there exists $B_{3}\in\mathcal{B}$ such
            that $x\in{B}_{3}$ and $B_{2}\subseteq{B}_{1}\cap{B}_{2}$.
        \end{definition}
        \begin{theorem}
            If $\topspace{X}$ is a topological space, then $\tau$ is a basis
            for $\tau$.
        \end{theorem}
        \begin{proof}
            For $X\in\tau$, and hence $X\subseteq\bigcup\tau$, but since
            $\tau\subseteq\powset{X}$, for all $\mathcal{U}\in\tau$ it is true
            that $\mathcal{U}\subseteq{X}$, and hence $X=\bigcup\tau$. If
            $\mathcal{U}_{1},\mathcal{U}_{2}$, and if
            $x\in\mathcal{U}_{1}\cap\mathcal{U}_{2}$, then
            $\mathcal{U}_{1}\cap\mathcal{U}_{2}\in\tau$ is such that
            $x\in\mathcal{U}_{1}\cap\mathcal{U}_{2}$ and
            $\mathcal{U}_{1}\cap\mathcal{U}_{2}\subseteq%
             \mathcal{U}_{1}\cap\mathcal{U}_{2}$, and hence $\tau$ is a basis
            for $\tau$.
        \end{proof}
        \begin{theorem}
            If $\topspace{X}$ is a topological space, if $\mathcal{B}$ is a
            basis for $\tau$, then:
            \begin{equation}
                \tau=\Big\{\,\mathcal{U}\in\powset{X}\;\big|\;
                    \textrm{There exists }\mathcal{O}\subseteq\mathcal{B}
                    \textrm{ such that }\mathcal{U}=\bigcup\mathcal{O}\Big\}
            \end{equation}
        \end{theorem}
        \begin{theorem}
            If $X$ is a set, if $\tau_{1}$ and $\tau_{2}$ are topologies on $X$,
            if $\tau_{1}\subseteq\tau_{2}$, if $\mathcal{B}_{1}$ is a basis for
            $\tau_{1}$, if $\mathcal{B}_{2}$ is a basis for $\tau_{2}$, if
            $x\in{X}$, and if $B_{1}\in\mathcal{B}_{1}$ is such that
            $x\in{B}_{1}$, there exists a $B_{2}\in\mathcal{B}_{2}$ such that
            $x\in{B}_{2}$ and $B_{2}\subseteq{B}_{1}$.
        \end{theorem}
        \begin{example}
            The previous theorem shows when two bases for a topology $\tau$ are
            the same, but different bases may give rise to different topologies.
            Consider on $\mathbb{R}$ the standard topology $\tau_{\mathbb{R}}$
            and the topology $\tau$ generated by the set:
            \begin{equation}
                \mathcal{B}=\{\,(x,\infty)\;|\;x\in\mathbb{R}\}
            \end{equation}
            We can see that $\tau\subseteq\tau_{\mathbb{R}}$ since every set
            in $\mathcal{B}$ can be obtained as the union of intervals as
            follows:
            \begin{equation}
                (x,\infty)=\bigcup_{n\in\mathbb{N}}(x,x+n)
            \end{equation}
            Since $(x,x+n)\in\tau_{\mathbb{R}}$ for all $n\in\mathbb{N}$ and for
            all $x\in\mathbb{R}$, and since topologies are closed under
            arbitrary unions, we thus have that every element of $\mathcal{B}$
            is contained in $\tau_{\mathbb{R}}$ and thus every element of
            $\tau$ is also contained in $\tau_{\mathbb{R}}$. However, these are
            not the same topology.
        \end{example}
        \begin{example}
            In $\mathbb{R}^{2}$, the set of all vertical open line segments
            forms the basis of a topology that is strictly finer than the
            standard Euclidean topology $\tau_{\mathbb{R}^{2}}$, and indeed this
            can be seen as the product topology on $\mathbb{R}$ with the
            standard topology multiplied by $\mathbb{R}$ with the discrete
            topology. The resulting space, being the product of metric spaces,
            is again a metric space. Since the discrete topology $\tau_{D}$ is
            strictly finer than the standard topology on $\mathbb{R}$, we see
            that the product topology $\tau_{\mathbb{R}}\times\tau_{\mathrm{D}}$
            is strictly finer than the standard one $\top_{\mathbb{R}^{2}}$.
        \end{example}
        Def second countable. Intersection of topologies is topology. Topology
        generated by a collection.
        \begin{theorem}
            If $X$ is a set, if $\mathcal{O}\subseteq\powset{X}$, if
            $\tau(\mathcal{O})$ is the topology generated by $\mathcal{O}$,
            and if $\mathcal{B}$ is the set:
            \begin{equation}
                \mathcal{B}=\Big\{\,\mathcal{U}\in\powset{X}\;|\;
                    \exists_{n\in\mathbb{N}}
                    \exists_{B:\mathbb{Z}_{n}\rightarrow\mathcal{O}}
                    \big(\mathcal{U}=\bigcap{B}_{k}\big)\,\Big\}
            \end{equation}
            Then $\mathcal{B}$ is a basis for $\tau(\mathcal{O})$.
        \end{theorem}
        That is, the collection of all finite intersections of elements of
        $\mathcal{O}$ forms a basis for the topology that $\mathcal{O}$
        generates.
        \begin{definition}
            A subbasis for a topological space $\topspace{X}$ is a subset
            $\mathcal{B}\subseteq\powset{X}$ such that the topology generated
            by $\mathcal{B}$ is equal to $\tau$.
        \end{definition}
        \begin{example}
            Consider the set of all open rays on $\mathbb{R}$. That is, the set
            of all subsets of $\mathbb{R}$ of the form:
            \twocolumneq{\mathcal{U}_{+}=(x,\infty)}
                        {\mathcal{U}_{\minus}=(\minus\infty,x)}
            Then this forms a subbasis for the standard topology on
            $\mathbb{R}$. To see this we simply need to show that the standard
            basis for $\mathbb{R}$ can be obtained from finite intersections of
            elements of $\mathcal{B}$, and also note that
            $\mathcal{B}\subseteq\tau_{\mathbb{R}}$. Given an open interval
            $(a,b)\subseteq\mathbb{R}$, let $B_{1}=(a,\infty)$ and
            $B_{2}=(\minus\infty,b)$. Then both $B_{1},B_{2}\in\mathcal{B}$,
            but $(a,b)=B_{1}\cap{B}_{2}$. Thus the topology generated by the
            subbasis $\mathcal{B}$ is the same as the topology generated by the
            standard basis of open intervals, and this is the standard
            topology on $\mathbb{R}$.
        \end{example}
        \begin{ltheorem}{Birkhoff's Topology Lattice Theorem}
            If $X$ is a set, if $T$ is the set of all topologies of $X$, and if
            $\subseteq$ is the inclusion relation, then $(T,\subseteq)$ is a
            complete lattice.
        \end{ltheorem}
        \begin{theorem}
            If $X$ is a set, if $\tau_{1},\tau_{2}$ are topologies on $X$,
            if $\tau_{1}\subseteq\tau_{2}$, and if
            $\identity{X}:\topspace[1]{X}\rightarrow\topspace[2]{X}$ is the
            identity mapping, then it open.
        \end{theorem}
        \begin{theorem}
            If $X$ is a set, if $\tau_{1},\tau_{2}$ are topologies on $X$,
            if $\tau_{2}\subseteq\tau_{1}$, and if
            $\identity{X}:\topspace[1]{X}\rightarrow\topspace[2]{X}$ is the
            identity mapping, then it continuous.
        \end{theorem}
        \begin{theorem}
            If $\topspace[Y]{Y}$ is a topological space, if $X$ is a set, if
            $\tau_{1},\tau_{2}$ are topologies on $X$, if
            $\tau_{1}\subseteq\tau_{2}$, and if $f:X\rightarrow{Y}$ is
            continuous with respect to $\tau_{1}$, then it is continuous with
            respect to $\tau_{2}$.
        \end{theorem}
        \begin{theorem}
            If $\topspace[X]{X}$ is a topological space, if $Y$ is a set, if
            $\tau_{1},\tau_{2}$ are topologies on $Y$, if
            $\tau_{1}\subseteq\tau_{2}$, and if $f:X\rightarrow{Y}$ is
            continuous with respect to $\tau_{2}$, then it is continuous with
            respect to $\tau_{1}$.
        \end{theorem}
        Similar result for open maps. Def metric space, metrics, open balls.
        \begin{example}
            Metrics are not topological entities, and two vastly different
            metrics on the same set may give the same topology. For example,
            given any metric $d$ on a set $X$, the metric $\tilde{d}$ formed by:
            \begin{equation}
                \tilde{d}(x,y)=\frac{d(x,y)}{1+d(x,y)}
            \end{equation}
            is bounded, yet forms the same topology. Hence boundedness is not a
            topological property but a metric one. Thus, there are questions
            about metric spaces that cannot be answered from the study of
            metrizable spaces. For example, if $\metspace{X}$ is a metric space
            one can ask if it has the midpoint property: For all $x,y\in{X}$
            is there a $z\in{X}$ such that $d(x,z)=d(y,z)=d(x,y)/2$? This cannot
            be answer topologically, for consider the closed unit interval
            $[0,1]\subseteq\mathbb{R}$ and the closed upper half unit circle in
            $\mathbb{R}^{2}$. That is:
            \begin{equation}
                S=\{\,(x,y)\in\mathbb{S}_{1}\;|y\geq{0}\,\}
            \end{equation}
            Then $[0,1]$ and $S$ are homeomorphic under the function
            $f:[0,1]\rightarrow{S}$ defined by:
            \begin{equation}
                f(x)=\big(\cos(\pi{x}),\,\sin(\pi{x})\big)
            \end{equation}
            But $[0,1]$ does have the midpoint property, whereas the upper
            semi-circle does not. That is, closed intervals are convex where
            circles (the boundaries of discs) are not. To topologies this
            question we might ask if there exists a metric with the midpoint
            property or if every metric has it. That is, can a given topological
            space $\topspace{X}$ be given a metric $d$ with the midpoint
            property? Are there topological spaces $\topspace{X}$ where every
            metric on $X$ that induces $\tau$ must have the midpoint property?
        \end{example}
        Def separable.
        \begin{theorem}
            If $\topspace{X}$ is a separable and metrizable topological space,
            then it is second countable.
        \end{theorem}
        \begin{proof}
            For if $\topspace{X}$ is separable, there exists a countable dense
            subset $A$, and if $\topspace{X}$ is metrizable, there exists a
            metric $d$ on $X$ that induces the topology $\tau$. Define
            $\mathcal{B}$ by:
            \begin{equation}
                \mathcal{B}=
                    \{B_{q}^{\metspace{X}}(x)\;|\;
                        x\in{A}\textrm{ and }q\in\mathbb{Q^{+}}\}
            \end{equation}
        \end{proof}
        \begin{example}
            A common \textit{non-theorem} that confuses students is the belief
            that separable and first countable impliy second countable, and this
            is not true. What we've shown is that first countable and separable
            imply second countable if we're working with metric spaces. For
            example, consider the particular point topology on $\mathbb{R}$.
            That is, a set $\mathcal{U}$ is open if and only if it is either
            empty or contains the origin. Then $\topspace{\mathbb{R}}$ is
            first countable. To see this, let $x\in\mathbb{R}$ and consider:
            \begin{equation}
                \mathcal{B}_{x}=\big\{\,\{x,\,0\}\,\big\}
            \end{equation}
            this is a neighborhood basis for $x$, and hence the space is first
            countable. Moreover it is separable since
            $\closure{\{0\}}=\mathbb{R}$. To see that it is not second countable
            we'll show that it is not $\sigma$ locally finite. For suppose
            $\mathcal{O}$ is a locally finite collection. That $0$ is contained
            in finitely many elements of $\mathcal{O}$, and thus $\mathcal{O}$
            must itself be finite. But then for any countable collection of
            locally finite sets we have that this can't be a a basis since
            $\mathbb{R}$ is uncountable. Hence $\topspace{\mathbb{R}}$ is not
            $\sigma$ locally finite, and therefore it is not secound countable.
        \end{example}
        \begin{example}
            As another example, let $H$ be the closed upper half plane in
            $\mathbb{R}^{2}$. That is, all point $(x,y)\in\mathbb{R}^{2}$ such
            that $y\geq{0}$. Consider the following basis: open balls in the
            interior of the upper half plane combined with open balls that lie
            tangent to the $x$ axis together with the tangential point $(x,0)$.
            Then the intersection of $\mathbb{Q}^{2}$ with the upper haf plane
            still forms a countable dense subset, but any basis needs to contain
            at least one set for every $(x,0)$. Since $\mathbb{R}$ is
            uncountable, this space cannot possibly be second countable.
        \end{example}
        Def cont func.
        \begin{theorem}
            If $\topspace[X]{X}$ and $\topspace[Y]{Y}$ are topological spaces,
            and if $f:X\rightarrow{Y}$ is a continuous function, then for all
            $x\in{X}$ and for all $\mathcal{V}\in\tau_{Y}$ such that
            $f(x)\in\mathcal{V}$, there exists a $\mathcal{U}\in\tau_{X}$ such
            that $f[\mathcal{U}]\subseteq\mathcal{V}$.
        \end{theorem}
        $\varepsilon-\delta$ def of cont for metric space.
        \begin{example}
            There are bijective continuous function $f:X\rightarrow{Y}$ that
            are not homeomorphisms. Let $\nsphere[1]$ be the unit circle and
            $[0,1)$ be the semi-open interval with the usual inherited metric
            topology from $\nspace[]$. Define $f:[0,1)\rightarrow\nsphere[1]$
            by:
            \begin{equation}
                f(x)=\big(\cos(2\pi{x}),\sin(2\pi{x})\big)
            \end{equation}
            This is simply wrapping the semi-open interval up into a circle.
            Since the left endpoint 0 is included, $f$ is bijective. It's also
            continuous, however $f^{\minus{1}}$ is not continuous at the point
            $(1,0)\in\nsphere[1]$. The function $f^{\minus{1}}$ effectively
            tears $\nsphere[1]$ at the point, and thus it is not continuous.
            Rigorously, we see that $\nsphere[1]$ is compact and $[0,1)$ is not
            (Both are true by the Heine-Borel theorem), and so $f^{\minus{1}}$
            can't be continuous since continuous functions preserve compactness.
        \end{example}
        Def connected (open def, closed def, clopen def).
        \begin{theorem}
            If $\topspace{\nspace[]}$ is the standard topological space on
            $\nspace[]$, then it is connected.
        \end{theorem}
        \begin{theorem}
            If $\topspace{X}$ is a topological space, then it is disconnected
            if and only if there exists non-empty subsets $A,B\subseteq{X}$ such
            that:
            \begin{equation}
                \big(\closure{A}\cap{B}\big)\cup\big(A\cap\closure{B}\big)
                =\emptyset
            \end{equation}
        \end{theorem}
        \begin{theorem}
            If $\topspace{X}$ is a topological space, if
            $\mathcal{C}\subseteq\powset{X}$ is a collection of connected
            subsets of $X$, and if $\bigcap\mathcal{C}\ne\emptyset$, then
            $\bigcup\mathcal{C}$ is connected.
        \end{theorem}
        This theorem provides a rather crude way of showing that $\nspace$ is
        connected. Consider $\nspace$ as the union of all lines through the
        origin. Since we know that $\nspace[]$ is connected, and since all of
        these lines have non-empty intersection (they intersect at the origin),
        their union must again be connected. But this union is simply the
        entirety of $\nspace$, and so Euclidean space is connected.
        \begin{theorem}
            If $n\in\mathbb{N}$, $n>1$, if $\topspace{\nspace}$ is the standard
            Euclidean topological space, and if $\vector{x}\in\nspace$, then
            $\nspace\setminus\{\vector{x}\}$ is connected.
        \end{theorem}
        \begin{proof}
            For it is path connected, and hence connected (connect two dots).
        \end{proof}
        \begin{theorem}
            If $n\in\mathbb{N}$, if $n>0$, and if $\topspace{\nsphere}$ is the
            usual subspace topology on the sphere, then it is connected.
        \end{theorem}
        \begin{proof}
            For $\nspace{n+1}\setminus\{\vector{0}\}$ is connected by the
            previous theorem, and the function
            $f:\nspace{n+1}\setminus\{\vector{0}\}\rightarrow\nsphere$ defined
            by:
            \begin{equation}
                f(\vector{x})=\frac{\vector{x}}{\norm{\vector{x}}}
            \end{equation}
            is surjective and continuous, and hence it's image is connected. But
            since it is surjective, it's image is the entire sphere, and thus
            $\nsphere$ is connected.
        \end{proof}
        IVP. Comp of cont is cont. Intervals are connected.
        \begin{theorem}
            If $n\in\mathbb{N}$, $n>1$, and if $\topspace{\nspace{n+1}}$ is the
            standard Euclidean topological space, then
            $\nspace{n+1}\setminus\nsphere$ has two open connected components.
        \end{theorem}
        \begin{proof}
            For the open ball $\nball{n+1}$ is path connected, and
            $\nspace{n+1}\setminus\closure{\nball{n+1}}$ is path connected.
        \end{proof}
        Removing a hyperplane leaves two open connected components.
        \begin{theorem}
            If $\topspace{\ntorus}$ is the $n$ torus with the product topology,
            then it is connected.
        \end{theorem}
        \begin{proof}
            For the product of connected is connected, and $\nsphere[1]$ is
            connected.
        \end{proof}
        Def cover, open cover, compactness, compact subset
        (compact in subspace topology).
        \begin{theorem}
            if $\topspace{X}$ is a topological space, and if $A\subseteq{X}$,
            then $A$ is compact if and only if for every subset
            $\mathcal{O}\subseteq\tau$ such that $\mathcal{O}$ is a cover of
            $A$, there is a finite subcover $\Delta\subseteq\mathcal{O}$.
        \end{theorem}
        \begin{proof}
            For if $A$ is a compact subset, then $\topspace[A]{A}$ is a compact
            space where $\tau_{A}$ is the subspace topology. But then:
            \begin{equation}
                \mathcal{O}_{A}=\{\,A\cap\mathcal{U}\;|\;
                    \mathcal{U}\in\mathcal{O}\,\}
            \end{equation}
            is an open cover of $A$ in the subspace topology. But $A$ is
            compact, and thus there is a finite subcover. Yadda yadda.
        \end{proof}
        \begin{fdefinition}{Finite Intersection Property}
                           {Finite_Intersection_Property}
            A topological space with the finite intersection property is a
            topological space $\topspace{X}$ such that for any set
            $\mathcal{C}\subseteq\powset{X}$ of closed sets such that for any
            finite subsets $\Delta\subseteq\mathcal{C}$ it is true that
            $\bigcap\Delta\ne\emptyset$, it is also true that
            $\bigcap\mathcal{C}\ne\emptyset$.
        \end{fdefinition}
        \begin{theorem}
            If $\topspace{X}$ is a topological space, then it is compact if and
            only if it has the finite intersection property.
        \end{theorem}
        \begin{proof}
            For suppose $\topspace{X}$ is compact and does not have the finite
            intersection property. Then there exists a sets
            $\mathcal{C}\subseteq\powset{X}$ such that
            $\bigcap\mathcal{C}=\emptyset$, yet for every finite subset
            $\Delta\subseteq\mathcal{C}$ it is true that
            $\bigcap\mathcal{C}\ne\emptyset$. But if
            $\bigcap\mathcal{C}=\emptyset$, then for all $x\in{X}$ there exists
            $C\in\mathcal{C}$ such that $x\notin{C}$. But then the set
            $\mathcal{O}$ defined by:
            \begin{equation}
                \mathcal{O}=\big\{\,\mathcal{U}\in\tau\;|\;
                    \exists_{C\in\mathcal{C}}
                    \big(\mathcal{U}=X\setminus{C}\big)\,\big\}
            \end{equation}
            is an open cover of $X$. But $X$ is compact, and thus there is a
            finite subcover $\Lambda\subseteq\mathcal{O}$. But then the set
            $\Delta\subseteq\mathcal{C}$ defined by the complements of $\Lambda$
            is finite and has empty intersection, a contradiction. Thus,
            $X$ has the finite intersection property. Next, suppose $X$ has the
            finite intersection property but is not compact. Then there is an
            open cover $\mathcal{O}$ of $X$ with no finite subcover. But
            then the complements $\mathcal{C}$ are a collection of closed sets
            such that $\bigcap\mathcal{C}=\emptyset$. But $X$ has the finite
            intersection property, and thus there exists a finite subset
            $\Delta\subseteq\mathcal{C}$ such that $\bigcap\Delta=\emptyset$.
            But then the set of complements of $\Delta$ is a finite subcover
            of $\mathcal{O}$, a contradiction. Thus, $X$ is compact.
        \end{proof}
        \begin{fdefinition}{Limit Point Compact}{Limit_Point_Compact}
            A limit point compact topological space is a topological space
            $\topspace{X}$ such that for every infinite subset $A\subseteq{X}$,
            the derived set $\derived{A}$ is non-empty.
        \end{fdefinition}
        Limit point compact is also often called
        \textit{weakly countably compact}. We've adopted the name limit point
        compact since this seems common in analysis, and the great theorems
        about compactness in metric spaces (for example, the Bolzano-Weierstrass
        theorem or the generalized Heine-Borel theorem) are often stated in
        terms of sequences or limit points. Moroever, some authors choose to
        make no distinction between limit point compact and countably compact,
        which we shall describe in a moment. The reason being that in an
        accessible space (a $T_{1}$ topological space), limit point compact and
        countably compact are equivalent. Since most spaces that are studied in
        the wild are Hausdorff, they are automatically accessible, and hence
        limit point compact and countably compact are usually the same thing.
        As the name suggests, countably compact is a weakening of compact.
        \begin{fdefinition}{Countably Compact}{Countably_Compact}
            A countably compact topological space is a topological space
            $\topspace{X}$ such that for every countable open cover
            $\mathcal{O}$ of $X$ there exists a finite subcover.
        \end{fdefinition}
        \begin{theorem}
            If $\topspace{X}$ is a compact topological space, then it is
            countably compact.
        \end{theorem}
        \begin{proof}
            For if $\mathcal{O}$ is a countable open cover of $X$, then it is
            an open cover of $X$, and since $X$ is compact there exists a finite
            subcover. Hence, $\topspace{X}$ is countably compact.
        \end{proof}
        In metric spaces, this result reverses, which is quite astounding. This
        is tied into the fact that in metric spaces, sequentially compact and
        compact are one in the same. In general, a countably compact space is
        compact if and only if it is Lindel\"{o}f. This can be seen quite easily
        since given a cover, the Lindel\"{o}f property can reduce this down to
        a countable subcover, and countable compactness then extracts a finite
        subcover. The reverse direction is true since compact implies both
        Lindel\"{o}f and countably compact.
        \begin{theorem}
            If $\topspace{X}$ is a topological space, then it is compact if and
            only if it is countably compact and Lindel\"{o}f.
        \end{theorem}
        \begin{proof}
            For if $\mathcal{O}$ is an open cover, then since $X$ is
            Lindel\"{o}f there exists a countable open subcover $\Delta$. But
            $X$ is countably compact, and thus if $\Delta$ is a countable open
            cover, then there exists a finite subcover $\Lambda$. Hence, $X$
            is compact.
        \end{proof}
        Countably compact always implies limit point compact as well.
        \begin{theorem}
            If $\topspace{X}$ is a countably compact topological space, then it
            is limit point compact.
        \end{theorem}
        \begin{proof}
            For suppose not and let $A\subseteq{X}$ be an infinite set with no
            limit point. Then since $A$ is infinite, there exists a countable
            subset $N\subseteq{A}$. But then there is a bijection
            $a:\mathbb{N}\rightarrow{N}$. But if $A$ has no limit point and
            $A\subseteq{N}$, then $N$ has no limit point. But then for all
            $n\in\mathbb{N}$ there is a $\mathcal{U}_{n}$ such that
            $N\cap\mathcal{U}_{n}=\{a_{n}\}$. But then for all $x\in{X}$ there
            exists an open subset $\mathcal{U}_{x}$ such that
            $\mathcal{U}_{x}\cap{N}$ is finite. Let
            $\mathcal{O}=\{\mathcal{U}_{x}\}$. Let $\mathcal{V}_{n}$ be the
            union of the $\mathcal{U}_{x}$ that contain $a_{n}$. This is a
            countable cover, and since $X$ is countably compact there is a
            finite subcover. From this we conclude that $N$ is finite, a
            contradiction. Thus, $\topspace{X}$ is limit point compact.
        \end{proof}
        \begin{theorem}
            If $\topspace{X}$ is a compact topological space, then it is limit
            point compact.
        \end{theorem}
        \begin{proof}
            For compact spaces are countably compact, and countably compact
            spaces are limit point compact.
        \end{proof}
        \begin{theorem}
            If $\topspace{X}$ is an accessible $(T_{1})$ limit point compact
            topological space, then it is countably compact.
        \end{theorem}
        The fact that limit point compact is very weak in a general topological
        space can be seen by studying the extreme value. If
        $\topspace[X]{X}$ is compact and $\topspace[Y]{Y}$ is an order topology,
        then for any continuous function $f:X\rightarrow{Y}$ the range must be
        bounded. This still holds for countably compact sets, but fails in the
        case of limit point compact.
        \begin{example}
            Let $\topspace[\empty]{\mathbb{Z}_{2}}$ be the trivial topological
            space on 2 points, and let $\mathcal[\powset]{\mathbb{Z}}$ be the
            discrete topology. The product topology on
            $\mathbb{Z}_{2}\times\mathbb{Z}$ is then limit point compact and the
            projection mapping
            $\pi_{2}:\mathbb{Z}_{2}\times\mathbb{Z}\rightarrow\mathbb{Z}$ is
            continuous. But the discrete topology on $\mathbb{Z}$ is the same as
            it's order topology, but $\pi_{Z}$ is not bounded. Indeed, the range
            of $\pi_{Z}$ is the entirety of $\mathbb{Z}$, and hence the range
            isn't even limit point compact.
        \end{example}
        If we consider accessible spaces ($T_{1}$), all of our problems
        disappear. The product topology of a trivial space (with at least two
        points) with a discrete space will not be accessible, and therein lies
        the issue.
        \par\hfill\par
        Compact subset of Hausdorff is closed, countable finite intersection
        property. Product space, $\nspace[n]$, torus $\ntorus[2]$ as the product
        of $\nsphere[1]$ with itself. Draw pictures. Box vs. Product topology.
        Axiom of Choice, Volterra set $[x-y\in\mathbb{Q}]$.
        \begin{theorem}
            Assuming axiom of choice, if $S$ is a non-empty set of disjoint
            non-empty sets, then there is a set $A$ such that for all
            $B\in{S}$, $A\cap{B}$ contains one point.
        \end{theorem}
        Partial order, total order, well order, well ordering theorem. Hausdorff
        maximality theorem.
    \section{Freddy P Topology Lectures.}
    \section{Algebraic Topology Notes}
        We can define a weaker notion of equivalence for topological spaces.
        Homeomorphism is an ideal definition of sameness, but many spaces that
        are not homeomorphic still share many of properties. We wish to expand
        our study of topology by defining \textit{homotopy equivalence}. We will
        show that homeomorphisms are stronger (i.e., homeomorphic spaces are
        homotopy equivalent) and also demonstrate that the converse fails. The
        concept is very pictorial, so there will be plenty of figures.
        \begin{fnotation}{Set of Continuous Functions}
                         {Set of Continuous Functions}
            The set of continuous functions $f:{X}\rightarrow{Y}$
            is denoted $\Ckspace[Y]{}{X}$.
        \end{fnotation}
        We now define what it means for continuous functions $f$ and $g$
        from a topological space $\topspace[X]{X}$ to a space
        $\topspace[Y]{Y}$ to be \textit{homotopic}.
        \begin{fdefinition}{Homotopy}{Homotopy}
            A homotopy from a continuous function $f:X\rightarrow{Y}$ to a
            continuous function $g:X\rightarrow{Y}$ with respect to topological
            spaces $\topspace[X]{X}$ and $\topspace[Y]{Y}$ is a continuous
            function $H:{X}\times{I}\rightarrow{Y}$ such that $H(x,0)=f(x)$ and
            $H(x,1)=g(x)$, where $I=[0,1]$ is the closed unit interval, and
            $X\times{I}$ carries the product topology
            $\tau_{X}\times\tau_{\nspace[]}|_{I}$.
        \end{fdefinition}
        To clarify this definition, $\tau_{\nspace[]}$ is the standard Euclidean
        topology on $\nspace[]$, and $\tau_{\nspace[]}|_{I}$ is the subspace
        topology induced on the closed unit interval $I$. The picture goes as
        follows: Given two functions $f,g:X\rightarrow{Y}$ we draw $f$ along a
        continuous path in $X\times{I}$ until we obtain the function $g$
        (see Fig.~\ref{fig:Homotopy_Diagram_for_Depicting_Homotopy}). Homotopy
        is a very weak notion since, as we will soon see, for \textit{every}
        pair of continuous functions $f,g:\nspace[m]\rightarrow\nspace$ there
        exists a homotopy $H:\nspace[m]\times{I}\rightarrow\nspace$ which drags
        $f$ to $g$.
        \begin{figure}[H]
            \centering
            \captionsetup{type=figure}
            \documentclass[crop,class=article]{standalone}
%----------------------------Preamble-------------------------------%
\usepackage{tikz}                       % Drawing/graphing tools.
\usetikzlibrary{arrows.meta}            % Latex and Stealth arrows.
%--------------------------Main Document----------------------------%
\begin{document}
    \begin{tikzpicture}[%
        line width=1pt,
        line cap=round,>={Stealth[black]},
        every edge/.style={draw=black,very thick},
        smalldot/.style={
            circle,
            fill=black,
            inner sep=0pt,
            outer sep=0
        }
    ]
        \begin{scope}[every node/.style=smalldot]
            % Set points defining the leftmost blob.
            \node at (0,0) (a) {};
            \node at (-0.5,0.2) (b) {};
            \node at (-1,1) (c) {};
            \node at (-0.4,2) (d) {};
            \node at (0, 1.8) (e) {};
            \node at (0.5, 1.7) (f) {};
            \node at (1,1) (g) {};
            \node at (0.7,0.4) (h) {};

            % Set points defining the rightmost blob.
            \node at (3,0) (a1) {};
            \node at (2.5,0.4) (b1) {};
            \node at (2,1.2) (c1) {};
            \node at (2.6,2.1) (d1) {};
            \node at (3, 2.2) (e1) {};
            \node at (3.5, 1.7) (f1) {};
            \node at (4,1) (g1) {};
            \node at (3.7,0.4) (h1) {};
        \end{scope}

        % Labels for the blobs X and Y.
        \node at (0,1) (i) {$X$};
        \node at (3,1) (i1) {$Y$};

        % Node indicating this is a homotopy.
        \node at (1.5,1) (ho) {$H$};

        % Nodes for drawing arrows between curves.
        \node at (1.1,0.15) (t1) {};
        \node at (1.1,1.85) (t2) {};
        \node at (1.9,0.15) (t3) {};
        \node at (1.9,1.85) (t4) {};

        % Draw a Hobby curve creating leftmost blob.
        \draw (a) to [out=180,in=-40] (b)
                  to [out=140,in=-75] (c)
                  to [out=105,in=170] (d)
                  to [out=-10,in=160] (e)
                  to [out=-20,in=160] (f)
                  to [out=-20,in=90] (g)
                  to [out=-90,in=50] (h)
                  to [out=-130,in=0] cycle;

        % Draw Hobby curve creating rightmost blob.
        \draw (a1) to [out=170,in=-50] (b1)
                   to [out=130,in=-80] (c1)
                   to [out=100,in=-150] (d1)
                   to [out=30,in=170] (e1)
                   to [out=-10,in=130] (f1)
                   to [out=-50,in=90] (g1)
                   to [out=-90,in=45] (h1)
                   to [out=-135,in=-10] cycle;

        % Draw arrows representing f and g.
        \path[shorten >=0.2cm,shorten <=0.2cm,->]
            (i) edge[bend left=40] node[above] {$f$} (i1);
        \path[shorten >=0.2cm,shorten <=0.2cm,->]
            (i) edge[bend right=40] node[below] {$g$} (i1);

        % Draw first arrow connecting f and g.
        \path[draw=black,dashed,<->]
            (t1) edge[bend right =10,semithick] (t2);

        % Draw second arrow connecting f and g.
        \path[draw=black,dashed,<->]
            (t3) edge[bend left =10,semithick] (t4);
    \end{tikzpicture}
\end{document}
            \caption{Homotopy Between Two Functions}
            \label{fig:Homotopy_Diagram_for_Depicting_Homotopy}
        \end{figure}
        \begin{theorem}
            If $\topspace[X]{X}$ is a topological space, if $\topvecspace[Y]{Y}$
            is a topological vector space over the real numbers, and if
            $f,g:X\rightarrow{Y}$ are continuous functions, then there is a
            homotopy $H:X\times{I}\rightarrow{Y}$ from $f$ to $g$.
        \end{theorem}
        \begin{proof}
            For let $H$ be defined by:
            \begin{equation}
                H(x,t)=(1-t)\cdot{f}(x)\vecadd[Y]t\cdot{g}(x)
            \end{equation}
            Since $Y$ is a topological vector space, $H$ is well defined for all
            $x\in{X}$ and $t\in[0,1]$. Moreover, $H(x,0)=f(x)$, $H(x,1)=g(x)$
            and $H$ is continuous. Hence, $H$ is a homotopy taking $f$ to $g$
            (Def.~\ref{def:Homotopy}).
        \end{proof}
        \begin{example}
            The simplest homotopy involves convex subsets of Euclidean spaces
            where we may again define the \textit{straight line} homotopy. Given
            two continuous functions $f,g:X\rightarrow\mathcal{U}$ from a
            topological space $X$ to a convex subset of $\nspace$, we again
            write $H(x,t)=f(x)(1-t)+g(x)t$. This is well defined since by
            hypothesis the space is convex and $H(x,t)$ represents a straight
            line between $f(x)$ and $g(x)$ for each point $x\in{X}$. Thus, $H$
            is a homotopy from $f$ to $g$. While it is perhaps to much to ask
            one to only consider convex spaces, any topological space that is
            homeomorphic to a convex subset of $\nspace$ can inherit this
            homotopy by using function composition with the hypothesized
            homeomorphism.
        \end{example}
        With homotopy now defined, we redundantly define \textit{homotopic}
        functions. Two avoid abuse of language we should first show that if
        the is a homotopy from $f$ to $g$, then there is a homotopy from $g$ to
        $f$. That is, we need not specify which function is the \textit{first}
        one.
        \begin{ltheorem}{Reflexivity of Homotopy}
            If $\topspace[X]{X}$ and $\topspace[Y]{Y}$ are topological spaces,
            if $f,g:X\rightarrow{Y}$ are continuous, and if
            $H:X\times{I}\rightarrow{Y}$ is a homotopy taking $f$ to $g$, then
            there exists a homotopy $G:X\times{I}\rightarrow{Y}$ taking $g$ to
            $f$.
        \end{ltheorem}
        \begin{proof}
            For let $G:X\times{I}\rightarrow{Y}$ be defined by:
            \begin{equation}
                G(x,t)=H(x,1-t)
            \end{equation}
            for all $x\in{X}$ and $t\in[0,1]$. Then $G$ is the composition of
            continuous functions and is therefore continuous. Moreover,
            $G(x,0)=H(x,1)=g(x)$ and $G(x,1)=H(x,0)=f(x)$. Hence $G$ is a
            homotopy taking $g$ to $f$ (Def.~\ref{def:Homotopy}).
        \end{proof}
        \begin{fdefinition}{Homotopic Functions}{Homotopic_Functions}
            Homotopic functions are continuous functions
            $f,g:{X}\rightarrow{Y}$, denoted ${f}\simeq{g}$,
            such that there exists a homotopy between them.
        \end{fdefinition}
        \begin{lexample}{Straight Line Homotopy}{Straight_Line_Homotopy}
            Fig.~\ref{fig:Homotopy_Diagram_for_Depicting_Homotopy} shows
            two topological spaces and two homotopic continuous functions.
            For a more concrete example, let $f,g:\nspace[m]\rightarrow\nspace$
            be continuous. The \textit{straight-line} homotopy is a homotopy
            between such functions. Define
            $H:\nspace\times{I}\rightarrow\nspace[m]$ by:
            \begin{equation}
                \label{eqn:Straight_Line_Homotopy}%
                H(x,t)=(1-t)f(x)+tg(x)
            \end{equation}
            Then $H(x,0)=f(x)$, $H(x,1)=g(x)$, and $H$ is continuous. Thus,
            ${f}\simeq{g}$. Note that $g(x)=constant$ is possible. Any
            continuous function $f:\mathbb{R}^{n}\rightarrow\mathbb{R}^{m}$
            is homotopic to a point.
        \end{lexample}
        We can visualize homotopy by letting $X=I$ and $Y\subset\nspace[2]$ be a
        nice blob, like the one shown in Fig.~\ref{fig:straight_line_homotopy}.
        Let $f,g:[0,1]\rightarrow Y$ be paths within the blob. Then the homotopy
        defined in Eqn.~\ref{eqn:Straight_Line_Homotopy} is the map that
        drags $f(x)$ to $g(x)$ via the straight line connecting the two points.
        This is done for every point $x\in [0,1]$.
        We now show that homotopic is an equivalence relation on
        $\Ckspace[Y]{}{X}$.
        \begin{figure}[H]
            \centering
            \captionsetup{type=figure}
            \documentclass[crop,class=article]{standalone}
%----------------------------Preamble-------------------------------%
\usepackage{tikz}                       % Drawing/graphing tools.
\usetikzlibrary{arrows.meta}            % Latex and Stealth arrows.
%--------------------------Main Document----------------------------%
\begin{document}
    \begin{tikzpicture}[%
        line width=1pt,
        line cap=round,
        >={Stealth[black]},
        every edge/.style={draw=black,very thick},
        smalldot/.style={
            circle,
            fill=black,
            inner sep=0pt,
            outer sep=0
        },
        dashcurve/.style={
            draw=black,
            dashed,
            <->
        }
    ]
        \begin{scope}[every node/.style=smalldot]
            % Set points for upper curve.
            \node at (2.5,1.3) (a0) {};
            \node at (2.7,1.4) (b0) {};
            \node at (2.9,1.7) (c0) {};
            \node at (3.2,1.8) (d0) {};
            \node at (3.4,1.9) (e0) {};

            % Set points for lower curve.
            \node at (2.8,0.6) (a1) {};
            \node at (3,0.6) (b1) {};
            \node at (3.3,0.5) (c1) {};
            \node at (3.5,0.9) (d1) {};
            \node at (3.7,1.2) (e1) {};

            % Points for the outer blob.
            \node at (3,0) (a2) {};
            \node at (2.5,0.4) (b2) {};
            \node at (2,1.2) (c2) {};
            \node at (2.6,2.1) (d2) {};
            \node at (4.2, 2.4) (e2) {};
            \node at (4.5, 2) (f2) {};
            \node at (4,1) (g2) {};
            \node at (3.7,0.4) (h2) {};
        \end{scope}

        % Nodes labelling the domain and co-domain.
        \node at (0,1) (i) {$[0,1]$};
        \node at (4,1.9) (i1) {$Y$};

        % Draw upper curve.
        \draw (a0) to [out=15,in=-135] (b0)
                   to [out=45,in=-130] (c0)
                   to [out=50,in=-170] (d0)
                   to [out=10,in=-135] (e0);

        % Draw lower curve.
        \draw (a1) to [out=0,in=170] (b1)
                   to [out=-10,in=170] (c1)
                   to [out=-10,in=-100] (d1)
                   to [out=80,in=-160] (e1);

        \begin{scope}[%
            every path/.style=dashcurve,
            every edge/.style=semithick
        ]
            % Draw dashed lines connecting curves.
            \path (a0) edge (a1);
            \path (b0) edge (b1);
            \path (c0) edge
                node[inner sep=0pt,outer sep=0pt,fill=white]
                {\scriptsize{\textit{H}}} (c1);
            \path (d0) edge (d1);
            \path (e0) edge (e1);
        \end{scope}

        % Draw curve defining the blob.
        \draw (a2) to [out=170,in=-45] (b2)
                   to [out=135,in=-85] (c2)
                   to [out=95,in=-150] (d2)
                   to [out=30,in=150] (e2)
                   to [out=-30,in=100] (f2)
                   to [out=-80,in=60] (g2)
                   to [out=-120,in=60] (h2)
                   to [out=-120,in=-10] cycle;


        % Draw curves representing maps f and g.
        \path[shorten >=0.2cm,shorten <=0.2cm,->]
            (i) edge[bend left=40]
            node[above] {$f$} (c0);
        \path[shorten >=0.2cm,shorten <=0.2cm,->]
            (i) edge[bend right=40]
            node[below] {$g$} (c1);
    \end{tikzpicture}
\end{document}
            \caption{Straight-Line Homotopy}
            \label{fig:straight_line_homotopy}
        \end{figure}
        \begin{theorem}
            If $\topspace[X]{X}$ and $\topspace[Y]{Y}$ are topological spaces,
            if $R$ is the relation on $\Ckspace[Y]{}{X}$ defined by:
            \begin{equation}
                R=\{\,(f,g)\in\Ckspace[Y]{}{X}^{2}\;|\;
                    f\textrm{ is homotopic to }g\,\}
            \end{equation}
            then $R$ is an equivalence relation.
        \end{theorem}
        \begin{proof}
            For $R$ is reflexive. Indeed, if $f\in\Ckspace[Y]{}{X}$, then let
            $H:X\times{I}\rightarrow{Y}$ be defined by $H(x,t)=f(x)$ for all
            $x\in{X}$, $t\in{I}$. Then $H$ is continuous, $H(x,0)=f(x)$ and
            $H(x,1)=f(x)$. Hence, $f$ is homotopic to itself and therefore
            $fRf$. Moreover, $R$ is symmetric. If $f,g\in\Ckspace[Y]{}{X}$ are
            homotopic, $fRg$, then there exists a continuous function
            $H:X\times{I}\rightarrow{Y}$ such that $H(x,0)=f(x)$ and
            $H(x,1)=g(x)$ for all $x\in{X}$
            (Def.~\ref{def:Homotopic_Functions}). Let
            $G:X\times{I}\rightarrow{Y}$ be defined by
            $G(x,t)=H(x,1-t)$. Then $G$ is the composition of continuous
            functions and is therefore continuous. But moreoever,
            $G(x,0)=H(x,1)=g(x)$ and $G(x,1)=H(x,0)=f(x)$, and therefore
            $G$ is a homotopy between $g$ and $f$. That is, $gRf$. Lastly, $R$
            is transitive. If $f,g,h\in\Ckspace[Y]{}{X}$, if $f$ is homotopic to
            $g$, and if $g$ is homotopic to $h$, then there exists continuous
            functions $H_{1},H_{2}:X\times{I}\rightarrow{Y}$ such that $H_{1}$
            is a homotopy from $f$ to $g$, and $H_{2}$ is a homotopy from $g$ to
            $H$. Let $H:X\times{I}\rightarrow{Y}$ be defined by:
            \begin{equation}
                H(x,t)=
                \begin{cases}
                    H_{1}(x,2t),&{0}\leq{t}\leq\frac{1}{2}\\
                    H_{2}(x,2t-1),&\frac{1}{2}<{t}\leq{1}
                \end{cases}
            \end{equation}
            By the pasting lemma, $H$ is continuous. But from the definition of
            $H_{1}$ and $H_{2}$, $H(x,0)=f(x)$ and $H(x,1)=h(x)$. Thus,
            $f$ is homotopic to $h$ and hence $R$ is an equivalence relation.
        \end{proof}
        \begin{ldefinition}{Homotopy Inverse}{Homotopy_Inverse}
            A homotopy inverse of a function $f\in{C}(X,Y)$ is a function
            $g\in{C}(Y,X)$ such that $g\circ{f}\simeq{id}_{X}$ and
            $f\circ{g}\simeq{id}_{Y}$.
        \end{ldefinition}
        \begin{theorem}
            If $X$ and $Y$ are topological spaces, $f:X\rightarrow{Y}$ is
            continuous, and if $g_{1}$ and $g_{2}$ are homotopy inverses
            of $f$, then $g_{1}\simeq{g}_{2}$.
        \end{theorem}
        \begin{proof}
            Since $g_{2}$ is a homotopy inverse of $f$,
            $f\circ{g}_{2}\simeq{id}_{Y}$. But then:
            \begin{equation}
                g_{1}\simeq{g}_{1}\circ(f\circ{g}_{2})
                =(g_{1}\circ{f})\circ{g}_{2}
            \end{equation}
            But $g_{1}$ is a homotopy of inverse of $f$,
            and thus $g_{1}\circ{f}\simeq{id}_{X}$. Thus:
            \begin{equation}
                (g_{1}\circ{f})\circ{g}_{2}\simeq{g}_{2}
            \end{equation}
            Since homotopic is a transitive relation, $g_{1}\simeq{g}_{2}$.
        \end{proof}
        \begin{ldefinition}{Homotopy Equivalence}{Homotopy_Equivalence}
            A homotopy equivalence from a topological space $X$ to a
            topological space $Y$ is a function $f\in{C}(X,Y)$ such that
            there exists a homotopy inverse $g$ of $f$.
        \end{ldefinition}
        \begin{ldefinition}{Homotopy Equivalent Spaces}
                           {Homotopy_Equivalent_Spaces}
            Homotopy equivalent spaces are topological spaces $X$ and $Y$
            such that there exists functions ${f}\in{C(X,Y)}$ and
            ${g}\in{C(Y,X)}$ such that ${f}\circ{g}\simeq{id_{Y}}$
            and ${g}\circ{f}\simeq{id_{X}}$.
        \end{ldefinition}
        From the definition of homotopy equivalent spaces it is important
        to note that it is not required that $g\circ{f}=id_{X}$, but rather
        that $g\circ{f}$ is \textit{homotopic} to the identity map $id_{X}$.
        Similarly, $f\circ{g}$ need only be homotopic to the identity map
        $id_{Y}$, and not equal to it. We can rephrase this by saying that
        homotopy equivalent spaces are topological spaces $X$ and $Y$ such
        that there exists a homotopy equivalence $f:X\rightarrow{Y}$
        between the two.
        \par\hfill\par
        Let's look at an example of what type of operations are allowed by
        homotopy equivalences. We'll start with a torus $\ntorus[]$ the has a
        disk inside of it (see Fig.~\ref{fig:Torus_with_Disc_Inside}).
        \begin{figure}[H]
            \centering
            \captionsetup{type=figure}
            \includegraphics{images/Torus_Wireframe_Gradient.pdf}
            \caption{Torus with a Disc Inside}
            \label{fig:Torus_with_Disc_Inside}
        \end{figure}
        The next step we'll do is shrink the entire disc down to a point,
        leaving the rest of the torus alone (Fig.~\ref{fig:Squished_Torus}).
        Such moves are not permitted by homeomorphisms since we are squeezing
        infinitely many elements together and identifying them with a single
        point, whereas homeomorphisms must be bijective per definition.
        \textit{Homotopy} does allow us to perform such an operation.
        \begin{figure}[H]
            \centering
            \captionsetup{type=figure}
            \includegraphics{images/Squished_Torus_3D.pdf}
            \caption{A Squished Torus}
            \label{fig:Squished_Torus}
        \end{figure}
        This can be made more concrete with
        equations since parametrizations of the torus are well known. We have:
        \begin{equation}
            F(\theta,\varphi)=\Big(
                \big(R+r\cos(\varphi)\big)\cos(\theta),\,
                \big(R+r\cos(\varphi)\big)\sin(\theta),\,
                r\sin(\varphi)\Big)
        \end{equation}
        Where $R$ and $r$ are the outer and inner radii, respectively. The
        squished torus can be obtain by shrinking the $x$ and $z$ coordinates
        as $y$ varies. The code used to generate Fig.~\ref{fig:Squished_Torus}
        adopted the following parametrization:
        \begin{equation}
                G(\theta,\varphi)=\Big(
                    \big(R+r\cos(\varphi)g(\theta)\big)\cos(\theta),\,
                    \big(R+r\cos(\varphi)\big)\sin(\theta),\,
                    r\sin(\varphi)g(\theta)\Big)
        \end{equation}
        \begin{figure}[H]
            \centering
            \captionsetup{type=figure}
            \includegraphics{images/Sphere_with_String_at_Poles_3D.pdf}
            \caption{A Sphere with String Attached}
            \label{fig:Sphere_with_String_Attached}
        \end{figure}
        where $g(\theta)=\sin(\theta/2)$. This shrinks the torus to a point when
        $\theta=0$ and has no effect at $\theta=\pi$, varying continuously in
        between which is precisely what we want. Now, one can imagine taking
        our croissant and stretching out the collapses point into a line,
        obtaining a half-moon with a line connecting the poles. We can then
        deform the crescent part of our object into a sphere, obtaining the
        figure shown in Fig.~\ref{fig:Sphere_with_String_Attached}. From here we
        can proceed and drag the endpoints of the string down towards the
        equator and obtain a kettle bell (see Fig.~\ref{fig:Kettle_Bell}).
        \begin{figure}[H]
            \centering
            \captionsetup{type=figure}
            \includegraphics{images/Kettle_Bell_3D.pdf}
            \caption{A Kettle Bell}
            \label{fig:Kettle_Bell}
        \end{figure}
        \begin{ltheorem}{Homeomorphic Implies Homotopy Equivalent}
                        {Homeomorphic_Implies_Homotopy_Equivalent}
            If $X$ and $Y$ are homeomorphic, then they are homotopy equivalent.
        \end{ltheorem}
        \begin{proof}
            If $X$ and $Y$ are homeomorphic topological spaces, then there
            is a homeomorphism $f:X\rightarrow Y$. But then $f$ is a
            continuous map from $X$ to $Y$, and $f^{-1}$ is a continuous
            map from $Y$ to $X$. Moreover, ${f}\circ{f^{-1}}=id_{Y}$, and
            ${f^{-1}}\circ{f}=id_{X}$, for $f$ is a bijection. But
            ${id_{X}}\simeq{id_{X}}$, and ${id_{Y}}\simeq{id_{Y}}$.
            Therefore, etc.
        \end{proof}
        The converse is not true, since homotopy equivalence is a very weak
        notion of equivalence as we will soon demonstrate. Many tools in the
        study of surgery theory asks about the converse of
        Thm.~\ref{thm:Homeomorphic_Implies_Homotopy_Equivalent}.
        \begin{theorem}
            \label{thm:homotopic_does_not_imply_homeomorphic}%
            There exist topological spaces that are homotopy equivalent but not
            homeomorphic.
        \end{theorem}
        \begin{proof}
            Let $X=\mathbb{R}^{2}$ and $Y=\{(0,0)\}$. Let $f:{X}\rightarrow{Y}$
            be defined by $f(x,y)=(0,0)$ and $g=\identity{Y}$. Then
            $g\circ{f}=(0,0)$. Let $H(x,y,t)=(1-t)(x,y)$.
            Then $H$ is continuous, $H(x,y,0)=(x,y)$,
            and $H(x,y,1)=(0,0)$. Thus, $H$ is a
            homotopy between ${g}\circ{f}$ and $id_{X}$, and
            therefore ${g}\circ{f}\simeq{id_{X}}$. But
            ${f}\circ{g}=id_{Y}$, and ${id_{Y}}\simeq{id_{Y}}$.
            Thus $X$ and $Y$ are homotopy equivalent.
            If $h:{X}\rightarrow{Y}$ is a homeomorphism, then it is a
            bijection. But then $\Card(X)=\Card(Y)$,
            where $\Card$ denotes the \textit{cardinality} of the space.
            But $\mathbb{R}^{2}$ is uncountable, and $\Card(Y)=1$,
            a contradiction. Therefore $X$ and $Y$ are not homeomorphic.
        \end{proof}
        \begin{figure}[H]
            \captionsetup{type=figure}
            \centering
            \begin{tikzpicture}[%
    scale=0.9,
    line width=1pt,
    line cap=round,
    >={Stealth[black]},
    every edge/.style={%
        draw=black,
        very thick
    },
    grayarrow/.style={%
        >=stealth,
        fill=gray,
        draw=gray,
        line width=0.7mm,
        ->
    }
]
    \filldraw[%
        even odd rule,
        inner color=gray,
        outer color=white,
        draw=white
    ] (0,0) circle (2);
    \draw[thick, <->] (-1.5,0)--(1.5,0);
    \draw[thick, <->] (0,-1.5)--(0,1.5);
    \draw[grayarrow] (1,1)--(0.5,0.5);
    \draw[grayarrow] (-1,-1)--(-0.5,-0.5);
    \draw[grayarrow] (1,-1)--(0.5,-0.5);
    \draw[grayarrow] (-1,1)--(-0.5,0.5);
    \node[%
        fill=black,
        circle,
        thick,
        draw,
        inner sep=2pt,
        outer sep=3pt
    ]
        at (0,0) (O) {};
    \node at (1.4,0) [below] {$x$};
    \node at (0,1.4) [right] {$y$};
\end{tikzpicture}
            \caption{Retraction of $\nspace[2]$ to $(0,0)$}
            \label{fig:homotopy_equivalence_of_plane_with_point}
        \end{figure}
        Fig.~\ref{fig:homotopy_equivalence_of_plane_with_point}
        shows the mapping $f$ between $\mathbb{R}^{2}$ and $\{(0,0)\}$.
        Thm.~\ref{thm:homotopic_does_not_imply_homeomorphic} relies on the
        fact that $\mathbb{R}^{2}$ and $\{(0,0)\}$ are of different
        \textit{cardinality}. However, even if the topological spaces $X$
        and $Y$ are homotopy equivalent, and are of the same cardinality,
        it is still possible that they are not homeomorphic. We will need
        to show that homeomorphisms preserve the notion of \textit{compactness}.
        \begin{ldefinition}{Compact Subsets}{Compact_Subsets}
            A compact subset of a topological space $X$ is a set $A\subseteq{X}$
            such that for every open cover $\mathcal{O}$ of $A$, there is a
            finite subcover $\Delta\subseteq\mathcal{O}$.
        \end{ldefinition}
        \begin{theorem}
            If $X$ is compact and $S\subset{X}$ is closed, then $S$ is compact.
        \end{theorem}
        \begin{proof}
            For let $\mathcal{O}$ be an open cover of $S$. Then since $S$ is
            closed, $S^{C}$ is open. But then $\mathcal{O}\cup\{S^{C}\}$ is
            an open cover of $X$. But $X$ is compact and therefore there is
            an open subcover $\Delta\subseteq\mathcal{O}\cup\{S^{C}\}$. But
            then $\Delta\setminus\{S^{C}\}$ is an finite subcover of $S$.
        \end{proof}
        \begin{theorem}
            If $a,b\in\mathbb{R}$ and $a<b$, then $[a,b]$ is compact.
        \end{theorem}
        \begin{proof}
            For suppose not. Then there is an oper cover $\mathcal{O}$ of
            $[a,b]$ with no finite subcover. Let $A$ be the set
            $A=\{r\in\mathbb{R}:[a,r]\textrm{ has a finite subcover}\}$.
            As $\mathcal{O}$ is an open cover, there is an open subset
            $\mathcal{U}_{1}\in\mathcal{O}$ such that $a\in\mathcal{U}_{1}$.
            Therefore $A$ is not empty. Moreoever, since $[a,b]$ is not compact,
            for all $r\in{A}$, $r<b$. Therefore $A$ is bounded above. By the
            least upper bound property there is a $\gamma\in\mathbb{R}$ such
            that for all $r\in{A}$, $r\leq\gamma$. But, as $\mathcal{U}_{1}$ is
            open and $a\in\mathcal{U}_{1}$, $a<\gamma\leq{b}$. But then
            $\gamma\in[a,b]$, and thus there is a
            $\mathcal{U}_{2}\in\mathcal{O}$ such that
            $\gamma\in\mathcal{U}_{2}$. But as $\mathcal{U}_{2}$ is open, there
            is an $\varepsilon>0$ such that
            $(\gamma-\varepsilon,\gamma+\varepsilon)\subset\mathcal{U}_{2}$.
            But then $[a,\gamma+\varepsilon/2]$ has a finite subcover, a
            contradiction since $\gamma$ is the least upper bound of $A$.
        \end{proof}
        \begin{theorem}
            \label{thm:product_of_compact_is_compact}%
            If $A$ and $B$ are compact, then $A\times{B}$ is compact.
        \end{theorem}
        \begin{proof}
            For let $\mathcal{O}$ be an open cover of $A\times{B}$. Then
            $\{\pi_{A}(\mathcal{U}):\mathcal{U}\in\mathcal{O}\}$, that is,
            the set of projections of open sets in $\mathcal{O}$ onto $A$,
            is an open cover of $A$. Similarly for $B$. But $A$ and $B$ are
            compact, and therefore there exists finite subcovers. Taking
            the union of these two gives a finite subcover of $A\times{B}$.
        \end{proof}
        \begin{theorem}
            \label{thm:finite_product_of_compact_is_compact}%
            If $A_{1},\hdots,A_{n}$ are compact, then
            $A_{1}\times\cdots\times{A_{n}}$ is compact.
        \end{theorem}
        \begin{proof}
            Apply induction to Thm.~\ref{thm:product_of_compact_is_compact}.
        \end{proof}
        The finiteness of the product in
        Thm.~\ref{thm:finite_product_of_compact_is_compact} is unnecessary.
        Tychonoff's Theorem, which is equivalent to the axiom of choice, says
        given an arbitrary collection of compact sets, the space formed by the
        product of these sets is also compact with respect to the product
        topology. We can now prove our main result.
        \begin{ftheorem}{Heine-Borel Theorem}{Heiner_Borel_Theorem}
            A subset $S\subset\nspace$ is compact if
            and only if it closed and bounded.
        \end{ftheorem}
        \begin{proof}
            Suppose $S$ is compact and suppose it is unbounded. Then the
            set of open balls about the origin
            $B_{r}(0)=\{\mathbf{x}\in\mathbb{R}^{n}:\norm{\mathbf{x}}<r\}$
            is an open cover of $S$, since it is an open cover of
            $\mathbb{R}^{n}$, and yet no finite subcover exists. For if
            one did, then there is a least $N\in\mathbb{N}$ such that
            $S\subset{B_{N}(0)}$, a contradiction as $S$ is unbounded.
            Therefore $S$ is bounded. Furthermore, suppose $S$ is
            not closed. Then there exists a point $\mathbf{x}\in{S^{C}}$
            such that, for all $r>0$,
            $B_{r}(\mathbf{x})\cap{S}\ne\emptyset$, where
            $B_{r}(\mathbf{x})=\{\mathbf{y}\in\mathbb{R}^{n}:%
             \norm{\mathbf{x}-\mathbf{y}}<r\}$.
            Let $\overline{B}_{r}(\mathbf{x})$ be the closure of these sets
            (That is, the closed ball about $\mathbf{x}$). Then the set of
            complements $\overline{B}_{r}(\mathbf{x})^{C}$ is an open cover
            of of $S$, for it is an open cover of
            $\mathbb{R}^{n}\setminus\{\mathbf{x}\}$, but no finite subcover
            exists, a contradiction. Thus $S$ is closed. Therefore, if $S$
            is compact then it is closed and bounded. If $S$ is bounded,
            then there is an $r\in\mathbb{R}$ such that
            $S\subset[-r,r]^{n}$. But $[-r,r]^{n}$ is the product of compact
            sets, and is therefore compact. But $S$ is closed, and closed
            subsets of compact spaces are compact. Therefore $S$ is compact.
        \end{proof}
        This will help find examples and counterexamples for the converse of
        Thm.~\ref{thm:Homeomorphic_Implies_Homotopy_Equivalent}.
        \begin{ltheorem}{Homeomorphisms Preserve Compactness}
                        {Homeomorphisms_Preserve_Compactness}
            If $X$ and $Y$ are homeomorphic, and if $X$ is compact,
            then $Y$ is compact.
        \end{ltheorem}
        \begin{proof}
            For if $X$ and $Y$ are homeomorphic, then there
            is a continuous bijection $f:X\rightarrow{Y}$.
            Let $\mathcal{O}$ be an open cover of $Y$.
            Then $\{f^{-1}(\mathcal{U}):\mathcal{U}\in\mathcal{O}\}$
            is an open cover of $X$. But $X$ is compact, and therefore
            there is a finite subcover $\Delta$. But, since $f$ is
            surjective,
            $\{\mathcal{U}\in\mathcal{O}:f^{-1}(\mathcal{U})\in\Delta\}$
            is a finite subcover of $Y$.
        \end{proof}
        The fact that $f$ is a homeomorphism is somewhat overkill.
        All that was necessary was that $f:X\rightarrow{Y}$ is continuous
        and surjective. Since $f$ is a homeomorphism, $f^{-1}$ is also
        continuous, and thus $X$ and $Y$ are compact, or not, together.
        \begin{theorem}
            \label{thm:Homotopy_Equivalance_of_Plane_without_point_%
                   and_unit_disc_but_not_homeomorphic}
            There exists topological spaces that have the same cardinality,
            are homotopy equivalent, but not homeomorphic.
        \end{theorem}
        \begin{proof}
            For let $X=\mathbb{R}^{2}\setminus\{(0,0)\}$,
            and let $Y=S^{1}$, where $S^{1}$ is the unit circle:
            \begin{equation}
                S^{1}=\{(x,y)\in\mathbb{R}^{2}:x^{2}+y^{2}=1\}
            \end{equation}
            Then $\Card(X)=\Card(Y)=\Card(\mathbb{R})$, and thus
            both sets are of the same cardinality. Moreover they
            are homotopy equivalent. For let $f:{X}\rightarrow{Y}$ and
            $g:{Y}\rightarrow{X}$ be defined by:
            \par
            \begin{subequations}
                \begin{minipage}[b]{0.49\textwidth}
                    \centering
                    \begin{equation}
                        f(x,y)=\frac{(x,y)}{\norm{(x,y)}}
                    \end{equation}
                \end{minipage}
                \hfill
                \begin{minipage}[b]{0.49\textwidth}
                    \centering
                    \begin{equation}
                        g(x,y)=(x,y)
                    \end{equation}
                \end{minipage}
            \end{subequations}
            \par\vspace{2.5ex}
            Define the function $H:X\times{I}\rightarrow{Y}$ by:
            \begin{equation}
                H(x,y,t)=(1-t)f(x,y)+tg(x,y)
            \end{equation}
            But then $H(x,y,0)=f(x,y)$, and $H(x,y,1)=g(x,y)$. Thus $H$ is a
            homotopy between ${g}\circ{f}$ and $id_{X}$. But also
            $({f}\circ{g})(x,y)=(x,y)$, for all $(x,y)\in S^{1}$.
            Thus ${f}\circ{g}=id_{Y}$, and ${id_{Y}}\simeq{id_{Y}}$,
            and therefore $X$ and $Y$ are homotopy equivalent.
            But $X$ is unbounded, and is therefore not compact,
            and $Y$ is closed and bounded, and is thus compact.
            But homeomorphisms preserve compactness. Therefore $X$
            and $Y$ are not homeomorphic.
        \end{proof}
        Thm.~\ref{thm:Homotopy_Equivalance_of_Plane_without_point_%
                  and_unit_disc_but_not_homeomorphic}
        relies on the fact that $S^{1}$ is compact and
        $\mathbb{R}^{2}\setminus\{(0,0)\}$ isn't. However, even if $X$ and
        $Y$ are both compact, and of the same cardinality, it is possible
        that they are homotopy equivalent but not homeomorphic. We'll need
        some results about connectedness to show this.
        \begin{figure}[H]
            \captionsetup{type=figure}
            \centering
            \documentclass[crop,class=article]{standalone}
%----------------------------Preamble-------------------------------%
\usepackage{amsfonts}                   % Blackboard Bold R.
\usepackage{tikz}                       % Drawing/graphing tools.
\usetikzlibrary{arrows.meta}            % Latex and Stealth arrows.
%--------------------------Main Document----------------------------%
\begin{document}
    \begin{tikzpicture}[%
        scale=0.9,
        line width=1pt,
        line cap=round,
        >={Stealth[black]},
        every edge/.style={%
            draw=black,
            very thick
        },
        grayarrow/.style={%
            >=stealth,
            fill=gray,
            draw=gray,
            line width=0.7mm,
            ->
        }
    ]
        \filldraw[%
            even odd rule,
            inner color=gray,
            outer color=white,
            draw=white
        ]
            (0,0) circle (2);

        % Draw axes.
        \draw[<->] (-1.5,0) -- (1.5,0);
        \draw[<->] (0,-1.5) -- (0,1.5);
        \draw[<->] (4,0) -- (7,0);
        \draw[<->] (5.5,-1.5) -- (5.5,1.5);

        % Draw gray arrows indicating homotopy equivalence.
        \begin{scope}[every edge/.style=grayarrow]
            \draw(1.1,1.1) edge (0.75,0.75);
            \draw(-1.1,-1.1) edge (-0.75,-0.75);
            \draw(1.1,-1.1) edge (0.75,-0.75);
            \draw(-1.1,1.1) edge (-0.75,0.75);
            \draw(0.2,0.2) edge (0.6,0.6);
            \draw(-0.2,-0.2) edge (-0.6,-0.6);
            \draw(0.2,-0.2) edge (0.6,-0.6);
            \draw(-0.2,0.2) edge (-0.6,0.6);
        \end{scope}

        \draw[dashed,draw=black,semithick] (0,0) circle (1);
        \node[%
            fill=white,
            circle,
            thick,
            draw,
            inner sep=2pt,
            outer sep=3pt
        ]
            at (0,0) (O) {};
        \node at (1.4,0) [below] {$x$};
        \node at (0,1.4) [right] {$y$};
        \node at (1.5,1.5) {$\mathbb{R}^{2}\setminus\{(0,0)\}$};
        \draw[>=Latex,draw=blue,->] (2.4,0) -- (3.6,0);
        \draw[draw=black,semithick] (5.5,0) circle (1);
        \node at (6.9,0) [below] {$x$};
        \node at (5.5,1.4) [right] {$y$};
        \node at (6.5,1.2) {$S^{1}$};
    \end{tikzpicture}
\end{document}
            \caption{Homotopy Equivalence of
                     $\nspace\setminus\{(0,0)\}$ and $\nsphere[1]$}
            \label{fig:homotopy_equivalence_between_the_plane_%
                   with_a_point_removed_and_the_unit_circle}
        \end{figure}
        \begin{ldefinition}{Disconnected Sets}{Disconnected_Sets}
            A disconnected subset of a topological space $X$ is a set
            $S\subseteq{X}$ such that there exist disjoint non-empty open
            sets $X_{1},X_{2}$ such that $S=X_{1}\cup{X_{2}}$.
        \end{ldefinition}
        \begin{ldefinition}{Connected Sets}{Connected_Sets}
            A connected subset of a topological space
            is a subset that is not disconnected.
        \end{ldefinition}
        \begin{ltheorem}{Homeomorphisms Preserve Connectedness}
                        {Homeomorphisms_Preserve_Connectedness}
            If $X$ and $Y$ are homeomorphic and
            $X$ is connected, then $Y$ is connected.
        \end{ltheorem}
        \begin{proof}
            Suppose not. If $Y$ is disconnected, then there are disjoint
            non-empty open sets $Y_{1},Y_{2}$ such that
            $Y=Y_{1}\cup{Y_{2}}$. But as $X$ and $Y$ are homeomorphic,
            there is a continuous bijection $f:X\rightarrow{Y}$. But then
            $f^{-1}(Y_{1})$ and $f^{-1}(Y_{2})$ are non-empty, as $f$ is a
            bijection, and moreoever they are disjoint open subsets of $X$,
            as $f$ is continuous. But then $X$ is disconnected,
            a contradiction. Thus, $Y$ is connected.
        \end{proof}
        \begin{theorem}
            There exists compact connected topological spaces of the same
            cardinality that are homotopy equivalent but not homeomorphic.
        \end{theorem}
        \begin{proof}
            Let $X=[-1,1]$, $Y=[-1,1]^{2}$ and define
            $f:X\rightarrow{Y}$ and $g:Y\rightarrow{X}$ by:
            \twocolumneq{f(x)=(x,0)}{g(x,y)=x}
            Then $g\circ{f}=id_{X}$, and thus $g\circ{f}\simeq{id_{X}}$. But
            also $H(x,y,t)=(1-t)(x,0)+(x,y)$ is a homotopy between
            $f\circ{g}$ and $id_{Y}$, and thus $f\circ{g}\simeq{id_{Y}}$.
            Therefore $X$ and $Y$ are homotopy equivalent. Moreover, they
            are both compact, connected, and have the cardinality
            of the continuum. Suppose $h$ is a homeomorphism
            $h:X\rightarrow{Y}$ and let $h(0)=\mathbf{x}\in{Y}$. If $h$ is
            a homeomorphism between $X$ and $y$, then the restriction of $h$
            to $X\setminus\{0\}$ is a homeomophism between $[-1,0)\cup(0,1]$
            and $[-1,1]^{2}\setminus\{\mathbf{x}\}$. But
            $[-1,1]^{2}\setminus\{\mathbf{x}\}$ is connected, and
            $[-1,0)\cup(0,1]$ is not. But homeomorphisms preserve
            connectedness. Therefore, $X$ and $Y$ are not homeomorphic.
        \end{proof}
        We have seen that cardinality, connectedness, compactness, and
        homotopy equivalence are not enough to guarantee that two spaces are
        homeomorphic. However, thus far every example has included two
        spaces that are of different \textit{dimension}. Adding dimension to
        the list still does not guarantee that the two spaces will be
        homeomorphic. First, we show that $S^{2}\setminus\{(0,0,1)\}$ is
        homeomorphic to $D^{2}$.
        \begin{theorem}
            \label{thm:sphere_without_point_homeomorphic_to_plane}%
            $S^{2}\setminus\{(0,0,1)\}$ is homeomorphic to $\mathbb{R}^{2}$
        \end{theorem}
        \begin{proof}
            For let $f:S^{2}\setminus\{(0,0,1)\}\rightarrow \mathbb{R}^{2}$
            be the stereographic projection mapping:
            \begin{equation}
                f(x,y,z)=\Big(\frac{x}{1-z},\frac{y}{1-z}\Big)
            \end{equation}
            If $(X,Y)\in\mathbb{R}^{2}$, let:
            \begin{align*}
                x&=\frac{2X}{\norm{(X,Y)}^{2}+1}&
                y&=\frac{2Y}{\norm{(X,Y)}^{2}+1}&
                z&=\frac{\norm{(X,Y)}^{2}-1}{\norm{(X,Y)}^{2}+1} 
            \end{align*}
            Then:
            \begin{subequations}
                \begin{align}
                    \Big(\frac{x}{1-z},\frac{y}{1-z}\Big)
                    &=\frac{\norm{(X,Y)}^{2}+1}{2}
                    \Big(\frac{2X}{\norm{(X,Y)}^{2}+ 1},
                         \frac{2Y}{\norm{(X,Y)}^{2}+1}\Big)\\
                    &=(X,Y)
                \end{align}
            \end{subequations}
            and
            \begin{subequations}
                \begin{align}
                    \norm{(x,y,z)}&=\sqrt{%
                        \frac{4\norm{(X,Y)}^{2}+\norm{(X,Y)}^{4}-
                              2\norm{(X,Y)}^{2}+1}{(\norm{(X,Y)}+1)^{2}}%
                    }\\
                    &=\sqrt{\frac{\norm{(X,Y)}^{4}+2\norm{(X,Y)}^{2}+1}
                                 {(\norm{(X,Y)}^{2}+1)^{2}}}\\
                    &=\sqrt{%
                        \frac{(\norm{(X,Y)}^{2}+1)^{2}}
                             {(\norm{(X,Y)}^{2}+1)^{2}}%
                    }
                \end{align}
            \end{subequations}
            Which evaluates to one. Thus,
            $(x,y,z)\in S^{2}\setminus\{(0,0,1)\}$, and $f$ is surjective.
            If $f(x_{1},y_{1},z_{1})=f(x_{2},y_{2},z_{2})$, then
            $z_{1}=z_{2}$. For since
            $(x_{1},y_{1},z_{1})\in S^{2}\setminus\{(0,0,1)\}$,
            and therefore $x_{1}^{2}+y_{1}^{2}=1-z_{1}^{2}$, we have:
            \begin{equation}
                \norm{(X,Y)}^{2}
                =\frac{x_{1}^2+y_{1}^2}{(1-z_{1})^{2}}
                =\frac{1-z_{1}^{2}}{(1-z_{1})^{2}}
                =\frac{x_{2}^{2}+y_{2}^{2}}{(1-z_{2})^{2}}
                =\frac{1-z_{2}^{2}}{(1-z_{2})^{2}}
            \end{equation}
            So we have:
            \begin{equation}
                \frac{1-z_{1}^{2}}{(1-z_{1})^{2}}
                =\frac{1-z_{2}^{2}}{(1-z_{2})^{2}}
                \Rightarrow
                \frac{1+z_{1}}{1-z_{1}}
                =\frac{1+z_{2}}{1-z_{2}}
            \end{equation}
            But the function $g(x)=\frac{1+x}{1-x}$ is an injective function
            and therefore $z_{1}=z_{2}$. From this $x_{1}=x_{2}$ and
            $y_{1}=y_{2}$. Thus, $f$ is a bijection. Moreoever,
            $f$ is continuous and:
            \begin{equation}
                f^{-1}(X,Y)
                =\Big(\frac{2X}{\norm{(X,Y)}^{2}+1},
                    \frac{2Y}{\norm{(X,Y)}^{2}+1},
                    \frac{\norm{(X,Y)}^{2}-1}{\norm{(X,Y)}^{2}+1}\Big)
            \end{equation}
            which is continuous. $f$ is a homeomorphism.
        \end{proof}
        Fig.~\ref{fig:stereographic_projection} depicts the stereographic
        projection used to prove
        Thm.~\ref{thm:sphere_without_point_homeomorphic_to_plane}.
        It can be seen that $(0,0,1)$ projects `to infinity'. Because of
        this, it is not uncommon to call this point infinity. Next, we
        prove that $\mathbb{R}^{2}$ is homeomorphic to $D^{2}$, almost
        completing our claim that $S^{2}\setminus\{(0,0,1)\}$
        is homeomorphic to $D^{2}$.
        \begin{figure}[H]
            \captionsetup{type=figure}
            \centering
            \documentclass[crop,class=article]{standalone}
%----------------------------Preamble-------------------------------%
%---------------------------Packages----------------------------%
\usepackage{geometry}
\geometry{b5paper, margin=1.0in}
\usepackage[T1]{fontenc}
\usepackage{graphicx, float}            % Graphics/Images.
\usepackage{natbib}                     % For bibliographies.
\bibliographystyle{agsm}                % Bibliography style.
\usepackage[french, english]{babel}     % Language typesetting.
\usepackage[dvipsnames]{xcolor}         % Color names.
\usepackage{listings}                   % Verbatim-Like Tools.
\usepackage{mathtools, esint, mathrsfs} % amsmath and integrals.
\usepackage{amsthm, amsfonts, amssymb}  % Fonts and theorems.
\usepackage{tcolorbox}                  % Frames around theorems.
\usepackage{upgreek}                    % Non-Italic Greek.
\usepackage{fmtcount, etoolbox}         % For the \book{} command.
\usepackage[newparttoc]{titlesec}       % Formatting chapter, etc.
\usepackage{titletoc}                   % Allows \book in toc.
\usepackage[nottoc]{tocbibind}          % Bibliography in toc.
\usepackage[titles]{tocloft}            % ToC formatting.
\usepackage{pgfplots, tikz}             % Drawing/graphing tools.
\usepackage{imakeidx}                   % Used for index.
\usetikzlibrary{
    calc,                   % Calculating right angles and more.
    angles,                 % Drawing angles within triangles.
    arrows.meta,            % Latex and Stealth arrows.
    quotes,                 % Adding labels to angles.
    positioning,            % Relative positioning of nodes.
    decorations.markings,   % Adding arrows in the middle of a line.
    patterns,
    arrows
}                                       % Libraries for tikz.
\pgfplotsset{compat=1.9}                % Version of pgfplots.
\usepackage[font=scriptsize,
            labelformat=simple,
            labelsep=colon]{subcaption} % Subfigure captions.
\usepackage[font={scriptsize},
            hypcap=true,
            labelsep=colon]{caption}    % Figure captions.
\usepackage[pdftex,
            pdfauthor={Ryan Maguire},
            pdftitle={Mathematics and Physics},
            pdfsubject={Mathematics, Physics, Science},
            pdfkeywords={Mathematics, Physics, Computer Science, Biology},
            pdfproducer={LaTeX},
            pdfcreator={pdflatex}]{hyperref}
\hypersetup{
    colorlinks=true,
    linkcolor=blue,
    filecolor=magenta,
    urlcolor=Cerulean,
    citecolor=SkyBlue
}                           % Colors for hyperref.
\usepackage[toc,acronym,nogroupskip,nopostdot]{glossaries}
\usepackage{glossary-mcols}
%------------------------Theorem Styles-------------------------%
\theoremstyle{plain}
\newtheorem{theorem}{Theorem}[section]

% Define theorem style for default spacing and normal font.
\newtheoremstyle{normal}
    {\topsep}               % Amount of space above the theorem.
    {\topsep}               % Amount of space below the theorem.
    {}                      % Font used for body of theorem.
    {}                      % Measure of space to indent.
    {\bfseries}             % Font of the header of the theorem.
    {}                      % Punctuation between head and body.
    {.5em}                  % Space after theorem head.
    {}

% Italic header environment.
\newtheoremstyle{thmit}{\topsep}{\topsep}{}{}{\itshape}{}{0.5em}{}

% Define environments with italic headers.
\theoremstyle{thmit}
\newtheorem*{solution}{Solution}

% Define default environments.
\theoremstyle{normal}
\newtheorem{example}{Example}[section]
\newtheorem{definition}{Definition}[section]
\newtheorem{problem}{Problem}[section]

% Define framed environment.
\tcbuselibrary{most}
\newtcbtheorem[use counter*=theorem]{ftheorem}{Theorem}{%
    before=\par\vspace{2ex},
    boxsep=0.5\topsep,
    after=\par\vspace{2ex},
    colback=green!5,
    colframe=green!35!black,
    fonttitle=\bfseries\upshape%
}{thm}

\newtcbtheorem[auto counter, number within=section]{faxiom}{Axiom}{%
    before=\par\vspace{2ex},
    boxsep=0.5\topsep,
    after=\par\vspace{2ex},
    colback=Apricot!5,
    colframe=Apricot!35!black,
    fonttitle=\bfseries\upshape%
}{ax}

\newtcbtheorem[use counter*=definition]{fdefinition}{Definition}{%
    before=\par\vspace{2ex},
    boxsep=0.5\topsep,
    after=\par\vspace{2ex},
    colback=blue!5!white,
    colframe=blue!75!black,
    fonttitle=\bfseries\upshape%
}{def}

\newtcbtheorem[use counter*=example]{fexample}{Example}{%
    before=\par\vspace{2ex},
    boxsep=0.5\topsep,
    after=\par\vspace{2ex},
    colback=red!5!white,
    colframe=red!75!black,
    fonttitle=\bfseries\upshape%
}{ex}

\newtcbtheorem[auto counter, number within=section]{fnotation}{Notation}{%
    before=\par\vspace{2ex},
    boxsep=0.5\topsep,
    after=\par\vspace{2ex},
    colback=SeaGreen!5!white,
    colframe=SeaGreen!75!black,
    fonttitle=\bfseries\upshape%
}{not}

\newtcbtheorem[use counter*=remark]{fremark}{Remark}{%
    fonttitle=\bfseries\upshape,
    colback=Goldenrod!5!white,
    colframe=Goldenrod!75!black}{ex}

\newenvironment{bproof}{\textit{Proof.}}{\hfill$\square$}
\tcolorboxenvironment{bproof}{%
    blanker,
    breakable,
    left=3mm,
    before skip=5pt,
    after skip=10pt,
    borderline west={0.6mm}{0pt}{green!80!black}
}

\AtEndEnvironment{lexample}{$\hfill\textcolor{red}{\blacksquare}$}
\newtcbtheorem[use counter*=example]{lexample}{Example}{%
    empty,
    title={Example~\theexample},
    boxed title style={%
        empty,
        size=minimal,
        toprule=2pt,
        top=0.5\topsep,
    },
    coltitle=red,
    fonttitle=\bfseries,
    parbox=false,
    boxsep=0pt,
    before=\par\vspace{2ex},
    left=0pt,
    right=0pt,
    top=3ex,
    bottom=1ex,
    before=\par\vspace{2ex},
    after=\par\vspace{2ex},
    breakable,
    pad at break*=0mm,
    vfill before first,
    overlay unbroken={%
        \draw[red, line width=2pt]
            ([yshift=-1.2ex]title.south-|frame.west) to
            ([yshift=-1.2ex]title.south-|frame.east);
        },
    overlay first={%
        \draw[red, line width=2pt]
            ([yshift=-1.2ex]title.south-|frame.west) to
            ([yshift=-1.2ex]title.south-|frame.east);
    },
}{ex}

\AtEndEnvironment{ldefinition}{$\hfill\textcolor{Blue}{\blacksquare}$}
\newtcbtheorem[use counter*=definition]{ldefinition}{Definition}{%
    empty,
    title={Definition~\thedefinition:~{#1}},
    boxed title style={%
        empty,
        size=minimal,
        toprule=2pt,
        top=0.5\topsep,
    },
    coltitle=Blue,
    fonttitle=\bfseries,
    parbox=false,
    boxsep=0pt,
    before=\par\vspace{2ex},
    left=0pt,
    right=0pt,
    top=3ex,
    bottom=0pt,
    before=\par\vspace{2ex},
    after=\par\vspace{1ex},
    breakable,
    pad at break*=0mm,
    vfill before first,
    overlay unbroken={%
        \draw[Blue, line width=2pt]
            ([yshift=-1.2ex]title.south-|frame.west) to
            ([yshift=-1.2ex]title.south-|frame.east);
        },
    overlay first={%
        \draw[Blue, line width=2pt]
            ([yshift=-1.2ex]title.south-|frame.west) to
            ([yshift=-1.2ex]title.south-|frame.east);
    },
}{def}

\AtEndEnvironment{ltheorem}{$\hfill\textcolor{Green}{\blacksquare}$}
\newtcbtheorem[use counter*=theorem]{ltheorem}{Theorem}{%
    empty,
    title={Theorem~\thetheorem:~{#1}},
    boxed title style={%
        empty,
        size=minimal,
        toprule=2pt,
        top=0.5\topsep,
    },
    coltitle=Green,
    fonttitle=\bfseries,
    parbox=false,
    boxsep=0pt,
    before=\par\vspace{2ex},
    left=0pt,
    right=0pt,
    top=3ex,
    bottom=-1.5ex,
    breakable,
    pad at break*=0mm,
    vfill before first,
    overlay unbroken={%
        \draw[Green, line width=2pt]
            ([yshift=-1.2ex]title.south-|frame.west) to
            ([yshift=-1.2ex]title.south-|frame.east);},
    overlay first={%
        \draw[Green, line width=2pt]
            ([yshift=-1.2ex]title.south-|frame.west) to
            ([yshift=-1.2ex]title.south-|frame.east);
    }
}{thm}

%--------------------Declared Math Operators--------------------%
\DeclareMathOperator{\adjoint}{adj}         % Adjoint.
\DeclareMathOperator{\Card}{Card}           % Cardinality.
\DeclareMathOperator{\curl}{curl}           % Curl.
\DeclareMathOperator{\diam}{diam}           % Diameter.
\DeclareMathOperator{\dist}{dist}           % Distance.
\DeclareMathOperator{\Div}{div}             % Divergence.
\DeclareMathOperator{\Erf}{Erf}             % Error Function.
\DeclareMathOperator{\Erfc}{Erfc}           % Complementary Error Function.
\DeclareMathOperator{\Ext}{Ext}             % Exterior.
\DeclareMathOperator{\GCD}{GCD}             % Greatest common denominator.
\DeclareMathOperator{\grad}{grad}           % Gradient
\DeclareMathOperator{\Ima}{Im}              % Image.
\DeclareMathOperator{\Int}{Int}             % Interior.
\DeclareMathOperator{\LC}{LC}               % Leading coefficient.
\DeclareMathOperator{\LCM}{LCM}             % Least common multiple.
\DeclareMathOperator{\LM}{LM}               % Leading monomial.
\DeclareMathOperator{\LT}{LT}               % Leading term.
\DeclareMathOperator{\Mod}{mod}             % Modulus.
\DeclareMathOperator{\Mon}{Mon}             % Monomial.
\DeclareMathOperator{\multideg}{mutlideg}   % Multi-Degree (Graphs).
\DeclareMathOperator{\nul}{nul}             % Null space of operator.
\DeclareMathOperator{\Ord}{Ord}             % Ordinal of ordered set.
\DeclareMathOperator{\Prin}{Prin}           % Principal value.
\DeclareMathOperator{\proj}{proj}           % Projection.
\DeclareMathOperator{\Refl}{Refl}           % Reflection operator.
\DeclareMathOperator{\rk}{rk}               % Rank of operator.
\DeclareMathOperator{\sgn}{sgn}             % Sign of a number.
\DeclareMathOperator{\sinc}{sinc}           % Sinc function.
\DeclareMathOperator{\Span}{Span}           % Span of a set.
\DeclareMathOperator{\Spec}{Spec}           % Spectrum.
\DeclareMathOperator{\supp}{supp}           % Support
\DeclareMathOperator{\Tr}{Tr}               % Trace of matrix.
%--------------------Declared Math Symbols--------------------%
\DeclareMathSymbol{\minus}{\mathbin}{AMSa}{"39} % Unary minus sign.
%------------------------New Commands---------------------------%
\DeclarePairedDelimiter\norm{\lVert}{\rVert}
\DeclarePairedDelimiter\ceil{\lceil}{\rceil}
\DeclarePairedDelimiter\floor{\lfloor}{\rfloor}
\newcommand*\diff{\mathop{}\!\mathrm{d}}
\newcommand*\Diff[1]{\mathop{}\!\mathrm{d^#1}}
\renewcommand*{\glstextformat}[1]{\textcolor{RoyalBlue}{#1}}
\renewcommand{\glsnamefont}[1]{\textbf{#1}}
\renewcommand\labelitemii{$\circ$}
\renewcommand\thesubfigure{%
    \arabic{chapter}.\arabic{figure}.\arabic{subfigure}}
\addto\captionsenglish{\renewcommand{\figurename}{Fig.}}
\numberwithin{equation}{section}

\renewcommand{\vector}[1]{\boldsymbol{\mathrm{#1}}}

\newcommand{\uvector}[1]{\boldsymbol{\hat{\mathrm{#1}}}}
\newcommand{\topspace}[2][]{(#2,\tau_{#1})}
\newcommand{\measurespace}[2][]{(#2,\varSigma_{#1},\mu_{#1})}
\newcommand{\measurablespace}[2][]{(#2,\varSigma_{#1})}
\newcommand{\manifold}[2][]{(#2,\tau_{#1},\mathcal{A}_{#1})}
\newcommand{\tanspace}[2]{T_{#1}{#2}}
\newcommand{\cotanspace}[2]{T_{#1}^{*}{#2}}
\newcommand{\Ckspace}[3][\mathbb{R}]{C^{#2}(#3,#1)}
\newcommand{\funcspace}[2][\mathbb{R}]{\mathcal{F}(#2,#1)}
\newcommand{\smoothvecf}[1]{\mathfrak{X}(#1)}
\newcommand{\smoothonef}[1]{\mathfrak{X}^{*}(#1)}
\newcommand{\bracket}[2]{[#1,#2]}

%------------------------Book Command---------------------------%
\makeatletter
\renewcommand\@pnumwidth{1cm}
\newcounter{book}
\renewcommand\thebook{\@Roman\c@book}
\newcommand\book{%
    \if@openright
        \cleardoublepage
    \else
        \clearpage
    \fi
    \thispagestyle{plain}%
    \if@twocolumn
        \onecolumn
        \@tempswatrue
    \else
        \@tempswafalse
    \fi
    \null\vfil
    \secdef\@book\@sbook
}
\def\@book[#1]#2{%
    \refstepcounter{book}
    \addcontentsline{toc}{book}{\bookname\ \thebook:\hspace{1em}#1}
    \markboth{}{}
    {\centering
     \interlinepenalty\@M
     \normalfont
     \huge\bfseries\bookname\nobreakspace\thebook
     \par
     \vskip 20\p@
     \Huge\bfseries#2\par}%
    \@endbook}
\def\@sbook#1{%
    {\centering
     \interlinepenalty \@M
     \normalfont
     \Huge\bfseries#1\par}%
    \@endbook}
\def\@endbook{
    \vfil\newpage
        \if@twoside
            \if@openright
                \null
                \thispagestyle{empty}%
                \newpage
            \fi
        \fi
        \if@tempswa
            \twocolumn
        \fi
}
\newcommand*\l@book[2]{%
    \ifnum\c@tocdepth >-3\relax
        \addpenalty{-\@highpenalty}%
        \addvspace{2.25em\@plus\p@}%
        \setlength\@tempdima{3em}%
        \begingroup
            \parindent\z@\rightskip\@pnumwidth
            \parfillskip -\@pnumwidth
            {
                \leavevmode
                \Large\bfseries#1\hfill\hb@xt@\@pnumwidth{\hss#2}
            }
            \par
            \nobreak
            \global\@nobreaktrue
            \everypar{\global\@nobreakfalse\everypar{}}%
        \endgroup
    \fi}
\newcommand\bookname{Book}
\renewcommand{\thebook}{\texorpdfstring{\Numberstring{book}}{book}}
\providecommand*{\toclevel@book}{-2}
\makeatother
\titleformat{\part}[display]
    {\Large\bfseries}
    {\partname\nobreakspace\thepart}
    {0mm}
    {\Huge\bfseries}
\titlecontents{part}[0pt]
    {\large\bfseries}
    {\partname\ \thecontentslabel: \quad}
    {}
    {\hfill\contentspage}
\titlecontents{chapter}[0pt]
    {\bfseries}
    {\chaptername\ \thecontentslabel:\quad}
    {}
    {\hfill\contentspage}
\newglossarystyle{longpara}{%
    \setglossarystyle{long}%
    \renewenvironment{theglossary}{%
        \begin{longtable}[l]{{p{0.25\hsize}p{0.65\hsize}}}
    }{\end{longtable}}%
    \renewcommand{\glossentry}[2]{%
        \glstarget{##1}{\glossentryname{##1}}%
        &\glossentrydesc{##1}{~##2.}
        \tabularnewline%
        \tabularnewline
    }%
}
\newglossary[not-glg]{notation}{not-gls}{not-glo}{Notation}
\newcommand*{\newnotation}[4][]{%
    \newglossaryentry{#2}{type=notation, name={\textbf{#3}, },
                          text={#4}, description={#4},#1}%
}
%--------------------------LENGTHS------------------------------%
% Spacings for the Table of Contents.
\addtolength{\cftsecnumwidth}{1ex}
\addtolength{\cftsubsecindent}{1ex}
\addtolength{\cftsubsecnumwidth}{1ex}
\addtolength{\cftfignumwidth}{1ex}
\addtolength{\cfttabnumwidth}{1ex}

% Indent and paragraph spacing.
\setlength{\parindent}{0em}
\setlength{\parskip}{0em}
%--------------------------Main Document----------------------------%
\begin{document}
    \begin{tikzpicture}[scale=5, >=triangle 45]
        % Draw the main coordinate system axes
        \draw[thick,->] (0,0,0)--(1.7,0,0) node[right] {$y$};
        \draw[thick,->] (0,0,0)--(0,1.3,0) node[above]{$z$};
        \draw[thick,->] (0,0,0)--(0,0,1.7) node[below left]{$x$};

        % The point (x,y,z)
        \coordinate (A) at (0.577350269,0.577350269,0.577350269);

        % The projection point (X,Y)
        \node[circle, inner sep=0pt, outer sep=1mm]
            (B) at (1.3660254,0,1.3660254) {};

        % Draw blue line "inside" sphere.
        \draw[draw=blue,semithick] (0,1,0) -- (A);

        % Draw sphere.
        \shade[ball color=gray, opacity=0.6]
            (1cm,0) arc (0:-180:1cm and 5mm)
            arc (180:0:1cm and 1cm);
        
        % Draw blue line "outside" of sphere.
        \draw[draw=blue, semithick, >=stealth, ->] (A)--(B);

        % Draw projection lines.
        \begin{scope}[every path/.style={dashed,thin}]
            \path (0,0,0) edge (1.3660254,0,1.3660254);
            \path (0,0,1.3660254) edge (1.3660254,0,1.3660254);
            \path (1.3660254,0,0) edge (1.3660254,0,1.3660254);
            \path (0,0,0.5774) edge (0.5774,0,0.5774);
            \path (0.5774,0,0) edge (0.5774,0,0.5774);
            \path (0.5774,0,0.5774) edge (0.5774,0.5774,0.5774);
            \path (0,0,0.5774) edge (0.5774,0,0.5774);
        \end{scope}

        % Draw the position vector.
        \draw[dashed,semithick] (0,0,0) -- (0.5774,0.5774,0.5774);

        \filldraw[black] (0.57735,0.57735,0.57735) circle (0.15mm);
        \filldraw[red] (1.3660254,0,1.3660254) circle (0.15mm);
        \node at (A) [above right] {$(x,y,z)$};
        \node at (B) [below right] {$(X,Y)$};
    \end{tikzpicture}
\end{document}
            \caption{Stereographic Projection of the Sphere onto the Plane.}
            \label{fig:stereographic_projection}
        \end{figure}
        \begin{theorem}
            $\mathbb{R}^{2}$ is homeomorphic to $D^{2}$.
        \end{theorem}
        \begin{proof}
            Let $f:D^{2}\rightarrow\mathbb{R}^{2}$
            be defined by:
            \begin{equation}
                f(\mathbf{x})=\frac{\mathbf{x}}{1-\norm{\mathbf{x}}}
            \end{equation}
            $f$ is surjective. For $\mathbf{0}\mapsto\mathbf{0}$ and if
            $\mathbf{y}\in\mathbb{R}^2\setminus\{\mathbf{0}\}$, then let
            $\mathbf{x}=\frac{\mathbf{y}}{1+\norm{\mathbf{y}}}$. Then:
            \begin{equation}
                \norm{\mathbf{x}}
                =\frac{\norm{\mathbf{y}}}{1+\norm{\mathbf{y}}}<1
            \end{equation}
            and thus $\mathbf{x}\in D^{2}$. But:
            \begin{equation}
                f(\mathbf{x})
                =\frac{\mathbf{y}}{1+\norm{\mathbf{y}}}
                \Big(
                    1-\frac{\norm{\mathbf{y}}}{1+\norm{\mathbf{y}}}
                \Big)^{-1}
                =\mathbf{y}
            \end{equation}
            Moreover, $f$ is injective.
            For if
            $f(\mathbf{x}_{1})=f(\mathbf{x}_{2})$,
            then:
            \begin{equation}
                \frac{\norm{\mathbf{x}_{1}}}{1+\norm{\mathbf{x}_{1}}}
                =\norm{f(\mathbf{x}_{1})}
                =\norm{f(\mathbf{x}_{2})}
                =\frac{\norm{\mathbf{x}_{2}}}{1+\norm{\mathbf{x}}_{2}}
            \end{equation}
            and therefore
            $\norm{\mathbf{x}}_{1}=\norm{\mathbf{x}_{2}}$.
            But from the definition of $f$:
            \begin{equation}
                \frac{\mathbf{x}_{1}}{1+\norm{\mathbf{x}_{1}}}
                =\frac{\mathbf{x}_{2}}{1+\norm{\mathbf{x}_{2}}}
            \end{equation}
            and therefore $\mathbf{x}_{1}=\mathbf{x}_{2}$.
            $f$ is bijective.
            Moreover, $f$ is continuous. Finally,
            \begin{equation}
                f^{-1}(\mathbf{y})
                =\frac{\mathbf{y}}{1+\norm{\mathbf{y}}}                    
            \end{equation}
            which is continuous. $f$ is a homeomorphism.
        \end{proof}
        \begin{theorem}
            $S^{2}\setminus\{(0,0,1)\}$ is homeomorphic to $D^{2}$.
        \end{theorem}
        \begin{proof}
            For $S^{2}\setminus\{(0,0,1)\}$ is homeomorphic to
            $\mathbb{R}^{2}$, and $\mathbb{R}^{2}$ is homeomorphic to
            $D^{2}$. But homeomorphism is an equivalence relation, and
            thus $S^{2}\setminus\{(0,0,1)\}$ is homeomorphic to $D^{2}$.
        \end{proof}
        \begin{figure}[H]
            \centering
            \captionsetup{type=figure}
            \includegraphics{images/Sphere_to_Disk_Homeo.pdf}
            \caption{Homeomorphism of a Sphere w/o a Point to an Open Disk}
            \label{fig:homeomorphism_S_2_wo_North_Pole_and_R_2_2}
        \end{figure}
        We can use the fact that $S^{2}\setminus \{(0,0,1)\}$ is
        homeomorphic to $D^{2}$ to construct examples of topological
        manifolds of the same dimensions that are homotopy equivalent, but
        not homeomorphic. We may generalize to $S^{2}$ with $n$ points
        removed is homeomorphic to $D^{2}$ with $n-1$ points removed.
        We now define the notion of \textit{manifold} and \textit{dimension}.
        It can be shown that if $X$ and $Y$ are homeomorphic manifolds,
        then they are of the same dimension. This is simply because
        $\mathbb{R}^{n}$ is homeomorphic to $\mathbb{R}^{m}$ if and only if
        $n=m$. Therefore, homeomorphisms preserve dimension. We use the
        fact that a sphere is not homeomorphic to a torus. This can be
        proved by considering the \textit{fundamental group} of the sphere
        and the torus, but can also be shown in a more elementary way if
        we're allowed to assume the Jordan Curve Theorem.
        \begin{ldefinition}{Jordan Curve}{Jordan_Curve}
            A Jordan Curve is a continuous injective function
            $f:S^{1}\rightarrow\mathbb{R}^{2}$.
        \end{ldefinition}
        With this we now state, but do not prove, the Jordan Curve Theorem.
        \begin{ftheorem}{Jordan Curve Theorem}{Jordan_Curve_Theorem}
            If $\Gamma$ is a Jordan Curve, then there exists two unique
            disjoint open connected sets $\interior[](\Gamma)$ and
            $\exterior[](\Gamma)$, called the interior and exterior of $\Gamma$,
            such that:
            \begin{equation*}
                \nspace[2]\setminus\Gamma
                    =\interior[]{\Gamma}\cup\exterior[]{\Gamma}
            \end{equation*}
            And such that $\partial\interior[](\Gamma)=\Gamma$ and
            $\partial\exterior[](\Gamma)=\Gamma$. $\interior[](\Gamma)$ is
            bounded and $\exterior[](\Gamma)$ is unbounded.
        \end{ftheorem}
        The Jordan Curve Theorem implies that the injective continuous
        image of $S^{1}$ into the sphere $S^{2}$ will cut the
        sphere into two parts. That is, given a circle and
        a continuous one-to-one mapping into the sphere, the
        result bisects the sphere into two parts.
        \begin{theorem}
            \label{thm:Sphere_Without_Circle_Is_Disconnected}
            If $f:S^{1}\rightarrow{S}^{2}$ is a continuous injective
            function, then there exists two disjoint open connected
            sets $\mathcal{U}_{1}$ and $\mathcal{U}_{2}$ such that:
            \begin{equation}
                S^{2}\setminus{f}(S^{1})
                =\mathcal{U}_{1}\cup\mathcal{U}_{2}
            \end{equation}
        \end{theorem}
        \begin{proof}
            First note that $f$ is not surjective. For if it were,
            remove two points from $S^{1}$ and remove the corresponding
            points from $S^{2}$. Then the restriction of $f$ to the
            remaining set is still bijective and continuous. But
            $S^{2}$ without two points is connected, whereas $S^{1}$
            without two points is disconnected. Therefore $f$ is not
            surjective. But then there is a point $\mathbf{y}\in{S}^{2}$
            such that, for all $x\in{S}^{1}$, $f(x)\ne\mathbb{R}$.
            But then $f:S^{1}\rightarrow{S}^{2}\setminus\mathbf{y}$
            is still a continuous injective function. But
            $S^{2}\setminus\mathbf{y}$ is homeomorphic to $\mathbb{R}^{2}$.
            Thus there is a
            $g:S^{2}\setminus\{\mathbf{y}\}\rightarrow\mathbb{R}$
            such that $g$ is a homeomorphism. But then $\Gamma=g\circ{f}$
            is a Jordan Curve. By the Jordan Curve Theorem there are two
            disjoint open connected sets $\interior[](\Gamma)$ and
            $\exterior[](\Gamma)$ such that
            $\interior[](\Gamma)\cup\exterior[](\Gamma)=\mathbb{R}^{2}\setminus\Gamma$.
            Define the following:
            \twocolumneq{%
                \mathcal{U}_{1}=g^{\minus{1}}\big[\interior[]{\Gamma}\big]
            }%
            {%
                \mathcal{U}_{2}=g^{\minus{1}}\big[\exterior[]{\Gamma}\big]
                                \cup\{\vector{y}\}
            }
            Then $\mathcal{U}_{1}$ and $\mathcal{U}_{2}$ are open, disjoint,
            connected, and cover $\nsphere[2]\setminus{f}[\nsphere[1]]$.
        \end{proof}
        We can now prove that the sphere and the torus are not homeomorphic.
        \begin{ltheorem}{$\nsphere[2]$ is not Homeomorphic to $\ntorus[]$}
                        {Sphere_Not_Homeo_to_Torus}
            There is no homeomorphism between the sphere $\nsphere[2]$ and
            the torus $\ntorus[]$.
        \end{ltheorem}
        \begin{proof}
            The Torus is the Cartesian product of two circles,
            $\ntorus[]=\nsphere[1]\times\nsphere[1]$, and by removing the inner
            circle we are left with a connected set. Suppose the two are
            homeomorphic and let $f:T^{1}\rightarrow{S}^{2}$ be a homeomorphism.
            Then the restriction of $f$ to $T^{1}\setminus{S}^{1}$ is a
            homeomorphism with $S^{2}\setminus{f}(S^{1})$. But by
            Thm.~\ref{thm:Sphere_Without_Circle_Is_Disconnected}
            $S^{2}\setminus{f}(S^{1})$ is disconnected, a contradiction.
            Therefore there is no homeomorphism.
        \end{proof}
        Finally, we'll use the following visual representation of a torus:
        \begin{figure}[H]
            \centering
            \captionsetup{type=figure}
            \includegraphics{images/Square_to_Torus.pdf}
            \caption{Turning a Square into a Torus}
            \label{fig:plane_representation_of_a_torus}
        \end{figure}
        It should be intuitively clear that both the torus and the sphere
        are two dimensional topological manifolds. If we remove a finite
        number of points from either of these we are still left with
        two dimensional manifolds.
        \begin{theorem}
            There exist manifolds $X$ and $Y$ such that
            $\dim(X)=\dim(Y)$, ${X}\simeq{Y}$,
            yet $X$ and $Y$ are not homeomorphic.
        \end{theorem}
        \begin{proof}
            For let $X=S^{2}\setminus\{(0,0,1),(0,1,0),(1,0,0)\}$, and let
            $Y=T^{2}\setminus\{(1,0,0)\}$. That is, $X$ is a sphere with
            three points removed, and $Y$ is a torus with one point removed.
            Then $\dim(X)=\dim(Y)=2$. Moreover, $X\simeq Y$. For $X$ is
            homeomorphic to the plane with $2$ points removed. This is
            homotopy equivalent to a figure $8$. Using the square
            representation of a torus in
            Fig.~\ref{fig:plane_representation_of_a_torus} we see that the
            torus with a point removed is also homotopy equivalent to a
            figure $8$. But homotopy equivalence is an equivalence relation,
            and thus $X\simeq Y$. But the a sphere is not homeomorphic to a
            torus, and similarly a sphere with $3$ points removed is not
            homeomorphic to a torus with $1$ point removed.
        \end{proof}
        \begin{figure}[H]
                \centering
                \captionsetup{type=figure}
                \resizebox{\textwidth}{!}{%
                    \begin{tikzpicture}[%
    every edge/.style={draw=black},
    scale=1.7,
    >=latex
]
    \draw[ball color=gray!40, opacity=0.4] (0,0) circle (1cm);
    \draw (-1,0) arc (180:360:1 and 0.3);
    \draw[dashed] (1,0) arc (0:180:1 and 0.3);
    \draw[fill=white] (0.55,0.55) circle (0.75pt);
    \draw[fill=white] (0.65,0.65) circle (0.75pt);
    \draw[fill=white] (0.5,0.72)  circle (0.75pt);
    \draw[->] (1.3,0) to (2.3,0);
    \draw[fill=gray, shading angle=215]
        (2.5,-0.5)--(3.2,0.5)--(5.5,0.5)--(4.8,-0.5)--cycle;
    \draw[densely dashed] (3.7,0) circle (0.295);
    \draw[densely dashed] (4.3,0) circle (0.295);
    \draw[fill=white] (3.7,0) circle (0.75pt);
    \draw[fill=white] (4.3,0) circle (0.75pt);
    \draw[->] (5.7,0) to [in=90,out=0] (7,-1);
    \draw[->] (1.2,-2.3)--(2,-2.3);
    \draw[%
        postaction={decorate},
        decoration={%
            markings,
            mark=at position .145 with \arrow{latex},
            mark=at position .375 with \arrow{latex},
            mark=at position .395 with \arrow{latex},
            mark=at position .615 with \arrowreversed{latex},
            mark=at position .855 with \arrowreversed{latex},
            mark=at position .875 with \arrowreversed{latex}
        },
        fill=gray,
        shading angle=215
    ] (-0.75,-3)--(0.75,-3)--(0.75,-1.5)--(-0.75,-1.5)--cycle;
    \draw[fill=white] (0,-2.3) circle (0.75pt);
    \draw[%
        postaction={decorate},
        decoration={%
            markings,
            mark=at position .5 with \arrow{latex},
            mark=at position .55 with \arrow{latex}
        }
    ]   (3.1,-2.3) arc[%
                start angle=0,
                delta angle=-360,
                x radius=.35,
                y radius=.75
        ];
    \draw (2.75,-1.55) -- (4.75,-1.55);
    \draw[%
        postaction={decorate},
        decoration={%
        markings,
        mark=at position .5 with \arrow{latex},
        mark=at position .55 with \arrow{latex}
        }
    ]   (5.1cm,-2.3cm) arc[%
            start angle=0,
            delta angle=-360,
            x radius=.35,
            y radius=.75
        ];
    \draw[->] (5.4,-2.3)--(6.3,-2.3);
    \draw (7,-1.8) circle (0.5);
    \draw (7,-2.8) circle (0.5);
\end{tikzpicture}%
                }
                \caption{Equivalency of $S^{2}\setminus\{a,b,c\}$,
                         $T^{2}\setminus\{\alpha\}$, and a figure-8.}
                \label{fig:homotopy_equivalence_sphere_%
                       with_3_holes_torus_with_1_hole}
        \end{figure}
        \begin{fdefinition}{Retract}{Retract}
            A retract of a topological space $\topspace{X}$ onto a subspace
            $\topspace[A]{A}$ is a continuous function $f:X\rightarrow{A}$ such
            that $f|_{A}=\identity{A}$.
        \end{fdefinition}
        \begin{example}
            Let $X\subseteq\nspace[2]$ be the annulus for radii
            $r_{1}=\frac{1}{2}$ and $r_{2}=\frac{3}{2}$. We can define a retract
            $f:X\rightarrow\nsphere[1]$ by writing:
            \begin{equation}
                f(\vector{x})=\frac{\vector{x}}{\norm{\vector{x}}}
            \end{equation}
            Then $f$ is continuous and $f|_{\nsphere[1]}=\identity{\nsphere[1]}$
            since all elements of the unit sphere have norm 1. Morever,
            $f[X]=\nsphere[1]$. Hence, $f$ is a retract of $X$ onto
            $\nsphere[1]$ (Def.~\ref{def:Retract}).
        \end{example}
        \begin{figure}[H]
            \centering
            \captionsetup{type=figure}
            \includegraphics{images/Homotopy_Circle.pdf}
            \caption{A Retract of an Annulus onto a Circle}
            \label{fig:Retract_Annulus_to_Circle}
        \end{figure}
        \begin{fdefinition}{Deformation Retraction}{Deformation_Retraction}
            A deformation retraction of a topological space $\topspace{X}$ onto
            a subspace $\topspace[A]{A}$ is a homotopy
            $F:X\times{I}\rightarrow{X}$ between a deformation retract
            $f:X\rightarrow{A}$ and the identity function $\identity{X}$.
        \end{fdefinition}
        Returning to the example of the annulus
        (Fig.~\ref{fig:Retract_Annulus_to_Circle}), we can convert this into a
        homotopy by considering the straight line homotopy between $\vector{x}$
        and $\uvector{x}$. We define:
        \begin{equation}
            H(\vector{x},t)=(1-t)\cdot\vector{x}+
                t\cdot\frac{\vector{x}}{\norm{\vector{x}}}
        \end{equation}
        and this gives us a deformation retraction of the annulus on
        $\nsphere[1]$. Another perhaps more difficult example is the plane with
        two holes in it, say at $(\minus{1},0)$ and $(1,0)$. There is a
        deformation retraction of this space onto a figure eight. For the
        definition of a figure eight, let's start with the lemniscate of
        Gerono, studied by the French mathematician Camille-Christophe Gerono in
        the $19^{th}$ century C.E. The defining implicit equation goes as
        follows:
        \begin{equation}
            x^{4}-x^{2}+y^{2}=0
        \end{equation}
        Setting $x(\theta)=\cos(\theta)$, we have:
        \begin{equation}
            y^{2}=\cos^{2}(\theta)\big(1-\cos^{2}(\theta)\big)
                 =\cos^{2}(\theta)\sin^{2}(\theta)
        \end{equation}
        so we parameterize the figure eight by:
        \begin{equation}
            \big(x(\theta),\,y(\theta)\big)
                =\big(\cos(\theta),\,\cos(\theta)\sin(\theta)\big)
        \end{equation}
        Now to retract the plane with two holes onto this object whilst leaving
        the lemniscate fixed. First, suppose we've retracted all far away points
        down to an oval, and points near the two holes we've pushed out so that
        the holes have been enlarged from points to circle. To finish the
        retraction we can use a bit of physics and differential equations
        (see Fig.~\ref{fig:Deformation_Retraction_lemniscate_of_Gerono}).
        \begin{figure}
            \centering
            \captionsetup{type=figure}
            \includegraphics{images/Homotopy_lemniscate_of_Gerono.pdf}
            \caption{Deformation Retraction onto the lemniscate of Gerono}
            \label{fig:Deformation_Retraction_lemniscate_of_Gerono}
        \end{figure}
        We put two identical electric charges at the centers of the blue
        circles. The \textit{electric field} exerted at a point
        $\vector{x}=(x_{0},x_{1})$ in the plane is given by Coulomb's law:
        \begin{equation}
            F(\vector{x})=\frac{\vector{x}-\vector{r}_{1}}
                               {\norm{\vector{x}-\vector{r}_{1}}_{2}^{3}}+
                \frac{\vector{x}-\vector{r}_{2}}
                     {\norm{\vector{x}-\vector{r}_{2}}_{2}^{3}}
        \end{equation}
        where $\vector{r}_{1}$ and $\vector{r}_{2}$ are the coordinates of the
        holes, we've chosen $(\minus{1},0)$ and $(1,0)$. The notation
        $\norm{\vector{x}}_{2}^{3}$ denotes the 2-norm raised to the third
        power. If this were an actual physics problem we would need some scale
        factor $Q/4\pi\epsilon_{0}$, but for our purposes we simply need the
        directions of the \textit{field lines}. Given a point inside of the
        lemniscate, we drag this point continuously outward along field lines
        until we hit the figure eight, and for points on the outside we
        contract inwards. All the while we leave the lemniscate fixed. The
        outcome is a deformation retraction onto our figure eight. For the sake
        of computation, one could use something like Euler's method of solving
        differential equations to numerical approximate this homotopy. For a
        more analytical approach with closed form solutions, we can consider
        Cassini ovals. If we let $a$ denote the distance from the first hole to
        the second, we can study the family of curves satisfying the following
        equation:
        \begin{equation}
            \label{eqn:Cassini_Ovals}%
            \big((x-a)^{2}+y^{2}\big)\big((x+a)^{2}+y^{2}\big)=b^{4}
        \end{equation}
        this asks for the set of all points $P$ such that the distance from
        $P$ to the first whole multiplied by the distance from $P$ to the second
        hole is equal to $b^{2}$. That is:
        \begin{equation}
            X_{b}=\{\,\vector{x}\in\nspace[2]\;|\;
                \norm{\vector{x}-\vector{r}_{1}}_{2}\cdot
                \norm{\vector{x}-\vector{r}_{2}}_{2}=b^{2}\,\}
        \end{equation}
        This problem was studied by the Italian astronomer Giovanni Domenico
        Cassini in 1680 C.E. and gives us a continuous means of retracting the
        plane with two holes onto a figure eight. The resulting figure eight is
        no longer the lemniscate of Gerono, but rather the lemniscate of
        Bernoulli, studied by Jakob Bernoulli in 1694 C.E. shortly after
        Cassini's investigations. Taking the gradient of
        Eqn.~\ref{eqn:Cassini_Ovals} gives us a vector field that once agains
        allows us to flow along field lines until we arrive at the figure eight
        (see Fig.~\ref{fig:Deformation_Retraction_Cassini_Ovals}). The gradient
        is computed to be:
        \begin{equation}
            \grad{f}=\big(4x(x^{2}+y^{2}-a^{2}), 4y(x^{2}+y^{2}+a^2)\big)
        \end{equation}
        \begin{figure}[H]
            \centering
            \captionsetup{type=figure}
            \includegraphics{images/Homotopy_Cassini_Ovals_001.pdf}
            \caption{Deformation Retraction Using Cassini Ovals}
            \label{fig:Deformation_Retraction_Cassini_Ovals}
        \end{figure}
        The gradient allows us to show what the paths of individual points will
        look like, and allows us to draw
        Fig.~\ref{fig:Deformation_Retraction_Cassini_Ovals}, but is unnecessary.
        Cassini's equation Eqn.~\ref{eqn:Cassini_Ovals} is all we need. As
        $b$ tends to zero we get two disconnected ovals closing in on our two
        points. When $b$ is equal to the square root of the distance between
        these two points we obtain our lemniscate, and as $b$ grows we get a
        single connected object that looks more and more like a circle as $b$
        gets large. This is precisely the description of a deformation
        retraction of the plane with two points missing onto the figure eight
        and the intermediate steps are shown in
        Fig.~\ref{fig:Homotopy_Cassini_Ovals}.
        \begin{figure}[H]
            \centering
            \captionsetup{type=figure}
            \includegraphics{images/Homotopy_Cassini_Ovals_002.pdf}
            \caption{Homotopy Using Cassini Ovals}
            \label{fig:Homotopy_Cassini_Ovals}
        \end{figure}
        Now we can use our imagination to consider other possible deformation
        retractions from the plane with two holes onto smaller subspaces. For
        one, we can take our central lemniscate of Bernoulli that was produced
        in Fig.~\ref{fig:Deformation_Retraction_Cassini_Ovals} using Cassini
        ovals and stretch the crossing point outwards until we achieve two
        circles attached by a straight line
        (see Fig.~\ref{fig:Homotopy_Two_Circles_and_String}). While the retract
        using electric fields and the retract using Cassini ovals resulted in
        two homeomorphic objects (the lemniscate of Gerono and the lemniscate
        of Bernoulli are homeomorphic spaces of $\nspace[2]$), this third object
        is \textit{not} homeomorphic to either of these.
        \begin{figure}[H]
            \centering
            \captionsetup{type=figure}
            \includegraphics{images/Homotopy_Two_Circles_and_String.pdf}
            \caption{Deformation Retraction onto Two Connnected Circles}
            \label{fig:Homotopy_Two_Circles_and_String}
        \end{figure}
        We can see that these are not homeomorphic as follows. Suppose $f$ is a
        homeomorphism from the lemniscate of Bernoulli to two circles with a
        straight line. If we remove the center of the lemniscate we are left
        with two objects that are homeomorphic to the open unit interval
        $(0,1)$. However, no matter where the crossing point maps to under $f$,
        removing a single point from the latter object does not result in two
        homeomorphic copies of the unit interval, even though homeomorphism
        preserve this notion of subspace. So while these are both the result of
        deformation retractions of the same space, they are \textit{not}
        homeomorphic. Further still we can imagine stretching the crossing point
        of our lemniscate vertically rather than horizontally, obtaining
        Fig.~\ref{fig:Homotopy_Oval_with_Line}.
        \begin{figure}[H]
            \centering
            \captionsetup{type=figure}
            \includegraphics{images/Homotopy_Oval_with_Lines.pdf}
            \caption{Homotopy onto a Different Figure Eight}
            \label{fig:Homotopy_Oval_with_Line}
        \end{figure}
        It is perhaps easiest to see that the retraction obtained in
        Fig.~\ref{fig:Homotopy_Oval_with_Line} is different then both
        Fig.~\ref{fig:Deformation_Retraction_lemniscate_of_Gerono} and
        Fig.~\ref{fig:Deformation_Retraction_Cassini_Ovals}. Removing any point
        from this final figure eight does not disconnect the space, however if
        one were to remove the central points from either the lemniscate of
        Bernoulli or the circles connected by a line, we disconnect these spaces
        into two separate parts. Since homeomophisms preserve such notions, this
        final object is not homeomorphic to either of the other two. Hence we
        see that none of these three things are homeomorphic to each other,
        though they are homotopy equivalent.
    \section{A Review of Topology}
        Fig.~\ref{fig:homotopy_equivalence_sphere_%
                       with_3_holes_torus_with_1_hole} shows how both
        $S^{2}$ with three points removed and $T^{2}$ with one point removed
        are homotopy equivalent. Recall that
        $\mathbb{R}^{2}\setminus \{(0,0)\}$ is homotopy equivalent to
        $S^{1}$. In a similar manner, the plane with two points removed is
        homotopy equivalent to two circles whose intersection contains a
        single points (That is, a figure-$8$). While the ``Proof,'' given
        was hand wavy, the fact that the sphere is not homeomorphic to the
        torus comes from the fact that these two objects have different
        boundary components, something preserved by homeomorphism. As we
        saw before, we can remove a circle from the torus, leaving one
        connected surface, but removing a circle from the sphere results in 
        two connected components. ``What about compact manifolds without
        boundary?''
        \begin{ltheorem}{Generalized Poincare Conjecture}
                        {Generalized_Poincare_Conjecture}
            If $X$ is an $n$ dimensional manifold that
            is homotopy equivalent to $S^{n}$, then $X$
            is homeomorphic to $S^{n}$.
        \end{ltheorem}
        \vspace{5pt}
        Recall that the boundary of a topological space
        is what remains when you remove it's interior.
        That is:
        \begin{equation}
            \partial{X}=\overline{X}\setminus\interior[](x)
        \end{equation}
        Where $\overline{X}$ is the closure of $X$.
        A topological space without boundary is one such that
        $\partial{X}=\emptyset$. With this we can now define
        closed and rigid manifolds.
        \begin{ldefinition}{Closed Manifolds}{Closed_Manifolds}
            A closed manifold of dimension $n\in\mathbb{N}$ is
            a compact topological manifold $\mathcal{M}$ of
            dimension $n$ without boundary.
        \end{ldefinition}
        \begin{ldefinition}{Closed Rigid Manifolds}
                           {Closed_Rigid_Manifolds}
            A closed rigid manifold of dimension $n$
            is a closed topological manifold $\mathcal{M}$
            such that, for all closed homotopy equivalent
            manifolds $\mathcal{N}$ of dimension $n$,
            $\mathcal{M}$ is homeomorphic to $\mathcal{N}$.
        \end{ldefinition}
        The question then becomes
        ``Which manifolds are rigid, and which are not?''
        From the Poincare theorem, $S^{n}$ is topologically
        rigid for all $n\in\mathbb{N}$. $S^{n}$ is not differentially
        rigid. That is, we cannot necessarily relax the definition
        of rigidity to include diffeomorphisms.
        The first example of a non-rigid closed
        manifold came in the 1930's from Franz, Reidemeister,
        and de Rham, and is called a Lens Space.
        Let $p$ and $q$ be coprime positive integers.
        Divide $S^{3}$ into $p$ equal parts, and then divide
        this into its northern and southern hemispheres.
        Take a piece of the northern hemisphere and move
        it over $q$ slices, and then glue this to the
        southern hemisphere. Take the piece that is already
        there and move it over $q$ pieces, and then glue
        that to the northern hemisphere. Repeat this
        process until all slices are done. The is called
        the Lens Space $L(p,q)$. $L(1,1)$ is simply the
        3-sphere, $L(2,1)$ is the real projective space
        $\mathbb{RP}^{3}$.
        See Fig.~\ref{fig:surgery_theory_lens_space_drawing}
        to see how this construction occurs.
        \begin{theorem}
            If $p,q_{1},q_{2}\in\mathbb{N}$, then $L(p,q_{1})$ and
            $L(p,q_{2})$ are homotopy equivalent if and only if there
            is an $n\in\mathbb{N}$ such that:
            \begin{equation}
                q_{1}q_{2}=n^{2}
            \end{equation}
        \end{theorem}
        \begin{theorem}
            If $p,q_{1},q_{2}\in\mathbb{N}$, then $L(p,q_{1})$ and
            $L(p,q_{2})$ are homotopy equivalent if and only if
            $q_{1}=q_{2}$.
        \end{theorem}
        \begin{figure}[H]
            \centering
            \captionsetup{type=figure}
            \documentclass[crop,class=article]{standalone}
%----------------------------Preamble-------------------------------%
\usepackage{pgfplots, tikz}             % Drawing/graphing tools.
\usetikzlibrary{arrows.meta}            % Latex and Stealth arrows.
\pgfplotsset{compat=1.9}                % Version of pgfplots.
%--------------------------Main Document----------------------------%
\begin{document}
    \begin{tikzpicture}[scale=1.3]
        \begin{axis}[%
            axis equal,
            width=14cm,
            height=14cm,
            hide axis,
            enlargelimits=0.3,
            view/h=45,
            view/v=30,
            scale uniformly strategy=units only,
            colormap={bluewhite}{%
                color=(blue) color=(white)%
            }
        ]
            \coordinate (X) at (axis cs: 1,0,0);
            \coordinate (-X) at (axis cs: -1,0,0);
            \coordinate (Y) at (axis cs: 0,1,0);
            \coordinate (-Y) at (axis cs: 0,-1,0);
            \coordinate (Z) at (axis cs: 0,0,1);
            \coordinate (-Z) at (axis cs: 0,0,-1);
            \draw[ball color=white] (axis cs: 0,0,0) circle (2.47cm);
            \coordinate (X) at (axis cs: 1,0,0);
            \draw (X) arc (0:45:100) (X) arc (0:-135:100);
            \foreach\a in {0,20,...,340}{%
                \pgfmathsetmacro{\Bound}{-60*cos(\a+45)}
                \addplot3[%
                    domain=\Bound:90,
                    samples=45,
                    samples y=0%
                ]   ({cos(\a)*cos(x)},{sin(\a)*cos(x)},{sin(x)});
                \addplot3[%
                    domain=-90:\Bound,
                    samples=45,
                    samples y=0,
                    densely dashed%
                ]   ({cos(\a)*cos(x)},{sin(\a)*cos(x)},{sin(x)});
            }
            \draw[draw=blue, >={stealth[blue]} ,->,thick]
                (3.2cm,3.5cm,3.5cm) to [in=330, out=0]
                node [below] {$q$} (2.0cm,6.6cm,5cm);
        \end{axis}
    \end{tikzpicture}
\end{document}
            \caption{How to construct $L(p,q)$.}
            \label{fig:surgery_theory_lens_space_drawing}
        \end{figure}
        We move on to the structure set of topological spaces,
        in particular closed topological
        manifolds $\mathcal{M}$.
        \begin{definition}
            Equivalent homotopies are homotopy equivalences
            $f_{1}:X_{1}\rightarrow Y$,
            $f_{2}:X_{2}\rightarrow Y$,
            denoted $f_{1}\sim{f_{2}}$, such that there
            exists a continuous function
            $g:X_{1}\rightarrow{X_{2}}$ and
            $f_{2}\circ{g}\simeq{f_{1}}$.
        \end{definition}
        The equivalent classes of $Y$ is called the
        structure set of $Y$,
        denoted $\mathcal{S}(Y)$. This set contains maps
        like $f_{1}$, $f_{2}$.
        If $g$ is a homeomorphism, then $f_{1}=f_{2}$.
        \begin{lexample}{}{}
            From the generalized Poincare Conjecture:
            \begin{equation}
                \mathcal{S}^{\textrm{Top}}(S^{n})
                =\{S^{n}\}
            \end{equation}
            From before, if $p,q\in\mathbb{N}$ are co-prime, then:
            \begin{equation}
                \Card\big(\mathcal{S}^{\textrm{Top}}(L(p,q)\big)>1
            \end{equation}
            Moreover, from the definition of rigid manifolds,
            ff $\Card\big(\mathcal{S}(X)\big)>1$, then $X$ is non-rigid.
            If $T^{n}$ denotes the $n$ torus, then:
            \begin{equation}
                \Card\big(\mathcal{S}^{\textrm{Top}}(T^{n})\big)
                =2^{n}
            \end{equation}
            However, not all surgery structure sets are finite.
        \end{lexample}
        A few questions naturally arise from the
        definition of the structure set. Is there a natural group
        structure that can be placed on $\mathcal{S}^{\textrm{Top}}(X)$?
        Is it possible for $\mathcal{S}^{\textrm{Top}}(X)$ to be infinite?
        By studying the real projective spaces $\mathbb{RP}^{n}$, we
        arrive at our first example of a space whose surgery structure
        set is infinite.
        \begin{equation}
            n\Mod{4}=3
            \Longrightarrow
            \Card\big(\mathcal{S}^{\textrm{Top}}(\mathbb{RP}^{n})\big)
            =\aleph_{0}
        \end{equation}
        A review of some concepts from algebraic topology.
        \begin{ldefinition}{Paths}{Paths}
            A path in a topological space $X$ is a
            continuous function $f:I\rightarrow{X}$
        \end{ldefinition}
        By further requiring that $f(0)=f(1)$, we call $f$ a loop. This is
        equivalent to the following:
        \begin{ldefinition}{Loops}{Loops}
            A loop in a topological space $X$ a continuous function
            $f:S^{1}\rightarrow{X}$.
        \end{ldefinition}
        \begin{ldefinition}{Fundamental Group}{Fundamental_Group}
            The fundamental group of a topological space
            $X$ with base point $p$ is the set:
            \begin{equation}
                \pi_{1}(X,p)=
                \{f\in{C(I,X)}:p=f(0)=f(1)\}/h
            \end{equation}
            where $h$ is the modulo of homotopy,
            equipped with the concatenation operation:
            \begin{equation}
                (f*g)(t)=
                \begin{cases}
                    f(2t),&0\leq{t}<\frac{1}{2}\\
                    g(2t-1),&\frac{1}{2}\leq{t}<1
                \end{cases}
            \end{equation}
        \end{ldefinition}
        \begin{ltheorem}{Homeomorphisms Preserve the Fundamental Group}
            If $(X,\tau_{X})$ and $(Y,\tau_{Y})$ are topological spaces,
            if $f:X\rightarrow{Y}$ is a homeomorphism, if $p_{X}\in{X}$ and
            if $p_{Y}=f(p_{X})$, then $\pi_{1}(X,p_{x})$ is isomorphic
            to $\pi_{1}(Y,p_{Y})$.
        \end{ltheorem}
        \begin{proof}
            If $X$ and $Y$ are homemorphic, then there is
            a continuous bijective function
            $f:X\rightarrow{Y}$ such that
            $f^{-1}$ is continuous. Define
            $\phi:\pi_{1}(X)\rightarrow\pi_{1}(Y)$ by:
            \begin{equation}
                \phi(x)=f\circ{x}
            \end{equation}
            But $f$ and $x$ are continuous and the composition
            of continuous functions is continuous, and therefore
            $\phi(x)$ is continuous. Moreover:
            \begin{align}
                \phi(x(0))=(f\circ{x})(0)=f(p_{X})=p_{Y}\\
                \phi(x(1))=(f\circ{x})(1)=f(p_{X})=p_{Y}
            \end{align}
            Therefore $\phi(x)\in\pi_{1}(X,p_{Y})$.
            But if $x_{1},x_{2}\in\pi_{1}(X)$, then:
            \begin{equation}
                \phi\big(x_{1}(t)*x_{2}(t)\big)
                =\phi\big(x_{1}(t)\big)*\phi\big(x_{2}(t)\big)
            \end{equation}
            Thus
            $\phi$ is a homomorphism. But as
            $f$ is a bijection, so is $\phi$, and
            therefore $\phi$ is an isomorphism.
            Thus, $\pi_{1}(X,p_{X})$ and
            $\pi_{1}(Y,p_{Y})$ are isomorphic.
        \end{proof}
        \begin{theorem}
            If $X$ and $Y$ are topological spaces, and if $\pi_{1}(X)$ and
            $\pi_{1}(Y)$ are not isomorphic, then
            $X$ and $Y$ are not homeomorphic.
        \end{theorem}
        Using this theorem we can tell whether or not
        certain spaces are homeomorphic. That is,
        the fundamental group is a
        \textit{topological invariant}.
        \begin{ldefinition}{Order of an Element of a Group}
                           {Order_of_an_Element_of_a_Group}
            The order of an element $g$ in a group $G$
            with neutral element $e$ is:
            \begin{equation}
                \Ord_{G}(g)=\inf\{n\in\mathbb{N}:a^{n}=e\}
            \end{equation}
            If there is no such $n\in\mathbb{N}$, then we write
            $\Ord_{G}(g)=\infty$.
        \end{ldefinition}
        \begin{ldefinition}{Group with Torsion}{Group_with_Torsion}
            A group with torsion is a group $(G,*)$ such that
            there exists $g\in\{G\}$ such that $g$ is not an identity
            of $(G,*)$ and $\Ord_{G}(g)<\infty$.
        \end{ldefinition}
        \begin{lexample}{}{}
            If $n\in\mathbb{N}$ and $n>1$, then $S^{n}$ is simply connected.
            Any loop in $S^{n}$ can be continuously transformed down to a
            point, and thus all loops in $S^{n}$ are homotopic. Therefore, for
            all $p\in{S}^{n}$:
            \begin{equation}
                \pi_{1}(S^{n},p)\simeq\{e\}
            \end{equation}
            The fundamental group of spheres is isomorphic to the trivial group.
            For the case of $n=1$, this is not true. Intuitively we can see
            that loops are uniquely determined by how many times they wrap around
            the circle. While the actual computation is difficult, for all
            $p\in{S}^{1}$ we have:
            \begin{equation}
                \pi_{1}(S^{1},p)\simeq\mathbb{Z}
            \end{equation}
            Thus, for all $n\in\mathbb{N}$, and for all $p\in{S}^{n}$,
            $\pi_{1}(X,p)$ is not a group with torsion. There are
            topological spaces, and base points in the spaces, such that
            their fundamental groups have torsion. The first two examples
            are the real projective plane $\mathbb{RP}^{n}$ and the Lens'
            Spaces
            $L(p,q)$. We have:
            \par
            \begin{subequations}
                \begin{minipage}[b]{0.49\textwidth}
                    \centering
                    \begin{equation}
                        \pi_{1}(\mathbb{RP}^{n})
                        =\mathbb{Z}_{2}
                    \end{equation}
                \end{minipage}
                \hfill
                \begin{minipage}[b]{0.49\textwidth}
                    \centering
                    \begin{equation}
                        \pi_{1}(L(p,q))
                        =\mathbb{Z}_{p}
                    \end{equation}
                \end{minipage}
                \par
                If $n>1$, then $\mathbb{Z}_{n}$ is a group with torsion
                since 1 has order $n-1$. Thus, these fundamental groups are
                groups with torsion.
            \end{subequations}
        \end{lexample}
        \begin{theorem}
            If $n\geq 5$, $n\equiv{3}\mod{4}$, and $\pi_{1}(X)$ is a torsion
            group, then:
            \begin{equation}
                \Card\big(S(X^{n})\big) =\infty
            \end{equation}
        \end{theorem}
        Some other gems: $S(\mathbb{C}\mathbb{P}^{n})=\mathbb{Z}_{2}$.
        Chern Manifolds are a thing.
        \subsubsection{The Unsolvable Word Problem}
            \begin{definition}
                A presentation of a group $G$ is a set $H\subset{G}$ of
                generators and a set $R$ of relations on $H$.
                This is denoted $G=\langle{H}|S\rangle$.
            \end{definition}
            \begin{example}
                \
                \begin{enumerate}
                    \item $\langle{a}|a^{n}=e\rangle$ is a the cyclic group
                          of order $n$ generated by $a$.
                    \item $\langle{g},h|hg=gh\rangle=\mathbb{Z}^{2}$
                    \item $\langle{g},h|g^{2}=e,h^{2}=e\rangle%
                           =\mathbb{Z}_{2}*\mathbb{Z}_{n}$
                    \item $\langle{g},h|f^{2}=e,h^{2}=e,%
                           gh=h^{-1}g\rangle=D_{2n}$
                \end{enumerate}
            \end{example}
            The word problem on unsolvability: Given two group
            presentations, there is no algorithm to show that they are
            isomorphic.
            \begin{definition}
                A finitely presented group is a group with a presentation
                $\langle{H}|R\rangle$ such that $H$ and $R$ are finite.
            \end{definition}
            \begin{theorem}
                If $n\geq 5$ and $G$ is finitely presented,
                then there is a closed $n$ dimensional manifold
                $\mathcal{M}$ such that $\pi_{1}(\mathcal{M})=G$.
            \end{theorem}
        \subsubsection{Exact Sequences and Surgery Exact Sequences}
            \begin{definition}
                An exact sequence
                $\cdots G_{3}\overset{f_{3}}{\rightarrow}G_{2}%
                 \overset{f_{2}}{\rightarrow}G_{1}%
                 \overset{f_{1}}{\rightarrow}G_{0}$
                is a sequence $f_{n}$ of homomorphisms and a sequence
                $G_{n}$ of groups such that $\Ima(f_{n+1})=\ker(f_{n})$
            \end{definition}
            Note, the definition requires that the $f_{n}$ are
            \textit{homomorphisms}, not homeomorphisms. Homeomorphism
            is a topological notion, not an algebraic one.
            \begin{example}
                $O\overset{f}{\rightarrow}G\overset{g}{\rightarrow}H$.
                $\Ima(f)=0\Rightarrow\ker(g)=0$. So $g$ is injective.
            \end{example}
            \begin{example}
                $G\overset{f}{\rightarrow}H\overset{g}{\rightarrow}O$,
                $\ker(g)=H\Rightarrow\Ima(f)=H$. So $f$ is surjective.
            \end{example}
            \begin{definition}
                A short exact sequence is an exact sequence
                $0\overset{f}{\rightarrow}G\overset{g}{\rightarrow}%
                 H\overset{h}{\rightarrow}L\overset{\ell}{\rightarrow}0$
            \end{definition}
            We have, from the previous examples, that in a short exact
            sequence $f$ must be injective and $g$ must be surjective. We
            now move onto surgery exact sequences (See Wall et. al).
            Let $n\geq 5$, and $\mathcal{M}$ be a closed manifold of
            dimension $n$. Let $\pi=\pi_{1}(\mathcal{M})$.
            Let Cat have the following meaning:
            \begin{itemize}
                \item Top: Category of continuous maps.
                      That is, the topological catagory.
                \item PL: Piece-Wise linear category.
                      Maps are piece-wise linear.
                \item Diff: Differentiable category.
                      Maps are diffeomorphisms.
            \end{itemize}
            \begin{example}
                \
                \begin{enumerate}
                    \begin{multicols}{2}
                        \item $S^{Top}(S^{n})=\{S^{n}\}$
                        \item $S^{PL}(S^{n})=\{S^{n}\}$
                        \item $|S^{Diff}(S^{2})|=28$ (Milnor)
                        \item $S^{PL}(T^{n})=\{S^{n}\}$ - Rigid
                        \item $|S^{PL}(T^{n})|=2^{n}$ - Non-Rigid.
                        \item $S^{Diff}(T^{n})$ - Difficult.
                    \end{multicols}
                \end{enumerate}
            \end{example}
            A surgery exact sequence is a sequence of the form:
            \begin{align*}
                S^{Cat}(M\times S')\rightarrow[M\times S',G/Cat]
                &\rightarrow L_{n+1}(\pi_{1}(\mathcal{M}))
                \rightarrow{S^{Cat}}(\mathcal{M})
                \rightarrow\cdots\\
                \cdots
                &\rightarrow{[M,G/Cat]}
                \rightarrow{L_{n}}(\pi_{1}(\mathcal{M}))
            \end{align*}
            Here, $L_{n}(X)$ is a \textit{Wall Group},
            and $[A,B]$ is a type of classifiying space.
\end{document}