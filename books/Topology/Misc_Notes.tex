%------------------------------------------------------------------------------%
\documentclass{article}                                                        %
%------------------------------Preamble----------------------------------------%
\makeatletter                                                                  %
    \def\input@path{{../../}}                                                  %
\makeatother                                                                   %
%---------------------------Packages----------------------------%
\usepackage{geometry}
\geometry{b5paper, margin=1.0in}
\usepackage[T1]{fontenc}
\usepackage{graphicx, float}            % Graphics/Images.
\usepackage{natbib}                     % For bibliographies.
\bibliographystyle{agsm}                % Bibliography style.
\usepackage[french, english]{babel}     % Language typesetting.
\usepackage[dvipsnames]{xcolor}         % Color names.
\usepackage{listings, lstlinebgrd}      % Verbatim-Like Tools.
\usepackage{mathtools, esint, mathrsfs} % amsmath and integrals.
\usepackage{amsthm, amsfonts}           % Fonts and theorems.
\usepackage{tabularx}
\usepackage{tcolorbox}                  % Frames around theorems.
\usepackage{upgreek}                    % Non-Italic Greek.
\usepackage{paracol}                    % Two-column styling.
\usepackage{wrapfig}                    % Wrap text around figure.
\usepackage{fmtcount, etoolbox}         % For the \book{} command.
\usepackage[newparttoc]{titlesec}       % Formatting chapter, etc.
\usepackage{titletoc}                   % Allows \book in toc.
\usepackage[nottoc]{tocbibind}          % Bibliography in toc.
\usepackage[titles]{tocloft}            % ToC formatting.
\usepackage{multicol, enumitem}         % Multi-column/enumerate.
\usepackage{import}                     % Import external files.
\usepackage{pgfplots, tikz}             % Drawing/graphing tools.
\usetikzlibrary{
    calc,                   % Calculating right angles and more.
    angles,                 % Drawing angles within triangles.
    arrows.meta,            % Latex and Stealth arrows.
    quotes,                 % Adding labels to angles.
    positioning,            % Relative positioning of nodes.
    decorations.markings,   % Adding arrows in the middle of a line.
    patterns,
    arrows,
    shapes,
    shapes.geometric,
    cd,
    hobby,
    babel
}                                       % Libraries for tikz.
\pgfplotsset{compat=1.9}                % Version of pgfplots.
\usepackage[font=scriptsize,
            labelformat=simple,
            labelsep=colon]{subcaption} % Subfigure captions.
\usepackage[font={scriptsize},
            hypcap=true,
            labelsep=colon]{caption}    % Figure captions.
\usepackage{hyperref}                   % Allows for hyperlinks.
\hypersetup{
    colorlinks=true,
    linkcolor=blue,
    filecolor=magenta,
    urlcolor=Cerulean,
    citecolor=SkyBlue
}                           % Colors for hyperref.
\usepackage[toc,acronym,nogroupskip]{glossaries} % Glossaries and acronyms.
\usepackage[subpreambles=false]{standalone}      % Complileable sub files.

% Various font stuff from kiwi.
% Use this for Times text and Computer Modern math
%\usepackage{times}

% Quite nice
%\usepackage[charter, greekfamily=, greekuppercase=italicized]{mathdesign}
%\usepackage[utopia, greekuppercase=italicized]{mathdesign}    % Math is narrower

% Use this for Times text and math
%\usepackage{newtxtext}
%\usepackage[libertine,cmintegrals]{newtxmath}
%\usepackage{fix-cm}

%\usepackage{txfontsb}
% or
%\usepackage{mathptmx}

%\usepackage[scaled=0.92]{helvet}
%\renewcommand{\rmdefault}{ptm}

%\usepackage{mathpazo}    % add possibly `sc` and `osf` options
%\usepackage{eulervm}

%\usepackage{fourier}
%\renewcommand{\rmdefault}{ptm}
%\usepackage{mathptm}

%\usepackage{fontspec}
%\setmainfont{lmodern}

%\usepackage[varg]{txfonts}
%\usepackage{fouriernc}
%\usepackage{mathpazo}

%\usepackage{bookman}
%\usepackage[scaled]{uarial}
%\usepackage[scaled]{helvet}
%\renewcommand*\familydefault{\sfdefault}
%\usepackage[math]{anttor}

%\newcommand\fgeorgia{\fontfamily{jvn}\selectfont}
%\newcommand\ftimes{\fontfamily{ptm}\selectfont}
%\newcommand\fhelvetica{\fontfamily{phv}\selectfont}
%\newcommand\fcourier{\fontfamily{pcr}\selectfont}
%\newcommand\fbookman{\fontfamily{pbk}\selectfont}
%\newcommand\fnewcentury{\fontfamily{pnc}\selectfont}
%\newcommand\fpalatino{\fontfamily{ppl}\selectfont}
%\newcommand\favantgarde{\fontfamily{pag}\selectfont}
%\newcommand\fnormal{\normalfont}
%\newcommand\fsize[1]{\ifnum#1>0\fontsize{#1}{#1}\selectfont\else\normalsize\fi}
%------------------------Theorem Styles-------------------------%
% Define theorem style for default spacing and normal font.
\newtheoremstyle{normal}
    {\topsep}               % Amount of space above the theorem.
    {\topsep}               % Amount of space below the theorem.
    {}                      % Font used for body of theorem.
    {}                      % Measure of space to indent.
    {\bfseries}             % Font of the header of the theorem.
    {}                      % Punctuation between head and body.
    {.5em}                  % Space after theorem head.
    {}

% Define theorem style for default spacing with italicized font.
\newtheoremstyle{normalit}{\topsep}{\topsep}
                {\itshape}{}{\bfseries}{}{.5em}{}

% Italic header environment.
\newtheoremstyle{thmit}{\topsep}{\topsep}{}{}{\itshape}{}{0.5em}{}

% Define italicized environments.
\theoremstyle{normalit}
\newtheorem{theorem}{Theorem}[section]
\newtheorem{lemma}{Lemma}[section]
\newtheorem{corollary}{Corollary}[section]
\newtheorem{proposition}{Proposition}[section]
\newtheorem*{theorem*}{Theorem}

% Define environments with italic headers.
\theoremstyle{thmit}
\newtheorem*{solution}{Solution}
\newtheorem*{fsolution}{Solution}

% Define default environments.
\theoremstyle{normal}
\newtheorem{example}{Example}[section]
\newtheorem{definition}{Definition}[section]
\newtheorem{problem}{Problem}[section]
\newtheorem{question}{Question}[section]
\newtheorem{remark}{Remark}[section]
\newtheorem{properties}{Properties}[section]
\newtheorem{notation}{Notation}[section]
\newtheorem{axiom}{Axiom}[section]
\newtheorem*{properties*}{Properties}
\newtheorem*{remark*}{Remark}
\newtheorem*{definition*}{Definition}
\theoremstyle{plain}

% Define framed environment.
\tcbuselibrary{most}
\newtcbtheorem[use counter*=theorem]{ftheorem}{Theorem}%
    {colback=green!5,colframe=green!35!black,
     fonttitle=\bfseries\upshape}{th}

\newtcbtheorem[use counter*=example]{fdefinition}{Definition}%
    {fonttitle=\bfseries\upshape,
     colback=blue!5!white,colframe=blue!75!black}{def}

\newtcbtheorem[use counter*=example]{fexample}{Example}%
    {fonttitle=\bfseries\upshape,
     colback=red!5!white,colframe=red!75!black}{ex}

\newtcbtheorem[use counter*=notation]{fnotation}{Notation}%
    {fonttitle=\bfseries\upshape,
     colback=SeaGreen!5!white,colframe=SeaGreen!75!black}{ex}

\newtcbtheorem[use counter*=corollary]{fcorollary}{Corollary}%
    {fonttitle=\bfseries\upshape,
     colback=Orchid!5!white,colframe=Orchid!75!black}{ex}

\newenvironment{bproof}{\textit{Proof.}}{\hfill$\square$}
\tcolorboxenvironment{bproof}{blanker,breakable,left=5mm,
                             before skip=10pt,after skip=10pt,
                             borderline west={1mm}{0pt}{red}}
\tcolorboxenvironment{fsolution}
    {enhanced jigsaw,colframe=cyan,interior hidden,breakable}

%--------------------Declared Math Operators--------------------%
\DeclareMathOperator{\Refl}{Refl}           % Reflection operator.
\DeclareMathOperator{\Span}{Span}           % Span of a set of vectors.
\DeclareMathOperator{\Card}{Card}           % Cardinality of set.
\DeclareMathOperator{\Ord}{Ord}             % Ordinal of ordered set.
\DeclareMathOperator{\Tr}{Tr}               % Trace of matrix.
\DeclareMathOperator{\adjoint}{adj}         % Adjoint of matrix.
\DeclareMathOperator{\rk}{rk}               % Rank of operator.
\DeclareMathOperator{\nul}{nul}             % Null space of operator.
\DeclareMathOperator{\sgn}{sgn}             % Sign of a number.
\DeclareMathOperator{\multideg}{mutlideg}   % Multi-Degree (Graphs).
\DeclareMathOperator{\GCD}{GCD}             % Greatest common denominator.
\DeclareMathOperator{\LM}{LM}               % Leading monomial
\DeclareMathOperator{\LC}{LC}               % Leading coefficient.
\DeclareMathOperator{\LT}{LT}               % Leading term.
\DeclareMathOperator{\LCM}{LCM}             % Least common multiple.
\DeclareMathOperator{\Mon}{Mon}             % Monomial.
\DeclareMathOperator{\Spec}{Spec}           % Spectrum.
\DeclareMathOperator{\proj}{proj}           % Projection.
\DeclareMathOperator{\comp}{comp}           % Component.
\DeclareMathOperator{\sinc}{sinc}           % Sinc function.
\DeclareMathOperator{\Ima}{Im}              % Image of operator.
\DeclareMathOperator{\Prin}{Prin}           % Principal value.
\DeclareMathOperator{\Mod}{mod}             % Modulus.
%------------------------New Commands---------------------------%
\DeclarePairedDelimiter\norm{\lVert}{\rVert}
\DeclarePairedDelimiter\ceil{\lceil}{\rceil}
\DeclarePairedDelimiter\floor{\lfloor}{\rfloor}
\newcommand*\diff{\mathop{}\!\mathrm{d}}
\newcommand*\Diff[1]{\mathop{}\!\mathrm{d^#1}}
\renewcommand{\mod}{\ \Mod}
\renewcommand*{\glstextformat}[1]{\textcolor{RoyalBlue}{#1}}
\renewcommand{\glsnamefont}[1]{\textbf{#1}}
\renewcommand\labelitemii{$\circ$}
\renewcommand\thesubfigure{\arabic{chapter}.\arabic{figure}}
\renewcommand\thesubfigure{%
    \arabic{chapter}.\arabic{figure}.\arabic{subfigure}}
\addto\captionsenglish{\renewcommand{\figurename}{Fig.}}
%------------------------Book Command---------------------------%
\makeatletter
\renewcommand\@pnumwidth{1cm}
\newcounter{book}
\renewcommand\thebook{\@Roman\c@book}
\newcommand\book{%
    \if@openright
        \cleardoublepage
    \else
        \clearpage
    \fi
    \thispagestyle{plain}%
    \if@twocolumn
        \onecolumn
        \@tempswatrue
    \else
        \@tempswafalse
    \fi
    \null\vfil
    \secdef\@book\@sbook
}
\def\@book[#1]#2{%
    \ifnum \c@secnumdepth >-3\relax
        \refstepcounter{book}%
        \addcontentsline{toc}{book}{
            \bookname\ \thebook:\hspace{1em}#1
        }
    \else
        \addcontentsline{toc}{book}{#1}%
    \fi
    \markboth{}{}%
    {\centering
     \interlinepenalty \@M
     \normalfont
     \ifnum \c@secnumdepth >-2\relax
       \huge\bfseries \bookname\nobreakspace\thebook
       \par
       \vskip 20\p@
     \fi
     \Huge \bfseries #2\par}%
    \@endbook}
\def\@sbook#1{%
    {\centering
     \interlinepenalty \@M
     \normalfont
     \Huge \bfseries #1\par}%
    \@endbook}
\def\@endbook{
    \vfil\newpage
        \if@twoside
            \if@openright
                \null
                \thispagestyle{empty}%
                \newpage
            \fi
        \fi
        \if@tempswa
            \twocolumn
        \fi
}
\newcommand*\l@book[2]{%
    \ifnum \c@tocdepth >-2\relax
        \addpenalty{-\@highpenalty}%
        \addvspace{2.25em \@plus\p@}%
        \setlength\@tempdima{3em}%
        \begingroup
            \parindent \z@ \rightskip \@pnumwidth
            \parfillskip -\@pnumwidth
            {
                \leavevmode
                \Large \bfseries #1\hfil \hb@xt@\@pnumwidth{
                    \hss #2
                }
            }
            \par
            \nobreak
            \global\@nobreaktrue
            \everypar{\global\@nobreakfalse\everypar{}}%
        \endgroup
    \fi}
\newcommand\bookname{Book}
\renewcommand{\thebook}{\texorpdfstring{\Numberstring{book}}{book}}
\providecommand*{\toclevel@book}{-2}
\makeatother
\titlecontents{chapter}[0pt]
    {\bfseries}
    {\chaptername\ \thecontentslabel:\quad}
    {}
    {\hfill\contentspage}
\titleformat{\part}[display]
    {\Large\bfseries}
    {\partname\nobreakspace\thepart}
    {0mm}
    {\Huge\bfseries}
    \titlecontents{part}[0pt]
    {\large\bfseries}
    {\partname\ \thecontentslabel: \quad}
    {}
    {\hfill\contentspage}
\newcommand{\MarkRightAngle}[4][.3cm]
    {\coordinate (tempa) at ($(#3)!#1!(#2)$);
     \coordinate (tempb) at ($(#3)!#1!(#4)$);
     \coordinate (tempc) at ($(tempa)!0.5!(tempb)$);%midpoint
     \draw (tempa) -- ($(#3)!2!(tempc)$) -- (tempb);}
%--------------------------LENGTHS------------------------------%
% Spacings for the Table of Contents.
\addtolength{\cftsecnumwidth}{1ex}
\addtolength{\cftsubsecindent}{1ex}
\addtolength{\cftsubsecnumwidth}{1ex}
\addtolength{\cftfignumwidth}{1ex}
\addtolength{\cfttabnumwidth}{1ex}

% Spacing for multi-column and enumerate environments.
\setlength{\multicolsep}{6pt}
\setlist[enumerate]{itemsep=0pt,topsep=3pt}

% Indent and paragraph spacing.
\setlength{\parindent}{0em}
\setlength{\parskip}{0em}                                                           %
%----------------------------Main Document-------------------------------------%
\begin{document}
    \title{Topology Notes}
    \author{Ryan Maguire}
    \date{\vspace{-5ex}}
    \maketitle
    \section{Hocking and Young (Chapter 1)}
        \begin{example}
            Let $\mathbb{R}$ be the real numbers with the usual topology and
            consider the subset $A\subseteq\mathbb{R}$ defined by:
            \begin{equation}
                A=\Big\{\,\frac{1}{n}\;|\;n\in\mathbb{N}\,\Big\}
            \end{equation}
            $A$ has only one limit point and this is zero. Note that
            $0\notin{A}$, and hence the derived set $\derived{A}$ is disjoint
            from $A$ itself. That is:
            \begin{equation}
                \derived[]{A}=\{0\}
            \end{equation}
            And thence $\derived[]{A}\cap{A}=\emptyset$.
        \end{example}
        \begin{example}
            Again consider $\mathbb{R}$ and define $B$ by:
            \begin{equation}
                B=\Big\{\,\frac{n+1}{n}+(\minus{1})^{n}\frac{n-1}{n}\;|\;
                    n\in\mathbb{N}\,\Big\}
            \end{equation}
            This oscillates between values that are asymptotically approaching
            zero and one. The derived set is then:
            \begin{equation}
                \derived[]{B}=\{0,\,1\}
            \end{equation}
            Once again, the derived set is disjoint from the original set.
        \end{example}
        \begin{example}
            In the usual topology, the derived set of $\mathbb{Q}$ is the
            entirety of $\mathbb{R}$. That is:
            \begin{equation}
                \derived[]{\mathbb{Q}}=\mathbb{R}
            \end{equation}
            So it is possible for the derived set to be strictly larger than the
            original set.
        \end{example}
        \begin{example}
            If $\topspace{X}$ is the chaotic topological space on some set $X$
            with the trivial topology $\tau=\{\emptyset,X\}$, then for any
            subset $A\subseteq{X}$, it is true that $\derived{A}=X$. That is,
            in the chaotic topology every point is a limit point of every other
            point.
        \end{example}
        \begin{example}
            If $\topspace{X}$ is the discrete topology, $\tau=\powset{X}$, then
            for any subset $A\subseteq{X}$, the derived set is empty:
            $\derived{A}=\emptyset$. That is, no point is a limit point of any
            subset. This is because every element in the discrete topology is
            isolated since $\{x\}$ is an open subset for every $x\in{X}$.
        \end{example}
        \begin{theorem}
            If $\topspace{X}$ is a topological space, if $A,B\subseteq{X}$, and
            if $A\subseteq{B}$, then $\derived{A}\subseteq\derived{B}$, where
            $\derived{Y}$ denotes the derived set of $Y$ in $\tau$.
        \end{theorem}
        \begin{theorem}
            If $\topspace{X}$ is a topological space, if $C\subseteq{X}$, then
            $C$ is closed if and only if $\closure{C}=C$.
        \end{theorem}
        \begin{theorem}
            If $\topspace{X}$ is a topological space, if
            $\mathcal{C}\subseteq\powset{X}$ is such that for all
            $C\in\mathcal{C}$ it is true that $C$ is closed, then
            $\bigcap\mathcal{C}$ is closed.
        \end{theorem}
        \begin{theorem}
            If $\topspace{X}$ is a topological space, if $C,D\subseteq{X}$ are
            closed in $X$, then $C\cup{D}$ is closed in $X$.
        \end{theorem}
        \begin{definition}
            A basis for a topology $\tau$ on a set $X$ is a subset
            $\mathcal{B}\subseteq\tau$ such that $\bigcup\mathcal{B}=X$ and such
            that for all $B_{1},B_{2}\in\mathcal{B}$ and for all
            $p\in{B}_{1}\cap{B}_{2}$, there exists $B_{3}\in\mathcal{B}$ such
            that $x\in{B}_{3}$ and $B_{2}\subseteq{B}_{1}\cap{B}_{2}$.
        \end{definition}
        \begin{theorem}
            If $\topspace{X}$ is a topological space, then $\tau$ is a basis
            for $\tau$.
        \end{theorem}
        \begin{proof}
            For $X\in\tau$, and hence $X\subseteq\bigcup\tau$, but since
            $\tau\subseteq\powset{X}$, for all $\mathcal{U}\in\tau$ it is true
            that $\mathcal{U}\subseteq{X}$, and hence $X=\bigcup\tau$. If
            $\mathcal{U}_{1},\mathcal{U}_{2}$, and if
            $x\in\mathcal{U}_{1}\cap\mathcal{U}_{2}$, then
            $\mathcal{U}_{1}\cap\mathcal{U}_{2}\in\tau$ is such that
            $x\in\mathcal{U}_{1}\cap\mathcal{U}_{2}$ and
            $\mathcal{U}_{1}\cap\mathcal{U}_{2}\subseteq%
             \mathcal{U}_{1}\cap\mathcal{U}_{2}$, and hence $\tau$ is a basis
            for $\tau$.
        \end{proof}
        \begin{theorem}
            If $\topspace{X}$ is a topological space, if $\mathcal{B}$ is a
            basis for $\tau$, then:
            \begin{equation}
                \tau=\Big\{\,\mathcal{U}\in\powset{X}\;\big|\;
                    \textrm{There exists }\mathcal{O}\subseteq\mathcal{B}
                    \textrm{ such that }\mathcal{U}=\bigcup\mathcal{O}\Big\}
            \end{equation}
        \end{theorem}
        \begin{theorem}
            If $X$ is a set, if $\tau_{1}$ and $\tau_{2}$ are topologies on $X$,
            if $\tau_{1}\subseteq\tau_{2}$, if $\mathcal{B}_{1}$ is a basis for
            $\tau_{1}$, if $\mathcal{B}_{2}$ is a basis for $\tau_{2}$, if
            $x\in{X}$, and if $B_{1}\in\mathcal{B}_{1}$ is such that
            $x\in{B}_{1}$, there exists a $B_{2}\in\mathcal{B}_{2}$ such that
            $x\in{B}_{2}$ and $B_{2}\subseteq{B}_{1}$.
        \end{theorem}
        \begin{example}
            The previous theorem shows when two bases for a topology $\tau$ are
            the same, but different bases may give rise to different topologies.
            Consider on $\mathbb{R}$ the standard topology $\tau_{\mathbb{R}}$
            and the topology $\tau$ generated by the set:
            \begin{equation}
                \mathcal{B}=\{\,(x,\infty)\;|\;x\in\mathbb{R}\}
            \end{equation}
            We can see that $\tau\subseteq\tau_{\mathbb{R}}$ since every set
            in $\mathcal{B}$ can be obtained as the union of intervals as
            follows:
            \begin{equation}
                (x,\infty)=\bigcup_{n\in\mathbb{N}}(x,x+n)
            \end{equation}
            Since $(x,x+n)\in\tau_{\mathbb{R}}$ for all $n\in\mathbb{N}$ and for
            all $x\in\mathbb{R}$, and since topologies are closed under
            arbitrary unions, we thus have that every element of $\mathcal{B}$
            is contained in $\tau_{\mathbb{R}}$ and thus every element of
            $\tau$ is also contained in $\tau_{\mathbb{R}}$. However, these are
            not the same topology.
        \end{example}
        \begin{example}
            In $\mathbb{R}^{2}$, the set of all vertical open line segments
            forms the basis of a topology that is strictly finer than the
            standard Euclidean topology $\tau_{\mathbb{R}^{2}}$, and indeed this
            can be seen as the product topology on $\mathbb{R}$ with the
            standard topology multiplied by $\mathbb{R}$ with the discrete
            topology. The resulting space, being the product of metric spaces,
            is again a metric space. Since the discrete topology $\tau_{D}$ is
            strictly finer than the standard topology on $\mathbb{R}$, we see
            that the product topology $\tau_{\mathbb{R}}\times\tau_{\mathrm{D}}$
            is strictly finer than the standard one $\top_{\mathbb{R}^{2}}$.
        \end{example}
        Def second countable. Intersection of topologies is topology. Topology
        generated by a collection.
        \begin{theorem}
            If $X$ is a set, if $\mathcal{O}\subseteq\powset{X}$, if
            $\tau(\mathcal{O})$ is the topology generated by $\mathcal{O}$,
            and if $\mathcal{B}$ is the set:
            \begin{equation}
                \mathcal{B}=\Big\{\,\mathcal{U}\in\powset{X}\;|\;
                    \exists_{n\in\mathbb{N}}
                    \exists_{B:\mathbb{Z}_{n}\rightarrow\mathcal{O}}
                    \big(\mathcal{U}=\bigcap{B}_{k}\big)\,\Big\}
            \end{equation}
            Then $\mathcal{B}$ is a basis for $\tau(\mathcal{O})$.
        \end{theorem}
        That is, the collection of all finite intersections of elements of
        $\mathcal{O}$ forms a basis for the topology that $\mathcal{O}$
        generates.
        \begin{definition}
            A subbasis for a topological space $\topspace{X}$ is a subset
            $\mathcal{B}\subseteq\powset{X}$ such that the topology generated
            by $\mathcal{B}$ is equal to $\tau$.
        \end{definition}
        \begin{example}
            Consider the set of all open rays on $\mathbb{R}$. That is, the set
            of all subsets of $\mathbb{R}$ of the form:
            \twocolumneq{\mathcal{U}_{+}=(x,\infty)}
                        {\mathcal{U}_{\minus}=(\minus\infty,x)}
            Then this forms a subbasis for the standard topology on
            $\mathbb{R}$. To see this we simply need to show that the standard
            basis for $\mathbb{R}$ can be obtained from finite intersections of
            elements of $\mathcal{B}$, and also note that
            $\mathcal{B}\subseteq\tau_{\mathbb{R}}$. Given an open interval
            $(a,b)\subseteq\mathbb{R}$, let $B_{1}=(a,\infty)$ and
            $B_{2}=(\minus\infty,b)$. Then both $B_{1},B_{2}\in\mathcal{B}$,
            but $(a,b)=B_{1}\cap{B}_{2}$. Thus the topology generated by the
            subbasis $\mathcal{B}$ is the same as the topology generated by the
            standard basis of open intervals, and this is the standard
            topology on $\mathbb{R}$.
        \end{example}
        \begin{ltheorem}{Birkhoff's Topology Lattice Theorem}
            If $X$ is a set, if $T$ is the set of all topologies of $X$, and if
            $\subseteq$ is the inclusion relation, then $(T,\subseteq)$ is a
            complete lattice.
        \end{ltheorem}
        \begin{theorem}
            If $X$ is a set, if $\tau_{1},\tau_{2}$ are topologies on $X$,
            if $\tau_{1}\subseteq\tau_{2}$, and if
            $\identity{X}:\topspace[1]{X}\rightarrow\topspace[2]{X}$ is the
            identity mapping, then it open.
        \end{theorem}
        \begin{theorem}
            If $X$ is a set, if $\tau_{1},\tau_{2}$ are topologies on $X$,
            if $\tau_{2}\subseteq\tau_{1}$, and if
            $\identity{X}:\topspace[1]{X}\rightarrow\topspace[2]{X}$ is the
            identity mapping, then it continuous.
        \end{theorem}
        \begin{theorem}
            If $\topspace[Y]{Y}$ is a topological space, if $X$ is a set, if
            $\tau_{1},\tau_{2}$ are topologies on $X$, if
            $\tau_{1}\subseteq\tau_{2}$, and if $f:X\rightarrow{Y}$ is
            continuous with respect to $\tau_{1}$, then it is continuous with
            respect to $\tau_{2}$.
        \end{theorem}
        \begin{theorem}
            If $\topspace[X]{X}$ is a topological space, if $Y$ is a set, if
            $\tau_{1},\tau_{2}$ are topologies on $Y$, if
            $\tau_{1}\subseteq\tau_{2}$, and if $f:X\rightarrow{Y}$ is
            continuous with respect to $\tau_{2}$, then it is continuous with
            respect to $\tau_{1}$.
        \end{theorem}
        Similar result for open maps. Def metric space, metrics, open balls.
        \begin{example}
            Metrics are not topological entities, and two vastly different
            metrics on the same set may give the same topology. For example,
            given any metric $d$ on a set $X$, the metric $\tilde{d}$ formed by:
            \begin{equation}
                \tilde{d}(x,y)=\frac{d(x,y)}{1+d(x,y)}
            \end{equation}
            is bounded, yet forms the same topology. Hence boundedness is not a
            topological property but a metric one. Thus, there are questions
            about metric spaces that cannot be answered from the study of
            metrizable spaces. For example, if $\metspace{X}$ is a metric space
            one can ask if it has the midpoint property: For all $x,y\in{X}$
            is there a $z\in{X}$ such that $d(x,z)=d(y,z)=d(x,y)/2$? This cannot
            be answer topologically, for consider the closed unit interval
            $[0,1]\subseteq\mathbb{R}$ and the closed upper half unit circle in
            $\mathbb{R}^{2}$. That is:
            \begin{equation}
                S=\{\,(x,y)\in\mathbb{S}_{1}\;|y\geq{0}\,\}
            \end{equation}
            Then $[0,1]$ and $S$ are homeomorphic under the function
            $f:[0,1]\rightarrow{S}$ defined by:
            \begin{equation}
                f(x)=\big(\cos(\pi{x}),\,\sin(\pi{x})\big)
            \end{equation}
            But $[0,1]$ does have the midpoint property, whereas the upper
            semi-circle does not. That is, closed intervals are convex where
            circles (the boundaries of discs) are not. To topologies this
            question we might ask if there exists a metric with the midpoint
            property or if every metric has it. That is, can a given topological
            space $\topspace{X}$ be given a metric $d$ with the midpoint
            property? Are there topological spaces $\topspace{X}$ where every
            metric on $X$ that induces $\tau$ must have the midpoint property?
        \end{example}
        Def separable.
        \begin{theorem}
            If $\topspace{X}$ is a separable and metrizable topological space,
            then it is second countable.
        \end{theorem}
        \begin{proof}
            For if $\topspace{X}$ is separable, there exists a countable dense
            subset $A$, and if $\topspace{X}$ is metrizable, there exists a
            metric $d$ on $X$ that induces the topology $\tau$. Define
            $\mathcal{B}$ by:
            \begin{equation}
                \mathcal{B}=
                    \{B_{q}^{\metspace{X}}(x)\;|\;
                        x\in{A}\textrm{ and }q\in\mathbb{Q^{+}}\}
            \end{equation}
        \end{proof}
        \begin{example}
            A common \textit{non-theorem} that confuses students is the belief
            that separable and first countable impliy second countable, and this
            is not true. What we've shown is that first countable and separable
            imply second countable if we're working with metric spaces. For
            example, consider the particular point topology on $\mathbb{R}$.
            That is, a set $\mathcal{U}$ is open if and only if it is either
            empty or contains the origin. Then $\topspace{\mathbb{R}}$ is
            first countable. To see this, let $x\in\mathbb{R}$ and consider:
            \begin{equation}
                \mathcal{B}_{x}=\big\{\,\{x,\,0\}\,\big\}
            \end{equation}
            this is a neighborhood basis for $x$, and hence the space is first
            countable. Moreover it is separable since
            $\closure{\{0\}}=\mathbb{R}$. To see that it is not second countable
            we'll show that it is not $\sigma$ locally finite. For suppose
            $\mathcal{O}$ is a locally finite collection. That $0$ is contained
            in finitely many elements of $\mathcal{O}$, and thus $\mathcal{O}$
            must itself be finite. But then for any countable collection of
            locally finite sets we have that this can't be a a basis since
            $\mathbb{R}$ is uncountable. Hence $\topspace{\mathbb{R}}$ is not
            $\sigma$ locally finite, and therefore it is not secound countable.
        \end{example}
        \begin{example}
            As another example, let $H$ be the closed upper half plane in
            $\mathbb{R}^{2}$. That is, all point $(x,y)\in\mathbb{R}^{2}$ such
            that $y\geq{0}$. Consider the following basis: open balls in the
            interior of the upper half plane combined with open balls that lie
            tangent to the $x$ axis together with the tangential point $(x,0)$.
            Then the intersection of $\mathbb{Q}^{2}$ with the upper haf plane
            still forms a countable dense subset, but any basis needs to contain
            at least one set for every $(x,0)$. Since $\mathbb{R}$ is
            uncountable, this space cannot possibly be second countable.
        \end{example}
\end{document}