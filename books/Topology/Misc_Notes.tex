%------------------------------------------------------------------------------%
\documentclass{article}                                                        %
%------------------------------Preamble----------------------------------------%
\makeatletter                                                                  %
    \def\input@path{{../../}}                                                  %
\makeatother                                                                   %
%---------------------------Packages----------------------------%
\usepackage{geometry}
\geometry{b5paper, margin=1.0in}
\usepackage[T1]{fontenc}
\usepackage{graphicx, float}            % Graphics/Images.
\usepackage{natbib}                     % For bibliographies.
\bibliographystyle{agsm}                % Bibliography style.
\usepackage[french, english]{babel}     % Language typesetting.
\usepackage[dvipsnames]{xcolor}         % Color names.
\usepackage{listings}                   % Verbatim-Like Tools.
\usepackage{mathtools, esint, mathrsfs} % amsmath and integrals.
\usepackage{amsthm, amsfonts, amssymb}  % Fonts and theorems.
\usepackage{tcolorbox}                  % Frames around theorems.
\usepackage{upgreek}                    % Non-Italic Greek.
\usepackage{fmtcount, etoolbox}         % For the \book{} command.
\usepackage[newparttoc]{titlesec}       % Formatting chapter, etc.
\usepackage{titletoc}                   % Allows \book in toc.
\usepackage[nottoc]{tocbibind}          % Bibliography in toc.
\usepackage[titles]{tocloft}            % ToC formatting.
\usepackage{pgfplots, tikz}             % Drawing/graphing tools.
\usepackage{imakeidx}                   % Used for index.
\usetikzlibrary{
    calc,                   % Calculating right angles and more.
    angles,                 % Drawing angles within triangles.
    arrows.meta,            % Latex and Stealth arrows.
    quotes,                 % Adding labels to angles.
    positioning,            % Relative positioning of nodes.
    decorations.markings,   % Adding arrows in the middle of a line.
    patterns,
    arrows
}                                       % Libraries for tikz.
\pgfplotsset{compat=1.9}                % Version of pgfplots.
\usepackage[font=scriptsize,
            labelformat=simple,
            labelsep=colon]{subcaption} % Subfigure captions.
\usepackage[font={scriptsize},
            hypcap=true,
            labelsep=colon]{caption}    % Figure captions.
\usepackage[pdftex,
            pdfauthor={Ryan Maguire},
            pdftitle={Mathematics and Physics},
            pdfsubject={Mathematics, Physics, Science},
            pdfkeywords={Mathematics, Physics, Computer Science, Biology},
            pdfproducer={LaTeX},
            pdfcreator={pdflatex}]{hyperref}
\hypersetup{
    colorlinks=true,
    linkcolor=blue,
    filecolor=magenta,
    urlcolor=Cerulean,
    citecolor=SkyBlue
}                           % Colors for hyperref.
\usepackage[toc,acronym,nogroupskip,nopostdot]{glossaries}
\usepackage{glossary-mcols}
%------------------------Theorem Styles-------------------------%
\theoremstyle{plain}
\newtheorem{theorem}{Theorem}[section]

% Define theorem style for default spacing and normal font.
\newtheoremstyle{normal}
    {\topsep}               % Amount of space above the theorem.
    {\topsep}               % Amount of space below the theorem.
    {}                      % Font used for body of theorem.
    {}                      % Measure of space to indent.
    {\bfseries}             % Font of the header of the theorem.
    {}                      % Punctuation between head and body.
    {.5em}                  % Space after theorem head.
    {}

% Italic header environment.
\newtheoremstyle{thmit}{\topsep}{\topsep}{}{}{\itshape}{}{0.5em}{}

% Define environments with italic headers.
\theoremstyle{thmit}
\newtheorem*{solution}{Solution}

% Define default environments.
\theoremstyle{normal}
\newtheorem{example}{Example}[section]
\newtheorem{definition}{Definition}[section]
\newtheorem{problem}{Problem}[section]

% Define framed environment.
\tcbuselibrary{most}
\newtcbtheorem[use counter*=theorem]{ftheorem}{Theorem}{%
    before=\par\vspace{2ex},
    boxsep=0.5\topsep,
    after=\par\vspace{2ex},
    colback=green!5,
    colframe=green!35!black,
    fonttitle=\bfseries\upshape%
}{thm}

\newtcbtheorem[auto counter, number within=section]{faxiom}{Axiom}{%
    before=\par\vspace{2ex},
    boxsep=0.5\topsep,
    after=\par\vspace{2ex},
    colback=Apricot!5,
    colframe=Apricot!35!black,
    fonttitle=\bfseries\upshape%
}{ax}

\newtcbtheorem[use counter*=definition]{fdefinition}{Definition}{%
    before=\par\vspace{2ex},
    boxsep=0.5\topsep,
    after=\par\vspace{2ex},
    colback=blue!5!white,
    colframe=blue!75!black,
    fonttitle=\bfseries\upshape%
}{def}

\newtcbtheorem[use counter*=example]{fexample}{Example}{%
    before=\par\vspace{2ex},
    boxsep=0.5\topsep,
    after=\par\vspace{2ex},
    colback=red!5!white,
    colframe=red!75!black,
    fonttitle=\bfseries\upshape%
}{ex}

\newtcbtheorem[auto counter, number within=section]{fnotation}{Notation}{%
    before=\par\vspace{2ex},
    boxsep=0.5\topsep,
    after=\par\vspace{2ex},
    colback=SeaGreen!5!white,
    colframe=SeaGreen!75!black,
    fonttitle=\bfseries\upshape%
}{not}

\newtcbtheorem[use counter*=remark]{fremark}{Remark}{%
    fonttitle=\bfseries\upshape,
    colback=Goldenrod!5!white,
    colframe=Goldenrod!75!black}{ex}

\newenvironment{bproof}{\textit{Proof.}}{\hfill$\square$}
\tcolorboxenvironment{bproof}{%
    blanker,
    breakable,
    left=3mm,
    before skip=5pt,
    after skip=10pt,
    borderline west={0.6mm}{0pt}{green!80!black}
}

\AtEndEnvironment{lexample}{$\hfill\textcolor{red}{\blacksquare}$}
\newtcbtheorem[use counter*=example]{lexample}{Example}{%
    empty,
    title={Example~\theexample},
    boxed title style={%
        empty,
        size=minimal,
        toprule=2pt,
        top=0.5\topsep,
    },
    coltitle=red,
    fonttitle=\bfseries,
    parbox=false,
    boxsep=0pt,
    before=\par\vspace{2ex},
    left=0pt,
    right=0pt,
    top=3ex,
    bottom=1ex,
    before=\par\vspace{2ex},
    after=\par\vspace{2ex},
    breakable,
    pad at break*=0mm,
    vfill before first,
    overlay unbroken={%
        \draw[red, line width=2pt]
            ([yshift=-1.2ex]title.south-|frame.west) to
            ([yshift=-1.2ex]title.south-|frame.east);
        },
    overlay first={%
        \draw[red, line width=2pt]
            ([yshift=-1.2ex]title.south-|frame.west) to
            ([yshift=-1.2ex]title.south-|frame.east);
    },
}{ex}

\AtEndEnvironment{ldefinition}{$\hfill\textcolor{Blue}{\blacksquare}$}
\newtcbtheorem[use counter*=definition]{ldefinition}{Definition}{%
    empty,
    title={Definition~\thedefinition:~{#1}},
    boxed title style={%
        empty,
        size=minimal,
        toprule=2pt,
        top=0.5\topsep,
    },
    coltitle=Blue,
    fonttitle=\bfseries,
    parbox=false,
    boxsep=0pt,
    before=\par\vspace{2ex},
    left=0pt,
    right=0pt,
    top=3ex,
    bottom=0pt,
    before=\par\vspace{2ex},
    after=\par\vspace{1ex},
    breakable,
    pad at break*=0mm,
    vfill before first,
    overlay unbroken={%
        \draw[Blue, line width=2pt]
            ([yshift=-1.2ex]title.south-|frame.west) to
            ([yshift=-1.2ex]title.south-|frame.east);
        },
    overlay first={%
        \draw[Blue, line width=2pt]
            ([yshift=-1.2ex]title.south-|frame.west) to
            ([yshift=-1.2ex]title.south-|frame.east);
    },
}{def}

\AtEndEnvironment{ltheorem}{$\hfill\textcolor{Green}{\blacksquare}$}
\newtcbtheorem[use counter*=theorem]{ltheorem}{Theorem}{%
    empty,
    title={Theorem~\thetheorem:~{#1}},
    boxed title style={%
        empty,
        size=minimal,
        toprule=2pt,
        top=0.5\topsep,
    },
    coltitle=Green,
    fonttitle=\bfseries,
    parbox=false,
    boxsep=0pt,
    before=\par\vspace{2ex},
    left=0pt,
    right=0pt,
    top=3ex,
    bottom=-1.5ex,
    breakable,
    pad at break*=0mm,
    vfill before first,
    overlay unbroken={%
        \draw[Green, line width=2pt]
            ([yshift=-1.2ex]title.south-|frame.west) to
            ([yshift=-1.2ex]title.south-|frame.east);},
    overlay first={%
        \draw[Green, line width=2pt]
            ([yshift=-1.2ex]title.south-|frame.west) to
            ([yshift=-1.2ex]title.south-|frame.east);
    }
}{thm}

%--------------------Declared Math Operators--------------------%
\DeclareMathOperator{\adjoint}{adj}         % Adjoint.
\DeclareMathOperator{\Card}{Card}           % Cardinality.
\DeclareMathOperator{\curl}{curl}           % Curl.
\DeclareMathOperator{\diam}{diam}           % Diameter.
\DeclareMathOperator{\dist}{dist}           % Distance.
\DeclareMathOperator{\Div}{div}             % Divergence.
\DeclareMathOperator{\Erf}{Erf}             % Error Function.
\DeclareMathOperator{\Erfc}{Erfc}           % Complementary Error Function.
\DeclareMathOperator{\Ext}{Ext}             % Exterior.
\DeclareMathOperator{\GCD}{GCD}             % Greatest common denominator.
\DeclareMathOperator{\grad}{grad}           % Gradient
\DeclareMathOperator{\Ima}{Im}              % Image.
\DeclareMathOperator{\Int}{Int}             % Interior.
\DeclareMathOperator{\LC}{LC}               % Leading coefficient.
\DeclareMathOperator{\LCM}{LCM}             % Least common multiple.
\DeclareMathOperator{\LM}{LM}               % Leading monomial.
\DeclareMathOperator{\LT}{LT}               % Leading term.
\DeclareMathOperator{\Mod}{mod}             % Modulus.
\DeclareMathOperator{\Mon}{Mon}             % Monomial.
\DeclareMathOperator{\multideg}{mutlideg}   % Multi-Degree (Graphs).
\DeclareMathOperator{\nul}{nul}             % Null space of operator.
\DeclareMathOperator{\Ord}{Ord}             % Ordinal of ordered set.
\DeclareMathOperator{\Prin}{Prin}           % Principal value.
\DeclareMathOperator{\proj}{proj}           % Projection.
\DeclareMathOperator{\Refl}{Refl}           % Reflection operator.
\DeclareMathOperator{\rk}{rk}               % Rank of operator.
\DeclareMathOperator{\sgn}{sgn}             % Sign of a number.
\DeclareMathOperator{\sinc}{sinc}           % Sinc function.
\DeclareMathOperator{\Span}{Span}           % Span of a set.
\DeclareMathOperator{\Spec}{Spec}           % Spectrum.
\DeclareMathOperator{\supp}{supp}           % Support
\DeclareMathOperator{\Tr}{Tr}               % Trace of matrix.
%--------------------Declared Math Symbols--------------------%
\DeclareMathSymbol{\minus}{\mathbin}{AMSa}{"39} % Unary minus sign.
%------------------------New Commands---------------------------%
\DeclarePairedDelimiter\norm{\lVert}{\rVert}
\DeclarePairedDelimiter\ceil{\lceil}{\rceil}
\DeclarePairedDelimiter\floor{\lfloor}{\rfloor}
\newcommand*\diff{\mathop{}\!\mathrm{d}}
\newcommand*\Diff[1]{\mathop{}\!\mathrm{d^#1}}
\renewcommand*{\glstextformat}[1]{\textcolor{RoyalBlue}{#1}}
\renewcommand{\glsnamefont}[1]{\textbf{#1}}
\renewcommand\labelitemii{$\circ$}
\renewcommand\thesubfigure{%
    \arabic{chapter}.\arabic{figure}.\arabic{subfigure}}
\addto\captionsenglish{\renewcommand{\figurename}{Fig.}}
\numberwithin{equation}{section}

\renewcommand{\vector}[1]{\boldsymbol{\mathrm{#1}}}

\newcommand{\uvector}[1]{\boldsymbol{\hat{\mathrm{#1}}}}
\newcommand{\topspace}[2][]{(#2,\tau_{#1})}
\newcommand{\measurespace}[2][]{(#2,\varSigma_{#1},\mu_{#1})}
\newcommand{\measurablespace}[2][]{(#2,\varSigma_{#1})}
\newcommand{\manifold}[2][]{(#2,\tau_{#1},\mathcal{A}_{#1})}
\newcommand{\tanspace}[2]{T_{#1}{#2}}
\newcommand{\cotanspace}[2]{T_{#1}^{*}{#2}}
\newcommand{\Ckspace}[3][\mathbb{R}]{C^{#2}(#3,#1)}
\newcommand{\funcspace}[2][\mathbb{R}]{\mathcal{F}(#2,#1)}
\newcommand{\smoothvecf}[1]{\mathfrak{X}(#1)}
\newcommand{\smoothonef}[1]{\mathfrak{X}^{*}(#1)}
\newcommand{\bracket}[2]{[#1,#2]}

%------------------------Book Command---------------------------%
\makeatletter
\renewcommand\@pnumwidth{1cm}
\newcounter{book}
\renewcommand\thebook{\@Roman\c@book}
\newcommand\book{%
    \if@openright
        \cleardoublepage
    \else
        \clearpage
    \fi
    \thispagestyle{plain}%
    \if@twocolumn
        \onecolumn
        \@tempswatrue
    \else
        \@tempswafalse
    \fi
    \null\vfil
    \secdef\@book\@sbook
}
\def\@book[#1]#2{%
    \refstepcounter{book}
    \addcontentsline{toc}{book}{\bookname\ \thebook:\hspace{1em}#1}
    \markboth{}{}
    {\centering
     \interlinepenalty\@M
     \normalfont
     \huge\bfseries\bookname\nobreakspace\thebook
     \par
     \vskip 20\p@
     \Huge\bfseries#2\par}%
    \@endbook}
\def\@sbook#1{%
    {\centering
     \interlinepenalty \@M
     \normalfont
     \Huge\bfseries#1\par}%
    \@endbook}
\def\@endbook{
    \vfil\newpage
        \if@twoside
            \if@openright
                \null
                \thispagestyle{empty}%
                \newpage
            \fi
        \fi
        \if@tempswa
            \twocolumn
        \fi
}
\newcommand*\l@book[2]{%
    \ifnum\c@tocdepth >-3\relax
        \addpenalty{-\@highpenalty}%
        \addvspace{2.25em\@plus\p@}%
        \setlength\@tempdima{3em}%
        \begingroup
            \parindent\z@\rightskip\@pnumwidth
            \parfillskip -\@pnumwidth
            {
                \leavevmode
                \Large\bfseries#1\hfill\hb@xt@\@pnumwidth{\hss#2}
            }
            \par
            \nobreak
            \global\@nobreaktrue
            \everypar{\global\@nobreakfalse\everypar{}}%
        \endgroup
    \fi}
\newcommand\bookname{Book}
\renewcommand{\thebook}{\texorpdfstring{\Numberstring{book}}{book}}
\providecommand*{\toclevel@book}{-2}
\makeatother
\titleformat{\part}[display]
    {\Large\bfseries}
    {\partname\nobreakspace\thepart}
    {0mm}
    {\Huge\bfseries}
\titlecontents{part}[0pt]
    {\large\bfseries}
    {\partname\ \thecontentslabel: \quad}
    {}
    {\hfill\contentspage}
\titlecontents{chapter}[0pt]
    {\bfseries}
    {\chaptername\ \thecontentslabel:\quad}
    {}
    {\hfill\contentspage}
\newglossarystyle{longpara}{%
    \setglossarystyle{long}%
    \renewenvironment{theglossary}{%
        \begin{longtable}[l]{{p{0.25\hsize}p{0.65\hsize}}}
    }{\end{longtable}}%
    \renewcommand{\glossentry}[2]{%
        \glstarget{##1}{\glossentryname{##1}}%
        &\glossentrydesc{##1}{~##2.}
        \tabularnewline%
        \tabularnewline
    }%
}
\newglossary[not-glg]{notation}{not-gls}{not-glo}{Notation}
\newcommand*{\newnotation}[4][]{%
    \newglossaryentry{#2}{type=notation, name={\textbf{#3}, },
                          text={#4}, description={#4},#1}%
}
%--------------------------LENGTHS------------------------------%
% Spacings for the Table of Contents.
\addtolength{\cftsecnumwidth}{1ex}
\addtolength{\cftsubsecindent}{1ex}
\addtolength{\cftsubsecnumwidth}{1ex}
\addtolength{\cftfignumwidth}{1ex}
\addtolength{\cfttabnumwidth}{1ex}

% Indent and paragraph spacing.
\setlength{\parindent}{0em}
\setlength{\parskip}{0em}                                                           %
%----------------------------Main Document-------------------------------------%
\begin{document}
    \title{Topology Notes}
    \author{Ryan Maguire}
    \date{\vspace{-5ex}}
    \maketitle
    \section{Hocking and Young (Chapter 1)}
        \begin{example}
            Let $\mathbb{R}$ be the real numbers with the usual topology and
            consider the subset $A\subseteq\mathbb{R}$ defined by:
            \begin{equation}
                A=\Big\{\,\frac{1}{n}\;|\;n\in\mathbb{N}\,\Big\}
            \end{equation}
            $A$ has only one limit point and this is zero. Note that
            $0\notin{A}$, and hence the derived set $\derived{A}$ is disjoint
            from $A$ itself. That is:
            \begin{equation}
                \derived[]{A}=\{0\}
            \end{equation}
            And thence $\derived[]{A}\cap{A}=\emptyset$.
        \end{example}
        \begin{example}
            Again consider $\mathbb{R}$ and define $B$ by:
            \begin{equation}
                B=\Big\{\,\frac{n+1}{n}+(\minus{1})^{n}\frac{n-1}{n}\;|\;
                    n\in\mathbb{N}\,\Big\}
            \end{equation}
            This oscillates between values that are asymptotically approaching
            zero and one. The derived set is then:
            \begin{equation}
                \derived[]{B}=\{0,\,1\}
            \end{equation}
            Once again, the derived set is disjoint from the original set.
        \end{example}
        \begin{example}
            In the usual topology, the derived set of $\mathbb{Q}$ is the
            entirety of $\mathbb{R}$. That is:
            \begin{equation}
                \derived[]{\mathbb{Q}}=\mathbb{R}
            \end{equation}
            So it is possible for the derived set to be strictly larger than the
            original set.
        \end{example}
        \begin{example}
            If $\topspace{X}$ is the chaotic topological space on some set $X$
            with the trivial topology $\tau=\{\emptyset,X\}$, then for any
            subset $A\subseteq{X}$, it is true that $\derived{A}=X$. That is,
            in the chaotic topology every point is a limit point of every other
            point.
        \end{example}
        \begin{example}
            If $\topspace{X}$ is the discrete topology, $\tau=\powset{X}$, then
            for any subset $A\subseteq{X}$, the derived set is empty:
            $\derived{A}=\emptyset$. That is, no point is a limit point of any
            subset. This is because every element in the discrete topology is
            isolated since $\{x\}$ is an open subset for every $x\in{X}$.
        \end{example}
        \begin{theorem}
            If $\topspace{X}$ is a topological space, if $A,B\subseteq{X}$, and
            if $A\subseteq{B}$, then $\derived{A}\subseteq\derived{B}$, where
            $\derived{Y}$ denotes the derived set of $Y$ in $\tau$.
        \end{theorem}
        \begin{theorem}
            If $\topspace{X}$ is a topological space, if $C\subseteq{X}$, then
            $C$ is closed if and only if $\closure{C}=C$.
        \end{theorem}
        \begin{theorem}
            If $\topspace{X}$ is a topological space, if
            $\mathcal{C}\subseteq\powset{X}$ is such that for all
            $C\in\mathcal{C}$ it is true that $C$ is closed, then
            $\bigcap\mathcal{C}$ is closed.
        \end{theorem}
        \begin{theorem}
            If $\topspace{X}$ is a topological space, if $C,D\subseteq{X}$ are
            closed in $X$, then $C\cup{D}$ is closed in $X$.
        \end{theorem}
        \begin{definition}
            A basis for a topology $\tau$ on a set $X$ is a subset
            $\mathcal{B}\subseteq\tau$ such that $\bigcup\mathcal{B}=X$ and such
            that for all $B_{1},B_{2}\in\mathcal{B}$ and for all
            $p\in{B}_{1}\cap{B}_{2}$, there exists $B_{3}\in\mathcal{B}$ such
            that $x\in{B}_{3}$ and $B_{2}\subseteq{B}_{1}\cap{B}_{2}$.
        \end{definition}
        \begin{theorem}
            If $\topspace{X}$ is a topological space, then $\tau$ is a basis
            for $\tau$.
        \end{theorem}
        \begin{proof}
            For $X\in\tau$, and hence $X\subseteq\bigcup\tau$, but since
            $\tau\subseteq\powset{X}$, for all $\mathcal{U}\in\tau$ it is true
            that $\mathcal{U}\subseteq{X}$, and hence $X=\bigcup\tau$. If
            $\mathcal{U}_{1},\mathcal{U}_{2}$, and if
            $x\in\mathcal{U}_{1}\cap\mathcal{U}_{2}$, then
            $\mathcal{U}_{1}\cap\mathcal{U}_{2}\in\tau$ is such that
            $x\in\mathcal{U}_{1}\cap\mathcal{U}_{2}$ and
            $\mathcal{U}_{1}\cap\mathcal{U}_{2}\subseteq%
             \mathcal{U}_{1}\cap\mathcal{U}_{2}$, and hence $\tau$ is a basis
            for $\tau$.
        \end{proof}
        \begin{theorem}
            If $\topspace{X}$ is a topological space, if $\mathcal{B}$ is a
            basis for $\tau$, then:
            \begin{equation}
                \tau=\Big\{\,\mathcal{U}\in\powset{X}\;\big|\;
                    \textrm{There exists }\mathcal{O}\subseteq\mathcal{B}
                    \textrm{ such that }\mathcal{U}=\bigcup\mathcal{O}\Big\}
            \end{equation}
        \end{theorem}
        \begin{theorem}
            If $X$ is a set, if $\tau_{1}$ and $\tau_{2}$ are topologies on $X$,
            if $\tau_{1}\subseteq\tau_{2}$, if $\mathcal{B}_{1}$ is a basis for
            $\tau_{1}$, if $\mathcal{B}_{2}$ is a basis for $\tau_{2}$, if
            $x\in{X}$, and if $B_{1}\in\mathcal{B}_{1}$ is such that
            $x\in{B}_{1}$, there exists a $B_{2}\in\mathcal{B}_{2}$ such that
            $x\in{B}_{2}$ and $B_{2}\subseteq{B}_{1}$.
        \end{theorem}
        \begin{example}
            The previous theorem shows when two bases for a topology $\tau$ are
            the same, but different bases may give rise to different topologies.
            Consider on $\mathbb{R}$ the standard topology $\tau_{\mathbb{R}}$
            and the topology $\tau$ generated by the set:
            \begin{equation}
                \mathcal{B}=\{\,(x,\infty)\;|\;x\in\mathbb{R}\}
            \end{equation}
            We can see that $\tau\subseteq\tau_{\mathbb{R}}$ since every set
            in $\mathcal{B}$ can be obtained as the union of intervals as
            follows:
            \begin{equation}
                (x,\infty)=\bigcup_{n\in\mathbb{N}}(x,x+n)
            \end{equation}
            Since $(x,x+n)\in\tau_{\mathbb{R}}$ for all $n\in\mathbb{N}$ and for
            all $x\in\mathbb{R}$, and since topologies are closed under
            arbitrary unions, we thus have that every element of $\mathcal{B}$
            is contained in $\tau_{\mathbb{R}}$ and thus every element of
            $\tau$ is also contained in $\tau_{\mathbb{R}}$. However, these are
            not the same topology.
        \end{example}
        \begin{example}
            In $\mathbb{R}^{2}$, the set of all vertical open line segments
            forms the basis of a topology that is strictly finer than the
            standard Euclidean topology $\tau_{\mathbb{R}^{2}}$, and indeed this
            can be seen as the product topology on $\mathbb{R}$ with the
            standard topology multiplied by $\mathbb{R}$ with the discrete
            topology. The resulting space, being the product of metric spaces,
            is again a metric space. Since the discrete topology $\tau_{D}$ is
            strictly finer than the standard topology on $\mathbb{R}$, we see
            that the product topology $\tau_{\mathbb{R}}\times\tau_{\mathrm{D}}$
            is strictly finer than the standard one $\top_{\mathbb{R}^{2}}$.
        \end{example}
        Def second countable. Intersection of topologies is topology. Topology
        generated by a collection.
        \begin{theorem}
            If $X$ is a set, if $\mathcal{O}\subseteq\powset{X}$, if
            $\tau(\mathcal{O})$ is the topology generated by $\mathcal{O}$,
            and if $\mathcal{B}$ is the set:
            \begin{equation}
                \mathcal{B}=\Big\{\,\mathcal{U}\in\powset{X}\;|\;
                    \exists_{n\in\mathbb{N}}
                    \exists_{B:\mathbb{Z}_{n}\rightarrow\mathcal{O}}
                    \big(\mathcal{U}=\bigcap{B}_{k}\big)\,\Big\}
            \end{equation}
            Then $\mathcal{B}$ is a basis for $\tau(\mathcal{O})$.
        \end{theorem}
        That is, the collection of all finite intersections of elements of
        $\mathcal{O}$ forms a basis for the topology that $\mathcal{O}$
        generates.
        \begin{definition}
            A subbasis for a topological space $\topspace{X}$ is a subset
            $\mathcal{B}\subseteq\powset{X}$ such that the topology generated
            by $\mathcal{B}$ is equal to $\tau$.
        \end{definition}
        \begin{example}
            Consider the set of all open rays on $\mathbb{R}$. That is, the set
            of all subsets of $\mathbb{R}$ of the form:
            \twocolumneq{\mathcal{U}_{+}=(x,\infty)}
                        {\mathcal{U}_{\minus}=(\minus\infty,x)}
            Then this forms a subbasis for the standard topology on
            $\mathbb{R}$. To see this we simply need to show that the standard
            basis for $\mathbb{R}$ can be obtained from finite intersections of
            elements of $\mathcal{B}$, and also note that
            $\mathcal{B}\subseteq\tau_{\mathbb{R}}$. Given an open interval
            $(a,b)\subseteq\mathbb{R}$, let $B_{1}=(a,\infty)$ and
            $B_{2}=(\minus\infty,b)$. Then both $B_{1},B_{2}\in\mathcal{B}$,
            but $(a,b)=B_{1}\cap{B}_{2}$. Thus the topology generated by the
            subbasis $\mathcal{B}$ is the same as the topology generated by the
            standard basis of open intervals, and this is the standard
            topology on $\mathbb{R}$.
        \end{example}
        \begin{ltheorem}{Birkhoff's Topology Lattice Theorem}
            If $X$ is a set, if $T$ is the set of all topologies of $X$, and if
            $\subseteq$ is the inclusion relation, then $(T,\subseteq)$ is a
            complete lattice.
        \end{ltheorem}
        \begin{theorem}
            If $X$ is a set, if $\tau_{1},\tau_{2}$ are topologies on $X$,
            if $\tau_{1}\subseteq\tau_{2}$, and if
            $\identity{X}:\topspace[1]{X}\rightarrow\topspace[2]{X}$ is the
            identity mapping, then it open.
        \end{theorem}
        \begin{theorem}
            If $X$ is a set, if $\tau_{1},\tau_{2}$ are topologies on $X$,
            if $\tau_{2}\subseteq\tau_{1}$, and if
            $\identity{X}:\topspace[1]{X}\rightarrow\topspace[2]{X}$ is the
            identity mapping, then it continuous.
        \end{theorem}
        \begin{theorem}
            If $\topspace[Y]{Y}$ is a topological space, if $X$ is a set, if
            $\tau_{1},\tau_{2}$ are topologies on $X$, if
            $\tau_{1}\subseteq\tau_{2}$, and if $f:X\rightarrow{Y}$ is
            continuous with respect to $\tau_{1}$, then it is continuous with
            respect to $\tau_{2}$.
        \end{theorem}
        \begin{theorem}
            If $\topspace[X]{X}$ is a topological space, if $Y$ is a set, if
            $\tau_{1},\tau_{2}$ are topologies on $Y$, if
            $\tau_{1}\subseteq\tau_{2}$, and if $f:X\rightarrow{Y}$ is
            continuous with respect to $\tau_{2}$, then it is continuous with
            respect to $\tau_{1}$.
        \end{theorem}
        Similar result for open maps. Def metric space, metrics, open balls.
        \begin{example}
            Metrics are not topological entities, and two vastly different
            metrics on the same set may give the same topology. For example,
            given any metric $d$ on a set $X$, the metric $\tilde{d}$ formed by:
            \begin{equation}
                \tilde{d}(x,y)=\frac{d(x,y)}{1+d(x,y)}
            \end{equation}
            is bounded, yet forms the same topology. Hence boundedness is not a
            topological property but a metric one. Thus, there are questions
            about metric spaces that cannot be answered from the study of
            metrizable spaces. For example, if $\metspace{X}$ is a metric space
            one can ask if it has the midpoint property: For all $x,y\in{X}$
            is there a $z\in{X}$ such that $d(x,z)=d(y,z)=d(x,y)/2$? This cannot
            be answer topologically, for consider the closed unit interval
            $[0,1]\subseteq\mathbb{R}$ and the closed upper half unit circle in
            $\mathbb{R}^{2}$. That is:
            \begin{equation}
                S=\{\,(x,y)\in\mathbb{S}_{1}\;|y\geq{0}\,\}
            \end{equation}
            Then $[0,1]$ and $S$ are homeomorphic under the function
            $f:[0,1]\rightarrow{S}$ defined by:
            \begin{equation}
                f(x)=\big(\cos(\pi{x}),\,\sin(\pi{x})\big)
            \end{equation}
            But $[0,1]$ does have the midpoint property, whereas the upper
            semi-circle does not. That is, closed intervals are convex where
            circles (the boundaries of discs) are not. To topologies this
            question we might ask if there exists a metric with the midpoint
            property or if every metric has it. That is, can a given topological
            space $\topspace{X}$ be given a metric $d$ with the midpoint
            property? Are there topological spaces $\topspace{X}$ where every
            metric on $X$ that induces $\tau$ must have the midpoint property?
        \end{example}
        Def separable.
        \begin{theorem}
            If $\topspace{X}$ is a separable and metrizable topological space,
            then it is second countable.
        \end{theorem}
        \begin{proof}
            For if $\topspace{X}$ is separable, there exists a countable dense
            subset $A$, and if $\topspace{X}$ is metrizable, there exists a
            metric $d$ on $X$ that induces the topology $\tau$. Define
            $\mathcal{B}$ by:
            \begin{equation}
                \mathcal{B}=
                    \{B_{q}^{\metspace{X}}(x)\;|\;
                        x\in{A}\textrm{ and }q\in\mathbb{Q^{+}}\}
            \end{equation}
        \end{proof}
        \begin{example}
            A common \textit{non-theorem} that confuses students is the belief
            that separable and first countable impliy second countable, and this
            is not true. What we've shown is that first countable and separable
            imply second countable if we're working with metric spaces. For
            example, consider the particular point topology on $\mathbb{R}$.
            That is, a set $\mathcal{U}$ is open if and only if it is either
            empty or contains the origin. Then $\topspace{\mathbb{R}}$ is
            first countable. To see this, let $x\in\mathbb{R}$ and consider:
            \begin{equation}
                \mathcal{B}_{x}=\big\{\,\{x,\,0\}\,\big\}
            \end{equation}
            this is a neighborhood basis for $x$, and hence the space is first
            countable. Moreover it is separable since
            $\closure{\{0\}}=\mathbb{R}$. To see that it is not second countable
            we'll show that it is not $\sigma$ locally finite. For suppose
            $\mathcal{O}$ is a locally finite collection. That $0$ is contained
            in finitely many elements of $\mathcal{O}$, and thus $\mathcal{O}$
            must itself be finite. But then for any countable collection of
            locally finite sets we have that this can't be a a basis since
            $\mathbb{R}$ is uncountable. Hence $\topspace{\mathbb{R}}$ is not
            $\sigma$ locally finite, and therefore it is not secound countable.
        \end{example}
        \begin{example}
            As another example, let $H$ be the closed upper half plane in
            $\mathbb{R}^{2}$. That is, all point $(x,y)\in\mathbb{R}^{2}$ such
            that $y\geq{0}$. Consider the following basis: open balls in the
            interior of the upper half plane combined with open balls that lie
            tangent to the $x$ axis together with the tangential point $(x,0)$.
            Then the intersection of $\mathbb{Q}^{2}$ with the upper haf plane
            still forms a countable dense subset, but any basis needs to contain
            at least one set for every $(x,0)$. Since $\mathbb{R}$ is
            uncountable, this space cannot possibly be second countable.
        \end{example}
        Def cont func.
        \begin{theorem}
            If $\topspace[X]{X}$ and $\topspace[Y]{Y}$ are topological spaces,
            and if $f:X\rightarrow{Y}$ is a continuous function, then for all
            $x\in{X}$ and for all $\mathcal{V}\in\tau_{Y}$ such that
            $f(x)\in\mathcal{V}$, there exists a $\mathcal{U}\in\tau_{X}$ such
            that $f[\mathcal{U}]\subseteq\mathcal{V}$.
        \end{theorem}
        $\varepsilon-\delta$ def of cont for metric space.
        \begin{example}
            There are bijective continuous function $f:X\rightarrow{Y}$ that
            are not homeomorphisms. Let $\nsphere[1]$ be the unit circle and
            $[0,1)$ be the semi-open interval with the usual inherited metric
            topology from $\nspace[]$. Define $f:[0,1)\rightarrow\nsphere[1]$
            by:
            \begin{equation}
                f(x)=\big(\cos(2\pi{x}),\sin(2\pi{x})\big)
            \end{equation}
            This is simply wrapping the semi-open interval up into a circle.
            Since the left endpoint 0 is included, $f$ is bijective. It's also
            continuous, however $f^{\minus{1}}$ is not continuous at the point
            $(1,0)\in\nsphere[1]$. The function $f^{\minus{1}}$ effectively
            tears $\nsphere[1]$ at the point, and thus it is not continuous.
            Rigorously, we see that $\nsphere[1]$ is compact and $[0,1)$ is not
            (Both are true by the Heine-Borel theorem), and so $f^{\minus{1}}$
            can't be continuous since continuous functions preserve compactness.
        \end{example}
        Def connected (open def, closed def, clopen def).
        \begin{theorem}
            If $\topspace{\nspace[]}$ is the standard topological space on
            $\nspace[]$, then it is connected.
        \end{theorem}
        \begin{theorem}
            If $\topspace{X}$ is a topological space, then it is disconnected
            if and only if there exists non-empty subsets $A,B\subseteq{X}$ such
            that:
            \begin{equation}
                \big(\closure{A}\cap{B}\big)\cup\big(A\cap\closure{B}\big)
                =\emptyset
            \end{equation}
        \end{theorem}
        \begin{theorem}
            If $\topspace{X}$ is a topological space, if
            $\mathcal{C}\subseteq\powset{X}$ is a collection of connected
            subsets of $X$, and if $\bigcap\mathcal{C}\ne\emptyset$, then
            $\bigcup\mathcal{C}$ is connected.
        \end{theorem}
        This theorem provides a rather crude way of showing that $\nspace$ is
        connected. Consider $\nspace$ as the union of all lines through the
        origin. Since we know that $\nspace[]$ is connected, and since all of
        these lines have non-empty intersection (they intersect at the origin),
        their union must again be connected. But this union is simply the
        entirety of $\nspace$, and so Euclidean space is connected.
        \begin{theorem}
            If $n\in\mathbb{N}$, $n>1$, if $\topspace{\nspace}$ is the standard
            Euclidean topological space, and if $\vector{x}\in\nspace$, then
            $\nspace\setminus\{\vector{x}\}$ is connected.
        \end{theorem}
        \begin{proof}
            For it is path connected, and hence connected (connect two dots).
        \end{proof}
        \begin{theorem}
            If $n\in\mathbb{N}$, if $n>0$, and if $\topspace{\nsphere}$ is the
            usual subspace topology on the sphere, then it is connected.
        \end{theorem}
        \begin{proof}
            For $\nspace{n+1}\setminus\{\vector{0}\}$ is connected by the
            previous theorem, and the function
            $f:\nspace{n+1}\setminus\{\vector{0}\}\rightarrow\nsphere$ defined
            by:
            \begin{equation}
                f(\vector{x})=\frac{\vector{x}}{\norm{\vector{x}}}
            \end{equation}
            is surjective and continuous, and hence it's image is connected. But
            since it is surjective, it's image is the entire sphere, and thus
            $\nsphere$ is connected.
        \end{proof}
        IVP. Comp of cont is cont. Intervals are connected.
        \begin{theorem}
            If $n\in\mathbb{N}$, $n>1$, and if $\topspace{\nspace{n+1}}$ is the
            standard Euclidean topological space, then
            $\nspace{n+1}\setminus\nsphere$ has two open connected components.
        \end{theorem}
        \begin{proof}
            For the open ball $\nball{n+1}$ is path connected, and
            $\nspace{n+1}\setminus\closure{\nball{n+1}}$ is path connected.
        \end{proof}
        Removing a hyperplane leaves two open connected components.
        \begin{theorem}
            If $\topspace{\ntorus}$ is the $n$ torus with the product topology,
            then it is connected.
        \end{theorem}
        \begin{proof}
            For the product of connected is connected, and $\nsphere[1]$ is
            connected.
        \end{proof}
        Def cover, open cover, compactness, compact subset
        (compact in subspace topology).
        \begin{theorem}
            if $\topspace{X}$ is a topological space, and if $A\subseteq{X}$,
            then $A$ is compact if and only if for every subset
            $\mathcal{O}\subseteq\tau$ such that $\mathcal{O}$ is a cover of
            $A$, there is a finite subcover $\Delta\subseteq\mathcal{O}$.
        \end{theorem}
        \begin{proof}
            For if $A$ is a compact subset, then $\topspace[A]{A}$ is a compact
            space where $\tau_{A}$ is the subspace topology. But then:
            \begin{equation}
                \mathcal{O}_{A}=\{\,A\cap\mathcal{U}\;|\;
                    \mathcal{U}\in\mathcal{O}\,\}
            \end{equation}
            is an open cover of $A$ in the subspace topology. But $A$ is
            compact, and thus there is a finite subcover. Yadda yadda.
        \end{proof}
        \begin{fdefinition}{Finite Intersection Property}
                           {Finite_Intersection_Property}
            A topological space with the finite intersection property is a
            topological space $\topspace{X}$ such that for any set
            $\mathcal{C}\subseteq\powset{X}$ of closed sets such that for any
            finite subsets $\Delta\subseteq\mathcal{C}$ it is true that
            $\bigcap\Delta\ne\emptyset$, it is also true that
            $\bigcap\mathcal{C}\ne\emptyset$.
        \end{fdefinition}
        \begin{theorem}
            If $\topspace{X}$ is a topological space, then it is compact if and
            only if it has the finite intersection property.
        \end{theorem}
        \begin{proof}
            For suppose $\topspace{X}$ is compact and does not have the finite
            intersection property. Then there exists a sets
            $\mathcal{C}\subseteq\powset{X}$ such that
            $\bigcap\mathcal{C}=\emptyset$, yet for every finite subset
            $\Delta\subseteq\mathcal{C}$ it is true that
            $\bigcap\mathcal{C}\ne\emptyset$. But if
            $\bigcap\mathcal{C}=\emptyset$, then for all $x\in{X}$ there exists
            $C\in\mathcal{C}$ such that $x\notin{C}$. But then the set
            $\mathcal{O}$ defined by:
            \begin{equation}
                \mathcal{O}=\big\{\,\mathcal{U}\in\tau\;|\;
                    \exists_{C\in\mathcal{C}}
                    \big(\mathcal{U}=X\setminus{C}\big)\,\big\}
            \end{equation}
            is an open cover of $X$. But $X$ is compact, and thus there is a
            finite subcover $\Lambda\subseteq\mathcal{O}$. But then the set
            $\Delta\subseteq\mathcal{C}$ defined by the complements of $\Lambda$
            is finite and has empty intersection, a contradiction. Thus,
            $X$ has the finite intersection property. Next, suppose $X$ has the
            finite intersection property but is not compact. Then there is an
            open cover $\mathcal{O}$ of $X$ with no finite subcover. But
            then the complements $\mathcal{C}$ are a collection of closed sets
            such that $\bigcap\mathcal{C}=\emptyset$. But $X$ has the finite
            intersection property, and thus there exists a finite subset
            $\Delta\subseteq\mathcal{C}$ such that $\bigcap\Delta=\emptyset$.
            But then the set of complements of $\Delta$ is a finite subcover
            of $\mathcal{O}$, a contradiction. Thus, $X$ is compact.
        \end{proof}
        \begin{fdefinition}{Limit Point Compact}{Limit_Point_Compact}
            A limit point compact topological space is a topological space
            $\topspace{X}$ such that for every infinite subset $A\subseteq{X}$,
            the derived set $\derived{A}$ is non-empty.
        \end{fdefinition}
        Limit point compact is also often called
        \textit{weakly countably compact}. We've adopted the name limit point
        compact since this seems common in analysis, and the great theorems
        about compactness in metric spaces (for example, the Bolzano-Weierstrass
        theorem or the generalized Heine-Borel theorem) are often stated in
        terms of sequences or limit points. Moroever, some authors choose to
        make no distinction between limit point compact and countably compact,
        which we shall describe in a moment. The reason being that in an
        accessible space (a $T_{1}$ topological space), limit point compact and
        countably compact are equivalent. Since most spaces that are studied in
        the wild are Hausdorff, they are automatically accessible, and hence
        limit point compact and countably compact are usually the same thing.
        As the name suggests, countably compact is a weakening of compact.
        \begin{fdefinition}{Countably Compact}{Countably_Compact}
            A countably compact topological space is a topological space
            $\topspace{X}$ such that for every countable open cover
            $\mathcal{O}$ of $X$ there exists a finite subcover.
        \end{fdefinition}
        \begin{theorem}
            If $\topspace{X}$ is a compact topological space, then it is
            countably compact.
        \end{theorem}
        \begin{proof}
            For if $\mathcal{O}$ is a countable open cover of $X$, then it is
            an open cover of $X$, and since $X$ is compact there exists a finite
            subcover. Hence, $\topspace{X}$ is countably compact.
        \end{proof}
        In metric spaces, this result reverses, which is quite astounding. This
        is tied into the fact that in metric spaces, sequentially compact and
        compact are one in the same. In general, a countably compact space is
        compact if and only if it is Lindel\"{o}f. This can be seen quite easily
        since given a cover, the Lindel\"{o}f property can reduce this down to
        a countable subcover, and countable compactness then extracts a finite
        subcover. The reverse direction is true since compact implies both
        Lindel\"{o}f and countably compact.
        \begin{theorem}
            If $\topspace{X}$ is a topological space, then it is compact if and
            only if it is countably compact and Lindel\"{o}f.
        \end{theorem}
        \begin{proof}
            For if $\mathcal{O}$ is an open cover, then since $X$ is
            Lindel\"{o}f there exists a countable open subcover $\Delta$. But
            $X$ is countably compact, and thus if $\Delta$ is a countable open
            cover, then there exists a finite subcover $\Lambda$. Hence, $X$
            is compact.
        \end{proof}
        Countably compact always implies limit point compact as well.
        \begin{theorem}
            If $\topspace{X}$ is a countably compact topological space, then it
            is limit point compact.
        \end{theorem}
        \begin{proof}
            For suppose not and let $A\subseteq{X}$ be an infinite set with no
            limit point. Then since $A$ is infinite, there exists a countable
            subset $N\subseteq{A}$. But then there is a bijection
            $a:\mathbb{N}\rightarrow{N}$. But if $A$ has no limit point and
            $A\subseteq{N}$, then $N$ has no limit point. But then for all
            $n\in\mathbb{N}$ there is a $\mathcal{U}_{n}$ such that
            $N\cap\mathcal{U}_{n}=\{a_{n}\}$. But then for all $x\in{X}$ there
            exists an open subset $\mathcal{U}_{x}$ such that
            $\mathcal{U}_{x}\cap{N}$ is finite. Let
            $\mathcal{O}=\{\mathcal{U}_{x}\}$. Let $\mathcal{V}_{n}$ be the
            union of the $\mathcal{U}_{x}$ that contain $a_{n}$. This is a
            countable cover, and since $X$ is countably compact there is a
            finite subcover. From this we conclude that $N$ is finite, a
            contradiction. Thus, $\topspace{X}$ is limit point compact.
        \end{proof}
        \begin{theorem}
            If $\topspace{X}$ is a compact topological space, then it is limit
            point compact.
        \end{theorem}
        \begin{proof}
            For compact spaces are countably compact, and countably compact
            spaces are limit point compact.
        \end{proof}
        \begin{theorem}
            If $\topspace{X}$ is an accessible $(T_{1})$ limit point compact
            topological space, then it is countably compact.
        \end{theorem}
        The fact that limit point compact is very weak in a general topological
        space can be seen by studying the extreme value. If
        $\topspace[X]{X}$ is compact and $\topspace[Y]{Y}$ is an order topology,
        then for any continuous function $f:X\rightarrow{Y}$ the range must be
        bounded. This still holds for countably compact sets, but fails in the
        case of limit point compact.
        \begin{example}
            Let $\topspace[\empty]{\mathbb{Z}_{2}}$ be the trivial topological
            space on 2 points, and let $\mathcal[\powset]{\mathbb{Z}}$ be the
            discrete topology. The product topology on
            $\mathbb{Z}_{2}\times\mathbb{Z}$ is then limit point compact and the
            projection mapping
            $\pi_{2}:\mathbb{Z}_{2}\times\mathbb{Z}\rightarrow\mathbb{Z}$ is
            continuous. But the discrete topology on $\mathbb{Z}$ is the same as
            it's order topology, but $\pi_{Z}$ is not bounded. Indeed, the range
            of $\pi_{Z}$ is the entirety of $\mathbb{Z}$, and hence the range
            isn't even limit point compact.
        \end{example}
        If we consider accessible spaces ($T_{1}$), all of our problems
        disappear. The product topology of a trivial space (with at least two
        points) with a discrete space will not be accessible, and therein lies
        the issue.
        \par\hfill\par
        Compact subset of Hausdorff is closed, countable finite intersection
        property, 
\end{document}