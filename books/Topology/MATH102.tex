%------------------------------------------------------------------------------%
\documentclass[oneside]{book}                                                  %
%------------------------------Preamble----------------------------------------%
\makeatletter                                                                  %
    \def\input@path{{../../}}                                                  %
\makeatother                                                                   %
%---------------------------Packages----------------------------%
\usepackage{geometry}
\geometry{b5paper, margin=1.0in}
\usepackage[T1]{fontenc}
\usepackage{graphicx, float}            % Graphics/Images.
\usepackage{natbib}                     % For bibliographies.
\bibliographystyle{agsm}                % Bibliography style.
\usepackage[french, english]{babel}     % Language typesetting.
\usepackage[dvipsnames]{xcolor}         % Color names.
\usepackage{listings}                   % Verbatim-Like Tools.
\usepackage{mathtools, esint, mathrsfs} % amsmath and integrals.
\usepackage{amsthm, amsfonts, amssymb}  % Fonts and theorems.
\usepackage{tcolorbox}                  % Frames around theorems.
\usepackage{upgreek}                    % Non-Italic Greek.
\usepackage{fmtcount, etoolbox}         % For the \book{} command.
\usepackage[newparttoc]{titlesec}       % Formatting chapter, etc.
\usepackage{titletoc}                   % Allows \book in toc.
\usepackage[nottoc]{tocbibind}          % Bibliography in toc.
\usepackage[titles]{tocloft}            % ToC formatting.
\usepackage{pgfplots, tikz}             % Drawing/graphing tools.
\usepackage{imakeidx}                   % Used for index.
\usetikzlibrary{
    calc,                   % Calculating right angles and more.
    angles,                 % Drawing angles within triangles.
    arrows.meta,            % Latex and Stealth arrows.
    quotes,                 % Adding labels to angles.
    positioning,            % Relative positioning of nodes.
    decorations.markings,   % Adding arrows in the middle of a line.
    patterns,
    arrows
}                                       % Libraries for tikz.
\pgfplotsset{compat=1.9}                % Version of pgfplots.
\usepackage[font=scriptsize,
            labelformat=simple,
            labelsep=colon]{subcaption} % Subfigure captions.
\usepackage[font={scriptsize},
            hypcap=true,
            labelsep=colon]{caption}    % Figure captions.
\usepackage[pdftex,
            pdfauthor={Ryan Maguire},
            pdftitle={Mathematics and Physics},
            pdfsubject={Mathematics, Physics, Science},
            pdfkeywords={Mathematics, Physics, Computer Science, Biology},
            pdfproducer={LaTeX},
            pdfcreator={pdflatex}]{hyperref}
\hypersetup{
    colorlinks=true,
    linkcolor=blue,
    filecolor=magenta,
    urlcolor=Cerulean,
    citecolor=SkyBlue
}                           % Colors for hyperref.
\usepackage[toc,acronym,nogroupskip,nopostdot]{glossaries}
\usepackage{glossary-mcols}
%------------------------Theorem Styles-------------------------%
\theoremstyle{plain}
\newtheorem{theorem}{Theorem}[section]

% Define theorem style for default spacing and normal font.
\newtheoremstyle{normal}
    {\topsep}               % Amount of space above the theorem.
    {\topsep}               % Amount of space below the theorem.
    {}                      % Font used for body of theorem.
    {}                      % Measure of space to indent.
    {\bfseries}             % Font of the header of the theorem.
    {}                      % Punctuation between head and body.
    {.5em}                  % Space after theorem head.
    {}

% Italic header environment.
\newtheoremstyle{thmit}{\topsep}{\topsep}{}{}{\itshape}{}{0.5em}{}

% Define environments with italic headers.
\theoremstyle{thmit}
\newtheorem*{solution}{Solution}

% Define default environments.
\theoremstyle{normal}
\newtheorem{example}{Example}[section]
\newtheorem{definition}{Definition}[section]
\newtheorem{problem}{Problem}[section]

% Define framed environment.
\tcbuselibrary{most}
\newtcbtheorem[use counter*=theorem]{ftheorem}{Theorem}{%
    before=\par\vspace{2ex},
    boxsep=0.5\topsep,
    after=\par\vspace{2ex},
    colback=green!5,
    colframe=green!35!black,
    fonttitle=\bfseries\upshape%
}{thm}

\newtcbtheorem[auto counter, number within=section]{faxiom}{Axiom}{%
    before=\par\vspace{2ex},
    boxsep=0.5\topsep,
    after=\par\vspace{2ex},
    colback=Apricot!5,
    colframe=Apricot!35!black,
    fonttitle=\bfseries\upshape%
}{ax}

\newtcbtheorem[use counter*=definition]{fdefinition}{Definition}{%
    before=\par\vspace{2ex},
    boxsep=0.5\topsep,
    after=\par\vspace{2ex},
    colback=blue!5!white,
    colframe=blue!75!black,
    fonttitle=\bfseries\upshape%
}{def}

\newtcbtheorem[use counter*=example]{fexample}{Example}{%
    before=\par\vspace{2ex},
    boxsep=0.5\topsep,
    after=\par\vspace{2ex},
    colback=red!5!white,
    colframe=red!75!black,
    fonttitle=\bfseries\upshape%
}{ex}

\newtcbtheorem[auto counter, number within=section]{fnotation}{Notation}{%
    before=\par\vspace{2ex},
    boxsep=0.5\topsep,
    after=\par\vspace{2ex},
    colback=SeaGreen!5!white,
    colframe=SeaGreen!75!black,
    fonttitle=\bfseries\upshape%
}{not}

\newtcbtheorem[use counter*=remark]{fremark}{Remark}{%
    fonttitle=\bfseries\upshape,
    colback=Goldenrod!5!white,
    colframe=Goldenrod!75!black}{ex}

\newenvironment{bproof}{\textit{Proof.}}{\hfill$\square$}
\tcolorboxenvironment{bproof}{%
    blanker,
    breakable,
    left=3mm,
    before skip=5pt,
    after skip=10pt,
    borderline west={0.6mm}{0pt}{green!80!black}
}

\AtEndEnvironment{lexample}{$\hfill\textcolor{red}{\blacksquare}$}
\newtcbtheorem[use counter*=example]{lexample}{Example}{%
    empty,
    title={Example~\theexample},
    boxed title style={%
        empty,
        size=minimal,
        toprule=2pt,
        top=0.5\topsep,
    },
    coltitle=red,
    fonttitle=\bfseries,
    parbox=false,
    boxsep=0pt,
    before=\par\vspace{2ex},
    left=0pt,
    right=0pt,
    top=3ex,
    bottom=1ex,
    before=\par\vspace{2ex},
    after=\par\vspace{2ex},
    breakable,
    pad at break*=0mm,
    vfill before first,
    overlay unbroken={%
        \draw[red, line width=2pt]
            ([yshift=-1.2ex]title.south-|frame.west) to
            ([yshift=-1.2ex]title.south-|frame.east);
        },
    overlay first={%
        \draw[red, line width=2pt]
            ([yshift=-1.2ex]title.south-|frame.west) to
            ([yshift=-1.2ex]title.south-|frame.east);
    },
}{ex}

\AtEndEnvironment{ldefinition}{$\hfill\textcolor{Blue}{\blacksquare}$}
\newtcbtheorem[use counter*=definition]{ldefinition}{Definition}{%
    empty,
    title={Definition~\thedefinition:~{#1}},
    boxed title style={%
        empty,
        size=minimal,
        toprule=2pt,
        top=0.5\topsep,
    },
    coltitle=Blue,
    fonttitle=\bfseries,
    parbox=false,
    boxsep=0pt,
    before=\par\vspace{2ex},
    left=0pt,
    right=0pt,
    top=3ex,
    bottom=0pt,
    before=\par\vspace{2ex},
    after=\par\vspace{1ex},
    breakable,
    pad at break*=0mm,
    vfill before first,
    overlay unbroken={%
        \draw[Blue, line width=2pt]
            ([yshift=-1.2ex]title.south-|frame.west) to
            ([yshift=-1.2ex]title.south-|frame.east);
        },
    overlay first={%
        \draw[Blue, line width=2pt]
            ([yshift=-1.2ex]title.south-|frame.west) to
            ([yshift=-1.2ex]title.south-|frame.east);
    },
}{def}

\AtEndEnvironment{ltheorem}{$\hfill\textcolor{Green}{\blacksquare}$}
\newtcbtheorem[use counter*=theorem]{ltheorem}{Theorem}{%
    empty,
    title={Theorem~\thetheorem:~{#1}},
    boxed title style={%
        empty,
        size=minimal,
        toprule=2pt,
        top=0.5\topsep,
    },
    coltitle=Green,
    fonttitle=\bfseries,
    parbox=false,
    boxsep=0pt,
    before=\par\vspace{2ex},
    left=0pt,
    right=0pt,
    top=3ex,
    bottom=-1.5ex,
    breakable,
    pad at break*=0mm,
    vfill before first,
    overlay unbroken={%
        \draw[Green, line width=2pt]
            ([yshift=-1.2ex]title.south-|frame.west) to
            ([yshift=-1.2ex]title.south-|frame.east);},
    overlay first={%
        \draw[Green, line width=2pt]
            ([yshift=-1.2ex]title.south-|frame.west) to
            ([yshift=-1.2ex]title.south-|frame.east);
    }
}{thm}

%--------------------Declared Math Operators--------------------%
\DeclareMathOperator{\adjoint}{adj}         % Adjoint.
\DeclareMathOperator{\Card}{Card}           % Cardinality.
\DeclareMathOperator{\curl}{curl}           % Curl.
\DeclareMathOperator{\diam}{diam}           % Diameter.
\DeclareMathOperator{\dist}{dist}           % Distance.
\DeclareMathOperator{\Div}{div}             % Divergence.
\DeclareMathOperator{\Erf}{Erf}             % Error Function.
\DeclareMathOperator{\Erfc}{Erfc}           % Complementary Error Function.
\DeclareMathOperator{\Ext}{Ext}             % Exterior.
\DeclareMathOperator{\GCD}{GCD}             % Greatest common denominator.
\DeclareMathOperator{\grad}{grad}           % Gradient
\DeclareMathOperator{\Ima}{Im}              % Image.
\DeclareMathOperator{\Int}{Int}             % Interior.
\DeclareMathOperator{\LC}{LC}               % Leading coefficient.
\DeclareMathOperator{\LCM}{LCM}             % Least common multiple.
\DeclareMathOperator{\LM}{LM}               % Leading monomial.
\DeclareMathOperator{\LT}{LT}               % Leading term.
\DeclareMathOperator{\Mod}{mod}             % Modulus.
\DeclareMathOperator{\Mon}{Mon}             % Monomial.
\DeclareMathOperator{\multideg}{mutlideg}   % Multi-Degree (Graphs).
\DeclareMathOperator{\nul}{nul}             % Null space of operator.
\DeclareMathOperator{\Ord}{Ord}             % Ordinal of ordered set.
\DeclareMathOperator{\Prin}{Prin}           % Principal value.
\DeclareMathOperator{\proj}{proj}           % Projection.
\DeclareMathOperator{\Refl}{Refl}           % Reflection operator.
\DeclareMathOperator{\rk}{rk}               % Rank of operator.
\DeclareMathOperator{\sgn}{sgn}             % Sign of a number.
\DeclareMathOperator{\sinc}{sinc}           % Sinc function.
\DeclareMathOperator{\Span}{Span}           % Span of a set.
\DeclareMathOperator{\Spec}{Spec}           % Spectrum.
\DeclareMathOperator{\supp}{supp}           % Support
\DeclareMathOperator{\Tr}{Tr}               % Trace of matrix.
%--------------------Declared Math Symbols--------------------%
\DeclareMathSymbol{\minus}{\mathbin}{AMSa}{"39} % Unary minus sign.
%------------------------New Commands---------------------------%
\DeclarePairedDelimiter\norm{\lVert}{\rVert}
\DeclarePairedDelimiter\ceil{\lceil}{\rceil}
\DeclarePairedDelimiter\floor{\lfloor}{\rfloor}
\newcommand*\diff{\mathop{}\!\mathrm{d}}
\newcommand*\Diff[1]{\mathop{}\!\mathrm{d^#1}}
\renewcommand*{\glstextformat}[1]{\textcolor{RoyalBlue}{#1}}
\renewcommand{\glsnamefont}[1]{\textbf{#1}}
\renewcommand\labelitemii{$\circ$}
\renewcommand\thesubfigure{%
    \arabic{chapter}.\arabic{figure}.\arabic{subfigure}}
\addto\captionsenglish{\renewcommand{\figurename}{Fig.}}
\numberwithin{equation}{section}

\renewcommand{\vector}[1]{\boldsymbol{\mathrm{#1}}}

\newcommand{\uvector}[1]{\boldsymbol{\hat{\mathrm{#1}}}}
\newcommand{\topspace}[2][]{(#2,\tau_{#1})}
\newcommand{\measurespace}[2][]{(#2,\varSigma_{#1},\mu_{#1})}
\newcommand{\measurablespace}[2][]{(#2,\varSigma_{#1})}
\newcommand{\manifold}[2][]{(#2,\tau_{#1},\mathcal{A}_{#1})}
\newcommand{\tanspace}[2]{T_{#1}{#2}}
\newcommand{\cotanspace}[2]{T_{#1}^{*}{#2}}
\newcommand{\Ckspace}[3][\mathbb{R}]{C^{#2}(#3,#1)}
\newcommand{\funcspace}[2][\mathbb{R}]{\mathcal{F}(#2,#1)}
\newcommand{\smoothvecf}[1]{\mathfrak{X}(#1)}
\newcommand{\smoothonef}[1]{\mathfrak{X}^{*}(#1)}
\newcommand{\bracket}[2]{[#1,#2]}

%------------------------Book Command---------------------------%
\makeatletter
\renewcommand\@pnumwidth{1cm}
\newcounter{book}
\renewcommand\thebook{\@Roman\c@book}
\newcommand\book{%
    \if@openright
        \cleardoublepage
    \else
        \clearpage
    \fi
    \thispagestyle{plain}%
    \if@twocolumn
        \onecolumn
        \@tempswatrue
    \else
        \@tempswafalse
    \fi
    \null\vfil
    \secdef\@book\@sbook
}
\def\@book[#1]#2{%
    \refstepcounter{book}
    \addcontentsline{toc}{book}{\bookname\ \thebook:\hspace{1em}#1}
    \markboth{}{}
    {\centering
     \interlinepenalty\@M
     \normalfont
     \huge\bfseries\bookname\nobreakspace\thebook
     \par
     \vskip 20\p@
     \Huge\bfseries#2\par}%
    \@endbook}
\def\@sbook#1{%
    {\centering
     \interlinepenalty \@M
     \normalfont
     \Huge\bfseries#1\par}%
    \@endbook}
\def\@endbook{
    \vfil\newpage
        \if@twoside
            \if@openright
                \null
                \thispagestyle{empty}%
                \newpage
            \fi
        \fi
        \if@tempswa
            \twocolumn
        \fi
}
\newcommand*\l@book[2]{%
    \ifnum\c@tocdepth >-3\relax
        \addpenalty{-\@highpenalty}%
        \addvspace{2.25em\@plus\p@}%
        \setlength\@tempdima{3em}%
        \begingroup
            \parindent\z@\rightskip\@pnumwidth
            \parfillskip -\@pnumwidth
            {
                \leavevmode
                \Large\bfseries#1\hfill\hb@xt@\@pnumwidth{\hss#2}
            }
            \par
            \nobreak
            \global\@nobreaktrue
            \everypar{\global\@nobreakfalse\everypar{}}%
        \endgroup
    \fi}
\newcommand\bookname{Book}
\renewcommand{\thebook}{\texorpdfstring{\Numberstring{book}}{book}}
\providecommand*{\toclevel@book}{-2}
\makeatother
\titleformat{\part}[display]
    {\Large\bfseries}
    {\partname\nobreakspace\thepart}
    {0mm}
    {\Huge\bfseries}
\titlecontents{part}[0pt]
    {\large\bfseries}
    {\partname\ \thecontentslabel: \quad}
    {}
    {\hfill\contentspage}
\titlecontents{chapter}[0pt]
    {\bfseries}
    {\chaptername\ \thecontentslabel:\quad}
    {}
    {\hfill\contentspage}
\newglossarystyle{longpara}{%
    \setglossarystyle{long}%
    \renewenvironment{theglossary}{%
        \begin{longtable}[l]{{p{0.25\hsize}p{0.65\hsize}}}
    }{\end{longtable}}%
    \renewcommand{\glossentry}[2]{%
        \glstarget{##1}{\glossentryname{##1}}%
        &\glossentrydesc{##1}{~##2.}
        \tabularnewline%
        \tabularnewline
    }%
}
\newglossary[not-glg]{notation}{not-gls}{not-glo}{Notation}
\newcommand*{\newnotation}[4][]{%
    \newglossaryentry{#2}{type=notation, name={\textbf{#3}, },
                          text={#4}, description={#4},#1}%
}
%--------------------------LENGTHS------------------------------%
% Spacings for the Table of Contents.
\addtolength{\cftsecnumwidth}{1ex}
\addtolength{\cftsubsecindent}{1ex}
\addtolength{\cftsubsecnumwidth}{1ex}
\addtolength{\cftfignumwidth}{1ex}
\addtolength{\cfttabnumwidth}{1ex}

% Indent and paragraph spacing.
\setlength{\parindent}{0em}
\setlength{\parskip}{0em}                                                           %
%----------------------------Main Document-------------------------------------%
\begin{document}
    \pagenumbering{roman}
    \title{MATH 102 Geometry Notes}
    \author{%
        Professor: Carolyn Gordon\\
        Notes by: Ryan Maguire%
    }
    \date{\vspace{-5ex}}
    \maketitle
    \tableofcontents
    \listoffigures
    \chapter{Lie Groups and Lie Algebras}
    \pagenumbering{arabic}
        \section{Lecture 1: Review}
            \subsection{Coordinate Charts in a Manifold}
                A coordinate chart in a manifold $M$ is an open subset
                $\mathcal{U}\subseteq{M}$ together with a homomorphism
                $\varphi:\mathcal{U}\rightarrow\mathcal{V}$ where $\mathcal{V}$
                is an open subset of $\nspace$ for some $n\in\mathbb{N}$.
                Equivalently, $\varphi:\mathcal{U}\rightarrow\nspace$ is a
                continuous injective open mapping
                (Fig.~\ref{fig:Coordinate_Chart}).
                \begin{figure}[H]
                    \centering
                    \captionsetup{type=figure}
                    \includegraphics{images/Chart_in_a_Manifold.pdf}
                    \caption{A Coordinate Chart in a Manifold}
                    \label{fig:Coordinate_Chart}
                \end{figure}
                \begin{example}
                    We can form a coordinate chart on the sphere by mapping the
                    upper hemisphere down to the plane. That is, if we consider
                    points $(x,y,z)\in\nsphere[2]$ such that $z>0$, we can take
                    this open set $\mathcal{U}^{+}$ and map it to $\nspace[2]$
                    by $\varphi(x,y,z)=(x,y)$. This is a continuous injective
                    open mapping, satisfying the criterion for a coordinate
                    chart. This type of chart is called an \textit{orthographic}
                    projection and is shown in
                    Fig.~\ref{fig:Sphere_Orthographic_Projection}.
                \end{example}
                \begin{figure}
                    \centering
                    \captionsetup{type=figure}
                    \resizebox{!}{0.8\height}{
                    \includegraphics{images/Sphere_Orthographic_Projection.pdf}
                    }
                    \caption{Orthographic Projection of a Sphere}
                    \label{fig:Sphere_Orthographic_Projection}
                \end{figure}
                The orthographic projection can be defined by taking an observer
                at the north pole and bringing them off to infinity in the
                \textit{up} direction. The orthographic projection is then what
                they would see if they had an amazing telescope. If we stop at
                some finite distance we wouldn't get the whole northern
                hemisphere (if you climb a small latter at the north pole you
                would not be able to see the equator). These
                \textit{near-sided} projections give us different coordinate
                charts. Suppose our observer is at Geosynchronous orbit which is
                fairly far away. The resulting coordinate chart is shown in
                Fig.~\subref{fig:Near_Sided_Proj}. Further still we can imagine
                removing what the observer can see, hollowing out the Earth, and
                then projecting what is left onto
                the plane. This results in the \textit{far-sided} projection
                and is shown in Fig.~\subref{fig:Far_Sided_Proj}.
                \begin{figure}[H]
                    \centering
                    \captionsetup{type=figure}
                    \begin{subfigure}[b]{0.49\textwidth}
                        \centering
                        \captionsetup{type=figure}
                        \resizebox{!}{0.8\height}{
                            \includegraphics{%
                                images/Sphere_GEO_Near_Sided_Projection.pdf%
                            }
                        }
                        \subcaption{Near-Sided Projection}
                        \label{fig:Near_Sided_Proj}
                    \end{subfigure}
                    \begin{subfigure}[b]{0.49\textwidth}
                        \centering
                        \captionsetup{type=figure}
                        \resizebox{!}{0.8\height}{
                            \includegraphics{%
                                images/Sphere_GEO_Far_Sided_Projection.pdf%
                            }
                        }
                        \subcaption{Far-Sided Projection}
                        \label{fig:Far_Sided_Proj}
                    \end{subfigure}
                    \caption{Different Projections of a Sphere}
                \end{figure}
                All of these are valid coordinate charts on the sphere
                $\nsphere[2]$.
            \subsection{Tangent Spaces}
                There are three ways we can think of the tangent space
                $\tanspace{\vector{x}}{\nspace}$ given a point
                $\vector{x}\in\nspace$. Firstly we can think of the classical
                calculus manner of tangent spaces, which is just a collection of
                vectors starting at $\vector{x}$. We can also use derivations.
                If we let $\Ckspace{\infty}{\nspace}$ denote the set of all
                smooth functions $f:\nspace\rightarrow\nspace[]$, then a
                \textit{derivation} is a function
                $D:\Ckspace{\infty}{\nspace}\rightarrow\nspace[]$ that is linear
                and Liebnizian. That is, for all $a,b\in\nspace[]$ and
                $f,g\in\Ckspace{\infty}{\nspace}$ the following holds:
                \begin{subequations}
                    \begin{align}
                        D(af+bg)&=aD(f)+bD(g)\\
                        D(fg)&=D(f)g(p)+f(p)D(g)
                    \end{align}
                \end{subequations}
                There is a basis for the space of derivations consisting of the
                partial derivatives at $\vector{x}$. Lastly, we can think of
                equivalence classes of smooth curves passing through the point
                $\vector{x}$. Using $\nspace[2]$ as an example, the three curves
                in figure Fig.~\ref{fig:Tan_Vec_as_Curves} passing through the
                point $p\in\nspace[2]$ belong to the same equivalence classe
                since they have the same tangent vector at $p$.
                \begin{figure}[H]
                    \centering
                    \captionsetup{type=figure}
                    \includegraphics{images/Tangent_Vector_as_Equiv_Curves.pdf}
                    \caption{Tangent Vectors as Curves}
                    \label{fig:Tan_Vec_as_Curves}
                \end{figure}
                There's a relationship between these three modes of thought.
                Given a vector $\vector{v}$ at $\vector{x}$ we can form a
                derivation $D$ by the equation:
                \begin{equation}
                    D_{\vector{v}}(f)=
                        \frac{\diff}{\diff{t}}\Big|_{t=0}
                        \big(f(\vector{x}+t\vector{v})\big)
                \end{equation}
                That is, we take $D$ to be the \textit{directional derivative}.
                A curve also gives rise to a derivation since a curve is simply
                a function $\gamma:(0,1)\rightarrow\nspace$. Given
                $f\in\Ckspace{\infty}{\nspace}$ we can compose
                $f\circ\gamma:(0,1)\rightarrow\nspace[]$, and on this we can
                perform ordinary calculus. We define the derivation by:
                \begin{equation}
                    D_{\gamma}(f)=\frac{\diff}{\diff{t}}\Big|_{t=t_{0}}
                        \big(f\circ\gamma\big)
                \end{equation}
                where $t_{0}$ is the time when $\gamma$ passes through $p$. This
                derivation corresponds to the equivalence class of curves with
                the same slope as the curve $\Gamma(t)=\vector{x}+t\vector{v}$.
                \par\hfill\par
                In the manifold setting we may lose the vector space structure
                offered in $\nspace$ and hence may not be able to use the first
                manner of tangent vectors, but the latter two are still well
                defined. Given a manifold $M$ and a point $p\in{M}$ one can
                obtain a basis for $\tanspace{p}{M}$ by taking a coordinate chart
                $(\mathcal{U},\varphi)$ with $p\in\mathcal{U}$ and defining the
                \textit{partials} at $p$ with respect to this chart. If we have
                a smooth real value function $f\in\Ckspace{\infty}{M}$ we can
                compose this with $\varphi^{\minus{1}}$ which gives us a
                function from an open subset of $\nspace$ to the real line 
                (Fig.~\ref{fig:Partials_on_Manifold}).
                \begin{figure}[H]
                    \centering
                    \captionsetup{type=figure}
                    \begin{tikzpicture}[>=Latex]
    \draw[<->] (-1,  0) to (3, 0);
    \draw[<->] ( 0, -1) to (0, 3);
    \draw[dashed, fill=red!70!white]
    (0.5, 1.5) to[out=30,  in=180]  (1.2,  1.9)
               to[out=0,   in=90]   (1.9,  1.4)
               to[out=-90, in=0]    (1.4,  0.2)
               to[out=180, in=-30]  (0.4,  0.3)
               to[out=150, in=-150] cycle;

    \node at (0.5, 3.0) {$\mathbb{R}^{n}$};
    \node at (1.1, 1.0) {$\varphi^{\minus{1}}[\mathcal{U}]$};
    \draw[<-] (1.3,2) to[out=90, in=180] node[above left] {$\varphi$} (4.7,4.7);

    \begin{scope}[xshift=3cm,yshift=5cm]
        \draw (0,0) to[out=90,  in=180] (1,  1.0)
                    to[out=0,   in=150] (2,  1.0)
                    to[out=-30, in=90]  (4,  0.0)
                    to[out=-90, in=0]   (2, -2.5)
                    to[out=180, in=-30] (0, -2.5)
                    to[out=150, in=-90] cycle;
        
        \draw (0.5, -1.7) to[in=-130, out=-50] (1.8, -1.7);
        \draw (0.6, -1.8) to[in=130,  out=50]  (1.7, -1.8);

        \draw[dashed, fill=cyan]
            (2.0, 0.0) to[out=30,  in=180]  (2.7,  0.5)
                       to[out=0,   in=90]   (3.5,  0.0)
                       to[out=-90, in=0]    (3.2, -0.9)
                       to[out=180, in=-30]  (2.2, -0.8)
                       to[out=150, in=-150] cycle;

        \node at (2.8, -0.2) {$\mathcal{U}$};
        \node at (1.0,  0.6) {$X$};
    \end{scope}
    \draw[<->] (6, 0) to (10, 0) node[above] {$\mathbb{R}$};
    \draw[->]  (6.7, 4.7) to[out=0, in=90] node[right] {$f$} (8, 0.2);
    \draw[blue, thick] (9, 0) arc (0:15:0.5);
    \draw[blue, thick] (9, 0) arc (0:-15:0.5);
    \draw[blue, thick] (7, 0) arc (180:195:0.5);
    \draw[blue, thick] (7, 0) arc (180:165:0.5);
    \draw[blue, thick] (7,0) to (9, 0);
    \draw[->] (1.2, -0.2) to[out=-30, in=-150]
        node[above] {$f\circ\varphi^{\minus{1}}$} (7.8, -0.2);
\end{tikzpicture}
                    \caption{Smooth Real-Valued Function on a Manifold}
                    \label{fig:Partials_on_Manifold}
                \end{figure}
                We define the partial derivative with respect to the
                $i^{th}$ coordinate as follows:
                \begin{equation}
                    \frac{\partial}{\partial{\varphi}_{i}}\Big|_{p}f
                    =\frac{\partial}{\partial{x}_{i}}\Big|_{\varphi(p)}
                        \big(f\circ\varphi^{\minus{1}}\big)
                \end{equation}
                where $x_{i}=\pi_{i}\circ\varphi$ with $\pi_{i}$ being the
                $i^{th}$ projection mapping.
            \subsection{Differentiation in Euclidean Space}
                Let $f:\nspace[m]\rightarrow\nspace$ be a function. We say that
                $f$ is differentiable at a point $\vector{x}\in\nspace[m]$ if
                there is a linear map $T:\nspace[m]\rightarrow\nspace$ such
                that:
                \begin{equation}
                    \underset{\vector{h}\rightarrow\vector{0}}{\lim}
                    \frac{\norm{f(\vector{x}+\vector{h})-
                        f(\vector{x})-T\vector{h}}_{2}^{n}}
                        {\norm{\vector{h}}_{2}^{m}}=0
                \end{equation}
                where $\norm{\cdot}_{2}^{k}$ denotes the \textit{Euclidean norm}
                in $\nspace[k]$. We denote $T$ by $f_{*_{\vector{x}}}$.
                \begin{theorem}
                    If $f:\nspace[m]\rightarrow\nspace$ is a linear function,
                    and if $\vector{x}\in\nspace[m]$, then $f$ is differentiable
                    at $\vector{x}$ and
                    $f_{*_{\vector{x}}}(\vector{u})=f(\vector{u})$
                \end{theorem}
                \begin{proof}
                    For if $f$ is linear, let $T:\nspace[m]\rightarrow\nspace$
                    be the linear operator $T\vector{h}=f(\vector{h})$. Then:
                    \begin{subequations}
                        \begin{align}
                            \underset{\vector{h}\rightarrow\vector{0}}{\lim}
                            \frac{%
                                \norm{%
                                    f(\vector{x}+\vector{h})-
                                    f(\vector{x})-
                                    T\vector{h}%
                                }_{2}^{n}%
                            }%
                            {%
                                \norm{\vector{h}}_{2}^{m}%
                            }
                            &=\underset{\vector{h}\rightarrow\vector{0}}{\lim}
                            \frac{%
                                \norm{%
                                    f(\vector{x})+
                                    f(\vector{h})-
                                    f(\vector{x})-
                                    T\vector{h}%
                                }_{2}^{n}%
                            }%
                            {%
                                \norm{\vector{h}}_{2}^{m}%
                            }\\
                            &=\underset{\vector{h}\rightarrow\vector{0}}{\lim}
                            \frac{%
                                \norm{%
                                    f(\vector{h})-
                                    T\vector{h}%
                                }_{2}^{n}%
                            }%
                            {%
                                \norm{\vector{h}}_{2}^{m}%
                            }\\
                            &=\underset{\vector{h}\rightarrow\vector{0}}{\lim}
                            \frac{\norm{\vector{0}}_{2}^{n}}
                                 {\norm{\vector{x}}_{2}^{m}}\\
                            &=0
                        \end{align}
                    \end{subequations}
                    and hence $f$ is differetiable. By definition
                    $f_{*_{\vector{x}}}(\vector{u})=T\vector{u}$ and
                    $T\vector{u}=f(\vector{u})$.
                \end{proof}
                If we consider the curve $\gamma(t)=\vector{x}+t\vector{v}$,
                then we see:
                \begin{equation}
                    f_{*_{\vector{x}}}(\vector{v})=D_{\gamma}f\big|_{\vector{x}}
                \end{equation}
                If $\vector{e}_{i}$ denotes the standard $i^{th}$ basis vector,
                then we have:
                \begin{equation}
                    f_{*}(\vector{e}_{i})=\frac{\partial}{\partial{x}_{i}}
                \end{equation}
                The matrix corresponding to $f_{*}$ is given by:
                \begingroup
                    \renewcommand*{\arraystretch}{1.5}
                    \begin{equation}
                        J=
                        \begin{bmatrix}
                            \frac{\partial{f}_{1}}{\partial{x}_{1}}&
                            \frac{\partial{f}_{1}}{\partial{x}_{2}}
                            &\dots&
                            \frac{\partial{f}_{1}}{\partial{x}_{n}}\\
                            \frac{\partial{f}_{2}}{\partial{x}_{1}}&
                            \frac{\partial{f}_{2}}{\partial{x}_{2}}
                            &\dots&
                            \frac{\partial{f}_{2}}{\partial{x}_{n}}\\
                            \vdots&\vdots&\ddots&\vdots&\\
                            \frac{\partial{f}_{m}}{\partial{x}_{1}}&
                            \frac{\partial{f}_{m}}{\partial{x}_{2}}
                            &\dots&
                            \frac{\partial{f}_{m}}{\partial{x}_{n}}
                        \end{bmatrix}
                    \end{equation}
                \endgroup
                \begin{example}
                    Let $m,n,p\in\mathbb{N}^{+}$ be positive integers and
                    $B:\nspace[m]\times\nspace\rightarrow\nspace[p]$ a bilinear
                    function. Given
                    $(\vector{u},\vector{v})\in%
                    \tanspace{(\vector{a},\vector{b})}{\nspace[m]\times\nspace}$
                    we can compute the derivative of $B$ as follows:
                    \begin{equation}
                        D_{(\vector{u},\vector{v})}B(\vector{a},\vector{b})=
                        \frac{\diff}{\diff{t}}\Big|_{t_{0}}
                        B(\vector{a}+t\vector{u},\vector{b}+t\vector{v})
                    \end{equation}
                    But since $B$ is bilinear we can factor the first variable
                    as follows:
                    \begin{equation}
                        D_{(\vector{u},\vector{v})}B(\vector{a},\vector{b})=
                        \frac{\diff}{\diff{t}}\Big|_{t_{0}}\Big(
                            B(\vector{a},\vector{b}+t\vector{v})+
                            tB(\vector{u},\vector{b}+t\vector{v})
                        \Big)
                    \end{equation}
                    But again from bilinearity we can simplify the second
                    slot, obtaining:
                    \begin{subequations}
                        \begin{align}
                            D_{(\vector{u},\vector{v})}B(\vector{a},\vector{b})
                            &=\frac{\diff}{\diff{t}}\Big|_{t_{0}}\Big(
                                B(\vector{a},\vector{b})+
                                tB(\vector{a},\vector{v})+
                                tB(\vector{u},\vector{b})+
                                t^{2}B(\vector{u},\vector{v})
                            \Big)\\
                            &=B(\vector{a},\vector{v})+B(\vector{u},\vector{b})
                                +2t_{0}B(\vector{u},\vector{v})
                        \end{align}
                    \end{subequations}
                    For this problem we have $t_{0}=0$, and so we arrive at the
                    result:
                    \begin{equation}
                        B_{*_{(\vector{a},\vector{b})}}(\vector{u},\vector{v})
                        =B(\vector{a},\vector{v})+B(\vector{u},\vector{b})
                    \end{equation}
                \end{example}
                \begin{ftheorem}{The Chain Rule}{Chain_Rule}
                    If $f:\nspace[m]\rightarrow\nspace$ is differentiable at a
                    point $\vector{x}\in\nspace[m]$, and if
                    $g:\nspace\rightarrow\nspace[p]$ is differentiable at
                    $f(\vector{x})\in\nspace$, then $g\circ{f}$ is
                    differentiable at $\vector{x}$ and:
                    \begin{equation*}
                        (g\circ{f})_{*_{p}}=g_{*_{f(p)}}\circ{f}_{*_{p}}
                    \end{equation*}
                \end{ftheorem}
                Our first use of this is with the Liebniz rule.
                \begin{ltheorem}{Liebniz Rule}{Liebniz_Rule}
                    If $B:\nspace[m]\times\nspace\rightarrow\nspace[p]$ is a
                    bilinear function, if
                    $F_{1}:\nspace[k]\rightarrow\nspace[m]$ and
                    $F_{2}:\nspace[k]\rightarrow\nspace[n]$ are differentiable
                    functions, and if $\vector{x}_{0}\in\nspace[k]$, then the
                    function $H:\nspace[k]\rightarrow\nspace[p]$ defined by:
                    \begin{equation*}
                        H(\vector{x})=
                            B\big(F_{1}(\vector{x}),F_{2}(\vector{x})\big)
                    \end{equation*}
                    then $H$ is differentiable at $\vector{x}_{0}$ and:
                    \begin{equation*}
                        H_{*_{p}}(\vector{u})=
                        B\big(%
                            F_{1_{*_{\vector{x}_{0}}}}(\vector{u}),\,%
                            F_{2}(\vector{u})\big)+
                        B\big(%
                            F_{1}(\vector{u}),\,%
                            F_{1_{*_{\vector{x}}}}(\vector{u})\big)
                    \end{equation*}
                \end{ltheorem}
                \begin{proof}
                    For let $G(x)=(F_{1}(x),F_{2}(x))$ and apply the chain rule
                    to $B\circ{G}$.
                \end{proof}
                \begin{example}
                    Let $\matspace{\nspace[]}$ denote the set of all
                    $n\times{n}$ matrices with entries in $\nspace[]$ and
                    $\GLnR{\nspace[]}$ the set of all invertible $n\times{n}$
                    matrices. That is, the set of all matrices with non-zero
                    determinant. Let
                    $F:\GLnR{\nspace[]}\rightarrow\GLnR{\nspace[]}$ be given by
                    $F(X)=X^{\minus{1}}$. Let's compute $F_{*_{A}}(U)$ for
                    $A\in\GLnR{\nspace[]}$ and
                    $U\in\tanspace{A}{\matspace{\nspace[]}}$. We'll use the fact
                    that $XX^{\minus{1}}=I$ is the identity matrix, and write
                    $G(X)=X$. Then $G(X)F(X)=I$ and hence by the Liebniz rule we
                    get:
                    \begin{equation}
                        G_{*_{A}}(U)F(A)+G(A)F_{*_{A}}(U)=0
                    \end{equation}
                    But $G$ is linear and hence $G_{*_{A}}(U)=G(U)=U$. Moreover,
                    $F(A)=A^{\minus{1}}$. Combining we get:
                    \begin{equation}
                        AF_{*_{A}}(U)=\minus{U}A^{\minus{1}}
                    \end{equation}
                    But $A\in\GLnR{\nspace[]}$ is invertible, and so we can
                    bring this to the other side obtaining:
                    \begin{equation}
                        F_{*_{A}}(U)=\minus{A}^{\minus{1}}UA^{\minus{1}}
                    \end{equation}
                \end{example}
                \begin{example}
                    Let $F:\matspace{\nspace[]}\rightarrow\matspace{\nspace[]}$
                    be defined by:
                    \begin{equation}
                        F(X)=AXA^{\minus{1}}
                    \end{equation}
                    where $A\in\GLnR{\nspace[]}$ is some invertible matrix.
                    Find $F_{*_{B}}$ given a matrix $B$. Since $F$ is linear we
                    know that $F_{*_{B}}(U)=F(U)$ for all $U$, and hence:
                    \begin{equation}
                        F_{*_{B}}(U)=AUA^{\minus{1}}
                    \end{equation}
                \end{example}
                \begin{example}
                    Let $H:\GLnR{\nspace[]}\rightarrow\GLnR{\nspace[]}$ be
                    defined by:
                    \begin{equation}
                        H(X)=XAX^{\minus{1}}
                    \end{equation}
                    with $A\in\matspace{\nspace[]}$. Compute $H_{*_{B}}$ for
                    $B\in\GLnR{\nspace[]}$. We can use the Liebniz rule if we
                    let $F(X)=XA$ and $G(X)=X^{\minus{1}}$. The derivative
                    becomes:
                    \begin{equation}
                        H_{*_{B}}(U)=F_{*_{B}}(U)G(B)+F(B)G_{*_{B}}(U)
                    \end{equation}
                    But $F$ is linear and hence $F_{*_{B}}(U)=F(U)$. Also we've
                    already computed the derivative of $X^{\minus{1}}$ so
                    $G_{*_{B}}(U)=\minus{B}^{\minus{1}}UB^{\minus{1}}$.
                    Combining, we get:
                    \begin{equation}
                        H_{*_{B}}=UAB^{\minus{1}}-BAB^{\minus{1}}UB^{\minus{1}}
                    \end{equation}
                \end{example}
                \begin{example}
                    Let $F:\matspace{\nspace[]}\rightarrow\matspace{\nspace[]}$
                    be defined by:
                    \begin{equation}
                        F(X)=X^{T}X
                    \end{equation}
                    where $X^{T}$ denotes the transpose of $X$. Compute
                    $F_{*_{A}}$. Let $G_{1}(X)=X^{T}$ and $G_{2}(X)=X$. By the
                    Liebniz property, we get:
                    \begin{equation}
                        F_{*_{A}}(U)=G_{1_{*_{A}}}(U)G_{2}(A)+
                            G_{1}(A)G_{2_{*_{A}}}(U)
                    \end{equation}
                    But the transpose is a linear operation, and hence
                    $G_{1_{*_{A}}}(U)=G_{1}(U)=U^{T}$. Similarly $G_{2}$ is
                    linear, and thus:
                    \begin{equation}
                        F_{*_{A}}(U)=U^{T}A+A^{T}U
                    \end{equation}
                \end{example}
                The function $F(X)=X^{T}X$ has it's image in the set of all
                symmetric matrices. If $Y\in{F}[\matspace{\nspace[]}]$, then
                $Y=F(X)$ for some $X\in\matspace{\nspace[]}$ and hence:
                \begin{equation}
                    Y^{T}=(X^{T}X)^{T}=X^{T}X=Y
                \end{equation}
                If $A\in\orthgroup{\nspace[]}$, the group of orthogonal
                matrices, then $F_{*_{A}}$ is surjective with respect to $P$,
                the set of symmetric matrices. For if $Y\in{P}$, then let
                $U=\frac{1}{2}AY$. Then:
                \begin{subequations}
                    \begin{minipage}[b]{0.56\textwidth}
                        \centering
                        \begin{align}
                            F_{*_{A}}(U)&=U^{T}A+A^{T}U\\
                                &=(\frac{1}{2}AY)^{T}A+A^{T}(\frac{1}{2}AY)\\
                                &=\frac{1}{2}\big(Y^{T}A^{T}A+A^{T}AY\big)
                        \end{align}
                    \end{minipage}
                    \hfill
                    \begin{minipage}[b]{0.43\textwidth}
                        \centering
                        \begin{align}
                            &=\frac{1}{2}\big(Y^{T}I+IY\big)\\
                            &=\frac{1}{2}(2Y)\\
                            &=Y
                        \end{align}
                    \end{minipage}
                \end{subequations}
            \subsection{Submanifolds}
                An immersed submanifold is given by an inective immersion
                $f:N\rightarrow{M}$. We write that $(N,f)$ is a submanifold of
                $M$ and usually identify $N$ with $f[N]$. If $f$ is an
                embedding, then we say $(N,f)$ is an embedded submanifold or a
                regular submanifold. The topology we endow $f[N]$ with is not
                necessarily the subspace topology, but rather the topology that
                makes $f$ a homeomorphism onto its image. An embedding is when
                this topology agrees with the subspace topology.
                \begin{example}
                    The first example of an immersed submanifold that is not
                    embedded comes from the function
                    $f:(\minus{1}.2,1)\rightarrow\nspace[2]$ defined by:
                    \begin{equation}
                        f(t)=\big(t^{2}-1,\,t^{3}-t\big)
                    \end{equation}
                    The graph of this parametric equation is shown in
                    Fig.~\ref{fig:Immersed_Submanifold_001}. This is not an
                    embedded submanifold since the subspace topology does not
                    give rise to a manifold. The origin locally looks like the
                    letter \textit{T}, which is not possible for topological
                    manifolds.
                \end{example}
                \begin{figure}[H]
                    \centering
                    \captionsetup{type=figure}
                    \includegraphics{images/Immersed_Not_Embedded_003.pdf}
                    \caption{An Immersed Submanifold}
                    \label{fig:Immersed_Submanifold_001}
                \end{figure}
                \begin{example}
                    A \textit{lemniscate} gives rise to another immersed
                    manifold. If we let $f$ be defined by:
                    \begin{equation}
                        f(t)=\big(\sin(t),\,\sin(2t)\big)
                    \end{equation}
                    If we set the domain to be $(-\pi,\pi)$, we obtain the
                    lemniscate shown in
                    Fig.~\subref{fig:Immersed_Not_Embedded_002}.
                \end{example}
                \begin{figure}[H]
                    \centering
                    \captionsetup{type=figure}
                    \begin{subfigure}[b]{0.49\textwidth}
                        \centering
                        \captionsetup{type=figure}
                        \includegraphics{images/Immersed_Not_Embedded_001.pdf}
                        \subcaption{Lemniscate with Domain $(-\pi,\,\pi)$}
                        \label{fig:Immersed_Not_Embedded_002}
                    \end{subfigure}
                    \begin{subfigure}[b]{0.49\textwidth}
                        \centering
                        \captionsetup{type=figure}
                        \includegraphics{images/Immersed_Not_Embedded_002.pdf}
                        \subcaption{Lemniscate with Domain $(0,\,2\pi)$}
                        \label{fig:Immersed_Not_Embedded_003}
                    \end{subfigure}
                    \caption{Immersed Lemniscates that are not Embeddings}
                    \label{fig:Immersed_Not_Embedded}
                \end{figure}
        \section{Lecture 2}
            \begin{fdefinition}{Topological Group}{Topological_Group}
                A topological group is a topological space $\topspace{G}$ with a
                binary operation $*$ such that $\monoid{G}$ is a group and and
                the functions $h:G\times{G}\rightarrow{G}$ and
                $\nu:G\rightarrow{G}$ defined by $h(x,y)=x*y$ and
                $\nu(g)=g^{\minus{1}}$ are continuous.
            \end{fdefinition}
            \begin{example}
                \label{ex:Top_Group_Indiscrete}%
                Given any group $\monoid{G}$, if we endow $G$ with the trivial
                topology $\tau=\{\emptyset,G\}$ then $\topgroup{G}$ is a
                topological group.
            \end{example}
            \begin{example}
                As another trivial example, if we endow a group $\monoid{G}$
                with the discrete topology $\tau=\powset{G}$, then
                $\topgroup{G}$ is a topological group. 
            \end{example}
            There's often the requirement that a topological group be Hausdorff.
            This is not completely unfounded since it is a remarkable fact that
            if the topology of your topology group is $T_{0}$ (points are
            topologically distinguishable), then the group structure allows you
            to prove that it is automatically $T_{2}$ (also known as Hausdorff).
            Since $T_{0}$ is a \textit{very} weak assumption, there's no harm in
            supposing topological groups are Hausdorff. But, as
            Ex.~\ref{ex:Top_Group_Indiscrete} shows it is possible to find
            counterexamples.
            \begin{fdefinition}{Topological Subgroup}{Topological_Subgroup}
                A topological subgroup of a topological group is a subgroup with
                the subspace topology.
            \end{fdefinition}
            \begin{example}
                If we let $\nspace[]$ have its usual topology and additive
                operation $+$, then $\mathbb{Q}$ is a topological subgroup. The
                subspace topology is the standard metric topology that makes it
                dense in $\nspace[]$.
            \end{example}
            If the topological space is locally Euclidean, then it is $T_{1}$
            and hence by the previous comment our topological group will be
            Hausdorff. Hence, so long as we consider second countable spaces,
            locally Euclidean topological groups are topological manifolds for
            free. With this we transition to Lie groups, which are topological
            groups with a smooth manifold structure.
            \begin{fdefinition}{Lie Group}{Lie_Group}
                A Lie group is a smooth manifold $\manifold{G}$ with a binary
                operation $*$ such that $\monoid{G}$ is a group, and the
                functions $h:G\times{G}\rightarrow{G}$ and $\nu:G\rightarrow{G}$
                defined by $h(x,y)=x*y$ and $\nu(g)=g^{\minus{1}}$ are smooth.
            \end{fdefinition}
            \begin{example}
                Taking the additive structure from the well known vector spaces
                $\nspace$ and $\mathbb{C}^{n}$ give rise to Lie groups.
                Topologically, $\mathbb{C}^{n}$ is homeomorphic to
                $\nspace[2n]$.
            \end{example}
            \begin{example}
                The unit circle $\nsphere[1]$ with the \textit{rotation}
                operation:
                \begin{equation}
                    \exp(i\theta_{1})*\exp(\theta_{2})
                        =\exp(i(\theta_{1}+\theta_{2}))
                \end{equation}
                Using this we can see that the $n$ torus $\ntorus$ is also a
                Lie group since it is the product of circles. We may endow it
                with the algebraic structure of $\nspace/\mathbb{Z}^{n}$.
            \end{example}
            \begin{example}
                Letting $H$ denote the quaternions, the topological structure is
                homeomorphic to $\nspace[4]$. Thus we may embed the 3-sphere
                $\nsphere[3]$ into $H$ by considering quaternions with norm 1.
                This shows that $\nsphere[3]$ has a Lie group structure.
            \end{example}
            \begin{example}
                The matrix groups $GL_{n}(\nspace[])$ are all examples of Lie
                groups when equipped with the topology of $\nspace[n^{2}]$.
            \end{example}
            The following question was posed by Hilbert in his famous list of
            problems released at the turn of the $20^{th}$ century. This is
            known as Hilbert's Fifth Problem.
            \begin{equation}
                \begin{split}
                    &\text{If }\topgroup{X}
                    \text{ is a topological group such that }
                    \topspace{X}\text{ is a topological}\\
                    &\text{manifold, then is this a Lie group?}
                \end{split}
            \end{equation}
            The answer is yes, the proof came in the 1950s. Moreover, the
            smooth structure is unique. This means there is no harm in defining
            a Lie group to be a a topological group with a topological manifold
            topology even though \textit{a priori} this seems weaker.
            \begin{fdefinition}{Lie Group Homomorphism}{Lie_Group_Homomorphism}
                A Lie group homomorphism from a Lie group
                $\topgroup[1]{G}$ to a Lie group $\topgroup[2]{G}$ is a smooth
                function $\phi:G_{1}\rightarrow{G}_{2}$ such that $\phi$ is a
                group homomorphism with respect to $\monoid[1]{G}$ and
                $\monoid[2]{G}$.
            \end{fdefinition}
            \begin{example}
                The exponential function gives rise to a Lie group homomorphism
                $\exp:\nspace[]\rightarrow\nsphere[1]$ by mapping
                $\theta\mapsto\exp(i\theta)$. Topologically this also a covering
                map.
            \end{example}
            \begin{example}
                Given a Lie group $G$ and an element $a\in{G}$, the inner
                automorphism $x\mapsto{a}*x*a^{\minus{1}}$ gives us a Lie group
                homomorphism.
            \end{example}
            \begin{theorem}
                If $\topgroup[1]{G}$ and $\topgroup[2]{G}$ are Lie groups and if
                $\phi:G_{1}\rightarrow{G}_{2}$ is a Lie group homomorphism, then
                $\phi$ is a map of constant rank.
            \end{theorem}
            \begin{proof}
                For $a\in{G}_{1}$, let $L_{a}$ be defined by $x\mapsto{a}*x$.
                This is a diffeomorphism.
            \end{proof}
            \begin{theorem}
                If $\phi:G_{1}\rightarrow{G}_{2}$ is a bijective Lie group
                homomorphism, then it is an isomorphism.
            \end{theorem}
            \begin{proof}
                By the inverse function theorem, together with the fact that the
                map has constant rank.
            \end{proof}
            \begin{fdefinition}{Lie Subgroup}{Lie Subgroup}
                A Lie subgroup of a Lie group $\topgroup[*][G]{G}$ is a Lie
                group $\topgroup[*][H]{H}$ and an injective Lie homomorphism
                $f:H\rightarrow{G}$, denoted $(H,f)$.
            \end{fdefinition}
            Note, given a Lie subgroup $(H,f)$ of a Lie group $G$ we often say
            tjat $f[H]$, the image of $H$ in $G$, is a Lie subgroup of $G$. This
            definition is well founded since a Lie subgroup is an immersed
            submanifold of $G$. This is because a Lie group homomorphism has
            constant rank, and if $f$ is injective then it must be an immersion
            by the theorem on rank. Note that a Lie subgroup of some Lie group
            $G$ need \textit{not} be a topological subgroup. A topological
            subgroup is a subgroup endowed with the subspace topology. Hence a
            Lie subgroup $(H,f)$ yields a topological subgroup if and only if
            $f$ is an embedding of $H$ into $G$.
            \begin{example}
                A line of irrational slope on a torus is an example of a Lie
                subgroup that is not a topological subgroup. It is an immersed
                submanifold, but not an embedded submanifold. If we take a line
                of \textit{rational} slope then we again obtain a Lie subgroup
                that is also a topological subgroup. This is homeomorphic to
                $\nsphere[1]$.
            \end{example}
            We'll need the following results:
            \begin{theorem}
                If $\iota:N\rightarrow{M}$ is an immersion, if $P$ is a
                manifold, and if $f:P\rightarrow{N}$ is a continuous function
                such that $\iota\circ{f}$ is smooth, then $f$ is smooth.
            \end{theorem}
            \begin{theorem}
                If $\iota:N\rightarrow{M}$ is an embedding, and if
                $\iota\circ{f}$ is smooth, then $f$ is smooth.
            \end{theorem}
            With this we may prove the following:
            \begin{theorem}
                If $\topgroup{G}$ is a Lie group, and if $(H,f)$ is a Lie
                subgroup such that $f:H\rightarrow{G}$ is an embedding, then
                $\topgroup[*][H]{H}$ is a Lie group.
            \end{theorem}
\end{document}