\documentclass[crop=false,class=book,oneside]{standalone}
%----------------------------Preamble-------------------------------%
%---------------------------Packages----------------------------%
\usepackage{geometry}
\geometry{b5paper, margin=1.0in}
\usepackage[T1]{fontenc}
\usepackage{graphicx, float}            % Graphics/Images.
\usepackage{natbib}                     % For bibliographies.
\bibliographystyle{agsm}                % Bibliography style.
\usepackage[french, english]{babel}     % Language typesetting.
\usepackage[dvipsnames]{xcolor}         % Color names.
\usepackage{listings}                   % Verbatim-Like Tools.
\usepackage{mathtools, esint, mathrsfs} % amsmath and integrals.
\usepackage{amsthm, amsfonts, amssymb}  % Fonts and theorems.
\usepackage{tcolorbox}                  % Frames around theorems.
\usepackage{upgreek}                    % Non-Italic Greek.
\usepackage{fmtcount, etoolbox}         % For the \book{} command.
\usepackage[newparttoc]{titlesec}       % Formatting chapter, etc.
\usepackage{titletoc}                   % Allows \book in toc.
\usepackage[nottoc]{tocbibind}          % Bibliography in toc.
\usepackage[titles]{tocloft}            % ToC formatting.
\usepackage{pgfplots, tikz}             % Drawing/graphing tools.
\usepackage{imakeidx}                   % Used for index.
\usetikzlibrary{
    calc,                   % Calculating right angles and more.
    angles,                 % Drawing angles within triangles.
    arrows.meta,            % Latex and Stealth arrows.
    quotes,                 % Adding labels to angles.
    positioning,            % Relative positioning of nodes.
    decorations.markings,   % Adding arrows in the middle of a line.
    patterns,
    arrows
}                                       % Libraries for tikz.
\pgfplotsset{compat=1.9}                % Version of pgfplots.
\usepackage[font=scriptsize,
            labelformat=simple,
            labelsep=colon]{subcaption} % Subfigure captions.
\usepackage[font={scriptsize},
            hypcap=true,
            labelsep=colon]{caption}    % Figure captions.
\usepackage[pdftex,
            pdfauthor={Ryan Maguire},
            pdftitle={Mathematics and Physics},
            pdfsubject={Mathematics, Physics, Science},
            pdfkeywords={Mathematics, Physics, Computer Science, Biology},
            pdfproducer={LaTeX},
            pdfcreator={pdflatex}]{hyperref}
\hypersetup{
    colorlinks=true,
    linkcolor=blue,
    filecolor=magenta,
    urlcolor=Cerulean,
    citecolor=SkyBlue
}                           % Colors for hyperref.
\usepackage[toc,acronym,nogroupskip,nopostdot]{glossaries}
\usepackage{glossary-mcols}
%------------------------Theorem Styles-------------------------%
\theoremstyle{plain}
\newtheorem{theorem}{Theorem}[section]

% Define theorem style for default spacing and normal font.
\newtheoremstyle{normal}
    {\topsep}               % Amount of space above the theorem.
    {\topsep}               % Amount of space below the theorem.
    {}                      % Font used for body of theorem.
    {}                      % Measure of space to indent.
    {\bfseries}             % Font of the header of the theorem.
    {}                      % Punctuation between head and body.
    {.5em}                  % Space after theorem head.
    {}

% Italic header environment.
\newtheoremstyle{thmit}{\topsep}{\topsep}{}{}{\itshape}{}{0.5em}{}

% Define environments with italic headers.
\theoremstyle{thmit}
\newtheorem*{solution}{Solution}

% Define default environments.
\theoremstyle{normal}
\newtheorem{example}{Example}[section]
\newtheorem{definition}{Definition}[section]
\newtheorem{problem}{Problem}[section]

% Define framed environment.
\tcbuselibrary{most}
\newtcbtheorem[use counter*=theorem]{ftheorem}{Theorem}{%
    before=\par\vspace{2ex},
    boxsep=0.5\topsep,
    after=\par\vspace{2ex},
    colback=green!5,
    colframe=green!35!black,
    fonttitle=\bfseries\upshape%
}{thm}

\newtcbtheorem[auto counter, number within=section]{faxiom}{Axiom}{%
    before=\par\vspace{2ex},
    boxsep=0.5\topsep,
    after=\par\vspace{2ex},
    colback=Apricot!5,
    colframe=Apricot!35!black,
    fonttitle=\bfseries\upshape%
}{ax}

\newtcbtheorem[use counter*=definition]{fdefinition}{Definition}{%
    before=\par\vspace{2ex},
    boxsep=0.5\topsep,
    after=\par\vspace{2ex},
    colback=blue!5!white,
    colframe=blue!75!black,
    fonttitle=\bfseries\upshape%
}{def}

\newtcbtheorem[use counter*=example]{fexample}{Example}{%
    before=\par\vspace{2ex},
    boxsep=0.5\topsep,
    after=\par\vspace{2ex},
    colback=red!5!white,
    colframe=red!75!black,
    fonttitle=\bfseries\upshape%
}{ex}

\newtcbtheorem[auto counter, number within=section]{fnotation}{Notation}{%
    before=\par\vspace{2ex},
    boxsep=0.5\topsep,
    after=\par\vspace{2ex},
    colback=SeaGreen!5!white,
    colframe=SeaGreen!75!black,
    fonttitle=\bfseries\upshape%
}{not}

\newtcbtheorem[use counter*=remark]{fremark}{Remark}{%
    fonttitle=\bfseries\upshape,
    colback=Goldenrod!5!white,
    colframe=Goldenrod!75!black}{ex}

\newenvironment{bproof}{\textit{Proof.}}{\hfill$\square$}
\tcolorboxenvironment{bproof}{%
    blanker,
    breakable,
    left=3mm,
    before skip=5pt,
    after skip=10pt,
    borderline west={0.6mm}{0pt}{green!80!black}
}

\AtEndEnvironment{lexample}{$\hfill\textcolor{red}{\blacksquare}$}
\newtcbtheorem[use counter*=example]{lexample}{Example}{%
    empty,
    title={Example~\theexample},
    boxed title style={%
        empty,
        size=minimal,
        toprule=2pt,
        top=0.5\topsep,
    },
    coltitle=red,
    fonttitle=\bfseries,
    parbox=false,
    boxsep=0pt,
    before=\par\vspace{2ex},
    left=0pt,
    right=0pt,
    top=3ex,
    bottom=1ex,
    before=\par\vspace{2ex},
    after=\par\vspace{2ex},
    breakable,
    pad at break*=0mm,
    vfill before first,
    overlay unbroken={%
        \draw[red, line width=2pt]
            ([yshift=-1.2ex]title.south-|frame.west) to
            ([yshift=-1.2ex]title.south-|frame.east);
        },
    overlay first={%
        \draw[red, line width=2pt]
            ([yshift=-1.2ex]title.south-|frame.west) to
            ([yshift=-1.2ex]title.south-|frame.east);
    },
}{ex}

\AtEndEnvironment{ldefinition}{$\hfill\textcolor{Blue}{\blacksquare}$}
\newtcbtheorem[use counter*=definition]{ldefinition}{Definition}{%
    empty,
    title={Definition~\thedefinition:~{#1}},
    boxed title style={%
        empty,
        size=minimal,
        toprule=2pt,
        top=0.5\topsep,
    },
    coltitle=Blue,
    fonttitle=\bfseries,
    parbox=false,
    boxsep=0pt,
    before=\par\vspace{2ex},
    left=0pt,
    right=0pt,
    top=3ex,
    bottom=0pt,
    before=\par\vspace{2ex},
    after=\par\vspace{1ex},
    breakable,
    pad at break*=0mm,
    vfill before first,
    overlay unbroken={%
        \draw[Blue, line width=2pt]
            ([yshift=-1.2ex]title.south-|frame.west) to
            ([yshift=-1.2ex]title.south-|frame.east);
        },
    overlay first={%
        \draw[Blue, line width=2pt]
            ([yshift=-1.2ex]title.south-|frame.west) to
            ([yshift=-1.2ex]title.south-|frame.east);
    },
}{def}

\AtEndEnvironment{ltheorem}{$\hfill\textcolor{Green}{\blacksquare}$}
\newtcbtheorem[use counter*=theorem]{ltheorem}{Theorem}{%
    empty,
    title={Theorem~\thetheorem:~{#1}},
    boxed title style={%
        empty,
        size=minimal,
        toprule=2pt,
        top=0.5\topsep,
    },
    coltitle=Green,
    fonttitle=\bfseries,
    parbox=false,
    boxsep=0pt,
    before=\par\vspace{2ex},
    left=0pt,
    right=0pt,
    top=3ex,
    bottom=-1.5ex,
    breakable,
    pad at break*=0mm,
    vfill before first,
    overlay unbroken={%
        \draw[Green, line width=2pt]
            ([yshift=-1.2ex]title.south-|frame.west) to
            ([yshift=-1.2ex]title.south-|frame.east);},
    overlay first={%
        \draw[Green, line width=2pt]
            ([yshift=-1.2ex]title.south-|frame.west) to
            ([yshift=-1.2ex]title.south-|frame.east);
    }
}{thm}

%--------------------Declared Math Operators--------------------%
\DeclareMathOperator{\adjoint}{adj}         % Adjoint.
\DeclareMathOperator{\Card}{Card}           % Cardinality.
\DeclareMathOperator{\curl}{curl}           % Curl.
\DeclareMathOperator{\diam}{diam}           % Diameter.
\DeclareMathOperator{\dist}{dist}           % Distance.
\DeclareMathOperator{\Div}{div}             % Divergence.
\DeclareMathOperator{\Erf}{Erf}             % Error Function.
\DeclareMathOperator{\Erfc}{Erfc}           % Complementary Error Function.
\DeclareMathOperator{\Ext}{Ext}             % Exterior.
\DeclareMathOperator{\GCD}{GCD}             % Greatest common denominator.
\DeclareMathOperator{\grad}{grad}           % Gradient
\DeclareMathOperator{\Ima}{Im}              % Image.
\DeclareMathOperator{\Int}{Int}             % Interior.
\DeclareMathOperator{\LC}{LC}               % Leading coefficient.
\DeclareMathOperator{\LCM}{LCM}             % Least common multiple.
\DeclareMathOperator{\LM}{LM}               % Leading monomial.
\DeclareMathOperator{\LT}{LT}               % Leading term.
\DeclareMathOperator{\Mod}{mod}             % Modulus.
\DeclareMathOperator{\Mon}{Mon}             % Monomial.
\DeclareMathOperator{\multideg}{mutlideg}   % Multi-Degree (Graphs).
\DeclareMathOperator{\nul}{nul}             % Null space of operator.
\DeclareMathOperator{\Ord}{Ord}             % Ordinal of ordered set.
\DeclareMathOperator{\Prin}{Prin}           % Principal value.
\DeclareMathOperator{\proj}{proj}           % Projection.
\DeclareMathOperator{\Refl}{Refl}           % Reflection operator.
\DeclareMathOperator{\rk}{rk}               % Rank of operator.
\DeclareMathOperator{\sgn}{sgn}             % Sign of a number.
\DeclareMathOperator{\sinc}{sinc}           % Sinc function.
\DeclareMathOperator{\Span}{Span}           % Span of a set.
\DeclareMathOperator{\Spec}{Spec}           % Spectrum.
\DeclareMathOperator{\supp}{supp}           % Support
\DeclareMathOperator{\Tr}{Tr}               % Trace of matrix.
%--------------------Declared Math Symbols--------------------%
\DeclareMathSymbol{\minus}{\mathbin}{AMSa}{"39} % Unary minus sign.
%------------------------New Commands---------------------------%
\DeclarePairedDelimiter\norm{\lVert}{\rVert}
\DeclarePairedDelimiter\ceil{\lceil}{\rceil}
\DeclarePairedDelimiter\floor{\lfloor}{\rfloor}
\newcommand*\diff{\mathop{}\!\mathrm{d}}
\newcommand*\Diff[1]{\mathop{}\!\mathrm{d^#1}}
\renewcommand*{\glstextformat}[1]{\textcolor{RoyalBlue}{#1}}
\renewcommand{\glsnamefont}[1]{\textbf{#1}}
\renewcommand\labelitemii{$\circ$}
\renewcommand\thesubfigure{%
    \arabic{chapter}.\arabic{figure}.\arabic{subfigure}}
\addto\captionsenglish{\renewcommand{\figurename}{Fig.}}
\numberwithin{equation}{section}

\renewcommand{\vector}[1]{\boldsymbol{\mathrm{#1}}}

\newcommand{\uvector}[1]{\boldsymbol{\hat{\mathrm{#1}}}}
\newcommand{\topspace}[2][]{(#2,\tau_{#1})}
\newcommand{\measurespace}[2][]{(#2,\varSigma_{#1},\mu_{#1})}
\newcommand{\measurablespace}[2][]{(#2,\varSigma_{#1})}
\newcommand{\manifold}[2][]{(#2,\tau_{#1},\mathcal{A}_{#1})}
\newcommand{\tanspace}[2]{T_{#1}{#2}}
\newcommand{\cotanspace}[2]{T_{#1}^{*}{#2}}
\newcommand{\Ckspace}[3][\mathbb{R}]{C^{#2}(#3,#1)}
\newcommand{\funcspace}[2][\mathbb{R}]{\mathcal{F}(#2,#1)}
\newcommand{\smoothvecf}[1]{\mathfrak{X}(#1)}
\newcommand{\smoothonef}[1]{\mathfrak{X}^{*}(#1)}
\newcommand{\bracket}[2]{[#1,#2]}

%------------------------Book Command---------------------------%
\makeatletter
\renewcommand\@pnumwidth{1cm}
\newcounter{book}
\renewcommand\thebook{\@Roman\c@book}
\newcommand\book{%
    \if@openright
        \cleardoublepage
    \else
        \clearpage
    \fi
    \thispagestyle{plain}%
    \if@twocolumn
        \onecolumn
        \@tempswatrue
    \else
        \@tempswafalse
    \fi
    \null\vfil
    \secdef\@book\@sbook
}
\def\@book[#1]#2{%
    \refstepcounter{book}
    \addcontentsline{toc}{book}{\bookname\ \thebook:\hspace{1em}#1}
    \markboth{}{}
    {\centering
     \interlinepenalty\@M
     \normalfont
     \huge\bfseries\bookname\nobreakspace\thebook
     \par
     \vskip 20\p@
     \Huge\bfseries#2\par}%
    \@endbook}
\def\@sbook#1{%
    {\centering
     \interlinepenalty \@M
     \normalfont
     \Huge\bfseries#1\par}%
    \@endbook}
\def\@endbook{
    \vfil\newpage
        \if@twoside
            \if@openright
                \null
                \thispagestyle{empty}%
                \newpage
            \fi
        \fi
        \if@tempswa
            \twocolumn
        \fi
}
\newcommand*\l@book[2]{%
    \ifnum\c@tocdepth >-3\relax
        \addpenalty{-\@highpenalty}%
        \addvspace{2.25em\@plus\p@}%
        \setlength\@tempdima{3em}%
        \begingroup
            \parindent\z@\rightskip\@pnumwidth
            \parfillskip -\@pnumwidth
            {
                \leavevmode
                \Large\bfseries#1\hfill\hb@xt@\@pnumwidth{\hss#2}
            }
            \par
            \nobreak
            \global\@nobreaktrue
            \everypar{\global\@nobreakfalse\everypar{}}%
        \endgroup
    \fi}
\newcommand\bookname{Book}
\renewcommand{\thebook}{\texorpdfstring{\Numberstring{book}}{book}}
\providecommand*{\toclevel@book}{-2}
\makeatother
\titleformat{\part}[display]
    {\Large\bfseries}
    {\partname\nobreakspace\thepart}
    {0mm}
    {\Huge\bfseries}
\titlecontents{part}[0pt]
    {\large\bfseries}
    {\partname\ \thecontentslabel: \quad}
    {}
    {\hfill\contentspage}
\titlecontents{chapter}[0pt]
    {\bfseries}
    {\chaptername\ \thecontentslabel:\quad}
    {}
    {\hfill\contentspage}
\newglossarystyle{longpara}{%
    \setglossarystyle{long}%
    \renewenvironment{theglossary}{%
        \begin{longtable}[l]{{p{0.25\hsize}p{0.65\hsize}}}
    }{\end{longtable}}%
    \renewcommand{\glossentry}[2]{%
        \glstarget{##1}{\glossentryname{##1}}%
        &\glossentrydesc{##1}{~##2.}
        \tabularnewline%
        \tabularnewline
    }%
}
\newglossary[not-glg]{notation}{not-gls}{not-glo}{Notation}
\newcommand*{\newnotation}[4][]{%
    \newglossaryentry{#2}{type=notation, name={\textbf{#3}, },
                          text={#4}, description={#4},#1}%
}
%--------------------------LENGTHS------------------------------%
% Spacings for the Table of Contents.
\addtolength{\cftsecnumwidth}{1ex}
\addtolength{\cftsubsecindent}{1ex}
\addtolength{\cftsubsecnumwidth}{1ex}
\addtolength{\cftfignumwidth}{1ex}
\addtolength{\cfttabnumwidth}{1ex}

% Indent and paragraph spacing.
\setlength{\parindent}{0em}
\setlength{\parskip}{0em}
%----------------------------GLOSSARY-------------------------------%
\makeglossaries
\loadglsentries{../../glossary}
\loadglsentries{../../acronym}
%--------------------------Main Document----------------------------%
\begin{document}
    \ifx\ifmathcourses\undefined
        \pagenumbering{roman}
        \title{Point-Set Topology}
        \author{Ryan Maguire}
        \date{\vspace{-5ex}}
        \maketitle
        \tableofcontents
        \clearpage
        \chapter*{Point-Set Topology}
        \addcontentsline{toc}{chapter}{Point-Set Topology}
        \markboth{}{POINT-SET TOPOLOGY}
        \vspace{10ex}
        \setcounter{chapter}{1}
        \pagenumbering{arabic}
    \else
        \chapter{Point-Set Topology}
    \fi
    \section{Topological Spaces}
        \begin{ldefinition}{Topology}{Topology}
            A topology on a set $X$ is a subset
            $\tau\subseteq\mathcal{P}(X)$ such that
            $\emptyset\in\tau$, $X\in\tau$, and for any
            subset $\mathcal{O}\subseteq\tau$, it is true that:
            \begin{equation}
                \bigcup_{\mathcal{U}\in\mathcal{O}}\mathcal{U}\in\tau
            \end{equation}
            And such that for any finite subset
            $\mathcal{C}\subseteq\tau$, it is true that:
            \begin{equation}
                \bigcap_{\mathcal{U}\in\mathcal{C}}\mathcal{U}\in\tau
            \end{equation}
            That is, $\tau$ is closed to arbitrary unions and
            finite intersections.
        \end{ldefinition}
        \begin{ldefinition}{Topological Space}{Topological_Space}
            A topological space, denote $(X,\tau)$ is a subset $X$
            and a topology $\tau$ on $X$.
        \end{ldefinition}
        \begin{ldefinition}{Open Subsets}{Open_Subsets}
            An open subset of a topological space $(X,\tau)$ is a
            set $\mathcal{U}\subseteq{X}$ such that
            $\mathcal{U}\in\tau$.
        \end{ldefinition}
        \begin{theorem}
            \label{thm:Emptyset_Is_Open}%
            If $(X,\tau)$ is a topological space, then
            $\emptyset$ is an open subset.
        \end{theorem}
        \begin{proof}
            For since $(X,\tau)$ is a topological space,
            $\tau$ is a topology on $X$
            (Def.~\ref{def:Topological_Space}). But then
            $X\in\tau$ (Def.~\ref{def:Topology}) and
            thus $X$ is an open subset of $(X,\tau)$
            (Def.~\ref{def:Open_Subsets}). Therefore, etc.
        \end{proof}
        \begin{theorem}
            \label{thm:Whole_Space_Is_Open}%
            If $(X,\tau)$ is a topological space, then
            $X$ is an open subset.
        \end{theorem}
        \begin{proof}
            For since $(X,\tau)$ is a topological space,
            $\tau$ is a topology on $X$
            (Def.~\ref{def:Topological_Space}). But then
            $\emptyset\in\tau$ (Def.~\ref{def:Topology}) and
            thus $\emptyset$ is an open subset of $(X,\tau)$
            (Def.~\ref{def:Open_Subsets}). Therefore, etc.
        \end{proof}
        \begin{ldefinition}{Closed Subsets}{Closed_Subsets}
            A closed subset of a topological space $(X,\tau)$
            is a subset $\mathcal{C}\subseteq{X}$ such that
            $X\setminus\mathcal{C}\in\tau$. That is, the complement
            of $\mathcal{C}$ is an open subset of $(X,\tau)$.
        \end{ldefinition}
        \begin{theorem}
            \label{thm:Emptyset_Is_Closed}%
            If $(X,\tau)$ is a topological space, then
            $\emptyset$ is closed.
        \end{theorem}
        \begin{proof}
            For since $(X,\tau)$ is a topological space, $X$ is an
            open subset (Thm.~\ref{thm:Whole_Space_Is_Open}).
            But $X\setminus\emptyset=X$, and thus the complement
            of $\emptyset$ is open. Thus, $\emptyset$ is
            closed (Def.~\ref{def:Closed_Subsets}).
            Therefore, etc.
        \end{proof}
        \begin{theorem}
            \label{thm:Whole_Space_Is_Closed}%
            If $(X,\tau)$ is a topological space, then $X$ is closed.
        \end{theorem}
        \begin{proof}
            For since $(X,\tau)$ is a topological space, $\emptyset$
            is an open subset (Thm.~\ref{thm:Emptyset_Is_Open}).
            But $X\setminus{X}=\emptyset$, and thus the complement
            of $X$ is open. Thus, $X$ is closed
            (Def.~\ref{def:Closed_Subsets}). Therefore, etc.
        \end{proof}
        \begin{theorem}
            \label{thm:Comp_of_Open_is_Closed}%
            If $(X,\tau)$ is a topological space, and if
            $\mathcal{U}\subseteq{X}$ is open, then
            $X\setminus\mathcal{U}$ is closed.
        \end{theorem}
        \begin{proof}
            For if $\mathcal{U}\subseteq{X}$ is a set, then
            $X\setminus(X\setminus\mathcal{U})=\mathcal{U}$. But
            $\mathcal{U}$ is open, and thus the complement of
            $X\setminus{U}$ is open. But then $X\setminus{U}$
            is closed (Def.~\ref{def:Closed_Subsets}).
        \end{proof}
        \begin{ldefinition}{Relative Topology}{Relative_Topology}
            The relative topology of a subset
            $A\subseteq{X}$ in a topological space $(X,\tau)$ is
            the set $\tau_{A}$ defined by:
            \begin{equation}
                \tau_{A}=
                \big\{\;A\cap\mathcal{U}\,:\,
                    \mathcal{U}\in\tau\;\big\}
            \end{equation}
        \end{ldefinition}
        \begin{theorem}
            If $(X,\tau)$ is a topological space, if $A\subseteq{X}$,
            and if $\tau_{A}$ is the relative topology on $A$,
            then $\tau_{A}$ is a topology on $A$.
        \end{theorem}
        \begin{proof}
            For since $\emptyset\in\tau$, and
            $\emptyset\cap{A}=\emptyset$, we have that
            $\emptyset\in\tau_{A}$. Similarly, since
            $A\subseteq{X}$, and sicne $X\in\tau$, we see that
            $A=A\cap{X}\in\tau_{A}$. If $\mathcal{O}$ is a subset
            of $\tau_{A}$, then there is a subset
            $\Delta\subseteq\tau$ such that:
            \begin{equation}
                \mathcal{O}=\big\{\;A\cap\mathcal{U}\,:\,
                    \mathcal{U}\in\Delta\;\}
            \end{equation}
            Define $\mathcal{D}$ by:
            \begin{equation}
                \mathcal{D}=\bigcup_{\mathcal{U}\in\Delta}\mathcal{U}
            \end{equation}
            But then:
            \begin{equation}
                \bigcup_{\mathcal{V}\in\mathcal{O}}\mathcal{V}
                =\bigcup_{\mathcal{U}\in\Delta}(A\cap\mathcal{U})
                =A\cap\Big(\bigcup_{\mathcal{U}\in\Delta}\mathcal{U}
                    \Big)
                =A\cap\mathcal{D}
            \end{equation}
            But $\tau$ is a topology on $X$, and thus
            $\mathcal{D}\in\tau$ (Def.~\ref{def:Topology}). But
            then $A\cap\mathcal{D}\in\tau_{A}$
            (Def.~\ref{def:Relative_Topology}). Thus, $\tau_{A}$
            is closed to arbitrary unions.
        \end{proof}
        \begin{definition}
            If $(X,\tau)$ is a topological space and $S\subset{X}$,
            then a set $\mathcal{U}\subset S$ is said to be open in
            $S$ if and only if $\mathcal{U}\in \mathscr{T}$, where
            $\mathscr{T}$ is the relative topology on $S$.
        \end{definition}
        \begin{ldefinition}{Continuous Functions}{Cont_Func_Top}
            A continuous function from a topological space
            $(X,\tau_{X})$ to a topological space $(Y,\tau_{Y})$ is
            a function $f:X\rightarrow{Y}$ such that for all
            $\mathcal{U}\in\tau_{Y}$ it is true that
            $f^{\minus{1}}(\mathcal{U})\in\tau_{X}$. That is,
            the pre-image of open sets is open.
        \end{ldefinition}
        \begin{theorem}
            If $(X,\tau_{X})$, $(Y,\tau_{Y})$, and $(Z,\tau_{Z})$
            are topological spaces, and if the functions
            $f:X\rightarrow{Y}$ and $g:Y\rightarrow{Z}$ are
            continuous, then $g\circ{f}:X\rightarrow{Z}$
            is continuous.
        \end{theorem}
        \begin{proof}
            For if $\mathcal{V}\in\tau_{Z}$ is and open set, then
            $g^{\minus{1}}(\mathcal{V})\in\tau_{Y}$, since $g$
            is continuous. But then since $f$ is continuous,
            $f^{\minus{1}}(g^{\minus{1}}(\mathcal{V}))\in\tau_{X}$
            (Def.~\ref{def:Cont_Func_Top}). Thus $g\circ{f}$
            is continuous.
        \end{proof}
        \begin{ldefinition}
              {Convergent Sequences In Topological Spaces}
              {Conv_Seq_Top}
            A convergent sequence in a topological space $(X,\tau)$
            is a sequence $a:\mathbb{N}\rightarrow{X}$ such that
            there is an $x\in{X}$ such that, for all
            $\mathcal{U}\in\tau$ such that $x\in\mathcal{U}$, there
            is an $N\in\mathbb{N}$ such that, for all $n>N$, it is
            true that $x_{n}\in\mathcal{U}$. We denote this by
            $a_{n}\rightarrow{x}$.
        \end{ldefinition}
        \begin{ldefinition}
              {Limits of Sequences in Topological Spaces}
              {Lim_Seq_Top}
            A limit of a sequence $a:\mathbb{N}\rightarrow{X}$ in
            a topological space $(X,\tau)$ is a point $x\in{X}$
            such that $a_{n}\rightarrow{x}$.
        \end{ldefinition}
        \begin{theorem}
            There exist topological spaces with convergent sequences
            that do not have unique limits.
        \end{theorem}
        \begin{proof}
            For let $X=\{1,2,3\}$, and let
            $\tau=\{\emptyset, \{1,2\},\{1,2,3\}\}$. We see that
            $\emptyset,X\in \tau$, unions and intersections are
            in $\tau,$ and thus $\tau$ is a topology. Let:
            \begin{equation}
                x_{n}=
                \begin{cases}
                    1,&n\textrm{ odd}\\
                    2,&n\textrm{ even}
                \end{cases}
            \end{equation}
            Then $x_n \rightarrow 1$ and $x_n \rightarrow 2$. To
            see this, let $\mathcal{U}$ be an open set such that
            $1\in \mathcal{U}$. Our choices are $\{1,2\}$ and
            $\{1,2,3\}$. Then for all $n\in \mathbb{N}$,
            $x_{n}\in\mathcal{U}$, and thus $x_{n}\rightarrow{1}$.
            Similarly, $x_n \rightarrow 2$. Convergence is not
            necessarily unique in topological spaces.
        \end{proof}
        \subsection{Separation Axioms}
        \begin{ldefinition}{Fr\'{e}chet Topological Space}
              {Frechet_Topological_Space}
            A Fr\'{e}chet Topological Space is a topological
            space $(X,d)$ such that for all distinct $x,y\in{X}$
            there is an open set $\mathcal{U}$ such that
            $x\in\mathcal{U}$ and $y\notin\mathcal{U}$.
        \end{ldefinition}
        \begin{theorem}
            If $(X,\tau)$ is a Fr\'{e}chet Topological space, and
            if $x\in{X}$, $\{x\}$ is closed subset.
        \end{theorem}
        \begin{proof}
            For if $x\in{X}$, then for all $y\in{X}$ such that
            $y\ne{x}$, there is a $\mathcal{U}_{y}\in\tau$ such
            that $x\notin\mathcal{U}_{y}$ and $y\in\mathcal{U}_{y}$.
            Define $\mathcal{V}$ by:
            \begin{equation}
                \mathcal{V}\;=
                \bigcup_{y\in{X}\setminus\{x\}}\mathcal{U}_{y}
            \end{equation}
            But then $\mathcal{V}\in\tau$, since $\tau$ is a
            topology (Def.~\ref{def:Topology}). And moreover, for
            all $y\in{X}$ such that $y\ne{x}$, it is true that
            $y\in\mathcal{V}$. Lastly, $x\notin\mathcal{V}$.
            Therefore $X\setminus\mathcal{V}=\{x\}$. But if
            $\mathcal{V}$ is open, then $X\setminus\mathcal{V}$ is
            closed (Thm.~\ref{thm:Comp_of_Open_is_Closed}).
            Therefore, etc.
        \end{proof}
        \begin{ldefinition}{Hausdorff Topological Space}
              {Hausdoff_Top}
            A Hausdorff Topological space is a topological space
            $(X,\tau)$ such that, for all distinct points
            $x,y\in{X}$, there are disjoint open subsets
            $\mathcal{U}_{x}$, $\mathcal{U}_{y}\in\tau$ such that
            $x\in\mathcal{U}_{x}$ and $y\in\mathcal{U}_{y}$.
        \end{ldefinition}
        \begin{theorem}
            If $(X,\tau)$ is a Hausdorff topological space, then
            it is a Fr\'{e}chet topological space.
        \end{theorem}
        \begin{proof}
            For if $x$ and $y$ are distinct points in $X$ and if
            $(X,\tau)$ is a Hausdorff topological space, then there
            is are disjoint open subsets $\mathcal{U}_{x}$ and
            $\mathcal{U}_{y}$ such that $x\in\mathcal{U}_{x}$ and
            $y\in\mathcal{U}_{y}$. But then there is a open subset
            such that $x\notin\mathcal{U}_{y}$ and
            $y\in\mathcal{U}_{y}$. Therefore, $(X,\tau)$ is a
            Fr'{e}chet topological space.
        \end{proof}
            \begin{theorem}
            Convergence in a Hausdorff Space $(X,\tau)$ is unique.
            \end{theorem}
            \begin{proof}
            $[x_n \rightarrow x\in X]\land [x_n \rightarrow y\in X]\land[x\ne y]\Rightarrow [\exists \mathcal{U},\mathcal{V}:\mathcal{U}\cap \mathcal{V}=\emptyset\land x\in \mathcal{U}\land y\in \mathcal{V}]$. $[x_n\rightarrow x]\Rightarrow [\exists N_1\in \mathbb{N}:n>N_1\Rightarrow x_n \in \mathbb{N}]$. $[x_n\rightarrow y]\Rightarrow [N_2\in \mathbb{N}:n>N\Rightarrow x_n \in \mathcal{V}]$. $[n>\max\{N_1,N_2\}]\Rightarrow [x_n \in \mathcal{U}\cap \mathcal{V}]$, a contradiction. Therefore, etc.
            \end{proof}
            \begin{definition}
            A topological space $(X,\tau)$ is said to be regular if for each closed subset $E\subset X$ and for each point $x\in E^c$, there exist disjoint open sets $\mathcal{U}$ and $\mathcal{V}$ such that $x\in \mathcal{U}$ and $E\subset \mathcal{V}$.
            \end{definition} 
            \begin{definition}
            In a topological space $(X,\tau)$, a point $p$ is said to have a neighborhood $S\subset X$ if and only if there is a set $\mathcal{U}\subset S$ such that $\mathcal{U}\in \tau$ and $p\in \mathcal{U}$.
            \end{definition}
            \begin{definition}
            A $T_3$ space is a regular $T_1$ space.
            \end{definition}
            \begin{theorem}
            A $T_3$ space $(X,\tau)$ is a $T_2$ space.
            \end{theorem}
            \begin{proof}
            Let $x,y\in X$ be distinct. As a $T_3$ space is $T_1$, $\{x\}$ is closed. Thus $\exists \mathcal{U},\mathcal{V}\in\tau: \mathcal{U}\cap\mathcal{V}=\emptyset, \{x\}\subset \mathcal{U}$, and $y\in \mathcal{V}$.
            \end{proof}
            \begin{definition}
            A topological space $(X,\tau)$ is said to be normal if and only if for all disjoint closed subsets $E,F\subset X$, there are disjoint open sets $\mathcal{U}$ and $\mathcal{V}$ such that $E\subset \mathcal{U}$ and $F\subset \mathcal{V}$.
            \end{definition}
            \begin{definition}
            A $T_4$ space is a normal $T_1$ space.
            \end{definition}
            \begin{theorem}
            A $T_4$ space $(X,\tau)$ is a $T_3$ space.
            \end{theorem}
            \begin{proof}
            A $T_4$ space is $T_1$. If $E\underset{Closed}\subset X$ and $x\in E^c$, then $\{x\}$ is closed. Thus $\exists \mathcal{U},\mathcal{V}\in\tau: \mathcal{U}\cap\mathcal{V}=\emptyset, \{x\}\subset \mathcal{U}$, and $E\subset \mathcal{V}$.
            \end{proof}
            \begin{definition}
            A homeomorphism between two topological spaces $(X,\tau)$ and $(Y,\tau)$ is a continuous bijection $f:X\rightarrow Y$ such that $f^{-1}:Y\rightarrow X$ is continuous.
            \end{definition}
            \begin{definition}
            If $(X,\tau)$ is a topological space, and $S\subset X$, then an open cover $\mathcal{O}$ of $S$ is a set of open sets $\mathcal{U}_{\alpha}$ such that $S\subset \cup_{\alpha\in A} \mathcal{U}_{\alpha}$, where $A$ is some index set.
            \end{definition}
            \begin{definition}
            A subcover of an open cover $\mathcal{O}$ is a subset of $\mathcal{O}$ that is also a cover.
            \end{definition}
            \begin{definition}
            If $(X,\tau)$ is a topological space and $S\subset X$, then $S$ is said to be compact if and only if every open cover of $S$ has a finite subcover.
            \end{definition}
            \begin{theorem}
            If $S$ is a compact subset of a Hausdorff space, then for all $x\in S^c$ there are disjoint open sets $\mathcal{U}$ and $\mathcal{V}$ such that $x\in \mathcal{U}$ and $S\subset \mathcal{V}$.
            \end{theorem}
            \begin{proof}
            For let $x\in S^c$. For all $y\in S$ there are disjoint open sets $\mathcal{U}_y$ and $\mathcal{V}_y$ such that $x\in \mathcal{U}$ and $y\in \mathcal{V}$. But then $\cup_{y\in S} \mathcal{U}_y$ is an open cover of $S$. As $S$ is compact, there is a finite subcover, that is sets $\mathcal{V}_{y_1},\hdots, \mathcal{V}_{y_n}$ that cover $S$. But then $\cap_{k=1}^{n} \mathcal{U}_{y_k}$ is open, contains $x$ and is disjoint from $\cup_{k=1}^{n} \mathcal{V}_{y_k}$. Therefore, etc.
            \end{proof}
            \begin{theorem}
            Every compact subset of a Hausdorff space $(X,\tau)$ is closed.
            \end{theorem}
            \begin{proof}
            Let $S$ be a compact subset of a X. $\forall x\in S^c, \exists \mathcal{U}_x\in \tau:\mathcal{U}_x\cap S = \emptyset:x\in \mathcal{U}_x$. But then $S^c \subset \underset{x\in S^c}\cup\mathcal{U}_x$. But also $S\cap (\cup_{x\in S^c}\mathcal{U}_x) = \emptyset$. Thus $S^c = \cup_{x\in S^c}\mathcal{U}_x$, and therefore $S^c$ is open. Thus $S$ is closed.
            \end{proof}
            \begin{theorem}
            If $S$ is a closed subset of a compact space $(X,\tau)$, $S$ is compact.
            \end{theorem}
            \begin{proof}
            For let $\mathcal{O}$ be an open cover of $S$. As $S$ is closed, $S^c$ is open, and thus $\{S^c\} \cup \mathcal{O}$ is an open cover $X$. As $X$ is compact, there is a finite subcover, call it $\mathscr{O}$. But then $\mathscr{O}\setminus \{S^c\}$ is a finite subcover $\mathcal{O}$ that covers $S$. Thus, etc.
            \end{proof}
            \begin{theorem}
            If $f:X\rightarrow Y$ is continuous and $X$ is compact, then $f(X)\subset Y$ is compact.
            \end{theorem}
            \begin{proof}
            Let $\mathcal{O}$ be an open cover of $f(X)$. As $f$ is continuous, $\mathcal{U}\in\mathcal{O}\Rightarrow f^{-1}(\mathcal{U})$ is open in $X$. Thus $\cup_{\mathcal{U}\in \mathcal{O}} f^{-1}(\mathcal{U})$ is an open cover of $X$. As $X$ is compact, there is a finite subcover, say $\mathscr{O}$. But then $\cup_{\mathcal{V}\in \mathscr{O}} \mathcal{V}$ is a finite subcover of $\mathcal{O}$. Therefore, etc.
            \end{proof}
            \begin{theorem}
            If $f:X\rightarrow Y$ is a continuous bijection, $X$ is compact and $Y$ is Hausdorff, then $f$ is a homeomorphism.
            \end{theorem}
            \begin{proof}
            If suffices to show that if $\mathcal{U}$ is open in $X$, then $f(\mathcal{U})$ is open in $f(X)$. Let $\mathcal{U}$ be open in $X$. As $X$ is compact and $\mathcal{U}$ is open, $\mathcal{U}^c$ is compact. But then $f(\mathcal{U}^c) = f(X)\setminus f(\mathcal{U})$ is compact. Thus $f(X)\setminus f(\mathcal{U})\underset{Closed}\subset f(X)\Rightarrow f(\mathcal{U})\underset{Open}\subset f(X)$.
            \end{proof}
            \begin{definition}
            A topological space $(X,\tau)$ is said to be disconnected if and only if there are two disjoint nonempty open sets $\mathcal{U}$ and $\mathcal{V}$ such that $X = \mathcal{U}\cup \mathcal{V}$.
            \end{definition}
            \begin{theorem}
            A topological space $(X,\tau)$ is disconnected if and only if there are two non-empty disjoint closed set $\mathcal{C}$ and $\mathcal{D}$ such that $X=\mathcal{C}\cup\mathcal{D}$.
            \end{theorem}
            \begin{proof}
            $\big[\exists \mathcal{U},\mathcal{V}\in \tau: [\mathcal{U}\cap \mathcal{V}=\emptyset]\land [X=\mathcal{U}\cup \mathcal{V}]\land [\mathcal{U},\mathcal{V}\ne \emptyset]\big]\Rightarrow [X = \mathcal{U}^c\cup \mathcal{V}^c]$ thus, $X$ is the union of disjoint, non-empty closed set. $[\mathcal{C}^c,\mathcal{D}^c\in \tau]\land[\mathcal{C},\mathcal{V}\ne\emptyset]\land[\mathcal{C}\cap \mathcal{D}=\emptyset]\land[\mathcal{C}\cup\mathcal{D}=X]\Rightarrow [X=\mathcal{C}^c\cup\mathcal{D}^c].$ Thus $X$ is disconnected.
            \end{proof}
            \begin{theorem}
            $(X,\tau)$ is disconnected if and only if there is a proper, nonempty set $A\subset X$ that is both open and closed.
            \end{theorem}
            \begin{proof}
            $\big[\exists \mathcal{U},\mathcal{V}\in \tau:[\mathcal{U}\cap \mathcal{V}=\emptyset]\land [X=\mathcal{U}\cup\mathcal{V}]\land[\mathcal{U},\mathcal{V}\ne \emptyset]\big]\Rightarrow [\mathcal{U}^c = \mathcal{V}]\Rightarrow [\mathcal{U}^c\in \tau]$. Thus, $\mathcal{U}$ is open and closed.
            \end{proof}
            \begin{definition}
            A topological space is called connected if and only if it is not disconnected.
            \end{definition}
            \begin{theorem}
            If $f:X\rightarrow Y$ is a continuous function and $X$ is connected, then $f(X)$ is connected.
            \end{theorem}
            \begin{proof}
            For let $f$ be continuous and $X$ be connected. Suppose $f(X)$ is disconnected. Then there are two nonempty open disjoint sets $\mathcal{U}$ and $\mathcal{V}$ such that $f(X) = \mathcal{U}\cap \mathcal{V}$. But then their preimage is open, and thus $X=f^{-1}(\mathcal{U})\cup f^{-1}(\mathcal{V})$, and thus $X$ is disconnected, a contradiction. Thus $f(X)$ is connected.
            \end{proof}
            \begin{definition}
            If $(X,\tau)$ and $(Y,\tau')$ are topological spaces, then the product topology on the set $X\times Y$ is the set $\mathscr{T} = \{\mathcal{U}\times \mathcal{V}:\mathcal{U}\in\tau,\mathcal{V}\in \tau'\}$.
            \end{definition}
            \begin{theorem}
            The product topology is a topology.
            \end{theorem}
            \begin{proof}
            \
            \begin{enumerate}
            \item As $\emptyset \in \tau$ and $\emptyset\in \tau'$, $\emptyset =\emptyset\times \emptyset \in \mathscr{T}$.
            \item If $\mathscr{U}_{\alpha}\in \mathscr{T}$, then $\cup_{\alpha} \mathscr{U}_{\alpha} = \cup_{\alpha} (\mathcal{U}_{\alpha},\mathcal{V}_{\alpha})$. As $\tau$ and $\tau'$ are topologies, $\cup_{\alpha} \mathcal{U} \in \tau$ and $\cup_{\alpha}\mathcal{V}_{\alpha} \in \tau'$. Thus, $\cup_{\alpha}\mathscr{U}_{\alpha} \in \mathscr{T}$.
            \item $\cap_{k=1}^{n} \mathscr{U}_{k} = \cap_{k=1}^{n} (\mathcal{U}_k,\mathcal{V}_k)$. As $\tau$ and $\tau'$ are topologies, $\cap_{k=1}^{n}\mathcal{U}_k \in \tau$ and $\cap_{k=1}^{n}\mathcal{V}_{k} \in \tau'$. Thus $\cap_{k=1}^{n} \mathscr{U}_k \in \mathscr{T}$
            \end{enumerate}
            \end{proof}
            \begin{definition}
            The projection map $\pi_1$ is defined as $\pi_1:X_1\times X_2\rightarrow X_1$ by $(x_1,x_2)\mapsto x_1$. Similarly for $\pi_2$.
            \end{definition}
            \begin{theorem}
            The projection map is continuous.
            \end{theorem}
            \begin{proof}
            Let $\pi_1:X_1\times X_2\rightarrow X_1$ be the projection map, $X\times Y$ having the project topology. Let $\mathcal{U}\underset{Open}\subset X_1$. Then $f^{-1}(\mathcal{U}) = \{(x_1,x_2):x_1\in \mathcal{U}, x_2\in X_2\}$. But $\mathcal{U}$ and $X_2$ are open, and thus $f^{-1}(\mathcal{U})$ is open (In the product topology).
            \end{proof}
            \begin{definition}
            An open mapping is a function $f:X\rightarrow Y$ such that $\mathcal{U}\underset{Open}\subset X\Rightarrow f(\mathcal{U}) \underset{Open}\subset Y$.
            \end{definition}
            \begin{theorem}
            The projection map is an open mapping.
            \end{theorem}
            \begin{proof}
            For let $\mathscr{U}$ be an open set in $X\times Y$ (With the product topology). That is, there are open sets $\mathcal{U}\subset X$ and $\mathcal{V}\subset Y$ such that $\mathscr{U}= \{(x,y):x\in \mathcal{U},y\in \mathcal{V}\}$, Then $\pi_1(\mathscr{U}) =\mathcal{U}$, which is open. Therefore, etc.
            \end{proof}
            \begin{theorem}
            If $X$ and $Y$ are compact, then $X\times Y$ is compact with the product topology.
            \end{theorem}
            \begin{proof}
            For let $\mathscr{O}$ be an open cover of $X\times Y$. Then $\{\pi_X(\mathscr{U}):\mathscr{U}\in \mathscr{O}\}$ is an open cover of $X$ and $\{\pi_{Y}(\mathscr{V}):\mathscr{V}\in \mathscr{O}\}$ is an open cover of $Y$. As $X$ and $Y$ are compact, there exist finite subcovers of each, say $\mathcal{O}_X$ and $\mathcal{O}_Y$. But then $\{\pi_{X}^{-1}(\mathcal{U}):\mathcal{U}\in \mathcal{O}_X\}\cup \{\pi_{Y}^{-1}(\mathcal{V}):\mathcal{V}\in \mathcal{O}_Y\}$ is a finite subcover of $\mathscr{O}$. Thus, $X\times Y$ is compact.
            \end{proof}
            \begin{theorem}
            If $X,Y\subset Z$ are compact, $X\cup Y$ is compact.
            \end{theorem}
            \begin{proof}
            Let $\mathcal{O}$ be an open cover of $X\cup Y$. Then there is a finite subcover of $X$ and a finite subcover of $Y$, and thus the combination of these subcovers is a cover of $X\cup Y$.
            \end{proof}
    \section{Old Notes}
        Elements of a topological space are called points.
        Elements of the topology are called open subset of $X$.
        A neighborhood of a point $x$ is a set $V$ that contains
        an open subste $U$ such that $x\in{U}$. An open neighborhood
        of $x$ is an open set $U$ such that $x\in{U}$.
        \begin{example}
            The chaotic topology on a set
            $X$ is the set $\tau=\{\emptyset,X\}$.
        \end{example}
        \begin{example}
            The discrete topology on a set $X$ is
            $\tau=\mathcal{P}(X)$.
        \end{example}
        \begin{example}
            The Sierpinski topology
            on $\{0,1\}$ is the set
            $\{\emptyset,\{0\},\{0,1\}\}$.
        \end{example}
        \begin{theorem}
            If $T_{\omega}$ is a set of topologies on
            a topological space $X$, then
            $\bigcap{T_{\omega}}$ is a topology on $X$.
        \end{theorem}
        However, the union of topologies may not be a
        topology. A topology $\tau_{1}$ is set to
        be finer than a topology $\tau_{0}$ if
        $\tau_{0}\subset\tau_{1}$. An accumulation point
        of a set $A$ is a point $x$ such that, for
        all open neighborhoods $U$ of $A$,
        $U\cap{A}\ne\emptyset$.
        \begin{theorem}[Bolzano-Weierstrass Theorem]
            If $X$ is a bounded, infinite subset of
            $\mathbb{R}$, then $X$ has at least
            one accumulation point.
        \end{theorem}
        \begin{definition}
            The Euclidean topology on
            $\mathbb{R}$ is the set of
            all open sets in the sense that
            $U$ is open if, for all $x\in{U}$,
            there is an $\varepsilon>0$ such
            that $(x-\varepsilon,x+\varepsilon)\subset{U}$.
        \end{definition}
        \begin{definition}
            A closed subset of a topological space
            $(X,\tau)$ is a set $A$ such that
            $A^{C}\in\tau$.
        \end{definition}
        \begin{theorem}
            The intersection of an arbitrary collection of
            closed sets is closed. The union of finitely
            many closed sets is closed.
        \end{theorem}
        \begin{proof}
            Apply DeMorgan's theorem to the properties
            of a topological space $\tau$.
        \end{proof}
        \begin{definition}
            The closure of a set $A$,
            denoted $\overline{A}$, is the
            intersection of all closed sets
            containing $A$.
        \end{definition}
        \begin{theorem}
            If $A$ is a set in a topological space
            $(X,\tau)$, then $A\subset\overline{A}$.
        \end{theorem}
        There's also something called the derived
        set of a set $A$. The interior of $A$
        is the union of all open subset of $A$.
        The boundary of $A$ is the set difference
        of the closure of $A$ and the interior of
        $A$.
        \begin{theorem}
            If $A$ is a set, then
            the interior of $A$ is equal to
            $(\overline{A^{C}})^{C}$
        \end{theorem}
        \begin{definition}
            A dense subset of a topological
            space $(X,\tau)$ is a set $A$
            such that $\overline{A}=X$.
        \end{definition}
        \begin{definition}
            The neighborhood system of a point
            $x$ in a topological space $(X,\tau)$
            is the set of all neighborhoods of
            $x$.
        \end{definition}
        \begin{definition}
            A sequence in a topological space
            $a_{n}$ converges to a point $a$ if,
            for all open neighborhoods $U$ of $a$,
            there is an $N\in\mathbb{N}$ such that,
            for all $n>N$, $a_{n}\in{U}$.
        \end{definition}
        Limits of sequences in topological spaces are NOT
        necessarily unique. This is different from convergence
        in $\mathbb{R}$, where convergence is always unique.
        \begin{definition}
            The relative topology of a
            topological space $(X,\tau)$ with
            respect to a subset $A\subset{X}$
            is $\tau_{A}=\{A\cap{U}:U\in\tau\}$
        \end{definition}
        \begin{theorem}
            If $(X,\tau)$ is a topological space and
            $A\subset{X}$, then
            $(A,\tau_{A})$ is a topological space.
        \end{theorem}
        $(A,\tau_{A})$ is called a subspace of
        $(X,\tau)$.
        \begin{definition}
            A basis of a topological space
            $(X,\tau)$ is a subset $B$ of
            $\tau$ such that every element
            of $\tau$ is the union of some of the
            elements of $B$.
        \end{definition}
        \begin{theorem}
            A subset $B\subset\tau$ is a basis
            for $\tau$ if and only if for all
            $U\in\tau$ and all $x\in{U}$, there is
            a $V\in{B}$ such that
            $x\in{V}\subset{U}$.
        \end{theorem}
        \begin{theorem}
            If $B$ is a basis of $\tau$, then
            $U$ is open if and only if for all
            $x\in{U}$ there is a $V\in{B}$ such that
            $x\in{V}\subset{U}$.
        \end{theorem}
        \begin{theorem}
            $\mathbb{R}$ has a countable basis.
        \end{theorem}
        \begin{proof}
            For the set of open intervals
            $(p,q)$, where $p$ and $q$ are rational
            numbers, forms a basis for the standard
            topology on $\mathbb{R}$. Moreover, this
            is countable.
        \end{proof}
        \begin{definition}
            If $(X,\tau)$ is a topological space
            and $S\subset\tau$, then $S$ is a subbase
            if a finite intersection of elements of $S$
            forms a base of $\tau$.
        \end{definition}
        \begin{definition}
            A local base for a point
            $x$ in a topological space $(X,\tau)$
            is a set of open neighborhods $B_{x}$ of
            $x$ such that for all open neighborhoods $G$
            of $x$, there is a $G_{x}\in{B_{x}}$ such that
            $x\in{G_{x}}\subset{G}$.
        \end{definition}
        \begin{theorem}
            If $(X,\tau)$ is a topological space, $x\in{X}$,
            and if $B$ is a base for $\tau$, then
            the set of elements $G_{x}$ in $B$ such that
            $x\in{G_{x}}$ is a local base for $x$.
        \end{theorem}
    \section{Product Topology}
    Some preliminaries. Given a set $X$, let
    $\mathcal{P}(X)$ denote the power set of $X$. That is,
    $\mathcal{P}(X)$ is the set of all subsets of $X$.
    \begin{ldefinition}{Topological Space}{Top_Space}
        A topological space is a set $X$ and a topology
        $\tau$ on $X$, denoted $(X,\tau)$.
    \end{ldefinition}
    It is common in the literature of mathematics to drop this
    ordered pair notation, and simply call $X$ a topological space.
    To prevent confusion, we will distinguish between the two:
    $X$ is a set, $(X,\tau)$ is a topological space.
    \par\hfill\par
    The notion of a topological space is a generalization of that
    of a \textit{metric space}. We discard all properties of
    metric spaces, with the exception of the fact that open sets
    are closed under arbitrary unions and finite intersections.
    As such, we call the elements of a topology $\tau$ on a set
    $X$ the \textit{open subsets} of $X$. We wish to talk about
    the \textit{product space} formed by the Cartesian product of
    two sets and their respective topologies. We'll need to define
    the \textit{generated} topology, so we prove the following:
    \begin{theorem}
        \label{thm:Intersec_of_Tops_is_Top}%
        If $X$ is a set, and if $T$ is a set of topologies
        on $X$, then:
        \begin{equation}
            \tau=\bigcap_{t\in{T}}t
        \end{equation}
        Is a topology on $X$.
    \end{theorem}
    \begin{proof}
        First note that, since for all $t\in{T}$, $t$ is a topology,
        it is true that $\emptyset\in{t}$, and thus $\emptyset$ is
        contained in the intersection. Therefore $\emptyset\in\tau$.
        Similarly, $X\in\tau$. Given a subset
        $\mathcal{O}\subseteq\tau$, it is true that
        $\mathcal{O}\subseteq{t}$ for all $t\in{T}$. But for all
        $t\in{T}$, $t$ is a topology on $X$, and therefore the union
        of the elements of $\mathcal{O}$ are contained in $t$, and
        thus this union is contained in $\tau$. Thus, $\tau$ is
        closed to arbitrary unions. Similarly for finite
        intersections. Thus, $\tau$ is a topology on $X$.
    \end{proof}
    \begin{ldefinition}{Generated Topology}{Generated_Topology}
        The topology generated by a subset $S\subseteq\mathcal{P}(X)$
        is the set:
        \begin{equation}
            \tau=\bigcap\{\tau_{S}:\tau_{S}
                \textrm{ is a topology on $X$ and }
                S\subseteq\tau_{S}\}
        \end{equation}
        That is, the smallest topology such that the elements of $S$
        are open.
    \end{ldefinition}
    By Thm.~\ref{thm:Intersec_of_Tops_is_Top} we see
    that the topology generated by some collection of subsets of
    $X$ is indeed a topology on $X$. This notion is similar to
    the one found when one studies measure theory. For
    arbitrary topologies it is often difficult, perhaps
    even impossible, to describe explicitly the elements of the
    topology. Analogously, consider the Borel
    $\sigma\textrm{-Algebra}$ on $\mathbb{R}$. We describe
    this as the $\sigma\textrm{-Algebra}$ generated by
    semi-intervals $[a,b)$. An explicit description of the elements
    of the Borel $\sigma\textrm{-Algebra}$ is almost certainly
    impossible. With this, we can move to product spaces.
    \begin{ldefinition}{Product of Two Topologies}
          {Product_of_Two_Topologies}
        The product of two topological spaces $(X,\tau_{X})$ and
        $(Y,\tau_{Y})$ is the topological space $(X\times{Y},\tau)$,
        where $X\times{Y}$ is the Cartesian product of $X$ and $Y$,
        and where $\tau$ is the topology generated by the sets:
        \begin{equation}
            \mathcal{O}=\big\{\mathcal{U}\times\mathcal{V}:
                \mathcal{U}\in\tau_{X},\mathcal{V}\in\tau_{Y}\big\}
        \end{equation}
    \end{ldefinition}
    It is important to note that we cannot simply set the topology
    $\tau$ to be the set of all sets of the form
    $\mathcal{O}=\big\{\mathcal{U}\times\mathcal{V}\big\}$, where
    $\mathcal{U}\in\tau_{X}$ and $\mathcal{V}\in\tau_{Y}$, for
    this will most likely \textbf{not} be a topology.
    The reason being that it may fail to be closed to unions.
    \begin{figure}[H]
        \centering
        \captionsetup{type=figure}
        \begin{subfigure}[b]{0.49\textwidth}
            \centering
            \begin{tikzpicture}[>=Latex]
                \draw[->, thick] (-0.4, 0) to (4, 0)
                    node [above] {$x$};
                \draw[->, thick] (0, -0.4) to (0, 4)
                    node [right] {$y$};
                \draw (1, -0.1) to (1, 0.1);
                \node at (1, -0.4) {$a$};
                \draw (3, -0.1) to (3, 0.1);
                \node at (3, -0.4) {$b$};
                \draw (-0.1, 1) to (0.1, 1);
                \node at (-0.4, 1) {$c$};
                \draw (-0.1, 3) to (0.1, 3);
                \node at (-0.4, 3) {$d$};
                \draw[fill=cyan, opacity=0.8, draw=white]
                    (1, 1) to (1, 3) to (3, 3) to (3, 1) to cycle;
                \draw[densely dashed] (0, 1) to (3, 1);
                \draw[densely dashed] (0, 3) to (3, 3);
                \draw[densely dashed] (1, 0) to (1, 3);
                \draw[densely dashed] (3, 0) to (3, 3);
            \end{tikzpicture}
            \subcaption{The Open Rectangle $(a,b)\times(c,d)$.}
        \end{subfigure}
        \begin{subfigure}[b]{0.49\textwidth}
            \centering
            \begin{tikzpicture}[>=Latex]
                \draw[->, thick] (-0.4, 0) to (4, 0)
                    node [above] {$x$};
                \draw[->, thick] (0, -0.4) to (0, 4)
                    node [right] {$y$};
                \draw[fill=cyan, opacity=0.8, densely dashed]
                    (1, 1) to (2, 1) to (2, 2) to (3, 2)
                           to (3, 3.5) to (1.5, 3.5)
                           to (1.5, 2) to (1, 2) to cycle;
            \end{tikzpicture}
            \subcaption{A Region That Cannot be Written as
                        $\mathcal{U}\times\mathcal{V}$.}
        \end{subfigure}
        \caption{Examples of Open Subsets of $\mathbb{R}^{2}$.}
        \label{fig:Point_Set_Top_Open_Subsets_R2}
    \end{figure}
    For consider $\mathbb{R}$. The standard topology
    on $\mathbb{R}^{2}$ is constructed by considering the
    collection of all open \textit{rectangles},
    $(a,b)\times(c,d)$. However, the set of open rectangles
    will not, by itself, be a topology on $\mathbb{R}^{2}$.
    For one, the union of two rectangles may not even be
    connected: Consider two disjoint non-empty open rectangles.
    This union will \textbf{not} be a rectangle.
    But even if two open rectangles are not disjoint,
    their union may not be a rectangle. See
    Fig.~\ref{fig:Point_Set_Top_Open_Subsets_R2} for examples.
    As a final example, consider the open unit disc in
    $\mathbb{R}^{2}$. This is the set:
    \begin{equation}
        D^{2}=\{(x,y)\in\mathbb{R}^{2}:x^{2}+y^{2}<1\}
    \end{equation}
    This is not of the form $\mathcal{U}\times\mathcal{V}$ for
    some pair of sets $\mathcal{U},\mathcal{V}\subseteq\mathbb{R}$.
    However, seeing as we've called it the open unit disc, we
    would certainly like it to be open. And indeed it is, for
    it lies in the topology that is \textit{generated}
    by open rectangles.
    \begin{figure}[H]
        \centering
        \captionsetup{type=figure}
        \begin{tikzpicture}[>=Latex]
            \draw[<->, thick] (-3.3, 0) to (3.3, 0) node [above] {$x$};
            \draw[<->, thick] (0, -3.3) to (0, 3.3) node [right] {$y$};
            \draw[densely dashed] (0, 0) circle (1in);

            % First Layer
            \draw[fill=cyan, opacity=0.6, densely dashed]
                (0.7071in, 0.7071in) to (-0.7071in, 0.7071in)
                                     to (-0.7071in, -0.7071in)
                                     to (0.7071in, -0.7071in)
                                     to cycle;
            
            % Second Layer
            \draw[fill=green, opacity=0.5, densely dashed]
                (0.68in, 0.3535in) to (0.935in, 0.3535in)
                                   to (0.935in, -0.3535in)
                                   to (0.68in, -0.3535in)
                                   to cycle;
            \draw[fill=green, opacity=0.5, densely dashed]
                (-0.68in, 0.3535in) to (-0.935in, 0.3535in)
                                    to (-0.935in, -0.3535in)
                                    to (-0.68in, -0.3535in)
                                    to cycle;
            \draw[fill=green, opacity=0.5, densely dashed]
                (0.3535in, 0.68in) to (0.3535in, 0.935in)
                                   to (-0.3535in, 0.935in)
                                   to (-0.3535in, 0.68in)
                                   to cycle;
            \draw[fill=green, opacity=0.5, densely dashed]
                (0.3535in, -0.68in) to (0.3535in, -0.935in)
                                    to (-0.3535in, -0.935in)
                                    to (-0.3535in, -0.68in)
                                    to cycle;

            % Third Layer.
            \draw[fill=orange, opacity=0.6, densely dashed]
                (0.68in, 0.3535in) to (0.8212in, 0.3535in)
                                   to (0.8212in, 0.5705in)
                                   to (0.68in, 0.5707in)
                                   to cycle;
            \draw[fill=orange, opacity=0.6, densely dashed]
                (0.68in, -0.3535in) to (0.8212in, -0.3535in)
                                    to (0.8212in, -0.5705in)
                                    to (0.68in, -0.5707in)
                                    to cycle;
            \draw[fill=orange, opacity=0.6, densely dashed]
                (-0.68in, -0.3535in) to (-0.8212in, -0.3535in)
                                     to (-0.8212in, -0.5705in)
                                     to (-0.68in, -0.5707in)
                                     to cycle;
            \draw[fill=orange, opacity=0.6, densely dashed]
                (-0.68in, 0.3535in) to (-0.8212in, 0.3535in)
                                    to (-0.8212in, 0.5705in)
                                    to (-0.68in, 0.5707in)
                                    to cycle;
            \draw[fill=orange, opacity=0.6, densely dashed]
                (0.3535in, 0.68in) to (0.3535in, 0.8212in)
                                   to (0.5705in, 0.8212in)
                                   to (0.5707in, 0.68in)
                                   to cycle;
            \draw[fill=orange, opacity=0.6, densely dashed]
                (0.3535in, -0.68in) to (0.3535in, -0.8212in)
                                    to (0.5705in, -0.8212in)
                                    to (0.5707in, -0.68in)
                                    to cycle;
            \draw[fill=orange, opacity=0.6, densely dashed]
                (-0.3535in, 0.68in) to (-0.3535in, 0.8212in)
                                    to (-0.5705in, 0.8212in)
                                    to (-0.5707in, 0.68in)
                                    to cycle;
            \draw[fill=orange, opacity=0.6, densely dashed]
                (-0.3535in, -0.68in) to (-0.3535in, -0.8212in)
                                     to (-0.5705in, -0.8212in)
                                     to (-0.5707in, -0.68in)
                                     to cycle;

            % Fourth Layer
            \draw[fill=red, opacity=0.5, densely dashed]
                (0.2in, 0.93in) to (0.2in, 0.9797in)
                                to (-0.2in, 0.9797in)
                                to (-0.2in, 0.93in)
                                to cycle;
            \draw[fill=red, opacity=0.5, densely dashed]
                (0.2in, -0.93in) to (0.2in, -0.9797in)
                                 to (-0.2in, -0.9797in)
                                 to (-0.2in, -0.93in)
                                 to cycle;
            \draw[fill=red, opacity=0.5, densely dashed]
                (0.93in, 0.2in) to (0.9797in, 0.2in)
                                to (0.9797in, -0.2in)
                                to (0.93in, -0.2in)
                                to cycle;
            \draw[fill=red, opacity=0.5, densely dashed]
                (-0.93in, 0.2in) to (-0.9797in, 0.2in)
                                 to (-0.9797in, -0.2in)
                                 to (-0.93in, -0.2in)
                                 to cycle;

            % Fifth Layer
            \draw[fill=blue, opacity=0.6, densely dashed]
                (0.82in, 0.3535in) to (0.8781in, 0.3535in)
                                   to (0.8781in, 0.4784in)
                                   to (0.82in, 0.4784in)
                                   to cycle;
            \draw[fill=blue, opacity=0.6, densely dashed]
                (0.82in, -0.3535in) to (0.8781in, -0.3535in)
                                    to (0.8781in, -0.4784in)
                                    to (0.82in, -0.4784in)
                                    to cycle;
            \draw[fill=blue, opacity=0.6, densely dashed]
                (-0.82in, 0.3535in) to (-0.8781in, 0.3535in)
                                    to (-0.8781in, 0.4784in)
                                    to (-0.82in, 0.4784in)
                                    to cycle;
            \draw[fill=blue, opacity=0.6, densely dashed]
                (-0.82in, -0.3535in) to (-0.8781in, -0.3535in)
                                     to (-0.8781in, -0.4784in)
                                     to (-0.82in, -0.4784in)
                                     to cycle;
            \draw[fill=blue, opacity=0.6, densely dashed]
                (0.3535in, 0.82in) to (0.3535in, 0.8781in)
                                   to (0.4784in, 0.8781in)
                                   to (0.4784in, 0.82in)
                                   to cycle;
            \draw[fill=blue, opacity=0.6, densely dashed]
                (0.3535in, -0.82in) to (0.3535in, -0.8781in)
                                    to (0.4784in, -0.8781in)
                                    to (0.4784in, -0.82in)
                                    to cycle;
            \draw[fill=blue, opacity=0.6, densely dashed]
                (-0.3535in, -0.82in) to (-0.3535in, -0.8781in)
                                     to (-0.4784in, -0.8781in)
                                     to (-0.4784in, -0.82in)
                                     to cycle;
            \draw[fill=blue, opacity=0.6, densely dashed]
                (-0.3535in, 0.82in) to (-0.3535in, 0.8781in)
                                    to (-0.4784in, 0.8781in)
                                    to (-0.4784in, 0.82in)
                                    to cycle;

            % Sixth Layer
            \draw[fill=yellow, opacity=0.6, densely dashed]
                (0.68in, 0.5705in) to (0.7641in, 0.5705in)
                                   to (0.7641in, 0.645in)
                                   to (0.68in, 0.645in)
                                   to cycle;
            \draw[fill=yellow, opacity=0.6, densely dashed]
                (0.68in, -0.5705in) to (0.7641in, -0.5705in)
                                    to (0.7641in, -0.645in)
                                    to (0.68in, -0.645in)
                                    to cycle;
            \draw[fill=yellow, opacity=0.6, densely dashed]
                (-0.68in, -0.5705in) to (-0.7641in, -0.5705in)
                                     to (-0.7641in, -0.645in)
                                     to (-0.68in, -0.645in)
                                     to cycle;
            \draw[fill=yellow, opacity=0.6, densely dashed]
                (-0.68in, 0.5705in) to (-0.7641in, 0.5705in)
                                    to (-0.7641in, 0.645in)
                                    to (-0.68in, 0.645in)
                                    to cycle;
            \draw[fill=yellow, opacity=0.6, densely dashed]
                (0.5705in, 0.68in) to (0.5705in, 0.7641in)
                                   to (0.645in, 0.7641in)
                                   to (0.645in, 0.68in)
                                   to cycle;
            \draw[fill=yellow, opacity=0.6, densely dashed]
                (0.5705in, -0.68in) to (0.5705in, -0.7641in)
                                    to (0.645in, -0.7641in)
                                    to (0.645in, -0.68in)
                                    to cycle;
            \draw[fill=yellow, opacity=0.6, densely dashed]
                (-0.5705in, -0.68in) to (-0.5705in, -0.7641in)
                                     to (-0.645in, -0.7641in)
                                     to (-0.645in, -0.68in)
                                     to cycle;
            \draw[fill=yellow, opacity=0.6, densely dashed]
                (-0.5705in, 0.68in) to (-0.5705in, 0.7641in)
                                    to (-0.645in, 0.7641in)
                                    to (-0.645in, 0.68in)
                                    to cycle;
        \end{tikzpicture}
        \caption{Tiling of the Open Unit Disc by Rectangles.}
        \label{fig:Point_Set_Top_Unit_Disc_Rect_Tiling}
    \end{figure}
    Note that, in the tiling of the unit disc shown in
    Fig.~\ref{fig:Point_Set_Top_Unit_Disc_Rect_Tiling}, many of
    the rectangles overlap. This is to avoid excluding any points
    within the circle, and to give a clear picture. Such a tiling
    is allowed in the topology generated by open rectangles, since
    \textit{arbitrary} unions are allowed. With this figure we have
    some evidence that the topology generated by open rectangles is
    most likely the same as the standard topology on $\mathbb{R}^{2}$.
    That is: The set of all sets $\mathcal{U}\subseteq\mathbb{R}^{2}$
    such that, for all $\mathbf{x}\in\mathcal{U}$, there is an $r>0$
    such that, for all $\mathbf{y}\in\mathbb{R}^{2}$ such that
    $\norm{\mathbf{x}-\mathbf{y}}_{2}<r$, it is true that
    $\mathbf{y}\in\mathcal{U}$. Here $\norm{\cdot}_{2}$ denotes the
    standard Euclidean norm, where we compute length by
    invoking the Pythagorean formula. Rather than carrying out
    complicated computations, we can simply note that open
    balls in the $\norm{\cdot}_{2}$ norm are of the form:
    \begin{equation}
        B_{r}^{(\mathbb{R}^{2},\norm{\cdot}_{2})}(\mathbf{x})
            =\big\{\mathbf{y}\in\mathbb{R}^{2}:
                \norm{\mathbf{x}-\mathbf{y}}_{2}<r\big\}
    \end{equation}
    These are circles centered at $\mathbf{x}$.
    Contrast that with open balls in the $\norm{\cdot}_{\infty}$
    metric:
    \begin{equation}
        B_{r}^{(\mathbb{R}^{2},\norm{\cdot}_{\infty})}(\mathbf{x})
            =\big\{\mathbf{y}\in\mathbb{R}^{2}:
                \max\{|x_{1}-y_{1}|,|x_{2}-y_{2}|\}<r\big\}
    \end{equation}
    These are just squares centered at $\mathbf{x}$. And we know
    that the $\norm{\cdot}_{2}$ and $\norm{\cdot}_{\infty}$ metrics
    are equivalent, so these topologies must be the same. Thus
    we've found a slightly more inconvenient way of describing
    the topology on $\mathbb{R}^{2}$. The plus side is that
    this alternative notion generalizes to $X\times{Y}$
    when $(X,\tau_{X})$ and $(Y,\tau_{Y})$ are more general
    topological spaces.
    \par\hfill\par
    In defining the product topology of two topological spaces,
    we used the familiar notion of a Cartesian product. Elements
    of the Cartesian product $X\times{Y}$ are ordered pairs
    $(x,y)$, where $x\in{X}$ and $y\in{Y}$. We can continue to
    ordered triples $(x,y,z)$ and the general $n$ tuple
    $(x_{1},\dots,x_{n})$ and similarly define the product
    topology of $n$ topological spaces
    $(X_{1},\tau_{1}),\dots(X_{n},\tau_{n})$.
    But what if we wanted to define an \textit{infinite} product
    of infinitely many spaces? If the product is countable, we
    have some intuition for we can think of \textit{infinite} tuples
    $(x_{1},\dots,x_{n},\dots)$, but this lacks clarity.
    Rather, let's go back to the product of two topological
    spaces and redefine it. Let $\mathbb{Z}_{2}=\{1,2\}$
    and define:
    \begin{equation}
        \prod_{i=1}^{2}X_{i}=\{f:\mathbb{Z}_{2}\rightarrow
            \bigcup_{k=1}^{2}X_{k}:
            f(1)\in{X}_{1},f(2)\in{X}_{2}\}
    \end{equation}
    That is, the set of all functions from $\mathbb{Z}_{2}$ into
    $X_{1}\cup{X}_{2}$ with the property that 1 maps into $X_{1}$
    and $2$ maps into $X_{2}$. There is a clear bijection between
    this new thing and $X_{1}\times{X}_{2}$, simply map
    $(x,y)$ to the function $f$, where $f(1)=x$ and $f(2)=y$.
    But now we have a definition that really didn't depend
    on how many products we were making. Let
    $\mathbb{Z}_{n}=\{1,\dots,n\}$, and let $X_{1},\dots,X_{n}$
    be sets. We can then define:
    \begin{equation}
        \prod_{i\in\mathbb{Z}_{n}}X_{i}=
            \{f:\mathbb{Z}_{n}\rightarrow
                \bigcup_{k\in\mathbb{Z}_{n}}X_{i}:
                \forall_{i\in\mathbb{Z}_{n}},f(i)\in{X}_{i}\}
    \end{equation}
    And we can go further, defining the product for any collection
    of sets. Let's first introduce some notation. An indexing set
    for a collection of sets is some set $I$ such that we can
    write all of the sets in our collection as $X_{i}$, for
    $i\in{I}$. To improve rigor, let's say that an indexing set
    for a collection of sets $\mathcal{O}$ is some set $I$ such
    that there is a surjective function $X:I\rightarrow\mathcal{O}$,
    and let's write $X(i)=X_{i}$, for all $i\in{I}$. That is,
    for all $i\in{I}$, $X_{i}$ is a set in $\mathcal{O}$.
    We can now define the general product of sets.
    \begin{ldefinition}{Product of Sets}{Product_Set}
        The product of a collection of sets indexed by a set $I$
        is the set:
        \begin{equation}
            \prod_{i\in{I}}X_{i}=
            \{f:I\rightarrow\bigcup_{i\in{I}}X_{i}:
                \forall_{i\in{I}},f(i)\in{X}_{i}\}
        \end{equation}
    \end{ldefinition}
    This notion is well-defined for arbitrary products, countable
    or not. It is important to note that the elements of the
    product space are \textit{functions}.
    \begin{lexample}
        Nothing in the definition of an indexing set requires
        $\mathcal{O}$ to contain many sets, so let
        $\mathcal{O}=\{\mathbb{R}\}$ and let $I=\mathbb{N}$.
        Then the product is simply:
        \begin{equation}
            \prod_{n\in\mathbb{N}}\mathbb{R}=
                \{a:\mathbb{N}\rightarrow\mathbb{R}\}
        \end{equation}
        That is, the set of all sequences of real numbers.
        Thus, the countable product of $\mathbb{R}$ can be
        thought of in two ways: The set of all
        \textit{infinite} tuples
        $(x_{1},\dots,x_{n},\dots)$, or the set of all
        \textit{sequences} of real numbers.
    \end{lexample}
    All of this has been purely set theoretic: There is no topology
    yet. Given a collection of topological spaces
    $(X_{i},\tau_{i})$, is there a good topology to place on the
    product? That is, can we form a nice product topological space?
    There are two well established ways to do this:
    The \textit{obvious} way, and the \textit{correct} way.
    We first let intuition lead us astray, and define the obvious
    answer: The Box Topology.
    \begin{ldefinition}{Box Topology}{Box_Topology}
        The box topology on a collection of topological spaces
        $(X_{i},\tau_{i})$ indexed by a set $I$ is the topology
        $\tau$ on the set $X$ where:
        \begin{equation}
            X=\prod_{i\in{I}}X_{i}
        \end{equation}
        And where $\tau$ is the topology generated by the sets:
        \begin{equation}
            \mathcal{U}=
                \big\{\prod_{i\in{I}}\mathcal{U}_{i}:
                    \mathcal{U}_{i}\in\tau_{i}\big\}
        \end{equation}
        That is, $\tau$ is generated by all of the open sets
        in all of the $X_{i}$.
    \end{ldefinition}
    This is precisely what we did for $\mathbb{R}^{2}$.
    We took the topology to be the one generated by all of the open
    rectangles in the plane. Unfortunately, when the product is
    infinite, the box topology is horrible. Some problems with
    the box topology:
    \begin{enumerate}
        \item The product of compact spaces need not be compact.
        \item The product of connected spaces need not be connected.
        \item The product of metric spaces need not be metrizable.
    \end{enumerate}
    Moreover, some functions that \textit{look} continuous, and that
    we would obviously want to be continuous, are not.
    For example, let $X$ be the set of sequences in $\mathbb{R}$,
    and define $f:\mathbb{R}\rightarrow{X}$ by mapping
    $x$ to the sequence $a_{n}=x$, $n\in\mathbb{N}$. That is:
    \begin{equation}
        f(x)=x,x,x,x,\dots,x,x,\dots
    \end{equation}
    This function is \textit{nowhere} continuous in the box topology.
    So now we devise a plan to make a \textit{better} topology
    with the following property:
    Suppose $g:\mathbb{R}\rightarrow{X}$,
    where $X$ is again the space of real-valued sequences, and
    suppose $g$ is of the form:
    \begin{equation}
        g(x)=g_{1}(x),g_{2}(x),\dots,g_{n}(x),\dots
    \end{equation}
    Where $g_{k}$ is continuous for all $k\in\mathbb{N}$. We
    \textbf{require} that $g$ be continuous in the product space.
    We could simply make $\tau$ be the chaotic topology,
    $\tau=\{\emptyset,X\}$, but then \textit{every} function
    $f:A\rightarrow{X}$ is continuous, for \textit{any}
    topological space $(A,\tau_{A})$, and this is rather boring.
    So we try another approach. Given a collection of topological
    spaces $(X_{i},\tau_{i})$, we require that the product space
    $(X,\tau)$ is such that all of the projection mappings
    $p_{i}:X\rightarrow{X}_{i}$ are continuous. The projection
    mappings can be defined set theoretically using the notation
    we've developed. Given a product $X$ of sets $X_{i}$,
    the projection mapping $p_{i}:X\rightarrow{X}_{i}$ is simply:
    \begin{equation}
        p_{i}(x)=x(i)
    \end{equation}
    This looks strange, but remember that
    we've defined the product space to be a set of
    \textit{functions}, and therefore $x\in{X}$ is a function.
    Thus, the $i^{th}$ projection mapping simply
    evaluates these functions in the $i^{th}$ coordinate.
    \par\hfill\par
    In the search for a topology on the product set $X$ that makes
    all of the projection mappings $p_{i}$ continuous, we could
    simply take $\tau=\mathcal{P}(X)$. Then, for \textit{any}
    topological space $(A,\tau_{A})$, and for \textit{every}
    function $f:X\rightarrow{A}$, $f$ is continuous.
    This is overkill and we see that this is larger than the
    box topology. So all of the problems with the box topology still
    exist! So, we require that $\tau$ is the \textit{smallest}
    such topology. We now define the initial topology.
    \begin{ldefinition}{Initial Topology}{Initial_Topology}
        The initial topology on a set $X$ generated by
        a set of functions $f_{i}$ from $X$ to topological
        spaces $(X_{i},\tau_{i})$ is the set:
        \begin{equation}
            \tau=\bigcap\big\{\tau_{X}:
                \tau_{X}\textrm{ is a topology on $X$ and }
                \forall_{i\in{I}},f_{i}
                \textrm{ is continuous.}\big\}
        \end{equation}
    \end{ldefinition}
    This collection is non-empty, since $\mathcal{P}(X)$ is contained
    in it, and by Thm.~\ref{thm:Intersec_of_Tops_is_Top},
    $\tau$ is a topology on $X$. We now define the product topology.
    \begin{ldefinition}{Product Topology}
        The product topology on a set $X$ defined as the
        product of topological spaces $(X_{i},\tau_{i})$
        indexed over $I$:
        \begin{equation}
            X=\prod_{i\in{I}}X_{i}
        \end{equation}
        Is the initial topology defined by the set:
        \begin{equation}
            \mathscr{F}=
            \{p_{i}:X\rightarrow{X}_{i},p_{i}(x)=x(i)\}
        \end{equation}
        That is, the set of projection mappings.
    \end{ldefinition}
    In this construction one might have noted that the
    projection mappings are continuous in the box topology.
    Thus one might very reasonably ask if the product topology
    and the box topology are the same thing. And indeed, for
    a \textit{finite} product, they are! This makes sense, for in
    $\mathbb{R}^{2}$ we can think of the topology generated by
    rectangles, or the topology generated by the projection mappings,
    and they are the same. What's crucial is that they differ for
    infinite products.
    \begin{ltheorem}{Product Topology Basis Theorem}
        If $(X,\tau)$ is the product topological space formed by
        the topological spaces $(X_{i},\tau_{i})$, indexed by
        a set $I$, then:
        \begin{equation}
            \tau=\bigcup\prod_{i\in{I}}
                \big\{\mathcal{U}_{i}\in\tau_{i}:
                    \mathcal{U}_{i}=X_{i}
                    \textrm{ for all but finitely many sets.}
                \big\}
        \end{equation}
        That is, $\tau$ is the set of all products of open sets
        $\mathcal{U}_{i}$, such that all but finitely many of
        the $\mathcal{U}_{i}$ are the entire space $X_{i}$.
    \end{ltheorem}
    \par\hfill\par
    This seems confusing, so we illustrate with some pictures. What's
    important to note is that, if the product is infinite, then
    the box topology and the product topology differ. To see this,
    in the box topology we allowed \textit{all} products of open
    sets, whereas now we only allow the product of open sets
    $\mathcal{U}_{i}$ where, for all but finitely many $i$, we
    have $\mathcal{U}_{i}=X_{i}$. It should then be clear that,
    if $\tau_{B}$ is the box topology, and $\tau_{P}$ is the
    product topology, then $\tau_{P}\subseteq\tau_{B}$, and for
    infinite products $\tau_{P}$ is a proper subset.
    \par\hfill\par
    Let's dumb down the theorem a bit, and imagine again a world
    where $X=Y=\mathbb{R}$. Let's consider the topology generated
    by sets $\mathcal{U}\times\mathcal{V}$, where $\mathcal{U}$
    is an open subset of $\mathbb{R}$, and
    $\mathcal{V}=\mathbb{R}$. That is, rather than allowing
    the product to be over finitely many arbitrary open sets,
    we allow it to be over one, and it must be in the $x$ axis.
    In doing this we can get a sense of what the product topology
    might look like.
    \begin{figure}[H]
        \centering
        \captionsetup{type=figure}
        \begin{subfigure}[b]{0.49\textwidth}
            \centering
            \begin{tikzpicture}[>=Latex]
                \draw[->, thick] (-0.2, 0) to (5, 0) node [above] {$x$};
                \draw[->, thick] (0, -1) to (0, 5) node [right] {$y$};
                \draw[fill=cyan, opacity=0.5, draw=white]
                    (2, -1) to (2, 5) to (4, 5) to (4, -1) to cycle;
                \draw (2, -0.1) to (2, 0.1);
                \node at (2, -0.4) [left] {$a$};
                \draw (4, -0.1) to (4, 0.1);
                \node at (4, -0.4) [right] {$b$};
                \draw[densely dashed] (2, -1) to (2, 5);
                \draw[densely dashed] (4, -1) to (4, 5);
            \end{tikzpicture}
            \subcaption{Sets of the Form $\mathcal{U}\times\mathbb{R}$.}
        \end{subfigure}
        \begin{subfigure}[b]{0.49\textwidth}
            \begin{tikzpicture}[>=Latex]
                \draw[->, thick] (-1, 0) to (4, 0) node [above] {$x$};
                \draw[->, thick] (0, -1) to (0, 5) node [right] {$y$};
                \draw[fill=cyan, opacity=0.5, draw=white]
                    (-1, 2) to (4, 2) to (4, 4) to (-1, 4) to cycle;
                \draw (-0.2, 2) to (0.1, 2);
                \node at (-0.4, 2) [above] {$c$};
                \draw (-0.1, 4) to (0.1, 4);
                \node at (-0.4, 4) [below] {$d$};
                \draw[densely dashed] (-1, 2) to (4, 2);
                \draw[densely dashed] (-1, 4) to (4, 4);
            \end{tikzpicture}
            \subcaption{Sets of the Form $\mathbb{R}\times\mathcal{V}$.}
        \end{subfigure}
        \caption{Strips in the Plane.}
        \label{fig:Point_Set_Topology_Strips_in_R2}
    \end{figure}
    We can do the same thing and consider sets of the form
    $\mathbb{R}\times\mathcal{V}$, where $\mathcal{V}$ is open
    in $\mathbb{R}$. Recall that open sets in $\mathbb{R}$ are
    intervals and arbitrary collections of intervals. Using this, we
    see that the topology generated by $\mathcal{U}\times\mathbb{R}$
    is the collection of all open vertical \textit{strips},
    and $\mathbb{R}\times\mathcal{V}$ form the horizontal strips.
    See Fig.~\ref{fig:Point_Set_Topology_Strips_in_R2}.
    We expand this game to $\mathbb{R}^{3}$, and think of sets
    of the form $\mathcal{U}\times\mathcal{V}\times\mathbb{R}$,
    or any permutation of the three coordinates.
    See Fig.~\ref{fig:Point_Set_Top_Blocks_in_R3}
    \begin{figure}[H]
        \centering
        \captionsetup{type=figure}
        \begin{tikzpicture}[>=Latex]
            \draw[->, thick] (0, 0, 0) to (4, 0, 0) node [above] {$x$};
            \draw[->, thick] (0, 0, 0) to (0, 4, 0) node [right] {$y$};
            \draw[->, thick] (0, 0, 0) to (0, 0, 8) node [left] {$z$};
            \node at (1.4, -0.15, 0) {$a$};
            \node at (3.6, -0.15, 0) {$b$};
            \node at (0, 1, 0.4) {$c$};
            \node at (0, 3, 0.4) {$d$};
            \draw[densely dashed] (1.5, 0, 0) to (1.5, 3, 0);
            \draw[densely dashed] (3.5, 0, 0) to (3.5, 3, 0);
            \draw[densely dashed] (0, 1, 0) to (3.5, 1, 0);
            \draw[densely dashed] (0, 3, 0) to (3.5, 3, 0);
            \draw[densely dashed] (1.5, 3, 0) to (1.5, 3, 6);
            \draw[densely dashed] (3.5, 3, 0) to (3.5, 3, 6);
            \draw[densely dashed] (1.5, 1, 0) to (1.5, 1, 6);
            \draw[densely dashed] (3.5, 1, 0) to (3.5, 1, 6);
            \draw[densely dashed] (1.5, 3, 6) to (1.5, 0, 6);
            \draw[densely dashed] (3.5, 3, 6) to (3.5, 0, 6);
            \draw[densely dashed] (0, 1, 6) to (3.5, 1, 6);
            \draw[densely dashed] (0, 3, 6) to (3.5, 3, 6);
            \draw[densely dashed] (0, 0, 6) to (4, 0, 6);
            \draw[densely dashed] (3.5, 0, 0) to (3.5, 0, 8);
            \draw[densely dashed] (0, 0, 6) to (0, 4, 6);
            \draw[fill=cyan, opacity=0.5, draw=none]
                (1.5, 1, 6) to (3.5, 1, 6)
                            to (3.5, 3, 6)
                            to (1.5, 3, 6)
                            to cycle;
            \draw[fill=cyan, opacity=0.5, draw=none]
                (3.5, 1, 6) to (3.5, 3, 6)
                            to (3.5, 3, 0)
                            to (3.5, 1, 0)
                            to cycle;
            \draw[fill=cyan, opacity=0.5, draw=none]
                (1.5, 3, 6) to (1.5, 3, 0)
                            to (3.5, 3, 0)
                            to (3.5, 3, 6)
                            to cycle;
        \end{tikzpicture}
        \caption{Blocks in Space.}
        \label{fig:Point_Set_Top_Blocks_in_R3}
    \end{figure}
    The product topology has the following wonderful features:
    \begin{enumerate}
        \item The product of compact topological spaces is compact.
        \item The product of connected spaces is connected.
        \item The product of metric spaces is metrizable.
    \end{enumerate}
\end{document}