%------------------------------------------------------------------------------%
\documentclass{article}                                                        %
%------------------------------Preamble----------------------------------------%
\makeatletter                                                                  %
    \def\input@path{{../../}}                                                  %
\makeatother                                                                   %
%---------------------------Packages----------------------------%
\usepackage{geometry}
\geometry{b5paper, margin=1.0in}
\usepackage[T1]{fontenc}
\usepackage{graphicx, float}            % Graphics/Images.
\usepackage{natbib}                     % For bibliographies.
\bibliographystyle{agsm}                % Bibliography style.
\usepackage[french, english]{babel}     % Language typesetting.
\usepackage[dvipsnames]{xcolor}         % Color names.
\usepackage{listings, lstlinebgrd}      % Verbatim-Like Tools.
\usepackage{mathtools, esint, mathrsfs} % amsmath and integrals.
\usepackage{amsthm, amsfonts}           % Fonts and theorems.
\usepackage{tabularx}
\usepackage{tcolorbox}                  % Frames around theorems.
\usepackage{upgreek}                    % Non-Italic Greek.
\usepackage{paracol}                    % Two-column styling.
\usepackage{wrapfig}                    % Wrap text around figure.
\usepackage{fmtcount, etoolbox}         % For the \book{} command.
\usepackage[newparttoc]{titlesec}       % Formatting chapter, etc.
\usepackage{titletoc}                   % Allows \book in toc.
\usepackage[nottoc]{tocbibind}          % Bibliography in toc.
\usepackage[titles]{tocloft}            % ToC formatting.
\usepackage{multicol, enumitem}         % Multi-column/enumerate.
\usepackage{import}                     % Import external files.
\usepackage{pgfplots, tikz}             % Drawing/graphing tools.
\usetikzlibrary{
    calc,                   % Calculating right angles and more.
    angles,                 % Drawing angles within triangles.
    arrows.meta,            % Latex and Stealth arrows.
    quotes,                 % Adding labels to angles.
    positioning,            % Relative positioning of nodes.
    decorations.markings,   % Adding arrows in the middle of a line.
    patterns,
    arrows,
    shapes,
    shapes.geometric,
    cd,
    hobby,
    babel
}                                       % Libraries for tikz.
\pgfplotsset{compat=1.9}                % Version of pgfplots.
\usepackage[font=scriptsize,
            labelformat=simple,
            labelsep=colon]{subcaption} % Subfigure captions.
\usepackage[font={scriptsize},
            hypcap=true,
            labelsep=colon]{caption}    % Figure captions.
\usepackage{hyperref}                   % Allows for hyperlinks.
\hypersetup{
    colorlinks=true,
    linkcolor=blue,
    filecolor=magenta,
    urlcolor=Cerulean,
    citecolor=SkyBlue
}                           % Colors for hyperref.
\usepackage[toc,acronym,nogroupskip]{glossaries} % Glossaries and acronyms.
\usepackage[subpreambles=false]{standalone}      % Complileable sub files.

% Various font stuff from kiwi.
% Use this for Times text and Computer Modern math
%\usepackage{times}

% Quite nice
%\usepackage[charter, greekfamily=, greekuppercase=italicized]{mathdesign}
%\usepackage[utopia, greekuppercase=italicized]{mathdesign}    % Math is narrower

% Use this for Times text and math
%\usepackage{newtxtext}
%\usepackage[libertine,cmintegrals]{newtxmath}
%\usepackage{fix-cm}

%\usepackage{txfontsb}
% or
%\usepackage{mathptmx}

%\usepackage[scaled=0.92]{helvet}
%\renewcommand{\rmdefault}{ptm}

%\usepackage{mathpazo}    % add possibly `sc` and `osf` options
%\usepackage{eulervm}

%\usepackage{fourier}
%\renewcommand{\rmdefault}{ptm}
%\usepackage{mathptm}

%\usepackage{fontspec}
%\setmainfont{lmodern}

%\usepackage[varg]{txfonts}
%\usepackage{fouriernc}
%\usepackage{mathpazo}

%\usepackage{bookman}
%\usepackage[scaled]{uarial}
%\usepackage[scaled]{helvet}
%\renewcommand*\familydefault{\sfdefault}
%\usepackage[math]{anttor}

%\newcommand\fgeorgia{\fontfamily{jvn}\selectfont}
%\newcommand\ftimes{\fontfamily{ptm}\selectfont}
%\newcommand\fhelvetica{\fontfamily{phv}\selectfont}
%\newcommand\fcourier{\fontfamily{pcr}\selectfont}
%\newcommand\fbookman{\fontfamily{pbk}\selectfont}
%\newcommand\fnewcentury{\fontfamily{pnc}\selectfont}
%\newcommand\fpalatino{\fontfamily{ppl}\selectfont}
%\newcommand\favantgarde{\fontfamily{pag}\selectfont}
%\newcommand\fnormal{\normalfont}
%\newcommand\fsize[1]{\ifnum#1>0\fontsize{#1}{#1}\selectfont\else\normalsize\fi}
%------------------------Theorem Styles-------------------------%
% Define theorem style for default spacing and normal font.
\newtheoremstyle{normal}
    {\topsep}               % Amount of space above the theorem.
    {\topsep}               % Amount of space below the theorem.
    {}                      % Font used for body of theorem.
    {}                      % Measure of space to indent.
    {\bfseries}             % Font of the header of the theorem.
    {}                      % Punctuation between head and body.
    {.5em}                  % Space after theorem head.
    {}

% Define theorem style for default spacing with italicized font.
\newtheoremstyle{normalit}{\topsep}{\topsep}
                {\itshape}{}{\bfseries}{}{.5em}{}

% Italic header environment.
\newtheoremstyle{thmit}{\topsep}{\topsep}{}{}{\itshape}{}{0.5em}{}

% Define italicized environments.
\theoremstyle{normalit}
\newtheorem{theorem}{Theorem}[section]
\newtheorem{lemma}{Lemma}[section]
\newtheorem{corollary}{Corollary}[section]
\newtheorem{proposition}{Proposition}[section]
\newtheorem*{theorem*}{Theorem}

% Define environments with italic headers.
\theoremstyle{thmit}
\newtheorem*{solution}{Solution}
\newtheorem*{fsolution}{Solution}

% Define default environments.
\theoremstyle{normal}
\newtheorem{example}{Example}[section]
\newtheorem{definition}{Definition}[section]
\newtheorem{problem}{Problem}[section]
\newtheorem{question}{Question}[section]
\newtheorem{remark}{Remark}[section]
\newtheorem{properties}{Properties}[section]
\newtheorem{notation}{Notation}[section]
\newtheorem{axiom}{Axiom}[section]
\newtheorem*{properties*}{Properties}
\newtheorem*{remark*}{Remark}
\newtheorem*{definition*}{Definition}
\theoremstyle{plain}

% Define framed environment.
\tcbuselibrary{most}
\newtcbtheorem[use counter*=theorem]{ftheorem}{Theorem}%
    {colback=green!5,colframe=green!35!black,
     fonttitle=\bfseries\upshape}{th}

\newtcbtheorem[use counter*=example]{fdefinition}{Definition}%
    {fonttitle=\bfseries\upshape,
     colback=blue!5!white,colframe=blue!75!black}{def}

\newtcbtheorem[use counter*=example]{fexample}{Example}%
    {fonttitle=\bfseries\upshape,
     colback=red!5!white,colframe=red!75!black}{ex}

\newtcbtheorem[use counter*=notation]{fnotation}{Notation}%
    {fonttitle=\bfseries\upshape,
     colback=SeaGreen!5!white,colframe=SeaGreen!75!black}{ex}

\newtcbtheorem[use counter*=corollary]{fcorollary}{Corollary}%
    {fonttitle=\bfseries\upshape,
     colback=Orchid!5!white,colframe=Orchid!75!black}{ex}

\newenvironment{bproof}{\textit{Proof.}}{\hfill$\square$}
\tcolorboxenvironment{bproof}{blanker,breakable,left=5mm,
                             before skip=10pt,after skip=10pt,
                             borderline west={1mm}{0pt}{red}}
\tcolorboxenvironment{fsolution}
    {enhanced jigsaw,colframe=cyan,interior hidden,breakable}

%--------------------Declared Math Operators--------------------%
\DeclareMathOperator{\Refl}{Refl}           % Reflection operator.
\DeclareMathOperator{\Span}{Span}           % Span of a set of vectors.
\DeclareMathOperator{\Card}{Card}           % Cardinality of set.
\DeclareMathOperator{\Ord}{Ord}             % Ordinal of ordered set.
\DeclareMathOperator{\Tr}{Tr}               % Trace of matrix.
\DeclareMathOperator{\adjoint}{adj}         % Adjoint of matrix.
\DeclareMathOperator{\rk}{rk}               % Rank of operator.
\DeclareMathOperator{\nul}{nul}             % Null space of operator.
\DeclareMathOperator{\sgn}{sgn}             % Sign of a number.
\DeclareMathOperator{\multideg}{mutlideg}   % Multi-Degree (Graphs).
\DeclareMathOperator{\GCD}{GCD}             % Greatest common denominator.
\DeclareMathOperator{\LM}{LM}               % Leading monomial
\DeclareMathOperator{\LC}{LC}               % Leading coefficient.
\DeclareMathOperator{\LT}{LT}               % Leading term.
\DeclareMathOperator{\LCM}{LCM}             % Least common multiple.
\DeclareMathOperator{\Mon}{Mon}             % Monomial.
\DeclareMathOperator{\Spec}{Spec}           % Spectrum.
\DeclareMathOperator{\proj}{proj}           % Projection.
\DeclareMathOperator{\comp}{comp}           % Component.
\DeclareMathOperator{\sinc}{sinc}           % Sinc function.
\DeclareMathOperator{\Ima}{Im}              % Image of operator.
\DeclareMathOperator{\Prin}{Prin}           % Principal value.
\DeclareMathOperator{\Mod}{mod}             % Modulus.
%------------------------New Commands---------------------------%
\DeclarePairedDelimiter\norm{\lVert}{\rVert}
\DeclarePairedDelimiter\ceil{\lceil}{\rceil}
\DeclarePairedDelimiter\floor{\lfloor}{\rfloor}
\newcommand*\diff{\mathop{}\!\mathrm{d}}
\newcommand*\Diff[1]{\mathop{}\!\mathrm{d^#1}}
\renewcommand{\mod}{\ \Mod}
\renewcommand*{\glstextformat}[1]{\textcolor{RoyalBlue}{#1}}
\renewcommand{\glsnamefont}[1]{\textbf{#1}}
\renewcommand\labelitemii{$\circ$}
\renewcommand\thesubfigure{\arabic{chapter}.\arabic{figure}}
\renewcommand\thesubfigure{%
    \arabic{chapter}.\arabic{figure}.\arabic{subfigure}}
\addto\captionsenglish{\renewcommand{\figurename}{Fig.}}
%------------------------Book Command---------------------------%
\makeatletter
\renewcommand\@pnumwidth{1cm}
\newcounter{book}
\renewcommand\thebook{\@Roman\c@book}
\newcommand\book{%
    \if@openright
        \cleardoublepage
    \else
        \clearpage
    \fi
    \thispagestyle{plain}%
    \if@twocolumn
        \onecolumn
        \@tempswatrue
    \else
        \@tempswafalse
    \fi
    \null\vfil
    \secdef\@book\@sbook
}
\def\@book[#1]#2{%
    \ifnum \c@secnumdepth >-3\relax
        \refstepcounter{book}%
        \addcontentsline{toc}{book}{
            \bookname\ \thebook:\hspace{1em}#1
        }
    \else
        \addcontentsline{toc}{book}{#1}%
    \fi
    \markboth{}{}%
    {\centering
     \interlinepenalty \@M
     \normalfont
     \ifnum \c@secnumdepth >-2\relax
       \huge\bfseries \bookname\nobreakspace\thebook
       \par
       \vskip 20\p@
     \fi
     \Huge \bfseries #2\par}%
    \@endbook}
\def\@sbook#1{%
    {\centering
     \interlinepenalty \@M
     \normalfont
     \Huge \bfseries #1\par}%
    \@endbook}
\def\@endbook{
    \vfil\newpage
        \if@twoside
            \if@openright
                \null
                \thispagestyle{empty}%
                \newpage
            \fi
        \fi
        \if@tempswa
            \twocolumn
        \fi
}
\newcommand*\l@book[2]{%
    \ifnum \c@tocdepth >-2\relax
        \addpenalty{-\@highpenalty}%
        \addvspace{2.25em \@plus\p@}%
        \setlength\@tempdima{3em}%
        \begingroup
            \parindent \z@ \rightskip \@pnumwidth
            \parfillskip -\@pnumwidth
            {
                \leavevmode
                \Large \bfseries #1\hfil \hb@xt@\@pnumwidth{
                    \hss #2
                }
            }
            \par
            \nobreak
            \global\@nobreaktrue
            \everypar{\global\@nobreakfalse\everypar{}}%
        \endgroup
    \fi}
\newcommand\bookname{Book}
\renewcommand{\thebook}{\texorpdfstring{\Numberstring{book}}{book}}
\providecommand*{\toclevel@book}{-2}
\makeatother
\titlecontents{chapter}[0pt]
    {\bfseries}
    {\chaptername\ \thecontentslabel:\quad}
    {}
    {\hfill\contentspage}
\titleformat{\part}[display]
    {\Large\bfseries}
    {\partname\nobreakspace\thepart}
    {0mm}
    {\Huge\bfseries}
    \titlecontents{part}[0pt]
    {\large\bfseries}
    {\partname\ \thecontentslabel: \quad}
    {}
    {\hfill\contentspage}
\newcommand{\MarkRightAngle}[4][.3cm]
    {\coordinate (tempa) at ($(#3)!#1!(#2)$);
     \coordinate (tempb) at ($(#3)!#1!(#4)$);
     \coordinate (tempc) at ($(tempa)!0.5!(tempb)$);%midpoint
     \draw (tempa) -- ($(#3)!2!(tempc)$) -- (tempb);}
%--------------------------LENGTHS------------------------------%
% Spacings for the Table of Contents.
\addtolength{\cftsecnumwidth}{1ex}
\addtolength{\cftsubsecindent}{1ex}
\addtolength{\cftsubsecnumwidth}{1ex}
\addtolength{\cftfignumwidth}{1ex}
\addtolength{\cfttabnumwidth}{1ex}

% Spacing for multi-column and enumerate environments.
\setlength{\multicolsep}{6pt}
\setlist[enumerate]{itemsep=0pt,topsep=3pt}

% Indent and paragraph spacing.
\setlength{\parindent}{0em}
\setlength{\parskip}{0em}                                                           %
\makeindex[intoc]                                                              %
%----------------------------Main Document-------------------------------------%
\begin{document}
    \pagenumbering{gobble}
    \title{MATH 114 Algebraic Topology - Assignment 5}
    \author{Ryan Maguire}
    \date{\vspace{-5ex}}
    \maketitle
    \pagenumbering{roman}
    \pagenumbering{arabic}
    \setcounter{section}{6}
    \begin{problem}
        Hatcher 1.3.3.
    \end{problem}
    \begin{solution}
        If $p:\tilde{X}\rightarrow{X}$ is a covering map and $\tilde{X}$ is
        compact and Hausdorff, then $X$ is compact since $p$ is surjective and
        the continuous image of a compact space is compact. $X$ is also
        Hausdorff. For suppose not and let $x,y\in{X}$ be distinct points that
        cannot be separated. But $p$ is a covering and hence there are open sets
        $\mathcal{U}_{x},\mathcal{U}_{y}$ containing $x$ and $y$, respectively,
        such that $p^{\minus{1}}[\mathcal{U}_{x}]$ is the disjoint union of
        homeomorphic copies of $\mathcal{U}_{x}$, and similarly for
        $\mathcal{U}_{y}$, and moreover $p$ is such a homeomorphism for these
        sheets. Since $\tilde{X}$ is compact and Hausdorff it is normal. But $p$
        is continuous, and hence the pre-image of closed sets are closed. Now
        points are closed in $X$. For let $z\in{X}$ and choose $\mathcal{U}_{z}$
        such that $p^{\minus{1}}[\mathcal{U}_{z}]$ is the disjoint union of
        homeomorphic copies of $\mathcal{U}_{z}$, with $z\in\mathcal{U}_{z}$ of
        course. Take any of these sheets. Since the subspace of a Hausdorff
        space is Hausdorff, it follows that $\mathcal{U}_{z}$ is Hausdorff in
        the subspace topology since it is homeomorphic to one of these sheets.
        Hence $X$ is locally Hausdorff, and thus $T_{1}$, which implies points
        are closed. Since $p$ is continuous, $p^{\minus{1}}[\{x\}]$ and
        $p^{\minus{1}}[\{y\}]$ is disjoint closed subsets. But $\tilde{X}$ is
        normal and hence there are open subsets separating them, say
        $\mathcal{V}_{x}$ and $\mathcal{V}_{y}$. But then
        $p^{\minus{1}}[\mathcal{U}_{x}]\cap\mathcal{V}_{x}$ and
        $p^{\minus{1}}[\mathcal{U}_{y}]\cap\mathcal{V}_{y}$ are disjoint open
        sets, and the forward image of these by $p$ are disjoint open subsets
        of $X$ containing $x$ and $y$, respectively. Hence, $X$ is Hausdorff.
        \par\hfill\par
        In the other direction, if $X$ is Hausdorff, then $\tilde{X}$ is. If
        $\tilde{x},\tilde{y}\in\tilde{X}$ are distinct, and if $p(\tilde{x})$
        and $p(\tilde{y})$ are also distinct then we are done since the images
        can be separated since $X$ is Hausdorff, taking the pre-image then gives
        disjoint open neighborhoods separating $\tilde{x}$ and $\tilde{y}$. If
        the images coincide, let $\mathcal{U}$ be a neighborhood of this image
        such that the pre-image is the homeomorphic copies or $\mathcal{U}$.
        Then since $\tilde{x}$ and $\tilde{y}$ map to the same points, they must
        live in different sheets, otherwise we have a sheet that is not
        Hausdorff in the subspace topology, but which is homeomorphic to an open
        subspace of a Hausdorff space, which is then itself Hausdorff, which is
        a contradiction. Hence $\tilde{x}$ and $\tilde{y}$ live in different
        sheets, and hence are separated by open sets. The compactness comes from
        the assumption that the pre-image of sheets is finite. Given a cover
        $\mathcal{O}$ of $\tilde{X}$, let $\Delta$ consist of the set of all
        open subsets of $X$ such that $\mathcal{U}\in\Delta$ implies that
        $p^{\minus{1}}[\mathcal{U}]$ is the disjoint union of homeomorphic
        copies of $\mathcal{U}$ and every copy is contained in some
        $\mathcal{V}\in\mathcal{O}$. Since $\mathcal{O}$ is an open covering of
        $\tilde{X}$, and since $p$ is a covering map, $\Delta$ is an open cover
        of $X$. That is, every point $x\in{X}$ has some $\mathcal{U}\in\Delta$
        with $x\in\mathcal{U}$. But $X$ is compact and hence there is a finite
        subcover. But for every $x\in{X}$ the pre-image is finite. Hence for
        each element of the subcover choose a representative point $x$ and
        choose all open sets in $\mathcal{O}$ that cover $p^{\minus{1}}[\{x\}]$.
        You'll need finitely many such sets, and you'll need to do this for
        finitely many such points in $X$, giving us a finite subcover of
        $\mathcal{O}$. Hence, $\tilde{X}$ is compact.
    \end{solution}
    \begin{problem}
        \label{prob:Hatcher_1_3_4}%
        Hatcher 1.3.4.
    \end{problem}
    \begin{solution}
        For the sphere with an equator we note that this is homeomorphic to a
        sphere with string attaching the poles, where the string remains on the
        exterior of the sphere. This is then homeomorphic to the image shown
        in Fig.~\ref{fig:Hatcher_1_3_4}.
        \begin{figure}[H]
            \centering
            \captionsetup{type=figure}
            \includegraphics{images/CW_Complex_Messy.pdf}
            \caption{An Object Homeomorphic to One in Problem~\ref{prob:Hatcher_1_3_4}}
            \label{fig:Hatcher_1_3_4}
        \end{figure}
        This shows us how to obtain our universal cover. The universal cover of
        a circle is the real line, and the universal cover of the sphere is
        itself. We need to attach a sphere along the prime meridian to the real
        line along $n$ to $n+1/2$ for every $n\in\mathbb{Z}$. This mimics the
        universal cover for the wedge of $\nsphere[1]$ and $\nsphere[2]$, which
        simply attaches a sphere to every integer at a single point. Since we
        can stretch the real line between $n$ and $n+1/2$ to form a half-circle,
        we may then just place the sphere on top of this. The result is shown
        in
        \begin{figure}[H]
            \centering
            \captionsetup{type=figure}
            \includegraphics{images/Universal_Cover_Example_001.pdf}
            \caption{Solution to Problem~\ref{prob:Hatcher_1_3_4}}
            \label{fig:Hatcher_1_3_4b}
        \end{figure}
        For a sphere with a circle intersecting at two points we do a similar
        construction but with the Cayley graph of the free group on two
        generators replacing the real line. This is because if we omit the
        sphere we have something homotopy equivalent to the wedge of two circles
        which has universal cover the Cayley graph.
    \end{solution}
    \begin{problem}
        Hatcher 1.3.8.
    \end{problem}
    \begin{solution}
        If $X$ is homotopy equivalent to $Y$, then there is a homotopy
        equivalence $f:X\rightarrow{Y}$ with a homotopy inverse
        $g:Y\rightarrow{X}$. But then there is a homotopy
        $H:X\times{I}\rightarrow{X}$ between $g\circ{f}$ and $\identity{X}$.
        Moreover, $f$ and $g$ lift to functions
        $\tilde{f}:\tilde{X}\rightarrow\tilde{Y}$ and
        $\tilde{g}:\tilde{Y}\rightarrow\tilde{X}$, and lastly we may lift each
        $H(\cdot,t)$ to a function $\tilde{H}(\cdot,t)$ on $\tilde{X}$. We are
        allowed to assume that $\tilde{H}$ glues together to form a continuous
        map on $\tilde{X}$. But then $\tilde{X}(\tilde{x},0)$ is the lift of
        $(f\circ{g})(x)$, which is just $(\tilde{f}\circ\tilde{g})(\tilde{x})$.
    \end{solution}
    \begin{problem}
        Hatcher 1.3.9
    \end{problem}
    \begin{solution}
        Firstly, any such mapping $f$ induces a homomorphism between the
        fundamental groups and the image of a homeomorphism is a subgroup. But
        the only finite subgroup of $\mathbb{Z}$ is the trivial group, and hence
        the image of $f$ is simply connected. But then we may lift the map $f$
        to the universal cover of $\nspace[1]$, which is the real line
        $\mathbb{R}$. But $\mathbb{R}$ is contractible, and hence every map into
        it is nullhomotopic. Hence, $f$ is also nullhomotopic.
    \end{solution}
\end{document}