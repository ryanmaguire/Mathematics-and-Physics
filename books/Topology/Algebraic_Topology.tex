\documentclass[crop=false,class=book,oneside]{standalone}
%----------------------------Preamble-------------------------------%
%---------------------------Packages----------------------------%
\usepackage{geometry}
\geometry{b5paper, margin=1.0in}
\usepackage[T1]{fontenc}
\usepackage{graphicx, float}            % Graphics/Images.
\usepackage{natbib}                     % For bibliographies.
\bibliographystyle{agsm}                % Bibliography style.
\usepackage[french, english]{babel}     % Language typesetting.
\usepackage[dvipsnames]{xcolor}         % Color names.
\usepackage{listings, lstlinebgrd}      % Verbatim-Like Tools.
\usepackage{mathtools, esint, mathrsfs} % amsmath and integrals.
\usepackage{amsthm, amsfonts}           % Fonts and theorems.
\usepackage{tabularx}
\usepackage{tcolorbox}                  % Frames around theorems.
\usepackage{upgreek}                    % Non-Italic Greek.
\usepackage{paracol}                    % Two-column styling.
\usepackage{wrapfig}                    % Wrap text around figure.
\usepackage{fmtcount, etoolbox}         % For the \book{} command.
\usepackage[newparttoc]{titlesec}       % Formatting chapter, etc.
\usepackage{titletoc}                   % Allows \book in toc.
\usepackage[nottoc]{tocbibind}          % Bibliography in toc.
\usepackage[titles]{tocloft}            % ToC formatting.
\usepackage{multicol, enumitem}         % Multi-column/enumerate.
\usepackage{import}                     % Import external files.
\usepackage{pgfplots, tikz}             % Drawing/graphing tools.
\usetikzlibrary{
    calc,                   % Calculating right angles and more.
    angles,                 % Drawing angles within triangles.
    arrows.meta,            % Latex and Stealth arrows.
    quotes,                 % Adding labels to angles.
    positioning,            % Relative positioning of nodes.
    decorations.markings,   % Adding arrows in the middle of a line.
    patterns,
    arrows,
    shapes,
    shapes.geometric,
    cd,
    hobby,
    babel
}                                       % Libraries for tikz.
\pgfplotsset{compat=1.9}                % Version of pgfplots.
\usepackage[font=scriptsize,
            labelformat=simple,
            labelsep=colon]{subcaption} % Subfigure captions.
\usepackage[font={scriptsize},
            hypcap=true,
            labelsep=colon]{caption}    % Figure captions.
\usepackage{hyperref}                   % Allows for hyperlinks.
\hypersetup{
    colorlinks=true,
    linkcolor=blue,
    filecolor=magenta,
    urlcolor=Cerulean,
    citecolor=SkyBlue
}                           % Colors for hyperref.
\usepackage[toc,acronym,nogroupskip]{glossaries} % Glossaries and acronyms.
\usepackage[subpreambles=false]{standalone}      % Complileable sub files.

% Various font stuff from kiwi.
% Use this for Times text and Computer Modern math
%\usepackage{times}

% Quite nice
%\usepackage[charter, greekfamily=, greekuppercase=italicized]{mathdesign}
%\usepackage[utopia, greekuppercase=italicized]{mathdesign}    % Math is narrower

% Use this for Times text and math
%\usepackage{newtxtext}
%\usepackage[libertine,cmintegrals]{newtxmath}
%\usepackage{fix-cm}

%\usepackage{txfontsb}
% or
%\usepackage{mathptmx}

%\usepackage[scaled=0.92]{helvet}
%\renewcommand{\rmdefault}{ptm}

%\usepackage{mathpazo}    % add possibly `sc` and `osf` options
%\usepackage{eulervm}

%\usepackage{fourier}
%\renewcommand{\rmdefault}{ptm}
%\usepackage{mathptm}

%\usepackage{fontspec}
%\setmainfont{lmodern}

%\usepackage[varg]{txfonts}
%\usepackage{fouriernc}
%\usepackage{mathpazo}

%\usepackage{bookman}
%\usepackage[scaled]{uarial}
%\usepackage[scaled]{helvet}
%\renewcommand*\familydefault{\sfdefault}
%\usepackage[math]{anttor}

%\newcommand\fgeorgia{\fontfamily{jvn}\selectfont}
%\newcommand\ftimes{\fontfamily{ptm}\selectfont}
%\newcommand\fhelvetica{\fontfamily{phv}\selectfont}
%\newcommand\fcourier{\fontfamily{pcr}\selectfont}
%\newcommand\fbookman{\fontfamily{pbk}\selectfont}
%\newcommand\fnewcentury{\fontfamily{pnc}\selectfont}
%\newcommand\fpalatino{\fontfamily{ppl}\selectfont}
%\newcommand\favantgarde{\fontfamily{pag}\selectfont}
%\newcommand\fnormal{\normalfont}
%\newcommand\fsize[1]{\ifnum#1>0\fontsize{#1}{#1}\selectfont\else\normalsize\fi}
%------------------------Theorem Styles-------------------------%
% Define theorem style for default spacing and normal font.
\newtheoremstyle{normal}
    {\topsep}               % Amount of space above the theorem.
    {\topsep}               % Amount of space below the theorem.
    {}                      % Font used for body of theorem.
    {}                      % Measure of space to indent.
    {\bfseries}             % Font of the header of the theorem.
    {}                      % Punctuation between head and body.
    {.5em}                  % Space after theorem head.
    {}

% Define theorem style for default spacing with italicized font.
\newtheoremstyle{normalit}{\topsep}{\topsep}
                {\itshape}{}{\bfseries}{}{.5em}{}

% Italic header environment.
\newtheoremstyle{thmit}{\topsep}{\topsep}{}{}{\itshape}{}{0.5em}{}

% Define italicized environments.
\theoremstyle{normalit}
\newtheorem{theorem}{Theorem}[section]
\newtheorem{lemma}{Lemma}[section]
\newtheorem{corollary}{Corollary}[section]
\newtheorem{proposition}{Proposition}[section]
\newtheorem*{theorem*}{Theorem}

% Define environments with italic headers.
\theoremstyle{thmit}
\newtheorem*{solution}{Solution}
\newtheorem*{fsolution}{Solution}

% Define default environments.
\theoremstyle{normal}
\newtheorem{example}{Example}[section]
\newtheorem{definition}{Definition}[section]
\newtheorem{problem}{Problem}[section]
\newtheorem{question}{Question}[section]
\newtheorem{remark}{Remark}[section]
\newtheorem{properties}{Properties}[section]
\newtheorem{notation}{Notation}[section]
\newtheorem{axiom}{Axiom}[section]
\newtheorem*{properties*}{Properties}
\newtheorem*{remark*}{Remark}
\newtheorem*{definition*}{Definition}
\theoremstyle{plain}

% Define framed environment.
\tcbuselibrary{most}
\newtcbtheorem[use counter*=theorem]{ftheorem}{Theorem}%
    {colback=green!5,colframe=green!35!black,
     fonttitle=\bfseries\upshape}{th}

\newtcbtheorem[use counter*=example]{fdefinition}{Definition}%
    {fonttitle=\bfseries\upshape,
     colback=blue!5!white,colframe=blue!75!black}{def}

\newtcbtheorem[use counter*=example]{fexample}{Example}%
    {fonttitle=\bfseries\upshape,
     colback=red!5!white,colframe=red!75!black}{ex}

\newtcbtheorem[use counter*=notation]{fnotation}{Notation}%
    {fonttitle=\bfseries\upshape,
     colback=SeaGreen!5!white,colframe=SeaGreen!75!black}{ex}

\newtcbtheorem[use counter*=corollary]{fcorollary}{Corollary}%
    {fonttitle=\bfseries\upshape,
     colback=Orchid!5!white,colframe=Orchid!75!black}{ex}

\newenvironment{bproof}{\textit{Proof.}}{\hfill$\square$}
\tcolorboxenvironment{bproof}{blanker,breakable,left=5mm,
                             before skip=10pt,after skip=10pt,
                             borderline west={1mm}{0pt}{red}}
\tcolorboxenvironment{fsolution}
    {enhanced jigsaw,colframe=cyan,interior hidden,breakable}

%--------------------Declared Math Operators--------------------%
\DeclareMathOperator{\Refl}{Refl}           % Reflection operator.
\DeclareMathOperator{\Span}{Span}           % Span of a set of vectors.
\DeclareMathOperator{\Card}{Card}           % Cardinality of set.
\DeclareMathOperator{\Ord}{Ord}             % Ordinal of ordered set.
\DeclareMathOperator{\Tr}{Tr}               % Trace of matrix.
\DeclareMathOperator{\adjoint}{adj}         % Adjoint of matrix.
\DeclareMathOperator{\rk}{rk}               % Rank of operator.
\DeclareMathOperator{\nul}{nul}             % Null space of operator.
\DeclareMathOperator{\sgn}{sgn}             % Sign of a number.
\DeclareMathOperator{\multideg}{mutlideg}   % Multi-Degree (Graphs).
\DeclareMathOperator{\GCD}{GCD}             % Greatest common denominator.
\DeclareMathOperator{\LM}{LM}               % Leading monomial
\DeclareMathOperator{\LC}{LC}               % Leading coefficient.
\DeclareMathOperator{\LT}{LT}               % Leading term.
\DeclareMathOperator{\LCM}{LCM}             % Least common multiple.
\DeclareMathOperator{\Mon}{Mon}             % Monomial.
\DeclareMathOperator{\Spec}{Spec}           % Spectrum.
\DeclareMathOperator{\proj}{proj}           % Projection.
\DeclareMathOperator{\comp}{comp}           % Component.
\DeclareMathOperator{\sinc}{sinc}           % Sinc function.
\DeclareMathOperator{\Ima}{Im}              % Image of operator.
\DeclareMathOperator{\Prin}{Prin}           % Principal value.
\DeclareMathOperator{\Mod}{mod}             % Modulus.
%------------------------New Commands---------------------------%
\DeclarePairedDelimiter\norm{\lVert}{\rVert}
\DeclarePairedDelimiter\ceil{\lceil}{\rceil}
\DeclarePairedDelimiter\floor{\lfloor}{\rfloor}
\newcommand*\diff{\mathop{}\!\mathrm{d}}
\newcommand*\Diff[1]{\mathop{}\!\mathrm{d^#1}}
\renewcommand{\mod}{\ \Mod}
\renewcommand*{\glstextformat}[1]{\textcolor{RoyalBlue}{#1}}
\renewcommand{\glsnamefont}[1]{\textbf{#1}}
\renewcommand\labelitemii{$\circ$}
\renewcommand\thesubfigure{\arabic{chapter}.\arabic{figure}}
\renewcommand\thesubfigure{%
    \arabic{chapter}.\arabic{figure}.\arabic{subfigure}}
\addto\captionsenglish{\renewcommand{\figurename}{Fig.}}
%------------------------Book Command---------------------------%
\makeatletter
\renewcommand\@pnumwidth{1cm}
\newcounter{book}
\renewcommand\thebook{\@Roman\c@book}
\newcommand\book{%
    \if@openright
        \cleardoublepage
    \else
        \clearpage
    \fi
    \thispagestyle{plain}%
    \if@twocolumn
        \onecolumn
        \@tempswatrue
    \else
        \@tempswafalse
    \fi
    \null\vfil
    \secdef\@book\@sbook
}
\def\@book[#1]#2{%
    \ifnum \c@secnumdepth >-3\relax
        \refstepcounter{book}%
        \addcontentsline{toc}{book}{
            \bookname\ \thebook:\hspace{1em}#1
        }
    \else
        \addcontentsline{toc}{book}{#1}%
    \fi
    \markboth{}{}%
    {\centering
     \interlinepenalty \@M
     \normalfont
     \ifnum \c@secnumdepth >-2\relax
       \huge\bfseries \bookname\nobreakspace\thebook
       \par
       \vskip 20\p@
     \fi
     \Huge \bfseries #2\par}%
    \@endbook}
\def\@sbook#1{%
    {\centering
     \interlinepenalty \@M
     \normalfont
     \Huge \bfseries #1\par}%
    \@endbook}
\def\@endbook{
    \vfil\newpage
        \if@twoside
            \if@openright
                \null
                \thispagestyle{empty}%
                \newpage
            \fi
        \fi
        \if@tempswa
            \twocolumn
        \fi
}
\newcommand*\l@book[2]{%
    \ifnum \c@tocdepth >-2\relax
        \addpenalty{-\@highpenalty}%
        \addvspace{2.25em \@plus\p@}%
        \setlength\@tempdima{3em}%
        \begingroup
            \parindent \z@ \rightskip \@pnumwidth
            \parfillskip -\@pnumwidth
            {
                \leavevmode
                \Large \bfseries #1\hfil \hb@xt@\@pnumwidth{
                    \hss #2
                }
            }
            \par
            \nobreak
            \global\@nobreaktrue
            \everypar{\global\@nobreakfalse\everypar{}}%
        \endgroup
    \fi}
\newcommand\bookname{Book}
\renewcommand{\thebook}{\texorpdfstring{\Numberstring{book}}{book}}
\providecommand*{\toclevel@book}{-2}
\makeatother
\titlecontents{chapter}[0pt]
    {\bfseries}
    {\chaptername\ \thecontentslabel:\quad}
    {}
    {\hfill\contentspage}
\titleformat{\part}[display]
    {\Large\bfseries}
    {\partname\nobreakspace\thepart}
    {0mm}
    {\Huge\bfseries}
    \titlecontents{part}[0pt]
    {\large\bfseries}
    {\partname\ \thecontentslabel: \quad}
    {}
    {\hfill\contentspage}
\newcommand{\MarkRightAngle}[4][.3cm]
    {\coordinate (tempa) at ($(#3)!#1!(#2)$);
     \coordinate (tempb) at ($(#3)!#1!(#4)$);
     \coordinate (tempc) at ($(tempa)!0.5!(tempb)$);%midpoint
     \draw (tempa) -- ($(#3)!2!(tempc)$) -- (tempb);}
%--------------------------LENGTHS------------------------------%
% Spacings for the Table of Contents.
\addtolength{\cftsecnumwidth}{1ex}
\addtolength{\cftsubsecindent}{1ex}
\addtolength{\cftsubsecnumwidth}{1ex}
\addtolength{\cftfignumwidth}{1ex}
\addtolength{\cfttabnumwidth}{1ex}

% Spacing for multi-column and enumerate environments.
\setlength{\multicolsep}{6pt}
\setlist[enumerate]{itemsep=0pt,topsep=3pt}

% Indent and paragraph spacing.
\setlength{\parindent}{0em}
\setlength{\parskip}{0em}
%----------------------------GLOSSARY-------------------------------%
\makeglossaries
\loadglsentries{../../glossary}
\loadglsentries{../../acronym}
%--------------------------Main Document----------------------------%
\begin{document}
    \ifx\ifmathcourses\undefined
        \pagenumbering{roman}
        \title{Algebraic Topology}
        \author{Ryan Maguire}
        \date{\vspace{-5ex}}
        \maketitle
        \tableofcontents
        \listoffigures
        \clearpage
        \chapter{Algebraic Topology}
        \pagenumbering{arabic}
    \else
        \chapter{Algebraic Topology}
    \fi
    \section{A Review of Algebra}
        A \textbf{Group} is a set $G$ and a
        binary operation $*$ such
        that the following are true:
        \begin{enumerate}
            \item[G1] $\forall_{a,b,c\in{G}},%
                       a*(b*c)=(a*b)*c$
                      \hfill[Associativity]
            \item[G2] $\exists_{e\in{G}}\forall_{a\in{G}}:a*e=a$
                      \hfill[Existence of Right Identity]
            \item[G3] $\forall_{a\in{G}}\exists_{a^{-1}\in{G}}:%
                       a*a^{-1}=e$
                      \hfill[Existence of Right Inverse]
        \end{enumerate}
        Usually one sees $a*e=e*a=a$ and
        $a*a^{-1}=a^{-1}*a=e$, but this can be relaxed to just
        left or right and then one can prove they are both
        equivalent. We often write $(G,*)$ to denote a group.
        An \textit{Abelian Group} is a group $(G,*)$ such that:
        \begin{enumerate}
            \item[G4] $\forall_{a,b\in{G}},a*b=b*a$
                      \hfill[Commutativity]
        \end{enumerate}
        A \textbf{Ring} is a set $R$ with
        two operations $+$ and $\cdot$, usually called
        addition and multiplication, respectively,
        with the following properties:
        \begin{enumerate}
            \item[R1] $(R,+)$ is an Abelian Group.
            \item[R2] $\forall_{a,b,c\in{R}},%
                       a\cdot(b\cdot{c})=(a\cdot{b})\cdot{c}$
                      \hfill[Associativity of Multiplication]
            \item[R3] $\forall_{a,b,c\in{R}},%
                       a\cdot(b+c)=(a\cdot{b})+(a\cdot{c})$
                      \hfill[Left-Distributive Law]
            \item[R4] $\forall_{a,b,c\in{R}}%
                       (b+c)\cdot{a}=(b\cdot{a})+(c\cdot{a})$
                      \hfill[Right-Distributive Law]
        \end{enumerate}
        We write $(R,+,\cdot)$ to denote a ring.
        A \textbf{Ring with Identity}, or a ring with unity, is
        a ring $(R,+,\cdot)$ such that:
        \begin{enumerate}
            \item[R5] $\exists_{1\in{R}}\forall_{a\in{R}}:%
                       1\cdot{a}=a\cdot{1}=a$
                      \hfill[Multiplicative Identity]
        \end{enumerate}
        A \textbf{Commutative Ring} is a ring $(R,+,\cdot)$
        such that:
        \begin{enumerate}
            \item[R6] $\forall_{a,b\in{R}},a\cdot{b}=b\cdot{a}$
                      \hfill[Commutativity of Multiplication]
        \end{enumerate}
        In a commutative ring, one can replace the
        Left-Distributive Law and the Right-Distributive Law with
        simply one Distributive Law. Commutativity implying
        they're equivalent. A $\textbf{Field}$ is a
        commutative ring with identity $(F,+,\cdot)$ such that:
        \begin{enumerate}
            \item[F1] $\forall_{a\in{F}, a\ne{0}}%
                       \exists_{a^{-1}\in{F}}:a\cdot{a^{-1}}=1$
                      \hfill[Multiplicative Inverse]
        \end{enumerate}
        One can prove some very intuitive results about the
        additive identity of a ring $R$.
        \begin{theorem}
            If $(R,+,\cdot)$ is a ring and $0$ is the additive
            identity of $R$, then for all $a\in{R}$,
            $a\cdot{0}=0$.
        \end{theorem}
        \begin{proof}
            For:
            \begin{align*}
                0&=a\cdot{0}-a\cdot{0}
                &
                &=a\cdot{0}+(a\cdot{0}-a\cdot{0})\\
                &=a\cdot(0+0)-a\cdot{0}
                &
                &=a\cdot{0}+0\\
                &=(a\cdot{0}+a\cdot{0})-a\cdot{0}
                &
                &=a\cdot{0}
            \end{align*}
        \end{proof}
        Thus, if one has a field $(F,+,\cdot)$, and if
        $0$ has a multiplicative inverse, then every element of
        $F$ is equal to $0$.
        \begin{theorem}
            If $(F,+,\cdot)$ is a field, $0$ is the additive
            identity, and if $0$ has a multiplicative inverse,
            then $F=\{0\}$.
        \end{theorem}
        \begin{proof}
            For $1=0\cdot{0}^{-1}$, but
            $0\cdot{0}^{-1}=0$, and thus $1=0$. But for all
            $a\in{F}$, $a=a\cdot{1}=a\cdot{0}=0$.
        \end{proof}
        Because of this, some require that $0\ne{1}$ in the
        definition of a field, and others call
        $F=\{0\}$ the trivial field.
        \begin{definition}
            A left module of a ring with identity $(R,+,\cdot)$
            is an abelian group $(M,+_{M})$ and a function
            $*:R\times{M}\rightarrow{M}$ such that:
            \begin{enumerate}
                \item $r*(a+_{M}b)=(r*a)+(r*b)$
                      \hfill[Left Scalar Distribution]
                \item $(r\cdot{s})*a=r*(s*a)$
                      \hfill[Scalar Associativity]
                \item $(r+s)*a=(r*a)+_{M}(s*a)$
                      \hfill[Right Module Distribution]
                \item $1*a=a$
                      \hfill[Identity Element]
            \end{enumerate}
        \end{definition}
        An attempt has been made to preserve the differences
        between the various operations in a left module.
        $+$ and $\cdot$ are binary operations that act on
        elements of $R$. That is, for $a,b\in{R}$, $a+b$
        gives another element of $R$, as does $a\cdot{b}$.
        However, $+_{M}$ is a binary operation over
        $M$. If $a,b\in{R}$, $a+_{M}b$ has no meaning.
        For $a,b\in{M}$, $a+_{M}b$ is well defined, and returns
        another element of $M$. The ``function,'' $*$
        takes an ordered pair $(r,a)$, where $r\in{R}$ and
        $a\in{M}$, and returns another element in $M$. For
        convenience we write $r*a$. If $a,b\in{R}$, then
        $a*b$ has no meaning, and if $a,b\in{M}$ then
        $a*b$ also has no meaning. Usually this is very
        unimportant, and $+$ and $+_{M}$ are given the same
        symbol, as are $\cdot$ and $*$. We can then more loosely
        rewrite the definition as, for all $r,s\in{R}$, and
        all $a,b\in{M}$:
        \begin{enumerate}
            \begin{multicols}{2}
                \item $r(a+b)=ra+rb$
                \item $(rs)(a)=r(sa)$
                \item $(r+s)(a)=(ra)+(rs)$
                \item $1a=a$
            \end{multicols}
        \end{enumerate}
        This is the more natural notation one finds when defining
        vector spaces. A module is analogous to a vector space:
        In a vector space one has a set $V$ and a
        \textit{field} $K$, whereas in a module one has a set
        $M$ and a \textit{ring with identity} $R$.
        \begin{theorem}
            If $(G,+)$ is an Abelian group, then there is a
            function $*:\mathbb{Z}\times{G}\rightarrow{G}$
            such that $(G,+_{G})$ is a left module of
            $(\mathbb{Z},+,\cdot)$, where $+$ and $\cdot$ are
            the standard arithmetic operations over $\mathbb{Z}$.
        \end{theorem}
        \begin{proof}
            For define $0*a=e$ and $1*a=a$ for all $a\in{G}$,
            and for all
            $n\in\mathbb{Z}$, $n>1$ inductively define
            $(n+1)*a=n*a+_{G}a$. For $n<0$ define
            $n*a=((-n)*a)^{-1}$, where the inverse is taken
            with respect to the group $G$. Then $*$ is a function
            $*:\mathbb{Z}\times{G}\rightarrow{G}$. If $n>0$, we
            have the following:
            \begin{align*}
                (n*a)+_{G}(n*b)
                &=\underset{n}{\underbrace{(a+_{G}\cdots+_{G}a)}}
                +_{G}
                \underset{n}{\underbrace{(b+_{G}\cdots+_{G}b)}}\\
                &=\underset{n}
                    {\underbrace{(a+b)+_{G}\cdots+_{G}(a+b)}}\\
                &=n*(a+b)
            \end{align*}
            If $n,m>0$, we have:
            \begin{equation*}
                n*a+_{G}m*a
                =\underset{n+m}{\underbrace{a+_{G}\cdots+_{G}a}}
                =(n+m)*a
            \end{equation*}
            And finally:
            \begin{equation*}
                (n\cdot{m})*a=
                \underset{n\cdot{m}}
                    {\underbrace{a+_{G}\cdots+_{G}a}}
                =n*(\underset{m}{\underbrace{a+_{G}\cdots+_{G}a}})
                =n*(m*a)
            \end{equation*}
            Similarly for when $n,m<0$, $n<0<m$, or $m<0<n$.
        \end{proof}
        Thus, every Abelian group $(G,+_{G})$ can be seen
        as a left module over $(R,+,\cdot)$. Moreover the
        function $*$ is unique, so this correspondence is
        unique as well. As another example, every vector space
        $V$ over a field $K$ is a left module over $K$, since
        any field $K$ is also a ring with identity.
        A \textbf{Left Ideal} of a ring $(R,+,*)$ is a subset
        $I\subseteq{R}$ such that:
        \begin{enumerate}
            \item $\forall_{a,b\in{I}},a+b\in{I}$
            \item $\forall_{r\in{R}}\forall_{a\in{I}},%
                   r\cdot{a}\in{I}$
        \end{enumerate}
        This can be rephrased by saying that $(I,+)$ is a subgroup
        of $(R,+)$, and $I$ absorbs left-multiplication. A
        \textbf{Right Ideal} replaces $r\cdot{a}$ with
        $a\cdot{r}$. An \textbf{Ideal} or \textbf{Two-Sided Ideal}
        is a subset that is both a left and a right ideal.
        \begin{theorem}
            If $(R,+,\cdot)$ is a ring with identity
            and $(I,+)$ is a left ideal
            of $R$, then there is a function
            $*:R\times{I}\rightarrow{I}$ such that
            $(I,+)$ is a left module over $R$.
        \end{theorem}
        \begin{proof}
            For let $*$ be the restriction of $\cdot$ to
            $R\times{I}$. Then, for all $a\in{I}$,
            $1\cdot{a}=a$. If $a,b\in{I}$ and $r\in{R}$, then:
            \begin{equation*}
                r*(a+b)
                =r\cdot(a+b)
                =(r\cdot{a})+(r\cdot{b})
            \end{equation*}
            If $r,s\in{R}$ and $a\in{I}$, then:
            \begin{gather*}
                (r\cdot{s})*a
                =(r\cdot{s})\cdot{a}
                =r\cdot(s\cdot{a})\\
                (r+s)*(a)
                =(r+s)\cdot{a}
                =(r\cdot{a})+(s\cdot{a})
            \end{gather*}
        \end{proof}
        \begin{definition}
            The Annihilator of a Left Module $(M,+_{M})$ over
            a ring with identity $(R,+,\cdot)$ is the set:
            \begin{equation*}
                I=\{r\in{R}:\forall_{m\in{M}},r*m=0\}
            \end{equation*}
        \end{definition}
        \begin{theorem}
            If $(M,+_{M})$ is a Left Module over a ring with
            identity $(R,+,\cdot)$, and if $I$ is the
            annihilator of $M$, then $I$ is a two-sided
            ideal of $R$.
        \end{theorem}
        \begin{proof}
            For if $r,s\in{I}$, then for all $m\in{M}$,
            $r*m=0$ and $s*m=0$. But $(r+s)*m=(r*m)+_{M}(s*m)=0$.
            Therefore $r+s\in{I}$. If $r\in{R}$ and $s\in{I}$,
            then $(r\cdot{s})*m=r*(s*m)=r*0=0$, and therefore
            $r\cdot{s}\in{I}$. Furthermore,
            $(s\cdot{r})*m=s*(r*m)=0$, and thus $s\cdot{r}\in{I}$.
            Therefore $I$ is a two-sided ideal.
        \end{proof}
    \section{The Fundamental Group}
    \section{Homology}
    \section{Cohomology}
    \section{Homotopy}
    \section{Notes from Dartmouth S2019}
        Categories. Certain types of objects with
        maps, like \textit{group homomorphisms}, or
        \textit{ring homomorphisms}, or
        \textit{continuous functions}. The objects
        are groups, rings, and topological spaces,
        respectively. Presently we care about
        topological spaces. This is a set $X$ with
        a \textit{topology} $\tau$ on $X$,
        which is a subset of $\mathcal{P}(X)$ such that
        $\emptyset\in\tau$, $X\in\tau$, the union of
        any collection is $\tau$ is again in $\tau$,
        and the intersection of any finite collection
        of elements of $\tau$ is again in $\tau$.
        That maps we talk about are continuous functions
        between topological spaces.
        \begin{example}
            $SO(3)$, $3\times{3}$ matrices that
            are orthogonal and have determinant 1, and
            $SU(2)$. Both of these are differentiable
            manifolds. How can we determine if they
            are homeomorphic or not? To every
            continuous map between topological spaces
            $X$ and $Y$ we can correspond groups
            $G(X)$ and $G(Y)$ and a group homomorphism
            $\varphi:G(X)\rightarrow{G}(Y)$ that
            corresponds somehow to the continuous
            function. By showing that certain properties
            are preserved by this correspondence we
            can show that $SO(3)$ and $SU(2)$ are not
            homeomorphic.
        \end{example}
        \begin{align}
            &f:X\rightarrow{Y}\rightarrow{Z}\\
            &\varphi:G(X)\rightarrow{G}(Y)
            \rightarrow{G}(Z)
        \end{align}
        The maps between topological spaces and groups
        are called \textit{Functors}. The category
        of topological spaces is somewhat to large.
        From point-set topology there are many
        pathelogical sets that are difficult to manage.
        In differential topology we study smooth
        manifolds which are locally like Euclidean space
        and have a lot of structure on them. Even
        nicer are \textit{Affine Varieties}, which
        are solution sets to polynomials that are
        studied in algebraic geometry. In the late
        1940s, Whitehead came up with the following
        category, that of CW Complexes. Not every
        topological space is a CW Complex, but in some
        respect every topological space is like a CW
        complex. All manifolds are CW complexes.
        \subsection{CW Complexes}
            The C in CW stands for \textit{cell}
            \begin{ldefinition}{Cell}
                A cell of dimension $n\in\mathbb{N}$ is
                a topological space that is homeomorphic
                to $D^{n}\setminus\partial{D}^{n}$
                Where $D^{n}$ is the unit disk:
                \begin{equation}
                    D^{n}=\{\mathbf{x}\in\mathbb{R}^{n}:
                    \norm{\mathbf{x}}=1\}
                \end{equation}
                And $\partial{D}^{n}$ is the boundary.
            \end{ldefinition}
            There's no intrinsic metric on the space.
            The open unit square is the a cell, since
            it is homeomorphic to the open unit disc.
            A zero dimensional cell is a point.
            One-cells are edges, two-cells are
            bubbles, and so forth.
            \begin{ldefinition}{CW Complexes}
                A CW Complex is a topological space
                $X$ that is the disjoint union of
                cells.
            \end{ldefinition}
            Typically CW Complexes are metric spaces.
            \begin{ldefinition}{$k$-Skeleton}
                The $k$-skeleton of a CW complex $X$,
                denoted $X^{k}$,
                is the disjoint union of all cells of
                dimension less than or equal to $k$.
            \end{ldefinition}
            $X^{0}$ is the disjoint union of points.
            In general:
            \begin{equation}
                X^{k}\simeq{X}^{k-1}\bigcup\Big(
                    \bigcup_{\alpha\in{I}}D_{\alpha}^{k}
                \Big)
            \end{equation}
            $X^{k}$ is obtained from $X^{k-1}$ by
            attaching $k$ cells $D_{\alpha}^{k}$ along
            their boundary $\partial{D}_{\alpha}^{k}$.
            \begin{ldefinition}{Quotient Space}
                The quotient space of a topological
                space $Z$ with respect to an equivalence
                relation $\sim$ is the set:
                \begin{equation}
                    Z/\sim=\{[x]:x\in{Z}\}
                \end{equation}
                Where $[x]$ is the equivalence class of
                $x$ with respect to $\sim$, and the
                topology:
                \begin{equation}
                    \tau_{\sim}=
                    \{p^{-1}(\mathcal{U}:
                    \mathcal{U}\in\tau\}
                \end{equation}
                Where $q$ is the projection mapping.
            \end{ldefinition}
            An equivalence relation identifies certain
            points in $Z$. With this we can see what it
            means to attach one space to another. Let
            $X$ and $Y$ be topological space, and let
            $A\subseteq{X}$. The attaching map is a
            continuous function $f:A\rightarrow{Y}$.
            Define the equivalence relation:
            \begin{equation}
                \sim=\{(a,f(a)):a\in{A}\}
            \end{equation}
            \begin{equation}
                Y\cup_{f}{X}=
                Y\cup{X}/\sim
            \end{equation}
            For each $k$ cell there is an attaching map
            $\varphi_{\alpha}:\partial{D}_{\alpha}^{k}\rightarrow{X}^{k-1}$.
            \begin{lexample}
                A zero dimensional CW complex is the
                disjoint union of a bunch of points.
                So $X=X^{0}$, and this has the
                discrete topology on it. That is,
                $\tau=\mathcal{P}(X)$, every set is
                open. The only defining characteristic
                is the cardinality of the space.
            \end{lexample}
            \begin{lexample}
                One dimensional CW complexes are a
                combination of edges and points. For
                each edge the endpoints must go to
                a point in the 0-skeleton $X^{0}$.
                We can send the two endpoints to the
                same point, which creates a loop, or
                to distinct points, which creates an
                edge.
            \end{lexample}
            \begin{lexample}
                Hawaiian ear-rings. A subset of
                $\mathbb{R}^{2}$ that looks like a
                CW complex, but is not.
            \end{lexample}
            Any manifold is a CW complex since all
            manifolds are triangulable. For example,
            $S^{2}$ is homeomorphic to a cube. The
            faces of a cube are homeomorphic to 
            $D^{2}$, and thus $S^{2}$ is a CW complex.
            A balloon is an example of a two dimensional
            CW complex.
            \begin{lexample}
                The real projective plane,
                $\mathbb{R}P^{2}$ is an important
                example that is encountered in topology.
                The easiest definition is the set of
                lines through the origin in
                $\mathbb{R}^{3}$. Equivalently, the
                set of all one dimensional
                vector subspaces of $\mathbb{R}^{3}$.
                We can think of this as a quotient
                space on $\mathbb{R}^{3}\setminus\{0\}$.
                \begin{equation}
                    \sim=\{(\mathbf{x},c\mathbf{x}):
                        \mathbf{c}\ne\mathbf{0},
                        c\in\mathbb{R}\}
                \end{equation}
                We can define:
                \begin{align}
                    \mathbb{R}P^{2}&=
                    \mathbb{R}^{3}\setminus
                    \{\mathbf{0}\}/\sim\\
                    &=S^{2}/\{(\mathbf{x},-\mathbf{x}:
                        \mathbf{x}\in{S}^{2}\}
                \end{align}
                That is, we identify all of the
                antipodal points on the unit sphere.
                This is a CW complex. To show this,
                take one zero-cell and one one-cell.
                This forms a loop. The attaching map
                $\varphi:S^{1}\rightarrow{S}^{1}$ is
                a two-to-one map.
            \end{lexample}
        \subsection{Homotopy Equivalence}
            \begin{ldefinition}{Homotopy}
                A homotopy of continuous maps
                $f_{t}:X\rightarrow{Y}$, $t\in[0,1]$, such that:
                \begin{equation}
                    F:X\times{I}\rightarrow{Y}
                    \quad\quad
                    F(x,t)=f_{t}(x)
                \end{equation}
                is continuous in the product space $X\times{I}$.
            \end{ldefinition}
            \begin{lexample}
                Let $X=S^{1}$, the unit circle, and let
                $Y=T^{2}$, the torus. This is
                $S^{1}\times{S}^{1}$. Then $X\times{I}$ is
                a cylinder. Circles on one of the inner circle
                can't be homotopic to circle on the other
                inner circle. Let
                $f:S^{1}\rightarrow\mathbb{R}^{2}$ be defined by:
                \begin{equation}
                    f(x,y)=(x,y)
                \end{equation}
                This is the inclusion map in the plane. Then
                $f\simeq{g}$, where $g(x,y)=0$. For let:
                \begin{equation}
                    F(x,y,t)=(tx,ty)
                \end{equation}
                This is continuous, for let:
                \begin{equation}
                    \tilde{F}:\mathbb{R}^{2}\times\mathbb{R}
                    \rightarrow\mathbb{R}^{2}\quad\quad
                    \tilde{F}(x,y,t)=(tx,ty)
                \end{equation}
                This is continuous since each of the
                coordinate functions is continuous. Thus
                $F$ is continuous in the subspace topology, which
                is the topology on $S^{1}\times{I}$. Functions
                that are homotopic to a point are called null
                homotopic.
            \end{lexample}
            \begin{theorem}
                Homotopic is an equivalence relation.
            \end{theorem}
            \begin{theorem}
                If $f:\mathbb{R}^{n}\rightarrow\mathbb{R}^{m}$
                is continuous, then $f$ is null homotopic.
            \end{theorem}
            \begin{theorem}
                If $f,g:\mathbb{R}^{n}\rightarrow\mathbb{R}^{m}$
                are continuous functions, then $f$ and $g$ are
                homotopic.
            \end{theorem}
            \begin{theorem}
                If $f:X\rightarrow\mathbb{R}^{n}$ is continuous,
                then $f$ is null homotopic.
            \end{theorem}
            \begin{proof}
                For let $F$ be defined by:
                \begin{equation}
                    F(x,t)=tf(x)
                \end{equation}
                Therefore, etc.
            \end{proof}
            \begin{ldefinition}{Homotopy Equivalent Spaces}
                Homotopy equivalent spaces are topological
                space $X$, $Y$ such that there exist continuous
                functions $f:X\rightarrow{Y}$ and
                $g:Y\rightarrow{X}$ such that
                $g\circ{f}\simeq{id}_{X}$ and
                $f\circ{g}\simeq{id}_{Y}$.
            \end{ldefinition}
            This definition is a weakening of the notion of
            homeomorphisms. Homeomorphisms are continuous
            functions $f,g$ such that $g\circ{f}=id_{X}$ and
            $f\circ{g}=id_{Y}$. Homeomorphic spaces are homotopy
            equivalent. The converse is not true, in general.
            \begin{lexample}
                Let $f:\mathbb{R}^{n}\rightarrow\{0\}$ be
                defined by $f(x)=0$, and let
                $g:\{0\}\rightarrow\mathbb{R}^{n}$ be defined
                by $g(x)=0$. Then $g\circ{f}=0$, and this
                is homotopy equivalent to the identity map, since
                the identity map is continuous and continuous
                maps are null homotopic. But also
                $f\circ{g}$ is the identity map for
                $\{0\}$. So $\mathbb{R}^{n}$ is homotopy
                equivalent to $\{0\}$. But $\mathbb{R}^{n}$
                is not homeomorphic to a point, simply because
                there is no bijection. A space that is
                homeomorphic to a point is called contractible.
            \end{lexample}
            \begin{ldefinition}{Retraction}
                A retraction of a space $X$ to a subset
                $A\subseteq{X}$ is a function $f:X\rightarrow{A}$
                such that $f$ is continuous and
                for all $a\in{A}$, $f(a)=a$.
            \end{ldefinition}
            \begin{ldefinition}{Deformation Retraction}
                A deformation retraction of a space
                $X$ onto a subset $A\subseteq{A}$ is a homotopy
                $f_{t}:X\rightarrow{X}$ such that $f_{0}=id_{X}$,
                $f_{1}:X\rightarrow{A}$ is a retract, and for
                all $t\in{I}$ and all $a\in{A}$, $f_{t}(a)=a$.
            \end{ldefinition}
            \begin{lexample}
                The contraction we saw from $\mathbb{R}^{n}$ to
                a point is an example of a deformation retract.
            \end{lexample}
            \begin{theorem}
                $X$ is a topological space, $A\subseteq{X}$, and
                if there is a deformation retract of $X$ onto
                $A$, then $X$ is homotopy equivalent to $A$.
            \end{theorem}
            Two steps or criteria that are common in algebraic
            topology.
            \begin{ldefinition}{CW Pair}
                A CW Pair, $(X,A)$, is a CW Complex $X$ and a
                subset $A\subseteq{X}$ such that $A$ is closed,
                and $A$ is the union of cells of $X$.
            \end{ldefinition}
            The set $A$ can be considered as a sub-complex
            of $X$.
            \begin{theorem}
                If $(X,A)$ is a CW pair and if $A$ is
                contracible, then $q:X\rightarrow{X}/A$
                is a homotopy equivalence.
            \end{theorem}
            So 1-Skeletons and 1 dimensional CW Complexes can
            be contracted down to a bunch of points with loops
            on the points. That is, homotopy equivalence is
            preserved. Circles are not contractible, as well
            we eventually see, so we cannot get rid of loops.
        \subsection{Homotopy Extension Property}
            Let $X(,\tau)$ be a topological space,
            $A\subseteq{X}$ a subspace, and let
            $f:A\rightarrow{Y}$ be continuous. Does
            $f$ extend to a map $\tilde{f}:X\rightarrow{X}$
            such that $\tilde{f}|_{A}=f$? Not always, let
            $A=S^{1}$ and $X$ be the torus. For which pairs
            $(X,A)$ is this always possible? Well, suppose
            it is true. Then $id_{A}:A\rightarrow{A}$ has
            an extension $\tilde{id}_{A}:X\rightarrow{A}$.
            But then $A$ is a retract of $X$. Thus, it
            is a necessary condition that $A$ is a retract
            of $X$ for it to be true. As it turns out, it is
            also a sufficient condition. If
            $r:X\rightarrow{A}$ is a retract
            \begin{ldefinition}{Homotopy Extension Property}
                A subset $A\subseteq{X}$ with the homotopy
                extension property is a set such that for all
                maps $f:X\rightarrow{Y}$ and for every
                homotopy $f_{t}:A\rightarrow{Y}$,
                $f_{0}=f|_{A}$, extends to a homotopy of
                $\tilde{f}_{t}:X\rightarrow{Y}$ with
                $\tilde{f}_{0}=f_{0}$.
            \end{ldefinition}
            \begin{theorem}
                Every CW Pair $(X,A)$ has the homotopy
                extension property.
            \end{theorem}
            \begin{proof}
                We will product a retract:
                \begin{equation}
                    X\times{I}\overset{r}{\longrightarrow}
                        X\times{0}\cup{A}\times{I}=Z
                \end{equation}
                First, $(D^{n},\partial{D}^{n})$, which is
                $(D^{n},S^{n-1})$, and this has the
                homotopy extension property.
            \end{proof}
            Recall that if $f:X\rightarrow{Y}$ is a map such that,
            for a given $A\subseteq{X}$, and for all $a\in{A}$,
            there is a fixed $y_{0}\in{Y}$ such that $f(a)=y$, then
            the quotient map $q:X\rightarrow{X/A}$ has a unique
            liften map $\overline{f}$ such that
            $\overline{f}\circ{q}=f$. Moreover, $\overline{f}$ is
            continuous with respect to the quotient topology.
            \begin{theorem}
                If $(X,A)$ is a CW pair, and if $A$ is
                homotopy equivalent to a point
                (Contractible), then the quotient map
                $q:X\rightarrow{X/A}$ is a homotopy
                equivalence.
            \end{theorem}
            \begin{proof}
                For let $(X,A)$ be a CW pair. Then it has the
                homotopy extension property. If $A$ is contractible
                then there is a homotopy of maps
                $f_{t}:A\rightarrow{A}$ such that $f_{0}=id_{A}$ and
                $f_{1}=a_{0}$, for some $a_{0}\in{A}$. We can
                extend this to $X$ by defined
                $f_{0}:X\rightarrow{X}$ as $f_{0}=id_{X}$. By
                the homotopy extension property, there's an
                extension $F_{t}:X\rightarrow{X}$ such that
                $F_{t}|_{A}=f_{t}$. Let $q$ be the quotient
                mapping and let $qf_{t}$ be the lifting mapping.
                Square $X-X$ on top, $X/A$ $X/A$ on bottom. Maps are
                $q$ and $qf_{t}$. Bottom map is
                $\overline{f}_{t}q$. Let $k$ be the mapping
                of $X/A$ to $X$. Then $q$ and $k$ are homotopy
                equivalences. Thus, $X$ and homotopy equivalent
                to $X/A$. Since $qk=\overline{f}_{1}\simeq{id}_{X/A}$
                and $kq=f_{1}\simeq{f}_{0}=id_{X}$.
            \end{proof}
            \begin{theorem}
                If $(X,A)$ is CW pair, $f,g:A\rightarrow{X_{0}}$
                are homotopic attaching maps, then:
                \begin{equation}
                    X_{0}\cup_{f}X\simeq{X}_{0}\cup_{g}X
                \end{equation}
            \end{theorem}
            \begin{theorem}
                $X\times{I}$ deformation retracts to
                $(X\times\{0\})\cup(A\times\{1\})$, then
            \end{theorem}
            This determines a deformation retract of:
            \begin{equation}
                Z=X_{0}\cup_{F}(X_{1}\times{I})
            \end{equation}
            As an aside, there are two things, a wedge product
            $\lor$ and a smash product $\land$. Given two pointed
            space $(X,x_{0})$, $(Y,y_{0})$, the wedge product is:
            \begin{equation}
                (X,x_{0})\lor(Y,y_{0})
                =X\coprod{Y}/\{x_{0}\sim{y}_{0}\}
            \end{equation}
            The smash product is:
            \begin{equation}
                X\land{Y}=
                X\times{Y}/\Big(
                    (\{0\}\times{Y})\cup(X\times\{y_{0}\})
                \Big)
            \end{equation}
            \begin{example}
                $S^{1}\land{S}^{1}$ is simply $S^{2}$.
                $S^{1}\times{S}^{1}$ is a torus, but we've collapsed
                the entire boundary down to a point. So we have a
                2-cell with no boundary, which is a sphere.
            \end{example}
        \subsection{Winding Number}
            \begin{theorem}
                If $f:[0,1]\rightarrow{S}^{1}$ is a continuous
                function, then there is a unique continuous lift
                $\tilde{f}:[0,1]\rightarrow\mathbb{R}$ such that
                $f=q\circ\tilde{f}$, where $q$ is the quotient map
                of $\mathbb{R}$ onto $S^{1}$.
            \end{theorem}
            \begin{proof}
                For let $\mathcal{U}_{1}=S^{1}\setminus\{(1,0)\}$ and
                $\mathcal{U}_{2}=S^{1}\setminus\{(0,1)\}$. This is an
                open cover of $S^{1}$. If $f:I\rightarrow{S}^{1}$ is
                continuous, then $f^{-1}(\mathcal{U}_{1})$ and
                $f^{-1}(\mathcal{U}_{2})$ form an open cover of
                $I$. But any open subset of $\mathbb{R}$ is the
                countable union of disjoint open intervals. Thus
                $f^{-1}(\mathcal{U}_{1})$ and
                $f^{-1}(\mathcal{U}_{2})$ are the countable union of
                disjoint open intervals. But $I$ is compact, and
                thus any open cover has a finite subcover. Therefore
                there are finitely many intervals $I_{1},\dots,I_{k}$
                such that $f(I_{k})\subseteq\mathcal{U}_{1}$ or
                $f(I_{k})\subseteq\mathcal{U}_{2}$.
            \end{proof}
            \begin{ldefinition}{Winding Number}
                The winding number of a loop
                $f:S^{1}\rightarrow{S}^{1}$ is:
                \begin{equation}
                    wind(f)+\tilde{f}(1)-\tilde{f}(0)
                \end{equation}
                Where $\tilde{f}$ is a lifting map of $f$ to
                $\mathbb{R}$.
            \end{ldefinition}
    \section{Stuff}
        HW Stuff. Chapter 0: 2, 3, 9, 16, 20. Bonus 6.
        \par\hfill\par
        A loop is a continuous function $f:S^{1}\rightarrow{X}$. Given
        a loop $f:S^{1}\rightarrow{S}^{1}$, there is a lifting map
        $\tilde{f}:S^{1}\rightarrow\mathbb{R}$. The winding number
        of the loop $f$ is defined as $w(f)=\tilde{f}(1)-\tilde{f}(0)$.
        \begin{theorem}
            If $f,g:S^{1}\rightarrow{S}^{1}$ are loops such that
            $f(0)=g(0)$, then $f\simeq{g}$ if and only if
            $w(f)=w(g)$.
        \end{theorem}
        \begin{proof}
            For let $\tilde{f},\tilde{g}$ be two lifts of $f$ and $g$,
            respectively, such that $\tilde{f}(0)=\tilde{g}(0)$.
            This can be done since given such lifts, they will differ
            by at most an integer. But if $w(f)=w(g)$, then
            $\tilde{f}(1)=\tilde{g}(1)$. Then let $H$ be the straight
            line homotopy between $\tilde{f}$ and $\tilde{g}$:
            \begin{equation}
                H(x,t)=\tilde{f}(x)t+(1-t)\tilde{g}(x)
            \end{equation}
            Let $h_{t}=qH$, where $q$ is the projection mapping. This
            is a homotopy between $f$ and $g$.
        \end{proof}
        Thus we have that the winding number is a homotopy invariant.
        Let $[X,y]$ be the set of homotopy equivalence classes of
        continuous maps $X\rightarrow{Y}$. Then $[S^{1},S^{1}]$ can
        be put into a one-to-one correspondence with $\mathbb{Z}$
        by mapping $f\mapsto{w}(f)$.
        \begin{theorem}
            There is no retract from $D^{2}$ onto it's boundary
            $\partial{D}^{2}=S^{1}$.
        \end{theorem}
        \begin{proof}
            Consider the identity map $id_{S^{1}}$. Then
            $w(id_{S^{1}})=1$. This is a loop in $D^{2}$ that is
            contractible to a point, and is thus null-homotopic.
            Suppose $r:D^{2}\rightarrow{S}^{1}$ is a retract. 
            Then $rf_{0}\simeq{r}f_{1}$, and thus
            $w(rf_{0})=w(rf_{1})$. But $w(rf_{0})=w(id_{S^{1}})=0$, and
            $rf_{1}$ is a point, so $w(rf_{1})=0$, a contradiction. Thus
            there is no retract.
        \end{proof}
        We now consider functors that take topological spaces to
        groups, mapping continuous functions to group homomorphisms.
        Recall that a path is a continuous function $f:I\rightarrow{X}$.
        A homotopy of paths $f,g$ is a function $F_{t}$ such that
        $F_{0}=f$ and $F_{1}=g$. The concatenation operation is:
        \begin{equation}
            (f*g)(t)=
            \begin{cases}
                f(2t),&0\leq{t}\leq\frac{1}{2}\\
                g(2t-1),&\frac{1}{2}<t\leq{1}
            \end{cases}
        \end{equation}
        \begin{theorem}
            If $U,V\subseteq{X}$ are closed sets, and if
            $U\cup{V}=X$, and if $k:X\rightarrow{Y}$ is such that
            $k|_{U}$ and $k|_{V}$ is continuous, then $k$ is continuous.
        \end{theorem}
        \begin{theorem}
            If $f_{0}\simeq{f}_{1}$, $g_{0}\simeq{g}_{1}$, then
            $f_{0}*g_{0}\simeq{f}_{1}*g_{1}$.
        \end{theorem}
        \begin{proof}
            Concatenate homotopies: $h_{t}=f_{t}*g_{t}$. This is a
            homotopy.
        \end{proof}
        A loop is a path $f:I\rightarrow{X}$ such that $f(0)=f(1)$, and
        this is called the base-point. We use this notion to define the
        fundamental group $\pi_{1}(X,x_{0})$.
        \begin{ldefinition}{Fundamental Group}
            The fundemtanl group of a topological space $X$ with a
            base point $x_{0}$ is the set:
            \begin{equation}
                \pi_{1}(X,x_{0})=\{[f]:f\in{C}(S^{1},X)\}
            \end{equation}
            Where equivalent functions are functions that are homotopy
            equivalent.
        \end{ldefinition}
        \begin{theorem}
            If $f:I\rightarrow{X}$, $\phi:I\rightarrow{I}$ is such that
            $\phi(0)=0$ and $\phi(1)=1$, then $f\circ\phi\simeq{f}$.
        \end{theorem}
        \begin{proof}
            Let $\phi_{t}(s)=(1-t)\phi(s)+ts$. This is a homotopy between
            $\phi$ and $id_{I}$. Let $f_{t}=f\phi_{t}$. This is a
            homotopy.
        \end{proof}
        \begin{theorem}
            If $X$ is a topological space, $x_{0}\in{X}$, and if
            $*$ is the concatenation operation, then
            $(\pi_{1}(X,x_{0}),*)$ is a group.
        \end{theorem}
        \begin{lexample}
            We have seen that $\pi_{1}(S^{1},x_{0})$ can be put into
            a correspondence with $\mathbb{Z}$, but moreover the
            group structure is preserved. That is,
            $(\pi_{1}(S^{1},x_{0}),*)$ is isomorphic to
            $(\mathbb{Z},+)$. While this group is Abelian,
            the fundamental of a topological space need not be, and
            in general it isn't.
        \end{lexample}
        \subsection{Lecture 6}
            Let $X$ be a space, and $x_{0}\in{X}$ a point in $X$.
            The fundamental group of $X$ with base point $x_{0}$
            is $\pi_{1}(X,x_{0})$, which is the set of loops with
            base point $x_{0}$, modulo homotopy, equipped with the
            concatenation operation.
            \begin{theorem}
                If $X$ and $Y$ are topological space, if
                $X\times{Y}$ has the product topology, and if
                $(x_{0},y_{0})\in{X}\times{Y}$, then:
                \begin{equation}
                    \pi_{1}(X\times{Y},(x_{0},y_{0}))
                    \simeq\pi_{1}(X,x_{0})\otimes\pi_{1}(Y,y_{0})
                \end{equation}
            \end{theorem}
            \begin{lexample}
                We have computed $\pi_{1}(S^{1},x_{0})$ for any
                point $x_{0}\in{S}^{1}$ by using the notion of
                winding number. We saw that:
                \begin{equation}
                    \pi_{1}(S^{1},x_{0})\simeq\mathbb{Z}
                \end{equation}
                But the torus $T^{2}$ can be seen as
                $S^{1}\times{S}^{1}$. Thus, for any point
                $x_{0}\in{T}^{2}$, we have:
                \begin{equation}
                    \pi_{1}(T^{2})\simeq
                    \mathbb{Z}\times\mathbb{Z}
                    =\mathbb{Z}^{2}
                \end{equation}
                We've seen that the torus can be identified by
                an equivalence relation on the square, which then
                creates two loops. These are the two inner circles
                of the torus. It turns out that any loop in the
                torus is going to be the product of these two loops.
                That is, the fundamental group is generated by
                two elements. By considering a trivial example,
                we see that this group has the relation:
                \begin{equation}
                    \pi_{1}(S^{1},x_{0})=
                    \langle{a,b}|aba^{-1}b^{-1}=e\rangle
                \end{equation}
                This relation says that:
                \begin{equation}
                    ab=ba
                \end{equation}
                And thus the group is Abelian. So we have that
                the fundamental group is an Abelian group
                generated by two elements, and thus we see the
                isomorphism with $\mathbb{Z}^{2}$.
            \end{lexample}
            Recall that a loop is a map $f:I\rightarrow{X}$
            such that $f(0)=f(1)$. This is equivalent to a
            function $f:S^{1}\rightarrow{X}$. We can use this to
            define the higher homotopy groups, $\pi_{n}(X,x_{0})$,
            by considering functions $f:S^{n}\rightarrow{X}$,
            modulo homotopy. For $n\geq{2}$,
            $\pi_{n}(X,x_{0})$ is Abelian. There's also
            $\pi_{0}(X,x_{0})$, which is not a group in general,
            but counts the number of path connected components.
            Moving on to the base point, this is mostly annoying.
            It seems that most of the time it doesn't matter which
            one we pick, provided that the space has one connected
            component. Recall that the fundamental group can be
            seen as a functor that maps pointed topological
            spaces to a group such that continuous maps
            $\varphi:(X,x_{0})\rightarrow(Y,y_{0})$ are mapped
            to group homomorphisms
            $\varphi_{*}:\pi_{1}(X,x_{0})\rightarrow\pi_{1}(Y,y_{0})$.
            Given a loop $f:S^{1}\rightarrow{X}$ and a continuous
            function $\varphi:X\rightarrow{Y}$, we take the
            composition $\varphi\circ{f}$ and map the equivalence
            class of $f$ to the equivalence class of
            $\varphi\circ{f}$. That is:
            \begin{equation}
                \varphi_{*}([f])=[\varphi\circ{f}]
            \end{equation}
            Then $\varphi$ is a group homomorphism between the
            two fundamental groups. Functors have the requirement
            that:
            \begin{align}
                (X,x_{0})\overset{f}{\longrightarrow}
                (Y,y_{0})\overset{g}{\longrightarrow}
                (Z,z_{0})\\
                \pi_{1}(X,x_{0})
                    \underset{\varphi_{*}}{\longrightarrow}
                \pi_{1}(Y,y_{0})
                    \underset{\psi_{*}}{\longrightarrow}
                \pi_{1}(Z,z_{0})
            \end{align}
            With the constraint that:
            \begin{equation}
                (\psi\varphi)_{*}=\psi_{1}\circ\varphi_{*}
            \end{equation}
            Given a homotopy of loops:
            \begin{equation}
                \varphi_{t}:(X,x_{0})\rightarrow(Y,y_{0})
            \end{equation}
            such that, for all $t\in{I}$, $\varphi_{t}(x_{0})=y_{0}$.
            Then:
            \begin{equation}
                (\varphi_{1})_{*}=(\varphi_{2})_{*}
            \end{equation}
            \begin{theorem}
                If $X$ is path connected, and if $x_{0},x_{1}\in{X}$,
                then:
                \begin{equation}
                    \pi_{1}(X,x_{0})\simeq\pi_{1}(X,x_{1})
                \end{equation}
            \end{theorem}
            \begin{proof}
                For let $h:I\rightarrow{X}$ be such that
                $h(0)=x_{0}$ and $h_{1}=x_{1}$. This is possible
                since $X$ is path connected. Define:
                \begin{equation}
                    \beta_{h}:\pi_{1}(X,x_{0})\rightarrow
                    \pi_{1}(X,x_{1})\quad\quad
                    \beta_{h}([f])=[h^{\minus{1}}\circ{f}\circ{h}]
                \end{equation}
                Then this is well defined and is a group homomorphism.
                Moreover it is an isomorphism since there's an
                inverse. Therefore, etc.
            \end{proof}
            \begin{theorem}
                If $\varphi_{t}:X\rightarrow{Y}$ is a homotopy,
                if $y_{0}=\varphi_{0}(x_{0})$, and if
                $y_{0}=\varphi_{1}(x_{0})$, then
                $\pi_{1}(Y,y_{0})$ is isomorphic to
                $\pi_{1}(Y,y_{1})$.
            \end{theorem}
            \begin{theorem}
                If $\varphi:X\rightarrow{Y}$ is a homotopy
                equivalence and if $\varphi(x_{0})=y_{0}$, then
                $\pi_{1}(X,x_{0})$ is isomorphic to
                $\pi_{1}(Y,y_{0})$.
            \end{theorem}
            \begin{proof}
                For if $\varphi:X\rightarrow{Y}$ is a homotopy
                equivalence then there is an
                $\psi:Y\rightarrow{X}$ such that
                $\varphi\psi\simeq id_{Y}$ and
                $\psi\varphi\simeq id_{X}$. Then
                $\psi_{*}\varphi_{*}=(\psi\varphi){*}$.
            \end{proof}
            That is, homotopy equivalent spaces have the same
            fundamental groups.
    \section{Lecture 7-ish, Maybe}
        We've computed the following:
        \begin{equation}
            \pi_{1}(\mathbb{Z},x_{0})=
            \begin{cases}
                \mathbb{Z},&n=1\\
                \{e\},&n\geq{2}
            \end{cases}
        \end{equation}
        Where $\{e\}$ denotes the trivial group. Let's
        now prove this.
        \begin{theorem}
            If $n\geq{2}$, and if $x_{0}\in{S}^{1}$, then
            $\pi_{1}(S^{n},x_{0})\simeq\{e\}$.
        \end{theorem}
        \begin{proof}
            What we want to do is take any loop $\gamma$,
            remove a point that
            $\gamma$ doesn't map to, and then use the fact
            that $S^{n}\setminus\{x\}$ is homeomorphic to
            $\mathbb{R}^{n}$. Since $\mathbb{R}^{n}$ is
            contractible, we are done. However, there exist
            space filling curves. So now we need to show that
            space filling curves can still be contracted.
            Pick $x\in{S]^[n}$ such that $x\ne{x}_{0}$. Then
            there is an ope ball $B$ centered about $x$ such
            that $x_{0}\notin{B}$. But $B$ is open and
            $\gamma$ is continuous Thus, $\gamma^{\minus{1}}$ is
            open in $[0,1]$. But since $x_{0}\notin{B}$,
            $0,1\notin\gamma^{\minus{1}}(B)$, and therefore
            the pre-image is an open subset of $\mathbb{R}$ as
            well. But every open subset of $\mathbb{R}$ is the
            disjoint union of open intervals. Let
            $(a,b)$ be one of these intervals. But then
            $f:(a,b)\rightarrow{B}$ is such that
            $f(a),f(b)\notin{B}$. But
            $\partial{B}\simeq{S}^{n-1}$ for $n\geq{2}$, and
            $S^{n-1}$ is path connected. Lift the image
            continuously to the bounded. This works if there
            are finitely many such intervals $(a,b)$, but there
            could be countably infinitely many. But
            $f^{\minus{1}}(x)$ is closed and is a subset of
            $S^{n}$, and is thereore closed and bounded and thus
            compact, by Heine-Borel. thus finitely many of the
            $(a_{i},b_{i})$ cover $f^{\minus{1}}(x)$. Move these
            ones. Thus, we can remove $x$ and complete the proof.
        \end{proof}
        We next move to Van Kampen's theorem. As an aside we must
        talk about free products of groups. The direct product
        of two groups $G_{1}$ and $G_{2}$ is defined by:
        \begin{equation}
            (x_{1},x_{2})*(y_{1},y_{2})=
            (x_{1}*y_{1},x_{2}*y_{2})
        \end{equation}
        There is a universal property on such groups that goes
        as follows:
        \begin{theorem}
            If $G_{1},G_{2}$, and $H$ are groups such that
            there are group homomorphisms
            $\varphi_{i}:H\rightarrow{G}_{i}$, then there is
            a unique group homomorphism
            $\psi:H\rightarrow{G}_{1}\times{G}_{2}$ such that:
            \begin{equation}
                \psi(h)=(\varphi_{1}(h),\varphi_{2}(h))
            \end{equation}
        \end{theorem}
        We want to flip this picture and define something called
        the \textit{free product} of $G_{1}$ and $G_{2}$. This
        is some group $G_{1}*G_{2}$ such that it contains,
        disjointly, all of the elements of $G_{1}$ and $G_{2}$,
        with no relations between elements of $G_{1}$ and $G_{2}$.
        This is the set of \textit{reduced words}. Here, a word
        is something in an \textit{alphabet}, or a set. We take
        as our alphabet the disjoint union of $G_{1}$ and $G_{2}$.
        A word is a finite ordered sequence from this alphabet.
        This includes the empty word, which is simply the empty
        set. A reduced word is a word such that $a_{n}$ and
        $a_{n+1}$ are not from the same group, and also none of
        the letters $a_{n}$ are the identity element in either
        $G_{1}$ or $G_{2}$. The product is then to concatenate
        two reduced words, and then further reduce the
        concatenation. Similarly, for a set of groups
        $\{G_{\alpha}:\alpha\in{I}\}$, the free product:
        \begin{equation}
            \ast_{\alpha\in{I}}G_{\alpha}
            =\{\textrm{Reduced Word in }
                \coprod_{\alpha\in{I}}G_{\alpha}\}
        \end{equation}
        Again, there is a unique group that satisfies the
        universal property mentioned earlier. As a special
        exmaple, consider $\mathbb{Z}\times\mathbb{Z}$. This has
        a Cayley group, and the Cayley actually reveals the
        group structure. That is, there are two generators,
        $a=(0,1)$ and $b=(1,0)$, and has the relation that
        $ab=ba$. The Cayley graph of $\mathbb{Z}*\mathbb{Z}$
        shows that this is the free group on 2 generators. It
        is a non-trivial question to show whether or not
        $F_{n}$ is isomorphic to $\mathbb{Z}^{n}$.
    \section{Van Kampen's Theorem}
        \begin{ltheorem}{Van Kampen's Theorem}
            If $X$ is path connected, if $x_{0}\in{X}$, and if
            $X$ is such that:
            \begin{equation}
                X=\bigcup_{\alpha\in{J}}A_{\alpha}
            \end{equation}
            Where $x_{0}\in{A}_{\alpha}$ for all
            $\alpha\in{I}$, $A_{\alpha}$ is path-connected,
            $A_{a}\cap{A}_{b}$ is path-connected, and
            $A_{a}\cap{A}_{b}\cap{A}_{c}$ is path-connected, then
            for all $[f]\in\pi_{1}(X,x_{0})$, $[f]$ can be
            factored in $\pi_{1}(X,x_{0})$ as the product:
            \begin{equation}
                [f]=[f_{1}]\cdot[f_{2}]\cdots[f_{m}]
            \end{equation}
            Where $f_{j}:I\rightarrow{A}_{\alpha_{j}}$ for each
            $j\in\mathbb{Z}_{m}$.
        \end{ltheorem}
        This is intuitively clear, given a loop in the space
        $X$ we can write it as a concatenation of different
        loops contained in each of the $A_{\alpha_{j}}$.
        But this theorem relates topology to algebra by the
        use of these free groups and free products.
        \subsection{Examples}
            \begin{lexample}
                Let $X$ be a torus with two internal circles
                identified. This is equivalent to two spheres
                attached by two lines, which is homotopic equivalent
                to two spheres and two loops, all joined at one
                point. That is,
                \begin{equation}
                    X\approx{S}^{1}\lor{S}^{1}\lor{S}^{2}\lor{S}^{2}
                \end{equation}
                The fundamental group of wedge sums has the
                following property:
                \begin{equation}
                    \pi_{1}(\lor_{\alpha}X_{\alpha})=
                    \star_{\alpha}\pi_{1}(X_{\alpha})
                \end{equation}
                The condition is that the base point in each
                $X_{\alpha}$ must have a neighborhood
                $\mathcal{U}_{\alpha}\in{X}_{\alpha}$ that
                deformation retracts onto the base point.
            \end{lexample}
            Given a graph, infinite or not, what is the fundamental
            group of the graph? First fint a maximal spanning
            tree. Tree's are contractible, and thus we have:
            \begin{equation}
                \pi_{1}(Graph)\simeq\pi_{1}(graph/tree)
            \end{equation}
            This last thing is the free group with generators of
            the edges the are not removed in the modulo.
            A know is a diffeomorphic copy of $S^{1}$ in
            $\mathbb{R}^{3}$. Links are several knots put together.
            Moving on to finite CW complex, consider $X$, which
            is $n\in\mathbb{N}$ dimensional, and has skeletons
            $X^{0},\dots,X^{n}$. Then $\pi_{1}(X^{1})$ is a
            free group since $X^{1}$ is a graph. What about
            $\pi_{1}(X^{2})$? We use Van Kampen's theorem to
            simplify this problem. $X^{2}$ is obtained from
            $X^{1}$ by attaching 2-cells $e_{\alpha}^{2}$,
            $\alpha\in{J}$, where $J$ is some index set, via
            attaching maps:
            $\varphi_{\alpha}:S^{1}\rightarrow{X}^{1}$.
            \begin{theorem}
                $\pi_{1}(X^{2})=\pi_{1}(X^{1})/N$
                Where $N$ is the normal subgroup generated by
                $[\gamma_{\alpha}\varphi_{\alpha}\gamma_{\alpha}]$
                in $\pi_{1}(X^{1})$.
            \end{theorem}
            \begin{theorem}
                $\pi_{1}(X)\simeq\pi_{1}(X^{2})$.
            \end{theorem}
            So what is $\pi_{1}$ of a two torus? We can write
            the two torus as an octogon with relations on the
            sides. There are four such relations, and we get
            that the fundamental group is the group generated by
            four elements with the condition that:
            \begin{equation}
                aba^{\minus{1}}b^{\minus{1}}
                cdc^{\minus{1}}d^{\minus{1}}=e
            \end{equation}
            \begin{theorem}
                If $X$ is a CW complex, and if
                $x_{0}$ is a vertex, or a point in the one skeleton
                of $X$, then:
                \begin{equation}
                    \pi_{1}(X,x_{0})=\pi_{1}(X^{2},x_{0})
                    =\pi_{1}(X,x_{0})/N
                \end{equation}
            \end{theorem}
            There are higher homotopy groups, $\pi_{k}(X,x_{0})$.
            These are homotopy classes of maps
            $f:S^{k}\rightarrow{X}$ such that $f(0)=x_{0}$.
            \begin{theorem}
                If $X$ and $Y$ are CW complexes, if
                $f:X\rightarrow{Y}$ is continuous, and if $f$ is
                a homotopy equivalence, then there are group
                homomorphisms
                $f_{*}:\pi_{k}(X,x_{0})\rightarrow{\pi}_{k}(Y,y_{0})$
                for all $k\in\mathbb{N}$.
            \end{theorem}
            Whitehead's theorem says the converse of this is true
            as well.
            \begin{ltheorem}{Whitehead's Theorem}
                The converse of the previous theorem is true.
            \end{ltheorem}
            \begin{theorem}
                If $X$ is a topological space, then there exists
                a CW complex $X'$ and a map
                $f:X'\rightarrow{X}$ such that
                $f_{*}:\pi_{k}(X')\simeq\pi_{k}(X)$.
            \end{theorem}
            This says that every topological space has a unique,
            up to homotopy equivalence, CW complex that approximates
            the space very well.
    \section{Covering Spaces}
        There's a deep link between covering spaces and Galois
        theory. When studying the fundamental group of a circle
        we came across loops $f:I\rightarrow{S}^{1}$ such that
        $f(0)=f(1)$. We studied the map
        $p(t)=\exp(2\pi{i}t)$ which spiralled $\mathbb{R}$ into
        the circle, and used this to define the winding number
        and calculate the fundamental group. This notion
        generalizes to other spaces.
        \begin{ldefinition}{Covering Space}
            A covering space of $X$ is a space $\tilde{X}$ such
            that, for al $x\in{X}$, there is an open neighborhood
            $\mathcal{U}_{x}\subseteq{X}$ such that
            $p^{\minus{1}}(\mathcal{U})$ is the disjoint union
            of open sets in $\tilde{X}$ and such that
            the image under $p$ of one of these sets is a
            homeomorphism with $\mathcal{U}$..
        \end{ldefinition}
        The first example is $\mathbb{R}$ and $S^{1}$. We will prove
        that nice enough topological spaces $X$ have a covering
        space $\tilde{X}$ such that $\tilde{X}$ is contractible.
        Such covers are unique up to homotopy. These are called
        the universal covers of the space. We will also show that
        there is a one-to-one correspondence between subgroups
        of $\pi_{1}(X)$ and covering spaces. That is, there
        is a function
        $\rho_{*}:\pi_{1}(\tilde{X})\rightarrow\pi_{1}(X)$ that
        is injective.
    \section{Universal Cover}
        We now discuss the existence of universal covers.
        This notion relations to Lie groups, representation theory,
        the special orthogonal group $SO(n)$, and it occurs
        in modern physics. $SO(n)$ is not simply connected, but
        it's universal cover, the spin group, is simply
        connected.
        \begin{ldefinition}{Simply Connected Space}
            A simply connected topological space is a path
            connected topological space $(X,\tau)$ such that
            for any two points $x,y\in{X}$ and any two paths
            $\gamma_{1},\gamma_{2}$ between $x$ and $y$,
            $\gamma_{1}$ and $\gamma_{2}$ are homotopic.
        \end{ldefinition}
        That is, $\pi_{0}(X)$ is trivial, there is only one
        path component, and $\pi_{1}(X,x_{0})$ is also
        trivial for all $x_{0}\in{X}$.
        \begin{lexample}
            $\mathbb{R}^{n}$ is simply connected for all
            $n\in\mathbb{N}$. Given two paths between the same
            two points, the straight line homotopy is a homotopy
            between such paths. For $n\geq{2}$, $S^{n}$ is
            also simply connected. However, $S^{1}$ is not
            simply connected. For $\mathbb{RP}^{2}$, we know
            that $\pi_{1}(\mathbb{RP}^{2})=\mathbb{Z}/2\mathbb{Z}$,
            which is a two element group. Thus there is, up to
            homotopy, only one non-trivial loop in
            $\mathbb{RP}^{2}$. Therefore the real projective plane
            is not simply connected. There's a clear candidate for
            the universal cover of $\mathbb{RP}^{2}$, and that is
            the sphere $S^{2}$.
        \end{lexample}
        \begin{lexample}
            We know that the torus $T^{2}=S^{1}\times{S}^{1}$ is
            not simply connected, since the fundamental group of
            $T^{2}$ is the product of the fundamental group
            of $S^{1}$ with itself, that is, $\mathbb{Z}^{2}$. The
            universal cover of $T^{2}$ is $\mathbb{R}^{2}$. If
            we had a surface with $n$ wholes, it is not
            automatically clear what the universal cover is, but
            this turns out to be $\mathbb{R}^{2}$ as well.
        \end{lexample}
        One question that arises is which topological spaces have
        a universal cover? It is true for all CW complexes, and
        there is the following necessary condition.
        If $X$ is a topological space, and if there is a
        universal cover $\tilde{X}$, then for all
        $x\in{X}$, if $\tilde{x}$ is such that
        $p(\tilde{x})=x$, then $\mathcal{U}\subseteq{X}$, where
        $x\in\mathcal{U}$, evenly covered and
        $\tilde{x}\in\tilde{\mathcal{U}}$, which is homeomorphic
        to $\mathcal{U}$. Given any loop
        $\gamma:I\rightarrow\mathcal{U}$, where $\gamma(0)=x$,
        this lifts to a loop $\tilde{\gamma}$ in
        $\tilde{\mathcal{U}}$. But $\tilde{\mathcal{U}}$ is
        simply connected, and thus $\tilde{\gamma}$ is null
        homotopic in $\tilde{X}$. Thus, we have the following
        definition:
        \begin{ldefinition}{Semi-Locally Simply Connected Space}
            A semi-locally connected space is a topological
            space $(X,\tau)$ such that for all $x\in{X}$ there is
            a neighborhood $\mathcal{U}$ such that for loop in
            $\mathcal{U}$ contracts in $X$.
        \end{ldefinition}
        When speaking of a local property, one usually requires that
        a certain property holds in every open set about a certain
        point. For example, locally compact spaces are topological
        spaces such that for every point $x$ and every open
        neighborhood of $x$ there is a compact subset $V$ contained
        in this neighborhood. We have the following chain:
        \begin{equation}
            \begin{split}
                \textrm{CW}\Longrightarrow
                &\textrm{Locally Contractible}\Longrightarrow
                \textrm{Locally Simply-Connected}
                \Longrightarrow\cdots\\
                &\cdots\Longrightarrow
                \textrm{Semi-Locally Simply-Connected}
            \end{split}
        \end{equation}
        There are two examples of spaces that look very similar,
        but not homeomorphic. One is the Hawaiian earrings, and
        the other is the countable $\lor{S}^{1}$. The Hawaiian
        earrings are not semi-locally simply connected, whereas
        the other set is locally contractible. Thus these two
        spaces are not homeomorphic. Now for a sufficient condition
        for $X$ to have a universal cover.
        \begin{theorem}
            If $X$ is path connected, locally path conneccted,
            and semi-locally path connected, then there exists
            a simply connected cover of $X$.
        \end{theorem}
        \begin{proof}
            Let $\tilde{X}$ be the set of homotopy equivalence
            classes of paths $\gamma:I\rightarrow{X}$ such that
            $\gamma(0)=x_{0}$. We need a function
            $p:\tilde{X}\rightarrow{X}$ such that
            $p([\gamma])=\gamma(1)$. Next we need a topology
            on $\tilde{X}$. Recall that for a basis for a topology
            of a topological space $Y$ is a collection
            $\mathcal{B}$ of open subsets of $Y$ with the property
            that for all $y\in{Y}$, $\mathcal{U}\subseteq{Y}$ is
            open and $y\in\mathcal{U}$, there is a $V\in\mathcal{B}$
            such that $y\in{V}\subseteq\mathcal{U}$. For a set
            $Z$, a collection $\mathcal{B}$ of subsets of $Z$
            can be used a the basis of a topology on $Z$ if
            for all $U,V\in\mathcal{B}$, and for all
            $x\in{U}\cap{V}$, there is a $W\in\mathcal{B}$ such
            that $x\in{W}\subseteq{U}\cap{V}$. Define
            $\mathcal{B}$ as follows:
            \begin{equation}
                \mathcal{B}=\{B\subseteq{X}:B\in\tau,
                    B\textrm{ path connected and }
                    \pi_{1}(B)\rightarrow\pi_{1}(X)
                    \textrm{ is trivial}\}
            \end{equation}
            $\mathcal{B}$ is a basis for the topology of
            $X$. For since $X$ is semi-locally simply connected,
            for all $x\in{X}$ there is a $V\subseteq{X}$ such
            that $x\in{V}$ and
            $\pi_{1}(V)\rightarrow\pi_{1}(X)$ is trivial. But
            $X$ is locally path-connected, and thus there is an
            open $W\subseteq{U}\cap{V}$ such that $x\in{W}$ and
            $W$ is path connected. From this we need to make a basis
            for $\tilde{X}$. For $\mathcal{U}\in\mathcal{B}$,
            let $\gamma:I\rightarrow{X}$ be such that
            $\gamma(0)=x\in\mathcal{U}$ and $\gamma(0)=x_{0}$.
            Define $\mathcal{U}_{\gamma}$ by:
            \begin{equation}
                \mathcal{U}_{\gamma}=\{
                    [\gamma\eta]:\eta(0)=\gamma(1)\}
            \end{equation}
            The set of all $\mathcal{U}_{\gamma}$ define a basis
            for a topology on $\tilde{X}$.
        \end{proof}
        \begin{theorem}
            If $X$ is a path connected CW complex, then
            $X$ has a simply connected cover.
        \end{theorem}
        \begin{theorem}
            If $X$ is a manifold, then $X$ has a simply
            connected cover.
        \end{theorem}
    \section{Group Actions}
        Let $G$ be a discrete group, and $X$. There is a notion
        called a group action of $G$ on $X$. There are two ways
        of viewing this, as a function
        $p:G\times{X}\rightarrow{X}$, or as a function
        $p:G\rightarrow\textrm{Perm}(X)$, where $\textrm{Perm}$
        denotes the set of permutations of $X$. A third definition
        is $\rho(g):X\rightarrow{X}$ for $g\in{G}$.
        A group action obeys the group law:
        \begin{equation}
            p(g_{1},g_{2})=p(g_{1})p(g_{2})
        \end{equation}
        Given a topological space $X$, $\rho(g)$ continuous
        would imply that it is a homeomorphism, since it has
        a continuous inverse.
        \begin{lexample}
            $\mathbb{Z}^{2}$ acts on $\mathbb{R}^{2}$:
            \begin{equation}
                \rho:\mathbb{Z}^{2}\rightarrow
                \textrm{Homeo}(\mathbb{R}^{2})
            \end{equation}
            This is defined by
            $\rho(m,n):\mathbb{R}^{2}\rightarrow\mathbb{R}^{2}$
            by $(x,y)\mapsto(x+m,y+n)$. This translates all points
            in $\mathbb{R}^{2}$ by the vector $(m,n)$. It suffices
            to define $\rho(g)$ on the generators $g$ of $G$.
            $\mathbb{Z}^{2}$ has the generators $(1,0)$ and $(0,1)$.
            So:
            \begin{subequations}
                \begin{align}
                    \rho((1,0))(x,y)&=(x+1,y)\\
                    \rho((0,1))(x,y)&=(x,y+1)
                \end{align}
            \end{subequations}
            We now need to chack that
            $\rho(a)\rho(b)=\rho(b)\rho(a)$.
        \end{lexample}
        \begin{lexample}
            Let $G$ be the group defined by the presentation:
            \begin{equation}
                G=\langle{a,b|ab=ba^{\minus{1}}}\rangle
            \end{equation}
            This acts on $\mathbb{R}^{2}$ by:
            \begin{subequations}
                \begin{align}
                    \rho((1,0))(x,y)&=(x+1,y)\\
                    \rho((0,1))(x,y)&=(\minus{x},y+1)
                \end{align}
            \end{subequations}
            We now need to check:
            \begin{equation}
                \rho(a)\big(\rho(b)(x,y)\big)=
                \rho(a)(\minus{x},y+1)=
                (\minus{x}+1,y+1)
            \end{equation}
            And also:
            \begin{equation}
                \rho(b)\big(\rho(a^{\minus{1}})(x,y)\big)=
                \rho(b)(x-1,y)=(\minus{x}+1,y+1)
            \end{equation}
            And thus this is a valid Group action.
        \end{lexample}
        Given a group $G$ and a set $X$, we consider the
        \textit{orbit} space $G/X$ as a set. The elements are
        called the orbits. Given a topology on $X$, we take
        the topology on the orbit space to be the quotient
        topology.
        \begin{lexample}
            Taking the group action discussed earlier of
            $\mathbb{Z}^{2}$ over $\mathbb{R}^{2}$, the orbit
            space $\mathbb{R}^{2}/\mathbb{Z}^{2}$. and this is
            homeomorphic to $T^{2}$. In the other example, with
            $G=\langle{a,b|ab=ba^{\minus{1}}}\rangle$,
            we have that the orbit space is
            $\mathbb{R}^{2}/G$ is the Klein bottle.
        \end{lexample}
        In general, if $\pi_{1}(X,x_{0})=G$, then $G$ acts on
        the covering space
        $p:\tilde{X}\rightarrow{X}$ with $\pi_{1}(X)=0$. In
        particular, if $X$ is a CW complex.
        \begin{lexample}
            Let $p:\mathbb{R}\rightarrow{S}^{1}$ be defined by
            $p(t)=\exp(2\pi{i}t)$. We know that
            $\pi_{1}(S)=\mathbb{Z}$. The generator of
            $\mathbb{Z}$ acts by mapping
            $x\mapsto{x}+1$. Note that:
            \begin{equation}
                \mathbb{R}/\pi_{1}(S^{1})\simeq
                \mathbb{R}/\mathbb{Z}\simeq
                S^{1}
            \end{equation}
            This is, in general, true. COnsider the Klein bottle
            The universal cover of the Klein bottle is
            $\mathbb{R}^{2}$. 
        \end{lexample}
        \begin{theorem}
            If $X$ is a topological space that is path connected
            and a CW complex, then for every subgroup
            $H\subseteq\pi_{1}(X)$, there exists a cover
            $p:\tilde{X}_{H}\rightarrow{X}$ such that:
            \begin{equation}
                p_{*}(\pi_{1}(\tilde{X}_{H},\tilde{x}_{0}))=H
            \end{equation}
            For suitable choice of $\tilde{x}_{0}$.
        \end{theorem}
        \begin{proof}
            Let $\tilde{X}\rightarrow{X}$ be a simply
            connected cover. Since $\pi_{1}(X,x_{0})=G$ acts on
            $\tilde{X}$, $H$ also acts on $\tilde{X}$. Now the
            quotient space $\tilde{X}/H$ has the desired
            properties.
        \end{proof}
        \subsection{Lifting Criterion}
            Given a space $X$ and a covering space
            $\tilde{X}$ with covering
            $p:\tilde{X}\rightarrow{X}$, and given a space
            $Y$ with a map $f:Y\rightarrow{X}$, when is there a
            \textit{lift} of the map $f$ to the covering
            space $\tilde{X}$? it seems natural to impose certain
            criterion on $Y$, and we require that $Y$ is
            path connected and locally path connected.
            1.33 in Hatcher.
            \begin{theorem}
                If $p:\tilde{X}\rightarrow{X}$ is a covering space,
                if $f:Y\rightarrow{X}$ is continuous, then a
                lift $\tilde{f}$ exists if and only if
                $f_{*}:\pi_{1}(Y,y_{0})\rightarrow\pi_{1}(X,x_{0})$
                is such that
                $\textrm{Ran}(f_{*})\subseteq\textrm{Ran}(p_{*})$,
                where $\textrm{Ran}$ denotes that range of the
                functions.
            \end{theorem}
            \begin{proof}
                If $\tilde{f}$ exists, then
                $f=\tilde{f}\circ{p}$, and thus
                $f_{*}=p_{*}\circ{f}_{*}$, so the range of
                $f_{*}$ is contained within the range of
                $p_{*}$. The hard part is going the other way.
                Now we need to construct such a lift.
            \end{proof}
    \section{Homology}
        \subsection{History}
            The basic root of the theory stems from the
            Euler number of a graph in the plane. That is,
            from his classic formula:
            \begin{align}
                F-E+V&=1
                \tag{Planar Graphs}\\
                F-E+V&=2
                \tag{Platonic Solids}
            \end{align}
            Similarly, if one triangulates a sphere, we again
            obtain $F-E+V=2$. So there's nothing to do with the
            nature of the platonic solids here, but rather that
            all of these objects are homeomorphic to the
            unit sphere. What about the torus? There's a formula
            for any surface with genus $g$, and we obtain:
            \begin{equation}
                F-E+V=2-2g
            \end{equation}
            So, for a torus we get zero. Riemann furthered the
            theory in the 1850's when he introduced the notion of
            connectivity number. Betti and Riemann continue this
            work around 1870 and introduced the idea of Betti
            numbers. This is the number of cuts of dimension $k$
            are needed to disconnect a space. For a genus
            two surface, we see that the first Betti number is
            four, since we need two cuts for each hole. Moving on,
            Emmy Noether added more to the theory in 1925
            when she related this to groups. For example the
            Betti number 4 somehow relates to the group
            $\mathbb{Z}^{4}$, with Torsion. The first modern
            definition of homology groups comes from Mayer and
            Vietoris in 1928. Between 1930 and 1950, the theory
            began to develop into its current form. It now extends
            beyond topology and into algebra and group theory.
            There is also the notion of cohomology that came along
            with this.
        \subsection{Singular Homology}
            \begin{ldefinition}{Simplex}
                A simplex of degree $n$ is a subset of
                $\mathbb{R}^{n}$ form by $n$ points
                that are affinely independent
                $v_{0},\dots,v_{N}$, defined by the
                convex hull:
                \begin{equation}
                    [v_{0},\dots,v_{n}]=
                    \{\sum_{k=0}^{n}t_{k}v_{k}:
                        \sum_{k=0}^{n}t_{k}=1,t_{k}\geq{0}\}
                \end{equation}
            \end{ldefinition}
            The standard simplex, denote $\Delta^{n}$, is the
            convex hull of $e_{1},\dots,e_{n+1}$. These are
            triangles, or tetrahedron's, etc.
            Every n-simplex with ordered vertices
            $[v_{0},v_{1},\dots,v_{n}]$ is canonically homeomorphic
            to $\Delta^{n}$. Take
            $t=(t_{0},t_{1},\dots,t_{n})$ and map this to
            $t_{0}v_{0}+\dots+t_{n}v_{n}$. In particular, each
            of the faces of an $n$ simplex comes with a map:
            \begin{equation}
                [v_{0},\dots,v_{j},\dots,v_{n}]
                \mapsto[v_{0},\dots,v_{j-1},v_{j+1},\dots,v_{n}]
            \end{equation}
            This is equivalent to the face mapping
            $\Delta^{n-1}\mapsto\Delta^{n}$.
            \subsubsection{Simplicial Homology}
                Defined for $\Delta$ complexs.
                \begin{ldefinition}{$\Delta$ Complex}
                    A topological space $X$ together with the
                    following structure:
                    \begin{enumerate}
                        \item A collection of continuous
                              maps $\sigma:\Delta^{n}\rightarrow{X}$.
                        \item $\sigma(\Delta^{n}\setminus%
                               \partial\Delta^{n})=e_{\alpha}^{n}$
                        \item The topology of $X$ is a CW topology.
                              That is, $A\subseteq{X}$ is open
                              if and only if
                              $\sigma_{\alpha}^{\minus{1}}(A)$ is
                              open in $\Delta^{n}$ for all
                              $\alpha\in{J}$.
                        \item For $\sigma_{\alpha}:%
                              \Delta^{n}\rightarrow{X}$ in the
                              structure, so is
                              $\sigma_{\alpha}\circ{F}_{j}^{n}$,
                              where $F_{j}^{n}$ is the face map.
                    \end{enumerate}
                \end{ldefinition}
                \begin{ldefinition}{Chain Complex}
                    A sequence of a Abelian groups
                    $C_{n}$, one for each $n\in\mathbb{Z}$,
                    together with group homomorphisms
                    $\partial_{n}:C_{n}\rightarrow{C}_{n-1}$.
                    Such that $\partial_{n-1}\circ\partial_{n}=0$. 
                    Equivalently,
                    $\mathrm{Im}(\partial_{n})\subseteq%
                     \mathrm{ker}(\partial_{n-1})$.
                \end{ldefinition}
                We can also let:
                \begin{equation}
                    C_{\cdot}=\oplus_{n\in\mathbb{Z}}C_{n}
                \end{equation}
                With a homomorphism:
                \begin{equation}
                    \partial:C_{\cdot}\rightarrow{C}_{\cdot}
                \end{equation}
                That is a degree -1 homomorphism such that
                $\partial^{2}=0$. The homology of a chain
                complex is the collection of groups:
                \begin{align}
                    Z_{n}&=\mathrm{ker}(\partial_{n})
                        \subseteq{C}_{n}\\
                    B_{n}&=\mathrm{Im}(\partial_{n+1})
                        \subseteq{C}_{n}\\
                    H_{n}&=Z_{n}/B_{n}
                \end{align}
            \subsubsection{Simplicial Homology}
                Let $X$ be a $\Delta$ complex. That is, the
                disjoint union of open simplices. We want to create
                a chain complex:
                \begin{equation}
                    \cdots\longrightarrow\Delta_{n}(X)
                    \overset{\partial_{n}}{\longrightarrow}
                    \Delta_{n-1}(X)
                    \overset{\partial_{n-1}}{\longrightarrow}
                    \cdots
                    \overset{\partial_{1}}{\longrightarrow}
                    \Delta_{0}(X)
                    \longrightarrow{0}
                \end{equation}
                Form a free abelian group generated by
                $\rho_{\alpha}^{n}$.
                \begin{equation}
                    \partial(\sigma)=
                    \sum(\minus{1})^{j}\sigma|
                    [v_{0},\dots,v_{j-1},v_{j+1},\dots,v_{n}]
                \end{equation}
                The $(\minus{1})^{j}$ term gives an orientation
                to simplices.
                \begin{theorem}
                    $\partial_{n-1}\circ\partial_{n}=0$.
                \end{theorem}
            \subsection{Singular Homology}
                Let $X$ be a topological space. A
                singular $n$ simplex in $X$ is a continuous map
                $\sigma:\Delta^{n}\rightarrow{X}$.
                \begin{theorem}
                    If $X$ has $k$ path-connected components, then
                    $H_{0}(X)=\mathbb{Z}^{k}$.
                \end{theorem}
                \begin{theorem}
                    If $X$ is the union f $X_{\alpha}$, where
                    $X_{\alpha}$ is path connected, thn:
                    \begin{equation}
                        H_{n}(X)=\oplus{H}_{n}(X_{\alpha})
                    \end{equation}
                \end{theorem}
                \begin{theorem}
                    If $X$ is path connected, then
                    $H_{0}(X)=\mathbb{Z}$.
                \end{theorem}
                What is an $n$ cycle? The abstract definition
                is some finite combination of singular simplices:
                $\sigma:\Delta^{n}\rightarrow{X}$, allowing
                repetitions.
\end{document}