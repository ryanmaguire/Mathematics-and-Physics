\documentclass[crop=false,class=book,oneside]{standalone}
%----------------------------Preamble-------------------------------%
%---------------------------Packages----------------------------%
\usepackage{geometry}
\geometry{b5paper, margin=1.0in}
\usepackage[T1]{fontenc}
\usepackage{graphicx, float}            % Graphics/Images.
\usepackage{natbib}                     % For bibliographies.
\bibliographystyle{agsm}                % Bibliography style.
\usepackage[french, english]{babel}     % Language typesetting.
\usepackage[dvipsnames]{xcolor}         % Color names.
\usepackage{listings, lstlinebgrd}      % Verbatim-Like Tools.
\usepackage{mathtools, esint, mathrsfs} % amsmath and integrals.
\usepackage{amsthm, amsfonts}           % Fonts and theorems.
\usepackage{tabularx}
\usepackage{tcolorbox}                  % Frames around theorems.
\usepackage{upgreek}                    % Non-Italic Greek.
\usepackage{paracol}                    % Two-column styling.
\usepackage{wrapfig}                    % Wrap text around figure.
\usepackage{fmtcount, etoolbox}         % For the \book{} command.
\usepackage[newparttoc]{titlesec}       % Formatting chapter, etc.
\usepackage{titletoc}                   % Allows \book in toc.
\usepackage[nottoc]{tocbibind}          % Bibliography in toc.
\usepackage[titles]{tocloft}            % ToC formatting.
\usepackage{multicol, enumitem}         % Multi-column/enumerate.
\usepackage{import}                     % Import external files.
\usepackage{pgfplots, tikz}             % Drawing/graphing tools.
\usetikzlibrary{
    calc,                   % Calculating right angles and more.
    angles,                 % Drawing angles within triangles.
    arrows.meta,            % Latex and Stealth arrows.
    quotes,                 % Adding labels to angles.
    positioning,            % Relative positioning of nodes.
    decorations.markings,   % Adding arrows in the middle of a line.
    patterns,
    arrows,
    shapes,
    shapes.geometric,
    cd,
    hobby,
    babel
}                                       % Libraries for tikz.
\pgfplotsset{compat=1.9}                % Version of pgfplots.
\usepackage[font=scriptsize,
            labelformat=simple,
            labelsep=colon]{subcaption} % Subfigure captions.
\usepackage[font={scriptsize},
            hypcap=true,
            labelsep=colon]{caption}    % Figure captions.
\usepackage{hyperref}                   % Allows for hyperlinks.
\hypersetup{
    colorlinks=true,
    linkcolor=blue,
    filecolor=magenta,
    urlcolor=Cerulean,
    citecolor=SkyBlue
}                           % Colors for hyperref.
\usepackage[toc,acronym,nogroupskip]{glossaries} % Glossaries and acronyms.
\usepackage[subpreambles=false]{standalone}      % Complileable sub files.

% Various font stuff from kiwi.
% Use this for Times text and Computer Modern math
%\usepackage{times}

% Quite nice
%\usepackage[charter, greekfamily=, greekuppercase=italicized]{mathdesign}
%\usepackage[utopia, greekuppercase=italicized]{mathdesign}    % Math is narrower

% Use this for Times text and math
%\usepackage{newtxtext}
%\usepackage[libertine,cmintegrals]{newtxmath}
%\usepackage{fix-cm}

%\usepackage{txfontsb}
% or
%\usepackage{mathptmx}

%\usepackage[scaled=0.92]{helvet}
%\renewcommand{\rmdefault}{ptm}

%\usepackage{mathpazo}    % add possibly `sc` and `osf` options
%\usepackage{eulervm}

%\usepackage{fourier}
%\renewcommand{\rmdefault}{ptm}
%\usepackage{mathptm}

%\usepackage{fontspec}
%\setmainfont{lmodern}

%\usepackage[varg]{txfonts}
%\usepackage{fouriernc}
%\usepackage{mathpazo}

%\usepackage{bookman}
%\usepackage[scaled]{uarial}
%\usepackage[scaled]{helvet}
%\renewcommand*\familydefault{\sfdefault}
%\usepackage[math]{anttor}

%\newcommand\fgeorgia{\fontfamily{jvn}\selectfont}
%\newcommand\ftimes{\fontfamily{ptm}\selectfont}
%\newcommand\fhelvetica{\fontfamily{phv}\selectfont}
%\newcommand\fcourier{\fontfamily{pcr}\selectfont}
%\newcommand\fbookman{\fontfamily{pbk}\selectfont}
%\newcommand\fnewcentury{\fontfamily{pnc}\selectfont}
%\newcommand\fpalatino{\fontfamily{ppl}\selectfont}
%\newcommand\favantgarde{\fontfamily{pag}\selectfont}
%\newcommand\fnormal{\normalfont}
%\newcommand\fsize[1]{\ifnum#1>0\fontsize{#1}{#1}\selectfont\else\normalsize\fi}
%------------------------Theorem Styles-------------------------%
% Define theorem style for default spacing and normal font.
\newtheoremstyle{normal}
    {\topsep}               % Amount of space above the theorem.
    {\topsep}               % Amount of space below the theorem.
    {}                      % Font used for body of theorem.
    {}                      % Measure of space to indent.
    {\bfseries}             % Font of the header of the theorem.
    {}                      % Punctuation between head and body.
    {.5em}                  % Space after theorem head.
    {}

% Define theorem style for default spacing with italicized font.
\newtheoremstyle{normalit}{\topsep}{\topsep}
                {\itshape}{}{\bfseries}{}{.5em}{}

% Italic header environment.
\newtheoremstyle{thmit}{\topsep}{\topsep}{}{}{\itshape}{}{0.5em}{}

% Define italicized environments.
\theoremstyle{normalit}
\newtheorem{theorem}{Theorem}[section]
\newtheorem{lemma}{Lemma}[section]
\newtheorem{corollary}{Corollary}[section]
\newtheorem{proposition}{Proposition}[section]
\newtheorem*{theorem*}{Theorem}

% Define environments with italic headers.
\theoremstyle{thmit}
\newtheorem*{solution}{Solution}
\newtheorem*{fsolution}{Solution}

% Define default environments.
\theoremstyle{normal}
\newtheorem{example}{Example}[section]
\newtheorem{definition}{Definition}[section]
\newtheorem{problem}{Problem}[section]
\newtheorem{question}{Question}[section]
\newtheorem{remark}{Remark}[section]
\newtheorem{properties}{Properties}[section]
\newtheorem{notation}{Notation}[section]
\newtheorem{axiom}{Axiom}[section]
\newtheorem*{properties*}{Properties}
\newtheorem*{remark*}{Remark}
\newtheorem*{definition*}{Definition}
\theoremstyle{plain}

% Define framed environment.
\tcbuselibrary{most}
\newtcbtheorem[use counter*=theorem]{ftheorem}{Theorem}%
    {colback=green!5,colframe=green!35!black,
     fonttitle=\bfseries\upshape}{th}

\newtcbtheorem[use counter*=example]{fdefinition}{Definition}%
    {fonttitle=\bfseries\upshape,
     colback=blue!5!white,colframe=blue!75!black}{def}

\newtcbtheorem[use counter*=example]{fexample}{Example}%
    {fonttitle=\bfseries\upshape,
     colback=red!5!white,colframe=red!75!black}{ex}

\newtcbtheorem[use counter*=notation]{fnotation}{Notation}%
    {fonttitle=\bfseries\upshape,
     colback=SeaGreen!5!white,colframe=SeaGreen!75!black}{ex}

\newtcbtheorem[use counter*=corollary]{fcorollary}{Corollary}%
    {fonttitle=\bfseries\upshape,
     colback=Orchid!5!white,colframe=Orchid!75!black}{ex}

\newenvironment{bproof}{\textit{Proof.}}{\hfill$\square$}
\tcolorboxenvironment{bproof}{blanker,breakable,left=5mm,
                             before skip=10pt,after skip=10pt,
                             borderline west={1mm}{0pt}{red}}
\tcolorboxenvironment{fsolution}
    {enhanced jigsaw,colframe=cyan,interior hidden,breakable}

%--------------------Declared Math Operators--------------------%
\DeclareMathOperator{\Refl}{Refl}           % Reflection operator.
\DeclareMathOperator{\Span}{Span}           % Span of a set of vectors.
\DeclareMathOperator{\Card}{Card}           % Cardinality of set.
\DeclareMathOperator{\Ord}{Ord}             % Ordinal of ordered set.
\DeclareMathOperator{\Tr}{Tr}               % Trace of matrix.
\DeclareMathOperator{\adjoint}{adj}         % Adjoint of matrix.
\DeclareMathOperator{\rk}{rk}               % Rank of operator.
\DeclareMathOperator{\nul}{nul}             % Null space of operator.
\DeclareMathOperator{\sgn}{sgn}             % Sign of a number.
\DeclareMathOperator{\multideg}{mutlideg}   % Multi-Degree (Graphs).
\DeclareMathOperator{\GCD}{GCD}             % Greatest common denominator.
\DeclareMathOperator{\LM}{LM}               % Leading monomial
\DeclareMathOperator{\LC}{LC}               % Leading coefficient.
\DeclareMathOperator{\LT}{LT}               % Leading term.
\DeclareMathOperator{\LCM}{LCM}             % Least common multiple.
\DeclareMathOperator{\Mon}{Mon}             % Monomial.
\DeclareMathOperator{\Spec}{Spec}           % Spectrum.
\DeclareMathOperator{\proj}{proj}           % Projection.
\DeclareMathOperator{\comp}{comp}           % Component.
\DeclareMathOperator{\sinc}{sinc}           % Sinc function.
\DeclareMathOperator{\Ima}{Im}              % Image of operator.
\DeclareMathOperator{\Prin}{Prin}           % Principal value.
\DeclareMathOperator{\Mod}{mod}             % Modulus.
%------------------------New Commands---------------------------%
\DeclarePairedDelimiter\norm{\lVert}{\rVert}
\DeclarePairedDelimiter\ceil{\lceil}{\rceil}
\DeclarePairedDelimiter\floor{\lfloor}{\rfloor}
\newcommand*\diff{\mathop{}\!\mathrm{d}}
\newcommand*\Diff[1]{\mathop{}\!\mathrm{d^#1}}
\renewcommand{\mod}{\ \Mod}
\renewcommand*{\glstextformat}[1]{\textcolor{RoyalBlue}{#1}}
\renewcommand{\glsnamefont}[1]{\textbf{#1}}
\renewcommand\labelitemii{$\circ$}
\renewcommand\thesubfigure{\arabic{chapter}.\arabic{figure}}
\renewcommand\thesubfigure{%
    \arabic{chapter}.\arabic{figure}.\arabic{subfigure}}
\addto\captionsenglish{\renewcommand{\figurename}{Fig.}}
%------------------------Book Command---------------------------%
\makeatletter
\renewcommand\@pnumwidth{1cm}
\newcounter{book}
\renewcommand\thebook{\@Roman\c@book}
\newcommand\book{%
    \if@openright
        \cleardoublepage
    \else
        \clearpage
    \fi
    \thispagestyle{plain}%
    \if@twocolumn
        \onecolumn
        \@tempswatrue
    \else
        \@tempswafalse
    \fi
    \null\vfil
    \secdef\@book\@sbook
}
\def\@book[#1]#2{%
    \ifnum \c@secnumdepth >-3\relax
        \refstepcounter{book}%
        \addcontentsline{toc}{book}{
            \bookname\ \thebook:\hspace{1em}#1
        }
    \else
        \addcontentsline{toc}{book}{#1}%
    \fi
    \markboth{}{}%
    {\centering
     \interlinepenalty \@M
     \normalfont
     \ifnum \c@secnumdepth >-2\relax
       \huge\bfseries \bookname\nobreakspace\thebook
       \par
       \vskip 20\p@
     \fi
     \Huge \bfseries #2\par}%
    \@endbook}
\def\@sbook#1{%
    {\centering
     \interlinepenalty \@M
     \normalfont
     \Huge \bfseries #1\par}%
    \@endbook}
\def\@endbook{
    \vfil\newpage
        \if@twoside
            \if@openright
                \null
                \thispagestyle{empty}%
                \newpage
            \fi
        \fi
        \if@tempswa
            \twocolumn
        \fi
}
\newcommand*\l@book[2]{%
    \ifnum \c@tocdepth >-2\relax
        \addpenalty{-\@highpenalty}%
        \addvspace{2.25em \@plus\p@}%
        \setlength\@tempdima{3em}%
        \begingroup
            \parindent \z@ \rightskip \@pnumwidth
            \parfillskip -\@pnumwidth
            {
                \leavevmode
                \Large \bfseries #1\hfil \hb@xt@\@pnumwidth{
                    \hss #2
                }
            }
            \par
            \nobreak
            \global\@nobreaktrue
            \everypar{\global\@nobreakfalse\everypar{}}%
        \endgroup
    \fi}
\newcommand\bookname{Book}
\renewcommand{\thebook}{\texorpdfstring{\Numberstring{book}}{book}}
\providecommand*{\toclevel@book}{-2}
\makeatother
\titlecontents{chapter}[0pt]
    {\bfseries}
    {\chaptername\ \thecontentslabel:\quad}
    {}
    {\hfill\contentspage}
\titleformat{\part}[display]
    {\Large\bfseries}
    {\partname\nobreakspace\thepart}
    {0mm}
    {\Huge\bfseries}
    \titlecontents{part}[0pt]
    {\large\bfseries}
    {\partname\ \thecontentslabel: \quad}
    {}
    {\hfill\contentspage}
\newcommand{\MarkRightAngle}[4][.3cm]
    {\coordinate (tempa) at ($(#3)!#1!(#2)$);
     \coordinate (tempb) at ($(#3)!#1!(#4)$);
     \coordinate (tempc) at ($(tempa)!0.5!(tempb)$);%midpoint
     \draw (tempa) -- ($(#3)!2!(tempc)$) -- (tempb);}
%--------------------------LENGTHS------------------------------%
% Spacings for the Table of Contents.
\addtolength{\cftsecnumwidth}{1ex}
\addtolength{\cftsubsecindent}{1ex}
\addtolength{\cftsubsecnumwidth}{1ex}
\addtolength{\cftfignumwidth}{1ex}
\addtolength{\cfttabnumwidth}{1ex}

% Spacing for multi-column and enumerate environments.
\setlength{\multicolsep}{6pt}
\setlist[enumerate]{itemsep=0pt,topsep=3pt}

% Indent and paragraph spacing.
\setlength{\parindent}{0em}
\setlength{\parskip}{0em}
%----------------------------GLOSSARY-------------------------------%
\makeglossaries
\loadglsentries{../../glossary}
\loadglsentries{../../acronym}
%--------------------------Main Document----------------------------%
\begin{document}
    \ifx\ifmathcourses\undefined
        \pagenumbering{roman}
        \title{Algebraic Topology}
        \author{Ryan Maguire}
        \date{\vspace{-5ex}}
        \maketitle
        \tableofcontents
        \listoffigures
        \clearpage
        \chapter{Algebraic Topology}
        %\markboth{}{ALGEBRAIC TOPOLOGY}
        %\setcounter{chapter}{1}
        \pagenumbering{arabic}
    \else
        \chapter{Algebraic Topology}
    \fi
    \section{A Review of Algebra}
        A \textbf{Group} is a set $G$ and a
        binary operation $*$ such
        that the following are true:
        \begin{enumerate}
            \item[G1] $\forall_{a,b,c\in{G}},%
                       a*(b*c)=(a*b)*c$
                      \hfill[Associativity]
            \item[G2] $\exists_{e\in{G}}\forall_{a\in{G}}:a*e=a$
                      \hfill[Existence of Right Identity]
            \item[G3] $\forall_{a\in{G}}\exists_{a^{-1}\in{G}}:%
                       a*a^{-1}=e$
                      \hfill[Existence of Right Inverse]
        \end{enumerate}
        Usually one sees $a*e=e*a=a$ and
        $a*a^{-1}=a^{-1}*a=e$, but this can be relaxed to just
        left or right and then one can prove they are both
        equivalent. We often write $(G,*)$ to denote a group.
        An \textit{Abelian Group} is a group $(G,*)$ such that:
        \begin{enumerate}
            \item[G4] $\forall_{a,b\in{G}},a*b=b*a$
                      \hfill[Commutativity]
        \end{enumerate}
        A \textbf{Ring} is a set $R$ with
        two operations $+$ and $\cdot$, usually called
        addition and multiplication, respectively,
        with the following properties:
        \begin{enumerate}
            \item[R1] $(R,+)$ is an Abelian Group.
            \item[R2] $\forall_{a,b,c\in{R}},%
                       a\cdot(b\cdot{c})=(a\cdot{b})\cdot{c}$
                      \hfill[Associativity of Multiplication]
            \item[R3] $\forall_{a,b,c\in{R}},%
                       a\cdot(b+c)=(a\cdot{b})+(a\cdot{c})$
                      \hfill[Left-Distributive Law]
            \item[R4] $\forall_{a,b,c\in{R}}%
                       (b+c)\cdot{a}=(b\cdot{a})+(c\cdot{a})$
                      \hfill[Right-Distributive Law]
        \end{enumerate}
        We write $(R,+,\cdot)$ to denote a ring.
        A \textbf{Ring with Identity}, or a ring with unity, is
        a ring $(R,+,\cdot)$ such that:
        \begin{enumerate}
            \item[R5] $\exists_{1\in{R}}\forall_{a\in{R}}:%
                       1\cdot{a}=a\cdot{1}=a$
                      \hfill[Multiplicative Identity]
        \end{enumerate}
        A \textbf{Commutative Ring} is a ring $(R,+,\cdot)$
        such that:
        \begin{enumerate}
            \item[R6] $\forall_{a,b\in{R}},a\cdot{b}=b\cdot{a}$
                      \hfill[Commutativity of Multiplication]
        \end{enumerate}
        In a commutative ring, one can replace the
        Left-Distributive Law and the Right-Distributive Law with
        simply one Distributive Law. Commutativity implying
        they're equivalent. A $\textbf{Field}$ is a
        commutative ring with identity $(F,+,\cdot)$ such that:
        \begin{enumerate}
            \item[F1] $\forall_{a\in{F}, a\ne{0}}%
                       \exists_{a^{-1}\in{F}}:a\cdot{a^{-1}}=1$
                      \hfill[Multiplicative Inverse]
        \end{enumerate}
        One can prove some very intuitive results about the
        additive identity of a ring $R$.
        \begin{theorem}
            If $(R,+,\cdot)$ is a ring and $0$ is the additive
            identity of $R$, then for all $a\in{R}$,
            $a\cdot{0}=0$.
        \end{theorem}
        \begin{proof}
            For:
            \begin{align*}
                0&=a\cdot{0}-a\cdot{0}
                &
                &=a\cdot{0}+(a\cdot{0}-a\cdot{0})\\
                &=a\cdot(0+0)-a\cdot{0}
                &
                &=a\cdot{0}+0\\
                &=(a\cdot{0}+a\cdot{0})-a\cdot{0}
                &
                &=a\cdot{0}
            \end{align*}
        \end{proof}
        Thus, if one has a field $(F,+,\cdot)$, and if
        $0$ has a multiplicative inverse, then every element of
        $F$ is equal to $0$.
        \begin{theorem}
            If $(F,+,\cdot)$ is a field, $0$ is the additive
            identity, and if $0$ has a multiplicative inverse,
            then $F=\{0\}$.
        \end{theorem}
        \begin{proof}
            For $1=0\cdot{0}^{-1}$, but
            $0\cdot{0}^{-1}=0$, and thus $1=0$. But for all
            $a\in{F}$, $a=a\cdot{1}=a\cdot{0}=0$.
        \end{proof}
        Because of this, some require that $0\ne{1}$ in the
        definition of a field, and others call
        $F=\{0\}$ the trivial field.
        \begin{definition}
            A left module of a ring with identity $(R,+,\cdot)$
            is an abelian group $(M,+_{M})$ and a function
            $*:R\times{M}\rightarrow{M}$ such that:
            \begin{enumerate}
                \item $r*(a+_{M}b)=(r*a)+(r*b)$
                      \hfill[Left Scalar Distribution]
                \item $(r\cdot{s})*a=r*(s*a)$
                      \hfill[Scalar Associativity]
                \item $(r+s)*a=(r*a)+_{M}(s*a)$
                      \hfill[Right Module Distribution]
                \item $1*a=a$
                      \hfill[Identity Element]
            \end{enumerate}
        \end{definition}
        An attempt has been made to preserve the differences
        between the various operations in a left module.
        $+$ and $\cdot$ are binary operations that act on
        elements of $R$. That is, for $a,b\in{R}$, $a+b$
        gives another element of $R$, as does $a\cdot{b}$.
        However, $+_{M}$ is a binary operation over
        $M$. If $a,b\in{R}$, $a+_{M}b$ has no meaning.
        For $a,b\in{M}$, $a+_{M}b$ is well defined, and returns
        another element of $M$. The ``function,'' $*$
        takes an ordered pair $(r,a)$, where $r\in{R}$ and
        $a\in{M}$, and returns another element in $M$. For
        convenience we write $r*a$. If $a,b\in{R}$, then
        $a*b$ has no meaning, and if $a,b\in{M}$ then
        $a*b$ also has no meaning. Usually this is very
        unimportant, and $+$ and $+_{M}$ are given the same
        symbol, as are $\cdot$ and $*$. We can then more loosely
        rewrite the definition as, for all $r,s\in{R}$, and
        all $a,b\in{M}$:
        \begin{enumerate}
            \begin{multicols}{2}
                \item $r(a+b)=ra+rb$
                \item $(rs)(a)=r(sa)$
                \item $(r+s)(a)=(ra)+(rs)$
                \item $1a=a$
            \end{multicols}
        \end{enumerate}
        This is the more natural notation one finds when defining
        vector spaces. A module is analogous to a vector space:
        In a vector space one has a set $V$ and a
        \textit{field} $K$, whereas in a module one has a set
        $M$ and a \textit{ring with identity} $R$.
        \begin{theorem}
            If $(G,+)$ is an Abelian group, then there is a
            function $*:\mathbb{Z}\times{G}\rightarrow{G}$
            such that $(G,+_{G})$ is a left module of
            $(\mathbb{Z},+,\cdot)$, where $+$ and $\cdot$ are
            the standard arithmetic operations over $\mathbb{Z}$.
        \end{theorem}
        \begin{proof}
            For define $0*a=e$ and $1*a=a$ for all $a\in{G}$,
            and for all
            $n\in\mathbb{Z}$, $n>1$ inductively define
            $(n+1)*a=n*a+_{G}a$. For $n<0$ define
            $n*a=((-n)*a)^{-1}$, where the inverse is taken
            with respect to the group $G$. Then $*$ is a function
            $*:\mathbb{Z}\times{G}\rightarrow{G}$. If $n>0$, we
            have the following:
            \begin{align*}
                (n*a)+_{G}(n*b)
                &=\underset{n}{\underbrace{(a+_{G}\cdots+_{G}a)}}
                +_{G}
                \underset{n}{\underbrace{(b+_{G}\cdots+_{G}b)}}\\
                &=\underset{n}
                    {\underbrace{(a+b)+_{G}\cdots+_{G}(a+b)}}\\
                &=n*(a+b)
            \end{align*}
            If $n,m>0$, we have:
            \begin{equation*}
                n*a+_{G}m*a
                =\underset{n+m}{\underbrace{a+_{G}\cdots+_{G}a}}
                =(n+m)*a
            \end{equation*}
            And finally:
            \begin{equation*}
                (n\cdot{m})*a=
                \underset{n\cdot{m}}
                    {\underbrace{a+_{G}\cdots+_{G}a}}
                =n*(\underset{m}{\underbrace{a+_{G}\cdots+_{G}a}})
                =n*(m*a)
            \end{equation*}
            Similarly for when $n,m<0$, $n<0<m$, or $m<0<n$.
        \end{proof}
        Thus, every Abelian group $(G,+_{G})$ can be seen
        as a left module over $(R,+,\cdot)$. Moreover the
        function $*$ is unique, so this correspondence is
        unique as well. As another example, every vector space
        $V$ over a field $K$ is a left module over $K$, since
        any field $K$ is also a ring with identity.
        A \textbf{Left Ideal} of a ring $(R,+,*)$ is a subset
        $I\subseteq{R}$ such that:
        \begin{enumerate}
            \item $\forall_{a,b\in{I}},a+b\in{I}$
            \item $\forall_{r\in{R}}\forall_{a\in{I}},%
                   r\cdot{a}\in{I}$
        \end{enumerate}
        This can be rephrased by saying that $(I,+)$ is a subgroup
        of $(R,+)$, and $I$ absorbs left-multiplication. A
        \textbf{Right Ideal} replaces $r\cdot{a}$ with
        $a\cdot{r}$. An \textbf{Ideal} or \textbf{Two-Sided Ideal}
        is a subset that is both a left and a right ideal.
        \begin{theorem}
            If $(R,+,\cdot)$ is a ring with identity
            and $(I,+)$ is a left ideal
            of $R$, then there is a function
            $*:R\times{I}\rightarrow{I}$ such that
            $(I,+)$ is a left module over $R$.
        \end{theorem}
        \begin{proof}
            For let $*$ be the restriction of $\cdot$ to
            $R\times{I}$. Then, for all $a\in{I}$,
            $1\cdot{a}=a$. If $a,b\in{I}$ and $r\in{R}$, then:
            \begin{equation*}
                r*(a+b)
                =r\cdot(a+b)
                =(r\cdot{a})+(r\cdot{b})
            \end{equation*}
            If $r,s\in{R}$ and $a\in{I}$, then:
            \begin{gather*}
                (r\cdot{s})*a
                =(r\cdot{s})\cdot{a}
                =r\cdot(s\cdot{a})\\
                (r+s)*(a)
                =(r+s)\cdot{a}
                =(r\cdot{a})+(s\cdot{a})
            \end{gather*}
        \end{proof}
        \begin{definition}
            The Annihilator of a Left Module $(M,+_{M})$ over
            a ring with identity $(R,+,\cdot)$ is the set:
            \begin{equation*}
                I=\{r\in{R}:\forall_{m\in{M}},r*m=0\}
            \end{equation*}
        \end{definition}
        \begin{theorem}
            If $(M,+_{M})$ is a Left Module over a ring with
            identity $(R,+,\cdot)$, and if $I$ is the
            annihilator of $M$, then $I$ is a two-sided
            ideal of $R$.
        \end{theorem}
        \begin{proof}
            For if $r,s\in{I}$, then for all $m\in{M}$,
            $r*m=0$ and $s*m=0$. But $(r+s)*m=(r*m)+_{M}(s*m)=0$.
            Therefore $r+s\in{I}$. If $r\in{R}$ and $s\in{I}$,
            then $(r\cdot{s})*m=r*(s*m)=r*0=0$, and therefore
            $r\cdot{s}\in{I}$. Furthermore,
            $(s\cdot{r})*m=s*(r*m)=0$, and thus $s\cdot{r}\in{I}$.
            Therefore $I$ is a two-sided ideal.
        \end{proof}
    \section{The Fundamental Group}
    \section{Homology}
    \section{Cohomology}
    \section{Homotopy}
    \section{Notes from Dartmouth S2019}
        Categories. Certain types of objects with
        maps, like \textit{group homomorphisms}, or
        \textit{ring homomorphisms}, or
        \textit{continuous functions}. The objects
        are groups, rings, and topological spaces,
        respectively. Presently we care about
        topological spaces. This is a set $X$ with
        a \textit{topology} $\tau$ on $X$,
        which is a subset of $\mathcal{P}(X)$ such that
        $\emptyset\in\tau$, $X\in\tau$, the union of
        any collection is $\tau$ is again in $\tau$,
        and the intersection of any finite collection
        of elements of $\tau$ is again in $\tau$.
        That maps we talk about are continuous functions
        between topological spaces.
        \begin{example}
            $SO(3)$, $3\times{3}$ matrices that
            are orthogonal and have determinant 1, and
            $SU(2)$. Both of these are differentiable
            manifolds. How can we determine if they
            are homeomorphic or not? To every
            continuous map between topological spaces
            $X$ and $Y$ we can correspond groups
            $G(X)$ and $G(Y)$ and a group homomorphism
            $\varphi:G(X)\rightarrow{G}(Y)$ that
            corresponds somehow to the continuous
            function. By showing that certain properties
            are preserved by this correspondence we
            can show that $SO(3)$ and $SU(2)$ are not
            homeomorphic.
        \end{example}
        \begin{align}
            &f:X\rightarrow{Y}\rightarrow{Z}\\
            &\varphi:G(X)\rightarrow{G}(Y)
            \rightarrow{G}(Z)
        \end{align}
        The maps between topological spaces and groups
        are called \textit{Functors}. The category
        of topological spaces is somewhat to large.
        From point-set topology there are many
        pathelogical sets that are difficult to manage.
        In differential topology we study smooth
        manifolds which are locally like Euclidean space
        and have a lot of structure on them. Even
        nicer are \textit{Affine Varieties}, which
        are solution sets to polynomials that are
        studied in algebraic geometry. In the late
        1940s, Whitehead came up with the following
        category, that of CW Complexes. Not every
        topological space is a CW Complex, but in some
        respect every topological space is like a CW
        complex. All manifolds are CW complexes.
        \subsection{CW Complexes}
            The C in CW stands for \textit{cell}
            \begin{ldefinition}{Cell}
                A cell of dimension $n\in\mathbb{N}$ is
                a topological space that is homeomorphic
                to $D^{n}\setminus\partial{D}^{n}$
                Where $D^{n}$ is the unit disk:
                \begin{equation}
                    D^{n}=\{\mathbf{x}\in\mathbb{R}^{n}:
                    \norm{\mathbf{x}}=1\}
                \end{equation}
                And $\partial{D}^{n}$ is the boundary.
            \end{ldefinition}
            There's no intrinsic metric on the space.
            The open unit square is the a cell, since
            it is homeomorphic to the open unit disc.
            A zero dimensional cell is a point.
            One-cells are edges, two-cells are
            bubbles, and so forth.
            \begin{ldefinition}{CW Complexes}
                A CW Complex is a topological space
                $X$ that is the disjoint union of
                cells.
            \end{ldefinition}
            Typically CW Complexes are metric spaces.
            \begin{ldefinition}{$k$-Skeleton}
                The $k$-skeleton of a CW complex $X$,
                denoted $X^{k}$,
                is the disjoint union of all cells of
                dimension less than or equal to $k$.
            \end{ldefinition}
            $X^{0}$ is the disjoint union of points.
            In general:
            \begin{equation}
                X^{k}\simeq{X}^{k-1}\bigcup\Big(
                    \bigcup_{\alpha\in{I}}D_{\alpha}^{k}
                \Big)
            \end{equation}
            $X^{k}$ is obtained from $X^{k-1}$ by
            attaching $k$ cells $D_{\alpha}^{k}$ along
            their boundary $\partial{D}_{\alpha}^{k}$.
            \begin{ldefinition}{Quotient Space}
                The quotient space of a topological
                space $Z$ with respect to an equivalence
                relation $\sim$ is the set:
                \begin{equation}
                    Z/\sim=\{[x]:x\in{Z}\}
                \end{equation}
                Where $[x]$ is the equivalence class of
                $x$ with respect to $\sim$, and the
                topology:
                \begin{equation}
                    \tau_{\sim}=
                    \{p^{-1}(\mathcal{U}:
                    \mathcal{U}\in\tau\}
                \end{equation}
                Where $q$ is the projection mapping.
            \end{ldefinition}
            An equivalence relation identifies certain
            points in $Z$. With this we can see what it
            means to attach one space to another. Let
            $X$ and $Y$ be topological space, and let
            $A\subseteq{X}$. The attaching map is a
            continuous function $f:A\rightarrow{Y}$.
            Define the equivalence relation:
            \begin{equation}
                \sim=\{(a,f(a)):a\in{A}\}
            \end{equation}
            \begin{equation}
                Y\cup_{f}{X}=
                Y\cup{X}/\sim
            \end{equation}
            For each $k$ cell there is an attaching map
            $\varphi_{\alpha}:\partial{D}_{\alpha}^{k}\rightarrow{X}^{k-1}$.
            \begin{lexample}
                A zero dimensional CW complex is the
                disjoint union of a bunch of points.
                So $X=X^{0}$, and this has the
                discrete topology on it. That is,
                $\tau=\mathcal{P}(X)$, every set is
                open. The only defining characteristic
                is the cardinality of the space.
            \end{lexample}
            \begin{lexample}
                One dimensional CW complexes are a
                combination of edges and points. For
                each edge the endpoints must go to
                a point in the 0-skeleton $X^{0}$.
                We can send the two endpoints to the
                same point, which creates a loop, or
                to distinct points, which creates an
                edge.
            \end{lexample}
            \begin{lexample}
                Hawaiian ear-rings. A subset of
                $\mathbb{R}^{2}$ that looks like a
                CW complex, but is not.
            \end{lexample}
            Any manifold is a CW complex since all
            manifolds are triangulable. For example,
            $S^{2}$ is homeomorphic to a cube. The
            faces of a cube are homeomorphic to 
            $D^{2}$, and thus $S^{2}$ is a CW complex.
            A balloon is an example of a two dimensional
            CW complex.
            \begin{lexample}
                The real projective plane,
                $\mathbb{R}P^{2}$ is an important
                example that is encountered in topology.
                The easiest definition is the set of
                lines through the origin in
                $\mathbb{R}^{3}$. Equivalently, the
                set of all one dimensional
                vector subspaces of $\mathbb{R}^{3}$.
                We can think of this as a quotient
                space on $\mathbb{R}^{3}\setminus\{0\}$.
                \begin{equation}
                    \sim=\{(\mathbf{x},c\mathbf{x}):
                        \mathbf{c}\ne\mathbf{0},
                        c\in\mathbb{R}\}
                \end{equation}
                We can define:
                \begin{align}
                    \mathbb{R}P^{2}&=
                    \mathbb{R}^{3}\setminus
                    \{\mathbf{0}\}/\sim\\
                    &=S^{2}/\{(\mathbf{x},-\mathbf{x}:
                        \mathbf{x}\in{S}^{2}\}
                \end{align}
                That is, we identify all of the
                antipodal points on the unit sphere.
                This is a CW complex. To show this,
                take one zero-cell and one one-cell.
                This forms a loop. The attaching map
                $\varphi:S^{1}\rightarrow{S}^{1}$ is
                a two-to-one map.
            \end{lexample}
        \subsection{Homotopy Equivalence}
            \begin{ldefinition}{Homotopy}
                A homotopy of continuous maps
                $f_{t}:X\rightarrow{Y}$, $t\in[0,1]$, such that:
                \begin{equation}
                    F:X\times{I}\rightarrow{Y}
                    \quad\quad
                    F(x,t)=f_{t}(x)
                \end{equation}
                is continuous in the product space $X\times{I}$.
            \end{ldefinition}
            \begin{lexample}
                Let $X=S^{1}$, the unit circle, and let
                $Y=T^{2}$, the torus. This is
                $S^{1}\times{S}^{1}$. Then $X\times{I}$ is
                a cylinder. Circles on one of the inner circle
                can't be homotopic to circle on the other
                inner circle. Let
                $f:S^{1}\rightarrow\mathbb{R}^{2}$ be defined by:
                \begin{equation}
                    f(x,y)=(x,y)
                \end{equation}
                This is the inclusion map in the plane. Then
                $f\simeq{g}$, where $g(x,y)=0$. For let:
                \begin{equation}
                    F(x,y,t)=(tx,ty)
                \end{equation}
                This is continuous, for let:
                \begin{equation}
                    \tilde{F}:\mathbb{R}^{2}\times\mathbb{R}
                    \rightarrow\mathbb{R}^{2}\quad\quad
                    \tilde{F}(x,y,t)=(tx,ty)
                \end{equation}
                This is continuous since each of the
                coordinate functions is continuous. Thus
                $F$ is continuous in the subspace topology, which
                is the topology on $S^{1}\times{I}$. Functions
                that are homotopic to a point are called null
                homotopic.
            \end{lexample}
            \begin{theorem}
                Homotopic is an equivalence relation.
            \end{theorem}
            \begin{theorem}
                If $f:\mathbb{R}^{n}\rightarrow\mathbb{R}^{m}$
                is continuous, then $f$ is null homotopic.
            \end{theorem}
            \begin{theorem}
                If $f,g:\mathbb{R}^{n}\rightarrow\mathbb{R}^{m}$
                are continuous functions, then $f$ and $g$ are
                homotopic.
            \end{theorem}
            \begin{theorem}
                If $f:X\rightarrow\mathbb{R}^{n}$ is continuous,
                then $f$ is null homotopic.
            \end{theorem}
            \begin{proof}
                For let $F$ be defined by:
                \begin{equation}
                    F(x,t)=tf(x)
                \end{equation}
                Therefore, etc.
            \end{proof}
            \begin{ldefinition}{Homotopy Equivalent Spaces}
                Homotopy equivalent spaces are topological
                space $X$, $Y$ such that there exist continuous
                functions $f:X\rightarrow{Y}$ and
                $g:Y\rightarrow{X}$ such that
                $g\circ{f}\simeq{id}_{X}$ and
                $f\circ{g}\simeq{id}_{Y}$.
            \end{ldefinition}
            This definition is a weakening of the notion of
            homeomorphisms. Homeomorphisms are continuous
            functions $f,g$ such that $g\circ{f}=id_{X}$ and
            $f\circ{g}=id_{Y}$. Homeomorphic spaces are homotopy
            equivalent. The converse is not true, in general.
            \begin{lexample}
                Let $f:\mathbb{R}^{n}\rightarrow\{0\}$ be
                defined by $f(x)=0$, and let
                $g:\{0\}\rightarrow\mathbb{R}^{n}$ be defined
                by $g(x)=0$. Then $g\circ{f}=0$, and this
                is homotopy equivalent to the identity map, since
                the identity map is continuous and continuous
                maps are null homotopic. But also
                $f\circ{g}$ is the identity map for
                $\{0\}$. So $\mathbb{R}^{n}$ is homotopy
                equivalent to $\{0\}$. But $\mathbb{R}^{n}$
                is not homeomorphic to a point, simply because
                there is no bijection. A space that is
                homeomorphic to a point is called contractible.
            \end{lexample}
            \begin{ldefinition}{Retraction}
                A retraction of a space $X$ to a subset
                $A\subseteq{X}$ is a function $f:X\rightarrow{A}$
                such that $f$ is continuous and
                for all $a\in{A}$, $f(a)=a$.
            \end{ldefinition}
            \begin{ldefinition}{Deformation Retraction}
                A deformation retraction of a space
                $X$ onto a subset $A\subseteq{A}$ is a homotopy
                $f_{t}:X\rightarrow{X}$ such that $f_{0}=id_{X}$,
                $f_{1}:X\rightarrow{A}$ is a retract, and for
                all $t\in{I}$ and all $a\in{A}$, $f_{t}(a)=a$.
            \end{ldefinition}
            \begin{lexample}
                The contraction we saw from $\mathbb{R}^{n}$ to
                a point is an example of a deformation retract.
            \end{lexample}
            \begin{theorem}
                $X$ is a topological space, $A\subseteq{X}$, and
                if there is a deformation retract of $X$ onto
                $A$, then $X$ is homotopy equivalent to $A$.
            \end{theorem}
            Two steps or criteria that are common in algebraic
            topology.
            \begin{ldefinition}{CW Pair}
                A CW Pair, $(X,A)$, is a CW Complex $X$ and a
                subset $A\subseteq{X}$ such that $A$ is closed,
                and $A$ is the union of cells of $X$.
            \end{ldefinition}
            The set $A$ can be considered as a sub-complex
            of $X$.
            \begin{theorem}
                If $(X,A)$ is a CW pair and if $A$ is
                contracible, then $q:X\rightarrow{X}/A$
                is a homotopy equivalence.
            \end{theorem}
            So 1-Skeletons and 1 dimensional CW Complexes can
            be contracted down to a bunch of points with loops
            on the points. That is, homotopy equivalence is
            preserved. Circles are not contractible, as well
            we eventually see, so we cannot get rid of loops.
        \subsection{Homotopy Extension Property}
            Let $X(,\tau)$ be a topological space,
            $A\subseteq{X}$ a subspace, and let
            $f:A\rightarrow{Y}$ be continuous. Does
            $f$ extend to a map $\tilde{f}:X\rightarrow{X}$
            such that $\tilde{f}|_{A}=f$? Not always, let
            $A=S^{1}$ and $X$ be the torus. For which pairs
            $(X,A)$ is this always possible? Well, suppose
            it is true. Then $id_{A}:A\rightarrow{A}$ has
            an extension $\tilde{id}_{A}:X\rightarrow{A}$.
            But then $A$ is a retract of $X$. Thus, it
            is a necessary condition that $A$ is a retract
            of $X$ for it to be true. As it turns out, it is
            also a sufficient condition. If
            $r:X\rightarrow{A}$ is a retract
            \begin{ldefinition}{Homotopy Extension Property}
                A subset $A\subseteq{X}$ with the homotopy
                extension property is a set such that for all
                maps $f:X\rightarrow{Y}$ and for every
                homotopy $f_{t}:A\rightarrow{Y}$,
                $f_{0}=f|_{A}$, extends to a homotopy of
                $\tilde{f}_{t}:X\rightarrow{Y}$ with
                $\tilde{f}_{0}=f_{0}$.
            \end{ldefinition}
            \begin{theorem}
                Every CW Pair $(X,A)$ has the homotopy
                extension property.
            \end{theorem}
            \begin{proof}
                We will product a retract:
                \begin{equation}
                    X\times{I}\overset{r}{\longrightarrow}
                        X\times{0}\cup{A}\times{I}=Z
                \end{equation}
                First, $(D^{n},\partial{D}^{n})$, which is
                $(D^{n},S^{n-1})$, and this has the
                homotopy extension property.
            \end{proof}
            Recall that if $f:X\rightarrow{Y}$ is a map such that,
            for a given $A\subseteq{X}$, and for all $a\in{A}$,
            there is a fixed $y_{0}\in{Y}$ such that $f(a)=y$, then
            the quotient map $q:X\rightarrow{X/A}$ has a unique
            liften map $\overline{f}$ such that
            $\overline{f}\circ{q}=f$. Moreover, $\overline{f}$ is
            continuous with respect to the quotient topology.
            \begin{theorem}
                If $(X,A)$ is a CW pair, and if $A$ is
                homotopy equivalent to a point
                (Contractible), then the quotient map
                $q:X\rightarrow{X/A}$ is a homotopy
                equivalence.
            \end{theorem}
            \begin{proof}
                For let $(X,A)$ be a CW pair. Then it has the
                homotopy extension property. If $A$ is contractible
                then there is a homotopy of maps
                $f_{t}:A\rightarrow{A}$ such that $f_{0}=id_{A}$ and
                $f_{1}=a_{0}$, for some $a_{0}\in{A}$. We can
                extend this to $X$ by defined
                $f_{0}:X\rightarrow{X}$ as $f_{0}=id_{X}$. By
                the homotopy extension property, there's an
                extension $F_{t}:X\rightarrow{X}$ such that
                $F_{t}|_{A}=f_{t}$. Let $q$ be the quotient
                mapping and let $qf_{t}$ be the lifting mapping.
                Square $X-X$ on top, $X/A$ $X/A$ on bottom. Maps are
                $q$ and $qf_{t}$. Bottom map is
                $\overline{f}_{t}q$. Let $k$ be the mapping
                of $X/A$ to $X$. Then $q$ and $k$ are homotopy
                equivalences. Thus, $X$ and homotopy equivalent
                to $X/A$. Since $qk=\overline{f}_{1}\simeq{id}_{X/A}$
                and $kq=f_{1}\simeq{f}_{0}=id_{X}$.
            \end{proof}
            \begin{theorem}
                If $(X,A)$ is CW pair, $f,g:A\rightarrow{X_{0}}$
                are homotopic attaching maps, then:
                \begin{equation}
                    X_{0}\cup_{f}X\simeq{X}_{0}\cup_{g}X
                \end{equation}
            \end{theorem}
            \begin{theorem}
                $X\times{I}$ deformation retracts to
                $(X\times\{0\})\cup(A\times\{1\})$, then
            \end{theorem}
            This determines a deformation retract of:
            \begin{equation}
                Z=X_{0}\cup_{F}(X_{1}\times{I})
            \end{equation}
            As an aside, there are two things, a wedge product
            $\lor$ and a smash product $\land$. Given two pointed
            space $(X,x_{0})$, $(Y,y_{0})$, the wedge product is:
            \begin{equation}
                (X,x_{0})\lor(Y,y_{0})
                =X\coprod{Y}/\{x_{0}\sim{y}_{0}\}
            \end{equation}
            The smash product is:
            \begin{equation}
                X\land{Y}=
                X\times{Y}/\Big(
                    (\{0\}\times{Y})\cup(X\times\{y_{0}\})
                \Big)
            \end{equation}
            \begin{example}
                $S^{1}\land{S}^{1}$ is simply $S^{2}$.
                $S^{1}\times{S}^{1}$ is a torus, but we've collapsed
                the entire boundary down to a point. So we have a
                2-cell with no boundary, which is a sphere.
            \end{example}
        \subsection{Winding Number}
            \begin{theorem}
                If $f:[0,1]\rightarrow{S}^{1}$ is a continuous
                function, then there is a unique continuous lift
                $\tilde{f}:[0,1]\rightarrow\mathbb{R}$ such that
                $f=q\circ\tilde{f}$, where $q$ is the quotient map
                of $\mathbb{R}$ onto $S^{1}$.
            \end{theorem}
            \begin{proof}
                For let $\mathcal{U}_{1}=S^{1}\setminus\{(1,0)\}$ and
                $\mathcal{U}_{2}=S^{1}\setminus\{(0,1)\}$. This is an
                open cover of $S^{1}$. If $f:I\rightarrow{S}^{1}$ is
                continuous, then $f^{-1}(\mathcal{U}_{1})$ and
                $f^{-1}(\mathcal{U}_{2})$ form an open cover of
                $I$. But any open subset of $\mathbb{R}$ is the
                countable union of disjoint open intervals. Thus
                $f^{-1}(\mathcal{U}_{1})$ and
                $f^{-1}(\mathcal{U}_{2})$ are the countable union of
                disjoint open intervals. But $I$ is compact, and
                thus any open cover has a finite subcover. Therefore
                there are finitely many intervals $I_{1},\dots,I_{k}$
                such that $f(I_{k})\subseteq\mathcal{U}_{1}$ or
                $f(I_{k})\subseteq\mathcal{U}_{2}$.
            \end{proof}
            \begin{ldefinition}{Winding Number}
                The winding number of a loop
                $f:S^{1}\rightarrow{S}^{1}$ is:
                \begin{equation}
                    wind(f)+\tilde{f}(1)-\tilde{f}(0)
                \end{equation}
                Where $\tilde{f}$ is a lifting map of $f$ to
                $\mathbb{R}$.
            \end{ldefinition}
\end{document}