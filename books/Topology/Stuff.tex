%------------------------------------------------------------------------------%
\documentclass{article}                                                        %
%------------------------------Preamble----------------------------------------%
\makeatletter                                                                  %
    \def\input@path{{../../}}                                                  %
\makeatother                                                                   %
%---------------------------Packages----------------------------%
\usepackage{geometry}
\geometry{b5paper, margin=1.0in}
\usepackage[T1]{fontenc}
\usepackage{graphicx, float}            % Graphics/Images.
\usepackage{natbib}                     % For bibliographies.
\bibliographystyle{agsm}                % Bibliography style.
\usepackage[french, english]{babel}     % Language typesetting.
\usepackage[dvipsnames]{xcolor}         % Color names.
\usepackage{listings}                   % Verbatim-Like Tools.
\usepackage{mathtools, esint, mathrsfs} % amsmath and integrals.
\usepackage{amsthm, amsfonts, amssymb}  % Fonts and theorems.
\usepackage{tcolorbox}                  % Frames around theorems.
\usepackage{upgreek}                    % Non-Italic Greek.
\usepackage{fmtcount, etoolbox}         % For the \book{} command.
\usepackage[newparttoc]{titlesec}       % Formatting chapter, etc.
\usepackage{titletoc}                   % Allows \book in toc.
\usepackage[nottoc]{tocbibind}          % Bibliography in toc.
\usepackage[titles]{tocloft}            % ToC formatting.
\usepackage{pgfplots, tikz}             % Drawing/graphing tools.
\usepackage{imakeidx}                   % Used for index.
\usetikzlibrary{
    calc,                   % Calculating right angles and more.
    angles,                 % Drawing angles within triangles.
    arrows.meta,            % Latex and Stealth arrows.
    quotes,                 % Adding labels to angles.
    positioning,            % Relative positioning of nodes.
    decorations.markings,   % Adding arrows in the middle of a line.
    patterns,
    arrows
}                                       % Libraries for tikz.
\pgfplotsset{compat=1.9}                % Version of pgfplots.
\usepackage[font=scriptsize,
            labelformat=simple,
            labelsep=colon]{subcaption} % Subfigure captions.
\usepackage[font={scriptsize},
            hypcap=true,
            labelsep=colon]{caption}    % Figure captions.
\usepackage[pdftex,
            pdfauthor={Ryan Maguire},
            pdftitle={Mathematics and Physics},
            pdfsubject={Mathematics, Physics, Science},
            pdfkeywords={Mathematics, Physics, Computer Science, Biology},
            pdfproducer={LaTeX},
            pdfcreator={pdflatex}]{hyperref}
\hypersetup{
    colorlinks=true,
    linkcolor=blue,
    filecolor=magenta,
    urlcolor=Cerulean,
    citecolor=SkyBlue
}                           % Colors for hyperref.
\usepackage[toc,acronym,nogroupskip,nopostdot]{glossaries}
\usepackage{glossary-mcols}
%------------------------Theorem Styles-------------------------%
\theoremstyle{plain}
\newtheorem{theorem}{Theorem}[section]

% Define theorem style for default spacing and normal font.
\newtheoremstyle{normal}
    {\topsep}               % Amount of space above the theorem.
    {\topsep}               % Amount of space below the theorem.
    {}                      % Font used for body of theorem.
    {}                      % Measure of space to indent.
    {\bfseries}             % Font of the header of the theorem.
    {}                      % Punctuation between head and body.
    {.5em}                  % Space after theorem head.
    {}

% Italic header environment.
\newtheoremstyle{thmit}{\topsep}{\topsep}{}{}{\itshape}{}{0.5em}{}

% Define environments with italic headers.
\theoremstyle{thmit}
\newtheorem*{solution}{Solution}

% Define default environments.
\theoremstyle{normal}
\newtheorem{example}{Example}[section]
\newtheorem{definition}{Definition}[section]
\newtheorem{problem}{Problem}[section]

% Define framed environment.
\tcbuselibrary{most}
\newtcbtheorem[use counter*=theorem]{ftheorem}{Theorem}{%
    before=\par\vspace{2ex},
    boxsep=0.5\topsep,
    after=\par\vspace{2ex},
    colback=green!5,
    colframe=green!35!black,
    fonttitle=\bfseries\upshape%
}{thm}

\newtcbtheorem[auto counter, number within=section]{faxiom}{Axiom}{%
    before=\par\vspace{2ex},
    boxsep=0.5\topsep,
    after=\par\vspace{2ex},
    colback=Apricot!5,
    colframe=Apricot!35!black,
    fonttitle=\bfseries\upshape%
}{ax}

\newtcbtheorem[use counter*=definition]{fdefinition}{Definition}{%
    before=\par\vspace{2ex},
    boxsep=0.5\topsep,
    after=\par\vspace{2ex},
    colback=blue!5!white,
    colframe=blue!75!black,
    fonttitle=\bfseries\upshape%
}{def}

\newtcbtheorem[use counter*=example]{fexample}{Example}{%
    before=\par\vspace{2ex},
    boxsep=0.5\topsep,
    after=\par\vspace{2ex},
    colback=red!5!white,
    colframe=red!75!black,
    fonttitle=\bfseries\upshape%
}{ex}

\newtcbtheorem[auto counter, number within=section]{fnotation}{Notation}{%
    before=\par\vspace{2ex},
    boxsep=0.5\topsep,
    after=\par\vspace{2ex},
    colback=SeaGreen!5!white,
    colframe=SeaGreen!75!black,
    fonttitle=\bfseries\upshape%
}{not}

\newtcbtheorem[use counter*=remark]{fremark}{Remark}{%
    fonttitle=\bfseries\upshape,
    colback=Goldenrod!5!white,
    colframe=Goldenrod!75!black}{ex}

\newenvironment{bproof}{\textit{Proof.}}{\hfill$\square$}
\tcolorboxenvironment{bproof}{%
    blanker,
    breakable,
    left=3mm,
    before skip=5pt,
    after skip=10pt,
    borderline west={0.6mm}{0pt}{green!80!black}
}

\AtEndEnvironment{lexample}{$\hfill\textcolor{red}{\blacksquare}$}
\newtcbtheorem[use counter*=example]{lexample}{Example}{%
    empty,
    title={Example~\theexample},
    boxed title style={%
        empty,
        size=minimal,
        toprule=2pt,
        top=0.5\topsep,
    },
    coltitle=red,
    fonttitle=\bfseries,
    parbox=false,
    boxsep=0pt,
    before=\par\vspace{2ex},
    left=0pt,
    right=0pt,
    top=3ex,
    bottom=1ex,
    before=\par\vspace{2ex},
    after=\par\vspace{2ex},
    breakable,
    pad at break*=0mm,
    vfill before first,
    overlay unbroken={%
        \draw[red, line width=2pt]
            ([yshift=-1.2ex]title.south-|frame.west) to
            ([yshift=-1.2ex]title.south-|frame.east);
        },
    overlay first={%
        \draw[red, line width=2pt]
            ([yshift=-1.2ex]title.south-|frame.west) to
            ([yshift=-1.2ex]title.south-|frame.east);
    },
}{ex}

\AtEndEnvironment{ldefinition}{$\hfill\textcolor{Blue}{\blacksquare}$}
\newtcbtheorem[use counter*=definition]{ldefinition}{Definition}{%
    empty,
    title={Definition~\thedefinition:~{#1}},
    boxed title style={%
        empty,
        size=minimal,
        toprule=2pt,
        top=0.5\topsep,
    },
    coltitle=Blue,
    fonttitle=\bfseries,
    parbox=false,
    boxsep=0pt,
    before=\par\vspace{2ex},
    left=0pt,
    right=0pt,
    top=3ex,
    bottom=0pt,
    before=\par\vspace{2ex},
    after=\par\vspace{1ex},
    breakable,
    pad at break*=0mm,
    vfill before first,
    overlay unbroken={%
        \draw[Blue, line width=2pt]
            ([yshift=-1.2ex]title.south-|frame.west) to
            ([yshift=-1.2ex]title.south-|frame.east);
        },
    overlay first={%
        \draw[Blue, line width=2pt]
            ([yshift=-1.2ex]title.south-|frame.west) to
            ([yshift=-1.2ex]title.south-|frame.east);
    },
}{def}

\AtEndEnvironment{ltheorem}{$\hfill\textcolor{Green}{\blacksquare}$}
\newtcbtheorem[use counter*=theorem]{ltheorem}{Theorem}{%
    empty,
    title={Theorem~\thetheorem:~{#1}},
    boxed title style={%
        empty,
        size=minimal,
        toprule=2pt,
        top=0.5\topsep,
    },
    coltitle=Green,
    fonttitle=\bfseries,
    parbox=false,
    boxsep=0pt,
    before=\par\vspace{2ex},
    left=0pt,
    right=0pt,
    top=3ex,
    bottom=-1.5ex,
    breakable,
    pad at break*=0mm,
    vfill before first,
    overlay unbroken={%
        \draw[Green, line width=2pt]
            ([yshift=-1.2ex]title.south-|frame.west) to
            ([yshift=-1.2ex]title.south-|frame.east);},
    overlay first={%
        \draw[Green, line width=2pt]
            ([yshift=-1.2ex]title.south-|frame.west) to
            ([yshift=-1.2ex]title.south-|frame.east);
    }
}{thm}

%--------------------Declared Math Operators--------------------%
\DeclareMathOperator{\adjoint}{adj}         % Adjoint.
\DeclareMathOperator{\Card}{Card}           % Cardinality.
\DeclareMathOperator{\curl}{curl}           % Curl.
\DeclareMathOperator{\diam}{diam}           % Diameter.
\DeclareMathOperator{\dist}{dist}           % Distance.
\DeclareMathOperator{\Div}{div}             % Divergence.
\DeclareMathOperator{\Erf}{Erf}             % Error Function.
\DeclareMathOperator{\Erfc}{Erfc}           % Complementary Error Function.
\DeclareMathOperator{\Ext}{Ext}             % Exterior.
\DeclareMathOperator{\GCD}{GCD}             % Greatest common denominator.
\DeclareMathOperator{\grad}{grad}           % Gradient
\DeclareMathOperator{\Ima}{Im}              % Image.
\DeclareMathOperator{\Int}{Int}             % Interior.
\DeclareMathOperator{\LC}{LC}               % Leading coefficient.
\DeclareMathOperator{\LCM}{LCM}             % Least common multiple.
\DeclareMathOperator{\LM}{LM}               % Leading monomial.
\DeclareMathOperator{\LT}{LT}               % Leading term.
\DeclareMathOperator{\Mod}{mod}             % Modulus.
\DeclareMathOperator{\Mon}{Mon}             % Monomial.
\DeclareMathOperator{\multideg}{mutlideg}   % Multi-Degree (Graphs).
\DeclareMathOperator{\nul}{nul}             % Null space of operator.
\DeclareMathOperator{\Ord}{Ord}             % Ordinal of ordered set.
\DeclareMathOperator{\Prin}{Prin}           % Principal value.
\DeclareMathOperator{\proj}{proj}           % Projection.
\DeclareMathOperator{\Refl}{Refl}           % Reflection operator.
\DeclareMathOperator{\rk}{rk}               % Rank of operator.
\DeclareMathOperator{\sgn}{sgn}             % Sign of a number.
\DeclareMathOperator{\sinc}{sinc}           % Sinc function.
\DeclareMathOperator{\Span}{Span}           % Span of a set.
\DeclareMathOperator{\Spec}{Spec}           % Spectrum.
\DeclareMathOperator{\supp}{supp}           % Support
\DeclareMathOperator{\Tr}{Tr}               % Trace of matrix.
%--------------------Declared Math Symbols--------------------%
\DeclareMathSymbol{\minus}{\mathbin}{AMSa}{"39} % Unary minus sign.
%------------------------New Commands---------------------------%
\DeclarePairedDelimiter\norm{\lVert}{\rVert}
\DeclarePairedDelimiter\ceil{\lceil}{\rceil}
\DeclarePairedDelimiter\floor{\lfloor}{\rfloor}
\newcommand*\diff{\mathop{}\!\mathrm{d}}
\newcommand*\Diff[1]{\mathop{}\!\mathrm{d^#1}}
\renewcommand*{\glstextformat}[1]{\textcolor{RoyalBlue}{#1}}
\renewcommand{\glsnamefont}[1]{\textbf{#1}}
\renewcommand\labelitemii{$\circ$}
\renewcommand\thesubfigure{%
    \arabic{chapter}.\arabic{figure}.\arabic{subfigure}}
\addto\captionsenglish{\renewcommand{\figurename}{Fig.}}
\numberwithin{equation}{section}

\renewcommand{\vector}[1]{\boldsymbol{\mathrm{#1}}}

\newcommand{\uvector}[1]{\boldsymbol{\hat{\mathrm{#1}}}}
\newcommand{\topspace}[2][]{(#2,\tau_{#1})}
\newcommand{\measurespace}[2][]{(#2,\varSigma_{#1},\mu_{#1})}
\newcommand{\measurablespace}[2][]{(#2,\varSigma_{#1})}
\newcommand{\manifold}[2][]{(#2,\tau_{#1},\mathcal{A}_{#1})}
\newcommand{\tanspace}[2]{T_{#1}{#2}}
\newcommand{\cotanspace}[2]{T_{#1}^{*}{#2}}
\newcommand{\Ckspace}[3][\mathbb{R}]{C^{#2}(#3,#1)}
\newcommand{\funcspace}[2][\mathbb{R}]{\mathcal{F}(#2,#1)}
\newcommand{\smoothvecf}[1]{\mathfrak{X}(#1)}
\newcommand{\smoothonef}[1]{\mathfrak{X}^{*}(#1)}
\newcommand{\bracket}[2]{[#1,#2]}

%------------------------Book Command---------------------------%
\makeatletter
\renewcommand\@pnumwidth{1cm}
\newcounter{book}
\renewcommand\thebook{\@Roman\c@book}
\newcommand\book{%
    \if@openright
        \cleardoublepage
    \else
        \clearpage
    \fi
    \thispagestyle{plain}%
    \if@twocolumn
        \onecolumn
        \@tempswatrue
    \else
        \@tempswafalse
    \fi
    \null\vfil
    \secdef\@book\@sbook
}
\def\@book[#1]#2{%
    \refstepcounter{book}
    \addcontentsline{toc}{book}{\bookname\ \thebook:\hspace{1em}#1}
    \markboth{}{}
    {\centering
     \interlinepenalty\@M
     \normalfont
     \huge\bfseries\bookname\nobreakspace\thebook
     \par
     \vskip 20\p@
     \Huge\bfseries#2\par}%
    \@endbook}
\def\@sbook#1{%
    {\centering
     \interlinepenalty \@M
     \normalfont
     \Huge\bfseries#1\par}%
    \@endbook}
\def\@endbook{
    \vfil\newpage
        \if@twoside
            \if@openright
                \null
                \thispagestyle{empty}%
                \newpage
            \fi
        \fi
        \if@tempswa
            \twocolumn
        \fi
}
\newcommand*\l@book[2]{%
    \ifnum\c@tocdepth >-3\relax
        \addpenalty{-\@highpenalty}%
        \addvspace{2.25em\@plus\p@}%
        \setlength\@tempdima{3em}%
        \begingroup
            \parindent\z@\rightskip\@pnumwidth
            \parfillskip -\@pnumwidth
            {
                \leavevmode
                \Large\bfseries#1\hfill\hb@xt@\@pnumwidth{\hss#2}
            }
            \par
            \nobreak
            \global\@nobreaktrue
            \everypar{\global\@nobreakfalse\everypar{}}%
        \endgroup
    \fi}
\newcommand\bookname{Book}
\renewcommand{\thebook}{\texorpdfstring{\Numberstring{book}}{book}}
\providecommand*{\toclevel@book}{-2}
\makeatother
\titleformat{\part}[display]
    {\Large\bfseries}
    {\partname\nobreakspace\thepart}
    {0mm}
    {\Huge\bfseries}
\titlecontents{part}[0pt]
    {\large\bfseries}
    {\partname\ \thecontentslabel: \quad}
    {}
    {\hfill\contentspage}
\titlecontents{chapter}[0pt]
    {\bfseries}
    {\chaptername\ \thecontentslabel:\quad}
    {}
    {\hfill\contentspage}
\newglossarystyle{longpara}{%
    \setglossarystyle{long}%
    \renewenvironment{theglossary}{%
        \begin{longtable}[l]{{p{0.25\hsize}p{0.65\hsize}}}
    }{\end{longtable}}%
    \renewcommand{\glossentry}[2]{%
        \glstarget{##1}{\glossentryname{##1}}%
        &\glossentrydesc{##1}{~##2.}
        \tabularnewline%
        \tabularnewline
    }%
}
\newglossary[not-glg]{notation}{not-gls}{not-glo}{Notation}
\newcommand*{\newnotation}[4][]{%
    \newglossaryentry{#2}{type=notation, name={\textbf{#3}, },
                          text={#4}, description={#4},#1}%
}
%--------------------------LENGTHS------------------------------%
% Spacings for the Table of Contents.
\addtolength{\cftsecnumwidth}{1ex}
\addtolength{\cftsubsecindent}{1ex}
\addtolength{\cftsubsecnumwidth}{1ex}
\addtolength{\cftfignumwidth}{1ex}
\addtolength{\cfttabnumwidth}{1ex}

% Indent and paragraph spacing.
\setlength{\parindent}{0em}
\setlength{\parskip}{0em}                                                           %
\geometry{b5paper, margin=1in}
%----------------------------Main Document-------------------------------------%
\begin{document}
    \title{A CW Complex That is Not First Countable}
    \author{Ryan Maguire}
    \date{\vspace{-5ex}}
    \maketitle
    Let $X^{0}=\vector{0}$ consist of a single point and
    $X^{1}=\coprod_{n\in\mathbb{N}}I$. That is, the disjoint union of countably
    many intervals. An element is denoted $(n,x)$ with $n\in\mathbb{N}$ and
    $x\in{I}$. The attaching maps send $(n,0)$ and $(n,1)$ to $\vector{0}$
    for all $n\in\mathbb{N}$, so we obtain countably many circles which are
    connected at a single point. The CW Complex topology is then just the
    normal quotient topology since this is 1 dimensional. The resulting space is
    not first countable since $[\vector{0}]$ does not have a countable
    neighborhood basis. For suppose it does and let $\mathcal{B}$ be such a
    basis. Since $\mathcal{B}$ is hypothesized to be countable, there is a
    surjection $\mathcal{U}:\mathbb{N}\rightarrow\mathcal{B}$. Since
    $\mathcal{U}_{n}\in\mathcal{B}$ for all $n\in\mathbb{N}$, $\mathcal{U}_{n}$
    is open. But a set is open if and only if $\pi^{\minus{1}}[\mathcal{U}_{n}]$
    is open where $\pi$ is the canonical projection map. Since
    $[\vector{0}]\in\mathcal{U}_{n}$, $[(n,0)],[(n,1)]\in\mathcal{U}_{n}$ since
    $[\vector{0}]=[(n,0)]=[(n,1)]$ for all $n\in\mathbb{N}$. But then $(n,0)$
    and $(n,1)$ are elements of $\pi^{\minus{1}}[\mathcal{U}_{n}]$. Since
    $\pi^{\minus{1}}[\mathcal{U}_{n}]$ is open there exists $a_{n}>0$ and
    $b_{n}>0$ such that
    $\{n\}\times[0,a_{n})\subseteq\pi^{\minus{1}}[\mathcal{U}_{n}]$ and
    $\{n\}\times(b_{n},1]\subseteq\pi^{\minus{1}}[\mathcal{U}_{n}]$. Let
    $r_{n}=\min\{a_{n},b_{n},\frac{1}{2}\}/2$ and define:
    \begin{equation}
        \mathcal{O}=\{\vector{0}\}\cup\bigcup_{n\in\mathbb{N}}\Big(
            \big(\{n\}\times[0,r_{n})\big)\cup\big(\{n\}\times(r_{n},1]\big)
        \Big)
    \end{equation}
    Then $\mathcal{O}$ is open in the disjoint topology and is saturated since
    $\pi^{\minus{1}}\big[\pi[\mathcal{O}]\big]=\mathcal{O}$. Therefore
    $[\mathcal{O}]=\pi[\mathcal{O}]$ is open and contains $[\vector{0}]$. But if
    $[\mathcal{O}]$ contains $[\vector{0}]$, then since $\mathcal{B}$ is a
    neighborhood basis of $[\vector{0}]$ there is a $\mathcal{V}\in\mathcal{B}$
    such that $\mathcal{V}\subseteq[\mathcal{O}]$. But $\mathcal{U}$ is
    surjective so there is an $N\in\mathbb{N}$ such that
    $\mathcal{U}_{N}=\mathcal{V}$. But
    $\pi[\{N\}\times(r_{N},a_{N})]\subseteq\mathcal{U}_{N}$ and
    $\pi[\{N\}\times(r_{N},a_{N})]\nsubseteq[\mathcal{O}]$, a contradiction
    since $\mathcal{U}\subseteq[\mathcal{O}]$. Hence, $\mathcal{B}$ is not a
    neighborhood basis.
    \par\hfill\par
    Some comments. We've used countable choice for the $r_{n}$. A sequential
    space is a topological space where sequences suffices to define openness.
    $\mathcal{U}\subseteq{X}$ is open if and only if for every sequence
    converging to a point in $\mathcal{U}$, the sequence is \textit{eventually}
    contained in $\mathcal{U}$. That is, open and sequentially open are
    identical. Open always implies sequentially open, the
    converse may fail (cocountable topology on $\mathbb{R}$). In a sequential
    space continuity and sequential continuity are identical. A very mild
    condition that a space be sequential
    is first countability. It is not necessary, just sufficient. A necessary and
    sufficient condition is the following:
    \begin{theorem}
        A topological space $\topspace{X}$ is sequential if and only if it is
        the quotient of a metric space.
    \end{theorem}
    Since the disjoint union of metric spaces is still a metric space (assuming
    axiom of choice), and since CW complex are obtained from quotients of
    these, CW complexes are indeed sequential. Hence, sequences suffice to prove
    continuity.
    \begin{definition}
        A sequential topological space is a topological space $\topspace{X}$
        such that $\mathcal{U}\subseteq{X}$ is open if and only if $\mathcal{U}$
        is sequentially open. That is, for every sequence
        $a:\mathbb{N}\rightarrow{X}$ with a limit $x\in\mathcal{U}$ there exists
        $N\in\mathbb{N}$ such that for all $n>N$ it is true that
        $a_{n}\in\mathcal{U}$.
    \end{definition}
    We say \textit{a} limit since no separation axioms are assumed, hence limits
    need not be unique. Since CW complexes are Hausdorff this does not cause
    concern.
    \begin{theorem}
        Open implies sequentially open.
    \end{theorem}
    \begin{proof}
        By the definition of convergence. If $a_{n}\rightarrow{x}$, then for
        every open $\mathcal{V}$ containing $x$ there is an $N\in\mathbb{N}$
        such that for all $n>N$ it is true that $a_{n}\in\mathcal{V}$. Setting
        $\mathcal{V}=\mathcal{U}$ gives the result.
    \end{proof}
    \begin{theorem}
        Continuous implies sequentially continuous.
    \end{theorem}
    \begin{proof}
        Standard topological result.
    \end{proof}
    \begin{theorem}
        First countable implies sequential.
    \end{theorem}
    \begin{proof}
        Another standard theorem.
    \end{proof}
    \begin{theorem}
        Metrizable spaces are sequential.
    \end{theorem}
    \begin{proof}
        For metrizable implies first countable, which implies sequential.
    \end{proof}
    \begin{theorem}
        If $\topspace[\alpha]{X_{\alpha}}$ are a collection of sequential
        spaces, if $Y=\coprod_{\alpha}X_{\alpha}$, and if $\tau$ is the
        disjoint union topology, then $\topspace{Y}$ is sequential.
    \end{theorem}
    \begin{proof}
        For suppose $\mathcal{V}\subseteq{Y}$ is sequentially open but not open.
        But $\mathcal{V}$ is open if and only if for all $\alpha$, the pre-image
        under the canonical inclusion mapping
        $\iota_{\alpha}:X_{\alpha}\rightarrow{Y}$ is open. That is,
        $\iota_{\alpha}^{\minus{1}}[\mathcal{V}]$ is open. Hence if
        $\mathcal{V}$ is \textit{not} open, then there exists an $\alpha$ such
        such that $\iota_{\alpha}^{\minus{1}}[\mathcal{V}]\subseteq{X}_{\alpha}$
        is not open. Let $\mathcal{U}=\iota_{\alpha}^{\minus{1}}[\mathcal{V}]$.
        But $X_{\alpha}$ is sequential and hence if $\mathcal{U}$ is not open,
        then it is not sequentially open. But then there is a sequence
        $a:\mathbb{N}\rightarrow{X}_{\alpha}$ with a limit $x\in\mathcal{U}$
        such that for all $N\in\mathbb{N}$ there is an $n>N$ with
        $a_{n}\notin\mathcal{U}$. But $\iota_{\alpha}$ is continuous and hence
        $\iota_{\alpha}(a_{n})\rightarrow\iota_{\alpha}(x)$. But
        $\iota_{\alpha}(x)\in\mathcal{V}$ and $\mathcal{V}$ is sequentially
        continuous. Therefore there is an $N\in\mathbb{N}$ such that for all
        $n>N$ it is true that $\iota_{\alpha}(a_{n})\in\mathcal{V}$. But then
        for all $n>N$, $a_{n}\in\mathcal{U}$, a contradiction. Hence,
        $\mathcal{U}$ is open and $\mathcal{V}$ is also open. That is, $Y$ is
        sequential.
    \end{proof}
    \begin{theorem}[Exercise 2.4G of Engelking's Topology]
        If $\topspace[X]{X}$ and $\topspace[Y]{Y}$ are topological spaces, if
        $X$ is sequential, and if $q:X\rightarrow{Y}$ is a quotient map, then
        $Y$ is sequential.
    \end{theorem}
    \begin{proof}
        For let $\mathcal{V}\subseteq{Y}$ be sequentially open. It suffices to
        prove $\mathcal{V}$ is open. Let
        $\mathcal{U}=q^{\minus{1}}[\mathcal{V}]$ and suppose this is not open.
        But $X$ is sequential and hence $\mathcal{U}$ is not sequentially open.
        But then there is a sequence $a:\mathbb{N}\rightarrow{X}$ with a limit
        $x\in\mathcal{U}$ such that for all $N\in\mathbb{N}$ there is an
        $n\in\mathbb{N}$ such that $n>N$ and $a_{n}\notin\mathcal{U}$. But $q$
        is continuous, and hence sequentially continuous, and therefore
        $q(a_{n})\rightarrow{q}(x)$. But $q(x)\in\mathcal{V}$ and $\mathcal{V}$
        is sequentially open, hence there is an $N\in\mathbb{N}$ such that
        for all $n>N$ it is true that $q(a_{n})\in\mathcal{V}$. But then for
        all $n>N$ we have $a_{n}\in\mathcal{U}$, a contradiction. Hence,
        $\mathcal{U}$ is open. But $\mathcal{V}$ is open if and only if
        $q^{\minus{1}}[\mathcal{V}]$ is open, and therefore $\mathcal{V}$ is
        open. That is, $Y$ is sequential.
    \end{proof}
    \begin{theorem}
        If $\topspace{X}$ is a finite dimensional CW complex, then it is
        sequential.
    \end{theorem}
    \begin{proof}
        We prove by induction on the dimension. The base case $n=0$ is trivially
        true since the disjoint union of points is a discrete space which is
        metrizable, and therefore sequential. Suppose it is true for
        $n\in\mathbb{N}$. If $X$ is an $n+1$ dimensional CW complex then it is
        of the form $X^{n}\coprod{X}^{n+1}/\sim$ where $X^{n}$ is an $n$
        dimensional CW complex, $X^{n+1}$ is the disjoint union of $n+1$
        dimensional closed balls, and $\sim$ is the equivalence relation induced
        by the attaching maps of the $n+1$ skeleton into $X^{n}$. But $X^{n}$ is
        sequential by hypothesis, and $X^{n+1}$ is
        the disjoint union of $n+1$ balls, which is metrizable, and hence
        sequential. But the disjoint union of sequential spaces is sequential,
        and hence $X^{n}\coprod{X}^{n+1}$ is sequential. But then
        $X^{n}\coprod{X}^{n+1}/\sim$ is the quotient of a sequential space, and
        is therefore sequential. Therefore, finite dimensional CW complexes are
        sequential.
    \end{proof}
    \begin{theorem}
        If $X$ is a CW complex, then it is sequential.
    \end{theorem}
    \begin{proof}
        If $X$ is a finite dimensional CW complex, then we are done. So suppose
        not and let $\mathcal{U}\subseteq{X}$ be sequentially open. To show that
        it is open suffices to check that $\mathcal{U}\cap{X}^{n}$ is an open
        subset of $X^{n}$ for all $n\in\mathbb{N}$. But $X^{n}$ is sequential,
        and hence if $\mathcal{U}\cap{X}^{n}$ is not open, then it is not
        sequentially open. But then there is a sequence
        $a:\mathbb{N}\rightarrow{X}^{n}$ with limit $x\in\mathcal{U}\cap{X}^{n}$
        such that for all $K\in\mathbb{N}$ there is an $k>K$ such that
        $a_{k}\notin\mathcal{U}\cap{X}^{n}$. But $\mathcal{U}$ is sequential,
        hence if $a_{k}\rightarrow{x}$ and $x\in\mathcal{U}$, then there is a
        $K\in\mathbb{N}$ such that for all $k>K$ it is true that
        $a_{k}\in\mathcal{U}$. But for all $k\in\mathbb{N}$ we have
        $a_{k}\in{X}^{k}$ by definition since $a$ is a sequence in $X^{n}$, and
        hence for all $k>K$ we have $a_{k}\in\mathcal{U}\cap{X}^{n}$, a
        contradiction. Therefore $\mathcal{U}\cap{X}^{n}$ is sequentially open,
        and thus open in $X^{n}$. But this is true of all $n\in\mathbb{N}$, and
        hence $\mathcal{U}$ is open.
    \end{proof}
\end{document}