%------------------------------------------------------------------------------%
\documentclass{article}                                                        %
%------------------------------Preamble----------------------------------------%
\makeatletter                                                                  %
    \def\input@path{{../../}}                                                  %
\makeatother                                                                   %
%---------------------------Packages----------------------------%
\usepackage{geometry}
\geometry{b5paper, margin=1.0in}
\usepackage[T1]{fontenc}
\usepackage{graphicx, float}            % Graphics/Images.
\usepackage{natbib}                     % For bibliographies.
\bibliographystyle{agsm}                % Bibliography style.
\usepackage[french, english]{babel}     % Language typesetting.
\usepackage[dvipsnames]{xcolor}         % Color names.
\usepackage{listings, lstlinebgrd}      % Verbatim-Like Tools.
\usepackage{mathtools, esint, mathrsfs} % amsmath and integrals.
\usepackage{amsthm, amsfonts}           % Fonts and theorems.
\usepackage{tabularx}
\usepackage{tcolorbox}                  % Frames around theorems.
\usepackage{upgreek}                    % Non-Italic Greek.
\usepackage{paracol}                    % Two-column styling.
\usepackage{wrapfig}                    % Wrap text around figure.
\usepackage{fmtcount, etoolbox}         % For the \book{} command.
\usepackage[newparttoc]{titlesec}       % Formatting chapter, etc.
\usepackage{titletoc}                   % Allows \book in toc.
\usepackage[nottoc]{tocbibind}          % Bibliography in toc.
\usepackage[titles]{tocloft}            % ToC formatting.
\usepackage{multicol, enumitem}         % Multi-column/enumerate.
\usepackage{import}                     % Import external files.
\usepackage{pgfplots, tikz}             % Drawing/graphing tools.
\usetikzlibrary{
    calc,                   % Calculating right angles and more.
    angles,                 % Drawing angles within triangles.
    arrows.meta,            % Latex and Stealth arrows.
    quotes,                 % Adding labels to angles.
    positioning,            % Relative positioning of nodes.
    decorations.markings,   % Adding arrows in the middle of a line.
    patterns,
    arrows,
    shapes,
    shapes.geometric,
    cd,
    hobby,
    babel
}                                       % Libraries for tikz.
\pgfplotsset{compat=1.9}                % Version of pgfplots.
\usepackage[font=scriptsize,
            labelformat=simple,
            labelsep=colon]{subcaption} % Subfigure captions.
\usepackage[font={scriptsize},
            hypcap=true,
            labelsep=colon]{caption}    % Figure captions.
\usepackage{hyperref}                   % Allows for hyperlinks.
\hypersetup{
    colorlinks=true,
    linkcolor=blue,
    filecolor=magenta,
    urlcolor=Cerulean,
    citecolor=SkyBlue
}                           % Colors for hyperref.
\usepackage[toc,acronym,nogroupskip]{glossaries} % Glossaries and acronyms.
\usepackage[subpreambles=false]{standalone}      % Complileable sub files.

% Various font stuff from kiwi.
% Use this for Times text and Computer Modern math
%\usepackage{times}

% Quite nice
%\usepackage[charter, greekfamily=, greekuppercase=italicized]{mathdesign}
%\usepackage[utopia, greekuppercase=italicized]{mathdesign}    % Math is narrower

% Use this for Times text and math
%\usepackage{newtxtext}
%\usepackage[libertine,cmintegrals]{newtxmath}
%\usepackage{fix-cm}

%\usepackage{txfontsb}
% or
%\usepackage{mathptmx}

%\usepackage[scaled=0.92]{helvet}
%\renewcommand{\rmdefault}{ptm}

%\usepackage{mathpazo}    % add possibly `sc` and `osf` options
%\usepackage{eulervm}

%\usepackage{fourier}
%\renewcommand{\rmdefault}{ptm}
%\usepackage{mathptm}

%\usepackage{fontspec}
%\setmainfont{lmodern}

%\usepackage[varg]{txfonts}
%\usepackage{fouriernc}
%\usepackage{mathpazo}

%\usepackage{bookman}
%\usepackage[scaled]{uarial}
%\usepackage[scaled]{helvet}
%\renewcommand*\familydefault{\sfdefault}
%\usepackage[math]{anttor}

%\newcommand\fgeorgia{\fontfamily{jvn}\selectfont}
%\newcommand\ftimes{\fontfamily{ptm}\selectfont}
%\newcommand\fhelvetica{\fontfamily{phv}\selectfont}
%\newcommand\fcourier{\fontfamily{pcr}\selectfont}
%\newcommand\fbookman{\fontfamily{pbk}\selectfont}
%\newcommand\fnewcentury{\fontfamily{pnc}\selectfont}
%\newcommand\fpalatino{\fontfamily{ppl}\selectfont}
%\newcommand\favantgarde{\fontfamily{pag}\selectfont}
%\newcommand\fnormal{\normalfont}
%\newcommand\fsize[1]{\ifnum#1>0\fontsize{#1}{#1}\selectfont\else\normalsize\fi}
%------------------------Theorem Styles-------------------------%
% Define theorem style for default spacing and normal font.
\newtheoremstyle{normal}
    {\topsep}               % Amount of space above the theorem.
    {\topsep}               % Amount of space below the theorem.
    {}                      % Font used for body of theorem.
    {}                      % Measure of space to indent.
    {\bfseries}             % Font of the header of the theorem.
    {}                      % Punctuation between head and body.
    {.5em}                  % Space after theorem head.
    {}

% Define theorem style for default spacing with italicized font.
\newtheoremstyle{normalit}{\topsep}{\topsep}
                {\itshape}{}{\bfseries}{}{.5em}{}

% Italic header environment.
\newtheoremstyle{thmit}{\topsep}{\topsep}{}{}{\itshape}{}{0.5em}{}

% Define italicized environments.
\theoremstyle{normalit}
\newtheorem{theorem}{Theorem}[section]
\newtheorem{lemma}{Lemma}[section]
\newtheorem{corollary}{Corollary}[section]
\newtheorem{proposition}{Proposition}[section]
\newtheorem*{theorem*}{Theorem}

% Define environments with italic headers.
\theoremstyle{thmit}
\newtheorem*{solution}{Solution}
\newtheorem*{fsolution}{Solution}

% Define default environments.
\theoremstyle{normal}
\newtheorem{example}{Example}[section]
\newtheorem{definition}{Definition}[section]
\newtheorem{problem}{Problem}[section]
\newtheorem{question}{Question}[section]
\newtheorem{remark}{Remark}[section]
\newtheorem{properties}{Properties}[section]
\newtheorem{notation}{Notation}[section]
\newtheorem{axiom}{Axiom}[section]
\newtheorem*{properties*}{Properties}
\newtheorem*{remark*}{Remark}
\newtheorem*{definition*}{Definition}
\theoremstyle{plain}

% Define framed environment.
\tcbuselibrary{most}
\newtcbtheorem[use counter*=theorem]{ftheorem}{Theorem}%
    {colback=green!5,colframe=green!35!black,
     fonttitle=\bfseries\upshape}{th}

\newtcbtheorem[use counter*=example]{fdefinition}{Definition}%
    {fonttitle=\bfseries\upshape,
     colback=blue!5!white,colframe=blue!75!black}{def}

\newtcbtheorem[use counter*=example]{fexample}{Example}%
    {fonttitle=\bfseries\upshape,
     colback=red!5!white,colframe=red!75!black}{ex}

\newtcbtheorem[use counter*=notation]{fnotation}{Notation}%
    {fonttitle=\bfseries\upshape,
     colback=SeaGreen!5!white,colframe=SeaGreen!75!black}{ex}

\newtcbtheorem[use counter*=corollary]{fcorollary}{Corollary}%
    {fonttitle=\bfseries\upshape,
     colback=Orchid!5!white,colframe=Orchid!75!black}{ex}

\newenvironment{bproof}{\textit{Proof.}}{\hfill$\square$}
\tcolorboxenvironment{bproof}{blanker,breakable,left=5mm,
                             before skip=10pt,after skip=10pt,
                             borderline west={1mm}{0pt}{red}}
\tcolorboxenvironment{fsolution}
    {enhanced jigsaw,colframe=cyan,interior hidden,breakable}

%--------------------Declared Math Operators--------------------%
\DeclareMathOperator{\Refl}{Refl}           % Reflection operator.
\DeclareMathOperator{\Span}{Span}           % Span of a set of vectors.
\DeclareMathOperator{\Card}{Card}           % Cardinality of set.
\DeclareMathOperator{\Ord}{Ord}             % Ordinal of ordered set.
\DeclareMathOperator{\Tr}{Tr}               % Trace of matrix.
\DeclareMathOperator{\adjoint}{adj}         % Adjoint of matrix.
\DeclareMathOperator{\rk}{rk}               % Rank of operator.
\DeclareMathOperator{\nul}{nul}             % Null space of operator.
\DeclareMathOperator{\sgn}{sgn}             % Sign of a number.
\DeclareMathOperator{\multideg}{mutlideg}   % Multi-Degree (Graphs).
\DeclareMathOperator{\GCD}{GCD}             % Greatest common denominator.
\DeclareMathOperator{\LM}{LM}               % Leading monomial
\DeclareMathOperator{\LC}{LC}               % Leading coefficient.
\DeclareMathOperator{\LT}{LT}               % Leading term.
\DeclareMathOperator{\LCM}{LCM}             % Least common multiple.
\DeclareMathOperator{\Mon}{Mon}             % Monomial.
\DeclareMathOperator{\Spec}{Spec}           % Spectrum.
\DeclareMathOperator{\proj}{proj}           % Projection.
\DeclareMathOperator{\comp}{comp}           % Component.
\DeclareMathOperator{\sinc}{sinc}           % Sinc function.
\DeclareMathOperator{\Ima}{Im}              % Image of operator.
\DeclareMathOperator{\Prin}{Prin}           % Principal value.
\DeclareMathOperator{\Mod}{mod}             % Modulus.
%------------------------New Commands---------------------------%
\DeclarePairedDelimiter\norm{\lVert}{\rVert}
\DeclarePairedDelimiter\ceil{\lceil}{\rceil}
\DeclarePairedDelimiter\floor{\lfloor}{\rfloor}
\newcommand*\diff{\mathop{}\!\mathrm{d}}
\newcommand*\Diff[1]{\mathop{}\!\mathrm{d^#1}}
\renewcommand{\mod}{\ \Mod}
\renewcommand*{\glstextformat}[1]{\textcolor{RoyalBlue}{#1}}
\renewcommand{\glsnamefont}[1]{\textbf{#1}}
\renewcommand\labelitemii{$\circ$}
\renewcommand\thesubfigure{\arabic{chapter}.\arabic{figure}}
\renewcommand\thesubfigure{%
    \arabic{chapter}.\arabic{figure}.\arabic{subfigure}}
\addto\captionsenglish{\renewcommand{\figurename}{Fig.}}
%------------------------Book Command---------------------------%
\makeatletter
\renewcommand\@pnumwidth{1cm}
\newcounter{book}
\renewcommand\thebook{\@Roman\c@book}
\newcommand\book{%
    \if@openright
        \cleardoublepage
    \else
        \clearpage
    \fi
    \thispagestyle{plain}%
    \if@twocolumn
        \onecolumn
        \@tempswatrue
    \else
        \@tempswafalse
    \fi
    \null\vfil
    \secdef\@book\@sbook
}
\def\@book[#1]#2{%
    \ifnum \c@secnumdepth >-3\relax
        \refstepcounter{book}%
        \addcontentsline{toc}{book}{
            \bookname\ \thebook:\hspace{1em}#1
        }
    \else
        \addcontentsline{toc}{book}{#1}%
    \fi
    \markboth{}{}%
    {\centering
     \interlinepenalty \@M
     \normalfont
     \ifnum \c@secnumdepth >-2\relax
       \huge\bfseries \bookname\nobreakspace\thebook
       \par
       \vskip 20\p@
     \fi
     \Huge \bfseries #2\par}%
    \@endbook}
\def\@sbook#1{%
    {\centering
     \interlinepenalty \@M
     \normalfont
     \Huge \bfseries #1\par}%
    \@endbook}
\def\@endbook{
    \vfil\newpage
        \if@twoside
            \if@openright
                \null
                \thispagestyle{empty}%
                \newpage
            \fi
        \fi
        \if@tempswa
            \twocolumn
        \fi
}
\newcommand*\l@book[2]{%
    \ifnum \c@tocdepth >-2\relax
        \addpenalty{-\@highpenalty}%
        \addvspace{2.25em \@plus\p@}%
        \setlength\@tempdima{3em}%
        \begingroup
            \parindent \z@ \rightskip \@pnumwidth
            \parfillskip -\@pnumwidth
            {
                \leavevmode
                \Large \bfseries #1\hfil \hb@xt@\@pnumwidth{
                    \hss #2
                }
            }
            \par
            \nobreak
            \global\@nobreaktrue
            \everypar{\global\@nobreakfalse\everypar{}}%
        \endgroup
    \fi}
\newcommand\bookname{Book}
\renewcommand{\thebook}{\texorpdfstring{\Numberstring{book}}{book}}
\providecommand*{\toclevel@book}{-2}
\makeatother
\titlecontents{chapter}[0pt]
    {\bfseries}
    {\chaptername\ \thecontentslabel:\quad}
    {}
    {\hfill\contentspage}
\titleformat{\part}[display]
    {\Large\bfseries}
    {\partname\nobreakspace\thepart}
    {0mm}
    {\Huge\bfseries}
    \titlecontents{part}[0pt]
    {\large\bfseries}
    {\partname\ \thecontentslabel: \quad}
    {}
    {\hfill\contentspage}
\newcommand{\MarkRightAngle}[4][.3cm]
    {\coordinate (tempa) at ($(#3)!#1!(#2)$);
     \coordinate (tempb) at ($(#3)!#1!(#4)$);
     \coordinate (tempc) at ($(tempa)!0.5!(tempb)$);%midpoint
     \draw (tempa) -- ($(#3)!2!(tempc)$) -- (tempb);}
%--------------------------LENGTHS------------------------------%
% Spacings for the Table of Contents.
\addtolength{\cftsecnumwidth}{1ex}
\addtolength{\cftsubsecindent}{1ex}
\addtolength{\cftsubsecnumwidth}{1ex}
\addtolength{\cftfignumwidth}{1ex}
\addtolength{\cfttabnumwidth}{1ex}

% Spacing for multi-column and enumerate environments.
\setlength{\multicolsep}{6pt}
\setlist[enumerate]{itemsep=0pt,topsep=3pt}

% Indent and paragraph spacing.
\setlength{\parindent}{0em}
\setlength{\parskip}{0em}                                                           %
\makeindex[intoc]                                                              %
%----------------------------Main Document-------------------------------------%
\begin{document}
    \pagenumbering{gobble}
    \title{MATH 114 Algebraic Topology - Assignments}
    \author{Ryan Maguire}
    \date{\vspace{-5ex}}
    \maketitle
    \pagenumbering{roman}
    \pagenumbering{arabic}
    \setcounter{section}{2}
\section{Wrong problems... Woops}
    \begin{problem}
        Show that the free product of non-trivial groups $G$ and $H$ has trivial
        center and that the only elements of finite order are the conjugates
        of finite order elements in $G$ and $H$.
    \end{problem}
    \begin{solution}
        For let $C\subseteq{G}*H$ be the center and suppose $C$ is non-trivial.
        Then there is an element $a\in{C}$ that is not the identity. That is,
        $a$ is not the empty word. Suppose $a$ is in reduced form and let
        $b_{1}$ be the first element and $b_{2}$ be the last,
        $a=b_{1}\cdots{b}_{2}$. If
        $b_{1},b_{2}\in{G}$, let $d\in{H}$ be an element other than the
        identity. Since $H$ is hypothesized to be non-trivial, such an element
        exists. But then $db_{1}$ cannot be reduced since $d\in{H}$ and
        $b_{1}\in{G}$. Similarly, $b_{2}d$ cannot be reduced. But
        $hb_{1}\cdots{b}_{2}$ and $b_{1}\cdots{b}_{2}h$ are distinct words, and
        thus $a$ is not in the center. For similar reasons, it is not possible
        that $b_{1},b_{2}\in{H}$. Thus, suppose $b_{1}\in{G}$ and $b_{2}\in{H}$
        and let $d=b_{1}$. Then $b_{2}d$ cannot be reduced, and $db_{1}$ may or
        may not be able to be reduced. If it cannot, then $da\ne{ad}$. If it can
        then we note that since $a$ is in reduced form it is an alternating
        sequence of elements in $G$ and $H$. Hence, after reducing $db_{1}$ we
        would have $db_{1}h_{1}g_{1}\cdots{b}_{2}$, which cannot be reduced
        further. Moreover, this is not equal to $ad$. Hence, the center is
        trivial. Next, given a word that is a conjugate of elements of finite
        order, it too has finite order. Since
        $(g^{\minus{1}}hg)^{n}=g^{\minus{1}}h^{n}g$, given the order $n$ of the
        element $h$, we would have $(g^{\minus{1}}hg)^{n}=g^{\minus{1}}h^{n}g$
        which simplifies to $g^{\minus{1}}e_{h}g=g^{\minus{1}}g=e_{g}$,
        which is equivalent to the empty word.
    \end{solution}
    \begin{problem}
        Show that $\nspace$ minus finitely many points is simply connected for
        $n\geq{3}$.
    \end{problem}
    \begin{solution}
        For $n\geq{2}$, $\nspace$ with finitely many points removed is a
        connected open subset of $\nspace$. Since $\nspace$ is locally path
        connected, this implies that $\nspace$ without finitely many points is
        path connected for $n\geq{2}$. Suppose first that we remove a single
        point from $\nspace$, $n\geq{3}$. Up to homeomorphism we may as well
        choose the origin, so let $X=\nspace\setminus\{\vector{0}\}$. Since
        $\gamma:\nsphere[1]\rightarrow{X}$ is a  continuous function from a
        compact set, its image is compact. But then there is a closed
        $\varepsilon>0$ ball around $\vector{0}$ that is disjoint with the image
        of $\gamma$. Using this we may project the image of $\gamma$ onto the
        $n-1$ sphere $\nsphere[n-1]$ using the following homotopy:
        \begin{equation}
            H(\theta,t)=(1-t)\cdot\gamma(\theta)
                +t\cdot\frac{\gamma(\theta)}{\norm{\gamma(\theta)}}
        \end{equation}
        Hence $\gamma$ is homotopic to a loop on $\nsphere[n-1]$. But for
        $k\geq{2}$ $\nsphere[k]$ is simply connected, and hence for
        $n\geq{3}$ we have that $\nsphere[n-1]$ is simply connected. Hence
        this loop is homotopic to a point on $\nsphere[n-1]$. Since
        $X$ is path connected, for any point $\vector{x}_{0}\in{X}$ the
        fundamental groups are isomorphic. This shows that, regardless of base
        point, the fundamental group is trivial. Therefore,
        $\nspace\setminus\{\vector{0}\}$ is simply connected. Now, for $m$
        points removed we may assume they are located at the coordinates
        $(k,0,\dots,0)$ for $k=0,\dots,m-1$. Placing spheres of radius
        $\frac{1}{2}$ around each of these we note that this is just the
        wedge of $\nsphere[n-1]$ $m$ times. Moreover, $\nspace$ with these $m$
        points removed deformation retracts down to this. Following a given
        loop $\gamma$ along this retraction gives us a loop on the wedge of
        $m$ spheres. We may deform the part of the loop in the last sphere
        (the sphere centered at $(m,0,\dots,0)$) to the point of intersection
        with the $m-1$ sphere since $\nsphere[n-1]$ is simply connected for
        $n\geq{3}$. But then the last point no longer affects our loop, and we
        essentially have a loop on the wedge of $m-1$ spheres. By induction,
        this is simply connected.
    \end{solution}
\section{Correct Problems}
    \begin{problem}
        Show that the composition of paths satisfies the following cancellation
        property. If $f_{0}\cdot{g}_{0}$ is homotopic to $f_{1}\cdot{g}_{1}$,
        and if $g_{0}$ is homotopic to $g_{1}$, then $f_{0}$ is homotopic to
        $f_{1}$.
    \end{problem}
    \begin{solution}
        This is true of any group. Given $a,b,c\in{G}$, $a*c=b*c$, we may
        multiply on the right by $c^{\minus{1}}$ obtaining
        $(a*c)*c^{\minus{1}}=(b*c)*c^{\minus{1}}$. Using associativity we have
        $a*(c*c^{\minus{1}})=a*e=a$ and similarly $b*(c*c^{\minus{1}})=b*e=b$.
        Since the equivalence class of paths is associative under concatenation
        we may adapt this argument. Given that $g_{0}$ is homotopic to $g_{1}$,
        there is a homotopy $H$. But then $H$  also gives us a homotopy between
        $g_{0}\cdot{g}_{0}^{\minus{1}}$ and $g_{1}\cdot{g}_{1}^{\minus{1}}$.
        Explicitly, let $G$ be defined by:
        \begin{equation}
            G(x,t)=
            \begin{cases}
                H(x,2t),&0\leq{t}\leq\frac{1}{2}\\
                H(x,2-2t),&\frac{1}{2}\leq{t}\leq{1}
            \end{cases}
        \end{equation}
        Since homotopic is transitive, we have
        $f_{0}\cdot(g_{0}\cdot{g_{0}}^{\minus{1}})$ is homotopic to
        $f_{1}\cdot(g_{1}\cdot{g}_{1}^{\minus{1}})$. But both of these
        concatenations on the right are homotopic to a point, and hence
        $f_{0}$ is homotopic to $f_{1}$.
    \end{solution}
    \begin{problem}
        Show that the change of basis homomorphism $\beta_{h}$ depends only on
        the homotopy class of $h$.
    \end{problem}
    \begin{solution}
        For suppose $g$ and $h$ are paths from $x_{0}$ and $x_{1}$ and that $g$
        is homotopic to $h$. Let $H$ be such a homotopy, $H(t,0)=g(t)$ and
        $H(t,1)=h(t)$. Then for any loop $f$ at $x_{1}$ we have that $g\cdot{f}$
        is homotopic to $h\cdot{f}$. Indeed, let $G(t)$ be defined by $H(2t)$
        for $t\in[0,\frac{1}{2}]$ and $f(t)$ otherwise. This is a continuous
        homotopy between $h\cdot{f}$ and
        $g\cdot{f}$. Similarly, $f\cdot\overline{h}$ and $f\cdot\overline{g}$
        are homotopic, further implying that $h\cdot{f}\cdot\overline{h}$ and
        $g\cdot{f}\cdot\overline{g}$ are homotopic. Hence:
        \begin{equation}
            \beta_{h}([f])=[h\cdot{f}\cdot\overline{h}]
                =[g\cdot{f}\cdot\overline{g}]=\beta_{g}([f])
        \end{equation}
    \end{solution}
    \begin{problem}
        Show that $\pi_{1}(X)$ is Abelian for a path connected $X$ if and only
        if every change of basepoint homomorphism $\beta_{h}$ depends only on
        the endpoint.
    \end{problem}
    \begin{solution}
        For let $x,y$ be points and $\gamma_{1},\gamma_{2}$ paths taking $x$ to
        $y$. Since $\pi_{1}(X,x)$ is Abelian, so is $\pi_{1}(X,y)$. But then
        given any loop $f$ at $y$,
        $\gamma_{1}\cdot{f}\cdot\overline{\gamma}_{1}$ is a loop at $x$ and
        similarly for $\gamma_{2}$. Moreover,
        $\gamma_{1}\cdot\overline{\gamma}_{2}$ is a loop at $x$. Thus:
        \begin{subequations}
            \begin{align}
                \beta_{\gamma_{1}}([f])
                &=[\gamma_{1}\cdot{f}\cdot\overline{\gamma}_{1}]\\
                &=[\gamma_{1}\cdot\overline{\gamma}_{2}
                    \cdot\gamma_{2}\cdot{f}\cdot\overline{\gamma}_{1}]\\
                &=[\gamma_{1}\cdot\overline{\gamma}_{2}]\cdot
                    [\gamma_{2}\cdot{f}\cdot\overline{\gamma}_{1}]\\
                &=[\gamma_{2}\cdot{f}\cdot\overline{\gamma}_{1}]\cdot
                    [\gamma_{1}\cdot\overline{\gamma}_{2}]\\
                &=[\gamma_{2}\cdot{f}\cdot\overline{\gamma}_{1}\cdot
                    \gamma_{1}\cdot\overline{\gamma}_{2}]\\
                &=[\gamma_{2}\cdot{f}\cdot\overline{\gamma}_{2}]\\
                &=\beta_{\gamma_{2}}([f])
            \end{align}
        \end{subequations}
        Hence $\beta$ only depends on the endpoints. Now, if $\beta$ only
        depends on the endpoint, then $\pi_{1}(X,x)$ is Abelian. Let
        $f,g$ be loops at $x$, and let $\gamma$ be the constant loop at $x$.
        But then:
        \begin{subequations}
            \begin{align}
                \beta_{f}([g])&=[f\cdot{g}\cdot\overline{f}]\\
                &=\beta_{\gamma}([g])\\
                &=[g]
            \end{align}
        \end{subequations}
        Hence $f\cdot{g}\cdot\overline{f}$ is homotopic to $g$. By concatenating
        both sides on the right by $f$ we have $[f]\cdot[g]=[g]\cdot[f]$ showing
        us the fundamental group is Abelian.
    \end{solution}
    \begin{problem}
        Show that the following are equivalent:
        \begin{enumerate}
            \item Every map $\nsphere[1]\rightarrow{X}$ is homotopic to a
                  constant map.
            \item Every map $\nsphere[1]\rightarrow{X}$ can be extended to a
                  continuous function $D^{2}\rightarrow{X}$.
            \item $\pi_{1}(X,x_{0})=\{e\}$ for all $x_{0}\in{X}$.
        \end{enumerate}
    \end{problem}
    \begin{solution}
        $(1\Rightarrow{2})$ If $f:\nsphere[1]\rightarrow{X}$ is homotopic to a
        constant map, then let $H:\nsphere[1]\times{I}\rightarrow{X}$ be such a
        homotopy. Then $H(x,1)=const$ for all $x\in\nsphere[1]$ so we may obtain
        a well-defined continuous map by passing $H$ to the quotient
        $(\nsphere[1]\times{I})/R$ where $R$ identifies $\nsphere[1]\times\{1\}$
        to a point. But the cylinder with it's top boundary identified is
        homeomorphic to $D^{2}$ so we may pass this function to $D^{2}$ giving
        us an extension of $f$.
        \par\hfill\par
        $(2\Rightarrow{3})$ Let $x_{0}\in{X}$ be arbitrary and let
        $f:I\rightarrow{X}$ be a loop based at $x_{0}$. We obtain a continuous
        function $\tilde{f}:\nsphere[1]\rightarrow{X}$ from $f$ since
        $f(0)=f(1)$, and since $I/R$ is homeomorphic to $\nsphere[1]$
        where $R$ is the equivalence relation identifying 0 and 1. Hence $f$ may
        be passed to the quotient. But by hypothesis any continuous function
        $\tilde{f}:\nsphere[1]\rightarrow{X}$ extends to a continuous function
        on $D^{2}$. But $D^{2}$ is contractible, and moreover there is a
        deformation retract of $D^{2}$ down to the point $(1,0)\in\nsphere[1]$
        which is the point identified with $f(0)=x_{0}$.
        But this deformation retraction gives us a homotopy of
        $f$ to the constant map $g:I\rightarrow{X}$ defined by $g(t)=x_{0}$.
        Since $x_{0}$ and $f$ are arbitrary, this shows $\pi_{1}(X,x_{0})$ is
        trivial for all $x_{0}\in{X}$.
        \par\hfill\par
        $(3\Rightarrow{1})$ This is more-or-less the definition of a trivial
        fundamental group. Let $f:\nsphere[1]\rightarrow{X}$ be a continuous
        function and let $x_{0}=f(1,0)$. Let $\tilde{f}$ be the loop defined by
        $\tilde{f}(t)=f(\exp(2\pi{i}t))$. Then $\tilde{f}$ is a loop at the
        point $x_{0}$, but $\pi_{1}(X,x_{0})$ is trivial and hence this is
        homotopic to the constant map $g:I\rightarrow{X}$ defined by
        $g(t)=x_{0}$. Lifting this homotopy to $f$ shows us that $f$ is also
        homotopic to a constant map.
        \par\hfill\par
        $X$ is simply connected if and only if all continuous maps from
        $\nsphere[1]$ into $X$ are homotopic. Firstly, this implies $X$ is
        path connected. For given $x,y\in{X}$ let $f$ be the constant map at
        $x$ and $g$ the constant map at $y$ (both having domains $\nsphere[1]$).
        By hypothesis these are homotopic so let $H$ be a homotopy. Then
        $H(0,t)$ is a path from $x$ to $y$. From the previous argument, if every
        map $\nsphere[1]\rightarrow{X}$ is homotopic to a point, then
        $\pi_{1}(X,x_{0})$ is trivial for all $x_{0}\in{X}$. Hence, $X$ is path
        connected with trivial fundamental group, and is therefore simply
        connected. In the other direction, if $X$ is simply connected then it is
        path connected and $\pi_{1}(X,x_{0})$ is trivial for all $x_{0}\in{X}$.
        Then, given any two maps $f,g:\nsphere[1]\rightarrow{X}$, let
        $x_{0}=f(0,0)$ and $x_{1}=g(0,0)$. But $\pi_{1}(X,x_{0})$ is trivial so
        $f$ is homotopic to the constant map at $x_{0}$, and similarly for
        $g$. But $X$ is path connected so the constant map at $x_{0}$ is
        homotopic to the constant map at $x_{1}$. Since homotopic is a
        transitive relation we have that $f$ is homotopic to $g$. Hence $X$ is
        simply connected if and only if every pair of continuous functions
        from $\nsphere[1]$ are homotopic.
    \end{solution}
\end{document}