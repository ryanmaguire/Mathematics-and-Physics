Contact topology and contact geometry can be very strange without the
proper motivation. Since we will be working with knot theory in a contact
setting, we first take some time to develop some of the theory. Historically it
is best to first consider \textit{symplectic geometry}. Consider a particle
in $\mathbb{R}^{3}$. We can consider both its position $\mathbb{p}$ and
velocity $\mathbf{v}$, and using the ordered pair $(\mathbf{p},\mathbf{v})$
we may consider this as a single point in $\mathbb{R}^{6}$. If there is a mass
$m$ associated to the particle, scaling the velocity component by $m$ yields
the \textit{phase-space coordinate} of the particle, the ordered pair
$(\mathbf{p},\,m\mathbf{v})$. The reasons for doing this are abundant. Firstly,
this is the setting of Hamiltonian mechanics
\cite{HamiltonMechanics1833}, a form of classical mechanics which extends and
reformulates the highly successful Lagrangian mechanics. But one need not even
go that far, the fourth order Runger-Kutte method, a numerical technique for
ordinary differential equations which has far superior error than the classical
Euler method, has formulations in phase space as well.
\par\hfill\par
The topological definition of symplectic manifolds is motivated by the
structure of physical phase spaces. Points $(\mathbf{p}(t),\,\mathbf{v}(t))$ in
phase space satisfy $\dot{\mathbf{p}}(t)=\mathbf{v}(t)$ and hence
$\mathbf{v}(t)$ can be seen to lie in the tangent space of $\mathbf{p}(t)$ for
each $t$. Labeling the position and velocity vectors
coordinates $\mathbf{p}=(p_{0},\,p_{1},\,p_{2})$ and
$\mathbf{v}=(v_{0},\,v_{1},\,v_{2})$, we can create two-form on
$\mathbb{R}^{6}$ defined by:
\begin{equation}
    \omega=\sum_{k=0}^{2}\textrm{d}p_{k}\land\textrm{d}v_{k}
\end{equation}
Symplectic topology begins by examining the characteristics of such a form
and axiomatizing them to a general smooth manifold. Two three key ingredients
are:
\begin{enumerate}
    \item The 2-form is closed: $\textrm{d}\omega=0$
    \item $\omega$ is non-degenerate: For every point in
        $\mathbf{u}\in\mathbb{R}^{6}$ if there is an
        $X\in{T}_{\mathbf{u}}\mathbb{R}^{6}$ such that
        $\omega(X,\,Y)=0$ for all $Y\in{T}_{\mathbf{u}}\mathbb{R}^{6}$, then
        $X=0$.
    \item The underlying manifold is even dimensional.
\end{enumerate}
This third point is redundant. The matrix $A$ representing $\omega$ is
skew-symmetric and such matrices have zero-determinant in odd dimensions
since:
\begin{equation}
    \textrm{det}(A)=\textrm{det}(-A^{T})=(-1)^{n}\textrm{det}(A)
\end{equation}
where $n$ is the dimension. For $\omega$ to be non-degenerate we require $A$
to be non-singular, and hence $n$ must be even.
\par\hfill\par
The previous discussion gives us the definition of a symplectic manifold.
\begin{definition}[\textbf{Symplectic Manifold}]
    A symplectic manifold is an ordered triple $(X,\,\mathcal{A},\,\omega)$
    where $(X,\,\mathcal{A})$ is a smooth manifold of even dimension and
    $\omega$ is a closed non-degenerate 2-form on $X$.
\end{definition}
Contact manifolds will be seen as the odd dimensional analogues of symplectic
manifolds, and many of the key theorems are motivated by symplectic results.
