Distinguishing the unknot from the trefoil seems intuitively clear. To prove
such a claim becomes trivial by introducing \textit{knot invariants}, which
are mathematical objects assigned to knot diagrams that do not change under
Reidemeister moves. The knot group is perhaps the easiest to describe, given
an embedding $\gamma:\mathbb{S}^{1}\rightarrow\mathbb{R}^{3}$ we simply
compute $\pi_{1}(\mathbb{R}^{3}\setminus\gamma[\mathbb{S}^{1}])$. Since
Reidemeister moves do not change the topology of the knot complement, this is
a valid knot invariant. Computing the knot group for the unknot produces
$\mathbb{Z}$, whereas the trefoil yields a non-Abelian group with presentation
$\langle{a,\,b}\;|\;aba=bab\rangle$.
\par\hfill\par
An easier and more pictorial method exists, \textit{tricolorability}, in which
we color the arcs of a knot diagram with three colors so that all three colors
meet at the crossings, or only one color is present at a crossing. We are
required to use each color at least once. Such a notion is indeed a knot
invariant, as can be checked via the Reidemeister moves. The trefoil is
tricolorable (see Fig.~\ref{fig:trefoil_tricolor}) whereas the unknot is not.
\par\hfill\par
\begin{figure}
    \centering
    \resizebox{0.4\textwidth}{!}{%
        \includegraphics{trefoil_tricolor.pdf}
    }
    \caption{Tricoloring of the Trefoil}
    \label{fig:trefoil_tricolor}
\end{figure}
There are a myriad of other invariants that have been invented throughout the
past century, but we will be concerned with knot polynomials and homology
theories. In particular, the Alexander, Jones, and HOMFLY-PT polynomials, and
Khovanov homology and its associated Poincar\'{e} polynomial. In this chapter
we'll define these invariants and discuss algorithms for their computation.
In the final chapter we tabulate these invariants for millions of knots
using the algorithms to be discussed.
