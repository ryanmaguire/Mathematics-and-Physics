%-----------------------------------LICENSE------------------------------------%
%   This file is part of Mathematics-and-Physics.                              %
%                                                                              %
%   Mathematics-and-Physics is free software: you can redistribute it and/or   %
%   modify it under the terms of the GNU General Public License as             %
%   published by the Free Software Foundation, either version 3 of the         %
%   License, or (at your option) any later version.                            %
%                                                                              %
%   Mathematics-and-Physics is distributed in the hope that it will be useful, %
%   but WITHOUT ANY WARRANTY; without even the implied warranty of             %
%   MERCHANTABILITY or FITNESS FOR A PARTICULAR PURPOSE.  See the              %
%   GNU General Public License for more details.                               %
%                                                                              %
%   You should have received a copy of the GNU General Public License along    %
%   with Mathematics-and-Physics.  If not, see <https://www.gnu.org/licenses/>.%
%----------------------------------Preamble------------------------------------%
\documentclass{book}
\usepackage{graphicx}   % Needed for figures.
\usepackage{amsmath}    % Needed for align.
\usepackage{amssymb}    % Needed for mathbb.
\usepackage{amsthm}     % For the theorem environment.
\usepackage{hyperref}   % Hyperlinks for URLs and references.

% Setup parameters for hyperlinks.
\hypersetup{colorlinks = true, linkcolor = blue}

%------------------------Theorem Styles-------------------------%
\theoremstyle{plain}
\newtheorem{theorem}{Theorem}[section]

% Define theorem style for default spacing and normal font.
\newtheoremstyle{normal}
    {\topsep}               % Amount of space above the theorem.
    {\topsep}               % Amount of space below the theorem.
    {}                      % Font used for body of theorem.
    {}                      % Measure of space to indent.
    {\bfseries}             % Font of the header of the theorem.
    {}                      % Punctuation between head and body.
    {.5em}                  % Space after theorem head.
    {}

% Define default environments.
\theoremstyle{normal}
\newtheorem{examplex}{Example}[section]
\newtheorem{definitionx}{Definition}[section]
\newtheorem{notationx}{Notation}[section]

\newenvironment{example}{%
    \pushQED{\qed}\renewcommand{\qedsymbol}{$\blacksquare$}\examplex%
}{%
    \popQED\endexamplex%
}

\newenvironment{definition}{%
    \pushQED{\qed}\renewcommand{\qedsymbol}{$\blacksquare$}\definitionx%
}{%
    \popQED\enddefinitionx%
}

\newenvironment{notation}{%
    \pushQED{\qed}\renewcommand{\qedsymbol}{$\blacksquare$}\notationx%
}{%
    \popQED\endnotationx%
}

\title{Khovanov Homology and Legendrian Simple Knots}
\author{Ryan Maguire}
\date{\today}

% No indent and no paragraph skip.
\setlength{\parindent}{0em}
\setlength{\parskip}{0em}

\begin{document}
    \pagenumbering{gobble}
    \begin{titlepage}
        \centering
        \LARGE{\bfseries{Khovanov Homology and Legendrian Simple Knots}}
        \par\vspace{3.5cm}
        \par\vspace{4cm}
        \Large{\itshape{Ryan Maguire}}
        \par\vspace{1.5ex}
        \normalsize{\today}
    \end{titlepage}
    \nopagecolor
    \pagenumbering{roman}
    \tableofcontents
    \listoffigures
    \listoftables
    \clearpage
    \chapter*{Preface}
        \addcontentsline{toc}{chapter}{Preface}
    \clearpage
    \chapter*{Acknowledgements}
        \addcontentsline{toc}{chapter}{Acknowledgements}
    \clearpage
    \pagenumbering{arabic}
    \chapter{Knots and Links}
        Knots have long been marveled as a source of art and beauty. In
        the Book of Kells, a Celtic work containing the four gospels of the
        New Testament,
        \par\hfill\par
        As a mathematical discipline, the origins of knot theory
        can date back to the $18^{\small\textrm{th}}$ and
        $19^{\small\textrm{th}}$ centuries with semi-rigorous treaties of
        the subject being formed by Vandermonde in 1771 \cite{Vanermonde1771}
        and Gauss (1860). Serious investigations into the field are often first
        attributed to Peter Tait who established some of the earliest
        tabulations of knots and put forward several conjectures in a treatise
        published in 1885. Tait's motivation was primarily the conjecture of
        Lord Kelvin who believed that chemical properties of matter could be
        explained by atoms being \textit{knotted}, in some sense.
        J.J. Thompson expanded this idea and developed some of the earlier
        mathematical properties of knots, only to abandon the hypothesis
        altogether with his discovery of the electron. An so vortex theory
        died, but knot theory lives on.
        \section{Basic Definitions and Invariants}
        \section{The Jones Polynomial and Kauffman Bracket}
        \section{Khovanov Homology}
        \section{Knot Recognition}
    \chapter{Contact Topology}
    \chapter{Algorithms and Computation}
    \chapter{Conjectures on Legendrian Simple Knots}
    \chapter{Numerical Results}

    % Print bibliographies from all texts.
    \clearpage
    \nocite{*}
    \bibliographystyle{annotate}
    \bibliography{bib.bib}
\end{document}
