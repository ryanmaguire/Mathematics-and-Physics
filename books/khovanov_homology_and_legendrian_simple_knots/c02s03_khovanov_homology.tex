\section{Khovanov Homology}
    The next invariant up is the one we're most concerned with.
    \textbf{Khovanov homology} is a link invariant that generalizes the
    Jones polynomial by \textit{categorifying} it. The concept was invented by
    Mikhail Khovanov in \cite{Khovanov1999CatJonesPoly} but we will be
    following the styling of \cite{BarNatanKhovanovJones} more closely.
    \par\hfill\par
    We start with the Kauffman relations, but instead of attaching polynomials
    to our diagram we use chain complexes. The \textit{Khovanov} bracket of a
    link diagram $L$ is given recursively via smoothings. We define:
    \begin{align}
        [[\emptyset]]&=0\rightarrow\mathbb{Z}\rightarrow{0}\\
        [[\mathbb{S}^{1}\sqcup{L}]]&=V\otimes[[L]]\\
        [[L]]&=\mathcal{F}
        \big(
            0\rightarrow[[L_{n,\,0}]]\rightarrow[[L_{n,\,1}]]\{1\}\rightarrow{0}
        \big)
    \end{align}
    where $V$ is a graded vector space,
    or more generally a free $\mathbb{Z}[q]$
    module of rank two with generators of degree $+1$ and $-1$
    \cite[p.~362]{Khovanov1999CatJonesPoly}, and where
    $\mathcal{F}$ is the \textit{flatten} operation that converts a
    double chain complex into a single chain complex by taking direct sums
    of the components along the diagonals \cite[p.~338]{BarNatanKhovanovJones}.
    The $\{1\}$ notation indicates the degree shift operation which shifts the
    grading by $1$. Explicitly, if we have decomposed $V$ into
    $V=\bigoplus_{n}V_{n}$, then $V\{1\}_{n}=V_{n-1}$ so that the graded
    dimension of $V\{1\}$ is $q$ times the graded dimension of $V$.
    \par\hfill\par
    The last ingrediant is the differential between $[[L_{n,\,0}]]$ and
    $[[L_{n,\,1}]]\{1\}$. This is also defined pictorially. We use the cube of
    resolutions of the knot or link diagram, for example
    Fig.~\ref{fig:trefoil_knot_cube_of_resolutions}. Recalling our previous
    binary notation, every complete smoothing corresponds uniquely to a string
    of length $N$ with entries in $\{\,0,\,1\,\}$ where $N$ is the number of
    crossings in the diagram. If two strings differ in only one place then
    there is an edge between them in the cube of resolutions.
    The edge describes a cobordism
    (a pair of pants) that either fuses two cycles into one or splits a cycle
    into two. Re-examining Fig.~\ref{fig:trefoil_knot_cube_of_resolutions} is
    strongly encouraged here. Fusing amounts to a homomorphism between
    $V\otimes{V}$ and $V$, whereas splitting needs a map from
    $V$ to $V\otimes{V}$. For generators $v_{-},\,v_{+}$ of $V$ we define
    $m:V\otimes{V}\rightarrow{V}$ to be the linear function induced by:
    \begin{align}
        m(v_{-}\otimes{v}_{-})&=\mathbf{0}\\
        m(v_{-}\otimes{v}_{+})&=v_{-}\\
        m(v_{+}\otimes{v}_{-})&=v_{-}\\
        m(v_{+}\otimes{v}_{+})&=v_{+}
    \end{align}
    and define $\Delta:V\rightarrow{V}\otimes{V}$ via:
    \begin{align}
            \Delta(v_{-})&=v_{-}\otimes{v}_{-}\\
            \Delta(v_{+})&=v_{+}\otimes{v}_{-}+v_{-}\otimes{v}_{+}
    \end{align}
    See \cite[p.~343]{BarNatanKhovanovJones} for further details. The
    differential is defined by an alternating sum of such maps over edges whose
    strings yield the same Hamming weight (sum of the entries in the string).
    \par\hfill\par
    By construction the Khovanov homology of a link contains the Jones
    polynomial. The Poincar\'{e} polynomial of the homology is defined by
    throwing away the torsion bits. We write:
    \begin{equation}
        \label{eqn:khovanov_polynomial}
        Kh_{L}(q,t)=
        \sum_{r,\ell}t^{r}q^{\ell}\textrm{dim}\big(Kh_{\ell}^{r}(L)\big)
    \end{equation}
    where we are writing the $r^{\small\textrm{th}}$ homology group
    $Kh^{r}(L)$ as the direct sum of homogeneous components
    $Kh_{\ell}^{r}(L)$ \cite{KatlasKhoHo}.
    The Jones polynomial is recovered as the graded Euler characteristic, or
    more explicitly:
    \begin{equation}
        J_{L}(q)=Kh_{L}(q,-1)
    \end{equation}
    This invariant is quite strong, but computationally expensive. This cost
    will be our greatest detriment in our work so when possible we first
    attempt to use the Jones polynomial. If this fails at differentiating
    two knots or links we then try to compute Khovanov homology or the
    \textbf{Khovanov polynomial}, which is the Poincar\'{e} polynomial of
    Khovanov homology (Eqn.~\ref{eqn:khovanov_polynomial}).
    \par\hfill\par
    The invariant is indeed stronger than the Jones polynomial and can
    distinguish quite a few more knots, though it is not a perfect
    invariant \cite{Watson2007KnotsWI}. That is, there are
    distinct knots with the same Khovanov homology, but it is a powerful
    invariant and is capable of detecting certain knot types.
    \begin{theorem}[Kronheimer and Mrowka, 2011]
        If a knot $K$ has the same Khovanov homology as the unknot, then $K$
        is equivalent to the unknot.
    \end{theorem}
    \begin{proof}
        \cite{KronheimerMrowka2011KhovanovUnknot}.
    \end{proof}
    As discussed, something as simple as determining if a diagram represents
    the unknot can require super exponential algorithms if we rely on
    Reidemeister moves alone. The definition of Khovanov homology produces a
    straight-forward algorithm that runs in
    $O(\textrm{poly}(N)\cdot{2}^{N})$ time \cite{BarNatanKhovanovJones},
    albeit with substantially more overhead than the Jones polynomial.
    Combining the previous theorem gives a considerably more efficient
    algorithm of unknot detection than randomly looping through
    Reidemeister combinations.
    \par\hfill\par
    Since the Khovanov polynomial is a generalization of
    the Jones polynomial it has been conjectured that if a
    knot has the same Jones polynomial as the unknot, then that knot is
    equivalent to the unknot. At the time of this writing it has not been
    proven, but there is evidence for and against the claim.
    Thistlewaite found links with the same Jones polynomial as the unlink
    \cite{Thistlethwaite2001LINKSWT}, and there is a 3-crossing virtual
    knot that has the same Jones polynomial as the unknot. For the claim,
    all knots of up to 24 crossings are either the unknot, or have a
    Jones polynomial different from the unknot
    \cite{VerificationUnknotJonesConjUpTo24}.
    \par\hfill\par
    The numerical results in this work are motivated by the following two
    theorems.
    \begin{theorem}[Baldwin and Sivek, 2022]
        If a knot $K$ has the same Khovanov homology as either of the
        trefoils, then $K$ is equivalent to one of them.
    \end{theorem}
    \begin{theorem}[Baldwin, Dowlin, Levine, Lidman, and Sazdanovic, 2021]
        If a knot $K$ has the same Khovanov homology as the figure-eight
        knot, then $K$ is equivalent to it.
    \end{theorem}
    See \cite{BaldwinSivekKhovanovTrefoils} and
    \cite{BaldwinDowlinKhovanovFigureEight}, respectively.
    \par\hfill\par
    The recursive nature of Khovanov homology provides a fairly
    straight-forward means of implementing an algorithm. This has been done in
    \cite{sage}. Like the Jones polynomial, the recursive algorithm suffers
    from exponential-in-space issues and is not usable for large knots.
    For 14 crossing knots it requires over
    100GB of data and days of computation for a single knot. 17 crossings
    knots are unapproachable by such methods. Fortunately modification of the
    symbolic calculus trick for the Jones polynomial can be performed for
    Khovanov homology \cite{BarNatan2006FASTKH} and with this we can
    remarkably compute with 60 crossing knots in about a day on modern hardware.
