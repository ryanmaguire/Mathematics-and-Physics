\section{Khovanov Homology}
    The Khovanov homology of a link is a powerful, if computationally
    expensive%
    \footnote{%
        The na\"{i}ve algorithm is exponential in the number of
        crossings. Improvements by Bar-Natan \cite{BarNatanFastKhoHo}
        have sped up computations but no polynomial-time algorithm is
        known at the time of this writing.
    },
    invariant first introduced by Mikhail Khovanov
    \cite{Khovanov1999CatJonesPoly} (See also \cite{barnatan2002khovanov}
    for an excellent introduction). It is closely related to the Jones
    polynomial, but able to distinguish many more knots and links. The
    homology groups $KH^{r}(L)$ of a link (or knot) $L$ are the direct sum
    of homogeneous components $KH_{\ell}^{r}(L)$ and the
    \textit{Khovanov Polynomial} (See \cite{KatlasKhoHo}) is given by:
    \begin{equation}
        Kh(L)(q,t)=
        \sum_{r,\ell}t^{r}q^{\ell}\textrm{dim}\big(KH_{\ell}^{r}(L)\big)
    \end{equation}
    The Jones polynomial of $L$ is recovered via:
    \begin{equation}
        J(L)(q)=Kh(L)(q,-1)
    \end{equation}
    Khovanov homology is not a perfect invariant
    \cite{Watson2007KnotsWI}. That is, there are
    distinct knots with the same Khovanov homology, but it is a powerful
    invariant and is capable of detecting certain knot types.
    \begin{theorem}[Kronheimer and Mrowka, 2001]
        If a knot $K$ has the same Khovanov homology as the unknot, then $K$
        is equivalent to the unknot.
    \end{theorem}
    The unknotting problem asks one to determine if a given knot diagram is
    equivalent to the unknot. Khovanov homology is a powerful enough tool
    to accomplish this task. The Khovanov polynomial is a generalization of
    the Jones polynomial and it has been conjectured that if a
    knot has the same Jones polynomial as the unknot, then that knot is
    equivalent to the unknot. At the time of this writing it has not been
    proven, but there is evidence for and against the claim.
    Thistlewaite found links with the same Jones polynomial as the unlink
    \cite{Thistlethwaite2001LINKSWT}, and there is a 3-crossing virtual
    knot that has the same Jones polynomial as the unknot. For the claim,
    all knots of up to 24 crossings are either the unknot, or have a
    Jones polynomial different from the unknot
    \cite{VerificationUnknotJonesConjUpTo24}.
    \begin{theorem}[Baldwin and Sivek, 2022]
        If a knot $K$ has the same Khovanov homology as either of the
        trefoils, then $K$ is equivalent to one of them.
    \end{theorem}
    \begin{theorem}[Baldwin, Dowlin, Levine, Lidman, and Sazdanovic, 2021]
        If a knot $K$ has the same Khovanov homology as the figure-eight
        knot, then $K$ is equivalent to it.
    \end{theorem}
    See \cite{BaldwinSivekKhovanovTrefoils} and
    \cite{BaldwinDowlinKhovanovFigureEight}, respectively.
