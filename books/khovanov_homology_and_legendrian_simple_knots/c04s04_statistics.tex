\section{Statistics of Knots Invariants}
    Using the \texttt{regina} and \texttt{JavaKh} libraries, the Jones
    polynomials of all knots with less than or equal to 19 crossings was
    tabulated, and the Khovanov polynomials of all knots up to 17 crossings
    have been computed.%
    \footnote{%
        This data is publicly available. See
        \cite{JonesData} and \cite{KhovanovData}.
    }
    This allows us to measure how common it is for a knot
    to have a unique Jones or Khovanov polynomial when compared with other
    low crossing knots. Unsurprisingly, the Jones polynomial does not
    distinguish nearly as many knots as the Khovanov polynomial.%
    \footnote{%
        The Khovanov polynomial throws away the torsion component of
        Khovanov homology. It would be interesting to compare how much
        stronger the \textit{full} Khovanov homology is.
    }
    The statistics for these knot invariants are shown in
    Tabs.~\ref{tab:jones_stat} and \ref{tab:kho_stat}, the key for the
    tables is given in Tab.~\ref{tab:key}.
    \par\hfill\par
    Computational limits (temporarily) prohibit us from expanding
    Tab.~\ref{tab:kho_stat} to 19 crossings, but this is being worked on.
    After examining the data one sees that the Khovanov polynomial
    distinguishes about $32\%$ of knots with up to 17 crossings. Put
    another way, about 7 out of 10 knots share their Khovanov polynomial
    with another knot in the list.
    \par\hfill\par
    An interesting observation is that the strengths of these invariants appear
    to be monotonically decreasing with crossing number. That is, if
    $K_{n}$ denotes the number of prime knots with crossing number bounded
    by $n$, if $J_{n}$ denotes the number of prime knots of up to $n$ crossings
    that are uniquely distinguished by their Jones polynomial among all other
    prime knots of up to $n$ crossings, we see that
    $J_{n}/K_{n}$ decreases monotonically. The \textit{rate} of its decrease
    seems to be slowing, leading one to ask \textit{does this limit exist}?
    \textit{Is it zero}?
    \par\hfill\par
    A similar observation is made with the Khovanov polynomial. If we label
    $Kh_{n}$ in a similar manner to $J_{n}$, we see that
    $Kh_{n}/K_{n}$ is decreasing monotonically. It would be interesting to
    know what the limit of this ratio is.
    \par\hfill\par
    Lastly in Tab.~\ref{tab:homflypt_stat} we find the statistics for the
    HOMFLY-PT polynomial. This was generated using the algorithms implemented
    in \cite{regina}. Something very interesting to note is the very impressive
    strength of the invariant, over $60\%$ of prime knots with up to 19
    crossings can be uniquely identified by their HOMFLY-PT polynomial, far
    more than the other invariants studied. If we label $H_{n}$ for HOMFLY-PT
    in a similar manner to $J_{n}$ and $Kh_{n}$ above, we find the interesting
    fact that $H_{n}/K_{n}$ is \textit{not} monotonic. After 17 crossings this
    ratio starts to \textit{increase}. It would be very interesting to
    know the $\lim\textrm{ inf}$ and $\lim\textrm{ sup}$ of this ratio.
    \par\hfill\par
    We can use this data to play around with the statistics and ask whether or
    not the conjectures we've made in the previous sections are sound. That is,
    we can now provide a heuristic argument.
    Since there were no 18 or 19 crossing knots
    found that had the same Jones polynomial as a torus or twist knot, other
    than torus and twist knots themselves, we need only consider the statistics
    generated in these tables up to 17 crossings. We see that there is a
    $0.25$ probability of a knot being uniquely distinguished among all
    other prime knots of up to 17 crossings by its Jones polynomial alone.
    That is, a knot chosen at random has a probability of $0.75$ that it has
    the same Jones polynomial as some other knot in our table. For the Khovanov
    polynomial these numbers are $0.32$ and $0.68$, respectively. Labeling
    these $J$ and $K$, respectively for the Jones and Khovanov polynomials,
    we have $\textrm{Pr}(K\;|\;\textrm{not }J)=\frac{0.32}{0.75}=0.43$ using
    Bayes' theorem. This
    is roughly the probability that a knot that is not distinguished by its
    Jones polynomial will still be distinguished by its Khovanov polynomial.
    \par\hfill\par
    Choosing a $p$ value of $0.05$, the null hypothesis we wish to test is
    whether the Khovanov polynomial distinguishing knots with matching Jones
    polynomials happens by chance or is statistically relevant. We can compute
    $\textrm{log}_{\textrm{Pr}(K\;|\;\textrm{not }J)}(p)$, which is
    $\textrm{log}_{0.43}(0.05)=3.54$. So with $N=4$ torus knots having the
    same Jones polynomial as a a non-torus knot, but being distinguished by
    Khovanov homology, we can conclude\footnote{%
        Assuming it is reasonable to claim that knots are random beings.
    }
    that the result is significant to $p=0.05$. For twist knots, where there
    were $11$ matches, we can take $p=0.0001$. For the conjectured Legendrian
    simple knots, of which there are 14 matches, we can push $p$ even lower.
    \par\hfill\par
    The premise behind this argument, that knots are random, is a point of
    contention. Nevertheless the above argument is not entirely meaningless
    and the author believes it does motivate further study of these conjectures.
    \begin{table}
        \centering
        \begin{tabular}{| l | l |}
            \hline
            Keyword & Description\\
            \hline
            Cr     & Crossing number, largest number of crossings considered.\\
            Unique & Number of polynomials that occur for one knot.\\
            Almost & Number of polynomials that occur for two knots.\\
            Total  & Total number of distinct polynomials in list.\\
            Knots  & Total number of knots in list.\\
            FracU  & Unique / Total\\
            FracT  & Total / Knots\\
            FracK  & Unique / Knots\\
            \hline
        \end{tabular}
        \caption{Legend for Tabs.~\ref{tab:jones_stat} and \ref{tab:kho_stat}}
        \label{tab:key}
    \end{table}
    \begin{table}
        \centering
        \begin{tabular}{| r | r | r | r | r | r | r | r |}
            \hline
            Cr &  Unique  &  Almost  &   Total   &   Knots    &  FracU   &  FracT   &  FracK\\
            \hline
            03 &        1 &        0 &         1 &         1 & 1.000000 & 1.000000 & 1.000000\\
            04 &        2 &        0 &         2 &         2 & 1.000000 & 1.000000 & 1.000000\\
            05 &        4 &        0 &         4 &         4 & 1.000000 & 1.000000 & 1.000000\\
            06 &        7 &        0 &         7 &         7 & 1.000000 & 1.000000 & 1.000000\\
            07 &       14 &        0 &        14 &        14 & 1.000000 & 1.000000 & 1.000000\\
            08 &       35 &        0 &        35 &        35 & 1.000000 & 1.000000 & 1.000000\\
            09 &       84 &        0 &        84 &        84 & 1.000000 & 1.000000 & 1.000000\\
            10 &      223 &       13 &       236 &       249 & 0.944915 & 0.947791 & 0.895582\\
            11 &      626 &       77 &       710 &       801 & 0.881690 & 0.886392 & 0.781523\\
            12 &     1981 &      345 &      2420 &      2977 & 0.818595 & 0.812899 & 0.665435\\
            13 &     6855 &     1695 &      9287 &     12965 & 0.738129 & 0.716313 & 0.528731\\
            14 &    25271 &     7439 &     37578 &     59937 & 0.672495 & 0.626958 & 0.421626\\
            15 &   105246 &    35371 &    170363 &    313230 & 0.617775 & 0.543891 & 0.336002\\
            16 &   487774 &   173677 &    829284 &   1701935 & 0.588187 & 0.487260 & 0.286600\\
            17 &  2498968 &   894450 &   4342890 &   9755328 & 0.575416 & 0.445181 & 0.256164\\
            18 & 13817237 &  4863074 &  24116048 &  58021794 & 0.572948 & 0.415638 & 0.238139\\
            19 & 82712788 & 27409120 & 141439472 & 352152252 & 0.584793 & 0.401643 & 0.234878\\
            \hline
        \end{tabular}
        \caption{Statistics for the Jones Polynomial}
        \label{tab:jones_stat}
    \end{table}
    \begin{table}
        \centering
        \begin{tabular}{| r | r | r | r | r | r | r | r |}
            \hline
            Cr & Unique  & Almost  &  Total  &  Knots  &  FracU   &  FracT   &  FracK\\
            \hline
            03 &       1 &       0 &       1 &       1 & 1.000000 & 1.000000 & 1.000000\\
            04 &       2 &       0 &       2 &       2 & 1.000000 & 1.000000 & 1.000000\\
            05 &       4 &       0 &       4 &       4 & 1.000000 & 1.000000 & 1.000000\\
            06 &       7 &       0 &       7 &       7 & 1.000000 & 1.000000 & 1.000000\\
            07 &      14 &       0 &      14 &      14 & 1.000000 & 1.000000 & 1.000000\\
            08 &      35 &       0 &      35 &      35 & 1.000000 & 1.000000 & 1.000000\\
            09 &      84 &       0 &      84 &      84 & 1.000000 & 1.000000 & 1.000000\\
            10 &     225 &      12 &     237 &     249 & 0.949367 & 0.951807 & 0.903614\\
            11 &     641 &      71 &     718 &     801 & 0.892758 & 0.896380 & 0.800250\\
            12 &    2051 &     326 &    2462 &    2977 & 0.833063 & 0.827007 & 0.688949\\
            13 &    7223 &    1636 &    9539 &   12965 & 0.757207 & 0.735750 & 0.557115\\
            14 &   27317 &    7441 &   39222 &   59937 & 0.696471 & 0.654387 & 0.455762\\
            15 &  118534 &   36867 &  182598 &  313230 & 0.649153 & 0.582952 & 0.378425\\
            16 &  578928 &  187639 &  919835 & 1701935 & 0.629382 & 0.540464 & 0.340159\\
            17 & 3167028 & 1001101 & 5033403 & 9755328 & 0.629202 & 0.515965 & 0.324646\\
            \hline
        \end{tabular}
        \caption{Statistics for the Khovanov Polynomial}
        \label{tab:kho_stat}
    \end{table}
    \begin{table}
        \centering
        \begin{tabular}{| r | r | r | r | r | r | r | r |}
            \hline
            Cr & Unique  & Almost  &  Total  &  Knots  &  FracU   &  FracT   &  FracK\\
            \hline
            03 &         1 &        0 &         1 &         1 & 1.000000 & 1.000000 & 1.000000\\
            04 &         2 &        0 &         2 &         2 & 1.000000 & 1.000000 & 1.000000\\
            05 &         4 &        0 &         4 &         4 & 1.000000 & 1.000000 & 1.000000\\
            06 &         7 &        0 &         7 &         7 & 1.000000 & 1.000000 & 1.000000\\
            07 &        14 &        0 &        14 &        14 & 1.000000 & 1.000000 & 1.000000\\
            08 &        35 &        0 &        35 &        35 & 1.000000 & 1.000000 & 1.000000\\
            09 &        84 &        0 &        84 &        84 & 1.000000 & 1.000000 & 1.000000\\
            10 &       241 &        4 &       245 &       249 & 0.983673 & 0.983936 & 0.967871\\
            11 &       730 &       34 &       765 &       801 & 0.954248 & 0.955056 & 0.911361\\
            12 &      2494 &      210 &      2724 &      2977 & 0.915565 & 0.915015 & 0.837756\\
            13 &      9475 &     1302 &     11044 &     12965 & 0.857932 & 0.851832 & 0.730814\\
            14 &     39401 &     7170 &     48329 &     59937 & 0.815266 & 0.806330 & 0.657374\\
            15 &    186799 &    38833 &    238614 &    313230 & 0.782850 & 0.761785 & 0.596364\\
            16 &    979987 &   209669 &   1266261 &   1701935 & 0.773922 & 0.744013 & 0.575808\\
            17 &   5559808 &  1157938 &   7175287 &   9755328 & 0.774855 & 0.735525 & 0.569925\\
            18 &  33722920 &  6480965 &  42857755 &  58021794 & 0.786857 & 0.738649 & 0.581211\\
            19 & 213355372 & 36387952 & 264839694 & 352152252 & 0.805602 & 0.752060 & 0.605861\\
            \hline
        \end{tabular}
        \caption{Statistics for the HOMFLY-PT Polynomial}
        \label{tab:homflypt_stat}
    \end{table}
