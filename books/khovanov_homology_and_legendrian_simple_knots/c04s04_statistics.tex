\section{Statistics of Knots Invariants}
    Using the \texttt{regina} and \texttt{JavaKh} libraries, the Jones
    polynomials of all knots with less than or equal to 19 crossings was
    tabulated, and the Khovanov polynomials of all knots up to 17 crossings
    have been computed.%
    \footnote{%
        This data is publicly available. See
        \cite{JonesData} and \cite{KhovanovData}.
    }
    This allows us to measure how common it is for a knot
    to have a unique Jones or Khovanov polynomial when compared with other
    low crossing knots. Unsurprisingly, the Jones polynomial does not
    distinguish nearly as many knots as the Khovanov polynomial.%
    \footnote{%
        The Khovanov polynomial throws away the torsion component of
        Khovanov homology. It would be interesting to compare how much
        stronger the \textit{full} Khovanov homology is.
    }
    The statistics for these knot invariants are shown in
    Tabs.~\ref{tab:jones_stat} and \ref{tab:kho_stat}, the key for the
    tables is given in Tab.~\ref{tab:key}.
    \par\hfill\par
    Computational limits (temporarily) prohibit us from expanding
    Tab.~\ref{tab:kho_stat} to 19 crossings, but this is being worked on.
    After examining the data one sees that the Khovanov polynomial
    distinguishes about $32\%$ of knots with up to 17 crossings. Put
    another way, about 7 out of 10 knots share their Khovanov polynomial
    with another knot in the list.
    \par\hfill\par
    An interesting observation is that the strengths of these invariants appear
    to be monotonically decreasing with crossing number. That is, if
    $K_{n}$ denotes the number of prime knots with crossing number bounded
    by $n$, if $J_{n}$ denotes the number of prime knots of up to $n$ crossings
    that are uniquely distinguished by their Jones polynomial among all other
    prime knots of up to $n$ crossings, we see that
    $J_{n}/K_{n}$ decreases monotonically. The \textit{rate} of its decrease
    seems to be slowing, leading one to ask \textit{does this limit exist}?
    \textit{Is it zero}?
    \par\hfill\par
    A similar observation is made with the Khovanov polynomial. If we label
    $Kh_{n}$ in a similar manner to $J_{n}$, we see that
    $Kh_{n}/K_{n}$ is decreasing monotonically. It would be interesting to
    know what the limit of this ratio is.
    \par\hfill\par
    Lastly in Tab.~\ref{tab:homflypt_stat} we find the statistics for the
    HOMFLY-PT polynomial. This was generated using the algorithms implemented
    in \cite{regina}. Something very interesting to note is the very impressive
    strength of the invariant, over $60\%$ of prime knots with up to 19
    crossings can be uniquely identified by their HOMFLY-PT polynomial, far
    more than the other invariants studied. If we label $H_{n}$ for HOMFLY-PT
    in a similar manner to $J_{n}$ and $Kh_{n}$ above, we find the interesting
    fact that $H_{n}/K_{n}$ is \textit{not} monotonic. After 17 crossings this
    ratio starts to \textit{increase}. It would be very interesting to
    know the $\lim\textrm{ inf}$ and $\lim\textrm{ sup}$ of this ratio.
    \par\hfill\par
    We can use this data to play around with the statistics and ask whether or
    not the conjectures we've made in the previous sections are sound. That is,
    we can now provide a heuristic argument.
    Since there were no 18 or 19 crossing knots
    found that had the same Jones polynomial as a torus or twist knot, other
    than torus and twist knots themselves, we need only consider the statistics
    generated in these tables up to 17 crossings.
    \par\hfill\par
    A first approach is to say that the probability of a prime knot having a
    unique Khovanov polynomial among all other prime
    knots with up to 17 crossings is 0.32 (Tab.~\ref{tab:kho_stat}).
    Ignoring mirrors, there are 16 twist knots and 14 torus knots with
    less than or equal to 17 crossings. For twist knots these are the knots
    $K_{m}$ for $m=0,\,1,\,\cdots,\,15$, and for torus knots these are given
    by the co-prime pairs
    $(2, 3)$, $(2, 5)$, $(2, 7)$, $(2, 9)$, $(2, 11)$, $(2, 13)$,
    $(2, 15)$, $(2, 17)$, $(3, 4)$, $(3, 5)$, $(3, 7)$, $(3, 8)$, $(4, 5)$,
    and the unknot.
    The (na\"{i}ve) probability of all of these knots having unique
    Khovanov polynomials is $0.32^{16+14-1}\approx{4.5}\times{10}^{-15}$, where
    the $-1$ in the exponent comes from double counting the unknot, once as
    a twist knot and once as a torus knot.
    For twist knots alone this is $0.32^{16}\approx{1.2}\times{10}^{-8}$ and
    for torus knots this is $0.32^{14}\approx{1.2}\times{10}^{-7}$.
    \par\hfill\par
    These are ridiculously small probabilities, which is good for our
    conjecture,
    but this assumes these knots are more-or-less \textit{random}.
    Let us use their Jones polynomials to demonstrate that they more-or-less
    are. The Jones polynomial
    has a $0.26$ probability of distinguishing a knot in our list of prime knots
    up to 17 crossings. Since having a unique Jones polynomial would imply
    having a unique Khovanov polynomial (recall $Kh_{K}(q,\,-1)=J_{K}(q)$),
    the set of knots
    that are distinguished by their Jones polynomial is a subset of the set of
    knots distinguished by the Khovanov polynomial. If we label
    \texttt{KH} as the event that a knot is determined by its Khovanov
    polynomial (when considered among prime knots with 17 crossings or less),
    and similarly label \texttt{J} for the Jones polynomial, the conditional
    probability is:
    \begin{equation}
        P(\textrm{not }\texttt{J}\;|\;\texttt{KH})
        =\frac{P(\texttt{KH})-P(\texttt{J})}{P(\texttt{KH})}
        \approx{0.21}
    \end{equation}
    So with a $0.21$ probability a knot that is known to be distinguished by
    its Khovanov polynomial will not be distinguished by its Jones polynomial.
    We have shown by brute force methods that the torus and twist knots are
    part of the Khovanov-distinguished family for prime knots up to 17
    crossings, so we can apply this probability. There were 4 torus knots that
    are not distinguished by their Jones polynomial and 11 such twist knots.
    When accounting for mirrors, this gives a
    $0.15$ probability of a torus knot being distinguished by the Khovanov
    polynomial and not the Jones polynomial, and a $0.36$ probability for twist
    knots. Roughly, this means torus knots are more likely to be distinguished
    by the Jones polynomial than the average Khovanov-distinguished knot,
    and the twist knots are less likely. The total probability for the union
    of these two families is $0.26$, which is close to the general
    probability of $0.21$. It is then fair to say that these
    families are behaving \textit{randomly}, as far as their
    Jones detection rates are concerned.
    \par\hfill\par
    We can also look at the conditional probability of finding a knot that is
    distinguished by its Khovanov polynomial given that we know it is not
    distinguished by its Jones polynomial. First, a quick review of probability.
    Given events $A,B\subseteq{X}$, for some probability space $X$, if $P(B)$
    is non-zero, then:
    \begin{align}
        P(B)&=P\big((A\cap{B})\cup(A^{C}\cap{B})\big)\\
        &=P(A\cap{B})+P(A^{C}\cap{B})\\
        \Rightarrow
        1&=\frac{P(A\cap{B})}{P(B)}+\frac{P(A^{C}\cap{B})}{P(B)}\\
        &=P(A\;|\;B)+P(A^{C}\;|\;B)
    \end{align}
    where $A^{C}=X\setminus{A}$, the complement of $A$ in $X$, and
    where $P(A\;|\;B)$ denotes the conditional probability. Note that
    $P(A^{C})$ is the same thing as $P(\textrm{not }A)$.
    Rearranging we have:
    \begin{equation}
        P(A\;|\;B)=1-P(A^{C}\;|\;B)
    \end{equation}
    a rather standard and intuitive result, we will now use this in our
    following argument. We seek the probability
    $P(\texttt{KH}\;|\;\textrm{not }\texttt{J})$.
    That is, the probability a knot will
    be distinguished by its Khovanov polynomial given that we know it is
    not distinguished by its Jones polynomial. Combining our previous derivation
    with Bayes' theorem, we obtain:
    \begin{align}
        P(\texttt{KH}\;|\;\textrm{not }\texttt{J})
        &=P(\textrm{not }\texttt{J}\;|\;\texttt{KH})
        \frac{P(\texttt{KH})}{P(\textrm{not }\texttt{J})}\\
        &=\big(1-P(\texttt{J}\;|\;\texttt{KH})\big)
            \frac{P(\texttt{KH})}{P(\textrm{not }\texttt{J})}\\
        &=\big(
            1-\frac{P(\texttt{KH}\;|\;\textrm{J})P(\texttt{J})}{P(\texttt{KH})}
        \big)
        \frac{P(\texttt{KH})}{P(\textrm{not }\texttt{J})}\\
        &=\big(
            1-\frac{P(\texttt{J})}{P(\texttt{KH})}
        \big)
        \frac{P(\texttt{KH})}{P(\textrm{not }\texttt{J})}\\
        &=\big(
            \frac{P(\texttt{KH})-P(\texttt{J})}{P(\texttt{KH})}
        \big)
        \frac{P(\texttt{KH})}{P(\textrm{not }\texttt{J})}\\
        &=\frac{P(\texttt{KH})-P(\texttt{J})}{P(\textrm{not }\texttt{J})}\\
        &=\frac{P(\texttt{KH})-P(\texttt{J})}{1-P(\texttt{J})}\\
        &=0.092
    \end{align}
    We have used the fact that $P(\texttt{KH}\;|\;\texttt{J})=1$ to obtain our
    result.
    To be on the conservative side we can round this up (smaller probabilities
    make our conjecture stronger) to $10\%$. So with roughly a $10\%$
    probability a knot that is not Jones-distinguished will still be
    distinguished by its Khovanov polynomial. We found 11 twist knots and 4
    torus knots that are not Jones-distinguished, but are still
    Khovanov-distinguished. Hence for the torus knots we may choose a $p$
    value of $0.001$, and for the twist knots we can go much lower, indicating
    that it is likely a statistically significant result.
    \par\hfill\par
    Now one may say that \textit{most} of the prime knots with 17 or less
    crossings happen to have 17 crossings, whereas most of the twist knots
    we've examined do not.
    A similar claim can be made for the torus knots. We can reformulate
    our probability argument as follows to account for this. Let
    $p(k)$ be the fraction of prime knots of up to $k$ crossings that are
    distinguished by their Khovanov polynomial. Let
    $Cr(K)$ denote the crossing number of $K$. The product
    $\prod_{m}p\big(Cr(K_{m})\big)$ over all twist knots with up to 17 crossings
    then serves as a better probability that all should be
    Khovanov-distinguished. Using our tables this number is
    $2.8\times{10}^{-5}$, which is still very small. For the torus
    knots we obtain $8.4\times{10}^{-6}$, a tiny probability.
    \par\hfill\par
    Lastly, one may argue that several of the torus and twist knots are known
    to be Khovanov-distinguished among all possible knots, like the unknot,
    and so should not be included in these probabilities.
    Our previous argument handles this since these knots have
    crossing number no greater than 5, and $p(k)=1$ for $k\leq{9}$.
    \par\hfill\par
    The premise behind these arguments, that knots are random, is a point of
    contention. Nevertheless the above argument is not entirely meaningless
    and the author believes it does motivate further study of these conjectures.
    \begin{table}
        \centering
        \begin{tabular}{| l | l |}
            \hline
            Keyword & Description\\
            \hline
            Cr     & Crossing number, largest number of crossings considered.\\
            Unique & Number of polynomials that occur for one knot.\\
            Almost & Number of polynomials that occur for two knots.\\
            Total  & Total number of distinct polynomials in list.\\
            Knots  & Total number of knots in list.\\
            FracU  & Unique / Total\\
            FracT  & Total / Knots\\
            FracK  & Unique / Knots\\
            \hline
        \end{tabular}
        \caption{Legend for Tabs.~\ref{tab:jones_stat} and \ref{tab:kho_stat}}
        \label{tab:key}
    \end{table}
    \begin{table}
        \centering
        \begin{tabular}{| r | r | r | r | r | r | r | r |}
            \hline
            Cr &  Unique  &  Almost  &   Total   &   Knots    &  FracU   &  FracT   &  FracK\\
            \hline
            03 &        1 &        0 &         1 &         1 & 1.000000 & 1.000000 & 1.000000\\
            04 &        2 &        0 &         2 &         2 & 1.000000 & 1.000000 & 1.000000\\
            05 &        4 &        0 &         4 &         4 & 1.000000 & 1.000000 & 1.000000\\
            06 &        7 &        0 &         7 &         7 & 1.000000 & 1.000000 & 1.000000\\
            07 &       14 &        0 &        14 &        14 & 1.000000 & 1.000000 & 1.000000\\
            08 &       35 &        0 &        35 &        35 & 1.000000 & 1.000000 & 1.000000\\
            09 &       84 &        0 &        84 &        84 & 1.000000 & 1.000000 & 1.000000\\
            10 &      223 &       13 &       236 &       249 & 0.944915 & 0.947791 & 0.895582\\
            11 &      626 &       77 &       710 &       801 & 0.881690 & 0.886392 & 0.781523\\
            12 &     1981 &      345 &      2420 &      2977 & 0.818595 & 0.812899 & 0.665435\\
            13 &     6855 &     1695 &      9287 &     12965 & 0.738129 & 0.716313 & 0.528731\\
            14 &    25271 &     7439 &     37578 &     59937 & 0.672495 & 0.626958 & 0.421626\\
            15 &   105246 &    35371 &    170363 &    313230 & 0.617775 & 0.543891 & 0.336002\\
            16 &   487774 &   173677 &    829284 &   1701935 & 0.588187 & 0.487260 & 0.286600\\
            17 &  2498968 &   894450 &   4342890 &   9755328 & 0.575416 & 0.445181 & 0.256164\\
            18 & 13817237 &  4863074 &  24116048 &  58021794 & 0.572948 & 0.415638 & 0.238139\\
            19 & 82712788 & 27409120 & 141439472 & 352152252 & 0.584793 & 0.401643 & 0.234878\\
            \hline
        \end{tabular}
        \caption{Statistics for the Jones Polynomial}
        \label{tab:jones_stat}
    \end{table}
    \begin{table}
        \centering
        \begin{tabular}{| r | r | r | r | r | r | r | r |}
            \hline
            Cr & Unique  & Almost  &  Total  &  Knots  &  FracU   &  FracT   &  FracK\\
            \hline
            03 &       1 &       0 &       1 &       1 & 1.000000 & 1.000000 & 1.000000\\
            04 &       2 &       0 &       2 &       2 & 1.000000 & 1.000000 & 1.000000\\
            05 &       4 &       0 &       4 &       4 & 1.000000 & 1.000000 & 1.000000\\
            06 &       7 &       0 &       7 &       7 & 1.000000 & 1.000000 & 1.000000\\
            07 &      14 &       0 &      14 &      14 & 1.000000 & 1.000000 & 1.000000\\
            08 &      35 &       0 &      35 &      35 & 1.000000 & 1.000000 & 1.000000\\
            09 &      84 &       0 &      84 &      84 & 1.000000 & 1.000000 & 1.000000\\
            10 &     225 &      12 &     237 &     249 & 0.949367 & 0.951807 & 0.903614\\
            11 &     641 &      71 &     718 &     801 & 0.892758 & 0.896380 & 0.800250\\
            12 &    2051 &     326 &    2462 &    2977 & 0.833063 & 0.827007 & 0.688949\\
            13 &    7223 &    1636 &    9539 &   12965 & 0.757207 & 0.735750 & 0.557115\\
            14 &   27317 &    7441 &   39222 &   59937 & 0.696471 & 0.654387 & 0.455762\\
            15 &  118534 &   36867 &  182598 &  313230 & 0.649153 & 0.582952 & 0.378425\\
            16 &  578928 &  187639 &  919835 & 1701935 & 0.629382 & 0.540464 & 0.340159\\
            17 & 3167028 & 1001101 & 5033403 & 9755328 & 0.629202 & 0.515965 & 0.324646\\
            \hline
        \end{tabular}
        \caption{Statistics for the Khovanov Polynomial}
        \label{tab:kho_stat}
    \end{table}
    \begin{table}
        \centering
        \begin{tabular}{| r | r | r | r | r | r | r | r |}
            \hline
            Cr & Unique  & Almost  &  Total  &  Knots  &  FracU   &  FracT   &  FracK\\
            \hline
            03 &         1 &        0 &         1 &         1 & 1.000000 & 1.000000 & 1.000000\\
            04 &         2 &        0 &         2 &         2 & 1.000000 & 1.000000 & 1.000000\\
            05 &         4 &        0 &         4 &         4 & 1.000000 & 1.000000 & 1.000000\\
            06 &         7 &        0 &         7 &         7 & 1.000000 & 1.000000 & 1.000000\\
            07 &        14 &        0 &        14 &        14 & 1.000000 & 1.000000 & 1.000000\\
            08 &        35 &        0 &        35 &        35 & 1.000000 & 1.000000 & 1.000000\\
            09 &        84 &        0 &        84 &        84 & 1.000000 & 1.000000 & 1.000000\\
            10 &       241 &        4 &       245 &       249 & 0.983673 & 0.983936 & 0.967871\\
            11 &       730 &       34 &       765 &       801 & 0.954248 & 0.955056 & 0.911361\\
            12 &      2494 &      210 &      2724 &      2977 & 0.915565 & 0.915015 & 0.837756\\
            13 &      9475 &     1302 &     11044 &     12965 & 0.857932 & 0.851832 & 0.730814\\
            14 &     39401 &     7170 &     48329 &     59937 & 0.815266 & 0.806330 & 0.657374\\
            15 &    186799 &    38833 &    238614 &    313230 & 0.782850 & 0.761785 & 0.596364\\
            16 &    979987 &   209669 &   1266261 &   1701935 & 0.773922 & 0.744013 & 0.575808\\
            17 &   5559808 &  1157938 &   7175287 &   9755328 & 0.774855 & 0.735525 & 0.569925\\
            18 &  33722920 &  6480965 &  42857755 &  58021794 & 0.786857 & 0.738649 & 0.581211\\
            19 & 213355372 & 36387952 & 264839694 & 352152252 & 0.805602 & 0.752060 & 0.605861\\
            \hline
        \end{tabular}
        \caption{Statistics for the HOMFLY-PT Polynomial}
        \label{tab:homflypt_stat}
    \end{table}
