\section{Transversely Simple Knots}
    The distribution of hyperplanes in the contact structure also gives rise to
    the notion of \textit{transverse} submanifolds. In particular we can define
    transverse knots.
    \begin{definition}[\textbf{Transverse Knot}]
        A transverse knot is a smooth embedding
        $\gamma:\mathbb{S}^{1}\rightarrow\mathbb{R}^{3}$ such that for all
        $t\in\mathbb{S}^{1}$ the three vectors
        $\partial{y}$, $\partial{x}+y\partial{z}$, and
        $\dot{\gamma}(t)$ span $\mathbb{R}^{3}$, where $(x,\,y,\,z)=\gamma(t)$.
    \end{definition}
    That is, at every point $p$ on the knot the velocity vector and the contact
    hyperplane at $p$ span $\mathbb{R}^{3}$. Transverse links can be similarly
    defined and these are links that are everywhere transverse to the contact
    structure. Any Legendrian link can be made transverse by a small
    perturbation in the direction normal to the given plane in the contact
    structure. We duplicate our discussion of Legendrian links and define
    transverse equivalence.
    \begin{definition}[\textbf{Transverse Ambient Isotopy}]
        A transverse ambient isotopy from a transverse knot
        $\gamma:\mathbb{S}^{1}\rightarrow\mathbb{R}^{3}$ to a transverse knot
        $\gamma':\mathbb{S}^{1}\rightarrow\mathbb{R}^{3}$ is an ambient
        isotopy $H:\mathbb{R}^{3}\times[0,\,1]\rightarrow\mathbb{R}^{3}$
        between $\gamma$ and $\gamma'$ such that the knot
        $\gamma_{t}:\mathbb{S}^{1}\rightarrow\mathbb{R}^{3}$ defined by:
        \begin{equation}
            \gamma_{t}\big((x,\,y)\big)
            =H\big((x,\,y,\,0),\,t\big)
            =H_{t}\big((x,\,y,\,0)\big)
        \end{equation}
        is transverse for all $t\in[0,\,1]$.
    \end{definition}
    \textbf{Transverse equivalence} is then taken to mean that there exists
    a transverse ambient isotopy between the two knots.
    \par\hfill\par
    As with the Legendrian setting we may define invariants for transverse
    knots and links. The \textbf{Bennequin number} of a
    transverse knot is defined by the \textit{algebraic crossing number}
    $e(K)$ and the \textit{braid index} $n(K)$. It is:
    \begin{equation}
        \beta(K)=e(K)-n(K)
    \end{equation}
    The braid index is the least integer $N\in\mathbb{N}$ such that $K$
    is representable as an element of the braid group $B_{N}$, and the
    algebraic crossing number is a synonym for the writhe of the diagram
    \cite{KawamuroAlgCrossNumberAndBraidIndex}.
    It is not an invariant of topological knots, but is an invariant under
    transverse equivalence. This is essentially the self-linking number of
    the corresponding topological knot with respect to the natural framing
    coming from the trivialization of the contact structure.
    \par\hfill\par
    Similar to Legendrian simple, we define a knot
    (or link) type to be transversely simple if all of its transverse
    representations are uniquely determined by their Bennequin number
    (See \cite{BirmanWrinkleTransversallySimpleKnots}) and by whether its
    velocity vectors point into the half space where the contact structure
    is positive or not. A paper by Etynre, Ng, and Vertesi
    \cite{EtnyreEtAlLegendrianAndTransverseTwistKnots}
    classifies when twist knots are transversely simple. In
    particular, infinitely many such knots are transversely simple giving
    us a family of knots to test conjectures with. It is also true that
    infinite families of non-Legendrian simple and non-transversely simple
    knots exist. See, for example, the works of Etnyre and Honda
    \cite{EtnyreHondaCabling} and Birman and Menasco
    \cite{BirmanMenasco2006}. More examples can be found in
    \cite{Foldvari2019legnonsimple}.
