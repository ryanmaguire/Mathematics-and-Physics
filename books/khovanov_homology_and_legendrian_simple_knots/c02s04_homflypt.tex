\section{HOMFLY-PT Polynomial}
    The last invariant we'll investigate is the HOMFLY-PT polynomial, which
    contains both the Alexander and Jones polynomials in it. Like the Jones
    polynomial it is defined via skein relations where we smooth crossings.
    Given a link diagram $L$ define $L_{n,\,0}$ to be the 0-smoothing of the
    $n^{\small\textrm{th}}$ crossing and $L_{n,\,\textrm{flip}}$ to be the
    link obtain by flipping the $n^{\small\textrm{th}}$ crossing, turning the
    over strand into the under and vice-versa. The HOMFLY-PT polynomial obeys
    the recursion relation:
    \begin{equation}
        P_{L}(\alpha,\,z)
        =\alpha^{2}P_{L_{n,\,\textrm{flip}}}-z\alpha{P}_{L_{n,\,0}}
    \end{equation}
    and we define $P_{\mathbb{S}^{1}}(\alpha,\,z)=1$, where
    $\mathbb{S}^{1}$ is taken to be an unknot. Our primary use of this
    invariant is for its strength in distinguishing knots. Of the more than
    352 million prime knots of up to 19 crossings, more than half of them are
    completely determined by their HOMFLY-PT polynomial. In trying to discern
    if two knots are equivalent this is an excellent tool, but it comes with a
    cost. As mentioned it contains both the Jones polynomial and the
    Alexander polynomial:
    \begin{align}
        \Delta_{K}(t)&=P_{K}(1,\,t-t^{-1})\\
        J_{K}(t)&=P_{K}(q^{-2},\,q-q^{-1})
    \end{align}
    and hence the computation is at least as hard as the Jones polynomial.
    Indeed, its computation is \textbf{NP-Hard}
    \cite[p.~653]{HOMFLYPTNPHard}. In
    \cite{Burton2018HOMFLFixedParameter} it is shown that the computation is
    also fixed-parameter tractable, and the algorithm presented has been
    implemented in \cite{regina}. Remarkably there is not too much of a time
    difference between computing HOMFLY-PT and computing the Jones polynomial.
    The calculation of the HOMFLY-PT polynomial of all primes knots of up to
    19 crossings can be done in a few days. This data has been tabulated by the
    author and is publicly available see \cite{HOMFLYData}.
