As we've seen khovanov homology is capable of detecting the
unknot, trefoils, and figure-eight knot and also that Khovanov homology with
coefficients in $\mathbb{Z}/2\mathbb{Z}$ is also capable of detecting
the cinquefoil knot \cite{BaldwinYingSivekCinquefoilKhovanov},
which is the $T(5,2)$ torus knot. We also saw that the Jones
polynomial is \textit{not} capable of detecting the
$T(5,2)$ torus knot since the $10_{132}$ knot yields the same
polynomial (See Tab.~\ref{table:matching_torus_knots}).
These are the smallest of the known Legendrian simple knots
and we now try to provide numerical evidence for the following.
\begin{conjecture}
    If a link type $L$ is Legendrian simple, then the Khovanov homology
    of $L$ distinguishes it. That is, if $\tilde{L}$
    is another link with the same Khovanov homology, then $\tilde{L}$ is
    equivalent to $L$.
\end{conjecture}
A tally is provided for torus knots against all prime knots up to 19 crossings.
There are many torus knots that have the same Jones polynomial as a non-torus
knot ($T(2,5)$ matches 10 and 17 crossing knots, $T(2,7)$
matches a 12 crossing knot, and $T(2,11)$ matches a 14 crossing knot)
so we cannot generalize the Jones unknot conjecture. Nevertheless, in
all cases the Khovanov homologies were different, strengthening the conjecture.
\par\hfill\par
The computations were done as follows. There are libraries in Python,
Sage, and C++ for working with knot polynomials. In particular,
Regina \cite{regina}, SnapPy \cite{SnapPy}, the Sage knot library
\cite{sage}, and two of the authors C libraries
\cite{MaguireJones}, \cite{MaguireLibtmpl}.
\par\hfill\par
We first gather the torus knots that could potentially have the same
Jones polynomial as some other knot based on the bounds of the degree.
The Jones polynomials of torus knots were computed using our previously
derived formula (see also \cite{jonesfordummyvjones}):
\begin{equation}
    \label{eqn:jones_poly_torus}%
    J(T(m,n))(q)=q^{(m-1)(n-1)/2}
        \frac{1-q^{m+1}-q^{n+1}+q^{m+n}}{1-q^{2}}
\end{equation}
The defining equation for the Kauffman bracket showed us that the degree of the
Jones polynomial of an $N$ crossing knot will be bounded by $5N+1$.
Since we are looking at knots up to $19$ crossings, we collect the
co-prime pairs $(m,\,n)$ with $1<m<n$ such that the degree
is less than $5\cdot{19}+1=96$. Since the Jones polynomial of a mirror
can be computed by substituting $q\mapsto{q}^{-1}$ we need not look at
negative values. The Khovanov polynomial makes a similar change,
$(q,\,t)\mapsto(q^{-1},\,t^{-1})$ \cite{WATSON2017915}.
\par\hfill\par
Of the aforementioned libraries regina had the best performance for the most
general knots.\footnote{%
    The sage implementation at the time of this writing uses a clever recursive
    trick that simplifies the computation via Type I and Type II moves as it
    goes. This yields a best case complexity of $O(N)$, for example when a
    twist knot is used. The worst case scenario is still $O(2^{N})$ and this
    is made worse by the fact the spatial component is also $O(2^{N})$ instead
    of $O(N)$ like the non-recursive methods.
}
This was used to compute the Jones polynomial of all prime knots of up to
19 crossings.\footnote{%
    The computation required about two and a half days.
}
If a knot outputted a polynomial that matched that of a torus
knot (using Eqn.~\ref{eqn:jones_poly_torus}) the alphabetical DT code of the
knot was saved in a text file for later examination. After computing the Jones
polynomial of all of our knots we look through the knots in our text file and
see which ones are indeed torus knots themselves. After discarding these
we end up with a total of 4 non-torus knots with the same Jones polynomial as a
torus knot (See Tab.~\ref{table:matching_torus_knots}).
Since the Khovanov polynomial contains the Jones polynomial in it
(recall $J(L)(q)=Kh(L)(q,-1)$) the only possible non-torus knots with
the same Khovanov homology as a torus knot were these 4.
\par\hfill\par
Using the algorithm from \cite{BarNatan2006FASTKH}, which is implemented in
\cite{JavaKhv2}\footnote{
    This is Nikolay Pultsin's fork of the original JavaKh library, written by
    Jeremy Greene under the supervision of Dror Bar-Natan. The original version
    fails to compile on a modern GNU/Linux system. Pultsin made some changes
    that allow the library to work with OpenJDK-17.
}
we found that these four knots with the same Jones polynomials as some
torus knot all had different Khovanov homologies. Thus, we have the
following claim:
\begin{theorem}
    If a prime knot $K$ has less than or equal to 19 crossings and has
    the Khovanov homology of a torus knot $T$,
    then $K$ is equivalent to $T$.
\end{theorem}
It should be remarked that Bar-Natan's fast Khovanov homology algorithm is
required for this computation to be performed. The na\"{i}ve algorithm that
works straight from the definition is implemented in the sage knot library
\cite{sage} and for a single 14 crossing knot requires over 100 GB of memory
and days of computations. 17 crossing knots are not feasible with such an
algorithm.
\par\hfill\par
A similar search through the twist knots yielded more results.
From our previous computation we know that the Jones polynomial of the twist
knots satisfy:
\begin{equation}
    J(m_{n})(q)=
    \begin{cases}
        (1+q^{-2}+q^{-n}+q^{-n-3})/(1+q),&n\textrm{ odd}\\
        (1+q-q^{3-n}+q^{-n})/(1+q),&n\textrm{ even}
    \end{cases}
\end{equation}
A search through all prime knots up to 19 crossings against twist knots
provided many matches for the Jones polynomial
(See Tab.~\ref{table:matching_twist_knots}), but
none for Khovanov homology. All twist knots with $K_{m}$ with
$m\geq{-3}$ are transversally simple
\cite{EtnyreEtAlLegendrianAndTransverseTwistKnots}, and hence we have evidence
for the following claim.
\begin{conjecture}
    If a link type $L$ is transversally simple, then the Khovanov
    homology of $L$ distinguish it. That is, if $\tilde{L}$
    is another link with the same Khovanov homology, then $\tilde{L}$ is
    equivalent to $L$.
\end{conjecture}
In the following sections we present the computations for three types of knots:
torus, twist, and conjectured Legendrian simple knots
(see \cite{LegendrianKnotAtlas}). In every case Khovanov homology Khovanov
homology seems able to completely distinguish these types of knots.
