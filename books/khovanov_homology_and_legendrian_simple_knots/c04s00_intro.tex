Khovanov homology is capable of detecting the
unknot, trefoils, and figure-eight knot. The Khovanov homology with
coefficients in $\mathbb{Z}/2\mathbb{Z}$ is also capable of detecting
the cinquefoil knot \cite{BaldwinYingSivekCinquefoilKhovanov},
which is the $T(5,2)$ torus knot. The Jones
polynomial, on the other hand, is not capable of detecting the
$T(5,2)$ torus knot since the $10_{132}$ knot yields the same
polynomial (See Tab.~\ref{table:matching_torus_knots}).
These are the easiest of the Legendrian simple knots
leading us to the following.
\begin{conjecture}
    If a link type $L$ is Legendrian simple, then the Khovanov homology
    of $L$ distinguishes it. That is, if $\tilde{L}$
    is another link with the same Khovanov homology, then $\tilde{L}$ is
    equivalent to $L$.
\end{conjecture}
Numerical evidence has been tallied for torus knots
against all prime knots up to 19 crossings. There are many
torus knots that have the same Jones polynomial as a non-torus knot
($T(2,5)$ matches 10 and 17 crossing knots, $T(2,7)$
matches a 12 crossing knot, and $T(2,11)$ matches a 14 crossing knot)
so we cannot generalize the Jones unknot conjecture. Nevertheless, in
all cases the Khovanov homologies were different
(see Numerical Results section).
\par\hfill\par
The computations were done as follows. There are libraries in Python,
Sage, and C++ for working with knot polynomials. In particular, we used
Regina \cite{regina}, SnapPy \cite{SnapPy}, the Sage knot library
\cite{sage}, and our own ever-growing C library.
\par\hfill\par
We first gather the torus knots that could potentially have the same
Jones polynomial as some other knot based on the bounds of the degree.
The Jones polynomials of torus knots were computed using the formula
\cite{jonesfordummyvjones}:
\begin{equation}
    \label{eqn:jones_poly_torus}%
    J(T(m,n))(q)=q^{(m-1)(n-1)/2}
        \frac{1-q^{m+1}-q^{n+1}+q^{m+n}}{1-q^{2}}
\end{equation}
The Kauffman bracket, of which the Jones polynomial is a simple
normalization \cite{barnatan2002khovanov},
of a knot $K$ with $N$ crossings has the following formula:
\begin{equation}
    \langle{K}\rangle=\sum_{n=0}^{2^{N}-1}(-q)^{w(n)}(q+q^{-1})^{c(n)}
\end{equation}
where $w(n)$ is the Hamming weight of $n$, the number of 1's in the
binary representation, and $c(n)$ is the number of disjoint cycles
that result in the complete smoothing of $K$ corresponding to
$0\leq{n}\leq{2}^{N}-1$. With this the degree of the bracket polynomial
is bounded by $3n$ ($w(n)$ is bounded by $n$, and $c(n)$ is bounded
by $2n$). The degree of the normalization is bounded by $2n$
\cite{barnatan2002khovanov}, so the
Jones polynomial of a knot with $n$ crossings has degree at most $5n$.
\par\hfill\par
Since we are looking at knots up to $19$ crossings, we collect the
co-prime pairs $(m,\,n)$ with $1<m<n$ such that the degree
is less than $5\cdot{19}=95$. Since the Jones polynomial of a mirror
can be computed by substituting $q\mapsto{q}^{-1}$
\cite{jonespolyjones}, we need not look at
negative values. The Khovanov polynomial makes a similar change,
$(q,\,t)\mapsto(q^{-1},\,t^{-1})$
\cite{WATSON2017915}.
\par\hfill\par
Using any of the aforementioned libraries, the Jones polynomial of all
prime knots up to 19 crossings were computed and compared against this
table of torus knot Jones polynomials (Eqn.~\ref{eqn:jones_poly_torus}).
If a match was found the regina library was used to determine if the
knots were actually identical. That is, if the knot whose Jones
polynomial was being compared against the torus knots was indeed a
torus knot itself. If the knots were distinct, this knot was saved in a
text file for later examination. At the end of the computation 4
non-torus knots had the same Jones polynomial as a torus knot
(See Tab.~\ref{table:matching_torus_knots}).
Since the Khovanov polynomial contains the Jones polynomial in it
(recall $J(L)(q)=Kh(L)(q,-1)$) the only possible non-torus knots with
the same Khovanov homology as a torus knot were these 4.
\par\hfill\par
Using the Java library JavaKh\footnote{%
    Thanks must be paid to Nikolay Pultsin who made edits to
    JavaKh-v2 so that it may run on a GNU/Linux machine using
    OpenJDK 17.
}
we found that these four knots with the same Jones polynomials as some
torus knot all had different Khovanov homologies. Thus, we have the
following claim:
\begin{theorem}
    If a prime knot $K$ has less than or equal to 19 crossings and has
    the Khovanov homology of a torus knot $T$,
    then $K$ is equivalent to $T$.
\end{theorem}
A similar search through the twist knots yielded more results.
The Jones polynomials of the twist knots are known, with the formula:
\begin{equation}
    J(m_{n})(q)=
    \begin{cases}
        (1+q^{-2}+q^{-n}+q^{-n-3})/(1+q),&n\textrm{ odd}\\
        (1+q-q^{3-n}+q^{-n})/(1+q),&n\textrm{ even}
    \end{cases}
\end{equation}
A search through all prime knots up to 19 crossings against twist knots
provided many matches for the Jones polynomial
(See Tab.~\ref{table:matching_twist_knots}), but
none for Khovanov homology. Infinitely many of the twists knots are
transversally simple, making them a good candidate to test the following
conjecture on.
\begin{conjecture}
    If a link type $L$ is transversally simple, then the Khovanov
    homology of $L$ distinguish it. That is, if $\tilde{L}$
    is another link with the same Khovanov homology, then $\tilde{L}$ is
    equivalent to $L$.
\end{conjecture}
A similar search for Knot Floer Homology (KFH), a homology theory
first introduced by Peter Ozsv\'{a}th and Zolt\'{a}n Szab\'{o}
\cite{ozsvathszabo2004}, using the Alexander polynomial
was performed, but a bug was found that caused knots with identical
Alexander polynomials to give false negatives. This has been corrected,
and when performing a new search there are several distinct knots
with the same Knot Floer Homology as a Legendrian simple knot. Steven
Sivek pointed
out that the pretzel knots $P(-3,3,2n+1)$ all have the same KFH, meaning
the $6_{1}$ twist knot matches the KFH of the $9_{46}$ knot in the
Rolfsen table. This is what first hinted at a bug in our KFH code.
Matthew Hedden also informed us that the $T(4,3)$ knot and the
$(2,3)$ cable of the trefoil also have matching KFH.
