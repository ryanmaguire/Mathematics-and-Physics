This work examines some of the computational aspects of knot theory and
provides evidence for new conjectures in the field. Chapter 1 contains a quick
summary of the results and introduces the problems that are to be studied.
In chapter 2 we discuss some elementary background, including several of the
combinatorial methods of representing knots that allow one to make use of
computers in their calculations. The challenge of discerning knots is then
tackled in chapter 3 where we discuss knot invariants. Here we introduce two
new algorithms for the computation of the Jones polynomial,
including a new modification of the Tait graph for virtual knots.
These new algorithms are still exponential in time (with respect to the number
of crossings in a knot diagram), but linear in their spatial requirement.
We also explore some known ideas for the calculation
of other invariants such as the Alexander and HOMFLY-PT polynomials, as well as
Khovanov homology, by reviewing definitions and analyzing algorithms.
In chapter 3 we discuss contact topology and Legendrian
simple knots, and then get in to the heart of
the thesis in chapter 4 which contains the numerical work. Here we provide
support for the conjecture that Khovanov homology is able to distinguish
Legendrian and transversally simple knots using the family of torus and twist knots as
representatives, respectively. By brute force methods we are able to prove that
the Khovanov polynomial, the Poincar\'{e} polynomial of Khovanov homology, is
able to distinguish torus and twist knots among all prime knots of up to 19
crossings, which amounts to more than 352 million knots. In the process of
performing these computations we have tallied several invariants for all
prime knots up to 19 crossings and these are now publicly available. We also
examine the conjectured Legendrian simple knots from the Legendrian knot atlas.
For each of these knots it is shown that the Khovanov polynomial is able to
distinguish them among all prime knots up to 19 crossings.
