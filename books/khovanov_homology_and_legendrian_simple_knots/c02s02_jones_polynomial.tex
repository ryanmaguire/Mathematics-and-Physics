\section{Jones Polynomial}
    The main invariant we will be working with is the Jones polynomial. It is
    far more powerful, statistically speaking, than the Alexander polynomial
    at distinguishing knots but this comes at the cost of increased
    computational complexity. Indeed, the computation is known to be
    \textbf{NP-Hard} \cite[p.~2]{Przytycka1991SubexponentiallyCT}
    and the best algorithms are of
    $O(\textrm{poly}(N)\cdot{2}^{c\sqrt{N}})$ where
    $N$ is the number of crossings and $c$ is some constant (which varies
    across algorithms). Polynomial-time quantum algorithms have been devised
    \cite{JonesQuantumAlgorithm},
    but these have yet to see practical use.
    \subsection{The Temperley-Lieb Algebra}
        The original definition uses braids and the Temperley-Lieb algebra
        over the commutative ring $\mathbb{Z}[t,\,t^{-1}]$. For a general
        commutative ring $R$ and fixed element $\delta\in{R}$ we construct
        an $R$-algebra with generators $e_{0},\,e_{1},\,\dots,\,e_{N-2}$,
        where $N\in\mathbb{N}$ is fixed, with the following constraints:
        \begin{align}
            e_{n}^{2}&=\delta{e}_{n}
            \tag{for all $0\leq{n}\leq{N-2}$}\\
            e_{n}e_{n+1}e_{n}&=e_{n}
            \tag{for all $0\leq{n}\leq{N-3}$}\\
            e_{n}e_{n-1}e_{n}&=e_{n}
            \tag{for all $1\leq{n}\leq{N-2}$}\\
            e_{n}e_{m}&=e_{m}e_{n}
            \tag{for all $0\leq{m,n}\leq{N-2}$ with $|m-n|\geq{2}$}
        \end{align}
        The algebra is denoted $TL_{N}(\delta)$, or just $TL_{N}$ if $\delta$
        is clear from context. In the case of the Jones polynomial we choose
        $R=\mathbb{Z}[t,\,t^{-1}]$ and $\delta=-t^{2}-t^{-2}$. For any
        $N\in\mathbb{N}$ there is a group representation
        $\rho:B_{N}\rightarrow{TL}_{N}$ where the $n^{\small\textrm{th}}$
        generator $\sigma_{n}$ of $B_{N}$ is mapped to
        $te_{n}+t^{-1}\textrm{id}$, where $\textrm{id}\in{TL}_{N}$ is the
        unital element.
        \par\hfill\par
        There exists a trace operation
        $\textrm{tr}:TL_{n}\rightarrow\mathbb{Z}[t,\,t^{-1}]$ called the
        \textit{Markov trace} and the Jones polynomial of an element
        $g\in{B}_{N}$ is defined as:
        \begin{equation}
            V_{N}(g)=\delta^{N-1}\textrm{tr}\big(\rho(g)\big)
        \end{equation}
        If two words $g_{0}\in{B}_{N}$, $g_{1}\in{B}_{M}$ have isotopic
        braid closures in the plane, then their Jones polynomials will be
        identical. By Alexander's theorem every knot has a
        representative braid, and hence the Jones polynomial of the knot is
        the Jones polynomial of any such braid. Since isotopic braid closures
        yield identical polynomials, we do indeed have a knot invariant.
        \par\hfill\par
        We will not be using this, for the most part, but the original
        definition allows us to compute the Jones polynomial of all
        \textbf{torus knots}. Given two coprime integers $(p,\,q)$, the
        $T_{p,\,q}$ torus knot is obtained by wrapping a loop around the
        torus, twisting $p$ times azimuthally for every $q$ turns around the
        meridian. After $p\cdot{q}$ wraps around the torus the loop closes
        up. We can explicitly embed the circle onto the $T_{p,\,q}$ torus knot
        via:
        \begin{equation}
            \gamma(t)=\Big(
                \big(2+\cos(2\pi{pt})\big)\cos(2\pi{qt}),\,
                \big(2+\cos(2\pi{pt})\big)\sin(2\pi{qt}),\,
                \sin(2\pi{pt})
            \Big)
        \end{equation}
        with $t\in[0,\,1]$
        (we obtain the circle by gluing $0$ and $1$ together). The right-handed
        trefoil corresponds to the $T_{3,\,2}$ torus knot
        (Fig.~\ref{fig:trefoil_2_3_torus_knot_on_torus}).
        \par\hfill\par
        For $p$ or $q$ equal to $1$ we obtain the unknot. Excluding these
        possibilities it is well established that
        $T_{p,\,q}$ and $T_{p',\,q'}$ are equivalent knots if and only if
        $\{\,p,\,q\}=\{\,p',\,q'\}$, viewed as unordered sets. Morover,
        $T_{p,\,q}$, $T_{-p,\,q}$, $T_{p,\,-q}$, and $T_{-p,\,-q}$ are related
        by mirrors and orientation reversals.
        \begin{figure}
            \centering
            \resizebox{0.5\textwidth}{!}{%
                \includegraphics{trefoil_2_3_torus_knot_on_torus.pdf}
            }
            \caption{Trefoil as a Torus Knot}
            \label{fig:trefoil_2_3_torus_knot_on_torus}
        \end{figure}
        In \cite{jonesfordummyvjones} the Jones polynomial for all torus knots
        is expliclity computed using the braid representation. For coprime
        pairs $(p,\,q)$ the Jones polynomial of $T_{p,\,q}$ is given by:
        \begin{equation}
            \label{eqn:jones_polynomial_torus_knot}
            J_{T_{p,\,q}}(t)
            =t^{\frac{(p-q)(q-1)}{2}}
            \frac{1-t^{p+1}-t^{q+1}+t^{q+p}}{1-t^{2}}
        \end{equation}
        This is a Laurent polynomial in $t^{1/2}$. In the next section we will
        be using Laurent polynomials in $q$. To swap between the two we make
        the substitution $q^{2}=t$. We will make free use of
        Eqn.~\ref{eqn:jones_polynomial_torus_knot} throughout our computations.
    \subsection{Kauffman Bracket}
        The polynomial is made pictorial, and lends itself more easily
        to computation, via the Kauffman bracket polynomial. The Jones
        polynomial can be computed by an appropriate normalization. The bracket
        polynomial is defined recursively in terms of
        \textit{smoothings} of a link diagram. Note, the definition we'll work
        with is similar to the description provided in
        \cite{BarNatanKhovanovJones}. It differs slightly from the original
        \cite{KauffmanStateModels}.
        \par\hfill\par
        Resolving, or smoothing, a crossing equates to removing it.
        There are two ways to do this. Given an unoriented knot diagram
        we rotate our heads until the over crossing travels from the top left
        to the bottom right. We will label the two smoothings as the
        0-smoothing and the 1-smoothing, respectively, and these can be seen
        in Fig.~\ref{fig:resolving_crossing}. Using such binary notation has
        computational benefits, but it is also common to label these $A$ and
        $B$ smoothings.
        \begin{figure}
            \centering
            \includegraphics{resolving_crossings.pdf}
            \caption{Resolving a Crossing}
            \label{fig:resolving_crossing}
        \end{figure}
        \par\hfill\par
        For a link diagram $L$ the definition of the Kauffman bracket
        polynomial is:
        \begin{align}
            \label{eqn:kauffman_bracket}
            \langle\emptyset\rangle&=1\\
            \langle{L\sqcup\mathbb{S}^{1}}\rangle&=(q+q^{-1})\langle{L}\rangle\\
            \langle{L}\rangle&=
                \langle{L_{n,0}}\rangle-q\langle{L_{n,1}}\rangle
        \end{align}
        where $L_{n,0}$ and $L_{n,1}$ are the links obtained from the
        0 and 1 smoothings of $L$ at the $n^{th}$ crossing, respectively. The
        notation $L\sqcup\mathbb{S}^{1}$ means the disjoint union of
        $L$ with an unknot. Hence the Kauffman bracket of the
        unknot is $q+q^{-1}$.
        \par\hfill\par
        The Kauffman bracket is invariant under Type II and Type III moves.
        Type I, on the other hand, scales the polynomial by $q$ or $q^{-1}$,
        depending on the sign of the introduced crossing. Hence the Kauffman
        bracket is an invariant of \textit{framed} or \textit{regular} knots
        and links. Since scaling by a power of $q$ does not alter the individual
        coefficients for the polynomial, this finite sequence is indeed a link
        invariant. There are two ways to make an invariant polynomial out of
        this. Should the link diagram $L$ be oriented, the
        \textit{unnormalized} Jones polynomial is given by:
        \begin{equation}
            \label{eqn:unnormalized_jones}
            \tilde{J}_{L}(q)=(-1)^{N_{-}}q^{N_{+}-2N_{-}}\langle{L}\rangle
        \end{equation}
        where $N_{+}$ and $N_{-}$ are the number of positive and negative
        crossings in the diagram, respectively. This introduction of the
        crossing signs does not change under Type II and Type III, essentially
        since the writhe does not change, and has the added bonus of making
        $J_{L}$ invariant under Type I as well.
        \par\hfill\par
        Noticing that the defining
        equations for the Kauffman bracket imply that for any non-empty link
        diagram $L$ the bracket polynomial is divisible by $q+q^{-1}$, we
        define the \textit{normalized} Jones polynomial to be:
        \begin{equation}
            J_{L}(q)=\frac{1}{q+q^{-1}}\tilde{J}_{L}(q)
        \end{equation}
        This final polynomial is in agreement with the original braid-theoretic
        definition, though using some alternative definitions one may need to
        make the substitution $q\mapsto{q}^{1/2}$.
        \par\hfill\par
        The other, far less common, method is to scale the bracket so that
        it is a genuine polynomial, and not a Laurent polynomial. That is,
        scale the bracket by $\pm{q}^{-\textrm{mindeg}(\langle{L}\rangle)}$
        ensuring that the zeroth coefficient is for the constant term and
        positive. This has two benefits. First, it is well-defined for
        unoriented links, whereas
        the previous definition yields different polynomials for different
        orientations of the same link. Second, it is a genuine polynomial and
        not a Laurent polynomial. Because of this one may study the Julia set
        and Newton basins of the associated meromorphic function
        $z\mapsto{z}-J_{L}(z)/J_{L}'(z)$, $z\in\mathbb{C}$. The topologies of
        such sets are dependent only on the Jones polynomial, and
        are hence knot invariants. The Newton basins for the right-handed
        trefoil are shown in
        Fig.~\ref{fig:newton_fractal_right_trefoil_jones}.
        \par\hfill\par
        An argument for the mainstream definition is that it satisfies
        $J_{L}(q)=J_{m(L)}(q^{-1})$ where $m(L)$ is the mirror of $L$. This
        alternative definition does not.
        \begin{figure}
            \centering
            \resizebox{\textwidth}{!}{%
                \includegraphics{newton_fractal_right_trefoil_jones.png}
            }
            \caption{Newton Basins for the Right-Handed Trefoil}
            \label{fig:newton_fractal_right_trefoil_jones}
        \end{figure}
    \subsection{An Explicit Algorithm Using Gauss Code}
        We now prove the following.
        \begin{theorem}
            There is an algorithm with input a valid extended Gauss code on $N$
            crossings, with output the Jones polynomial of the (possibly virtual)
            knot described by the extended Gauss code, such that the best, worst,
            and average time complexity is identical and of
            $O(\textrm{poly}(N)\cdot{2}^{N})$, and with spatial requirement $O(N)$.
        \end{theorem}
        If we assume certain arithmetic operations are constant, then
        $\textrm{poly}(N)$ can be replaced by $N$. The proof is contained in
        this section by explicitly providing the algorithm. An attempt is made
        to put this in terms of \textit{literate programming}, and the source
        code is embedded along with the details of the algorithm. It is written
        in plain \texttt{C}.
        \par\hfill\par
        The splitting operation
        $\langle{L}\rangle=\langle{L_{n,\,0}}\rangle-q\langle{L_{n,\,1}}\rangle$
        at the $n^{\small\textrm{th}}$ crossing in a link diagram $L$ gives us a
        recursive algorithm for the computation of the Kauffman bracket,
        and hence the Jones polynomial. Such an algorithm is not only
        exponential in time, but exponential in space as well. This is because
        the entirety of the recursion tree must be stored in memory near the
        end of the calculation, and this is about
        $\textrm{poly}(N)\cdot{2}^{N}$ nodes. Analysis of the
        memory requirements for algorithms has become increasingly rare in
        modern times since 64GB and 128GB memory is becoming increasingly
        commonplace even for personal computers, but an exponential-in-space
        algorithm will eat through this very quickly. Most implementations
        avoid this by rewriting the recursive step in some other way.
        We adopt a similar tactic. By expanding the
        recursive definition into an iterative one we can reduce the memory
        requirements to linear. The time-complexity will still be exponential.
        \par\hfill\par
        Given a link diagram $L$ label the $N$ crossings in some way from $0$
        to $N-1$. Given any number $0\leq{n}\leq{2}^{N}-1$ we may associate a
        \textbf{complete smoothing} of the link diagram to $n$ as follows.
        Write out the number $n$ in binary. Since it falls between $0$ and
        $2^{N}-1$ this can be done using $N$ bits. If the
        $m^{\small\textrm{th}}$ bit of $n$ is zero, perform a 0-smoothing at
        the $m^{\small\textrm{th}}$ crossing. Otherwise perform a 1-smoothing.
        Every possible combination of smoothings is then uniquely represented
        by a single integer between $0$ and $2^{N}-1$.
        \par\hfill\par
        \begin{figure}
            \centering
            \includegraphics{trefoil_knot_cube_of_resolutions.pdf}
            \caption{Cube of Resolutions for the Right-Handed Trefoil}
            \label{fig:trefoil_knot_cube_of_resolutions}
        \end{figure}
        Take, for example, the right-handed trefoil in
        Fig.~\ref{fig:right_handed_trefoil_gauss_code} with the same labeling
        scheme given in the figure. There are three crossings and so we have
        $2^{3}=8$ possible resolutions. The diagram
        in Fig.~\ref{fig:trefoil_knot_cube_of_resolutions} is called the
        \textit{cube of resolutions} for the right-handed trefoil.
        This language is slightly misleading since the cube of
        resolutions is dependent on the knot diagram, so it is better to say
        this is the cube of resolutions for the standard diagram of
        the right-handed trefoil.
        \par\hfill\par
        The edges in the cube connect resolutions that differ in precisely one
        bit. Hence there is an edge between \texttt{001} and \texttt{011},
        as well as \texttt{001} and \texttt{101}, but there is no edge between
        \texttt{001} and \texttt{010}. This is related to
        \textit{Hamming weights}, soon to be discussed, and \textit{cobordisms},
        which will not be discussed.
        \par\hfill\par
        In general, for an $N$ crossing knot, the cube of resolutions will
        generate an $N$ dimensional hypercube with a different complete
        smoothing at each vertex. The image becomes complicated very quickly
        since the size of the cube is exponential in the number of crossings.
        The cube of resolutions for the figure-eight knot is shown in
        Fig.~\ref{fig:figure_eight_knot_cube_of_resolutions}.
        Connecting the appropriate images with arrows results in the tesseract
        graph. For neatness, this has been omitted.
        \begin{figure}
            \centering
            \resizebox{\textwidth}{!}{%
                \includegraphics{figure_eight_knot_cube_of_resolutions.pdf}%
            }
            \caption{Cube of Resolutions for the Figure-Eight}
            \label{fig:figure_eight_knot_cube_of_resolutions}
        \end{figure}
        \par\hfill\par
        The third defining equation for the Kauffman bracket
        (Eqns.~\ref{eqn:kauffman_bracket})
        reduces an $N$ crossing link into two $N-1$ crossing
        links, one link obtained from the 0 smoothing, the other from the
        1 smoothing. If we continue this recursive step we'll end up with
        $2^{N}$ completely resolved links, each being the disjoint union of
        circles. Applying the first and second equations, we can inductively
        prove the following formula for the Kauffman bracket:
        \begin{equation}
            \label{eqn:kauffman_bracket}%
            \langle{L}\rangle=\sum_{n=0}^{2^{N}-1}
                (-q)^{w(n)}(q+q^{-1})^{c(n)}
        \end{equation}
        Here, $w(n)$ is the Hamming weight of $n$, the number of 1's
        that occur in the binary expansion of $n$. Recalling that an integer
        $0\leq{n}\leq{2}^{N}-1$ represents a complete smoothing of a knot,
        $c(n)$ is the number of cycles that result from the $n$ smoothing.
        \par\hfill\par
        The Hamming weight is a well-studied function and efficient methods of
        computing it are known, including constant-time algorithms if one
        restricts their attention to fixed-width integers
        (usually 32 or 64 bit).%
        \footnote{
            By \textit{constant time} on a finite set of inputs we mean the
            required \textit{human} time for an input $n=1$ is about the same
            as $n=2^{32}-1$.
        }
        A general and portable algorithm can be done in
        $\log(N)$ time as follows.
\begin{lstlisting}[style=CStyle]
unsigned int hamming_weight(unsigned int val)
{
    unsigned int result = 0U;

    while(val != 0U)
    {
        result += val & 1U;
        val = val >> 1U;
    }

    return result;
}
\end{lstlisting}
        The code \texttt{result += val \& 1U} may need an explanation.
        \texttt{val \& 1U} performs a bit-wise \texttt{AND} with \texttt{val} and
        1. Since the binary representation of 1 is zero in all slots except the
        zeroth bit, \texttt{val \& 1U} is simply checking the zeroth bit of
        \texttt{val}. We add this value to \texttt{result}, ultimately giving
        us the Hamming weight. The code \texttt{val = val >> 1U} takes val and
        shifts it 1 bit to the right. So, if we had $111_{2}$ and shift it to
        the right, we'd end up with $011_{2}$. Because of this after
        \texttt{log(val)} steps \texttt{val} would be zero, exiting the
        while-loop.
        \par\hfill\par
        For the Kauffman bracket this function is used with inputs of size
        $2^{N}-1$, where $N$ is the number of crossings, so the worst-case
        time complexity at this step is $O(N)$. For example, considering the
        resolution of the right-handed trefoil
        in Fig.~\ref{fig:trefoil_knot_cube_of_resolutions} corresponding to
        $7=111_{2}$, the loop will end after 3 steps, one step for each
        crossing. The $000_{2}$ resolution exits immediately.
        \par\hfill\par
        If speed is our game we can use compiler-intrinsics, pending the
        compiler we are using. \texttt{GCC} offers the \texttt{popcount}
        function. We could write:
\begin{lstlisting}[style=CStyle]
unsigned int hamming_weight(unsigned int val)
{
    return __builtin_popcount(val);
}
\end{lstlisting}
        This will greatly improve performance, but breaks portability.
        \texttt{MSVC} users (Windows) may find the following helpful:
\begin{lstlisting}[style=CStyle]
#include <intrin.h>

unsigned int hamming_weight(unsigned int val)
{
    return __popcnt(val);
}
\end{lstlisting}
        Again, recognizing that this is not a portable solution.
        \par\hfill\par
        With the Hamming weight settled, the remaining information needed to
        compute the Kauffman bracket is the number of cycles that result from
        the $n^{th}$ smoothing ($0\leq{n}\leq{2}^{N}-1$). This section gives a
        simple algorithm using extended Gauss code. The code outlined works for
        knots, but simple extensions could be made for links. In particular, the
        algorithm works for \textit{virtual knots} and there is no restriction
        to the classical setting.
        \par\hfill\par
        \begin{figure}
            \centering
            \includegraphics{trefoil_knot_framed_001.pdf}
            \caption{Thickened Trefoil Knot}
            \label{fig:trefoil_knot_framed_001}
        \end{figure}
        We modify the algorithm for computing the virtual knot genus so that we
        may instead count the number of cycles in a complete smoothing of all
        the crossings. Once again we thicken the knot
        (like in Fig.~\ref{fig:trefoil_knot_framed_001}). Every
        crossing becomes a four-way intersection
        (Fig.~\ref{fig:thickened_crossings}), and the idea is to loop
        over the $4N$ roads in this thickened knot, $N$ being the number of
        crossings.
        \par\hfill\par
        We first examine the negative crossing. The 0 and 1
        smoothings are shown in
        Fig.~\ref{fig:thickened_negative_crossing_smoothings}. Suppose we start
        on the bottom-left road of the 0-smoothing and walk towards the
        crossing. The smoothing in the visual equates to a road block, the thick
        black line we are unable to pass through. We are thus forced to go down
        the bottom-right road. The arrow for this road is going the wrong way,
        pointing towards the crossing and we are moving away from it.
        To preserve orientation we must walk backwards. This translates as
        follows for extended Gauss code. Approaching a negative crossing from
        the bottom left road means we have approached \texttt{Un-} in
        the code reading left-to-right, \texttt{n} being the index. The
        road block tells us to go down the bottom right road, but in reverse.
        That is, look for \texttt{On-} in the code and then go
        to the \textit{previous} entry.
        This is equivalent to walking backwards. We end up at a new crossing.
        We now know the rule for approaching a crossing from the bottom left
        road for a negative crossing given the 0-smoothing. In total there are
        16 possible cases: Is the sign \texttt{+1} or \texttt{-1}? Is the type
        \texttt{U} or \texttt{O}? Is the smoothing $0$ or $1$?
        Is the direction forwards or backwards? The
        visuals for these 16 combinations are shown in
        Fig.~\ref{fig:thickened_negative_crossing_smoothings} and
        Fig.~\ref{fig:thickened_positive_crossing_smoothings}.
        \par\hfill\par
        \begin{figure}
            \centering
            \includegraphics{thickened_crossings.pdf}
            \caption{Signed Crossings in a Framed Knot}
            \label{fig:thickened_crossings}
        \end{figure}
        \begin{figure}
            \centering
            \includegraphics{thickened_negative_crossing_smoothings.pdf}
            \caption{Smoothing a Negative Crossing in a Framed Knot}
            \label{fig:thickened_negative_crossing_smoothings}
        \end{figure}
        \begin{figure}
            \centering
            \includegraphics{thickened_positive_crossing_smoothings.pdf}
            \caption{Smoothing a Positive Crossing in a Framed Knot}
            \label{fig:thickened_positive_crossing_smoothings}
        \end{figure}
        Jumping from the over-crossing to the under-crossing, and vice-verse,
        as is needed in the algorithm is made easier if we have an array
        \texttt{ind} such that \texttt{ind[n]} is an ordered pair whose
        zeroth entry is the index of the under-crossing for the
        $n^{\small\textrm{th}}$ crossing, and first entry is the index of the
        over-crossing. It is then
        useful to describe a \texttt{struct} for this ordered pair, and an
        algorithm to obtain this array.
\begin{lstlisting}[style=CStyle]
struct crossing_indices {
    unsigned int under;
    unsigned int over;
};

struct crossing_indices *get_indices(const struct knot *K)
{
    unsigned int n;
    struct crossing_indices *ind;

    if (!K)
        return NULL;

    if (K->number_of_crossings == 0U)
        return NULL;

    ind = malloc(sizeof(*ind) * K->number_of_crossings);

    if (!ind)
        return NULL;

    for (n = 0U; n < 2U * K->number_of_crossings; ++n)
    {
        if (K->type[n] == over_crossing)
            ind[K->crossing_number[n]].over = n;
        else
            ind[K->crossing_number[n]].under = n;
    }

    return ind;
}
\end{lstlisting}
        The error checks are for good programming practice: Check if your
        input pointer is \texttt{NULL}, don't assume \texttt{malloc} was
        successful, and don't pass zero to \texttt{malloc}. The actual algorithm
        is contained in the for loop. We loop over the Gauss code, ask which
        crossing number we have, and then add this index to our array. The
        loop requires $2N$ steps, $N$ being the number of crossings, so this
        step is $O(N)$. Unlike the Hamming weight and the number of circles
        from a resolution, this index-finding algorithm is only needed once. We
        execute the algorithm at the start and then store the result in
        memory.
        \begin{figure}
            \centering
            \includegraphics{thickened_crossing_labeled.pdf}
            \caption{Thickened Crossings with Labels}
            \label{fig:thickened_crossings_labeled}
        \end{figure}
        \begin{figure}
            \centering
            \includegraphics{thickened_crossings_resolved_labeled.pdf}
            \caption{Thickened Resolved Crossings with Labels}
            \label{fig:thickened_crossings_resolved_labeled}
        \end{figure}
        \par\hfill\par
        Because of the 16 possibilities, the full algorithm is lengthy. To make
        this easier we label the roads so that we can make a table. This is
        done in Fig.~\ref{fig:thickened_crossings_labeled} and we call this the
        \textit{road number}. By examining
        Fig.~\ref{fig:thickened_crossings_resolved_labeled} we can tell where
        we need to go as we walk along the knot. This is done explicitly in
        Tab.~\ref{tab:circle_counting_algorithm_where_go}.
        The table reads as follows.
        The left \textit{In} column tells you which road you are approaching the
        crossing from. Note if you are approaching from roads 0 or 1, you are
        walking forward, and if you are approaching from roads 2 or 3, you are
        walking backwards. Using the information about the sign and resolution
        of the crossing you know which road to leave from. In the Gauss code
        if you approach an under crossing, find the corresponding over crossing
        in the code (and vice-versa). Then go to the previous entry if you are
        leaving from roads 0 and 1, and the next entry if leaving from roads
        2 or 3. Eventually you will return to an entry in
        the Gauss code you have already been to. This amounts to 1 cycle from
        the resolution. Move on to the next element of the Gauss code you have
        yet to encounter and repeat.
        \begin{table}
            \centering
            \begin{tabular}{c c c c}
                In&Sign&Resolution&Out\\
                \hline
                0&-&0&1\\
                0&-&1&3\\
                0&+&0&3\\
                0&+&1&1\\
                \hline
                1&-&0&0\\
                1&-&1&2\\
                1&+&0&2\\
                1&+&1&0\\
                \hline
                2&-&0&3\\
                2&-&1&1\\
                2&+&0&1\\
                2&+&1&3\\
                \hline
                3&-&0&2\\
                3&-&1&0\\
                3&+&0&0\\
                3&+&1&2
            \end{tabular}
            \caption{The Circle Counting Algorithm - Where to Go}
            \label{tab:circle_counting_algorithm_where_go}
        \end{table}
        Tab.~\ref{tab:circle_counting_algorithm_where_start}
        then tells which road number you are entering based on the sign and type
        of the crossing, and the direction you are walking. Eventually the code
        will be exhausted, and the resulting tally is the number of
        circles from this resolution.
        \par\hfill\par
        \begin{table}
            \centering
            \begin{tabular}{c c c c}
                Type&Sign&Direction&In\\
                \hline
                $O$&-&Forward&1\\
                $O$&-&Backward&3\\
                $O$&+&Forward&0\\
                $O$&+&Backward&2\\
                \hline
                $U$&-&Forward&0\\
                $U$&-&Backward&2\\
                $U$&+&Forward&1\\
                $U$&+&Backward&3
            \end{tabular}
            \caption{The Circle Counting Algorithm - Where to Start}
            \label{tab:circle_counting_algorithm_where_start}
        \end{table}
        We now give the full details of the algorithm as a
        (semi) literate program. Since we are using all
        numbers between 0 and $2^{N}-1$ to describe complete resolutions of the
        knot the intermediate step must be performed with (at least) $N$-bit
        wide integers. Since the width of \texttt{int} is usually 32-bits, we
        restrict ourselves to knots with less than 32 crossings. If we change
        all of our types to \texttt{long} we can increase this to 64 crossings.
        We define \texttt{MAX\_CROSSINGS} to be either 32 or 64, pending size
        of the integers we are using.
        \par\hfill\par
        We create a buffer that is \texttt{4*MAX\_CROSSINGS} long to keep track
        of which roads we have visited. Our code starts as follows.
\begin{lstlisting}[style=CStyle]
unsigned int
circle_count(const struct knot *K,
             const struct crossing_indices *ind,
             unsigned int resolution)
{
    unsigned int n, number_of_circles, road_index;
    unsigned int crossing, code_index;
    unsigned char direction, crossing_resolution, road_number;
    enum crossing_sign sign;

    unsigned char have_visited[4U * MAX_CROSSINGS];

    const unsigned char forward = 0x00U;

    /* Empty knot, no cycles. */
    if (!K)
        return 0U;

    /* Unknot, 1 cycle. */
    if (K->number_of_crossings == 0U)
        return 1U;

    /* Invalid input, return 0. */
    if (!ind)
        return 0U;

    /* Initialze the buffer to have zero entries. */
    for (n = 0U; n < 4U * K->number_of_crossings; ++n)
        have_visited[n] = 0x00U;

    /* Initialize count to zero. */
    number_of_circles = 0U;
\end{lstlisting}
        with all variables and arrays initialized we start the bulk of the
        algorithm and count the number of cycles corresponding to the given
        resolution. We loop over the $4N$ roads in our thickened knot.
\begin{lstlisting}[style=CStyle, firstnumber = 34]
    for (n = 0U; n < 4U*K->number_of_crossings; ++n)
    {
        /* If we've already checked this road skip it. */
        if (have_visited[n])
            continue;
\end{lstlisting}
        Now we need the crossing index, sign of the crossing, and the road
        number (between 0 and 3). All of this can be obtained from the road
        index, which is the dummy variable \texttt{n}. This road index is
        $\texttt{n=4c+r}$ where \texttt{c} is the crossing index and
        \texttt{r} is the road number. We compute \texttt{c} by shifting
        \texttt{n} down by two bits, and we compute \texttt{r} via
        \texttt{r=n mod 4}. This yields the following:
\begin{lstlisting}[style=CStyle, firstnumber = 40]
        road_index = n;
        crossing = road_index >> 2U;
        road_number = road_index & 0x03U;
\end{lstlisting}
        Since the sign does not change between occurences of a crossing, we can
        grab the sign from either the over or under crossing entry of the Gauss
        code. We use the crossing table \texttt{ind} that we've already computed
        to aid us.
\begin{lstlisting}[style=CStyle, firstnumber = 44]
        sign = K->sign[ind[crossing].over];
\end{lstlisting}
        Lastly we need the direction we are travelling, either forwards or
        backwards. We set \texttt{forward = 0}, and so have implicitly
        declared \texttt{backward = 1}. This has the following advantage. By
        examining Fig.~\ref{fig:thickened_crossings_labeled} we see that the
        forward direction corresponds to road number 0 and 1, and backwards
        corresponds to 2 and 3. That is, the direction is determine by the
        2's bit in the road number. We compute the direction via:
\begin{lstlisting}[style=CStyle, firstnumber = 46]
        direction = road_number >> 1U;
\end{lstlisting}
        The last bit of information we need is the location of the entry in the
        Gauss code. This will be called the \textit{code index}. It can be
        computed by examining the figures above. We have:
\begin{lstlisting}[style=CStyle, firstnumber = 48]
        if (sign == positive_crossing)
        {
            if (road_number & 0x01U)
                code_index = ind[crossing].under;
            else
                code_index = ind[crossing].over;
        }

        else
        {
            if (road_number & 0x01U)
                code_index = ind[crossing].over;
            else
                code_index = ind[crossing].under;
        }
\end{lstlisting}
        We now loop over all of the roads and compute the number of cycles. We
        start with:
\begin{lstlisting}[style=CStyle, firstnumber = 64]
        while (!have_visited[road_index])
        {
            have_visited[road_index] = 0x01U;

            /* How to smooth the mth crossing. Get the mth bit. */
            crossing_resolution = (resolution >> crossing) & 0x01U;
\end{lstlisting}
        Now we need to know which road to leave on. We appeal to
        Tab.~\ref{tab:circle_counting_algorithm_where_go}. We convert this table
        into a more algebraic expression as follows. We see from
        Fig.~\ref{fig:thickened_positive_crossing_smoothings} that if we have a
        positive crossing with a 0 smoothing, then
        $0\mapsto{3}$, $1\mapsto{2}$, $2\mapsto{1}$, and $3\mapsto{0}$. We note
        that this is described by $y=3-x$. Similarly, for a 1 smoothing we have
        $0\mapsto{1}$, $1\mapsto{0}$, $2\mapsto{3}$, and $3\mapsto{2}$. This
        is described by $y=(5-x)\textrm{ mod }4$. Negative crossings may be
        handled by mirroring this. We obtain:
\begin{lstlisting}[style=CStyle, firstnumber = 71]
            if (sign == positive_crossing)
            {
                if (crossing_resolution == 0x00U)
                    road_number = 3U - road_number;

                else
                {
                    road_number = (5U - road_number) & 0x03U;
                    direction = 0x01U - direction;
                }
            }

            else
            {
                if (crossing_resolution == 0x00U)
                {
                    road_number = (5U - road_number) & 0x03U;
                    direction = 0x01U - direction;
                }

                else
                    road_number = 3U - road_number;
            }

            have_visited[4U*crossing + road_number] = 0x01U;
\end{lstlisting}
        Changing which road we're on means we've changed the strand we're on.
        Over goes to under and under goes to over. We need to update this by
        going to the correct entry in the Gauss code.
\begin{lstlisting}[style=CStyle, firstnumber=97]
            if (K->type[code_index] == over_crossing)
                code_index = ind[crossing].under;
            else
                code_index = ind[crossing].over;
\end{lstlisting}
        We now move along to the new entry in the Gauss code by moving either
        forwards or backwards, pending our current direction. If we're at the
        final entry in the Gauss code and are moving forward, we loop back
        around to the start. Similarly if we're at the first entry and a walking
        backwards we loop around to the final entry. This gives us:
\begin{lstlisting}[style=CStyle, firstnumber=102]
            if (direction == forward)
            {
                if (code_index == 2U*K->number_of_crossings - 1U)
                    code_index = 0U;
                else
                    ++code_index;
            }

            else
            {
                if (code_index == 0U)
                    code_index = 2U*K->number_of_crossings - 1U;
                else
                    --code_index;
            }
\end{lstlisting}
        Nearly done, we're now at the next entry in the Gauss code, we just need
        to know which road we're entering from. We use
        Tab.~\ref{tab:circle_counting_algorithm_where_start} for this.
\begin{lstlisting}[style=CStyle, firstnumber = 118]
            if (K->sign[code_index] == positive_crossing)
            {
                if (K->type[code_index] == over_crossing)
                {
                    if (direction == forward)
                        road_number = 0U;
                    else
                        road_number = 2U;
                }

                else
                {
                    if (direction == forward)
                        road_number = 1U;
                    else
                        road_number = 3U;
                }
            }

            else
            {
                if (K->type[code_index] == over_crossing)
                {
                    if (direction == forward)
                        road_number = 1U;
                    else
                        road_number = 3U;
                }

                else
                {
                    if (direction == forward)
                        road_number = 0U;
                    else
                        road_number = 2U;
                }
            }
        }
\end{lstlisting}
        As a slight reduction in code size, these nested conditionals can be
        replaced by an algebraic expression in our search of the
        \textit{in} road. We first note that the negative and positive crossings
        have mirrored conditional statements. We search for four variables
        $a,b,c,d$ such that:
        \begin{equation}
            r=ast+bs+ct+d
        \end{equation}
        where $r$ is the road number, $s$ is the sign, and $t$ is the type.
        Since the sign $s$ and type $t$ are represented by \texttt{enum} data
        types we may treat them as integers and add them.
        Tab.~\ref{tab:circle_counting_algorithm_where_start} tells gives us
        $a=-2$, $b=1$, $c=1$, and $d=2\cdot\texttt{dir}$, where \texttt{dir} is
        the direction ($\texttt{forward}=0$, $\texttt{backward}=1$). We can
        compute the output road number via:
\begin{lstlisting}[style=CStyle]
unsigned char
get_road_number(const struct knot *K,
                unsigned int code_index,
                unsigned char direction)
{
    const unsigned char b = K->sign[code_index];
    const unsigned char c = K->type[code_index];
    const unsigned char a = (b * c) << 1U;
    const unsigned char d = direction << 1U;
    return b + c + d - a;
}
\end{lstlisting}
        Regardless of preference (nested conditionals or algebraic expressions)
        we now know which road we are entering the crossing from.
        We update our variables and go back to the top of the
        \texttt{while} loop and continue until
        we encounter a road we've already traveled on. Once this occurs we
        increment the circle count. We do this until all $4N$ entries of the
        \texttt{have\_visited} table are set to 1. At this point the
        \texttt{for} loop will exit. We return the count and exit the function.
\begin{lstlisting}[style=CStyle, firstnumber = 157]
            crossing = K->crossing_number[code_index];
            sign = K->sign[code_index];
            road_index = (crossing << 2U) + road_number;
        }

        ++number_of_circles;
    }

    return number_of_circles;
}
\end{lstlisting}
        Let's use the trefoil as an example. The $000_{2}$ resolution results
        in 2 cycles (Fig.~\ref{fig:trefoil_knot_cube_of_resolutions}). Let's
        check that the algorithm detects this. The Gauss code is
        \texttt{O0+ U1+ O2+ U0+ O1+ U2+}.
        We start by entering the zeroth crossing
        (road 0). It is an over crossing, so we look ahead in the code and
        find the corresponding under crossing. It is a positive crossing with
        the zero resolution, so
        Tab.~\ref{tab:circle_counting_algorithm_where_go} tells
        us to leave through road 3. We mark road 0 and road 3 of the zeroth
        crossing as traveled and proceed. Leaving through road 3 means we
        travel forward. Hence we wind up at \texttt{O1+} in the code and we are
        walking forwards. Over-crossing, positive sign, walking forwards means
        we are entering the crossing from road 0
        (Tab.~\ref{tab:circle_counting_algorithm_where_start}). We will again
        leave through road 3 and enter the \texttt{O2+}
        crossing walking forwards. We again enter road 0 and leave
        through road 3 and wind up at \texttt{O0+} walking forward,
        completing our first cycle. The next untouched
        road is road 1 for the zeroth crossing. This corresponding to
        \texttt{U0+} walking forwards.
        Tab.~\ref{tab:circle_counting_algorithm_where_go}
        tells us to leave through road 2. We end up at \texttt{U1+}, road 1.
        Again we leave through road 2 and end up at \texttt{U2+} road 1.
        We leave through road 2 entering $U0+$, completing our cycle.
        All of the roads have been marked, and we have 2 cycles total,
        in agreement with Fig.~\ref{fig:trefoil_knot_cube_of_resolutions}.
        \par\hfill\par
        To compute the Kauffman bracket we loop over all integers
        $0\leq{n}\leq{2}^{N}-1$, $N$ being the number of crossings, compute the
        number of circles corresponding to the $n^{\small\textrm{th}}$
        resolution using this
        algorithm, and perform the sum in Eqn.~\ref{eqn:kauffman_bracket}. The
        circle counting algorithm requires $4N$ steps since we loop over the
        $4N$ roads, and the Hamming weight has worst-case complexity $O(N)$,
        meaning each stage is $O(4N)+O(N)=O(N)$ in time complexity. There are
        $2^{N}$ total steps, so $O(N2^{N})$ is the total complexity of the
        computation. This assumes certain arithmetic operations are constant
        in time. For arbitrary precision arithmetic beyond 32-bit and 64-bit
        fixed-width integers this is not so. At any rate, these complexities are
        polynomial so our total complexity is $O(\textrm{poly}(N)\cdot{2}^{N})$.
        \par\hfill\par
        The computation of the array of indices at the beginning
        is $O(N)$, but as mentioned is only required once and is not part of
        the inner for-loop. The result for the entire algorithm is
        $O(N+\textrm{poly}(N)\cdot{2}^{N})=O(\textrm{poly}(N)\cdot{2}^{N})$.
        The primary benefit here is that the space
        requirement is linear, only storing the $4N$ array of Booleans in
        memory and the Laurent polynomial itself (which is $O(N)$ in length).
        The algorithm also operates on any valid extended Gauss code, meaning
        we can use it to compute the Kauffman bracket of virtual knots.
        \par\hfill\par
        To obtain the unnormalized Jones polynomial we need to shift
        the degree of the resulting Kauffman bracket polynomial
        (Eqn.~\ref{eqn:unnormalized_jones}). The computation of $N_{-}$ and
        $N_{+}$ is $O(N)$, loop through the Gauss code and count.
        We represent Laurent polynomials by the following struct:
        \begin{lstlisting}[style=CStyle, gobble=12]
            struct laurent_polynomial {
                signed int lowest_degree;
                signed int highest_degree;
                signed int *coeffs;
            };
        \end{lstlisting}
        Once $N_{-}$ and $N_{+}$ have been computed,
        multiplication by $q^{N_{+}-2N_{-}}$ thus takes $O(1)$ in time, we
        simply add $N_{+}-2N_{-}$ to \texttt{highest\_degree} and
        \texttt{lowest\_degree}. This step does not impact the computation time
        in any significant way. A full implementation by the author is
        available as free software under the GPLv3 license
        \cite{MaguireJones}.
    \subsection{Symbolic Calculus}
        Another method of computation is worth noting since its modification
        will be used to compute Khovanov homology
        (as in \cite{BarNatan2006FASTKH}). This method is outlined in
        \cite{KatlasJones} and is the implementation used by the
        \texttt{Mathematica} package \texttt{KnotTheory`}. The algorithm
        can be modified to general programming languages lacking symbolic
        manipulations using doubly linked lists and an implementation
        is provided by the author \cite{MaguireJones}.
        \par\hfill\par
        The computation begins by examining how smoothing a crossing modifies
        the PD code of a knot. The algorithm can work for link diagrams too
        where we consider sequences of PD codes, one for each component of the
        link.
        \begin{figure}
            \centering
            \includegraphics{resolving_crossings_pd_label.pdf}
            \caption{Resolving a Labeled Crossing}
            \label{fig:resolving_crossings_pd_label}
        \end{figure}
        The Kauffman relation tells us to replace ordered quadruples
        \texttt{X[a,b,c,d]} with ordered pairs
        \texttt{P[a,b]P[c,d]-$q$P[a,d]P[b,c]}
        (Fig.~\ref{fig:resolving_crossings_pd_label}).
        Gathering these \textit{affine combinations}
        at each crossing gives us a product:
        \begin{equation}
            \langle{K}\rangle
            =\prod_{n=0}^{N-1}(
                \texttt{P[an,bn]P[cn,dn]-$q$P[an,dn]P[bc,cn]}
            )
        \end{equation}
        where \texttt{X[an,bn,cn,dn]} is the
        $n^{\small\textrm{th}}$ entry in the PD code. You might think
        \textit{hey, that is linear}! Indeed, we have the product of
        $N$ affine terms, but expanding this product out into a polynomial
        will require $2^{N}$ monomial products. This expansion will be
        necessary in order to retrieve integer coefficients from our formal
        polynomial with (products of) ordered pair coefficients.
        \par\hfill\par
        Expanding the product we will obtain a polynomial where the coefficients
        are formal products of ordered pairs. The ordering of the product
        does not matter, we can take it to be commutative. Should we find a
        product of the form \texttt{P[a,b]P[b,c]} the PD code is telling us
        the \texttt{a} arc is now connected to the \texttt{b} arc after we've
        resolved the crossings, and similar that the \texttt{b} arc is
        connected to the \texttt{c} arc. \texttt{b} is now redundant and we
        may replace this with \texttt{P[a,c]}. That is, we make the formal
        substitution $\texttt{P[a,b]P[b,c]}\mapsto\texttt{P[a,c]}$. Similarly
        \texttt{P[a,b]P[a,b]} tells us \texttt{a} and \texttt{b} are forming a
        cycle. We may substitute $\texttt{P[a,b]P[a,b]}\mapsto\texttt{P[a,a]}$.
        In both of the substitutions the ordering of the pairs is immaterial.
        We simply wish to know which arcs are now connected to which.
        \texttt{P[a,a]} tells us we have a loop. The Kauffman bracket says
        we replace $\texttt{P[a,a]}\mapsto{q}+q^{-1}$.
        \par\hfill\par
        If symbolic manipulations are available to you this defines a very
        compact algorithm.\footnote{%
            \cite{KatlasJones} implements this in three lines.
        }
        As mentioned, in a general programming language we can mimic these
        formal Laurent polynomial using doubly linked lists where the elements
        of the linked list are used to represent formal products of ordered
        pairs.
        \par\hfill\par
        The computational complexity in both time and space are on par with the
        previous Gauss code algorithm, though the actual time comparisons show
        the former method is a bit faster. Nevertheless we can modify this
        idea slightly and achieve a remarkable speed boost.
        \par\hfill\par
        Repeat the previous algorithm by picking some element in the PD code,
        undergoing the formal substitution
        $\texttt{X[a,b,c,d]}\mapsto\texttt{P[a,b]P[c,d]-$q$P[a,d]P[b,c]}$.
        Rather than doing this for each crossing,
        forming a product of affine terms,
        and then expanding, we instead find the entry in the PD code with the
        \textit{most} number of elements in common with our starting one.
        That is, the entry \texttt{X[e,f,g,h]} with the highest number of
        matching integers in \texttt{X[a,b,c,d]}. We take this PD code entry
        and expand it using
        $\texttt{X[e,f,g,h]}\mapsto\texttt{P[e,f]P[g,h]-$q$P[e,h]P[f,g]}$. We
        then simplify the product
        \begin{equation}
            \nonumber
            (\texttt{P[a,b]P[c,d]-qP[a,d]P[b,c]})
            (\texttt{P[e,f]P[g,h]-qP[e,h]P[f,g]})
        \end{equation}
        using the rules before for substituting formal products of ordered
        pairs. Once simplified we add the entry in the next PD code with the
        highest number of matching entries among
        \texttt{a,b,c,d,e,f,g,h}. We continue doing this until all entries in
        the PD code have been added.
        \par\hfill\par
        We can envision this modification as follows. We pick a crossing in
        the diagram and draw a small circle around it. We find the next
        crossing with the highest number of strands going into this circle.
        We add this crossing, and enlarge our circle into a blob the engulfs
        both crossings. We then find the next crossing with the largest number
        of strands going into our blob, and enrich the blob by adding this
        crossing. The blob is our \textit{computational front}, and by
        adding the crossing with the highest number of strands in the blob we
        are increasing the size of the blob minimally. We are effectively
        ensuring we do as little simplifying as possible in our final
        expansion.
        \par\hfill\par
        While simple enough to describe, this modification of the symbolic
        calculus yields significant improvements in computational time (the
        complexity is about the same). It is not uncommon to find the
        computation is about 100x faster with this trick. On a more somber
        note, 100x faster for an exponential-like algorithm means you can
        increase the crossing number by 7 or 8 before you're back to similar
        times.
        \subsection{Tait Graph Expansion}
            The next method of computation has two uses for us. Firstly, it is
            very efficient in terms of speed, and secondly it will give us an
            explicit formula for the Jones polynomial of \textbf{twist knots}.
            Later on we will appeal to this formula for calculations
            involving twist of over 90 crossings. The exponential nature of
            the algorithms we've thus far presented would otherwise prevent us
            from doing this.
            \par\hfill\par
            We start with the Tait graph. Given a knot or link diagram
            the smoothing operations for the Kauffman bracket can translate to
            operations on the Tait graph. This can be seen in
            Fig.~\ref{fig:tait_graph_kauffman_negative_smoothing} for negative
            (red) edges in the graph. For positive edges we would rotate the
            picture.
            \par\hfill\par
            \begin{figure}
                \centering
                \resizebox{0.5\textwidth}{!}{%
                    \includegraphics{tait_graph_kauffman_negative_smoothing.pdf}
                }
                \caption{Smoothing a Negative Edge on the Tait Graph}
                \label{fig:tait_graph_kauffman_negative_smoothing}
            \end{figure}
            We see that the Kauffman relation tells us to contract and delete
            edges, which corresponds to merging and separating faces in the
            knot diagram. That is, if $T$ is the Tait graph of a knot diagram
            $K$, smoothing at the $n^{\small\textrm{th}}$ crossing corresponds
            to smoothing the $n^{\small\textrm{th}}$ edge via:
            \begin{align}
                \label{eqns:kauffman_tait_relation}
                \langle{T}\rangle
                &=\langle{{T^{n_{-}}_{\textrm{delete}}}}\rangle
                -q\langle{{T^{n_{-}}_{\textrm{contract}}}}\rangle\\
                \langle{T}\rangle
                &=\langle{{T^{n_{+}}_{\textrm{contract}}}}\rangle
                -q\langle{{T^{n_{+}}_{\textrm{delete}}}}\rangle
            \end{align}
            where $n_{\pm}$ indicates whether the edge is positive (blue) or
            negative (red), respectively.
            Separating disks will still result in disks, but
            merging is more subtle. If we are merging two distinct disks we are
            essentially creating a dogbone, which is still topologically a
            disk. The issue arises when the two dots in the left hand side of
            Fig.~\ref{fig:tait_graph_kauffman_negative_smoothing} represent
            the same face. In such an event we are creating a topological
            annulus, which is not allowed for Tait graphs since the vertices
            must represent disks in the knot diagram. The Kauffman reduction
            $\langle{\mathbb{S}^{1}\sqcup{K}}\rangle=(q+q^{-1})\langle{K}\rangle$
            comes to the rescue. The inner circle of our annulus corresponds to
            a cycle in the smoothing of the knot diagram. We may contract it to
            a point, turning our annulus into a disk, at the cost of a
            $q+q^{-1}$ term floating around.
            \begin{figure}
                \centering
                \resizebox{0.5\textwidth}{!}{%
                    \includegraphics{tait_graph_reidemeister_1_dual.pdf}
                }
                \caption{Loops in the Tait Graph}
                \label{fig:tait_graph_reidemeister_1_dual}
            \end{figure}
            \par\hfill\par
            In the case of a blue edge we get:
            \begin{align}
                \langle{T}\rangle
                &=\langle{T^{n_{+}}_{\textrm{contract}}}\rangle
                -q\langle{T^{n_{+}}_{\textrm{delete}}}\rangle\\
                &=(q+q^{-1})\langle{T^{n_{+}}_{\textrm{delete}}}\rangle
                -q\langle{T^{n_{+}}_{\textrm{delete}}}\rangle\\
                &=q^{-1}\langle{T^{n_{+}}_{\textrm{delete}}}\rangle
            \end{align}
            For red we have:
            \begin{align}
                \label{eqn:tait_graph_kauffman_bracket_negative_loop_reduction}
                \langle{T}\rangle
                &=\langle{{T^{n_{-}}_{\textrm{delete}}}}\rangle
                -q\langle{{T^{n_{-}}_{\textrm{contract}}}}\rangle\\
                &=\langle{{T^{n_{-}}_{\textrm{delete}}}}\rangle
                -q(q+q^{-1})\langle{{T^{n_{-}}_{\textrm{delete}}}}\rangle\\
                &=-q^{2}\langle{{T^{n_{-}}_{\textrm{delete}}}}\rangle
            \end{align}
            This computation merely amounts to the fact that the Kauffman
            bracket polynomial is not invariant under Type I moves. It does,
            however, show that the effect of Type I moves on a diagram is
            is simply shifts of the degree and, possibly, changes in the signs
            of the coefficients.
            \par\hfill\par
            At the end of this deletion and contraction algorithm one will have
            dots on the plane. These corresponds to isolated faces bounded
            by cycles. To complete the computation we remove all of these
            points and replace them with $(q+q^{-1})^{m}$ where $m$ is the
            number of dots.
            \par\hfill\par
            As noted there is a lot of practical uses for this idea. Firstly
            it presents another algorithm, but one that operates on graphs.
            Second, and perhaps most important for us, it gives us a fairly
            simple means of computing the Jones polynomial of twist knots.
            Lastly, and perhaps most important for knot theorists, it can be
            modified and used to prove some of the original Tait conjectures.
            In \cite{ThistlethwaiteSpanningTree} the Kauffman bracket is
            computed by attaching monomials to each of the spanning trees of
            the Tait graph and then summing over them. This is a great
            theoretical tool, but also a powerful computational one.
            It is known that planar graphs have at most
            $O(r^{M})$ spanning trees with $r\approx{5.3}$ and where $M$ is the
            number of vertices \cite{NumberOfSpanningTrees}. This is the
            worst-case analysis, however, and luck occasionally strikes and
            gives us a much better bound. The Tait graph for the standard
            right and left trefoil diagrams have three spannings trees. Indeed,
            for $T_{2,\,p}$ torus knots there are $p$ spanning trees, reducing
            the computation to linear.\footnote{%
                Though, the Jones polynomial has already been computed for
                all torus knots in Eqn.~\ref{eqn:jones_polynomial_torus_knot}.
            }
            \par\hfill\par
            We now put Eqns.~\ref{eqns:kauffman_tait_relation} to good use
            and compute the Jones polynomial of all twist knots. To do this
            we first describe twist knots. The twist knot with $n\in\mathbb{Z}$
            twists starts by performing $n$ Type I moves on the unit circle,
            creating a \textit{positive} Type I loop for $n>0$ and a
            \textit{negative} Type I loop for $n<0$. To create a non-trivial
            knot we then stretch the ends around and link them together by
            cutting at a point, lacing, and then tying
            (see Fig.~\ref{fig:twist_knot_001}).
            \begin{figure}
                \centering
                \resizebox{0.6\textwidth}{!}{%
                    \includegraphics{twist_knot_001.pdf}
                }
                \caption{Twist Knots}
                \label{fig:twist_knot_001}
            \end{figure}
            \begin{figure}
                \centering
                \resizebox{0.6\textwidth}{!}{%
                    \includegraphics{twist_knot_tait_graph_001.pdf}
                }
                \caption{Tait Graph of Twist Knots}
                \label{fig:twist_knot_tait_graph_001}
            \end{figure}
            By choosing one of the inner loop faces as the start we obtain a
            pleasant Tait graph to work with
            (Fig.~\ref{fig:twist_knot_tait_graph_001}). By performing the
            Kauffman relation on the bottom right-most edge we obtain
            Fig.~\ref{fig:twist_knot_tait_graph_kauffman_relation}. The
            1-smoothing yields a digon with a chain attached ending in a leaf.
            The dual of a leaf is a loop with the sign flipped
            (see Fig.~\ref{fig:tait_graph_reidemeister_1_dual}).
            Appealing to
            Eqn.~\ref{eqn:tait_graph_kauffman_bracket_negative_loop_reduction}
            deleting a positive leaf scaled the Kauffman bracket by
            $-q^{2}$. For the $K_{N}$ twist knot there will be
            $N-1$ vertices along the chain, deleting all of them yields
            $(-q^{2})^{N-1}\langle{H}\rangle$ where $H$ is the knot diagram
            corresponding to the digon, which happens to be the Hopf link.
            \par\hfill\par
            The 0-smoothing is even nicer, we have obtained the Tait graph for
            the $K_{N-1}$ twist knot. We thus have the following recursion
            relation.
            \begin{figure}
                \centering
                \includegraphics{twist_knot_tait_graph_kauffman_relation.pdf}
                \caption{Kauffman Relation for the Tait Graph of Twist Knots}
                \label{fig:twist_knot_tait_graph_kauffman_relation}
            \end{figure}
            \begin{align}
                \langle{K_{N+1}}\rangle
                &=\langle{K_{N}}\rangle
                -q(-q^{2})^{N}\langle{H}\rangle\\
                &=\langle{K_{N}}\rangle
                +(-1)^{N+1}q^{2N+1}\langle{H}\rangle
            \end{align}
            We may expand this recusion out $N$ times until we get $K_{1}$,
            which is the left-handed trefoil. Our formula becomes
            \begin{equation}
                \langle{K_{N+1}}\rangle
                =\langle{K_{1}}\rangle
                +\langle{H}\rangle
                \sum_{k=1}^{k}(-1)^{k+1}q^{2k+1}
            \end{equation}
            Using the geometric series we get
            \begin{equation}
                \langle{K_{N+1}}\rangle
                =\langle{K_{1}}\rangle+\langle{H}\rangle
                \frac{q^{3}}{1+q^{2}}\big(1+(-1)^{N+1}q^{2N}\big)
            \end{equation}
            so the computation of the Kauffman bracket for twist knots is
            reduced to the left-handed trefoil and the Hopf link (in their
            standard diagrams). The left-handed trefoil produces
            \begin{equation}
                \langle{K_{1}}\rangle=(q+q^{-1})(q^{-2}-1-q^{4})
            \end{equation}
            and the Hopf link yields
            \begin{equation}
                \langle{H}\rangle=
                (q+q^{-1})(q^{-1}+q^{3})
            \end{equation}
            After computing the writhes of the diagram and normalizing
            by $q+q^{-1}$ we obtain the Jones polynomial:
            \begin{equation}
                J_{K_{N}}(q)
                =\begin{cases}
                    \frac{q^{6}+q^{2}-q^{6-2N}+q^{-2N}}{1+q^{2}}
                    &N\textrm{ even}\\
                    \frac{1+q^{-4}+q^{-2N}-q^{-2N-6}}{1+q^{2}}
                    &N\textrm{ odd}
                \end{cases}
            \end{equation}
            This computation holds for $N>0$. For $N<0$ we use the
            reflection formula by noticing that $K_{N}$ and $K_{-1-N}$ are
            mirrors of each other. We could also repeat the Tait graph
            computation using red edges, either works.
            \par\hfill\par
            We can get an algorithm for virtual knots out of the Tait graph
            if we use the modifications presented in the first chapter. Given
            the virtual Tait graph for a virtual knot diagram, for any
            crossing (hollow vertex) there are three possibilities:
            \begin{enumerate}
                \item There are two red edges incident.
                \item There are two blue edges incident.
                \item There are two red edges and two blue edges incident.
            \end{enumerate}
            Smoothings in the knot diagram equate to deleting hollow vertices,
            and the Kauffman relations tell us to get a polynomial out of this
            from the modified Tait graph. For the first two possibilities we
            have Eqns.~\ref{eqns:kauffman_tait_relation}. An algorithm for
            virtual knots will be obtained once we've figured out how to
            resolve a hollow vertex of degree 4. To do this we note that
            a 0-smoothing will cut the red edges and contract the blue ones,
            whereas a 1-smoothing does the opposite, cutting blue edges and
            contracting red ones. As before we must be careful in this
            contraction, should a loop be created we must remove it and then
            scale the polynomial by $-q^{2}$ or $q^{-1}$, depending on the
            color of the edge for the loop. And that's it, we can now compute
            the Kauffman bracket for virtual Tait graphs as well.
