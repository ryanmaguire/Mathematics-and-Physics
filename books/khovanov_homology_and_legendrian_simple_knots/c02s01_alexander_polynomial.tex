\section{Alexander Polynomial}
    The Alexander polynomial is one of the first great invariants of knot theory.
    It is capable of distinguishing most knots with low crossing numbers and
    its computational complexity is polynomial in time. We start with a
    knot embedded into $\mathbb{R}^{3}$ and examine the knot complement.
    The fundamental group (the knot group of the knot) contains a subgroup
    isomorphic to $\mathbb{Z}$ as can be seen by wrapping a circle $n$ times
    around a small section of the knot, doing this for any $n\in\mathbb{Z}$.
    By the Galois correspondence of covering spaces
    \cite[Sec.~1.3]{HatcherAlgTop} there is an infinite cyclic cover,
    a covering space of the knot complement with fundamental group
    $\mathbb{Z}$. Following \cite[p.~53]{LickorishKnotTheory} we can be quite
    explicit. Take a Seifert surface $S\subseteq\mathbb{R}^{3}$, an embedded
    orientable surface that has the knot as its boundary, and remove a
    thickened form of it from $\mathbb{R}^{3}$. That is, we are removing
    $S\times(-1,\,1)$ from Euclidean space. The result is a manifold with
    boundary, where the boundary is two copies of $S$, label them $S_{+}$ and
    $S_{-}$. Label this space $Y$ and take infinitely many spaces $Y_{n}$,
    $n\in\mathbb{Z}$, each homeomorphic to $Y$ with homeomorphism
    $h_{n}:Y\rightarrow{Y}_{n}$. Take the disjoint union of them all and
    form the quotient space by identifying $h_{n}[S_{-}]$ with $h_{n+1}[S_{+}]$.
    The result is a covering space $X$ with fundamental group $\mathbb{Z}$
    for the knot complement. There is a Deck transformation
    $t:X\rightarrow{X}$ given by $t|Y_{n}=h_{n+1}h_{n}^{-1}$ which
    shifts the sheets by one. This acts on the first homology group
    $H_{1}(X,\,\mathbb{Z})$ allowing the group to be viewed as a
    module over $\mathbb{Z}[t,\,t^{-1}]$. This is the
    \textbf{Alexander module}. It is finitely presentable and can be
    represented by the \textbf{Alexander matrix}. If there are $r$ generators,
    we can consider the ideal generated by all $r\times{r}$ minors of the
    matrix, and this is called the \textbf{Alexander ideal}. In
    \cite{AlexanderTopologicalInvariants} it is shown such an ideal is
    principle and non-zero. Given a generator, scaling by $\pm{t}^{\pm{n}}$ will
    also be a generator. Alexander's original polynomial is the generator with
    positive constant term.
    \par\hfill\par
    A combinatorial description is given in
    \cite[p.~49]{LivingstonKnotTheory} that makes computation easier. Starting
    with an oriented knot diagram. Label the $N$ crossings and $N$ arcs
    between under crossings. That is, an arc starts at an under crossings and
    continues over all over crossings it may see until it reaches another
    under crossing. The entirety of this curve counts as one arc.
    Form an $N\times{N}$ matrix with coefficents in $\mathbb{Z}[t]$
    as follows. Travel to the end of the $\ell^{\small\textrm{th}}$ arc.
    There is an over crossing in front with the $m^{\small\textrm{th}}$
    arc passing through, and the $(\ell+1)^{\small\textrm{th}}$ arc is on the
    other side. If the crossing is positive, place $1-t$ in the
    $(n,\,\ell)$ entry, where $n$ is the number of this particular crossing,
    and $-1$ in the $(n,\,m)$ slot. For negative crossings put
    $1-t$ in $(n,\,\ell)$, $t$ in $(n,\,m)$, and
    $-1$ in $(n,\,m+1)$. Should any of the three integers $m$, $\ell$, and $m+1$
    be equal, sum the corresponding values for that entry. All other entries
    are zero. The Alexander polynomial is computed by the
    determinant of the $(N-1)\times(N-1)$ matrix formed by crossing out the
    last row and last column. Should $N-1=0$ we take the determinant to be
    one. Choice of labeling can scale the polynomial by $\pm{t}^{n}$. We thus
    define the \textbf{Alexander polynomial} $\Delta_{K}(t)$ to the result of
    the determinant shifted by $\pm{t}^{n}$ so that
    $\Delta_{K}(t)=\Delta_{K}(t^{-1})$ and so that the constant term is
    non-negative.
    \par\hfill\par
    Using sparse matrices we can fill up the matrix in $O(N)$ time. A full
    matrix where we'd be require to enter in the zero values can be rendered
    in $O(N^{2})$ time. The bulk of the computation then lies in the
    determinant. The na\"{i}ve Laplace expansion runs in $O(N!)$ time but it
    is a well-known fact in computational linear algebra that $LU$
    decomposition can be used to perform the calculation in cubic time,
    $O(N^{3})$.
    \par\hfill\par
    Being a polynomial time invariant
    it is often the first thing a knot theorist will
    compute when trying to differentiate two knots. As the number of crossings
    increases the strength seriously diminishes and it becomes more likely
    for distinct knots to have identical Alexander polynomials. Nevertheless,
    its computation is quite fast and has even been improved in recent years
    \cite{BarNatanPolynomialTimeKnotPolynomials}. The implementation used
    for tabulating the Alexander polynomial is based on this newer method.
