\section{Basic Definitions and Invariants}
    Following \cite[p.~15]{LivingstonKnotTheory}, we define a knot as follows.
    \begin{definition}[\textbf{Knot}]
        A knot is a polygonal (piece-wise linear) embedding
        $\gamma:\mathbb{S}^{1}\rightarrow\mathbb{R}^{3}$ of the unit circle
        into three dimensional Euclidean space.
    \end{definition}
    A triangle serves as a polygonal example of the unknot, and the trefoil is
    given a polygonal embedding in Fig.~\ref{fig:trefoil_knot_polygonal}.
    \begin{figure}
        \centering
        \includegraphics{trefoil_knot_polygonal.pdf}
        \caption{Polygonal Trefoil Knot}
        \label{fig:trefoil_knot_polygonal}
    \end{figure}
    While it is possible to define knots using \textit{continuous} embeddings,
    such a definition provides for the existence of
    \textit{wild knots} \cite{FoxArtinWildKnots1948}.\footnote{%
        See \cite{BrowneWildKnots} for how to procedurally generate wild knots.
    }
    \textit{Smooth} embeddings can similarly be used, albeit at the expense of
    a slightly harder definition of \textit{knot equivalence}. In the polygonal
    case this comes with ease. Given a piece-wise linear embedding
    $\gamma:\mathbb{S}^{1}\rightarrow\mathbb{R}^{3}$ with vertices
    $P_{1},\,\dots,\,P_{n}$ we are allowed to deform $\gamma$ into a new
    embedding $\gamma'$ formed by the (ordered) vertices
    $P_{1},\,\dots,\,P_{n},\,P'$
    such that the triangle $\Delta{P}_{n}P'P_{1}$ only intersects $\gamma$
    along the line segment $\overline{P_{1}P_{n}}$. This is called an
    \textbf{elementary knot equivalence}. With this we define equivalent knots.
    \begin{definition}[\textbf{Equivalent Knots}]
        Equivalent knots are polygonal embeddings
        $\gamma,\gamma':\mathbb{S}^{1}\rightarrow\mathbb{R}^{3}$ that differ by
        a finite number of elementary knot equivalences.
    \end{definition}
    Knot equivalence induces an equivalence relation on the set of all knots,
    the \textbf{knot type} of a knot
    $\gamma:\mathbb{S}^{1}\rightarrow\mathbb{R}^{3}$ is the equivalence class
    $K=[\gamma]$. Many texts do not distinguish between a knot and its knot
    type.
    \par\hfill\par
    While the polygonal definitions of knot and knot equivalence are perhaps
    the easiest to describe, it is also convenient to think of knots as smooth
    embeddings of $\mathbb{S}^{1}$ into $\mathbb{R}^{3}$. Indeed, most of the
    figures to follow will depict smooth knots. It is then fortunate that the
    two descriptions are equivalent for most purposes, and both exclude the
    existence of wild knots \cite[p.~147]{CrowellFoxKnotTheory}. As mentioned,
    the smooth setting has a more verbose definition of knot equivalence.
    Smooth embeddings $\gamma,\gamma':\mathbb{S}^{1}\rightarrow\mathbb{R}^{3}$
    are considered equivalent if there is an \textbf{ambient isotopy}
    between them, a smooth family of diffeomorphisms
    $H_{t}:\mathbb{R}^{3}\rightarrow\mathbb{R}^{3}$ such that
    $H_{0}|_{\mathbb{S}^{1}}=\gamma$ and $H_{1}|_{\mathbb{S}^{1}}=\gamma'$.%
    \footnote{%
        Here we take $\mathbb{S}^{1}\subseteq\mathbb{R}^{3}$
        to lie in the $xy$ plane.
    }
    \footnote{%
        The definition in \cite[p.~4]{CrowellFoxKnotTheory} only requires a
        single diffeomorphism $f:\mathbb{R}^{3}\rightarrow\mathbb{R}^{3}$ that
        takes the image of $\gamma$ to the image of $\gamma'$. Such a
        definition is unable to distinguish a knot from its \textit{mirror},
        the result of composing the knot with the reflection $z\mapsto{-z}$.
        I will not require a knot to be considered equivalent to it's mirror.
    }
    \par\hfill\par
    We often study knots via the use of \textbf{knot diagrams}. Given a knot
    $\gamma:\mathbb{S}^{1}\rightarrow\mathbb{R}^{3}$ and a vector
    $\mathbf{v}\in\mathbb{S}^{2}$, by composing $\gamma$ with the projection
    $\textrm{proj}_{\mathbf{v}}:\mathbb{R}^{3}\rightarrow\mathbb{R}^{2}$ we get
    a drawing in the plane. For a polygonal knot, a \textbf{regular} projection
    is one where for each $\mathbf{p}\in\mathbb{R}^{2}$ the pre-image consists
    of at most two points in $\mathbb{S}^{1}$, and such that vertices map to
    unique points in $\mathbb{R}^{2}$. For a smooth knot, a
    regular projection is again one such that the pre-image of each
    $\mathbf{p}\in\mathbb{R}^{2}$ consists of at most two points in
    $\mathbb{S}^{1}$, but also such that the resulting composition is an
    \textbf{immersion} from $\mathbb{S}^{1}$ into $\mathbb{R}^{2}$.
    \par\hfill\par
    For the general knot $\gamma$ and vector $\mathbf{v}\in\mathbb{S}^{2}$, the
    resulting projection may not be regular. Fortunately we have the following
    theorem \cite[p.~22]{LivingstonKnotTheory}.
    \begin{theorem}
        If $\gamma:\mathbb{S}^{1}\rightarrow\mathbb{R}^{3}$ is a
        (smooth or polygonal) knot, if $\mathbf{v}\in\mathbb{S}^{2}$, and if
        $\varepsilon>0$, then there is an equivalent knot
        $\gamma':\mathbb{S}^{1}\rightarrow\mathbb{R}^{3}$ such that:
        \begin{equation}
            \max\big(
                \{\,||\gamma(s)-\gamma'(t)||\;:\;\,s,t\in\mathbb{S}^{1}\,\}
            \big)<\varepsilon
        \end{equation}
        and such that the projection of $\gamma'$ along $\mathbf{v}$ is regular.
    \end{theorem}
    \begin{figure}
        \centering
        \includegraphics{trefoil_knot_001.pdf}
        \caption{Trefoil Knot Diagram}
        \label{fig:trefoil_knot_001}
    \end{figure}
    A regular projection of a knot is not enough to completely identify the
    embedding, we need to keep track of crossing information. This is done by
    leaving gaps in the curve to indicate which strand is \textit{under} and
    which is \textit{over}, as in Fig.~\ref{fig:trefoil_knot_001}. A
    knot projection with the over and under crossings is called a
    \textbf{knot diagram}.
    \par\hfill\par
    Alexander and Briggs \cite{AlexanderBriggs1926}, and independently
    Kurt Reidemeister \cite{Reidemeister1927}, proved in the 1920s that
    knot equivalence can be demonstrated using knot diagrams and finite
    sequences of \textbf{Reidemeister moves}, of which there are three types.
