%-----------------------------------LICENSE------------------------------------%
%   This file is part of Mathematics-and-Physics.                              %
%                                                                              %
%   Mathematics-and-Physics is free software: you can redistribute it and/or   %
%   modify it under the terms of the GNU General Public License as             %
%   published by the Free Software Foundation, either version 3 of the         %
%   License, or (at your option) any later version.                            %
%                                                                              %
%   Mathematics-and-Physics is distributed in the hope that it will be useful, %
%   but WITHOUT ANY WARRANTY; without even the implied warranty of             %
%   MERCHANTABILITY or FITNESS FOR A PARTICULAR PURPOSE.  See the              %
%   GNU General Public License for more details.                               %
%                                                                              %
%   You should have received a copy of the GNU General Public License along    %
%   with Mathematics-and-Physics.  If not, see <https://www.gnu.org/licenses/>.%
%----------------------------------Preamble------------------------------------%
\section{Basic Definitions}
    Following \cite[p.~15]{LivingstonKnotTheory}, we define a knot as follows.
    \begin{definition}[\textbf{Knot}]
        A knot is a polygonal (piece-wise linear) embedding
        $\gamma:\mathbb{S}^{1}\rightarrow\mathbb{R}^{3}$ of the unit circle
        into three dimensional Euclidean space.
    \end{definition}
    A triangle serves as a polygonal example of the unknot, and the trefoil is
    given a polygonal embedding in Fig.~\ref{fig:trefoil_knot_polygonal}.
    \begin{figure}
        \centering
        \includegraphics{trefoil_knot_polygonal.pdf}
        \caption{Polygonal Trefoil Knot}
        \label{fig:trefoil_knot_polygonal}
    \end{figure}
    While it is possible to define knots using \textit{continuous} embeddings,
    such a definition provides for the existence of
    \textit{wild knots} \cite{FoxArtinWildKnots1948}.\footnote{%
        See \cite{BrowneWildKnots} for how to procedurally generate wild knots.
    }
    \textit{Smooth} embeddings can similarly be used, albeit at the expense of
    a slightly harder definition of \textit{knot equivalence}. In the polygonal
    case this comes with ease. Given a piece-wise linear embedding
    $\gamma:\mathbb{S}^{1}\rightarrow\mathbb{R}^{3}$ with vertices
    $P_{0},\,\dots,\,P_{n-1}$ we are allowed to deform $\gamma$ into a new
    embedding $\gamma'$ formed by the (ordered) vertices
    $P_{0},\,\dots,\,P_{n-1},\,P_{n}$
    such that the triangle $\Delta{P}_{n-1}P_{n}P_{0}$ (interior included)
    only intersects $\gamma$
    along the line segment $P_{0}P_{n-1}$. This is called an
    \textbf{elementary knot equivalence}. With this we define equivalent knots.
    \newpage
    \begin{definition}[\textbf{Equivalent Knots}]
        Equivalent knots are polygonal embeddings
        $\gamma,\gamma':\mathbb{S}^{1}\rightarrow\mathbb{R}^{3}$ that differ by
        a finite number of elementary knot equivalences.
    \end{definition}
    A triangle and a square are equivalent knots, as are all regular
    polygons in the plane, which serve as examples of the unknot.
    Moreover, by the Jordan curve theorem any two embeddings that
    lie in a plane $S\subseteq\mathbb{R}^{3}$ are equivalent to the unknot as
    well. Application of Chazelle's algorithm
    \cite{ChazelleTriangulationAlgorithm} tells us this equivalence
    can be demonstrated in linear time.\footnote{%
        Explicitly, use Chazelle's algorithm to triangulate the polygonal
        figure. \textit{Collapse} any triangle with a vertex on the boundary
        of the polygon by deleting said vertex. Continue doing this until there
        is one triangle left. Both Chazelle's algorithm and this collapsing
        process run in linear time. Equivalence of two triangles in the
        plane can be done in constant time.
    }
    \par\hfill\par
    Knot equivalence induces an equivalence relation on the set of all knots,
    and this motivates a definition.
    \begin{definition}[\textbf{Knot Type}]
        The knot type of a knot $\gamma:\mathbb{S}^{1}\rightarrow\mathbb{R}^{3}$
        is the equivalence class $\mathcal{K}=[\gamma]$ of all knots that are
        equivalent to $\gamma$.
    \end{definition}
    Many texts do not distinguish between a knot and its knot type.
    While the polygonal definitions of knot and knot equivalence are perhaps
    the easiest to describe, it is also convenient to think of knots as smooth
    embeddings of $\mathbb{S}^{1}$ into $\mathbb{R}^{3}$. Indeed, most of the
    figures to follow will depict smooth knots. It is then fortunate that the
    two descriptions are equivalent for most purposes, and both exclude the
    existence of wild knots \cite[p.~147]{CrowellFoxKnotTheory}. As mentioned,
    the smooth setting has a more verbose definition of knot equivalence.
    Smooth embeddings $\gamma,\gamma':\mathbb{S}^{1}\rightarrow\mathbb{R}^{3}$
    are considered equivalent if there is an \textit{ambient isotopy}
    between them. The definition is quite precise.
    \newpage
    \begin{definition}[\textbf{Ambient Isotopy}]
        An ambient isotopy from a smooth embedding
        $\gamma:\mathbb{S}^{1}\rightarrow\mathbb{R}^{3}$ to a smooth
        embedding $\gamma':\mathbb{S}^{1}\rightarrow\mathbb{R}^{3}$ is a
        smooth function
        $H:\mathbb{R}^{3}\times[0,\,1]\rightarrow\mathbb{R}^{3}$
        such that for each $t\in[0,\,1]$ the function
        $H_{t}:\mathbb{R}^{3}\rightarrow\mathbb{R}^{3}$ defined by
        $H_{t}(\mathbf{x})=H(\mathbf{x},\,t)$ is a diffeomorphism,
        and such that:
        \begin{subequations}
            \begin{align}
                H_{0}\big((x,\,y,\,0)\big)&=\gamma\big((x,\,y)\big)\\
                H_{1}\big((x,\,y,\,0)\big)&=\gamma'\big((x,\,y)\big)
            \end{align}
        \end{subequations}
        for each $(x,\,y)\in\mathbb{S}^{1}$. Smoothness of $H$ is taken in
        the sense of the product smooth structure between a smooth manifold and
        a smooth manifold with boundary.
    \end{definition}
    An ambient isomorphism is a smooth family of diffeomorphisms that
    drags the first (smooth) knot to the second. A few comments on this.
    Alternative and inequivalent definitions do exist in the literature.
    The definition in \cite[p.~4]{CrowellFoxKnotTheory}, for example,
    requires a single diffeomorphism
    $f:\mathbb{R}^{3}\rightarrow\mathbb{R}^{3}$ that
    takes the image of $\gamma$ to the image of $\gamma'$. Such a
    definition is unable to distinguish a knot from its \textit{mirror},
    the result of composing the knot with the reflection $z\mapsto{-z}$.
    If one requires $f$ be \textit{orientation preserving}, then the
    two definitions are equivalent.
    \begin{definition}[\textbf{Smoothly Equivalent Knots}]
        Smoothly equivalent knots are smooth embeddings
        $\gamma:\mathbb{S}^{1}\rightarrow\mathbb{R}^{3}$ and
        $\gamma':\mathbb{S}^{1}\rightarrow\mathbb{R}^{3}$ such that there
        exists an ambient isotopy between them.
    \end{definition}
    As in the polygonal case, smooth equivalence yields an equivalence
    relation. Any smooth knot can be made polygonal by choosing a sufficiently
    small enough $\varepsilon>0$ and sampling the embedding such that no two
    successive points are more than $\epsilon$ apart, and then connecting
    the dots with line segments. By choosing $\epsilon$ small enough we ensure
    the topology of the complement of the embeddings are identical. This yields
    a bijection between the knot types of polygonal knots and the knot types
    of smooth knots, and we may freely swap between the two notions without
    ambiguity.
    \subsection{Knot Diagrams}
        Explicit embeddings are rarely given for smooth or polygonal knots.
        Instead we often study knots via the use of \textbf{knot diagrams}.
        Given a knot $\gamma:\mathbb{S}^{1}\rightarrow\mathbb{R}^{3}$ and a
        vector $\mathbf{v}\in\mathbb{S}^{2}$, by composing $\gamma$ with the
        projection
        $\textrm{proj}_{\mathbf{v}}:\mathbb{R}^{3}\rightarrow\mathbb{R}^{2}$ we
        get a drawing in the plane. It is possible for the projection to lose
        too much information, disabling us from identifying the knot from its
        \textit{shadow}. We call projections that do not do this
        \textit{regular}.
        \begin{definition}[\textbf{Regular Knot Projection}]
            A regular projection of a knot
            $\gamma:\mathbb{S}^{1}\rightarrow\mathbb{R}^{3}$ is a projection
            $\textrm{proj}_{\mathbf{v}}:\mathbb{R}^{3}\rightarrow\mathbb{R}^{2}$
            along a vector $\mathbf{v}\in\mathbb{S}^{2}$ such that for each
            $\mathbf{p}\in\mathbb{R}^{2}$ the pre-image consists of at most two
            points in the image of $\gamma$, and such that vertices of
            $\gamma$ map to unique points in $\mathbb{R}^{2}$.
        \end{definition}
        For a smooth knot a regular projection is again one such that the
        pre-image of each $\mathbf{p}\in\mathbb{R}^{2}$ consists of at most two
        points in the image of $\gamma$, but also such that any two
        $s,t\in\mathbb{S}^{1}$ that project to the same point do so
        \textit{transversally}
        \par\hfill\par
        For the general knot $\gamma$ and vector $\mathbf{v}\in\mathbb{S}^{2}$
        the resulting projection may not be regular. Fortunately we have the
        following theorem \cite[p.~22]{LivingstonKnotTheory}.
        \begin{theorem}
            If $\gamma:\mathbb{S}^{1}\rightarrow\mathbb{R}^{3}$ is a
            (smooth or polygonal) knot, if $\mathbf{v}\in\mathbb{S}^{2}$, and if
            $\varepsilon>0$, then there is an equivalent knot
            $\gamma':\mathbb{S}^{1}\rightarrow\mathbb{R}^{3}$ such that:
            \begin{equation}
                \max\big(
                    \{\,||\gamma(t)-\gamma'(t)||\;:\;\,t\in\mathbb{S}^{1}\,\}
                \big)<\varepsilon
            \end{equation}
            and such that the projection of $\gamma'$ along $\mathbf{v}$
            is regular.
        \end{theorem}
        \begin{figure}
            \centering
            \includegraphics{trefoil_knot_001.pdf}
            \caption{Trefoil Knot Diagram}
            \label{fig:trefoil_knot_001}
        \end{figure}
        A regular projection of a knot is not enough to completely identify the
        embedding, we need to keep track of crossing information. This is done
        by leaving gaps in the curve to indicate which strand is \textit{under}
        and which is \textit{over}, as in Fig.~\ref{fig:trefoil_knot_001}. A
        knot projection with the over and under crossings labeled is called a
        \textbf{knot diagram}.
        \par\hfill\par
        Alexander and Briggs \cite{AlexanderBriggs1926}, and independently
        Kurt Reidemeister \cite{Reidemeister1927}, proved in the 1920s that
        knot equivalence can be demonstrated using knot diagrams and finite
        sequences of \textbf{Reidemeister moves}, of which there are three
        types.
        \par\hfill\par
        \begin{figure}
            \centering
            \begin{minipage}[b]{0.4\textwidth}
                \centering
                \resizebox{\textwidth}{!}{%
                    \includegraphics{reidemeister_1_move.pdf}
                }
                \caption{Reidemeister I}
                \label{fig:reidemeister_1_move}
            \end{minipage}
            \hfill
            \begin{minipage}[b]{0.5\textwidth}
                \centering
                \resizebox{\textwidth}{!}{%
                    \includegraphics{reidemeister_2_move.pdf}
                }
                \caption{Reidemeister II}
                \label{fig:reidemeister_2_move}
            \end{minipage}
        \end{figure}
        \begin{figure}
            \centering
            \includegraphics{reidemeister_3_move.pdf}
            \caption{Reidemeister III}
            \label{fig:reidemeister_3_move}
        \end{figure}
        \textbf{Type I} (Fig.~\ref{fig:reidemeister_1_move}) allows us to undo
        \textit{loops}. That is, if we find a portion of a knot diagram in which
        a strand passes over itself, with no other strands of the knot being
        included, we may cancel out this crossing and reduce it to a straight
        line. This reduces the number of crossings in the diagram by one.
        \par\hfill\par
        \textbf{Type II} (Fig.~\ref{fig:reidemeister_2_move}) involves one
        strand passing over another twice. If we see see one strand go over
        another, and then again go over the same strand with no new crossings
        in between, we may \textit{pull} the bottom strand from underneath
        the upper one, reducing the number of crossings in the diagram by two.
        \par\hfill\par
        \textbf{Type III} (Fig.~\ref{fig:reidemeister_3_move}) is the most
        complicated, and has the unfortunate fact that it does not reduce the
        number of crossings in the diagram. If one strand passes entirely
        beneath two other strands that are making a crossing, we may pull the
        first strand below this crossing and on to the other side. In both
        diagrams we have three crossings drawn.
        \par\hfill\par
        In addition to these three types we are also allowed to use their
        \textit{mirrors}. We may also reverse these three processes by
        introducing loops or placing one strand on top of another. This is not
        a redundant task as will be made clearer after a definition.
        \begin{definition}[\textbf{Crossing Number of a Knot}]
            The crossing number of a knot is the minimum number of crossings
            possible for all equivalent diagrams of the knot.
        \end{definition}
        Repeated application of Type I and Type II may result in
        a knot diagram that has fewer crossings than the one you started with,
        but this may only be a \textit{local minimum}. To achieve the true
        crossing number you may first need to make the knot diagram
        \textit{more} complicated by introducing new crossings via the reverse
        of these two moves. Indeed, as we will discuss soon, you may need to
        make the knot diagram a \textit{lot} more complicated before you can
        make it simpler.
        \par\hfill\par
        As mentioned, the sufficiency of these three moves was proved in the
        1920s by Alexander, Briggs, and Reidemeister. It is a relatively new
        fact that all three are \textit{necessary}
        \cite{OstlundReidemeisterMoves2001},
        \cite{HaggeReidemeisterRequired2005}. We can easily prove the necessity
        of Type I by introducing the \textbf{writhe} of a knot diagram, which
        will be needed later for the Jones polynomial. To do this requires a
        discussion of \textbf{oriented knots} and \textbf{crossing signs}.
        \par\hfill\par
        The circle $\mathbb{S}^{1}$ is an orientable manifold. You can orient it
        clockwise or counterclockwise with respect to the $x$ axis. That is,
        start at the $x$ axis and travel to the negative $y$ axis
        (clockwise), or travel to the positive $y$ axis (counterclockwise).
        We can copy our chosen orientation onto an embedding of
        $\mathbb{S}^{1}$ into $\mathbb{R}^{3}$, giving us an oriented knot.
        \begin{definition}[\textbf{Oriented Knot}]
            An oriented Knot is a knot
            $\gamma:\mathbb{S}^{1}\rightarrow\mathbb{R}^{3}$ equipped with an
            orientation, or a chosen direction.
        \end{definition}
        If we project an oriented knot onto a knot diagram we can make note of
        the orientation by marking the diagram with arrows indicating the
        directions at several points.\footnote{%
            Technically one arrow is sufficient, but this may cause your
            reader to become annoyed.
        }
        This will be useful for several constructions.
        \begin{definition}[\textbf{Oriented Knot Diagram}]
            An oriented knot diagram is a knot diagram of an oriented knot
            where the orientation of the knot is indicated by directed arrows.
        \end{definition}
        The trefoil is given an oriented knot diagram in
        Fig.~\ref{fig:trefoil_knot_oriented_001}. While slightly artificial (it
        is always possible to give a knot one of the two possible
        orientations), we will use oriented knot diagrams to describe
        things like Gauss codes, writhe, and the Jones polynomial, so the
        definition is not entirely useless. Indeed, orientations give rise to
        the notion of \textbf{invertible knots}, embeddings
        $\gamma:\mathbb{S}^{1}\rightarrow\mathbb{R}^{3}$ that are equivalent to
        $\tilde{\gamma}:\mathbb{S}^{1}\rightarrow\mathbb{R}^{3}$ with
        $\tilde{\gamma}\big((x,\,y)\big)=\gamma\big((x,\,-y)\big)$. That is,
        knots that are equivalent to the same knot but with the orientation
        reversed. The existence of \textit{non-invertible knots} was proven in
        1963 by Hale Trotter (1931-2022 C.E.)
        \cite{TrotterInvertibleKnots1963}.\footnote{%
            Alexander and Briggs mention in
            \cite[p.~563-564]{AlexanderBriggs1926} that a knot need not
            be equivalent to its inverse, but do not provide any examples.
        }
        \par\hfill\par
        \begin{figure}
            \centering
            \includegraphics{trefoil_knot_oriented_001.pdf}
            \caption{Oriented Trefoil}
            \label{fig:trefoil_knot_oriented_001}
        \end{figure}
        Once an orientation is given, we may discuss the \textbf{sign} of a
        crossing. A rigorous definition can be given for smooth embeddings using
        linear algebra.
        \begin{definition}[\textbf{Sign of a Crossing}]
            The sign of a crossing in an oriented knot diagram is the sign of
            the determinant of the matrix $[\dot{\gamma}(s)\;\dot{\gamma}(t)]$
            formed by the velocity vectors of $\gamma$ at the unique points
            $s,t\in\mathbb{S}^{1}$ that project to the crossing, $s$
            representing the over strand and $t$ representing the under strand.
        \end{definition}
        For the sake of being well-defined, and for the sake of computation, the
        notions \textit{under} and \textit{over} need to be recoverable from the
        embedding $\gamma$ and the vector $\mathbf{v}$ defining the projection.
        For $s,t\in\mathbb{S}^{1}$ projecting to the same point along
        $\mathbf{v}$ the point whose image under $\gamma$ has a greater
        component along $\mathbf{v}$ is the over strand, and the lesser
        is the under strand. This makes sense if one thinks of projecting
        along $\mathbf{v}$ as an orthographic operation, placing an observer
        infinitely far away in the direction of $\mathbf{v}$. The strand
        with greater component along $\mathbf{v}$ is the one such an
        observer would see as the over strand. Note that since $\gamma$ is
        an embedding the components of $\gamma(s)$ and $\gamma(t)$ along
        $\mathbf{v}$ will differ, otherwise we'd have $\gamma(s)=\gamma(t)$.
        \par\hfill\par
        Since knot diagrams are required to produce transversal crossings, the
        velocity vectors will span the plane and the determinant will be
        non-zero. The sign is thus plus or minus one. In principle we now have
        a straight-forward way of computing the sign of any crossing in a
        diagram, but this can be made far more visual.
        Given an oriented knot diagram $K$ and any crossing in our
        diagram, we rotate our heads until the \textit{forward} directions for
        both strands are pointing upwards in the plane
        (Fig.~\ref{fig:crossing_signs}). If left-to-right passes over
        right-to-left we call the crossing \textit{positive}. Otherwise,
        if right-to-left passes over left-to-right, we call it negative.
        The sign of the crossing is the value $-1$ or $+1$, depending on if
        the crossing is negative or positive. Being integer valued we may sum
        the signs of all crossings of a knot diagram.
        \begin{figure}
            \centering
            \includegraphics{crossing_signs.pdf}
            \caption{Signs of Crossings}
            \label{fig:crossing_signs}
        \end{figure}
        \begin{definition}[\textbf{Writhe of a Knot Diagram}]
            The writhe of an oriented knot diagram is the sum of the signs of
            all of the crossings in the diagram.
        \end{definition}
        It is worth noting that if one were to reverse the orientation the
        crossings in Fig.~\ref{fig:crossing_signs} will flip. After turning our
        heads 180 degrees (carefully as to not fracture ones neck) we will
        find that the image does not change. Using the rigorous definition this
        can also be seen since the mapping $(x,\,y)\mapsto(-x,\,-y)$ is
        orientation preserving, being given by a rotation, and does not change
        the sign of the original determinant. Hence the writhe is independent of
        the orientation and is dependent only on the knot diagram itself. It is
        not an invariant of the knot. While unchanged by the second and third
        moves, Type I moves alter the writhe of the diagram by $+1$ or
        $-1$. By introducing loops we see that the writhe a knot diagram $K$
        can be any integer for a given knot type $\mathcal{K}$.
        Since Type II and Type III do not
        alter writhe, we see that the Type I move is required.
    \subsection{Links}
        We now generalize knots slightly and discuss \textbf{links}.
        \begin{definition}[\textbf{Link}]
            A link with $N\in\mathbb{N}$ components is a polygonal embedding of
            $\sqcup_{n=0}^{N-1}\mathbb{S}^{1}$ into $\mathbb{R}^{3}$.
            That is, $N$ distinct knots
            $\gamma_{n}:\mathbb{S}^{1}\rightarrow\mathbb{R}^{3}$ such that for
            all $m\ne{n}$ we have
            $\gamma_{m}[\mathbb{S}^{1}]\cap\gamma_{n}[\mathbb{S}^{1}]=\emptyset$.
        \end{definition}
        \begin{figure}
            \centering
            \includegraphics{hopf_link_diagram.pdf}
            \caption{Hopf Link}
            \label{fig:hopf_link_diagram}
        \end{figure}
        We consider $0\in\mathbb{N}$ to be valid and allow zero component links.
        This can be useful for computational purpose, for example in the
        discussion of the Kauffman bracket. $N=1$ is also allowed, in which
        case we have a knot. The simplest non-trivial link is the
        \textit{Hopf link} which consists of two components linked together
        with two crossings (Fig.~\ref{fig:hopf_link_diagram}).
        This link can be found in Jewish\footnote{%
            The Josefov Jewish Council building in Prague contains the
            Hopf link in the form of the star of David \cite{KatlasHopfLink},
            similar to Fig.~\ref{fig:hopf_link_star_of_david}.
        }
        and Japanese\footnote{%
            The \textit{Chigai Kuginuki}, the emblem, or \textit{mon},
            of the As\~{o} family depicts a Hopf link \cite{KatlasHopfLink}
            similar to the one shown in Fig.~\ref{fig:hopf_link_chigai_kuginuki}.
        }
        cultures, and almost certainly others. Another celebrated example is
        the Whitehead link\footnote{%
            Named after J. H. C. Whitehead who used the link in his
            construction of the Whitehead
            manifold, a contractible 3-manifold without boundary that is not
            homeomorphic to Euclidean 3-space \cite{WhiteheadManifold}.
            Whitehead had hoped the non-existence of such an object could be
            used to prove the Poincar\'{e} conjecture
            \cite{WhiteheadPoincareConjecture}.
        }
        (Fig.~\ref{fig:whitehead_link}) which can be found on the
        hammer Mjolnir of the Norse god Thor
        \cite[p.~309]{MonteliusMjolnirHopfLink}.
        \begin{figure}
            \centering
            \begin{minipage}[b]{0.49\textwidth}
                \centering
                \resizebox{\textwidth}{!}{%
                    \includegraphics{hopf_link_star_of_david.pdf}
                }
                \caption{Star of David}
                \label{fig:hopf_link_star_of_david}
            \end{minipage}
            \hfill
            \begin{minipage}[b]{0.49\textwidth}
                \centering
                \resizebox{\textwidth}{!}{%
                    \includegraphics{hopf_link_chigai_kuginuki.pdf}
                }
                \vspace{2em}
                \caption{Chigai Kuginuki}
                \label{fig:hopf_link_chigai_kuginuki}
            \end{minipage}
        \end{figure}
        \begin{figure}
            \centering
            \resizebox{\textwidth}{!}{%
                \includegraphics{whitehead_link_heraldic.pdf}
            }
            \caption{Whitehead Link}
            \label{fig:whitehead_link}
        \end{figure}
        \par\hfill\par
        Many notions for knots may be mimicked for links. In particular we may
        consider \textit{link equivalence}, \textit{link types},
        \textit{regular link projections}, and \textit{link diagrams}, and we
        may perform all of this in the smooth setting where the embeddings are
        done smoothly. We may also orient our links and define the
        writhe of an oriented link
        diagram. One must be careful in the computation of writhe. Unlike the
        one-component knot setting, where the writhe is independent of the
        orientation, for link diagrams the writhe is indeed orientation
        dependent. In general, given an $N\in\mathbb{N}$ component link,
        there are $2^{N}$ possible ways to orient it. If we order the
        components and denote $0$ for a clockwise orientation and $1$ for a
        counterclockwise one, we obtain a string on $N$ characters where the
        $k^{\small\textrm{th}}$ entry corresponds to the orientation of the
        $k^{\small\textrm{th}}$ component. The writhe associated to this
        combination of orientations will be the same as the writhe of
        \textit{complement} of this string, the string obtained by swapping all
        zeros with ones and vice-versa, by an argument identical to the
        one-component knot case. So there are at most $2^{N-1}$ possible writhes
        for a link, and this is sharp. The Hopf link in
        Fig.~\ref{fig:hopf_link_diagram} can take on writhes of $+2$ and $-2$,
        pending the orientation given. This will cause a slight inconvenience
        in our definition of the Jones polynomial.
