%-----------------------------------LICENSE------------------------------------%
%   This file is part of Mathematics-and-Physics.                              %
%                                                                              %
%   Mathematics-and-Physics is free software: you can redistribute it and/or   %
%   modify it under the terms of the GNU General Public License as             %
%   published by the Free Software Foundation, either version 3 of the         %
%   License, or (at your option) any later version.                            %
%                                                                              %
%   Mathematics-and-Physics is distributed in the hope that it will be useful, %
%   but WITHOUT ANY WARRANTY; without even the implied warranty of             %
%   MERCHANTABILITY or FITNESS FOR A PARTICULAR PURPOSE.  See the              %
%   GNU General Public License for more details.                               %
%                                                                              %
%   You should have received a copy of the GNU General Public License along    %
%   with Mathematics-and-Physics.  If not, see <https://www.gnu.org/licenses/>.%
%----------------------------------Preamble------------------------------------%
\begin{figure}
    \centering
    \resizebox{\textwidth}{!}{%
        \includegraphics{endless_knot_celtic_style.pdf}
    }
    \caption{The Endless Knot}
    \label{fig:endless_knot_celtic_style}
\end{figure}
\begin{figure}
    \centering
    \resizebox{\textwidth}{!}{%
        \includegraphics{basket_weave_knot_celtic_style.pdf}
    }
    \caption{The Basket Weave Knot}
    \label{fig:basket_weave_knot_celtic_style}
\end{figure}
\begin{figure}
    \centering
    \resizebox{!}{0.4\textheight}{%
        \includegraphics{borromean_rings_tricursal_valknut.pdf}
    }
    \caption{The Tricursal Valknut}
    \label{fig:borromean_rings_tricursal_valknut}
\end{figure}
Knots have long been marveled as a source of art and beauty. In the Book of
Kells, a Celtic work containing the four gospels of the New Testament created
sometime between the the $7^{\small\textrm{th}}$ and $9^{\small\textrm{th}}$
centuries \cite[p.~108]{Nordenfalk1977}, intricate drawings of complicated
knots and links are found. Many pages contained depictions of the endless knot
(Fig.~\ref{fig:endless_knot_celtic_style}) and the basket weave knot
(Fig.~\ref{fig:basket_weave_knot_celtic_style}). The endless knot also appears
in Tibetan Buddhism, being one of the ``eight auspicious symbols''
\cite[p.~11]{BeerTibetanSymbols}. The Borromean rings
(Fig.~\ref{fig:borromean_rings_no_shadow}) appear in Celtic, Tibetan,
and Viking cultures
\cite[p.~129]{VikingWomenJesch}%
\footnote{%
    The Legend of Hildr is depicted on the stone carving in this
    reference. The tricursal \textit{Valknut}
    (Fig.~\ref{fig:borromean_rings_tricursal_valknut}),
    the Viking-Germanic version of Borromean rings,
    can be seen on the third carving from the top.
},
and the trefoil (Fig.~\ref{fig:trefoil_knot})
appears in Celtic, Islamic, Norse, and Tibetan art as well.
\par\hfill\par
As a mathematical discipline, the origins of knot theory
date back to the $18^{\small\textrm{th}}$ and
$19^{\small\textrm{th}}$ centuries with semi-rigorous treaties of
the subject being formed by Vandermonde in 1771 \cite{Vanermonde1771}
and Gauss in 1833 \cite[p.~1327]{RiccaNipotaGaussLinkingNumber}.
Serious investigations into the field are often first attributed to Peter Tait
who established some of the earliest tabulations of knots and put forward
several conjectures in a treatise published in 1885. Tait's motivation was
primarily the conjecture of Lord Kelvin who believed that chemical properties
of matter could be explained by atoms being \textit{knotted}, in some sense.
J.J. Thompson expanded this idea and developed some of the earlier mathematical
properties of knots, only to abandon the hypothesis altogether with his
discovery of the electron. An so vortex theory died, but knot theory lives on.
\begin{figure}
    \centering
    \resizebox{!}{0.4\textheight}{%
        \includegraphics{borromean_rings_no_shadow.png}
    }
    \caption{The Borromean Rings}
    \label{fig:borromean_rings_no_shadow}
\end{figure}
