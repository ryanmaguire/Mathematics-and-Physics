%-----------------------------------LICENSE------------------------------------%
%   This file is part of Mathematics-and-Physics.                              %
%                                                                              %
%   Mathematics-and-Physics is free software: you can redistribute it and/or   %
%   modify it under the terms of the GNU General Public License as             %
%   published by the Free Software Foundation, either version 3 of the         %
%   License, or (at your option) any later version.                            %
%                                                                              %
%   Mathematics-and-Physics is distributed in the hope that it will be useful, %
%   but WITHOUT ANY WARRANTY; without even the implied warranty of             %
%   MERCHANTABILITY or FITNESS FOR A PARTICULAR PURPOSE.  See the              %
%   GNU General Public License for more details.                               %
%                                                                              %
%   You should have received a copy of the GNU General Public License along    %
%   with Mathematics-and-Physics.  If not, see <https://www.gnu.org/licenses/>.%
%----------------------------------Preamble------------------------------------%
\begin{figure}
    \centering
    \begin{minipage}[b]{0.49\textwidth}
        \centering
        \resizebox{\textwidth}{!}{%
            \includegraphics{endless_knot_celtic_style.pdf}
        }
        \caption{Endless Knot}
        \label{fig:endless_knot_celtic_style}
    \end{minipage}
    \hfill
    \begin{minipage}[b]{0.49\textwidth}
        \centering
        \resizebox{\textwidth}{!}{%
            \includegraphics{basket_weave_knot_celtic_style.pdf}
        }
        \vspace{1.2em}
        \caption{Basket Weave Knot}
        \label{fig:basket_weave_knot_celtic_style}
    \end{minipage}
\end{figure}
\begin{figure}
    \centering
    \begin{minipage}[b]{0.49\textwidth}
        \centering
        \resizebox{\textwidth}{!}{%
            \includegraphics{borromean_rings_tricursal_valknut.pdf}
        }
        \caption{Tricursal Valknut}
        \label{fig:borromean_rings_tricursal_valknut}
    \end{minipage}
    \hfill
    \begin{minipage}[b]{0.49\textwidth}
        \centering
        \resizebox{\textwidth}{!}{%
            \includegraphics{borromean_rings_no_shadow.png}
        }
        \caption{Tubular Borromean Rings}
        \label{fig:borromean_rings_no_shadow}
    \end{minipage}
\end{figure}
\begin{figure}
    \centering
    \begin{minipage}[b]{0.49\textwidth}
        \centering
        \resizebox{\textwidth}{!}{%
            \includegraphics{trefoil_knot_celtic_style.pdf}
        }
        \caption{Celtic Trefoil}
        \label{fig:trefoil_knot_celtic_style}
    \end{minipage}
    \begin{minipage}[b]{0.49\textwidth}
        \centering
        \resizebox{\textwidth}{!}{%
            \includegraphics{trefoil_unicursal_valknut.pdf}
        }
        \caption{Unicursal Valknut}
        \label{fig:trefoil_unicursal_valknut}
    \end{minipage}
\end{figure}
Knots have long been marveled as a source of art and beauty. The Book of
Kells, a Celtic rendition of the four gospels of the New Testament created
between the $7^{\small\textrm{th}}$ and $9^{\small\textrm{th}}$
centuries \cite[p.~108]{Nordenfalk1977}, contains intricate drawings of
complicated knots and links. Many pages depict the endless knot
(Fig.~\ref{fig:endless_knot_celtic_style}) and the basket weave knot
(Fig.~\ref{fig:basket_weave_knot_celtic_style}). The endless knot also appears
in Tibetan Buddhism, being one of the ``eight auspicious symbols''
\cite[p.~11]{BeerTibetanSymbols}. The Borromean rings
(Fig.~\ref{fig:borromean_rings_no_shadow}) are found in Celtic, Tibetan,
and Viking cultures
\cite[p.~129]{VikingWomenJesch}%
\footnote{%
    The Legend of Hildr is depicted on the stone carving in this
    reference. The tricursal \textit{Valknut}
    (Fig.~\ref{fig:borromean_rings_tricursal_valknut}),
    the Viking-Germanic version of Borromean rings,
    can be seen on the third carving from the top.
},
and the trefoil in Celtic (Fig.~\ref{fig:trefoil_knot_celtic_style}),
Islamic, Norse (Fig.~\ref{fig:trefoil_unicursal_valknut}), and Tibetan art.
\par\hfill\par
As a mathematical discipline, the origins of knot theory
date back to the $18^{\small\textrm{th}}$ and
$19^{\small\textrm{th}}$ centuries with semi-rigorous treaties of
the subject being formed by Vandermonde (1735-1796 C.E.) in 1771
\cite{Vanermonde1771} and Gauss
(1777-1855 C.E.) in 1833 \cite[p.~1327]{RiccaNipotaGaussLinkingNumber}.
James Clerk Maxwell (1831-1879 C.E.) reinterpreted Gauss' work for the theory
of electromagnetism in his seminal 1873 work
\textit{A Treatise on Electricity and Magnetism}
\cite{MaxwellTreatist1873} and made a few contributions to link theory as well.
\par\hfill\par
Serious investigations into the field are often first attributed to Peter Tait
(1831-1901 C.E.) who established some of the earliest tabulations of knots
in 1885 \cite{TaitOnKnots1885}. Several of his methods are still used today,
and the \textit{Tait graph} will occupy one of the later sections.
In his three treatise on knots he also put forward the famous
\textit{Tait conjectures}, all three of which were proven in the 1980s and
1990s (see \cite{MurasugiJonesPolynomial}, \cite{KauffmanStateModels},
\cite{ThistlethwaiteSpanningTree}, \cite{ThistlethwaiteKauffmanPolynomial},
and \cite{ThistlethwiateMenascoAlternatingLinks}).

\par\hfill\par
Tait was motivated by Lord Kelvin's hypothesis that chemical properties
of matter could be explained by atoms being \textit{knotted}
\cite{ThompsonVortex1867}, in some sense. J.J. Thompson expanded this idea and
developed some of the earlier mathematical properties of knots
\cite{ThompsonVortexRings1883}, only to abandon the hypothesis altogether with
his discovery of the electron \cite{ThompsonStructureOfAtoms1904}.\footnote{%
    This article also contains the famous \textit{Thompson problem}, asking for
    the minimum equilibrium configuration for $n$ equally charged particles
    constrained to a lie on a sphere.
}
And so vortex theory died, but knot theory lived on as it became a part of
\textit{topology}, which was brought to the main stage of mathematics
by Henri Poincar\'{e}'s famous \textit{analysis situs}
\cite{PoincareAnalysisSitus1895}. Within the following decades mathematicians
such as James Alexander (1888-1971 C.E.), R. H. Bing (1914-1986 C.E.),
Max Dehn (1878-1952 C.E.), Heinz Hopf (1894-1971 C.E.),
Kurt Reidemeister (1893-1971 C.E.), Herbert Seifert (1907-1996 C.E.),
and John Whitehead (1904-1960 C.E.) began making serious and rigorous strides
into the theory.
