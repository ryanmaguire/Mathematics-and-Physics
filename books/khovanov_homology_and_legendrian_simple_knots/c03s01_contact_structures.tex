\section{Contact Structures}
    Continuing with our motivations from classical mechanics, we now discuss
    contact structures. If we consider the Lagrangian $\mathcal{L}$ of a
    mechanical system, Newton's laws tell us the \textit{Euler-Lagrange}
    equation will be satisfied:
    \begin{equation}
        \frac{\partial\mathcal{L}}{\partial\mathbf{v}}
        =\frac{\textrm{d}}{\textrm{d}t}
        \frac{\partial\mathcal{L}}{\partial\dot{\mathbf{v}}}
    \end{equation}
    where again we are working with phase space coordinate
    $(\mathbf{p},\,\mathbf{q})$. The Hamiltonian $\mathcal{H}$ is defined as
    the quantity emanating from this that satisfies the following system of
    equations:
    \begin{align}
        \dot{\mathbf{p}}&=-\frac{\partial\mathcal{H}}{\partial\mathbf{v}}\\
        \dot{\mathbf{v}}&=\frac{\partial\mathcal{H}}{\partial\mathbf{p}}
    \end{align}
    The \textit{de facto} standard introduction to this study is
    \cite{GoldsteinClassicalMechanics}. See chapter 8.
    \par\hfill\par
    $\mathcal{H}$ may be thought of as a smooth function from phase space to
    the real numbers. We may rephrase this to a smooth function from the
    cotangent bundle of $\mathbb{R}^{3}$ to the real numbers using the
    canonical dual basis expansion for the dual of $\mathbb{R}^{3}$. Given an
    initial starting condition, conservation of energy tells us the
    time-derivative of $\mathcal{H}$ is zero, and so the particle will be
    constrained to lie in the hypersurface in $\mathbb{R}^{6}$ with the same
    starting value for $\mathcal{H}$. This hypersurface is co-dimension 1 and
    the properties of it are what we wish to axiomatize for contact manifolds.
    \par\hfill\par
    We start with a smooth manifold $(X,\mathcal{A})$ of dimension $2n+1$.
    We select from $TX$ a family of co-dimension 1 hyperplanes,
    one for each point in $X$,
    that satisfy a \textit{maximal non-integrability} condition,
    the antonym of the complete integrability conditions for the Frobenius
    theorem. We require that for any smooth submanifold $S$
    (open, closed, with boundary, without boundary, etc.)
    of dimension greater than $n$ that there is at least one point
    $x\in{S}$ such that $T_{x}S$ is not given by the chosen hyperplane in $TX$
    with base point $p$. That is, there is no submanifold of dimension greater
    than $n$ that is everywhere tangent to our selection of hyperplanes.
    \par\hfill\par
    The maximal non-integrability condition translates locally to the existence
    of a particular 1-form. Given $p\in{X}$, a co-dimension 1 hyperplane in
    $T_{x}X$ can be described as the kernel of a linear functional
    $\alpha_{x}:T_{x}X\rightarrow\mathbb{R}$. Locally in a smooth chart
    $(\mathcal{U},\,\varphi)\in\mathcal{A}$ about $x$, we obtain a 1-form
    $\alpha:T\mathcal{U}\rightarrow\mathbb{R}$. The maximal non-integrability
    condition translates to the following equation:
    \begin{equation}
        \alpha\land(\textrm{d}\alpha)^{n}\ne{0}
    \end{equation}
    where
    $(\textrm{d}\alpha)^{n}=\textrm{d}\alpha\land\cdots\land\textrm{d}\alpha$
    performed $n$ times. This equation yields the definition of a contact
    manifold.
    \begin{definition}[\textbf{Contact Manifold}]
        A contact manifold is an ordered triple $(X,\,\mathcal{A},\,\alpha)$
        such that $(X,\,\mathcal{A})$ is a smooth manifold, and
        $\alpha$ is a family of locally defined 1-forms whose domains of
        definition cover $X$ and $\alpha\land(\textrm{d}\alpha)^{n}\ne{0}$.
    \end{definition}
    The definition is a slight abuse of notation since we've used $\alpha$ to
    describe both a \textit{family} of 1-forms, and the individual 1-forms
    themselves, but the practice is fairly common.
    \par\hfill\par
    The canonical examples of symplectic manifolds are provided by the
    cotangent bundle of a smooth manifold. In a similar vein the
    \textit{spherical cotangent bundle}, $ST^{*}M$, has a canonical
    1-form associated to it (it is the symplectic form $\omega$ restricted to
    $ST^{*}M$). The spherical cotangent bundle has 2 common means
    of construction. Firstly, there is an action of the positive
    real number $\mathbb{R}^{+}$ on the non-zero cotangent vectors in
    $T^{*}_{x}X$ for any point $x\in{X}$, namely
    $(r,\,\phi)\mapsto{r}\phi$ for covector $\phi\in{T}^{*}_{x}X$ and
    $r\in\mathbb{R}^{+}$. Quotient out by this action by considering two
    elements $(p_{0},\,\phi_{0})$ and $(p_{1},\,\phi_{1})$ equivalent if
    $p_{0}=p_{1}$ and $\phi_{0}$ and $\phi_{1}$ differ by a positive multiple.
    This is the spherical cotangent bundle.
    \par\hfill\par
    More naturally, if $(X,\,\mathcal{A})$ is equipped with a Riemannian metric
    $g$ (which is always possible for smooth manifolds), then there is a
    canonical isomorphism between $TX$ and $T^{*}X$. Namely, given a point
    $(p,\,v)$ with $v\in{T}_{p}X$, we define
    $v^{*}(w)=g_{p}(v,\,w)$. The bilinearity of Riemannian metrics tells us
    that $v^{*}$ is a covector at $p$, and indeed this mapping is bijective
    and linear (again, guaranteed since $g$ is a Riemannian metric). Under this
    identification we may consider the spherical cotangent bundle to be
    precisly the spherical tangent bundle $STX$, which is the set of elements
    $(p,\,v)\in{TX}$ satisfying $g_{p}(v,\,v)=1$. The canonical symplectic
    2-form $\omega$ on $T^{*}X$ restricts to a 1-form $\alpha$ on
    $ST^{*}X$, and this will indeed satisfy the maximal non-integrability
    conditions. The connection between classical mechanics is made immediate
    as one can identify the spherical cotangent bundle as a hypersurface of
    constant kinetic energy since:
    \begin{equation}
        K(v)=\frac{1}{2}mg_{p}(v,\,v)
    \end{equation}
    We can be quite explicit here. Choosing a chart
    $(\mathcal{U},\,\varphi)$ for $T^{*}M$ with
    $\varphi=(x_{0},\,\dots,\,x_{n-1},\,v_{0},\,v_{n-1})$ and with the
    Riemannian metric $g$ represented by a matrix with components
    $g_{i,\,j}$, the \textit{Liouville} form becomes:
    \begin{equation}
        \omega=\sum_{i=1}^{n}\sum_{j=1}^{n}
            \tilde{g}_{i,\,j}\textrm{d}x_{i}\land\textrm{d}v_{j}+
        \sum_{k=1}^{n}\sum_{i=1}^{n}\sum_{j=1}^{n}
            \frac{\partial\tilde{g}_{i,\,j}}{\partial{x}_{k}}
            v_{i}\textrm{d}x_{j}\land\textrm{d}x_{k}
    \end{equation}
    where $\tilde{g}$ is the isomorphism between $TM$ and $T^{*}M$ given by
    the Riemannian metric $g$.
    Restricting this to $ST^{*}M$ yields the contact form $\alpha$.
    \par\hfill\par
    The main theorem for contact manifolds tells us two things. Firstly,
    they the locally \textit{boring} in that any two contact structures look
    identical in a small enough neighborhood. Contact topology is thus a global
    phenomenon. Secondly, and very conveniently, it tells us how to
    \textit{draw} contact manifolds.
    \begin{theorem}[\textbf{Darboux's Theorem}]
        If $(X,\,\mathcal{A},\,\alpha)$ is a $2n+1$ dimensional
        contact manifold, and if $x\in{X}$, then there is a smooth chart
        $(\mathcal{U},\,\varphi)\in\mathcal{A}$ such that $x\in\mathcal{U}$
        and:
        \begin{equation}
            \alpha=
            \sum_{k=1}^{n-1}
            \textrm{d}\varphi_{2k}-\varphi_{2k+1}\textrm{d}\varphi_{0}
        \end{equation}
        where $\varphi_{k}$ is the $k^{\small\textrm{th}}$ component of
        $\varphi:\mathcal{U}\rightarrow\mathbb{R}^{2n+1}$ and
        $\textrm{d}\varphi_{k}$ is the associated 1-form on $\mathcal{U}$.
    \end{theorem}
    Three dimensional Euclidean space can be given a global contact structure
    via:
    \begin{equation}
        \label{eqn:euclidean_contact_form}
        \alpha=\textrm{d}z-y\textrm{d}x
    \end{equation}
    By the Darboux theorem any other
    contact structure must \textit{locally} look exactly the same.\footnote{%
        Globally contact structures can look quite different and the
        classification of such forms is an active area of research.
    }
    Most importantly for the sake of visualization we now have a means of
    drawing the contact structure. The kernel of $\alpha$ at $(x,\,y,\,z)$ is a
    2-dimensional plane in $\mathbb{R}^{3}$ and the Darboux form gives us the
    spanning vectors. Name, we seek linearly independent vectors $v_{0},v_{1}$
    starting at $(x,\,y,\,z)$ with $\alpha(v_{k})=0$, $k=0,\,1$.
    $\partial{y}$ is one such vector since
    $\alpha$ does not contain $\textrm{d}y$. Almost as easy, by swapping $x$
    and $z$ in the formula we see that $\partial{x}+y\partial{z}$ is in the
    kernel of $\alpha$ as well. The vectors $(1,\,0,\,y)$ and $(0,\,1,\,0)$
    are always linearly independent, and so we have found a basis for the
    kernel of $\alpha$ at each point $(x,\,y,\,z)\in\mathbb{R}^{3}$.
    Drawing this yields Fig.~\ref{fig:darboux_form_001}. This is the
    \textbf{Euclidean contact structure}, and as we discuss the interplay
    between knot theory and contact topology we shall always implicitly have
    this contact form in mind.
    \begin{figure}
        \centering
        \resizebox{\textwidth}{!}{%
            \includegraphics{darboux_form_001}
        }
        \caption{Standard Euclidean Contact Structure}
        \label{fig:darboux_form_001}
    \end{figure}
    \par\hfill\par
    The maximal non-integrability condition tells us there is no 2-dimensional
    surface $S\subseteq\mathbb{R}^{3}$ nor 3-dimensional submanifold
    $\mathcal{U}\subseteq\mathbb{R}^{3}$ where the tangent space at each point
    is given by the Darboux form. It is, however, possible for 1-dimensional
    submanifolds (i.e., knots) to be everywhere tangent. More generally, for
    a $2n+1$ dimensional contact manifold $(X,\,\mathcal{A},\,\alpha)$ it is
    possible for there to exist $n$ dimensional submanifolds
    $S\subseteq{X}$ that are everywhere tangent to hyperplane distribution
    given by $\alpha$. Such submanifolds
    are called \textbf{Legendrian submanifolds}.\footnote{%
        The symplectic analogues are called \textbf{Lagrangian} submanifolds.
    }
    \par\hfill\par
    In the case of $n=1$ we have 1-dimensional submanifolds of a 3-dimensional
    manifold. Excluding manifolds with boundary, such a submanifold is
    intrinsically homeomorphic to either $\mathbb{R}$ or $\mathbb{S}^{1}$.
    In the latter case we call these \textbf{Legendrian knots}
    (see \cite{JoshuaMSabloffWhatIsLegendrianKnot} for a nice introduction).
    Any knot can be perturbed into a Legendrian embedding
    (see the introduction of \cite{VeraVertessiTransNonSimpleKnots}) and
    Eqn.~\ref{eqn:euclidean_contact_form} gives us much insight into the
    layout of such embeddings. Suppose
    $\gamma:\mathbb{R}\rightarrow\mathbb{R}^{3}$ is a curve such that
    $\alpha(\dot{\gamma}(t))=0$ for all $t\in\mathbb{R}$ where $\alpha$
    is the Euclidean contact form from Eqn.~\ref{eqn:euclidean_contact_form}.
    Writing $\gamma(t)=(x(t),\,y(t),\,z(t))$, and slightly abusing notation,
    this says:
    \begin{subequations}
        \label{eqn:euclidean_legendrian_knot_relations}
        \begin{align}
            \textrm{d}z(t)-y(t)\textrm{d}x(t)
            &=0\\
            \Rightarrow
            \frac{\textrm{d}z}{\textrm{d}t}(t)-
            y(t)\frac{\textrm{d}x}{\textrm{d}t}(t)
            &=0\\
            \Rightarrow
            \frac{\textrm{d}z/\textrm{d}t}{\textrm{d}x/\textrm{d}t}(t)
            &=y(t)\\
            \Rightarrow
            \frac{\textrm{d}z}{\textrm{d}x}(t)
            &=y(t)
        \end{align}
    \end{subequations}
    This tells us Legendrian knots have two degrees of freedom, as opposed to
    topological knots which have three. Once the $x$ and $z$ coordinates are
    known, $y$ is determined. We can use
    Eqns.~\ref{eqn:euclidean_legendrian_knot_relations} to indeed give explicit
    embeddings of Legendrian knots. For $y(t)$ to have a well-defined formula,
    whenever $\dot{x}(t)$ approaches zero so must $\dot{z}(t)$. Indeed,
    $\dot{z}(t)$ must approach zero \textit{faster}, in some sense, so that
    we apply some elementary technique such as L'H\^{o}pital's rule. The
    simplest formula for a circle in the $xz$ plane,
    $(\cos(t),\,0,\,\sin(t))$, lacks such a feature. By simply
    cubing the $z$ component this is salvaged. We define:
    \begin{equation}
        \gamma_{xz}(t)=\big(\cos(t),\,0,\,\sin^{3}(t)\big)
    \end{equation}
    Eqns.~\ref{eqn:euclidean_legendrian_knot_relations} tells us for an
    embedding to be Legendrian, $y(t)$ \textit{must} be given by:
    \begin{equation}
        y(t)=\frac{\dot{z}(t)}{\dot{x}(t)}=-3\cos(t)\sin(t)
    \end{equation}
    Our \textbf{Legendrian unknot} is then defined by:
    \begin{equation}
        \gamma(t)=\big(
            \cos(t),\,
            -3\cos(t)\sin(t),\,
            \sin^{3}(t)
        \big)
    \end{equation}
    \begin{figure}
        \centering
        \includegraphics{legendrian_unknot_002.pdf}
        \caption{Legendrian Unknot}
        \label{fig:legendrian_unknot_002}
    \end{figure}
    Simple formulas such as this give us the ability to create animations. In
    Fig.~\ref{fig:legendrian_unknot_002} we see this embedding together with
    several of the hyperplanes defined by the Euclidean contact form. Note the
    knot is indeed tangent to each plane in the distribution.
    \par\hfill\par
    \begin{figure}
        \centering
        \resizebox{\textwidth}{!}{%
            \includegraphics{cusps_in_the_plane_001.pdf}
        }
        \caption{Curve with Cusps}
        \label{fig:cusps_in_the_plane_001}
    \end{figure}
    The requirement that $\dot{z}(t)$ approach zero whenever $\dot{x}(t)$ does
    tells us that the projection of a Legendrian knot down the $y$ axis will
    have \textit{cusps}, the quintessential model for this is the curve given
    by $y^{3}=x^{2}$, see Fig.~\ref{fig:cusps_in_the_plane_001}.
    Indeed, for our Legendrian unknot in Fig.~\ref{fig:legendrian_unknot_002}
    we chose to project down a vector nearly parallel to the $z$ axis in order
    to show the tangency with the plane distribution. Projecting down the
    $y$ axis yields Fig.~\ref{fig:legendrian_unknot_001}. This is called the
    \textbf{front projection} of the Legendrian knots, and as alluded to we can
    see cusps at the left and right edges where $\dot{x}(t)$ approaches zero.
    \begin{figure}
        \centering
        \includegraphics{legendrian_unknot_001.pdf}
        \caption{Front Projection of the Legendrian Unknot}
        \label{fig:legendrian_unknot_001}
    \end{figure}
    \par\hfill\par
    As in the topological knot setting, to work with Legendrian knots requires
    a means of working with their projections in the plane. A
    \textbf{Legendrian knot diagram} of a Legendrian knot is a knot diagram
    with crossing information recorded, and additionaly with the cusps indicated
    by (locally) appealing to Fig.~\ref{fig:cusps_in_the_plane_001}. For
    the Legendrian unknot we have been working with see
    Fig.~\ref{fig:legendrian_unknot_cusps_001}.
    \begin{figure}
        \centering
        \includegraphics{legendrian_unknot_cusps_001}
        \caption{Legendrian Unknot Diagram}
        \label{fig:legendrian_unknot_cusps_001}
    \end{figure}
