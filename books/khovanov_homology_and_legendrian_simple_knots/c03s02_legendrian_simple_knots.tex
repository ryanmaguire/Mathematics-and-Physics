\section{Legendrian Simple Knots}
    We may repeat our discussion of topological knot types for Legendrian
    knots with little alteration. An \textit{ambient Legendrian isotopy} from
    a Legendrian knot $\gamma:\mathbb{S}^{1}\rightarrow\mathbb{R}^{3}$ to a
    Legendrian knot $\gamma':\mathbb{S}^{1}\rightarrow\mathbb{R}^{3}$ is an
    ambient isotopy $H:\mathbb{R}^{3}\times[0,\,1]\rightarrow\mathbb{R}^{3}$
    between $\gamma$ and $\gamma'$ such that for all $t\in[0,\,1]$ the
    restriction of $H$ to the unit circle is a Legendrian knot. Legendrian
    equivalent knots are those that can be deformed into each other by a
    smooth family of Legendrian knots. The information in a Legendrian knot is
    entirely captured by the crossings and the cusps, and we may thus define
    \textbf{extended Legendrian Gauss code} and
    \textbf{Legendrian Reidemeister moves}. The theory of Legendrian knots is
    richer than topological ones since two knots may belong to the same
    \textit{topological} knot type, but may not be equivalent as Legendrian
    knots. To handle such subtleties requires Legendrian knot invariants, which
    are invariants of Legendrian knot diagrams. Such objects are almost never
    invariants of topological knots.
    \par\hfill\par
    There are two so-called \textit{classical} invariants of Legendrian knot
    diagrams, the \textit{Thurston-Bennequin number} and the
    \textit{rotation number} \cite{Ding2006LEGENDRIANHA}. We first describe the
    Thurston-Bennequin number. Take a Legendrian knot diagram $K$ can compute
    the writhe $w(K)$. The
    cusps in the diagram come in pairs, for every \textit{left} cusp there is
    a \textit{right} cusp. The Thurston-Bennequin number thus has two
    equivalent phrasings: compute:
    \begin{equation}
        tb(K)=w(K)-\frac{1}{2}c(K)
    \end{equation}
    where $c(K)$ is the \textit{total} number of cusps, or compute
    $w(k)-R(K)$ where $R(K)$ is the total number of \textit{right cusps only}.
    It is indeed a Legendrian knot invariant. The Legendrian Type II and III
    moves do not change the writhe nor the number of cusps, and the Legendrian
    Type I move changes the number of cusps by two and the writhe by one, so
    this cancels in the difference.
    \par\hfill\par
    \begin{figure}
        \centering
        \includegraphics{legendrian_unknot_zig_zag_001.pdf}
        \caption{Zig-Zag Legendrian Unknot Diagram}
        \label{fig:legendrian_unknot_zig_zag_001}
    \end{figure}
    By repeatedly zig-zagging, as in
    Fig.~\ref{fig:legendrian_unknot_zig_zag_001}, we leave the writhe unchanged
    and increase the number of cusps by two, hence decreasing the
    Thurston-Bennequin number by one. The result is \textit{topologically}
    equivalent, but Legendrianly different. Since we may repeat this process
    \textit{ad infinitum} we see that the Thurston-Bennequin number is not
    bounded below for any topological knot type. It is possible that it is
    bounded above, however, and the \textit{maximum} possible
    Thurston-Bennequin number of a knot among all Legendrian knot diagrams
    yields a topological knot invariant.
    \par\hfill\par
    The other classical invariant of Legendrian knot diagrams is the
    \textit{rotation number}, which counts the number of times the knot spins
    around. This is given a combinatorial computation once an orientation is
    given to a Legendrian knot diagram. At each of the right cusps count the
    number $r_{+}$ of times the orientation goes \textit{east-to-west} as you
    cross the cusp, and the number $r_{-}$ of times you go west-to-east.
    The rotation number is then:
    \begin{equation}
        rot(K)=\frac{1}{2}(r_{+}+\ell_{+})
    \end{equation}
    The Legendrian Reidemeister moves do not change the rotation number yielding
    an invariant of \textit{oriented} Legendrian knot diagrams. Taking the
    absolute value gives us an invariant of unoriented diagrams.
    \par\hfill\par
    Our zig-zagging construction in Fig.~\ref{fig:legendrian_unknot_zig_zag_001}
    changes the rotation number by either $+1$ or $-1$, depending on the
    orientation given. This operation yields a \textit{mountain range} of
    inequivalent Legendrian knots of the same topological knot type. Given
    a point $(r,\,t)$ in the mountain range, representing a Legendrian knot
    diagram $K$ with $tb(K)=t$ and $rot(K)=r$, we obtain points
    $(r-1,\,t-1)$ and $(r+1,\,t-1)$ by applying our zig-zags. By
    Bennequin's theorem there are a finite number of diagrams with
    maximum Thurston-Bennequin number, and these serve as the
    \textit{peaks} of the mountain.
    \par\hfill\par
    For a given topological knot type $\mathcal{K}$ it is possible for two
    inequivalent Legendrian knot diagrams to represent the same position in
    the mountain diagram. While it was conjectured such knots may knot exist,
    Chekanov found such an example \cite{ChekanovDifAlgOfLegLinks},
    the $m_{3}$ twist knot, also known as the $5_{2}$ knot on the Rolfsen
    table. This motivates a definition.
    \begin{definition}[\textbf{Legendrian Simple Knot}]
        A Legendrian simple knot is a topological knot type $\mathcal{K}$
        such that for any two Legendrian knot diagrams $K$ and $K'$ of
        $\mathcal{K}$ with the same Thurston-Bennequin and rotation numbers,
        it is true that $K$ and $K'$ are Legendrian equivalent.
    \end{definition}
    Legendrian simple knots are the knots where the classical Legendrian
    invariants completely classify the Legendrian diagrams of the knot. Since
    not all knots are Legendrian simple there is then a need for other
    invariants that are able to distinguish Legendrian knot diagrams having
    the same $(tb,\,rot)$ pair. There has been success in inventing such
    invariants, and Legendrian knot polynomials and homology theories now
    exist. Fortunately for us, we seek to study Legendrian simple knots. The
    classification of which knots are Legendrian simple has made some real
    progress.
    \begin{theorem}[Eliashberg-Fraser, 1998]
        The unknot is Legendrian simple.
    \end{theorem}
    \begin{proof}
        See \cite{EliashbergFraserClassificationTopTrivialLegKnots}.
    \end{proof}
    Further knots have been found to be Legendrian simple, including all of the
    torus knots and the figure eight knot \cite{EtnyreHondaContactTopologyI}.
    More examples can be found in the Legendrian knot atlas
    \cite{LegendrianKnotAtlas}, including several knots types which have been
    conjectured to be Legendrian simple.
