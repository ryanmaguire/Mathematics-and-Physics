This thesis examines computational methods in knot theory. At first glance
this may seem to be a light-hearted endeavor since knots and links are very
intuitive objects. Indeed, many of the ideas behind the theory could be
adequately understood by a high school student, or even a junior high school
student with sufficient motivation. But as one dives deeper into the field they
discover many of the calculations to be performed are quite daunting. Three of
the four \textit{knot invariants} we are to discuss are known to fall into
the category of \textbf{NP-Hard} problems. To put it simply, being able to
compute any of these invariants for general knots in polynomial time
would lead one to proving $\textbf{P}=\textbf{NP}$,
\footnote{%
    Very briefly, this equality means that the ability to \textit{check} if a
    proposed solution is correct in polynomial time (\textbf{NP}) means it is
    possible to \textit{find} a solution in polynomial time (\textbf{P}). A
    simple example, 2 and 7 are the prime factorization of 14. You can
    \textit{check} this very quickly by multiplying 2 and 7. It is a much
    harder task to \textit{find} a prime factorization, in general.
}
earning the researcher
one million dollars from the Clay institute. This may, perhaps, lead to the
collapse of encryption as we know it, global banking would fail, and society
would fall into disarray.\footnote{%
    The author seriously doubts these statements, but many believe in them.
}
Thousands of other problems fall into the realm of
\textbf{NP-Hard} and \textbf{NP-Complete}, and finding quick solutions to any
of these would also prove $\textbf{P}=\textbf{NP}$. Since this has not been
done for any of these problems, many mathematicians and computer scientists
believe $\textbf{P}\ne\textbf{NP}$, meaning these problems do not have
efficient means of solving. For us this means our knot invariants will forever
be difficult to calculate. Failing to be dissuaded, we continue onward and
see what can be done.
\section{Algorithms}
    The primary result in this thesis is a new algorithm for the Jones
    polynomial. There are several that exist in the literature already with
    various trade-offs in time and space. This new one has excellent spatial
    properties.
    \begin{theorem}
        There is an algorithm with input a valid extended Gauss code on $N$
        crossings, with output the Jones polynomial of the (possibly virtual)
        knot described by the extended Gauss code, such that the best, worst,
        and average time complexity is identical and of
        $O(\textrm{poly}(N)\cdot{2}^{N})$, and with spatial requirement $O(N)$.
    \end{theorem}
    The straight-forward recursive algorithm discussed in chapter 3 has
    $O(\textrm{poly}(N)\cdot{2}^{N})$ requirements in both time and space.
    Simple improvements can be made on this that create best-case $O(N)$ in
    time and space, but this is for very specific knots. The worst-case scenario
    is still $O(\textrm{poly}(N)\cdot{2}^{N})$. An implementation of the
    modified-recursive algorithm is available in \cite{sage}.
    \par\hfill\par
    The Jones polynomial is generalized by the HOMFLY-PT polynomial, and the
    Kauffman skein algorithm is able to compute the Jones polynomial in
    $O(\textrm{poly}(N)\cdot{2}^{N})$ time, and by performing a depth-first
    search through the decision tree one can achieve $O(N)$ space
    \cite{Burton2018HOMFLFixedParameter}. Modifications of this have been made
    (see \cite{GOUESBET1999271}) providing
    $O(\textrm{poly}(N)\cdot{2}^{N})$ time requirements and polynomial
    spatial needs. With these algorithms there is some sensitivity to the
    \textit{initial conditions}, the data representing the knot, that can
    make the computation much longer or much shorter. The number of
    \textit{states} used can be more than double size of the
    \textit{cube of resolutions}, or it can be a fraction of it. What this
    means is the best and worst case scenario complexities vary, even for
    nearly identical inputs, and it becomes a bit of a science trying to get the
    best computational time for a given knot. This new algorithm picks a nice
    middle ground and the best and worst case scenarios are identical.
    \par\hfill\par
    The fastest algorithm known to the author is provided in
    \cite{Burton2018HOMFLFixedParameter} with
    $O(\textrm{poly}(N)\cdot{2}^{\sqrt{N}})$ time requirements, but
    $(2\sqrt{N})!^{2}$ spatial requirements, far from polynomial. A graph of
    $f(x)=\Gamma(2\sqrt{x}+1)^{2}/2^{x}$ should convince the reader that this
    tends to zero as $x$ approaches infinity, and indeed using some of the
    derivative rules of the Gamma function one can prove this. Hence this
    non-polynomial space algorithm is still better than the na\"{i}ve recursive
    algorithm, but $(2\sqrt{N})!^{2}$ still grows enormously quick. The
    speed of the algorithm makes it worthwhile to implement, and this is
    done in \cite{regina}. This library was used extensively throughout the
    making of this thesis.
    \par\hfill\par
    Finding improvements for these algorithms is still an active area of
    research. Many methods are limited to
    \textit{classical} knots, embeddings of
    $\mathbb{S}^{1}$ in to $\mathbb{R}^{3}$ or $\mathbb{S}^{3}$, but the
    polynomial is still well defined for \textit{virtual knots}. These are
    embeddings of $\mathbb{S}^{1}$ in to $M\times\mathbb{R}$ where $M$ is a
    compact orientable surface. In chapter 3 we provide a new generalization
    of the Tait graph for classical knots that allows one to define
    \textit{virtual} Tait graphs. The graph-theoretical algorithm for the
    Jones polynomial which is based on a knots Tait graph can then be mimicked
    with only slight alteration, providing a graph-theoretical algorithm for
    the computation of the Jones polynomial of virtual knots.
\section{Conjectures and Results}
    After studying new algorithms for the Jones polynomial, and discussing
    old algorithms for Khovanov homology, we present new tabulations of these
    invariants (as well as the HOMFLY-PT polynomial) and state some conjectures.
    One of the classic conjectures of knot theory asks if the Jones polynomial
    is capable of uniquely determining if a knot is equivalent to the unknot.
    That is, if $J_{K}$ denotes the Jones polynomial of a knot $K$, and if
    $J_{K}$ is equal to the Jones polynomial of the unknot, must it be true
    that $K$ is equivalent to the unknot? The conjecture has been verified for
    knots up to 24 crossings \cite{VerificationUnknotJonesConjUpTo24} but
    remains open. If one instead considers Khovanov homology, the answer is
    yes. That is, if a knot has the same Khovanov homology as the unknot, then
    it is equivalent to the unknot. In chapter 4 we expand upon this idea and
    conjecture that certain knot types may be categorized by their Khovanov
    homologies. In particular, we pose the following question.
    \begin{question}
        If a knot type is Legendrian or transversally simple, does its Khovanov
        homology uniquely distinguish it among all other topological knot types?
    \end{question}
    We use the torus and twist knots, as well as Ng's table of conjectured
    Legendrian simple knots, as a test case for this conjecture. All
    torus knots are Legendrian simple, and the Legendrian and transversal
    classification of twist knots has been completed in
    \cite{EtnyreEtAlLegendrianAndTransverseTwistKnots}. We prove by brute
    force methods the following theorem.
    \begin{theorem}
        If a prime knot $K$ has crossing number less than or equal to 19 and
        has the Khovanov polynomial (the Poincar\'{e} polynomial
        of its Khovanov homology) of a torus or twist knot $K'$, then $K$ is
        equivalent to $K'$. In particular, if $K$ and $K'$ have the same
        Khovanov homology they are equivalent since the Khovanov polynomial is
        the Poincar\'{e} polynomial of Khovanov homology.
    \end{theorem}
    By similar brute force methods we check the conjectured Legendrian simple
    knots in the Legendrian knot atlas and can provide the following claim.
    \begin{theorem}
        If a prime knot $K$ has crossing number less than or equal to 19 and
        has the Khovanov polynomial of any conjectured
        Legendrian simple knot $K'$ found
        in \cite{LegendrianKnotAtlas}, then $K$ is equivalent to $K'$.
    \end{theorem}
    These theorems are proved using computers. While it would be simple enough
    to program a Khovanov homology algorithm, and press \textit{go},
    the exponential nature of Khovanov homology makes this a challenge.
    Some methods of computation, as in \cite{sage}, using exponential in
    space algorithms that can not get past 15 crossings without needing 100+ GB
    of memory. We instead resort to the algorithm outlined in
    \cite{BarNatan2006FASTKH}, which is both the fastest and most efficient
    in memory algorithm presently available, as far as the author is aware.
    Even with this the challenge of tabulating up to 19 crossings would have
    taken about 3 years. This was not done. We instead use the Jones polynomial
    to weed out possible matches for Khovanov homology, and only when a knot $K$
    is found to have the same Jones polynomial as a torus knot, or a twist knot,
    or a conjectured Legendrian simple knot $T$, are the Khovanov polynomials
    of $K$ and $T$ computed and compared. In this manner we can reduce the
    calculation to two and a half days on somewhat decent hardware. As a side
    effect we get a tabulation of the Jones polynomial of all prime knots up
    to 19 crossings, and can play around with the statistics. This data is
    freely available and hosted via Dartmouth's servers, see
    \cite{JonesData}.
