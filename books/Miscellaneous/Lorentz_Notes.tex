\documentclass{article}

%---------------------------Packages----------------------------%
\usepackage{geometry}
\geometry{a5paper, margin=14mm}
\usepackage{graphics, float}            % Graphics/Images.
\usepackage[english]{babel}             % Language typesetting.
\usepackage[dvipsnames]{xcolor}         % Color names.
\usepackage{mathtools, esint, mathrsfs} % amsmath and integrals.
\usepackage{amsthm, amsfonts, amssymb}  % Fonts and theorems.
\usepackage{upgreek}                    % Non-Italic Greek.
\usepackage{titletoc}                   % Allows \book in toc.
\usepackage[nottoc]{tocbibind}          % Bibliography in toc.
\usepackage[titles]{tocloft}            % ToC formatting.
\usepackage[hang,multiple]{footmisc}
\usepackage{imakeidx}                   % Used for index.
\usepackage[font={scriptsize},
            hypcap=true,
            labelsep=colon]{caption}    % Figure captions.
\usepackage[pdftex,
            pdfauthor={Ryan Maguire},
            pdftitle={Mathematics and Physics},
            pdfsubject={Mathematics, Physics, Science},
            pdfkeywords={Mathematics, Physics, Computer Science, Biology},
            pdfproducer={LaTeX},
            pdfcreator={pdflatex}]{hyperref}
\hypersetup{
    colorlinks=true,
    linkcolor=blue,
    filecolor=magenta,
    urlcolor=Cerulean,
    citecolor=SkyBlue
}                           % Colors for hyperref.
\usepackage[toc,acronym,nogroupskip,nopostdot]{glossaries}
\usepackage{glossary-mcols}

%------------------------Theorem Styles-------------------------%
\theoremstyle{plain}
\newtheorem{theorem}{Theorem}[section]

% Define theorem style for default spacing and normal font.
\newtheoremstyle{normal}
    {\topsep}               % Amount of space above the theorem.
    {\topsep}               % Amount of space below the theorem.
    {}                      % Font used for body of theorem.
    {}                      % Measure of space to indent.
    {\bfseries}             % Font of the header of the theorem.
    {}                      % Punctuation between head and body.
    {.5em}                  % Space after theorem head.
    {}

% Italic header environment.
\newtheoremstyle{thmit}{\topsep}{\topsep}{}{}{\itshape}{}{0.5em}{}

% Define environments with italic headers.
\theoremstyle{thmit}
\newtheorem*{solution}{Solution}

% Define default environments.
\theoremstyle{normal}
\newtheorem{example}{Example}[section]
\newtheorem{definition}{Definition}[section]
\newtheorem{axiom}{Axiom}[section]
\newtheorem{notation}{Notation}[section]

\author{Ryan Maguire}
\title{Notes on Lorentz Geometry}
\begin{document}
    \maketitle
    \tableofcontents
    \listoffigures
    \section{Preliminary Material}
        \begin{definition}
            An $(r,s)$ tensor of a Module $M$ over a ring $R$ is a
            multilinear function $A:(M^{*})^{r}\times{M}^{s}\rightarrow{R}$,
            where $M^{*}$ is the dual of $M$, and multilinear means that the
            restriction of $A$ to any of the components is linear.
        \end{definition}
        Of interest in manifold theory are tensor fields. The set
        $\mathfrak{X}(M)$ of vector fields over $M$ can be given the structure
        of a module over $C^{\infty}(M,\mathbb{R})$. Viewing a vector field
        $X:M\rightarrow{TM}$ as a smooth section from a manifold $M$ to it's
        tangent space, we have, for all $X,Y\in\mathfrak{X}(M)$ and all
        $f\in{C}^{\infty}(M,\mathbb{R})$:
        \begin{subequations}
            \begin{align}
                (X+Y)(p)&=X(p)+Y(p)\\
                (fX)(p)&=f(p)X(p)
            \end{align}
        \end{subequations}
        \begin{definition}
            A tensor field $A$ on a smooth manifold $M$ is an $(r,s)$ tensor
            $A:\mathfrak{X}^{*}(M)^{r}\times\mathfrak{X}(M)^{s}\rightarrow{C}^{\infty}(M,\mathbb{R})$,
            where $\mathfrak{X}(M)$ is the module of vector fields over $M$,
            and $\mathfrak{X}^{*}(M)$ is the dual of this.
        \end{definition}
        We can then define a metric tensor on a manifold to be a $(0,2)$ tensor
        field with certain properties. There is another way to do this. We want
        a metric tensor to give us a dot product, or a means of measuring
        angles, for each point. So for each point we want a bilinear function
        $g_{p}$ that takes in two tangent vectors at $p$ and returns a real
        number. It should also be symmetric since the dot product is, and
        non-degenerate. In terms of the tangent space at $p$, this is a function
        $g_{p}:T_{p}M\times{T}_{p}M\Rightarrow\mathbb{R}$. We want our function
        $g$ to provide, for every $p\in{M}$, a metric tensor $g(p)=g_{p}$, and
        we want this to occur smoothly. First we note that a bilinear function
        that takes in two tangent vectors at $p$ and returns a real number is an
        element of the dual $(T_{p}M\otimes{T}_{p}M)^{*}$. Since our manifold
        $M$ is finite dimensional, and since $T_{p}M$ has its vector space
        dimension equal to the manifold dimension of $M$,
        $T_{p}M\otimes{T}_{p}M$ is the tensor product of two finite dimensional
        vector spaces, and hence $(T_{p}M\otimes{T}_{p}M)^{*}$ is isomorphic
        to $T_{p}M^{*}\otimes{T}_{p}M^{*}$. So we can define a metric tensor
        to be a smooth section $g:M\rightarrow{T}M^{*}\otimes{T}M^{*}$, where
        $TM^{*}$ is the cotangent bundle.
        \par\hfill\par
        There is a way to get a quadratic form on $M$ from a metric tensor.
        Define, for each $p\in{M}$, the quadratic form
        $Q_{p}:T_{p}M\rightarrow\mathbb{R}$ by:
        \begin{equation}
            Q_{p}(v)=g_{p}(v,v)
        \end{equation}
        Since $g_{p}$ is a bilinear form, $Q_{p}$ is a quadratic form. By
        Sylvester's Law of Inertia, there is a basis for $T_{p}M$ such that
        $Q_{p}$ has the form:
        \begin{equation}
            Q_{p}(v)=\sum_{k=1}^{n}a_{k}v_{k}^{2}
        \end{equation}
        Where $a_{k}$ is either $0$, $1$, or $-1$. Sylvester's theorem also says
        the number of $0$, $1$, and $-1$ that occur is an invariant. That is,
        choosing another basis with such a representation of $Q_{p}$ will yield
        the same number of $a_{k}$ being 0, the same number being 1, and the
        same number being $-1$. Since a metric tensor is non-degenerate, then
        can be no zeros in this expansion, so a metric tensor has two invariant
        numbers $(p,n)$, the number of positive 1's and the number of negative
        1's. This ordered pair $(p,n)$ is locally constant on the manifold, and
        if the manifold is connected that means it is constant. This gives us
        the following:
        \begin{definition}
            The signature of a metric tensor $g$ on a connected manifold $M$ is
            the unique ordered pair $(p,n)$ where $p$ is the number of postive
            1's given by Sylvester's theorem and $n$ is the number of negative
            1's.
        \end{definition}
        \begin{definition}
            A Lorentz manifold is a smooth manifold $M$ with a metric tensor
            $g:M\rightarrow{T}M^{*}\otimes{T}M^{*}$ with signature $(n-1,1)$,
            where $n$ is the dimension of $M$.
        \end{definition}
        \begin{example}
            From Sylvester's theorem we know that for any Lorentz manifold
            $(M,g)$ and for any point $p\in{M}$ there is a chart
            $(\mathcal{U},\varphi)$ such that for all $v\in{T}_{p}M$ we have:
            \begin{equation}
                g_{p}(v,v)=-\textrm{d}\varphi_{1}^{2}(v)+
                    \sum_{k=2}^{n}\textrm{d}\varphi_{k}^{2}(v)
            \end{equation}
            Where $\textrm{d}\varphi_{k}$ is the 1-form defined by
            $\textrm{d}\varphi_{k}(\partial\varphi_{\ell})=\delta_{k\ell}$,
            where $\delta_{k\ell}$ is the Kronecker delta, and
            $\partial\varphi_{j}$ is the $j^{th}$ standard basis vector for
            $T_{p}M$ given by the chart $(\mathcal{U},\varphi)$. If we let
            $M=\mathbb{R}^{3}$, then we can cover $M$ with a single global chart
            $(\mathbb{R}^{3},\textrm{Id}_{\mathbb{R}^{3}})$ and can define the
            metric tensor $g$ by:
            \begin{equation}
                g=-\textrm{d}t^{2}+\textrm{d}x^{2}+\textrm{d}y^{2}
            \end{equation}
            The $z$ axis is replaced by a $t$ since we usually think of this as
            representing time. This is called the 3-dimensional Minkowski space.
        \end{example}
        A metric tensor does not need to be positive-definite, like in the case
        of a Riemannian metric. It is possible for a non-zero tangent vector to
        have $g_{p}(v,v)=0$. In the case of Minkowski space this occurs when
        the sum of the squares of the spatial components equals the square of
        the time component. Solving
        $\textrm{d}t^{2}=\textrm{d}x^{2}+\textrm{d}y^{2}$ gives us a cone:
        \begin{equation}
            \textrm{d}t=\pm\sqrt{\textrm{d}x^{2}+\textrm{d}y^{2}}
        \end{equation}
        This is called the \textit{light cone} of the point $p$. Since every
        Lorentz manifold locally has some chart that represents the metric
        tensor in this form, the light cone of a point $p$ in an arbitrary
        Lorentz manifold is a well-defined concept. This notion gives us a few
        new definitions. If the time component is greater than the spatial
        component, then $g_{p}(v,v)$ is negative. We give this a name:
        \begin{definition}
            A timelike vector at a point $p$ in a Lorentz manifold $(M,g)$ is a
            tangent vector $v\in{T}_{p}M$ such that $g_{p}(v,v)<0$.
        \end{definition}
        If $g_{p}(v,v)$ is positive, the above equation tells us the spatial
        components sum to a greater value than the square of the time one. This
        is also given a name:
        \begin{definition}
            A space like vector at a point $p$ in a Lorentz manifold $(M,g)$ is
            a tangent vector $v\in{T}_{p}M$ such that $g_{p}(v,v)>0$.
        \end{definition}
        Now for the case when we have equality. The naming convention comes from
        physics. If we try to describe the displacement of a photon of light
        with respect to time, we get the equation $r^{2}=c^{2}t^{2}$, where
        $c$ is the speed of light. Translating this for our Lorentz manifold,
        if a tangent vector $v$ at the point $p$ satisfies
        $g_{p}(v,v)$, then there is some chart where we have:
        \begin{equation}
            \textrm{d}t^{2}(v)=\sum_{k=2}^{n}\textrm{d}x_{k}^{2}(v)
        \end{equation}
        This mimics our $r^{2}=c^{2}t^{2}$ equation if we set $c=1$. And indeed,
        in \textit{natural units}, which is a unit system used in modern
        physics, the speed of light is taken to be 1. This suggest that we name
        such a vector $v$ as follows:
        \begin{definition}
            A lightlike vector at a point $p$ in a Lorentz manifold $(M,g)$ is a
            non-zerotangent vector $v\in{T}_{p}M$ such that $g_{p}(v,v)=0$.
        \end{definition}
        And the last case combines lightlike vectors with the zero vector.
        \begin{definition}
            A null vector at a point $p$ in a Lorentz manifold $(M,g)$ is a
            tangent vector $v\in{T}_{p}M$ such that either $v$ is lightlike, or
            $v$ is zero. That is, any element $v\in{T}_{p}M$ such that
            $g_{p}(v,v)=0$.
        \end{definition}
        \begin{definition}
            The light cone of a point $p$ in a Lorentz manifold $(M,g)$ is the
            set of all points $v\in{T}_{p}M$ such that $g_{p}(v,v)=0$.
        \end{definition}
\end{document}